%%%%\renewcommand{\thefootnote}{\arabic{footnote}}

\chapter*{}%%% chap 30

\section*{Prof. Gowravaram Ramachandran: Profile of a Theoretical Physicist}
 
 Gowravaram Ramachandran was born on 29$^{\rm th}$ October, 1936, at Vangal (near Karur), Andhra Pradesh. He was the elder of the two sons of his parents, Shri Gowravaram Venkata Krishnaiah and Smt Kamala. In a life-span of 84 years, G. Ramachandran etched for himself an illustrious life in Physics and contributed significantly to quantum physics, nuclear physics, particle physics and astrophysics.
 
He obtained B.Sc.(Hons) in Physics and Mathematics from the Madras Christian College, Tambaram, Chennai, in 1955, and Master's degree in Physics from Madras University in the year 1957. Immediately thereafter, G. Ramachandran pursued his doctoral research under the guidance of Prof. Alladi Ramakrishnan and obtained his Ph.D. degree in the year 1964 from the Madras University. He had the honour of being one of the founding members of the Theoretical Physics Seminar Group, which was founded by Dr. Alladi Ramakrishnan in 1960 and, under the aegis of this Group, he had the privilege of interacting with several eminent scientists, including Niels Bohr, George Gamow, W. Heitler,  Subrahmanyan Chandrasekhar, Murray Gell-Mann, Abdus Salam, R.H. Dalitz, Donald Glaser, E.C.G. Sudarshan, Leon Rosenfeld, and A.M. Lane. Prof. Niels Bohr, who had visited India as a guest of the Government of India, made a positive mention of the Group's accomplishments to the then Prime Minister, Pandit Jawaharlal Nehru. The then Education and Finance Minister of Madras State, Sri C. Subramaniam (Bharat Ratna), facilitated a meeting of this Group with the Prime Minister. This resulted in the creation of the Institute of Mathematical Sciences (IMSc) also known as MATSCIENCE) at Chennai, in 1962.
 
 \eject
 
 Dr. Ramachandran pursued his post-doctoral research at the Research and Training School, Indian Statistical Institute (ISI), Kolkata, and continued to work there as Associate Professor till 1971. His work was greatly appreciated by Dr. C. R. Rao, FRS, the then Director of ISI, Kolkata, and with his encouragement, Dr. Ramachandran developed the infrastructure for research in Particle Physics at the Institute.
 
On an invitation from Professor G.N. Ramachandran, FRS, Dr. Ramachandran joined the Molecular Biophysics Unit (MBU), Indian Institute of Science, Bangalore, as a Visiting Fellow in 1971 and continued in that post till 1973. He joined the Department of Physics, University of Mysore, as a Reader in 1973, and rose to the position of Professor in 1984. After superannuation in 1996, he was CSIR Emeritus Scientist at Mysore University till 2001. Subsequently, on an invitation from Prof. R. Cowsik, the then Director of Indian Institute of Astrophysics (IIA), Bangalore, he joined IIA as a Visiting Senior Professor and worked there till 2007.
 
Prof. Ramachandran continued to be academically active in teaching and research after 2007 under the umbrella of GVK Academy, which he had set up in the revered memory of his father. He left this world on April 9, 2020.

Prof. Ramachandran was deeply interested in a variety of topics in Theoretical Physics. Some of his contributions, resulting from research done in collaboration with several colleagues and students, are listed below:
\begin{itemize}
\item Beginnings of the so-called Medium Energy Physics: Theoretical studies on the use of nuclear targets with particle beams for studying nuclear structure as well as particle properties in the strong interaction regime. This lies at the interface of particle and nuclear physics and the pioneering work began in Madras in late fifties and early sixties.

\item Classification of spin systems into oriented and non-oriented spin systems.

\item Formulation of Multi-Axial representation of spin-j systems, which is applicable to both pure and mixed spin systems, similar to Majorana representation of a spin-j state, which is applicable to pure states only.

\item Extraction of reaction amplitudes from spin (polarization) measurements.

\item Development of a new approach to spin in nuclear and particle reactions.

\item A new description of atoms interacting with lasers and squeezed spin states.

\item Astrophysical spectro-polarimetry.

\item Meson production in N-N collisions and photo and electro production of higher spin mesons.

\item Neutron-Proton fusion and photodisintegration of deuteron at astrophysical energies relevant to big bang nucleosynthesis.

\item Investigation of Einstein-Podolsky-Rosen (EPR) phenomena.

\item Formulation of Multivariate representation of a spin-j density matrix, the variates being the three spin components.

\item Studies in three body problem.
\end{itemize}

He was the recipient of the Mysore University Golden Jubilee Award for research in science and technology in 1980. He trained several Ph.D. students and, with their collaboration, contributed more than 100 research papers to prominent international and national journals and presented more than 150 papers and invited talks at conferences and symposia.

His zeal for mentoring and training students never diminished and the same zeal found expression in the two books he authored:
\begin{enumerate}
\item[1.] Preamble to Quantum Theory of Angular Momentum, Gateway to Atoms, Nuclei and Elementary Particles (Prayoga, 2016)

\item[2.] Introduction to vectors, axial vectors, tensors and spinors (Vijayalakshmi Prakashana, 2017)
\end{enumerate}

Prof. Ramachandran's wife, Smt. Seethalakshmi, and his children Lata, Devi, Ramakrishna, Gowri, Krishnakumar and Anuradha were pillars of strength and support for him throughout his life and, together, they welcomed his many students into their family.

\eject

\begin{center}
\textbf{Appendix -- A}\\[4pt]
\textbf{List of students guided by Prof. G. Ramachandran for the Ph. D. degree.}
\end{center}

{\renewcommand{\arraystretch}{1.2}
\fontsize{7.5}{9.5}\selectfont
\noindent
\begin{longtable}{@{}|p{2cm}|p{2cm}|p{1.3cm}|p{2.5cm}|}
\hline
\multirow{2}{2cm}{\centering\textbf{Student}} & \multirow{2}{2cm}{\centering\textbf{Institution}} & \multicolumn{1}{p{1.3cm}|}{\centering\textbf{Degree Awarded  by}} &  \multirow{2}{2.5cm}{\centering\textbf{Present Position}} \\
\hline
Dr. Anup Kumar Rej & Research and Training School, Indian Statistical Institute, Kolkata &  University of Calcutta &\\
\hline
Dr. M. V. N. Murthy &  Department of Physics, University of Mysore & University of Mysore & Retd. Professor, Institute of Mathematical Sciences, Chennai
\\
\hline
Dr. K. Venkatesh & Department of Physics, University of Mysore & University of Mysore & Retd. Professor, M S Ramaiah Institute of Technology, Bangalore\\
\hline
Dr. R. S. Keshavamurthy &  Department of physics, University of Mysore & University of Mysore & Retd. Scientist, Indira Gandhi Centre for Atomic Research, Kalpakkam\\
\hline
Dr. V. Ravishankar & Department of Physics, University of Mysore & University of Mysore & Professor, Department of Physics, IIT Delhi\\
\hline
Dr. K. S. Mallesh & Department of Physics, University of Mysore & University of Mysore & Retd. Professor, Department of Physics, University of Mysore\\
\hline
Dr. S. N. Sandhya & Department of Physics, University of Mysore & University of Mysore & Faculty Member at Miranda House, New Delhi.\\
\hline
Dr. Sudha Rao Alike & Department of Physics, University of Mysore &  University of Mysore & Retd. Associate Professor, Teresian College, Mysore\\
\hline
Dr. Swarnamala Sirsi &  Department of Physics, University of Mysore & University of Mysore & Retd. Associate Professor, Yuvaraja College, University of Mysore\\
\hline
Dr. M.S. Vidya & Department of Physics, University of Mysore & University of Mysore & Founder Trustee, Vidya Online Charitable Trust, New Delhi. \\
\hline
Dr. A.R. Usha Devi & Department of Physics, University of Mysore & University of Mysore & Professor, Department of Physics, Bangalore University\\
\hline
Dr. P.N. Deepak & Department of Physics, University of Mysore and Indian Institute of Astrophysics, Bangalore & University of Mysore & Assistant Professor, Department of Physics, Birla Institute of Technology and Science, Goa Campus \\
\hline
Dr. J. Balasubramanyam & Indian Institute of Astrophysics, Bangalore & Bangalore University & Faculty Member at BASE, Bangalore \\
\hline
Dr. Vikas M. Shelar & GVK Academy & National Institute of Technology Karnataka, Surathkal & Assistant Professor, Physics Dept., Ramaiah University, Bangalore\\
\hline
\end{longtable}}

He mentored several other students, unofficially, during their research work for Ph.D., including Late Dr. R. K. Umerjee, Dr. K. Ananthanarayanan, Sri S. B. Patangi, Dr. Yee Yee Oo and Dr. S. P. Shilpashree.

\begin{center}
\textbf{Appendix -- B}\\[4pt]
\textbf{Publications of Prof. G. Ramachandran}
\end{center}

Prof.G.Ramachandran contributed more than 100 papers in reputed journals that include :
\begin{itemize}
\item Physical Review and Physics Review Letters (18)
\item Nuclear Physics and Physics Letters (17)
\item Journal of Physics (17)
\item Modern Physics Letters and International Journal of Modern Physics (10) 
\item Pramana (10)
\item Nuovo Cimento (4)
\item Journal of Quantitative Spectroscopy and Radiative Transfer (3)
\item Foundations of Physics (1)
\end{itemize}

The following is the list of his papers.







