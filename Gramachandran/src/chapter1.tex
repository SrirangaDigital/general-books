\chapter[QCD Breaks Lorentz Invariance and Colour]{QCD Breaks Lorentz Invariance and Colour$^{*}$}\label{chap1}

\footnotetext[1]{G.Ramachandran was my teacher and friend for nearly sixty years. He has had an illustrious career training and guiding generations of students and accomplishing novel research. I dedicate this article to his memory.}

\Authorline{A. P. Balachandran\footnote[2]{\url{balachandran38@gmail.com}}}

\begin{center}
\textit{Physics Department, Syracuse University, Syracuse,}\\ 
\textit{New York 13244-1130, U.S.A.}
\end{center}


\section*{Abstract}

\begin{quote}
In a previous work \cite{key1}, we have argued that the algebra of non-abelian superselection rules is spontaneously broken to its maximal abelian subalgebra, that is, the algebra generated by its completing commuting set (the two Casimirs and a basis of its Cartan subalgebra). In this paper, alternative arguments confirming these results are presented. In addition, Lorentz invariance is shown to be broken in QCD, just as it is in QED. The experimental consequences of these results include fuzzy mass and spin shells of coloured particles like quarks, and decay life times which depend on the frame of observation \cite{key2,key3,key4}.
\end{quote}

\section{Introduction}\label{chap1-sec1}

Quantum field theory (QFT) is defined by the algebra $\mathcal{A}$ of local observables and an irreducible representation (IRR) $\pi$ of $\mathcal{A}$ on a Hilbert space $\mathcal{H}$. In general, there are many inequivalent IRR's $\pi_{0}$, $\pi_{1}$,... of $\mathcal{A}$ defining its superselection sectors.

For example, in QED, $\pi_{0}$ can be the sector with total charge $q_{0}= 0$, while $\pi_{n}$ can be the sector with total charge $q_{n}$. No observation can mix these sectors.

In QED, the charge operator $Q$ generates the abelian $U(1)$ group. In QCD, the $U(1)$ is replaced that the non-abelian $SU(3)$ of colour. Superficially, the superselection algebra seems to be the group algebra $\mathbb{C}SU(3)$. We have previously argued \cite{key1} that in reality it is a maximal abelian subalgebra of $\mathbb{CC}SU(3)$ generated by a complete commuting set (CCS). (See in this connection the Kadison-Singer conjecture and its proof \cite{key5}.) We can assume the CCS to be generated by the two Casimir operators and the operators on $\mathcal{H}$ representing the $\lambda_{3}$ and $\lambda_{8}$ of the Gell-Mann matrices. The remaining independent generators of $\mathbb{C}SU(3)$ are spontaneously broken.

There exist elegant proofs of Roepstorff \cite{key6}, Buchholz et. al. \cite{key2} and especially Fr\"ohlich et. al. \cite{key3} that infrared effects in QED break Lorentz invariance. We adapt these arguments here to show that generic elements of $SU(3)$ map an IRR $\pi$ of $\mathcal{A}$ to a distinct IRR $\pi'\neq \pi$ of $\mathcal{A}$. In Higgs theories, this phenomenon is well-known: when the Higgs field $\varphi$ breaks $SO(3)$ to $SO(2)$ say, as in the 't Hooft-Polyakov model \cite{key7}, $SO(3)$ transformations which change the direction of $\varphi$ at spatial infinity cannot be represented as unitary operators on $\mathcal{H}$. In the same way, here, any operator which disturbs the eigenvalues of the CCS is spontaneously broken.

In section \ref{chap1-sec2}, we review the QED result on Lorentz violation. This is then generalised to QCD in sections \ref{chap1-sec3} and \ref{chap1-sec4}.

The results of this paper can be adapted to any non-abelian gauge group.

\section{The Case of QED}\label{chap1-sec2}

QED is classified by a continuous family of superselection sectors.

The first is its classification by the charge $q_{n}$. In the representation $\pi_{n}$ of $\mathcal{A}$, the charge operator $Q$ has the eigenvalue $q_{n}$:
\begin{equation}
\pi_{n}: Q|n,P,\cdot \rangle=q_{n}|n,P,\cdot\rangle\label{chap1-eq2.1}
\end{equation}
where $P = (P^{\mu})$ denotes the total momenta. Local observables cannot change $q_{n}$.

Then, there are the sectors with ``in'' state vectors \cite{key4}
\begin{align}
& e^{q_{n}\int d^{3}x[A^{-}_{i}(x)\omega^{+}_{i}(x)-A^{+}_{i}(x)\omega^{-}_{i}(x)]}|n,P,\cdot\rangle := |n,P,\omega,\cdot\rangle,\label{chap1-eq2.2}\\
& |n,P,0,\cdot\rangle \equiv |n,P,\cdot\rangle,\qquad q_{n}\neq 0\label{chap1-eq2.3}
\end{align}
created by the infrared photons. Here $A^{\pi}_{i}$ are the positive and negative frequency parts of the electromagnetic potential in the Coulomb gauge, and the functions $\omega^{+}_{i}$, $\omega^{-}_{i}=\overline{\omega}^{+}_{i}$ are transverse:
\begin{equation}
\partial_{i}\omega^{\pi}_{i}(x)=0,\label{chap1-eq2.4}
\end{equation}

Also they do not vanish fast as we approach infinity:
\begin{equation}
\lim\limits_{r\to\infty}r^{2}\hat{x}_{i}\omega_{i}(x)^{\pm}\neq 0.\label{chap1-eq2.5}
\end{equation}

One such typical $\omega^{+}_{i}$ has the Fourier transform
\begin{equation}
\hat{\omega}_{i}^{+}(k)=\int d^{3}xe^{i\overrightarrow{k}\cdot \overrightarrow{x}}\omega^{+}_{i}(x)=\frac{1}{P\cdot k_{i}+i_{\epsilon}}(P_{i}-\overrightarrow{P}\cdot \widehat{k}\widehat{k}_{i})\label{chap1-eq2.6}
\end{equation}
(with $\epsilon$ decreasing to zero as usual). The momentum $P_{\mu}$ is the total momentum of the charged system. (We have not shown the individual momenta and charges of which $P$ and $q_{n}$ are composed as they are not importatnt for our considerations.) The important point here is that $\widehat{\omega}^{+}_{i}$ is not square-integrable:
\begin{equation}
\langle \omega,\omega\rangle := \lim\limits_{|\overrightarrow{k}|\to 0}\int^{\infty}_{|\overrightarrow{k}|}\frac{d^{3}k}{2|\overrightarrow{k}|}|\widehat{\omega}_{i}(k)|^{2}=\infty.\label{chap1-eq2.7}
\end{equation}
It is then a theorem \cite{key6} that the representation of $\mathcal{A}$ built on \eqref{chap1-eq2.2} is superselected: it is not the Fock space representation.

The appendix gives a derivation of \eqref{chap1-eq2.2}.

We now elaborate on the physical meaning of \eqref{chap1-eq2.2}. Consider the current
\begin{equation}
J^{\mu}(x)=q_{n}\int d\tau \delta^{4}(x-z(\tau))\frac{dz^{\mu}(\tau)}{d\tau}\label{chap1-eq2.8}
\end{equation}
It radiates photons of momenta $k := (| \overrightarrow{k}|, \overrightarrow{k})$. We are interested in infrared photons, so we assume that
\begin{equation}
\frac{dz^{\mu}(\tau)}{d\tau}=\dfrac{P^{\mu}}{m},\quad z^{\mu}(\tau)=\tau\frac{P^{\mu}}{m},\quad m^{2}=P^{\mu}P_{\mu},\quad m>0,\label{chap1-eq2.9}
\end{equation}
where $P^{\mu}$ is constant.

Now \eqref{chap1-eq2.8} generates the additional interaction
\begin{equation}
\int d^{3}xA_{\mu}(x)J^{\mu}(x).\label{chap1-eq2.10}
\end{equation}
It changes the ``in'' state to $|n,\omega\ldots\rangle$ as in shown in the appendix.

The following is a further important point. If the Lorentz boost
\begin{equation}
K_{i}=\int d^{3}xx_{i}[\overrightarrow{E}^{2}(x)+\overrightarrow{B}^{2}(x)]+\text{matter part}\label{chap1-eq2.11}
\end{equation}
is well-defined in $|n, 0;\cdot\rangle$, then it diverges in the sector $|n,\omega;\cdot \rangle$, $\omega\neq 0$ : Lorentz invariance is broken in the latter. We can see this as follows:
\begin{align}
& \rangle n,\omega;\cdot | \int d^{3}xx_{i}[\overrightarrow{E}^{2}(x)+\overrightarrow{B}^{2}(x)]|n,\omega;\cdot\rangle\notag\\
& =\langle n,0;\cdot |\int d^{3}xx_{i}[(\overrightarrow{E}-i\overrightarrow{\omega})^{2}(x)+\overrightarrow{B}^{2}(x)]|n,0;\cdot\rangle,\notag\\
&\omega_{i} := \omega^{+}_{i}+\omega^{-}_{i}\label{chap1-eq2.12}
\end{align}
and this diverges since $\overrightarrow{\omega}^{2}(x)=\mathcal{O}(\frac{1}{|\overrightarrow{x}|^{4}})$ as $|\overrightarrow{x}|\to \infty$.

There is an alternative approach for these considerations due to Roepstorff \cite{key6}. It is based on the Weyl algebra and the GNS construction. For free scalar fields, in four-dimensional spacetime, the Weyl algebra $\mathcal{W}$ has elements $W(f)$ where $f$ is a compactly supported real test function, $f\in \mathcal{C}^{\infty}_{0}$. If $g$ is another such test function,
\begin{align}
& W(f)W(g)=W(f+g)e^{i\sigma(f,g)/2},\label{chap1-eq2.13}\\
& \sigma (f,g)=i \int d^{4}xd^{4}yf(x)D(x-y)g(y),\label{chap1-eq2.14}\\
& D(x-y)=\text{commutator function} = \int \dfrac{d^{3}P}{(2\pi)^{3}}\dfrac{1}{2|P_{0}|}[e^{-iP\cdot (x-y)}-e^{iP\cdot (x-y)}].\label{chap1-eq2.15}
\end{align}
Since $(\Box + m^2)D = 0$ if the field has mass $m$, $\sigma$ vanishes on any function $f$ of the form $(\Box + m^2)h$, with $h\in \mathcal{C}^{\infty}_{0}$. On quotenting out such functions, $\sigma$ becomes a symplectic form.

Also, the function $\sigma (f,\cdot)$ defined by
\begin{equation}
\sigma(f,\cdot)(y)=i\int d^{4}xf(x)D(x-y)\label{chap1-eq2.16}
\end{equation}
fulfills the equations of motion.

Let us introduce the scalar product
$$
(f,g)=\frac{1}{(2\pi)^{3}}\int \dfrac{d^{3}k}{2|k_{0}|}\overline{\widetilde{f}(k)}\widetilde{g}(k),\qquad |k_{0}|=(\overrightarrow{k}^{2}+m^{2})^{1/2},
$$
where
\begin{equation}
\widetilde{f}(k)=\int d^{4}xe^{-ik\cdot x}f(x),\qquad \widetilde{g}(k)=\int d^{4}xe^{-ik\cdot x}g(x).\label{chap1-eq2.17}
\end{equation}
Then the Fock representation with the corresponding Hilbert space $\mathcal{H}$ is given by the following state $\omega_{0}$ on the Weyl algebra and the GNS construction:
\begin{equation}
\omega_{0}(W(f))=e^{-(f,f)/2}.\label{chap1-eq2.18}
\end{equation}
If $|0\rangle$ is the Fock vacuum, we have, as can be checked,
\begin{equation}
\omega_{0}(W(f))=\langle 0|W (f)|0\rangle.\label{chap1-eq2.19}
\end{equation}

Now if $F$ is a linear functional on test functions, we can twist $\omega_{0}$ to a new state $\omega_{F}$:
\begin{equation}
\omega_{F}(W(f))=\omega_{0}(W(f))e^{i \Iim F(f)}.\label{chap1-eq2.20}
\end{equation}

\begin{thebibliography}{99}
\bibitem{key1} A. P. Balachandran and S. Vaidya, \textit{Spontaneous Lorentz Violation in Gauge Theories}, Eur. Phys. J. Plus \textbf{128}, 118 (2013) [arXiv:1302.3406 [hep-th]]. 
\bibitem{key2} D. Buchholz, \textit{Gauss' Law and the Infraparticle Problem}, Phys. Lett. B \textbf{174}, 331 (1986); D. Buchholz and K. Fredenhagen, \textit{Locality and the Structure of Particle States}, Commun. Math. Phys. \textbf{84}, 1 (1982). 
\bibitem{key3} J. Fr\"ohlich, G. Morchio, and F. Strocchi, \textit{Infrared problem and spontaneous breaking of the lorentz group in qed}, Physics Letters B \textbf{89}, 1, 61--64 (1979).
\bibitem{key4} A. P. Balachandran, S. K\"urk\c{c}\"uo\u{g}lu, A. R. de Queiroz and S. Vaidya, \textit{Spontaneous Lorentz Violation: The Case of Infrared QED}, Eur. Phys. J. C \textbf{75}, 2, 89 (2015) [arXiv:1406.5845 [hep-th]].

For earlier work especially in $2+1$ dimensions, see A. P. Balachandran, S. K\"urk\c{c}\"uo\u{g}lu and A. R. de Queiroz, \textit{Spontaneous Breaking of Lorentz Symmetry and Vertex Operators for Vortices}, Mod. Phys. Lett. A \textbf{28}, 1350028 (2013) [arXiv:1208.3175 [hep-th]].

\bibitem{key5} N. J. A. Harvey, \textit{An introduction to the Kadison-Singer Problem and the Paving Conjecture} (2013) \url{http://www.cs.ubc.ca/ nickhar/Publications/KS/KS.pdf} 

B. Roberts, \textit{Philosophy and physics in the kadison-singer conjecture}, Soul Physics, Philosophical Foundations of Physics (2013)
\bibitem{key6} G. Roepstorff, \textit{Coherent Photon States and Spectral Condition}, Commun. math. Phys. 19, 301--314 (1970).
\bibitem{key7} G. 't Hooft, \textit{Magnetic monopoles in unified gauge theories}, Nuc. Phys. B \textbf{79}, 2, 276--284 (1974);

A. M. Polyakov, Zh. Eksp. Teor. Fiz. Pis'ma. Red. \textbf{20}, 430 (1974) [JETP Lett. \textbf{20}, 194 (1974)].
\bibitem{key8} M. E. Peshkin and D. V. Schr\"oeder, \textit{An Introduction To Quantum Field Theory (Frontiers in Physics)}, Perseus Books, Reading, Massachusetts (1995).
\bibitem{key9} F. Riesz and B. S. Nagy, \textit{Functional Analysis}, Dover Books on Mathematics, Reprint Edition (1956).
\bibitem{key10} A. P. Balachandran, A. R. de Queiroz and S. Vaidya, \textit{Quantum Entropic Ambiguities: Ethylene}, Phys. Rev. D \textbf{88}, no. 2, 025001 (2013) [arXiv:1302.4924 [hep-th]].
\end{thebibliography}
