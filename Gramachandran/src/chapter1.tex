\chapter[QCD Breaks Lorentz Invariance and Colour]{QCD Breaks Lorentz Invariance and Colour$^{*}$}\label{chap1}

\footnotetext[1]{G.Ramachandran was my teacher and friend for nearly sixty years. He has had an illustrious career training and guiding generations of students and accomplishing novel research. I dedicate this article to his memory.}

\Authorline{A. P. Balachandran\footnote[2]{\url{balachandran38@gmail.com}}}

\begin{center}
Physics Department, Syracuse University, Syracuse,\\ 
New York 13244-1130, U.S.A.
\end{center}


\section*{Abstract}

\begin{quote}
In a previous work \cite{key1}, we have argued that the algebra of non-abelian superselection rules is spontaneously broken to its maximal abelian subalgebra, that is, the algebra generated by its completing commuting set (the two Casimirs and a basis of its Cartan subalgebra). In this paper, alternative arguments confirming these results are presented. In addition, Lorentz invariance is shown to be broken in QCD, just as it is in QED. The experimental consequences of these results include fuzzy mass and spin shells of coloured particles like quarks, and decay life times which depend on the frame of observation \cite{key2,key3,key4}.
\end{quote}

\section{Introduction}\label{chap1-sec1}

