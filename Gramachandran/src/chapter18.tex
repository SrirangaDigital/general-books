\chapter[In memory of Professor G Ramachandran]{In memory of Professor G Ramachandran}\label{chap18}

\Authorline{A. R. Usha Devi}


I cannot believe that Professor G. Ramachandran is no more and I am writing this in his memory! On 9th April 2020 I received a phone call from Professor K. S. Mallesh around 9 PM, informing me that Prof. G Ramachandran passed away after suffering a massive heart attack. A shocking news - Professor G. Ramachandran was strong and energetic till this unexpected end. It is a difficult reality to come to terms with.

Professor G. Ramachandran (fondly referred to as GR by his students) inspired many students by his vast knowledge, sharp intellect and his child like enthusiasm. GR shaped my thinking, my views on life and my work during the years 1988-1998 that I spent in the Department of Studies in Physics, Manasagangotri, University of Mysore, Mysore (as a M.Sc student during 1988-90 and as a Ph.D student under GR’s supervision during 1991-98).

I vividly recall my first interaction with GR during late 1988. GR did not teach any course during our Junior M.Sc. We had heard a lot of appreicating remarks about GR from our senior students. A group of four students, including me and my sister Sudha, wanted to meet GR, though we did not know how to approach him! One fine day we saw GR standing alone outside his chamber. We decided to go and talk to him. I told GR “We are very much interested in theoretical physics and want to get some exposure from you on this.” GR warmly welcomed us into his office and started conversing casually. He began describing (for around 2 hours) Dirac’s relativistic quantum theory (which we had not studied in our first year M.Sc) and went on to explain several subtleties of quantum field theory. For us, it was like “Alice in the wonderland” experience. We were spell-bound during GR’s inspiring narrative. We missed our lunch in the hostel that day - never mind - we decided to take up theoretical physics as our special subject. Unfortunately only six students opted for this special subject at the end of the academic year 1988-89 and we were told that theoretical physics is not going to be offered; we had to alter our choice. It was indeed a helpless situation. But both GR and Professor A. V. Gopal Rao (AVG) encouraged us to continue interacting with them on specific topics of theoretical physics. Sudha and me used to join discussions of GR’s research group, where Professors Mallesh, Swarnamala Sirsi, Sandhya, Sudha Rao used to work out details of their research work on the black board; GR’s sharp remarks, his eye for details made us realize for the first instance what a critical thinking is all about. We did not understand much during these discussions. But we were keen to develop our interest in the research topics being discussed and \textit{see} their intricate subtleties.

GR taught us “Advanced quantum mechanics” during our senior M.Sc year. His relativistic quantum theory classes were remarkable. GR’s grasp of physics awed and inspired me simultaneously. I used to note down every word GR used to utter (including his remarks like \textit{“this was another feather in the cap of Mr. Pauli”} - in the context of explaining Dirac’s \textit{(negative energy sea)} in his class and most of my classmates teased me for that! During my senior M.Sc year I chose the topic “Pion-Nucleon scattering experiments” for my seminar. GR took me through a detailed excursion on this topic. I fondly remember his powerful description laced with wonderful anecdotes and his enormous patience when I raised silly questions. I always wondered how a sequence of relevant thoughts used to emerge so effortlessly when GR spoke. Before the end of my second year M.Sc I asked GR if I can work for my Ph.D under his supervision. GR readily agreed. I got qualified under UGC-CSIR National Level Entrance Examination during 1991 and I joined the Department of Studies in Physics as a Junior Research Fellow.
