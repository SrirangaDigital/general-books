\chapter{A Tribute to my Guru}

\Authorline{Swarnamala Sirsi\footnote[*]{Email: \url{ssirsi@uomphysics.net}}}
\authinfo{Yuvaraja's College, University of Mysore, Mysuru}

Some things are easier said than done - accepting that Professor G. Ramachandran
(G.R.) is no longer with us, is no different. A mentor to me throughout, I cannot thank
him enough for several reasons; the most important of them being, a sense of realization for me that Physics was what I wanted to do for the rest of my life. I owe this to him.

My association with Professor G.R. began as his student in 1980 - it was my dream
to pursue and learn Physics and hence, I made my way to Mysore in 1979 to make that
dream a reality. Although I was not his student until my final year, he was not new to
any of us - his classes were legendary and I had heard how dynamic his lectures were.
Some of my seniors including Malini Kamath and Usha Sathyanarayana revered him,
which I would soon realise myself - he was no ordinary teacher. I had opted for Theoretical Physics as my specialization and the first five minutes into his class, I was floored.
I knew just then that this was how I wanted to learn Physics and to continue doing this
for a long time to come. All the wonderful things I had heard about his teaching and
classes were pale in comparison to my own experience in his class. His sessions were
magical - they seemed surreal. Too good to be true, they said and I could not agree
more. My passion for the subject grew tremendously and every one of his classes only
intensified that. His classes would transport you to a different world - a world full of
possibilities and adventures. The inspiration I drew from him then, continues to this day.

I realised early on that comprehending all of his teachings would not be an easy task
for me. Therefore, I began noting down everything including all anecdotes and trivia he
shared during his lectures. I realise today, as I go through some of my notes, that these
snippets of information and original references he provided are hard to come by.
During my M. Sc, G.R. taught us an entire paper comprising of many important topics
- according to our schedule, this meant 4 hours per week. However, his classes were so
full of life, learning and information that three days a week, his classes went on from 2
PM until 6 PM! It amazed me then and it amazes me today as a teacher myself, the
sheer amount of topics he covered and how! I remember he started with the concept of
Hilbert space and taught Angular momentum theory extensively. I would like to share
a memory from the time - I attended an interview conducted by CSIR in New Delhi for
my Post Doctoral Fellowship. The committee had some of the best people in the field
including Professor V. Balakrishnan, Professor Rohini Godbole and Professor Avinash
Khare. The minute they realised that I was G.R's Ph.D. student, one of them immediately said - “then we will not be asking you about Angular momentum. You would know
it better than us”. I was astonished and honoured to have been under his guidance and
this was such a defining moment for me. Experts in the field did not fail to acknowledge	
his mastery in the field of Angular momentum and we, his students, were also on the
receiving end of this recognition!


After teaching Angular momentum exhaustively, GR took up Dirac equation and then
moved onto scattering theory. He taught us quantization of Dirac, Klein-Gordon,
Maxwell fields and later, processes like pair creation/annihilation, Compton scattering,
Bhabha scattering among others. After QED, he introduced us to strong interaction -
right from n-p scattering to basics of QCD. He then went on to teach us Fermi's theory of
$\beta$-Decay and Lee-Yang's theory of parity violation, introducing electro-weak unification
of Weinberg and Salaam. He taught us the density matrix formalism and its application which involved his research work with MVN Murthy, RS Keshavamurthy and KS
Mallesh.


My attempt at describing G.R. and his imprint on some of us will be an exercise in
futility, however, I shall try. There are not many like him. His immeasurable patience
and willingness to go out of his way to ensure his students had no ambiguity with his
lessons, cannot be compared. His insistence on making the class interactive by encouraging us to question him and his marathon discussions with the class - all unforgettable.
He would insist on strong discussions in the class and outside- these have instilled in
me the imperative need of having open interactions in my sessions today as well. Here’s
an example G.R. himself shared with us one day - G.R. was having a discussion with a
visitor- a famous Physicist at IISC, Bangalore, when a young student asked the visitor a
question about his work. The great man snubbed the student with an evasive reply and
the said student, dejectedly walked away. G.R., unable to digest this, went in search of
the student and clarified his doubts. I have never known him to turn away a student
for want of his time. He was always there for his students. Always. This trait in him,
to go that extra mile and ensuring that no student ever left without learning something
valuable and clarifying doubts without an ounce of judgement, was what drew many
students to him. He was that professor in the university who was constantly surrounded
by students. If the students came up with frivolous questions, he would gently steer
them towards more profound issues. This would be the beginning of a lifelong love affair
with Physics for all his students. Case in point is yours truly.


I was privileged to have experienced the greatness of G.R. - the teacher and the inimitable researcher. 
I was able to witness first hand, the thought process of a brilliant
mind. My journey with Physics continues in his absence by following the path he paved
for us so efficiently and effortlessly.

G.R. had a child like curiosity and enthusiasm which was incredibly infectious! Professor 
K.R. Parthasarathy was invited by the University of Mysore to give a series of
lectures in the Mathematics and Statistics departments. After the lectures, G.R. was
brimming with joy and enthusiasm and told us - “Now I know how to view the spin
density matrix in terms of a statistical distribution!”. That was the beginning of our
sojourn in the field of quasi-probability distributions of spin systems, which eventually
led to my Ph.D. thesis `Theoretical Study of Spin Distributions in external electric and
magnetic fields'. My Ph.D. journey would not have been the fantastic experience that it
was, without him. Those intense and stimulating discussions with Professor G.R. and
Ushadevi are truly unforgettable and later, paved the way for the publication of half a
dozen papers.

An admirable quality of G.R. was his willingness to admit his ignorance in front of
his students. Surely, it takes great humility to have no inhibition in admitting one’s
unawareness. According to G.R, this was a trait that even his guide Professor Alladi
Ramakrishnan possessed. Once, Professor Alladi brought in a research paper which
used natural units ($\hslash = c = 1$) to the classroom openly admitting that he did not quite
understand it. This of course, led his students to work out possible solutions on their
own. G.R. too, adopted this method and I have such fond memories of these exercises!
The fact that he was quite generous with his compliments to the students who solved
the problems, needs to be mentioned.


Everyone is someone's teacher. But not everyone can have the effect G.R. did, on
me and several others too. He was an extraordinarily gifted teacher who had abundant
patience and determination to ensure we understood the most difficult of concepts in
Physics. If we were found wanting, he would not hesitate to begin from the basics and
patiently guide us all the way in order to encourage us to appreciate the physics behind
it. He was our role model in a true sense. Should I stumble upon a complex concept
to explain or teach, it fascinates me to this day that I tend to visualise all the possible
ways Professor G.R.,would have done it.


Most people go through their lives without having a ‘Guru’ - I am proud and honoured to say that I am not one of them. I had a Guru - a bond that I will cherish and
honour, till the end of time. Thank you, Professor G.R.
