\chapter{Dr. G. Ramachandran - Some memories}\label{chap24}

%\Authorline{N.G.Deshpande// Professor of Physics, University of Oregon}

\begin{center}
\textbf{B. Ramachandran}\\
\textbf{\textit{Former Professor, Indian Statistical Institute, Delhi Centre 7, S. J. S. S. Marg, New Delhi 110016}}
\end{center}

I am happy to learn that Dr. Mallesh and other disciples of Dr. G. Ramachandran are bringing out a commemoration volume in celebration of his life
and work. I have had the pleasure and privilege of knowing him since early
1962: over continuous periods in 1962-63 at Chennai, in 1966-69 at Kolkata,
and frequent meetings, (mostly at Bangalore) outside these periods. His abilities and potential came to the attention of my \textit{guru}, Dr. C. R. Rao, and he was
given his first post doctoral assignment in 1964 at Kolkata campus of the Indian
Statistical Institute, (my alma mater and principal place of work. Interestingly,
the year 2020 also marks the birth centenary of Dr. C. R. Rao, celebrated by
the family, friends, admirers and former students of Dr. Rao). At I.S.I, Kolkata,
Dr. G. R. also came to interact during 1964-66 with a small group of young
mathematicians interested in Quantum Probability Theory.

During 1966-69 our careers intersected at I.S.I, Kolkata and I could observe
his total dedication to work, his eagerness to learn and to impart and also wit-
ness peals of his spasmodic laughter. At Mysore and after his formal retirement
from there, he used to run, not a run-of-the-mill University Department, but a
\textit{gurukula}. I used to call him an ``apostle of angular momentum". His response
would be his usual gentle smile.

\eject

In conclusion, let me offer my best wishes to the organisers of this project
for its successful completion and express the hope that it will be followed up
with a full-fledged symposium honouring Dr. G. R., when the times are more
propitious.
