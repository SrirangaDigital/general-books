\chapter[QCD Breaks Lorentz Invariance and Colour]{QCD Breaks Lorentz Invariance and Colour$^{*}$}\label{chap1}

\footnotetext[1]{G.Ramachandran was my teacher and friend for nearly sixty years. He has had an illustrious career training and guiding generations of students and accomplishing novel research. I dedicate this article to his memory.}

\Authorline{A. P. Balachandran\footnote[2]{\url{balachandran38@gmail.com}}}

\begin{center}
\textit{Physics Department, Syracuse University, Syracuse,}\\ 
\textit{New York 13244-1130, U.S.A.}
\end{center}


\section*{Abstract}

\begin{quote}
In a previous work \cite{key1}, we have argued that the algebra of non-abelian superselection rules is spontaneously broken to its maximal abelian subalgebra, that is, the algebra generated by its completing commuting set (the two Casimirs and a basis of its Cartan subalgebra). In this paper, alternative arguments confirming these results are presented. In addition, Lorentz invariance is shown to be broken in QCD, just as it is in QED. The experimental consequences of these results include fuzzy mass and spin shells of coloured particles like quarks, and decay life times which depend on the frame of observation \cite{key2,key3,key4}.
\end{quote}

\section{Introduction}\label{chap1-sec1}

Quantum field theory (QFT) is defined by the algebra $\mathcal{A}$ of local observables and an irreducible representation (IRR) $\pi$ of $\mathcal{A}$ on a Hilbert space $\mathcal{H}$. In general, there are many inequivalent IRR's $\pi_{0}$, $\pi_{1}$,... of $\mathcal{A}$ defining its superselection sectors.

For example, in QED, $\pi_{0}$ can be the sector with total charge $q_{0}= 0$, while $\pi_{n}$ can be the sector with total charge $q_{n}$. No observation can mix these sectors.

In QED, the charge operator $Q$ generates the abelian $U(1)$ group. In QCD, the $U(1)$ is replaced that the non-abelian $SU(3)$ of colour. Superficially, the superselection algebra seems to be the group algebra $\mathbb{C}SU(3)$. We have previously argued \cite{key1} that in reality it is a maximal abelian subalgebra of $\mathbb{CC}SU(3)$ generated by a complete commuting set (CCS). (See in this connection the Kadison-Singer conjecture and its proof \cite{key5}.) We can assume the CCS to be generated by the two Casimir operators and the operators on $\mathcal{H}$ representing the $\lambda_{3}$ and $\lambda_{8}$ of the Gell-Mann matrices. The remaining independent generators of $\mathbb{C}SU(3)$ are spontaneously broken.

There exist elegant proofs of Roepstorff \cite{key6}, Buchholz et. al. \cite{key2} and especially Fr\"ohlich et. al. \cite{key3} that infrared effects in QED break Lorentz invariance. We adapt these arguments here to show that generic elements of $SU(3)$ map an IRR $\pi$ of $\mathcal{A}$ to a distinct IRR $\pi'\neq \pi$ of $\mathcal{A}$. In Higgs theories, this phenomenon is well-known: when the Higgs field $\varphi$ breaks $SO(3)$ to $SO(2)$ say, as in the 't Hooft-Polyakov model \cite{key7}, $SO(3)$ transformations which change the direction of $\varphi$ at spatial infinity cannot be represented as unitary operators on $\mathcal{H}$. In the same way, here, any operator which disturbs the eigenvalues of the CCS is spontaneously broken.

In section \ref{chap1-sec2}, we review the QED result on Lorentz violation. This is then generalised to QCD in sections \ref{chap1-sec3} and \ref{chap1-sec4}.

The results of this paper can be adapted to any non-abelian gauge group.

\section{The Case of QED}\label{chap1-sec2}

QED is classified by a continuous family of superselection sectors.

The first is its classification by the charge $q_{n}$. In the representation $\pi_{n}$ of $\mathcal{A}$, the charge operator $Q$ has the eigenvalue $q_{n}$:
\begin{equation}
\pi_{n}: Q|n,P,\cdot \rangle=q_{n}|n,P,\cdot\rangle\label{chap1-eq2.1}
\end{equation}
where $P = (P^{\mu})$ denotes the total momenta. Local observables cannot change $q_{n}$.

Then, there are the sectors with ``in'' state vectors \cite{key4}
\begin{align}
& e^{q_{n}\int d^{3}x[A^{-}_{i}(x)\omega^{+}_{i}(x)-A^{+}_{i}(x)\omega^{-}_{i}(x)]}|n,P,\cdot\rangle := |n,P,\omega,\cdot\rangle,\label{chap1-eq2.2}\\
& |n,P,0,\cdot\rangle \equiv |n,P,\cdot\rangle,\qquad q_{n}\neq 0\label{chap1-eq2.3}
\end{align}
created by the infrared photons. Here $A^{\pi}_{i}$ are the positive and negative frequency parts of the electromagnetic potential in the Coulomb gauge, and the functions $\omega^{+}_{i}$, $\omega^{-}_{i}=\overline{\omega}^{+}_{i}$ are transverse:
\begin{equation}
\partial_{i}\omega^{\pi}_{i}(x)=0,\label{chap1-eq2.4}
\end{equation}

Also they do not vanish fast as we approach infinity:
\begin{equation}
\lim\limits_{r\to\infty}r^{2}\hat{x}_{i}\omega_{i}(x)^{\pm}\neq 0.\label{chap1-eq2.5}
\end{equation}

One such typical $\omega^{+}_{i}$ has the Fourier transform
\begin{equation}
\hat{\omega}_{i}^{+}(k)=\int d^{3}xe^{i\overrightarrow{k}\cdot \overrightarrow{x}}\omega^{+}_{i}(x)=\frac{1}{P\cdot k_{i}+i_{\epsilon}}(P_{i}-\overrightarrow{P}\cdot \widehat{k}\widehat{k}_{i})\label{chap1-eq2.6}
\end{equation}
(with $\epsilon$ decreasing to zero as usual). The momentum $P_{\mu}$ is the total momentum of the charged system. (We have not shown the individual momenta and charges of which $P$ and $q_{n}$ are composed as they are not importatnt for our considerations.) The important point here is that $\widehat{\omega}^{+}_{i}$ is not square-integrable:
\begin{equation}
\langle \omega,\omega\rangle := \lim\limits_{|\overrightarrow{k}|\to 0}\int^{\infty}_{|\overrightarrow{k}|}\frac{d^{3}k}{2|\overrightarrow{k}|}|\widehat{\omega}_{i}(k)|^{2}=\infty.\label{chap1-eq2.7}
\end{equation}
It is then a theorem \cite{key6} that the representation of $\mathcal{A}$ built on \eqref{chap1-eq2.2} is superselected: it is not the Fock space representation.

The appendix gives a derivation of \eqref{chap1-eq2.2}.

We now elaborate on the physical meaning of \eqref{chap1-eq2.2}. Consider the current
\begin{equation}
J^{\mu}(x)=q_{n}\int d\tau \delta^{4}(x-z(\tau))\frac{dz^{\mu}(\tau)}{d\tau}\label{chap1-eq2.8}
\end{equation}
It radiates photons of momenta $k := (| \overrightarrow{k}|, \overrightarrow{k})$. We are interested in infrared photons, so we assume that
\begin{equation}
\frac{dz^{\mu}(\tau)}{d\tau}=\dfrac{P^{\mu}}{m},\quad z^{\mu}(\tau)=\tau\frac{P^{\mu}}{m},\quad m^{2}=P^{\mu}P_{\mu},\quad m>0,\label{chap1-eq2.9}
\end{equation}
where $P^{\mu}$ is constant.

Now \eqref{chap1-eq2.8} generates the additional interaction
\begin{equation}
\int d^{3}xA_{\mu}(x)J^{\mu}(x).\label{chap1-eq2.10}
\end{equation}
It changes the ``in'' state to $|n,\omega\ldots\rangle$ as in shown in the appendix.

The following is a further important point. If the Lorentz boost
\begin{equation}
K_{i}=\int d^{3}xx_{i}[\overrightarrow{E}^{2}(x)+\overrightarrow{B}^{2}(x)]+\text{matter part}\label{chap1-eq2.11}
\end{equation}
is well-defined in $|n, 0;\cdot\rangle$, then it diverges in the sector $|n,\omega;\cdot \rangle$, $\omega\neq 0$ : Lorentz invariance is broken in the latter. We can see this as follows:
\begin{align}
& \rangle n,\omega;\cdot | \int d^{3}xx_{i}[\overrightarrow{E}^{2}(x)+\overrightarrow{B}^{2}(x)]|n,\omega;\cdot\rangle\notag\\
& =\langle n,0;\cdot |\int d^{3}xx_{i}[(\overrightarrow{E}-i\overrightarrow{\omega})^{2}(x)+\overrightarrow{B}^{2}(x)]|n,0;\cdot\rangle,\notag\\
&\omega_{i} := \omega^{+}_{i}+\omega^{-}_{i}\label{chap1-eq2.12}
\end{align}
and this diverges since $\overrightarrow{\omega}^{2}(x)=\mathcal{O}(\frac{1}{|\overrightarrow{x}|^{4}})$ as $|\overrightarrow{x}|\to \infty$.

There is an alternative approach for these considerations due to Roepstorff \cite{key6}. It is based on the Weyl algebra and the GNS construction. For free scalar fields, in four-dimensional spacetime, the Weyl algebra $\mathcal{W}$ has elements $W(f)$ where $f$ is a compactly supported real test function, $f\in \mathcal{C}^{\infty}_{0}$. If $g$ is another such test function,
\begin{align}
& W(f)W(g)=W(f+g)e^{i\sigma(f,g)/2},\label{chap1-eq2.13}\\
& \sigma (f,g)=i \int d^{4}xd^{4}yf(x)D(x-y)g(y),\label{chap1-eq2.14}\\
& D(x-y)=\text{commutator function} = \int \dfrac{d^{3}P}{(2\pi)^{3}}\dfrac{1}{2|P_{0}|}[e^{-iP\cdot (x-y)}-e^{iP\cdot (x-y)}].\label{chap1-eq2.15}
\end{align}
Since $(\Box + m^2)D = 0$ if the field has mass $m$, $\sigma$ vanishes on any function $f$ of the form $(\Box + m^2)h$, with $h\in \mathcal{C}^{\infty}_{0}$. On quotenting out such functions, $\sigma$ becomes a symplectic form.

Also, the function $\sigma (f,\cdot)$ defined by
\begin{equation}
\sigma(f,\cdot)(y)=i\int d^{4}xf(x)D(x-y)\label{chap1-eq2.16}
\end{equation}
fulfills the equations of motion.

Let us introduce the scalar product
$$
(f,g)=\frac{1}{(2\pi)^{3}}\int \dfrac{d^{3}k}{2|k_{0}|}\overline{\widetilde{f}(k)}\widetilde{g}(k),\qquad |k_{0}|=(\overrightarrow{k}^{2}+m^{2})^{1/2},
$$
where
\begin{equation}
\widetilde{f}(k)=\int d^{4}xe^{-ik\cdot x}f(x),\qquad \widetilde{g}(k)=\int d^{4}xe^{-ik\cdot x}g(x).\label{chap1-eq2.17}
\end{equation}
Then the Fock representation with the corresponding Hilbert space $\mathcal{H}$ is given by the following state $\omega_{0}$ on the Weyl algebra and the GNS construction:
\begin{equation}
\omega_{0}(W(f))=e^{-(f,f)/2}.\label{chap1-eq2.18}
\end{equation}
If $|0\rangle$ is the Fock vacuum, we have, as can be checked,
\begin{equation}
\omega_{0}(W(f))=\langle 0|W (f)|0\rangle.\label{chap1-eq2.19}
\end{equation}

Now if $F$ is a linear functional on test functions, we can twist $\omega_{0}$ to a new state $\omega_{F}$:
\begin{equation}
\omega_{F}(W(f))=\omega_{0}(W(f))e^{i \Iim F(f)}.\label{chap1-eq2.20} 
\end{equation}
Suppose we can write
\begin{equation}
F(f) = \langle\eta, f\rangle, \qquad \eta \in \mathcal{H} \label{chap1-eq2.21}
\end{equation}
Then, by Schwarz inequality,
\begin{equation}
|| F(f) || \leq ||\eta || ~ || f ||.\label{chap1-eq2.22}
\end{equation}
Conversely, by the Riesez-Frechet theorem \cite{key9}, we can write $F(f)= \langle \eta, f\rangle$ for $\eta \in \mathcal{H}$ iff
\begin{equation}
|| F(f) || < c || f||, \qquad c= \text{constant}. \label{chap1-eq2.23}
\end{equation}

Now if there is such an $\eta$, we can check that
\begin{equation}
\omega_F (W (f))= \langle 0| W (\Iim \eta)^\dagger W (f)W (\Iim \eta)|\rangle. \label{chap1-eq2-24}
\end{equation}
Since $W(\Iim \eta | 0 \rangle)$ is in the Fock space, the GNS representation from $\omega_F$ is unitarily equivalent to the one from $\omega_0$.

Any smooth $\xi \in \mathcal{C}^\infty$ which is not in $\mathcal{H}$ also gives an $F$:
\begin{equation}
  F(f)= (\xi, f), \label{chap1-eq2.25}
\end{equation}
since $f$ is a test function and hence compactly supported. In this case,
\begin{equation}
\omega_F (W(f))= \langle 0| W (\Iim \xi)^\dagger W (f)W (\Iim \xi)|0\rangle, \label{chap1-eq2.26}
\end{equation}
but $W (\Iim \xi)|0 \rangle$ is not in the fock space. So the representation of $\mathcal{W}$ built on $W (\Iim \xi)|0 \rangle$ is not unitarily equivalent to the Fock representation.

Thise considerations can be adapted to QED. Let $\alpha_\mu$ be the test function for the potential in the Lorentz gauge $\partial^\mu \alpha_\mu =0$. If
\begin{equation}
f_{\mu \nu} (\alpha) = \partial_{\mu} \alpha_{\nu}- \partial_\nu \alpha_\mu, \label{chap1-eq-2.27}
\end{equation}
the symplectic form (modulo the kernel) is $\sigma:$
\begin{equation}
  \sigma (\alpha_1, \alpha_2) = i \int d^4 x d^4 y \alpha^\mu_1 (x) D (x-y) \alpha_{2 \mu} (y). \label{chap1-eq2.28}
\end{equation}
It depends only on $f_{i, \mu \nu}= \partial_\mu \alpha_{i \nu}- \partial_\nu \alpha_{i, \mu}$, as it is gauge invariant: $\sigma(\alpha_1, \alpha_2)$ remains invariant under $\alpha_\mu \to \alpha_\mu + \partial_{\mu} \eta$. Thus, the Weyl algebra is defined by
\begin{equation}
W (\alpha_1) W (\alpha_2)= W ((\alpha_1 + \alpha_2)) e^{\frac{i}{2} \sigma(\alpha_1, \alpha_2)}. \label{chap1-eq2.29}
\end{equation}
With
\begin{align}
  &\tilde{\alpha_\mu} (k) = \int d^4 x \alpha_\mu (x)e^{ik\cdot x}, \qquad k_0 = |\overrightarrow{k}|,\label{chap1-eq2.30}\\
  &\Longrightarrow k^{\mu} \tilde{\alpha_\mu} (k)=0,\label{chap1-eq2.31}
\end{align}
We introduce the scalar product
\begin{equation}
(\alpha_1, \alpha_2)= \frac{1}{(2\pi)^3} \int \frac{d^3 k}{2|\overrightarrow{k}|} \sum^3_{i=1} \left(\overline{\tilde{\alpha}}_{1, i} -  \frac{k_i}{k_0} \overline{\tilde{\alpha}}_{1, 0}\right) (\tilde{\alpha}_{2, i}- \frac{k_i}{k_0} \tilde{\alpha}_{2, 0})(k) \label{chap1-eq2.32}
\end{equation}
We can then adapt the scalar field considerations to the electromagnetic field.

The choice
\begin{equation}
F(\alpha) = (\omega, \alpha) \label{chap1-eq2.33}
\end{equation}
when $\omega$ is given by \eqref{chap1-eq2.6} leada to the vertex operator in \eqref{chap1-eq2.2}.

We note that
\begin{equation}
F(\alpha) = \int dx dy {J}^\mu (x) {D}(x-y) \alpha_\mu (y) \label{chap1-eq2.34}
\end{equation}
if the $\tau$ integration in ${J}_\mu$ is restricted to $\tau \leq 0$ and the Lorentz gauge is changed to the Coulomb gauge. See Appendix.

\section{The Case of QCD}\label{chap1-sec3}

By the axioms of QFT \cite{key9}, observables are local. They generate the algebra $\mathcal{A}$ of local observables.

The group $SU(3)$ of QCD commutes with all elements of $\mathcal{A}$, just as the $U(1)$ of charge commutes with all elements of  $\mathcal{A}$. But $SU(3)$ unlike $U(1)$ is non-abelian. For this reason, we argued \cite{key1, key4} that $SU(3)$ is spontaneously broken.

Our arguments were as follows. If $\mathbb{C}SU(3)$ is the group algebra of $SU(3)$, $\mathbb{C}SU(3)$ also commutes with $\mathcal{A}$. A maximal abelian subalgebra of $\mathbb{C}SU(3)$ is spanned by the complete commuting set (CCS) $c_2, c_3, \hat{\lambda}_3 \hat{\lambda}_8$ where $c_2, c_3$ are the quadratic and cubic Casimir operators and $\hat{\lambda}_3 \hat{\lambda}_8$ are the operators which represent the $\hat{\lambda}_3 \hat{\lambda}_8$ of the Gell-Mann matrices on the Hilbert space. In any IRR of $\mathcal{A}$, we can diagonalise the CCS. The basis vectors in this IRR can be written as
\begin{equation}
  |c_2, c_3, i_3, y;\cdot \rangle,\label{chap1-eq3.1}
\end{equation}
where $c_2, c_3, i_3, y$ are the eigenvalues of $c_2, c_3, \hat{\lambda}_3 \hat{\lambda}_8$. No observable can affect them. They label a superselection sector.

But in a coloured representation $(c_2, c_3 ~\text{or both}~\neq 0)$, a generic $SU(3)$ transformation will change $i_3, y$. That is, it will change the superselection sector. Hence, such an $SU(3)$ transformation is spontaneously broken.

We have analysed a similar situation which occurs for the ethylene molecule \cite{key10}. The gauge group there is $D^\ast_8$, the binary dihedrall group isomorphic to the group
\begin{equation}
\left\{ \pm \mathbb{I}_{2 \times 2}, \pm i \tau_j :\quad j=1, 2, 3, \quad \tau_j = \text{Pauli matrices}\right\}.\label{chap1-eq3.2}
\end{equation}
This is non-abelian. Its two-dimensional representation is relevant for certain confirmations of the molecule. We proved explicitly that its maximal abelian subgroup, say
\begin{equation}
H = \left\{ \pm \mathbb{I}_{2 \times 2}, \pm  i \tau_3\right\}\label{chap1-eq3.3}
\end{equation}
is diagonal in an IRR of the molecule: no observable changes the eigenvalues of elements of $H$.

A consequence of the above observation is that coloured states, such as
\begin{equation}
|c_2, c_3, i_3, y; \cdot \rangle ~ \langle c_2, c_3, i_3, y; \cdot | \label{chap1-eq3.4}
\end{equation}
for $c_2, c_3$ or both $\neq 0$, are \textit{mixed} \cite{key4}. That has consequence for colour confinement \cite{key4}.

In what follows, we give another argument to show that the $SU(3)$ of colour is spontaneously broken. It uses the infrared cloud in QCD which dresses the coloured particle. An extra result we find is that Lorentz invariance is broken in QCD too.

We approach this problem as in the QED case. The charged particle in QED gets dressed by the infrared radiation and becomes the in and out state. The latter is not in the Fock space and breaks Lorentz invariance.

In a similar manner, we can expect novel physical consequences from the gluon radiation of Yang-Mills particles such as quarks.

The dynamics of Yang-Mills gluous is non-linear. Hence, gluous can also radiate gluons. We do not treat this radiation. Perhaps by using a null four vector for $dz_\mu (\tau)/\tau$ in what follows we can get an adequate description of infra-gluon radiation buy gluons.

Nor do we treat effects of confinement.

The electric charge $e$ is replaced by an ``internal'' vector of operators $\hat{\lambda}= (\hat{\lambda}_\alpha, \alpha = 1, 2, \ldots , 8)$ for a Yang-Mills or Wong \cite{key11} particle. (We absorb the Yand-Mills coupling constant in the gluon field.) The operators $\hat{\lambda}$ are $SU(3)$ Lie algebra valued with the standard commutators
\begin{equation}
\left[ \hat{\lambda}_\alpha \hat{\lambda}_\beta \right] = if_{\alpha \beta} {}^\gamma \hat{\lambda}_\gamma. \label{chap1-eq3.5}
\end{equation}
Their representation on the quantum Hilbert space identifies the particle as a quark, a di-quark, etc.

The quantum Yang-Mills current is
\begin{equation}
J_\mu^\alpha (x) = \int d \tau \delta^4 (x -z) (\tau) \hat{\lambda} {}^\alpha (\tau) \frac{dz_\mu (\tau)}{d \tau}. \label{chap1-eq3.6}
\end{equation}
The vector $\hat{\lambda} (\tau)$ can change with $\tau$, fulfilling \eqref{chap1-eq3.5}.

There is a consistency condition on $J_\mu$: it must be covariantly constant. Thus the field equation gives
\begin{equation}
D^\mu F_{\mu \nu} = J_\nu, \qquad F= \text{curvature of the gluon connection}~ A. \label{chap1-eq3.7}
\end{equation}
where
\begin{equation}
D^\mu D^\nu F_{\mu \nu}=0 \label{chap1-eq3.8}
\end{equation}
as an indentity. Hence,
\begin{equation}
D^\mu J_\mu =0 \label{chap1-eq3.9}
\end{equation}
or
\begin{equation}
D_\tau \hat{\lambda} (\tau) \equiv \partial_\tau \hat{\lambda} (\tau) + i \frac{dz^\mu(\tau)}{d \tau} \left[A_\mu (z(\tau)), \hat{\lambda}(\tau)\right] =0. \label{chap1-eq3.10}
\end{equation}
Thus if
\begin{align}
  & g(\tau) = P \exp \left(-i \int^\tau_0 d \tau' \frac{dz^\mu(\tau')}{d \tau'} A_\mu (z(\tau')) \right), \label{chap1-eq3.11}\\
  & \Longrightarrow [\partial_\tau +i A_\tau (z(\tau))] g (\tau)=0, \label{chap1-eq3.12}
\end{align}
then
\begin{equation}
\hat{\lambda} (\tau) = g (\tau) \hat{\lambda} (0) g (\tau)^{-1}. \label{chap1-eq3.13}
\end{equation}
This equation determines the evolution of $\hat{\lambda}$ with $\tau$. Since $g(\tau) \in SU(3)$, this is by a gauge transformation.

Because of \eqref{chap1-eq3.13}, the $\tau$-evolution of $J$ does not mix different IRR's of $\hat{\lambda}$ and commutes with the Casimir invariants. We therefore assume that $\hat{\lambda}$ belongs to a fixed IRR of \underline{$SU(3)$} (the underline denoting Lie algebra).

The Wong equations have a Lagrangian description found by us long ago \cite{key12}.

\subsection{The Gauge Choice}\label{chap1-sec3.1}

Let $K$ be a fixed element of \underline{$SU(3)$} which is in the orbit of $\hat{\lambda}(0)$. Then we can find $h (\tau) \in SU(3)$ such that
\begin{equation}
  K= h (\tau) \hat{\lambda} (\tau) h(\tau)^{-1}.\label{chap1-eq3.14}
\end{equation}
That is because the $\tau$-evolution \eqref{chap1-eq3.13} is by gauge transformation.

Let us assume that $K$ is a ``generic'' element of \underline{$SU(3)$}, so that its stability group $[U(1) \times U (1)]$ is abelian. (We comment on-abelian stability groups below.) In that case,
\begin{equation}
A_\tau [z(\tau)]: = \frac{dz^\mu(\tau)}{d\tau} A_\mu [z(\tau)] \in \underline{U(1) \times U(1)}.\label{chap1-eq3.15}
\end{equation}
That is because $A_\tau [z(\tau)]$ commutes with $K$ by \eqref{chap1-eq3.10}.

The current is now abelian, in the direction $K$:
\begin{equation}
J^\mu (x) = K \int d \tau \delta^4 [x-z(\tau)] \frac{dz^\mu(\tau)}{d\tau}. \label{chap1-eq3.16}
\end{equation}

We must now solve \eqref{chap1-eq3.7} for the emitted radiation at low frequencies by this current. That can be done as in Peshkin and Schr\"oeder \cite{key8}, by setting
\begin{equation}
A_\mu = a_\mu K + b_\mu L, \label{chap1-eq3.17}
\end{equation}
where $Tr (KL)=0, K, L$ are the basis of the two \underline{$U(1)$}'s and commute, and $a_\mu(x)$, $b_\mu(x)\in \mathbb{R}$. Then, from \eqref{chap1-eq3.7},
\begin{equation}
\partial^\mu (\partial_\mu a \nu - \partial_\nu a_\mu) K = J^\mu, \label{chap1-eq3.18}
\end{equation}
whereas $b$ has no source.

This equation for $a$ can be treated exactly in QCD. the particle state vectors have eigenvalues $k_i$ of $K$. so the radiation is $k_i \omega_\mu$ for a fixed $k_j$ and its dressed in state is
\begin{equation}
|\text{in}, \omega, \ldots \rangle = e^{kj \int d^3 x (Tr\lambda_K[A_i^- (x) \omega_i^+ (x)- a_i^+ (x) \omega_i^-(x)])} | P; k_j \rangle |0\rangle_\gamma, \label{chap1-eq3.19} 
\end{equation}
where $\lambda_K = K / \sqrt{TrK^2}, |0\rangle_\gamma$ is the gluon Fock vacuum and $|P; k_j \rangle$ is the particle vacuum with eigenvalue $k_j$ for $K$.

In \cite{key4}, we have shown how the twist of the in state vector in \eqref{chap1-eq3.19} can be absorbed as an instanton-like term in the action.

\subsection{Lorentz Invariance}\label{chap1-sec3.2}

Just as in QED, Lorentz invariance is spontaneously broken in QCD. The proof follows from \eqref{chap1-eq3.19} as in QED.

\section{QCD Breaks Colour}\label{chap1-sec4}

The considerations leading to the gauge choice $K$ imply that the test functions $\alpha$ are proportional to $K: \alpha = K \beta$, where $\beta$ are real-valued. The linear functional $F$ in \eqref{chap1-eq2.34} now involves a trace over the Lie algebra elements. Correspondingly, there is now a trace in \eqref{chap1-eq2.34}. Also, the electric charge $q_n$ is replaced by $k_j$.

Now $K$ has in general manuy eigenvalues and we can transform one eigenvalue $k_j$ to another, $k_l$, by conjugating $K$ by a suitable $SU(3)$ element. Consider the case where $k_j \neq k_l$. The change in $F(\alpha)$ is then by
\begin{equation}
(\zeta, \alpha) \label{chap1-eq4.1}
\end{equation}
where
\begin{equation}
  \zeta^\mu = (k_l - k_j) \int_{\tau \leq 0} d^4 y d\tau \delta^4 (y -z(\tau)) \frac{dz^\mu}{d \tau} D (y-x). \label{chap1-eq4.2}
\end{equation}
But $\zeta$ is not square integrable and $(\zeta, f)$ does not fulfill the bound $|(\zeta, f)|\leq c || f ||$, with $c=$ constant. Thus by the theorem in \cite{key6}, this transformation cannot be unitarily implemented. Thus the gauge transformation on $K$ which changes its eigenvalue from $k_j$ to $k_l$ cannot be unitarily implemented. That is, $SU(3)$ is spontaneously broken.

The change from $k_j$ to $k_l \neq k_j$ is very much like changing the charge in QED from $q_n$ to $q_m \neq q_n$. We see that just as in the QED case, such a change is not unitarily implementable.

\textit{Remarks on phenomenogy:} After the gauge choice which leads to \eqref{chap1-eq3.19}, the vertex operator there is as in QED \cite{key4}, with $k_j$ replacing the total electric charge. Therefore, we can twist the mass term of coloured particles as in that paper. In QCD as well, that explicitly realises the known QED results \cite{key2, key3} about the mass and spin shells of particles not being sharp. It also gives simple models for phenomenolgy and shows, for example, that life times and decays of coloured particles are frame dependent.

\subsection{The Case where $K$ is Non-Generic}\label{chap1-sec4.1}

In this case, $K$ is parallel to $\hat{\lambda}_8$ and its stability group is non-abelian, being $s(U(2) \times U(1))$, which is isomorphic to $U(2)$. Its Lie algebra is the direct sum of those of $SU(2)$ and $U(1)$, where the latter has $\lambda_8$ as a basis. Also, \underline{$U(1)$} commutes with \underline{$SU(2)$}. The generators of \underline{$SU(2)$} and \underline{$U(1)$} are trace orthogonal while the current $J_\mu$ has only the \underline{$U(1)$} or $\lambda_8$ component. Thus the current is non-zero only for the abelian component $\lambda_8$ of $A_\tau(z (\tau))$. The \underline{$SU(1)$} part of $A_\tau (z(\tau))$ decouples from this \underline{$U(1)$} part. So the \underline{$U(1)$} part can be treated as before, with similar conclusions.

\noindent \textbf{\large Acknowledgements}
\medskip

The original claim that non-abelian superselection rules are spontaneously broken is contained in the paper with Vaidya \cite{key1}. In papers with Queiroz and Baidya \cite{key10, key13}, The related result was shown that coloured states are mixed and hence cannot be produced from colour singlets. The work on Lorentz symmetry breaking was done with K\"urk\c{c}\"{u}o\u{g}lu, Queiroz and Vaidya \cite{key4}. I am grateful to all these colleagues and to Kumar Gupta for valuable inputs. I bery much appreciate the invaluable help I received from Nirmalendu Acharyya and Ver\'onica Errasti D\'iez in preparing the paper. Vero was especially helpful, carefully reading the paper and suggesting improvements. I also thank Andr\'es Reyes at the Universidad de Los Andes for hosting me as the Sanford Professor and for warm hospitality while this work was being completed.

\medskip
\noindent \textbf{\large A Apprndix: Derivation of \eqref{chap1-eq2.2}}
\medskip

This appendix gives a derivation of equation \eqref{chap1-eq2.2}. It is not new and its variations may be found in several papers (cf. \cite{key3, key6, key14}). We describe it here to help the reader.

Equation \eqref{chap1-eq2.2} shows the in-state of the charged particle of momentum $P$ and charge $q_n$ after it is dressed by the cloud of infrared photons. If there are several particles with total momentum $P$ and charge $q_n$, we can approximate them by a single particle of momentum $P$ and charge $q_n$ as explained by previous works.

We start with non-interacting charges and photons, the latter being in their vacuum state:
\begin{align*}
  |n, P, \cdots \rangle \otimes | 0 \rangle_\gamma. \label{chap1-eqA.1}\tag{A.1}
\end{align*}
Here $|0 \rangle_\gamma$ is the photon vacuum and $n$ as before denotes charge $q_n$. Let $H_0$ be the Hamiltonian associated with \eqref{chap1-eqA.1}

The current $j^\mu$ of the charged particle can be approximated by that of a particle moving with charge $q_n$ and constant velocity $\dfrac{P^\mu}{(P\cdot P)^{\frac{1}{2}}}: = \dfrac{P^\mu}{M}$. It radiates photons of various momenta, but since we are interested in the infrared limit where they approach zero, we van assume that $\dfrac{P^\mu}{M}$ is constant.

The current $J^\mu$ is thus
\begin{align*}
  J^\mu (x) & = q_n \int^\infty_{-\infty} d \tau \delta^4 (x- z (\tau)) \frac{dz^\mu(\tau)}{d\tau}\\
  & = \infty^\infty_{-\infty} d \tau \delta^4 \left(x- \tau \frac{P}{M}\right) \frac{P^\mu}{M}\label{chap1-eqA.2}\tag{A.2}
\end{align*}
where for the charged particle tragectory $z(\tau)$, we assume that
\begin{equation*}
z^\mu (\tau) = \tau \frac{P^\mu}{M}. \label{chap1-eqA.3}\tag{A.3}
\end{equation*}

The interaction of $A_\mu$ with $J^\mu$ is described by the interaction Hamiltonian
\begin{align*}
  H_I(t) & = q_n \int d^3 x a_\mu (x) \int d \tau \frac{P^\mu}{M} \delta^4 \left( x- \tau \frac{P}{M} \right) \label{chap1-eqA.4}\tag{A.4}\\
  & = q_n \left( \frac{P^\mu}{P_0}\right) A_\mu \left( x_0 \frac{P}{P^0}\right), \quad t= x_0 \label{chap1-eqA.5}\tag{A.5}
\end{align*}

The in-state associated with \eqref{chap1-eqA.1} is obtained by applying the M\o{}ller operator
\begin{equation*}
\Omega = \lim\limits_{t \to -\infty} e^{iHt}e^{-iH_0t}\equiv \lim\limits_{t \to -\infty} U(t)\label{chap1-eqA.6}\tag{A.6}
\end{equation*}
to \eqref{chap1-eqA.1}. It is \eqref{chap1-eq2.2} where $\omega$ will make its appearance below. Thus
\begin{align*}
  | n, P, \omega, \cdots \rangle & = \Omega | n, P, \cdots \rangle \otimes | 0 \rangle_\gamma\\
  & \equiv \Omega|n, P, 0, \cdots \rangle \otimes |0 \rangle_\gamma \label{chap1-eqA.7}\tag{A.7}
\end{align*}

Now
\begin{align*}
  i \frac{dU(t)}{dt} & = -U (t) \hat{H}_I(t)\\
  \hat{H}_I (t) & = e^{iH_0t} H_I (t) e^{-iH_0t}, \label{chap1-eqA.8}\tag{A.8}
\end{align*}
$\hat{H}_I(t)$ being the interaction Hamiltonian in the interaction representation.

Thus,
\begin{equation*}
U(t)= T\exp \left[-i \int^0_t dt' \hat{H}_I (t') \right]\label{chap1-eqA.9}\tag{A.9}
\end{equation*}
If $\hat{A}_\mu$ is the vector potential in interaction representation, we can also write \eqref{chap1-eqA.9} as a $T$-ordered Dirac-Wilson line integral
\begin{equation*}
U(t) = T \exp \left[-i \int^0_t dz^\mu (x_0) A_\mu  \left( x_0 \frac{P}{M}\right)\right]. \label{chap1-eqA.10}\tag{A.10}
\end{equation*}

Since
\begin{equation*}
  [[\hat{H}_I (t), \hat{H}_I(t')], \hat{H}_I (t'')]=0 \label{chap1-eqA.11}\tag{A.11}
\end{equation*}
it is known that \cite{key14}
\begin{equation*}
U(t) = \exp \left[ -i \int^0_t dt' \hat{H}_I (t')\right] \times u(t) \label{chap1-eqA.12}\tag{A.12}
\end{equation*}
where $u(t)$ is a (time- and $P$-dependent) phase commuting with all operators of the electromagnetic field. It does not affect us. Hence we can drop it form $\Omega$ and write
\begin{align*}
  \Omega & = \exp \left[-i \int^0_{-\infty} dt' \hat{H}_I (t') \right] \label{chap1-eqA.13}\tag{A.13}\\
  & = \exp \left[ -iq_n \int^0_{-\infty} dz^\mu (x_0) a_\mu \left( x_0 \frac{P}{M}\right)\right] \label{chap1-eqA.14}\tag{A.14}
\end{align*}

Let
\begin{equation*}
  A_\mu^+ (x) = \int \frac{d^3k}{\sqrt{2k}} \frac{1}{(2\pi)^3}a_\mu^+ (k)e^{ik\cdot x}, \qquad k \cdot x = k_0 x_0- \vec{k} \cdot \vec{x} \label{chap1-eqA.15}\tag{A.15}
\end{equation*}
denote the positive frequency part of $A_\mu(x)$ and let
\begin{equation*}
A_\mu^- (x) = (a_\mu^+ (x))^\dagger, \qquad a_\mu^- (k) = (a_\mu^+ (k))^\dagger. \label{chap1-eqA.16}\tag{A.16}
\end{equation*}

Then
\begin{align*}
  q_n \int^0_{-\infty} dx_0 \frac{P^\mu}{P^0} A_\mu \left( \frac{P}{P^0}\right) & = -i \int \frac{d^3k}{\sqrt{2k_0}} \frac{1}{(2 \pi)^3} a_\mu^+ (k) \tilde{\omega}^{\mu -} (k)\\
  & \qquad + i \int \frac{d^3k}{\sqrt{2k_0}} \frac{1}{(2 \pi)^3} a_\mu^- (k) \tilde{\omega}^{\mu +}(k), \label{chap1-eqA.17}\tag{A.17}\\
  \tilde{\omega}^{\mu +} (k) & = \frac{q_n P^\mu}{k \cdot P + i\epsilon} = \bar{\tilde{\omega}}^{\mu +} (k) \label{chap1-eqA.18}\tag{A.18}
\end{align*}
where $\epsilon$ as usual decreases to zero when all calculations are done.

Thus we get
\begin{equation*}
\Omega = \exp - \left\{ \int \frac{d^3 k}{\sqrt{2k_0}} \frac{1}{(2 \pi)^3} \left[a_\mu^+ (k) \tilde{\omega}^{\mu -} (k) - a_\mu^- (k) \tilde{\omega}^{\mu + (k)} \right]\right\}. \label{chap1-eqA.19}\tag{A.19}
\end{equation*}

The gauge choice in \eqref{chap1-eqA.19} is Lorentz: $k^\mu a_\mu^+ (k) =0$. so we can replace $a_\mu^{\pm} (k)$ by
\begin{equation*}
{a_\mu^{\pm}}' (k) = a_\mu^{\pm} (k) - \frac{k_\mu}{k_0} a_0^\pm (k) \label{chap1-eqA.20}\tag{A.20}
\end{equation*}
which is in the Coulomb gauge:
\begin{equation*}
  {a_0^{+}}' (k) =0, \qquad k_i {a_i^\pm} ' (k) =0. \label{chap1-eqA.21}\tag{A.21}
\end{equation*}

Hence
\begin{equation*}
a_\mu^{\mp} (k) \tilde{\omega}^{\mu \pm} \to {a_i^{\mp}}' (k) \tilde{\omega}^{i \pm}. \label{chap1-eqA.22}\tag{A.22}
\end{equation*}
Since ${a_i^{\pm}}'$ are transverse by \eqref{chap1-eqA.21}, we can now change $\tilde{\omega}^{i \pm }$ to
\begin{equation*}
\hat{\omega}^{i \pm} (k) = \tilde{\omega}^{i \pm} (k) - k^i \hat{k}^i \cdot \tilde{\omega}^{i \pm}(k). \label{chap1-eqA.23}\tag{A.23}
\end{equation*}

On fourier transforming
\begin{equation*}
\int \frac{d^3k}{\sqrt{2k_0}} \frac{1}{(2 \pi)^3} \left[ {a_\mu^-}' (k) \hat{\omega}^{\mu +} (k) - {a_\mu^+}' (k) \hat{\omega}^{\mu -} (k)\right]' (k) e^{ik \cdot x} \label{chap1-eqA.24}\tag{A.24}
\end{equation*}
and calling
\begin{equation*}
\int \frac{d^3k}{\sqrt{2k_0}} \frac{1}{(2 \pi)^3} \left[ {a_\mu^+}' (k) e^{ik \cdot x} + {a_\mu^-}' (k) e^{-i k \cdot x}\right] \label{chap1-eqA.25}\tag{A.25}
\end{equation*}
as $A_i (x)$ (with no prime), we get \eqref{chap1-eq2.2}. This $A_i$ is in Coulomb gauge.


\begin{thebibliography}{99}
\bibitem{key1} A. P. Balachandran and S. Vaidya, \textit{Spontaneous Lorentz Violation in Gauge Theories}, Eur. Phys. J. Plus \textbf{128}, 118 (2013) [arXiv:1302.3406 [hep-th]]. 
\bibitem{key2} D. Buchholz, \textit{Gauss' Law and the Infraparticle Problem}, Phys. Lett. B \textbf{174}, 331 (1986); D. Buchholz and K. Fredenhagen, \textit{Locality and the Structure of Particle States}, Commun. Math. Phys. \textbf{84}, 1 (1982). 
\bibitem{key3} J. Fr\"ohlich, G. Morchio, and F. Strocchi, \textit{Infrared problem and spontaneous breaking of the lorentz group in qed}, Physics Letters B \textbf{89}, 1, 61--64 (1979).
\bibitem{key4} A. P. Balachandran, S. K\"urk\c{c}\"uo\u{g}lu, A. R. de Queiroz and S. Vaidya, \textit{Spontaneous Lorentz Violation: The Case of Infrared QED}, Eur. Phys. J. C \textbf{75}, 2, 89 (2015) [arXiv:1406.5845 [hep-th]].

For earlier work especially in $2+1$ dimensions, see A. P. Balachandran, S. K\"urk\c{c}\"uo\u{g}lu and A. R. de Queiroz, \textit{Spontaneous Breaking of Lorentz Symmetry and Vertex Operators for Vortices}, Mod. Phys. Lett. A \textbf{28}, 1350028 (2013) [arXiv:1208.3175 [hep-th]].

\bibitem{key5} N. J. A. Harvey, \textit{An introduction to the Kadison-Singer Problem and the Paving Conjecture} (2013) \url{http://www.cs.ubc.ca/ nickhar/Publications/KS/KS.pdf} 

B. Roberts, \textit{Philosophy and physics in the kadison-singer conjecture}, Soul Physics, Philosophical Foundations of Physics (2013)
\bibitem{key6} G. Roepstorff, \textit{Coherent Photon States and Spectral Condition}, Commun. math. Phys. 19, 301--314 (1970).
\bibitem{key7} G. 't Hooft, \textit{Magnetic monopoles in unified gauge theories}, Nuc. Phys. B \textbf{79}, 2, 276--284 (1974);

A. M. Polyakov, Zh. Eksp. Teor. Fiz. Pis'ma. Red. \textbf{20}, 430 (1974) [JETP Lett. \textbf{20}, 194 (1974)].
\bibitem{key8} M. E. Peshkin and D. V. Schr\"oeder, \textit{An Introduction To Quantum Field Theory (Frontiers in Physics)}, Perseus Books, Reading, Massachusetts (1995).
\bibitem{key9} F. Riesz and B. S. Nagy, \textit{Functional Analysis}, Dover Books on Mathematics, Reprint Edition (1956).
\bibitem{key10} A. P. Balachandran, A. R. de Queiroz and S. Vaidya, \textit{Quantum Entropic Ambiguities: Ethylene}, Phys. Rev. D \textbf{88}, no. 2, 025001 (2013) [arXiv:1302.4924 [hep-th]].
\bibitem{key11} S. K. Wong \textit{Field and particle equations for the classical Yang-Mills field and particles with isotopic spin}, Nuove Cimento A \textbf{65}, 689-694 (1970).
\bibitem{key12} A. P. Balachandran, S. Borchardt and A. Stern, \textit{Lagrangian and Hamiltonian Descriptions of Yang-Mills Particles}, Phys. rev. D \textbf{17}, 3247 (1978);
\bibitem{key13} A. P. Balachandran, S. Vaidya and A. R. de Queiroz, \textit{A Matrix Model for QCD}, Mod. Phys. Lett. A \textbf{30}, no. 16, 1550080 (2015) [arXiv:1412.7900 [hep-th]];
\bibitem{key14} K. E. Eriksson, \textit{Asymptotic states in quantum electrodynamics}, Phys. Scripta \textbf{1}, e (1970). doi:10.1088/0031-8949/1/1/001
\end{thebibliography}
