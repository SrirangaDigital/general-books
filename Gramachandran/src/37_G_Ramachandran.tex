\chapter{Prof. Gowravaram Ramachandran: Profile of a Theoretical Physicist}

\lhead[\small\thepage]{\small\it\thechapter. Prof. Gowravaram Ramachandran: Profile of a Theoretical Physicist}
\rhead[\small\it\thechapter. Prof. Gowravaram Ramachandran: Profile of a Theoretical Physicist]{\small\thepage}

Gowravaram Ramachandran was born on 29$^{\rm th}$ October, 1936, at Vangal (near Karur), Andhra Pradesh. He was the elder of the two sons of his parents, Shri Gowravaram Venkata Krishnaiah and Smt Kamala. In a life-span of 84 years, G. Ramachandran etched for himself an illustrious life in Physics and contributed significantly to quantum physics, nuclear physics, particle physics and astrophysics.
 
He obtained B.Sc.(Hons) in Physics and Mathematics from the Madras Christian College, Tambaram, Chennai, in 1955, and Master's degree in Physics from Madras University in the year 1957. Immediately thereafter, G. Ramachandran pursued his doctoral research under the guidance of Prof. Alladi Ramakrishnan and obtained his Ph.D. degree in the year 1964 from the Madras University. He had the honour of being one of the founding members of the Theoretical Physics Seminar Group, which was founded by Dr. Alladi Ramakrishnan in 1960 and, under the aegis of this Group, he had the privilege of interacting with several eminent scientists, including Niels Bohr, George Gamow, W. Heitler,  Subrahmanyan Chandrasekhar, Murray Gell-Mann, Abdus Salam, R.H. Dalitz, Donald Glaser, E.C.G. Sudarshan, Leon Rosenfeld, and A.M. Lane. Prof. Niels Bohr, who had visited India as a guest of the Government of India, made a positive mention of the Group's accomplishments to the then Prime Minister, Pandit Jawaharlal Nehru. The then Education and Finance Minister of Madras State, Sri C. Subramaniam (Bharat Ratna), facilitated a meeting of this Group with the Prime Minister. This resulted in the creation of the Institute of Mathematical Sciences (IMSc) also known as MATSCIENCE) at Chennai, in 1962.
 
 Dr. Ramachandran pursued his post-doctoral research at the Research and Training School, Indian Statistical Institute (ISI), Kolkata, and continued to work there as Associate Professor till 1971. His work was greatly appreciated by Dr. C. R. Rao, FRS, the then Director of ISI, Kolkata, and with his encouragement, Dr. Ramachandran developed the infrastructure for research in Particle Physics at the Institute.
 
On an invitation from Professor G.N. Ramachandran, FRS, Dr. Ramachandran joined the Molecular Biophysics Unit (MBU), Indian Institute of Science, Bangalore, as a Visiting Fellow in 1971 and continued in that post till 1973. He joined the Department of Physics, University of Mysore, as a Reader in 1973, and rose to the position of Professor in 1984. After superannuation in 1996, he was CSIR Emeritus Scientist at Mysore University till 2001. Subsequently, on an invitation from Prof. R. Cowsik, the then Director of Indian Institute of Astrophysics (IIA), Bangalore, he joined IIA as a Visiting Senior Professor and worked there till 2007.
 
Prof. Ramachandran continued to be academically active in teaching and research after 2007 under the umbrella of GVK Academy, which he had set up in the revered memory of his father. He left this world on April 9, 2020.

Prof. Ramachandran was deeply interested in a variety of topics in Theoretical Physics. Some of his contributions, resulting from research done in collaboration with several colleagues and students, are listed below:
\begin{itemize}
\item Beginnings of the so-called Medium Energy Physics: Theoretical studies on the use of nuclear targets with particle beams for studying nuclear structure as well as particle properties in the strong interaction regime. This lies at the interface of particle and nuclear physics and the pioneering work began in Madras in late fifties and early sixties.

\item Classification of spin systems into oriented and non-oriented spin systems.

\item Formulation of Multi-Axial representation of spin-j systems, which is applicable to both pure and mixed spin systems, similar to Majorana representation of a spin-j state, which is applicable to pure states only.

\item Extraction of reaction amplitudes from spin (polarization) measurements.

\item Development of a new approach to spin in nuclear and particle reactions.

\item A new description of atoms interacting with lasers and squeezed spin states.

\item Astrophysical spectro-polarimetry.

\item Meson production in N-N collisions and photo and electro production of higher spin mesons.

\item Neutron-Proton fusion and photodisintegration of deuteron at astrophysical energies relevant to big bang nucleosynthesis.

\item Investigation of Einstein-Podolsky-Rosen (EPR) phenomena.

\item Formulation of Multivariate representation of a spin-j density matrix, the variates being the three spin components.

\item Studies in three body problem.
\end{itemize}

He was the recipient of the Mysore University Golden Jubilee Award for research in science and technology in 1980. He trained several Ph.D. students and, with their collaboration, contributed more than 100 research papers to prominent international and national journals and presented more than 150 papers and invited talks at conferences and symposia.

His zeal for mentoring and training students never diminished and the same zeal found expression in the two books he authored:
\begin{enumerate}
\item Preamble to Quantum Theory of Angular Momentum, Gateway to Atoms, Nuclei and Elementary Particles (Prayoga, 2016)

\item Introduction to vectors, axial vectors, tensors and spinors (Vijayalakshmi Prakashana, 2017)
\end{enumerate}

Prof. Ramachandran's wife, Smt. Seethalakshmi, and his children Lata, Devi, Ramakrishna, Gowri, Krishnakumar and Anuradha were pillars of strength and support for him throughout his life and, together, they welcomed his many students into their family.

\eject

\begin{center}
\textbf{Appendix -- A}\\[4pt]
\textbf{List of students guided by Prof. G. Ramachandran for the Ph.D. degree.}
\end{center}

{\renewcommand{\arraystretch}{1.2}
\fontsize{7.5}{9.5}\selectfont
\noindent
\begin{longtable}{@{}|>{\raggedright}p{2cm}|p{2cm}|p{1.3cm}|p{2.5cm}|}
\hline
\multirow{2}{2cm}{\centering\textbf{Student}} & \multirow{2}{2cm}{\centering\textbf{Institution}} & \multicolumn{1}{p{1.3cm}|}{\centering\textbf{Degree Awarded  by}} &  \multirow{2}{2.5cm}{\centering\textbf{Present Position}} \\
\hline
Dr. Anup Kumar Rej & Research and Training School, Indian Statistical Institute, Kolkata &  University of Calcutta &\\
\hline
Dr. M. V. N. Murthy &  Department of Physics, University of Mysore & University of Mysore & Retd. Professor, Institute of Mathematical Sciences, Chennai
\\
\hline
Dr. K. Venkatesh & Department of Physics, University of Mysore & University of Mysore & Retd. Professor, M S Ramaiah Institute of Technology, Bangalore\\
\hline
Dr. R. S. Keshavamurthy &  Department of physics, University of Mysore & University of Mysore & Retd. Scientist, Indira Gandhi Centre for Atomic Research, Kalpakkam\\
\hline
Dr. V. Ravishankar & Department of Physics, University of Mysore & University of Mysore & Professor, Department of Physics, IIT Delhi\\
\hline
Dr. K. S. Mallesh & Department of Physics, University of Mysore & University of Mysore & Retd. Professor, Department of Physics, University of Mysore\\
\hline
Dr. S. N. Sandhya & Department of Physics, University of Mysore & University of Mysore & Faculty Member at Miranda House, New Delhi.\\
\hline
Dr. Sudha Rao Alike & Department of Physics, University of Mysore &  University of Mysore & Retd. Associate Professor, Teresian College, Mysore\\
\hline
Dr. Swarnamala Sirsi &  Department of Physics, University of Mysore & University of Mysore & Retd. Associate Professor, Yuvaraja College, University of Mysore\\
\hline
Dr. M.S. Vidya & Department of Physics, University of Mysore & University of Mysore & Founder Trustee, Vidya Online Charitable Trust, New Delhi. \\
\hline
Dr. A.R. Usha Devi & Department of Physics, University of Mysore & University of Mysore & Professor, Department of Physics, Bangalore University\\
\hline
Dr. P.N. Deepak & Department of Physics, University of Mysore and Indian Institute of Astrophysics, Bangalore & University of Mysore & Assistant Professor, Department of Physics, Birla Institute of Technology and Science, Goa Campus \\
\hline
Dr. J. Balasubramanyam & Indian Institute of Astrophysics, Bangalore & Bangalore University & Faculty Member at BASE, Bangalore \\
\hline
Dr. Vikas M. Shelar & GVK Academy & National Institute of Technology Karnataka, Surathkal & Assistant Professor, Physics Dept., Ramaiah University, Bangalore\\
\hline
\end{longtable}}

He mentored several other students, unofficially, during their research work for Ph.D., including Late Dr. R. K. Umerjee, Dr. K. Ananthanarayanan, Sri S. B. Patangi, Dr. Yee Yee Oo and Dr. S. P. Shilpashree.

\begin{center}
\textbf{Appendix -- B}\\[4pt]
\textbf{Publications of Prof. G. Ramachandran}
\end{center}

Prof.G.Ramachandran contributed more than 100 papers in reputed journals that include :
\begin{itemize}
\item Physical Review and Physics Review Letters (18)
\item Nuclear Physics and Physics Letters (17)
\item Journal of Physics (17)
\item Modern Physics Letters and International Journal of Modern Physics (10) 
\item Pramana (10)
\item Nuovo Cimento (4)
\item Journal of Quantitative Spectroscopy and Radiative Transfer (3)
\item Foundations of Physics (1)
\end{itemize}

The following is the list of his papers.

\newpage

\begin{center}
{\large\bfseries List of Publications of Prof. Ramachandran G.}
\end{center}

\noindent
\textbf{I. Papers published by Prof. Ramachandran G.}

\medskip

\noindent
\textbf{1961}
\begin{enumerate}
\item A note on photo-mesons from deuterons: Devanathan V. and Ramachandran G., \textit{Nuclear Physics}, {\bf 23} (1961) 312--318.
\item Elastic photoproduction of neutral pions from deuterium: Alladi Ramakrishnan, Devanathan V. and Ramachandran G., \textit{Nuclear Physics}, {\bf 24} (1961) 163--168.
\item A Time-Dependent Approach to Rearrangement Collisions: Alladi Ramakrishnan, Ramachandran G. and Devanathan V., \textit{Nuovo Cimento}, {\bf 21} (1961) 145--154.
\end{enumerate}
\noindent
\textbf{1962}
\begin{enumerate}
\setcounter{enumi}{3}
\item Photo-production of charged pions from nuclei: Devanathan V. and Ramachandran G., \textit{Nuclear Physics}, {\bf 38} (1962) 654--660.

\item Nuclear polarization following Photoproduction of pions from nuclei. Part I: Ramachandran G. and Devanathan V., \textit{Nuclear Physics (Netherlands) Divided
into Nucl. Phys. A and Nucl. Phys. B}, {\bf 48} (1963) 369--374.

\item Photoproduction of charged pions from nuclei. Part II: Devanathan V. and Ramachandran G., \textit{Nuclear Physics (Netherlands) Divided into Nucl. Phys. A and
Nucl. Phys. B}, {\bf 42} (1963) 254--263.
\end{enumerate}
\noindent
\textbf{1964}
\begin{enumerate}
\setcounter{enumi}{6}
\item Deuteron polarization following neutral pion photoproduction: Ramachandran G. and Umerjee R. K., \textit{Nuclear Physics}, {\bf 54} (1964) 665--672.
\item Photoproduction of pions from $^{3}$H and $^{3}$He: Ramachandran G. and Ananthanarayanan K., \textit{Nuclear Physics}, {\bf 59} (1964) 633--640.
\item Nuclear polarisation following photoproduction of pions from nuclei (II): Ramachandran G. and Devanathan V., \textit{Nuclear Physics}, {\bf 50} (1964) 593--598.
\end{enumerate}
\textbf{1965}
\begin{enumerate}
\setcounter{enumi}{9}
\item Scattering of pions from $^{3}$H and $^{3}$He: Ramachandran G. and Ananthanarayanan K., \textit{Nuclear Physics}, {\bf 64} (1965) 652--656.
\item Photoproduction of charged pions from nuclei (III): Ramachandran G. and Devanathan V., \textit{Nuclear Physics}, {\bf 66} (1965) 595--608.
\item Electron Helicity in the Final State of Elastic e- p and e- d Scattering: Ramachandran G. and Umerjee R. K., \textit{Physical Review}, {\bf 137} (1965) B978
\item Photoproduction of charged pions from nuclei. Part III: Ramachandran G. and Devanathan V., \textit{Nuclear Physics (Netherlands) Divided into Nucl. Phys. A and Nucl. Phys. B}, {\bf 66} (1965).
\end{enumerate}
\textbf{1966}
\begin{enumerate}
\setcounter{enumi}{13}
\item Pion scattering on nuclei: Ramachandran G., \textit{Nuclear Physics}, {\bf 87} (1966) 107--120.
\item Possible test for the $|\Delta I|=0$ weak nuclear force: Ramachandran G., \textit{Nuovo Cimento A}, 44 (1966) 218--221.
\end{enumerate}
\textbf{1967}
\begin{enumerate}
\setcounter{enumi}{15}
\item Recoil polarization and structure of the deuteron: Ramachandran G., \textit{Nuclear Physics B}, {\bf 2} (1967) 565--580.
\end{enumerate}
\textbf{1969}
\begin{enumerate}
\setcounter{enumi}{16}
\item Recoil-Deuteron Vector Polarization in Elastic Electron Scattering: Ramachandran G., \textit{Physical Review Letters}, {\bf 22} (1969) 794.
\item Deuteron polarization in elastic nucleon deuteron scattering: Ramachandran G., \textit{Lettere al Nuovo Cimento}, {\bf 1} (1969) 39--41.
\end{enumerate}
\textbf{1970}
\begin{enumerate}
\setcounter{enumi}{18}
\item Photoproduction of neutral pions on nuclei: Rej A. K., Prabhakar N. D. and Ramachandran G., \textit{Nuclear Physics B}, {\bf 15} (1970) 56--60.
\item Photoproduction of neutral pions on $^{7}$Li: Ramachandran G., Prabhakar N. D., and Rej A. K., \textit{Nuclear Physics B}, {\bf 20} (1970) 369--380.
\end{enumerate}
\textbf{1972}
\begin{enumerate}
\setcounter{enumi}{20}
\item Comment on a recent paper of Olkhovsky and Recami: Ramachandran G., Tagare S. G. and Kolaskar A. S., \textit{Lettere al Nuovo Cimento}, {\bf 4} (1972) 141--143.
\item Polarization of the emitted neutron in muon capture: Devanathan V., Parthasarathy R. and Ramachandran G., \textit{Annals of Physics}, {\bf 72} (1972) 428--444.
\item A semi-classical model of the electron containing tachyonic matter: Ramachandran G. N. and Ramachandran G. and Tagare S. G., \textit{Physics Letters A}, {\bf 39} (1972) 383--384.
\end{enumerate}
\textbf{1976}
\begin{enumerate}
\setcounter{enumi}{23}
\item Bremsstrahlung in neutron-electron collisions: Ramachandran G. and Keshavamurthy R. S., \textit{J. Mys. Univ.}, {\bf XXVI} (1976) 141.
\end{enumerate}
\textbf{1977}
\begin{enumerate}
\setcounter{enumi}{24}
\item Neutrino scattering on polarized deuterons: Murthy M. V. N., Ramachandran G. and Sarma K. V. L., \textit{Pramana, J. of Phys.}, {\bf 9} (1977) 11.
\end{enumerate}
\textbf{1978}
\begin{enumerate}
\setcounter{enumi}{25}
\item Recoil deuteron vector polarization in elastic electron-deuteron scattering: Ramachandran G. and Singh S. K., \textit{Phys. Lett.}, {\bf D18} (1978) 1441.
\item Magnetic contribution to high energy inverse bremsstrahlung process: Ramachandran G. and Keshavamurthy R.S., \textit{Phys. Rev.}, {\bf 10} (1978) 559.
\item Photoproduction of charged pions on neutrons from deuteron targets: Ramachandran G., Keshavamurthy R. S. and Murthy M. V. N., \textit{Phys. Lett.}, {\bf B77} (1978) 65.
\item Theoretical study of deuteron polarization in $\gamma+d\to \pi^{0}+d$: Ramachandran G. and Murthy M. V. N., \textit{Nucl. Phys.}, {\bf A 302} (1978) 404.
\end{enumerate}
\textbf{1979}
\begin{enumerate}
\setcounter{enumi}{29}
\item Photoproduction of charged pions on neutrons.: Ramachandran G., Keshavamurthy R. S. and Murthy M. V. N., \textit{J. Phys.}, {\bf G5} (1979) 1525.
\item Parity violation in polarised electron-deuteron scattering: Murthy M. V. N., Ramachandran G. and Singh S. K., \textit{Phys. Lett.}, {\bf 81B} (1979) 129.
\item Target asymmetry and effective neutron polarization with polarized deuteron targets: Ramachandran G., Keshavamurthy R. S. and Murthy M. V. N., \textit{Phys. Lett.}, {\bf 87B} (1979) 252.
\item A new representation for the density matrix: Ramachandran G. and Murthy M. V. N., \textit{Nucl. Phys.}, {\bf A323} (1979) 403.
\item Photoproduction of charged pions on neutrons from deuteron targets: Ramachandran G., Keshavamurthy R. S. and Murthy M. V. N., \textit{Phys. Lett.}, {\bf B86} (1979) 426.
\end{enumerate}
\textbf{1980}
\begin{enumerate}
\setcounter{enumi}{34}
\item SU(3) representation for the polarization of light: Ramachandran G., Murthy M. V. N. and Mallesh K. S., \textit{Pramana, J. of Phys.}, {\bf 15} (1980) 357.
\item A new representation for the density matrix II: Equation of motion: Ramachandran G. and Murthy M. V. N., \textit{Nucl. Phys.}, {\bf A337} (1980) 301.
\end{enumerate}
\textbf{1981}
\begin{enumerate}
\setcounter{enumi}{36}
\item Study of $^{3}$He$(\gamma,\pi^{0})^{3}$ He as the nuclear probe: Ramachandran G., Keshavamurthy R. S. and Venkatesh K., \textit{Pramana, J. of Phys.}, {\bf 17} (1981) 337.
\item Longitudinal asymmetry and polarization in $n(\gamma,\pi^{-})p$ on deuteron targets: Keshavamurthy R. S., Murthy M. V. N. and Ramachandran G., \textit{J. Phys. G: Nucl. Phys.}, {\bf 7} (1981) L137--L140.
\item Photoproduction of charged pions on neutrons II: Keshavamurthy R. S. and Ramachandran G., \textit{J. Phys. G: Nucl. Phys.}, {\bf 7} (1981) 867--880.
\end{enumerate}
\textbf{1982}
\begin{enumerate}
\setcounter{enumi}{39}
\item Threshold photoproduction of charged pions on $^{14}$N: Murthy M. V. N., Ramachandran G. and Singh S. K., \textit{Z. Phys.}, {\bf A306} (1982) 117.
\end{enumerate}
\textbf{1983}
\begin{enumerate}
\setcounter{enumi}{40}
\item Weak neutral current effects in elastic e-d scattering: Murthy M. V. N., Ramachandran G. and Singh S. K., \textit{Pramana, J. of Phys.}, {\bf 20} (1983) 221.
\item Fermi motion and Pauli exclusion principle effects in $d(\pi^{-},\pi^{-}p)n$ in the $\Delta$-resonance region: Ramachandran G., Keshavamurthy R. S., Patangi S. B. and Ravishankar V., \textit{Phys. Rev. C}, {\bf 29} (1983) 2198.
\end{enumerate}
\textbf{1984}
\begin{enumerate}
\setcounter{enumi}{42}
\item On polarized spin-1 systems: Ramachandran G., Mallesh K. S. and Ravishankar V., \textit{J. Phys. G: Nucl. Part. Phys.}, {\bf 10} (1984) L163--L166.
\item Oriented spin systems: Ramachandran G. and Mallesh K. S., \textit{Nucl. Phys.}, {\bf A422} (1984) 327.
\item Fermi motion, Pauli principle effects and polarization phenomena in medium energy processes on Nuclei: Ramachandran G., \textit{J. Madras Univ.}, {\bf 17} (1984) 38.
\end{enumerate}
\textbf{1985}
\begin{enumerate}
\setcounter{enumi}{45}
\item Photon emission from non-oriented spin systems: Ramachandran G. and Ravishankar V., \textit{Pramana, J. of Phys.}, {\bf 24} (1985) 813.
\item Nuclear parity violation in $\gamma d\to \pi^{0}d$: Ravishankar V., Ramachandran G. and Murthy M. V. N., \textit{Phys. Rev. C}, {\bf 32} (1985) 640.
\end{enumerate}
\textbf{1986}
\begin{enumerate}
\setcounter{enumi}{47}
\item Redundant polarisation measurements in particle interactions with arbitrary spin j: Ravishankar V. and Ramachandran G., \textit{J. Mod. Phys. Lett.}, {\bf A1} (1986) 333.
\item On polarized spin j assemblies: Ramachandran G. and Ravishankar V., \textit{J. Phys. G: Nucl. Phys.}, {\bf 12} (1986) L143.
\item Theoretical study of recoil nuclear polarization in $^{10}$B$(\gamma^{-},\pi^{+})^{10}$B: Ramachandran G. and Ravishankar V., \textit{J. Phys. G: Nucl. Phys.}, {\bf 12} (1986) 1221.
\item Inelastic scattering of pions on $^{7}$Li: Patangi S. B. and Ramachandran G., \textit{Pramana, J. of Phys.}, {\bf 26} (1986) 337.
\item Polarization parameters in systems with spin-spin interactions: Mallesh K. S. and Ramachandran G., \textit{Pramana, J. of Phys.}, {\bf 26} (1986) 43.
\end{enumerate}
\textbf{1987}
\begin{enumerate}
\setcounter{enumi}{52}
\item Non-oriented spin systems: Ramachandran G., \textit{J. Madras Univ.}, {\bf B50} (1987) 383.
\item 2j-th rank tensor polarization in reactions involving a spin-j particle: Ravishankar V. and Ramachandran G., \textit{Phys. Rev.}, {\bf C35} (1987) 62.
\item Non-oriented spin-1 system: Ramachandran G., Ravishankar V., Sandhya S. N. and Swarnamala Sirsi, \textit{J. Phys. G: Nucl. Phys.}, {\bf 13} (1987) L271--L273.
\end{enumerate}
\textbf{1988}
\begin{enumerate}
\setcounter{enumi}{55}
\item Complete determination of reaction amplitudes: Ramachandran G. and Sandhya S. N., \textit{Mod. Phys. Lett.}, {\bf A3} (1988) 1113.
\item Ambiguity free polarization measurements with mixture initial state preparations: Ramachandran G. and Ravishankar V., \textit{Phys. Rev.}, {\bf C37} (1988) 553.
\end{enumerate}
\textbf{1989}
\begin{enumerate}
\setcounter{enumi}{57}
\item Polarization parameters of a spin-one system bounds and geometrical representation: Ramachandran G. and Mallesh K. S., \textit{Phys. Rev.}, {\bf C40} (1989) 1641.
\item Invariants of motion for an N-level system: Mallesh K. S. and Ramachandran G., \textit{J. Phys. B: At. Mol. Opt. Phys.}, {\bf 22} (1989) 2311.
\end{enumerate}
\textbf{1990}
\begin{enumerate}
\setcounter{enumi}{59}
\item On $\pi-d$ elastic scattering using aligned deuteron targets: Sudha Rao A. and Ramachandran G., \textit{Pramana, J. of Phys.}, {\bf 35} (1990) 61.
\item Spin observables and reconstruction of $\pi-d$ elastic scattering amplitudes in transverse frame: Sandhya S. N., Sudha Rao A. and Ramachandran G., \textit{J. Phys. G: Nucl. Phys.}, {\bf 16} (1990) L95.
\end{enumerate}
\textbf{1992}
\begin{enumerate}
\setcounter{enumi}{61}
\item Channel selection in collisions involving particles with spins: Ramachandran G. and Vidya M. S., \textit{J. Mys. Univ.}, {\bf 32B} (1992) 545.
\item Ambiguity free empirical determination of $^{10}$C$(p, p')$ $^{10}$C$^{*}$(1$^{\dagger}$) inelastic scattering amplitudes: Sudha Rao A., Mallesh K. S. and Ramachandran G., \textit{Mod. Phys. Lett. A}, {\bf 7} (1992) 175.
\item Multiaxial decomposition of a cartesian tensor: Swarnamala Sirsi, Usha Devi A. R. and Ramachandran G., \textit{J. Mys. Univ.}, {\bf 32B} (1992) 541.
\end{enumerate}
\textbf{1993}
\begin{enumerate}
\setcounter{enumi}{64}
\item Radar polarimetry: The need aspect and the density matrix approach: Vishwanathan G., Ramachandran G. and Vidya M. S., \textit{Ind. J. Rad. Space Phys.}, 22 (1993) 180.
\end{enumerate}
\textbf{1994}
\begin{enumerate}
\setcounter{enumi}{65}
\item Photon polarization asymmetries in $(p, p'\gamma)$ experiments: Usha Devi A. R., Sudha Rao A. and Ramachandran G., \textit{Pramana, J. of Phys.}, {\bf 42} (1994) 97.
\item Determination of inelastic scattering amplitudes in $(p, p'\gamma)$ reactions: Ramachandran G., Usha Devi A. R. and Sudha Rao A., \textit{Phys. Rev. C}, {\bf 49} (1994) R623.
\item Joint probabilities for aligned spin-1 system: Usha Devi A. R., Swarnamala Sirsi, Devi G. and Ramachandran G., \textit{J. Phys. G: Nucl. Phys.}, {\bf 20} (1994) 1859.
\item Non-oriented nuclei in external electric and magnetic fields: Ramachandran G., Swarnamala Sirsi and Devi G., \textit{Perspectives in Theoretical Nuclear Physics, Eds. K. Srinivasa Rao and L. Sathpathy, Wiley-Eastern(India) Ltd.}, (1994) 76.
\end{enumerate}
\textbf{1995}
\begin{enumerate}
\setcounter{enumi}{69}
\item Polarized light: Ramachandran G., Usha Devi A. R. and Vardhana N. S. S. K., \textit{Pramana, J. of Phys.}, {\bf 45} (1995) 319.
\item A complement to Goldstein-Moravesik theorem: Usha Devi A. R., Sudha Rao A. and Ramachandran G., \textit{Mod. Phys. Lett.}, {\bf A10} (1995) 1449.
\end{enumerate}
\textbf{1996}
\begin{enumerate}
\setcounter{enumi}{71}
\item Quasi-probability distributions for arbitrary spin-j particles: Ramachandran G., Usha Devi A. R., Devi P. and Swarnamala Sirsi, \textit{Foundations of Physics}, {\bf 26} (1996) 401.
\end{enumerate}
\textbf{1997}
\begin{enumerate}
\setcounter{enumi}{72}
\item Non-central interactions in elastic scattering with arbitrary spins: The case of $N\Delta\to N\Delta$: Ramachandran G., Vidya M. S. and Prakash M. M., \textit{Phys. Rev. C}, {\bf 56} (1997) 2882.
\item $\Delta$ excitation in inelastic scattering of nucleons on nuclei: Ramachandran G. and Vidya M. S., \textit{Phys. Rev. C}, {\bf 56} (1997) R12.
\item Neutrino scattering on polarised deuterons: Murthy M. V. N., Ramachandran G. and Sarma K. V. L., \textit{Pramana, J. of Phys.}, {\bf 9} (1997) 11.
\item Non-locality in EPRB spin correlations: Usha Devi A. R., Swarnamala Sirsi and Ramachandran G., \textit{Physics Teacher}, {\bf 39} (1997) 68.
\item Nonlocality in Einstein-Podolsky-Rosen spin correlations: Usha Devi A. R., Swarnamala Sirsi and Ramachandran G., \textit{Int. J. Mod. Phys., A}, {\bf 12} (1997) 5279.
\item Trivariate quasi-probability distributions for polarised spin-1 nuclei: Usha Devi A. R., Swarnamala Sirsi, Ramachandran G. and Devi P., \textit{Int. J. Mod. Phys. A}, {\bf 12} (1997) 2779--2790.
\end{enumerate}
\textbf{1998}
\begin{enumerate}
\setcounter{enumi}{78}
\item Tensor force and NN scattering with polarized beam and target: Ramachandran G. and Deepak P. N., \textit{Mod. Phys. Lett. A}, 13 (1998) 3063.
\item Effective spin dependent interactions for reaction: The case of $NN \to N\Delta$: Ramachandran G. and Vidya M. S., \textit{Phys. Rev. C}, {\bf 58} (1998) 3008.
\end{enumerate}
\textbf{1999}
\begin{enumerate}
\setcounter{enumi}{80}
\item $pd$ fusion with polarized deuterons: Ramachandran G., Deepak P. N. and Prasanna Kumar S., \textit{J. Phys. G: Nucl. Part. Phys.}, {\bf 25} (1999) L155--L158.
\end{enumerate}
\textbf{2000}
\begin{enumerate}
\setcounter{enumi}{81}
\item Irreducible tensor phenomenology for $pd\to {}^{3}$He $\pi^{+}\pi^{-}$: Ramachandran G. and Deepak P. N., \textit{J. Phys. G: Nucl. Part. Phys.}, {\bf 26} (2000) 343--345.
\item Total cross sections for scattering and reactions with polarized beam and polarized target: Ramachandran G. and Deepak P. N., \textit{J. Phys. G: Nucl. Part. Phys.}, {\bf 26} (2000) 1809--1815.
\item Pion production in \textit{N N} collisions: Ramachandran G., Deepak P. N. and Vidya M. S., \textit{Phys. Rev. C}, {\bf 62} (2000) 011001(R).
\item Irreducible tensor phenomenology for pd$\to{}^{3}$He $\pi^{+}\pi^{-}$: Ramachandran G. and Deepak P. N., \textit{Journal of Physics G: Nuclear and Particle Physics}, {\bf 26} (2000) 343.
\item Squeezing and non-oriented spin states: Mallesh K. S., Swarnamala Sirsi, Mahmood A. A. Sbaiah, Deepak P. N. and Ramachandran G., \textit{J. Phys. A}, {\bf 33} (2000) 779.
\end{enumerate}
\textbf{2001}
\begin{enumerate}
\setcounter{enumi}{86}
\item Model-independent determination of Doublet and Quartet cross sections for N$\to$d$\to$ fusion: Ramachandran G. and Deepak P. N., \textit{Nucl. Phys. A}, {\bf 695} (2001) 175.
\item Empirical analysis for differential cross section measurements for p$\to$p$\to$pp$\pi^{0}$: Ramachandran G. and Deepak P. N., \textit{Phys. Rev. C}, {\bf 63} (2001) 051001.
\item Non-central interactions in inelastic scattering of nucleons on nuclei: The case of $^{12}$C(p,p$'$)$^{12}$C$^{*}$($1^{+}$): Ramachandran G., Vidya M. S. and Sudha Rao A., \textit{Phys. Rev. C}, {\bf 63} (2001) 034604.
\item Spin squeezing of mixed systems: Mallesh K. S., Swarnamala Sirsi, Mahmoud A. A. Sbaih, Deepak P. N. and Ramachandran G., \textit{J. Phys. A: Math. Gen.}, {\bf 34} (2001) 3293--3308.
\end{enumerate}
\textbf{2002}
\begin{enumerate}
\setcounter{enumi}{90}
\item Singlet and triplet differential cross sections for pp$\to$pp$\pi^{0}$: Deepak P. N. and Ramachandran G., \textit{Phys. Rev. C}, {\bf 65} (2002) 027601.
\item Spin Squeezing and Quantum Correlations,: Mallesh K. S., Swarnamala Sirsi, Mahmood A. A. Sbaih, Deepak P. N. and Ramachandran G., \textit{Pramana, J. of
Phys.}, {\bf 59} (2002) 175.
\end{enumerate}
\textbf{2003}
\begin{enumerate}
\setcounter{enumi}{92}
\item Squeezing of a coupled state of two spinors: Usha Devi A. R., Mallesh K. S., Mahmoud A. A. Sbaih, Nalini K. B. and Ramachandran G., \textit{J. Phys. A: Math.
Gen.}, {\bf 36} (2003) 5333--5347.
\item Multivariate Quasi Probability Distribution for Polarized Spin Systems: Ramachandran G., Devi, Usha Devi A. R., Putcha D. and Swarnamala Sirsi, \textit{Stochastic Point Processes}, {\bf 256} (2003).
\item Photon polarization in np fusion : Ramachandran G., Deepak P. N., Kumar S. P., \textit{Journal of Physics G: Nuclear and Particle Physics}, {\bf 29} (2003) L45.
\end{enumerate}
\textbf{2004}
\begin{enumerate}
\setcounter{enumi}{95}
\item Radiative capture of polarized neutrons by polarized protons: Ramachandran G.and Deepak P. N., \textit{Modern Physics Letters A}, {\bf 19} (2004) 1411--1419.
\item Spin-dependence of meson production in nucleon-nucleon collisions: Deepak P. N., Hanhart C., Ramachandran G. Vidya M. S., \textit{Verhandlungen der Deutschen
Physikalischen Gesellschaft}, {\bf 39} (2004) 28.
\end{enumerate}
\textbf{2005}
\begin{enumerate}
\setcounter{enumi}{97}
\item Polarization of line radiation in the presence of external electric quadrupole and uniform magnetic fields: II. Arbitrary orientation of magnetic field: Yee Yee Oo, Nagendra K. N., Sharath Ananthamurthy, Swarnamala Sirsi, Vijayashankar R. and Ramachandran G., \textit{Journal of Quantitative Spectroscopy and Radiative Transfer}, {\bf 90} (2005) 343--366.
\item Scattering of Polarized Radiation by Atoms in Magnetic and Electric Fields: Yee Yee Oo, Nagendra K. N., Sharath Ananthamurthy and Ramachandran G., \textit{arXiv:astro-ph/0509775}, (2005).
\item Spin-dependence of meson production in NN collisions: Deepak P. N., Hanhart C., Ramachandran G. Vidya M. S., \textit{International Journal of Modern Physics A}, {\bf 20} (2005) 599--601.
\item $\omega$ production in pp collisions: Ramachandran G., Vidya M. S., Deepak P. N., Balasubramanyam J., \textit{Physical Review C}, {\bf 72} (2005) 031001.
\end{enumerate}
\textbf{2006}
\begin{enumerate}
\setcounter{enumi}{101}
\item Empirical Determination of Threshold Partial Wave Amplitudes in pp$\to$pp$\omega$: Ramachandran G., Balasubramanyam J., Vidya M. S. and Venkataraya, \textit{Mod. Phys. Lett. A}, {\bf 21} (2006) 2009--2017.
\item Deuteron photodisintegration with polarized photons at astrophysical energies: Ramachandran G., Shilpashree S. P., \textit{Phys. Rev. C}, {\bf 74} (2006) 052801.
\item Photodisintegration of polarized deuterons at astrophysical energies: Ramachandran G., Yee Yee Oo and Shilpashree S. P., \textit{J. Phys. G: Nucl. Part. Phys.}, {\bf 32} (2006) B17.
\end{enumerate}
\textbf{2007}
\begin{enumerate}
\setcounter{enumi}{104}
\item Irreducible tensor approach to spin observables in the photoproduction of mesons with arbitrary spin-parity s $\pi$: Ramachandran G., Vidya M. S. and Balasubramanyam J., \textit{Physical Review C}, {\bf 75} (2007) 065201.
\item Tensor polarization of $\omega$ produced at threshold in p-p collisions: Ramachandran G., Balasubramanyam J., Shilpashree S. P. and Padmanabha G., \textit{J. Phys. G}, {\bf 34} (2007) 661--666.
\item Unified approach to photo-and electro-production of mesons with arbitrary spins: Ramachandran G., Vidya M. S. and Balasubramanyam J., \textit{Pramana}, {\bf 68} (2007) 31--41.
\item Scattering Polarization in the Presence of Magnetic and Electric Fields: Yee Yee Oo, Sampoorna M., Nagendra K. N., Sharath Ananthamurthy and Ramachandran G., \textit{Journal of Quantitative Spectroscopy and Radiative Transfer}, 108(2) (2007).
\end{enumerate}
\textbf{2008}
\begin{enumerate}
\setcounter{enumi}{108}
\item Omega meson production in pp collisions with a polarized beam: Balasubramanyam J., Venkataraya, Ramachandran G., \textit{Phys. Rev. C}, {\bf 78} (2008) 012201.
\item Polarization phase matrices for radiation scattering on atoms in external magnetic fields: The case of forbidden transitions in astrophysics: Yee Yee Oo, Phyu Phyu San, Sampoorna M., Nagendra K. N. and Ramachandran G., \textit{arXiv:astro-ph/0805.3860}, (2008).
\end{enumerate}
\textbf{2009}
\begin{enumerate}
\setcounter{enumi}{110}
\item Decay of Polarized Delta: Ramachandran G., Venkataraya, Vidya M. S, Balasubramanyam J. and Padmanabha G., \textit{arXiv:nucl-th/0901.0659}, (2009).
\item Final state polarization of protons in $pp\to pp\omega$: Ramachandran G., Venkataraya and Balasubramanyam J., \textit{Phys. Rev. C.}, {\bf 81} (2009) 027601.
\end{enumerate}
\textbf{2010}
\begin{enumerate}
\setcounter{enumi}{112}
\item Phase ambiguity of the threshold amplitude in $pp \to pp\pi^{0}$: Ramachandran G., Padmanabha G. and Sujith Thomas, \textit{Phys. Rev. C.}, {\bf 81} (2010).
\end{enumerate}
\textbf{2011}
\begin{enumerate}
\setcounter{enumi}{113}
\item Deuteron photodisintegration with polarized lasers: Ramachandran G.and Shilpashree S. P., \textit{arXiv:nucl-th/1104.1308}, (2011).
\end{enumerate}
\textbf{2013}
\begin{enumerate}
\setcounter{enumi}{114}
\item Photodisintegration of aligned deuterons at astrophysical energies using linearly polarized photons: Shilpashree S. P., Swarnamala Sirsi and Ramachandran G., \textit{International Journal of Modern Physics E}, {\bf 22} (2013) 1350030.
\end{enumerate}
\textbf{2019}
\begin{enumerate}
\setcounter{enumi}{115}
\item $\Delta$ Contribution to $pp\to pp\pi^{0}$: Venkataraya, Sujith Thomas and Ramachandran G., \textit{arXiv:nucl-th/1909.03456}, (2019).
\item Statistical Assemblies of Particles with Spin: Ramachandran G., \textit{arXiv:nucl-th/1909.03931 quant-ph}, (2019) [Dedicated to Dr. C R Rao on the occasion of his 100$^{\text{th}}$ Birthday]
\end{enumerate}

%~ \newpage

\noindent
\textbf{II. Books authored by Prof. Ramachandran G.}
\begin{enumerate}
\item Preamble to Quantum Theory of Angular Momentum, Gateway to Atoms, Nuclei and Elementary Particles: Ramachandran G., \textit{Prayoga}, (2016).
\item Introduction to vectors, axial vectors, tensors and spinors: Ramachandran G., Vidya M. S. and Venkataraya, \textit{Vijayalakshmi Prakashana}, (2017).
\end{enumerate}
