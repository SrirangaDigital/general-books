\chapter{The Ensemble Nature of the Schr\"{o}dinger Equation and its Wavefunction, and a New Universal Action Mechanics}\label{chap27}

%\Authorline{N.G.Deshpande// Professor of Physics, University of Oregon}

\begin{center}
\textbf{C. S. Unnikrishnan}\\
\textbf{\textit{Tata Institute of Fundamental Research, Homi Bhabha Road, Mumbai 400005}}
\end{center}

\begin{center}
(Date textdate)\\

\medskip

Abstract
\end{center}


All researchers of quantum mechanics assume that the Schr\"{o}dinger equation is the equation for
dynamics, akin to Newton's equation of motion or the Hamilton-Jacobi equation for the evolution
of action. The wavefunction $\psi(x)$ is assumed to be some way associated with a single particle,
or a single quantum history. Hence, Born's ad hoc interpretation is necessary to connect the psi-function to the statistical results of an ensemble of observations. Here I show that the Schr\"{o}dinger
equation is not the equation for dynamics, but it is in fact the equation for the evolution of
the ensemble probability density $\rho(x)$, written in terms of the single valued square root χ(x)
of the probability density, \textit{with the constraint that $\rho(x)$ is that of a dynamical system}. It then
follows that the Born's rule is an exact relation $\chi \chi^{\ast} \equiv \rho$, by definition. This implies that a
matter-wave is absent in quantum mechanics; $\psi(x)$ is not the matter-wave. Neither is it the
representative of a single quantum history. The psi-function is an ensemble averaged quantity.
The true dynamical equation is a modified Hamilton's equation for the action-wave $\zeta(S)$. This
completes the Hamiltonian dynamics to a universal dynamics, with the encompassing uncertainty
relation $\Delta S \geq \hbar$. These findings define the kernel of quantum dynamics without any of the
foundational problems of interpretation or ontology, while retaining all the statistical results of
quantum mechanics.

\section{Introduction}

``\textbf{It will be impossible to answer any one question completely without at the
same time answering them all}", P. A. M. Dirac, Proc. Roy. Soc. Lond. A, Vol. 114,
243 (1927).

The Schr\"{o}dinger equation marked a decisive milestone of modern quantum mechanics
(QM), after two decades of isolated results of importance in atomic physics. The equation
was also considered as the completion of non-relativistic QM, apart from the addition of the
Born's rule. Its more detailed structure, compared to Heisenberg's earlier breakthrough of
finding the `rule of quantization', made it the basis of most discussions and the calculations
of the theory. Schr\"{o}dinger arrived at the first order equation for the time evolution of
`something like a wave' after many steps, spread over four papers in 1926, starting with a
second order wave equation that resembled conventional second order wave equations [1].
The differential equation was written for the time evolution of the ``wavefunction", which was
notionally related to, and inspired by, de Broglie's matter-wave. The Schr\"{o}dinger equation
is
\begin{equation*}
i \hbar \frac{\partial \psi (x)}{\partial t} = \hat{H} \psi (x) \tag{1}
\end{equation*}
where $\hat{H}$ is a differential operator corresponding to the Hamiltonian of the system. However,
as well known, the wavefunction or the `psi-function' $\psi (x)$ cannot be interpreted as a physical
entity is real space, though it is a function of the $3N$ coordinates of the $N$ particles in the
quantum system. It does not directly represent the material particle or a wave corresponding
to the particle, evolving in real space. Thus, the complex valued psi-function remains as
much of a mystery now as it was about a century ago. The only additional interpretational
insight was M. Born's proposal that the absolute square of the $\psi$-function, $\psi(x) \psi^{\ast}(x)$, is to
be equated to the probability density $\rho(x)$ for evaluating the statistical results of all possible
observations of the system.

Subsequent formal developments clarified the mathematical structure of QM and clearly
delineated how to use the theory for the precise calculations of ensemble averaged quantities
of interest. The generalization to the Pauli equation for the two-component `spinor' psi-
function and the further generalization to the relativistic case with the invention of the Dirac
equation completed the quantum theory of particles. The comparison with experimental
observations, and the excellent empirical agreement in every known case, suggest that the
present QM is the correct statistical description of physical effects in the microscopic atomic
world, to the tiniest scales.

I intend to take a close look at the Schr\"{o}dinger equation, to reveal its true structure
and meaning, including the real nature of the elusive `psi-function'. I will show that \textit{the
Schr\"{o}dinger equation is not the equation for dynamics}. Rather, it is the equation for the
time evolution of a probability density, written in terms of its single valued square root, with
the constraint that the probability density corresponds to the dynamics of an ensemble of
particles. In other words, the Schr\"{o}dinger equation is an ensemble equation, and not the
equation for the single particle or single quantum history. This drastically changes the basis
for the entire understanding of quantum mechanics, hitherto followed in all approaches
to the interpretation of the theory. Then the question arises immediately, what then is
the core equation of quantum dynamics?! The correct equation for dynamics is found by
modifying and completing Hamilton's equation for the evolution of the action S, to a new
equation for the true ``wavefunction of action" $\zeta(S)$, which describes the universal dynamics
of particles at all scales and all velocities \cite{chap27-key2}. The new dynamics reveals the reason for the
intrinsic uncertainty, with the new general relation $\Delta S \geq \hbar$. It is argued, from empirical
considerations, that there is no matter-wave, and that matter itself does not have any wave
property. The sole `wave' in quantum mechanics, and indeed in all mechanics, is the `action-
wave'. It is shown, with explicit demonstrations and calculations, that the new mechanics
solves the vexing problems of the collapse of the quantum state and quantum measurement,
as well as the alleged problem of nonlocality in multi-particle correlations. In addition, the
new mechanics is free of the divergence of zero-point energy in the vacuum wave modes.
With these clarifications and features, quantum mechanics seems completed, both in its
structure and in its physical interpretation, a century after its formulation. Now, we can
not only calculate with quantum mechanics, but also understand the dynamics consistently
and causally, without conceptual clashes or ambiguities.

\section{The Schr\"{o}dinger Equation}

Schr\"{o}dinger started his exploration of a wave equation, for the matter wave conjectured
by de Broglie, with the familiar differential equation for waves,
\begin{equation*}
\nabla^2 \psi (x,t) - \frac{\partial^2 \psi (x,t)}{u^2 \partial t^2} = 0 \tag{2}
\end{equation*}
where the velocity $u = E/ [2m(E - V )]^{1/2}$. With the time dependence of the $\psi$-function
that obeys the relation $E = h\nu$ limited to $\psi(t) \infty \exp \pm i2\pi \nu t$, one gets $\ddot{\psi} (t) = -4 \pi^2 E^2 \psi /h^2$.
The resulting equation with the Coulomb potential in the Hydrogen atom $V (r) = e^2 /r$,
\begin{equation*}
\nabla^2 \psi + \frac{8 \pi^2 m}{h^2} (E + e^2 / r) \psi = 0 \tag{3}
\end{equation*}
\textit{did provide the Bohr spectrum of Hydrogen atom} \cite{chap27-key1}. Since the interpretation of $\psi$ as the
matter wave of the single electron around the nucleus sort of fitted what the equation was
suggesting, the initial interpretation of the Schr\"{o}dinger equation as a wave equation repre-
senting dynamics, and its wavefunction as the matter wave in space, was widely influential.
This general identification of matter as a dual entity, \textit{as both a particle and a wave}, determined the rest of the history of the physical interpretation of QM. However, it may be noted that there was no compulsion from experiments to equate the particles with a wave,
both propagating in space; what was empirically required was the association of only the
dynamics (motion) of the particle, and the material particle itself, with an unobservable
wave-like entity, with the periodicity $\hbar$ in the action $S$.

What was Schr\"{o}dinger's route from the equation
\begin{equation*}
\nabla^2 \psi(x,t) + \frac{2m}{\hbar^2} (E-V) \psi = 0 \tag{4}
\end{equation*}
to the final Schr\"{o}dinger equation? Stated concisely, he eliminated the quantity $E$ in the
equation by writing the time derivative $\psi(t) = 2\pi i E \psi /h$. Then we get
\begin{equation*}
\nabla^2 \psi (x,t)  + \frac{2m}{\hbar^2} [-i \hbar \dot{\psi} (t)] - \frac{2m}{\hbar^2} V_{\psi} = 0 \tag{5}
\end{equation*}
which is the Schr\"{o}dinger equation
\begin{equation*}
i \hbar \frac{\partial \psi (x,t)}{\partial t} = - \frac{\hbar^2}{2m} \nabla^2 \psi (x,t) + V \psi  \tag{6}
\end{equation*}

Clearly, the logical path followed was not one that guaranteed, or even suggested, an
equation of dynamics. Note that there was no relation made between the combination
$(\nabla^2 + V )$ and an operator corresponding to the Hamiltonian of the system. The total energy
was in fact represented by the time derivative because that was obtained by replacing $E$
with $\dot{\psi}$, after assuming harmonic solutions. Also, one just assumed its general validity for
time-dependent potentials as well. Schr\"{o}dinger invoked `success' as the justification for the
procedure. However, the price paid for the success was that he had to hesitantly abandon
any acceptable physical interpretation of the psi-function and the wave equation. The first
order time derivative may remind us of the first order derivative in the celebrated Hamilton's
equation (called the Hamilton-Jacobi equation) of the action,
\begin{equation*}
\frac{\partial S (x,p,t)}{\partial t} = - H \tag{7}
\end{equation*}

But this is misleading, because there is no obvious relation between the function ψ and the
action S. The reason I retraced Schr\"{o}dinger's path to the final equation was to remind
that there is no direct correspondence between the action $S(x, p, t)$ and the wavefunction
$\psi(x, t)$ in spite of the fact that Schr\"{o}dinger derived much of the motivation and guidance
from Hamilton's work while formulating a wave equation for quantum dynamics. However,
the superficial resemblance, especially the first order time derivative, has been serving as a
misdirecting factor.

During his struggle to find a physical interpretation, Schr\"{o}dinger realized that only the
product $\psi(x) \psi^{\ast}(x)$ had any clear physical meaning; he saw $e|\psi|^2$ as representing the effective
charge density of a `fuzzy' electron. Even that was beyond an interpretation consistent
with the conventional notion of localized matter. It was only after Born's interpretation of
$\psi(x)\psi^{\ast} (x)$ as a probability density, rather than a real density of any physical quantity in
space, that one could claim a satisfactory structure in the theory of QM. Then, one got the
option to keep the particle as `point' matter, and $|\psi|^2$ as referring to the (statistical) ensemble
of particles. However, it also marks a missed opportunity for a complete understanding of
the new mechanics. If $\psi(x) \psi^{\ast} (x)$ is indeed the probability density, an evolution equation
for $\psi(x)$ should most naturally be related to the time evolution of the probability density,
rather than to the dynamics of a particle! And it is known that the time evolution of the
probability density of a closed ensemble is given by a first order partial differential equation
in time, called the equation of continuity or conservation,
\begin{equation*}
\frac{\partial \rho (x,t)}{\partial t} = - \nabla \cdot (\rho \upsilon) \tag{8}
\end{equation*}

This equation does not appear to be related to either the Hamilton's equation or the
Schr\"{o}dinger equation, especially with its first order spatial derivative. However, there are
two ways in which the second spatial derivatives are implicit in the continuity equation. One
is because the dynamical velocity $\upsilon$ is related to the action $S$ in the Hamilton's equation
by the relation $p = m \upsilon =\nabla S$. Another is when the statistical ensemble has a diffusion-
like behaviour, in which case the probability current is proportional to the gradient of the
probability density, $\rho \upsilon = - D \nabla \rho$. Then
\begin{equation*}
\upsilon = - \frac{1}{\rho} D \nabla \rho  = - D \nabla (l n \rho) \tag{9}
\end{equation*}


Thus, the evolution equation for the probability density in the general case is
\begin{equation*}
\frac{\partial \rho (x,t)}{\partial t} = - \nabla \cdot (\rho \upsilon) = - \nabla \cdot \left(\frac{\rho \nabla S}{m} - D \nabla \rho \right) = - \frac{\rho}{m} \nabla^2 S - \frac{1}{m} \nabla \rho \cdot \nabla S + D \nabla^2 \rho \tag{10} 
\end{equation*}

This has second spatial derivatives on the right. But it is not an equation for dynamics. It
is explicitly an equation for the statistical ensemble and the quantity $\rho (x, t)$ characterizes
the ensemble. In fact, the structure of the Schr\"{o}dinger equation is of this kind, rather than
of the Hamilton's equation, because the time evolution of quantity is related to the spatial
derivatives of the same quantity. If indeed this is the case, then the psi-function in the
Schr\"{o}dinger equation is definitely not be the matter wave of de Broglie! When all these
aspects are evident, why do we still associate $\psi$ with a matter wave and such a wave with a
particle? We shouldn't, as the rest of the discussion will prove.

The probability density $\rho$ in the equation 8 is of course a positive quantity. Therefore,
$\rho = \psi \psi^{\ast}$ where $\psi = Ae^{i\phi}$ is complex and single valued, with $A$ positive. Then, $+\sqrt{\rho} = A$.
We call $A$ the probability amplitude. The equation 8 is equivalent to
\begin{equation*}
\frac{\partial A}{\partial t} = \frac{A}{2} \nabla \cdot \upsilon - \nabla A \cdot \upsilon \tag{11}
\end{equation*}

A diffusive term $D \nabla^2 A$ can be added to this as well, keeping the linear nature of the equation,
but its justification for the ensemble of single particle dynamics comes from another quarter,
to be explained later.

But
\begin{equation*}
\frac{\partial A}{\partial t} = \frac{\partial \psi e^{-i \phi}}{\partial t} = e^{-i \phi} \frac{\partial \psi}{\partial t} -  i \psi e^{-i \phi} \frac{\partial \phi}{\partial t}  \tag{12}
\end{equation*}

Therefore, the time evolution equation for either $A$ or $\psi$, both representing the probability
density, also contains the time derivative of the dummy phase $\phi$. However, irrespective
of this fact, $\psi$ here is the (complex) square root of the ensemble probability density and
$\partial \psi /\partial t$ is its time evolution, given by the continuity equation. Its association with a single
particle is only in the sense of a probability density (the physical quantities associated with
the particle are associated with this probability density). Anticipating the result that this
$\psi (x, t) = Ae^{i \phi}$ is indeed the wavefunction of the Schr\"{o}dinger equation, we establish the first
rigorous relation on our journey, straight from the definition.
\begin{equation*}
\psi (x, t) \psi (x, t)^{\ast} = \rho(x, t) \tag{13}
\end{equation*}

So, Born's rule connecting the wavefunction and the probability density is exact. It does
not depend on what the dummy phase physically is.

As the next step, I show that, just as $A(x, t) = + \sqrt{\rho}$, \textit{the function $\psi (x, t)$ obeys the
continuity equation for the probability density, when the system is a dynamical system}, i.e.,
when the ensemble obeys the equation of dynamics. I emphasize that the function $\psi(x, t)$
does not represent a single particle or single dynamical history. Note than $A$ and $\psi$ are
dimensionless, as $\rho$ is. Writing $\phi = S/\varepsilon$,
\begin{align*}
\frac{\partial \psi}{\partial t} & = e^{i S/\varepsilon} \frac{\partial A}{\partial t} + \frac{i}{\varepsilon} \psi \frac{\partial S}{\partial t} \\
\frac{\partial \psi}{\partial t} - \frac{i}{\varepsilon} \psi \frac{\partial S}{\partial t}() & = e^{i S  / \varepsilon} \left(- \frac{A}{2} \nabla \cdot \upsilon - \nabla A \cdot \upsilon + D \nabla^2 A \right) \tag{14}
\end{align*}

The spatial derivatives of $\psi$ are

%%%%%%%% 
