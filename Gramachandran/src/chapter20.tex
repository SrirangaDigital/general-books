\chapter[Deuteron Photodisintegration and its application to Astrophysics]{Deuteron Photodisintegration and its application to Astrophysics}\label{chap20}

\Authorline{S. P. Shilpashree}

\begin{center}
School of Engineering and Technology,\\ 
CHRIST (Deemed to be University)\\
Bangalore 
\end{center}

I am indeed honoured to contribute to ``Prof. G. Ramachandran Memorial Volume''. 

I met Late. Prof. G Ramachandran at Indian Institute of Astrophysics during Summer project program in the year 2003. The interactions and research discussions during this time, motivated me to take up research career. His passion for science and his remarkable insights in Physics has enriched my growth as a student. His involvement with his originality has nourished my intellectual maturity that I will benefit from, for a long time to come. I am grateful to Late. Prof. G. Ramachandran for his motivation, advice and guidance from the very early stage of my research career. I would sincerely acknowledge his contributions of time, ideas and immense knowledge. Above all and the most needed, he provided me unflinching encouragement and support in various ways. His truly scientist intuition has inspired me and enriched my growth as a student and a researcher. He encouraged me to not only grow as a theoretical physicist but also as an independent thinker. I am deeply indebted to him.

I  recall with great pleasure the numerous discussions we had at his house on various research topics. Even after the award of my PhD degree, Prof. GR constantly used to motivate me to think on various research problems. I started working as a Research Trainee with him when he was at Indian Institute of Astrophysics. During the training period we started working on a problem, photodisintegration of deuterons and its application in the field of Astrophysics. I would like present here the summary of the work that I did with Late. Prof. Ramachandran.

As the Universe cools from an astounding $10^{32}K$ to temperatures $\approx 10^9K$, in the `first three minutes', four light nuclei viz., $^2H$, $^3He$, $^4He$ and $^7Li$ are produced in significant amounts \cite{key1,key2}, which depend on the baryon density $\Omega_B$. The yield of $^4He$ is the largest and is known to a theoretical uncertainty of less than $0.1\%$ \cite{key3}. At this accuracy, the uncertainty in the abundance is dominated by the experimentaluncertainties in the neutron life time. The recent precision determinations \cite{key4} of neutron life time using Penner trap techniques have thus attracted considerable attention, in this context.  The first step in the success story of BBN was in fact taken while establishing the primeval abundance of $^4He$ \cite{key5}.  Measurements of $^3He$ were also carried out in the solar wind, meteorites and lunar soil \cite{key6}. These  provide constraints \cite{key7} on the deuterium abundance. Based on solar wind and meteoritic measurements \cite{key8}, Reeves, Audouze, Fowler and Schramm \cite{key9} emphasized the cosmological origin of deuterium. Therefore, the detection of deuterium provided an important evidence in favour of BBN.

The number of deuterons produced in reactions like $^3He(\gamma,p)D$ or $^3H(\gamma,n)D$ or $^7Li(\gamma,^4He)D$ in the 'first 3 minutes' are small by 8 or 9 orders of magnitude \cite{key10} as compared to those produced through $p(n,\gamma)D$. Interstellar measurements \cite{key11} made by the Copernicus Satellite and the conclusive  argument \cite{key12} which showed that no realistic astrophysical process could produce significant deuterium led to the realization that premieval deuterium is burnt to $^3He$. Determination of the deuterium abundances has been notoriously difficult and plagued by large uncertainties, since deuterium with a small binding energy of 2.226 MeV is fragile and most of it gets destroyed in stellar interiors with high temperatures, even before the stars reach the main sequence.

 `The ratio of primordial abundance of deuterium to that observed to day could be any where between 1 and 50' \cite{key10}.
 
The estimates depend, of course, on the input nuclear rates \cite{key13}. The predicted primordial abundances of the light elements have been used to constrain the number $N_\nu$ of light neutrino species \cite{key1}. This number was confirmed experimentally \cite{key14}, by measuring the cross section of the single photon events in positron electron collisions near the $Z^0$ resonance. 

The primordial nucleosynthesis starts with the production of deuterium through
$$
n+p \to D+ \gamma
$$

Considerable interest centers on the determination of the primordial deuterium abundance, not only because it facilitates, in turn, the accurate predictions of the abundances of $^3He$, $^4He$ and $^7Li$, but also because it pins down the primordial baryon density, since it varies sharply with  the density. It is therefore referred to as the BARYOMETER.

The measurements of deuterium abundance in high red shift quasar absorption systems \cite{key15} reduce these uncertainties. With corresponding advances in $He$ and $Li$ observations, a `precision era for BBN' is said to have dawned \cite{key16},when an estimate for the density was given as $(4.0 \pm 0.8) 10^{-31}$ g cm$^{-3}$ or as a fraction of critical density $\Omega_B \;h^{2} = (0.022\pm 0.004)$, where $h=0.72\pm 0.08$ denotes the Hubble parameter.   In an effort to sharpen the predictions of BBN, Burles et al \cite{key17} have observed,
"Our method breaks down for the process $n + p \to d+\gamma$....BBN" and identified the 
range for the neutron kinetic energy E$_n$ as between 25 to 200 keV in c.m. frame at 
which input data is required. Their estimate for baryon density  is 
$\Omega_B \;h^{2} = (0.019\pm 0.0024)$. 

It has been remarked:
".. as the observational uncertainties shrink, the uncertainties on the  calculated 
abundances begin to dominate....." \cite {KMN2000}. Along with  developments 
in astronomical observations, precise laboratory measurements can be invoked to remove 
crucial ambiguities in nuclear  physics input parameters to sharpen the theoretical 
predictions in the astrophysical context. 


\begin{thebibliography}{99}
\bibitem{key1} P. J. E. Peebles, Phys. Rev. Lett. {\bf 16} (1966) 410
\bibitem{key2} R. V. Wagoner, W. A. Fowler and F. Hoyle, Astrophys. J. {\bf 148} (1967) 3 
\bibitem{key3} R. E. Lopez and M. S. Turner, Phys. Rev. D {\bf 59} (1999) 103502
\bibitem{key4} G. J. Mathews, T. Kajino and T. Shima,arXiv: astro-ph/0408523 v2 (2004)

A. Serebrov et al, Phys. Lett. B {\bf 605} (2005) 72
\bibitem{key5} C. R. O'Dell, M. Peimbert and T. D. Kinman, Astrophys. J. {\bf 140} (1964) 119 

F. Hoyle and R. J. Tayler, Nature {\bf 203} (1964) 1108
\bibitem{key6} J. Geiss, in Origin and Evolution of the Elements, edited by N. Prantzos, E. Vangioni-Flam and M. Casse (Cambridge University Press, Cambridge, 1993) p. 89
\bibitem{key7} N. Hata et al, Phys. Rev. Lett. {\bf 75} (1995) 3977
\bibitem{key8} D. C. Black, Nature (London) {\bf 234} (1971) 148 

J. Geiss and H. Reeves, Astron. Astrophys. {\bf 18} (1972) 126
\bibitem{key9} H. Reeves, F. Audouze, W. A. Fowler and D. N. Schramm Astrophys. J. {\bf 179}(1973) 909
\bibitem{key10} M. S. Smith, L. H. Kawano and R. A. Malaney, Astrophys. J. Suppl. Ser. {bf 85} (1993) 219
\bibitem{key11} J. Rogerson and D. York Astrophys. J. {\bf 186} (1973) L95
\bibitem{key12} R. I. Epstein et al, Nature (London) {\bf 263} (1976) 198
\bibitem{key13} E. M. Burbidge, G. R. Burbidge, W. A. Fowler and F. Hoyle, Rev. Mod. Phys. {\bf 29} (1957) 547

N. Hata et al., Phys. Rev. Lett. {\bf 75} (1995) 3977

D. N. Schramm and M. S. Turner, Rev. Mod. Phys. {\bf 70} (1998) 303

K. Langanke and M. Wiescher, Rep. Prog. Phys. {\bf 64} (2001) 1657
\bibitem{key14} M. Acciarri et al, L3 Collaboartion Phys. Lett. B {\bf 437} (1998) 199
\bibitem{key15} S. Burles and D. Tytler, Astrophys. J. {\bf 499} (1998) 699

S. Burles and D. Tytler, Astrophys. J. {\bf 507} (1998) 732
\bibitem{key16} D. N. Schramm and M. S. Turner, Rev. Mod. Phys. {\bf 70} (1998) 303
\bibitem{key17} S. Burles, K. M. Nollett, J. W. Truran and M. S. Turner, Phys. Rev. Lett. {\bf 82} (1999) 4176
\bibitem{key18} K. M. Nollett and S. Burles, Phys. Rev. D {\bf 61} (2000) 123505
\bibitem{key19} E. Fermi, Recera Scient. {\bf 7} (1936) 13

I. E. Amaldi and Fermi, Phys. Rev. {\bf 50} (1936) 899
\bibitem{key20} T. S. Suzuki, Y. Nagai, T. Shima, T. Kikuchi, H. Sato, T. Kii and M. Igashira, Astrophys. J. {\bf 439} (1995) L59
\bibitem{key21} Y. Nagai et al., Phys. Rev. C {\bf 56} (1997) 3173
\bibitem{key22} A. E. Cox, S. A. R Wynchank and C. H. Collie, Nucl. Phys. {\bf 74} (1965) 497

D. Cokinos and E. Melkonian, Phys. Rev. C {\bf 15} (1977) 1636
\bibitem{key23} N. Austern, Phys. Rev. {\bf 92} (1953) 670

N. Austern and E. Rost, Phys. Rev. {\bf 117} (1960) 1506
\bibitem{key24} G. Breit and M. L. Rustgi, Nucl. Phys. {\bf A161} (1971) 337
\bibitem{key25} D. O. Riska and G. E. Brown, Phys. Lett. B {\bf 38} (1972) 193
\bibitem{key26} T. Sato, M. Niwa and H. Ohtsubo, in Proc. of International Symp on Weak and Electromagnetic Interactions in Nuclei Eds. H. Ejiri, T. Kishimaw and T. Sato (World Scientific, Singapore 1995) p 488
\bibitem{key27} G.R. Bishop et al., Phys. Rev. {\bf 80} (1950) 211
\bibitem{key28} H. Arenh$\ddot{\rm o}$vel and M. Sanzone, Photodisintegration of the deuteron. A review of theory and experiment (Springer-Verlag, Berlin,1991)
\bibitem{key29} R. W. Jewell, W. John, J. E. Sherwood and D. H. White, Phys. Rev. {\bf 139B} (1965) 71

L. Rustgi, Reeta Vyas and Manoj Chopra, Phys. Rev. Lett. {\bf 50} (1983) 236
\bibitem{key30} R. J. Holt, K. Stephenson and J. R. Specht, Phys. Rev. Lett. {\bf 50} (1983) 577
\bibitem{key31} J. P. Soderstrum and L. D. Knutson, Phys. Rev. C {\bf 35} (1987) 1246
\bibitem{key32} E. Hadjimichael, Phys. Lett. {\bf 46B}, (1973) 147

H. Arenh$\ddot{\rm o}$vel, N. Fabian and H. G. Miller, Phys. Lett. {\bf 52 B} (1974) 303
\bibitem{key33} E. A. Kolomenskii  et al., Yad. Fiz. {\bf 25} (1977) 233
\bibitem{key34} V. A. Vesna  et al., Nucl. Phys. A {\bf 352} (1981) 181
\bibitem{key35} I. I. Grach and M. Zh. Shmatikev, Yad. Fiz. {\bf 45} (1987) 933
\bibitem{key36} A. P. Burichenko and I. B. Khriplovich, Nucl. Phys. A {\bf 515} (1990) 139
\bibitem{key37} A. N. Bazhenov et al., Phys. Lett. B {\bf 289} (1992) 17
\bibitem{key38} G. Ramachandran, P. N. Deepak and S. Prasanna Kumar, J. Phys. G: Nucl. Part. Phys. {\bf 29} (2003) L45
\bibitem{key39} T. M. M$\ddot{\rm u}$ller et al., Nucl. Inst. Meth. A {\bf 440} (2000) 736
\bibitem{key40} J.- W. Chen, G. Rupak and M. J. Savage, Phys. Lett. B {\bf 464} (1999)1
\bibitem{key41} T.-S. Park, K. Kubodera, D.-D. Min and M. Rho, Phys. Lett. B {\bf 472} (2000) 232
\bibitem{key42} J.-W. Chen and M. J. Savage, Phys. Rev. C {\bf 60} (1999) 065205
\bibitem{key43} G. Rupak, Nucl. Phys. A {\bf 678} (2000) 405
\bibitem{key44} S. Ando et al, arXiv: nucl-th/0511074 v1 28 Nov 2005 
\bibitem{key45}E. C. Schreiber  et al., Phys. Rev. C {\bf 61} (2000) 061604(R)
\bibitem{key46} W. Tornow  et al., Mod. Phys. Lett. A {\bf 18} (2003) 282
\bibitem{key47} M. W. Ahmed et al. 2008 {\it Phys. Rev.} C{\bf 77}, 044005.
\bibitem{key48} M. A. Blackston et al. 2008 {\it Phys. Rev.} C {\bf 78} 034003.
\bibitem{key49} H. R. Weller  et al,  At. Data Nucl. Data Tables {\bf 50} (1992) 29
\bibitem{key50} G. Ramachandran and S. P. Shilpashree, Phys. Rev. C. {\bf 74} (2006) 052801(R)
\bibitem{key51} M. E. Rose 1957 Elementary Theory of Angular momentum (New York: John Wiley)
\bibitem{key52} G. Ramachandran, Yee Yee Oo and S. P. Shilpashree,

J. Phys. G: Nucl. Part. Phys. {\bf 32} (2006) B17
\bibitem{53}  G. Ramachandran and M. S. Vidya Phys Rev. C {\bf 56} (1997) R12

G. Ramachandran and M. S. Vidya Invited talk in Proc. DAE Symp. on Nucl. Phys. ed V. M. Datar and S. Kumar vol 39A (1996) p 47
\bibitem{key54} G. Ramachandran and S. P. Shilpashree 2006 {\it Phys. Rev. C}  {\bf 74} 052801(R)
\bibitem{key55} Blaine E. Norum, Private Communication (2006)
\bibitem{key56} Bradley David Sawatzky, Ph.D. Thesis (2005) (unpublished)
\end{thebibliography}
