\chapter[Deuteron Photodisintegration and its application to Astrophysics]{Deuteron Photodisintegration and its application to Astrophysics}\label{chap20}

\Authorline{S. P. Shilpashree}

\begin{center}
School of Engineering and Technology,\\ 
CHRIST (Deemed to be University)\\
Bangalore 
\end{center}

I am indeed honoured to contribute to ``Prof. G. Ramachandran Memorial Volume''. 

I met Late. Prof. G Ramachandran at Indian Institute of Astrophysics during Summer project program in the year 2003. The interactions and research discussions during this time, motivated me to take up research career. His passion for science and his remarkable insights in Physics has enriched my growth as a student. His involvement with his originality has nourished my intellectual maturity that I will benefit from, for a long time to come. I am grateful to Late. Prof. G. Ramachandran for his motivation, advice and guidance from the very early stage of my research career. I would sincerely acknowledge his contributions of time, ideas and immense knowledge. Above all and the most needed, he provided me unflinching encouragement and support in various ways. His truly scientist intuition has inspired me and enriched my growth as a student and a researcher. He encouraged me to not only grow as a theoretical physicist but also as an independent thinker. I am deeply indebted to him.

I  recall with great pleasure the numerous discussions we had at his house on various research topics. Even after the award of my PhD degree, Prof. GR constantly used to motivate me to think on various research problems. I started working as a Research Trainee with him when he was at Indian Institute of Astrophysics. During the training period we started working on a problem, photodisintegration of deuterons and its application in the field of Astrophysics. I would like present here the summary of the work that I did with Late. Prof. Ramachandran.

As the Universe cools from an astounding $10^{32}K$ to temperatures $\approx 10^9K$, in the `first three minutes', four light nuclei viz., $^2H$, $^3He$, $^4He$ and $^7Li$ are produced in significant amounts \cite{key1,key2}, which depend on the baryon density $\Omega_B$. The yield of $^4He$ is the largest and is known to a theoretical uncertainty of less than $0.1\%$ \cite{key3}. At this accuracy, the uncertainty in the abundance is dominated by the experimentaluncertainties in the neutron life time. The recent precision determinations \cite{key4} of neutron life time using Penner trap techniques have thus attracted considerable attention, in this context.  The first step in the success story of BBN was in fact taken while establishing the primeval abundance of $^4He$ \cite{key5}.  Measurements of $^3He$ were also carried out in the solar wind, meteorites and lunar soil \cite{key6}. These  provide constraints \cite{key7} on the deuterium abundance. Based on solar wind and meteoritic measurements \cite{key8}, Reeves, Audouze, Fowler and Schramm \cite{key9} emphasized the cosmological origin of deuterium. Therefore, the detection of deuterium provided an important evidence in favour of BBN.

The number of deuterons produced in reactions like $^3He(\gamma,p)D$ or $^3H(\gamma,n)D$ or $^7Li(\gamma,^4He)D$ in the 'first 3 minutes' are small by 8 or 9 orders of magnitude \cite{key10} as compared to those produced through $p(n,\gamma)D$. Interstellar measurements \cite{key11} made by the Copernicus Satellite and the conclusive  argument \cite{key12} which showed that no realistic astrophysical process could produce significant deuterium led to the realization that premieval deuterium is burnt to $^3He$. Determination of the deuterium abundances has been notoriously difficult and plagued by large uncertainties, since deuterium with a small binding energy of 2.226 MeV is fragile and most of it gets destroyed in stellar interiors with high temperatures, even before the stars reach the main sequence.

 `The ratio of primordial abundance of deuterium to that observed to day could be any where between 1 and 50' \cite{key10}.
 
The estimates depend, of course, on the input nuclear rates \cite{key13}. The predicted primordial abundances of the light elements have been used to constrain the number $N_\nu$ of light neutrino species \cite{key1}. This number was confirmed experimentally \cite{key14}, by measuring the cross section of the single photon events in positron electron collisions near the $Z^0$ resonance. 

The primordial nucleosynthesis starts with the production of deuterium through
$$
n+p \to D+ \gamma
$$

Considerable interest centers on the determination of the primordial deuterium abundance, not only because it facilitates, in turn, the accurate predictions of the abundances of $^3He$, $^4He$ and $^7Li$, but also because it pins down the primordial baryon density, since it varies sharply with  the density. It is therefore referred to as the BARYOMETER.

The measurements of deuterium abundance in high red shift quasar absorption systems \cite{key15} reduce these uncertainties. With corresponding advances in $He$ and $Li$ observations, a `precision era for BBN' is said to have dawned \cite{key16},when an estimate for the density was given as $(4.0 \pm 0.8) 10^{-31}$ g cm$^{-3}$ or as a fraction of critical density $\Omega_B \;h^{2} = (0.022\pm 0.004)$, where $h=0.72\pm 0.08$ denotes the Hubble parameter.   In an effort to sharpen the predictions of BBN, Burles et al \cite{key17} have observed, "Our method breaks down for the process $n + p \to d+\gamma$....BBN" and identified the range for the neutron kinetic energy $E_n$ as between 25 to 200 keV in c.m. frame at which input data is required. Their estimate for baryon density  is $\Omega_B \;h^{2} = (0.019\pm 0.0024)$. 

It has been remarked: ``.. as the observational uncertainties shrink, the uncertainties on the  calculated abundances begin to dominate....." \cite {key18}. Along with  developments in astronomical observations, precise laboratory measurements can be invoked to remove crucial ambiguities in nuclear  physics input parameters to sharpen the theoretical predictions in the astrophysical context. 

Although laboratory  measurements on n-p fusion date back to 1936 by Fermi and collaborators \cite{key19}, the experiments were done employing thermal neutrons for which $E_n$ is of order $10^{-6}$, where as experiments on  photodisintegration (which is related to the fusion reaction by time reversal) could also claim the same antiquity with lab photon energies $> 2.62$ MeV which corresponds to $E_n = 189 keV$. However, it had not been possible  for a long time to measure the cross section atastrophysical energies due to the tendency of the neutrons to thermalize atlow energies. The first cross section measurements between 20 keV and 64 keVhave been reported in 1995 by Suzuki et al \cite{key20} and subsequently byNagai et al \cite{key21} at 550 keV. 

The physics here is highly interesting.The thermal neutron cross section \cite{key22} has traditionally been interpretted in terms of the dominant isovector $M1$ amplitude for radiative capture from the initial $^1S_0$ state of the $n-p$ system in the continuum. The first theoretical calculations \cite{key23} based on potential models led to a $10\%$ discrepancy with the experimental measurements. Breit and Rustgi \cite{key24} proposed a polarized target-beam-test to detect the possibility of radiative capture from the initial $^3S_1$ state as well, which can take place through isoscalar $M1$ and possibly also isoscalar $E2$ transitions. However, the surprising accuracy with which Riska and Brown \cite{key25} explained the $10\%$ discrepancy by including Meson Exchange Current (MEC) contributions, set the trend for theoretical discussion in later years. It has been noted by Nagai et al \cite{key21} that the measured cross section is in agreement with the theoretical calculations by Sato et al \cite{key26} including MEC's, isobar currents and pair currents. They have also pointed out that ``the theory is in good agreement with the cross section measured for neutrons above 14 MeV, but it deviates by about 15$\%$ from the measured cross section of the $D(\gamma,n)p$ reaction by using the $\gamma$ ray of between 2.5 and 2.75 MeV \cite{key27}, corresponding to neutron energies of 550 and 1080 keV" \cite{key21}. Experimental studies on photodisintegration of the deuteron for photon energies from 2.62 MeV and above is well documented  \cite{28}. The cross section at 2.62 MeV is $1.30\pm 0.029 mb$ which increases slowly to $2.430 \pm 0.17 mb$ at 4.45 MeV and starts slowly decreasing with energy thereafter. The disintegration process is dominantly through $E1$ transitions leading to final triplet $P$-states of the $n-p$ system in the continuum. Apart from the 15$\%$ discrepancy with the measured cross section~\cite{key27} noted by Nagai et al., \cite{key21}, the measured angular distribution and neutron polarization at photon energy of 2.75 MeV \cite{key29} and in the range 6 to 13 MeV \cite{key30} were found to be in disagreement with theoretical predictions which included the meson exchange currents. Measurements of the analyzing power \cite{key31} in $p(\vec n, \gamma)d$ at neutron energies of 6.0 and 13.43 MeV were consistent with the measurements \cite{key30} and theoretical calculations \cite{key32} showed that meson exchange currents produce a significant change but the effect is to move the theoretical curve to more negative values, thus making the discrepancy between theory and experiment more pronounced. An observable which is sensitive to the presence of isoscalar $M1$ and $E2$ transitions from the triplet $S$-state is the circular polarization of the emitted radiation with initially polarized neutrons. The first measurement \cite{key33} to detect the presence of isoscalar amplitudes was not quite encouraging but a subsequent measurement \cite{key34} yielded a value $P_\gamma= -(2.29 \pm 0.9)\times 10^{-3}$. An attempt \cite{key35} to explain the large measured value by introducing a six quark admixture in the deuteron wave function led however to a disagreement with the well known deuteron magnetic moment. Later calculations \cite{key36} in the zero range approximation and the wavefunction for a Reid soft core potential led to a theoretical prediction $P_\gamma $ of the order of $-1.1 \times 10^{-3}$ with an estimated accuracy of 25$\%$. The measured value \cite{key37} of $P_\gamma =-(1.5 \pm 0.3)\times 10^{-3}$ is in reasonable agreement with the theoretical calculation \cite{key36}. The importance of measuring the photon polarization with initially polarized neutrons incident on a polarized proton target has been pointed out \cite{key38}. When the initial preparation of the neutron and proton polarizations $P(n)$ and $P(p)$ are such that they are either opposite to each other or orthogonal to each other, the interference of the small isoscalar amplitudes with the large isovector amplitude could substantially contribute to the observable photon polarization.

Anticipating the experimental results of polarized thermal neutron capture by polarized protons by $M\ddot{\rm u}$ller et al., \cite{key39}, the possibility of the initial $^3S_1$ state contributions at thermal neutron energies was discussed using two different versions of effective field theory \cite{key40,key41} Although the measured value of $(1.0\pm 2.5)\times 10^{-4}$ for the $\gamma$ anisotropy $\eta$ was not sufficiently sensitive to distinguish between the two theoretical predictions, we may use equation \eqref{chap20-eq2} of M$\ddot{\rm u}$ller et al., \cite{key39} to estimate the ratio $R$ of the triplet to singlet capture cross sections to be $1.202 \times 10^{-3} $. If we multiply $R$ by the well-known cross section  \cite{key22}, we get an estimate of 401.7 $\mu b$ for the $^3S_1$ contribution to the cross section at thermal neutron energies. Quite surprisingly, this number is of the same order as the measured cross sections for capture at astrophysical energies of 20, 40 and 64 keV \cite{key20}. In fact, it is even larger by a factor of 10 than the measured cross section at 550 keV \cite{key21}. This raises an open question as to what could possibly be the ratio $R$ at astrophysical energies relevant to BBN. 

The influential paper of Burles, Nollett, Truran and Turner \cite{key17} has inspired several theoretical \cite{key42,key43} as well as experimental \cite{key45,key46,key47,key48} studies. Since photodisintegration of the deuteron is well documented \cite{key28} for photon energies of 2.62 MeV and above and is known to be dominated by $E1$ transitions leading to final triplet  $P$-states in the $n-p$ continuum, these studies~\cite{key42,key43,key44,key45,key46} were  motivated towards the determination of the relative $M1$ and $E1$ contributions to the process at astrophysical energies. The experiment \cite{key45} was concerned with the measurement of the near threshold beam analyzing power using for the first time a laser based $\gamma$-ray source at 3.58 MeV. This was followed by measurements at seven $\gamma$-ray energies between 2.39 and 4.05 MeV \cite{key46}. These measurements with 100$\%$ linearly polarized photons have been analyzed, making several  simplifying assumptions viz., 
\begin{enumerate}[a)]
\item only $l=0,1$ partial waves were considered in the final state  due to the low energies involved,
\item of the allowed two $M1$ and four $E1$ transitions, the isoscalar $E1$ leading to $^1P_1$ is set to zero,
\item the isoscalar $M1$ term leading to $^3S_1$ is neglected, using the traditional agruments for its supression,
\item the three isovector $E1$ terms were combined to form a single $P$-wave amplitude, using the theoretical formalism \cite{key49}, where $M1$ and $E1$ contributions were calculated separately.
\end{enumerate}

The reaction $d(\vec \gamma, n)p$ is studied theoretically, using a model independent formalism, without making any simplifying assumptions  except that only the dipole transitions are considered with $l=0,1$  partial waves in the final state \cite{key50}. We choose the linearly polarized photon momentum ${\bf k}$ in c.m. frame to be along z-axis and with the linear polarization along the x-axis of a right handed cartesian coordinate system and the neutron momentum ${\bf p}$ in c.m. frame to have polar coordinates $(p,\theta,\phi)$, following \cite{key45}. The left and right circular states of photon polarization are denoted by $\mu =\pm 1$  following Rose \cite{key51}.  We use natural units, $\hbar = c=1$.  The unpolarized differential cross section for the reaction $d(\gamma, n)p$, in c.m frame at energy $E$ is given by 
\makeatletter
\counterwithout{equation}{chapter}
\makeatother
\begin{eqnarray}
{d\sigma_0 \over d\Omega}&=&{1 \over 6} { E_n E_p E_d |{\bf p}| \over (2\pi E)^2 } \sum_{\mu=-1,1} Tr({\bf T}(\mu) {\bf T}^\dagger(\mu)) \nonumber \\ 
 &=& {1 \over 6} \sum_{\mu=-1,1} Tr[{\bf M}(\mu) {\bf M}^\dagger(\mu)], \label{chap20-eq1}
\end{eqnarray}
where $Tr$ denotes the trace or spur and ${\bf T}(\mu)$ denotes the on-energy-shell matrix for $d(\vec \gamma, n)p$,  when photons are in the polarized state ${\bf u}_\mu$. The c.m. energies of the neutron, proton and deuteron are denoted respectively by $E_n$, $E_p$ and  $E_d$.  Following \cite{key52}, we express 
\begin{equation}
{\bf M}(\mu) = \sum_{s=0}^1 \sum_{\lambda = |s-1|}^{s+1} (S^\lambda(s,1) \cdot {\mathcal F}^\lambda(s,\mu)), \label{chap20-eq2}
\end{equation}
in terms of irreducible tensor operators, $S^\lambda_{\nu}(s,1)$ of rank $\lambda$ in hadron spin space \cite{key53} connecting the initial spin 1 state of the deuteron with the final singlet and triplet states, $s=0,1$ of  the $n-p$ system in the continuum.  The differential cross section relevant to \cite{key45,key46} for $d(\vec \gamma, n)p$ with linearly  polarized photons is given, in c.m. frame, by  
\begin{eqnarray}
{d\sigma \over d\Omega} ={1 \over 6} Tr {\bf M M^\dagger}, \label{chap20-eq3}
\end{eqnarray}
where
\begin{equation} 
{\bf M= M}(+1)+{\bf M}(-1).\label{chap20-eq4}
\end{equation}
Using known properties \cite{key52} of the irreducible tensor operators and
standard Racah algebra, we have
\begin{equation}
{d\sigma \over d\Omega}={2\pi^2 \over 6} [a+b \sin^2\theta (1+\cos 2\phi) -c\cos\theta],\label{chap20-eq5}
\end{equation}
where 
\begin{eqnarray}
a &=& \big[ 8 |M1_v|^2+ 24|M1_s|^2 + 36|E1_s|^2 + 8 |E1_v^{j=0}|^2 \nonumber \\ 
	&+&18 |E1_v^{j=1}|^2 + 26 |E1_v^{j=2}|^2 -16 Re(E1_v^{j=0} E1_v^{j=2*}) \nonumber \\ 
	&-&36 Re(E1_v^{j=1} E1_v^{j=2*}) \big], \label{chap20-eq6}
\end{eqnarray}
\begin{eqnarray}
 b &=& \big[ 9 |E1_v^{j=1}|^2+21 |E1_v^{j=2}|^2 +24 Re(E1_v^{j=0} E1_v^{j=2*}) \nonumber \\
&+&54 Re(E1_v^{j=1} E1_v^{j=2*})-18|E1_s|^2 \big], \label{chap20-eq7}
\end{eqnarray}
and
\begin{equation}
c = 4\sqrt 6 Re \left[(2 E1_v^{j=0} + 3 E1_v^{j=1} -5 E1_v^{j=2})M1_s^*\right] \label{chap20-eq8}.
\end{equation}
It is readily seen from  eq. \eqref{chap20-eq8} that the third term $c \cos\theta$ in eq. \eqref{chap20-eq5} arises due to the interference of the $M1_s$ amplitude with the $E1_v$ amplitudes. This term does not find place in \cite{key49}, since the calculations there have been carried out separately for the $E1$ and $M1$ transitions.  This work was published in \cite{key54}.

After our paper was published, we were informed \cite{key55} by Prof. Blaine E. Norum of Virginia University that their recent measurements \cite{key56} with linearly polarized photons does indeed show a front back polar angle asymmetry, whose magnitude systematically rises as the photon energy falls towards the threshold. This observation supports our expectation that the isoscalar amplitude could be substantial at astrophysical energies. A theoretical understanding of the reaction together with such precise measurements in  the energy region of astrophysical interest  would certainly go a long way in reducing the uncertainties associated with the determination of the primordial deuteron abundance. 

\begin{thebibliography}{99}
\bibitem{key1} P. J. E. Peebles, Phys. Rev. Lett. {\bf 16} (1966) 410
\bibitem{key2} R. V. Wagoner, W. A. Fowler and F. Hoyle, Astrophys. J. {\bf 148} (1967) 3 
\bibitem{key3} R. E. Lopez and M. S. Turner, Phys. Rev. D {\bf 59} (1999) 103502
\bibitem{key4} G. J. Mathews, T. Kajino and T. Shima,arXiv: astro-ph/0408523 v2 (2004)

A. Serebrov et al, Phys. Lett. B {\bf 605} (2005) 72
\bibitem{key5} C. R. O'Dell, M. Peimbert and T. D. Kinman, Astrophys. J. {\bf 140} (1964) 119 

F. Hoyle and R. J. Tayler, Nature {\bf 203} (1964) 1108
\bibitem{key6} J. Geiss, in Origin and Evolution of the Elements, edited by N. Prantzos, E. Vangioni-Flam and M. Casse (Cambridge University Press, Cambridge, 1993) p. 89
\bibitem{key7} N. Hata et al, Phys. Rev. Lett. {\bf 75} (1995) 3977
\bibitem{key8} D. C. Black, Nature (London) {\bf 234} (1971) 148 

J. Geiss and H. Reeves, Astron. Astrophys. {\bf 18} (1972) 126
\bibitem{key9} H. Reeves, F. Audouze, W. A. Fowler and D. N. Schramm Astrophys. J. {\bf 179}(1973) 909
\bibitem{key10} M. S. Smith, L. H. Kawano and R. A. Malaney, Astrophys. J. Suppl. Ser. {bf 85} (1993) 219
\bibitem{key11} J. Rogerson and D. York Astrophys. J. {\bf 186} (1973) L95
\bibitem{key12} R. I. Epstein et al, Nature (London) {\bf 263} (1976) 198
\bibitem{key13} E. M. Burbidge, G. R. Burbidge, W. A. Fowler and F. Hoyle, Rev. Mod. Phys. {\bf 29} (1957) 547

N. Hata et al., Phys. Rev. Lett. {\bf 75} (1995) 3977

D. N. Schramm and M. S. Turner, Rev. Mod. Phys. {\bf 70} (1998) 303

K. Langanke and M. Wiescher, Rep. Prog. Phys. {\bf 64} (2001) 1657
\bibitem{key14} M. Acciarri et al, L3 Collaboartion Phys. Lett. B {\bf 437} (1998) 199
\bibitem{key15} S. Burles and D. Tytler, Astrophys. J. {\bf 499} (1998) 699

S. Burles and D. Tytler, Astrophys. J. {\bf 507} (1998) 732
\bibitem{key16} D. N. Schramm and M. S. Turner, Rev. Mod. Phys. {\bf 70} (1998) 303
\bibitem{key17} S. Burles, K. M. Nollett, J. W. Truran and M. S. Turner, Phys. Rev. Lett. {\bf 82} (1999) 4176
\bibitem{key18} K. M. Nollett and S. Burles, Phys. Rev. D {\bf 61} (2000) 123505
\bibitem{key19} E. Fermi, Recera Scient. {\bf 7} (1936) 13

I. E. Amaldi and Fermi, Phys. Rev. {\bf 50} (1936) 899
\bibitem{key20} T. S. Suzuki, Y. Nagai, T. Shima, T. Kikuchi, H. Sato, T. Kii and M. Igashira, Astrophys. J. {\bf 439} (1995) L59
\bibitem{key21} Y. Nagai et al., Phys. Rev. C {\bf 56} (1997) 3173
\bibitem{key22} A. E. Cox, S. A. R Wynchank and C. H. Collie, Nucl. Phys. {\bf 74} (1965) 497

D. Cokinos and E. Melkonian, Phys. Rev. C {\bf 15} (1977) 1636
\bibitem{key23} N. Austern, Phys. Rev. {\bf 92} (1953) 670

N. Austern and E. Rost, Phys. Rev. {\bf 117} (1960) 1506
\bibitem{key24} G. Breit and M. L. Rustgi, Nucl. Phys. {\bf A161} (1971) 337
\bibitem{key25} D. O. Riska and G. E. Brown, Phys. Lett. B {\bf 38} (1972) 193
\bibitem{key26} T. Sato, M. Niwa and H. Ohtsubo, in Proc. of International Symp on Weak and Electromagnetic Interactions in Nuclei Eds. H. Ejiri, T. Kishimaw and T. Sato (World Scientific, Singapore 1995) p 488
\bibitem{key27} G.R. Bishop et al., Phys. Rev. {\bf 80} (1950) 211
\bibitem{key28} H. Arenh$\ddot{\rm o}$vel and M. Sanzone, Photodisintegration of the deuteron. A review of theory and experiment (Springer-Verlag, Berlin,1991)
\bibitem{key29} R. W. Jewell, W. John, J. E. Sherwood and D. H. White, Phys. Rev. {\bf 139B} (1965) 71

L. Rustgi, Reeta Vyas and Manoj Chopra, Phys. Rev. Lett. {\bf 50} (1983) 236
\bibitem{key30} R. J. Holt, K. Stephenson and J. R. Specht, Phys. Rev. Lett. {\bf 50} (1983) 577
\bibitem{key31} J. P. Soderstrum and L. D. Knutson, Phys. Rev. C {\bf 35} (1987) 1246
\bibitem{key32} E. Hadjimichael, Phys. Lett. {\bf 46B}, (1973) 147

H. Arenh$\ddot{\rm o}$vel, N. Fabian and H. G. Miller, Phys. Lett. {\bf 52 B} (1974) 303
\bibitem{key33} E. A. Kolomenskii  et al., Yad. Fiz. {\bf 25} (1977) 233
\bibitem{key34} V. A. Vesna  et al., Nucl. Phys. A {\bf 352} (1981) 181
\bibitem{key35} I. I. Grach and M. Zh. Shmatikev, Yad. Fiz. {\bf 45} (1987) 933
\bibitem{key36} A. P. Burichenko and I. B. Khriplovich, Nucl. Phys. A {\bf 515} (1990) 139
\bibitem{key37} A. N. Bazhenov et al., Phys. Lett. B {\bf 289} (1992) 17
\bibitem{key38} G. Ramachandran, P. N. Deepak and S. Prasanna Kumar, J. Phys. G: Nucl. Part. Phys. {\bf 29} (2003) L45
\bibitem{key39} T. M. M$\ddot{\rm u}$ller et al., Nucl. Inst. Meth. A {\bf 440} (2000) 736
\bibitem{key40} J.- W. Chen, G. Rupak and M. J. Savage, Phys. Lett. B {\bf 464} (1999)1
\bibitem{key41} T.-S. Park, K. Kubodera, D.-D. Min and M. Rho, Phys. Lett. B {\bf 472} (2000) 232
\bibitem{key42} J.-W. Chen and M. J. Savage, Phys. Rev. C {\bf 60} (1999) 065205
\bibitem{key43} G. Rupak, Nucl. Phys. A {\bf 678} (2000) 405
\bibitem{key44} S. Ando et al, arXiv: nucl-th/0511074 v1 28 Nov 2005 
\bibitem{key45}E. C. Schreiber  et al., Phys. Rev. C {\bf 61} (2000) 061604(R)
\bibitem{key46} W. Tornow  et al., Mod. Phys. Lett. A {\bf 18} (2003) 282
\bibitem{key47} M. W. Ahmed et al. 2008 {\it Phys. Rev.} C{\bf 77}, 044005.
\bibitem{key48} M. A. Blackston et al. 2008 {\it Phys. Rev.} C {\bf 78} 034003.
\bibitem{key49} H. R. Weller  et al,  At. Data Nucl. Data Tables {\bf 50} (1992) 29
\bibitem{key50} G. Ramachandran and S. P. Shilpashree, Phys. Rev. C. {\bf 74} (2006) 052801(R)
\bibitem{key51} M. E. Rose 1957 Elementary Theory of Angular momentum (New York: John Wiley)
\bibitem{key52} G. Ramachandran, Yee Yee Oo and S. P. Shilpashree,

J. Phys. G: Nucl. Part. Phys. {\bf 32} (2006) B17
\bibitem{53}  G. Ramachandran and M. S. Vidya Phys Rev. C {\bf 56} (1997) R12

G. Ramachandran and M. S. Vidya Invited talk in Proc. DAE Symp. on Nucl. Phys. ed V. M. Datar and S. Kumar vol 39A (1996) p 47
\bibitem{key54} G. Ramachandran and S. P. Shilpashree 2006 {\it Phys. Rev. C}  {\bf 74} 052801(R)
\bibitem{key55} Blaine E. Norum, Private Communication (2006)
\bibitem{key56} Bradley David Sawatzky, Ph.D. Thesis (2005) (unpublished)
\end{thebibliography}
