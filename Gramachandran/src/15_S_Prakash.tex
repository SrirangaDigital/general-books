\chapter{GR - A Rare Physicist of a bygone era}\label{chap8}

%\Authorline{N.G.Deshpande// Professor of Physics, University of Oregon}

\begin{center}
\textbf{J. Schwinger}\\
\textbf{\textit{`Feynman brought Quantum Field Theory to the masses'}}
\end{center}

This is a quotation cited in the Chapter on Feynman Diagrams in the popular book on the intimidating subject of Quantum Field Theory by A. Zee. The essence of Schwinger's statement is that space-time pictures called Feynman diagrams, instead of being obstacles, provide an opening window to first-time learners of Quantum Field Theory even though there is a lot more to quantum field theory than Feynman diagrams. In 1990, GR delivered a few lectures on Feynman diagrams to a group of Physics teachers of Undergraduate colleges of Mysore attending the first UGC sponsored refresher course in the Postgraduate Department of Physics, University of Mysore. In hindsight, his attempt to introduce the rudiments of Quantum Field Theory through Feynman diagrams to teachers whose knowledge of Electrodynamics did not go beyond Maxwell's field equations seemed to be in resonance with Schwinger's observation. But it looked na\"{i}ve and utopian to some of us. But in history, we have often seen what is na\"{i}ve today can be common sense tomorrow. Abolition of slavery or rise of Democracy were seen as being impossible until they happened or as Oscar Wilde once said, `Progress is the realisation of Utopias'. GR's attempt seemed to exemplify this. Any interaction with him would leave one with the feeling that mysterious and difficult concepts in Physics could be demystified and he was always willing to walk the extra mile with anyone interested in the subject. GR, who had rubbed shoulders with scientists of eminence like G.N. Ramachandran, C.N.R. Rao and Alladi Ramakrishnan was always willing to teach deeply physical concepts in Quantum Mechanics to even a novice with a cursory knowledge of Group Theory,  Linear Vector Spaces etc., as he generally believed that serious topics could be made reachable by reducing conceptual barriers. In that sense, he was a rare Physicist.

I first met GR in 1973 at the Molecular Biophysics Unit of Indian Institute of Science, Bangalore. GR was a visiting scientist on invitation from Late Professor G.N. Ramachandran (popularly known as GNR). GNR, a pioneer in the field of Protein structure and conformation was nominated for the Nobel Prize (which eluded him) for proposing a triple helical structure for Collagen, a protein and development of the well known Ramachandran plot. Few research students including me developed a warm relationship with GR with time. Sometimes GR used to enlighten us about his work with GNR such as application of Hartree-Fock method or Self-consistent Field Method (SCF) to the Hydrogen molecule by taking into account the electron-electron interaction in the molecules and produce the best possible one-electron wave function for a multi-electron system. During that time, GNR a man of wide interests was working on varied problems, such as the significance of non-planar peptide unit, the possibility of the left-handed structure of DNA, proof of Four colour theorem etc., During a telephonic conversation with GR sometime during 2017, I referred to the insightful elucidation of characteristics of Polar and Axial Vectors in his book, \textit{Quantum theory of Angular Momentum}. Immediately, he recalled his discussion with GNR about L ~\&~ D amino acids (mirror images) and told me that he could clear certain doubts of GNR by giving examples of parity violation in the celebrated experiment of C.S.Wu where the mirror image of the weak interaction is not observed. In his book, he explains how either mirror reflection or inversion (parity) converts a right-handed frame S to a left-handed frame $S'$ The consequence is the direction associated with an axial vector-like Angular momentum. This momentum is different in $S \& S'$ unlike polar vectors (eg. Velocity) whose Cartesian components change with no change in direction under inversion. Thus, the convention for assignment of direction to axial vectors representing spin or angular momentum is essential for understanding parity violation in weak interactions. This book of GR along with his other book on \textit{Vectors, Axial Vectors, Tensors and Spinors} (co-authored with M.S. Vidya and Venkataraya) will be of great help to students in achieving a better understanding of some essential concepts. Before GR could collaborate fruitfully with GNR, he left the Institute to join the Post-graduate Department of Physics, Mysore.  

During his brief stint in the Biophysics Department, I, along with a few of my friends had the benefit of his exposition of basic concepts in Quantum Mechanics, Nuclear Magnetic Resonance (NMR) - an important tool in the study of molecular structures and related topics. Lack of necessary mathematical knowledge proved to be a barrier in assimilating his insights. He always used to say that as you start working on a problem, you can acquire the necessary mathematical skills. Many times, we would walk with him to his house in Malleswaram and savour hot coffee with snacks offered by his wife. In course of time, we discovered that he was deeply religious and orthodox. He would generally come to the department around 2 pm after finishing a fairly elaborate pooja and lunch and stay till late evening. His scientific acumen and religiosity seemed contradictory to some of us who had this notion that science is based on facts whereas religion is all faith with very little factuality. GR was not an exception. In the Institute we were witness to eminent physicist E.C.G. Sudarshan who had joined the Centre for Theoretical Studies (CTS) in the Institute speaking as eloquently about his work on Tachyons (particles that travel faster than light) as on Transcendental Meditation of Mahesh Yogi.  I would also like to mention here that GNR started the Department of Mathematical Philosophy in the Institute during the late seventies. This trend of hard-core theoretical Physicists (including Schrodinger and Heisenberg) who deploy a lot of mathematical abstractions, finding meaning in Religion and Philosophy may be attributed to the abstractive nature of the disciplines. Or a simplistic explanation could be that most people have multiple identities. One also sees many Bengalis who found no contradictions in hero-worship of Tagore, Subash Chandra Bose, Satyajit Ray and Marxists even though Tagore opposed the nationalism of Bose and ideological rigidity of the Left as they do not promote critical thinking. Looking back with nostalgia to the era of sixties, seventies and eighties, the Institute with a fair number of eminent personalities reflected the ethos of the society at large. Possibly, the society at that time was still influenced by the broad thinking, deep commitment and persistent efforts to acquire knowledge for its own sake exhibited by the leaders of Pre and Post Independent India, in different walks of life. This is in sharp contrast to the present-day society where everyone or everything is viewed in binaries - either good or bad, true or false, friend or enemy, national or anti-national etc.; and the middle ground with shades of grey, a result of nuanced thinking is sorely missing.  American novelist F. Scott Fitzgerald once described grey thinkers as people possessing a first-rate mind with the ability to hold two opposing thoughts at the same time, while still retaining the ability to function. One can perhaps say E.C.G.Sudarshan, GNR, and GR were exemplars of grey thinking.

It is difficult to write about GR without his Physics. His presentations were informal, serious and profound without frills or showmanship. His extrapolations of some basic concepts may not be present in any textbook. For example, many of us have this notion that only particles with rotation possess angular momentum about a point. According to GR, one can associate angular momentum about a point (origin) with a particle moving along a straight path.  If the particle is moving away from origin with constant speed, the magnitude of its angular momentum ${\rm r}^{\ast}{\rm p}^{\ast}$sin($\theta$) (r-distance from the origin, p - linear momentum) remains constant (conservation of angular momentum) since any increase in r is compensated by a decrease in $\theta$ and hence sin$\theta$. His lucid exposition of multipole expansion of the potential of arbitrary charge distribution is worthy of mention. After he joined the University of Mysore, the Theoretical Physics group headed by our respected teacher Prof. K.N. Srinivasa Rao gained immense popularity with students. Many bright minds gravitated towards GR and did their PhD with his guidance. My association with him led to my family developing a close relationship with his illustrious students M.V.N. Murthy and V.Ravishankar and we cherish the many intellectually stimulating and happy moments spent with them. Another special characteristic of GR was his propensity to discuss Physics unmindful of time, place and surroundings. Once when I met him near his house in Saraswathipuram in the evening, he talked about some topic, which I forget, for nearly an hour, completely oblivious of the evening traffic. Another instance was during the occasion of the marriage of my daughter Pavitra. When I introduced GR to my son-in-law, Navaneeth and mentioned that he was doing PhD in Theoretical Fluid Mechanics, GR engaged Navaneeth in a serious discussion about his work for about 5 to 6 minutes. This shows that perhaps Physics was in his veins.

G.R had a warm and respectful relationship with my father Prof. K. Srinivasan (Retired), Yuvaraja's College, Mysore. The relationship started the day he landed in Mysore for attending the interview for the faculty position in the University of Mysore. My father received him at the bus stand and took him to Crawford Hall, the venue of the Interview. After the Interview, my father brought him home for lunch, but he refused to have lunch. My parents were a little perplexed and later in the evening before his departure offered him a cup of coffee.  GR refused saying that he does not eat anything outside. My parents used to recall this incident many times later. Quite often, my father would help him in sorting out financial matters connected with the University. My father was greatly appreciative of his academic scholarship and achievements but was greatly disappointed when GR informed him about his rejecting the Chairmanship of the Physics Department. My father felt that GR could have moulded the Department to reach greater heights.  Maybe, the academically oriented GR, like GNR, did not possess the mindset required for tackling the nitty-gritty of day to day administration and lacked the necessary PR skills. After my father’s demise, when I requested him to contribute an article to the memorial volume on my father, he readily agreed and wrote a nice and affectionate article.

Sometime during October 2018, I met him along with my mother and wife, Devaki and gave him the book on my father. He looked worn out and briefly narrated his health-related issues. My last meeting was a few months later with Ravishankar and his former student Dr K.S. Gundu Rao. Though he was not his usual self, he evinced a keen interest in the research activities of Ravishankar. Due to the Corona pandemic, I could not pay my respects to him after his demise and I deeply regret it. This brief write-up is my respects to GR - a rare Physicist of a bygone era.

%~ \begin{tabular}{V{2.5}cp{14.2cm}V{2.5}}
%~ \clineB{1-2}{2.5}
 %~ &\\
%~ \raisebox{-4cm}{\includegraphics{src/figures/authors/ .jpg}} & 

%~ \centerline{\large\bf S.Prakash}

%~ \bigskip
%~ After obtaining PhD from I.I.Sc. in 1981, Dr.S.Prakash joined the Physics Department, J.S.S College of Arts, Commerce & Science, Mysore. He retired from there as a Professor of Physics in 2011. Currently he is leading a retired life at S1, Mana Jardin, Doddakannalli, Bengaluru.\\
%~ &\\ 
%~ \clineB{1-2}{2.5}
%~ \end{tabular}
