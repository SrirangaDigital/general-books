\chapter[Cabibbo angle and universality of weak interactions]{Cabibbo angle and universality of weak interactions}\label{chap21}

\Authorline{G Rajasekaran}
\addtocontents{toc}{\protect\contentsline{section}{{\sl G Rajasekaran}\smallskip}{}}

\authinfo{Institute of Mathematical Sciences, Chennai 600113\\
         and Chennai Mathematical Institute, Siruseri 603103}


         
Nicola Cabibbo (who passed away in September 2010)
made a fundamental contribution to the 
development of High Energy Physics (HEP). It 
goes by the name "Cabibbo angle" and appears to be
a simple parameter. Hence its true significance in the
conceptual development of the theory of HEP, that we now
call Standard Model (SM) of HEP, is not generally appreciated.  
An attempt will be made in this article to explain how and why
the introduction of this angle and its further development 
were in fact very important ingredients in the construction
of the SM.

After the strange particles were discovered in the mid 50's
and the study of their weak decays progressed, it was realized
that their weak decay probabilities were smaller (by an 
order of magintude) than what a simple application of Fermi's
nuclear beta-decay theory extended to strange particle decays
predicted. To be a little more precise, in this calculation
of the decay rates, it is the current x current theory of
Feynman and Gell-Mann (FG) that was used. A few words on 
that theory will be in in order.
 
In 1958, Feynman and Gell-Mann proposed 
their famous current x current form of weak interaction theory
which extended Fermi's theory and made it applicable to all 
forms of weak decays: for instance, beta decay of the neutron and the
nuclei, the decay of the muon, the decay of the pion, the decays of the
strange particles such as K, Lambda etc. This was a beautiful
generalization of Fermi theory and it also included the axial
vector current in addition to the original vector current of
Fermi, in the form of the V-A interaction of Sudarshan 
and Marshak, thus incorporating the maximal parity violation 
discovered by T D Lee, C N Yang, C S Wu and others in 1956.

An important feature of the FG theory was the idea that weak
interactions are universal. Just as the universality of 
electromagnetic interactions implies the equality of the 
charge of the electron and the proton (apart from the sign),
universality was supposed to imply equality of the strength
of the weak interaction. Since the strength of the beta decay
of the neutron is specified by the Fermi coupling constant $G_{F}$,
all the other weak decays also must have the same strength,
if the hypothesis of universality of weak interactions is valid.
This was borne out in the decays of the neutron, the muon and
the pion, thus confirming universality. But, not in the decays
of the strange particles kaon and lambda. These decays were 
suppressed as compared to the decays of non-strange particles,
implying a breakdown of the universality of weak couplings.

Cabibbo succeeded in reformulating the hypothesis of universality 
in such a way that the discrepancy noted in the comparison of
the decay rates of the strange particles with those of the
non-strange particles could be explained beautifully. This
was done in his seminal paper of 1963 (N Cabibbo, Unitary symmetry
and leptonic decays, Phys Rev Lett,10,531,(1963)).

By 1963, SU(3) symmetry of Gell-Mann and Ne'eman which was
at that time called Unitary Symmetry was already becoming a part
of HEP. Following Gell-Mann, Cabibbo took the strangeness-
conserving and strangeness-violating weak hadronic
currents as members of the same octet representation under 
SU(3). This allowed him a more precise formulation of the
weak interaction. Call the S-conserving and S-violating
currents as $J$ and $J'$ respectively. Cabibbo notes that
if we take the total hadronic current as $aJ + bJ'$ with $a=b=1$,
and use it in the FG current $x$ current theory, this would
not ensure universality in the usual sense (equal coupling
for all currents), because the orthogonal current $bJ-aJ'$ is not
coupled. So he proposes instead his form of universality
which came to be known as Cabibbo universality by taking
the hadronic current as $\cos(\theta)J + \sin(\theta)J'$ so that
the current has unit 'length' (since $\cos^{2}(\theta)+\sin^{2}(\theta)
=1)$. This is the crucial point of the whole paper. Theta is
the Cabibbo angle. (Note that this choice still leaves the
current $-\sin(\theta)J + \cos(\theta) J'$ uncoupled. We will come 
back to this point later.)

By comparing the decay probabilities of the kaon and pion
Cabibbo determines the value of theta to be 0.26 (the present
value is closer to 0.22). Since $\sin_{2}(\theta)$ is about 0.05
while $\cos_{2}(\theta)$ is 0.95, the suppression of the kaon
decay as compared to pion decay is explained. This suppression
of strange decays with respect to non-strange decays is now
known as Cabibbo suppression.

In his paper Cabibbo goes on to calculate the rates
for the leptonic decays of all the strange baryons lambda,
sigma and cascade on the basis of the current that he has
proposed. This was followed in the subsequent years by 
many papers (by a host of authors including Cabibbo)
which applied the Cabibbo form of the current
to many weak decays and confronted their results with the
experimental data on these decays which were steadily coming.
Agreement was very good in all cases, thus confirming 
Cabibbo universality.

All this was done within the framework of SU(3) and hence
the agreement with experimental results helped to establish
the usefulness of SU(3) not only in the classification of
hadrons, but also in correlating their weak decays. 

It is important to point out that this form of universality
is not a discovery of Cabibbo. In the famous Gell-Mann-Levy
paper (Gell-Mann and Levy, The axial vector current in beta
decay, Nuovo Cimento, XVI,705 (1960)), there is a footnote
which had exactly the Cabibbo universality, using a pre-SU(3)
form of the weak current (this was even before Gell-Mann wrote
his paper on SU(3)!) Hence Cabibbo angle should really be
called Gell-Mann-Levy-Cabibbo angle, but the name Cabibbo
angle has stuck.

Although Cabibbo's paper refers to the Gell-Mann-Levy paper,
we must give substantial credit to Cabibbo; the Gell-Mann-Levy 
remark remained as a footnote for almost three years 
and it was Cabibbo who took it seriously and developed it 
into a full-fledged theory of the leptonic decays of hadrons 
with predictions that were verified experimentally.

\section*{Interpretation and aftermath}

In 1964 came the quark model of Gell-Mann and Zweig. The
introduction of quarks revolutionized Cabibbo universality
(and HEP in general) and allowed a much simpler interpretation
of the universality. What participates in the weak interaction 
is neither the d quark nor the s quark, but a linear superposition
$\cos(\theta)d + \sin(\theta)s$, called Cabibbo-rotated quark. 
One can see the simplicity of this interpretation of what 
Cabibbo had done.

Why is this a revolutionary change for the Cabibbo form? Because
of what happened in the aftermath.

Cabibbo himself in his original paper was rather modest; he said
he will restrict himself to a weaker form of universality. The
Cabibbo universality so far was infact weaker, since, as we 
pointed out earlier, the orthogonal
combination of currents $-\sin(\theta)J + \cos(\theta)J'$ was left
uncoupled. One can be sure that Cabibbo himself was acutely aware
of this deficiency in his universality hypothesis. But he could
not have done anything about it until the currents were rewritten
in quark language. 

The point is this: the particle defined by the linear superposition
$\cos(\theta) d + \sin(\theta) s$ is coupled to the u quark in
the weak current with the universal strength, but the orthogonal particle 
- $\sin(\theta)d + \cos(\theta)s$ is left with zero strength. Where is
the universality then? The loss of universality is thus starkly 
felt if the quark language is used.
 
So, another quark (similar to u quark, with
electric charge +2/3) is needed to couple to this orthogonal combination
and that turned out to be the charmed quark c ! The moral is that
Cabibbo universality alone could have been used to predict the
existence of the charmed quark. But, as it is, in the framework of
the Yang-Mills theory of weak interactions (that later became the 
Electroweak part of the SM), Glashow, Iliopoulos and Maiani
invoked the c quark coupled to precisely the "orthogonal quark", 
in order to forbid the unseen strangeness-changing
neutral current weak decays. This famous GIM mechanism was proposed
in 1970 and subsequently the hadrons $(\psi/J)$ containing the c quark was 
discovered experimentally in November 1974 (the so-called November
revolution, which saw fantastically narrow peaks - narrower by a factor
10,000 as compared to all the old hadrons - in electron-positron colliding
rings).      
 
To sum up, by the end of 1974 there were 4 quarks $u,d,s$ and $c$ and
they were organized into 2 doublets, now called 2 generations: $(u,d)$ and 
$(c,s)$. The weak current however required a modified form of the 2 doublets:
$(u,d')$ and $(c,s')$ where the primed quarks were the mutually orthogonal
pair: $d'=d \cos(\theta)+ s \sin(\theta),s'= -d \sin(\theta)+ s \cos(\theta)$.
Weak interaction picks the Cabibbo-rotated quarks $d'$ and $s'$. This is
a rotation in the 2-dimensional $(d,s)$ space and hence the Cabibbo
rotation is a $2 \times 2$ matrix charecterized by one quark-mixing angle
which is the original Cabibbo angle, theta.

\section*{From 2 to 3}

Enter Kobayashi and Maskawa (1973). Theirs was a very simple and
elementary observation made even before the second generation,
which required the discovery of c in 1974, was complete! But
it had a profound consequence. By counting the number of
angles and irremovable (and hence physically significant) phases
that any $n \times n$ unitary matrix has, they showed that the minimum 
number of $n$ (that is, the minimum number of quark generations)
must be 3, if we want to have at least one irremovable phase.
Such a phase is essential for CP violation which is matter-
antimatter asymmetry and CP violation was discovered experimentally
already in 1964 by Cronin and Fitch in the neutral kaon decays.
As pointed out by Andre Sakharov in 1967, CP violation is one of
the neccessary conditions for explaining how a fireball that was 
created in the initial Big Bang with equal amount of matter and 
antimatter could evolve into the observed matter-dominated 
universe. That is the deep cosmological significance of CP
violation and hence of the phase angle in the $3 \times 3$ quark mixing
matrix, the Cabibbo-Kobayashi-Maskawa (CKM) matrix.
 
Since the required additional quarks c,b and t were discovered
experimentally in 1974, 1977 and 1994 respectively, K-M's simple
remark attained the status of a prediction that was perfectly
confirmed by subsequent experiments.

Further, the single CKM phase was found to correlate all known
CP-violating phenomena - both in the old K-K bar as well as 
the new B-B bar systems. The pains-taking experiments using the
B-factories at Stanford (USA) and KEK (Japan) that were done in 
the last decade have given the stamp of established theory on the
CKM quark mixing. All this led to the Nobel Committee choosing
Kobayashi and Maskawa for sharing the Physics Nobel Prize(2008)   
with Nambu.
  
The fact that the $3\times3$ quark mixing matrix is called CKM matrix
by itself shows how intricately connected the ideas of Cabibbo
and Kobayashi-Maskawa are, although one must add the GIM
contribution for completeness. It shows how pointless picking
out a subset of names for Nobel Prize is. It has become
particularly painful because of the increased media-hype of
recent times.

Perhaps the Nobel Committee sought to free themselves from
possible criticism by stressing the K-M contribution to the 
understanding of broken symmetry (which is really
broken matter-antimatter symmetry). But this defense
cannot be maintained since K and M built on the structure
whose first brick was laid by Cabibbo. It would have been a
happier decision if the Nobel Committee had chosen Nambu
and Goldstone for 2008 and CKM for a later year.

In any case, let us ignore the Prize and return to Cabibbo. 
It is his seminal work in 1963 that laid the foundation for our
modern understanding of the weak interactions among the quarks.
Theorists who boldly followed his idea of universality of weak 
interactions to its logical completion, not only successfully
predicted the existence of the charmed quark, but also the top
and bottom quarks which were neccessary for CP violation.
Finally, the universality that was formulated by Gell-Mann-Levy 
and Cabibbo got enshrined in the Standard Model of High Energy 
Physics, in the form of the equality of gauge couplings to all 
the particles. It became a corner stone of the Standard Model. 

I thank MGK Menon for inducing me to write this article and
Sandip Pakvasa for critical comments on an earlier version
of the article. Thanks are due to Rahul Sinha for helpful
information.
