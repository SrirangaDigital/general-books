\chapter[Grand Unified Theories]{Grand Unified Theories\footnote{Invited talk given at the VI High Energy Physics Sympoaim, Mysuru, December, 1982}}\label{chap26}

\Authorline{G. Rajasekaran}
\addtocontents{toc}{\protect\contentsline{section}{{\sl G. Rajasekaran}\smallskip}{}}
\authinfo{Department of Theoretical Physics\\ University of Madras, Guindy Campus\\ Madras 600025, India.}

The Field of Grand Unified Theories (GUT) is quite vast Here we choose to discuss the following topics which seem interesting at the present ???.

\begin{itemize}
\item[1.] Standard model of QCD and QFD.
\item[2.] Model-independent results of unifiestion.
\item[3.] Popular models for unification ($SU_{3}, So_{10}, E_{6}$)
\item[4.] Patterns of symmetry breaking in $So_{10}$
\item[5.] Towards a physical understanding of the grand-unifying group.
\item[6.] Catalysis of bary on number violation by aanopoles.
\item[7.] Higher dimensional unification.
\end{itemize}

Major comissions are superaymmetry and super gravity,but there are two talks$^{1, 2}$ in this symposium devoted to these topics.

\section{Standard Model of QCD and QFD}\label{sec-1}

The standard model describes all of presently known High Energy Physics. It is based on the gauge group $SU_{3C} \times SU_{2L} \times U_{1}$. While $SU_{3C}$ leads to quantu m chromo-dynamics (QCD) and describes strong interactions, $SU_{2L} \times U_{1}$ leads to quantum flavour dynamics (QFD) and describes electro weak interactions. the gauge bosons of QCD are the 8 gluons $G_{\mu}^{i} (i= 1 \ldots 8)$. The gauge bosons of QFD are $W^{a}_{\mu}(a=1,2,3)$ and $B_{\mu}$ or, the physical weak bosons $W^{\pm}_{\mu}$ and $Z_{\mu}$ and photon $A_{\mu}$.

The physical low-energy separation between weak and electromagnetic interactions i  achived by breaking the symmetry at the scale $M_{L} \approx 100 GeV$.
$$
SU_{2L} \times U_{1} \rightarrow  U_{1Q}
$$
where $U_{1Q}$ is the electromagnetic gauge group. The symmetry breaking is supposedly induced by the nonvanishing vacuum exprectation value of the Higgs scalar $\phi$ whcih is choosen to be (1, 2) under $SU_{3C} \times SU_{2L}$. The colour symmetry $SU_{3C}$ is supposed to be unbroken.

The particle sector comprising leptons and quarks is taken to be the 3 generations of fermions:
\begin{equation*}
\underbrace{\begin{pmatrix}v_{e} \\ e^{-}  \end{pmatrix} \begin{pmatrix}u_{\alpha} \\ d_{\alpha}  \end{pmatrix}  }_{1}\quad \underbrace{\begin{pmatrix}v_{\mu} \\ \mu^{-}  \end{pmatrix} \begin{pmatrix}C_{\alpha} \\ s_{\alpha}  \end{pmatrix}  }_{2}\quad \underbrace{\begin{pmatrix}v_{\tau} \\ \tau^{-}  \end{pmatrix} \begin{pmatrix}t_{\alpha} \\ b_{\alpha}  \end{pmatrix}  }_{3}
\end{equation*}

In the above, $\alpha$ denotes the colour index : $\alpha = 1,2,3$. Among these fermions, $'t'$ has not yet been experimentally discovered. It is generally believed that it will be discovered at the higher energies. Separating the $L$ and $R$ helicities, we have 15 particles for each generation. For the first generation, they are 
\begin{equation*}
\begin{pmatrix}
v_{e} \\ e^{-}
\end{pmatrix}_{L}
,\quad
\begin{pmatrix}
u_{\alpha} \\ d_{\alpha}
\end{pmatrix}_{L},\quad
e^{-}_{R},\quad u_{\alpha R},\quad d_{\alpha R},
\end{equation*}
where the doublets and singlets of $SU_{2L}$ are explicitly indicated.

What is the present experimental status of the standard model ?

{\bf QCD~:} There is intense activity at an unprecedented level to fit all handronic phenomena to QCD. Althought there is general confirmation, it must be admitted that these is no direct verification of QCD. Infact, because of the dogma of colour confinement, QCD is docmed to indirect verification only. 

{\bf QFD~:} All neutral current seotors are in beautiful agreement with $SU_{2L} \times U_{1}$ with $\sin^{2} \theta_{w} \approx 0.21$ where $\sin \theta_{w}$ is the coupling constant ratio $e/g_{L}$. But the crucial test will be the existence of the $W$ and $Z$ bosons with masses at about 84 and 94 GeV respectively. Their discovery\footnote{The W discovery has been announced recently. See ref.3.} at the $p\bar{p}$ collider is awaited eagerly.

Finally, it must be mentioned agin that there is as yet no established experimental phenomenon which requires us to go beyond the standard model on $SU_{3C} \times SU_{2L} \times U_{1}$. But, theoretical physicistes have a craze for unification and have invented the subject of Grand Unification Theory to which we turn.

\section{Model-independent results of unification}\label{sec-2}

Before we describe models based on particular unifying groups, let us mention here some general results$^{4}$ which are independent of the particular unifying group.

\subsection{2.1~~Coupling constant ratios in the unification limit}\label{subsec-2.1}

These can be derived once the fermion representation is given. If the number of fermions is $N_{f}$, the number of left and right handed components is $2N_{f}$. Consider the maximal unifying group : $G_{\rm max} =SU_{2N_{f}}$ and construct the $2N_{f} \times 2N_{f}$ matrices $q, T_{L}$ and $T_{C}$ corresponding to electric charge, the third compoent of the $SU_{2L}$ generators and the third component of the colour generators respectively. Then,
$$
\frac{\alpha}{\alpha_{s}} = \frac{T_{r} T_{c}^{2}}{T_{r} Q^{2}} ; \quad \sin^{2} \theta_{w} = \frac{e^{2}}{g^{2}_{L}} = \frac{T_{r} T_{L}^{2}}{T_{r} Q^{2}}.
$$
Choosing the doublet scheme
\begin{equation*}
\begin{pmatrix}
u^{\alpha} \\ \alpha_{\alpha}
\end{pmatrix}
, \quad 
\begin{pmatrix}
v^{e} \\ e^{-}
\end{pmatrix}
\end{equation*}
for each generation, and writing the $16 \times 16$ matrices $Q, T_{L}$ and $T_{C}$ for these states, we get
$$
\frac{\alpha}{\alpha_{3}} = \frac{3}{8}\quad ; \sin^{2} \theta_{w} = \frac{3}{8}.
$$
These results are independent of the actural unifying group as long as it si a subgroup of the above $G_{\rm max} = SU_{16}$ and as long as the doublet scheme is maintained.

\subsection{2.2~~Renormalization Effects}\label{subsec-2.2}

The coupling constant ratios derived above are valid only in the unification limit. Once the unified symmetry is broken, the coupling constants undergo different renormalization and hence their ratios relevant after symmetry breaking are different.

Let us assume that the unification group G is broken at the scale $M_{u}$ in a single step into the statndard model group $SU_{3c} \times SU_{2L} \times U_{1}$. Then one gets, for the coupling constant ratios at the scale $M_{L} \approx 100 GeV$,
\begin{align*}
\frac{\alpha}{ \alpha_{s}} &= \frac{3}{8} - \frac{33}{8 \pi}~~ \alpha~~ ln~~ \frac{M_{u}}{M_{L}}\\
\sin^{2}\theta_{w} & = \frac{3}{8} - \frac{55}{24 \pi}~~ \alpha~~ ln~~ \frac{M_{u}}{M_{L}}
\end{align*}
Here, only the gauge boson contributions at the one loop level are taken inti account. It can be assumed that the complete fermion multiplet is light, in which case they do not contribute to the renormalization of the coupling constant ratios. Such as assumption is not valid for the Higgs and we shall take them into account latex.

The first of the above equations can be rewritten as
$$
\frac{3}{8} \frac{1}{\alpha} - \frac{1}{\alpha_{s}} = \frac{33}{8 \pi}~~ ln ~~ \frac{M_{u}}{M_{L}}
$$

On the left-hand side we substitute the empirical value of the coupling constants for QCD at the 'low' energy scale $M_{L}$:
$$
\frac{1}{\alpha_{s}} \approx 5;\quad \frac{3}{8} \frac{1}{\alpha} \approx 50;\quad \frac{3}{8} \frac{1}{\alpha} - \frac{1}{\alpha_{s}} \approx 45.
$$

This leads to the superheavy unification scale:
$$
M_{u} \approx 100 GeV\quad x-exp \frac{45 \times 8\pi}{ 33} \approx 10^{15} GeV.
$$
substitution of this valude of $M_{u}$ in the renormalization equation for $\sin^{2} \theta_{w}$ gives
$$
\sin^{2}\theta_{w} \approx 0.21
$$
which is in very good  agreement with the current experimental value.

Further, the above unification scale leads to another important consequence. Nothin that the unifying group G, in general, contains B (baryon number) violating process, small values of $M_{u}$ would have been catastrophic for the stability of the proton. Assuming a single exchange of the superheavy gauge boson of mass $\sim$ $M_{u}$, the life-time for proton decay can be estimated as 
$$
\tau_{p} \sim \frac{M_{u}^{4}}{m_{p}^{5}} \sim 10^{31}~~ {\rm years}
$$
which is consistant with the current experiment information$^{5}$.

Thus, not only the correct value of the electroweak mixing angle $\sin^{2}\theta_{w}$, but also the longevity of the proton is explained. The is the spectacular sucess of the grand unification scheme.

\section{Popular models for unification ($SU_{3}, SO_{10}, E_{6}$)}\label{sec-3}

Here we provide only the bare bones of the three popular OUT models. For more details, see ref.6. 3.1. The group $SU_{5}$

The gauge bosons belong to the adjoint representation: 24 and their decomposition under $SU_{3c} \times SU_{2L}$ are as follows:
\begin{align*}
24 & = (8,1) + (1,3) + (1,1) + (3,2) + (3^{*}, 2)\\
   & G^{\alpha}_{\beta}~~ (w^{\pm}, w^{0})~~ B~~ (x_{\alpha}, y_{\alpha})~~ (x^{\alpha}, y^{\alpha})
\end{align*}

The 24 gauge bosons are comprised of teh octlet of colour gluons $C^{\alpha}_{\beta}$, the 4 electroweak boson $(W^{\pm}, W^{0}, B)$ and teh 12 leptoquarks $(x_{\alpha}, y_{\alpha}, X^{\alpha}, Y^{\alpha})$, where $\alpha$ is the colour index.

The fermions of one family belong to $5^{*} + 10$ and they are respectively 
\begin{equation*}
\begin{pmatrix}
v_{e} & \alpha^{c}_{\alpha}\\
e^{-} &   
\end{pmatrix}_{L}
{\rm and}
\begin{pmatrix}
 & u^{\alpha}& \\
e^{+} &   & u^{c}_{\alpha}\\
      &d^{\alpha} &   
\end{pmatrix}_{L}
\end{equation*}
This is repeated for the $\mu -c -s$  family and $\tau-t-b$ family.

The symmetry-breaking occurs in two stages at two mass scales $M_{x}$ and $M_{L}$ :
$$
SU_{5}\xrightarrow[M_{x}]{} SU_{3c} \times SU_{2L} \times U_{1r} \xrightarrow[M_{L}]{} SU_{3c} \times U_{1Q}
$$

This can be achieved by choosing two Higgs multiplets 24 and 5 with vaccum expectation values equal to $M_{x}$ and $M_{L}$ respectively.

\subsection{The group $SO_{10}$}\label{subsec-3.2}

The gauge bosons belong to 45 and their decomposition according to $SU_{3c} \times SU_{2L} \times SU_{2R}$ is given by
\begin{align*}
45  &= (8,1,1) + (1,3,1) + (1,1,3) + (1,1,1)\\
     &G^{\alpha}_{\beta} \quad W^{\pm, o}_{L} \quad W^{\pm 0}_{R} \quad B\\
     &+ (3^{*}, 2,2) + (3,2,2) + (3,1,1) + (3^{*}, 1, 1)\\ 
     &\begin{pmatrix} x & \bar{Y'}\\ Y & \bar{x'}\end{pmatrix}\quad  \begin{pmatrix} x' & \bar{Y}\\ Y' & \bar{x}\end{pmatrix}\quad x_{s} \quad \bar{x}_{s}
\end{align*}

Now, there are weak bosons corresponding to $SU_{2L}$ as well as $SU_{2R}$ and the lepto-quarks are also more in number.

All the fermions of one family can be assigned to a single representation 16 consisting of
\begin{equation*}
\begin{pmatrix}
u^{\alpha} \\
d^{\alpha}
\end{pmatrix}_{L}
\begin{pmatrix}
\gamma \\
e^{-}
\end{pmatrix}_{L}
\begin{pmatrix}
d^{c}_{\alpha}\\
-u^{c}_{\alpha}
\end{pmatrix}_{L}
\begin{pmatrix}
e^{+}\\
-\gamma^{c}
\end{pmatrix}_{L}
\end{equation*}

In contrast to $SU_{5}$, many patterns of symmetry-breaking are possible for $SO_{10}$ and we shall describe them in detail later.

\subsection{The group $E_{6}$}

The gauge boson belong to 78:
\begin{align*}
78 &= (8,1,1) + (1,8,1) + (1,1,8) + (3,3^{*}, 3^{*}) + (3^{*}, 3, 3)\\
   &= 45 + 16 + 16^{*} + 1 
\end{align*}
where the decomposition given in the first and second lines correspond to representation of $SU_{3L} \times SU_{3L} \times SU_{3R}$ and $SO_{10}$ respectively.

The fermions can be taken to be in 27 and two types of models are possible, depending on whether $SU_{3C} \times SU_{3L} \times SU_{3R}$ or $SO_{10}$ is the intermediate symmetry group.

\subsection*{(a) $SU_{3C} \times SU_{3L} \times SU_{3R}$ type}

It is possible to encompess the observed fermions (plus unobserved ones, of course) in two families of 27 each:
\begin{equation*}
27 = 
\begin{pmatrix}
N^{c}_{\tau} & \tau^{+} & e^{+}\\
\tau^{-} & \gamma_{\tau} & \alpha^{c}_{e}\\
e^{-} & \gamma_{e} & \beta_{e}
\end{pmatrix}
 +
 (u^{\alpha} d^{\alpha} b^{\alpha})_{L} + 
 \begin{pmatrix}
 u^{c}_{\alpha} \\
 d^{c}_{\alpha}\\
 b^{c}_{\alpha}
 \end{pmatrix}_{L}
\end{equation*}
\begin{equation*}
27 = 
\begin{pmatrix}
N^{c}_{\tau} & M^{+} & \mu^{+}\\
M^{-} & \gamma_{M} & \alpha^{c}_{mu}\\
\mu^{-} & \gamma_{\mu} & \beta_{\mu}
\end{pmatrix}
 +
 (C^{\alpha} s^{\alpha} h^{\alpha})_{L} + 
 \begin{pmatrix}
 C^{c}_{\alpha} \\
 s^{c}_{\alpha}\\
 h^{c}_{\alpha}
 \end{pmatrix}_{L}
\end{equation*}
with the $SU_{3C} \times SU_{3L} \times SU_{3R}$ decomposition :
$27 = (1, 3^{*}, 3) + (3, 3, 1) + (3^{*}, 1, 3)$.

The top quark of charge + 2/3 is not needed in this model; instead a h-quark of charge -1/3 is required.

\subsection*{(b) $SO_{10}$ type}

In this type of model, 3 families are needed and each family is a 27 with the $SO_{10}$ decomposition:
$$
27 = 10 + 10 + 1
$$
where the 16 is the standard $SO_{10}$ multiplet given earlier, to 10 is comprised of new fermions belonging to 5 and $5^{*}$ of $SU_{5}$:
\begin{equation*}
\begin{pmatrix}
D^{c}_{\alpha}& N \\
  & E
\end{pmatrix}_{L}
\begin{pmatrix}
D^{\alpha}& N^{c} \\
  & -E^{c}
\end{pmatrix}_{L}
\end{equation*}
and the 1 is a neutral fermion $N_{OL}$.

\subsection{comparison and Comments}

A copparison between the good and bad points of the three unification groups $SU_{5}, SO_{10}$ and $E_{6}$ is offered in Table 3.1. $SU_{5}$ is the minimal group with no more then the 4 commuting generators of the standard $SU_{3} \times SU_{2} \times U_{1}$ model, whereas $SO_{10}$ and $E_{6}$ have 1 and 2 additional commuting generatora respectively. Consquently, $SU_{5}$ has only a single step of symmetry-breaking, leading to the 'almost' unique predictions of $\sin^{2} \theta_{w}$ and $\tau_{p}$ already mentioned in Sec.2. In contrast, no unique prediction is possible in $SO_{10}$ and $E_{6}$. This is the good point of $WU_{5}$. On the other hand, this is bought at the expense of the sterils desert which exists in $SU_{5}$ extending from 100 GeV to $10^{14}$ GeV. Because of the many interesting patterns of aymmetry breaking possible in $SO_{10}$ and $E_{6}$, these groups can lead to much richer physics.

All the fermions of one family can be put in a single irreducible representation of $SO_{10}$ and $E_{6}$; because of anomalies, this is not possible in $SU_{5}$. Another good point of $SO_{10}$ and $E_{6}$ is their left-right symmetry can a basic unified theory be as asymmetric as $SU_{5}$ ? Finally, if one regards quark-lepton symmetry as a desirable feature, then only $SO_{10}$ scores a point.

table????


\section{Patterns of Symmetry Breaking in $SO_{10}$}

As an example of the rich physics which may be expected in the bigger groups, we shall study the patterns of symmetry breaking possible in $SO_{10}$. Recently, considerable amount of work has been done in this field (see Ref. 7-9) and much of the material in this section is taken from these references.

Basically, ther are 2 routes for symmetry breaking in $SO_{10}$- either via $SU_{5}$ or via $SU_{4c} \times SU_{2L} \times SU_{2R}$ (see fig 4.1). The formet take one back to $SU_{5}$, but the latter is more interesting. The intermediate group $SU_{4c} \times SU_{2L} \times SU_{2R}$. which may be called the pati-salam group, is an interesting one. It contains a 4-colour group with leptons being treated as the objects with teh 4$^{th}$ colour and it also contains the left-right symmetric weak group $SU_{2L} \times SU_{2R}$.

figure???

Actually, there is a fine structure in the subsequent breaking of $G_{PS}$. Either the 4-colour group may break first into the usual 3-colour group and a $U_{1}$, or the right-handed weak group $SU_{2R}$ may break fist. This gives rise to the two sub routes given in Fig. 4.2.

figure????

How are the various symmetry-breaking scales $M_{u}, M_{c}, M_{R}$ and $M_{B-L}$ to be fixed? How are the masses of the multitude of Higgs scalars to be determined? The so-called extended survival hypothesis$^{10}$ is designed to answer these questions. It helps us to fix the Higgs masses and thus tring bring some order into an otherwise chaotic situation.

Consider the symmetry breaking chain:
$$
G \xrightarrow[M_{u}]{} G_{1} \xrightarrow[M_{1}]{} G_{2} \xrightarrow[M_{2}]{} \cdots G_{n} \xrightarrow[M_{n}]{} G_{n+1} \xrightarrow[M_{n+1}]{} \cdots G_{std} \xrightarrow[M_{L}]{} G_{Ex}
$$
where $G_{std}= SU_{3c} \times SU_{2L} \times U_{1}$ and $G_{ex}= SU_{3c} \times U_{1 Q}$.

Let ust fix our attention on a particular link in the chain:
$$
G_{n} \xrightarrow[M_{n}]{} G_{n+1}
$$

\newpage
Let this symmetry-breaking be achieved by the non-vanishing vacuum expectation value of some Higgs field $H_{n}$:
$$
\langle H_{n} \rangle \neq 0
$$

This Higgs field $H_{n}$ will be a singlet under $G_{n+1}$ and part of some represenations $R_{n}$ of the group $G_{n}$, $R_{n}$ will be part of some representation $R^{n-1}_{n}$ of $G_{n-1}$ and so on and finally all these will be contained in some representation $R^{0}_{n}$ or the group G. Thus,
$$
R^{0}_{n} \supset R'_{n} \supset R^{2}_{n} \cdot R^{n-1}_{n} \supset R_{n} \supset H_{n}
$$
corresponding groups being
$$
G \supset G_{1} \supset G_{2} \cdot G_{n-1} \supset G_{n}.
$$

The extended survival hypothesis (ESH) now states:

\begin{itemize}
\item[(a)] All members of $R_{n}$ acquires mass of order $M_{n}$, where $M_{n}$ is the scale of breaking of $G_{n}$.
\item[(b)] All Higgs scalars contained in $R^{j}_{n}$ but not in $R_{n}^{j+1}$ acquire mass of order $M_{j}$, where $M_{j}$ is the scale of breaking of $G_{j}$.
\end{itemize}

A simpler statement of the ESH hypothesis is: only those particles which have to be light will remain light. this hypothesis is an extension of the survival hypothesis of Georgi which was designed ot explain how the fermions survive as light particles in GUTs like $SU_{5}$.

Let us fist illustrate the application of ESH hypothesis to symmetry breaking in $SU_{5}$. The symmetry breaking occurs in 2 stages:
$$
SU_{5} \xrightarrow[M_{u}]{} SU_{3c} \times SU_{2L} \times U_{1r} \xrightarrow[M_{L}]{} SU_{sc} \times U_{1Q}
$$
and the Higgs representations are given in table 4.1. The groups are indioated only by their indices. Thus, $SU_{3c} \times SU_{2L} \times U_{1Y}$ is indicated by $3c \times 2L \times 1Y$.
\begin{center}
\begin{tabular}{|c|c|c|c|}
\hline
 & 5 & 3c $\times$ 2L $\times$ 1Y & 3c $\times$ 1Q\\
 \hline
 $\phi$ & 24 & (1,1,0) &  \\
 H  & 5  & (1,2, ) & (1, 0)\\
 \hline
\end{tabular}\\[.2cm]
\centering{Table 4.1} 
\end{center}

Appling the ESH hypothesis, we see that all members of 24 acquire masses of order $M_{U}$. As for 5, (1,2,) of $3c \times 2L \times 1Y$ acquire mass of order $M_{L}$ while the rest of 5 acquire mass of order $M_{U}$. The physical reason behind this mass pattern is clear; it is connected to the fact that the Higgs acalars supply the longitudinal components of the gauge bosons at the vaious stages of symmetry breaking.

Let us next consider the $SO_{10}$ symmetry breaking chain $SO_{10}\xrightarrow[M_{u}]{} SU_{4c} \times SU_{2L} \times SU_{2R} \xrightarrow[M_{c}]SU_{3c} \times SU_{2L} \times SU_{2R} \times U_{1B-L}\xrightarrow[M_{R}] SU_{3c} \times SU_{2L} \times U_{1r} \xrightarrow[M_{L}]{} SU_{2c} \times U_{1Q}$
where we have contracted one of the links in Fig.4.2 by putting $M_{R}\approx M_{S-L}$. The various Higgs reps. are given in Table4.2.
\begin{center}
\begin{tabular}{|c|c|c|c|c|}
\hline
10 & $4 \times 2L \times 2R$ & 3 $\times$ 2L $\times$ 2R $\times$ 1QL & $G_{std}$ & $G_{Ex}$\\
\hline
$\rho$ 54 & (1,1,1) &  & & \\
$\xi$ 45 & (15, 1,1) & (1,1,1,0) & &\\
$\Delta_{R}$ 126 & (10,1,3) & (1,1,3, $\sqrt{\frac{3}{2}}$ )& (1,1,0) & \\
$\Delta_{L}$ 126 & (10, 3, 1) & (1,1,3, $\sqrt{\frac{3}{2}}$ ) & (1, 3, $\sqrt{\frac{3}{2}}$) & (1,0)\\
$\phi 126$ & (15,2,2) & (1,2,2,0)&(1,2,$\sqrt{\frac{3}{20}}$) & (1,0)\\
\hline
\end{tabular}\\[.2cm]
\centering{Table 4.2}
\end{center}

The application of ESH now leads to the following masses for the Higgs scalars. The Higgs representation are written down, without specifying the groups, which can be easily read off from the table 4.2.
\begin{align*}
&\rho (54) \sim M_{U}.\\
&\xi  (45) : (15,1,1) \sim M_{x};~~ 45~~ {\rm minus} (15,1,1) \sim M_{U}\\
&\Delta_{R}(126) : \left(1,1,3, \sqrt{\frac{3}{2}}\right) \sim M_{R}; (10,1,3)~~ {\rm minus}~~ \left(1,1,3, \sqrt{{\frac{3}{2}}}\right)\\
&\sim M_{c}; \Delta_{R} (126)~~ {\rm minus}~~ (10,1,3)\sim M_{u}.\\
&\Delta_{L}(126): \left(1,3, \sqrt{\frac{3}{5}}\right)\sim M_{L}; \left(1,3,1,\sqrt{\frac{3}{2}}\right) ~~{\rm minus}~~ \left(1, 3, \sqrt{\frac{3}{5}}\right) \sim M_{R};\\
& (10,3,2)~~ {\rm minus}~~ \left(1,3,1, \sqrt{\frac{3}{2}}\right)\sim M_{c}; \Delta_{L}~~ {\rm minus}~~ (10,3,1)\sim M_{u}.\\
&\phi (126) : \left(1,2, \sqrt{\frac{3}{20}}\right)\sim M_{L}; (1,2,2,0) ~~{\rm minus}~~ \left(1,2, \sqrt{\frac{3}{20}}\right) \sim M_{R};\\
&(15,2,2)~~ {\rm minus}~~ (1,2,2,0) \sim M_{c}; \phi (126) ~~{\rm minus}~~ (15,2,2)\sim M_{u}.
\end{align*}

The renormalization effects on $\sin^{2}\theta_{w}$ and $\alpha /\alpha_{s}$ can now be worked out for the multistap symmetry breaking of Fig.4.2. Except for $\rho$ (54), for all other Higgs multiplets of $SO_{10}$, there are large mass difference within the multiplet. Hence, in contrast to fermions, Higgs scalars necessarily contribute to the renormalization effects. The results are given below for the two cases (a) $M_{c} > M_{R}$ and (b) $M_{c}> M_{R}$, corresponding to the two subroutes indicated in Fig.4.2.
\begin{align*}
sin^{2} \theta_{w} &= \frac{3}{8} + \frac{\alpha}{16 \pi} \left[ \left(\frac{44}{3}-4 \right) {\rm ln} \frac{M_{u}}{M_{c}}\right.\\
& - \left(\frac{44}{3} + 6 \right) ln \frac{M_{c}}{M_{R}} - \left(\frac{110}{3} + 1\right)ln \frac{M_{R}}{M_{B-L}} - \left(\frac{110}{3}-1 \right)ln \frac{M_{B-L}}{M_{L}}\\
&1-\frac{B}{3} \frac{\alpha}{\alpha_{s}}= \frac{\alpha}{2\pi} \left[\frac{1}{3} (44-4) ln \frac{M_{u}}{M_{c}} + \frac{1}{3} (44 + 14) ln \frac{M_{c}}{M_{R}}\right.\\
&\left (22 + 3) ln \frac{M_{R}}{M_{B-L}} + (22 + \frac{7}{3}) ln \frac{M_{B-L}}{M_{L}}\right]
 \end{align*}
\begin{align*}
\sin^{2}\theta_{w} &= \frac{3}{8} + \frac{\alpha}{16 \pi} \left[ \left(\frac{44}{3}-4\right) ln \frac{M_{u}}{M_{R}}\right.\\
 &-\left.\left(\frac{22}{3} -12 \right) ln \frac{M_{R}}{M_{C}} - \left(\frac{110}{3} + 1 \right) ln \frac{M_{C}}{M_{B}} - \left(\frac{110}{3}-1\right) ln \frac{M_{B-L}}{M_{L}}\right]\\ 
 & 1-\frac{8}{3} \frac{\alpha}{\alpha_{s}} = \frac{\alpha}{2\pi} \left[ \frac{1}{3} (44-4) ln \frac{M_{U}}{M_{R}} + \left(22- \frac{4}{3} \right) ln \frac{M_{R}}{M_{C}}\right.\\
 &\left. (22 + 13) ln  \frac{M_{c}}{M_{B-L}} + \left(22  + \frac{7}{3}\right) ln frac{M_{B-L}}{M_{L}}\right]
\end{align*}

The above equations replace the simpler equation of sec.2 which were derived for the single step symmetry breaking with neglect of Higgs contributions, The second number within the brackets refers to the Higgs contribution and one can see that these contributions are quits substantial in some cases.

In view of the presence of many scale parameters, no unique prediction is possible now. However, using the values of $\sin^{2}\theta_{w}$ and $\alpha/\alpha s$ as input, one can work out allowed regions for $M_{U}$, $M_{R}$ etc, and then the consequences for B, L isolation can be investigated.

As an example, let us consider the possibility of Hydrogen-antihydro\break gen oscillation\footnote{I thank M.K. Parida and U. Sarkar for discussion on this topic.}. This is a $\Delta B =2$, $D L =2$ process, for which the lowest-dimensional operator required is $q^{6}$$l^{2}$ where q and l are the quark and lepton fields respectively. The mass dimension of this operator is 12 and has to be multiplied by $M^{-8}$ (where M is some mass scale to be determined). so that the effective Lagrangian given by
$$
\mathcal{L}_{eff} \sim \frac{q^{2}l^{2}}{M^{B}}
$$
will have the required mass dimension 4. The amplitude for $H-\bar{H}$ oscillation is thus determined by the scale parameter M. What is the value of M ?

If the theory contains two scalar multiplets S and T which are (6,3,\-2/3) and (1,3,-2) under $SU_{3C} \times SU_{2L} \times U_{1Y}$ respectively, the $H-\bar{H}$ oscillation can arise through the diagram depioted in Fig. 4.3. It can be seen that S and T have the quantum number of a diquark and a dilepton respectively. The amplitude for the oscillation can be eatimated to be $A (H \leftrightarrow \bar{H}) \sim \lambda f^{3}q fl/M^{6}_{s} M^{2}_{T}$.

figure???


Hence, the effective mass parameter occuring in the effective Lagrangian above is 
$$
M \sim M^{3/4}_{s} M^{1/4}_{T}
$$

Let us now see how to fix the masses of these scalars S and T. It turns out that both these scalars are present in the symmetry-breaking chain:
$$
G_{ps} \xrightarrow[M_{c}]{} SU_{3C} \times SU_{2L} \times SU_{2R} \times U_{1B-L}\xrightarrow[M_{R}]{} G_{std}\xrightarrow[M_{L}]{G_{Ex}}
$$
where $G_{PS} = SU_{4C} \times SU_{2L} \times SU_{2R}$ may be an intermediate step in the symmetry breaking of $SO_{10}$, $E_{6}$ or some other group. Let us consider the Higgs scalars (10,1,3) and (10,3,1) of $G_{PS}$. The decomposition of these representations under $SU_{3C} \times SU_{2L} \times SU_{2R} \times U_{1B-L}$ and the subsequent breaking history is given in Table 4.3.  (compare with Table 4.2). The required scalars S and T have been indetified in the table. By applying ESH we see that
$$
M_{s}\sim M_{c} ; M_{T}\sim M_{L}
$$

tables???????

These can be substituted into the amplitude for $H \leftrightarrow \bar{H}$ oscillation given earlier. The allowed limits for $M_{C}$ determined from an analysis of the renormalization effects on $\sin^{2}\theta_{w}$ and $\alpha/\alpha s$ ca than be used to provide allowed limits for the oscillation amplitude.

\section{Towards a Physical understanding of the Grand-unifying Group}

We would like to pause here and ask: what is the physical meaning of the grand unifying group? Generally, the model-builders do not seem to be bothered about this, but at some stage, one has to ask this question$^{11}$. Here we may draw a lesson from history.

When flavour $SU_{3}$ was originally proposed by Sakataetal, it was based on $p, n, \Lambda $ as the triplet of basic objects. The weakness of this model waw that $p, n, \Lambda$ were treated as basic while the physically very similar baryons $\epsilon$ and $\Xi$ were assigned to a composite multiplet. By bringing together $p, n , \Lambda$ along with $\epsilon$ and $\Xi$ into an octet, Gell-Mann and Neeman construcuted the more successfull model. For quite some time this 'Eight-fold Way', based on $SU_{3}$ without triplets ruled the field. The physical meaning of flavour $SU_{3}$ remained obsoure however, untio the quarks u, d, s were identified as the basic triplet. It was this step, taken by Gell-Mann and Zweig, which ushered in the modern under standing of hadrons.

Similarly, the physical meaning of $SU_{3}$ may remain obsoure, until the basic quintet of $SU_{3}$ is identified. The $5^{*}$ multiplet $(\gamma, e^{-}, d^{c}_{d})_{L}$ of $SU_{3}$ can  hardly be regarded as the basic quintet since the particles $(e^{+}, u^{\alpha}, d^{\alpha}, u^{c}_{\alpha})_{L}$ which are physically very similar are  assigned to a decimet. The analogy with sakata model is very striking.

The next step\footnote{Ofcourse, it should be mentioned here that a single basic multiplet is not possible for $SU_{3}$ because the axial anomaly would prevent renormalizability.} os assigning all the 5 + 10 (with an additional singlet) to one basic multiplet 16 of $SO_{10}$ appers to be analogous to the step taken by Gell-Mann and Neeman in assigning all of the eight baryons to the octet of $SU_{3}$. This naturally leads us to expect that the next step should be the construction of the analogue of the quark model. Thus, some sort of composite structure for the quarks and leptons seems inevitable.

Actually, the situation is more involved. Whereas the physical mena\-ing of $SU_{n}$ group is the invariance under unitary trnasformation of the scalar product : $\sum^{n}_{i=1} \phi^{*}_{i}, \phi_{i}$, the physical meaning of $SO_{n}$ group is the invariance under orthognal transformation of the scalar product : $\sum_{i=1}^{n} x^{2}_{i}$. In the former, $\phi_{i}$ refers to a complex field while the $x_{i}$ of letter corresponds to a real coordinate. So, the physical meaning of $SO_{10}$ should be sought perphaps in higher dimensional space time. We shall come back to this possibilitu in a later section.

If these ideas are correct, then what is the physical meaning of the exceptional group such as $E_{6}$ ?

Before we discuss that, let us briefly refer to the Quaternion and the Octonion which seem to be intimately connected to the exceptional groups.$^{12, 13}$

There is a remarkable theorem due to Hurwitz which states that there are only 4 algebra whcih satisfy both the following properties:
\begin{itemize}
\item[(a)] If $xy=0$, then $x=0$ or $y=0$
\item[(b)] The norm satisfies $|xy|=|x| |y|$. 
\end{itemize}

These 4 algebras are Real numbers (R), Comples numbers (C), quaternions (Q) and Octonions ($\omega$) and these are called Hurwitz algebras. Q's are noncommutative, but $\omega's$ are not only noncommutative, but nonassociative also, ie.
$$
x(yz) \neq (xy)z
$$

Just as C's can be constructed interms of R by using $1({\rm with} 1^{2}=-1)$, Q's can be constructed in terms of C by introducing one more quantity j satisfying $j^{2}=-1$ and $\omega$ can be built from Q, by introducing further anities of the same type as 1 and j. But the important point about the Hurwitz theorem is that the process stops at $\omega$. If we try to extend the algebra further, then either property (a) or property (b) mentioned above, fails.

Now, construct Lie algebras and Lie groups in terms of antihermition matrices whose elements are members of Hurwitz algebras. As is well known, antihermition matrices with Roal elements leads to $SO_{n}$ whereas the same with complex elements leads to $SU_{n}$. Extending the procedure with Q and $\omega$ leads to only a finite number of Lie groups which are the exceptional groups. The procedure is quite involved and leads to the so-called magic square [see Frendenthal$^{14}$].

Coming back to our earlier question, we have the corresondence:
\begin{align*}
R &: \sum_{i=1}^{n} x_{i} x_{i} \rightarrow SO_{n}\\
C &: \sum_{l=1}^{n} \phi_{i}^{*} \phi_{i} \rightarrow SU_{n}\\
\end{align*}
 
 \begin{equation*}
   \left.
   \begin{array}{cc}
     Q & ?  \\
     \omega & ?
   \end{array}
   \right\} \rightarrow {\rm Exceptional Groups}
\end{equation*}

In the above, the question marks refer to new forms of scalar product. What is the physical significance of these new forms of scalar product? These possibly arise in new forms o Quantum Mechnics.$^{15, 16}$

In the present form of Quantum Mechanics, we knopw that comnplex number plays a curcial role. The probability amplitude is taken to be a comples number and thr probalility is the norm of this complex number. Now that we know that complex number is merely one of the four Hurwitz algebras, a  clear direction for the extension of Quantum Mechanics emerges. In these new forms of quantum mechanics, the probability amplitude may become a quaternion or even the nonassociative ootonion, with correspindingly new definition of scalar product which will lead to the exceptional groups.

To summarize, search for the physical meaning of CUT based on $SO_{n}$ may lead to extension of space-time, while the same based on exceptional groups may lead to extension of Quantum Mechanics.

Finally let us note teh remarkable chain
$$
SU_{5} - SU_{10}-E_{6}-E_{7}-E_{8}.
$$

The connection between these algebras is brought out by their Dynkin diagrams, givben in Fig. 5.1. Because of the similarity, $SU_{3}$ (which is $A_{4}$ in Cartan's notation) can be called $E_{4}$ and $SO_{10}$ ($d_{3}$ in certan's notation) can be called $E_{5}$.

figure???

The above chain is interesting because the first two members are the most popular GUTs. Hence, it would seem that GUTs based on the exceptional groups $E_{6}$, $E_{7}$ and $E_{8}$ are worth serious consideraton. In any case, one important advantage of this chain is bobvious. In contrest to the never ending chains $SU_{n}$ or $SO_{n}$, this chain is finite. There is hope that the game of model building has an end.

\section{Catalysis of Baryon Number Violation by Monopoles}

Stable monopole solutions occur for any gauge theory in which a semisimple gauge group G  is broken to a subgroup H which contains an expliot $U_{1}$ factor. The mass of the monopole is given by
$$
m_{mon} \sim \frac{4\pi}{g^{2}} m_{x}
$$
where $m_{x}$ is the energy scale at which the group G was broken and g is the gauge coupling constant. Hence, super heavy monopoles should exist in Nature if Nature is described by $C_{GUT} \sim SU_{5}, SO_{10}$ etc. For $m_{x} \sim 10^{14}$ GeV.

These superheavy monopoles are vey important for GUT, since they probe the original GUT, Actually, if we can procure a monopole  M and a antimonopole M, we can produce the X bosoms of the GUT:
$$
M + \bar{M} \rightarrow X + \bar{X}
$$

The superheavy monopoles can be used to recreate the scenerios of the big Bang in the laboratory. In particular, since baryon number violetion is an intrinsic part of the GUT interaction, the monopole interaction with teh nucleon will induce baryon number violation. the cross section for this is expected to be of order $1/m^{2}_{x}$ since it is governed by the geometrical area of the monopole. This is much bigger than the typical x-boson mediated baryon number violation (in the absenoe of monopole) which is of order $1/m^{4}_{x}$ in the rate. However, this is well-known (see for instance ref.17).

Recently, the possibility of a much more spectacular enhancement in the rare of baryon-number-violating process induced by monopoles has been brought to light (ref. 18-20). This is caused by the light fermions, which had so far been left out in most discussions of magnetic monopoles. It has been realized rather belatedly that light fermions can lead to rather drastic changes in the structure and properties of monopoles. Because such light fermions are present in the realistic GUT, they cannot be ignored. As a consequence of these light fermions, the cross section for monopole-induced baryon number violation is now claimed to be that of typical strong interaction $\sigma \sim 1/(GeV)^{2}$.

In this strong catalysis of baryon number violation, the well-known axial anomaly play an important role. Axial anomaly in the presence of Euclidean field configurations (instantons) can lead to fermion number violation which can in turn lead to B and L violation as was pointed out by t'Hooft long ago. However, this violation i suppersed by a huge factor exp(- const/$e^{2}$) and hence is undetactable at the present moment. What ????? pointed out os that, in contrast to the vacum sector where the action for the Euclidean configuation is of order $1/e^{2}$ and so one gets the above supperssion factor, in the monopole sector, the Euclidean configuration gives an action of order zero and therefore there is no such supperssion factor. Hence the expectation value of the B and L violating operator u u d e in the one-monopole state $|M > $ turns out to be
$$
\langle M| m(\vec{r}) u (\vec{r}) d (\vec{r}) e (\vec{r}) | M \rangle = \frac{C}{r^{6}}
$$
where $\vec{r}$ is the distance measured from the centre of teh monopole. The r-dependence of the right-hand side merely follows from the dimension of the operator  u u d e and c is a calculable  numerical coefficient, which is of order unity. Thus, there is no suppression factor.

To go from the above, to calculate the B and L violation cross section such as $\sigma (p  m \rightarrow e^{+} M)$ where $p$ is a proton and M is a monopole, is montrivial, but it is claimed that the cross section also will have no suppression factor and hence will have the typical strong interaction value. A definitive calculation of this cross action has not yet been done.

\section{Higher-Dimensional Unification}

Local coordinate invariance of the 4-d world leads to Einstein's theory of Gravity while local internal symmetry leads to the Gauge Theory of strong and Electroweak forces. If the number of dimensions $d > 4$, the latter can in fact be derived from the former. This is the higher-dimensional unification of Gravity with the other forces of nature.$^{21, 22}$ The mechanism by which this is achieved is so elegent that Nature would have aurely used it at some level. The question is : Is not that level already accessible to present-day Theoretical Physics?

Let us first discuss the original 5 dimensional Kaluza-Klein theory, We have
$$
x^{A} = (x^{\mu}, x^{5})
$$ 
where $A = 1 \ldots 5, = 1 \ldots 4$, but the $5^{\rm th}$ coordinate is curled up into a tiny circle of radius a so that $x^{5}=0$ a and $\theta$ goes from 0 to 2x. Consider the metric tensor $g_{AB}$. Under the usual 4-d Lorentz transformations, $B_{\mu 5}$ transforms as a 4-vector. Consider the following transformation:
\begin{align*}
x^{\mu} &= x^{'\mu}\\
x^{5} & = x^{' 5} + a\lambda(x^{'v}), \quad {\rm or},\quad \theta = \theta' + \lambda(x^{'v})\\ 
\end{align*}
ie. the circle at each point is rotated by a space-time dependent amount. under this transformation, $g_{\mu 5}$ transforms as follows:
\begin{align*}
g_{\mu r} &\rightarrow g_{AB} \frac{\delta x^{A}}{\delta x^{'\mu}} \frac{\delta x^{B}}{\delta x^{'5}}\\
          &= g_{\sigma \rho} \frac{\delta x^{\sigma}}{\delta x^{' \mu}} \frac{\delta x^{5}}{\delta x^{'5}} + g_{\sigma 5} \frac{\delta x^{\sigma}}{\delta x^{' \mu}} \frac{\delta x^{5}}{\delta x^{' 5}}\\
          & + g_{5 \rho} \frac{\delta x^{5}}{\delta x^{' \mu}} \frac{\delta x^{\rho}}{\delta x^{' 5}} + g_{55} \frac{\delta x^{5}}{\delta x^{' \mu}} \frac{\delta x^{5}}{\delta x^{'5}}\\
          &= g_{\mu 5 } + a \delta_{\mu} \Lambda
\end{align*}
where we have put $g_{55}= 1$. We see that $g_{\mu 5}$ undergoes the same trnasformation as the gauge transforamtion of the electromagnetic vector potential $A_{\mu}$ and hence can be idantified with it:
$$
g_{\mu g} \equiv a A_{\mu}.
$$

Thus, we have ahown that gauge transformations are contained in the coordinate transformations of the 5-dimensional theory. 

We partition the 5-dimensiona; metric tensor and write the ansatz
\begin{equation*}
g_{AB} (x^{\alpha}, x^{5}) = \left(
\begin{array}{c|c}
g_{\mu v} (x^{\alpha}) & a A_{\mu} (x^{\alpha})\\[0.2cm]
\hline
a A_{v}(x^{\alpha}) & g_{55} (x^{\alpha})\\[0.2cm]
\end{array}
\right)
\end{equation*}

We assume that $g_{AB}$ depends on $x^{\alpha}$ only. This is the ansatz which breaks the original 5-dimensional symmetry and the 5-d manifold $M^{5}$ aplits into the 4 dimensional manifolds $M^{4}$ and the circle $s'$:
$$
M^{5} \rightarrow M^{4} \times S'.
$$

Substituting the above anatz into the 5 dimensional Einstein-Hilbert action we get the 4 dimensional Einstein-Hilbert action plus the Maxwell action:
$$
\int d^{5}x \sqrt{-g^{5}} R^{(5)} \rightarrow \int \alpha^{4} x \sqrt{-g^{4}} \left\{\frac{1}{a^{2}} R^{(4)} + F_{\mu v}F^{\mu v} \right\}
$$
where $R^{d}$ refers to the curvature scalar in d-dimension and $F^{\mu v}$ is the usual Maxwell field. Thus, we see that the relative strength between the gravitational and  electromanetic interactions is fixed by the radius 'a'. Indentifying with the usual stengtgh of gravitational interaction, one finds that 'a' is essentially the Planck length. Thus we understand why the experimental physicists have not so far discovered the existence of these tiny circles.

The 5 dimensional theory unifies electromagnetism with gravituy. What about unification of Electroweak \& strong forces with gravity ? For that we require $d > 5$.

Let us take $d=4 + n$. The $4 + n$ dimensional theory can be worked out in analogy to the 5 dimensional case. The mainfold $M^{4 + n}$ splits into $M^{4} \times B$ where B is compact mainfold of dimension n. This is implemented by the ansatz:
\begin{equation*}
g_{AB} (x^{\alpha}, \theta^{k}) = \left(
\begin{array}{c|c}
g_{\mu v} (x^{\alpha}) & a \sum_{c} A_{\mu}^{c} (x^{\alpha} K^{c}_{i} (\theta^{k})\\[0.2cm]
\hline
a\sum_{c}A_{v}^{c} (x^{\alpha}) K_{j}^{c}(\theta^{k}) & r_{ij}(\theta^{k})\\[0.2cm]
\end{array}
\right)
\end{equation*}
where $\theta^{k}$ $(k= 1\ldots n)$ are the coordinates for the internal space of $B$,$T^{0} (c = 1 \ldots N)$ are the generators of the symmetry group G of B, the generators $T^{c}$, action on $e^{1}$ produce $Q^{1} + K^{c}_{1}(\theta)$, and $K^{c}_{1}(\theta)$ are the killing vectors. Substitution of this ansatz into the $4+ n $ dimensional action produces 4-d gravity along with the gauge theory of the group G:
$$
\int d^{4 + n} x \sqrt{-g^{4+ n}} R^(4 + n) \rightarrow \int d^{4}x \sqrt{-g^{(4)}} \left\{\frac{1}{a^{2}}R^{(4)} + \sum_{c}F^{c}_{\mu v} F^{c}_{\mu v} \right\}
$$
One can also verigy that for transformation of the type:
$$
(x^{\alpha}, \theta^{i}) \rightarrow (x^{\alpha}, \theta^{i} + \sum_{c} \epsilon^{c} (x^{\alpha}) K^{c}_{i}(\theta)),
$$
we get
$$
A_{\mu}^{c} (\theta) \rightarrow A^{c}_{\mu} (x) + f^{abc} \epsilon A_{\mu}^{c} + \delta_{\mu} \epsilon^{a}.
$$

Thus, the non-abelian gauge transformations are contained in the coordinate transformations.

The important questions now is : what is the dimension n of the internal space B ? witten$^{21}$ has shown recently that for the standard model group $G_{std} = SU_{3} \times SU_{2} \times U_{1}$, minimal manifold is $CP^{2} \times S^{2} \times S^{1} $ with dimension $=4 + 2+ 1 = 7$. Hence, adding the 4 of space-time, the minimum number of dimensions needed to describe the world is $d= 4 + 7 =11$.

This 11-dimensional worl turns up, from an enetirely differen direction also. It is well-known that d=11 is the maximal number of dimensions for supergravity theory; 11-dimensional $N=1$ supergravity is known to eb euqivalent to the 4-dimensional $N=8$ extended supergravity and 8 is the maximal extension possible if spins $> 2$ are to be avoided. This remarkable coincidence between the minimal number of dimension for Kaluza-Klein type of unification end the maximal number of dimension for super-unificationsuggests the physical reality of $d=11$.

Let us finally discuss briefly the alternative point of view of sugawara$^{23}$. who asks: why not consider fields and coordinates to be at the same level? Let  $\phi_{\alpha}$ for $\alpha = 1, 2, 3,,4 \ldots n+ 4$, represent this amalgam of fields and coordinates. ie. we may have $\phi_{1} \ldots \phi_{4}$ representing the usual coordinates $x_{1}\ldots x_{4}$ while $\phi_{1} \ldots \phi_{4}$ representing the usual coordinates $x_{1}\ldots x_{4}$ while $\phi_{5} \ldots \phi_{n+4}$ refer to fields. And one has full summetry under the $SO_{n+4}$ group. In a certain sense, this seems a natural step. For, remember the Lagrangian for a $SO_{n}$ gut is actually invariant under $SO_{n} \times SO_{4}$ where $SO_{4}$ comes from the Lorentz group. (Let us ingore the difference between $SO_{3, 1}$ and $SO_{4}$). Sugawara extands $SO_{n} \times SO_{4}$ to $SO_{n+4}$.

The questions is : does this make sense? can the usual fields having dependence on the independent coordinates namely $\phi_{\alpha}(x_{1} \ldots x_{4})$ for $\alpha = 5 \ldots n+ 4$ emerge as a symmetry-breaking effect? sugawara and Kaneko$^{23}$ (KEK pre-print 1982) claim to show that this is possible. Space-time emerges as a 4-dimensional surface in the $n+ 4$ dimensional continum. What is more, even Kaluza-Klein type of unification comes out as a special case of the sugarwara-Kaneko theory.

\section{Concluding Remarks}

Grand Unified Theories based on $SU_{5}$ or some other group are generally considered to be the natural step after the successful construct of the standard gauge model of electroweak and strong interactions. The outcome of the ongoing experiments$^{5}$ an baryon number violation will play a crucial role in testing the validity of this approach.

At a more fundemental level, the following question bas to be faced. Grand Unified Theories involve a leapfrog over the fantastic intervel of $10^{-15}$ cm to $10^{-30}$ cm. When a theory claima to describe physics over such a large regime, we have a right to ask about the physical meaning of its basic framework. Such an enquiry might lead to composite strucutre for quarks and leptons, or even fundamental revision of our concepts of space-time(higher dimensions), or of quantum mechanics (nonassociative algebras). There seem to be many ways open, only future can tell.

\begin{thebibliography}{99}
\bibitem{} R. Kaul, Superaymetry in GUTs, Review talk at VU HEP Symposium, Mysore, 1982.
\bibitem{} P. Majundar, Supergravity, Review talk at VI HEP Symposium, Mysore, 1982.
\bibitem{} UA1 Collaboration: G. Arnison etal, Phys. Lett.122B, 103 (1983).
\bibitem{} J.K. Bajaj and G. Rajasekaran, Pramana 14, 395 and 411 (1980).
\bibitem{} V.S. Narasimhan, Experimental Status of Baryon Non-Conservation, Review talk at VI HEP Symposium, Mysore, 1982.
\bibitem{} P. Langackar, Phys. Rep. C72, 185 (1981).
\bibitem{} A. Raychandhuri and Probir Roy, Neutron-Antineutron Oscillations and $SO_{10}$  Grand Unification, Tata Institute Preprint TIFR/TH/82-16.
\bibitem{} U. Sarkar, S. P. Misra and M.K. Parida, Hydrogen-Antihydrogen oscillations-signature of intermadiate mass scales in GUTs, Bhudenewar preprint IP/BBSR/82-14.
\bibitem{} M.K. Parida and C.C Hazra, Left-Right Symmetry, $K_{L}$-$K_{s}$ Mass difference and a possible new formula for neutrino mass, sambaipur preprint SU/PHY/82-3.
\bibitem{} F. del Aguila and L.E. Ibanez, Nuc;l-Phys. B177, 60 (1981).
\bibitem{} H. Sugawara, Review talk on QFD and unification at International Conference on High Energy Physics, Madison, 1980.
\bibitem{} P. Ranond, Introduction to Exceptional Lie Groups and Algebras, Clatech preprint CALT-68-577 (1976).
\bibitem{} R. D. Schafer, Introduction to Non-Associative Algebras, Academic Press, 1966.
\bibitem{} H. Freudenthal, Lie Groups in the foundation of Geometry, advances in Math.1, 143 (1964).
\bibitem{} P. Jordan, J. Von Neumann and E. P. Wigner, Ann. Math. 36, 29 (1934).
\bibitem{} L.P. Horwitz and L.C. Biedenharn, J. Math. Phys. 20, 269 (1979).
\bibitem{} C.P. Dokos and T.N. Tomoras Phys. Rev.D21, 2940 (1980).
\bibitem{} V.A. Rubakov, JETP letters 33,644 (1981); Nucl.Phys, B 203, 311 (1982).
\bibitem{} C.G. Callan, Jr Phys. Rev.D25, 2141 (1982); D26, 2058 (1982).
\bibitem{} F.A. Wilozek, Phys. Rev.Lett.48, 1146 (1982).
\bibitem{} E. witten, Nucl.Phys. B186, 412 (1981).
\bibitem{} See for instance A. Zee in Grand Unified Theories and Related Topics ($4^{\rm th}$ Kyoto Summer Institute), World Scientific, (1981).
\bibitem{} T. Kaneke and H. Sugawara, on the Structure of Space, Time and Field, KEK preprint KEK-TH 43 (1982).
\end{thebibliography}

\section*{Discussion} 

\noindent
U. Sarker: Comments
\begin{itemize}
\item[1.] The three generations of fermions cannot be associated in only two 27 dimensional representation of E(6) (topless model) [calculations diacussed].
\item[2.] Extended survival hypothesis is not only a hypothesis, rather it is a consequence of the minimal fine tuning (R. N. Mohapatra and G. Senjanovic, report no.BNL-31719).
\end{itemize}

\vspace{0.2cm}

\noindent
Probir Roy : How about noncompactness, specially in the sugawara programme? spacetime symmetry corresponds to SO(1,3) rather than sO(4). 

\vspace{0.2cm}
\noindent
G. Rajasekaran : In fact there is a further essential noncompactness coming from translations in space-time. The higher dimensional theories including sugawara's start with $n+4$ dimensional noncompact manifold. Finally, the n-dimension somehow collapse into a compact manifold while the 4 dimensional space-time remains noncompact.

The metric negative sign for the time coordinate can be taken into account, for instance, bu starting with such a sign in the original $n+4$ manifold itself.

\vspace{0.2cm}
\noindent
R. Ramachandran : In view of the fact the fermions are in left, right symmetric represenations in SO(10) and E(6) group, is Rubakov effect expected to be absent in then?

\vspace{0.2cm}
\noindent
G. Rajasekaran : That appeare to be true.


\vspace{0.2cm}
\noindent
G. Bhanot : comment- I was interested by your comment that space-time and fields may be manifestations of the some thing. I want to make a comment about this. Recently, t' Hooft has suggested that the "true" theory at short distance is SU() gauge theory (because it might become Borel aumable and therorefore defined by perturbation theory plus it may be finite etc.). Im addition, it has been shown recently by myself along with two other people (G. B. Neubeyer, Heller) generalising and correcton some work of Eguchi and xawai that the SU() theory constructs space-time out of the gauge group, These considerations may be an laternative to the usual Kaluta-Klein compactification and extensions by Sugawara.  


