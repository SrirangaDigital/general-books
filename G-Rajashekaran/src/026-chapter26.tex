\chapter[Grand Unified Theories]{Grand Unified Theories\footnote{Invited talk given at the VI High Energy Physics Sympoaim, Mysuru, December, 1982}}\label{chap26}

\Authorline{G. Rajasekaran}
\addtocontents{toc}{\protect\contentsline{section}{{\sl G. Rajasekaran}\smallskip}{}}
\authinfo{Department of Theoretical Physics\\ University of Madras, Guindy Campus\\ Madras 600025, India.}

The Field of Grand Unified Theories (GUT) is quite vast Here we choose to discuss the following topics which seem interesting at the present ???.

\begin{itemize}
\item[1.] Standard model of QCD and QFD.
\item[2.] Model-independent results of unifiestion.
\item[3.] Popular models for unification ($SU_{3}, So_{10}, E_{6}$)
\item[4.] Patterns of symmetry breaking in $So_{10}$
\item[5.] Towards a physical understanding of the grand-unifying group.
\item[6.] Catalysis of bary on number violation by aanopoles.
\item[7.] Higher dimensional unification.
\end{itemize}

Major comissions are superaymmetry and super gravity,but there are two talks$^{1, 2}$ in this symposium devoted to these topics.

\section{Standard Model of QCD and QFD}\label{sec-1}

The standard model describes all of presently known High Energy Physics. It is based on the gauge group $SU_{3C} \times SU_{2L} \times U_{1}$. While $SU_{3C}$ leads to quantu m chromo-dynamics (QCD) and describes strong interactions, $SU_{2L} \times U_{1}$ leads to quantum flavour dynamics (QFD) and describes electro weak interactions. the gauge bosons of QCD are the 8 gluons $G_{\mu}^{i} (i= 1 \ldots 8)$. The gauge bosons of QFD are $W^{a}_{\mu}(a=1,2,3)$ and $B_{\mu}$ or, the physical weak bosons $W^{\pm}_{\mu}$ and $Z_{\mu}$ and photon $A_{\mu}$.

The physical low-energy separation between weak and electromagnetic interactions i  achived by breaking the symmetry at the scale $M_{L} \approx 100 GeV$.
$$
SU_{2L} \times U_{1} \rightarrow  U_{1Q}
$$
where $U_{1Q}$ is the electromagnetic gauge group. The symmetry breaking is supposedly induced by the nonvanishing vacuum exprectation value of the Higgs scalar $\phi$ whcih is choosen to be (1, 2) under $SU_{3C} \times SU_{2L}$. The colour symmetry $SU_{3C}$ is supposed to be unbroken.

The particle sector comprising leptons and quarks is taken to be the 3 generations of fermions:
\begin{equation*}
\underbrace{\begin{pmatrix}v_{e} \\ e^{-}  \end{pmatrix} \begin{pmatrix}u_{\alpha} \\ d_{\alpha}  \end{pmatrix}  }_{1}\quad \underbrace{\begin{pmatrix}v_{\mu} \\ \mu^{-}  \end{pmatrix} \begin{pmatrix}C_{\alpha} \\ s_{\alpha}  \end{pmatrix}  }_{2}\quad \underbrace{\begin{pmatrix}v_{\tau} \\ \tau^{-}  \end{pmatrix} \begin{pmatrix}t_{\alpha} \\ b_{\alpha}  \end{pmatrix}  }_{3}
\end{equation*}

In the above, $\alpha$ denotes the colour index : $\alpha = 1,2,3$. Among these fermions, $'t'$ has not yet been experimentally discovered. It is generally believed that it will be discovered at the higher energies. Separating the $L$ and $R$ helicities, we have 15 particles for each generation. For the first generation, they are 
\begin{equation*}
\begin{pmatrix}
v_{e} \\ e^{-}
\end{pmatrix}_{L}
,\quad
\begin{pmatrix}
u_{\alpha} \\ d_{\alpha}
\end{pmatrix}_{L},\quad
e^{-}_{R},\quad u_{\alpha R},\quad d_{\alpha R},
\end{equation*}
where the doublets and singlets of $SU_{2L}$ are explicitly indicated.

What is the present experimental status of the standard model ?

{\bf QCD~:} There is intense activity at an unprecedented level to fit all handronic phenomena to QCD. Althought there is general confirmation, it must be admitted that these is no direct verification of QCD. Infact, because of the dogma of colour confinement, QCD is docmed to indirect verification only. 

{\bf QFD~:} All neutral current seotors are in beautiful agreement with $SU_{2L} \times U_{1}$ with $\sin^{2} \theta_{w} \approx 0.21$ where $\sin \theta_{w}$ is the coupling constant ratio $e/g_{L}$. But the crucial test will be the existence of the $W$ and $Z$ bosons with masses at about 84 and 94 GeV respectively. Their discovery\footnote{The W discovery has been announced recently. See ref.3.} at the $p\bar{p}$ collider is awaited eagerly.

Finally, it must be mentioned agin that there is as yet no established experimental phenomenon which requires us to go beyond the standard model on $SU_{3C} \times SU_{2L} \times U_{1}$. But, theoretical physicistes have a craze for unification and have invented the subject of Grand Unification Theory to which we turn.

\section{Model-independent results of unification}\label{sec-2}

Before we describe models based on particular unifying groups, let us mention here some general results$^{4}$ which are independent of the particular unifying group.

\subsection{2.1~~Coupling constant ratios in the unification limit}\label{subsec-2.1}

These can be derived once the fermion representation is given. If the number of fermions is $N_{f}$, the number of left and right handed components is $2N_{f}$. Consider the maximal unifying group : $G_{\rm max} =SU_{2N_{f}}$ and construct the $2N_{f} \times 2N_{f}$ matrices $q, T_{L}$ and $T_{C}$ corresponding to electric charge, the third compoent of the $SU_{2L}$ generators and the third component of the colour generators respectively. Then,
$$
\frac{\alpha}{\alpha_{s}} = \frac{T_{r} T_{c}^{2}}{T_{r} Q^{2}} ; \quad \sin^{2} \theta_{w} = \frac{e^{2}}{g^{2}_{L}} = \frac{T_{r} T_{L}^{2}}{T_{r} Q^{2}}.
$$
Choosing the doublet scheme
\begin{equation*}
\begin{pmatrix}
u^{\alpha} \\ \alpha_{\alpha}
\end{pmatrix}
, \quad 
\begin{pmatrix}
v^{e} \\ e^{-}
\end{pmatrix}
\end{equation*}
for each generation, and writing the $16 \times 16$ matrices $Q, T_{L}$ and $T_{C}$ for these states, we get
$$
\frac{\alpha}{\alpha_{3}} = \frac{3}{8}\quad ; \sin^{2} \theta_{w} = \frac{3}{8}.
$$
These results are independent of the actural unifying group as long as it si a subgroup of the above $G_{\rm max} = SU_{16}$ and as long as the doublet scheme is maintained.

\subsection{2.2~~Renormalization Effects}\label{subsec-2.2}

The coupling constant ratios derived above are valid only in the unification limit. Once the unified symmetry is broken, the coupling constants undergo different renormalization and hence their ratios relevant after symmetry breaking are different.

Let us assume that the unification group G is broken at the scale $M_{u}$ in a single step into the statndard model group $SU_{3c} \times SU_{2L} \times U_{1}$. Then one gets, for the coupling constant ratios at the scale $M_{L} \approx 100 GeV$,
\begin{align*}
\frac{\alpha}{ \alpha_{s}} &= \frac{3}{8} - \frac{33}{8 \pi}~~ \alpha~~ ln~~ \frac{M_{u}}{M_{L}}\\
\sin^{2}\theta_{w} & = \frac{3}{8} - \frac{55}{24 \pi}~~ \alpha~~ ln~~ \frac{M_{u}}{M_{L}}
\end{align*}
Here, only the gauge boson contributions at the one loop level are taken inti account. It can be assumed that the complete fermion multiplet is light, in which case they do not contribute to the renormalization of the coupling constant ratios. Such as assumption is not valid for the Higgs and we shall take them into account latex.

The first of the above equations can be rewritten as
$$
\frac{3}{8} \frac{1}{\alpha} - \frac{1}{\alpha_{s}} = \frac{33}{8 \pi}~~ ln ~~ \frac{M_{u}}{M_{L}}
$$

On the left-hand side we substitute the empirical value of the coupling constants for QCD at the 'low' energy scale $M_{L}$:

