\chapter[Grand Unified Theories]{Grand Unified Theories\footnote{Invited talk given at the VI High Energy Physics Sympoaim, Mysuru, December, 1982}}\label{chap26}

\Authorline{G. Rajasekaran}
\addtocontents{toc}{\protect\contentsline{section}{{\sl G. Rajasekaran}\smallskip}{}}
\authinfo{Department of Theoretical Physics\\ University of Madras, Guindy Campus\\ Madras 600025, India.}

The Field of Grand Unified Theories (GUT) is quite vast Here we choose to discuss the following topics which seem interesting at the present ???.

\begin{itemize}
\item[1.] Standard model of QCD and QFD.
\item[2.] Model-independent results of unifiestion.
\item[3.] Popular models for unification ($SU_{3}, So_{10}, E_{6}$)
\item[4.] Patterns of symmetry breaking in $So_{10}$
\item[5.] Towards a physical understanding of the grand-unifying group.
\item[6.] Catalysis of bary on number violation by aanopoles.
\item[7.] Higher dimensional unification.
\end{itemize}

Major comissions are superaymmetry and super gravity,but there are two talks$^{1, 2}$ in this symposium devoted to these topics.

\section{Standard Model of QCD and QFD}\label{sec-1}

The standard model describes all of presently known High Energy Physics. It is based on the gauge group $SU_{3C} \times SU_{2L} \times U_{1}$. While $SU_{3C}$ leads to quantu m chromo-dynamics (QCD) and describes strong interactions, $SU_{2L} \times U_{1}$ leads to quantum flavour dynamics (QFD) and describes electro weak interactions. the gauge bosons of QCD are the 8 gluons $G_{\mu}^{i} (i= 1 \ldots 8)$. The gauge bosons of QFD are $W^{a}_{\mu}(a=1,2,3)$ and $B_{\mu}$ or, the physical weak bosons $W^{\pm}_{\mu}$ and $Z_{\mu}$ and photon $A_{\mu}$.

The physical low-energy separation between weak and electromagnetic interactions i  achived by breaking the symmetry at the scale $M_{L} \approx 100 GeV$.
$$
SU_{2L} \times U_{1} \rightarrow  U_{1Q}
$$
where $U_{1Q}$ is the electromagnetic gauge group. The symmetry breaking is supposedly induced by the nonvanishing vacuum exprectation value of the Higgs scalar $\phi$ whcih is choosen to be (1, 2) under $SU_{3C} \times SU_{2L}$. The colour symmetry $SU_{3C}$ is supposed to be unbroken.

The particle sector comprising leptons and quarks is taken to be the 3 generations of fermions:
\begin{equation*}
\underbrace{\begin{pmatrix}v_{e} \\ e^{-}  \end{pmatrix} \begin{pmatrix}u_{\alpha} \\ d_{\alpha}  \end{pmatrix}  }_{1}\quad \underbrace{\begin{pmatrix}v_{\mu} \\ \mu^{-}  \end{pmatrix} \begin{pmatrix}C_{\alpha} \\ s_{\alpha}  \end{pmatrix}  }_{2}\quad \underbrace{\begin{pmatrix}v_{\tau} \\ \tau^{-}  \end{pmatrix} \begin{pmatrix}t_{\alpha} \\ b_{\alpha}  \end{pmatrix}  }_{3}
\end{equation*}

In the above, $\alpha$ denotes the colour index : $\alpha = 1,2,3$. Among these fermions, $'t'$ has not yet been experimentally discovered. It is generally believed that it will be discovered at the higher energies. Separating the $L$ and $R$ helicities, we have 15 particles for each generation. For the first generation, they are 
\begin{equation*}
\begin{pmatrix}
v_{e} \\ e^{-}
\end{pmatrix}_{L}
,\quad
\begin{pmatrix}
u_{\alpha} \\ d_{\alpha}
\end{pmatrix}_{L},\quad
e^{-}_{R},\quad u_{\alpha R},\quad d_{\alpha R},
\end{equation*}
where the doublets and singlets of $SU_{2L}$ are explicitly indicated.

What is the present experimental status of the standard model ?

{\bf QCD~:} There is intense activity at an unprecedented level to fit all handronic phenomena to QCD. Althought there is general confirmation, it must be admitted that these is no direct verification of QCD. Infact, because of the dogma of colour confinement, QCD is docmed to indirect verification only. 

{\bf QFD~:} All neutral current seotors are in beautiful agreement with $SU_{2L} \times U_{1}$ with $\sin^{2} \theta_{w} \approx 0.21$ where $\sin \theta_{w}$ is the coupling constant ratio $e/g_{L}$. But the crucial test will be the existence of the $W$ and $Z$ bosons with masses at about 84 and 94 GeV respectively. Their discovery\footnote{The W discovery has been announced recently. See ref.3.} at the $p\bar{p}$ collider is awaited eagerly.

Finally, it must be mentioned agin that there is as yet no established experimental phenomenon which requires us to go beyond the standard model on $SU_{3C} \times SU_{2L} \times U_{1}$. But, theoretical physicistes have a craze for unification and have invented the subject of Grand Unification Theory to which we turn.

\section{Model-independent results of unification}\label{sec-2}

Before we describe models based on particular unifying groups, let us mention here some general results$^{4}$ which are independent of the particular unifying group.

\subsection{2.1~~Coupling constant ratios in the unification limit}\label{subsec-2.1}

These can be derived once the fermion representation is given. If the number of fermions is $N_{f}$, the number of left and right handed components is $2N_{f}$. Consider the maximal unifying group : $G_{\rm max} =SU_{2N_{f}}$ and construct the $2N_{f} \times 2N_{f}$ matrices $q, T_{L}$ and $T_{C}$ corresponding to electric charge, the third compoent of the $SU_{2L}$ generators and the third component of the colour generators respectively. Then,
$$
\frac{\alpha}{\alpha_{s}} = \frac{T_{r} T_{c}^{2}}{T_{r} Q^{2}} ; \quad \sin^{2} \theta_{w} = \frac{e^{2}}{g^{2}_{L}} = \frac{T_{r} T_{L}^{2}}{T_{r} Q^{2}}.
$$
Choosing the doublet scheme
\begin{equation*}
\begin{pmatrix}
u^{\alpha} \\ \alpha_{\alpha}
\end{pmatrix}
, \quad 
\begin{pmatrix}
v^{e} \\ e^{-}
\end{pmatrix}
\end{equation*}
for each generation, and writing the $16 \times 16$ matrices $Q, T_{L}$ and $T_{C}$ for these states, we get
$$
\frac{\alpha}{\alpha_{3}} = \frac{3}{8}\quad ; \sin^{2} \theta_{w} = \frac{3}{8}.
$$
These results are independent of the actural unifying group as long as it si a subgroup of the above $G_{\rm max} = SU_{16}$ and as long as the doublet scheme is maintained.

\subsection{2.2~~Renormalization Effects}\label{subsec-2.2}

The coupling constant ratios derived above are valid only in the unification limit. Once the unified symmetry is broken, the coupling constants undergo different renormalization and hence their ratios relevant after symmetry breaking are different.

Let us assume that the unification group G is broken at the scale $M_{u}$ in a single step into the statndard model group $SU_{3c} \times SU_{2L} \times U_{1}$. Then one gets, for the coupling constant ratios at the scale $M_{L} \approx 100 GeV$,
\begin{align*}
\frac{\alpha}{ \alpha_{s}} &= \frac{3}{8} - \frac{33}{8 \pi}~~ \alpha~~ ln~~ \frac{M_{u}}{M_{L}}\\
\sin^{2}\theta_{w} & = \frac{3}{8} - \frac{55}{24 \pi}~~ \alpha~~ ln~~ \frac{M_{u}}{M_{L}}
\end{align*}
Here, only the gauge boson contributions at the one loop level are taken inti account. It can be assumed that the complete fermion multiplet is light, in which case they do not contribute to the renormalization of the coupling constant ratios. Such as assumption is not valid for the Higgs and we shall take them into account latex.

The first of the above equations can be rewritten as
$$
\frac{3}{8} \frac{1}{\alpha} - \frac{1}{\alpha_{s}} = \frac{33}{8 \pi}~~ ln ~~ \frac{M_{u}}{M_{L}}
$$

On the left-hand side we substitute the empirical value of the coupling constants for QCD at the 'low' energy scale $M_{L}$:
$$
\frac{1}{\alpha_{s}} \approx 5;\quad \frac{3}{8} \frac{1}{\alpha} \approx 50;\quad \frac{3}{8} \frac{1}{\alpha} - \frac{1}{\alpha_{s}} \approx 45.
$$

This leads to the superheavy unification scale:
$$
M_{u} \approx 100 GeV\quad x-exp \frac{45 \times 8\pi}{ 33} \approx 10^{15} GeV.
$$
substitution of this valude of $M_{u}$ in the renormalization equation for $\sin^{2} \theta_{w}$ gives
$$
\sin^{2}\theta_{w} \approx 0.21
$$
which is in very good  agreement with the current experimental value.

Further, the above unification scale leads to another important consequence. Nothin that the unifying group G, in general, contains B (baryon number) violating process, small values of $M_{u}$ would have been catastrophic for the stability of the proton. Assuming a single exchange of the superheavy gauge boson of mass $\sim$ $M_{u}$, the life-time for proton decay can be estimated as 
$$
\tau_{p} \sim \frac{M_{u}^{4}}{m_{p}^{5}} \sim 10^{31}~~ {\rm years}
$$
which is consistant with the current experiment information$^{5}$.

Thus, not only the correct value of the electroweak mixing angle $\sin^{2}\theta_{w}$, but also the longevity of the proton is explained. The is the spectacular sucess of the grand unification scheme.

\section{Popular models for unification ($SU_{3}, SO_{10}, E_{6}$)}\label{sec-3}

Here we provide only the bare bones of the three popular OUT models. For more details, see ref.6. 3.1. The group $SU_{5}$

The gauge bosons belong to the adjoint representation: 24 and their decomposition under $SU_{3c} \times SU_{2L}$ are as follows:
\begin{align*}
24 & = (8,1) + (1,3) + (1,1) + (3,2) + (3^{*}, 2)\\
   & G^{\alpha}_{\beta}~~ (w^{\pm}, w^{0})~~ B~~ (x_{\alpha}, y_{\alpha})~~ (x^{\alpha}, y^{\alpha})
\end{align*}

The 24 gauge bosons are comprised of teh octlet of colour gluons $C^{\alpha}_{\beta}$, the 4 electroweak boson $(W^{\pm}, W^{0}, B)$ and teh 12 leptoquarks $(x_{\alpha}, y_{\alpha}, X^{\alpha}, Y^{\alpha})$, where $\alpha$ is the colour index.

The fermions of one family belong to $5^{*} + 10$ and they are respectively 
\begin{equation*}
\begin{pmatrix}
v_{e} & \alpha^{c}_{\alpha}\\
e^{-} &   
\end{pmatrix}_{L}
{\rm and}
\begin{pmatrix}
 & u^{\alpha}& \\
e^{+} &   & u^{c}_{\alpha}\\
      &d^{\alpha} &   
\end{pmatrix}_{L}
\end{equation*}
This is repeated for the $\mu -c -s$  family and $\tau-t-b$ family.

The symmetry-breaking occurs in two stages at two mass scales $M_{x}$ and $M_{L}$ :
$$
SU_{5}\xrightarrow[M_{x}]{} SU_{3c} \times SU_{2L} \times U_{1r} \xrightarrow[M_{L}]{} SU_{3c} \times U_{1Q}
$$

This can be achieved by choosing two Higgs multiplets 24 and 5 with vaccum expectation values equal to $M_{x}$ and $M_{L}$ respectively.

\subsection{The group $SO_{10}$}\label{subsec-3.2}

The gauge bosons belong to 45 and their decomposition according to $SU_{3c} \times SU_{2L} \times SU_{2R}$ is given by
\begin{align*}
45  &= (8,1,1) + (1,3,1) + (1,1,3) + (1,1,1)\\
     &G^{\alpha}_{\beta} \quad W^{\pm, o}_{L} \quad W^{\pm 0}_{R} \quad B\\
     &+ (3^{*}, 2,2) + (3,2,2) + (3,1,1) + (3^{*}, 1, 1)\\ 
     &\begin{pmatrix} x & \bar{Y'}\\ Y & \bar{x'}\end{pmatrix}\quad  \begin{pmatrix} x' & \bar{Y}\\ Y' & \bar{x}\end{pmatrix}\quad x_{s} \quad \bar{x}_{s}
\end{align*}

Now, there are weak bosons corresponding to $SU_{2L}$ as well as $SU_{2R}$ and the lepto-quarks are also more in number.

All the fermions of one family can be assigned to a single representation 16 consisting of
\begin{equation*}
\begin{pmatrix}
u^{\alpha} \\
d^{\alpha}
\end{pmatrix}_{L}
\begin{pmatrix}
\gamma \\
e^{-}
\end{pmatrix}_{L}
\begin{pmatrix}
d^{c}_{\alpha}\\
-u^{c}_{\alpha}
\end{pmatrix}_{L}
\begin{pmatrix}
e^{+}\\
-\gamma^{c}
\end{pmatrix}_{L}
\end{equation*}

In contrast to $SU_{5}$, many patterns of symmetry-breaking are possible for $SO_{10}$ and we shall describe them in detail later.

\subsection{The group $E_{6}$}

The gauge boson belong to 78:
\begin{align*}
78 &= (8,1,1) + (1,8,1) + (1,1,8) + (3,3^{*}, 3^{*}) + (3^{*}, 3, 3)\\
   &= 45 + 16 + 16^{*} + 1 
\end{align*}
where the decomposition given in the first and second lines correspond to representation of $SU_{3L} \times SU_{3L} \times SU_{3R}$ and $SO_{10}$ respectively.

The fermions can be taken to be in 27 and two types of models are possible, depending on whether $SU_{3C} \times SU_{3L} \times SU_{3R}$ or $SO_{10}$ is the intermediate symmetry group.

\subsection*{(a) $SU_{3C} \times SU_{3L} \times SU_{3R}$ type}

It is possible to encompess the observed fermions (plus unobserved ones, of course) in two families of 27 each:
\begin{equation*}
27 = 
\begin{pmatrix}
N^{c}_{\tau} & \tau^{+} & e^{+}\\
\tau^{-} & \gamma_{\tau} & \alpha^{c}_{e}\\
e^{-} & \gamma_{e} & \beta_{e}
\end{pmatrix}
 +
 (u^{\alpha} d^{\alpha} b^{\alpha})_{L} + 
 \begin{pmatrix}
 u^{c}_{\alpha} \\
 d^{c}_{\alpha}\\
 b^{c}_{\alpha}
 \end{pmatrix}_{L}
\end{equation*}
\begin{equation*}
27 = 
\begin{pmatrix}
N^{c}_{\tau} & M^{+} & \mu^{+}\\
M^{-} & \gamma_{M} & \alpha^{c}_{mu}\\
\mu^{-} & \gamma_{\mu} & \beta_{\mu}
\end{pmatrix}
 +
 (C^{\alpha} s^{\alpha} h^{\alpha})_{L} + 
 \begin{pmatrix}
 C^{c}_{\alpha} \\
 s^{c}_{\alpha}\\
 h^{c}_{\alpha}
 \end{pmatrix}_{L}
\end{equation*}
with the $SU_{3C} \times SU_{3L} \times SU_{3R}$ decomposition :
$27 = (1, 3^{*}, 3) + (3, 3, 1) + (3^{*}, 1, 3)$.

The top quark of charge + 2/3 is not needed in this model; instead a h-quark of charge -1/3 is required.

\subsection*{(b) $SO_{10}$ type}

In this type of model, 3 families are needed and each family is a 27 with the $SO_{10}$ decomposition:
$$
27 = 10 + 10 + 1
$$
where the 16 is the standard $SO_{10}$ multiplet given earlier, to 10 is comprised of new fermions belonging to 5 and $5^{*}$ of $SU_{5}$:
\begin{equation*}
\begin{pmatrix}
D^{c}_{\alpha}& N \\
  & E
\end{pmatrix}_{L}
\begin{pmatrix}
D^{\alpha}& N^{c} \\
  & -E^{c}
\end{pmatrix}_{L}
\end{equation*}
and the 1 is a neutral fermion $N_{OL}$.

\subsection{comparison and Comments}

A copparison between the good and bad points of the three unification groups $SU_{5}, SO_{10}$ and $E_{6}$ is offered in Table 3.1. $SU_{5}$ is the minimal group with no more then the 4 commuting generators of the standard $SU_{3} \times SU_{2} \times U_{1}$ model, whereas $SO_{10}$ and $E_{6}$ have 1 and 2 additional commuting generatora respectively. Consquently, $SU_{5}$ has only a single step of symmetry-breaking, leading to the 'almost' unique predictions of $\sin^{2} \theta_{w}$ and $\tau_{p}$ already mentioned in Sec.2. In contrast, no unique prediction is possible in $SO_{10}$ and $E_{6}$. This is the good point of $WU_{5}$. On the other hand, this is bought at the expense of the sterils desert which exists in $SU_{5}$ extending from 100 GeV to $10^{14}$ GeV. Because of the many interesting patterns of aymmetry breaking possible in $SO_{10}$ and $E_{6}$, these groups can lead to much richer physics.

All the fermions of one family can be put in a single irreducible representation of $SO_{10}$ and $E_{6}$; because of anomalies, this is not possible in $SU_{5}$. Another good point of $SO_{10}$ and $E_{6}$ is their left-right symmetry can a basic unified theory be as asymmetric as $SU_{5}$ ? Finally, if one regards quark-lepton symmetry as a desirable feature, then only $SO_{10}$ scores a point.

table????


\section{Patterns of Symmetry Breaking in $SO_{10}$}

As an example of the rich physics which may be expected in the bigger groups, we shall study the patterns of symmetry breaking possible in $SO_{10}$. Recently, considerable amount of work has been done in this field (see Ref. 7-9) and much of the material in this section is taken from these references.

Basically, ther are 2 routes for symmetry breaking in $SO_{10}$- either via $SU_{5}$ or via $SU_{4c} \times SU_{2L} \times SU_{2R}$ (see fig 4.1). The formet take one back to $SU_{5}$, but the latter is more interesting. The intermediate group $SU_{4c} \times SU_{2L} \times SU_{2R}$. which may be called the pati-salam group, is an interesting one. It contains a 4-colour group with leptons being treated as the objects with teh 4$^{th}$ colour and it also contains the left-right symmetric weak group $SU_{2L} \times SU_{2R}$.

figure???

Actually, there is a fine structure in the subsequent breaking of $G_{PS}$. Either the 4-colour group may break first into the usual 3-colour group and a $U_{1}$, or the right-handed weak group $SU_{2R}$ may break fist. This gives rise to the two sub routes given in Fig. 4.2.

figure????

How are the various symmetry-breaking scales $M_{u}, M_{c}, M_{R}$ and $M_{B-L}$ to be fixed? How are the masses of the multitude of Higgs scalars to be determined? The so-called extended survival hypothesis$^{10}$ is designed to answer these questions. It helps us to fix the Higgs masses and thus tring bring some order into an otherwise chaotic situation.

Consider the symmetry breaking chain:
$$
G \xrightarrow[M_{u}]{} G_{1} \xrightarrow[M_{1}]{} G_{2} \xrightarrow[M_{2}]{} \cdots G_{n} \xrightarrow[M_{n}]{} G_{n+1} \xrightarrow[M_{n+1}]{} \cdots G_{std} \xrightarrow[M_{L}]{} G_{Ex}
$$
where $G_{std}= SU_{3c} \times SU_{2L} \times U_{1}$ and $G_{ex}= SU_{3c} \times U_{1 Q}$.

Let ust fix our attention on a particular link in the chain:
$$
G_{n} \xrightarrow[M_{n}]{} G_{n+1}
$$

Let this symmetry-breaking be achieved by the non-vanishing vacuum expectation value of some Higgs field $H_{n}$:
$$
\langle H_{n} \rangle \neq 0
$$

This Higgs field $H_{n}$ will be a singlet under $G_{n+1}$ and part of some represenations $R_{n}$ of the group $G_{n}$, $R_{n}$ will be part of some representation $R^{n-1}_{n}$ of $G_{n-1}$ and so on and finally all these will be contained in some representation $R^{0}_{n}$ or the group G. Thus,
$$
R^{0}_{n} \supset R'_{n} \supset R^{2}_{n} \cdot R^{n-1}_{n} \supset R_{n} \supset H_{n}
$$
corresponding groups being
$$
G \supset G_{1} \supset G_{2} \cdot G_{n-1} \supset G_{n}.
$$

The extended survival hypothesis (ESH) now states:

\begin{itemize}
\item[(a)] All members of $R_{n}$ acquires mass of order $M_{n}$, where $M_{n}$ is the scale of breaking of $G_{n}$.
\item[(b)] All Higgs scalars contained in $R^{j}_{n}$ but not in $R_{n}^{j+1}$ acquire mass of order $M_{j}$, where $M_{j}$ is the scale of breaking of $G_{j}$.
\end{itemize}

A simpler statement of the ESH hypothesis is: only those particles which have to be light will remain light. this hypothesis is an extension of the survival hypothesis of Georgi which was designed ot explain how the fermions survive as light particles in GUTs like $SU_{5}$.

Let us fist illustrate the application of ESH hypothesis to symmetry breaking in $SU_{5}$. The symmetry breaking occurs in 2 stages:
$$
SU_{5} \xrightarrow[M_{u}]{} SU_{3c} \times SU_{2L} \times U_{1r} \xrightarrow[M_{L}]{} SU_{sc} \times U_{1Q}
$$
and the Higgs representations are given in table 4.1. The groups are indioated only by their indices. Thus, $SU_{3c} \times SU_{2L} \times U_{1Y}$ is indicated by $3c \times 2L \times 1Y$.
\begin{center}
\begin{tabular}{|c|c|c|c|}
\hline
 & 5 & 3c $\times$ 2L $\times$ 1Y & 3c $\times$ 1Q\\
 \hline
 $\phi$ & 24 & (1,1,0) &  \\
 H  & 5  & (1,2, ) & (1, 0)\\
 \hline
\end{tabular}\\[.2cm]
\centering{Table 4.1} 
\end{center}

Appling the ESH hypothesis, we see that all members of 24 acquire masses of order $M_{U}$. As for 5, (1,2,) of $3c \times 2L \times 1Y$ acquire mass of order $M_{L}$ while the rest of 5 acquire mass of order $M_{U}$. The physical reason behind this mass pattern is clear; it is connected to the fact that the Higgs acalars supply the longitudinal components of the gauge bosons at the vaious stages of symmetry breaking.

Let us next consider the $SO_{10}$ symmetry breaking chain $SO_{10}\xrightarrow[M_{u}]{} SU_{4c} \times SU_{2L} \times SU_{2R} \xrightarrow[M_{c}]SU_{3c} \times SU_{2L} \times SU_{2R} \times U_{1B-L}\xrightarrow[M_{R}] SU_{3c} \times SU_{2L} \times U_{1r} \xrightarrow[M_{L}]{} SU_{2c} \times U_{1Q}$
where we have contracted one of the links in Fig.4.2 by putting $M_{R}\approx M_{S-L}$. The various Higgs reps. are given in Table4.2.
\begin{center}
\begin{tabular}{|c|c|c|c|c|}
\hline
10 & $4 \times 2L \times 2R$ & 3 $\times$ 2L $\times$ 2R $\times$ 1QL & $G_{std}$ & $G_{Ex}$\\
\hline
$\rho$ 54 & (1,1,1) &  & & \\
$\xi$ 45 & (15, 1,1) & (1,1,1,0) & &\\
$\Delta_{R}$ 126 & (10,1,3) & (1,1,3, $\sqrt{\frac{3}{2}}$ )& (1,1,0) & \\
$\Delta_{L}$ 126 & (10, 3, 1) & (1,1,3, $\sqrt{\frac{3}{2}}$ ) & (1, 3, $\sqrt{\frac{3}{2}}$) & (1,0)\\
$\phi 126$ & (15,2,2) & (1,2,2,0)&(1,2,$\sqrt{\frac{3}{20}}$) & (1,0)\\
\hline
\end{tabular}\\[.2cm]
\centering{Table 4.2}
\end{center}

The application of ESH now leads to the following masses for the Higgs scalars. The Higgs representation are written down, without specifying the groups, which can be easily read off from the table 4.2.
\begin{align*}
&\rho (54) \sim M_{U}.\\
&\xi  (45) : (15,1,1) \sim M_{x};~~ 45~~ {\rm minus} (15,1,1) \sim M_{U}\\
&\Delta_{R}(126) : \left(1,1,3, \sqrt{\frac{3}{2}}\right) \sim M_{R}; (10,1,3)~~ {\rm minus}~~ \left(1,1,3, \sqrt{{\frac{3}{2}}}\right)\\
&\sim M_{c}; \Delta_{R} (126)~~ {\rm minus}~~ (10,1,3)\sim M_{u}.\\
&\Delta_{L}(126): \left(1,3, \sqrt{\frac{3}{5}}\right)\sim M_{L}; \left(1,3,1,\sqrt{\frac{3}{2}}\right) ~~{\rm minus}~~ \left(1, 3, \sqrt{\frac{3}{5}}\right) \sim M_{R};\\
& (10,3,2)~~ {\rm minus}~~ \left(1,3,1, \sqrt{\frac{3}{2}}\right)\sim M_{c}; \Delta_{L}~~ {\rm minus}~~ (10,3,1)\sim M_{u}.\\
&\phi (126) : \left(1,2, \sqrt{\frac{3}{20}}\right)\sim M_{L}; (1,2,2,0) ~~{\rm minus}~~ \left(1,2, \sqrt{\frac{3}{20}}\right) \sim M_{R};\\
&(15,2,2)~~ {\rm minus}~~ (1,2,2,0) \sim M_{c}; \phi (126) ~~{\rm minus}~~ (15,2,2)\sim M_{u}.
\end{align*}

The renormalization effects on $\sin^{2}\theta_{w}$ and $\alpha /\alpha_{s}$ can now be worked out for the multistap symmetry breaking of Fig.4.2. Except for $\rho$ (54), for all other Higgs multiplets of $SO_{10}$, there are large mass difference within the multiplet. Hence, in contrast to fermions, Higgs scalars necessarily contribute to the renormalization effects. The results are given below for the two cases (a) $M_{c} > M_{R}$ and (b) $M_{c}> M_{R}$, corresponding to the two subroutes indicated in Fig.4.2.
\begin{align*}
sin^{2} \theta_{w} &= \frac{3}{8} + \frac{\alpha}{16 \pi} \left[ \left(\frac{44}{3}-4 \right) {\rm ln} \frac{M_{u}}{M_{c}}\right.\\
& - \left(\frac{44}{3} + 6 \right) ln \frac{M_{c}}{M_{R}} - \left(\frac{110}{3} + 1\right)ln \frac{M_{R}}{M_{B-L}} - \left(\frac{110}{3}-1 \right)ln \frac{M_{B-L}}{M_{L}}\\
&1-\frac{B}{3} \frac{\alpha}{\alpha_{s}}= \frac{\alpha}{2\pi} \left[\frac{1}{3} (44-4) ln \frac{M_{u}}{M_{c}} + \frac{1}{3} (44 + 14) ln \frac{M_{c}}{M_{R}}\right.\\
&\left (22 + 3) ln \frac{M_{R}}{M_{B-L}} + (22 + \frac{7}{3}) ln \frac{M_{B-L}}{M_{L}}\right]
 \end{align*}
\begin{align*}
\sin^{2}\theta_{w} &= \frac{3}{8} + \frac{\alpha}{16 \pi} \left[ \left(\frac{44}{3}-4\right) ln \frac{M_{u}}{M_{R}}\right.\\
 &-\left.\left(\frac{22}{3} -12 \right) ln \frac{M_{R}}{M_{C}} - \left(\frac{110}{3} + 1 \right) ln \frac{M_{C}}{M_{B}} - \left(\frac{110}{3}-1\right) ln \frac{M_{B-L}}{M_{L}}\right]\\ 
 & 1-\frac{8}{3} \frac{\alpha}{\alpha_{s}} = \frac{\alpha}{2\pi} \left[ \frac{1}{3} (44-4) ln \frac{M_{U}}{M_{R}} + \left(22- \frac{4}{3} \right) ln \frac{M_{R}}{M_{C}}\right.\\
 &\left. (22 + 13) ln  \frac{M_{c}}{M_{B-L}} + \left(22  + \frac{7}{3}\right) ln frac{M_{B-L}}{M_{L}}\right]
\end{align*}

The above equations replace the simpler equation of sec.2 which were derived for the single step symmetry breaking with neglect of Higgs contributions, The second number within the brackets refers to the Higgs contribution and one can see that these contributions are quits substantial in some cases.

In view of the presence of many scale parameters, no unique prediction is possible now. However, using the values of $\sin^{2}\theta_{w}$ and $\alpha/\alpha s$ as input, one can work out allowed regions for $M_{U}$, $M_{R}$ etc, and then the consequences for B, L isolation can be investigated.

As an example, let us consider the possibility of Hydrogen-antihydrogen oscillation\footnote{I thank M.K. Parida and U. Sarkar for discussion on this topic.}. This is a $\Delta B =2$, $D L =2$ process, for which the lowest-dimensional operator required is $q^{6}$$l^{2}$ where q and l are the quark and lepton fields respectively. The mass dimension of this operator is 12 and has to be multiplied by $M^{-8}$ (where M is some mass scale to be determined). so that the effective Lagrangian given by
$$
\mathcal{L}_{eff} \sim \frac{q^{2}l^{2}}{M^{B}}
$$
will have the required mass dimension 4. The amplitude for $H-\bar{H}$ oscillation is thus determined by the scale parameter M. What is the value of M ?

If the theory contains two scalar multiplets S and T which are (6,3,2/3) and (1,3,-2) under $SU_{3C} \times SU_{2L} \times U_{1Y}$ respectively, the $H-\bar{H}$ oscillation can arise through the diagram depioted in Fig. 4.3. It can be seen that S and T have the quantum number of a diquark and a dilepton respectively. The amplitude for the oscillation can be eatimated to be $A (H \leftrightarrow \bar{H}) \sim \lambda f^{3}q fl/M^{6}_{s} M^{2}_{T}$.

figure???


Hence, the effective mass parameter occuring in the effective Lagrangian above is 
$$
M \sim M^{3/4}_{s} M^{1/4}_{T}
$$

Let us now see how to fix the masses of these scalars S and T. It turns out that both these scalars are present in the symmetry-breaking chain:
$$
G_{ps} \xrightarrow[M_{c}]{} SU_{3C} \times SU_{2L} \times SU_{2R} \times U_{1B-L}\xrightarrow[M_{R}]{} G_{std}\xrightarrow[M_{L}]{G_{Ex}}
$$


