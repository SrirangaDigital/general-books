\chapter[GUEST EDITORIAL]{Guest Editorial}\label{chap13}

\Authorline{G. Rajasekaran}
\addtocontents{toc}{\protect\contentsline{section}{{\sl G. Rajasekaran}\smallskip}{}}

\authinfo{Institute of Mathematical Sciences, Taramani,\\
Chennai 600 113, India and\\
Chennai Mathematical Institute, Siruseri,\\
Chennai 603 103, India\\ Email url{graj@imsc.res.in}}


\section*{A crisis in fundamental physics}

One hundred years of fundamental physics starting with
the understanding of the atom, the atomic nucleus and
what lies inside have culminated in a theory known as the
standard model of high energy physics (SM of HEP).
This theory has now been proved to be the correct law of
Nature valid down to a length scale of $10^{-17}$ cm, nine
orders of magnitude smaller than the atomic size, i.e.
$10^{8}$ cm. In this theory, the three fundamental forces of
Nature, namely electromagnetic, weak and strong forces
are incorporated in the dynamics of Yang–Mills gauge
field which is a generalization of Maxwell’s electro-
dynamics. So this theory is the fundamental basis of all
phenomena resulting from these three forces. SM is a
quantum field theory that combines quantum mechanics
and special relativity in a mathematically consistent way.

SM was constructed by theorists more than 40 years
ago. Experimenters verified each component of this
theory in the next 40 years, except for one component,
namely the Higgs boson. This last missing piece was dis-
covered only in 2012 and that was big news. This discov-
ery was made at the Large Hadron Collider (LHC) at
CERN, Geneva. Where do we go from here? SM is not
the end of physics. There are many things yet to be dis-
covered beyond SM, but the biggest omission in SM is
gravity, the gravitational force. Remember SM is a theory
of only three fundamental forces. The fourth, gravita-
tional force, which in some sense is the most fundamental
force is missing in SM.

It is a deep irony of Nature that the twin revolutions of
quantum and relativity that powered the conceptual
advances of the 20th century and that underlie all the sub-
sequent scientific developments, have a basic incompati-
bility between them. The marriage between quantum
mechanics and relativity has not been possible. By
relativity, here we mean general relativity, since special
relativity, has already been combined with quantum
mechanics leading to quantum field theory which has
been used in constructing SM. Gravity which gets sub-
sumed into the very fabric of space and time in Einstein’s
general relativity has resisted all attempts at being com-
bined with the quantum world. Hence, quantum gravity
has become the most fundamental problem of physics.

The most successful attempt to construct quantum
gravity is string theory, but there are also other candi-
dates competing for this honour, such as loop quantum
gravity. Actually, string theory offers much more than a
quantum theory of gravity. It provides a quantum theory
of all the other forces too. In other words, it can incorpo-
rate the SM of HEP also, within a unifying framework
that includes gravity. Indian physicists have made top-
level contributions in string theory which continues to at-
tract many bright students in the country.

In string theory, a point particle is replaced by a one-
dimensional object called a string as the fundamental
entity. Its length is about $10^{–33}$ cm, which is the length
scale of any theory of quantum gravity, including string
theory. The various vibrational modes of the string corre-
spond to the elementary particles. String theory automati-
cally contains quantum gravity and that is its speciality.
However, it is bought at a price. It works only if the
number of space dimensions is nine and including time it
is 10. Where are the extra six dimensions? They are
curled up to form space bubbles at distance scales of the
same $10^{–33}$ cm. Both the string and the extra curled-up
dimensions will be revealed only when we can access
such length scales.

Strings live in a 10-dimensional world having six com-
pact space dimensions in addition to our familiar four-
dimensional space–time. We now know that in addition
to the one-dimensional strings, string theory automati-
cally contains two-dimensional membranes and branes of
higher dimensions too. String theory is the relativistic
quantum dynamics of a mind-boggling variety of interact-
ing extended objects (analogues of chairs, tables, cars,
etc.) living in a 10-dimensional world. It has rich mathe-
matics and physics. Its richness is continuously being
discovered. No wonder, string theory is so difficult. But it
can be mastered through mathematics. String theory also
requires creation of new mathematics.

To sum up, string theory is the top candidate for a cor-
rect theory of quantum gravity, which is the next frontier
in fundamental physics after the spectacular success of
the SM of HEP.

So has string theory solved the problem of quantum
gravity? Perhaps yes, but how do we know? Where is the
experimental support for string theory? Remember it took
40 years to verify SM as the correct theory of Nature. It
required the construction of particle accelerators of
higher and higher energy ultimately culminating in the
construction of the LHC reaching energies in the trillion
electron volt (TeV) region. This machine is a behemoth.
It is a circular ring of 28 km in circumference and its
construction took 20,000 physicists and engineers work-
ing for 20 years. In relativistic quantum mechanics there
is an inverse relationship between the length scale and the
energy required to probe it. Remember that we have de-
scended down to a length scale of $10^{–17}$ cm. To probe
this, we needed the TeV energies of LHC. So, to probe
the length scale of quantum gravity ($10^{–33}$ cm), we need
16 orders more energy, i.e. $10^{16}$ TeV. This is the Planck
energy, that is required to experimentally test string the-
ory or any theory of quantum gravity. Most people think
this is not possible. This is the crisis in fundamental
physics.

If current ideas in cosmology are correct, then the early
Universe provides us with a HEP laboratory where parti-
cle energies were almost unlimited. So it is believed by
many that our theories of quantum gravity can be tested
by appealing to events in the early Universe. At the risk
of getting a flak from many of my respected colleagues, I
would like to strike a note of caution. We know of only
one Universe and the events presumably occurred only
once, that too quite a long time ago. Modern science
owes its existence to the advent of repeatable experiments
under controllable conditions, whereas history provides
only a single sequence of events. History cannot be a
substitute for science.

Galileo decreed ‘Laws of Physics are written in the
language of Mathematics, but those Laws can be proved
or disproved only by Experiments or Direct Observa-
tions’. For 400 years, physics has progressed only by fol-
lowing the path opened by Galileo. If we give up this
path now, that will be the end of fundamental physics.
Then all the theories that we build for quantum gravity
will remain as mere metaphysics. What is the way out?

Actually this pessimism is unwarranted. Human inge-
nuity knows no bounds. The energy barrier will be
crossed. Instead of merely scaling up the sizes of the
accelerating machines, we must discover new principles
of particle acceleration. Either new principles of accelera-
tion have to be discovered, or there will be an end to HEP
by about 2040. Actually this conclusion has nothing to do
with quantum gravity or Planck energy. Conventional ac-
celerator technology cannot take us above a few orders of
magnitude beyond the present TeV energies.

Growth of accelerator energies over the past 80 years
has been phenomenal. The energy has been increasing by
a factor 10 every 6 years. I interpret this exponential
growth as an optimistic sign for the future of fundamental
physics. So, if the same growth can be maintained, 16
powers of 10 can be reached in 96 years. It is long, but
not infinite. Even SM took 40 years to be experimentally
proved. But this growth is possible only if new principles
of acceleration and newer technologies are continuously
invented. I have been saying all this for the past many
years and advocating pursuit of newer accelerator tech-
nologies in our country.

What are the new principles of acceleration? I will give
one example. In the last 30 years, many ideas on laser-
plasma acceleration are being pursued. Using laser
excitation of plasma wakefields, electrons have been suc-
cessfully accelerated to 1 GeV in 1 cm (compared to
kilometre-size conventional accelerators to get similar
energies). So table-top accelerators are perhaps not far
way. Maybe this will lead to breakthroughs that will help
us to cross the super-high energy barrier. It is not claimed
that laser-plasma accelerator will take us Planck energy,
but it will be an important step.

Our country must enter into this field of laser-plasma
acceleration. I have been discussing this matter with the
leading laser and plasma physicists in IPR, BARC, TIFR
and RRCAT, and they are quite enthusiastic. One may
envisage the following plan of action: (1) Set up a Natio-
nal Task Force for planning R\& D for the laser-plasma
accelerator. (2) Build a National Centre for Laser-Plasma
Accelerator.

To succeed in this venture we need all the good will
and help from the international scientific community. The
International Conference on Ultrahigh Intensity Lasers
held in Goa during 12–17 October 2014, revealed the
worldwide progress in the field and also showed that
India can expect good support from the world if we take
up this challenge. So this is the right time to take the
plunge.

This megascience project will need the joint support of
many scientific wings of the Indian Government. This is
an ambitious plan. It will work only if we think big and
dream big. We have to develop an appropriate mindset
and a collaborative spirit to succeed.
\begin{flushright}
{G. Rajasekaran}
\end{flushright}
Institute of Mathematical Sciences, Taramani,\\
Chennai 600 113, India and\\
Chennai Mathematical Institute, Siruseri,\\
Chennai 603 103, India\\
e-mail: \url{graj@imsc.res.in}
