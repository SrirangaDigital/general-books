\chapter{Dr Abdul Kalam and the India-based Neutrino Observatory}\label{chap19}

\Authorline{G Rajasekaran}
\addtocontents{toc}{\protect\contentsline{section}{{\sl G Rajasekaranr}\smallskip}{}}

\authinfo{Institute of Mathematical Sciences, Chennai\\    
          and Chennai Mathematical Institute, Siruseri}
    

India-based Neutrino Observatory (INO) is India's megascience project
already approved by the Government of India with an outlay of about
Rs 1400 crores. It has many components: an under-the-mountain laboratory
in Theni District in Tamil Nadu, the Inter-University Centre for High
Energy Physics (IICHEP) on the outskirts of Madurai and more than 120
physicists, engineers and students from 25 research institutions, 
Universities and IITs spread over the length and breadth of the country.
The Government of Tamil Nadu has already given land in Theni District
and Madurai for this project.

The following is a brief account of the interaction of INO scientists with
Dr Abdul Kalam and his support for INO.

I met Dr Kalam for the first time in 2009. We had invited him for a talk
at the Institute of Mathematical Sciences at Chennai during the Birth 
Centenary celebrations of Homi Bhabha. We discussed INO with him and
asked for his support. After that we met Kalam many times for his help
and support in our continuing INO-outreach activities among people.

\vspace{.2cm}

On 11 January 2010, Dr Naba Mondal (Tata Institute of Fundamental Research,
Mumbai), Dr Vivek Datar (Bhabha Atomic Research Centre, Mumbai) and myself
met Dr Kalam at his residence in Delhi and discussed all aspects of INO with 
him for almost two hours. We gave him many articles on INO and we asked him
for his opinion that we can take to the people. He said he will do it
a little later after digesting the material that we had presented to him
orally as well as in written form.

In the first week of February Dr Chinnaraj Joseph (the then Principal of
The American College, Madurai and the Chairman of the Madurai-Theni INO
Outreach Cell) sent us detailed documents to us listing the steps that
we needed to take in dealing with the village situation. One of the points
he made, was, in his own words: " I hear from the village representatives
and my team that the people are very curious about the stand taken by our
former President Dr Abdul Kalam. Everybody here from within and outside
believe that Kalam's words would do real magic."

So, I wrote a detailed letter to Dr Kalam renewing our request in the light
of the urgency of the village agitations that we faced. When I learnt that
he was coming to Chennai on 11 February, I wrote to him seeking an
appointment. This time his response was immediate and positive. All our
earlier work with him was bearing fruit. When Dr MVN Murthy and Dr D Indumathi
(both from Institute of Mathematical Sciences, Chennai) along with me met
him on 11 February at Raj Bhavan, he had his answer ready in his hand and
read it to us. This was his reply to a letter from a student of Gandhigram
Rural University and it contains the following statements:

"Scientific institutions and Colleges around the Theni area will be
reinforced through the neutrino project. Just as CERN is famous for
its Large Hadron Collider project, Theni and surrounding region will
become famous for neutrino particle physics experiments. I expect 
great scientific and technological activity in the project site
and the neighboring academic institutions."

Dr Kalam gave us permission to spread this among people. Dr Jeyapragasam of
Madurai helped us to publish it in the Tamil daily "Dinamalar" on 27 February.
When I was addressing a meeting in the Theni region I read Kalam's
letter and immediately there was an elecrifying reaction from the people.
We used Kalam's statements in many INO outreach meetings.

Kalam continued to support INO. Finally on 17 June 2015 he along with
Srijan Pal Singh published an article on INO in The HINDU
in which he cleared the public misconceptions about INO and gave his
strong opinion stressing the importance of INO for the scientific advance
of India. We learnt that Dr Kalam had planned to visit many Colleges in Theni
District and Madurai and give talks about INO to the students. But
fate decided differently. On 27 July Kalam passed away.

People must take the statements in his article on 17 June
to their heart. " We must dream. It is dreams that will
take India forward." These are the famous words of Kalam repeated  
by him on many occasions. INO is one such dream. We must make this
dream come true. We must defeat the forces of ignorance opposing INO,
through the weapon of rational thought. INO must be
established in Tamil Nadu. The Government of Tamil Nadu, the people
of Tamil Nadu and especially the students must make every effort
towards realising this dream project.  
