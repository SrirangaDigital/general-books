\chapter{Are Neutrinos Majorana Particles?}\label{chap9}
{\centering\hspace{3.5cm} (Keynote Address)}

\vspace{.2cm}
\Authorline{G. Rajasekaran}
\addtocontents{toc}{\protect\contentsline{section}{{\sl G. Rajasekaran}\smallskip}{}}
\lhead[\small\thepage]{\small\leftmark}

\authinfo{Institute of Mathematical Sciences,\\ Madras 600113.\\ e-mail: graj@imsc.res.in} 

Dirac introduced the concept of antiparticles while trying to solve the negative energy
problem in the famous relativistic equation for the electron that he had discovered in 1928.
Now we know that the existence of antiparticles is one of the most important consequences
of combining quantum mechanics with special relativity. For every particle there exists
an antiparticle. However some particles could be self-conjugate, in the sense that particle
and antiparticle could be the same. Of course such particles have to be electrically neu-
tral. Among the elementary particles of the Standard Model, photon and the Z boson are
self-conjugate. Both these are bosons. The possibility of a self-conjugate fermion was first
pointed out by Majorana in 1937 and hence they are called Majorana fermions while the
other fermions (with distinct particles and antiparticles) are called Dirac Fermions. Among
the fermions of the Standard Model, only neutrinos are electrically neutral and hence qualify
to be Majorana particles. But it is still an open question whether neutrinos are Majorana
particles or Dirac particles, in other words, whether neutrino is a self-conjugate particle or
not.


This question can be shown to be a pure semantic one if neutrinos are massless. In that
case one can prove, by renaming the left-handed neutrino and the right-handed antineutrino
suitably, that there is no physical distinction between Dirac and Majorana neutrinos. This
is sometimes called ”the confusion theorem”. But after the discovery of neutrino mass in
the last decade, we know that such a distinction exists and physicists must determine the
category to which neutrino belongs.

There are important reasons why theoreticians prefer Majorana neutrinos. If neutrinos
are Majorana particles, there exists an elegant mechanism called sea-saw to explain why
the neutrino masses, although not zero, are so tiny. Further, if neutrinos are Majorana
particles, lepton number L is not conserved and this opens the door to generate an excess
of leptons over antileptons in the early universe which can subsequently generate an excess
of baryons over antibaryons, thus explaining how after annihilation of most of the particles
with antiparticles, a finite but small residue of particles was left, to make up the present
Universe. Hence the fundamental importance of the question whether neutrinos are Majo-
rana particles is clear.

Inspite of the great attractiveness of the idea that neutrinos could be Majorana parti-
cles, it is only a theoretical idea. It has to be either confirmed or refuted by experiment.
At the present stage, the only experiment that can answer this question is the neutrinoless
double beta decay (NDBD). In neutrinoless double beta decay,the two Majorana neutrinos
that are virtually produced can annihilate each other leaving only two electrons in the final
state, thus violating lepton number by two units. So establishment of the existence of this
decay would be proof of the Majorana nature of the neutrino and violation of lepton number.

\vspace{.2cm}

Usual double beta decay (in which two neutrinos along with two electrons are emitted)
itself is a very rare process because it is doubly weak as compared to the standard beta
decay and further the phase space is suppressed by the larger number of particles in the
final state. Inspite of its rarity, this decay has been experimentally detected and now well
studied. However as we saw above, the signature for the Majorana character of the neutrino
is the absence of the two neutrinos and this process without neutrinos in the final state has
not yet been detected. Although the neutrinoless double beta decay has reduced number of
particles in the final state and hence more phase space, it is a much rarer process than the
two neutrino decay because the decay amplitude is proportional to the tiny neutrino mass
(consistent with the confusion theorem).

Actually the decay amplitude for neutrinoless double beta decay is proportional to a
neutrino mass factor mee which is a linear combination of the masses of the three neutrinos.
This linear combination also involves the neutrino mixing angles and the very important CP
violating phases. Of course this has to be multiplied by the relevant nuclear matrix element
to get the decay amplitude. Thus if neutrinoless double beta decay is detected and the rate
is measured and if the nuclear matrix element is known, one can then extract important
information on the neutrino parameters. Especially to be noted is the fact that mee contains
the overall scale of the neutrino masses and this mass-scale is not obtainable from neutrino
oscillation experiments. However the success of such an extraction of neutrino parameters
from NDBD depends very crucially on our knowing the nuclear matrix elements and hence
nuclear theory will play an essential role in NDBD activity.

Can the NDBD neutrino mass factor $m_{ee}$ be zero? Unfortunately the answer is yes. The
particular linear combination of neutrino masses that enters here may be zero (or very small)
for some reason and then the rate for NDBD may be vanishingly small. In this unfortu-
nate case, NDBD experiments will not throw any light on whether neutrinos are Majorana
particles. One may have to go to processes such as a $\mu^{-}$ colliding with a nucleus leading
to a $\mu^{+}$ in the final state. The amplitude for this reaction is proportional to a different
linear combination of neutrino masses which hopefully is not zero. But this lepton number
violating reaction will be even harder to study than NDBD.

Is it possible to enhance the rates for the lepton number violating processes by some
mechanism, for instance by shining an intense laser? Such ideas are being considered and
one of them might work. But for the present let us return to NDBD.

Inspite of the many experiments that have been mounted for searching for neutrinoless double beta decay, none has borne fruit sofar. However,in 2004 Klapdor and his group
reported observing the decay in $^{76}Ge$. This created quite an excitement because of the
importance of such a discovery. However this result was soon controverted by many critical
physicists. I was reading Galileo’s biography when this news came. My mind went back
by 400 years to the time when Galileo was facing his sceptical opponents who refused to
believe that Galileo really saw the things that he claimed to have seen through his spy glass.
Of course times have changed and the situation is quite different. In any case there is no
religious dogma concerning the nature of the neutrino! (There is more about Galileo later
in this article.)It is very important to settle the issue by independant experiments.

The National Workshop on ”Neutrinos in Nuclear,Particle and Astrophysics (NUPA04)”
held at IIT,Kharagpur in Feb 2004 was the first such workshop covering the impact of the
recent discoveries of Neutrino Physics in all the three fields of Nuclear Physics, High Energy
Physics and Astrophysics. In that Workshop I stressed the importance of the neutrinoless
double beta decay experiment and suggested that the Double Beta Decay Theorists Prof P
K Raina and Prof P K Rath must organize and coordinate the activity that could lead to
mounting a Neutrinoless Double Beta Decay Experiment in the INO cavern. Their magnif-
icent enthusiasm led to two focussed Workshops DBD05 at IIT,Kharagpur in March 2005
and NDBD05 at Lucknow in November 2005. More importantly, they have succeeded in
bringing two excellent experimenters Vivek Datar and RG Pillay and their groups into the
NDBD project and this augurs well for the ultimate success of the project.We must acknowl-
edge the important role played by the late CVK Baba who was a pillar of strength for the
project because of his wisdom and experience. His absence is missed very much.

Utpal Sarkar and VKB Kota brought the experimenters and theorists together for the
NDBD07 at Ahmedabad in February 2007. Now in October 2007, we are here for NDBD07
at Mumbai, thanks to Vandana Nanal and her Organizing Committee.

\section*{Roadmap}

Let me close this talk with the following remarks relevant to the roadmap ahead:

1. The fundamental importance of neutrinoless double beta decay must be stressed again
and again. Repetition of this manthra is not a waste of time since the most important un-
known in all of neutrino physics is the answer to the question whether neutrino is Majorana
or Dirac. The answer will have a bearing on High Energy Physics as well as Cosmology. At
the present time, NDBD experiment is the only way to answer it.

2. Hence at the India-based Neutrino Observatory (INO), NDBD must be pursued with
full vigour. Although the major focus of the INO as of now is to construct the magnetised
iron calorimeter (ICAL) to be used for atmospheric neutrinos (Phase I) and neutrinos from
muon storage rings and beta beams (Phase II), parallel processing of NDBD must go on.
NDBD activity must soon gain sufficient strength so that the NDBD detector will become
as (if not more) important as ICAL to INO. But this requires considerable spadework and
R and D.


3. It is absolutely essential to scout for good experimenters and augment the NDBD
group. We must scan the whole spectrum of likely candidates in research institutions and
universities inside and outside the country and attract them to NDBD. Also we must not
restrict ourselves to HEP and NP experimenters only. We must interact with atomic physi-
cists, condensed matter physicists, material scientists, chemists, engineers... All of these can
make useful contributions to mounting a viable NDBD experiment in the country. Students
must form an important component of the human resource that we seek for NDBD.The task
is a gigantic one, but it can be done and it must be done.

4. Finally we must mention that experiments on the search for dark matter have reached
very high sensitivities and are becoming capable of detecting it, if it really exists in the form
and abundance generally expected. Hence we must include the possibility of the Indian
NDBD project leading to an Indian Dark Matter project in the future.

I now add a few remarks which, although not on NDBD, will be about neutrinos and
some history.

\section*{A Majorana Puzzle}

Dirac Equation has negative energy solutions. Dirac solved the negative energy problem
filling the negative energy sea (Dirac sea) using Pauli exclusion principle. But then he had
to explain the possible vacancy or hole in the sea. So he had to predict the antiparticle:
Hole = Antiparticle. This is the famous Hole Theory of Dirac.

What happens to all this, for Majorana particles? Majorana particle also satisfies Dirac
equation and so there are negative energy solutions. How is the negative energy problem to
be solved? Answer is given at the end of the article.

\section*{Mossbauer Effect for Neutrinos}

This is a very interesting idea by R S Raghavan (R S Raghavan,hep-ph/0601079). Con-
sider the following two processes:
\begin{align}
^{3}H \rightarrow \bar{\nu}_{e} & + ^{3}He + e^{-}(bound)\\
\bar{\nu}_{e} +^{3}He & + e^{-}(orbital)\rightarrow^{3} H
\end{align}


The first process is the beta decay of tritium, but the electron in the final decay product
is bound to the $^{3}He$ nucleus. Hence it is a two-body decay with the antineutrino emitted
with unique energy 18.6 keV. The second process is just the inverse in which the same 18.6
antineutrinos are used to cause the capture reaction which is again a two-body reaction
with the initial electron as an orbital electron in $^{3}He$. For process (1) embed $^{3}H$ in fcc
metal tritide and do the same for the initial $^{3}He$ in process(2). Thus nuclear recoil is completely avoided in both processes just as in Mossbauer effect and by using the antinuetrinos
emitted in process (1) to initiate process(2), one is achieving recoilless resonant capture of
antineutrinos. Raghavan estimates the resonance width and gets from which he calculates the recoilless resonant capture crosssection for process (2) to be $\sim 5\times 10^{-32} cm^{2}$
. This is 10 orders of magnitude larger than the typical capture crosssection
of antineutrinos on protons which is $\sim 10^{-42} cm^{2}$.

\begin{equation}
\frac{\Delta E}{E} \sim 2 \times 10^{-17}\label{eq-3}
\end{equation}


We thus have the tools to do ultraprecise very low energy neutrino experiments. We have
a monochomatic antineutrino beam from process (1) which can be detected with a very high
crosssection. Before detection, the antineutrinos can be made to fall through a gravitational
field and thus the gravitational red-shift of  $\bar{\nu}_{e}$ can be measured. Flavour oscillations in tabletop experiments with 1gm to 1 Kg materials can be observed. This is the route to Precision
Neutrino Physics. This will revolutionize Neutrino Physics. Of course all this is possible
only if the challenges involved in the physics and technology of the embedding mentioned
above can be met.

\section*{Directed Monoenergetic Neutrino Beam?}

We envisage a directed monochromatic neutrino beam of low energy. A possible way of
realising it was considered by R S Raghavan. There exist proposals to make high energy
beams of monochromatic neutrinos by accelerating nuclei that undergo electron capture.
(See J Sato, hep-ph/0503144, J Bernabeu, hep-ph/0505054). These will not however have
the advantage of the recoilless resonant capture. So let us consider the bound-state beta
decay again:
\begin{equation}
^{3}H \rightarrow \bar{\nu}_{e} ^{3}He + e^{-}(bound)\label{eq-4}
\end{equation}
and let us use a magnetic field to polarize the nuclei $^{3}H$ and $^{3}He$. Let z be the direction of
the magnetic field. We consider angular momentum conservation. For the initial state, we
have $^{3}H$ of spin 1/2 and the 1s atomic electron of spin 1/2. By using the external magnetic
field to polarize the nucleus and the nucleus to polarize the electron through hyperfine
interaction we can prepare the initial state such that the total spin of the atom is in the
triplet state with the z-component having value +1.

For the final state, the $^{3}He$ nucleus also has spin 1/2, but the two atomic electrons are
in the filled 1s shell thus contributing zero angular momentum. As for the antineutrino, in
allowed beta decay only S-wave antineutrino participates, for the higher partial waves are
suppressed by $kR \sim 10^{-4}$, where k is its momentum and R is the nuclear radius. Hence,
to balance the angular momentum, the $^{3}He$e nucleus and the antineutrino must have their
$z$-component of spins +1/2 making up the total z-component as +1, which is the initial
value.

But if the z-component of the spin of the antineutrino is +1/2, its momentum also has
to be in the z direction, since antineutrinos have unique helicity. Hence the antineutrinos
emitted in this process in the presence of a sufficiently strong magenetic field are all emitted
along the $z$-direction. In other words, we have a monoenergetic unidirectional (anti)neutrino
beam! Now combine it with the recoilless emission and absorption of neutrinos (Mossbauer
effect), with consequent enormous enhancement in production and detection rates, as already
described. This is the Ultimate Neutrino Device.

One can easily imagine any number of fantastic applications with such a device. By
having pulsed polarizing magnetic fields, even neutrino-communication through the Earth
to the antipodes is possible.

Does this make sense? See the answer at the end of the article.

\section*{More on Galileo}

I have drawn the analogy of looking for the signature of Majorana neutrinos in NDBD
detector with Galileo’s looking for the signature of the new Astronomy in his spyglass. The
analogy may prove even closer in view of the cosmological significance of the Majorana
nature of neutrinos. Because of this and also because of the colourful picture of the Galileo
episode that Arthur Koestler paints in his book ”The Watershed”, I am tempted to quote
an excerpt from there. After describing the great impact of Galileo’s discoveries with his
optic tube on the world at large, Koestler says:

”But to understand the reactions of the small academic world in his own country, we
must also take into account the subjective effect of Galileo’s personality. Copernicus had
been a kind of invisible man throughout his life. Nobody who met the disarming Kepler,
in the flesh or by correspondence, could seriously dislike him. But Galileo had a rare gift
of provoking enmity - not the affection alternating with rage which Tycho aroused, but
the cold, unrelenting hostility which genius plus arrogance minus humility creates among
mediocrities.

”Without this personal background, the controversy that followed the publication of the
Sidereus Nuncius would remain incomprehensible. For the subject of the quarrel was not
the significance of the Jupiter’s satellites, but their existence, which some of Italy’s most
illustrious scholars flatly denied. Galileos’s main academic rival was Magini in Bologna.
In the month following the publication of the Star Messenger, on the evenings of April
24 and 25, 1610, a memorable party was held in a house in Bologna, where Galileo was
invited to demonstrate the Jupiter moons in his spyglass. Not one among the numerous and
illustrious guests declared himself convinced of their existence. Father Clavius, the leading
mathematician of Rome, equally failed to see them; Cremonini, teacher of philosophy at
Padua, refused even to look into the telescope; so did his colleague Libri. The latter,
incidentally, died soon afterward, providing Galileo with one more opportunity to make
more enemies with the much-quoted sarcasm: ”Libri did not choose to see my celestial
trifles while he was on earth; perhaps he will do so now he has gone to heaven.”

”These men may have been partially blinded by passion and prejudice, but they were
not quite as stupid as theIn any case, we have discussed the recoilless reactions for neutrino physics and the aborted
proposal to make a low energy unidirectional monochromatic (anti)neutrino beam, mainly
to emphasize the importance of ingenious ideas to take neutrino physics further. What we
need are a hundred crazy ideas. Maybe one of them will work and help us to sove the
neutrino mystery.y may seem. Galileo’s telescope was the best available, but it was
still a clumsy instrument without fixed mountings, and with a visual field so small that, as
somebody has said, ”the marvel is not so much that he found Jupiter’s moons, but that he
was able to find Jupiter itself.” The tube needed skill and experience in handling, which none
of the others possessed. Sometimes a fixed star appeared in duplicate. Moreover, Galileo
himself was unable to explain why and how the thing worked; and the Siderius nuncius
was conspicuously silent on this essential point. Thus it was not entirely unreasonable to
suspect that the blurred dots which appeared to the strained and watering eye pressed to
the spectacle-sized lense might be optical illusions in the atmosphere, or somehow produced
by the mysterious gadget itself. This, infact, was asserted, in a sensational pamphlet, Refutation of the Star Messenger, published by Magini’s assistant, a young fool called Martin
Horky.”

\section*{Answer to the Majorana Puzzle}

Majorana spinor does not have a well-defined energy eigenvalue. It is not the eigenfunction of the single-particle Hamiltonian. A many-particle description (field quantization) is
neccessary for a correct understanding of the majorana particle, just as in the case of the
spin-0 Klein-Gordon particle. (This was pointed out in my article on Dirac in Dirac and
Feynman:Pioneers in Quantum mechanics, Edited by Ranabir Dutt and Asim K ray, Wiley
Eastern Ltd, 1993,p 9.)

The above of course has nothing to do with the well-known Majorana puzzle: the tragic
disappearance of Ettore Majorana in 1938. On Enrico Fermi’s appeal, Benito Mussolini
had mobilised the whole state resources to search for Majorana, but to no avail. There is
some evidence that Majorana joined a monastery. Was he disappointed that he could not
make important contributions? Would the importance of Majorana neutrinos to physics and
cosmology (that is now recognized) have changed his mind?

\section*{Answer to the Unidirectionality Question}

The argument that led to the unidirectionality of the (anti)neutrino is wrong. The crucial assumption in the argument was that the (anti)neutrino was emitted in S-wave; a pure
S-wave is not possible and this is a consequence of quantum mechanics and relativity.

The Dirac wavefunction for a particle with well-defined momentum p can be written as
\begin{equation}
\psi = e^{i \vec{p}\cdot \vec{r}}\begin{pmatrix}
u\\
v
\end{pmatrix}\label{eq-5}
\end{equation}
where $u$ and $v$ are two-component spinors and $v$ is given by
\begin{equation}
v = \frac{\vec{\sigma} \cdot \vec{p}}{E + m}u\label{eq-6}
\end{equation}
For $m \approx 0$, $E=p$ and so $v$ becomes
\begin{equation}
v=\frac{\vec{\sigma} \cdot \vec{p}}{p}u\label{eq-7}
\end{equation}
More generally, if energy is well-defined, but direction of $\vec{p}$ is not well-defined, the above
should be rewritten as
\begin{equation}
\psi = \begin{pmatrix}
F(\vec{r})\\
G(\vec(r))
\end{pmatrix}\label{eq-8}
\end{equation}
where $F(r)$ is a 2-component wavefunction satisfying
\begin{equation}
(\nabla^{2} + p^{2}) F(\vec{r}) =0\label{eq-9}
\end{equation}
and
\begin{equation}
G(\vec{r}) = -i \frac{\vec{\sigma \cdot \vec{\nabla}}}{p} F(\vec{r})\label{eq-10}
\end{equation}

Because of the presence of $\vec{\sigma} \cdot \vec{\nabla}$ in the wavefuncion, it is clear that a pure $S$-wave is not
possible. Even if we take $F(r)$ to be a pure $S$-wave (no angular dependance), $G(r)$ will
contain a $P$-wave. Thus, a pure $S$-wave is not possible for a relativistic fermion.

For an antineutrino of mass zero, we can now impose the helicity condition
\begin{equation}
\frac{i\vec{\sigma \cdot \vec{\nabla}}}{p}\psi = \psi.\label{eq-11}
\end{equation}

It is important to note that that this is the correct way of writing the helicity condition
which is more general than the usual condition
\begin{equation}
\frac{\vec{\sigma \cdot \vec{p}}}{p}\psi = -\psi\label{eq-12}
\end{equation}
which is valid only when the momentum direction is well-defined. One can easily verify that
the wave function satisfying the correct helicity condition of Eq.~\eqref{eq-11} takes the form
\begin{align}
\psi &= \begin{pmatrix}
H(\vec{r})\\
-H(\vec{r})
\end{pmatrix}\label{eq-13}\\
H(\vec{r}) &= (1+ i\frac{\vec{\sigma \cdot \vec{\nabla}}}{p}) K (\vec{r})\label{eq-14}
\end{align}
where $H(\vec{r})$ and $K(\vec{r})$ are two-component wavefunctions and
\begin{equation}
(\nabla^{2} + p^{2}) K(\vec{r}) = 0.\label{eq-15}
\end{equation}

Eq.~\eqref{eq-13} gives the form of the wavefunction $\psi$ that must be kept in mind while discussing
angular momentum conservation. While considering eigenstates of angular momentum, we
cannot use eigenstates of linear momentum $\vec{p}$, since they do not commute with each other.
On the other hand, as far as eigenstates of angular momentum are concerned, pure S-wave is
impossible; even if we take $K(r)$ to be a function of r alone without any angular dependance
it is clear that the combination that occurs in $\psi$ will contain both S and P waves. With
P-wave present, our argument falls to the ground.

In the above discussion, we have put the neutrino mass to be zero and this approximation
is sufficient since the mass is very small compared to its energy 18.6 keV. If we want to
consider a purely theoretical case of the neutrino kinetic energy comparable to or smaller
than its mass, a more refined analysis is required, but in this case, even the helicity condition
(Eq.~\eqref{eq-11} or ~\eqref{eq-12}) is not valid and so there is no argument for the (anti)neutrino emission in
a single direction.

Thus a correct understanding of neutrino requires both quantum mechanics and relativity. Further, to say that the spin of the neutrino is pointing in the direction of its motion
is not always correct. The correct statement is Eq.~\eqref{eq-11}. These points may have some
pedagogical value since they might not have appeared in textbooks.

In any case, we have discussed the recoilless reactions for neutrino physics and the aborted
proposal to make a low energy unidirectional monochromatic (anti)neutrino beam, mainly
to emphasize the importance of ingenious ideas to take neutrino physics further. What we
need are a hundred crazy ideas. Maybe one of them will work and help us to sove the
neutrino mystery.
