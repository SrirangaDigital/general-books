\chapter[Is Quantum Mechanics Forever? ]{Is Quantum Mechanics Forever?}\label{chap8}  

\Authorline{G. RAJASEKARAN\footnote{On leave of absence from Institute of Mathematical Sciences, Madras}}  


\addtocontents{toc}{\protect\contentsline{section}{{\sl G. RAJASEKARAN } \smallskip}{}}
\lhead[\small\thepage]{\small\thechapter. }

\authinfo{Tata Institute of Fundamental Research,\\ Homi Bhabha Road, Bombay 400005, India’ }

\section*{Abstract}

Noting that theoretical high energy physicists have already constructed theories de- scribing all of physics including gravity and that these theories are characterized by the the question is posed whether quantum mechanics will continue length scale ~ $10^{-33}$ cm,
An attempt is made to focus attention on those aspects of the to be valid in these realms. recent developments in high energy physics which have a bearing on this question. 

Talk delivered at the meeting on philosophical Foundations of Quantum Theory, March 24-26, 1988, New Delhi (to appear in the Proceedings; eds Ranjit Nair, World Scientific.

\section{Introduction}

Classical Mechanics was eminently successful for macroscopic physics. Quantum mechanics was discovered in our attempt to describe physical phenomena at the atomic scale,
namely about $10^{-8}$ cm. The same quantum mechanics coupled with special relativity has
been successfully applied in our journey towards even deeper regions and, at present, high
energy physics experiments probe down to $10^{-16}$ cm (or a little less) in the subnuclear world. Theoretical physicists have gone much further in their imagination; they have already constructed theories claimed to be relevant at $10^{-33}$ cm! Will quantum mechanics continue to be valid in these realms? Will it reign supreme for ever? Or will it be superceded by a newer mechanics to be discovered? These are the questions which will be
posed in this talk, but, of course, no answers will be given! 

It is the struggle to apply the laws of electrodynamics to atoms that gave
birth to quantum mechanics and later to quantum electrodynamics which is a relativistic quantum
field theory. Now we have enlarged the domain of applicability of quantum mechanics
to cover the weak and strong nuclear forces as well. In fact electro
dynamics itself has
been incorporated into the enlarged electroweak dynamics which
combines electromag-
netism and the weak force into a single theory. And the strong |nuclea
r force is believed
to be a manifestation of the so-called quantum chromodynamics,
Quantum electroweak
dynamics together with quantum chromodynamics constitute
the so-called gauge theory
or the standard model of high energy physics which can be now
regarded as the basis of
all of physics except gravity. The energy scale probed so far
in these phenomena i is upto
about 100 GeV which corresponds to about $2 \times 10^{-16}$ cm in the
length scale. (The energy
and length scales have an inverse relationship in quantum
mechanics and table 1 gives
some of the important land-marks.)

The established part of physics stops at the standard model.
In their attempt to probe
deeper regions of space-time beyond $10^{16}$ cm (or 100 GeV),
theoretical physicists have
invented many ideas such as supersymmetry and higher dimen
sions and constructed new
theories such as unification, supergravity or superstrings.

Unification is an attempt to unify the two parts of the standa
rd model, namely elec-
troweak dynamics and chromodynamics. This already causes
a big Jump in the energy
scale from 100 GeV to $10^{15}$ GeV, and thus the unification scale
is $10^{15}$ GeV or $10^{-29}$ cm.

Supergravity, higher-dimensional theories and superstring theory are attempts at con-
structing a theory of quantum gravity and among these, superstring theory promises to be
successful. The scale of quantum gravity is the Planck length:
$$
\ell P_{\ell} = \left(\dfrac{\hbar G_{N}}{c^{3}} \right)^{1/2} ~10^{-23} {\text{cm}}
$$

or the Planck mass:

$$
m_{p \ell} = \left(\frac{\hbar c}{G_{N}} \right)^{1/2} ~10^{19} \text{GeV}
$$

where $G_{N}$ is Newton’s constant, $c$ is velocity of light and $\hbar$ is Planck constant.

In this talk we focus attention on those aspects of the recent developments in theoretical
high energy physics which may have a bearing on the question: Is quantum mechanics
for
ever? Only as much of detail will be provided as is necessary to serve as a background for
the questions to be posed.$^{1}$)

It must be kept in mind however that none of the ideas and theories beyond the
standard model has an iota of experimental support 80 far. Revolutionary developments
in technology and experimental technique may be required before a direct test
of those
ideas and theories taking us to $10^{15}$ GeV or $10^{19}$ GeV in he energy scale becomes possible.
Until that happens, all these ideas and theories as well as whatever is said in this
talk are
likely to remain as nothing more than mere flights of fancy.

The rest of the material in this report is organized as follows. In sec. 2, the exceptional
unification groups and their possible connection to new forms of quantum mechani
cs is
discussed. Sec. 3 deals with supersymmetry, supergravity and higher dimensions. In sec.
4, the necessity of quantizing gravity is “proved”. Sec. 5 is devoted to a discussion
of the
problems of quantum gravity and a description of superstring theory. Possible
points of
departure from the present quantum mechanics are noted at the appropriate places
in the
various sections. The final section contains some general remarks.

\section{Quaternions, Octonions, Exceptional\\ Groups and Unification}

In the present form of quantum mechanics we know that complex number plays a
crucial role. Probability amplitude is taken to be a complex number and probability is
the norm of this complex number.
However, complex number is merely one of the four
Hurwitz algebras$^{7}$) (defined below) and so a clear direction for the extension of quantum
mechanics emerges.

The theorem due to Hurwitz states that there are only four algebras which satisfy both
the following properties:-

\begin{itemize}
\item (a) If $xy = 0$, then $x=0$ or $y=0$.
\item The norm satisfies $|xy| = |x| |y|$.
\end{itemize}

These four algebras are Real numbers $(R)$, Complex numbers $(C)$, Quaternions $(Q)$ and
Octonions $(\Omega)$ and these are called Hurwitz algebras.

Quaternions are noncommutative,
but octonions are not only noncommutative but
nonassociative as well, i.e.

$$
x(yx) \neq (xy)z.
$$


ust as the complex numbers can be constructed in terms of the real numbers by using
$i$ (with $i^{2} = —1$), the quaternions can be constructed in terms of the complex numbers
by introducing one more “imaginary number” $j$ satisfying $j^{2} = —1$ and the octonions can
be built from the quaternions by introducing further entities of the same type. But the
important point about the Hurwitz theorem is that the process stops at the octonion. If
we try to extend the algebra further, then either property (a) or property (b) mentioned
above fails.


In the new forms of quantum mechanics the probability amplitude may become a
quaternion or an octonion$^{8}$), This will call for a new definition of scalar product and thus
leads to exceptional groups which is a class of Lie groups. From this point of view it
is remarkable that exceptional groups have become important unification groups in high
energy physics.
So, before we make the connection of the exceptional groups with the
quaternions and octonions, we touch on some aspects of Lie groups$^3$) in general and their
use by high energy physicists for unifying$^3$) the forces of nature.

Cartan’s classification of compact Lie groups leads to the four infinite families $A_{n}$, $B_{n}$,
$C_{n}$, and $D_{n}$, which respectively correspond to $SU(n +1)$, $SO(2n +1)$, $S_{p}(2n)$ and $SO(2n)$
and the five exceptional groups $G_{2}$, $F_{4}$, $E_{g}$, $E_{7}$ and $E_{8}$. Each of these groups can be
characterized by its Dynkin diagram given in Fig. 1.

Many models have been constructed for unifying the electroweak and strong forces.
The most successful attempts among these were based on the groups $SU(5)$, $SO(10)$ and
$E_{6}$ in increasing order of complexity and nonminimality. Thus, a representative each from
the unitary, orthogonal and exceptional groups seem to be relevant for unification. It turns
out that these three Lie groups have closely related Dynkin diagrams as shown in Fig. 2
and in fact one is encouraged to complete the chain by including the last two exceptional
groups $E_{7}$ and $E_{8}$:-

$$
SU(5) - SO(10)-E_{6}-E_{7}-E_{8}.
$$

Remarkably enough, in the most preferred unified- superstring theory which unifies the
nongravitational forces with gravity, it is the last exceptional group in the chain, namely
$E_{8}$ that occurs in a natural way.


What is the connection of the scalar product to the ning
group? Whereas the definition of the unitary group $SU(n)$ involves the invariance of the scalar product $\sum\limits_{i-1}^{n} \psi_{i}^{\ast} \psi_{i}^{\ast}$
under unitary transformations, for the orthogonal group $SO(n)$, it is the invariance of the
scalar product $\sum\limits_{i-1}^{n} x_{i}^{2}$ under orthogonal transformations. In the former, $\psi_{i}$ refers to a
complex field while the $x_{i}$ of the latter corresponds to a real coordinate.
corresponding definition of the exceptional groups?
the Hurwitz algebra become relevant.

An equivalent way of proceeding is as follows.
Construct Lie algebras and Lie groups
in terms of antihermitian matrices whose elements are members of the Hurwitz algebra.
As is well known, antihermitian matrices with real elements lead to $SO(n)$, whereas the
same with complex elements lead to $SU(n)$.
Extending the procedure with quaternions
and octonions, one gets only a finite number of Lie groups which are the exceptional
groups. [The procedure is quite involved and it leads to the so- called magic square. See
Freudenthal$^{10}$) for more details.] Thus, quaternions and octonions are intimately related
to the exceptional groups.

To sum up, we have the correspondence:-

\begin{align}
R : &\sum_{i=1}^{n} x_{i} x_{i} \rightarrow SO(n)\\
C: &\sum_{i=1}^{n}\phi_{i}^{\ast} \phi_{i} \rightarrow SU(n)\\
\end{align}
\begin{equation*}
\begin{rcases}
Q : ?
\Omega : ?
\end{rcases}
\rightarrow {Exceptional Groups} \tag{2.5}
\end{equation*}

The question marks refer to new forms of scala
r product. The familiar sca.'ar products
for $R$ and $C$ have been used in our descriptio
n of space-time and quantum mechanics
respectively. The new forms of scalar product
involving $Q$ and $\Omega$ possibly ar.’se in new
forms of quantum dynamics. Thus, the emergenc
e of exceptional groups in unification
physics may signal the relevance of such
new dynamics at the unification scale ($10^{15}$
GeV)
or beyond.

\section{Supersymmetry, Supergravity and Higher Dimensions}

Supersymmetry$^{4}$) transformations connect fermions with bosons. Supersymmetry of-
fers a revolutionary jump in our conceptual framework; however it is not yet organically in-
tegrated into physics. Supersymmetry involves an enlargement of ordinary space-time into
superspace whose coordinates include anticommuting numbers (called Grassmann numbers)
in addition to the familiar (commuting) space-time coordinates. There must be some direct
physical manifestation of superspace. This must be identified and looked for, experimen-
tally. However this is for the future.

A practical motivation for the immediate introduction of supersymmetry into high en-
ergy physics is the better ultraviolet behaviour of supersymmetric quantum field theories.
This better behaviour is due to the cancellation of divergences between the contributions
from the fermionic and bosonic members of the supermultiplet. There even exists a su-
persymmetric quantum field theory which is completely finite! This is the Yang-Mills
theory with $N = 4$ (ie. with four supersymmetries). So, for the first time in history we
have a quantum field theory in four-dimensional space-time which is free of the notori-
ous ultraviolet divergences which have plagued relativistic quantum field theory from the
beginning.

Another remarkable fact of supersyrametry is that it provides us with a natural entry
into quantum gravity. Two successive supersymmetric transformations lead to space-time
translation and so if supersymmetry is made into a local or gauge symmetry, that contains
gravity or rather supergravity (which is a generalization of Einstein’s gravity and includes
the gravitino as the spin- 3/2 supersymmetric partner of the spin-2 graviton.)

Again supergravity has better ultraviolet behaviour than ordinary gravity and many of
the divergences do cancel, but unfortunately divergences do remain for Feynman diagrams
with higher number of loops. So the problem of quantum gravity is not solved at the level
of supersymmetry.

Kaluza had the beautiful idea of a geometrical unification$^{5}$) of electromagnetism with
gravity through five-dimensional space-time. It is based on the beautiful equation:
\begin{align*}
(\text{gravity in} d = 4) &+ (\text{electromagnetism in} d=4)\\
                          & (\text{gravity in } d = 5). 
\end{align*}
where $d$
is the total number of space-time dime
nsions. This idea was subsequently
developed by Klein. The fifth dimension
is supposed to be curled up into a tiny
circle of redius
equal to Planck length
$$
\ell p_{\ell} = \left(\frac{\hbar G_{N}}{c^{3}}\right)^{1/2} ~ 1-^{-33} cm.
$$

This explains why experimental phys
icists have not yet discovered the
existence of this
tiny extra dimension to our ordinary
four-dimensional space-time!

What about the other interactions?
Generalizing the Kaluza-Klein theory$^{5}$)
one finds
that a minimum of eleven space-time
dimensions is required to take into acco
unt of all the
known forces of nature. Thus, we have
\begin{align*}
(\text{gravity in} d = 4) &+ (\text{electroweak and strong forces in} d=4)\\
                          & (\text{gravity in } d = 11). 
\end{align*}

So, at each point in our familiar four-dimensional world, there exists a
seven-dimensional
compact manifold of scale size ~
$10^{-33}$ cm, All the observed nongravi
tational interac-
tions of the physical world are manifest
ations of the geometrical properties
of this internal
manifold. All of physics is thus reduced
to geometry.

However, the orginal programme of
higher-dimensional unification has encountered two
serious problems:-
\begin{itemize}
\item Fermions (quarks and leptons) with
the required chiral properties do not
exist in this
theory.
\item  The ultraviolet divergence problem
of quantum gravity is not solved. 
\end{itemize}

This is the reason for going to supe
rstring theory which is also formul
ated in higher dimens
ions.

Should the present form of quan;um
mechanics remain intact, when spac
e-time may
include Grassmann coordinates ag well
as hidden compact dimensions? Over
the years, con-
siderable amount of discussion has
focussed on the interpretation of qua
ntum mechanics,
Especially interesting have been the
discussions on the Einstein- Podols
ky-Rosen corre-
lations and the possibility of violatio
n of local realism in quantum mech
anics. All these
discussions naturally take place with
in the framework of four-dimensional
space-time. On the other hand, theoretical high-energy physicists have been taking higher-dimensional
space-time and Grassmann dimensions seriously, since in their attempts at understand-
ing the microworld, such concepts seem to be necessary.
Is it realistic to ignore such
developments completely in the discussion of the interpretation of quantum mechanics?
Unconventional solutions to the problem of local realism may be possible.


\section{Semiclassical Gravity}

Should gravity be quantized? If gravity is not quantized and the rest of physics is
gravity? In this
quantized, we have semiclassical gravity. Can’t we live with semiclassical
section we shall show what is wrong with it.

Semiclassical gravity is described by the equation
\begin{equation*}
G_{\mu \nu}(\Gamma_{\alpha \beta}) = 8 \pi G_{N}\langle \psi | T_{\mu \nu} (\phi)| \psi \tag{1} 
\end{equation*}
where $G_{\mu \nu}$ is the Einstein tensor defined by
$$
G_{\mu \nu} = R_{\mu \nu} - \frac{1}{2} \Gamma_{\mu \nu} R
$$

and involves the unquantized classical metric $g_{\alpha \beta}$ whereas the stress-energy tensor
 of mater and other fields $T_{\mu \nu}$ is a quantum operator and is a function of the quantum operator $\phi$
 denoting the denoting the quantized matter field and taken to be a single scalar field for simplicity.
The quantum state of matter is represented by the state vector $|\phi \rangle$. It is clear from eq.(1)
that the classical matric $\Gamma_{\alpha \beta}$ depends on the quantum state vector
$|\phi \rangle$.


It is convenient to use Heisenberg picture since Schrodinger picture is not covariant.
The covariant Heisenberg equation of motion for ¢ in the presence of the gravitational field
described by the classical metric $\Gamma_{\mu \nu}$ is
\begin{equation*}
\partial_{\mu} (\Gamma^{\mu \nu} \partial_{\nu} \phi)- m^{2} \phi = 0\tag{2a} 
\end{equation*}

Allowing the operators to act on the statevector $|\psi \rangle$ and also explicitly indicating the
dependence of $\Gamma_{\alpha \beta}$ on |$\psi\rangle$, we rewrite this equation as
\begin{equation*}
\partial_{\mu}(\Gamma^{\mu \nu} (|\psi \rangle)\partial_{\nu} \phi) |\psi \rangle - m^{2} \psi | \psi \rangle =0. \tag{2b}
\end{equation*}

This equation makes it obvious that the quantum evolution of matter has a nonlinear
dependence on the wave function or the state vector $|\psi \rangle$.

This is an example of nonlinear quantum mechanics. Nonlinearity in $|\psi\rangle$ may be one
way in which the present form of quantum mechanics may be changed in the future. The
linearity of quantum mechanics, on which superposition principle is based, is an anomaly
in physics. Everywhereelse in physics, linearity is only an approximation to be replaced by
a more exact nonlinear theory. Will the same thing happen to quantum mechanics? For
more on nonlinear quantum mechanics see ref. 11.

Coming back to semiclassical gravity, let us ask: what happens during measurement?
This becomes crucial because of the nonlinearity. In the conventional (Copenhagen) in-
terpretation of quantum mechanics, |p) collapses to an eigenstate during measurement, ie.
\begin{equation*}
| \psi \rangle = \sum_{i} C_{i}(t) | \psi_{i}\rangle \tag{3}
\end{equation*}

where $|\psi\rangle$ are the constant state vectors of the Heisenberg picture, while $C_{i}(t)$ are time
dependent coefficients denoting the collapse during the measurement. Equation (3) is not
consistant with eq.(1). For, taking the covariant derivative of eq.(1), the left-hand-side
gives zero:
$$
G_{\mu \nu_{i} \nu} = 0
$$
while the right-hand-side gives
$$
8\pi G_{N} \langle \psi | T_{\mu \nu_{i} \nu}| | \psi \rangle + 8\pi G_{N} \sum_{i, j} (C_{i}^{*} C_{j})_{i \nu} \langle \psi_{i} | T_{\mu \nu} | \psi_{j}\rangle.
$$

The first term vanishes, but the second term does not. Thus, to retain semiclassical gravity,
we must assume that $|\psi \rangle$ never collapses.

Some other formulation of quantum mechanics in which the wave function never col-
lapses, is needed. The many-world\break formulation$^{12}$) of quantum mechanics is such a formula-
tion. Thus, at this point of the argument, we assume that quantum mechanics — especially
the measurement process in quantum mechanics — is to be understood within the frame-
‘work of the many-world formulation and proceed to construct a thought-experi\break ment to
test, semiclassical gravity.

The thought-experiment consists of two parts - detection of radioactive decay and
measurement of gravity — coupled by the action of an experimenter:-

The radioactive part and the gravity part are shown in Fig.
(3a) and (3b) respectively.
The experimenter counts the number of photons in the (identical) Geiger counters 1 and
2 simultaneously for, say, 30 seconds.
If the number of counts in 1 is greater than in 2,
-(to be called decision $\alpha$), the experimenter places the large masses in position A for 30
minutes and then in position B for the same time. If the number of counts in 1 is less than
in 2, (to be called decision $\beta$), the masses are placed in position B first and then moved
to A. In either case, the change in gravitational field is measured by a Cavendish torsion
balance.

The quantum process (decay of $Co^{60}$) causes the wave function to have amplitudes of
comparable weight for both decisions $\alpha$ and $\beta$ and hence the comeapenginy positioning of
the masses leads to simultaneously occuring amplitude, for both the mass configurations
$AB$
and $BA$.
The basic assumption is that the full wave function never collapses and
that it includes all aspects of the experiment including the experimenter who recorded the
Geiger tube counts, classified the decision and then placed the masses in the corresponding
positions.
This is the many - world formulation.
In other words, the wave function $|\psi \rangle$
is supposed to have both the (4- dimensional) configurations shown in Fig. 4 with equal
weights:-
$$
| \psi \rangle = C_{\alpha} | \psi \rangle + C_{\beta} | \psi_{\beta} \rangle ; |C_{\alpha}|^{2} \approx |C_{\beta}|^{2}. 
$$

In the Copenhagen formulation, the wave function collapses to either $|\psi_{\alpha}\rangle$ or $|\psi_{\beta}\rangle$ and 
so we have the gravitational field given by either of the following possibilities:
$$
G_{\mu \nu}^{(\alpha)} = 8\phi G_{N}\langle \psi_{\alpha} | T_{\mu \nu} | \psi_{\alpha}
$$
 or
 $$
 G_{\mu \nu}^{(\beta)} = 8\phi G_{N}\langle \psi_{\beta} | T_{\mu \nu} | \psi_{\beta}.
 $$
 
 So, the gravitational field is correlated with the macroscopically different mass configura-
- tion in $\alpha$ or $\beta$, as expected intuitively.


In the many-world formulation, all possible outcomes exist simultaneously in the su-perposition. Therefore we get
\begin{align*}
G_{\mu \nu} &= 8\pi G_{N} \langle \psi | T_{\mu \nu} | \psi\rangle\\
            &= 8 \pi G_{N} \left\{|C_{\alpha}|^{2} \langle \psi_{\alpha} | T_{\mu \nu} | \psi_{\alpha}\rangle + |C_{\beta}|^{2} \langle \psi_{\beta} | T_{\mu \nu} \psi_{\beta} \rangle + 2 Re C_{\alpha} C_{\beta}^{*} \langle \psi_{\beta}| T_{\mu \nu} | \psi_{\alpha} \rangle\right\}. 
\end{align*}

The last term can be neglected because of the rapidly varying phases connected with
the gravitational field is the
the macroscopically large energies in the experiment. Thus,
of experimental conditions,
weighted average of the two situations $\alpha$ and $\beta$. By choice
ted contrary to physical
$|C_{\alpha}|^{2} \approx |C_{\beta}|^{2}$. Thus, no response of the torsion balanee is expected contrary to physical intuition.

It must be pointed out that in linear quantum mechanics, the Copenhagen formulation
same results in practice$^{12}$),
and the many-world formulation are supposed to lead to the
near quantum mechanics and
but semiclassical gravity, which we are considering, is nonli
hence the difference.

As we have already argued, in semiclassical gravity, we have to choose the many-world
formulation for the consistency of the equations. Since this now leads to the counterin-tuitive
throw away semiclassical gravity.
 result for the thought-experiment we may be tempted to
and let an actual exper-iment decide the issue.
 However, it is more logical to leave the choice to Nature
Geilker!) in 1981.
 This experiment was performed by Page and
the mass configuration in
The gravitational field was indeed found to be correlated with
. $\alpha$ or $\beta$, as expected intuitively, thus ruling out semiclassical gravity.

\section{Quantum Gravity and Superstrings}

Gravity must be incorporated into the rest of physics.
It is intolerable to have one
world where quantum mechanics reigns supreme but gravity is ignored and another world
where gravity cannot be ignored but use of quantum mechanics to describe it leads to
meaningless divergent results. Coexistence of quantum mechanics with classical gravity is
not possible as shown in sec. 4.

So-far any attempted theory of quantum gravity has been afflicted with the worst ul-
traviolet divergence problems known in quantum field theory — much worse than the diver-
gences in quantum electrodynamics, electroweak dynamics or quantum chromodynamics,
Whereas the divergences in quantum electrodynamics are comparable to the divergence of
the infinite geries
$$
1 + 1+ 1 + \ldots
$$
the divergences in quantum gravitodynamics are comparable to the divergent series
$$
1 + 2 + 3 \ldots
$$

This is because, in contrast to the photon whose coupling to the electron (say) is constant,
graviton couples to energy and so the coupling increases with the energy of the virtual
particles in the intermediate state.
(Actually the series must be replaced by "integral.)
Another complication of gravity is the multigraviton vertices (of arbitrarily high order)
which is again absent in electrodynamics, In Fig. 5 the virtual emission and reabsorption
of the photon ($\gamma$) and the graviton ($\Gamma$) giving rise to the electromagnetic and gravitational
self-energies of the electron are shown. These differences turn out to be very important.
Whereas the mild divergences of the non-gravitational fields are “renormalizable” and are
under control, the divergences of quantum gravity are uncontrollable and “nonrenorma- °
lizable”.
No meaningful quantum theory could be constructed in the presence of these
divergences.

It is possible that just as quantum mechanics had its birth in the infrared catastrophe
of blackbody radiation, a new form of mechanics may be born in the ultraviolet catastrophe
of quantum gravity.

However, such a radically new step does not seem to be necessary at present, since there is a theory which claims to be the correct theory of quantum gravity without the ultraviolet catastrophe. The is the superstring theory.

Superstring theory$^{6}$) seems to succeed where every other theory has failed. For the
first time in history we are having a glimpse of a possible solution to the age-old problem of constructing a theory of quantum gravity. Superstring solves the problem of ultraviolet divergences by 2 (nonlocal) atring-generalization of the usual local quantum field theory based on point particles. The expected finiteness of superstring theory, which is based on some calculations as well as general arguments, is yet to be proved. If not finite, at least renormalizability is expected.

based on point particles. The expected by a one-dimensional object as a more basic entity. The  linear scale of this one-dimensional object (the string) is given by the Planck length ($\sim 10^{-33}$ cm). The string can be a open one with two open ends or a closed one with no open ends (see Fig. 6). Remarkably enough, it turns out that these simplest configurations, namely open and closed, that one can imagine for this one-dimensional object correspond to the gauge boson and the graviton respectively. The gauge field and gravity together encompass all known forces of nature.

More precisely, the ground state of the open and closed strings are respectively the massless vector gauge boson and the massless tensor graviton, but there are excited states as required in the quantum mechanics of a system with internal structure (albeit one-dimensional), All these excited states occur at a mass scale determined by the inverse of the length scale of the string, and hence at Planck mass $(\sim 10^{19} GeV)$.

Some of the miracles of string theory are the following:-
\begin{itemize}
\item[(a)] Every consistent string theory necessarily inclues gravity. This is because open strings can be shown to lead necessarily to closed strings. This is the first time in physics that gravity can be said to be “derived” from the rest of physics.

\item[(b)] Consistency of string theory requires higher dimensions and supersymmetry. In classical mechanics, a string can exist in any number of space-time dimensions. In contrast, the relativistic quantum mechanics of a string requires a specific number of space-time dimensions for consistency. This number is 26 for a bosonic string and 10 for a supersymmetric string which includes bosons and fermions. However the 26-dimensional bosonic string has an unstable ground state (called tachyon) and so the 10-dimensional supersymmetric string or superstring is chosen as the consistent system.
Thus, in contrast to classical mechanics, a consistent quantum mechanical theory of strings requires both higher dimensions and supersymmetry.

We must note the restrictive nature of quantum mechanical formalism. We may ask,
why not relax the rules of quantum mechanics, if superstring theory is relevant only
at $10^{-33}$ cm? In that case, arbitrary types of strings living in four-dimensional space
time may be possible. Which is the more conservative approach, ebandoning quan-
tum mechanics or abandoning four-dimensional space- time? High-energy physicists
have taken to the latter route. The relevant question is: which is the route followed
by Nature? 

\item[(c)] There exists a string analogue of the Kaluza- Klein unification of gravity with other
forces. This is the so-called heterotic string theory wherein both gravity and gauge
forces in a lower dimension (10) can be derived from pure gravity (namely closed
strings) partly living in a higher dimension (26). Among the unification groups for
the gauge forces allowed in this theory, the exceptional group $E_{8} \times E_{8}$ has emerged
as a better candidate. The possible connection of exceptional groups for newer forms
of quantum mechanics based on quaternions or octonions was already pointed out in
sec.2.
\end{itemize}

\textbf{String theory is a rapidly developing subject. The last word on it is not yet said.}


\section{Discussion}

From its birth, quantum mechanics has given rise to a lively debate involving possible
paradoxes, counterintuitive physical concepts and many other issues. Quantum mechanics
has survived all these controversies and at present it is the only known dynamical framework for all of physics. This does not mean that it will remain so forever and that is the
main point of the present talk. 

However, is it possible that the well-known paradoxes of quantum mechanics are manmade? They may be the result of first subjecting the human mind to classical ideas and
then confronting it with quantum mechanics. One must perhaps perform an experiment in
which some bright and young students are taught quantum physics and quantum mechanics
before they learn classical physics and classical mechanics. It should be now possible to do
this, since phenomena such as superconductivity and superfluidity and quantum electronic
instruments and devices such as the superconducting quantum interference device (SQUID)
provide us with macroscopic contacts with the quantum world. Important lessons may be
learnt by physicists and philosophers from such an experiment in teaching. 

The marriage of quantum mechanics with general relativity may lead to a breakdown
of quantum mechanics. Since the general theory of relativity of Einstein is the classical
theory of gravitation, this is the problem of quantum gravity already discussed in sec. 5.
As already pointed out, there is a hope that the theory of superstrings will provide the
correct theory of quantum gravity although nobody knows the shape of such a theory as
yet. However, even without a complete theory of quantum gravity, many brave attempts
have been made in envisaging the strange world of quantum gravity populated by metric
fluctuations, worm holes, baby universes, disconnected universes, space- time foam etc.
- All these phenomena are expected to occur at the Planck scale$\sim 10^{-33}$ cm. It has been
pointed out that these strange goings-on lead to a breakdown of quantum mechanics such
as pure states evolving into mixed states thus causing a loss of quantum coherence$^{14}$).
This is a controversial$^{15}$) but important subject and it reinforces the urgency of a complete
theory of quantum gravity (possibly superstrings). 

Classical physics failed to explain the stability of the atom and thus quantum mechanics
was born. It is possible that quantum mechanics may fail to explain the stability of spacetime and so a new form of mechanics may. be necessary. Such new discoveries may come only through the study of deeper regions of space-time. From this point of view, the
onward march to shorter distances (which means higher energies) assumes an enormous
significance. Ultimately, the only justification for pushing the frontier of high energy
physics to higher energies is the expectation that we will reach the boundaries of validity
of our present view of the physical universe - based on our present conceptual framework
of space-time (namely relativity) and our present understanding of dynamics (namely
quantum mechanics). 

\section*{Bibliography}

\begin{thebibliography}{99}
\itemsep=0pt

\bibitem{} For more extensive information on the relevant developments in high energy physics,
the following references may be conslulted. See ref. P. Langacker, Phys. Reports 72C, 185 (1981).  2 for the standard model of high
energy physics, ref. 3 for unification, ref. 4 for supersymmetry and supergravity, ref.
5 for higher dimensions and ref. 6 for superstrings. 

\bibitem{} T.P. Cheng and L.F. Li, Gauge Theory of Elementary Particle Physics (Oxford University Press, 1983). 
\bibitem{} S. Ferrara, Supersymmetry (North Holland and World Scientific, 1987). 
\bibitem{} E. Witten, Nucl. Phys. B186, 412 (1981). |
A. Zee, in Grand Unified Theories and Related Topics, 4th Kyoto Summer Institute
(World Scientific, 1981). 
\bibitem{} B. Green, J. Schwarz and E. Witten, Superstring Theory, Vol. 1 and 2 (Cambridge
University Press, 1986). 
\bibitem{} P. Ramond, Introduction to Exceptional Lie Groups and Algebras, Caltech preprint
CALT-68-577 (1976). 
\bibitem{} R.D. Schafer, Introduction to Non-Associative Algebras, (Academic Press, 1966). 
\bibitem{} P. Jordan, J. Von Neumann and E.P. Wigner, Ann. Math. 36, 29 (1934).
L.P. Horwitz and L.C. Biedenharn, J. Math. Phys. 20, 269 (1979). 
\bibitem{} R. Gilmore, Lie Groups, Lie Algebras and some of their applications (John Wiley,
New York, 1974). 
\bibitem{} H. Freudenthal, Lie Groups in the Foundation of Geometry, Advances in Math. 1,
143 (1964). 
\bibitem{}  B. Mielnik, Commun. Math. Phys. 37, 221 (1974).
T.W.B. Kibble, Commun. Math. Phys. 64, 73 (1978). 
\bibitem{} H. Everett III, Rev. Mod. Phys. 29, 454 (1957).
J.A. Wheeler, Rev. Mod. Phys. 29, 463 (1957).
B.S. De Witt, Phys. Today 23, No. 9, 30 (1970).
The Many-World Interpretation of Quantum Mechanics: A Fundamental Exposition, Edited by B.S. De Witt and N. Graham (Princeton Univ. Press, Princeton,
1973). 
\bibitem{} D.N. Page and C.D. Geilker, Phys. Rev. Lett. 47, 979 (1981). 
\bibitem{} 5.W. Hawking, Commun. Math. Phys. 87, 395 (1982).
S.W. Hawking, D.N. Page and C.N. Pope, Nucl. Phys. B170, 283 (1980).
T. Banks, M. Peskin and L. Susskind, Nucl. Phys. B244, 125 (1984).
J. Ellis et. al., Nucl. Phys. B241, 381 (1984).
S.W. Hawking, Phys. Lett. 195B, 337 (1987). 
\bibitem{} S. Coleman, Nucl. Phys. B307, 854 (1988). 
\end{thebibliography}

\begin{tabular}{|c|c|c|}
\hline
Landmark & Energy & Length\\
\hline
Nuclear Physics & 200 MeV & $10^{-13}$ cm\\
Standard Model & 100 GeV & $2 \times 10^{-16}$ cm\\
Unification & $10^{15}$ GeV & $2 \times 10^{-29}$ cm \\
Quantum gravity & $10^{19}$ GeV & $2 \times 10^{-33}$ cm\\
\hline
%~ \caption{Table 1. The energy and length scales of quantum physics}
\end{tabular}


\vspace{.5cm}

figures??????????
