\chapter{The Story of The Neutrino}\label{chap23}

\Authorline{G Rajasekaran\footnote[*]{Email: \url{graj@imsc.res.in}}}
\addtocontents{toc}{\protect\contentsline{section}{{\sl G Rajasekaran}\smallskip}{}}

\authinfo{Institute of Mathematical Sciences, Taramani, Chennai\\
\& Chennai Mathematical Institute, Siruseri, Chennai}

\section*{Abstract}

This is an elementary review of the history and physics of neutrinos. The story of the discovery of neutrino mass through neutrino
oscillations is described in some detail. Experiments on solar neutrinos and atmospheric neutrinos played an important part. Recent
advances are summarized and future developments are indicated.

\section{Introduction}

In the recent past, two Nobel Prizes were given to Neutrino Physics.
In 2002 Ray Davis of USA and Matoshi Koshiba of Japan got the Nobel
Prize for Physics while last year (2015) Arthur McDonald of Canada and
Takaaki Kajita of Japan got the Nobel Prize. To understand the importance
of neutrino research it is necessary to go through the story of the neutrino
in some detail.

Starting with Pauli and Fermi, the early history of the neutrino is described culminating in its experimental detection by Cowan and Reines. Because of its historical importance the genesis of the solar neutrino problem
and its solution in terms of neutrino oscillation are described in greater detail.
In particular, we trace the story of the 90-year-old thermonuclear hypothesis
which states that the Sun and the stars are powered by thermonuclear fusion
reactions and the attempts to prove this hypothesis experimentally. We go
through Davis’s pioneering experiments to detect the neutrinos emitted from
these reactions in the Sun and describe how the Sudbury Neutrino Observatory in Canada was finally able to give a direct experimental proof of this
hypothesis in 2002 and how, in the process, a fundamental discovery i.e. the
discovery of neutrino oscillation and neutrino mass was made.

We next describe the parallel story of cosmic-ray-produced neutrinos and
how their study by SuperKamioka experiment in Japan won the race by
discovering neutrino oscillations in 1998.

Many other important issues are briefly discussed at the end.

\section{What is a Neutrino?}

Neutrino is an elementary particle like electron. But unlike electron which
has a negative electric charge, it is neutral. Also, unlike electrons which are
constituents of all atoms, neutrinos do not exist within atoms. But they
are created through many processes all over the Universe in large numbers
and are flying everywhere at almost the speed of light. Every second more
than $10^{12}$ neutrinos are passing through our body without affecting us in any
way. Since the probability of neutrinos interacting with matter is negligible,
they simply pass through all matter. Hence it requires huge detectors and
sophisticated instruments to study them.

Until some years ago, neutrinos were regarded as massless particles like photons. But in 1998 neutrinos were discovered to have mass. This discovery
is expected to lead to fundamental changes in our knowledge of physics and
astronomy. Many more discoveries about neutrinos are yet to be made.

\section{Early History of Neutrino}

After radioactivity was discovered by Becquerrel in 1897, many properties
of radioactivity were revealed by the researches of a host of scientists including the famous ones Marie Curie and Ernest Rutherford. Among those, the
so-called beta radioactivity turned out to be a puzzle. The electrons that
came out in the beta activity did not come out with a single energy unlike
the case of alpha and gamma activity where the alpha particle or the gamma
photon emitted by a particular nucleus came with a single energy. The beta
electrons had a continuous spectrum of energies. This seemed to contradict
the principle of conservation of energy which is a cornerstone of Physics.
Wolfgang Pauli in 1930 suggested a way to resolve this puzzle. If another
unseen particle was emitted along with the electron, it could take away part
of the energy and thus the principle of conservation of energy could be saved.
This was Pauli’s suggestion.

Although neutrino was born in the mind of Pauli, it was Enrico Fermi
who made neutrino the basis of his famous theory of beta decay in 1932 and
showed how in the beta decay of a nucleus an electron and a neutrino are
simultaneously created [1]. It is this that remained as the basic theory of the
decays of all elementary particles for more than 40 years. It was also Fermi
who christened the particle as ’Neutrino’.

In the subsequent decades beta decays of many atomic nuclei were experimentally studied. All of them were in beautiful agreement with Fermi’s
theory and hence it was clear to theoretical physicists at least that Pauli’s
neutrinos were indeed emitted in beta decay. But Cowan and Reines did not
agree. If neutrinos exist, their existence must be experimentally proved, they
said. And they proved it in 1954.

Before we describe their experiment, it is necessary to explain beta decays
of nuclei.

\section{Beta decays and the Cowan-Reines\\ Experiment}

Every atomic nucleus contains $Z$ number of protons and $N$ number of
neutrons. For example, the nucleus of the Hydrogen atom is a single proton.
Helium nucleus contains 2 protons and 2 neutrons. Uranium nucleus contains
92 protons and 146 neutrons. Many nuclei undergo beta decay spontaneously.
The nucleus $(Z,N)$ which contains $Z$ protons and $N$ neutrons emits an electron
$(e^{-})$ and an antineutrino $\bar{\nu}_{e}$ and becomes the nucleus $(Z+1,N-1)$ containing
$Z+1$ protons and $N-1$ neutrons. This is shown below.

\newpage

\begin{equation}
(Z, N) \rightarrow (Z + 1, N-1)+ e^{-} + \bar{\nu}_{e}(\beta^{-} \text{decay})
\end{equation}

In the same way, neutron (n) decays and becomes a proton as shown below.
\begin{equation}
n \rightarrow p + e^{-} + \bar{\nu_{e}} (\beta^{-} \text{decay of} n)
\end{equation}

If we transfer the antineutrino from the right side of the eq.(1) to the left
side, it will be a neutrino. As shown below, this then denotes the reaction in
which a neutrino $\nu_{e}$ collides with a nucleus $(Z,N)$ and the result is another
nucleus $(Z+1, N-1)$ and an electron $e^{-}$.
\begin{equation}
\nu_{e} + (Z, N) \rightarrow (Z + 1, N-1) + e^{-} (inverse \beta decay)
\end{equation}

This is sometimes called inverse beta decay and it is through such reactions
experimental physicists detected neutrinos.

In eq.(4) shown below, nucleus $(Z,N)$ emits a positron $e^{+}$ and a neutrino
$\nu_{e}$ and becomes the nucleus $(Z-1, N+1)$.
\begin{equation}
(Z, N) \rightarrow (Z-1, N+1) + e^{+} + \nu_{e} (\beta^{+} decay)
\end{equation}

Here, if we transfer the neutrino to the left side, it will be an antineutrino $\bar{\nu}_{e}$
and we will have a reaction of the antineutrino (shown in eq.(5)).
\begin{equation}
\bar{\nu}_{e} + (Z, N) \rightarrow (Z + 1, N-1) + e^{+} (inverse \beta^{+} decay)
\end{equation}
As an example, an antineutrino and a proton collide and become a positron
and a neutron (eq.(6)).
\begin{equation}
\bar{\nu_{e}} + p \rightarrow + e^{+} (Cowan - Reines reaction)
\end{equation}

It is this reaction that Cowan and Reines used to prove the real existence of
the neutrino (actually the antineutrino).

Every nuclear reactor is a copious source of antineutrinos. How? When
nuclei such as Uranium fission in the nuclear reactor, a variety of radioactive
nuclei are produced. Many of them undergo beta decay and emit antineutrinos. Cowan and Reines used a hydrogenous material as their detector.
Hydrogen nucleus is a proton. If the antineutrino from the reactor interacts
with the proton, a positron and a neutron are produced, as we already saw
(eq.(6)). Reines and Cowan proved the appearance of the positron and neutron in their detector placed near the nuclear reactor. Thus the emission of
antineutrinos from the nuclear reactor was experimentally proved by Cowan
and Reines in 1954. Reines received the Nobel Prize in 1995. Cowan had
passed away before that.

There are two interesting episodes connected to the Cowan-Reines experiment. In that period (1945-55) many nuclear bomb tests were being
conducted. In the explosion of the nuclear bomb also, Uranium nucleus fissions and antineutrinos are produced. Cowan and Reines had planned to
catch those antineutrinos, but were prevented from pursuing that dangerous
venture. They then changed their plan and went to the Savannah River Reactor (USA) to do their experiment and succeeded. Pauli had apparently
sent a cable telegram to the Committee which was to decide on the sanction of financial support for the Cowan-Reines experiment, saying that ”his
particle” cannot be detected by anybody and so asking the Committee not
to support such an experiment. However that telegram did not reach the
Committee in time; support was given and the antineutrino was caught in
the experiment!

\section{Neutrinos from the Sun}

It is the Sun that is giving us light and heat. Without it, life on Earth
is impossible. How does Sun produce its energy and continue to shine for
billions of years? In the 19th century, the source of the energy in the Sun and
the stars remained a major puzzle in science, which led to many controversies.
Finally, after the discovery of the atomic nucleus and the tremendous amount
of energy locked up in the nucleus, Eddington in 1920 suggested nuclear energy as the source of solar and stellar energy. It took many more years for the
development of nuclear physics to advance to the stage when Bethe,the Master Nuclear Physicist, analysed all the relevant facts and solved the problem
completely in 1939. A year earlier,Weisszacker had given a partial solution.

Bethe’s paper is a masterpiece [2]. It gave a complete picture of the thermonuclear reactions that power the Sun and the stars. However, a not-sowell-known fact is that Bethe leaves out the neutrino that is emitted along
with the electron, in the reactions enumerated by him. Neutrino, born in
Pauli’s mind in 1932, named and made the basis of weak interaction by
Fermi in 1934, was already a well-known entity in nuclear physics. And it
is Fermi’s theory that Bethe used in his work. So it is rather inexplicable
why he ignored the neutrinos in his famous paper. The authority of Bethe’s
paper was so great that the astronomers and astrophysicists who followed
him in the subsequent years failed to note the presence of neutrinos. Even
many textbooks in Astronomy and Astrophysics written in the 40’s and 50’s
do not mention neutrinos! This was unfortunate, since we must realize that,
in spite of the great success of Bethe’s theory, it is nevertheless only a theory. Observation of neutrinos from the Sun is the only direct experimental
evidence for Eddington’s thermonuclear hypothesis and Bethe’s theory of
energy production. That is the importance of detecting solar neutrinos.

The basic process of thermonuclear fusion in the Sun and stars is four
protons (which are the same as Hydrogen nuclei) combining into a Helium
nucleus and releasing two positrons, two neutrinos and 26.7 MeV of energy.
$$
p + p + p + p \rightarrow He^{4} + e^{+} +e^{+} + \nu_{e} + \nu_{e}
$$

This can be regarded as the most important reaction for all life, for without it Sun cannot shine and there can be no life on Earth!

However,the probability of four protons meeting at a point is negligibly
small even at the large densities existing in the solar core. Hence the actual
series of nuclear reactions occurring in the solar and stellar cores are given
by the so-called carbon cycle and the pp-chain. In the carbon cycle the four
protons are successively absorbed in a series of nuclei,starting and ending
with carbon. In the pp-chain two protons combine to form the deuteron and
further protons are added.

We shall not go into details here [3] except noting that both in the carbon
cycle and the pp-chain, the net process is the same as what was mentioned
above, namely the fusion of four protons to form alpha particle with the
emission of two positrons and two neutrinos.

It is these thermonuclear fusion reactions that are responsible for the Sun
and the stars continuing to shine for billions of years. This fact remained as a theoretical fact for many decades although it was accepted as generally
correct by scientists. So even Nobel Prize was given to Bethe in 1967.

The only way to prove Bethe’s theory is to detect the neutrinos coming
from the Sun.

It is easy to calculate from the solar luminosity the total number of neutrinos emitted by the Sun; for ‘every’ 26.7 MeV of energy received by us,
we must get 2 neutrinos. Thus one gets the solar neutrino flux at the earth
as 70 billion per square cm per sec. These many solar neutrinos are passing
through our body and the Earth.

\section{The Davis Experiment}

About 50 years ago, Ray Davis started his pioneering experiments to
detect the solar neutrinos. His experiment was based on the inverse beta
decay:
$$
\nu_{e} + Cl^{37} \rightarrow e^{-} + Ar^{37}
$$

Chlorine-37 absorbs the solar neutrino to yield Argon-37 and an electron.
(See Section 4 for explanation of beta decay and inverse beta decay.)

A tank containing 615 tons of a fluid rich in chlorine called tetrachloroethylene was placed in the Homestake gold mine in South Dakota(USA). The
Chlorine-37 atoms in the fluid were converted into Argon-37 atoms by the
above reaction. The fluid was periodically purged with Helium gas to remove
the Argon-37 atoms which were then counted by means of their radioactivity.
Davis continued his experiment for almost 30 years and the result was that
about one neutrino in three days was caught in his experiment.

Two points must be noted. In three days billions of neutrinos fall on
Davis’s tank, but only one among them reacted with Chlorine-37 and got
caught. All others escape without any interaction, thus showing how tiny
is the probability of interaction of a neutrino. The experiment also proves
the extraordinary capability of Davis in counting radioactive atoms. If you
colour one grain of sand red and mix it in the sand of Sahara desert, can one
find that red grain of sand? The achievement of Davis is comparable to that.

Although solar neutrinos were detected by Davis, a new puzzle appeared.
Actually Davis detected only about a third of the solar neutrinos that must
have been detected in his tank. What is the reason for this discrepancy
between the theoretical number of solar neutrinos that must be detected in Davis’s detector and the actual number detected? Are the thermonuclear
fusion hypothesis and Bethe’s theory based on it wrong? This became known
as the solar neutrino puzzle and the puzzle lasted for many years.

\section{Kamioka and Superkamioka}

A few other experiments were undertaken in the attempt to resolve the
solar neutrino puzzle. The most important one among them was the Kamioka
experiment in Japan led by Matoshi Koshiba.

One must also note that Davis’s radio chemical experiment was a passive
experiment.There was actually no proof that he detected any solar neutrinos.In particular if a critic claimed that all the radioactive atoms that he
detected were produced by some background radiation, there was no way of
conclusively refuting it. That became possible through the Kamioka experiment that went into operation in the 80’s.

In contrast to Davis’s chlorine tank which was a passive detector,the
Kamioka water Cerenkov detector is an active real time detector. Solar neutrino kicks out an electron in the water molecule by elastic scattering and
the electron is detected through the Cerenkov radiation it emits. Since the
electron is mostly kicked toward the forward direction, the detector is directional.A plot of the number of events against the angle between the electron
track and Sun’s direction gives an unmistakable peak at zero angle,proving
that neutrinos from the Sun were being detected.The original Kamioka detector had 2 kilotons of water and the Cerenkov light was collected by an
array of 1000 photomultiplier tubes, each 20” diameter and this was later
superceded by the SuperKamioka detector which had 50 kilotons of water
faced by 11,000 photomutiplier tubes. Both Kamioka and SuperK gave convincing proof of the detection of solar neutrinos.The ratio of the measured
solar neutrino flux to the predicted flux was about 0.5, thus confirming the
solar neutrino puzzle.

There is a difficulty in resolving the solar neutrino puzzle. To understand
that, we have to know more details about the Sun.

\newpage

\section{Standard Solar Model and the Gallium Experiment}

In the Sun, the dominant thermonuclear fusion process is the pp-chain.
Although the 70 billion neutrinos per square centimeter per sec as the total
number of solar neutrinos falling on the Earth could be trivially calculated
from the solar luminosity,their energy spectrum which is crucial for their
experimental detection,requires a detailed model of the Sun, the so-called
Standard Solar Model (SSM). SSM is based on the thermonuclear hypothesis
and Bethe’s theory, but uses a lot more physics input about the interior of
the Sun.

A knowledge of the neutrino energy spectrum is needed since the neutrino
detectors are strongly energy sensitive.In fact all detectors have an energy
threshold and hence miss out the very low energy neutrinos.

Leaving out the details [3], the solar neutrino spectrum is roughly characterized by a dominant (0.9975 of all neutrinos) low energy spectrum ranging
from 0 to 0.42 MeV and a very weak (0.0001 of all the neutrinos) high energy
part extending from 0 to 14 MeV. Most of the neutrino detectors detect only
the tiny high-energy branch of the spectrum.

While the dominant low-energy neutrino flux is basically determined by
the solar luminosity,the flux of the high-energy neutrino flux is very sensitive
to the various physical processes in the Sun and hence is a test of SSM.
In fact,this latter flux is a very sensitive function of the temperature of the
solar core,being proportional to the 18th power of this temperature and hence
this neutrino flux provides a very good thermometer for the solar core. In
contrast to the photons which hardly emerge from the core,the neutrinos
escape unscathed and hence give us direct knowledge about the core.

There is a simple physical reason for this sharp dependence on temperature. It is related to the quantum-mechanical tunnelling formula, the famous
discovery of George Gamow. The probability for tunnelling through the repulsive Coulomb barrier has a sharp exponential dependence on the kinetic
energy of the colliding charged particles.

\newpage

The detection threshold in Davis’s experiment was 0.8 MeV and thus
only the high-energy neutrinos were detected. SSM could be used to get the
number of neutrinos expected above this threshold and the detected number
was less than the predicted number by a factor of about 3. Over the three
decades of operation of Davis’s experiment,this discrepancy has remained and has been known as the solar neutrino puzzle.

The energy threshold of the Kamioka and SuperKamioka detectors was
about 7 MeV and so only the high-energy part of the neutrino spectrum was
being detected.

The next input came from the gallium experiments. The high-energy
neutrino flux is very sensitive to the details of the SSM and so SSM could
be blamed for the detection of a lower flux.On the other hand the low energy neutrinos are not so sensitive to SSM. So the gallium detector based
on the inverse beta decay of Gallium-71 was constructed.Although this was
also a passive radiochemical detector,its threshold was 0.233 MeV and hence
it was sensitive to a large part of the low-energy branch extending up to
0.42 MeV.Actually two gallium detectors were mounted,called SAGE and
GALLEX and both succeeded in detecting the low energy neutrinos in addition to the high energy neutrinos but again at a depleted level by a factor of
about 0.5.

To sum up, there were three classes of neutrino detectors with different
energy thresholds,all of which detected solar neutrinos, but at a depleted
rate.The ratio R of the measured flux to the predicted flux was 0.33$\pm$0.028 in
the chlorine experiment, 0.56$\pm$0.04 in the two gallium experiments (average)
and 0.475$\pm$0.015 in the SuperK experiment.

Actually it must be regarded as a great achievement for both theory
and experiment that the observed flux was so close to the theoretical one,
especially considering the tremendous amount of physics input that goes into
the SSM. After all R does not differ from unity by orders of magnitude! This
is all the more significant since the large uncertainties in some of the low
energy thermonuclear cross sections do lead to a large uncertainty in the
SSM prediction. But astrophysicists led by John Bahcall were ambitious
and claimed that the discrepancy is real and must be explained. Two points
favour this view.As already stated,the gallium experiments sensitive to the
low-energy flux which is comparatively free of the uncertainties of SSM, also
showed a depletion in the flux. Second, SSM has been found to be very
successful in accounting for many other observed features of the Sun, in
particular the helioseismological data i.e data on solar quakes.

Hence something else is the reason for R being less than unity and that
is neutrino oscillation.

\section{Three kinds of Neutrinos}

In addition to the well-known electron,two heavier types of electrons are
known to exist.Reserving the name electron to the well-known particle of
mass 0.5 MeV,the heavier ones are called muon ($\mu$) and tauon ($\tau$ ) and their
masses are 105 and 1777 MeV respectively. Correspondingly there are three
types or flavours of neutrinos called eneutrino $(\nu_{e})$, mu neutrino $(\nu_{\mu})$ or tau
neutrino ($\nu_{\tau}$ ) that are respective companions of electron,muon or tauon. Just
is electron and eneutrino are emitted in beta decay, in the processes involving
muon or taon, muneutrino or tauneutrino will appear. The three doublets
are shown below:
\begin{equation*}
\begin{pmatrix}
\nu_{e}\\
e
\end{pmatrix}
\begin{pmatrix}
\nu_{\mu}\\
\mu
\end{pmatrix}
\begin{pmatrix}
\nu_{\tau}\\
\tau
\end{pmatrix}
\end{equation*}

What is produced in the thermonuclear reactions in the Sun is the antielectron (positron) and eneutrino. This eneutrino produced an electron when
it converted the Chlorine-37 nucleus in Davis’s detector into an Argon-37 nucleus:
$$
\nu_{e} + Cl^{37} \rightarrow e^{-} + Ar^{37}
$$

If some of the eneutrinos oscillate to the muneutrinos or the tauneutrinos
on the way to the earth, the reactions in Davis’s detector must be
$$
\nu_{\mu} + Cl^{37} \rightarrow \mu^{1} + Ar^{37}
$$
$$
\nu_{\tau} + Cl^{37} \rightarrow \tau^{-} + Ar^{37}
$$

Just as the eneutrino produces an electron in the inverse beta decay process,
the muneutrino or the tauneutrino has to produce a muon or a tauon respectively in the final state. But since the energy of the solar neutrinos are
limited to 14 MeV,the muon or tauon with the high masses of 105 and 1777
MeV cannot be produced in the inverse beta decay. According to Einstein’s
famous equation,
$$
E=mc^{2}
$$
it is energy E which is converted into mass m. So the neutrinos that have
been converted into the mu or tau flavour through oscillation escape detection
in the Chlorine and Gallium experiments.

Although elastic scattering of neutrinos on electron which is used as the
detecting mechanism in the Kamioka and SuperK water Cerenkov detectors can detect the converted mu or tau flavours also, it has a much reduced
efficiency. Hence the depletion of the number of neutrinos observed in the
water detector also is attributable to oscillation.

There was a famous painting called ”The Cow and Grass”.But nothing
except a blank convass was visible.When asked to show the grass, the painter
said the cow had eaten the grass. When pressed to show at least the cow,he
said it went away after eating the grass.

Our neutrino story so far is like that.We said thermonuclear reactions in
the Sun must produce so many neutrinos.We did not see so many neutrinos,
but then explained them away through oscillations.

In Science we have to do something better.If we say that neutrinos have
oscillated into some other flavour, we have to see the neutrinos of those
flavours too.

This is precisely what is done in a two-in-one experiment.

\section{Two-in-One Experiment}

The beta decay and inverse beta decay processes that we have described
so far are charge-changing (CC) weak interaction processes. Another kind
of weak interaction, known as charge-non\break changing or neutral current (NC)
weak interaction was discovered in 1973. These two kinds of processes are
shown below:
$$
\nu_{e} + (Z, N) \rightarrow (Z + 1, N -1) + e^{-} (CC)\\
\nu + (Z, N) \rightarrow (Z, N)^{\ast} + \nu (NC)
$$

In the CC process eneutrino changes into electron. Neutrino does not have
charge while electron does have charge. So charge of the particle changes in
the process and hence CC. The nucleus also changes from $(Z,N)$ to $(Z+1,
N-1)$ and so its charge changes. But in the $NC$ process, neutrino remains as
neutrino. The nucleus $(Z,N)$, without changing its charge, either gets exited
to a higher energy state or disintegrates. We have denoted such a state of
the nucleus as $(Z, N)^{\ast}$ in the $NC$ reaction above.

The important point is that the solar eneutrinos that oscillated into the
mu type or the tao type cannot undergo the appropriate CC process as we
already explained. But since the NC process does not create the heavier muon
or taon, they can undergo the NC process. So if we design an experiment
in which both the CC and NC modes are detected, and if the number of neutrinos involved in NC reactions is found to be larger than those in CC
reactions, oscillation will be proved.

While the CC mode will give the number of eneutrinos, the NC mode
will give the total number of e, mu and tau type of neutrinos. The total
number detected will be a test of SSM independent of oscillations while the
NC minus CC events will give the number that had oscillated away.

This is the ’two-in-one” experiment. A huge two-in-one detector based on
Boron called BOREX was proposed by Sandip Pakvasa and Raju Raghavan
(who passed away in 2011), but that has not materialized.The two-in-one
detector based on deuteron in heavy water proposed by Chen was constructed
at the Sudbury Neutrino Observatory (SNO), Canada and it finally solved
the solar neutrino puzzle. SNO uses 1000 tons of heavy water borrowed from
the Canadian Atomic Energy Commission.

Just as water is made of $H_{2}O$ molecules, heavy water is made of $D_{2}O$
molecules. The nucleus of the heavy hydrogen D is made up of one proton
and one neutron. Solar neutrino breaks up the deuteron D by CC and NC
modes. While CC mode leads to two protons and an electron, NC mode
leads to a neutron, a proton and a neutrino.
$$
\nu_{e} + D \rightarrow p + p + e^{-} (CC)
$$
$$
\nu + D \rightarrow p + n + \nu (NC)
$$

The threshold of detection was again high like SuperK so that only the
high energy neutrinos were detected. Let us now straightaway go to the
exciting results of SNO that came out in April 2002.

The CC mode gave the flux (million neutrinos per sq cm per sec) as
1.76$\pm$0.11 while the NC gave 5.09$\pm$0.65 in the same units. (The numbers are
in millions rather than in billions since the threshold of detection was again
high like in SuperK so that only the high-energy neutrinos were detected.)
Thus we conclude that the flux of e + mu + tau neutrinos is 5.09$\pm$0.65 while
that of the e flavour alone is 1.76$\pm$0.11. The difference 3.33$\pm$0.66 is the flux
of the mu + tau flavours. Hence oscillation is confirmed. Roughly two third
of the eneutrinos have oscillated to the other flavours. Further, comparing
with the SSM prediction of 5.05$\pm$0.40, SSM also is confirmed. So at one
sweep the SNO results confirmed both the SSM based on the thermonuclear
fusion hypothesis and neutrino oscillation.

What is the moral of the story? When we said in the beginning that
the thermonuclear hypothesis for the Sun has to be proved, it was not a
question of proof before a court of law. Science does not progress that way.

In trying to prove the hypothesis experimentally through the detection of
solar neutrinos, Davis and the other scientists have made a discovery of
fundamental importance, namely that the neutrinos have mass. Only if they
have mass, they can oscillate.

\section{Neutrino Oscillation}

To understand neutrino oscillation, one must think of neutrino as a wave
rather than a particle (remember quantum mechanics).Neutrino oscillation
is a simple consequence of its wave property.Let us consider the analogy with
light wave.Consider a light wave travelling in the z-direction. Its polarization
could be in the x-direction, y-direction or any direction in the x-y plane.This
is the case of plane-polarized wave. However the wave could have circular
polarization too, either left or right. Circular polarization can be composed as
a linear superposition of the two plane polarizations in the x and y directions.
Similarly plane polarization can be regarded as a superposition of the left and
right circular polarizations.

Now consider plane polarized wave travelling through an optical medium.
During propagation through the medium, it is important to resolve the plane
polarized light into its circularly polarized components since it is the circularly polarized wave that has well-defined propagation characteristics such as
the refractive index or velocity of propagation. In fact in an optical medium
waves with the left and right circular polarizations travel with different velocities. And so when light emerges from the medium, the left and right circular
polarizations have a phase difference proportional to the distance travelled.
If we recombine the circular components to form plane polarized light, we
will find the plane of polarization to have rotated from its initial orientation.
Or, if we start with a polarization in the x-direction, a component in the ydirection would be generated at the end of propagation through the optical
medium.

For the neutrino wave, the analogues of the two planes of polarizations
of the light wave are the three flavours (e, mu or tau) of the neutrino (see
table 1). When the neutrinos are produced in the thermonuclear reactions
in the solar core, they are produced as the e type. When the neutrino wave
propagates, it has to be resolved into the analogues of circular polarization
which are energy eigenstates or mass eigenstates of the neutrino. These states
have well-defined propagation characteristics with well-defined frequencies (remember frequency is the same as energy divided by Planck’s constant).
The e type of neutrino wave will propagate as a superposition of three mass
eigenstates which pick up different phases as they travel. At the detector,
we recombine these waves to form the flavour states.Because of the phase
differences introduced during propagation, the recombined wave will have
rotated ”in flavour space”. In general, it will have a mu component and tau
component in addition to the e component it started with. This is what is
called neutrino oscillation or neutrino flavour conversion through oscillation.
\begin{longtable}{cc}
\hline
Light wave & Neutrino wave \\
\hline
Plane polarization & Flavour state\\
x or y & $\nu_{e}$, $\nu_{\mu}$ or $\nu_{\tau}$\\
\hline
Circular polarization & Mass eigenstate\\
right or left & $\nu_{1}$, $\nu_{2}$ or $\nu_{3}$\\
\hline
\caption*{Table 1 : The analogy between light wave and neutrino wave}
\end{longtable}

\newpage

Flavour conversion is directly due to the phase difference arising from the
frequency difference or energy difference which in turn is due to the mass
difference. Mass difference cannot come without mass.Hence discovery of
flavour conversion through neutrino oscillation amounts to the discovery of
neutrino mass. This is the fundamental importance of neutrino oscillation,
since so far neutrinos were thought to be massless particles like photons.

Since it is an oscillatory phenomenon, the probability of flavour conversion
is given by oscillatory functions of the distance travelled by the neutrino
wave, the characteristic ”oscillation length” being proportional to the average
energy of the neutrino and inversely proportional to the difference of squares
of masses. Further,the overall probability for conversion is controlled by the
mixing coefficients that occur in the superposition of the mass eigenstates to
form the flavour states and vice versa. These mixing coefficients form a 3x3
unitary matrix.

Neutrino oscillations during neutrino propagation in matter become much
more complex and richer in physics, but we shall not go into the details here.
After Wolfenstein calculated the important effect of matter on the propagating neutrino and Mikheyev and Smirnov drew attention to the dramatic
effect on neutrino oscillation when the neutrino passes through matter of varying density, it was Bethe who gave an elegant explanation of the MSW
(Mikheyev-Smirnov-Wolfenstein) effect based on quantum mechanical levelcrossing. In fact most people (including the present author) appreciated the
beauty of MSW effect only after Bethe’s paper came out. One may comment that Bethe redeemed himself for his earlier omission of neutrinos in his
famous paper on the energy production in stars.

We next go to the cosmic-ray-produced neutrinos since their study and
its interplay with solar neutrino research constitute a fascinating chapter in
the story of the neutrino.

\section{Cosmic-Ray-Produced or Atmospheric\\ neutrinos}

Cosmic Rays were discovered around the year 1900. They are mostly very
energetic protons. They are created in many parts of the Universe and are
flying in all directions everywhere. They fall on Earth too. Since Earth is
surrounded by atmosphere, these protons collide on the nitrogen or oxygen
nuclei of the atmosphere and in these collisions many kinds of elementary
particles are created. All these move in the direction of the Earth. Figure 1
shows such a cosmic-ray shower. Many elementary particles such as muon
($\mu$), pion ($\pi$), and Kaon (K) were originally discovered in cosmic ray research
only. As seen from the Figure 1, all these particles decay and give rise
to neutrinos. They are cosmic-ray produced neutrinos although they are
generally called atmospheric neutrinos.

Homi Jahangir Bhabha who founded the Tata Institute of Fundamental
Research in Mumbai was well-known for cosmic ray research. Around 1950,
he suggested to B V Sreekantan that cosmic ray research must be conducted
in Kolar Gold Field (KGF) mine which is one of the deepest mines in the
world. His idea was to measure the flux of cosmic ray particles as we go down
the depth of one or two kilometers below the Earth and verify experimentally whether the penetrating component of the cosmic rays was composed
of muons alone (as he had concluded in his earlier theoretical research) or
whether there was any other particle.

Sreekantan, Ramanamurthy and Naranan followed Bhabha’s suggestion
and thus started the pioneering KGF experiments and the experiments continued for more than two decades. The scientists determined how the muon flux decreased as a function of the depth. When the experiments were continued at greater and greater depths, at a certain depth the number of the
muons detected became zero. At that depth (which was about 2 kilometer
from the surface of the Earth) all the muons are absorbed by the rock above,
but neutrinos are not absorbed and hence could be detected without any
disturbance from other particles such as muons. The scientists succeeded in
detecting these neutrinos. This happened in the year 1965. This was the
first detection of cosmic-ray-produced neutrinos in the world. The credit for
this achievement goes to the Tata Institute of Fundamental Research and
two other collaborating institutions Durham University, UK and Osaka University, Japan. Last year 2015 was the Golden Jubilee Year of this milestone
in the story of the neutrino.

figure?????
\newpage

The atmospheric neutrino research that started in India progressed further especially in Japan and brought great success to the Japanese physicists.
We have already described how the Kamioka and SuperKamioka experiments
succeeded in catching the solar neutrinos. The same experiments caught the
atmospheric neutrinos also. Further study led to another discovery which we
describe now.

The pion born from cosmic rays decays into a muon and a muneutrino.
Then the muon also decays into an electron, an eneutrino and a muneutrino.
The decay of the Kaon also leads to same results. Hence as shown in Figure 1, the number of muneutrinos reaching the Earth is twice the number
of eneutrinos. In the Kamioka experiments, it was possible to distinguish
the two kinds of neutrinos. Since the cosmic ray protons had a very high
energy, about 1000 MeV, the neutrinos born from them have very high energy and so can create the muons of 107 MeV. The muneutrinos colliding
with the nuclei in the detector produce muons and eneutrinos produce electrons. Since muons and electrons emit different kinds of Cerenkov light, the
Kamioka and SuperK experimenters succeeded in counting the number of
colliding muneutrinos and eneutrinos separately.

The underground SuperK detector and the directions in which neutrinos arrive at the detector are shown in Figure 2. The sky and atmosphere
surround the Earth in all directions and so the neutrinos arrive from all directions. In the downward direction, the ratio of muneutrinos to eneutrinos was
experimentally shown to be 2 as expected. But this ratio gradually decreased
from 2 as the direction changed and became unity for the upward moving
neutrinos. Although Kamioka detector and a few other detectors saw this
anomaly in 1990, it required the bigger SuperK detector with its superior
statistics to establish the effect in 1998.

About half of the upward moving muneutrinos have disappeared. How?
The maximum height of the atmosphere is about 20 kilometer. So neutrinos coming downwards from above travel only a few kilometers and reach
the detector without oscillation. Neutrinos coming upwards have to cross a
distance of 13,000 kilometers which is Earth’s diameter and undergo oscillation. Half of the muneutrinos oscillate to the tauneutrinos. Although the
cosmic-ray produced neutrinos have high enough energy to create the muon,
their energy is not sufficient to create the taon of mass 1777 MeV. So the
tauneutrinos escape undetected. Thus the SuperK experiment discovered the
oscillation of cosmic-ray produced neutrinos.

figure??????

\section{The Nobel Prizes: Solar and Atmospheric Neutrinos}

One may say that it is in the Davis experiment on solar neutrinos that
neutrino oscillation and neutrino mass were discovered first. However it was
not possible to accept these conclusions as firm on the basis of the Davis
experiment. For, as we mentioned earlier the question as to whether the
flux of the higher energy neutrinos from the Sun was calculated correctly
could not be settled without any doubt. This doubt was completely removed only by the results of the two-in-one experiment of SNO, since the inference
of neutrino oscillation from SNO results was completely independent of the
calculation of the solar neutrino flux.

SNO results came out only in 2002. Much before that, in 1998, SuperK
discovered the oscillations of cosmic-ray-produced neutrinos. Their discovery
concerned the ratio of muneutrinos to eneutrinos and hence did not depend
on the uncertainties of calculated fluxes of the neutrinos produced by cosmic
rays. Hence it was accepted that the discovery of neutrino oscillation and
neutrino mass by SuperK in cosmic-ray-produced neutrino experiments was
free from doubts of the kind that plagued the interpretation of Davis and
SuperK experiments on solar neutrinos. In the race for the discovery of
oscillations experiments on cosmic-ray-produced neutrinos won over those
on solar neutrinos.

In 2002, Nobel Prize was given to Ray Davis who pioneered solar neutrino
research, was the first to detect solar neutrinos and continued the experiments
for more than 30 years and Matoshi Koshiba who was the leader of the
Kamioka and SuperK experiments that detected solar neutrinos, cosmicray-produced neutrinos and Supernova neutrinos. The Nobel Prize of 2015
was given to Arthur McDonald who was the leader of SNO which proved
thermonuclear fusion as the source of solar energy and firmly established
oscillation of solar neutrinos and to Takaaki Kajita who was the leader of
SuperK that discovered the oscillations of cosmic-ray- produced neutrinos.

\section{Neutrino masses and Mixing}

As we already mentioned, nuclear reactors produce antineutrinos copiously. High energy protons from particle accelerators produce pions whose
decays ultimately lead to neutrinos. This is in fact the same process as in
case of cosmic-ray protons which we mentioned earlier.

Solar neutrinos, atmospheric neutrinos, reactor neutrinos and accelerator
neutrinos – many experiments on all these have been done and considerable amount of information on neutrino oscillations have been learnt. Most
importantly, the mass-differences between the three kinds of neutrinos have
been determined and they are very very tiny:
\begin{align*}
m^{2}_{2}-m_{1}^{2} &= 0.00007ev^{2}\\
|m_{3}^{2} -m_{2}^{2}| &= 0.002ev^{2}
\end{align*}

Note one of the mass difference is known only in magnitude and its sign
has yet to be determined and so the ordering of the three mass levels is not
yet known.

Oscillation experiments give only mass differences. To determine the
mass itself a different kind of experiment has to be done. From the precision
experimental study of the continuous energy distribution of the electrons
emitted in the beta decay of Tritium (heavy Hydrogen), an upper limit of
2.2 eV for the neutrino mass or masses has been determined. So, all the three
neutrino masses are clustered close to each other at a value smaller than 2.2
eV. Among all the massive elementary particles, electron has the lowest mass
0.5 MeV. Neutrino masses are a million times smaller. But many secrets of
the Universe are hidden in this tiny number.

As a culmination of hundred years of fundamental research a theory called
the Standard Model of High Energy Physics [4] has been shown to be the
basis of almost All of physics except gravity. But according to this theory
neutrinos are massless. Hence the importance of the discovery that neutrinos
have mass. Neutrino mass may be the portal to go beyond Standard Model.

The oscillation experiments also determined the $3 \times 3$ mixing matrix that
tells us how the three massive neutrinos are superposed to give the three
flavours e, $\mu$, $\tau$ of neutrinos. This unitary matrix is characterized by three
angle parameters and a phase. The values of the three angles as determined
by the oscillation experiments [5] are
\begin{align*}
\theta_{12} &= 30^{o}\\
\theta_{23} & = 45^{o}\\
\theta_{31} & = 9^{o}
\end{align*}

The phase however is not yet determined. This phase is very important since it signals matter-antimatter asymmetry which in turn can play an
important role in the evolution of the Universe as pointed out below.

\section{Continuing Story}

There are many more things in the neutrino story. We shall describe
them briefly.

Generally there is an antiparticle for every particle. This is a Law of Nature which is a consequence of combining quantum mechanics with relativity
and was discovered by Dirac. Positron is the antiparticle of electron. Their electric charges are equal in magnitude but opposite in sign. However, when
the electric charge is zero as is the case for neutrino, its antiparticle, namely
the antineutrino could be the same as the neutrino itself. If this is true, the
particle is called a Majorana particle, named after Majorana who envisaged
such a possibility. A particle whose antiparticle is different, such as the electron is called a Dirac particle. Is neutrino a Majorana particle?[6] This is
the most important question in Neutrino Physics and this question can be
answered only by the ‘neutrinoless double beta decay experiment’. These
experiments are going on, but have not yet yielded a definitive answer.

Cosmologists have found good evidence that the Universe was born 14
billion years ago in a gigantic explosion called the Big Bang. At that point,
the Universe must have contained equal number of particles and antiparticles.
However there are only particles now. All the atoms in the Universe are made
of protons, neutrons and electrons only. What happened to the antiprotons,
antineutrons and positrons? How did they disappear? How was the matterantimatter symmetry that existed at the beginning of the Universe destroyed?
This is an important cosmological puzzle. The key to solving this puzzle is
contained in the neutrino. If neutrino and antineutrino can be proved to be
the same and if the phase in the mixing matrix (see above) is proved to be
nonzero, this puzzle can be answered. Hence, neutrino plays an important
role in cosmological research.

Supernova explosion is the end stage of most of the stars. Most of the
energy of the explosion is released through the neutrinos that are emitted in a
very large number. The neutrinos emitted in the so-called Supernova 1987-a
were detected in the SuperK detector. This was one of the reasons for the
Nobel Prize given to Koshiba in 2002 since this was the first time neutrinos
from outside the solar system were detected on the Earth and supernova
neutrino research was thus initiated.

Ultrahigh energy neutrinos with energy greater than 1012 eV coming from
outer space have been detected in the year 2013. This was achieved by using
ice as a detector in the Antarctica Continent near the South Pole. The size
of this ice detector is one kilometer in length, one kilometer in breadth and
one kilometer in height and it is called Ice Cube.

Radioactive Uranium and Thorium ores lying buried in the deep bowels
of the Earth emit neutrinos. These geoneutrinos have been detected in the
KamLAND detector in Japan and the BOREXINO detector of the Gran
Sasso laboratory in Italy.\break Through this, one can map where and at what
depths Uranium and Thorium ores lie and this knowledge will be used in Geochronology. Thus a new window on Earth Science has been opened by
neutrino research.

The bulk of the low-energy neutrinos constituting more than 90 percent
of the solar neutrinos which had eluded detection have now been detected
by the BOREXINO detector. The measured flux is in very good agreement
with SSM.

Neutrinos are the most penetrating radiation known to us. A typical neutrino can travel through a million earth diameters without getting stopped.
However because of the MSW effect the neutrino senses the density profile
of the matter through which it travels and so the flavour composition of the
final neutrino beam can be decoded to give information about the matter
through which it has travelled. Hence tomography of the Earth’s interior
through neutrinos will be possible which may even lead to the prediction of
earthquakes in future. This requires our mastery of neutrino technology. But
neutrino technology will be mastered and neutrino tomography will come.


Efforts are going on all over the world to create new underground laboratories for neutrinos. As already mentioned, India was a pioneer in neutrino
research. The cosmic-ray-produced neutrinos first detected in KGF in India
in 1965 led to two Nobel Prizes for the Japanese physicists. But the KGF
mines were closed in 1995. To recover this lost initiative the India-based
Neutrino Observatory (INO) has been planned [7]. The underground laboratory will be created in a huge cavern to be dug out in a mountain in Theni
District near Madurai and the main Centre of INO will be built in Madurai
City. In the first stage, a neutrino oscillation experiment using atmospheric
neutrinos will be performed in a gigantic 50,000 ton magnetised iron detector
which will be mounted inside the underground laboratory [7].

\section*{Milestones in the neutrino story}

\noindent
1930 Birth of Neutrino: Pauli\\
1932 Theory of beta decay, ”Neutrino” named: Fermi\\
1954 First detection of neutrino: Cowan and Reines\\
1964 Discovery of muneutrino: Lederman, Schwartz and Steinberger\\
1965 Detection of atmospheric neutrino: KGF\\
1970 Start of the solar neutrino experiment: Davis\\
1987 Detection of neutrinos from supernova: SuperKamioka\\
1998 Discovery of neutrino oscillation and mass: SuperKamioka\\
2001 Discovery of tauneutrino: DONUT\\
2002 Solution of the solar neutrino puzzle: SNO\\
2005 Detection of geoneutrinos: KamLAND\\
2013 Detection of ultra high energy neutrinos from space: Ice Cube

\newpage

\begin{thebibliography}{99}
\bibitem{} For an elementary account of Fermi’s theory see G Rajasekaran,
Fermi and the theory of weak interactions, Resonance, 19, 18
(2014)

\bibitem{} H A Bethe, Energy production in stars, Phys.Rev, 55, 434
(1939)
\bibitem{} For more details see G Rajasekaran, Hans Bethe, the Sun and
the neutrinos, Resonance, October 2005, p 49-67
\bibitem{} For an elementary account of the Standard Model see G
Rajasekaran, Standard Model, Higgs boson and what next?,
Resonance, 17, 956 (2012)
\bibitem{} For recent history see G Rajasekaran, An angle to tackle the
neutrinos, Current Science 103, 622 (2012)
\bibitem{} For more on this question see G Rajasekaran, Are neutrinos
Majorana particles? arXiv:0803 4387
\bibitem{} \url{http://www.ino.tifr.res.in/ino}

\end{thebibliography}
