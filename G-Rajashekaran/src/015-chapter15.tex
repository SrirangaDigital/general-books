\chapter{Profound Truths}\label{chap15}

\Authorline{G Rajasekaran} 
\addtocontents{toc}{\protect\contentsline{section}{{\sl G Rajasekaran}\smallskip}{}}
\lhead[\small\thepage]{\small\leftmark}
\authinfo{Institute of Mathematical Sciences, Chennai 600113\\   
        and Chennai Mathematical Institute, Siruseri 603103}

                   

Niels Bohr, the famous theoretical physicist, who was one
of the pioneers of quantum theory that revolutionized physics
almost hundred years ago, is reported to have said that a
profound truth is one whose opposite also is true. I shall
discuss a few Profound Truths that satisfy this criterion.

\section{Is light made of particles or waves?}

In this famous controversy, Newton was an advocate of the
corpuscular theory of light, while Fresnel, Young and Huygens
developed the wave theory of light which was in beautiful
agreement with the observed phenomena such as interference
and diffraction of light. But the great merit of Newton was 
that he sensed the inadequacy of a pure corpuscular theory
and so endowed the corpuscles with a bit of wave-like nature,
which he called "fits". Maxwell discovered that light is an
electromagnetic wave and thus solved the problem of what 
kind of wave light is. Finally in quantum mechanics, light
is described both as a wave and as a particle, the photon.
In fact the same turns out to be true of all elementary
particles like electron; electron is both a particle and
a wave in quantum mechanics.

\newpage

\section{Is gravitation a force or is it a property of space?}

Aristotle is supposed to have taught that while smoke
goes up to the sky a stone drops down to the ground because
this is the nature of the 'up" and "down". Galileo and Newton
attributed the motion of bodies to gravitational force.
Galileo discovered the correct way to describe motion and
Newton completed the picture by enunciating his laws of
motion and the law of gravitational force. So the stone
falls down due to the gravitational force of the Earth.
But Einstein came and replaced Newton's gravitational force
by the property of space (actually space-time) around the
Earth. In Einstein's theory, the stone just follows the
path allowed by the curvature of space caused by the Earth.

George Bernard Shaw, who introduced Einstein to a London
audience when the latter made his first visit to the city
after he became famous, is reported to have said:
"From the time of Aristotle everybody knew the stone drops
down while smoke goes up, because that is in the nature of
things and in the nature of space. A famous English man,
Newton, told us that is all wrong. Bodies move because of 
gravitational force. Now I am introducing to you a German
Professor who has recently proved that Newton is wrong and
Aristotle was correct." 

Of course there is a bit of literary licence in what Bernard 
Shaw said, but one cannot deny that there is an element of
Truth in what he said. Einstein restored gravity to a property
of space (or rather space-time) which is nearer to the
Aristotlean view.

\section{Ptolemy or Copernicus: who is right?}


For more than a millenium the geocentric model of the
Universe prevailed and Ptolemy and his successors succeeded
in erecting a fairly accurate model of the motion of
the planets using a complicated system of circles and epicycles.
Copernicus replaced it by the solar model in which all the
planets including the Earth go around the Sun. This is called
the Copernican revolution since the dethroning of the Earth
as the Centre of the Universe had very profound consequences
in human thought going much beyond Astronomy. Astronomy itself
progressed greatly and the story of Galileo, Tycho Brahe, Kepler
and Newton are well-known. Physics and all of Science got a big
boost after Newton showed that objects in the Heavens as well
as on the Earth are governed by the same Laws of Nature.

So it is an accepted truth that planets go around the Sun rather
than the Earth. But what is the big difference? Both are in fact
correct (as was first pointed out by Fred Hoyle). One can do 
Astronomy as well as Physics in both frames
of reference - in one frame of reference Sun is at rest and in
the other Earth is at rest. One can go from one frame of reference
to the other by a transformation of coordinates. 
In fact in Einstein's General Relativity all frames of reference 
are equivalent.

However we must point out that although both Ptolemy and
Copernicus are right, the Ptolemaic system led to a deadend
as far as Science is concerned. It is the Copernican view that
led to real progress in Astronomy and Physics, as we indicated
above. Often, one version of a profound truth may be more
fruitful for the progress of science, atleast for a while.

\section{S Matrix Theory vs Quantum Field Theory}

The inward bound path of discovery unravelling the mysteries of
matter and the forces holding it together - at deeper and ever
deeper levels - culminated after a hundred years
in the following picture of structure repeating inside structure:
$$
\text{Atoms} \rightarrow \text{Nuclei} \rightarrow \text{Nucleons} \rightarrow \text{Quarks} \rightarrow ?
$$

The distance scale travelled thus far is from $10^{-8}$ cm (size of the
atom) to less than $10^{-17}$ cm. Quarks which are the constituents
of the nucleons (protons and neutrons) are seen to be point-like 
down to the scale of $10^{-17}$ cm probed so far. Are there
further structures below that scale? Only future can tell.

But at our present level of knowledge, quarks and electrons are
the fundamental particles out of which all known forms of matter
are composed and their dynamics is governed by Quantum Field 
Theory (QFT).

In this inward bound journey, the question: "Are some particles
more fundamental than others?" was faced and answered.
The decade 1956 to 1965 was an important epoch in this inward
bound journey. Actually it was the golden age of hadrons (strongly
interacting particles like the nucleons). Hundreds of these
particles were discovered and under the influence of this deluge,
QFT was declared dead and an alternate philosophy called
S Matrix theory was proposed, its chief proponent being 
G F Chew. In this theory all the hundreds of hadrons were
regarded as equally elementary and this was called the
principle of nuclear democracy. Ultimately this approach turned out
to be a dead end.  

A different line of attack spearheaded by M Gell-Mann proved more
successful. We have to omit many interesting technical points here.
Starting with quarks as the elementary constituents of hadrons, this 
finally led to the QFT of all hadrons. 
 
Although Gell-Mann and Zweig independantly
proposed the idea of quarks in 1964, it took many years before
quarks emerged as a physical reality. Gell-Mann himself was
tentative and said quarks are only mathematical. The chief reason
for the reluctance to accept quarks as constituents of hadrons
was the prevalent S Matrix philosophy at that time. The idea of
quarks as being more elementary or more primary than the other 
hadrons was repugnant to the whole scheme of nuclear democracy.
In fact in a public lecture at TIFR, Mumbai in the late 1960's
Chew claimed that relativity theory and quantum mechanics inplied
the impossibility of anything more fundamental than the hadrons.
Nevertheless quarks have been vindicated.

However one must not conclude that S Matrix approach was a complete
failure. Although it was a failure as a theory of hadrons, it is
this approach that gave rise to String Theory which may turn out to
be the correct theory of much more than hadrons! Most importantly
this theory in which all elementary particles occur as mere vibrations 
of a string, promises to solve the sofar-intractable problem of
Quantum Gravity too. Further, in current research in QFT, there are
already signs of S Matrix concepts creeping in. 

For more on the tortuous history of quarks and the many twists
and turns in the historical panorama of high energy physics, one
may read the following: From atoms to quarks and beyond: a historical
panorama, by G Rajasekaran (arXiv:physics/0602131), pp 361-392,
India in the World of Physics: Then and Now, Ed: Asoke N Mitra,
Pearson Longman, Delhi.

\section{Big Bang vs Steady State Universe} 

Was the Universe born at some time in the past and is destined 
to die at some time in the future? Or is the Universe for ever?
Surely this is a profound question.

It is now believed by cosmologists that the Universe started
about 13 billion years ago as a tiny fireball and expanded
to its present size and is still expanding. This is the
Big Bang model of the Universe oroginally due to Lemaitre
and George Gamow. There exists the rival model of the
steady state Universe developed by Bondi, Gold, Hoyle and 
Narlikar which provided an interesting alternative. However
in the past decades observational evidence especially the
discovery of cosmic microwave radiation which is the relic radiation
left over from the initial fireball and the detailed precision data
on the radiation matching what is expected from the Big Bang Universe
seems to have more or less killed the steady state model.

But, is the steady state Universe really dead? No.
Although the Universe does evolve in time, it may be cyclic,
expansion following contraction and the whole cycle repeating
for ever. Already there are attempts
by cosmologists to answer the question: what existed before
the Big Bang? All this might lead to a bigger picture 
in which the Big Bang explosion may be only one of the episodes and the
Universe on the whole might be for ever! 

So the question cannot be decided that easily. The jury 
is still out.

\section{Evolution vs Intelligent Design?}

Finally we come to the most profound of all questions. Is science
complete, or is there something beyond, that is hidden from us?

Scientists have traced the history of the Universe for 13 billion
years - from the Big Bang, to the formation of stars and galaxies,
to the formation of planets - evolution of life on the Earth from 
primitive forms to the human being with a wonderful mind that is now
asking what is the meaning of all this. The general scientific view is
that everything evolved by itself by the inherent "Laws of Nature".
The specific path that evolution of life itself took on this 
insignificant corner of the Universe has a lot to do with 
randomness and chance. Randomness and the survival of the fittest
have resulted in what we see and experience. 

Science has also probed the inner workings of matter 
and the forces that hold them together down to a size 
seventeen orders of magnitute smaller than a centimeter. 
No one has any clue about the meaning of all the beauty of 
the structures and concepts that human intellect has revealed 
in this inward bound quest.

The amazing success of Science has led to the dominant view
among the scientists: "Do not look for any meaning hidden
behind all that science has revealed."

There is however the opposite view:
" There are more things in Heaven and Earth, Horatio,
Than are dreamt of in your philosophy" 
                     \hspace{2cm}{- Shakespeare (in Hamlet, Act I, Scene V).}
Great thinkers with perceptive minds have realised this.
They have been inspired by something beyond science which they have
grasped intuitively. Even scientists are uneasy by 
the emptiness of a world view that does not go beyond science. 
Some scientists articulate it, some do not. 

This essay started with Niels Bohr. He enunciated the concept of
complementarity while wrestling with the conundrums and 
contradictions of early quantum theory. The contrary views must be
regarded as complementary with each other and both are required 
for a complete perception of physical reality, he advocated.
Maybe the same applies to the question under discussion. A fuller
picture of the Universe may require going beyond Science.



 

