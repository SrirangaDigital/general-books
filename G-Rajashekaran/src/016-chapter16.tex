\chapter{High Energy Physics- Essays and Reviews}\label{chap16}

\Authorline{}
\addtocontents{toc}{\protect\contentsline{section}{{\sl }\smallskip}{}}

\authinfo{}
\lhead[\small\thepage]{\small\thechapter. High Energy Physics- Essays and Reviews}


             
\section{Preface and Notes}

This book contains articles that i wrote and the lectures
that i gave at various venues during the last five decades.
They provide a historical panorama of high energy physics incuding the 
physics of many discoveries. Infact the most interesting way to
learn Physics is through History.

Here in these notes i will try to mention the context of the
articles, where the lectures were given, and where the articles
were published.

\section{Essays}
\begin{itemize}
\item Profound Truths: published in Current Science
\item Einstein and a Century of Physics: lecture given at various venues during the Year of Physics 2005
\item Dirac Equation and its aftermath: published in Resonance
\item Hans Bethe, the Sun and Neutrinos: published in Resonance
\item Cabibbo and the Universality of Weak Interactions; pub. in Resonance
\item An angle to tackle the neutrinos; pub. in Current Science
\item A crisis in Fundamental Physics: Guest Editorial in Current Science
\item A great opportunity for Indian Physics: It is about INO and was published in the Hindu, dated
\item Abdul Kalam and INO: Tamil version published in Mulumai Ariviyal Udayam
\item Neutrinos and Life: Tamil version published in Mulumai Ariviyal Udayam
\item Is neutrino a Majorana Particle: published in Proc of NDBD Workshop
\item Linking Weak Interactions with Electrodynamics: pub in Science Today
\item Elecroweak Symmetry: pub. in Proc. of Raman Centenary Symposium
\item Integral versus fractional charged quarks;pub. in Few Body Conference
\item Baryon number nonconservation in unorthodox models: published in the International Conference on baryon number nonconservation, Bombay, 1981
\item Is Quantum Mechanics for ever? Talk at Workshop on Foundations of QM
\item Manpower for Fundamental Physics: published in Current Science
\item Two topics in EW Physics: Pub in WHEP
\item Dark Universe: Tamil version was published in Mulumai Ariviyal Udayam
\item Murray Gell-Mann and the story of Strong Interaction: pub in Wire
\end{itemize}

\section{Reviews}

\begin{itemize}
\item Yang-Mills fields and the theory of weak interactions: lecture given in SINP, 1971. This was perhaps the very first lecture on what became known later as the Elecroeak Theory 
\item Higgs Boson and the Standard Model: pub. in Resonance
\item The Story of the neutrino: pub.in Modern Atomism, Ed: J Pasupathy, History of Science,Philosophy and Culture in Indian Civilization, Vol XII, Part 4
\item Fermi and the theory of weak interactions: published in Resonance
\item Atoms, quarks and beyond- a historical panorama
\item Solitons, monopoles and and bags: pub. in Proc. of Solid State and Nuclear Physics Symposium
\item New Forms of quantum statistics and  generalized Fock spaces: Lecture given at the S N Bose Centenary Symposium
\item Perspectives in High Energy Physics: pub. in the  HEP Symposium, Calcutta
\item Gauge Theories- a Review, pub. in the Proc. of Einsten Centenary Symposium
\item Grand Unified Theories: pub. in Proc of HEP Symposium, Mysore
\item Magnetic Monopoles: Pub.in " 50 years of SINP"
\item Electroweak Radiative Corrections: Proc. of WHEPP 1989, Bombay
\item Superstrings-an elementary review: Symposium on "Changing faces of particle nuclear physics". This talk was given before the duality symmetries was discovered. The duality symmetries make the five consistent string theories as different views of the same theory. So there is only one consistent string theory!  

\section{Lectures in High Energy Physics}
                    
\item Unification of weak and electromagnetic interactions: Proc of First HEP Sumposium 1972. This is the very first review of electroweak theory in India
\item Building up the Standard Model of HEP: pub in Proceedings of the UGC School on Gauge Theories, Gravity and Cosmology
\item Recent Developments in High Energy Physics: Panchgani Lectures and nuclear physics, BHU. This was given before the duality symmetries was
\item String Theory: Lectures at the UGC School, Madras. These lectures were given before the duality symmetries were discovered. As a consequence of the duality symmetries, the five theories are to be regarded as different views of the same theory. So there is only one consistent string theory!
  
