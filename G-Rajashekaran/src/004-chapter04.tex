\chapter{Hans Bethe, the Sun and the Neutrinos}\label{chap4}

\Authorline{G Rajasekaran}
 \addtocontents{toc}{\protect\contentsline{section}{{\sl G Rajasekaran}\smallskip}{}}
\authinfo{Institute of Mathematical Sciences,\\ 
Madras-600113}

\section{INTRODUCTION}

This is an elementary account of the recent
discoveries in Neutrino Physics.Because of its historical
importance and its role in the story of Hans Bethe, 
the genesis of the solar neutrino problem and
its solution in terms of neutrino oscillation are described
in greater detail. In particular, we trace the story of the
80-year-old thermonuclear hypothesis which states that the
Sun and the stars are powered by thermonuclear fusion reactions.
We describe how the Sudbury Neutrino Observatory in Canada was
finally able to give a direct experimental proof of this
hypothesis in 2002 and how, in the process, a fundamental
discovery i.e. the discovery of neutrino mass was made.

Atmospheric neutrinos and reactor neutrinos
are important for a complete analysis of neutrino oscillations.
These and many other equally important issues are briefly
discussed at the end.

\section{SOLAR NEUTRINOS}

In the 19th century, the source of the energy in the Sun
and the stars remained a major puzzle in science, which
led to many controversies. Finally, after the
discovery of the tremendous amount of energy locked up
in the nucleus, Eddington in 1920 suggested nuclear energy as
the source of solar and stellar energy. It took
many more years for the development of nuclear physics
to advance to the stage when Bethe,the Master Nuclear 
Physicist, analysed all the
relevant facts and solved the problem completely in 1939.
A year earlier,Weisszacker had given a partial solution.

Bethe's paper is a masterpiece.It gave a  complete
picture of the thermonuclear reactions that power the
Sun and the stars. However, a not-so-well-known fact
is that Bethe leaves out the neutrino that is emitted
along with the electron, in the reactions enumerated
by him. Neutrino, born in Pauli's mind in 1932, named
and made the basis of weak interaction by Fermi in
1934, was already a well-known entity in nuclear
physics. So it is rather inexplicable why Bethe ignored
the neutrinos in his famous paper.The authority of
Bethe's paper was so great that the astronomers and
astrophysicists who followed him in the subsequent
years failed to note the presence of neutrinos.Even
many textbooks in Astronomy and Astrophysics written
in the 40's and 50's do not mention neutrinos! This
was unfortunate,since we must realize that,inspite
of the great success of Bethe's theory,it is
nevertheless only a theory.Observation of neutrinos
from the Sun is the only direct experimental
evidence for Eddington's thermonuclear hypothesis
and Bethe's theory of energy production. That is
the importance of detecting solar neutrinos.

The basic process of thermonuclear fusion in the Sun and stars
is four protons combining into an alpha particle and releasing
two positrons,two neutrinos and 26.7 MeV of energy.So it is
trivial to calculate from the solar luminosity the total
number of neutrinos emitted by the Sun; for "every" 26.7 MeV
of energy received by us, we must get 2 neutrinos.
Thus one gets the
solar neutrino flux at the earth as 70 billion per square
cm per sec. So an enormous number of neutrinos are passing
through our body!

However,the probability of four protons meeting at a point
is negligibly small even at the large densities existing
in the solar core.Hence the actual series of nuclear reactions
occuring in the solar and stellar cores are given by the
so-called carbon cycle (Box 1) and the pp-chain (Box 2).
In the carbon cycle
the four protons are successively absorbed in a series of
nuclei,starting and ending with carbon.In the pp-chain two
protons combine to form the deuteron and further protons
are added.We shall not go into details here except noting that
both in the carbon cycle and the pp-chain,the net process is
the same as what was mentioned above, namely the fusion of
four protons to form alpha particle with the emission of two
positrons and two neutrinos.

In the Sun,the dominant process is the pp-chain.Although the
total number of neutrinos emitted by the Sun could be trivially
calculated from the solar luminosity,their energy spectrum which
is crucial for their experimental detection,requires a detailed
model of the Sun, the so-called Standard Model of the Sun (SSM).
SSM is based on the thermonuclear hypothesis and Bethe's theory,
but uses a lot more physics input.
A knowledge of the neutrino energy spectrum is needed since
the neutrino detectors are strongly energy sensitive.Infact all
detectors have an energy threshold and hence miss out the very
low energy neutrinos.

Leaving out the details, the solar neutrino spectrum is
roughly characterized by a dominant (0.9975 of all neutrinos)
low energy spectrum ranging from 0 to 0.42 MeV and a very weak 
(0.0001 of all the neutrinos)
high energy part extending from 0 to 14 MeV (Box 3). 
The former arises from the 
pp reaction of two protons combining to form a deuteron,
a positron and a neutrino.The latter comes from the beta decay
of Boron-8 which is produced in a thermonuclear reaction
initiated by a proton combining with Beryllium-7.Most of the
neutrino detectors detect only the tiny high-energy branch of
the spectrum, the so-called Boron-8 neutrinos.

While the dominant low-energy neutrino flux is basically 
determined by the solar luminosity,the flux of the high-energy
Boron-8 neutrino flux is very sensitive to the various physical
processes in the Sun and hence is a test of SSM. Infact,this
latter flux is a very sensitive function of the temperature
of the solar core,being proportional to the 18th power of
this temperature and hence this neutrino flux provides 
a very good thermometer for the solar core. In contrast to
the photons which hardly emerge from the core,the neutrinos
escape unscathed and hence give us direct knowledge about
the core.

There is a simple physical reason for this sharp dependance
on temperature. It is related to the quantum-mechanical
tunnelling formula, the famous discovery of George Gamow.
The probablity for tunnelling through the repulsive Coulomb
barrier has a sharp exponential dependance on the kinetic
energy of the colliding charged particles.

The pioneering experiment on solar neutrinos started by Davis
and collaborators in the 60's is based on the inverse beta decay
process:Chlorine-37 absorbs the neutrino to yield Argon-37 and
an electron. (See Box 4 for beta decay and inverse beta decay.)
A tank containing 615 tons of a fluid rich in chlorine
called tetrachloroethylene was placed in the Homestake gold mine
in South Dakota(USA).The fluid was periodically purged with
Helium gas to remove the argon atoms which were then counted
by means of their radioactivity.In a typical series of 62 runs
during 1970-1983, the number of radioactive Argon-37 atoms
detected per day was 0.44$\pm$0.04.Of this,0.08$\pm$0.03 was attributed
to cosmic ray and other background and so the number of argon
atoms produced by solar neutrino capture was 0.36$\pm$0.05 per day.
These numbers give an idea of the level of achievement of Davis
in devising methods of extracting the argon atoms and counting
them.No wonder it has been likened to finding a particular grain
of sand in the whole of the Sahara desert.

The detection threshold in Davis's experiment was 0.8 MeV and
thus only the high-energy Boron-8 neutrinos were detected.SSM
could be used to get the number of neutrinos expected above
this threshold and the detected number was less than the
predicted number by a factor of about 3. Over the three decades
of operation of Davis's experiment,this discrepancy has
remained and has been known as the solar neutrino puzzle.

Davis's radiochemical experiment was a passive experiment.There
was actually no proof that he detected any solar neutrinos.In
particular if a critic claimed that all the radioactive atoms
that he detected were produced by some background radiation,
there was no way of conclusively refuting it. That became
possible through the Kamioka experiment that went into operation
in the 80's.

In contrast to Davis's chlorine tank,the Kamioka water Cerenkov
detector is a real time detector.Solar neutrino kicks out an
electron in the water molecule (elastic scattering) and the
electron is detected through the Cerenkov radiation it emits.
Since the electron is mostly kicked toward the forward direction,
the detector is directional.A plot of the number of events against
the angle between the electron track and Sun's direction gives
an unmistakable peak at zero angle,proving that neutrinos from
the Sun were being detected.The original Kamioka detector had
2 KiloTons of water and the Cerenkov light was collected by an
array of 1000 photomultiplier tubes,each 20" diameter and this
was later superceded by the SuperKamioka detector which had
50 KiloTons of water faced by 11,000 photomutiplier tubes.
Both Kamioka and SuperK gave convincing proof of the detection
of solar neutrinos.The energy threshold of these detectors
was about 7 MeV and so only the high-energy part of the Boron-8
spectrum was being detected.The ratio of the measured solar
neutrino flux to the predicted flux was about 0.5, thus
confirming the solar neutrino puzzle.

The next input came from the gallium experiments. The Boron-8
neutrino flux is very sensitive to the details of the SSM and
so SSM could be blamed for the detection of a lower flux.On the
other hand the low energy pp neutrinos are not so sensitive to
SSM.So the gallium detector based on the inverse beta decay of
Gallium-71 was constructed.Although this was also a passive
radiochemical detector,its threshold was 0.233 MeV and hence
it was sensitive to a large part of the pp flux extending upto
0.42 MeV.Actually two gallium detectors were mounted,called
SAGE and GALLEX and both succeeded in detecting the pp neutrinos
in addition to the B-8 neutrinos but again at a depleted level
by a factor of about 0.5.

To sum up, there were three classes of neutrino detectors with
different energy thresholds,all of which detected solar neutrinos,
but at a depleted rate.The ratio R of the measured flux to the
predicted flux was 0.33$\pm$0.028 in the chlorine experiment,
0.56$\pm$0.04 in the two gallium experiments (average) and
0.475$\pm$0.015 in the SuperK experiment.

Actually it must be regarded as a great achievement for both
theory and experiment that the observed flux was so close to the
theoretical one, especially considering the tremendous amount of
physics input that goes into the SSM.After all R does not differ
from unity by orders of magnitude!This is all the more significant
since the large uncertainties in some of the low energy thermonuclear
crosssections do lead to a large uncertainty in the SSM prediction.
But astrophysicists led by Bahcall are ambitious and claim
that the discrepancy is real and must be explained.Two points
favour this view.As already stated,the gallium experiments
sensitive to the pp flux which is comparatively free of the
uncertainties of SSM, also showed a depletion in the flux.Second,
SSM has been found to be very successful in accounting for many
other observed features of the Sun, in particular the 
helioseismological data i.e data on solar quakes.

Hence something else is the reason for R being less than unity
and that is neutrino oscillation.

In addition to the well-known electron,two heavier types of
electrons are known to exist.Reserving the name electron to
the well-known particle of mass 0.5 MeV,the heavier ones are
called muon and tauon and their masses are 105 and 1777 MeV
respectively.Correspondingly there are three types or flavours
of neutrinos called e, mu or tau neutrino that go respectively with the
electron,muon or tauon in the beta decay as well as
inverse beta decay interactions (See Box 5).

What is produced in the thermonuclear reactions in the Sun
is the e neutrino.If some of the e neutrinos oscillate
to the mu or the tau neutrinos on the way to the earth,
the depletion in the number detected on the earth can be
explained since the chlorine and gallium detectors cannot
detect the mu or tau neutrinos.Just as the e neutrino
produces an electron in the inverse beta decay process,
the mu or tau neutrino has to produce a muon or a tauon
respectively in the final state (See Box 6). But since the energy of the
solar neutrinos are limited to 14 MeV,the muon or tauon with
the high masses of 105 and 1777 MeV cannot be produced in
the inverse beta decay and so the neutrinos that have been
converted into the mu or tau flavour through oscillation
escape detection.

Although elastic scattering of neutrinos on electron which
is used as the detecting mechanism in the Kamioka and SuperK
water Cerenkov detectors can detect the converted mu or tau
flavours also, it has a much reduced efficiency. Hence the
depletion of the number of neutrinos observed in the water
detector also is attributable to oscillation.

There was a famous painting called "The Cow and Grass".But
nothing except a blank convass was visible.When asked to
show the grass, the painter said the cow had eaten the grass.
When pressed to show at least the cow,he said it went away
after eating the grass.

Our neutrino story so far is like that.We said thermonuclear reactions
in the Sun must produce so many neutrinos.We did not see so
many neutrinos, but then explained them away through oscillations.

In Science we have to do something better.If we say that neutrinos have
oscillated into some other flavour, we have to see the
neutrinos of those flavours too.

This is precisely what is done in a two-in-one experiment (Box 7).
There are two kinds of weak interaction processes.Beta decay
in which a nucleus decays into another nucleus emitting a
neutrino along with an electron as well as the related
inverse beta decay in which a neutral neutrino colliding
with a nucleus leads to a charged lepton and a different
final nucleus are both charged current (CC) weak interaction
processes.(Here charged lepton means electron,muon or tauon.)
There is a second class of weak interaction known as neutral
current (NC) weak interaction in which the neutrino colliding
with the nucleus excites or disintegrates it but remains as
the neutrino in the final state.A low energy mu or tau
flavoured neutrino will not cause the CC interaction in
the nucleus as we already stated, but it can cause NC interaction.
So if we design an experiment in which both the CC and NC modes
are detected, although the CC mode will give only the number
of e neutrinos, the NC mode will give the total number of e, mu and
tau neutrinos. The total number detected will be a test of
SSM independant of oscillations while the NC minus CC events
will give the number that had oscillated away.

A huge two-in-one detector (BOREX) was proposed by Pakvasa and
Raghavan but that has not materialized.The two-in-one detector
based on deuteron in heavy water proposed by Chen has come up.
This is the Sudbury Neutrino Observatory (SNO) that has finally
solved the solar neutrino problem.

SNO uses 1000 tons of heavy water.Solar neutrino breaks up the
deuteron by CC and NC modes.While CC mode leads to two protons
and an electron, NC mode leads to a neutron, a proton and a neutrino.
The threshold of detection was again high like SuperK so that 
only the B-8 neutrinos were detected.
Let us now straightaway go to the exciting results
of SNO that came out in April 2002.

The CC mode gave the flux (million neutrinos per sq cm per sec)
as 1.76$\pm$0.11 while the NC gave 5.09$\pm$0.65 in the same units.
Thus we conclude that the flux of e + mu + tau neutrinos is
5.09$\pm$0.65 while that of the e flavour alone is 1.76$\pm$0.11.
The difference 3.33$\pm$0.66 is the flux of the mu + tau flavours.
Hence oscillation is confirmed.Roughly two third of the e
neutrinos have oscillated to the other flavours.Further,
comparing with the SSM prediction of 5.05$\pm$0.40, SSM also
is confirmed. So at one sweep the SNO results
confirmed both the SSM and neutrino oscillation.

What is the moral of the story? When we said in the beginning
that the thermonuclear hypothesis for the Sun has to be proved,
it was not a question of proof before a court of law. Science
does not progress that way. In trying to prove the hypothesis
experimentally through the detection of solar neutrinos, Davis
and the other physicists have helped in making a discovery of
fundamental importance, namely that the neutrinos oscillate
and hence have mass.

\section{NEUTRINO OSCILLATION}

To understand neutrino oscillation, one must think
of neutrino as a wave rather than than a particle
(remember quantum mechanics).Neutrino oscillation
is a simple consequence of its wave property.Let
us consider the analogy with light wave.Consider a
light wave travelling in the z-direction.Its 
polarization could be in the x-direction,y-direction
or any direction in the x-y plane.This is the case
of plane-polarized wave.However the wave could have
circular polarization too,either left or right.
Circular polarization can be composed as a linear
superposition of the two plane polarizations in the
x and y directions.Similarly plane polarization can
be regarded as a superposition of the left and right
circular polarizations.

Now consider plane polarized wave travelling through an optical medium.
During propagation through the medium,it is important to resolve the
plane polarized light into its circularly polarized components
since it is the circularly polarized wave that has well-defined
propagation characteristics such as the refractive index or velocity of
propagation.In fact in an optical medium waves with the left
and right circular polarizations travel with different velocities.
And so when light emerges from the medium, the left and right
circular polarizations have a phase difference proportional
to the distance travelled.If we recombine the circular
components to form plane polarized light,we will find the
plane of polarization to have rotated from its initial
orientation.Or,if we start with a polarization in the
x-direction,a component in the y-direction would be
generated at the end of propagation through the optical medium.

For the neutrino wave, the analogues of the two planes of
polarizations of the light wave are the three flavours (e,
mu or tau) of the neutrino (See Box 8).When the neutrinos are produced
in the thermonuclear reactions in the solar core, they are
produced as the e type.When the neutrino wave propagates,
it has to be resolved into the analogues of circular polarization
which are energy eigenstates or mass eigenstates of the neutrino.
These states have well-defined propagation characteristics with
well-defined frequencies (remember frequency is the same as energy
divided by Planck's constant).The e type of neutrino wave will
propagate as a superposition of three mass eigenstates which
pick up different phases as they travel.At the detector, we
recombine these waves to form the flavour states.Because of
the phase differences introduced during propagation, the
recombined wave will have rotated "in flavour space". In general,
it will have a mu component and tau component in addition to
the e component it started with.This is what is called neutrino
oscillation or neutrino flavour conversion through oscillation.

Flavour conversion is directly due to the phase difference
arising from the frequency difference or energy difference
which in turn is due to the mass difference.Mass difference
cannnot come without mass.Hence discovery of flavour conversion
through neutrino oscillation amounts to the discovery of
neutrino mass. This is the fundamental importance of neutrino
oscillation, since sofar neutrinos were thought to be massless
particles like photons.

Since it is an oscillatory phenomenon,the probability of
flavour conversion is given by oscillatory functions of the
distance travelled by the neutrino wave, the characteristic
"oscillation length" being proportional to the average energy
of the neutrino and inversely proportional to the difference
of squares of masses.Further,\break the overall probability for 
conversion is controlled by the mixing coefficients that
occur in the superposition of the mass eigenstates to form
the flavour states and vice versa.These mixing coefficients
form a 3x3 unitary matrix.

Neutrino oscillations during neutrino propagation in matter
become much more complex and richer in physics, but we shall
not go into the details here. However it is important to mention
two things. Hans Bethe redeemed himself for his earlier omission
of neutrinos in his famous paper on the energy production in stars.
This redemption came in the following way. After Wolfenstein
pointed out the important effect of matter on the propagating
neutrino and Mikheyev and Smirnov drew attention to the dramatic
effect on neutrino oscillation when the neutrino passes through
matter of varying  density, it was Bethe who gave an elegant
explanation of the MSW (Mikheyev-Smirnov-Wolfenstein) effect
based on quantum mechanical level-crossing. In fact most people
(including the present author) appreciated the beauty of MSW effect
only after Bethe's paper came out.

The second important thing about the matter effect is the possibility of neutrino
tomography. Neutrinos are the most penetrating radiation known to us. A typical
neutrino can travel through a million earth diameters without getting stopped.
However because of the MSW effect the neutrino senses the density profile
of the matter through which it travels and so the flavour composition
of the final neutrino beam can be decoded to give information about the
matter through which it has travelled. Hence tomography of the Earth's interior
through neutrinos will be possible. Of course this requires 
our mastery of neutrino technology.
But neutrino technology will be mastered and neutrino tomography will come.

\section{ATMOSPHERIC NEUTRINOS}

Solar neutrinos are MeV neutrinos.We now shift to GeV neutrinos.
Cosmic rays, which are mostly protons, collide on the nitrogen
and oxygen nuclei of the earth's atmosphere and produce a large
number of pions which ultimately decay into neutrinos and
electrons.These are called atmospheric neutrinos.A pioneering
experiment was done in India more than 35 years ago.This was
the underground cosmic ray experiment in the Kolar Gold Field
(KGF) mine which is one of the deepest mines in the world.
When the experiment was done at deeper and deeper levels,the
cosmic ray detector became silent at a certain depth.It was
realized that at that depth and beyond,the other cosmic ray
produced particles such as muons were completely shielded by
the overlying rock ( a few km thick) and hence one reaches
the capability of detecting the cosmic ray produced neutrinos
at those depths.The experimenters went ahead and did detect
the neutrinos.That was in 1965.

Detailed studies of these atmospheric neutrinos were undertaken
in many underground laboratories around the world in the
succeeding decades.A well-known fact of weak interaction physics
is that a pion mostly decays into a muon and a mu type neutrino.
Subsequently the muon decays into an electron and two neutrinos
,an e type and a mu type. So in the final debri of neutrinos
detected deep underground, for every e neutrino there must be
two mu neutrinos.In other words, the ratio R of mu type to
e type must be 2.One could distinguish between the two types of
neutrinos by observing either the electron or the muon that
will be respectively emitted when the e neutrino or the mu neutrino
has a CC interaction in the detector. Since the atmospheric neutrinos
have energies in the GeV range, both the e-type and the mu-type
of neutrinos can induce the CC reactions, in contrast to the
solar neutrinos.

The Kamioka water Cerenkov detector in Japan could distinguish
between the electron and the muon and thus measure the ratio R. It was
found that the ratio was in fact 2 for the neutrinos coming in
the downward direction,but it deviated considerably from 2 and
was about 1 for those neutrinos coming in the upward direction
which have obviously travelled through the 13,000 Km of the
earth diameter.Although Kamioka detector and a few other
detectors saw this anomaly  in 1990,it required the SuperKamioka
detector with its superior statistics to establish the effect
in 1998.

The explanation of this anomaly is again neutrino oscillation.
Since the anomaly was in the ratio of the fluxes of two types
of neutrinos, unlike the solar neutrino problem before the advent
of SNO, the inference of neutrino oscillation from the atmospheric
neutrino anomaly was relatively free of the large uncertainties
of the absolute flux.Hence in the discovery of neutrino mass through
oscillation, SuperK and the atmospheric neutrino experiment won
the race.


\section{REACTOR NEUTRINOS}

A fission reactor is a copious source of neutrinos (actually
e type antineutrinos).The very first experimental detection
of neutrino was in fact made with reactor neutrinos.Fermi's
theory of beta decay which was based on the existence of
neutrino was in such beautiful agreement with experimental
data on the beta decays of nuclei that hardly anybody doubted
the existence of neutrinos.Nevertheless Cowan and Reines
realised the importance of directly detecting the antineutrinos
produced in a fission reactor and succeeded doing it in 1954,
thus ushering the experimental study of neutrinos.They
used inverse beta decay for the detection.The antineutrino
is absorbed by a proton,giving a positron and a neutron, both
of which are detected by delayed coincidence.

A very important result on neutrinos was obtained in 1998 in a 
reactor neutrino experiment at Chooz in France.The reactor
was so powerful that the neutrino detector could be placed
even 1 Km away.The detected flux agreed with the calculated 
flux within about 2 percent, thus showing that there was
no oscillation upto 1 Km.Although this was a null result,
this played a crucial role in the global analysis of neutrino
oscillations.

Even more recently, antineutrinos from a dozen power reactors
in Japan were detected in a scintillation detector called
KamLAND. Although the reactors were at various distances
from the detector, they were all at about 180 Km and at
such a distance the antineutrinos must oscillate and this
has been confirmed beautifully.

\section{OTHER MATTERS IN BRIEF}

A combined analysis of the solar,atmospheric and reactor neutrino
data has led to a rough determination of the two differences of
neutrino mass squares and the 3x3 mixing matrix, although there
are still many uncertainties to be resolved. The mass-square
differences are found to be very tiny: 0.002 and 0.00007 in units
of electron-Volt (eV) squared.

Neutrinos from supernovae which are exploding stars is an important
topic we have not discussed. In fact Bethe made fundamental
contributions to the theory of supernova explosion and continued
to work on it almost to the last day. Neutrinos in cosmology is
another fascinating subject we have not touched.

Neutrino Physics has only started. There are still many questions
to be answered. Although neutrinos are now known to be massive
from the existence of neutrino oscillations, we do not know the
values of the masses, since only differences in neutrino mass-
squares can be determined from the oscillation phenomena. 
However nuclear beta decay experiments can give the absolute
masses, although sofar they have led only to an upper limit to the
neutrino mass and that limit is 2.2 eV.
Even the fundamental nature of the
neutrino is still not known, namely whether neutrino is its
own antiparticle or not. This question can be answered only
by the "neutrinoless double beta decay experiment". 

There are many plans to start new neutrino laboratories all over
the world. India was a pioneer in neutrino experiments. As we
already pointed out, atmospheric neutrinos were first detected
in the deep mines of KGF in 1965. It is planned to revive 
underground neutrino experiments in India. A multi-institutional
Neutrino Collaboration has been formed with the objective of
creating the India-based Neutrino Observatory (INO). More
information on INO is available at the website http://www.imsc.res.in/~ino
