\chapter{Neutrinos and Life}\label{chap17}

\Authorline{G Rajasekaran}
\addtocontents{toc}{\protect\contentsline{section}{{\sl G Rajasekaran}\smallskip}{}}
\authinfo{Institute of Mathematical Sciences, Chennai\\ 
        and Chennai Mathematical Institute, Siruseri}
\lhead[\small\thepage]{\small\leftmark}

                 
                 
When we met Dr Abdul Kalam on 11 January 2010 in connection
with INO (See the article: Dr Abdul Kalam and INO), he raised 
an interesting question. It was " Is there a connection between
neutrinos and the origin of life? " He raised it many times but
we could not answer it. He asked us to think about it.

After I returned to Chennai this question was bothering my mind.
After a few days I found the answer. I am giving below the 
answer that I sent to Dr Kalam.

As we know, it is the Sun that is giving us light and heat.
Without it, life on Earth is impossible. How does Sun produce 
its energy and continue to shine for billions of
years? The answer was found by Hans Bethe, the famous
nuclear physicist in 1939. It is the thermonuclear fusion
reaction. Through a chain of reactions, four protons
(four Hydrogen nuclei) join together to form a Helium
nucleus emitting two positrons and two neutrinos and
releasing 27 MeV of energy. This can be regarded as the
most important reaction for all life, for, without it
there can be no life on Earth!

For a long time, the above mechanism for producing energy
which the Sun and all the other stars are using remained
a theory although it was believed to be correct and
Bethe himself received the Nobel Prize for it.

The only way of establishing or proving Bethe's theory 
as correct is to detect or observe the neutrinos coming
from the above reaction. For every 27 MeV of energy that
we receive from the Sun as heat or light, we must receive
two neutrinos. Forty years ago, Ray Davis and his team
in USA started the pioneering solar neutrino
experiments with the aim of detecting the neutrinos from
the Sun. The experiments continued for for more than three
decades! These as well as a few other neutrino experiments
done in USA, Europe and Canada
detected the neutrinos from the Sun and conclusively proved
Bethe's theory. In the process, they discovered something
new also, namely neutrinos have mass. This was an important
discovery.

Davis received the Nobel Prize in 2002 and McDonald the leader
of the Canadian experiment is receiving it this year. It is 
these experiments that have given a great impetus for world-wide
interest in pursuing further neutrino experiments including 
our own INO.

Thus one may conclude that neutrino research gave proof for
the life-producing and life-sustaining thermonuclear fusion
reactions taking place in Sun and other stars.

There is one more connection of these processes with life. The very
materials that make up all living bodies are also made in
such thermonuclear fusion reactions only. All of living matter
is made of heavier nuclei such as Carbon, Oxygen... in
addition to Hydrogen nuclei. It is inside the stars that these
heavier nuclei are made from the lighter ones. The first step
is the fusion of hydrogen into helium that we described
earlier. After that, the heavier nuclei such as Carbon, Oxygen 
.... Iron are successively cooked in the interior of the star.
The process stops at Iron. 

The star cannot produce any more energy 
after producing Iron and hence collapses,throwing out
most of the stellar matter in a supernova explosion. This
matter is dispersed throughout the Universe. And life starts
from this stellar matter which has the required mix of
heavier nuclei. Actually most of the energy of the exploding
star is released in the form of neutrinos which also were
detected in 1987. But that is another story.

It is Dr Kalam's persistent questioning which induced
me to think of the deep connection between Neutrinos and
Life. All the parts of the Universe are deeply connected.
The mysterious secrets of the Universe lie hidden even
in a tiny grain of sand. We can dig them out through 
science and reflection.

\begin{quote} 
        To see a World in a grain of sand,
        And a Heaven in a wild flower,
        Hold Infinity in the palm of your hand,
        And Eternity in an hour.
                             
                                William Blake
\end{quote} 
