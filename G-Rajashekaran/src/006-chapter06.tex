\chapter[Murray Gell-Mann and the Story of Strong Interaction]{Murray Gell-Mann and the Story of Strong Interactions}\label{chap6}


\Authorline{G.Rajasekaran}
 \addtocontents{toc}{\protect\contentsline{section}{{\sl G.Rajasekarane}\smallskip}{}}

 \authinfo{}

Murray Gell-Mann who passed away on 24 May 2019 was a giant of modern
particle physics. His most well-known contribution is the proposal that
protons and neutrons are not indivisible but are made up of smaller
particles called quarks.

However there is another story - that of the strong interactions - that
is less popularly known but is intimately connected with Gell-Mann's
life.

After Ernest Rutherford discovered the proton in 1917 and James Chadwick
discovered the neutron in 1932, physicists began to recognise the existence
of a new force of nature, called the strong force, that binds the protons
and neutrons to form the atomic nucleus.

Sometime after the discovery of the neutron, the German physicist Werner
Heisenberg introduced a concept called isospin symmetry to make sense of
the fact they are so similar in many respects. This symmetry is now called
SU(2) symmetry because it concerns two objects, the proton and the neutron.

In 1935, the Japanese physicist Hideki Yukawa propounded a now-famous theory
that the protons and neutrons were bound together because they exchanged
particles later called mesons. Further studies with cosmic rays confirmed
the existence of these particles. They are pi mesons or pions.

Infact, physicists began to find a variety of particles in Cosmic Rays,
such as the Lambda, Sigma and the K meson. These particles had a strange
behavior. Working independantly, Kazuhiko Nishijima and Gellp-Mann
found, in 1953 and 1956 respectively that their strange behavior could
be explained if they had a new attribute or quantum number called
'Strangeness'. This explanation was enscapulated in the Gell-Mann - Nishijima
formula and became an important part of particle physics.

In 1956, Shoichi Sakata and his group in Japan propounded a new idea, that
the proton, neutron and Lambda were fundamental particles and all other
composite particles were made of them. This idea became known as SU(3)
symmetry, the symmetry of three particles proton, neutron and Lambda.

It was at this juncture, in 1961, that Gell-Mann's most influential
contribution to the study of fundamental properties of matter arrived.
Gell-Mann noticed that apart from proton, neutron and Lambda, there
existed five other particles similar to them. They are Sigma${^+}$, Sigma$^{-}$,
Sigma${^0}$, Xi${^0}$ and Xi${^-}$. Gell-Mann and Yuval Neeman independantly proposed
that these eight particles together formed a family under SU(3). Gell-Mann
called it the Eightfold Way (taking the name from Buddhist scriptures).
Eightfold Way became highly successful.

Using the Eightfold Way Gell-Mann predicted the existence of another particle
Omega${^-}$. It was later discovered in 1964 and that cemented the Eightfold Way as
correct.

However there was still one problem. Even though proton, neutron and Lambda
was not a triplet, SU(3) needed a triplet. Without a triplet, SU(3) cannot
be even defined.


At this point I would like to describe what I call the 'Bangalore event'.
In August 1961, the first summer school in theoretical physics of the
Tata Institute of Fundamental Research (TIFR) was held at the Indian
Institute of Science, Bangalore. The lecturers were Gell-Mann and Richard
Dalitz and the audience included Homi Bhabha, M G K Menon, Yash Pal and
other physicists and graduate students.

Gell-Mann lectured on SU(3) symmetry and Eightfold Way, fresh from the
anvil, even before they were published!

During one of these lectures, Dalitz questioned Gell-Mann about the triplets.
Why was he ignoring them? Gell-Mann managed to evade a direct response inspite
of Dalitz's repeated quostioning. If Gell-Mann had answered the question, he
might have predicted the existence of a new kind of particle called quark,
in Bangalore in 1961 instead of somewhere else three years later.

\newpage

Indeed if any of the other Indian participants had succeeded in answering
Dalitz's question, we would have got the quarks and this would have been
a major Indian contribution. It was a missed opportunity.

Gell-Mann and Zweig independantly proposed the idea of quarks in 1964,
but it would take many more years before they were confirmed to exist.
Gell-Mann himself was rather tentative: the title of his published paper
was "A schematic model of hadrons". (Hadron is a name for any particle
having strong interaction, like proton or neutron). To him, quarks were
only mathematical, an on-paper technique to break down a ccomplicated
problem.

Three types of quarks u, d and s, called up, down and strange replace
Sakata's proton, neutron and Lambda. Proton and neutron are made of
three quarks each. Pions and K mesons are composed of a quark and
antiquark.

The principal reason Gell-Mann did not accept the reality of quarks
as the constituents of hadrons was the prevalent S-Matrix philosophy.
The idea that some strongly interacting paricles like quarks were
'more' elementary than others was repugnant to the whole scheme of
S-Matrix theory. In fact, G F Chew, the chief proponent of the S-Matrix
theory, delivered a lecture at TIFR, Mumbai in the late 1960's
where he claimed he could prove that because of relativity and quantum
mechanics, no particle more elementary than proton or neutron was
possible.

But in spite of this quark-phobia, a few bold souls took the idea
seriously and worked out the consequences. They included A N Mitra,
G Morpurgo and Dalitz.

Dalitz in particular - perhaps as a response to the Bangalore event -
began work on the quark idea. He surprised everyone at the Rochester
Conference in 1966 when he claimed all the properties of the hadrons came
out correctly if one assumed they were made of quarks.
Nonetheless, although these calculations provided compelling evidence
for their correctness, there was no clinching evidence for quarks yet.

In the meantime, Gell-Mann was making steady progress on his programme to
exploit the mathematical quarks to glean more infomation about strong
interactions, through what he called current algebra.

Quarks eventually became 'real' after two important experimental discoveries.
The first was when, in the late 60's, high-energy electrons were used
at the Stanford Linear Accelerator Center to probe the internal
structure of protons. We will say more about it later.

The second was the discovery of the psi particle which had an extraordinarily
long life time of decay. It was as if anthropologists stumbled upon a
group of humans living for 10,000 years in some remote corner of the world!
This puzzle was solved by the invention of a new quark c with a new quantum
number 'charm'.

Most sceptics started believing in quarks after these developments.

The success of the quark model has a parellel in the history of atoms.
Sceptics like Ernest Mach, Wilhelm Ostwald and others did not believe
in the reality of atoms, while Ludwig Boltzmann waged a heroic battle
against their conservative notions. He emerged triumphant after Jean
Baptise Perrin experimentally verified Einstein's formula for Brownian
motion, which was predicated on the idea that atoms and molecules exist.

How were the quarks made visible? In the SLAC experiments very high-energy
electrons were shot at protons and some of the electrons rebounded since
they strick point particles inside the proton. These point particles were
identified as quarks.

This Stanford experiment is very similar to Rutherford's experiment in 1911.
He shot alpha particles on gold atoms and the alpha particles rebounded
since they struck the point nuclei inside the atoms. That was the discovery
of the atomic nucleus.

Further the Stanford experiments showed that as seen by the high-energy electron,
quarks inside the proton behaved like non-interacting free particles. This
was called asymptotic freedom and soon the theory which had asymptotic freedom
was identified as the theory of gluons based on a new quantum number called 'colour'.
Gell-Mann along with Harald Fritz and Heinrich Leutwyler proposed this as the theory
of strong interactions. Quarks interact by exchanging gluons just as electrons
interact by exchanging photons. This is a generalisation of Electrodynamics
and is called 'Chromodynamics'. Gell-Mann had a knack for coining new words:
strangeness,quark,colour and chromodynamics were all coined by him.

Including charmed quark, there were four quarks: u, d, s and c. Two
more quarks were discovered later, t and b, t being discovered only in 1995.
So we have six quarks in total and their antiparticles.

The physics of strong interactions was complete... right? Not exactly. There
was another problem. If you smash an atom with high-energy particles, elecrons
pop out. If you smash a nucleus, protons and neutrons pop out. But if you smash
a proton, quarks do't pop out. This is because quarks are said to be permanently
confined within a volume of space the size of the proton. Physicists believe
confinement is a consequence of chromodynamics, but this is yet to be proven.
Many of the world's brightest minds snd biggest computers are working on it but
the answer remains out of reach.

Gell-Mann was awarded the Nobel Prize for Physics in 1969. From 1955 to 1999 he
was working at the California Institute of Technology. In his later years he
devoted himself to studies of complexity theory at the Santa Fe Institute which
he had co-founded with other scientists in 1984.
