\chapter{Linking Weak Interactions with Electrodynamics}

\Authorline{G. Rajasekaran}
\addtocontents{toc}{\protect\contentsline{section}{{\sl }\smallskip}{}}
\authinfo{}

{\bf The force that causes the decay of a neutron inot a proton and the electromagnetic interaction between and electron a proton are traced to a common origin.}

Recent developments in high energy physics have led to a breakthorugh in our understanding of one of the fundamental force of nature, namely, the weak forces. It is this force which causes radioactie nuclei to disintegrate thorugh beta-decay. There is every indication now that this mysterious weak forec is but another facet of the well-known electromagnetic force that, for instance, causes a magnetic needle to move when an electric current is passed through a wire in the vicinity. In other words, high energy physicists have just glimpsed a grand unity of Nature connecting beta-decay with electrodynamics operating at a deep submicroscopic level.

Until now, it has been coustomary for any account of subatomic physics to start with the statement that there are four fundamental forces of nature gravitational, electromagnetic. strong and weak. This text-book classification (see box below) is breaking down as a consequence of recent developments.

The story of weak interactions starts with Henri Becquerel's discovery of radioactivity in 1896 and its subsequent classification into alpha, beta and gamma raditations. But the real understanding of beta-decay in the sense we it now came only after Enrico Fermi invented a physical mechanism for the decay process in 1934.

The basic ingredient for Fermi's theory had been provided earlier by Wolfgang Pauli. To Solve the puzzle of the continuous energy spectrum of the electrons emitted in the beta-decay of nuclei, Pauli had suggested that along with the electron, a massless neutral practicle also was emitted (see Table 1 on p.23). Fermi succeeded in incorpaortating Pauli's suggestion, and thus was born the theory of weak interactions. Fermi also christened the massless neutral particle as {\it neutrino}.

Fermi's idea waw simple. He drew an analogy with electromagnetic inter-action which, at the quantum level, can be characterised as the emission of a photon by and electron (see bos on p. 18). Fermi pictured the weak interaction responsible for the beta-decay of the neutron in the same manner as the emission of an electron-neutrino pair, the neutron converting itself into a proton in the process (Fig. 1).

In electrodynamics, the electric current interacts, with the photon, whereas in Fermi's theory of weak interactions, the weak current of the proton-neutron pair interacts with the weak current of the electron-neutrino pair. We should also note that all these currents are {\it vector} quantities.

This theory of weak interactions proposed by Fermi more than 40 years ago purly in an intutive basis has stood the ground successfully except for one important amendment which came in 1956. This was the discovery of parity-violation in weak interactions by T. D. Lee, C. N. Yang, C. S Wu and others which amounted to the overthrow of the symmetry between left and right hadedness in weak interation (see box on p. 19). But Fermi interaction survived even this revolution in our basic concepts and the only modification in it was to replace the vector weak current of Fermi by an equal mixture of vector (V) and axial vector (A) currents. (Vectors and axial vectors behave differently when we go from left to right-handed co-ordinate systems and hence, the partiy violation.) This is the famopus V-A interaction discovered in 1957 by E. C. G Sudarshan and R. E Marshak.

During 1947-55, many new particles such as muons, pions, kaons and hyperons were discovered. All of them were found to decay by weak inter-actions. The field of weak interacitions thus got enriched with a multitude of phenomena, of which nuclear beta decay is just one. Weak interaction is indeed a universal property of all fundamental particles.

\section*{A form for the weak interaction}

Remarkably enough, all the weak phenomena, namely the weak decays of all these fundamental particles, could be incorporated in a straight forward generalisation of the original Fermi interaction. This is to write the interaction in the form $G_{F}$ J, J, where J and J are the total weak currents and $G_{F}$ is the Fermi coupling constant. Just as the sterngth of the electromagnetic force is measured by the electric charge of the lelectron, the strength of the weak force is measured by the Fermi coupling constant $G_{F}$ whose value is about $10^{-5}$ (expressed in units of inverse of proton-mass-squared). It is because of the smallness of this number that this force is called {\it weak} in contrast to the nuclear forces which are {\it strong}. These currents them selves can be symbolically expressed as :
$$
J = pn + ve + v' \mu + \ldots 
$$
$$
J = np + ev + \mu v + \ldots 
$$ 

\begin{center}
\begin{tabular}{|llll|}
\hline
\multicolumn{4}{|l|}{{\bf The four forces of Nature}}\\
\hline
Force & Major role in Nature &Stength of  & Range of \\
      &                      &the force   & the force \\     
\hline
Gravita- & Binds matter to form  & $10^{-19}$ & infinite\\
tional   & stars and galaxies    &           &          \\
Electro- & Binds electron and nucle & $\frac{1}{137}$ & infinite \\
magnetic & to form atoms and molecules. &              &  \\
                &Binds atoms and molecules to  &              &   \\
                &form all matter &      &   \\ 
                &(animate and inanimate)& & \\
Strong & Binds neutrons and protons & 1 & $10^{-13}$ cm \\
       &  to form nuclei            &   &               \\
Weak & Disrupts nuclei, neutrons  &$10^{-n}$  & almost \\
     & and other elementary particles &       & zero   \\
\hline
\end{tabular}
\end{center}


\section*{Electrodynamics, photons and diagrams}

Electrodynamics is well understood From the days of Faraday and Max\-well. They introduceds the idea of the elestromagnetic field pervading all of space for a consistent description of all electromagnetic phenomena. The structure built by Faraday and Maxwell has survived even the two major revolutions in our fundamental concepts that shook the world of physics in the twentieth century Relativity and Quantum Theory. The only change wat that the continuous electromagnetic field has to be replaced by quanta of energy, of {\it photons}.

Thus, the electromagnetic force between two charge particles, say, proton and electron, can be represented by the diagram:

figure??


Here, the wavy line denotes the quantum or photom which exchanged between the proton the electron. This is the modern quantised version of the classical picture. wherein the proton would be considered to produce and electromagnetic field which would then influence the electron placed in the field. Both these pictures elestromagnetism as a continuous field and as quantise packet of energy (photon)- anre necessary for a proper understanding of modern physics.

Quantum Theory leads to an inverse relationshio between the mass of the exchanged quantum and the range R of the force:
$$
R \sim \frac{1}{m}
$$

Since the mass m of the photon is zero, the force between two charge particles is of infinite range. This is just the well known fact that the Coulomb force between charge acts even at infinite separation.

One can split up the above diagram into two parts, one describing the emission of a photon by the proton and the other describing the absorption of the photon by the electron. Therefore, the basic electrodynamic interaction is in fact, the intercation of both electrons and protons with the photon and so it can be represented by the diagrams in Fig. 3.

In symbolic form, the interaction is written or the vector potential of the electromagnetic field and J denotes the electric current of the electrons and protons. The symbol e is the numerical value of the electric charge on the electron and characterises the strenght of the electromagnetic force.

We shall make a rather liberal use of pictorial represenatations of processes and interactions, after Feynman whose use of such diagrams in an intuitive interpretation of comples calculations was an important step in the elucidation of the fundamental processes of Nature. Especially important was the idea discovered by Feynman and Stuckelberg that antiparticles could be regarded as particles traveling backwards in time.

figure????

A diagrammatic representation of this is given in Fig 4. Positively charged negatively charged and neutral particles denoted by white, colou\-red and black line in the diagrams. The Current $J^{+}$ describes a neutron turning into a proton, an electron turning into a neutrino or a muon turning into a neutrino all these transitions result in an increase of electric charge. Hence, $J^{+}$ is called the charge-raising current. $J^{-}$, which describes the opposite transitions, is called the charge-lowering current.

However, one can see that Fermi's original form of the interaction describing the beta-decay of the neutron is just one term (pn) (ev) in the product $J^{+}, J^{-}$, The decay of the muon by the proton are secribed by the terms $(v/ \mu)$ $(e v)$ and $(v' \mu)$  $(n p)$. (Fig. 5). In the diagrams, antiparticle line is obtained by changing the directions of the arrow. The dots in the expression for $J^{+}$ and $J^{-}$ refer to other terms which are added in order to incorporate the weak decays of other particles.

This current $\times$ current form of the interaction as a description of all the weak decay phenomena was proposed by R. Feynman and M. Gell-Mann in 1957.

\section*{The elusive neutrino}

Pauli proposed the neutrion in the yeay 1930. Although the sucess of Fermi's theory based on neutrino emission in explaining experimental data could be taken as indirect veri- fieation of Pauli' prediction, a direct detection of the neutrino was achieved only in 1956 by F. Reines and C. L . Cowan.

The neutrino is unique among elementary particles. Every other particle is also subject to electromagnetic or nuclear force. In contrast, the neutrino in influenced by the weak force only.

The neutrino can, therefore, travel throught matter over distances of many Earth-diameters without a single interaction. Nothing can stop it. If only we can control it, a neutrino beam would be the ultimate in penetrability. However, for the time being, we shall leave that to science fiction.

For the same reason, the neutrino is extremely difficult to detect. Since the probability of interaction for a neutrino is vanishingly small, we need a very large supply of neutrinos before detection becomes practical.

An ideal place to took for neutrinos is thus a large reactor because of the intense neutrino flux from decays of fission fragments. These are actually antineutrinos (v) and Reines and Cowan succeeded in detecting these reactor-produced antineutrinos by observing the weak reaction.
$$
v + P \rightarrow  e^{+} + n
$$

Here $e^{+}$ denotes the positron which is teh antiparticle of the electron.

That was in 1956 subsequently, it became possible to detect the neutrinos coming out of the decays of pions and kaons which themselves are produced in high energy accelerators. This was rendered possible both because of the copious supply of neutrinos available through decays as well as because the probability of interaction of neutrionos increases with energy. As a result, the study of neutrino-interactions in accelerator experiments now competes successfully with other conventional experiments.

Further, evene neutrinos produced by cosmic rays have now been detected. The underground laboratory (of the Tata Institute of Fundamental Research) at the deep mine in the Kolar Gold Field was one of the first to detect cosmic-ray-produced neutrinos in 1965.

So much for history. We now go over to the recent developments.

\section*{W bosons}

We have already drawn attention to the analogy between weak interactions and electrodynamics which Fermi exploited in constructing his theory. One may attribute it to the intuition of Fermi's genius or two jut goog luck. Whatever it is, it turns out that the analogy with electrodymamics which Fermi banked upon not only yielded basically the correction form of weak interaction the vector form in contrast to the scalar to serve even now as a fruitful analogy in the search for a more complete theory of weak interactions.

In fact, the basic idea of the new theory of weak interaction is to push the analogy with electrodynamics as far as possible.

In beta-decay, Fermi had imagined the n-p line and the e -v line interaction at the same space-time point. But, clearly, the correspondence with electrodynamics is greatly enhanced if the two pairs of line are separated and an exchange of a quantum W between the n-p line and the e-v line is allowed for (Fig.8).

\section*{Violation of right-left symmetry}

In the right-handed co-ordinate system, the directions of the x,y and z axes are such that, if we imagine a screw (actually, a right-handed screw which is what we normally use) being rotated from x to y, the screw wll advace along x. The left-handed co-ordinate system is obtained by mirror-reflection. Can the laws of physics distinguish between these two co-ordinate systems? As far as we know, if we ignore the phenomena involving the weak force. all other laws of physics ar symmetric under mirror relection and, hence, cannot distinguish between the left and right co-ordinate systems.

The significance of this right-left symmetry as well as its violation can be appreciated better, if we think of the following. Suppose we want co commnunication, suppose we want to communicate with sombody in a distant galaxy through radio waves. Let us see how we will go about defining a right handed co-ordinate system for him. Screws will not help here since we do not know whether they use a right-handed screw or a left-handed screw in the galaxy! We can use any of the laws of physics for this purpose. If noen of the laws distinguishes between the two co-ordinate systems, we will never be able to convey a definitions of the right-handed co-ordinate system to a beign in a distant galaxy.

However, thanks to the weak force, this can be done. The following instruction can be conveyed: "Take Co" nuclei which undergo beta-decay and arrange a large number of electron to go  around these nuclei, thus forming a circulating electric current. If a rotatin a circulating electric current. If a rotation from the x-axis to the y-axis is in the direction of the circulating electrons, then the z-axis is the direction in which more beta-decay electrons are emitted. This would define the right-handed co-ordinate system for our friend in the distant galaxy. Thus , weak interaction allows us to define a right-handed co-ordinate system by using natural physical laws.

A word of caution, however. We have to make sure that the planet inhabited by our friend is made up of matter and not of antimatter. If it is mage up of antimatter he would really take nuclei of anti-co and positrons (antielectron) and would end up with a left-handed co-ordinate system by following our intructions!.

What are the properties of this new particle (a) W has to be charged, in contrast to the photon, as can be seen by conserving charge at the two vertices of the W exchange diagram in Fig. 8. The neutron turns into a proton by emitting a W ans so this W should be negatively charged. The anti-W should be positively charged. (b) just like the photon, the W particle is also spinning on its axis (with spin angular momentum of one $h/2\pi$ unit). Both photon and W are bosons. (c) In contrast to the photon, the W boson has to be a very massive object. For, in beta-decay, we know that  the interaction between the $n\sim p$ line and the e-v line acts almost at the same space-time point. In other words, the Fermi contact interaction is really in good agreement with expriments so far. If the mass of the W boson were zero, or even small, the weak force would act even at a long range (just as the electric force between two charges) which, would not agree with experiments facts on beta-decay. So, to preserve the agreement of Fermi contact interactions with expreiment, the mass $m_{w}$ of the W boson has to be very large.

In Fermi's theory, remeber the coupling constrat was $G_{F}$. In the W-boson theory, we have a coupling constant g at each vertex and so, for the same process, $G_{F}$ is replaced by a factor $g^{2}$ multiplie by the propagatin factor for the W boson. The propagation factor for the W boson is $1/m^{2}_{w}$ at the small moment relevent in beta-decay. Hence, the get agreement with Fermi's theory, we have the important relationship: $G_{F} \approx \frac{g^{2}}{m^{2}_{w}}$. 

By introducing the fields W and W for the positively and negatively charge W bosons, the current $\times$ current form of the interaction can be split into the form
$$
g(J^{+} W^{-} + J^{-} W^{+}).
$$

The meaning of this expression in the the weak currents $J^{+}$ and $J^{-}$ interact with the W-boson fields with the coupling constant g. This form of the weak interaction is very similar to the electrodynamic interaction written symbolically as 
$$
eJ^{\gamma} A
$$

It may be remarked that A is also the vector potential of electrodynamics. It actually stands for four quantities comprising the three components of the three-dimensional vector potential A and the electrostatic potential V. Similarly, each of our current s J statnds collectively for the three-dimensional vector current J and the corresponding charge desity $\rho$.

Fig. 9 shows that we have achieved a greater degree of symmetry between weak and electromagnetic interactions.

\section*{Current conservation and gauge invariance}

The next step of the argument is to realise that the symmetry between W-boson theory and electrodynamics noted above is only apparent and does not hold at a deeper level.

Conservation of electric charge is a cornerstone of electrodynamics. The total charge in a isolated system cna neither be icreased nor decreased, and remains constant. In fact, this is a guiding principle in constructing all diagrams of particle-interactions.  This constancy of the total charge is ensured by the continuity equation satisfied by the continuity equation satisfied by the electric charge desity $\rho$ and the electric current $J$:
$$
\frac{\delta \rho}{\delta t} + {\rm div}  {\rm J} = 0
$$

Does such an equation hold for the weak currents $J^{-1}$ and $J^{-} ?$ The answer turns out to the be negative. In other words, if we define a {\it weak charge} for each particle on the analogy of the electric charge, this weak charge is not conserved in the W-boson theory.

A reltatd question concerns gauge incariance. Electrodynamics is gaguge incariant, which simply means that different potentials leas to the same physical situation as long as the electric field E and the magnetic field B are the same.

An elementary example of gauge invariance is provided by the well known fact that if electric potential {\it everywhere} is changed by a constant amount, the electric field is unaffected and so nothing of consequence will be changed (Fig. 10). Electrodynamics is so formulatex as to be invariant under such gauge transformation.

Is such a property valid for the W-boson fields? Again, the answer is in the negagtive for the simple interaction formulated above.

Certain important structural modifications have to be made in the W-boson theory in order to achieve current theory in order to achieve current conservation and gauge invariance which leads to gaguge theory.

\section*{Gauge Theory}

The required basic theoretical structure has been known since the work of C. N. Yanga and R. Mills in 1954 in 'non-abelian' gauge fields. But many other ideas had to be evolved before theory could be tailored to meet the experimentals facts of weak interactions. the resulting theory has come to be known as the gauge theory of weak and electromagnetic interactions, or simply, gauge theory. A number of theoretical physicsts have made key contributions towards the construction of gauge theory-the names usually associated with the theory being S. Weinberg and A. Salam. Here we can attempt only a qualitative description of gauge theory.

Gauge theory generalises the concept of charge. The single electric charge of electrodynamics is replaced in the new theory by many types of charges or charge-like attributes. The current corresponding to each type of charge interacts with its own boson called gauge boson. But the essential point of gauge theory is that the twin requirements of generalised current conservation and generalised gauge invariance force one to combine weak and electromagnetic interactions dynamically into a single framework. They can no longer be regarded as distinct interactions with separate coupling constants. Further, as a consequence of this dynamical mixing between weak and electromagnetic interactions, as new interaction with a neutral massive boson Z is also generated. (More on this neutral current interaction later). neutral massive boson Z is also generated. (More on this neutral current interaction later). 

The combined interaction in gauge theory can, therefore, be expressed schematically in the following form.
$$
g \left\{J \gamma A + J^{+} W^{-} + J^{-} W^{+} + J^{n} Z \right\}
$$

There are four "charges" whose currents interact with the four gauge bosons. Thus, the new theory introduces a symmetry between the photon and the massive bosons of weak interactions. Stated differently, photon is just a member of a large family of bosons comprising photon, $W^{+}$, $W^{-}$ and $Z$.

This generalisation of the concept of charge leads to self-interactions among the gauge to self-interactions among the gauge bosons, which are pictured in Fig. 11. This is a new feature not present in electrodynamics. The photon interacts with every electrically charged object. But the photon itself being uncharged, does not interact with ifself. The bosons of the generalised theory interact with every thing that carries a generalised charge Since the bosons themselves carry these charges, they have to interact with themselves.

In this respect, the new theory is nearer to Einstein's theory of gravitation. The gravitational field interacts with everything having mass or energy. Since the gravitational field itself has energy, it has to interact with itself. So, in Einstein's theory, gravitational field has self-interaction.

An important ingredient in the construction of gauge theory is the concept of {\it phase transition} from the symmetric world, in which the bosons of weak interactions together with photon are all massless, to the asysmmetric world of massive W bosons and massless photon. It is this phase trasition which is believed to be reponsible for the observed dis-parties between the short-ranged weak force and the long-ranged weak force. This phase transition in the submicroscopic world is skon to the well-known  phase transition between a solid and a liquid or between a liquid and a gas. Thus, even an apparently down-to-earth phenomenon like the boiling of water has something important to teach us in our study of the fundamental force of nature.

So, we have completed a full circle. We started with Fermi who made his theory of beta-decay by copying electrodynamics. We tried to make that copying more and more perfect. We end up by unifying electrodynamics and beta-decay into the same framework. The myriad electrodynamic and weak decay phenomena are manifestations of just one fundamental {\it electroweak} force (Fig. 12).

Unification of weak interaction with electrodynamics gives us a bonus,  It has been known for a long time that Fermi's original form of the interaction can only be considered an effective potential to be used in a lowest order approximation in the calculation of decay probalilities. Any attempt to improve the approximation leads to infinite interaction which does not make any sense.

Contruction of a dynamical theory of weak interaction free from this defect has been one of fundamental problems in high energy physics. Gauge theory solves this problem. Stated in technical language, gauge theory has been found to be {\it renormalisable} in the same sens as {\it Quantum Electrodynamics} is. It is, in fact, the discovery of the renomalisability of gauge theory (by G. t' Hooft in 1971) that is responsible for its increased popularity among physicists. 

We shall now discuss some of the simple consequences of gauge theory.

\section*{Why is Weak interaction Weak?}

An immediate consequene of the dynamical connection between weak and electromagnetic interactions is that the two coupling constants are essentially the same, $g\approx e$. This allows us to calculate the W-boson mass from the known values of $G_{F}$ and e (Fig. 13), and the result comes out to be about 40 times the proton mass. The mass of the Z boson also is exprected to be the same.

This is only an order-of-magnitude calculation. Nevertheless, it is good enough to suggest that the W boson is a very massive particle-in fact, too massive to be produced in the highest energy accelerator existing in the world today.

An important lesson of this exercise is that the weak interaction,in fact, are as strong as the electromagnetic interactions. Only because the W boson is so massive that the weak energies. At high energies, namely, at energies high as compared to the rest mass-energy of the W boson $(E = m_{w}C^{2})$, weak interaction regains its full strength.

\section*{Neutral Current}

The unified gauge theory encompasses not only the known electromagnetic and weak interactions, but also a new type of weak interaction $j^{n} Z$. The current $J^{n}$ consists of terms $J^{n} \sim \bar{p} p + n n + \bar{v} v + \bar{e}  e$ to J + and J, the current $J^{n}$ decsribes transitions in which the electric charge does not change. It is called the neutral current although a more nonchanging current. The neutral current interacts with the neutral boson Z which is massive like the W boson.

The neutral current interacition would lead to neutrino scattering processes in which, after the scattering event, the neutrino emerges as a neutrino rather that getting converted into a charged particle (see fig. 14). Such neutral current processes have now been exprimentally detected. This discovery of neutral-current weak interaction was made at the European High Energy Laboratory (CERN) at Geneva in 1973.

This discovery has its own intrinsic importance, because it opens up a whole new class of weak interactions which had remained undetected in all the 80 years' history of weak interactions. From the point of view of gauge theory, this has an added significance, for the newtral-current interactions acts as something like a bridege between electrodynamics and the usual electric current, but involves a massive boson just like the usual charged current weak interaction. So, its discovery and the subsequent study which showed its properties to be of the type expected in gauge theory have helped to confirm that physicist are on the right track.

\section*{SLAC Experiment}

Another spectacular confirmation of gauge theory has come from experiments done last yeat (1978) at the Stanford Linear Accelerator Center (SLAC), USA. The neutral-current interaction would also lead to electron-proton scattering vai exchange of a Z boson indicated in Fig. 15. This process exists in addition to the usual  electromagnetic electron-proton scattering via the exchange of photon (Fig.2).

an important distinction between the electromagnetic and teh neutral current electron-proton scattering processes is that partity (that is, right leftsymmetry) is preserved in the electromagnetic process, but violated in the neutral-current process. The extent of parity violation in the neutral-current is related to the dynamical mixing between weak and electromagnetic interactions and hence is of fundamental significance.

Earlier experiments which looked for his parity violation by sending a laser beam through heavy atoms (bismuth) had not led to definitive results. But the new experiment at SLAC gives a much clearer indication of the subtle effect.

The principle of the experiment is simple (see Fig. 16). A beam of longitudinally polarised electrons of very high energy hit a liquid deuterium target and the inelastically scattered electrons were detected. The aim was to look for a differences in the rate of detection of the scattered electrons when the polarisation (that is, the spin direction) of the electron was reversed. If the experiment were to detect such a difference between the scattering of right-handedly spinning electron and the left-handedly spinning electron, it would mean parity violation.

The SLAC experiment in fact yields a positive result for parity violation in neutral current and teh amount of parity violation agrees with the predication of the particular version of gauge theory put forward by Weinberg and Salam.

\section*{Some Dissents}

In spite of the prevailing optimism for gauge theory generated by the successful outcome of the recent experiments on neutral currents, we have to inject certain amount of caution.

The Weinberg-Salam version of gauge does not quit succeed in unifying weak interactions and electrodynamics through a single coupling constant. Another free parameter enters the relationshio between the two coupling constants. (For simplicity, we had ignored this complication in our computation of the W-boson mass). It is hoped that in a true unification to be achieved by going beyond Weinberg and Salam, this parameter would be uniquely determined. Attempts in this direction generally require a large number of gauge boson, larger than the four of Weinberg-Salam.

As we have alerady seen, an important consequence of gauge theory is the existence of W and Z bosons of mass of the order of 40 times the proton mass. A new generation of high energy accelerators is needed to produce such massive particles. The major high energy experimental progreammes of the world are being geared towards this task. One may expect an intesive search for these massive gauge bosons to start within a few years. Until the gauge bosons are detected experimentally, gauge theory cannot be regarded as established.

We have already referred to the important role of the idea of a phase transition in the construction of gauge theory. A further class of bosons, called Higgs bosons, has been invoked to facilitate this phase transition in the submicroscopic world. How far does Nature agree with these imaginative constructions of theoretical physicists is yet to be seen.

Finally, to put the whole thing into proper perspective, let us come back to electrodynamics. The basic laws of electrodynamics (Table 2) were first formulated by Maxwell more than 100 years ago. In this, Maxwell was greatly influenced by the intutitive physical pictures which Faraday had already build up for the understanding of electromagnetic phenomena.

Important predictions followed, once a complete and consistent system of laws was found. For instance, Maxwll predicted light itself to be an electromagnetic phenomenon.  Earlier discoveries of Oersted, Ampere and Faraday had unified electricity and magnetism into a single science of electromagnetism. Maxwell then unified optics with electrodynamics.

\begin{center}
\begin{tabular}{llll}
\multicolumn{4}{l}{{\bf TABEL 1 : THE FOUR FORCES OF NATURE}}\\
\hline
1896 & \multicolumn{3}{l}{Discovery of radiactivity (Becquerel)} \\
1930 & \multicolumn{3}{l}{Invention of the neutrion (Pauli)} \\     
1934     & \multicolumn{3}{l}{Theory of beta-decay (Fermi)} \\
1956     & \multicolumn{3}{l}{Dicovery of parity-violation (Lee, Yang and Wu)} \\
1956     & \multicolumn{3}{l}{Detection of the neutrino (Cowan and Reines)} \\
1957     & \multicolumn{3}{l}{Discovery of V-A interaction (Sudarshan and Marshak)} \\
1957     & \multicolumn{3}{l}{Current $\times$ current formulation (Feynman and Gell-Mann)} \\
1967     & \multicolumn{3}{l}{Gauge theory (Weinberg and Salam)} \\
1968      & \multicolumn{3}{l}{Gauge theory (Weinberg and Salam)} \\
1973     & \multicolumn{3}{l}{Discovery of neutral current (55 physicists at CERN)} \\
1978     & \multicolumn{3}{l}{Discovery of parity violation in neutrall current } \\ 
         &  \multicolumn{3}{l}{(20 physicists at SLAC) } \\               
\hline
\end{tabular}
\end{center}

\begin{center}
\begin{tabular}{llll}
\multicolumn{4}{l}{{\bf TABEL 2 : LAWS OF ELECTRODYNAMICS}}\\
\hline
Div & $E = 4 \pi \rho$ & \multicolumn{2}{l}{(GAUSS'S LAW OF ELECTROSTATICS)}\\
Curl & $E + \frac{1}{c} \frac{\delta B}{\delta t} =0$ & \multicolumn{2}{l}{(FARADAY'S LAW OF INDUCTION)}\\
Div & $B = 0$ & \multicolumn{2}{l}{(ABSENCE OF ISOLATED )}\\
    &         & \multicolumn{2}{l}{(MAGNETIC POLES)}\\
Curl & $B-\frac{1}{c} \frac{\delta  E}{\delta t} = \frac{4}{c} J$ & \multicolumn{2}{l}{AMPERE-MAXWELL'S LAW}\\
\hline
\end{tabular}
\end{center}

 
\begin{center}
\begin{tabular}{llll}
\multicolumn{4}{l}{{\bf TABEL 3 : LAWS OF ELECTROWEAK DYNAMICS}}\\
\hline
Div & \multicolumn{3}{l}{$E^{i} + {\rm NONLINEAR TERMS = 4\pi \rho^{i}}$}\\
Curl & \multicolumn{3}{l}{$E^{i} + \frac{1}{c} \frac{\delta B^{i}}{\delta t} + {\rm NONLINEAR TERMS} = 0$ } \\
Div & \multicolumn{3}{l}{$B^{i} +  {\rm NONLINEAR TERMS} = 0$} \\
Curl & \multicolumn{3}{l}{$B^{i} - \frac{1}{c} \frac{\delta E^{i}}{\delta t} + {\rm NONLINEAR TERMS} = \frac{4\pi}{c} J^{i} $} \\
\hline
\end{tabular}
\end{center}

\section*{Unification of other forces}

In the history of modern physic, there have been many attemptes to unify the various forces of nature. The most heroic were those of Einstein and Weyl, who osught a unified theory of gravitaton and eletrodynamics. But these earlier attempts did not have the benefit of the new insight into the possibility of electroweak synthesis which physicists have now obtained.

The successful synthesis of electro dynamics and weak force prompts us to look for further unification with the other forces. There is good indication now that the strong force is also mediated by gauge bosons called gluons and, hence the chances of a grand synthesis of electroweak dynamics with the strong nuclear force appear bright.

These new developments have opened the way for the possible ultimate unification of all the forces with gravitation.

These fundamental discoveries opened up vast areas of technological development. Electromagnetic machinery and wireless communications have become part of everyday life.

All this is old stuff. Maxwell's equations have stood the test of time for these 100 years. Will they remain in isolated glory for ever?

The answer has to be in the negative, Progress in physics generally consists in the discovery of more general laws of wider applicability which reduce to the older laws in some restricted domain.

This is precisely what we are witnessing now. The successful development of gauge theory shows that there exist more general laws of electrodynamics. There exist more general fields $E'$ and $B'$ (the index i running over the set of gauge bosons) and these fields are generalisations of the usual electric and magnetic fields and describe not only electrodynamics but weak interactions also. Actually, $E^{i}$ and $B^{i}$ (for $i=1,2,3,4$ ) are related to the gauge-boson fields A. W, W and Z in the same way as the usual electric and magnetic fields are related to the electromaganetic potentials.

These electroweak fields $E'$ and $B'$ satisfy laws of motion which are similar to Maxwell's laws of electrodynamics (see Table 3). There are some curcial differences arising from the self-interactions (Fig. 11) already pointed out. The self-interactions lead to non-linearities in the equations. Further, the phase transition to the asymmetric world of massive W and Z bosons affects the physical interpretation of these equations. Never theless, the basic similarity to electrodyamics stands.

Thus, Maxwell's laws have been incorporated into a more general system of laws which unify electrodynamics and weak interactions. The implications of this grand unification are yet to be fully understood or realised. There is no doubt that the consequences of this unification will be as profound and as far0reaching as those of Faraday's unification of electricity and magnetism and of Maxwell's unification of electrodynamics and optics.

Dr. Rajasekaran (43) , who  took his Ph.D from the University of Chicago, USA, is Professor and Head of the Department of Theoretical Physics, University of Madras. Earlier, he was with the Tata Institute of Fundamental Research, Bombay. His areas of research include high energy physics and quantum field theory.























 


