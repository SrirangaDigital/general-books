\chapter[In Memory of Prof.\ G Ramachandran: A tribute]{STANDARD MODEL, HIGGS BOSON AND WHAT NEXT?}\label{chap5}

\Authorline{G. Rajasekaran}
\addtocontents{toc}{\protect\contentsline{section}{{\sl G. Rajasekaran}\smallskip}{}}

\begin{center}
Institute of Mathematical Sciences, Chennai 600113\\
and Chennai Mathematical Institute, Siruseri 603103.\\
e-mail: graj@imsc.res.in
\end{center}


Abstract : One hundred years of Fundamental Physics, starting with discover-
ies such as radioactivity and electron, have culminated in a theory which is called
the Standard Model of High Energy Physics. This theory is now known to be the
basis of almost ALL of known physics except gravity. We give an elementary
account of this theory in the context of the recently announced discovery of the
Higgs boson. We conclude with brief remarks on possible future directions that
this inward bound journey may take.


\section*{Hundred Years of Fundamental Physics}

The earlier part of the 20th Century was marked by two revolutions that rocked
the Foundations of Physics.

\begin{tabular}{|cc|}
\hline
1. Quantum Mechanics\& & 2. Relativity\\
\hline 
\end{tabular}

Quantum Mechanics became the basis for understanding Atoms, and then, cou-
pled with Special Relativity, Quantum Mechanics provided the framework for
understanding the Atomic Nucleus and what lies inside.

At the beginning of the twentieth century, the quest for the understanding of
the atom topped the agenda of fundamental physics. This quest successively led
to the unravelling of the atomic nucleus and then to the nucleon (the proton or the
neutron). Now we know that the nucleon itself is made of three quarks. This is the
level to which we have descended at the end of the twentieth century. The depth
(or the distance scale) probed thus far is $10^{-17}$ cm.

\begin{center}
{INWARD BOUND}
\end{center}
{\fontsize{8pt}{10pt}\selectfont{
\begin{tabular}{|llllllll|}
\hline
      %~ &               &        & INWARD BOUND  &          &               &                      &\\  
Atoms & $\rightarrow$ & Nucler & $\rightarrow$ & Nucleons & $\rightarrow$ & Quarks $\rightarrow$ & ?\\
$10^{-8}$ cm &  & $10^{-12}$ cm & & $10^{-13}$ cm & & $10^{-17}$ cm & \\
\hline
\end{tabular}}}

\vspace{.2cm}

This inward bound path of discovery unraveling the mysteries of matter and the
forces holding it together – at deeper and ever deeper levels – has culminated,
at the end of the 20th century, in the theory of \textit{Fundamental Forces based on
Nonabelian Gauge Fields}, for which we have given a rather prosaic name :

\vspace{.2cm}
{\fontsize{8pt}{10pt}\selectfont{
\begin{tabular}{|lcl|}
\hline
 & THE STANDARD MODEL OF HIGH ENERGY PHYSICS & \\
 \hline
\end{tabular}}}

\vspace{.2cm}

In this theory, the strong forces operating within the nuclei and within the
nucleons, as well as the weak forces that were revealed through the discovery of
radioactivity hundred years ago are understood to be generalizations of the

\vspace{.2cm}
{\fontsize{8pt}{10pt}\selectfont{
\begin{tabular}{|lcl|}
\hline
 & ELECTRODYNAMICS OF FARADY AND MAXWELL. & \\
 \hline
\end{tabular}}}

\vspace{.2cm}

Electrodynamics was formulated around the year 1875 and its applications
came in the 20th century. We owe a lot to the Faraday-Maxwell Electrodynamics,
for the applications of electrodynamic technology have become a part of modern
life. People take out a small gadget from their pockets and speak to their friends
living hundreds or thousands of kilometers away; somebody in a spacelab turns
a knob of an instrument and controls a spacecraft that is hurtling across millions
of kilometers away to a distant planet. All this has been possible only because of
electromagnetic waves.

It turns out that the dynamics of strong and weak forces was formulated around
1975 almost exactly 100 years after the year of electrodynamics. We may expect
that equally profound applications will follow, once the technologies of the strong
and weak forces are mastered. That may be the technology of the 21st century.
After this bird’s eye view of one century of developments we now describe the
four forces of Nature and then take up the Standard Model.

\section*{Fundamental Forces of Nature}

The four fundamental forces are the strong, electromagnetic,weak and gravita-
tional forces. Strong forces are responsible for binding nucleons into the nucleus
(and for binding quarks into the nucleons). They are characterised by a strength
parameter which is roughly one and their range is $10^{-13}$cm. Electromagnetic
forces bind nuclei and electrons to form atoms and molecules and bind atoms or
molecules to form solid matter. Their strength is measured by the fine structure
constant whose value is about 1/137 and their range is infinite. Weak interactions
cause the beta decay of nuclei and also are responsible for the fusion reactions that
power the Sun and stars. Their strength is $10^{-5} m_{p}^{-2}$
p and their range is less than
 $10^{-14}$ cm. Here  $m_{p}$ is the mass of the proton. Gravity binds the planets into the
solar system, stars into galaxies and so on. Although gravity is the weakest force,
its strength being $10^{-40} m_{p}^{-2}$
, it becomes the dominant force for the Universe at
large, because of its infinite range and because of its being attractive only (unlike
electromagnetism where attraction can be cancelled by repulsion).

In quantum theory the range of a force is inversely proportional to the mass
of the quantum that is exchanged. Since the photon mass is zero, electromagnetic
force mediated by the exchange of photons is of infinite range. Since the strong
interaction between nucleons has finite range, it has to be mediated by a quantum
(or particle) of finite mass. This is how Yukawa predicted the particle that was
later identified as the pion,which we now know to be a composite of a quark and
an antiquark. About the finite range of the weak force and the quantum exchanged
we shall say more later. Since gravity has infinite range, quantum theory of grav-
ity (if it is constructed) will have its quantum, called graviton with zero mass.

The above text-book classification of the four fundamental forces has broken
down. We now know that weak force and electromagnetism are two facets of one
entity called electroweak force. Can one go further and unify the strong force with
the electroweak force? It is possible to do so and it is called grand unification, but
that is a speculative step which may be confirmed only in the future. The grander
unification will be unification with gravitation which we may call ”Total Unifi-
cation” which was the dream of Einstein. Perhaps that will be realized by string
theory and in the future.


For the present we have the Standard Model which is a theory of the elec-
troweak and strong interactions and is based on a generalization of elecrodynam-
ics. So let us start with electrodynamics.


\section*{Laws of Electrodynamics}

The laws of electrodynamics are expressed in terms of the following parial
differential equations:
\begin{align*}
\vec{\nabla} \cdot \vec{E} &= 4 \pi \rho\\
\vec{\nabla} \times \vec{E} + \frac{1}{c} \frac{\partial \vec{B}}{\partial t} &=0\\
\vec{\nabla} \cdot \vec{B} &= 0\\
\vec{\nabla} \times \vec{E} + \frac{1}{c} \frac{\partial \vec{B}}{\partial t} &= \frac{4\pi}{c} \vec{j}
\end{align*}

These laws were formulated by Maxwell on the basis of earlier experimental
discoveries by Oerstead, Ampere, Faraday and many others. Actually, from his
observations and deep experimental studies of the electromagnetic phenomena, Faraday had actually built up an intuitive physical picture of the electromagnetic
field and Maxwell made this picture precise by his mathematical formulation.
Once Maxwell wrote down the complete and consistent system of laws, very im-
portant consequences followed. He could show that his equations admitted the
existence of waves that travelled with a velocity that he could calculate purely
from electrical measurements to be $3 \times 10^{10}$ cm per sec. Since the velocity of
light was known to be this number, Maxwell proposed that light was an electro-
magnetic wave. This was a great discovery since until that time nobody knew
what light was. Subsequently Hertz experimentally demonstrated the existence of
the electromagnetic waves prediced by Maxwell.

figure???


Faraday had actually built up an intuitive physical picture of the electromagnetic
field and Maxwell made this picture precise by his mathematical formulation.
Once Maxwell wrote down the complete and consistent system of laws, very im-
portant consequences followed. He could show that his equations admitted the
existence of waves that travelled with a velocity that he could calculate purely
from electrical measurements to be 3 × 1010 cm per sec. Since the velocity of
light was known to be this number, Maxwell proposed that light was an electro-
magnetic wave. This was a great discovery since until that time nobody knew
what light was. Subsequently Hertz experimentally demonstrated the existence of
the electromagnetic waves prediced by Maxwell.

Let us briefly compare the Faraday-Maxwell picture with quantum field the-
ory. In the former, a charged particle, say, a proton is surrounded by an electro-
magnetic field existing at every point in space-time. If another charged particle,
an electron is placed in this field, the field will interact with the electron and that
is how the electromagnetic interaction between proton and electron is to be under-
stood in the classical electromagnetic theory. In quantum field theory,the proton
emits an electromagnetic quantum which is called the photon and the electron ab-
sorbs it and this is how the interaction between the proton and electron is to be
understood. Exchange of the field-quanta is responsible for the interaction. This
is depicted in the ”Feynman diagram” shown in Fig 1. This is our brief description
of quantum field theory which is the basic language in which Standard Model of
High Energy Physics is written.

\section*{Standard Model of High Energy Physics}


Standard Model consists of two parts, electroweak dynamics that unifies elec-
tromagnetic and weak interactions and chromodynamics that governs strong in-
teractions.


In electrodynamics we have an electromagnetic field described by the pair
of vector fields $(\vec{E}, \vec{B})$ and the corresponding quantum is the photon. Analo-
gously, in electroweak dynamics we have four types of generalized electromag-
netic fields ($\vec{E_{i}}$, $\vec{B_{i}}$) with the index i going over 1 to 4, one of them being the
old Faraday-Maxwell electromagnetic field. Correspondingly there exist four ele-
croweak quanta, also called electroweak gauge bosons. One of them is the photon
$\gamma$, mediating electromagnetic interaction and the other three $W^{+}$ ,$W^{-}$ and Z me-
diate weak interaction.

In Fig 2 we illustrate an example of weak interaction, namely the decay of the
neutron into proton,electron and antineutrino. Neutron and proton are depicted as
composites of three quarks $udd$ and $uud$ respectively. The $d$ quark turns into a
$u$ quark by emitting the weak quantum $W^{-}$ which turns into a pair of ”leptons”
(electron and antineutrino). The electromagnetic and weak interactions among
the quarks and the leptons mediated by the electroweak quanta are pictured in the
Feynman diagrams of Fig 3 (a to d).


The laws of electroweak dynamics (EWD) are given by the equations:

figures?????

\begin{align}
\vec{\nabla} \cdot \vec{E_{i}} &= 4 \pi \rho_{i}\\
\vec{\nabla} \times \vec{E_{i}} + \frac{1}{c} \frac{\partial \vec{B_{i}}}{\partial t} &=0\\
\vec{\nabla} \cdot \vec{B_{i}} &= 0\\
\vec{\nabla} \times \vec{B_{i}} + \frac{1}{c} \frac{\partial \vec{E_{i}}}{\partial t}+ \ldots &= \frac{4\pi}{c} \vec{j_{i}}
\end{align}

We shall explain the dots in the equations soon.


In Quantum Chromodynamics(QCD) governing strong interactions we have
eight types of generalized electromagnetic fields $(\vec{E}_{\alpha} , \vec{B}_{\alpha})$ with the index  rang-
ing over 1 to 8. The corresponding quanta are called gluons $G_{\alpha}$ since it is their
exchange between quarks that bind or glue the quarks together to form the proton
or neutron. This exchange of gluons between the quarks $q$ (which may be $u$ or
$d$) is shown in Fig 4. The laws of QCD are given by the same equations as those
given above for EWD with the index $i$ replaced by $\alpha$.The analogy of the laws of
EWD and QCD to the original laws of electrodynamics is obvious.


If these generalizations of Maxwell’s equations are as simple as made out
above, why did Standard Model take another hundred years to be constructed?
The answer lies in the dots in the equations expressing the laws of EWD and
QCD. Let us go back to electrodynamics in which every electrically charged par-
ticle interacts with the electromagnetic field or (in the quantized version) emits
or absorbs a photon. But photon itself does not have charge and hence does not
interact with itself. In the generalization described above, there are twelve gen-
eralized charges, four in EWD and eight in QCD, corresponding to the similar
number of generalized electromagnetic fields. In contrast to electric charge which
is just a number (positive, negative or zero), these generalized charges are matri-
ces which do not commute with each other and hence can be called nonabelian
charges. (In mathematics, algebras with commuting and noncommuting objects
are respectively called abelian and nonabelian algebras.) Electrodynamics which
is based on the abelian charge is called abelian gauge theory and the generaliza-
tion based on nonabelian charges is called nonabelian gauge theory. In contrast to
photon, which is the abelian gauge quantum and does not carry the abelian electric
charge, the nonabelian gauge quanta themselves carry the nonabelian charges and
hence are self-interacting. These self-interactions are shown in Fig 5; both a cubic
and a quartic coupling exist. The nonlinear terms expressing these couplings are
hidden behind our dots and it is these which make the theory of nonabelian gauge
fields much more complex than the simple Maxwell theory. Nonabelian gauge
fields were introduced by Yang and Mills in 1954 and hence are also called Yang-
Mills(YM) fields, but it took many more important steps in the next two decades
before this theory could be used to construct the correct Standard Model.

figures????

A remark on gravity is appropriate at this point. What plays the role of
”charge” in gravity? Obviously it is mass in Newton’s theory, but is replaced
by energy in Einstein’s theory. Since the gravitational field itself has energy, it
has to be self-interacting exactly as in the case of the nonabelian gauge field car-
rying the nonabelian charge. However, unlike in that case where Yang and Mills
showed a cubic and a quartic interaction completes the theory, in the gravitational
case one has to add vertices of all orders (quintic, sextic...). This is what makes
Einstein’s theory of gravitation much more intractable in the quantized version.
In complexity, Yang-Mills theory comes between the simple Maxwell theory and
the complex Einstein theory.

figures??

\section*{The Field and Particle Sectors}

The constituents of the universe according to the Standard Model come in two
categories which may be called the field sector and the particle sector.


In the field sector, we have the twelve gauge fields $\gamma, W^{+} , W^{-}, Z,$\break $ G_{1} ,
G_{2} ,....G_{8}$ . Their quanta are all particles with spin 1 (in units of $\hbar$), exactly like
the first and most familiar one among them, the photon ($\gamma$). All such particles
having spin equal to integral multiple of $\hbar$ belong to the great family of ”bosons”
(particles obeying Bose-Einstein statistics).

The particle sector consists of spin $\frac{1}{2}$ particles belonging to the other great
family of ”fermions” (particles that obey Fermi-Dirac statistics.) Among these,
we have already encountered the two quarks $(u,d)$ and the two ”leptons” $(v,e)$.
The quarks make up the nucleon and the nucleons make nuclei. Nuclei and elec-
trons make atoms, molecules and all known matter. The weak radioactive decays
involve the $v$. Thus the quartet of particles consisting of a quark doublet and a
lepton doublet seems to be sufficient to make up the whole universe. However
Nature has chocen to repeat this quartet twice more, so that there actually exist
three ”generations” of particle quartets each consisting of a quark doublet and a lepton doublet:
\begin{itemize}
\item (u, d), ($\nu_{e}$, $e$)
\item (c, s), ($\nu_{mu}$, $\mu$)
\item (t, b), ($\nu_{\tau}$, $\tau$)
\end{itemize}

The existence of three generations is required to explain (the experimentally
observed) matter-antimatter asymmetry which can solve the cosmological puzzle:
how did the universe which started as a fireball with equal proportion of matter
and antimatter evolve into a state which has only matter? But we shall not delve
into this question here except to mention that Kobayashi and Maskawa predicted,
on this basis, the existence of three generations of quarks, even before the three
generations were experimentally discovered.


An important remark about quantum field theory is in order here. Although
we divided the stuff of the universe into a field sector and a particle sector, fields
have their quanta which are particles and in quantum field theory each particle in
the particle sector also has its quantum field; electron, for instance, is the quantum
of the electron field. Thus quantum field theory unifies field and particle concepts.


There is an incompleteness in our description of the QCD sector. Both quarks
and gluouns are not seen directly in any experiment. They are supposed to be
permanently confined inside the proton and neutron. But this hypothesis of con-
finement which is supposed to be a property of QCD has not been proved. This im-
portant theoretical challenge remains as a loophole. This problem is so intractable
that it has been announced as one of the millennium problems of mathematics.


\section{Symmetry breaking and Higgs}


Remember the vast disparity between electromagnetism and the weak force
as regards their ranges; one is of infinite range and the other is short-ranged.
How does electroweak unification cope with this breakdown of the electroweak
symmetry that is intrinsic to the unification? This is achieved by a spontaneous
breakdown of symmetry engineered by the celebrated Higgs mechanism which keeps photon massless while raising the masses of W and Z to finite values. Thus
weak interaction gets a finite range. The experimental discovery of W and Z with
the masses predicted by the electroweak theory was a great triumph for the theory.


The idea of spantaneous breakdown of symmetry (SBS) in high energy physics
originates from Nambu although he applied it in a different context. But the stum-
bling block was the Goldstone theorem. This predicted the existence of a massless
spin zero boson as the consequence of SBS and prevented the application of SBS
to construct any physically relevant theory, since such a massless particle is not
observed. [See the Appendix at the end of this article for more on SBS.] Thus
apparently one had to choose between the devil (massless W boson) and the deep
sea (massless spin zero boson).


It was Higgs who, in 1964, showed that this is not correct. By using Gold-
stone’s model (which is much simpler than the original Nambu model), he showed
that there is no Goldstone theorem if the symmetry that is broken is a gauge sym-
metry. The devil drinks up the deep sea and comes out as a regular massive spin
one gauge boson. No massless spin zero boson is left. This is called Higgs mech-
anism. Many other authors also contributed to this idea.

Earlier, Glashow had identified the correct version of the Yang-Mills theory
for the electrowaek unification. By combining that with Higgs mechanism Wein-
berg and Salam independantly constructed the elecroweak part of SM in 1967.


There is a bonus. Higgs mechanism postulates the existence of a universal
all-pervading field called the Higgs field and this field which gives masses to W
and Z also gives masses to all the fermions of the particle sector, except to the
neutrinos. Thus, in particular, the masses of the quarks and electron come from
the Higgs field.


But there is an important byproduct of the Higgs mechanism: a massive spin
zero boson, called the Higgs boson, must exist as a relic of the original Higgs
field. High energy physicists searching for it in all the earlier particle accelerators
had failed to find it. So the announcement on 4 July 2012 that the Higgs boson has
been sighted finally at a mass of 125 GeV at the gigantic particle collider called
Large Hadron Collider(LHC) at CERN, Geneva has been welcomed by every-
body. More tests have to be performed to establish that the particle seen is indeed
the Higgs boson.


In the last four decades, experimenters have succeeded in confirming every
component of the full SM with three generations of fermions. Higgs boson re-
mained as the only missing piece. So with its discovery (assuming that the dis-
covery will be established by further tests), Standard Model has emerged as the
Standard Theory describing Nature. This is a great scientific achievement. SM
now deserves a better name!


We have now completed our description of the SM. We list below the Nobel
Prizes sofar awarded to some of the makers of the SM, the theorists who proposed
it and the experimentalists who proved it to be right.

{\fontsize{6}{8}\selectfont{
\begin{tabular}{lll}
Year & Winners & Contribution to SM\\
1978 & Glashow,Salam,Weinberg & Construction of Electroweak Theory \\
1983 & Rubbia, Van der Meer & Discovery of W and Z \\
1990 & Friedman,Kendal,Taylor & ”Observation” of quarks inside proton\\
1999 & ’t Hooft, Veltman & Proof of renormalizability of EW Theory\\
2004 & Gross, Politzer, Wilczek & Asymptotic freedom of YM Theory\\
2008 & Nambu  & Spontaneous breaking of symmetry\\
2008 & Kobayashi, Maskawa &  Matter-antimatter asymmetry\\
\end{tabular}}}

\section*{Beyond Standard Model}


Neutrinos: Neutrinos are massless in the Standard Model. As already men-
tioned, Higgs mechanism does not give mass to neutrinos. About 15 years ago,
experimenters discovered that neutrinos do have tiny masses and this has been
hailed as a great discovery since this may show us how to go beyond the Standard
Model. Neutrino may be the portal to go beyond SM and that is the importance of
the India-based Neutrino Observatory (INO) which is about to come up in Tamil
Nadu.


Dark matter: Astronomers have discovered that most of the matter in the uni-
verse is not the kind we are familiar with. It is called dark matter since it does
not emit or absorb light. Although this discovery has been made already, nobody
knows what this dark matter is and only physicists can discover that. A dark mat-
ter experiment also will be mounted in the INO cavern (suitably extended).

In the last four decades after the Standard Model was constructed, theoreti-
cians have not been idle but have constructed many theories that go beyond the SM. Of these we have already mentioned one, namely grand unification. Another
is supersymmetry that postulates the existence of a boson corresponding to every
known fermion and vice versa. This is a very elegant symmetry that leads to a
better quantum field theory than the one on which SM has been built. But if it
is right, we have to discover a whole new world of particles equalling our known
world; remember we took a hundred years to discover the known particles starting
with the electron.


There are many more theoretical speculations apart from grand unification and
supersymmetry. But none of them has seen an iota of experimental support sofar,
even in the LHC. However LHC will have many more years of operation; let us
hope new things will be discovered.


\section*{Quantum Gravity}


The biggest loophole in SM is that gravity has been left out. The most suc-
cessful attempt to construct quantum gravity is the String Theory.


The role of quantum mechanics coupled with special relativity in providing the
basis for the understanding of what lies inside the atomic nucleus (the microcosm)
was mentioned at the beginning of this article. On the other hand, it is General
Relativity, which is also the theory of Gravitation that provides the framework for
understanding the Universe at large (the macrocosm).


It is a deep irony of Nature that the twin revolutions of quantum and relativity
that powered the conceptual advances of the 20th century and that underlie all the
subsequent scientific developments, have a basic incompatibility between them.
The marriage between quantum mechanics and relativity has not been possible.
By relativity, here we mean general relativity since special relativity has already
been combined with quantum mechanics leading to quantum field theory.

Gravity which gets subsumed into the very fabric of space and time in Ein-
stein’s General Relativity has resisted all attempts at being combined with the
quantum world. Hence, Quantum Gravity had become the most fundamental
problem of physics at the turn of the twentieth century.

This is in contrast to all the other fundamental forces of Nature, namely elec-
tromagnetic, strong and weak forces, which have all been successfully incorpo-rated into the quantum mechanical framework. The Standard Model of High En-
ergy Physics that we have described is just that, and the Standard Model leaves
out Gravity. This is the reason for the rise of String Theory, for it promises to be
a theory of Quantum Gravity. For the first time in history, we may be glimpsing
at a possible solution to the puzzle of Quantum Gravity. Actually, String Theory
offers much more than a quantum theory of gravity. It provides a quantum theory
of all the other forces too. In other words, it can incorporate the Standard Model
of HEP also, within a unifying framework that includes gravity.


So, String Theory has been hailed as the ”Theory Of Everything” and some
theoretical physicists have even had ”Dreams of a Final Theory” - which is actu-
ally the title of an excellent book by Weinberg. But one is tempted to say
”There are more things in heaven and earth, Horatio,
Than are dreamt of in your philosophy” - Shakespeare (in Hamlet, Act I, Scene V)


I do not believe that String Theory or any Theory will be a Theory of Every-
thing or a Final Theory. But it is very likely that String Theory is the Fundamental
Theory for the 21st century.


In string theory, a point particle is replaced by a one-dimensional object called
string as the fundamental entity. Its length is about 10-33 cm which is the length
scale of any theory of quantum gravity including string theory. The various vibra-
tional modes of the string correspond to all the elementary particles. String theory
automatically contains quantum gravity and that is its chief attraction. However
that is bought at a price. It works only if the number of space dimensions is 9
and including time it is 10. Where are the extra six dimensions? They are curled
up to form space bubbles existing at distance scales of the same 10-33 cm! Both
the string and the extra curled-up dimensions will be revealed only when we can
access such length scales. Remember we have sofar reached only 10-17 cm. We
have a long way to go.

Apart from string theory there are other approaches to quantum gravity. Only
future will tell us which is the right one. In any case, quantum gravity is an im-
portant (albeit distant) frontier and the journey continues.

\section*{Appendix: A Tutorial on SBS}

Consider a simple mechanical example: a ball placed on the top of a hill of circular cross section surrounded by a circular valley (Fig 6). This system has a
circular symmetry, but the ball is in an unstable equilibrium and will roll down
into the bottom of the valley where it will reach a point of stable equilibrium. The
ball could have come to any point along the circular bottom of the valley, but once
it has done it, the circular symmetry has been broken.


Now replace the hill and valley problem with a problem of field theory. This
is Goldstone’s model of the scalar field which has two components $\phi_{1}$ and $\phi_{2}$ .
The quanta of the scalar field have spin zero and hence are bosons. The potential
energy V of the field system as a function of the field components is chosen to
be exactly like in the mechanical example and has circular symmetry in the field
space (see Fig 7). It has a maximum energy at point A where $\phi_{1}$ and $\phi_{2}$ are zero
and a minimum all along a circle.


It is wrong to choose the maximum of the potential (point A) as the ground
state of the field system although the field has zero value at that point since it is a
state of unstable equilibrium. We can choose any one point along the circle of the
minimum of V, as the ground state of the system; however once we choose it, the
circular symmetry is broken. This is the mechanism of spontaneous breaking of
symmetry.


An important consequence follows. Since it does not cost any energy to move
around the circular trough of minimum potential, there exists a massless particle
(the $\alpha$ mode). As can be seen from Fig 7, movement along a direction normal
to this circle (the $\beta$ mode) costs positive potential energy and this corresponds to
the massive particle. The massless mode is called the Nambu-Goldstone boson
and this result is called the Goldstone Theorem (proved by Goldstone, Salam and
Weinberg) which states that SBS of any continuous symmetry results in the exis-
tence of the spin zero massless Nambu-Goldstone boson.


By the addition of a massless spin one gauge boson to the Goldstone model,
Higgs showed that the massless spin zero boson is eaten up by the massless spin
one boson and as a result, the massless spin one boson becomes massive and the
massless spin zero boson disappears.This is the Higgs mechanism. The massive
spin zero boson (the beta mode) however exists and this is the Higgs boson which
was eagerly searched for, and presumably found now.

Note that in the ground state, the field is not zero, but is equal to the radius of the circle of minimum potential. This is the universal Higgs field existing
everywhere, that gives mass to all the particles.


figures????????????
