\chapter{Gauge Theories}\label{chap24}

\Authorline{G. RAJASEKARAN }
\addtocontents{toc}{\protect\contentsline{section}{{\sl G. RAJASEKARAN}\smallskip}{}}

\authinfo{}

\section{Introduction}

Until recently it has been customary for any account of subatomic physics
to start with the observation that there. are four fundamental forces in the
universe: strong, electromagnetic, weak and: gravitational. 

\begin{center}
\begin{tabular}{ccc}
Force & Strength & Range\\[0.2cm]
Strong & 1 & $10^{-13}$ cm\\
Electromagnetic & 1/37 &$\infty$\\
Weak & $10^{-5}$ & $\sim$ 0\\
Gravitational & $10^{-39}$ & $\infty$\\
\end{tabular}
\end{center}

This textbook classification is breaking down as a consequence of
recent developments in high energy physics. These developments have led
to a breakthrough in our understanding of one of the fundamental forces,
namely, the weak force. It is this force which causes radioactive nuclei to 
disintegrate through beta-decay. There is every indication now that this
mysterious weak force and the familiar electromagnetic force are facets of
the same entity, which has been named the {\it electro-weak force}. ‘This
synthesis of electro and weak forces has been achieved through a nonabelian gauge theory. 

The successful synthesis of olectrodynamics and weak. force prompts us
to look for further unification with the other forces. There is good. indica
tion now that the strong force which is what binds the neutrons anc
protons to form nuclei is also. a manifestation of a non-abelian gauge
theory. Hence the chances of a grand synthesis of electro Weak dynamics
with the strong nuclear force appear bright. 

In the history of modern physics there have been many attempts to a .
unify the various forces of nature. The most heroic were those of Einstein
and Weyl who sought a unified theory of gravitation and electrodynamics.
But these earlier attempts did not have the benefit of the new insight into
the possibility of electro-weak synthesis which physicists. have now
obtained. 

The new developments have opened the way for the powsible ultimate unification of all the forces with gravitation a goal which was very dear to Einstein and which still remains a. Major goal of physics. Perhaps supersymmetry and supergravity, which are under intensive study at present will provide the next step towards this goal.

However, the following cautionary remark may be in order. All our present activity in high energy phvsics is concerned with gaining an under- standing of phenomena at the level of $10^{-13}$ to $10^{-5}$ cm. Gravitation will play a role in subatomic phenomena only when distances of the order of the Planck length, $(Gh/c^{3})^{1/2} \sim 10^{-33}$ cm are approached. Thus, something like 20 orders of magnitude have to be crossed, before a ral physical connection can be made between gravitation and other forces. We may. encounter entirely new types of forces in this journey from $10^{-4}$ cm to $10^{-3}$ cm. It is even possible that our present. concepts. of. space-time and quantum mechanics may undergo revolutionary changes before we reach down to the level of the gravitational length $10^{-3}$ cm. So, attempts at unification of the subatomic forces with gravitation may be premature at the
present stage of our knowledge. 

In a any case this talk will be restricted to the gauge theory of electroweak and strong forces only. Since gauge theory now covers all of particle—physics, our short review cannot be an exhaustive survey$^{1}$. An attempt will be made to provide a perspective on the subject, albeit a coloured perspective. 

\section{Flavour and colour dynamics} 


All matter is made dynamics leptons and ee a hadrons, Leptons are point particles having no strong interactions, Neutrinos, electrons and muons are — examples of leptons. Nucleons, pions, kaons and various hyperons are called hadrons. Hadrons have finite size with complex strong interactions among themselves. According fo current philosophy, hadrons are not regarded as elementary particles, but made of certain more elementary objects, called quarks. Nucleon and other baryons are supposed to be. composed of three quarks while the mesons are made of quark-antiquark pairs. So, we start with the quarks and leptons. 

figure????

The quark matrix is characterized by two types of quantum numbers, flavour which changes in the vertical direction and colour which changes in the horizontal direction. Flavour includes the well-known set of
quantum numbers such as- isospin, strangeness, etc., as well as newer 
quantum numbers. charm; truth, beauty, etc. Colour is a new concept 
which first originated in the following way. The three-quar wavefunction 
of a baryon was found to be totally. symmetric. So, to avoid contradiction with Fermi-Dirac statistics for the quarks, a new quantum number The leptonic multiplet was introduced. It has now been called colour.
has flavour only. 

So, we have quarks, flavour and colour... In‘ high-energy physics, either
arbitrary meaningless words are coined or words are used without their
the physics behind these words is not conventional meaning. Hopefully,
so meaningless. 

Now construct a non-abelian gauge field theory in flavour space. That 
is called quantum flavour dynamics (QFD) and it describes the unified 
theory of weak and electromagnetic interactions. The non-abelian gauge
theory in colour space is called quantum colour dynamics or quantum
chromodynamics (QCD) and this is believed to be the theory of strong 
interactions. 

The following table. gives a compact description of both QED and QCD. 

\begin{center}
\begin{tabular}	{|m{.5\textwidth}|m{.5\textwidth}|}
\hline
\hspace{2.3cm}QFD & \hspace{2.3cm}QCD\\
\hline
Gauge theory in flavour: space. Flavourchanging (vertical) transitions and corres-ponding emission or absorption of gauge bosons.& Gauge theory in colour space. Colourchanging (horizontal) transitions with emission of colour gauge bosons (called gluons). \\
 figure & figure\\
 \hline
 Describe weak and electromagnetic intercations & Strong interactions\\
 \hline
 Based ona gauge group $G_{F}$ acting on the flavour quantum numbers. What is $G_{F}$ The most favoured candidate is the $SU(2) \times  U(1)$ of Weinberg and. Salam. & According to current dogma, the colour gauge group $G_{C}$ is believed to be $SU(3)$ acting on the colour quantum numbers.\\
 \hline
 $G_{F}$ acts on both quarks and leptons. & $G_{c}$ acts an quarks only and the quarks are taken to be triples under colour $SU(3)$\\
 \hline
 The gauge symmetry $G_{F}$ is supposed to be spontaneously broken down to the level of electromagnetic U (1) symmetry. & In contrast, colour SU(3) is regarded as an exact symmetry.\\
 \hline
 Exept one gauge boson which is identifled with the photon all other gauge bosons of $G_{F}$ are massive. They are the various intermediate bosons of weak intercation: $W\pm$, $Z$, etc. & The octet of colour gauge bosons which are the gluons are all massless.\\
 \hline
\end{tabular}
\end{center}

\section{Quantum flavour dynamics}

We shall now consider some more details about QFD. Let us take the 
Weinberg-Salam version for simplicity. It is based on the gauge group:
$SU(2)_{L} \times U(1)$. The left-handed weak isospin group $SU(2)_{L}$ is associated
with the triplet of gauge fields $W_{\mu}^{i} (i = 1, 2, 3)$ while the weak hypercharge
group. U(1) is as8ociated with an abelian gauge field $B_{\mu}$. All these four
vector bosons are massless, to start with. 

Massive bosons are obtained through the spontaneous of
symmetry. The mechanism used for this purpose is the celebrated HiggsKibble mechanism, namely, non-vanishing  vacuum expectation values of certain chosen scalar fields $\phi$:
$$
\langle 0 | \phi 0 \rangle \neq 0
$$

As a result, we end up with three massive vector bosons $w^{+}$, $W^{-}$ and
$Z$ and one massive boson which is the photon $\gamma$. The charged bosons $W^{\pm}$
mediate the well-known charged-current weak interaction such as the beta
decay of the neutron. the neutral boson $Z$ mediates the neutral-current
weak interaction discovered in 1973. and $\gamma$ mediates the electromagnetic
interaction. Al these are illustrated in the following diagrams: 
 
figures?????

This is the unified theory of electro-weak interactions. Let us now look
at some simple consequences of this unified picture. 

{\it Masses of the weak bosons W. and Z }

In the pre-gauge theoretical description of $\beta$-decay due originally to
Fermi, the process was pictured as the
'four-fermion interaction and the coupling
constant for this interaction was 
$$
G_{w}\simeq \frac{10^{-5}}{m^{2}_{N}}
$$

where $m_{N}$ is the nucleon mass. In gauge
theory, this $G_{w}$ is replaced by $g^{2}/m^{2}_{w}$, where
$g$ is the gauge coupling constant and $m_{w}$
is the mass of the $W$ boson. This arises
from a factor $g$ each for emission and absorption of the $W$ boson and a
factor $1/m^{2}_{w}$, for the propagation of the $W$ boson (for small momentum
transfers). 

So we have 
$$
G_{w}\simeq \frac{g^{2}}{m^{2}_{w}}
$$

Unification of the weak and electromagnetic interactions ends to a con-
nection between the weak coupling constant $g$. and the electromagnetic
coupling constant: 
$$
g \sim eq e
$$

Combining these relationships, we get 
$$
m_{w} \simeq \frac{e}{\sqrt{G_{w}}} \simeq GeV
$$

where we have used $G_{w}= 10^{-5/m^{2}_{N}}$ and $e^{2}/4\pi = 1/137$.

This is only an order of magnitude calculation, but itis confirmed by a
more precise analysis. One finds 
$$
m_{w} {\rm and } m_{z} \simeq 80-100 GeV
$$

These masses are so high. that $W$ and $Z$ cannot be produced in the
highest energy accelerator existing in the world today. A new generation.
of high energy accelerators is needed to produce such massive particles. 
The major high energy experimental programmes of the world are being
geared towards this task. One may expect an intensive search for $W$ and
$Z$ bosons to start within a few years. 

An important lesson of the above simple calculation is that the weak
interactions in fact are as strong as the electromagnetic interactions. Only
because the $W$ boson is i so massive that the weak interaction appears weak
at low energies. — "At high energies, that is, at centre-of mass energies of
the order of 100 GeV, weak interaction regains its full strength.

{\it Neutral current}

The existence of neutral-current weak interaction-reactions such as a
with a strength comparable to that of the usual charged current weak
interaction as in (2) is a natural consequence of unification with electrodynamics. In fact, because of the dynamical mixing between weak and electromagnetic interactions inherent QFD, neutral current acts as something  like a bridge between conventional weak and electromagnetic phenomena.
Hence the discovery of the neutral current weak interaction in the neutrino
reactions in 1973 and the subsequent detailed. study which showed its
properties to be of the type expected i in QED have helped to confirm that
physicists are on. the right track. 

In fact, there are: four parameters $\alpha$, $\beta$ $\gamma$ and $\delta$ chatacterizing the
neutral current interactions in general among the neutrinos and hadrons.
a Their values as predicted by Weinberg-Salam (W-S) version of QFD are  compared to the experimentally determined values below: 

\begin{center}
\begin{tabular}	{ccc}
\hline 
Interaction& W-S ($\sin^{2} \theta = 3.22$) & Experimental\\
\hline
$\alpha$ & 0.56 & 0.58 $\pm$ 0.14\\
$\beta$ & 1 &  0.92 $\pm$ 0.14\\
$\gamma$ & -0.15 & -0.28 $\pm$ 0.14\\
$\delta$ & 0 & 0.06$\pm$ 0.14\\
\hline
\end{tabular}
\end{center}

einberg-Salam model has one: free parameter $sin^{2} \theta$ which has been fixed.
at 0.22 for the above comparison. One can see the remarkable nereomen|
between theory and experiment.

Since there are a number of astrophysicists i in the audience, it may be
relevant to point out that the neutral-current interaction of the neutrinos.
may be of astrophysical significance. This interaction leads to coherent
 scattering of neutrinos on nuclei and hence to neutrino pressure. (Without neutral currents, coherent scattering of neutrinos is not possible).
Possible importance. of this neutrino pressure on supernova explosion has been considered in recent literature.  

\textbf{{\it Weak-electromagnetic interference}}

The agreement between theory and experiment noted above does not establish $G_{F}$ to be $SU(2) \times U(1)$, for it concerns only one sector of the neutral current, namely the neutrino-hadron sector. Let us now. consider
the electron-hadron sector, where the neutral current interaction corresponding to the diagram (1) should exist. This process should exist along
with the usual electromagnetic interaction via the exchange of a photon (2). 

figures???

Since both the above interaction exist simultaneously, how can one
disentangle the weak neutral-current effect from the dominating electromagnetic effect? 

Whereas the electromagnetic interaction is parity-conserving, the weak
effect is expected to be parity-violating. In fact it is definitely parityviolating in the Weinberg-Salam model. So, look for parity-violation in
e-p interaction. 

Such a. parity-violation has been looked for, in 1 atomic physics, hoy ever
initial results were not so definitive. But, spectacular verification of this
parity-violation i in electron-nucleon interaction came from the high-energy.
electron-deuteron scattering experiment performed at the Stanford Linear
Accelerator Centre (SLAC) USA last year. 

It is easy to see the advantage of high energy oxporiment. in this context, if we estimate the expected amount of parity violation. The effect arises due to the interference between the weak and electromagnetic 
amplitudes given by the diagrams: 

figures???

So, we have 
$$
A_{pv} \sim \frac{(W K)(EM)}{|WK|^{2} + |EM|^{2}} \sim \frac{WK}{EM} \sim \frac{G_{w}}{e^{2}/q^{2}} \sim 10^{-4} \frac{q^{2}}{(GeV)^{2}}
$$

where we have again used $G_{w} = 10^{-5}/m^{2}_{N}$ and $e
^{2}/4\pi = 1/137$. Thus, the
parity- “Violating coefficient $A_{pv}$ is seen to increase with the. momentum
transfer $q^{2}$ and hence the advantage of going for a high-energy experiment. 

he principle of the SLAC experiment is simple. A beam of longitudinally polarized electrons of 20 GeV hit a liquid deuterium target and
the inelastically scattered electrons were detected. If $\sigma_{+}$ and $\sigma_{-}$ denote the
cross-sections for the process with the two directions of polarization, then
the parity-violating coefficient $A_{pv}$ is simply given by 
$$
A_{pv}=\frac{\sigma_{+}-\sigma_{-}}{\sigma_{+}+ \sigma_{-}}
$$
The experimental result was 
$$
A_{pv}= -(0.95 \pm 0.16) \times 10^{-4} \frac{q^{2}}{(GeV)^{2}}
$$
which is i of the same order of magnitude estimated above. Further it
agrees very well with the prediction of the Weinberg-Salam model (with
$\sin^{2}\theta = 0.22$): $— 0.87 x 10^{-4} q^{2}/(GeV)$. Hence: the jubilation over
gauge theory! 

Nevertheless, it is good to point out that there are 17 measurable
parameters in all the various neutral-current sectors and only 5 have so.
far been measured with sufficient accuracy, In any case, until $W\pm$ and
$Z$ are detected experimentally, gauge. theory cannot ¢ be regarded as
established. 

\subsubsection*{Renormalizability and unification}

QFD is renormalizable. The incorporation of weak interaction into
a gauge-theoretic framework elevates the theory of weak interactions to
the rank of a renormaliza ble theory—a position which it now shares with
QED. 

Here a remark concerning the unification of weak ane electromagnetic
‘interactions is in order. Apart from any aesthetic or other considerations,
there is a logical reason for linking weak and electromagnetic interactions
and that is just renormalizability. It is possible to construct a renormalizable gauge theory of weak interactions alone. But since ‘such a theory
would necessarily contain charged vector bosons. (mediating the neu ron
beta decay for instance), one has to have a ‘meaningful theory for these
electrically charged vector bosons. It has been known for a lon
that the theory of the charged vector boson interacting with the electro
magnetic field does not make much sense. Not only is the theory unrenormalizable, but is, in general, not even relativistically. invariant.$^{2}$ Al
these problems are neatly solved, once the charged vector bosons and |
photon are combined into a gauge multiplet, as is done in QFD. It is
worthwhile to note the close similarity between this and what. happens to
the troubles of higher-spin fields when these are incorporated in supersym-
. metric theories, There is surely a moral in all this: Nature does not deal
with arbitrary interactions. 


Thus a meaningful theory for charged vector bosons is obtained only
by dynamically combining their weak and electromagnetic. interactions.
This is a very satisfying thing about gauge theory. 

\subsubsection{Back to electrodynamics}

To put the whole thing into proper perspective, let us “how go back to electrodynamics. The basic laws of electrodynamics were first formulated by Maxwell more than 100 years ago. (The year 1979 does not only mark the centenary of the birth of Einstein, but also the centenary of the death of Maxwell. In his work Maxwell was greatly influenced by the intuitive physical pictures which Faraday had already built up for the understanding of electromagnetic phenomena. 

Earlier discoveries of Oersted, Ampere and Faraday had unified .
electricity and magnetism into a single science of electromagnetism. Maxwell then unified optics with electrodynamics. 

\begin{center}
\begin{tabular}{|ccc|}
\hline
& $\bigtriangledown \cdot E = 4 \pi \rho$ &\\[0.2cm]
& $\bigtriangledown \times E + \frac{1}{c} \frac{\partial B}{\partial t} = 0$&\\[0.2cm]
& $\bigtriangledown \cdot B =0$ &\\[0.2cm]
& $\bigtriangledown \times B - \frac{1}{c} \frac{\partial E}{\partial t} = \frac{4\pi}{c}J$&\\[0.2cm]
& {\rm Laws of Electrodynamics  } & \\[0.2cm]
\hline
\end{tabular}
\end{center}

All this is i old stuff Maxwell’ s equations have Stood the test of time
for these 100 years. Will they remain in isolated glory for ever? 

The answer has to be in the negative. Progress in physics generally con- sists in the discovery of more general laws of wider applica ality which
reduse the older laws to some restricted domain. 

This is precisely what we are witnessing now. The successful develop: ment of gauge theory shows that there exist more general laws which subsume Maxwell’s laws of electrodynamics. Thre exist more general 
fields $E_{i}$ and $B_{i}$ (the index  $i$ running over the set of gauge bosons) and these fields are generalizations of the usual electric and. magnetic fields and describe not only electrodynamics but weak interactions also. Actually,
$E_{i}$, and $B_{i}$ (for $i= 1, 2,3, 4$) are related to the gauge boson fields $A$, $W^{+}$, $W^{-}$ and $Z$ in an ‘analogous way as the usual electric and magnetic
fields are related to the electromagnetic potentials. These electro-weak
fields KE, and B, satisfy laws of motion which. are similar to Maxwells laws of electrodynamics: 
\begin{center}
\begin{tabular}{|ccc|}
\hline
& $\bigtriangledown \cdot E_{1}+ \cdots = 4 \pi \rho$ &\\[0.2cm]
& $\bigtriangledown \times E_{1} + \frac{1}{c} \frac{\partial B_{t}}{\partial t}+ \ldots = 0$&\\[0.2cm]
& $\bigtriangledown \cdot B_{t} =0$ &\\[0.2cm]
& $\bigtriangledown \times B_{t} - \frac{1}{c} \frac{\partial E_{t}}{\partial t} = \frac{4\pi}{c}J_{1}$&\\[0.2cm]
& {\rm Laws of Electrodynamics  } & \\[0.2cm]
\hline
\end{tabular}
\end{center}

There are some crucial differences between the laws of electrodynamics
and those of electro-weak dynamics. The. self-interaction between the gauge fields indicated by the dia- :
grams: which is. a : characteristic
feature of non-abelian gauge theory
is absent in the abelian electromag-
netic theory. These self-interactions
lead to non-linearities which are denoted by dots in the above equations. Further, the spontaneous breakdown:
of symmetry leading tothe asymmetric world of. massive $W$ and $Z$ but massless $\gamma$ affects the physical interpretation of these equations. 


Nevertheless, it is. correct to say that Maxwell's laws have been.
incorporated into a more general system of laws which unify electr xdynamics and weak interactions. The implications of this grand unification 
are yet to be fully understood or realised. There is no doubt that the
consequences of this unification will be as profound and as far-reaching
as those of Faraday’s unification of electricity and magnetism and of .
Maxwell's unification of electrodynamics and optics. 

The truly profound consequences of the electro-weak synthesis have
yet to be realised. However, the following two results are worth
mentioning. 

\subsubsection*{Phase transition and restor ‘ation of weak-electr omagnetic symmetry}

There exists a similarity between the spontaneous. breakdown of symmetry
and the phenomenon of phase transition. In particular Kirzhnitz. and
Linde$^{3}$ in 1972 pointed out the close analogy between the Higgs Lagrangian
of the spontaneously broken quantum field theory and the free energy
. expression in Landau-Ginzburg’s phenomenological theory of phase
transitions. As a consequence of this analogy, there exists a critical
temperature, $T_{c}$, above which the symmetry between weak and electromagnetic interactions is restored. So, a collection of leptons with conventional weak and electromagnetic interactions will behave entirely differently
if. their temperature is raised above $T_{c}$. The striking physical differences are as given below: 

figures ??? page no 11


However, the critical temperature $T_{c}$, is of the order 

$$
T_{c} \sim \phi \sim \frac{m_{w}}{e} \sim G_{w}^{-4} \sim 10^{3} GeV \sim 10^{16} {^{0}K}.
$$

Perhaps this is too hot even for astrophysicists! 

\subsubsection*{QFD with external electromagnetic field}

A large part of classical physics and even quantum physics, in the atomic
as well as cosmological scale makes use of the concept of external classical
electromagnetic field.” In view of the unification, does this: require any
change? The following example suggests this may be so.

The electric charge of the neutrino is zero, but it can have an “electro.
magnetic form factor induced by the weak interaction through the following diagrams: 

figures ??

However, the result comes out divergent  in models of the Weinbers-Salam
type. How can this happen in a renormalizable theory? 

The answer is that only the physical S-matrix elements and decay
matrix elements on the mass-shall are finite after renormalization. The
form-factor, being an off-mass-shall Green's function, remains divergent.

In other words, the electromagnetic form factor is not a measur
quantity. To measure it, virtual photons of (mass)$^{2}= q^{4} \neq 0$ is necessary.
So, one should consider the whole process including the charged particle which emits the virtual photon. But then, in this unified model, there is the $Z$ boson which interacts. with all the charged. particles. So, one e should
include the $Z$ exchange diagrams also. Once we add both these classes
of diagrams together, a finite result t is obtained: 


figure


But, since electromagnetic form factors can be used to describe the
interaction of a particle with an external classical electromagnetic field,
this leads to. the question posed above. 

\section{Quantum chromodynamics}

We now come to QCD where the gauge group is $SU (3)_{c}$. The quarks are
colour triplets while the gluons are colour octets and play the role of gauge
bosons. The QCD Lagrangian is 
\begin{align*}
\mathcal{L} &= - \frac{1}{4} (\partial_{\mu} A_{v}^{i}- \partial_{v}A^{i}_{\mu}- gf^{ijk}A^{j}_{\mu}A_{v}^{k})^{2}\\
&+ \bar{\psi}(i_{\gamma}^{\mu} \partial_{\mu} - g\gamma^{\mu} \frac{\lambda^{i}}{2}A_{\mu}^{i})\psi.
\end{align*}

Here $\psi$ is the quark field, $A_{\mu}^{i}$ is the gluon field, $g$ is the gauge coupling
constant, 7 goes over 1 to 8, $f^{ijk}$ are the structure constants of $SU (3)$ and
$\lambda^{i}/2$ are the $SU(3)$ generators in. the triplet representation. The following -
interaction vertices are contained in $\mathcal{L}$:

figures

The gluon-gluon vertices are characteriatic of non-abelian gauge theory
and distinguish QCD from the abelian QED.  

The first question we have to ask is: Why QCD? What i is the! reason
for believing QCD to be the teow of S.I.?

\subsubsection*{Why QCD?}

Fora long time,” physicists had given up. field theory a as a tefl approach for understanding strong interactions and taken to the S-matrix approach. : So, what caused the recent resurgence of field theory in strong interaction physics and what is the reason for going for this non-abelian gauge field theory (QCD)? 

The reason comes from an experiment the so-called deep inelastic scattering of leptons on the nucleon: 
$$
e + N \rightarrow e + {rm hadrons}
$$

It was found that, as observed by a high. $q^{2}$ probe, the nucleon behaves as if it were composed of {\it free}, point-like constituents (called {\it partons} by Feynman). The lepton scatters off each parton, elastically and incoherently. The incoherent sum of all parton cross-sections gives a very good descrip- tion of the experimental results. Thus, the complete cross-section for the electron scattering off the nucleon can be written (schematically) as 
$$
\sigma_{N} \sim \sum_{i} \int^{1}_{0} dx f_{i}(x) \sigma_{i}(x)
$$

where $\sigma_{i}(x)$ is the electron-scattering cross-section on the ith parton with. fractional longitudinal momentum $x$ and $f_{i}(x)$) is the probability for finding the ith parton with fractional longitudinal momentum $x$ inside the nucleon. Integrating over all the fractions and summing over all the partons $i$ incoherently, we get the electron-nucleon cross-section. It was a remarkable discovery that such a complicated process could be described by such a simple formula. Similar results were found for the neutrino-nucleon scattering processes also:
$$
v + N \rightarrow \mu + {\rm hadrons(charged- current weak interaction)} 
$$ 
$$
\rightarrow + {\rm hadrons (neutral-current weak interaction)}
$$

This phenomenon has a rather close resemblance to Rutherford's famous $\alpha$-particle scattering’ experiments which led to the discovery of the nucleus inside the atom. Thomson’s spread-out atomic model would lead to soft scattering (Le. small Scattering angles) only. . Experimentally, Rutherford and collaborators found hard Scattering (i.e. large scattering angles), thus showing the presence of the point-nucleus ‘inside the atom. In the same way, in the deep inelastic lepton-nucleon scattering, even for large $q^{2}$ (i.e. large scattering angle), scattering was observed to take place, in contrast to what would be expected for a spread
out nucleon. This leads to the discovery of point: elie constituents. deep
inside the nucleon. 

More detailed study of the experimental data devealed: that these
partons are in fact quarks; they seemed to have the same spins and charges
as expected for quarks. (We shall say more on this point later.) 

Attention should now be drawn to the adjective ‘free’. In. addition to
being point- -like, the quark-partons behave as if they are free. If they are
interacting, the cross-section formula would not be so simple. 

Now, the quarks are bound by tremendous attractive forces to make
up the nucleon. So, the interaction between quarks Should really be.
superstrong. And yet, when observed through high $q^{2}$ probes, this superstrong interaction weakens to such an extent that the quarks. behave as
free particles. 

For. quite sometime this was a mystery. On the other hand, this
provided an important clue about the nature of the strong interaction
itself. We can now say that any theory of strong interactions should
satisfy this property, namely, it should tend to a free particle theory or a
free field theory at high $q^{2}$. Is there any such theory? 

Consider non-relativistic potential scattering, i.e. non- “relativistic particles: interacting through well-defined smooth potentials. Since the total
energy can be written as $E= T+ V$, as the kinetic energy $T$ increases, the
potential energy $V$ becomes less and less important in comparison, so that
for high energies the: the theory does tend to a theory. of free particles, for
properly defined: smooth potentials. 


But, of course, this is not useful for high energy physics which has to
be described by relativistic quantum mechanics. Here, particle-production -
dominates at high energies and potential description fails. 

So, we should ask the same question in the realm of relativistic quantum
field theories. Here it is renormalization group which provides the required
technique. By using renormalization group, one can define a momentum- .
dependent coupling constant $g(q^{2})$, also. called effective coupling constant.
‘So, what we need is a theory in which 
$$
g(q^{2}) \rightarrow 0 \quad {\rm for}\quad q^{2} \rightarrow \infty.
$$

Such a theory is called {\rm asymptotically free}, i.e. the.
theory. tends. to a free. field theory for asymptotic
momenta. 

To cut the long story short, it was soon discovered
that none of the conventional field theories such as
$\phi^{4}$ Yukawa interaction $\bar{\psi}\psi\phi$ or. QED $\bar{\psi}\gamma_{\mu} \psi A^{\mu}$ is
asymptotically free. Of all the re-normalizable
quantum field theories, only non-abelian guage theory
was found to possess the unique distinction of being
asymptotically free. The characteristic triple gluon 
vertex is the essential ingradient that makes this theory asymtotically free.

So, asymtotically free non-abelian gauge theory emerged as a good
choice for a theory of strong interactions. Since the colour degree of
with three colours was already around, the gauge group was taken as the colour SU (3) and QCD was born.  

\subsubsection*{Asymptotic freedom}

Some details on the theory of asymptotic freedom may be given. The
group equation for the effective coupling constant $g(t)$ as. can be written
$$
\frac{dg(t)}{dt}= \beta(g(t))
$$

where $t=l q^{2}, q^{2}$ beging the momentum transfer. Here $\beta(g)$ is a well defined finite functions of $g$ which can be calculated in every re-normalizable quantum field theory at least for small $g$, and it characterizes the corresponding theory in a very impoertant way. It is $\beta(g)$ which controls the asymptotic behaviour of the theory.

First consider the case where $\beta(g)$ is positive. So, $dg/dt$ is positive and $g(t)$ increases with $t$ or $q^{2}$. This theory is not asymptotically free. If, on the other
hand, $\beta (g)$ is negative near the origin, g(t) will decrease with $t$ to zero for asymptotic values of $t$. So, for an asymptotically free theory $\beta (g)$ should be negative near the origin, 

For QCD, $\beta (g)$ turns out to be 
$$
\beta(g) = \left(-11 + \frac{2}{3} f \right) \frac{g^{3}}{(4 \pi)^{2}} + 0(g^{5})
$$
where $f$ is the number of quark flavours.  Actually, $\beta(g)$ can be calculated using the gluon-self-energy diagrams: 


The first term within the bracket comes from the gluon-loop-diagram and hence is negative because of the three-gluon coupling while the second 
term proportional to $f$ arises from the asymptotically not-free quark-gluon :
coupling. QCD is. asymptotically free if the first term wins. over the
second, i.e. if $f < 16$.  

So far-so good. But the major problem: in QCD is. the following,
Where are the quarks and Eons Why: are they not produced i in hadronic
‘collisions?

One possibility which: is  believed by. many people. with varying degrees
of vehemence is that quarks and gluons are permanently confined within
the hadrons and can never be liberated. This dogma is called {\it colour
confinement}. Any coloured object (ice. anything which is not a singlet
under colour SU (3)) is confined to be within the hadrons.  Quarks and
gluons being colour triplet and octet respectively, they are confined. But
the familiar hadrons such as nucleons and: pions are: colour singlets and
hence can be produced in the free state. However, to this day, the hypothesis of colour confinement has not been. proved in QCD. 

\section{QCD with a difference (a heretical view)}

So far we have been discussing the orthodox version of QCD. The rest
of the talk will be devoted to a heretical view. 

The orthodox and unorthodox versions of QCD are based on the two
types of quarks, the fractionally- charged quarks of Gell-Mann and Zweig
(G-Z) and the integrally-charged quarks of Han and Nambu ay
respectively: 

figure

The electric charges are indicated:as superscripts. For the Han-Nambu
quarks, the charges vary in the colour (horizontal) direction; the colouraveraged charge is fractional and equal to that of the G-Z quarks. 


It had been generally supposed that only the G-Z assignment is in
agreement with experimental data on deep-inelastic scattering experiments
which. probe the charges of the constituents of the nucleon. But as was 
shown sometime agot, this is a false conclusion. A proper. treatment of 
the mixing between the gluons of QCD and the gauge bosons of QFD
: yields the following result for the effective charge. of the H-N quark as a
Seen 1 by a high-momentum probe: 
$$
\mathcal{Q}^{H \cdot N}_{{\rm eff.}} = \mathcal{Q}^{G-Z} + \mathcal{Q}^{c} \frac{m^{2}_{g}}{m^{2}_{g}-q^{2}}
$$

For momentum-transfer $g^{2} > m^{2}_{g}$, we gee that the colour-dependent part of the charge $\mathcal{Q}^{c}$ is damped out and hence
$$
\mathcal{Q}^{H.N}_{eff.} \rightarrow \mathcal{Q}^{G-z}
$$ 

In other words, although the quarks may be in fact integrally charged, they manifest only the colour-independent parts of their charges in deep- inelastic scattering experiments and hence masquerade as G-Z quarks.

How then can we distinguish between the fractionally-charged model and the integrally-charged quark model? The answer is through the gluons. 

The gluons also are different in the two models. Whereas in the orthodox QCD based on G-Z quarks the gluons are electrically neutral, in the QCD. based on integrally-charged quarks the gluons also are charged, This will be clear from a look at the quark matrix in the H-N case. ‘Since the gluons cause the horizontal transitions such as 
$$
u^{1} \rightarrow u^{0} + G^{+}
$$
they have to be charged. So, the possible electromagnetic effects of the gluons will signal integral charges for the quarks. 

Recently we$^{5}$ have looked for such gluon effects in two places which can be described here briefly. 

(a) {\it Two-gluon Jets in $e^{+}$ $e^{-}$ annihilation:} The study of jets in$e^{+}$ $e^{-}$ annihila- tion is potentially a rich source of information on the hadronic constituents. The $e^{+}$ $e^{-}$ annihilation producing a quark-antiquark $(q — \bar{q})$ pair which in turn produces the hadronic jets is illustrated in the figure. 

The annihilation into the quark-antiquark $(q-\bar{q})$. pair which sub- sequently materializes as two hadronic jets would give the angular distri- bution: 
$$
\left(\frac{d \sigma}{d \omega}\right)\sim \sum_{a} \mathcal{Q}^{2}_{a} (1+ \cos^{2}\theta)
$$
where $\mathcal{Q}_{a}$, are the quark-charges. On the other hand, the annihilation. into a charged gluon-antigluon pair which again materializes as two jets would
lead to 
$$
\left(\frac{d \sigma}{d \omega} \right)_{gluons} \sim \sum_{i} \mathcal{Q}^{2}_{i}(1-\cos^{2}\theta)
$$
where $\mathcal{Q}_{i}$ are the gluon-charges. We find the over-all angular distribution to be 
$$
\left(\frac{d \sigma}{d \omega} \right)_{total} \sim 1 + A \cos^{2}\theta
$$
where $A=37/43  \thickapprox 0.86$. This is to be compared to the experimental regult: $A=0.97\pm 0.17$. So, the gluon contribution is not ruled out.

(b) {Effect of charged gluons on lepton-pair production:} The mechanism
for the production of a lepton pair ($\mu^{+}$ $\mu^{-}$ or $e^{+} e^{-}$) of high invariant mass,
in hadronic collisions is illustrated in the following figure: 

figure??


This is essentially inverse to the $e^{+}$ $e^{-}$ annihilation process. Again cone —
tributions from quarks as well as charged gluons are possible: 
\begin{align*}
\left(\frac{d^{2}\sigma}{dq^{2} dy d\omega} \right)_{quarks} &\sim \sum_{a} \mathcal{Q}^{2}_{a} f_{a}(y_{1}) \bar{f}_{a}(y_{2})(1+ \cos^{2} \theta)\\
\left(\frac{d^{2}\sigma}{dq^{2} dy d\omega} \right)_{gluons} & \sim \sum_{i} \mathcal{Q}^{2}_{i}g(y_{1}) g(y_{2}) (1-\cos^{2} \theta)
\end{align*}

Here, $f_{a}(y_{1})$ and $\bar{f}_{a}(y_{2})$ are the probabilities for finding a quark in one
proton and an antiquark in the other proton. with fractional longitudinal
momenta $y_{1}$ and $y_{2}$ respectively. $g(y_{1})$ and $g(y_{2})$ are similar probabilities
for gluons. 


We find that the ratio of the gluon to the quark contribution can be
quite sizable in some kinematic regions. Also, the angular distribution
is drastically modified from the $(1 + cos^{4}\theta)$ form characteristic of quarkcontribution. Hopefully, these features can be confronted with experiment
soon. 

To sum up, integrally-charged quarks are also consistent with experi:
ments so far. 

\textbf{{\it reference for integral charges}}

Fairbank. et al.$^{6}$ have claimed to have detected fractional charges in a
modern version of Millikan’s oil-drop experiment with superconducting
balls. It is too early to say whether this will be confirmed by an independent experiment. If it is confirmed, then the heretical views expressed here
would have to be abandoned. 

Here, we would like: to suggest: that there may be some deeper reason
for prefering integral charges for all particles of nature. 

One such deepet reason is: provided’ by magnetic monopoles. Dirac?
showed long ago (in 1931) that, if magnetic monopoles exist, the laws of
quantum mechanics impose a constraint on the allowed electric charges.
He showed that the allowed electric charges $e’$ and magnetic charges $g'$
should satisfy the quantum condition: 
$$
e'g' = \frac{n \hbar c}{2}
$$
where $n$ is an integer.

It is important to note that once a magnetic monopole with a well-defined magnetic charge $g'$ is discovered, the Dirac's condition severely restricts the possible electric charge. In 1975 Price et al.$^{8}$ claimed the observation of a track of a magnetic monopole of stength $g'=\hbar c/e$ where $e$ is the electronic charge. This evidance has been contoverted and so magnetic monopole is not yet discovered. But, just for the sake of argument, if we belive this evidence, and if we also believe that objects with charge $e'-e/3$ also exist (as claimed by Fairbank et al.), then there is a contradiction with Dirac's quantum condition. For, now
$$
e'g'=\frac{\hbar  c}{3}
$$
So, {\it both} Fairbank and Price cannot be right!

{\it Leptons as quarks} 

In this last part of the talk, all caution will be abandoned and a hypothesis will be put forward which is even further removed from conventional views. This can  be stated in the form of a question. Can leptons be identified with quarks? In other words, are the hadrons made of leptons such as $e, \mu, v$ etc.?

The author has had this heretical idea for quite some time, even before.
gauge theory became a court religion. The chief motivation for it was the
mysterious principle or working hypothesis called {\it lepton-hadron Universality} which has guided the development of weak interaction theory for
quite some time—through current algebra, Cabibbo universality and now
even in the construction. of gauge models. 

The total weak current is always written as $l_{\mu}+ h_{\mu}$ and then leptonhadron universality is invoked to demand that the leptonic current $I_{\mu}$ and
the hadronic current $h_{\mu}$. should have the same algebraic. properties. Why
should this be so? 
  
Is not the simplest and the most. natural explanation that $I_{\mu}$ and $h_{\mu}$ are
identical? This would be the case indeed, if the hadronic constituents
are simply the leptons. There is then no need to write a hadronic current
separately, the leptonic current would describe the hadrons as well.  So,
the only current is $I_{\mu}$.

Of: course; such an identification of quarks with leptons would make
sense only if quarks are integrally charged. And we have already argued
that integral charged quarks are consistent with high-energy experiments
so far.

So, let us boldly  trypothesise:  
\begin{center}
quarks $\mod$ leptons
\end{center}

The problem one faces is: How to explain the absence of: strong
interactions for the leptons $e$, $\mu$, etc? This problem, although a difficult
one, may not be insurmountable. A definitive solution cannot be given
immediately, but one can envisage the following possibility. 

We can exploit the idea that vacuum is not simple. Vacuum can
undergo aphase transition, perhaps through acquiring a non-vanishing
expectation value of a Higgs field $\langle \phi \rangle$ which, however, is spacedependent. A crude model follows:

In the leptonic phase, the leptons have only weak and electromagnetic
interactions. There is a phase transition. Some bubbles develop, which
are regions corresponding to non-vanishing $\langle \phi \rangle$. If three leptons ora
lepton-antilepton pair gets trapped in a bubble then it behaves like a
hadron.  

We shall stop here, by concluding that if the above idea turns out to be
right, it would provide a very natural unification of the leptonic and hadronic worlds. 

\begin{thebibliography}{99}
\bibitem{} For. more’ details. and original references, the extensive reviews in the following
- conferences and. symposia may. be.consulted:—:
- ‘Tata Institute Winter School in High Energy Physics, Panchgani (December 1977);
High Energy Physics Symposium, Jaipur (Dec: 1978); Proceedings of Summer
Institate-on' Particle Physics;: Stanford: (July 1978). SLAC Report /No.:215; Inter---
national Conference on High Energy Physics,.Tokyo (Aug. 1978).

\bibitem{} Lee, T.D.,; and Yang, C.N., {\it Phys. Rev.} 128, 885 (1962).

Nakamura, N., {\it Prog, Theor. Phys.} 33, 279 (1965).

Tzou, K.H., {\it Nuovo. Cimento}, 33 286 (1964).  

\bibitem{} Kirzhnitz, DA. and Linde,’A.D., {\it Phys. Lett.} 42B, 471 (1972). 

\bibitem{} Rajasekaran,'G, and Roy; P., {\it Pramana}- 5,302 (1975).

Pati, J.C., and Salam, A., {\it Phys. Rev. Lett.} 36, 11 (1976).

\bibitem{} Rajasekaran,.G.. and. -Rindani, S.D., Madras. University. preprint MUTP-79/4,
Phys. Lett. preprint.   

Jayaraman; T. and: G., Rajasekaran, Madras University preprint MUTP-78/13
(to. be published). 

\bibitem{} La Rue, G.S: et ali, {\it Phys. Rev. Lett.,} 42,:142 (1979).

\bibitem{} Dirac, P:A.M., {\it Proc. Roy. Soc.} Gaondon), A133. 60 (1931). 

\bibitem{} Price; P.B., et al, {\it Phys, Rev. Lett}, 35,-487 (1975),    
\end{thebibliography}

\begin{enumerate}
  \item[{\it CSivaram: }] What i is. the strength of the neutral current as compared. to: the charged
current? Does it also have a. V-A- structure? 
\item[{G. R. :}] The strength: is comparable to: that: of the charged current. The
: structure of the: neutral current is 
$$
(V-A)-\sin^{2}\theta V
$$
where $\sin^{2} \theta$ is Weinberg's mixing angle.

\item[{\it CSivaram: }] The: Weinberg-Salam- theory. seems to: Suggest a suppression of the
ous - Cabibbo angle above-a certain critical magnetic field strength so that the.
A. particle may have a much enhanced lifetime in: certain hypernuclear
excited states.of $Co^{57}$ and Nb. © This is:analogous: to phase transition
above a certain critical field strength in superconductors. 

\item[{\it G. R.: }] Yes, the phenomenological description. of conventional weak interactioninvolves a number of: parameters such as Cabibbo angle which arise
from spontaneous breakdown of symmetry. By analogy to phasetransitions these parameters will. undergo drastic changes above a certain critical magnetic field. 
 
\item[{\it M Mphan: }]  Since electromagnetic: and weak interactions are connected, would it
mean that the particle. decay processes like radioactivity will be affected
by. strong electric. or magnetic: fields? 

\item[{\it G.R. :}] The effect of the magnetic field above a: critical strength in changing
weak interaction phenomena was noted by Dr. Sivaram in the above 
comment. However, this has nothing to do with unification of weak
and. electromagnetic interactions. Direct manifestations” of the
a! unification are hard to. find: 

\item[{\it S C Tiwarl:}] Is axial photon a. necessary outcome of unified weak and 
magnetic. theories? In four-fermion interacting type pseudoscalr
Lagrangian approach, it is found that photon should appear. 

\item[{\it G. R.:}] This is not generally true, although it: may be so in some models, Tn
particular, in Weinberg-Salam model there is no-axial photon. 

\item[{\it Yash Pal:}] Are there any deep attempts being made to understand the value of the
Weinberg-Salani mixing parameter $sin^{2}\theta$; 

\item[{\it G. R.:}] Yes, Weinberg-Salam model does not provide a genuine unification of
weak. and electromagnetic: interactions; For, the relative strength of the
weak and electromagnetic interactions is not completely fixed, but is:
related. to a parameter. $\sin^{2}\theta$ which: is not determined within the framework of this model. However, if the product group SU(2) $\times$ U(1) is
embedded within a simple group. such as: SU(5), then true unification
occurs and the value of. $sin^{2}\theta$ - is then. determined.” theoretically, Such
attempts are being made. 
\item[{\it AK Pandey:}] Suppose: theoreticians find by some quantization scheme, a value for
fine-structnre: constant: consistent: with electron ‘charge; then. this will
contradict: the . experiments giving fractional charges of quarks. Does
the experiment responsible. for the notion of fractional charge of quark
contain sufficient loopholes to be. reconciled with above. theory: or
should the theoreticians tty to get” a value of. fine-structure constant
nine times snialler? 
\item[{\it G. R.:}] The theory you are -referring to;.is. hypothetical and: the: experiment
is not established. So, the question can be faced; if and. when it arises. 

\item[{\it P-Achuthan:}] Identification’ of: leptons’ and. hadrons is “radical: indeed. “But. then
many. structures built’: with: so. much. effort may “have: to: be broken:
Your comments. please! 

\item[{\it G. R.:}] Newer ideas need ‘not break the older “structures completely. Usually,
the essence. ofthe older ideas. do. find a niche in the more complete
newer structures. It may “be that: the newer theory. (yet to be con.
structed) will show under what Physical conditions the separation:
between leptons and ‘hadrons become approximately valid: 

\item[{\it P Achuthan:}] Hadrons: as bubbles in: ideal liquid (ether) have been pictured earlier. by
V. Weisskopf. Your way of looking at hadrons as bubbles seems to be
a fruitful idea. 

\item[{\it S Ramddurai: }] Phase transition in the early universe has been examined by Ruderman - -
and he finds that the early universe does not affect this. 
  \end{enumerate}
