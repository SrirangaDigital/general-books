\chapter[Einstein and A Century of Physics]{Einstein and A Century of Physics}

\Authorline{G Rajasekaran}
\addtocontents{toc}{\protect\contentsline{section}{{\sl G Rajasekaran}\smallskip}{}}

\authinfo{Institute of Mathematical Sciences\\ 
Chennai}
        

{\centering{(Based on a lecture given at The Children's Club)}}

In the year 1905, Albert Einstein wrote three papers in Physics:
one that founded the Special Theory of Relativity, the second that
proposed that Light is made of Quanta and the third on Brownian
Motion that enabled the Atomic Hypothesis to be verified experimentally.
Thus Einstein made three fundamental discoveries in a single year.
Such a feat is unique in the Annals of Science. Hence 1905 is called
the Year of Miracles or the Miraculous Year and 2005 is to be
celebrated as the World Year of Physics.

      I will give an elementary account of these three contributions
of Einstein and also give an idea of their impact on the development
of Physics in these one hundred years 1905 to 2005.

\section*{Classical Physics}

To appreciate the import of Einstein's revolutionary contributions
one must start with Classical Physics, namely Physics as it existed
upto 1900. All of classical physics at the fundamental level could
be summarized by two great systems of dynamical laws, one was due to
Newton and governed "Matter" and the other was due to Maxwell and
governed "Field", the electromagnetic field.

Newton's law of motion for material bodies stated that mass times
acceleration is the applied force and his universal law of gravitation
stated that the gravitational force of attraction between two bodies
is proportional to the product of their masses and inversely
proportional to the square of their distance of separation. With this
system of laws Newton could unify the dynamics of bodies in the
Heavens with those on the Earth.

Maxwell's dynamical laws for the electromagnetic field were based on
the intuitive picture of the electric and magnetic fields that Faraday
had built out of the results of many famous experiments. Once Maxwell
succeeded in constructing his complete and consistent system of
mathematical equations dictating the dynamical behavior of the
electromagnetic field, he could predict that electromagnetic waves
existed and that their speed would be 300,000 km per second. By that
time the speed of light was already known to be the same number. So
Maxwell could identify that light was also an electromagnetic wave
and thus solved the puzzle of what light was.

In classical physics there is a third dynamical system dealing with
heat, called thermodynamics,but unlike the other two described above
this is not an independant dynamical system. Heat is nothing but
molecular motion and so thermodynamics is derivable from the other
two.Thermodynamics is the same as statistical mechanics of a large
collection of atoms or molecules. This was the great contribution
of Boltzmann. However existence of atoms and molecules was not a
universally accepted hypothesis at the end of the nineteenth
century.

We shall now take up Einstein's 1905 discoveries one by one.

\section*{Brownian motion}

In 1828, a botanist Robert Brown discovered that pollen grains
from plants suspended in water showed zig-zag motions. Although
these motions were attributed to some life-force, it became clear
that that was not true; infact inanimate particles of matter
also showed it.This Brownian motion is now known to be due to
the impact of molecules of water on the suspended particle.
These impacts will be on all sides and directions and so tend
to cancel on the average but not completely.A residual force
does remain and so the particle moves in that direction.This
will be a random motion.

\newpage

Einstein gave the correct quantitative explanation of Brownian
motion.He gave a formula for the Brownian displacement during an
interval of time relating the displacement to the time interval
and the number of molecules in a cc (cubic centimeter) of water.
Later Jean Perrin experimentally verified Einstein's formula
and then used it to determine the number of molecules in a
cc of water.This helped to establish the reality of atoms and
molecules and this is the importance of Einstein's work on
Brownian motion. Even the skeptics now had to accept that all
MATTER IS MADE OF ATOMS.

\section*{Theory of Relativity}

Sometimes it is said that Einstein proved that everything is
relative and this is Relativity Theory. This is wrong.
Einstein's Theory of Relativity is about relative motion.

A momentous confrontation between Newton's dynamical laws for
matter and Maxwell's dynamical laws for electromagnetic field
developed towards the end of the nineteenth century. This
confrontation which had to do with relative motion was
resolved by Einstein in favour of Maxwell's laws. He had to
change Newton's laws. This is the story of relativity.

Galileo and Newton had removed absolute motion or absolute
rest from dynamics.We must remember that Newton's law of
motion deals with acceleration which is rate of change of
speed. The absolute value of the speed of motion does not
enter into it. Hence Newtonian dynamics retains its form
in a moving platform as long as the platform moves with
uniform speed.

How about Maxwell's dynamics? To answer this let us think
of a light wave travelling with the speed already mentioned,
since light wave propagation is a part of Maxwellian dynamics.
Again think of a moving platform such as a space vehicle and
let us observe the light wave from this moving platform. If the
platform is moving in the same direction as light, we would
have expected that light as observed from the platform will
move at a smaller speed. This was the expectation
based on our pre-Einsteinian notions of relative motion
and relative speed. But since the speed of light was
calculated from Maxwell's laws to have the value already
mentioned, it would seem that these laws do not
retain their form in the moving platform. In other words,
Maxwell's laws seemed to depend on absolute
motion, in contrast to Newton's laws.

What did experiments say? Michelson and Morley,in a famous
series of experiments showed that the speed of light
was the same even in a moving frame (actually they
used the moving Earth as the moving frame). The
conclusion is that both experiment and Maxwell's
theory demanded that the speed of light did not
depend on absolute motion; it is the same whether
you observe it from a stationary or moving platform.

Einstein took this as the basic premise and modified
space and time so that the constancy of the speed of
light could be maintained even when the platform moves.
In this modified view of space and time Maxwellian
dynamics retains its form even when the platform moves.
But then Newtonian dynamics will no longer retain its
form when the platform moves.In other words, Newtonian
dynamics was not consistent with this new view of space
and time and had to be modified. Einstein did this and
the consequence was $E= mc^{2}$ with all its terrible
aftermath.

The view of space and time that Einstein introduced
was radical. In the words of Herman Minkowski who
was Einstein's teacher, "Henceforth space by itself
and time by itself are doomed to fade away into mere
shadows and only a kind of union of the two will
preserve an independant reality." This space-time
union has become the arena for all developments in
Modern Physics.This is the revolutionary change
unleashed by Einstein's Special Theory of Relativity
in 1905.

Einstein could not stop with Special Relativity. Remember
Newtonian dynamics had another important component, Newton's
law of gravitation. This law also had to be brought into
consistence with the new view of space-time. This
took 10 years of hard work by Einstein, but he finally
succeeded in 1915. In that year his General Theory of
Relativity (which has been called the supreme triumph
of the human intellect) was born. In this theory gravitation
is nothing but curved space-time!

\section*{Photoelectric Effect and Quantum Theory}

Of the two ingredients of classical physics, matter was known
to be made of particles (atoms) while field was continuous.
But in Quantum Theory, field also became discrete or discontinuous.

Quantum Revolution began in 1900 when Planck introduced quanta
for describing heat radiation. in 1905 Einstein took it one step
further through his light-quantum hypothesis: light is also made
of quanta (photons).Using this he could successfully explain
photoelectric effect in which light incident on a metal ejects
electrons.

The full development of Quantum Theory required many more steps
spanning almost the whole of the first half of the 20th
century. Some of these are Niels Bohr's construction of his model
of the atom, Satyendra Nath Bose's discovery of Quantum Statistics
for photons and de Broglie's brilliant idea of matter waves.Finally
in 1924 Heisenberg and Schrodinger arrived at the definitive
formulation of Quantum Mechanics. Dirac in 1928 extended it to
be consistent with Special Relativity and created Relativistic
Quantum Mechanics.

Matter waves restored the symmetry between matter and field. The
continuous electromagnetic field had acquired a discrete particle
aspect through the quanta of Planck and Einstein. Particulate
matter acquired a continuous wave or field aspect through de Broglie's
matter wave concept. Thus both matter and electromagnetic field now
show a dual nature; both are discrete and continuous!

Quantum Mechanics of Heisenberg and Schrodinger dealt with matter.
Construction of the corresponding dynamics for the field took
much longer. Dirac took the first major step in quantising the
electromagnetic field but its final definitive formulation was
achieved only during 1945-50 through the work of Feynman, Schwinger,
Tomonaga and Dyson. This is Quantum Field Theory and it describes
the relativistic quantum dynamics of both matter and field.

In the development of Quantum Theory Einstein's role was not over
with his 1905 contribution. He was the first to successfully apply
Quantum Theory to solids and explain their specific heat. He
generalized Bose's photon statistics to what is now called Bose-
Einstein Statistics.

His subsequent interventions in the history of Quantum Theory
took the form of a strong critique of the theory. His debates
with Niels Bohr and later what has come to be known as the
Einstein-Podolsky-Rosen paradox deal with issues which have not
been resolved yet.These concern the highly counter-intuitive
physical notions embedded in Quantum Mechanics and the dabate on
these started by Einstein is still on.

\section*{The Present Problem}

Special Relativity and Quantum Mechanics together gave birth to
Quantum Field Theory which has become the language for present-day
fundamental physics. Quantum Field Theory which was originally
constructed for describing Quantum Electrodynamics, has now been
found to correctly describe the quantum dynamics of the Weak and
Strong forces too, these being the forces that operate within the
atomic nuclei.This theory of the subnuclear forces is based on
an important generalisation of Maxwell's laws. This
is a success story of the last half of the 20th century.

On the other hand, the marriage between General Relativity and
Quantum Mechanics has not been possible. Gravitation which gets
incorporated into the very fabric of space-time in Einstein's
General Relativity has resisted all attempts at being combined
with the quantum world. Hence Quantum Gravity has become the
most fundamental problem of the 21st century. That is the reason
for the rise of String Theory, for it promises to be a theory
of Quantum Gravity. But this is not yet proved and the quest is on.

\section*{The Bomb and Beyond}

In any discussion of Physics in the 20th century, the topic of
nuclear energy cannot be omitted. It was inevitable that during
their inward bound journey to the inside of the atom, physicists
would encounter the tremendous energy locked up inside the atomic
nucleus. And they devised the means to release it. But it was to
the eternal shame of the physicists that it was first released for
the destruction of humanity. The genie which was let out of the
bottle more than half a century ago, is still at large. We have
yet to put it back into the bottle.

Einstein was keenly aware of the catastrophe that confronted the
world. After nuclear bombs were dropped in Japan in August 1945,
he said " The only salvation for civilization and the human race
lies in the creation of a world government, with security of
nations founded upon law."

Until his death in 1955 he strove hard for stopping the
nuclear lunacy and replacing violence by a superior moral force.
On several occasions he expressed his views in the following way:
" Mankind can be saved only if a supranational system based
on law is created to eliminate the methods of brute force."
"The problem of bringing peace to the world on a supranational basis
will be solved only by employing Gandhi's method on a large scale."
"What ought the minority of intellectuals to do against evil?
Frankly I can see only the revolutionary way of noncooperation
in the sense of Gandhi's."

His words about Mahatma Gandhi are powerful: " Generations to come
will wonder whether such a man in flesh and blood walked on the
earth."

Einstein's creed can by summed up by what he said: " Science
without religion is lame, religion without science is blind."
He lived by a deep faith - that there are laws of Nature to be
discovered. His lifelong pursuit was to discover them. His
realism and optimism are illuminated by his famous remark
"Subtle is the Lord, but malicious he is not."





























