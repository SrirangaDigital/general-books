\chapter{Manpower for fundamental physics experiments}\label{chap18}

\Authorline{G Rajasekaran}
\addtocontents{toc}{\protect\contentsline{section}{{\sl G Rajasekaran}\smallskip}{}}

\authinfo{Institute of Mathematical Sciences, Chennai 600113\\ and Chennai Mathematical Institute, Siruseri 603103}



  

Let me start by mentioning a few major experimental projects 
in fundamental physics that are seriously considered
for starting in India.

\section*{The experimental projects}

1.A neutrino oscillation experiment involving a gigantic
(50 Kton) magnetised iron detector to be mounted inside
a huge cavern of the India-based Neutrino Observatory(INO)  
that has to be dug inside a mountain in Theni District,Tamil 
Nadu. This detector will be even larger than the huge detectors 
which are taking data at the Large Hadron Collider (LHC) at CERN, 
Geneva. So our students will be able to work in the construction 
of such a detector and use it, right here in India.

2.Search for neutrinoless double beta decay(NDBD) which
is actually the most fundamental of all neutrino experiments
since it will tell us about the nature of the neutrino itself
(whether it is a Dirac or Majorana particle). This also
will be installed in the INO cavern.

3.A low-energy neutrino experiment called LENS (Low Energy
Neutrino Spectroscopy) which will detect the pp neutrinos
from the Sun. These are the most abundant neutrinos from
the Sun (amounting to more than 90 percent of the solar neutrinos) 
and have not been detected so far. Hence LENS has the capability
of revolutionizing solar neutrino physics once again. This
experiment which will use Indium-loaded liquid scintillator 
will be mounted either in the INO cavern (or another
existing cavern or tunnel inside a mountain).

4.Astronomers have discovered that most of the matter in
the Universe is not the kind we are familiar with. It is
called Dark Matter since it does not emit or absorb light.
Although this discovery has already been made, nobody knows
what this dark matter is and only physicists can discover
that. A dark matter experiment will be mounted in INO cavern
(suitably extended). This has been called DINO (Dark matter
at INO) and this will be preceded by a smaller experiment
at a shallower depth. 

5.A neutron-antineutron oscillation experiment in India
is being thought of. In fact a Workshop to discuss this
was held in Kolkata last year. Such an experiment
will put India back in the world scene for the search
for baryon number violation which has not yet been observed.

6. A gravitational wave detector is being planned to be
set up in India. This is the goal of the INDIGO (Indian 
Gravitational Wave Observatory) project. Although this
is of great importance in astronomy, direct detection
of gravitational waves predicted by Einstein's General
Relativity is also an important area of fundamental 
physics. So we include it here.

\section*{Technology}

Although all these projects concern high energy physics,
nuclear physics or astronomy, the technology and material 
science component involved in all of them must not be 
lost sight off. The RPC-based magnetised detector to be
set up in the INO cavern will require 30,000 sensitive
detector elements and 3 million electronic
channels. The NDBD, LENS and DINO will
need sophisticated cryogenics, chemistry, semiconductor
crystal fabrication and other techniques of modern
material science. Construction of gravitational wave
detector will require sophistication at an unimaginable 
level.Hence execution of the above fundamental
physics projects will lead to the development of state-of-
the-art infrastructure in all these fields. This important 
off-shoot of "aiming for the Moon" must be kept in mind.     

But where is the manpower for all this? None of the above
projects can succeed unless the crucial problem of
manpower is solved. 

\section{Manpower creation}

A few suggestions are offered here towards this aim.

1. Much of the manpower for the Department of Atomic Energy
came from the innovative Training School started by Homi Bhabha 
in 1957. Inspired by this, INO started its own training
programme 5 years ago. The scope of this programme could be 
enlarged to cover the other experiments.

However we need more people at the faculty level to train
these young students. 
  
2. We have to contact those bright young Indian scientists
who went abroad in search of fertile pastures and lure
them back with assurance of those fertile pastures here.
There are many good experimental physicists who would
be willing to return. A high-level drive has to be
undertaken to achieve this. Heads of scientific institutions 
must go with "a blank cheque" during their travel abroad and  
offer jobs straightaway when they meet deserving candidates.

That is what Bhabha did in the 1950's and 60's and that
is how the School of Mathematics, the Cosmic Ray
group, the Radio Astronomy group and the Molecular
Biology group, all at TIFR, were built by K Chandrasekaran,
Bernard Peters, Govind Swarup and Obaid Siddiqui, all of
whom Bhabha brought. (Of course the times were very
different then, but still those glorious examples can
light our path even now.) Recently the Chinese followed this
path very successfully.

There are many reputed Indian physicists abroad who can
identify good candidates and help us in such a recruitment
drive (the inverse brain-drain).

3. Where are these new recruits to be placed? All
of them need not and should not go to the established
institutions such as TIFR, IISc or SINP.
We must persuade the IITs, IISERs and the Central
Universities to recruit the bulk of the returning
experimental physicists. We have already got positive
response from the heads of a few of these institutions
and we must continue to try and extract similar response
from the other institutions also. IISERs and NISER
have been founded especially to attract bright youngsters
into science. What better way to attract than to show
them the possibility of joining front-ranking
fundamental science experiments in India? The bright
students in their fourth year must be put into
project work connected to one of the experiments in
the list above.

4. Many privately funded engineering institutions have come up
in the country. Unfortunately most of them are money-making
institutions rather than the money-spending variety. We need
the latter. Academic institutions must earn, not money, but
the reputation of excellence in the advancement of knowledge. 
Nevertheless one must not write them off. Recently
some of them are showing promise; they are capable
of aiming for excellence. We may be able to induct good 
science and engineering faculty into them.

5. I now come to the most important aspect of the
manpower problem. It is a sad fact that because of the
continuing neglect of our more-than-four-hundred universities
and thousands of colleges, these languish in academic
slumber. Since most of our student-power lies in these
institutions, it is no wonder that all our plans for
major scientific projects suffer from lack of manpower.
So it is clear that mobilizing the universities and
coupling them to the National Science Projects is the
only correct way forward. It will remedy both these ills.

However this is a gigantic task. I will restrict myself to 
three brief points. Because of the importance of this 
problem, I suggest that DST should confer with UGC and
come out with innovative solutions. Second, we must try to
influence the universities in the physical as well as
intellectual neighborhood of each of us and persuade them to
facilitate the participation of their students in a
major scientific project. Third, in many of the university
departments, a large fraction of the faculty strength has
been kept vacant for many years. These must be filled with
experimenters who can contribute to one of the experiments
in the list above.

There may be many more ideas, but what is needed is action.
