\chapter[Unification of Weak and Electromaghetig Interactions]{Unification of Weak and Electromaghetig Interactions}\label{chap25}

\Authorline{}
\addtocontents{toc}{\protect\contentsline{section}{{\sl G. Rajasekaran}\smallskip}{}}
\authinfo{Tata	 Institute of Fundamental Research, Bombay-5}

\section{Introduction}

The similarity in structure of the electromagnetic and: weak
interactions has, in the past, prompted several attempts at unification
of these two fundamental interactions. Recently much interest has been
aroused in a theory which achieves this unification by extending the
abelian gauge Invariance of the electromagnetic theory to non-abelian
gauge invariance, the noneebelianness being due to the charge-changing
character of the weak currents. 

Gauge invariance can be defined as the invariance under space=
time-dependent transformations of the group ard so ney be called
general invariance to be distinguished from special invariance
ig the usual invartance under constant transformations. This general
invariance is possible only if there exist corresponding vector fields,
which in the case of non-abelian symmetry are the self-coupled Yang-
 Mills fielde$^{1}$, Such a generally-invariant model unifying the weak
and electromagnetic interactions was first constructed by Weinberg$^{2}$
in 1967. 

An important ingredient of Welnberg's model is the idea of
apontaneous breakdown of the general invariance. Without this, the
vector fields will be maesiegs and the theory will lack physical relevance. The thing that prevented a succesful application of the idea of spontaneous breakdown of symmetry in praticle physics had been the Goldston Theorme, namely the appearance of a massless ????????????????? advance was made by Higgs$^{3}$ in 1964 when he showed that the
"Theorem" can be circumvented when the symmetry involved is a general
syumetry. It is this discovery which enabled Weinberg to construct
the generally invariant wmified theory. 

Once we achieve this unification, we get a bonus, namely a
renormalizable theory of weak interactions. Although renormalizabi Hi ty
of this theory was conjectured earlier, it is only after a paper by
Hooft$^{4}$ last year, that one is reasonably hopeful of ite validity.
This has generated much enthusiasm in the subject since the construc
tion of a renormalizable vector-axial vector theory of weak interac
tions hag been one of the fundamental problems in Particle Payaics. 

\subsection*{The SU(2) $\times$ U(1) Lepton Model}

The integrated charges corresponding to the weak and electromagnetic currents of the known leptons satisfy a 90(2) (\&) U(1) algebra.
So, one may take the general symmetry to be SU(2) $\times$ U(1) and intro- .
duce a Yang-Mills field $\vec{W}_{\mu}$ and an abelian-gauge vector field $B_{\mu}$
sorresponding respactively to the SU(2) and ua parts. This is.
Veinberg's noder$^{2}$.

Let us call the SU(2) and U(1) generators, the "weak isospin"
$\vec{k}$ and the week hypercharge" $x$ respectively. The electric charge is
given by $Q= K_{3} + x$. The leptons end their quentwe numbers are then As
ne following :
\begin{center}
\begin{tabular}{ccc}
 & k & x\\
$L\mod \frac{1-r_{5}}{2} \begin{pmatrix} K \\ X \end{pmatrix}$ & $\frac{1}{2}$ & $-\frac{1}{2}$\\[0.4cm]
$R\mod \frac{1+r_{5}}{2} e$ & $0$ & $-1$
\end{tabular}
\end{center}


The case of $\nu_{\mu}$ and $\mu$ exactly parallel aml so Wil be ignored.

The ????? which is invariant under the general $SU(2) \times U(1)$  transformation is 
\begin{multline*}
\mathcal{L} = \frac{1}{4} (\partial_{\mu} \vec{w}_{v}- \partial_{v}\vec{w}_{\mu} + g \vec{w}_{\mu} \times \vec{w}_{v})^{2} + \frac{1}{4} (\partial_{\mu} B_{v}-\partial_{v}B_{\mu})^{2} + \bar{L}r_{\mu} D_{\mu}L\\
+\bar{R}r^{\mu}D_{\mu} R + \frac{1}{2} |D_{\mu} \phi|^{2} + f_{e}(\bar{L}\phi R + \bar{R}\phi^{+}L) + v(\phi)\tag{1}
\end{multline*}
where $D_{\mu} = \partial_{\mu}-ig \bar{W}_{\mu}\cdot \bar{k}-ig'B_{\mu}x$. The Higgs scalar multiplet $\phi$ is choser to have the quantum number $k=\frac{1}{2}$ and $x= \frac{1}{2}$ so that it charge-structure is given by $\begin{pmatrix} \phi^{+}\\ \phi^{1}\end{pmatrix}$. $V(\phi)$ is an invariant function of $\phi$ containing quadration and quartic terms and the signs of these terms are so chosen that the vacum expection value $\langle \phi \rangle$ is not zero.

Because of the invariance requriment the ????? cannot contain any quadratic terms in the vector boson of Lepton fields. So there are no masses to start with. But, because of spontaneous breakdown of the symmetry, implemented by the non-vanishing $\langle \phi \rangle$, mass terms are generated.

To see this in a simple manner, one may first substitute








