\chapter[Is There A Future for High Energy Physics?]{Is There A Future for High Energy Physics?}\label{chap11}

\Authorline{G.Rajasekaran}
\addtocontents{toc}{\protect\contentsline{section}{{\sl G.Rajasekaran}\smallskip}{}}

\authinfo{Institute of Mathematical Sciences Madras 600 113}

\section*{Abstract}

The present state of High Energy Physics is critically examined.
In spite of the spectacular success of the standard model, there is a serious crisis facing the
field. Since the Planck scale is now recognized to be the true fundamental scale of
physics, the importance of research on new methods of acceleration that can take
us to superhigh energies is emphasized.

The major events which culminated in the construction of the Standard Model of High
Energy Physics are presented in Table 1 in chromological order. Using nonabelian gauge
theory with Higgs mechanism, the electroweak (EW) theory was already constructed in
1967, although it attracted the attention of most theorists only after another four years,
when it was shown to be renormalizable. The discovery of asymptotic freedom of non
abelian gauge theory and the birth of QCD in 1973 were the final inputs that led to the
full standard model.


On the experimental side, the discovery of scaling in deep inelastic scattering (DIS)
which led to the asymptotic free QCD and the discovery of the neutral current which
helped to confirm the electroweak theory can be regarded as crucial experiments. To
this list, one may add the polarized electron-deuteron experiment which showed that
$SU(2)_{L},	XU(1)$ is the correct gauge group for electroweak theory, the discovery of gluonic
jets in electron-positron annihilation confirming QCD and the discovery of $W$ and $Z$ in
1983 that established the electroweak theory. The experimental discoveries of charm, $\tau$
and beauty were fundamental for the concrete 3-generation standard model.


However, note the blanks after 1973 and 1983 on the theoretical and experimental sides
respectively. Theoretical physicists have been working even after 1973 and experiments
are being done even after 1983. But the tragic fact is that none of the bright ideas
proposed by theorists in the past 20 years has received any experimental support and
none of the experiments done in the past 10 years has led to any discovery. Even the
famous $W$, $Z$ discovery was only a confirmation of a theory proposed 16 years earlier.\footnote{The evidence for top quark recently announced$^{1}$ is important, but top is expected within the statndard model.}

It. is clear that if such a situation persists for long, it may become difficult to continue
to be optimistic about the future of high energy physics.
\setcounter{chapter}{0}
\begin{center}
\begin{table}[h]
\caption{History of the Standard Model}
\begin{tabular}{|l|l|}
\hline
{\textbf Theory} & {\textbf Experiment}\\
\hline
1954 Nonabeliab & \\
gauge fields & \\
1960 & 1960\\
1964 Higgs mechanism & \\
1967 EW Theory & 1968 Scaling in DIS \\
1970 & 1970\\
1971 Renormalizability of EW Theory& \\
1973 Asymptotic freedom & 1973 Neutral current\\
$\rightarrow$ QCD & \\
 & 1974 Charm\\
 & 1975 $\tau$-lepton\\
 & 1977 Beauty \\
 & 1978 $\vec{e} d$ expt\\
 & 1979 gluonic jets\\
 1980 & 1980\\
 & 1983 W, Z\\
 1990 & 1990\\
 \hline 
\end{tabular}
\end{table}
\end{center}

It may be argued that the current lean period of discoveries in high energy physics is
just a natural consequence of the spectacular success achieved in the past decades. The construction of the standard model is certainly a watershed. In the standard model we
now have a theory for {\it all} that is known in high energy physics. So, there is nothing more
to do!


Clearly the above sentiments are quite detrimental to the progress of of any scientific
field. In any case, there are too many loopholes in the standard model to be satisfied
with it, the biggest of these being the omission of gravitation, the most important force
of nature.


There are still many interesting questions and unsolved problems within the standard
model : Higgs and symmetry breaking, QCD, neutrinos, CP etc and there may be other
surprises and discoveries (supersymmetry, compositeness etc) which may take us beyond
the standard model. Nevertheless, an examination of the current scene reveals a serious
crisis facing high energy physics - namely, the widening gap between theory and experi-
ment. The primary factor that is responsible for this crisis is the recognition that quantum
gravity is the next frontier of high energy physics and that the true fundamental scale of
Physics is the Planck mass $M_{p} ~ 10^{19}$ GeV, which is the scale of quantum gravity. As a
result, all of present-day experimental activity in high energy physics has been reduced to
zero-energy physics. On the other hand, enchanted by the theories at the Planck mass,
many active theorists are drowning themselves in the depths of mathematics. Physics is
an experimantal science. Hence this gap between theory and experiment will ultimately
spell the ruin of high energy physics.

Can this energy barrier separating experiment and theory be surmounted?
Are con-trolled experiments at $M_{p}$, possible? Can Planck energy be obtained in the laboratory?
The future of high energy physics hangs on the answers to these questions.

Already, grand unification had pushed the fundamental scale to $10^{15}$ GeV and quantum
gravity takes it to $10^{19}$ GeV. Preoccupation with such superhigh energies without the
sobering control of experiments is bound to lead to Metaphysics. \textit{The pre-eminence of
experiments in physics must be reestablished}. So it is imperative that physicists and
technologists put their minds together to solve this crucial problem of the energy barrier.
After all, there is no Law of Nature (such as the Second Law of Thermodynamics) which
forbids the attainment of such energies in the laboratory\footnote{If there is such a Law, prove it!}. Human ingenuity knows no
bounds and a method will be found, to reach the superhigh energies so that \textit{controlled
laboratory experiments} can be done to test quantum gravity, superstrings or even theories
beyond.

How do we reach Planckian energies? Can we envisage a Planckian accelerator? Before
we answer this question we examine a few indirect methods.


\section*{\textit{(i) Cosmology and Early Universe}}


If current ideas in Cosmology
and Astrophysics are correct, then early universe pro-
vides us with a High Energy Physics Laboratory where particle energies were not limited by any budget cuts or other restrictions. Hence,
it is thought that all our theories of High
Energy Physics are testable in principle by appea
ling to events in the early universe. We
seem
to have come a long way from Landau’s dictum
: ``Astrophysicists are often wrong
but seldom in doubt”.


However, we know of only one universe and
the events presumably occured only once,
that
too, quite a long time ago. Modern Science owes
its existence to the advent of
repeatable experiments under controlled conditions
whereas History provides only a single
sequence of events. History cannot be a substitute
for Science. Cosmology cannot provide
cruci
al and definitive tests for fundamental theor
ies of physics. On the other hand, high
energy physics can definitely be applied to the
study of early universe. Laws of physics
inferred from and tested in laboratory experime
nts can and must be applied to the study
of the history of the universe. \textit{No definitive law
of physics can be inferred from speculative
theories of the beginning of the universe}.


In other words, it’ is advocated that the only
healthy traffic between High Energy
Physics and Cosmology is a one-way-traffic :
$$
High Energy Physics \rightarrow Cosmology.
$$


\newpage

\section*{\it (ii) Nonaccelerator Particle Physics}


Although the characteristic scale of weak interactio
ns is 100 GeV, this did not deter
physi
cists from learning much of weak intereaction
phenomenology through experiments
at the available lower energies ever since the
discovery of beta decay. Similarly even
theories with characteristic scales at $10^{15} - 10^{19}$ GeV
are expected to leave their signals
(albeit weak) in the lower-energy phenomena.
Proton decay, neutrino masses and
mixing,
neutrino oscillations, double $\beta$ decay and fifth force
are such signals and experiments ded-
icated to the study of these phenomena provide
us with indirect but important windows
on the superhigh energy scales.


The importance of deep underground laboratori
es for nonaccelerator physics experi-
ments is well recognized. In this context, we must
record here the unfortunate closure of
the deep mine at Kolar which was an important asset
for this country, especially because
of the absence of high energy accelerators in this
part of the world. An excellent oppor-
tunity to develop this facility into a first-rate
underground laboratory for nonaccelerator
particle physics has been lost.


In spite of the importance of non-accelerator
particle physics experiments, these must
be regarded as only our first and preliminary
attack at the unknown frontier. These
experiments can give us only indirect evidence
on the physics at superhigh energy scales.
Just as the real nature of the weak force, namely
that it is a gauge froce mediated by
gauge bosons, becomes manifest only at 100 GeV, in
the same way, the real nature of the
unknown physics beyond the standard model
will become clear only by experiments at
the superhigh energy scales

\section*{\it (iii) Monopoles}


Grand unified theories predict the existence of magnetic
monopoles with masses of
the order of $10^{16}$ GeV. If such superheavy monopoles
exist in nature and if monopoles
and antimonopoles can be caught in sufficient numbers which
could be kept in separate
“bottles”,
then
by letting the monopole
and antimonopole
collide and annihilate each
other, we can create the fireballs with superhigh energies (~ $10^{16}$
GeV) right here in the
laboratory :
\newpage



$$
M + \bar{M} \rightarrow 10^{16} GeV.
$$



However, success of this venture obviously depends
objects!

\section*{\it (iv) Planckian Accelerator}


None of the above avenues - historical research on the early
universe, nonaccelerator
experiments and monopole search can compare with dedicat
ed experiments in the lab-
oratory directly bearing on the superhigh Planckian energies.
Physicists cannot remain
satisfied with indirect attacks on the superhigh-energy frontier
. Planck energy must be
attained in the laboratory.

{\it This must be regarded as the most important problem in High
Energy Physics}. A
breakthrough in the discovery of a new mechanism of acceleration
which can take us to
Planck energy will advance High Energy Physics much more than all
the beautiful theories
at Planck energy which theorists are constructing. This will be a revoluti
on.

Some of the ideas being pursued are : lasertron, wakefield acceleration,
switched power
linac, collective accelerator, laser-driven grating linac, inverse free
electron laser, inverse
Cerenkov acceleration, plasma accelerator, laser beat-wave method
\footnote{C. Joshi and his colleagues$^{2}$ at the University of California, Los Angeles have succeeded in using this method to produce an accelerating electric field of 2.8 Gev per meter, whic is the largest coherent man-made accelerating field yet produed, and 30 times larger than the limit imposed by radiofrequency breakdown in conventional accelerators.} etc. What we need
are a hundred crazy ideas. May be, one of them will work.

A word about India. We need not feel disheartened by our lag
in accelerator tech-
nology. Perhaps there is no point in repeating all the well-tried
accelerator mechanisms
(which may be irrelevant as far as the Planckian accelerator
is concerned). We may be
able to leap-frog on accelerator technology!


A look at the past history of accelerators will show that the
growth of the energy of
the accelerators over the years has been phenomenal$^{3}$. In an overall
sense, the energy of
the accelerators increases by a factor 10 in every 6 years. We
interpret this exponential
growth of the energy as an optimistic sign for the future of High
Energy Physics. Pes-
simists will point out that the required money as well as the dimensi
ons of the accelerator
also grow exponentially. This is true for conventional method
s of acceleration. What we are envisaging are newer methods and newer technologies which will overcome these
limitations.

For machines employing the same technique of acceleration, the growth lines have a
shape that is not’ an exponential, but taper off. It is only the overall growth including
all types of accelerators that is an exponential with the slope given above. This only
shows that the growth of accelerators of a given kind has generally slowed down after
the associated technology has matured and emphasizes the importance of new ideas of
acceleration at every stage, in order to go further.


By an optimistic extrapolation of this exponential growth, one can show$^{4}$ that Plack
energy $10^{19}$ GeV in the c.m. system can be reached in the year 2086 A.D. If this scenerio
is realized, the c.m. energies available for High Energy Physics experiments in future
will be as in Table 2.
So, by 2062, we will produce the $X$ bosons
(the leptoquarks)
and other objects of the grand unified theories and by 2086, higher dimensions of space-
time will no longer be hidden, superstring theory will be directly tested and quantum
gravity experiments will be done in the laboratory! Of course, entirely new things not
contemplated by any theorist so far may be discovered.

\vspace{.2cm}

\begin{table}[h]
\caption{Progress towards Planck energy}
\begin{tabular}{|c|c|c|c|c|c|c|}
\hline
Year & 1990 & 196 & 2002 & 2020 & 2062 & 2086\\
\hline
Energy in Gev & $10^{3}$ & $10^{4}$ & $10^{5}$ & $10^{8}$ & $10^{15}$ & $10^{19}$\\
\hline
\end{tabular}
\end{table}



The period we have to wait may look too long. But, if one compares it with the
time which elapsed between our first glimpse of the weak decays (Becquerel’s discovery
of radioactivity in 1896) and the production and identification of the carrier of the weak
force in the laboratory (1983), it is not much longer.

To put the whole thing into proper perspective, let us contemplate Maxwell’s equations
for electrodynamics
\begin{align}
\overrightarrow{\bigtriangledown} \cdot  \overrightarrow{E} &=4 \pi \rho\\
\overrightarrow{\bigtriangledown} \times \overrightarrow{E} + \frac{\partial \overrightarrow{B}}{\partial t} &=0\\
\overrightarrow{\bigtriangledown} \cdot  \overrightarrow{B} &=0\\
\overrightarrow{\bigtriangledown} \times \overrightarrow{B} - \frac{\partial \overrightarrow{E}}{\partial t} &= 4 \pi \overrightarrow{j} 
\end{align}
and compare them with the dynamical equations of the standard model :
\begin{align}
\overrightarrow{\bigtriangledown} \cdot  \overrightarrow{E}_{i} + \ldots &=4 \pi \rho_{i}\\
\overrightarrow{\bigtriangledown} \times \overrightarrow{E}_{i} + \frac{\partial \overrightarrow{B_{i}}}{\partial t} + \ldots &=0\\
\overrightarrow{\bigtriangledown} \cdot  \overrightarrow{B_{i}}+ \ldots &=0\\
\overrightarrow{\bigtriangledown} \times \overrightarrow{B_{i}} - \frac{\partial \overrightarrow{E_{i}}}{\partial t} + \ldots &= 4 \pi \overrightarrow{j_{i}} 
\end{align}

where 2 goes over 1 to 12 corresponding to the four electroweak gauge fields $\gamma, W^{+}, W^{-},Z$
and the eight gluons. The dots in eqs(5-8) refer to the complications arising from the
nonabelian nature of the gauge fields of the standard model.


All the accelerators so far are based on electrodynamics. As compared to electrody-
namic technology, the technology of the standard model is in a very primitive stage ; we
are perhaps at a level comparable to the study of electricity by rubbing amber on wool!
We now know that electrodynamics does not stand alone ; it is only a part of the unified
electroweak dynamics. The deeper implications of the electroweak unification may be as
profound and far reaching as those of Faraday’s unification of electricity and magnetism
or of Maxwell’s unification of electrodynamics and optics. Our understanding of QCD is
at an even more primitive stage, because of colour confinement. But chromodynamics will
be mastered and chromodynamic technology also will come. Electrodynamic technology
led to acceleration of particles upto TeV energies. By releasing the forces of the standard
model and putting them to work, the goal of acceleration upto Planckian energies may be
achieved even earlier than the prediction above. “Prediction is a difficult art, especially
when it concerns the future”.


\begin{thebibliography}{99}


\bibitem{1}. F.Abe etal, Fermilab preprint (1994), Fermilab-Pub-94/097-E.

\bibitem{2}. M.Everett etal, Nature 368 (1994) 527.

\bibitem{3}. C.H.Llewellyn Smith, Oxford preprint (1986) 34/86.

\bibitem{4}. G.Rajasekaran, in {\textit Proc. of VII] HEP Symposium (Vol II)} ed. M.K.Pal and G.Bhattacharya
(SINP, Calcutta, 1986) p.399.\\
G.Rajasekaran, in \textit{Particle Phenomenology in the 90’s} ed. A. Datta etal (World
Scientific, 1991) p.1.
\end{thebibliography}





