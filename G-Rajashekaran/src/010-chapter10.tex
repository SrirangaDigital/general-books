\chapter{High Energy Physics in 2014 and its Future }\label{chap10}

\Authorline{G. Rajasekaran }
\addtocontents{toc}{\protect\contentsline{section}{{\sl G. Rajasekaran}\smallskip}{}}

\authinfo{email: graj@imsc.res.in\\  Institute of Mathematical Sciences, Chennai - 600 118, India\\
Chennai Mathematical Institute, Chennai - 603 108, India }
\smallskip

\section*{Abstract}
After a brief history, we focus on the present status of HEP
and its possible future. Ideas to ensure a healthy growth of HEP in India are
discussed. This involves a few major experimental projects in fundamental
physics. None of these projects can succeed unless the crucial problem of
manpower is solved. A few suggestions are offered towards this aim. 

\section*{Hundred Years of Fundamental Physics} 

The earlier part of the 20$^{th}$ century was marked by two revolutions that
rocked the Foundations of Physics:
\begin{itemize}
\item[1]. Quantum Mechanics
\item[2]. Relativity 
\end{itemize}

Quantum Mechanics became the basis for understanding ATOMS, and
then, coupled with Special Relativity, Quantum Mechanics provided the
framework for understanding the atomic nucleus and what lies inside. ‘This
history can be summarized as follows: 

\section{Inward Bound} 

\begin{center}
\begin{tabular}{c@{\hspace{.1cm}}c@{\hspace{.1cm}}c@{\hspace{.1cm}}c@{\hspace{.1cm}}c@{\hspace{.1cm}}c@{\hspace{.1cm}}c@{\hspace{.1cm}}c@{\hspace{.1cm}}cc}
Atoms        & $\rightarrow$ & Nuclei        & $\rightarrow $ & Nucleons      & $\rightarrow$   & Quarks         & $\rightarrow$ & ?&\\
$10^{-8}$ cm &               & $10^{-12}$ cm &                & $10^{-13}$ cm &                 & $10^{-17}$ cm  &               &  &
\end{tabular}
\end{center}

This inward bound path of discovery unravelling the mysteries of matter
and the forces binding it together - at deeper and ever deeper levels - has
culminated, at the end of the 20th century, in the theory of Fundamental
Forces based on Nonabelian Gauge Fields, for which we have given a rather
prosaic name: 

\vspace{-.3cm}

\section*{The Standard Model of High Energy Physics} 


What is called High Energy Physics (HEP) is just the continuation of
the era of discoveries that saw the discovery of the electron, the discovery
of radioactivity and X rays, the discovery of the nucleus and the neutron and the discovery of cosmic rays and the positron. HEP is the front end or
cutting edge of the human intellect advancing into the unknown territory in
its inward bound journey. 

The construction of the Standard Model (SM) was complete in the 70’s.
In the next 4 decades, experimenters have succeeded in confirming every
component of the full SM. Higgs boson remained as the only missing piece.
High energy physicists searching for it in all the earlier particle accelerators
had failed to find it. So the discovery of Higgs boson of mass 126 GeV at
the Large Hadron Collider (LHC) at CERN in 2012 has been welcomed by
everybody. This is a great scientific and engineering achievement. 

Let us backtrack a little. Please allow me the indulgence of putting in a
bit of my personal history. More than 50 years ago when I started my research
in High Energy Physics (HEP) which was then called Particle Physics there
was no Standard Model. We were all groping in the dark. I had the good
fortune to witness the Standard Model being built step by step. After each
step was taken, I learnt of it with a pang of regret that I did not do it. It
was an agonizing period for me. But although I was not on the stage, I was
almost in the first row, seeing history in the making. 

But all that is old story. By 1973, what we now call Standard Model was
in place. This is now known to be the basis of almost ALL known physics
except Gravity. 

After that glorious period of early 70's, it has been a long sterile period of
almost four decades. Let me explain. During this period theorists have not
been idle, but none of their theories has seen an iota of experimental support.
Experimenters have also been busy, but all they have done in the last four
decades is only to confirm one or other component of the full Standard Model
with three generations of fermions and all their details. I am aware that these
are rather drastic statements, but they are true. Even Higgs boson is only a
part of the Standard Model. 

After that glorious period of early 70's, it has been a long sterile period of
almost four decades. Let me explain. During this period theorists have not
been idle, but none of their theories has seen an iota of experimental support.
Experimenters have also been busy, but all they have done in the last four
decades is only to confirm one or other component of the full Standard Model
with three generations of fermions and all their details. I am aware that these
are rather drastic statements, but they are true. Even Higgs boson is only a
part of the Standard Model. 

After the long sterile period, we now have at CERN, the Large Hadron
Collider (LHC) which is capable of making discoveries. It can confirm or
refute the numerous speculations that theorists have made. The day of reck oning for theoretical high energy physicists has come. That is the importance
of LHC. 


Unfortunately, inspite of the brilliant performance of LHC and its detectors, no discovery of anything beyond SM has been made sofar, altho’ many
theories beyond SM have been constructed. But it is only the beginning.
Many more years are to come. Hopefully Nature will be kind to us and LHC
will make discoveries. Our immediate situation is very positive. The LHC
machine and its array of detectors ATLAS, CMS, ALICE and LHCb are all
performing beautifully. Thousands of experimenters and theorists all over
the world working together are bound to discover something new. 


\textbf{A word about India:} More young people must be brought into HEP
(both experiment and theory). This is the right time since LHC has started
working. Many new institutions in India are opening up and we must see
that strong HEP groups are built up in most of them. India is a big country.
We must think big. No small measures or small steps will do. Our agenda is
to discover whole new worlds. 


\section*{Theoretical Speculations} 


\textbf{Grand Unification:} Grand Unification idea is more than 40 years old.
Although there are indirect evidences for its correctness such as the meeting
of the three gauge coupling constants at around 10$^16$ GeV, the crucial prediction of proton decay is still not borne out. On the other hand one can turn
the table around and claim that a spectacular success for grand unification is
the natural explanation of the longevity of the proton. As one more evidence
for unification, I would like to cite the meeting of the mixing angles in the
quark sector with those in the leptonic sector. 

\textbf{Supersymmetry:} This theoretical idea postulates the existence of a boson corresponding to every known fermion like electron and vice versa. This
is a very elegant symmetry that leads to better quantum field theory than
the one on which SM has been built. But, if it is right, we have to discover a
whole new world of particles equalling our known world; remember we took
a. 100 years to discover all the particles of the 5M, starting with the electron.
Patience is needed,  

\textbf{Technicolour:} A whole new world of strong interactions! Having lived
through the old strong interactions in the 50's and 60's without knowing
what it is, that is not my cup of tea. But if Nature had decided to repeat
her tricks, who are we to refute her? There is one point that is striking Technicolour had to be replaced by Extended Technicolour and then came
Walking Technicolor. All this has to be done to take care of one phenomenological detail or other. Are we building epicycle after epicycle? After the
discovery of Higgs, perhaps Technicolour is not that popular. But the idea
of compositeness cannot be ruled out. 

\textbf{Extra dimensions:} Again we are building a whole new world of extra
dimensions. It took us thousands of years to understand the four (three space
plus one time) dimensional world where we live. Now the theorists are constructing worlds with more dimensions added to the three plus one. Can this
be done so fast? Many of the constructions in extra dimensions again remind
us of Ptolemy: fitting phenomenological details with epicycles after epicycles. 


\textbf{Is there a Balmer formula?:} In the SM, all the 12 fermion masses are
arbitrary parameters fixed only by experiment. Perhaps one has to extend
SM to include a theory of generations for understanding the pattern of the
fermion masses. Enormous amount of theoretical work has been done to
attack this problem, but there is no memorable result. 

However in 1982, Yoshio Koide found a remarkable empirical formula: 
$$
m_{e} + m_{\mu} + m_{\tau} = \frac{2}{3}(\sqrt{m_{e}} + \sqrt{m_{\mu}} + \sqrt{m_{\tau}})^{2}
$$

which is satisfied to an accuracy of 1 part in 10$^{5}$. There does not exist any
other relation of comparable accuracy in all of HEP (except of course the
precision results calculated from QED and electroweak theory). But to this
day nobody has succeeded in deriving the Koide relation from any theory. Is
this the much-needed Balmer formula which can serve as the guide-post for
discovering the correct theory of generations? 



\textbf{Quantum Gravity and String Theory:} The biggest loophole in SM is
that it leaves out Gravity. The most successful attempt to construct quantum
gravity is String Theory, in which area Indian theorists have made fundamental contributions. But, to incorporate it into physics, one needs experiments
bearing on the Planck scale of energy which is far far beyond our present
capabilities. 



\textbf{Preons:} A brief look at the history of atoms, nucleons and then quarks
would suggest that preons must be the next natural step. Nature may really
be a never-ending layered structure. In the hey-days of S-matrix and bootstrap philosophy in the early 60’s, it was even proposed that we have reached the end of the road and no more constituent structure below the hadrons was
possible. But the subsequent development of physics has shown this to be
wrong. We now know that hadrons are made of quarks. Are quarks in turn
made of preons? Many preonic models were proposed in the past, but none is
as yet required by experimental data. Down to a distance scale of 10$^-17$" cm,
quarks and leptons behave like point particles. Nevertheless, Nature might
have already chosen one preonic model and future experiments might reveal
it! 

\vspace{.2cm}

\textbf{Dark matter:} In contrast to all the above topics, Dark matter is already
established to exist. This discovery is due to astronomers. But its nature
is left to physicists to discover. Dark matter is more abundant than visible
matter (about 4 to 5 times). Dark matter also may have all the variety and
complication of visible matter which we took 100 years to understand. So,
characterizing dark matter by one or two parameters (like the relic density
and the mass of the dark matter particle) may be far from the truth. 


\textbf{Cosmology:} If current ideas in Cosmology and Astrophysics are correct,
then early universe provides us with a HEP laboratory where particle energies
were almost unlimited. So it is believed by many among us that all our
theories of HEP can be tested by appealing to events in the early universe.
At the risk of getting a flak from many of my respected colleagues, I would
like to strike a note of caution. 


There is no doubt that the era of ``precision cosmology” dawned with the
measurement of CMBR anisotropies whose accuracy is awe-inspiring. The
recent measurement of the B-mode polarization is creating waves. We seem
to have come a long way from Landau’s dictum: 

{\centering “Astrophysicists are often wrong but seldom in doubt”.} 


However we know of only one universe and the events presumably occurred only once, that too quite a long time ago, Modern Science owes its
existence to the advent of repeatable experiments under controllable conditions whereas History provides only a single sequence of events. History
cannot be a substitute for Science. 

Cosmology cannot provide crucial and definitive tests for fundamental
theories of Physics. On the other hand, laws of physics inferred from and
tested in laboratory experiments can and must be applied to the study of
the universe and its history. In other words the only healthy traffic between
HEP and cosmology is a one-way traffic: 

$$
\text{HEP} \rightarrow \text{Cosmology.} 
$$

\textbf{New ideas on particle acceleration:} Physics is not theory alone.
Even beautiful theories have to be confronted with experiments and either
confirmed or thrown out. Here we encounter a serious crisis facing HEP. In
the next 25 years, new accelerator facilities with higher energies such is the
Linear Electron Collider will be built so that the prospects for HEP in the
immediate future appear bright. Beyond that period, the current accelerator
route seems to be closed because known accelerator methods cannot take
us perhaps beyond 100 TeV. It is here that one turns to Cosmology and
Non-Accelerator Particle Physics, such as from Underground, Underwater or
Underice laboratories. However these must be regarded as only our first and
preliminary attack on the unknown frontier. These can give only hints of new
physics. Physicists cannot remain satisfied with hints and indirect attacks
on the superhigh energy barrier. 


There are many interesting fundamental theories taking us to 10$^{16}$ to
10$^{19}$ GeV, but unless the experimental barrier is crossed, these will remain
only as metaphysical theories. Either, new ideas of acceleration have to be
discovered, or, there will be an end to HEP by about the year 2040. 


In the last 30 years, many ideas on laser-plasma acceleration are being
pursued, Using laser excitation of plasma wakefields, electrons have been
successfully accelerated to 1 GeV in 1 cm (compared to kilometre-size conventional accelerators to get similar energies). So table-top accelerators are
perhaps not far way. Maybe this will lead to breakthroughs that will help
us to cross the superhigh energy barrier. What we need are a hundred crazy
ideas. Maybe one of them will work! 


By an optimistic extrapolation of the growth of accelerator technology in
the past 70 years, one can show that even 10$^{19}$ GeV can be reached before
the end of the 21st century. But this is possible only if newer methods and
newer technologies are continuously invented, 


\section*{The experimental projects} 


I now describe a few major experimental projects in fundamental physics
that are either on already, or being seriously considered for starting in India,
or must be started. 


1. The only experimental programme in HEP that is pursued sofar in
the country is the participation of Indian groups in international acceleratorbased experiments. This is inevitable at the present stage, because of the
nature of present-day HEP experiments that involve accelerators, experimental groups and financial resources that are all gigantic in size. 


2. While our participation in international collaborations must continue
with full vigour, at the same time, for a balanced growth of experimental HEP in the country, we must have in-house activities also. Construction of
an accelerator in India, in a suitable energy range which may be initially
10-20 GeV and its utilization for research as well as student-training will
provide this missing link. 


3. As already explained, known methods of particle-acceleration cannot
take us beyond tens of TeV or utmost 100 TeV. Hence in order to ensure
the continuing vigour of HEP in the 21st century, it is absolutely essential to
discover new principles of acceleration, such as laser-plasma acceleration or
something even newer. Here lies an opportunity that our country should not
miss. (I have been stressing this at every opportunity for the past 30 years.) 


4. A multi-institutional neutrino collaboration is creating the India-based
Neutrino Observatory(INO). A neutrino oscillation experiment using atmospheric neutrinos will be performed in a gigantic (50 Kton) magnetised iron
detector to be mounted inside a huge cavern that has to be dug inside a
mountain in Theni District,Tamil Nadu. This detector will be even larger
than the huge detectors which are taking data at the Large Hadron Collider (LHC) at CERN, Geneva. So our students will be able to work in the
construction of such a detector and use it, right here in India. 


5. Search for neutrinoless double beta decay (NDBD) which is actually
the most fundamental of all neutrino experiments since it will tell us about
the nature of the neutrino itself (whether it is a Dirac or Majorana particle).
This also will be installed in the INO cavern. 


6. A low-energy neutrino experiment called LENS (Low Energy Neutrino
Spectroscopy) which will detect the pp neutrinos from the Sun. These are
the most abundant neutrinos from the Sun (amounting to more than 90 percent of the solar neutrinos) and have not been detected so far. Hence LENS
has the capability of revolutionizing solar neutrino physics once again. This
experiment which will use Indium-loaded liquid scintillator will be mounted
either in the INO cavern (or another existing cavern or tunnel inside a mountain). 


7. Astronomers have discovered that most of the matter in the Universe
is not the kind we are familiar with. It is called Dark Matter since it does not
emit or absorb light. Although this discovery has already been made, nobody
knows what this dark matter is and only physicists can discover that. A dark
matter experiment will be mounted in INO cavern (suitably extended). This
has been called DINO (Dark matter at INO) and this will be preceded by a
smaller experiment at a shallower depth. 


8. A neutron-antineutron oscillation experiment in India is being thought
of. In fact a Workshop to discuss this was held in Kolkata two years ago.
Such an experiment will put India back in the world scene for the search for
baryon number violation which has not yet been observed. 


9. A gravitational wave detector is being planned to be set up in India.
This is the goal of the INDIGO (Indian Gravitational Wave Observatory)
project. Although this is of great importance in astronomy, direct detection
of gravitational waves predicted by Einstein's General Relativity is also an
important area of fundamental physics. So we include it here. 


\textbf{Technology:} Although all these projects concern high energy physics,
nuclear physics or astronomy, the technology and material science component
involved in all of them must not be lost sight off. Building an accelerator
needs accelerator engineers; discovering new principles of acceleration needs
laser and plasma physicists. The RPC-based magnetised detector to be set
up in the INO cavern will require 30,000 sensitive detectors elements and 3
million electronic channels. The NDBD, LENS and DINO will need sophisticated cryogenics, chemistry, semiconductor crystal fabrication and other
techniques of modern material science. Construction of gravitational wave
detector will require sophistication at an unimaginable level. Hence execution of the above fundamental physics projects will lead to the development
of state-of-the-art infrastructure in all these fields. This important off-shoot
of ”aiming for the Moon” must be kept in mind. 


\vspace{-.3cm}

\section*{Manpower creation} 


But where is the manpower for all this? None of the above projects can
succeed unless the crucial problem of manpower is solved. A few suggestions
are offered here towards this aim. 

1. Much of the manpower for the Department of Atomic Energy came
from the innovative Training School started by Homi Bhabha in 1957. Inspired by this, INO started its own training programme 6 years ago. The
scope of this programme could be enlarged to cover the other experiments.
However we need more people.at the faculty level to train these young students. 

2. We have to contact those bright young Indian scientists who went
abroad in search of fertile pastures and lure them back with assurance of
those fertile pastures here. There are many good experimental physicists who
would be willing to return. A high-level drive has to be undertaken to achieve
this. Heads of scientific institutions must go with ”a blank cheque” during
their travel abroad and offer jobs straightaway when they meet deserving
candidates. 


That is what Bhabha did in the 1950’s and 60’s and that is how the School
of Mathematics, the Cosmic Ray group, the Radio Astronomy group and the
Molecular Biology group, all at TIFR, were built by K Chandrasekaran, Bernard Peters, Govind Swarup and Obaid Siddiqui, all of whom Bhabha
brought. (Of course the times were very different then, but still those glorious
examples can light our path even now.) Recently the Chinese followed this
path very successfully. There are many reputed Indian physicists abroad who
can identify good candidates and help us in such a recruitment drive (the
inverse brain-drain). 


3. Where are these new recruits to be placed? All of them need not and
should not go to the established institutions such as TIFR, IISc or SINP. We
must persuade the IITs, IISERs and the Central Universities to recruit the
bulk of the returning experimental physicists. We have already got positive
response from the heads of a few of these institutions and we must continue
to try and extract similar response from the other institutions also. IISERs
and NISER have been founded especially to attract bright youngsters into
science. What better way to attract than to show them the possibility of
joining front-ranking fundamental science experiments in India? The bright
students in their fourth year must be put into project work connected to one
of the experiments in the list above. 


4. Many privately funded engineering institutions have come up in the
country. Unfortunately most of them are money-making institutions rather
than the money-spending variety. We need the latter. Academic institutions
must earn, not money, but the reputation of excellence in the advancement
of knowledge. Nevertheless one must not write them off. Recently some of
them are showing promise; they are capable of aiming for excellence. We
may be able to induct good science and engineering faculty into them. 


5. I now come to the most important aspect of the manpower problem.
It is a sad fact that because of the continuing neglect of our more-thanfour-hundred universities and thousands of colleges, most of these languish
in academic slumber. Since our student-power lies in these institutions, it
is no wonder that all our plans for major scientific projects suffer from lack
of manpower. So it is clear that mobilizing the universities and coupling
them to the National Science Projects is the only correct way forward. It
will remedy both these ills. 


However this is a gigantic task.. I will restrict myself to three brief points.
Because of the importance of this problem, I suggest that DST should confer
with UGC and come out, with innovative solutions. Second, each one of us
must try to influence the universities in the physical as well as intellectual
neighborhood of each of us and persuade them to facilitate the participation
of their students in a major scientific project. Third, in many of the university
departments, a large fraction of the faculty strength has been kept vacant
for many years. These must be filled with experimenters who can contribute
to one of the experiments in the list above. 


There may be many more ideas, but what is needed is action. 


Imagine the enormous excitement that students will get by seeing frontline fundamental physics experiments right here. No wonder students are not
excited by science and hence do not enter the field in large numbers. The
only way this can change is by ensuring that experiments are done here. 

\textbf{Manpower in Theoretical HEP:} Theoretical HEP continues to attract
the best students and as a consequence its future in the country appears
bright. However, this important national resource is being underutilized.
Well-trained HEP theorists are ideally suited to teach any of the basic components of Physics such as QM, Relativity, QFT, Gravitation and Cosmology, Many Body Theory, Statistical Mechanics and Advanced Mathematical
Physics since all these ingredients go to make up the present-day HEP theory.
Ways must be found so that a large fraction of these bright young theoretical physicists can be absorbed in the Universities. Even if just one of them
joins each of the 400 universities in the country, there will be a qualitative
improvement in physics teaching throughout the country. But, this will not
happen unless the young theoreticians gain a broad perspective and train
themselves for teaching-cum-research careers. 
