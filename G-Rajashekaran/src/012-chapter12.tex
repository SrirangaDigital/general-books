\chapter[A great opportunity for Indian science]{A great opportunity for Indian science}\label{chap12}

\Authorline{G. Rajasekaran}
\addtocontents{toc}{\protect\contentsline{section}{{\sl G. Rajasekaran}\smallskip}{}}

\authinfo{}




\textbf{The importance of the India-based Neutrino Observatory (INO) in the context of international science cannot be overemphasised. Now is the time to regain an initiative lost}


India is a land of ancient civilisation that had made fundamental contributions to human knowledge in the hoary past. Even the discovery of the positional number system with the accompanying concept of ‘zero’ is attributed to ancient India. We already have a population of more than a billion and may soon become the most populous nation on Earth. 

Should we rest content with our ancient heritage and keep repeating that our contribution to knowledge is ‘zero’? Should we continue to be mere borrowers and users of modern knowledge and modern scientific technology? When do we give back? When do we become creators of fundamental knowledge? The opening up of neutrino physics offers us a great opportunity to do that.

Very important discoveries have been made recently in neutrino physics and neutrino astronomy. Scientists from the United States and Japan received the Nobel Prize in 2002 for these discoveries. Neutrinos are elementary particles that are filling the Universe in abundance but are very elusive. Trillions of neutrinos are passing through our bodies every second without affecting us. One of the most important discoveries of the last decade is that neutrinos have mass. Until this discovery, it was thought that neutrinos are massless particles like photons, the quanta of light.
This has led to active planning of many more neutrino laboratories round the world, especially considering that a considerable part of neutrino physics is yet to be discovered. A grand race is on.

India was a pioneer in neutrino physics. The very first detection of cosmic-ray produced neutrinos was made in the Kolar Gold Fields (KGF) experiment in 1965. But the KGF laboratory was closed in the 1990s because the KGF mines were closed.

Can we recover the lost initiative?  We can. The India-based Neutrino Observatory (INO) project has been conceived with that objective in view. A

group of scientists and engineers spread over 25 scientific research institutions and universities in India is actively involved in the creation of INO. It is a unique basic science collaboration in the country. It has been approved for funding by the Department of Atomic Energy and the Department of Science and Technology and included by the Planning Commission as a mega science project under the Eleventh Five-Year Plan.  (Information about the INO is available at \url{www.imsc.res.in/~ino}.)

A rock of at least a kilometre thickness is needed to filter all other cosmic-ray-produced particles to enable the detector to detect the elusive neutrinos. Hence we have to go inside a mountain. The Nilgiri mountains were chosen as the suitable site for the underground laboratory because of the stability and safety of the Nilgiri rock. A huge cavern of size 120 m $\times$ 25 m $\times$ 30 m will be dug under the Nilgiri mountains at 1.3 km below the peak and this will be accessed through a horizontal tunnel of more than 2 km in length. A gigantic magnetised detector weighing 50,000 tonnes will be constructed inside this cavern and will be used to detect and study the neutrinos. 

In the beginning, this detector will be used to study the neutrinos produced by cosmic rays. Further progress in neutrino physics will depend on catching neutrinos that will be produced in the so-called ‘neutrino factories.’ Such plans are being made in Japan, Europe, and the U.S. We are in dialogue with scientists abroad who are involved in these plans. Neutrinos produced in the neutrino factories thousands of kilometres away will travel through the Earth and be detected in the INO. Such long-baseline neutrino experiments are needed to reveal further neutrino secrets.
Although the first priority will be to establish those parts of neutrino physics that are still unknown or uncertain, once that is accomplished attention will shift to mastering neutrino technology. But that will take time. Some of the exciting applications of neutrino technology will be these:  (1) Since neutrinos are the most penetrating radiation known to mankind (a typical neutrino can travel a million Earth diameters of matter without getting stopped), neutrino beams will be the ultimate tools for the tomography of Earth. (2) A new window on geophysics opened a few years ago when a neutrino detector in Japan detected geoneutrinos emitted by radioactive uranium and thorium ore buried in the bowels of the Earth. This leads to the possibility of mapping the whole Earth as far as its radioactive content is concerned.

The 50,000 tonnes of steel used in the detector does not deteriorate since the neutrinos hardly interact. If this steel could be lent by Mittal Steel Company or the Tata Iron and Steel Company, it could be returned to them later. There is a precedent for this. The Sudbury Neutrino Observatory in Canada used the heavy water loaned from the Canadian Atomic Energy Commission and returned it to the AEC after making a crucial contribution to neutrino physics.

Such a contribution by Indian industry will be a trend-setter for building synergy between science and industry. It is much needed for taking the country to the next stage of development.

Since the proposed INO site in Nilgiris is near an environmentally sensitive area, the INO group has taken great pains to formulate an environmental management plan, in consultation with environmental scientists. The INO group is committed to prevent any damage to the environment and, in fact, plans to contribute positively towards its preservation through its own resources – so that INO becomes a model project to establish that basic science and environmental awareness can go hand in hand.

In addition to making major discoveries, the INO will benefit generations of students and young scientists and engineers by training them through participation in a major scientific experiment. Student recruitment and training for the INO has started. The project will also include an INO Centre devoted to R \& D in detector technology, which will have far-reaching applications in diverse fields.

In spite of progress on all other fronts, the project has been bogged down because of the delay in procuring the required government clearances.
The importance of the India-based neutrino observatory in the context of international science cannot be overemphasised. Other groups in other countries are eagerly waiting for the operation of, and results from, the INO. However they are not going to wait indefinitely. Plans are afoot both in the U.S. as well as China for building huge underground neutrino laboratories. The INO’s competitive edge is slipping away and any further delay will be detrimental to the success of the project. A number of reputed scientists from different parts of the world, including two Nobel Laureates and several Directors of neutrino laboratories, have recently appealed to Prime Minister Manmohan Singh for urgent action.

I hope that the governments of India and Tamil Nadu together will act soon so that this great opportunity for Indian science is not lost.

(The author is a former Joint Director of, and Distinguished Professor at, the Institute of Mathematical Sciences, Chennai. He is now Adjunct Professor at the Chennai Mathematical Institute. Email: \url{graj@imsc.res.in})

