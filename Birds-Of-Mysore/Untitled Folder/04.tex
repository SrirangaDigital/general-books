\chapter{Wetland Birds in Mysore District} 

Wetlands in general include lakes, rivers, streams, 
tanks, marshes, esturities, deltas, seashores, mangrove swamps 
and coral reefs. Lakes include small and large irrigation 
tanks also. Birds are usually abundant on lakes and esturies. 

There are a large number of irrigation tanks scattered 
all over Mysore District. In the south west region, they are 
perennial (8-10 months full of water). There are 20 major 
tanks with water spread area exceeding 200 hectares. Most 
of these tanks are fed by streams, and water level increases 
after monsoon showers. 

A typical tank scenario consists of uplands bordered 
with trees, with lowland grasses sloping towards the shallow 
waterzone of the tank. The interface between land and water 
shows clear zones with sedges, rushes, reeds, submerged water 
weeds and floating vegetation from shallow to deeper waters 
in that order. This presents a very scenic view with placid 
open waters. There is a luxuriant growth of plankton in the 
open waters due to abundance of sunshine in the region. This 
gives rise to richness of crustacean and insect larvae 
supporting myriads of fish life. This in turn supports 
diversity of bird life. The richness of avifauna depends on 
variety of microhabitats created by vegetation and 
topography. 

The marshy areas at the edge of water are more
productive and a large number of birds are found actively 
feeding here especially the waders. Some of ducks and geese 
which are herbivorous also prefer marsh water interface. The 
colour of the water could be a good indicator of the degree 
of pollution. The water may be clear or turbid or greenish 
with nutrient-rich influx from surroundings or dirty, being 
highly polluted. 

The sandpipers, stilts and wagtails will be waiting in 
the low lands covered with grasses for feeding on insect 
larvae and worms. In the sedges will be found warblers 
strutting and flying from bed to bed. 

The zone of emergent vegetation like reedbeds are 
inhabited by moorhens and other gallinules. 

Sometimes weaver birds build nests in colonies by 
slinging together the long leaves of macrophytes typha. The 
gallinules also make use of them building platform nests. 
Diving ducks will be found in fishing open waters The coots 
will be found ranging over the entire wetlands. 

The trees provide perching places for cormorants, storks 
and King fisher. The conmorants will be found drying their 
wings on the rocks after diving for fish since they cannot 
fly efficiently with wet feathers. Pelicans and coots are 
usually found on large tanks where fish is abundant. On the 
large trees around, a number of water birds will be roosting 
and build their nests in single or mixed colonies. Herons 
nest in colonies in places called as heronaries. 

Herons are also wading birds, patiently waiting for 
their prey on the edge of the water. Gray heron, a 
solitary large bird (3ft), will be seen on most of the tanks. 
Its neck is slender with kink and has large toes to balance 
its body on soft mud while foraging. 

The spot billed ducks are common resident ducks on most 
of the tanks. Their numbers are seen augmented during winter 
when they are found amidst migratory ducks. The floating 
vegetations provide walking pads for jacanas which balance 
themselves with long toes, seeking invertebrates and tiny 
frogs. 

The common kinsfisher founds singly by the tank plunge 
dives to fish while the pied kingfisher hovers over the water 
in a stationery position and hurls itself with great speed at 
water to catch fish. 


\smallskip
\bigskip
\noindent
\heading{Migratory Birds:} 

During every winter (beginning with October-November 
onwards) a large number of ducks, geese and waders start 
arriving on water bodies in South with unfailing accuracy. 
They come from eastern Germany, Europe, Siberia, Central Asia 
and as far as China, crossing thousands of miles. They 
scatter over all the major tanks. It is a great spectacle to 
be obeserved on the wetlands around Mysore. They include 
garneys (blue winged teal), cotton teal, pintai1s, 
pochards, shovellers, Black winged stilts, sandpipers, the 
barheaded goose and some small birds. They come in large 
flock of hundreds, with exception of shovellers who are seen 
in few numbers. Gargneys and pintails are commonest. A 
marsh harrier is also a common winter visitor seen singly 
hovering over the reedbeds to pick up unspecting prey. Only 
a few migratory storks could be seen. One may occassionally 
come across a small flock of lesser flamingoes  or a few 
stragglers left behind. On some tanks painted storks are 
common. They have been found breeding near Karanjikere on the tall 
trees surrounding the area. 

Most of the migratory birds depart by March. Every year
since 1987, Asian Waterfowl Census is carried out- in the 
month of January all over India and neighbouring S.Asia. 
Since bird watchers from Mysore very actively participate in 
this programme of international significance, the amatures 
and new comers with background in bird watching should join 
and enjoy the watching the unusual spectacle of bird 
migration right at their doorstep. 

\vskip 1cm

\bigskip

\hfill{Prepared by N.P. Duni \qquad ~}

\smallskip
\hfill{Retd. Scientist, CFTRI \qquad ~}

\smallskip
\hfill{(A List of Wetlandbirds is appended) }
