\part{Content and structure of Haṭhayogapradīpikā}


\section*{Haṭhayoga \&  Haṭhayogapradīpikā}

The definition of the term - Haṭhayoga is as follows -

\begin{shloka}
\textbf{hakāraḥ kīrtitaḥ sūryaḥ ṭhakāraḥ candra ucyate|}\\
\textbf{sūryācandramasoryogāt haṭhayogo nigadyate||}

\verseauthor{(Verse quoted in jyotsnā commentary of haṭhayogapradīpika 1.1)}
\end{shloka}

Ha refers to the Sun and ṭha refers to the moon. The conjoining (Yoga) of Sun and Moon is Haṭhayoga. Sun inturn points to Prāṇa and moon refers to the Apāna. The aikya (oneness)-Yoga of Prāṇa and Apāna which is attained through Prāṇāyāma is Haṭhayoga\footnote{haśca ṭhaśca haṭhau sūryacandrau tayoryogo haṭhayogaḥ, etena haṭhaśabdavācyayoḥ sūryacandrākhyayoḥ prāṇāpānayoraikyalakṣaṇaḥ prāṇāyāmo haṭhayoga iti haṭhayogasya lakṣaṇaṁ siddham  - jyotsnā commentary of haṭhayogapradīpika 1.1}.

Haṭhayoga traces its origins to Yogins like Matsyendranātha and Gorakṣanātha. They have Śaiva orientation. Hence they may inturn have derived their inspiration from the Śaivāgama tradition. Scholars also hod that Haṭhayoga has connections with Kashmir Tantra tradition.  The Āgamas have their origins the beginnings of the Common Era.\ Yoga is one among the four basic aspects in Āgama which include- caryā, kriya, jñāna/ vidyā and Yoga. From being one among the limbs of practices on Āgama, Haṭhayoga seems to have developed gradually establishing its unique set of ideals and practices.

The oldest known treatise in Haṭhayoga tradition is understood to be kaula-jñana-nirṇaya of Matsyendranātha (900 Cent.\ CE). Siddha-\break siddhānta-paddhati, Gorakṣaśataka (10 -- 12$^{\rm th}$ Century CE) Haṭha\-yoga\-pra\-dī\-pika (14$^{\rm th}$ Century CE), Gheraṇḍasaṃhitā, Śivasaṃhitā (17$^{\rm th}$ Century CE) are a few other noteworthy treatises in this tradition.

\section*{Haṭhayogapradīpika – The light on Haṭhayoga}

Pradīpikā in saṃskṛta means Light. Thus haṭhayogapradīpikā means “Light on Haṭhayoga”. Among the various treatises of Haṭhayoga - Haṭhayogapradīpika by Svātmārāma is respected, accepted and studied widely for its compact and comprehensive approach to all aspects of Haṭhayoga. A popular Saṃskṛta commentary that throws light on subtle aspects of Haṭhayogapradīpika is  jyotsnā by Brahmānanda (19th Cent CE).  The author of Haṭhayogapradīpika is variously called as Svātmārāma or cintāmaṇi or ātmārāma. He is known to be the Son of Sahajānanda Yogīndra.

\section*{Chapterization in Haṭhayogapradīpika}

The text is in Four chapters and contains 389 verses. The chapters and the distribution of the verses in them are as follows -  

\begin{enumerate}
\itemsep=0pt
\renewcommand{\theenumi}{\Roman{enumi}}
\renewcommand{\labelenumi}{\theenumi.}
\item Āsanas and other prerequisites for Yoga - 67
\item Prāṇāyāma - 78
\item Mudrās  - 130
\item Samādhi- Nādānusandhāna - 114
\end{enumerate}

As can be seen above, the four chapters indicate the four limbs of the Haṭhayoga – namely Āsana, Prāṇāyāma, Mudrās and Nādānusandhānana. It is to be noted, that the first three are the means that facilitate the practice of the last limb – Samādhi – which is also termed as Rājayoga in this treatise. There too, the practice of Nādānusandhānana has been preferred in this text towards attainment of Samādhi. 
