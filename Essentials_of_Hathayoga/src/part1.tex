\part{Content and Structure of \textit{Haṭhayogapradīpikā}}


\section*{\textit{Haṭhayoga \&  Haṭhayogapradīpikā}}

The definition of the term \textit{Haṭhayoga} is as follows:

\begin{shloka}
\textit{\textbf{hakāraḥ kīrtitaḥ sūryaḥ ṭhakāraḥ candra ucyate|}}\\
\textit{\textbf{sūryācandramasoryogāt haṭhayogo nigadyate||}}

\verseauthor{(Verse quoted in \textit{Jyotsnā} commentary of \textit{Haṭhayogapradīpika} 1.1)}
\end{shloka}

\textit{`Ha'} refers to the Sun and \textit{`ṭha'} refers to the moon. The conjoining \textit{(Yoga)} of Sun and Moon is \textit{Haṭhayoga}. Sun inturn points to \textit{Prāṇa} and moon refers to the \textit{Apāna.} The \textit{Aikya} (oneness)-\textit{Yoga} of \textit{Prāṇa} and \textit{Apāna} which is attained through \textit{Prāṇāyāma} is \textit{Haṭhayoga}\footnote{\textit{haśca ṭhaśca haṭhau sūryacandrau tayoryogo haṭhayogaḥ, etena haṭhaśabdavācyayoḥ sūryacandrākhyayoḥ prāṇāpānayoraikyalakṣaṇaḥ prāṇāyāmo haṭhayoga iti haṭhayogasya lakṣaṇaṁ siddham  - jyotsnā} commentary of haṭhayogapradīpika 1.1}.

\textit{Haṭhayoga} traces its origins to \textit{Yogins} like \textit{Matsyendranātha} and \textit{Gorakṣanātha.} They have \textit{Śaiva} orientation. Hence they may inturn have derived their inspiration from the \textit{Śaivāgama} tradition. Scholars also had that \textit{Haṭhayoga} has connections with Kashmir Tantra tradition.  The \textit{Āgama-s} have their origins the beginnings of the Common Era.\ Yoga is one among the four basic aspects in \textit{Āgama} which include- \textit{Caryā, Kriya, Jñāna/ Vidyā and Yoga.} From being one among the limbs of practices on \textit{Āgama, Haṭhayoga} seems to have developed gradually establishing its unique set of ideals and practices.

The oldest known treatise in \textit{Haṭhayoga} tradition is understood to be \textit{kaula-jñana-nirṇaya} of \textit{Matsyendranātha} (900 Cent.\ CE). \textit{Siddha-\break siddhānta-paddhati, Gorakṣaśataka} (10 -- 12$^{\rm th}$ Century CE) \textit{Haṭha\-yoga\-pra\-dī\-pika} (14$^{\rm th}$ Century CE), \textit{Gheraṇḍasaṃhitā, Śivasaṃhitā} (17$^{\rm th}$ Century CE) are a few other noteworthy treatises in this tradition.

\section*{\textit{Haṭhayogapradīpika} – The Light }
%~ on Haṭhayoga
\textit{Pradīpikā} in \textit{Saṃskṛta} means Light. Thus \textit{Haṭhayogapradīpikā} means “Light on \textit{Haṭhayoga}.” Among the various treatises of \textit{Haṭhayoga} - \textit{Haṭhayogapradīpika} by \textit{Svātmārāma} is respected, accepted and studied widely for its compact and comprehensive approach to all aspects of \textit{Haṭhayoga.} A popular \textit{Saṃskṛta} commentary that throws light on subtle aspects of \textit{Haṭhayogapradīpika} is  \textit{jyotsnā} by \textit{Brahmānanda} (19th Cent CE).  The author of \textit{Haṭhayogapradīpika} is variously called as \textit{Svātmārāma} or \textit{Cintāmaṇi} or \textit{Ātmārāma.} He is known to be the Son of \textit{Sahajānanda Yogīndra.}

\section*{Chapterization in \textit{Haṭhayogapradīpika}}

The text is in Four chapters and contains 389 verses. The chapters and the distribution of the verses in them are as follows :  

\begin{enumerate}
\itemsep=0pt
\renewcommand{\theenumi}{\Roman{enumi}}
\renewcommand{\labelenumi}{\theenumi.}
\item \textit{Āsana-s} and other prerequisites for \textit{Yoga} - 67
\item \textit{Prāṇāyāma} - 78
\item \textit{Mudrā-s}  - 130
\item \textit{Samādhi}- \textit{Nādānusandhāna} - 114
\end{enumerate}

As can be seen above, the four chapters indicate the four limbs of the \textit{HaṭhayogaHaṭhayoga} – namely \textit{Āsana, Prāṇāyāma, Mudrā-s} and \textit{Nādānusandhānana.} It is to be noted, that the first three are the means that facilitate the practice of the last limb \textit{Samādhi} which is also termed as \textit{Rājayoga} in this treatise. There too, the practice of Nādānusandhānana has been preferred in this text towards attainment of \textit{Samādhi.} 
