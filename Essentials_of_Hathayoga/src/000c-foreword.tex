\thispagestyle{empty}

\chapter*{Foreword}

\begin{shloka}
\centering
\noindent
 \emph{\textbf{vede yogo bahudhā proktaḥ}}\\
 \emph{\textbf{tatrāṣṭāṅgo yogo mukhyaḥ |}}\\
 \emph{\textbf{patañjaliḥ śrībhagavānittham}}\\
 \emph{\textbf{dideśa sūtraiḥ svakṛtaiḥ kṛpāluḥ ||}}
\smallskip

\textit{Veda-s Speak of Yoga in many ways}\\
\textit{Aṣṭāṅga is the best among them}\\
\textit{which compassionate Patañjali}\\
\textit{expounded through his sūtra-s,}
\end{shloka}

\noindent \textit{Yoga} traces its origin to \textit{Veda-s. Veda-s} are the first known compendium of knowledge in the human civilization.  The \textit{Veda-s} are in Sanskrit \textit{(Saṁskṛtam)}. The \textit{Veda-s} are huge, and the message contained therein is difficult to decode. Great Sages \textit{(Ṛṣi-s)} passed on the messages through their teachings to a lineage \textit{(Parampara)} of students, which even now continues. The vastness of the topics covered in the \textit{Veda-s} are such it covers everything that one would like to know, follow and practice to achieve anything from material benefits to spiritual benefits. One such important practical knowledge is \textit{Yoga}.  References to \textit{Yoga} is found in several places in the \textit{Veda-s}, both in the segment relating to rituals \textit{(Karma Kāṇḍa)} and enquiry (\textit{Jñāna Kāṇḍa}). The credit of putting them together at one place goes to Sage \textit{Patañjali} who gave the great \textit{Yoga} manual known as \textit{“Pātañjalayogadarśanam”} or popularly called as \textit{“Yogasūtra-s of Patañjali.”} Subsequently, the messages of \textit{Yogi-s} who studied and practised \textit{Yoga} under a \textit{Guru} in the lineage way were recorded and are available as \textit{Yoga} texts, together called \textit{‘Yoga Śāstra-s.’}  One such branch of Yogic practice is \textit{Haṭha-yoga,} the name given to the set of techniques followed by the \textit{Yogi-s} of the \textit{‘Nāthasampradāya,”} the origin of which is traced to Lord \textit{Śiva} (one of the Trinities – Vedic Gods).

\textit{Haṭhayogapradīpikā} of \textit{Svātmārāma} is acclaimed as one of the important Yogic texts for study under a \textit{Guru} for those who aspire to seriously pursue \textit{Yoga} and obtain its wholesome benefits. Any Yogic text needs to be studied under a \textit{Guru} only, as the messages, particularly those relating to practices, are ‘subtle’ and needs individual explanation. The text which is in Saṁskṛtam is terse and even great Yoga teachers will use the commentaries while explaining the aspects. In our tradition, this text is studied along with the commentary by \textit{Swami Brahmānanda} called \textit{“Jyotsna.”} The commentary is special as it contains lucid explanations of every word of the main text and with supplementary references to a vast collection of texts, such as \textit{Liṅgapurāṇa, Nandikeśvarapurāṇa, Yogacintāmaṇi, Siddhasiddhāntapaddhati, 
Yogavāsiṣṭha, Bhagavadgītā, etc.} Our teacher, Sri T K V Desikachar used to read out the commentary and give the meaning with practical explanations.

Sri T Krishnamacharya considered \textit{Haṭhayogapradīpikā} of \textit{Svātmārāma} as one of the important texts to be studied by any aspiring student.  In this text, important parts were taken, and some parts were omitted. Krishnamacharya Yoga Mandiram (KYM) continues to give special importance to the study of this text in detail.

Our teacher has been insisting on studying the texts in the original form itself which calls for knowledge of Saṁskṛtam.  He would make us to take notes and one need to write the Saṁskṛtam words in the original form.  A working knowledge of Saṁskṛtam will automatically set in as one studies these texts. \textit{Yoga} being a technical subject, translations of Saṁskṛtam words will not bring out the intended meaning. The word \textit{‘Yoga’} itself continues to be used in its Saṁskṛtam form and not translated.

The textual immersion Workshops on \textit{‘Haṭhayogapradīpikā} of \textit{Svātmā\-rāma’} organised by the Research Department of the Krishnamacharya Yoga Mandiram was led by Dr M Jayaraman, Former Director, Textual Research. They received enthusiastic response from serious \textit{Yoga} teachers and practitioners from India and Abroad. The present attempt is to bring out the proceedings of the unique workshops in a book form for preservation and use by those interested in understanding the nuances of this great text.

I personally congratulate Dr M Jayaraman in organising the workshops and in bringing out this book.  I am sure this book will find wide readership and will find its place as a source of quotation.
\bigskip

\centerline{\textit{“Śrīgurubhyonamah’}}
\bigskip

\begin{flushright} 
\begin{tabular}{c}
\textbf{\textit{Yogācārya} S Śrīdharan}\\
Chennai\\
14 May 2022
\end{tabular}
\end{flushright}
