\section*{Chapter-wise Purport and Salient Points}

\subsection*{Chapter 1}

The first chapter of Haṭhayogapradīpikā can be seen in the three parts - 
\vspace{-15pt}

\begin{enumerate}
\itemsep=0pt
\renewcommand{\theenumi}{\arabic{enumi}}
\renewcommand{\labelenumi}{\theenumi)}
\item Part 1 --- 1--16      =  16 Verses --- invocation, prerequisites for Yoga 
\item Part 2 --- 17--54    =  38 Verses --- Āsanas --- definitions and benefits (15)
\item Part 3 --- 55--67    =  13 Verses --- dietary and other lifestyle prescriptions
\end{enumerate}

\section*{1. Salient Points of Part 1 --- Chapter 1}

\textbf{Verse 1--3:}  Invocation and Objective: Ādinātha –  Śiva is saluted in the first verse as the one who gave the teaching of Haṭhayoga,. It is also clarified in the very first verses that Haṭhayoga is a ladder to Rājayoga (this is emphasized in the 2nd verse and also the last verse (67)  of this chapter). This indicates that the first three limbs of Haṭha leads to the practice of the last limb of Samādhi through the practice of Nādānusandhāna. Further it is stated in the third verses that Svātmārāma, the Yogin has written Haṭhayogapradīpikā to help clear the confusion about various paths that lead to Rājayoga. By implication this establishes Haṭhayoga as a systematic means to attain Rājayoga.

\textbf{Verses  4--9:} Lineage: The text lists 34 teachers of Haṭha Lineage. These teachers are stated to live (forever) in the universe, overcoming the limitation of time.

\begin{multicols}{3}
\begin{enumerate}
\itemsep=0pt
\item ādinātha
\item matsyendranātha
\item śābara
\item ānanda
\item bhairava 
\item cairāṅgī 
\item mīnanātha
\item gorakṣa 
\item virūpākṣa 
\item bileśaya 
\item manthāna
\item bhairava 
\item siddhi
\item buddhi
\item kanthaḍi 
\item koraṇṭaka 
\item surānanda
\item siddhapāda 
\item Carpaṭi
\item kānerī
\item pūjyapāda
\item nityanātha
\item nirañjana
\item kapālī 
\item bindunātha
\item Kākacanḍīśvara
\item allāma
\item prabhudeva
\item ghoḍācolī
\item ṭiṇṭiṇi
\item bhānukī
\item nāradeva
\item khaṇḍa
\item kāpālika
\end{enumerate}
\end{multicols}

\textbf{Verses 10--11:} Solace and Secrecy: It mentioned herein that Haṭha practices are the source of solace and they have to be learnt and passed on in secrecy – to indicate the sanctity and the need of seriousness in pursuing this discipline of knowledge.

\textbf{Verses 12--14:} Maṭha-lakṣaṇa: The choice of the country, the choice of place and also the structure/building in which Haṭha has to be practiced have been enumerated herein.  The choice of country and place: Well governed, In the righteous path, Prosperous, Without thieves etc, Away from rocks, fire, water, Solitude (12). The Building/structure: Small door, without any window, burrow or slant/tilt, neither high, low or wide, Free from pests and insects and animals, On the outside: Small hall, Altar, Well. Surrounding wall (13). What to do in such a place?  In such a place, setting aside all worries, one should only follow the instructions of the teacher. (14)

\textbf{Verses 15, 16:} Six factors each for failure and success in Haṭha: 

\noindent
\begin{minipage}[t]{.45\linewidth}
6 Factors each for failure in Haṭha ---
\begin{enumerate}
\item Over eating 
\item Over exertion 
\item Excessive talk
\item Observance of unsuitable disciplines 
\item Company with people 
\item Unsteadiness  (15)
\end{enumerate}
\end{minipage}
\smallskip~
\begin{minipage}[t]{.45\linewidth}
6 Factors each for failure/success in Haṭha ---
\begin{enumerate}
\item Enthusiasm-zeal ---\hfil\break  strong will to succeed 
\item Forging ahead --- not worrying too much 
\item Courage --- in not getting discouraged by setbacks
\item Being  endowed with awareness of the ultimate goal --- the consciousness  
\item faith/firmness 
\item Not mixing too much with people (16) 
\end{enumerate}
\end{minipage}

\subsection*{2. Salient Points of Part 2 --- Chapter 1}

\textbf{Verses  17--54 :} Āsana-s: Fifteen Āsanas are presented here. The fifteen Āsanas are - 

\begin{multicols}{2}
\begin{enumerate}
\itemsep=0pt
\item Svastikāsana (19)
\item Gomukhāsana (20)
\item Vīrāsana  (21)
\item Kūrmāsana (22)
\item Kukkuṭāsana (23)                                        
\item Uttānakūrmakāsana (24)
\item Dhanurāsana (25)
\item Matsyendrāsana (26,27) 
\item Paścimatānāsana (28,29)
\item Mayūrāsana (Hand balance) (30,31) 
\item Śavāsana (32)  	 		
\item Siddhāsana-5\hfil\break Variations  (35--43) 
\item Padmāsana-3\hfil\break Variations (44--49)
\item Siṃhāsana (50--52)   
\item Bhadrāsana  (53,54)
\end{enumerate}
\end{multicols}

Āsanas 1--11 are presented as one set and Āsanas 12--15 are presented as another set. The second set is stated to be very important. Even among the second set of Āsanas --- the 12th Āsana --- Siddhāsana is glorified as the greatest among the Āsanas. It is to be noted that – nomenclature and method of performing of Āsanas 1-7 are given. Nomenclature, method of performing and also benefits are given for Āsanas 8--15.  The types of Āsanas given here in are  - Āsanas in seated position (1--9, 12--15), an Āsana  Commenced from a kneeling postion (10) and an Āsana in Lying postion (11). It is also to be noted that five variations of Siddhāsana and three variations of Padamāsana can be seen from the text and commentary.

\subsection*{3. Salient Points  of Part 3  --- chapter 1}

\textbf{Verses 55--66 :} This salient aspects of the verses of this section is summarized in the form of six short questions and answers for easy and quick understanding.

\begin{enumerate}
\item What is the sequence of Haṭha practices?  Āsana, prāṇāyama, mudrā, nādānusandhāna is the sequence (56)
\item How long will it take to succeed in Haṭha? If a person is Celibate, part takes Moderate food, has a dispasstionate nature, a regular practitioner – success in Yoga is achived in one year (57).
\item What is the diet for a Haṭhayogin? (includes prescribed as conducive and proscribed as unconducive) (58-63)   

Prescribed as conducive: Pleasant/agreeable and sweet is conducive. Further while eating food  one should not eat to fill. 1/4th  empty space should be there in the abodomen. The food should be consumed as an offering to indwelling śiva (divinity). (58). Wheat, barley, milk, ghee, palm sugar, butter, honey, dry ginger, green gram etc  are also conducive(62, 63)

Proscribed as unconducive: Food that is bitter, sour, pungent, salty, generates heat (jaggery), green vegetables, sour gruel, (Sesame or mustard) oil, sesame, mustard, alcohol, fish, meat including that of goat, curds, buttermilk, horse gram, fruit of jujube (bore -ilanthai), oil cakes, asafoetiḍā and garlic. (59) Reheated food, dry, excess of salt or sourness, food that is pungent , too much of vegetables are to be avoided. (60)
\item What are the other precautions at the initial stages?    Fire during winter (avoid luxury), sexual intercourse, journeys, bathing early in the morning, fasting and other hard physical activity should be avoided. (61)  
\item What is the age and condition for taking up Haṭha-Yoga? Youth, old, extremely old, sick and weak can also succeded in Yoga. Practice without lethargy leads to success. (64)
\item How to succeed in Haṭhayoga? The one who practices in a sustained way succeeds. Mere reading of yoga texts is of no avail. (65) Wearing a yogic costume and eloquently talking about it will not help. Only practice will lead to success. (66) 
\end{enumerate}

\section*{Chapter 2}

The contents of the second chapter can be seen in five parts. It is as follows -  

\begin{enumerate}
\item Need  of prāṇāyāma ---  verses 1--3 
\item Nāḍīśodhana  - verses 4--20
\item Ṣaṭkarmas and Gajakaraṇī --- verses 21--38
\item Importance of Prāṇāyāma and Aṣṭakumbhakas (eight techniques of Prāṇāyāma) --- verses 39--70
\item Classification of prāṇāyāma and  Indicators of  Success in haṭha verses 71--78
\end{enumerate}

\begin{enumerate}
\item \textbf{verses 1--3:} Need of Prāṇāyāma: Importance of prāṇāyāma  --- In these verses, Prāṇāyāma is advised  to be practiced after attaining  firmness is āsana to regulate the mind  and also to live long. The importance of taking the guiḍānce of a Guru in  prāṇāyāma is emphasized.  

\item \textbf{Verses 4--20:} Nāḍīśodhana: The need to cleanse the Nāḍīs, that are loaded with impurities, through Prāṇāyāma (nāḍīśodhana) to facilitate the movement of Prāṇa in Suṣumnā is highlighted in three verses (4--6). The technique of Nāḍīśodhana\footnote{(inhaling through left nostril --- Holding the breath --- Exhaling through the Right Nostril and Inhaling through the right nostril --- Holding the breath --- Exahling through right nostril)} sitting in Padmāsana is stated (7--10). The need to practice Nāḍīśodhana for four times (morning- midday-evening-midnight) a day and eighty cycles per practice is mentioned (11). Indicators of progress in this practice like sweating\footnote{Rubbing such sweat on to one’s body to experience lightness and firmness in the body is stated in verse 13}, trembling and experiencing Prāṇa in suṣumnā (12) are stated thereafter. Diet of milk and rice conducive for Nāḍīśodhana is stated (14). Practicing Nāḍīśodhana slowly and systematically is emphasized. (15,16).  Consequences of Wrong practice --- Hiccups, asthma, bronchial diseases, pain in the head, ears and eyes (17) are indicated. The need to practice the components of Prāṇāyāma correctly (inhale --- to one’s capability, hold --- with Bandhas  and exhale --- slowly) is emphasized (18).  The indicators of cleansed Nāḍīs such as --- leanness and glow in the body, ability to hold breathe comfortably, increase of gastric fire, freedom from illnesses and experiencing Nāda (inner Yogic (Anahata) sounds) are stated (19, 20).
\item \textbf{Ṣaṭkarmas and Gajakaraṇī} - Verses  21-38

The Ṣaṭkarmas (six practices)  that would cleanse the system from excess kapha and Medas (fat) is stated. This is stated as a prerequisite from Prāṇāyāma. If there is no excess of Kapaha and Medas these are not required to be practiced.  (21) The six cleansing practices are described in detail with their benefits (22--35). The six practices are  -
\begin{enumerate}
\item dhauti (swallowing a wet and clean cloth and drawing it back --- cleansing of the path of food), 
\item vasti (standing in water and intake of water through anus and cleansing the anus region)          
\item netī (introducing a soft, wet tube through the nostril to cleanse nasal passage),                                  
\item traṭaka (staring at a very small object without blinking eyes, cleansing eyes)                            
\item naulika (moving the abdomen in a circular way from right to left and cleaning the abdominal region), 
\item kapālabhāti (rapid inhalation and exhalation focusing on cleansing the sinus region).
\end{enumerate}

It is also clarified that according to certain teachers, that there is no need for Ṣaṭkarmas. By prāṇāyāma all impurities are removed (37). Another Standalone practice called Gajakaraṇī is stated. It is a practice by which the food and water in the stomach are voluntarily vomited to cleanse the abdominal region. (38) 

\item \textbf{Importance of Prāṇāyāma and Aṣṭakumbhakas --- verses 39--70}

The efficacy of prāṇāyāma in overcoming death, ensuring longevity, regulating the mind , making the prāṇa flow in suṣumnā and by that stabilizing the mind and gradually attaining Samādhi (unmanī state) is mentioned (39--43)

After this, the eight prāṇāyāma techniques that are called as Aṣṭakumbhakas are stated, with their nomenclature, technique of practice and benefits (44--70)\footnote{In verses 45--47 the need to practice the three Bandhas (mūlabandha, uḍyānabandha and jālandharabandha) during prāṇāyāma techniques  (while holding the breath within) is emphasized to facilitate the flow of prāṇa in suṣumnā.}. This can be called as the core part of this chapter\footnote{Additionally, In the Jyotsnā Commentary to Verse 48, the ideal/beneficial daily routine for the practitioner of Hatha is also described}. The eight Kumbhaka practices are ---

\begin{enumerate}
\item sūryabhedana
\item ujjāyī
\item sītkārī
\item śītalī
\item bhastrikā
\item bhrāmarī
\item mūrchā
\item plāvinī
\end{enumerate}

It is to be noted that Prāṇāyāma techniques 1-6 have stated to bestow therapeutic benefits. Techniques 7 and 8 grant capabilities to stay as if in a state of stupor (mūrchā) and also float on water respectively.

\item \textbf{Classification of prāṇāyāma and   Indicators of  Success in haṭha - verses 71--78}
\end{enumerate}

Three components of Prāṇāyāma (recaka, pūraka andkumbhaka) and Two types (sahita and kevala) of Prāṇāyāmas are stated. The need to use Sahita Kumbhaka to move to the state of Kevala kumbhaka is emphasized. It is also stated that excellence in Kumbhaka leads to awakening of Kuṇḍalinī. Awakening of Kuṇḍalinī opens the path of Suṣumnā (for prāṇa). By this Haṭha is achieved (the combining of Prāṇa and apāna and their entry into suṣumnā).  From this, the state of Rājayoga (Samādhi) is attained. (71--77). The chapter ends with the  enumeration of eight fold indicators of success in Haṭha that include --- slimness, brightness, manifestation of inner sound, clear eyes, freedom from disease, semen control (sensuality control), stimulation of digestive fire and purification of nāḍi (78). 

\section*{Chapter 3}

The third chapter can be seen in two divisions –

\begin{enumerate}
\item About awakening of kuṇḍalinī --- verses 1--5
\item Ten Mudrā-s --- verses 6--130 (the list, details and benefits)
\end{enumerate}

\subsection*{1) About awakening of kuṇḍalinī --- verses 1--5}

Herein - the foundational nature of (awakening) Kundalni to haṭha practices (1), the need of Guru’s grace in awakening Kuṇḍalinī (2), the fact that suṣumnā becomes free by awakening Kuṇḍalinī (3), Seven synonyms of Kuṇḍalinī (4) and the need to practice Mudrās to awaken Kuṇḍalinī (5) are discussed. 

\subsection*{2) Ten Mudrā-s --- verses 6--130 (the list, details and benefits)}

The Ten mudrās are enlisted and elaborated with their benefits in this section. Initially the list of the mudrās are given and it is stated that these are directly given by Ādinātha (Śiva) and hence are to be protected secretly (6--9). The ten mudrās are then elaborated systematically. The ten  mudrās  and the distribution of verses in their description is given below -  

\begin{enumerate}
\item Mahāmudrā (10--18)
\item Mahābandha (19--25) 
\item Mahāvedha (26--31) 
\item Khecarī (32--54) 
\item uḍyāna-bandha (55--60)
\item mūlabandha (61--69)
\item jālandharabandha (70--76)
\item viparīta-karaṇī (77--82) 
\item vajorlī/sahajoil/amaroli (83--103) 
\item śakticālana (104--126)
\end{enumerate}

Towards the end of the chapter - the need to be attentive/mindful in the practices related to breathing is stressed. The efficacy of the Mudrās in giving various powers is stated and the importance of learning Mudrās from a Guru is emphasized (127--130) 

\textbf{Note: It is to be noted that in Krishnamacharya Yoga Mandiram only five mudrās are practiced. Of the five --- three are Bandhas (uḍyānabandha, mūlabandha and jālandharabandha). The other two are Mahāmudrā and viparīta-karaṇī. The other five practices are stated as Vāmācāra (self-mortifying practices) by Sri Krishnamacharya and have not been taught or prescribed.}

\section*{Chapter 4}

The fourth chapter can be seen in four divisions. They are as follows -

\begin{enumerate}
\item samādhi --- 1--33
\item Laya practices 29--64
\item Nādānusandhāna --- verses 65--102
\item Transcendental experiences of a Yogin in samādhi and concluding advice ---  verses 103--114
\end{enumerate}

A brief overview of the chapter is present below based on the four divisions.

\begin{enumerate}
\item \textbf{Verses 1--28:  samādhi:} The initial verses are in the form of Saluation to Lord Śiva and it also stated that Samādhi-krama will be elaborated (1 \& 2). Fifteen synonyms of Samādhi are stated in two verses after this (3,4). The definition of Samādhi as --- Samattva and also union of the jīvātman and paramātman are presented. Dissolution of Salt in water is given as an example to illustrate the dissolution of mind in the ātman, which is Samādhi.(5--7). The need for grace of Guru is mentioned (8,9). The need of practices of āsanas, Prāṇāyāma and Mudrās (Haṭha practices) leading to awakening of Kuṇḍalinī and thereby facilitating the entry of Prāṇa into Suṣumnā is summarized. This is presented as the path that leads to the state of Samādhi. It is emphasized that only the practices of Haṭha through the absorption(laya) of Prāṇa and Manas lead to  knowledge and liberation (10--15). After this a few teachings on absorption of prāṇa (Prāṇa-laya) is described. The intimate connection of Prāṇa and Manas are described with the example of the mixture of water and milk. It is also stated that Vāsanas and Prāṇa are the two causes that activate the citta. If one is removed the other is also taken care of and the mind settles down. The fickleness of the mind is compared to the mercury and the benefits of control of both Mercury (Rasa) and Prāṇa are described. This section ends with an emphasize on the need to regulate/stabilize the prāṇa, mind, senses and strengthen the body.(16--28)

\item \textbf{Verse 29--64: Laya:} Laya is introduced in this section.  The absorption (Laya) of the mind and the prāṇa itself is eulogized as the state of liberation as it grants peace and happiness. Laya is praised in various verses and defined as (due to absorption/stabilization of the mind/prāṇa) non arousal of vāsanas and forgetting of the objects of the senses. (29--34) Śāmbhavī mudrā that leads to Prāṇa-laya through Citta-laya is stated. (35--42). After this, another type of Khecarī mudrā is stated as a prāṇalaya method (43--53). After the two practices of Prāṇalaya, a few methods of manolaya is stated --- such as meditating upon the Kuṇḍalinī (54), meditating upon the space/Brahman (55). After this, a few verses describe the need to focus and stabilize the mind. The need to make it  dissolvein the object of meditation like the camphor in the fire is presented (56-64).

\item \textbf{Verses 65--102:} Nādānusandhāna: This is the central portion of the chapter that expounds Nādānusandhāna, the chief and best among the laya practices. This is a Manolaya practice. To glorify this practice it is initially stated that even those simple aspirants who cannot attain philosophical clarity intellectually, can practice Nādānusandhāna. (65--67). Two types of Nādānusandhāna practices are presented in this text. The first type is in four stages ārambha (commence), Ghata (consoliḍāte), Paricaya (intro to higher stages), niṣpatti (reaching the goal). Here a Yogin is expected to sit in Muktāsana assuming the śāmbhavī mudrā and start focusing on the Nāda manifesting in the right ear with a focused mind.  As the mind is focussed, the Yogin experiences the aforementioned four stages where sounds of varying nature is heard at four various places beginning from anāhatacakra (midchest) to Brahmarandhra (crown region of the head).(68--81).

The second type of Nādānusandhāna is stated through the principles of Pratyāhāra --- implying moving from grosser sounds to subtler and refined sounds. Three stages of focusing on Nādas are presented with  no specifc place of focus. The three sets of sound that are to be focussed are, the sounds of --- a) Ocean, cloud, kettle drum, b) then sounds of conch, bell and horn c) and finally sounds of flute, vina and buzzing sound of the bees.(82--89). The next set of verses speak about how the mind is captivated by this practice of Nādānusandhānana using exampels from the nature like --- bees, elephants, deers, horses and serpents. Finally it is stated that, as the fire gets extinguished when the fuel is exhausted, similarly, the mind that is captivated in Nāda gets dissolved when Nāda dies down by Paravairāgya (90--102).
 
\item \textbf{Verses 103-114 :} Transcendental experiences of a Yogin in\break samādhi and concluding advice: The transcendental experiences of a Yogin who experiences Samādhi through aaforementioned practices is stated in this section. The Yogins freedom from objects of senses, freedom from relative states of awareness, firmness in the state of awareness are described. A Yogin which such experiences is stated as Jīvan-mukta (liberated while being alive). (103--113). The chapter and the text ends with a strong parting advice with emphasize upon the importance of practice to attain the goals of Haṭhayoga. The last verse is as follows - “As long as prāṇa does not flow in suṣumnā and reach Brahmarandhra, as long as sensual desires are not overcome, as long as the mind does not reflect the natural state (svarūpa) so long, those who talk of spiritual knowledge indulge only in boastful and false prattle (114)   
\end{enumerate}
