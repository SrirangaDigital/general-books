\section*{Chapter-wise Purport and Salient Points}

\subsection*{Chapter 1}

The first chapter of \textit{Haṭhayogapradīpikā} can be seen in the three parts :

\begin{enumerate}
\itemsep=0pt
\renewcommand{\theenumi}{\arabic{enumi}}
\renewcommand{\labelenumi}{\theenumi)}
\item Part 1 --- 1--16      =  16 Verses --- invocation, prerequisites for \textit{Yoga} 
\item Part 2 --- 17--54    =  38 Verses --- \textit{Āsana-s} --- definitions and benefits (15)
\item Part 3 --- 55--67    =  13 Verses --- dietary and other lifestyle prescriptions
\end{enumerate}

\section*{1. Salient Points of Part 1 --- Chapter 1}

\textbf{Verse 1--3:}  \textbf{Invocation and Objective:} \textit{Ādinātha} –  \textit{Śiva} is saluted in the first verse as the one who gave the teaching of \textit{Haṭhayoga},. It is also clarified in the very first verses that \textit{Haṭhayoga}  is a ladder to \textit{Rājayoga} (this is emphasized in the 2nd verse and also the last verse (67)  of this chapter). This indicates that the first three limbs of \textit{Haṭha} leads to the practice of the last limb of \textit{Samādhi} through the practice of \textit{\textit{Nādānusandhāna}}. Further it is stated in the third verses that \textit{Svātmārāma}, the \textit{Yogin} has written \textit{Haṭhayogapradīpikā} to help clear the confusion about various paths that lead to \textit{Rājayoga}. By implication this establishes \textit{Haṭhayoga}  as a systematic means to attain \textit{Rājayoga}.

\textbf{Verses  4--9:} \textbf{Lineage:} The text lists 34 teachers of \textit{Haṭha} Lineage. These teachers are stated to live (forever) in the universe, overcoming the limitation of time.

\begin{multicols}{3}
\begin{enumerate}
\itemsep=0pt
\item \textit{Ādinātha}
\item \textit{Matsyendranātha}
\item \textit{Śābara}
\item \textit{Ānanda}
\item \textit{Bhairava }
\item \textit{Cairāṅgī }
\item \textit{Mīnanātha}
\item \textit{Gorakṣa }
\item \textit{Virūpākṣa }
\item \textit{Bileśaya }
\item \textit{Manthāna}
\item \textit{Bhairava }
\item \textit{Siddhi}
\item \textit{Buddhi}
\item \textit{Kanthaḍi }
\item \textit{Koraṇṭaka }
\item \textit{Surānanda}
\item \textit{Siddhapāda }
\item \textit{Carpaṭi}
\item \textit{Kānerī}
\item \textit{Pūjyapāda}
\item \textit{Nityanātha}
\item \textit{Nirañjana}
\item \textit{Kapālī }
\item \textit{Bindunātha}
\item \textit{Kākacanḍīśvara}
\item \textit{Allāma}
\item \textit{Prabhudeva}
\item \textit{Ghoḍācolī}
\item \textit{Ṭiṇṭiṇi}
\item \textit{Bhānukī}
\item \textit{Nāradeva}
\item \textit{Khaṇḍa}
\item \textit{Kāpālika}
\end{enumerate}
\end{multicols}

%~ \vspace{-5cm}

\textbf{Verses 10--11:} \textbf{Solace and Secrecy:} It mentioned herein that \textit{Haṭha} practices are the source of solace and they have to be learnt and passed on in secrecy – to indicate the sanctity and the need of seriousness in pursuing this discipline of knowledge.

\textbf{Verses 12--14:} \textbf{Maṭha-lakṣaṇa:} The choice of the country, the choice of place and also the structure/building in which \textit{Haṭha} has to be practiced have been enumerated herein.  The choice of country and place: Well governed, In the righteous path, Prosperous, Without thieves etc, away from rocks, fire, water, Solitude (12). The Building/structure: Small door, without any window, burrow or slant/tilt, neither high, low or wide, Free from pests and insects and animals, on the outside: small hall, altar, well. Surrounding wall (13). What to do in such a place?  In such a place, setting aside all worries, one should only follow the instructions of the teacher. (14)

\newpage

\textbf{Verses 15, 16:} Six factors each for failure and success in \textit{Haṭha}: 

\noindent
\begin{minipage}[t]{.45\linewidth}
6 Factors each for failure in \textit{Haṭha} ---
\begin{enumerate}
\item Over eating 
\item Over exertion 
\item Excessive talk
\item Observance of unsuitable disciplines 
\item Company with people 
\item Unsteadiness  (15)
\end{enumerate}
\end{minipage}
\smallskip~
\begin{minipage}[t]{.45\linewidth}
6 Factors each for failure/success in \textit{Haṭha} ---
\begin{enumerate}
\item Enthusiasm-zeal ---\hfil\break  strong will to succeed 
\item Forging ahead --- not worrying too much 
\item Courage --- in not getting discouraged by setbacks
\item Being  endowed with awareness of the ultimate goal --- the consciousness  
\item faith/firmness 
\item Not mixing too much with people (16) 
\end{enumerate}
\end{minipage}

\subsection*{2. Salient Points of Part 2 --- Chapter 1}

\textbf{Verses  17--54 :} \textbf{Āsana-s:} Fifteen \textit{Āsana-s} are presented here. The fifteen \textit{Āsana-s} are :

\begin{multicols}{2}
\begin{enumerate}
\itemsep=0pt
\item \textit{Svastikāsana (19)}
\item \textit{Gomukhāsana (20)}
\item \textit{Vīrāsana  (21)}
\item \textit{Kūrmāsana (22)}
\item \textit{Kukkuṭāsana (23)}
\item \textit{Uttānakūrmakāsana (24)}
\item \textit{Dhanurāsana (25)}
\item \textit{Matsyendrāsana (26,27) }
\item \textit{Paścimatānāsana (28,29)}
\item \textit{Mayūrāsana (Hand balance) (30,31) }
\item \textit{Śavāsana (32)  	 		}
\item \textit{Siddhāsana-5\hfil\break Variations  (35--43) }
\item \textit{Padmāsana-3\hfil\break Variations (44--49)}
\item \textit{Siṃhāsana (50--52)   }
\item \textit{Bhadrāsana  (53,54)}
\end{enumerate}
\end{multicols}

Āsanas 1--11 are presented as one set and \textit{Āsana-s} 12--15 are presented as another set. The second set is stated to be very important. Even among the second set of \textit{Āsana-s}, the 12th \textit{Āsana}, \textit{Siddhāsana} is glorified as the greatest among the \textit{Āsana-s}. It is to be noted that – nomenclature and method of performing of \textit{Āsana-s} 1-7 are given. Nomenclature, method of performing and also benefits are given for \textit{Āsana-s} 8--15.  The types of \textit{Āsana-s} given here are \textit{Āsana-s} in seated position (1--9, 12--15), an \textit{Āsana}  commenced from a kneeling postion (10) and an \textit{Āsana} in lying postion (11). It is also to be noted that five variations of \textit{Siddhāsana} and three variations of \textit{Padamāsana} can be seen from the text and commentary.

\subsection*{3. Salient Points  of Part 3  --- chapter 1}

\textbf{Verses 55--66 :} This salient aspects of the verses of this section is summarized in the form of six short questions and answers for easy and quick understanding.

\begin{enumerate}
\item What is the sequence of \textit{Haṭha} practices?  \textit{Āsana}, \textit{Prāṇāyama}, \textit{Mudrā}, \textit{Nādānusandhāna} is the sequence (56)
\item How long will it take to succeed in \textit{Haṭha}? If a person is celibate, part takes moderate food, has a dispasstionate nature, a regular practitioner – success in \textit{Yoga} is achived in one year (57).
\item What is the diet for a \textit{Haṭhayogin}? (includes prescribed as conducive and proscribed as unconducive) (58-63)   

\textbf{Prescribed as conducive:} Pleasant/agreeable and sweet is conducive. Further while eating food  one should not eat to fill. 1/4th  empty space should be there in the abodomen. The food should be consumed as an offering to indwelling \textit{Śiva} (divinity). (58). Wheat, barley, milk, ghee, palm sugar, butter, honey, dry ginger, green gram etc  are also conducive(62, 63)

\textbf{Prescribed as unconducive:} Food that is bitter, sour, pungent, salty, generates heat (jaggery), green vegetables, sour gruel, (Sesame or mustard) oil, sesame, mustard, alcohol, fish, meat including that of goat, curds, buttermilk, horse gram, fruit of jujube (bore -ilanthai), oil cakes, asafoetiḍā and garlic. (59) Reheated food, dry, excess of salt or sourness, food that is pungent , too much of vegetables are to be avoided. (60)
\item What are the other precautions at the initial stages?    Fire during winter (avoid luxury), sexual intercourse, journeys, bathing early in the morning, fasting and other hard physical activity should be avoided. (61)  
\item What is the age and condition for taking up \textit{Haṭha-Yoga?} Youth, old, extremely old, sick and weak can also succeded in \textit{Yoga}. Practice without lethargy leads to success. (64)
\item How to succeed in \textit{Haṭhayoga?} The one who practices in a sustained way succeeds. Mere reading of \textit{Yoga} texts is of no avail. (65) Wearing a \textit{Yogic} costume and eloquently talking about it will not help. Only practice will lead to success. (66) 
\end{enumerate}

\section*{Chapter 2}

The contents of the second chapter can be seen in five parts. It is as follows:  

\begin{enumerate}
\item Need  of \textit{Prāṇāyāma} ---  verses 1--3 
\item \textit{Nāḍīśodhana}  - verses 4--20
\item \textit{Ṣaṭkarmas} and \textit{Gajakaraṇī} --- verses 21--38
\item Importance of \textit{Prāṇāyāma} and \textit{Aṣṭakumbhakas} (eight techniques of \textit{Prāṇāyāma} ) --- verses 39--70
\item Classification of \textit{Prāṇāyāma} and indicators of  success in \textit{Haṭha} verses 71--78
\end{enumerate}

\begin{enumerate}
\item \textbf{Verses 1--3:} \textbf{Need of \textit{Prāṇāyāma}} : Importance of \textit{Prāṇāyāma} in these verses, \textit{Prāṇāyāma} is advised  to be practiced after attaining  firmness is \textit{Āsana} to regulate the mind  and also to live long. The importance of taking the guidance of a Guru in  \textit{Prāṇāyāma} is emphasized.  

\item \textbf{Verses 4--20:} \textbf{Nāḍīśodhana}: The need to cleanse the Nāḍīs, that are loaded with impurities, through \textit{Prāṇāyāma} ( \textit{Nāḍīśodhana}) to facilitate the movement of \textit{Prāṇa} in \textit{Suṣumnā} is highlighted in three verses (4--6). The technique of \textit{Nāḍīśodhana}\footnote{(inhaling through left nostril --- Holding the breath --- Exhaling through the Right Nostril and Inhaling through the right nostril --- Holding the breath --- Exahling through right nostril)} sitting in \textit{Padmāsana} is stated (7--10). The need to practice \textit{Nāḍīśodhana} for four times (morning- midday-evening-midnight) a day and eighty cycles per practice is mentioned (11). Indicators of progress in this practice like sweating\footnote{Rubbing such sweat on to one’s body to experience lightness and firmness in the body is stated in verse 13}, trembling and experiencing \textit{Prāṇa}in \textit{Suṣumnā} (12) are stated thereafter. Diet of milk and rice conducive for \textit{Nāḍīśodhana} is stated (14). Practicing \textit{Nāḍīśodhana} slowly and systematically is emphasized. (15,16).  Consequences of wrong practice --- Hiccups, asthma, bronchial diseases, pain in the head, ears and eyes (17) are indicated. The need to practice the components of \textit{Prāṇāyāma} correctly (inhale --- to one’s capability, hold --- with \textit{Bandha-s}  and exhale --- slowly) is emphasized (18).  The indicators of cleansed \textit{Nāḍī-s} such as leanness and glow in the body, ability to hold breathe comfortably, increase of gastric fire, freedom from illnesses and experiencing \textit{Nāda} (inner \textit{Yogic} (\textit{Anahata}) sounds) are stated (19, 20).

\item \textbf{\textit{Ṣaṭkarma-s} and \textit{Gajakaraṇī}} - Verses  21-38

The \textit{Ṣaṭkarma-s} (six practices)  that would cleanse the system from excess \textit{Kapha} and \textit{Medas} (fat) is stated. This is stated as a prerequisite from \textit{Prāṇāyāma}. If there is no excess of \textit{Kapha} and \textit{Medas} these are not required to be practiced.  (21) The six cleansing practices are described in detail with their benefits (22--35). The six practices are  -
\begin{enumerate}
\item \textit{Dhauti} (swallowing a wet and clean cloth and drawing it back --- cleansing of the path of food), 
\item \textit{Vasti} (standing in water and intake of water through anus and cleansing the anus region)          
\item \textit{Netī} (introducing a soft, wet tube through the nostril to cleanse nasal passage),                                  
\item \textit{Traṭaka} (staring at a very small object without blinking eyes, cleansing eyes)                            
\item \textit{Naulika} (moving the abdomen in a circular way from right to left and cleaning the abdominal region), 
\item \textit{Kapālabhāti} (rapid inhalation and exhalation focusing on cleansing the sinus region).
\end{enumerate}

It is also clarified that according to certain teachers, that there is no need for \textit{Ṣaṭkarma-s}. By \textit{Prāṇāyāma} all impurities are removed (37). Another Standalone practice called \textit{Gajakaraṇī} is stated. It is a practice by which the food and water in the stomach are voluntarily vomited to cleanse the abdominal region. (38) 

\item \textbf{Importance of \textit{Prāṇāyāma} and \textit{Aṣṭakumbhakas} --- verses 39--70}

The efficacy of \textit{Prāṇāyāma} in overcoming death, ensuring longevity, regulating the mind, making the \textit{Prāṇa}flow in \textit{Suṣumnā} and by that stabilizing the mind and gradually attaining \textit{Samādhi} (unmanī state) is mentioned (39--43)

After this, the eight \textit{Prāṇāyāma} techniques that are called as \textit{Aṣṭakumbhaka-s} are stated, with their nomenclature, technique of practice and benefits (44--70)\footnote{In verses 45--47 the need to practice the three \textit{Bandha-s} (mūlabandha, uḍyānabandha and jālandharabandha) during \textit{Prāṇāyāma} techniques  (while holding the breath within) is emphasized to facilitate the flow of \textit{Prāṇa}in \textit{Suṣumnā}.}. This can be called as the core part of this chapter\footnote{Additionally, In the Jyotsnā Commentary to Verse 48, the ideal/beneficial daily routine for the practitioner of Hatha is also described}. The eight \textit{Kumbhaka} practices are:

\begin{enumerate}
\item \textit{Sūryabhedana}
\item \textit{Ujjāyī}
\item \textit{Sītkārī}
\item \textit{Śītalī}
\item \textit{Shastrikā}
\item \textit{Bhrāmarī}
\item \textit{Mūrchā}
\item \textit{Plāvinī}
\end{enumerate}

It is to be noted that \textit{Prāṇāyāma} techniques 1-6 have stated to bestow therapeutic benefits. Techniques 7 and 8 grant capabilities to stay as if in a state of stupor (\textit{Mūrchā}) and also float on water respectively.

\item \textbf{Classification of \textit{Prāṇāyāma} and indicators of success in \textit{Haṭha} - verses 71--78}
\end{enumerate}

Three components of \textit{Prāṇāyāma} (\textit{Recaka, Pūraka} and \textit{Kumbhaka}) and two types (sahita and kevala) of \textit{Prāṇāyāma} s are stated. The need to use Sahita \textit{Kumbhaka} to move to the state of Kevala \textit{Kumbhaka} is emphasized. It is also stated that excellence in \textit{Kumbhaka} leads to awakening of \textit{Kuṇḍalinī}. Awakening of \textit{Kuṇḍalinī}  opens the path of \textit{Suṣumnā} (for \textit{Prāṇa}). By this \textit{Haṭha} is achieved (the combining of \textit{Prāṇa}and \textit{Apāna} and their entry into \textit{Suṣumnā}).  From this, the state of \textit{Rājayoga} (\textit{Samādhi}) is attained. (71--77). The chapter ends with the  enumeration of eight fold indicators of success in \textit{Haṭha} that include slimness, brightness, manifestation of inner sound, clear eyes, freedom from disease, semen control (sensuality control), stimulation of digestive fire and purification of \textit{Nāḍi} (78). 

\section*{Chapter 3}

The third chapter can be seen in two divisions –

\begin{enumerate}
\item About awakening of \textit{Kuṇḍalinī}  --- verses 1--5
\item Ten \textit{Mudrā-s} --- verses 6--130 (the list, details and benefits)
\end{enumerate}

\subsection*{1) About awakening of \textit{Kuṇḍalinī}  --- verses 1--5}

Herein the foundational nature of (awakening) \textit{Kundalni} to \textit{Haṭha} practices (1), the need of \textit{Guru’s} grace in awakening \textit{Kuṇḍalinī}  (2), the fact that \textit{Suṣumnā} becomes free by awakening \textit{Kuṇḍalinī}  (3), Seven synonyms of \textit{Kuṇḍalinī}  (4) and the need to practice \textit{Mudrā-s} to awaken \textit{Kuṇḍalinī}  (5) are discussed. 

\subsection*{2) Ten \textit{Mudrā-s} --- verses 6--130 (the list, details and benefits)}

The Ten \textit{Mudrā-s} are enlisted and elaborated with their benefits in this section. Initially the list of the \textit{Mudrā-s} are given and it is stated that these are directly given by \textit{Ādinātha} (Śiva) and hence are to be protected secretly (6--9). The ten \textit{Mudrā-s} are then elaborated systematically. The ten  \textit{Mudrā-s}  and the distribution of verses in their description is given below -  

\begin{enumerate}
\item \textit{Mahāmudrā (10--18)}
\item \textit{Mahābandha (19--25) }
\item \textit{Mahāvedha (26--31) }
\item \textit{Khecarī (32--54) }
\item \textit{Uḍyāna-bandha (55--60)}
\item \textit{Mūlabandha (61--69)}
\item \textit{Jālandharabandha (70--76)}
\item \textit{Viparīta-karaṇī (77--82) }
\item \textit{Vajorlī/sahajoil/amaroli (83--103) }
\item \textit{Śakticālana (104--126)}
\end{enumerate}

Towards the end of the chapter, the need to be attentive/mindful in the practices related to breathing is stressed. The efficacy of the \textit{Mudrā-s} in giving various powers is stated and the importance of learning \textit{Mudrā-s} from a \textit{Guru} is emphasized (127--130) 

\textbf{Note: It is to be noted that in Krishnamacharya \textit{Yoga} Mandiram only five \textit{Mudrā-s} are practiced. Of the five, three are \textit{Bandha-s} (\textit{Uḍyānabandha, Mūlabandha} and \textit{Jālandharabandha}). The other two are \textit{Mahāmudrā} and \textit{Viparīta-karaṇī}. The other five practices are stated as \textit{Vāmācāra} (self-mortifying practices) by Śrī Krishnamacharya and have not been taught or prescribed.}

\section*{Chapter 4}

The fourth chapter can be seen in four divisions. They are as follows:

\begin{enumerate}
\item \textit{Samādhi} --- 1--33
\item \textit{Laya} practices 29--64
\item \textit{Nādānusandhāna} --- verses 65--102
\item Transcendental experiences of a \textit{Yogin} in \textit{Samādhi} and concluding advice ---  verses 103--114
\end{enumerate}

A brief overview of the chapter is present below based on the four divisions.

\begin{enumerate}
\item \textbf{Verses 1--28:  \textit{Samādhi} :} The initial verses are in the form of Saluation to Lord \textit{Śiva} and it also stated that \textit{Samādhi-krama} will be elaborated (1 \& 2). Fifteen synonyms of \textit{Samādhi} are stated in two verses after this (3,4). The definition of \textit{Samādhi} as --- \textit{Samattva} and also union of the \textit{Jīvātman} and \textit{Paramātman} are presented. Dissolution of salt in water is given as an example to illustrate the dissolution of mind in the \textit{Ātman}, which is \textit{Samādhi}.(5--7). The need for grace of \textit{Guru} is mentioned (8,9). The need of practices of \textit{Āsana-s}, \textit{Prāṇāyāma} and \textit{Mudrā-s} (\textit{Haṭha} practices) leading to awakening of \textit{Kuṇḍalinī}  and thereby facilitating the entry of \textit{Prāṇa}into \textit{Suṣumnā} is summarized. This is presented as the path that leads to the state of \textit{Samādhi}. It is emphasized that only the practices of \textit{Haṭha} through the absorption(\textit{Laya}) of \textit{Prāṇa}and \textit{Manas} lead to  knowledge and liberation (10--15). After this a few teachings on absorption of \textit{Prāṇa}(\textit{Prāṇa-laya}) is described. The intimate connection of \textit{Prāṇa} and \textit{Manas} are described with the example of the mixture of water and milk. It is also stated that \textit{Vāsanas} and \textit{Prāṇa}are the two causes that activate the citta. If one is removed the other is also taken care of and the mind settles down. The fickleness of the mind is compared to the mercury and the benefits of control of both Mercury (\textit{Rasa}) and \textit{Prāṇa} are described. This section ends with an emphasize on the need to regulate/stabilize the \textit{Prāṇa}, mind, senses and strengthen the body.(16--28)

\item \textbf{Verse 29--64: \textit{Laya}:} \textit{Laya} is introduced in this section.  The absorption (Laya) of the mind and the \textit{Prāṇa}itself is eulogized as the state of liberation as it grants peace and happiness. \textit{Laya} is praised in various verses and defined as (due to absorption/stabilization of the mind/\textit{Prāṇa}) non arousal of \textit{Vāsana-s} and forgetting of the objects of the senses. (29--34) \textit{Śāmbhavī} \textit{mudrā} that leads to \textit{Prāṇa-laya} through \textit{Citta-laya} is stated. (35--42). After this, another type of Khecarī \textit{mudrā} is stated as a \textit{Prāṇalaya} method (43--53). After the two practices of \textit{Prāṇalaya}, a few methods of \textit{Manolaya} is stated such as meditating upon the \textit{Kuṇḍalinī}  (54), meditating upon the space/\textit{Brahman} (55). After this, a few verses describe the need to focus and stabilize the mind. The need to make it  dissolvein the object of meditation like the camphor in the fire is presented (56-64).

\item \textbf{Verses 65--102:} \textit{\textbf{Nādānusandhāna}}: This is the central portion of the chapter that expounds \textit{Nādānusandhāna}, the chief and best among the \textit{Laya} practices. This is a Manolaya practice. To glorify this practice it is initially stated that even those simple aspirants who cannot attain philosophical clarity intellectually, can practice \textit{Nādānusandhāna}. (65--67). Two types of \textit{Nādānusandhāna} practices are presented in this text. The first type is in four stages \textit{Ārambha} (commence), \textit{Ghata} (consoliḍāte), \textit{Paricaya} (intro to higher stages), \textit{Niṣpatti} (reaching the goal). Here a \textit{Yogin} is expected to sit in \textit{Muktāsana} assuming the \textit{Śāmbhavī} \textit{Mudrā} and start focusing on the \textit{Nāda} manifesting in the right ear with a focused mind.  As the mind is focussed, the \textit{Yogin} experiences the aforementioned four stages where sounds of varying nature is heard at four various places beginning from \textit{Anāhatacakra} (midchest) to \textit{Brahmarandhra} (crown region of the head).(68--81).

The second type of \textit{Nādānusandhāna} is stated through the principles of \textit{Pratyāhāra} implying moving from grosser sounds to subtler and refined sounds. Three stages of focusing on \textit{Nādas} are presented with  no specifc place of focus. The three sets of sound that are to be focussed are, the sounds of --- a) Ocean, cloud, kettle drum, b) then sounds of conch, bell and horn c) and finally sounds of flute, vina and buzzing sound of the bees.(82--89). The next set of verses speak about how the mind is captivated by this practice of \textit{Nādānusandhānana} using exampels from the nature like bees, elephants, deers, horses and serpents. Finally it is stated that, as the fire gets extinguished when the fuel is exhausted, similarly, the mind that is captivated in \textit{Nāda} gets dissolved when Nāda dies down by Paravairāgya (90--102).
 
\item \textbf{Verses 103-114 :} \textbf{Transcendental experiences of a \textit{Yogin} in\break \textit{Samādhi} and concluding advice:} The transcendental experiences of a \textit{Yogin} who experiences \textit{Samādhi} through aaforementioned practices is stated in this section. The Yogins freedom from objects of senses, freedom from relative states of awareness, firmness in the state of awareness are described. A \textit{Yogin} which such experiences is stated as \textit{Jīvan-mukta} (liberated while being alive). (103--113). The chapter and the text ends with a strong parting advice with emphasize upon the importance of practice to attain the goals of \textit{Haṭhayoga}. The last verse is as follows - “As long as \textit{Prāṇa}does not flow in \textit{Suṣumnā} and reach \textit{Brahmarandhra}, as long as sensual desires are not overcome, as long as the mind does not reflect the natural state (\textit{Svarūpa}) so long, those who talk of spiritual knowledge indulge only in boastful and false prattle (114)   
\end{enumerate}
