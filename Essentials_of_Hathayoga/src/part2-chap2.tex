\chapter{\textit{Prāṇāyāma}}

\subsection*{Summary}

\section*{I. \textit{Prāṇāyāma-s} (\textit{Kumbhaka-s}) in \textit{Haṭhayogapradī\-pikā}}

\textit{Sūryabhedanam, Ujjāyī, Sītkārī, Śītalī, Bhastrikā, Bhrāmarī, Mūrcchā And Plāvinī Are The Eight Prāṇāyāmas/kumbhakas.} 2.44

\heading{\textit{Prāṇāyāma -- 1: Sūryabhedanam}}

A practitioner of \textit{Yoga} should be seated in a comoftable seat. After assuming an \textit{Āsana} (posture) the air from outside has to be inhaled through the right nostril.  Then having the hair on the head and the tip of the nails of the fingers of the feet as the limit (as the air held within is felt all over the body mentioned in this manner) the breath should be held. Then slowly the through the left nostril the air has to be exhaled.  2.48,49

\heading{Benefits of \textit{Sūryabhedanam}}

This best practice cleanses the skull region, removes diseases caused by vititation of \textit{Vāta} and worms (in the abdomen). Hence this has to be done again and again. 2.50

\heading{\textit{Prāṇāyāma} -- 2: \textit{Ujjāyī}}

With closed mouth the air has to be drawn slowly through the nostrils till the sound is felt from throat to the chest/heart. One has to hold the breath as stated earlier (in \textit{Sūryabhedana}) and exhale through the \textit{Iḍā nāḍī} / left nostril. 2.51

\heading{Benefits of \textit{Ujjāyī}}

This \textit{Prāṇāyāma} called \textit{Ujjāyī} stimulates (digestive) fire in the body, removes diseases caused by \textit{Kapha}, and removes problmes related to \textit{Nāḍī-s}, water (infection), stomach and stomach. This can be done by a person who is standing or walking.  2.52,53

\heading{\textit{Prāṇāyāma -- 3: Sītkārī}}

Inhalation should be done throught he mouth while making the sound \textit{‘Sīt’} and exhalation should be done through the nostril(s) only. 2.54-1

\heading{Benefits of \textit{Sītkārī}}

By practicing in this manner, one shall become \textit{Kāmadeva} (the god of love – handsome). He will be respected by all the \textit{Yoginis} and becomes the creator and annihilator. There will be no hunger, thirst, sleep or laziness. By this practice, the body also becomes strong and a person will beomce king among the practitioners on the earth and will be free from any sufferings.   2.54.2-56

\heading{\textit{Prāṇāyāma -- 4:  Śītalī}}

An intelligent person should inhale the air (slowly) through the tongue and hold the breath as earlier (as in \textit{Sūryabhedana}). Then, the air has to be exhaled through both the nostrils slowly.  2.57

\heading{Benefits of \textit{Śītalī}}

This \textit{Śītalī}, which is a well known practice of \textit{Prāṇāyāma}, completely removes \textit{gulma}\footnote{Gulma refers to abdominal pain due to various causes including formation of mass of cells, enlargement of abdominal glands. Pliha refers to the enlargement of spleen.}, pliha (and such other diseases), hunger, thirst and poisons.  2.58

\heading{\textit{Prāṇāyāma -- 5: Bhastrikā}}

One should stay seated placing the upward facing soles of feet on the (opposite) thighs. This is \textit{Padmāsana} that removes all Sins(59). Aligning the neck and the adomen (in a straight line) a steady-minded person having firmly assumed \textit{Padmāsana}, should exhale Prāṇa with sound through one nostril (the right nostril) with effort, sealing the mouth firmly(60). The breath being exhaled should be felt from the chest-throat region till the skull region. Then, the breath has to be inhaled quickly (through one/right nostril) (61). In this manner repeatedly exhalation and inhalation has to be done. Just like the bellows of the backsmith which is operated rapidly (62) the breath in the body has to be operated mindfully. When tiredness is experienced in the body, then inhale through the right nostril(63). One should inhale in the manner by which the abdomen is filled with air quickly. One should (then) hold the nose with the fingers excluding the middle and index fingers(64). After holding the breath as per injunctions (stated while explaining \textit{Sūryabhedana}) one should exhale through the left nostril. 2.58-65.1

\heading{Benefits of \textit{Bhastrikā}}

This (\textit{Bhastrikā}) removes (imbalances) Vata, Pitta and Kapaha. It enhances the bodily (digestive) fire. It also destroys the obstructions like Kapha that lie in the mouth of the \textit{Suṣumnā Nāḍī}. Bhastrikā also pierces the three granthis (knots) that are firmly entrenched in the body. (Hence) this practice called Bhastrikā, has to be done especially. 2.65-2 - 67

\heading{\textit{Prāṇāyāma -- 6: Bhrāmarī}}

Inhalation has to be done rapidly with sound that resembles the sound of a male bee. Exhalation has to be done slowly, slowly with the sound resembling the sound of a female bee. 2.68-1

\heading{Benefits of \textit{Bhrāmarī}}

By being involved in the practice in this manner, an inexplicable bliss manifests in the mind of the best among the \textit{Yogins}.

\heading{Other Practices -- \textit{Nāḍīśodhana} and its Benefits}

A \textit{Yogi} seated in the \textit{Padmāsana} should inhale through the left nostril. After holding the breath to the level of one’s capability one should exhale through the right nostril(7). (Then) The abdomen has to be filled slowly with air by inhaling through the right nostril and after holding the breath as per injunctions (with \textit{Bandha-s}) one should exhale through the left nostril (8). One has to inhale through that nostril by which exhalation has been done. Then one should hold the breath with great intensity and exhale the breath slowly and not quickly, through the other nostril (9).  If the breath is inhaled through the left nostril then after holding the breath, exahaltion has to be done thorugh the other nostril. Then through the right nostril the breath has to be inhaled. After holding the breath it has to be exhaled through the left nostril. This activity of the Sun and moon (the left and right-nostrils) done by the self-restrained practitioners will result in the cleansing of the \textit{Nāḍī-s} after three months. 

\heading{\textit{Kapālabhāti} and its Benefits}

Like the bellows of the blacksmith - exhaling and inhaling rapidly is the well known \textit{Kapāla-bhāti}. It dries out the vitiation of \textit{Kapha}. 2.35
\newpage

\thispagestyle{empty}
~
\vfill
\begin{center}
\textbf{\Huge Textual Immersion}
\end{center}
\vfill
\eject

\section*{I. \textit{Prāṇāyāma-s} (\textit{Kumbhaka-s}) in \textit{Haṭhayogapradīpikā}}

\noindent 
\textbf{Verses 2.44}

\begin{verse}
\textit{sūryabhedanamujjāyī sītkārī śītalī tathā |\\
bhastrikā bhrāmarī mūrcchā plāvinītyaṣṭakumbhakāḥ ||}
\end{verse}

\subsection*{Word Split}

\textit{Sūryabhedanam Ujjāyī Sītkārī Śītalī Tathā, Bhastrikā, Bhrāmarī, Mūrcchā, Plāvinī, Iti, Aṣṭakumbhakāḥ}

\subsection*{Paraphrased Word Meaning}

\textit{Sūryabhedanam, Ujjāyī, Sītkārī, Śītalī, Bhastrikā, Bhrāmarī, Mūrcchā}

\begin{multicols}{2}
\textit{Tathā} --- and \\
\textit{Plāvinī}  \\
\textit{Iti}  --- thus are \\
\textit{Aṣṭa} --- eight \\
\textit{Kumbhakāḥ} --- \textit{Prāṇāyāmas} 
\end{multicols}

\subsection*{Purport}

\textit{Sūryabhedanam, Ujjāyī, Sītkārī, Śītalī, Bhastrikā, Bhrāmarī, Mūrcchā And Plāvinī Are The Eight prāṇāyāmas/kumbhakas.}


\section*{\textit{Prāṇāyāma -- 1: Sūryabhedanam}}

\noindent \textbf{Verses 2.48. 49}

\begin{verse}
\textit{āsane sukhade yogī baddhvā caivāsanaṁ tataḥ |\\
dakṣanāḍyā samākṛṣya bahiḥsthaṁ pavanaṁ śanaiḥ ||\\
ā keśādānakhāgrācca nirodhāvadhi kumbhayed |\\
tataḥ śanaiḥ savyanāḍyā recayet pavanaṁ śanaiḥ ||}
\end{verse}

\subsection*{Word Split -- Verse 2.48}

\textit{Āsane, Sukhade, Yogi, Badhvā, Ca, Eva, Āsanaṁ Tataḥ, Dakṣanāḍyā Samākṛṣya Bahiḥsthaṁ Pavanaṁ, Śanaiḥ}

\subsection*{Paraphrased Word Meaning}


\noindent \textbf{Verses 48}

\begin{multicols}{2}
\textit{Yogi} --- A practitioner of Yoga \\
\textit{Sukhade} --- in a comfortable \\
\textit{Āsane} --- seat \\
\textit{Āsanaṁ} --- posture  \\
\textit{Ca} --- and \\
\textit{Baddhvā} --- assuming\\
\textit{Eva} --- only\\
\textit{Tataḥ} --- then\\ 
\textit{Bahiḥsthaṁ} --- external\\
\textit{Pavanaṁ} --- the air \\
\textit{Dakṣa-nāḍyā} --- the right nostril/\textit{Nāḍī}\\
\textit{Śanaiḥ} --- slowly \\
\textit{Samākṛṣya} --- after drawing in
\end{multicols}

\subsection*{Word Split – Verse 2.49}

\textit{Ā, Keśād, Ānakhāgrāt, Ca, Nirodhāvadhi, Kumbhayed, Tataḥ Śanaiḥ, Savyanāḍyā, Recayet, Pavanaṁ Śanaiḥ}

\noindent \textbf{Verses 49}

\begin{multicols}{2}
\textit{Ā  keśād}--- from the hair \\
\textit{Ā nakhāgrāt} --- till the tip of the nails (of the fingers of the feet) \\
\textit{Ca}  --- and \\
\textit{Nirodhāvadhi} --- as the limits of Nirodha \\
\textit{Kumbhayed} --- should hold the breath\\
\textit{Tataḥ} --- then/after that \\
\textit{Savyanāḍyā} --- through the left nostril/\textit{Nāḍī} \\
\textit{Śanaiḥ} --- slowly \\
\textit{Śanaiḥ} --- slowly \\
\textit{Pavanaṁ} --- the air \\
\textit{Recayet} --- should exale 
\end{multicols}

\subsection*{Purport}

A practitioner of \textit{Yoga} should be seated in a comoftable seat. After assuming an \textit{Āsana} (posture) the air from outside has to be inhaled through the right nostril.  Then having the hair on the head and the tip of the nails of the fingers of the feet as the limit (as the air held within is felt all over the body mentioned in this manner) the breath should be held. Then slowly the through the left nostril the air has to be exhaled. 

\subsection*{Inputs from the \textit{Jyotsnā} Commentary}

\begin{enumerate}
\itemsep=0pt
\item The seat for \textit{Prāṇāyāma} can constitue cloth, dry \textit{Darbha} grass mat etc
\item \textit{Svastikāsana}, \textit{Virāsana} or \textit{Padmāsana} is preferable. Or \textit{Siddhāsana} is the best for \textit{Prāṇāyāma}. 
\item Right \textit{Nāḍī} refers to \textit{Piṅgala} and the left \textit{Nāḍī} refers to \textit{Iḍā}. 
\item Holding the breath should be so intense that \textit{Prāṇa} is felt all over the body. 
\item Though it has been stated that \textit{Kumbhaka} has to done with intense effort, the intensity here refers to the intensity of intention – to succed in \textit{Prāṇāyāma} and it does not refer to forcing the breath. Holding the breath has to be done gradually. The capability has to be built gradually – like taming an elephant or a lion. 
\item Exahalation has to be done only slowly. If one exhales quickly it - may result in loss of strength (\textit{Bala-hani}). To emphasize this, the word \textit{Śanaiḥ} is repated twice.       
\end{enumerate}
\newpage
	
\subsection*{Benefits of \textit{Sūryabhedanam}}


\noindent \textbf{Verses 2.50}

\begin{verse}
\textit{kapālaśodhanaṁ vātadoṣaghnaṁ kṛmidoṣahṛt |\\
punaḥ punariḍāṁ kāryaṁ sūryabhedanamuttamam || 50 ||}
\end{verse}

\subsection*{Word Split}


\textit{Kapālaśodhanaṁ, Vātadoṣaghnaṁ, Kṛmidoṣahṛt, Punaḥ, Punaḥ, Iḍāṁ, Kāryam, Sūryabhedanam, Uttamam}

\subsection*{Paraphrased Word Meaning}


\begin{multicols}{2}
\textit{Iḍāṁ} --- This\\
\textit{Uttamam} --- best\\
\textit{Sūrya-bhedanam} \\
\textit{Kapāla-śodhanam} --- cleanses the skull region\\
\textit{Vāta-doṣa-ghnam} --- destroys diseases caused by \textit{Vāta}\\
\textit{Kṛmi-doṣa-hṛt }---  removes diseases caused by worms \\
\textit{Punaḥ} --- again \\
\textit{Punaḥ} --- again \\
\textit{Kāryaṁ} --- should be done
\end{multicols}

\subsection*{Purport}

This best practice cleanses the skull region, removes diseases caused by vititation of \textit{Vāta} and worms (in the abdomen). Hence this has to be done again and again.

\subsection*{Inputs from \textit{Jyotsnā} Commentary}

\begin{enumerate}
\itemsep=0pt
\item 80 types of \textit{Vāta} related diseases are addressed by this practice. 
\item The worms here refer to the worms in the abdomen. Diseases caused by them are overcome by this practice.
\end{enumerate}

\section*{\textit{Prāṇāyāma – 2 Ujjāyī}}

\noindent \textbf{Verses 2.51}

\begin{verse}
\textit{mukhaṁ saṁyamya nāḍībhyām ākṛṣya pavanaṁ śanaiḥ |\\
yathā lagati kaṇṭhāttu hṛdayāvadhi sasvanam |\\
pūrvavat kumbhayet prāṇaṁ recayediḍayā tathā || | 51 ||}
\end{verse}

\subsection*{Word Split}

\textit{Mukham, Saṁyamya, Nāḍībhyām, Ākṛṣya, Pavanaṁ, Śanaiḥ, Yathā, Lagati, Kaṇṭhāt, Tu, Hṛdayāvadhi, Sasvanam , Pūrvavat, Kumbhayet, Prāṇam, Recayed, Iḍayā, Tathā}

\subsection*{Paraphrased Word Meaning}

\begin{multicols}{2}
\textit{Mukham} --- mouth\\
\textit{Saṁyamya} --- having closed\\
\textit{Kaṇṭhāt} --- from the throat\\
\textit{Hṛdayāvadhi} --- till the chest\\
\textit{Yathā} --- as\\
\textit{Sa-svanam} --- with (hissing) sound\\
\textit{(pavanaḥ)} --- air\\
\textit{Lagati} --- touches\\
\textit{(tathā)} --- in that manner\\
\textit{Nāḍībhyām} --- by both the nostrils\\
\textit{Pavanaṁ} --- air\\
\textit{Śanaiḥ} --- slowly\\
\textit{Tu }--- (filler)\\
\textit{Ākṛṣya} --- having drawn in \\
\textit{Pūrvavat} --- as earlier \\
\textit{Prāṇam} --- the breath \\
\textit{Kumbhayet} --- hold \\
\textit{Tathā} --- and \\
\textit{Iḍayā} --- through left nostril\\
\textit{Recayed} --- (one) should exhale
\end{multicols}

\subsection*{Purport}

With closed mouth the air has to be drawn slowly through the nostrils till the sound is felt from throat to the chest/heart. One has to hold the breath as stated earlier (in \textit{Sūryabhedana}) and exhale through the \textit{Iḍā Nāḍī} / left nostril.    
\newpage

\heading{Benefits of \textit{Ujjāyī}}

\noindent \textbf{Verses 2.52,53}

\begin{verse}
\textit{śleṣmadoṣaharaṁ kaṇṭhe dehānalavivardhanam ||\\
nāḍījalodarādhātugatadoṣavināśanam |\\
gacchatā tiṣṭhatā kāryamujjāyyākhyaṁ tu kumbhakam} ||
\end{verse}

\subsection*{Word Split - Verse 2.52}

\textit{Śleṣmadoṣaharam, Kaṇṭhe, Dehānalavivardhanam}

\subsection*{Paraphrased Word Meaning}

\noindent \textbf{Verses 52}

\begin{multicols}{2}
(\textit{Ujjāyyākhyaṁ tu Kumbhakam})\\
\textit{Dehānala-vivardhanam }--- stimulates fire of the body \\
\textit{Kaṇṭhe} --- in the throat  \\
\textit{Śleṣma-doṣa-haram }--- removes the diseases associated with \textit{Kapha} 
\end{multicols}

\subsection*{Word Split - Verse 2.53}

\textit{Nāḍījalodarādhātugatadoṣavināśanam, Gacchatā, Tiṣṭhatā, Kāryam,\break Ujjāyyākhyaṁ, Tu, Kumbhakam}

\noindent \textbf{Verses 53}

\begin{multicols}{2}
\textit{Nāḍī-jalodarādhātu-gata-doṣavi\-nāśanam  }--- removes problems related to \textit{Nāḍīs}, water, stomach and the dhatus\\
\textit{Gacchatā} --- by a person who is going \\
\textit{Tiṣṭhatā} --- by a person who is standing \\
\textit{Ujjāyyākhyaṁ} --- the practice known as \textit{Ujjāyī}\\
\textit{Tu} --- indeed \\
\textit{Kumbhakam} --- \textit{Prāṇāyāma} \\
\textit{Kāryam} --- should be done
\end{multicols}

\subsection*{Purport - Verse 2.52,53}


This \textit{Prāṇāyāma} called \textit{Ujjāyī} stimulates (digestive) fire in the body, removes diseases caused by \textit{Kapha}, and removes problmes related to \textit{Nāḍī-s}, water (infection), stomach and stomach. This can be done by a person who is standing or walking.

\subsection*{Inputs from \textit{Jyotsnā} Commentary}


\begin{enumerate}
\item \textit{Kāsa} (cough) and other such \textit{Kapha} related problems are addressed by \textit{Ujjāyī} 
\item Though \textit{Ujjāyī} can be done while walking and also standing – no bandhas should be attempted during \textit{Ujjāyī} practice while standing and walking. This means that \textit{Bandha-s} can be practiced in \textit{Ujjāyī} only in the seated postion. 
\end{enumerate}
\newpage

\section*{\textit{Prāṇāyāma -- 3 Sītkārī}}

\noindent \textbf{Verses 2.54-1 }

\begin{center}
sītkāṁ kuryāt tathā vaktre ghrāṇenaiva vijṛmbhikām || 54-1||
\end{center}

\subsection*{Word Split}


\textit{Sītkām, Kuryāt, Tathā, Vaktre, Ghrāṇena, Eva, Vijṛmbhikām}

\subsection*{Paraphrased Word Meaning}


\begin{multicols}{2}
\textit{Vaktre} --- in the mouth \\
\textit{Sītkāṁ} --- \textit{‘Sīt’} sound \\
\textit{Kuryāt} --- (one) should make \\
\textit{Tathā}  --- and \\
\textit{Ghrāṇena} --- through the nostril(s)\\
\textit{Eva} --- only \\
\textit{Vijṛmbhikām }--- exhalation
\end{multicols}

\subsection*{Purport}


Inhalation should be done throught he mouth while making the sound \textit{‘Sīt’} and exhalation should be done through the nostril(s) only.

\subsection*{Inputs from \textit{Jyotsnā} Commentary}


\begin{enumerate}
\itemsep=0pt
\item The inhalation through the mouth should happen when the tongue is placed in between the lips with the sound-\textit{‘Sīt’}.
\item Though it has been stated as nostril – exhalation should be done through both the nostrils. 
\item It should be understood that the word Eva – only – is used to indicate that exhalation has to be done only through nostril and this prohibits exhalation through mouth. Even after the completion of the practice of this \textit{Prāṇāyāma} one should not exhale through the mouth as it leads to loss of energy /weakness. 
\item Though the verse above does not mention about the practice of \textit{Kumbhaka} ( holding the breath after inhalation), as this entire section is on \textit{Kumbhaka} – even without being mentioned – holding the breath after inhalation has to be practiced. 
\end{enumerate}

\subsection*{Benefits of \textit{Sītkārī}}


\noindent \textbf{Verses 2.54-2 -- 56}

\begin{verse}
\textit{evamabhyāsayogena kāmadevo dvitīyakaḥ || 54-2 ||\\
yoginīcakrasaṁmānyaḥ sṛṣṭisaṁhārakārakaḥ |\\
na kṣudhā na tṛṣā nidrā naivālasyaṁ prajāyate || 55 ||\\
bhavet sattvaṁ ca dehasya sarvopadravavivarjitaḥ |\\
anena vidhinā satyaṁ yogīndro bhūmimaṇḍale || 56 ||}
\end{verse}

\subsection*{Word Split Verse 2.54-2}


\textit{Evam, Abhyāsayogena, Kāmadevaḥ, Dvitīyakaḥ}

\subsection*{Paraphrased Word Meaning}

\begin{multicols}{2}
\textit{Evam }--- in this manner\\
\textit{Abhyāsa-yogena} --- by this yogic practice \\
\textit{Kāma-devaḥ} --- the god of love (handsome)\\
\textit{Dvitīyakaḥ }--- second (the practitioner becomes)
\end{multicols}

\subsection*{Word Split - Verse2. 55}


\textit{Yoginīcakrasaṁmānyaḥ, Sṛṣṭisaṁhārakārakaḥ, Na, Kṣudhā, Na, Tṛṣā, Nidrā, Na, Eva, Ālasyam, Prajāyate}

\subsection*{Paraphrased Word Meaning}


\begin{multicols}{2}
\textit{Yogini-cakra-saṁmānyaḥ} --- (becomes)\\
Respected by the host of \textit{Yoginis} \\
\textit{Sṛṣṭi-saṁhāra-kārakaḥ} --- (becomes) capable\\
\textit{Of} creating and annihilating  \\
\textit{Na} --- no \\
\textit{Kṣudhā} --- hunger \\
\textit{Na} --- no\\
\textit{Tṛṣā} --- thirst \\
\textit{Nidrā} --- sleep \\
\textit{Naiva} --- indeed not \\
\textit{Ālasyaṁ} --- laziness \\
\textit{Prajāyate} --- manifests 
\end{multicols}

\subsection*{Word Split -- Verse 2.56}


\textit{Bhavet, Sattvam, Ca, Dehasya, Sarvopadravavivarjitaḥ, Anena, Vidhinā, Satyam, Yogīndraḥ, Bhūmimaṇḍale}

\subsection*{Paraphrased Word Meaning}


\begin{multicols}{2}
\textit{Anena} --- by this\\
\textit{Vidhinā} --- method\\ 
\textit{Dehasya} --- of the body \\
\textit{Sattvam} --- strength \\
\textit{Ca} --- and \\
\textit{Bhavet} --- shall be \\
\textit{Bhūmi-maṇḍale} --- in the earth \\
\textit{Yogīndraḥ} --- (one becomes)king among the Yogins\\
\textit{Sarvopadrava-vivarjitaḥ} --- free from all sufferings (bhavet) \\
\textit{Satyam} --- (this is) true.
\end{multicols}

\subsection*{Purport}
\vspace{-5pt}

By practicing in this manner, one shall become \textit{Kāmadeva} (the god of love – handsome). He will be respected by all the \textit{Yoginis} and becomes the creator and annihilator. There will be no hunger, thirst, sleep or laziness. By this practice, the body also becomes strong and a person will beomce king among the practitioners on the earth and will be free from any sufferings.

\subsection*{Inputs from \textit{Jyotsnā} Commentary}
\vspace{-5pt}

\begin{enumerate}
\itemsep=0pt
\item Due to excellence in \textit{Rūpa} (form) and \textit{Lāvaṇya} (beauty) the practitioner  will resemble \textit{Kāmadeva}. 
\item Creation and destruction of the universe is referred to here. (one becomes powerful like the almighty)
\item \textit{Ālasyam} is defined as heaviness in the body that prohibits movement/taking initiative/effort. The heaviness in the body is caused due to \textit{Kapaḥ} etc and the heaviness in the mind is due to preponderance of \textit{Tamas}. 
\item \textit{Sattva} here refers to \textit{Bala}- strength (and not \textit{Sattva-guṇa}). 
\end{enumerate}

\section*{\textit{Prāṇāyāma -- 4 Śītalī}}

\noindent \textbf{Verses 2.57}

\begin{verse}
\textit{jihvayā vāyumākṛṣya pūrvavat kumbhasādhanam |\\
śanakairghrāṇarandhrābhyāṁ recayet pavanaṁ sudhīḥ ||}
\end{verse}

\subsection*{Word Split}

\textit{Jihvayā, Vāyum, Ākṛṣya, Pūrvavat, Kumbhasādhanam, Śanakaiḥ Ghrāṇarandhrābhyāṁ, Recayet Pavanaṁ, Sudhīḥ}

\subsection*{Paraphrased Word Meaning}
\vspace{-10pt}

\begin{multicols}{2}
\textit{Sudhīḥ} --- intelligent person \\
\textit{Jihvayā} --- by the tongue \\
\textit{Vāyum} --- the air \\
\textit{(śanaiḥ}) --- (slowly)\\
\textit{Ākṛṣya} --- after drawing \\
\textit{Pūrvavat} --- as earlier \\
\textit{Kumbha-sādhanam} --- the practice of \textit{Kumbhaka} (holding the breath)\\
\textit{(kṛtvā}) --- (after doing)\\
\textit{Ghrāṇa-randhrābhyāṁ} --- by both the nostrils \\
\textit{Pavanaṁ} --- the air \\
\textit{Śanakaiḥ} --- slowly \\
\textit{Recayet} --- (one)should exhale
\end{multicols}

\subsection*{Purport}


An intelligent person should inhale the air (slowly) through the tongue and hold the breath as earlier (as in \textit{Sūryabhedana}). Then, the air has to be exhaled through both the nostrils slowly.

\subsection*{Inputs from \textit{Jyotsnā} Commentary}
\vspace{-10pt}

\begin{enumerate}
\itemsep=0pt
\item Inhaling has to be done through the tongue that is extened outside the lips and is folded (inwards) resembling the lower beak of a bird.
\item Inhalation has to be done slowly.
\item \textit{Pūrvavat} - As earlier (holding the breath) - here refers to holding the breath as has been described for \textit{Sūryabhedana}. 
\end{enumerate}

\subsection*{Benefits of Śītalī}

\noindent \textbf{Verses 2.58}

\begin{verse}
\textit{gulmaplīhādikān rogān jvaraṁ pittaṁ kṣudhāṁ tṛṣām |\\
viṣāṇi śītalī nāma kumbhikeyaṁ nihanti hi ||}
\end{verse}

\subsection*{Word Split}


\textit{Gulmaplīhādikān, Rogān, Jvaram Pittam, Kṣudhām, Tṛṣām, Viṣāṇi, Śītalī, Nāma, Kumbhikā, Iyaṁ Nihanti, Hi}

\subsection*{Paraphrased Word Meaning}


\begin{multicols}{2}
\textit{Iyaṁ} --- This \textit{Śītalī} \\
\textit{Nāma} --- well known \\
\textit{Kumbhikā} --- \textit{Prāṇāyāma}\\
\textit{Gulma-plīhādikān} --- Gulma-pliha and others \\
\textit{Rogān} --- diseases \\
\textit{Jvaram} --- fever\\
\textit{Pittam} --- Pitta (vitiation)\\
\textit{Kṣudhām} --- hunger \\
\textit{Tṛṣām} --- thirst \\
\textit{Viṣāṇi} --- poisons\\
\textit{Nihanti} --- completely removes\\
\textit{Hi} --- indeed
\end{multicols}

\subsection*{Purport}


This \textit{Śītalī}, which is a well known practice of \textit{Prāṇāyāma}, completely removes \textit{Gulma}\footnote{Gulma refers to abdominal pain due to various causes including formation of mass of cells, enlargement of abdominal glands. Pliha refers to the enlargement of spleen.}, pliha (and such other diseases), hunger, thirst and poisons.

\subsection*{Inputs from \textit{Jyotsnā} Commentary}


\begin{enumerate}
\item Illnesses caused due to \textit{Pitta} is removed (It is to be understood that \textit{Pitta} itself is not removed)
\item \textit{Viṣā} --- poison here refers to the poision of snakes etc.
\end{enumerate}

\section*{\textit{Prāṇāyāma – 5 Bhastrikā}}

\noindent \textbf{Verses 2.59-65-1}

\begin{verse}
\textit{ūrvorupari saṁsthāpya śubhe pādatale ubhe |\\
padmāsanaṁ bhavedetat sarvapāpapraṇaśanam || 59 ||\\
samyak Padmāsanaṁ baddhvā samagrīvodaraḥ sudhīḥ |\\ 
mukhaṁ saṁyamya yatnena prāṇaṁ ghrāṇena recayet || 60 ||\\
yathā lagati hṛtkaṇṭhe kapālāvadhi sasvanam |\\
vegena pūrayeccāpi hṛtpadmāvadhi mārutam || 61 ||\\
punarvirecayet tadvat pūrayecca punaḥ punaḥ |\\
yathaiva lohakāreṇa bhastrā vegena cālyate || 62 ||\\
tathaiva svaśarīrasthaṁ cālayet pavanaṁ dhiyā |\\
yadā śramo bhaveddehe tadā sūryeṇa pūrayet || 63 ||\\
yathodaraṁ bhavet pūrṇamanilena tathā laghu |\\
dhārayennāsikāṁ madhyātarjanībhyāṁ vinā dṛḍham || 64 ||\\
vidhivat Kumbhakaṁ kṛtvā recayediḍayānilam |}
\end{verse}

\subsection*{Word Split --- Verse 2.59}


\textit{Ūrvoḥ, Upari, Saṁsthāpya, Śubhe, Pādatale, Ubhe, padmāsanam, Bhaved, Etat, Sarvapāpapraṇaśanam}

\subsection*{Paraphrased Word Meaning}


\begin{multicols}{2}
\textit{Upari} --- facing upwards \\
\textit{Ubhe} --- both \\
\textit{Śubhe} --- clean\\ 
\textit{Pādatale} --- soles of feet \\
\textit{Ūrvoḥ} ---  of the thighs \\
\textit{Saṁsthāpya} --- having placed firmly \\
(vaset) --- (one should stay)\\
\textit{Etat} --- this \\
\textit{Sarva-pāpa-praṇaśanam} --- all sin-destroying \\
\textit{Padmāsanam} \\
\textit{Bhaved} --- should be
\end{multicols}

\subsection*{Word Split --- Verse 60}


\textit{Samyak, Padmāsanam, Baddhvā, Samagrīvodaraḥ, Sudhīḥ, Mukham, Saṁyamya, Yatnena, Prāṇam Ghrāṇena, Recayet}

\subsection*{Paraphrased word Meaning}


\begin{multicols}{2}
\textit{Sama-grīvodaraḥ} --- with neck and abdomen straight \\
\textit{Sudhīḥ}  --- a steady-minded person \\
\textit{Samyak} --- firmly \\
\textit{Padmāsanam} \\
\textit{Baddhvā} --- having assumed \\
\textit{Mukham} --- the mouth \\
\textit{Saṁyamya} --- having sealed/closed \\
\textit{Yatnena} --- with effort \\
\textit{Prāṇam} --- the air \\
\textit{Ghrāṇena} --- through one nostril \\
\textit{Recayet} --- should exhale 
\end{multicols}

\subsection*{Word Split --- Verse 2.61}

\textit{Yathā, Lagati, Hṛtkaṇṭhe, Kapālāvadhi, Sasvanam, Vegena, Pūrayet, Ca, Api, Hṛtpadmāvadhi, Mārutam}
\newpage
\subsection*{Paraphrased Word Meaning}

\begin{multicols}{2}
\textit{Hṛt-kaṇṭhe} --- in the chest and the throat\\
\textit{Kapālāvadhi} --- till the skull  region\\
\textit{Sa-svanam} --- with sound \\
\textit{Yathā} --- as \\
\textit{Lagati} --- is felt \\
\textit{(Prāṇah)}  (the Prāṇa)\\
\textit{(Tathā)} (in that manner)\\
\textit{(Recayet)} (one should exhale)\\
\textit{Hṛt-padmāvadhi} --- till the lotus of the heart  \\
\textit{Vegena} --- rapidly \\
\textit{Mārutam} --- the air\\ 
\textit{Pūrayet} --- (one) should inhale \\
\textit{Ca api} --- And also 
\end{multicols}

\subsection*{Word Split - Verse 2.62}


\textit{Punaḥ, Virecayet, Tadvat, Pūrayet, Ca, Punaḥ, Punaḥ,  Yathā, Eva, Lohakāreṇa, Bhastrā, Vegena Cālyate}

\subsection*{Paraphrased Word Meaning}


\begin{multicols}{2}
\textit{Tadvat}  --- in the same way\\
\textit{Punaḥ} --- again\\
\textit{Virecayet} --- one should exhale\\
\textit{Punaḥ} --- again\\
\textit{Punaḥ} --- again\\
\textit{Ca} --- and\\
\textit{Pūrayet} --- one should inhale\\
\textit{Yathā} ---  as\\
\textit{Eva} --- indeed \\
\textit{Loha-kāreṇa} --- by the blacksmith\\
\textit{Bhastrā} --- the bellows\\
\textit{Vegena} --- rapidly\\
\textit{Cālyate} --- is operated
\end{multicols}

\subsection*{Word Split --- Verse 2.63}


\textit{Tathā, Eva, Svaśarīrastham, Cālayet, Pavanam, Dhiyā, Yadā, Śramo, Bhaved, Dehe, Tadā, Sūryeṇa Pūrayet}
\newpage
\subsection*{Paraphrased Word Meaning}


\begin{multicols}{2}
\textit{Tathā} --- in the same manner \\
\textit{Eva} --- indeed \\
\textit{Svaśarīrastham} --- in one’s body \\
\textit{Pavanam}  --- the air \\
\textit{Dhiyā} --- mindfully \\
\textit{Cālayet} --- one should move \\
\textit{Yadā} --- when \\
\textit{Dehe} --- in the body \\
\textit{Śramaḥ} --- tiredness \\
\textit{Bhaved}  --- should be experienced \\
\textit{Tadā} --- then\\
\textit{Sūryeṇa} --- through the right nostril \\
\textit{Pūrayet} --- one should inhale 
\end{multicols}

\subsection*{Word Split - Verse 2.64}

\textit{Yathā, Udaram, Bhavet, Pūrṇam, Anilena, Tathā, Laghu Dhārayet, Nāsikām, Madhyātarjanībhyām, Vinā, Dṛḍham}


\subsection*{Paraphrased Word Meaning}


\begin{multicols}{2}
\textit{Yathā} --- as \\
\textit{Udaram} --- the abdomen \\
\textit{Anilena} --- by the air \\
\textit{Laghu} --- quickly \\
\textit{Pūrṇam} --- filled \\
\textit{Bhavet} --- it should become \\
\textit{Tathā (sūryeṇa pūrayet)} --- in that manner (inhale through the right nostril)\\
\textit{Madhyātarjanībhyām} --- (by) the middle and the index fingers \\
\textit{Vinā}  --- excluding\\
\textit{Dṛḍham} --- firmly\\
\textit{Nāsikām} --- the nose\\
\textit{Dhārayet} --- one should hold
\end{multicols}

\subsection*{Word Split - Verse 2.65-1}

\textit{Vidhivat, Kumbhakam, Kṛtvā, Recayed, Iḍayā, Anilam}
\newpage
\subsection*{Paraphrased Word Meaning}

\begin{multicols}{2}
\textit{Vidhivat} --- as per injunctions\\
\textit{Kumbhakam} --- holding of breath \\
\textit{Kṛtvā} --- having done \\
\textit{Anilam} --- the air \\
\textit{Iḍayā} --- through the left nostril\\
\textit{Recayed} --- (one) should exhale
\end{multicols}

\subsection*{Purport}


One should stay seated placing the upward facing soles of feet on the (opposite) thighs. This is \textit{Padmāsana} that removes all Sins(59). Aligning the neck and the adomen (in a straight line) a steady-minded person having firmly assumed \textit{Padmāsana}, should exhale \textit{Prāṇa} with sound through one nostril (the right nostril) with effort, sealing the mouth firmly(60). The breath being exhaled should be felt from the chest-throat region till the skull region. Then, the breath has to be inhaled quickly (through one/right nostril) (61). In this manner repeatedly exhalation and inhalation has to be done. Just like the bellows of the backsmith which is operated rapidly (62) the breath in the body has to be operated mindfully. When tiredness is experienced in the body, then inhale through the right nostril(63). One should inhale in the manner by which the abdomen is filled with air quickly. One should (then) hold the nose with the fingers excluding the middle and index fingers(64). After holding the breath as per injunctions (stated while explaining \textit{Sūryabhedana}) one should exhale through the left nostril.

\subsection*{Inputs from \textit{Jyotsnā} Commentary}


\begin{enumerate}
\item The method of practice of \textit{Bhastrikā} is like this:

The left nostril has to be blocked by the little and ring fingers of right hand and the rapid inhalation and exhalation has to be done by the right nostril. When one gets tired then inhalation has to be done by the same right nostril and \textit{Kumbhaka} has to be practiced (with the \textit{Bandha-s}). Then exhalation has to be done through the left nostril. After that, the right nostril has to be blocked with the thumb and rapid inhalation and exhalation like the bellows has to be done through the left nostril. Then when one tires, inhalation has to be done through the left nostril and it has to be closed with the ring and the little fingers (of the right hand). After holding the breath according one’s capacity, exahaltion has to be done through the right nostril. This is one method of practicing \textit{Bhastrikā}.
 
\item Second method:

The left nostril has to be blocked with the ring and little fingers (of the right hand). Quick Inhalation has to be done through the right nostril and immediately blocking it with the thumb, exhaltion has to be done through the left nostril.  This has to be done for hundred times. When one feels tired, inhalation has to be done through right nostril. After holding the breath along with the \textit{Bandha-s}, exhalation has to be done through the left nostril. Again, the right nostril has to be blocked with the thumb and inhalation has to be done through the left nostril. Immediately closing the left nostril with the ring and little fingers, exhalation has to be done through the right nostril like the bellows. After having done this inhalation and exhalation again and again (for hundered times), when one tires, inhalation has to be done through the left nostril. Then it has to be closed with the ring and little fingers. \textit{Kumbhaka} has to be practiced and exhalation has to be done through the right nostril. This is the second method. 
\end{enumerate}
\newpage

\subsection*{Benefits of \textit{Bhastrikā}}


\noindent \textbf{Verses 2.65-2 - 67}

\begin{verse}
\textit{vātapittaśleṣmaharaṁ śarīrāgnivivardhanam ||\\
kuṇḍalībodhakaṁ kṣipraṁ pavanaṁ sukhadaṁ hitam |\\
brahmanāḍīmukhe saṁsthakaphādyargalanāśanam ||\\
samyaggātrasamudbhūtagranthitrayavibhedakam |\\
viśeṣeṇaiva kartavyaṁ bhastrākhyaṁ Kumbhakaṁ tviḍām ||}
\end{verse}

\subsection*{Word Split --- Verse 2.65-2}


\textit{Vātapittaśleṣmaharam, Śarīrāgnivivardhanam}

\subsection*{Paraphrased Word Meaning}


\begin{multicols}{2}
\textit{Vāta-pitta-śleṣma-haram }--- (it) removes (imbalances in) Vata, pitta and \textit{Kapha}\\
\textit{Śarīrāgni-vivardhanam} --- (It) also kindles the bodily (digestive) fire
\end{multicols}

\subsection*{Word Split - Verse 2.66}


\textit{Kuṇḍalībodhakam, Kṣipram, Pavanam, Sukhadam, Hitam, Brahmanāḍīmukhe, Saṁsthakaphādyargalanāśanam}

\subsection*{Paraphrased Word Meaning}


\begin{multicols}{2}
\textit{Kṣipram} --- quickly \\
\textit{Kuṇḍalī-bodhakam} --- awakens \textit{Kuṇḍalnī}\\
\textit{Pavanam} --- purifies \\
\textit{Sukhadam} --- gives comfort \\
\textit{Hitam }--- beneficial\\
\textit{Brahmanāḍī}-mukhe --- in the mouth of \textit{Suṣumnā}\\
\textit{Saṁstha-kaphādyargala-nāśanam} --- destroys the obstructions like Kapaha
\end{multicols}

\subsection*{Word Split --- Verse 2.67}


\textit{Samyaggātrasamudbhūtagranthitrayavibhedakam,  Viśeṣeṇa, Eva, Kartavyam, Bhastrākhyam, Kumbhakam, Tu, Iḍām}

\subsection*{Paraphrased Word Meaning}


\begin{multicols}{2}
\textit{Samyag-gātra-samudbhūta-granthi-\\traya-vibhedakam} ---  it  pierces three knots that are firmly entrenched  in the body\\
\textit{Iḍām}  --- this \\
\textit{Bhastrākhyam} --- known as \textit{Bhastrā}\\
\textit{Kumbhakam} --- the \textit{Prāṇāyāma} \\
\textit{Tu} --- indeed \\
\textit{Viśeṣeṇa} --- especially \\
\textit{Eva} --- must \\
\textit{Kartavyam} --- be done
\end{multicols}

\subsection*{Purport}


This (\textit{Bhastrikā}) removes (imbalances) \textit{Vata, Pitta} and \textit{Kapaḥa}. It enhances the bodily (digestive) fire. It also destroys the obstructions like \textit{Kapha} that lie in the mouth of the  \textit{Suṣumnā Nāḍī. Bhastrikā} also pierces the three  \textit{Granthis} (knots) that are firmly entrenched in the body. (Hence) this practice called  \textit{Bhastrikā}, has to be done especially.

\subsection*{Inputs from \textit{Jyotsnā} Commentary}


\begin{enumerate}
\item  \textit{Bhastrikā} is \textit{“Hita”} – benefical to all at all times- because it removes the imbalances in all the three  \textit{Doṣas}. 
\item Though all the \textit{Prāṇāyāmas} are benefical at all times, still – \textit{Sūryabhedana} and \textit{Ujjāyī} are hot. Hence they are beneifical in cold season.  \textit{Sitkari} and  \textit{Sitali} are cool and hence they are benefical in hot seasons.  But there is a balance in the hot and cold experiences through  \textit{Bhastrikā} \textit{Prāṇāyāma} and hence it is beneficial always. 
\item Though all the \textit{Prāṇāyāmas} can address all ailments, still,  \textit{Sūryabhedana} (especially) removes imbalances in  \textit{Vata} and \textit{Ujjāyī} mostly removes \textit{Kapha}.  \textit{Sitkari} and  \textit{Sitali} remove imbalances in  \textit{Pitta}. It should be understood that Bhastrikā \textit{Prāṇāyāma} balances all the three  \textit{Doṣas}.  
\item In the  \textit{Suṣumnā} there are three  \textit{Granthis} – knots – called  \textit{Brahmagranthi, Viṣṇugranthi} and  \textit{Rudragranthi}.  All the three are pierced by this practice. 
\item Hence, this \textit{Prāṇāyāma} has to be certainly done. While  \textit{Sūryabhedana} and others can be done based on requirement.
\end{enumerate}
\newpage
    
\section*{\textit{Prāṇāyāma - 6 Bhrāmarī}}

\noindent \textbf{Verses 2.68-1}

\textit{vegād ghoṣaṁ pūrakaṁ bhṛṅganādaṁ  bhṛṅgīnādaṁ recakaṁ manda\-mandam||}

\subsection*{Word Split}


\textit{Vegād, Ghoṣam, Pūrakam, Bhṛṅganādam, Bhṛṅgīnādam, Recakam, Mandamandam, Yogīndrāṇām, Evam Abhyāsāyogāt, Cite, Jātā, Kācid, Ānandalīlā}

\subsection*{Paraphrased Word Meaning}


\begin{multicols}{2}
\textit{Vegād} --- rapidly\\
\textit{Ghoṣam} --- (with) sound \\
\textit{Bhṛṅga-nādam} --- (resembling) the sound of a male bee\\
\textit{Pūrakam} --- inhalation \\
\textit{(kuryāt)} --- (one) should do \\
\textit{Bhṛṅgī-nādam} --- (resembling) the sound of the female bee \\
\textit{Manda-mandam} --- slowly, slowly \\
\textit{Recakam} --- exhalation \\
\textit{(kuryāt)} --- ((one) should do)
\end{multicols}

\subsection*{Purport}

Inhalation has to be done rapidly with sound that resembles the sound of a male bee. Exhalation has to be done slowly, slowly with the sound resembling the sound of a female bee.

\subsection*{Inputs from \textit{Jyotsnā} Commentary}


\begin{enumerate}
\item The breath has to be held (within) after inhalation. As this is part of the \textit{Kumbhakas} already, \textit{Kumbhaka} has not been stated especially (it is taken to be understood). 
\end{enumerate}
\newpage
\subsection*{Benefits of \textit{Bhrāmarī}}


\noindent \textbf{Verses 2.68-2}

\begin{center}
\textit{yogīndrāṇāmevamabhyāsa-yogā-ccitte jātā kācidānanda-līlā ||}
\end{center}

\subsection*{Word Split}

\textit{Yogīndrāṇām, Evam Abhyāsāyogād, Cite, Jātā, Kācid, Ānandalīlā}

\subsection*{Paraphrased Word Meaning}


\begin{multicols}{2}
\textit{Evam} --- in this manner \\
\textit{Abhyāsa-yogād} ---  by being involved in the practice \\
\textit{Yogīndrāṇām} --- of the best of the Yogins \\
\textit{Citte}  --- in the mind \\
\textit{Kācid} --- inexplicable \\
\textit{Ānanda-līlā} --- play of bliss \\
\textit{Jātā} ---  manifests
\end{multicols}

\subsection*{Purport}


By being involved in the practice in this manner, an inexplicable bliss manifests in the mind of the best among the Yogins.

\subsection*{Note on  \textit{Mūrcchā} and  \textit{Plāvinī}   ( \textit{Prāṇāyāmas} – 7\&8 in the series of Eight \textit{Kumbhakas})}


Thus far six \textit{Kumbhakas}/ \textit{Prāṇāyāmas} out of eight as part of the  \textit{Aṣṭakumbhakas} were elaborated. The last two  \textit{Mūrcchā}  and  \textit{Plāvinī} are not elaborated as they are not practiced in the Krishnnamacharya Tradition. Basically, the two \textit{Prāṇāyāmas} are centred on attaining extraordinary capabilities.  The gist of the two practices is as follows – The practice of  \textit{Mūrcchā}  grants a state of unconsciousness that is blissfull. It is practiced by inhaling (whether through nostrils or by mouth is not specified –even in the commentary) and doing very firm  \textit{Jālandhara bandha} amd exhalaing very slowly.   \textit{Plāvinī} grants the capability to float on the surface of even very deep waters. The practice is done by inhaling excessively and filling  the abdomen (no further descrptions are found either in the text or the commentary). 
\newpage

\section*{Other Practices}

\subsection*{1. \textit{Nāḍīśodhana} and its Benefits}

\noindent \textbf{Verses 2.7-10}

\begin{verse}
\textit{baddhapadmāsano yogī prāṇaṁ candreṇa pūrayet |\\
dhārayitvā yathāśakti bhūyaḥ sūryeṇa recayet ||}
\end{verse}

\begin{verse}
\textit{prāṇaṁ sūryeṇa cākṛṣya pūrayedudaraṁ śanaiḥ |\\
vidhivat Kumbhakaṁ kṛtvā punaścandreṇa recayet ||}
\end{verse}

\begin{verse}
\textit{yena tyajet tena pītvā dhārayedatirodhataḥ |\\
recayecca tato'nyena śanaireva na vegataḥ ||}
\end{verse}

\begin{verse}
\textit{prāṇaṁ cediḍayā pibenniyamitaṁ bhūyo'nyayā recayet\\
pītvā piṅgalayā samīraṇamatho baddhvā tyajedvāmayā |\\ 
sūryācandramasoranena vidhinābhyāsaṁ sadā tanvatāṁ\\
śuddhā nāḍigaṇā bhavanti yamināṁ māsatrayādūrdhvataḥ ||}
\end{verse}

\subsection*{Word Split – Verse 2.7}


\textit{Baddhapadmāsanaḥ, Yogi, Prāṇam, Candreṇa, Pūrayet Dhārayitvā, Yathāśakti, Bhūyaḥ, Sūryeṇa, Recayet}

\subsection*{Paraphrased Word Meaning}


\begin{multicols}{2}
\textit{Baddha-padmāsanaḥ} --- being in \textit{Padmāsana} \\
\textit{Yogi} --- a Yoga practitioner \\
\textit{Prāṇam} --- the prāṇa \\
\textit{Candreṇa}  through the left nostril \\
\textit{Pūrayet} --- should inhale \\
\textit{Yathā-śakti} --- to the level of one’s capability \\
\textit{Dhārayitvā} --- after holding (the breath)\\
\textit{Bhūyaḥ} --- again \\
\textit{Sūryeṇa} --- through the right nostril \\
\textit{Recayet} --- (one) should exhale 
\end{multicols}

\subsection*{Word Split - Verse 2.8}


\textit{Prāṇam, Sūryeṇa, Ca, Ākṛṣya, Pūrayed, Udaram, Śanaiḥ, Vidhivad, Kumbhakam, Kṛtvā, Punaḥ, Candreṇa, Recayet}

\subsection*{Paraphrased Word Meaning}


\begin{multicols}{2}
\textit{Sūryeṇa} --- by the right nostril \\
\textit{Ca} --- and \\
\textit{Prāṇam} --- the \textit{Prāṇa} \\
\textit{Ākṛṣya} --- having drawn \\
\textit{Udaram} --- the abdomen \\
\textit{Śanaiḥ} ---  slowly \\
\textit{Pūrayed} --- (one) should fill \\
\textit{Vidhivad} --- as per injunctions \\
\textit{Kumbhakam} --- holding of the breath\\
\textit{Kṛtvā} --- after doing \\
\textit{Punaḥ} --- again \\
\textit{Candreṇa} --- through the left nostril \\
\textit{Recayet} --- one should exhale
\end{multicols}

\subsection*{Word Split - Verse 2.9}


\textit{Yena, Tyajet, Tena, Pītvā, Dhārayed, Atirodhataḥ, Recayet, Ca,  Tataḥ, Anyena Śanaiḥ, Eva, Na Vegataḥ}

\subsection*{Paraphrased Word Meaning}


\begin{multicols}{2}
\textit{Yena}- that (nostril) by which\\
\textit{Tyajet} --- one exhales\\
\textit{Tena} --- by that (same nostril) \\
\textit{Pītvā} --- after inhaling \\
\textit{Atirodhataḥ} ---  with great intensity \\
\textit{Dhārayed} --- one should hold \\
\textit{Tataḥ} ---  then \\
\textit{Anyena} --- through the other(nostril)\\
\textit{Śanaiḥ} --- slowly\\
\textit{Eva} --- only\\
\textit{Na} --- not \\
\textit{Vegataḥ} --- quickly \\
\textit{Recayet} --- (one) should exhale \\
\textit{Ca}  --- and
\end{multicols}

\subsection*{Word Split - Verse 2.10}


\textit{Prāṇam, Ced, Iḍayā, Pibed, Niyamitam, Bhūyaḥ, Anyayā, Recayet,  Pītvā, Piṅgalayā, Samīraṇam, Atho, Baddhvā, Tyajed, Vāmayā, Sūryācandramasoḥ, Anena, Vidhinā, Abhyāsam, Sadā, Tanvatām, Śuddhāḥ, Nāḍigaṇāḥ, Bhavanti, Yaminām, Māsatrayād, Ūrdhvataḥ}

\subsection*{Paraphrased Word Meaning}


\begin{multicols}{2}
\textit{Prāṇam} --- the air \\
\textit{Iḍayā} --- by the left nostril \\
\textit{Ced} --- if \\
\textit{Pibed} --- (one) should exhale \\
\textit{Niyamitam} (prāṇam) --- and should hold (the prāṇa) \\
\textit{Bhūyaḥ} --- then \\
\textit{Anyayā} --- through the other (right nostril) \\
\textit{Recayet} --- (one) should exhale \\
\textit{Atho} --- then \\
\textit{Piṅgalayā} --- through the right nostril \\
\textit{Samīraṇam} --- the air \\
\textit{Pītvā} --- after inhaling \\
\textit{Baddhvā} --- after holding\\ 
\textit{Vāmayā } --- through the left nostril \\
\textit{Tyajed} --- (one) should exhale \\
\textit{Anena} --- by this \\
\textit{Sūryācandramasoḥ} --- of the sun and the moon (left and right nostrils)\\
\textit{Vidhinā} --- method/activity \\
\textit{Sadā}  --- always \\
\textit{Abhyāsam} --- practice \\
\textit{Tanvatām} --- those who continue \\
\textit{Yaminām} --- of the self-restrained practitioners\\
\textit{Māsatrayād} --- three months \\
\textit{Ūrdhvataḥ} --- after \\
\textit{Nāḍi}-gaṇāḥ --- the host of \textit{Nāḍīs} \\
\textit{Śuddhāḥ} --- cleansed \\
\textit{Bhavanti}  --- become
\end{multicols}

\subsection*{Purport}


A \textit{Yogi} seated in the \textit{Padmāsana} should inhale through the left nostril. After holding the breath to the level of one’s capability one should exhale through the right nostril(7). (Then) The abdomen has to be filled slowly with air by inhaling through the right nostril and after holding the breath as per injunctions (with  \textit{Bandhas}) one should exhale through the left nostril (8). One has to inhale through that nostril by which exhalation has been done. Then one should hold the breath with great intensity and exhale the breath slowly and not quickly, through the other nostril (9).  If the breath is inhaled through the left nostril then after holding the breath, exahaltion has to be done thorugh the other nostril. Then through the right nostril the breath has to be inhaled. After holding the breath it has to be exhaled through the left nostril. This activity of the Sun and moon (the left and right-nostrils) done by the self-restrained practitioners will result in the cleansing of the \textit{Nāḍī-s} after three months. 

\subsection*{Inputs from the \textit{Jyotsnā} Commentary}


\begin{enumerate}
\item In the context of verse 7 - \textit{Jyotsnā} commentary defines  \textit{Pūraka, Recaka} and \textit{Kumbhaka} as follows –  \textit{Pūraka} – The intake of the air from outside with special (mindful/slow) effort  is Pūraka/inhalation. \textit{Kumbhaka} – Holding of the breath with  \textit{Bandhas} like  \textit{Jālandhara} is \textit{Kumbhaka}.  \textit{Recaka} – This is expelling of the breath that was held with special (mindful/long) effort. The rapid inhalation and exhalation which has been described as part of \textit{Kapālabhāti} cannot be termed as  \textit{Pūraka} and \textit{Recaka} as they do not confirm to the definition given above. 
\item \textit{Vidhivat} (as per injunctions)– this term appears in the verse 8 to describe \textit{Kumbhaka}. It indicates the practice of the  \textit{Bandhas} during holding the breath. 
\item  \textit{Atirodhataha} (with great intensity) – this term appears in the verse 9 to describe \textit{Kumbhaka}. Holding the breath till one experiences sweat, trembling is to be understood as the great intensity ( \textit{Atiroadhatah}) during holding the breath.
\item Though verses 7-9 have described the process of \textit{Nāḍīshodhan} verse 10 serves the purpose of summarizing the content of three verses (7-9) in one verse and it also mentions benefits of \textit{Nāḍīśodhana}. (“the \textit{Nāḍīs} of the self-restrained practitioners will be cleansed after three months”). 
\end{enumerate}

\section*{2. \textit{Kapālabhāti} and its Benefits}

\noindent \textbf{Verses 2.35}

\begin{verse}
\textit{bhastrāvallohakārasya recapūrau sasambhramau |\\
kapālabhātirvikhyātā kaphadoṣaviśoṣiṇī ||}
\end{verse}

\subsection*{Word Split}


\textit{Bhastrāvad, Lohakārasya, Recapūrau, Sasambhramau, Kapālabhātiḥ, Vikhyātā, Kaphadoṣaviśoṣiṇī}

\subsection*{Paraphrased Word Meaning}


\begin{multicols}{2}
\textit{Loha-kārasya} --- of the black smith \\
\textit{Bhastrāvad} --- like the bellows \\
\textit{Sa-sambhramau} --- rapid\\
\textit{Reca-pūrau} --- exhalation and inhalation \\
\textit{Kapāla-bhātiḥ} --- (this is) \textit{Kapāla-bhāti}\\
\textit{Vikhyātā ---} the well-known (iyam) (this)\\
\textit{Kapha-doṣa-viśoṣiṇī} --- dires out the vitiation of \textit{Kapha} 
\end{multicols}

\subsection*{Purport}


Like the bellows of the Blacksmith - exhaling and inhaling rapidly is the well known \textit{kapāla-bhāti}. It dries out the vitiation of \textit{Kapha}.

\subsection*{Inputs from \textit{Jyotsnā}}


Twenty types of discomforts/ illnesses caused by Kapha is removed by \textit{Kapālabhāti}. (20 \textit{Kapha} reated ailments are detailed in  \textit{Carakasamhitā sūtrasthāna} 20.17  -  contentment, drowsiness, excess sleep, excess cold sensation, heaviness in body, lassitude, sweet taste in mouth, salivation, expectoration of mucous, excess accumulation of waste, loss of strength, indigestion, adherence of waste surrounding heart, adherence of waste surrounding throat, adherence of waste in blood vessels/ atherosclerosis, goiter, morbid obesity and decreased agni, Urticarial rashes and Pale look)
