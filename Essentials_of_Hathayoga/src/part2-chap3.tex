\chapter{Mudrās and Bandhas}

\subsection*{Summary}

\section*{The purpose of the Mudrās}

Hence, to awaken the Kuṇḍalinī that is asleep in the mouth of Suṣumnā one should repeat the practices Mudrā. 3.5

\heading{The List of Ten Mudrās}

Mahā-mudrā,  mahā-bandha,  mahā-vedha,  khecarī, uḍyāna-bandha, mūla-bandha, jālandhara-bandha, viparīta- karaṇī, vajrolī, and  śakti-cālanam are the ten mudrās that destroy  aging and death. 3.6,7

\heading{Mahā-mudrā}

By the heel of the left leg, the perinueum has to be pressed. The right leg has to be extended and has to be held firmly by the hands.   3.10

After assuming bandha in the throat regions prāṇa has to be made to flow above (through suṣumnā by the practice  mūlabandha). Then kuṇḍalinī energy becomes straight like the snake that becomes straight when being hit by a stick. (11) When kuṇḍalinī becomes straight like this, the state of death based on the two sides(prāṇa deserting iḍā and piṅgalā) (and entering suṣumnā) is attained. (12) – 3.11,12

After that one should very slowly exhale and not quickly. This practice is called as mahāmudrā (great mudrā). It has been shown (to us) by great Siddhas (those who have attained higher states of Yoga).  3.13

After practicing this through the left nostril one should do the practice through the right nostril till the number of times on each side becomes equal. After that the practice of the Mudrā has to be ended. 3.15

\heading{Benefits of Mahāmudrā}

Great afflictions etc, death etc which are great difficulites/afflictions get reduced. Hence this practice is called as the Great mudrā by the best among the learned. 3.14

As there is nothing which is conducive or nonconductive (for a practitioner of Mahamudrā) (food of)  all tastes and (which are even) tasteless are also digested. Even the terrible, poison is also digested like nectar. 3.16

The one who practices mahāmudrā becomes free from diseases (arising out of Vitiations), like tuberculosis, leprosy, constipation, abdominal disease, indigestion etc  3.17

The sated Mahāmudrā bestows great powers. Hence it has to be guarded safely and should not be given to anyone (undeserving). 3.18

\heading{Uḍḍīyānabandha}

That (practice), by which the prāṇa being locked (out of other channels and is directed towards suṣumnā) flies/soars in suṣumnā, is called as uḍḍīyana (uḍḍīyāna) by the masters of Yoga 3.55

As the great entity that moves about (within) in the space (of the body) tirelessly (by the practice of this Bandha) flies (into suṣumnā). It will now be stated about the Bandha. 3.56

One should pull back the abdomen above (and also below) the navel. Such a practice is called as uḍḍīyāna. This is the lion that torments the elephant of death. 3.57

uḍḍīyāna that is taught by the Guru is always natural. The one who pratices this always, even if he is old attains youthfulness. 3.58

One should practice drawing the abdomen above and below the navel with effort. By practicing this for six months death is conquered. There is no doubt about this. 3.59

\heading{(only) Benefits of Uḍḍīyāna-bandha}

Among all the Bandhas uḍḍiyāna is the best. Because, if uḍḍiyāna is firmly attained, then liberation is naturally attained.  3.60

\heading{Mūlabandha}

The perineum has to be pressed by the heel and the anus has to be constricted. The Apāna has to be drawn upwards. This is called as mūla-bandha. 3.61

Constricting the anus with effort and thereby making the downward moving Apāna travel upwards is called as mūla-bandha by the masters of Yoga. 3.62

But, having pressed the anus with the heel one should repeatedly constrict the air (by the constriction of the anus) so that the it travels upwards. (This is mūlabandha)  3.63

\heading{Benefits of Mūla-bandha and other related inputs}

Prāṇa and apāna, nāda and bindu – all these beomce one by the practice of mūlabandha. And they grant the success of Yoga. There is no doubt on this.  3.64

By the constant practice of mūlabandha apāna and prāṇa unite. (Accumlated) Urine and feces (in the body) are removed. Aged will also become youthful. 3.65

When the apāna has turned upwards (by the practice of mūlabandha) and reaches the sphere of fire (region of jaṭharāgni), the flame of fire (jaṭharāgni) becomes elongated being hit/fanned the wind (apāna). 3.66

After that the fire and apāna reach prāṇa which is already hot. By that the fire of the body (digestive fire) becomes intensely hot. 3.67

By that the kuṇḍalinī that is asleep is scorched and like a female snake being hit by a stick, exhaling, it straightens outs. 3.68

After becoming straight, like a snake entering a burrow,   kuṇḍalinī enters suṣumnā-nāḍī. Hence, mūlabandha has to be done by practitioners of Yoga always. 3.69

\heading{Jālandhara-bandha}

The throat has to be constricted and the chin has to be placed on the chest firmly. This is called as Jālandhara Bandha which destroys aging and death.  3.70

The jālandhara bandha blocks the web of channles of Prāṇa, (and consequently) the water of the ether (the diluted Candra – life energy present in the skull region closer to the eyebrows) that moves downwards is also blocked. Jālandhara Bandha removes host of sufferings related to throat. 3.71

\heading{Benefits of Jālandhara-bandha}

When Jālaadhara Bandha, that is characterized by constricting the throat is done, the nectar does not fall into the (digestive) fire and the air/ prāṇa does not get vitiated. 3.72

Only by constricting the throat the two channels of Prāṇa are blocked. This is jālandhara Bandha. And this (the throat) is the middle Cakra, that blocks (all) the 16 ādhāras (centres of Prāṇa). 3.73

\heading{Additional Inputs on the three Bandhas}

After constricting the anus region (mūlabandha) one should do uḍḍiyāna-bandha. Block iḍā (the left nāḍī) and piṅgalā (the right nāḍī) (by Jālandhara bandha). Make the  prāṇa flow in the posterior path (suṣumnā).  3.74

By this method, the prāṇa attains steadiness/stillness. By this death, aging, disease etc do not manifest. 3.75

This set of three Bandhas is great. It has been practiced by great siddhas. It is considerd by the Yogins as the method of accomplishment of all techniques of Yoga.   3.76

\heading{Three verses on bandhas from Chapter 2}

The jālandhara Bandha has to be done after inhalation and uḍḍiyāna Bandha has to be done at the end of the Kumbhaka (holding the breath within) but before the beginning of exhalation. 2.45

After constricting the throat (jālandharabandha) and then quickly constricting the lower region (mūlabandha) and then by drawing (the abdomen) backwards (uḍyānabandha), the prāṇa will flow in the brahma-nāḍi (suṣumnā). 2.46

By drawing the apāna above and taking the prāṇa below, a Yogi shall become free from aging and shall become (a youth of) sixteen years of age. 2.47

\heading{Viparīta Karaṇī (5)}

For the one whose navel is above and the palate is below, the sun is above and the moon is below. This called as the practice of inversion which is attained by the words of the Guru. 3.79

For the one who practices constantly, it enhances gastric fire. Food in plenty has to be arranged for such a practitioner. 3.80

If one consumes food scantily the fire scorches immediately. In the first day one should be in (this) head-down-feet-up position for just a moment. 3.81

Everyday one should practice a little more than a moment (than the previous day). A person who regulary practices (Viparīta Karaṇī) for a duration of a yāma (3 hours), he (the practitioner) conquers time. 3.82
\newpage

\thispagestyle{empty}
~\vfill
\centerline{\textbf{\LARGE Textual Immersion}}
\vfill
\eject

\section*{1. The purpose of the Mudrās}

\noindent 
\textbf{Verses  3.5}

\begin{shloka}
tasmāt sarva-prayatnena prabodhayitumīśvarīm |\\
brahma-dvāra-mukhe suptāṁ mudrābhyāsaṁ samācaret ||
\end{shloka}

\subsection*{Word Split}

Tasmāt, sarva-prayatnena, prabodhayitum, īśvarīm, brahma-dvāra-mukhe, suptām, mudrābhyāsam, samācaret

\subsection*{Paraphrased Word Meaning}

\begin{multicols}{2}
tasmāt --- hence \\
īśvarīm --- the all powerful/goddess\\
brahma-dvāra-mukhe --- in the mouth of suṣumnā\\
suptāṃ --- the one who is asleep \\
prabodhayitum --- to awaken\\
sarva-prayatnena --- with all efforts \\
mudrābhyāsaṃ --- the practice of mudrā \\
samācaret ---  one should do
\end{multicols}

\subsection*{Purport}

Hence, to awaken the Kuṇḍalinī that is asleep in the mouth of Suṣumnā one should repeat the practices Mudrā.

\subsection*{Inputs from Jyotsnā Commentary}

\begin{enumerate}
\item The word Tasmāt (Hence) means – as the six cakras are pierced (by Prāṇa) only when the Kuṇḍalinī is awakened (hence one should do Mudrās). 
\item Brahmadvāra – the word Brahman here refers to the Pure consciousness – Sacchiḍānanda. As the Suṣumnā Nāḍī is the dvāra (path) of attainment of realization of Brahman hence it is called as Brahma-dvāra. 
\item Abhyāsa (mudrā-abhyāsa)– the word abhyāsa here means – āvritti – repetition. (It can be understood that by repeated practice of Mudrās only Kuṇḍalinī is awakened and not just by doing it once or twice)
\end{enumerate}
\newpage

\section*{2. The List of Ten Mudrās}

\noindent 
\textbf{Verses 3.6,7}

\begin{shloka}
Mahā-mudrā mahā-bandho mahā-vedhaśca khecarī |\\
uḍyānaṁ mūla-bandhaśca bandho jālandharābhidhaḥ || 3.6 ||\\
karaṇī viparītākhyā vajrolī śakti-cālanam |\\
iḍāṁ hi mudrā-daśakaṁ jarā-maraṇa-nāśakam || 3.7 ||
\end{shloka}

\noindent 
\textbf{Word Split (3.6)}

Mahā-mudrā, mahā-bandhaḥ, mahā-vedhaḥ, ca khecarī, uḍyānam, mūla-bandhaḥ, ca bandhaḥ, jālandharābhidhaḥ 

\subsection*{Word Split (3.7)}

karaṇī, viparītākhyā, vajrolī, śakti-cālanam,  iḍām, hi, mudrā-daśakam, jarā-maraṇa-nāśakam 

\subsection*{Paraphrased Word Meaning (3.6,7)}

\begin{multicols}{2}
mahā-mudrā, maha-bandhaḥ, mahā-vedhaḥ \\
ca  --- and \\
khecarī, uḍyānam, mūla-bandhaḥ \\
ca  --- and \\
jālandharābhidhaḥ --- the one known as jālandhara\\
bandhaḥ --- the Bandha \\
viparītākhyā karaṇī --- the practice called as viparīta (inverse)\\
vajrolī, śakti-cālanam \\
iḍāṃ --- this \\
hi --- indeed \\
jarā-maraṇa-nāśakam --- destroyer of aging and death\\
mudrā-daśakam – the set of ten Mudrās
\end{multicols}

\subsection*{Purport}

Mahā-mudrā,  mahā-bandha,  mahā-vedha,  khecarī, uḍyāna-bandha, mūla-bandha, jālandhara-bandha, viparīta- karaṇī, vajrolī, and  śakti-cālanam are the ten mudrās that destroy  aging and death.

\section*{1. Mahā-mudrā }

\noindent \textbf{Verses 3.10}

\begin{shloka}
pāda-mūlena vāmena yoniṁ sampīḍya dakṣiṇam |\\
prasāritaṁ padaṁ kṛtvā karābhyaṁ dhārayed dṛḍham || 3.10 ||
\end{shloka}

\subsection*{Word Split}

pāda-mūlena, vāmena, yonim, sampīḍya, dakṣiṇam, prasāritam, padam, kṛtvā, karābhyam, dhārayed, dṛḍham

\subsection*{Paraphrased Word Meaning}

\begin{multicols}{2}
vāmena --- by the left \\
pāda-mūlena --- (by) heel \\
yoniṃ --- perineum \\
sampīḍya --- after pressing\\
dakṣiṇam --- the right \\
padaṃ --- leg\\
prasāritam --- being extended  \\
kṛtvā --- after doing \\
karābhyaṃ --- by the hands \\
dṛḍham --- firmly \\
dhārayed --- one should hold 
\end{multicols}

\subsection*{Purport}

By the heel of the left leg, the perinueum has to be pressed. The right leg has to be extended and has to be held firmly by the hands.    

\subsection*{Inputs from Jyotsnā Commentary}

\begin{enumerate}
\item Yoni (perineum) --- this refers to the portion inbetween the anus and the genital.
\item prasāritam (extension) --- in the process of extending the right leg, the leg has to be extended staright as a stick and the heel has to be placed on the ground and the fingers of the leg should point upwards. 
\item karābhyaṃ (by both the hands) --- according to the tradition, by the two index fingers that are bent (hook-shaped)  the foot has to be held in the big toe.  
\end{enumerate}

\begin{center}
\textbf{Mahā-mudrā Continued…}
\end{center}

\noindent \textbf{Verses 3.11,12}

\begin{shloka}
kaṇṭhe bandhaṁ samāropya dhārayed vāyumūrdhvataḥ |\\
yathā daṇḍāhataḥ sarpaḥ daṇḍākāraḥ prajāyate || 3.11 ||\\
ṛjvībhūtā tathā śaktiḥ kuṇḍalī sahasā bhavet |\\
tadā sā maraṇāvasthā jāyate dvipuṭāśrayā || 3.12 ||
\end{shloka}

\subsection*{Word Split (3.11)}

kaṇṭhe, bandham, samāropya, dhārayed, vāyum, ūrdhvataḥ, yathā, daṇḍāhataḥ, sarpaḥ, daṇḍākāraḥ, prajāyate

\subsection*{Word Split (3.12)}

ṛjvībhūtā,  tathā, śaktiḥ, kuṇḍalī, sahasā, bhavet, tadā, sā, maraṇāvasthā, jāyate, dvipuṭāśrayā

\subsection*{Paraphrased Word Meaning}

\noindent \textbf{(3.11)}

\begin{multicols}{2}
kaṇṭhe --- in the throat \\
bandhaṃ --- the lock \\
samāropya --- after assuming \\
ūrdhvataḥ --- above \\
vāyum --- prāṇa (the air)\\
dhārayed --- one should hold \\
yathā --- as \\
daṇḍahataḥ --- hit by the stick \\
sarpaḥ --- snake \\
daṇḍākāraḥ --- the shape of stick \\
prajāyate --- attains 
\end{multicols}

\noindent \textbf{(3.12)}

\begin{multicols}{2}
tathā --- in the same manner \\
kuṇḍalī  śaktiḥ --- the kuṇḍalī(nī)  energy \\
sahasā --- quickly \\
ṛjvībhūtā --- straight \\
bhavet ---  (it) should become \\
tadā --- then \\
sā --- that \\
dvipuṭāśrayā --- depended on the two sides (nāḍīs)\\
maraṇāvasthā --- the state of death \\
jāyate --- happens
\end{multicols}

\subsection*{Purport}

After assuming bandha in the throat regions prāṇa has to be made to flow above (through suṣumnā by the practice  mūlabandha). Then kuṇḍalinī energy becomes straight like the snake that becomes straight when being hit by a stick. (11) When kuṇḍalinī becomes straight like this, the state of death based on the two sides(prāṇa deserting iḍā and piṅgalā) (and entering suṣumnā) is attained. (12)

\subsection*{Inputs from Jyotsnā Commentary}

\begin{enumerate}
\item kaṇṭhe, bandham, samāropya – This refers to Jālandharabandha.
\item vāyumūrdhvataḥ - This refers to Mūlabandha. Some traditional practitoners state that the effect of Mūlabandha is also achieved by pressing the perineum (Yoni) and doing jihvābandha. 
\item daṇḍākāraḥ - This means Kuṇḍalinī sheds its coiled shape and becomes straight 
\item maraṇāvasthā – When Prāṇa leaves iḍā and piṅgalā and enters suṣumnā  it is(like) the state of death. (Normally, only a person who breathes in and out is known to live. Breathing in and out does not happen when Prāṇa enters suṣumnā. Hence it is called as state of death. The Yogi is very much alive even if Prāṇa enters suṣumnā)
\end{enumerate}

\heading{Mahā-mudrā Continued…}

\noindent \textbf{Verses 3.13}

\begin{center}
tataḥ śanaiḥ śanaireva recayennaiva vegataḥ | 
iyaṁ khalu mahāmudrā mahāsiddhaiḥ pradarśitā ||
\end{center}

\subsection*{Word Split}

tataḥ,  śanaiḥ,  śanaiḥ, eva recayet, na, eva vegataḥ, iyam, khalu, mahāmudrā,  mahāsiddhaiḥ, pradarśitā 

\subsection*{Paraphrased Word Meaning}

\begin{multicols}{2}
tataḥ --- After that \\
śanaiḥ --- slowly \\
eva --- only \\
recayet --- one should exhale \\
na eva --- not at all \\
vegataḥ --- quickly \\
iyaṃ --- This  (is) mahāmudrā – great mudrā\\
mahā-siddhaiḥ --- by the great siddhas\\
khalu --- indeed \\
pradarśitā --- has been shown 
\end{multicols}

\subsection*{Purport}

After that one should very slowly exhale and not quickly. This practice is called as mahāmudrā (great mudrā). It has been shown (to us) by great Siddhas (those who have attained higher states of Yoga).

\subsection*{Inputs from Jyotsnā Commentary}

\begin{enumerate}
\item naiva vegataḥ - one should not exhale quickly as it might lead to loss of strength 
\item mahā-siddhaiḥ - this refers to Great Siddhas like ādinātha etc.
\end{enumerate}

\heading{Mahā-mudrā Continued…}

\noindent \textbf{Verses 3.15}

\begin{shloka}
candrāṅge tu samabhyasya sūryāṅge punarabhyaset |\\
yāvat tulyā bhavet saṅkhyā tato mudrāṁ visarjayet ||
\end{shloka}

\subsection*{Word Split}

candrāṅge, tu, samabhyasya, sūryāṅge, punaḥ, abhyaset, yāvat, tulyā, bhavet, saṅkhyā, tataḥ, mudrām, visarjayet 

\subsection*{Paraphrased Word Meaning}

\begin{multicols}{2}
candrāṅge --- on the left side /nostril \\
tu ---  (filler)\\
samabhyasya --- after practicing \\
yāvat --- till \\
saṅkhyā --- the number \\
tulyā --- equal \\
bhavet --- should become \\
sūryāṅge --- on the right side \\
punaḥ --- again \\
abhyaset --- one should practice \\
tataḥ --- after that \\
mudrām ---  the mudrā\\
visarjayet ---  one should dislodge/end 
\end{multicols}

\subsection*{Purport}

After practicing this through the left nostril one should do the practice through the right nostril till the number of times on each side becomes equal. After that the practice of the Mudrā has to be ended. 

\subsection*{Inputs from Jyotsnā Commentary}

\begin{enumerate}
\item candrāṅge --- This refers to the iḍā nāḍī 
\item sūryāṅge --- this refers to the piṅgalā nāḍī 
\item saṅkhyā --- the number here indicates the number of Kumbhakas practiced on each side. 
\item This is the entire sequence – Practicing on the left refers to (practicing the mudrā  as stated by) folding the left leg and pressing the perineum with left heel and holding the big toe of the right foot that has been extended with the bent index fingers of both the hands. Prāṇa stays in the left side by this practice.  Practicing on the right side refers to (practicing the mudrā  by) folding of the right leg and pressing the perenuum with the right heel and holding the big toe of the left foot that has been extended with the bent index fingers of both the hands. Prāṇa stays in the right side by this practice.
\end{enumerate}

\subsection*{Benefits of Mahāmudrā}

\noindent \textbf{Verses 3.14}

\begin{center}
Mahā-kleśādayo doṣāḥ kṣīyante maraṇādayaḥ |\\ 
mahā-mudrāṁ ca tenaiva vadanti vibudhottamāḥ || 3.14 ||
\end{center}

\subsection*{Word Split}

mahā-kleśādayaḥ, doṣāḥ, kṣīyante, maraṇādayaḥ, mahā-mudrām, ca, tena, eva vadanti, vibudhottamāḥ

\subsection*{Paraphrased Word Meaning}

\begin{multicols}{2}
mahā-kleśādayaḥ --- Great afflictions etc \\
maraṇādayaḥ --- death etc \\
doṣāḥ  --- difficulties/afflictions\\ 
kṣīyante  --- get reduced  \\
tena --- by that/hence \\
eva ---   only \\
ca ---  and \\
vibudhottamāḥ --- best among the learned   \\
mahāmudrām  --- Great mudrā\\
vadanti  --- (they) say
\end{multicols}

\subsection*{Purport}

Great afflictions etc, death etc which are great difficulites/afflictions get reduced. Hence this practice is called as the Great mudrā by the best among the learned.

\subsection*{Inputs from Jyotsnā Commentary}

\begin{enumerate}
\item mahā-kleśādayaḥ --- This refers to the great afflictions like  avidyā(ignorance)  asmitā (ego) raga(passion) dveṣa (hatred) abhiniveśa (fear of death), and also their effects such as śoka (sorrow), moha(delusion). 
\item maraṇādayaḥ --- this refers to aging, death etc. 
\item mahāmudrā --- As this practice pacifies (mudrāyati/śamayati) great afflictions (mahā-kleśās etc) hence this is called as mahā-mudrā.
\end{enumerate}

\subsection*{Benefits of Mahāmudrā Continued -}

\noindent \textbf{Verses 3.16}

\begin{shloka}
na hi pathyam-apathyaṁ vā rasāḥ sarve'pi nīrasāḥ |\\ 
api bhuktaṁ viṣaṁ ghoraṁ pīyūṣamiva jīryati || 16 ||
\end{shloka}

\subsection*{Word split}

Na, hi, pathyam, apathyam, vā, rasāḥ, sarve, api, nīrasāḥ, api, bhuktam, viṣam, ghoram, pīyūṣam, iva, jīryati 

\subsection*{Paraphrased Word Meaning}

\begin{multicols}{2}
hi --- as \\
na --- no \\
pathyam --- beneficial \\
apathyam --- not beneficial\\
vā --- or \\
sarve --- all\\
rasāḥ --- tastes \\
nīrasāḥ --- tasteless \\
api --- also \\
bhuktam --- consumed food \\
ghoram --- terrible/inconducive \\
viṣam --- poison \\
pīyūṣam --- nectar \\
iva --- like \\
jīrya(n)ti --- digested 
\end{multicols}

\subsection*{Purport}

As there is nothing which is conducive or nonconductive (for a practitioner of Mahamudrā) (food of)  all tastes and (which are even) tasteless are also digested. Even the terrible, poison is also digested like nectar. 

\subsection*{Inputs from Jyotsnā Commentary}

\begin{enumerate}
\item rasāḥ --- this refers to tastes such as spicy, salty (tastes) etc 
\item ghoram --- this refers to food that is very difficult to digest 
\item jīrya(n)ti --- If posion itself is digested, then what remains to be said about other kinds of food. 
\end{enumerate}

\subsection*{Benefits of Mahāmudrā contined…}

\noindent \textbf{Verses 3.17}

\begin{shloka}
kṣaya-kuṣṭha-gudāvarta-gulmājīrṇa-purogamāḥ |\\ 
tasya doṣā kṣayaṁ yānti mahā-mudrāṁ tu yo'bhyaset ||
\end{shloka}

\subsection*{Word Split}

kṣaya-kuṣṭha-gudāvarta-gulmājīrṇa-purogamāḥ, tasya, doṣāḥ, kṣayam, yānti, mahā-mudrām, tu, yaḥ, abhyaset

\subsection*{Paraphrased Word Meaning}

\begin{multicols}{2}
yaḥ --- The one who \\
mahāmudrāṃ ---  mahāmudrā\\
tu --- (filler)\\
abhyaset --- should practice \\
tasya --- for him \\
kṣaya-kuṣṭha-gudāvarta-gulmājīrṇa-\\
purogamāḥ --- tuberculosis, leprosy, \\
constipation, abdominal disease, \\
indigestion etc \\
doṣāḥ --- afflictions \\
kṣayaṃ --- destruction/reduction \\
yānti --- attain 
\end{multicols}

\subsection*{Purport}

The one who practices mahāmudrā becomes free from diseases (arising out of Vitiations), like tuberculosis, leprosy, constipation, abdominal disease, indigestion etc 

\subsection*{Inputs from Jyotsnā Commentary}

\begin{enumerate}
\item kṣaya-kuṣṭha-gudāvarta-gulmājīrṇa-purogamāḥ --- This includes illnesses like Mahodara (dropsy) etc.   
\end{enumerate}

\subsection*{Benefits of Mahāmudrā contined…}

\noindent \textbf{Verses 3.18}

\begin{shloka}
kathiteyaṁ mahā-mudrā mahā-siddhikarī nṛṇām |\\
gopanīyā prayatnena na deyā yasya kasyacit || 3.18 ||
\end{shloka}

\subsection*{Word Split}

Kathitā, iyam, mahā-mudrā, mahā-siddhikarī, nṛṇām, gopanīyā, prayatnena, na, deyā, yasya, kasyacit

\subsection*{Paraphrased Word Meaning}

\begin{multicols}{2}
iyam --- This \\
kathitā --- stated \\
mahāmudrā --- mahāmudrā\\
nṛṇām --- for human beings \\
mahā-siddhikarī --- bestower of great powers \\
prayatnena --- with effort \\
gopanīyā --- has to be guarded \\
yasya kasyacit --- to anyone (underserving)\\
na --- not \\
deyā --- should be given 
\end{multicols}

\subsection*{Purport}

The sated Mahāmudrā bestows great powers. Hence it has to be guarded safely and should not be given to anyone (undeserving).

\subsection*{Inputs from Jyotsnā Commentary}

\begin{enumerate}
\item nṛṇām --- this refers to those who practice 
\item mahāsiddhikarī --- Great powers include aṇimā etc. (Becoming small like atom etc)
\item yasya kasyacit --- This is a prohibition of giving the knowledge of this practice to anyone who does not deserve  (anadhikārī) to learn this. 
\end{enumerate}
\newpage

\section*{2. Uḍḍīyānabandha}

\noindent \textbf{Verses 3.55}

\begin{shloka}
baddho yena suṣumnāyāṁ prāṇastūḍḍīyate yataḥ |\\
tasmāduḍḍīyanākhyo'yaṁ yogibhiḥ samudāhṛtaḥ ||
\end{shloka}

\subsection*{Word Split}

baddhaḥ yena suṣumnāyām,  prāṇaḥ, tu, uḍḍīyate, yataḥ, tasmād, uḍḍīyanākhyaḥ, ayam, yogibhiḥ, samudāhṛtaḥ 

\subsection*{Paraphrased Word Meaning}

\begin{multicols}{2}
yataḥ --- as \\
yena --- by which \\
baddhaḥ --- locked\\
suṣumnāyām --- in the suṣumnā\\
prāṇaḥ --- the prāṇa\\
tu --- (filler)\\
uḍḍīyate --- flies  \\
tasmād --- Hence \\
ayam --- This \\
uḍḍīyanākhyaḥ --- named as uḍḍīyana\\
yogibhiḥ ---  by the masters of Yoga\\
samudāhṛtaḥ --- has been stated well
\end{multicols}

\subsection*{Purport}

That (practice), by which the prāṇa being locked (out of other channels and is directed towards suṣumnā) flies/soars in suṣumnā, is called as uḍḍīyana (uḍḍīyāna) by the masters of Yoga

\subsection*{Inputs from Jyotsnā Commentary}

\begin{enumerate}
\item Before describing the bandha the meaning of the term uḍḍīyāna is given in this verse. 
\item yogibhiḥ --- Matysendra and others are the Yogins referred to here. 
\item uḍḍīyāna --- The word is made up of prefix ut (above) and the root is ḍīṅ - to fly. (This indicates the upward movement of prāṇa by the practice the bandha) 
\end{enumerate}

\heading{Uḍḍīyāna-bandha Continued…}

\noindent \textbf{Verses 3.56}

\begin{shloka}
uḍḍīnaṁ kurute yasmād-aviśrāntaṁ mahā-khagaḥ |\\
uḍḍīyānaṁ tadeva syāt tatra bandho'bhidhīyate ||
\end{shloka}

\subsection*{Word Split}

uḍḍīnam,  kurute, yasmāt, aviśrāntam, mahā-khagaḥ, uḍḍīyānam, tad, eva, syāt, tatra, bandhaḥ, abhidhīyate 

\subsection*{Paraphrased Word Meaning}

\begin{multicols}{2}
mahā-khagaḥ --- The great one who moves \\
about in the space  \\
yasmād --- as \\
aviśrāntam --- tirelessly \\
uḍḍīnam --- flying \\
kurute --- does \\
tad --- That \\
eva --- only\\
uḍḍīyānam --- uḍḍīyāna (the name of the Bandha)\\
syāt --- should be \\
tatra --- About that \\
bandhaḥ --- (the practice of) lock\\
abhidhīyate --- is being stated 
\end{multicols}

\subsection*{Purport}

As the great entity that moves about (within) in the space (of the body) tirelessly (by the practice of this Bandha) flies (into suṣumnā). It will now be stated about the Bandha.

\noindent \textbf{Verses 3.57}

\begin{shloka}
udare paścimaṁ tānaṁ nābherūrdhavaṁ ca kārayet |\\
uḍḍīyāno hyasau bandho mṛtyu-mātaṅga-kesarī ||
\end{shloka}

\subsection*{Word Split}

Udare, paścimam, tānam, nābheḥ, ūrdhavam, ca, kārayet, uḍḍīyānaḥ, hi, asau bandhaḥ, mṛtyu-mātaṅga-kesarī

\subsection*{Paraphrased Word Meaning}

\begin{multicols}{2}
udare --- in the abdomen \\
nābheḥ --- of the navel \\
ūrdhvam --- above \\
ca --- also \\
paścimaṃ --- backwards \\
tānam --- pull \\
kārayet --- one should do \\
asau --- That \\
bandhaḥ --- lock \\
uḍḍīyānaḥ --- uḍḍīyāna \\
hi --- indeed \\
mṛtyu-mātaṅga-kesarī --- the lion (that) \\
torments the elephant of death
\end{multicols}

\subsection*{Purport}

One should pull back the abdomen above (and also below) the navel. Such a practice is called as uḍḍīyāna. This is the lion that torments the elephant of death. 

\subsection*{Inputs from Jyotsnā Commentary}

\begin{enumerate}
\item Ca (also) --- the word indicates that one should pull the abdomen region below the navel also.
\item paścimaṁ tānaṁ --- this refers to the pulling back of the aforementioned region of abdomen in such a(n instense) way that (as if) the abdominal region above an below the navel touches the back. 
\end{enumerate}

\heading{Uḍḍīyāna-bandha Continued…}

\noindent \textbf{Verses 3.58}

\begin{shloka}
uḍḍīyānaṁ tu sahajaṁ guruṇā kathitaṁ sadā |\\ 
abhyaset satataṁ yastu vṛddho'pi taruṇāyate ||
\end{shloka}

\subsection*{Word Split}

uḍḍīyānam, tu, sahajam, guruṇā, kathitam, sadām, abhyaset, satatam, yastu, vṛddhaḥ, api, taruṇāyate

\subsection*{Paraphrased Word Meaning}

\begin{multicols}{2}
uḍḍīyānam --- uḍḍīyāna(Bandha)\\
tu --- (filler) \\
guruṇā --- by the teacher \\
sadā --- always \\
sahajam --- natural \\
kathitam --- has been stated \\
yaḥ --- the one \\
satatam --- always \\
abhyaset ---  should practice \\
tu --- (filler)\\
vṛddhaḥ --- aged  \\
api --- also \\
taruṇāyate --- acts as a Youth 
\end{multicols}

\subsection*{Purport}

uḍḍīyāna that is taught by the Guru is always natural. The one who pratices this always, even if he is old attains youthfulness. 

\subsection*{Inputs from Jyotsnā Commentary}

\begin{enumerate}
\item Guru --- the one who advises that which is beneficial 
\item Sahaja --- this practice is natural because, when the air is exhaled, everyone can always experience the inward movement of the abdomen. 
\end{enumerate}

\heading{Uḍḍīyāna-bandha Continued…}

\noindent \textbf{Verses 3.59}

\begin{shloka}
nābherūrdhavam-adhaścāpi tānaṁ kuryāt prayatnataḥ |\\
ṣaṇmāsam-abhyasen-mṛtyuṁ jayatyeva na saṁśayaḥ ||
\end{shloka}

\begin{multicols}{2}
nābheḥ --- of the navel \\
ūrdhvam --- above \\
adhaḥ --- below \\
ca --- and \\
api --- also \\
prayatnataḥ --- with effort \\
tānam --- drawing   \\
kuryāt --- one should do \\
ṣaṇmāsam --- six months \\
abhyaset --- one should practice \\
mṛtyum --- death \\
jayati --- is conquered \\
eva --- indeed  \\
na ---  no\\
saṃśayaḥ --- doubt
\end{multicols}

\subsection*{Purport}

One should practice drawing the abdomen above and below the navel with effort. By practicing this for six months death is conquered. There is no doubt about this. 

\subsection*{Inputs from Jyotsnā Commentary}

\begin{enumerate}
\item Abhyaset --- One should practice again and again 
\item ṣaṇmāsam etc ---  These words are uttered to praise the practice. 
\end{enumerate}

\section*{(only) Benefits of Uḍḍīyāna-bandha}

\noindent \textbf{Verses 3.60}

\begin{shloka}
sarveṣām-eva bandhānām-uttamo hyuḍḍiyānakaḥ |\\
uḍḍiyāne dṛḍhe bandhe muktiḥ svābhāvikī bhavet ||
\end{shloka}

\subsection*{Word Split}

sarveṣām, eva, bandhānām, uttamaḥ, hi, uḍḍiyānakaḥ, uḍḍiyāne, dṛḍhe, bandhe, muktiḥ, svābhāvikī, bhavet 
\newpage

\subsection*{Paraphrased Word Meaning}

\begin{multicols}{2}
sarveṣām --- of all\\
bandhānām --- Bandhas (locks)\\
uḍḍiyānakaḥ --- uḍḍiyāna\\
eva --- only \\
uttamaḥ --- (is) the best \\
hi --- because \\
uḍḍiyāne --- on uḍḍiyāna\\
bandhe --- Bandha (lock)\\
dṛḍhe --- firmly (being attained)\\
muktiḥ --- liberation \\
svābhāvikī ---  naturally \\
bhavet --- should be attained
\end{multicols}

\noindent \textbf{Purport}

Among all the Bandhas uḍḍiyāna is the best. Because, if uḍḍiyāna is firmly attained, then liberation is naturally attained. 

\noindent \textbf{Inputs from Jyotsnā Commentary}

\begin{enumerate}
\item sarveṣām eva bandhānām --- Among 16 types of foundational Bandhas. (16 places of Bandhas can be seen in the notes to verse 3.73)
\item muktiḥ svābhāvikī --- When once through this Bandha Prāṇa enters suṣumnā and reaches the head region (mūrdha) then, Samādhi is attained and liberation follows naturally. 
\end{enumerate}
\newpage

\section*{3. Mūlabandha}

\begin{shloka}
pārṣṇibhāgena saṁpīḍya yonim-ākuñcayed gudam |\\
apānam-ūrdhvam-ākṛṣya mūla-bandho'bhidhīyate || 3.61 ||
\end{shloka}

\subsection*{Word Split}

pārṣṇibhāgena, saṁpīḍya, yonim, ākuñcayed, gudam, apānam, ūrdhvam, ākṛṣya, mūlabandhāḥ, abhidhīyate

\subsection*{Paraphrased Word Meaning}

\begin{multicols}{2}
pārṣṇi-bhāgena --- by the heel\\
yonim --- the perineum \\
saṃpīḍya --- having pressed \\ 
gudam --- the anus  \\
ākuñcayed --- one should constrict\\
apānam --- apāna (vāyu)\\
ūrdhvam --- upwards \\
ākṛṣya --- having drawn (one should stay)\\
mūla-bandhaḥ --- mūla-bandha (the base lock)\\
abhidhīyate --- this is stated 
\end{multicols}

\subsection*{Purport}

The perineum has to be pressed by the heel and the anus has to be constricted. The Apāna has to be drawn upwards. This is called as mūla-bandha.

\subsection*{Inputs from Jyotsnā Commentary}

\begin{enumerate}
\item pārṣṇi --- This refers to the heel region that is below the ankle
\item Yoni --- This refers to the region in between the anus and the genital  
\item Apānam --- the Apāna that always moves downwards has to be drawn upwards 
\end{enumerate}

\noindent \textbf{Verses 3.62}

\begin{shloka}
adhogatim-apānaṁ vā ūrdhva-gaṁ kurute balāt |\\
ākuñcanena taṁ prāhuḥ mūla-bandhaṁ hi yoginaḥ || 3.62 ||
\end{shloka}

\subsection*{Word Split}

Adhogatim, apānam, vā, ūrdhvagam, kurute, balāt, ākuñcanena, tam, prāhuḥ, mūlabandham, hi, yoginaḥ

\subsection*{Paraphrased Word Meaning}

\begin{multicols}{2}
adhogatim --- the one that moves downwards \\
apānam ---  the apāna vāyu\\
vai (vā) ---  certainly \\
ākuñcanena --- by constriction  \\
balāt --- with effort \\
ūrdhvagam --- moving upwards \\
kurute --- (the one who) does  \\
yoginaḥ --- the masters of Yoga\\
tam --- that  \\
mūla-bandham ---  mūla-bandha\\
prāhuḥ --- state\\
hi --- certainly 
\end{multicols}

\subsection*{Purport}

Constricting the anus with effort and thereby making the downward moving Apāna travel upwards is called as mūla-bandha by the masters of Yoga.

\subsection*{Inputs from Jyotsnā Commentary}

\begin{enumerate}
\item Ūrdhvagam --- upward movement here refers to the movement of the  prāṇa into suṣumnā
\item (Though apparently there is not much of diference between this verse and the previous verse, it has to be understood that) this verse is centred on explaining the meaning of the word mūla-bandham whereas the previous verse focusses on explaining the steps involved in the practice of mūla-bandham.
\end{enumerate}

\heading{Mūla-bandha Continued…}

\noindent \textbf{Verses 3.63}

\begin{shloka}
gudaṁ pārṣṇyā tu saṁpīḍya vāyum-ākuñcayed balāt |\\
vāraṁ vāraṁ yathācordhvaṁ samāyāti samīraṇaḥ ||
\end{shloka}

\subsection*{Word Split}

Gudam, pārṣṇyā, tu, saṁpīḍya, vāyum, ākuñcayed, balāt, vāram, vāram, yathā, ca, ūrdhvaṁ samāyāti, samīraṇaḥ

\subsection*{Paraphrased Meaning}

\begin{multicols}{2}
pārṣṇyā --- by the heel \\
tu ---  but  \\
gudam --- anus \\
saṃpīḍya --- having pressed \\
yathā --- as\\
ca --- also \\
samīraṇaḥ --- the air \\
ūrdhvam --- upwards \\
samāyāti --- travels/moves  \\
(tathā) --- (in that manner)\\
balāt --- with effort \\
vāram vāram --- again and again \\
vāyum --- the air\\
ākuñcayed --- one should constrict
\end{multicols}

\subsection*{Purport}

But, having pressed the anus with the heel one should repeatedly constrict the air (by the constriction of the anus) so that the it travels upwards. (This is mūlabandha)  

\subsection*{Inputs from Jyotsnā Commentary}

\begin{enumerate}
\item The practice of mūlabandha described here is according to the text Yogabija.
\item Tu (but) --- the word here indicates that how the prescription (of mūlabandha) is different here from the one in the previous verse.
\item The sentence “This is mūlabandha” has to be added to the complete (the descpition of the verse).
\item By repeatedly constricting the anus the air is drawn upwards.
\end{enumerate}

\subsection*{Benefits of Mūla-bandha and other related inputs}

\noindent \textbf{Verses 3.64}

\begin{shloka}
prāṇāpānau nāda-bindū mūla-bandhena caikatām |\\
gatvā yogasya saṁsiddhiṁ yacchato nātra saṁśayaḥ ||
\end{shloka}

\subsection*{Word Split}

prāṇāpānau, nādabindū, mūlabandhena, ca, ekatām, gatvā, yogasya, saṁsiddhim, yacchataḥ, na, atra, saṁśayaḥ 

\subsection*{Paraphrased word meaning}

\begin{multicols}{2}
prāṇāpānau ---  prāṇa and apāna \\
nādabindū --- nāda and bindū\\
ca --- and \\
mūlabandhena --- by mūlabandha \\
ekatām --- onness \\
gatvā --- having attained \\
yogasya --- of Yoga \\
saṃsiddhim --- success \\
yacchataḥ --- grant \\
atra --- here \\
saṃśayaḥ --- doubt \\
na --- no
\end{multicols}

\subsection*{Purport}

Prāṇa and apāna, nāda and bindu --- all these beomce one by the practice of mūlabandha. And they grant the success of Yoga. There is no doubt on this. 

\subsection*{Inputs from Jyotsnā Commentary}

\begin{enumerate}
\item prāṇāpānau --- This refers to the two vital airs that travel upwards and downwards repectively 
\item nādabindū --- nāda refers to anāhata sound (the sound that is experienced in the suṣumnā when prāṇa enters it). Bindu refers to anusvāra (the dot that is seen above in oṃ - ओं)
\item The meaing of this is --- When mūlabandha is practiced apāna becomes one with prāṇa and enters suṣumnā. By that the nāda manifests. Then the along the the nāda, prāṇa and apāna travel above hṛdaya (chest) and facilitate the onness of Nāda (śakti) with Bindu (śiva) and finally all these enter the top of the head. (binduḥ śivātmako bījaṃ śaktirnādaḥ - sāradātilakaḥ quoted in śabdakalpadruma - \url{https://tinyurl.com/2y8kbnkm}). Then success in Yoga is attained.  
\end{enumerate}

\heading{Benefits of Mūla-bandha continued…}

\noindent \textbf{Verses 3.6}

\begin{shloka}
apāna-prāṇayor-aikyaṁ kṣayo mūtra-purīṣayoḥ |\\
yuvā bhavati vṛddho'pi satataṁ mūla-bandhanāt ||
\end{shloka}

\subsection*{Word Split}

apānaprāṇayoḥ, aikyam, kṣayaḥ, mūtrapurīṣayoḥ, yuvā, bhavati, vṛddhaḥ, api, satatam, mūlabandhanāt 

\subsection*{Paraphrased Word Meaning}

\begin{multicols}{2}
satatam ---  always\\
mūlabandhanāt --- by the practice of mūlabandha\\
apāna-prāṇayoḥ --- of apāna and prāṇa\\
aikyam --- oneness \\
mūtrapurīṣayoḥ --- of urine and feces \\
vṛddhaḥ --- aged\\
api --- also\\
yuvā --- youth\\
bhavati --- will become
\end{multicols}

\subsection*{Purport}

By the constant practice of mūlabandha apāna and prāṇa unite. (Accumlated) Urine and feces (in the body) are removed. Aged will also become youthful.

\subsection*{Inputs from Jyotsnā Commentary}

\begin{enumerate}
\item kṣayo mūtra-purīṣayoḥ  --- This refers to the removal of the accumulated urine and feces in the body. 
\end{enumerate}

\heading{Benefits of Mūla-bandha continued…}

\noindent \textbf{Verses 3.66}

\begin{shloka}
apāna ūrdhvage jāte prāyāte vahni-maṇḍalam |\\
tadānala-śikhā dīrghā jāyate vāyunāhatā ||
\end{shloka}

\subsection*{Word Split}

Apāne, ūrdhvage, jāte, prāyāte, vahnimaṇḍalam, tadā, analaśikhā, dīrghā, jāyate, vāyunā, āhatā

\subsection*{Paraphrased Word meaning}

\begin{multicols}{2}
apāne --- on the apāna \\
ūrdhvage --- upward moving  \\
jāte --- having become \\
vahnimaṇḍalam --- the region/sphere of fire \\
prāyāte --- having attained \\
tadā --- then  \\
vāyunā --- by the wind \\
āhatā --- hit/fanned \\
analaśikhā  --- the flame/tuft of fire \\
dīrghā --- elongated  \\
jāyate --- (it) becomes
\end{multicols}

\subsection*{Purport}

When the apāna has turned upwards (by the practice of mūlabandha) and reaches the sphere of fire (region of jaṭharāgni), the flame of fire (jaṭharāgni) becomes elongated being hit/fanned the wind (apāna). 

\subsection*{Input from Jyotsnā commentary}

\begin{enumerate}
\item apāna ūrdhvage --- when the apāna which has the nature tendency of moving downwards has turned upwards.
\item vahnimaṇḍalam --- this is the sphere of fire which is triaṅgular in the shape. It is just below the navel. This has been stated by yājñavalkya – In the dehamadhya (centre of the body) there is the place of the fire. It is of the hue of molten gold. It is triaṅgular in shape in human beings and for fourlegged being it is of the shape of quadrilateral. For birds it is in the form of a sphere. What I state is true. In the middle of that always exisits a tiny flame in the fire.   (yogayājñavalkyasaṃhitā 4.11,12)   
\item vāyunāhatā --- the wind(vāyu) herein refers to  apāna
\end{enumerate}

\heading{4. Benefits of Mūla-bandha continued…}

\noindent \textbf{Verses 3.67}

\begin{shloka}
tato yāto vahnyapānau prāṇam uṣṇa-svarūpakam |\\
tenātyanta-pradīptastu jvalano deha-jas-tathā ||
\end{shloka}

\subsection*{Word Split}

tataḥ, yātaḥ,  vahnyapānau, prāṇam, uṣṇasvarūpakam, tena, atyantapradīptaḥ, tu, jvalanaḥ, dehajaḥ, tathā

\subsection*{Paraphrased Word Meaning}

\begin{multicols}{2}
tataḥ --- after that \\
vahnyapānau --- fire and apāna\\
uṣṇasvarūpakam --- the hot \\
prāṇam --- prāṇa\\
yātau --- both reach \\
tena --- by that \\
atyantapradīptaḥ --- intensely heated  \\
tu --- (filler)\\
dehajaḥ --- born in the body\\
jvalanaḥ --- the fire \\
tathā --- (filler)
\end{multicols}

\subsection*{Purport}

After that the fire and apāna reach prāṇa which is already hot. By that the fire of the body (digestive fire) becomes intensely hot. 

\subsection*{Inputs from Jyotsnā Commentary}

\begin{enumerate}
\item prāṇam uṣṇa-svarūpakam --- the one that has the nature of being hot. Or because of the elongated nature of the flame of the fire, the one that has (now) become hot. 
\item The fire that was hot being fanned by apāna, now becomes intensely hot after joining with prāṇa. 
\end{enumerate}

\heading{Benefits of Mūla-bandha continued…}

\noindent \textbf{Verses 3.68}

\begin{shloka}
tena kuṇḍalinī suptā santaptā saṁprabudhyate |\\
daṇḍāhatā bhujaṅgīva niśvasya ṛjutāṁ vrajet || 3.68 ||
\end{shloka}

\begin{multicols}{2}
tena --- by that \\
suptā --- asleep \\
kuṇḍalinī --- kuṇḍalinī    \\
santaptā --- being scorched \\
saṃprabudhyate ---  awakens \\
daṇḍāhatā --- being hit by a stick \\
bhujaṅgī --- female serpent \\
iva --- like \\
niśvasya ---  exhaling \\
ṛjutāṃ --- straightness \\
vrajet --- shall attain  
\end{multicols}

\subsection*{Purport}

By that the kuṇḍalinī that is asleep is scorched and like a female snake being hit by a stick, exhaling, it straightens outs. 
\newpage

\heading{Benefits of Mūla-bandha continued…}

\noindent \textbf{Verses 3.69}

\begin{shloka}
bilaṁ praviṣṭeva tato brahma-nāḍyantaraṁ vrajet |\\
tasmān-nityaṁ mūla-bandhaḥ kartavyo yogibhiḥ sadā ||
\end{shloka}

\subsection*{Word Split}

Bilam, praviṣṭā, iva, tataḥ, brahmanāḍyantaram, vrajet, tasmāt, nityam, mūlabandhaḥ, kartavyaḥm yogibhiḥ, sadā

\subsection*{Paraphrased Word Meaning}

\begin{multicols}{2}
tataḥ --- after that \\
bilam --- burrow \\
praviṣṭā --- having entered \\
iva --- as if \\
brahmanāḍyantaram --- into the suṣumnā -nāḍī\\
vrajet --- shall enter \\
tasmāt --- because of that \\
yogibhiḥ --- by the practitoners of Yoga \\
nityam --- constantly\\
sadā --- always\\
mūlabandhaḥ --- mūlabandha\\
kartavyaḥ --- should be done 
\end{multicols}

\subsection*{Purport}

After becoming straight, like a snake entering a burrow,   kuṇḍalinī enters suṣumnā-nāḍī. Hence, mūlabandha has to be done by practitioners of Yoga always.

\subsection*{Inputs from Jyotsnā Commentary}

\begin{enumerate}
\item Nityam --- everyday 
\item Sadā --- at all times 
\item kartavyaḥ --- should be done 
\end{enumerate}

\section*{4. Jālandhara-bandha}

\noindent \textbf{Verses 3.70 –}

\begin{shloka}
kaṇṭham-ākuñcya hṛdaye sthāpayeccibukaṁ dṛḍham |\\
bandho jālandharākhyo'yaṁ jarā-mṛtyu-vināśakaḥ ||
\end{shloka}

\subsection*{Word Split}

kaṇṭham, ākuñcya, hṛdaye, sthāpayet, cibukam, dṛḍham, bandhaḥ, jālandharākhyaḥ, ayam,  jarāmṛtyuvināśakaḥ 

\subsection*{Paraphrased Word Meaning}

\begin{multicols}{2}
kaṇṭham --- the throat\\
ākuñcya --- after constricting \\
hṛdaye --- in the chest \\
cibukam --- the chin \\
dṛḍham --- firmly \\
sthāpayet --- (one) should place \\
ayam --- This \\
jālandharākhyaḥ --- known as jālandhara\\
bandhaḥ --- bandha \\
jarā-mṛtyu-vināśakaḥ ---  destroyer of aging and death 
\end{multicols}

\subsection*{Purport}

The throat has to be constricted and the chin has to be placed on the chest firmly. This is called as Jālandhara Bandha which destroys aging and death. 

\subsection*{Inputs from Jyotsnā Commentary}

\begin{enumerate}
\item hṛdaye sthāpayeccibukam --- The chin has to be placed on the chest at a distance of 4 Aṅgulas (finger breadths) (from the throat).
\end{enumerate}
\newpage

\section*{Jālandhara-bandha}

\noindent \textbf{Verses 3.71}

\begin{shloka}
badhnāti hi sirā-jālam-adho-gāmi nabho-jalam |\\
tato jālandharo bandhaḥ kaṇṭha-dukhaugha-nāśanaḥ ||
\end{shloka}

\subsection*{Word Split}

Badhnāti, hi, sirā-jālam, adho-gāmi, nabho-jalam,  tataḥ, jālandharaḥ, bandhaḥ, kaṇṭhadukhaughanāśanaḥ 

\subsection*{Paraphrased Word Meaning}

\begin{multicols}{2}
hi --- indeed \\
sirā-jālam --- the web of channels of Prāṇa\\
badhnāti --- blocks \\
adho-gāmi ---  downward moving \\
nabho-jalam --- waters of the ether\\
tataḥ --- hence \\
jālandharaḥ bandhaḥ --- jālandhara bandha\\
kaṇṭha-dukhaugha-nāśanaḥ --- destroyer of the host of sufferings of the throat
\end{multicols}

\subsection*{Purport}

The jālandhara bandha blocks the web of channles of Prāṇa, (and consequently) the water of the ether (the diluted Candra – life energy present in the skull region closer to the eyebrows) that moves downwards is also blocked. Jālandhara Bandha removes host of sufferings related to throat.

\subsection*{Inputs from Jyotsnā Commentary}

\begin{enumerate}
\item This verse basically presents the meaning of the word jālandhara. As it blocks the jāla (web of channels of Prāṇa) this is jālandhara Bandha. Or the Jala (water) from the space above is blocked (Bandha) hence also it is jālandhara Bandha.
\item nabho-jalam --- this refers to the nectar (amṛta/ life force) that has the tendency to flow downwards from the space in the skull (from near the region of the eyebrows)
\end{enumerate}

\subsection*{Benefits of Jālandhara-bandha}

\noindent \textbf{Verses 3.72}

\begin{shloka}
jālandhare kṛte bandhe kaṇṭha-saṅkoca-lakṣaṇe |\\
na pīyūṣaṃ patatyagnau na ca vāyuḥ prakupyati ||
\end{shloka}

\subsection*{Word Split}

Jālandhare, kṛte, bandhe, kaṇṭhasaṅkocalakṣaṇe, na, pīyūṣam, patati, agnau, na, ca, vāyuḥ, prakupyati

\subsection*{Paraphrased Word Meaning}

\begin{multicols}{2}
kaṇṭha-saṅkoca-lakṣaṇe --- chacracteried by the constricting of the throat\\
jālandhare --- the jālandhara \\
bandhe --- Bandha \\
kṛte --- on being done \\
pīyūṣam ---  the nectar \\
agnau --- in the fire \\
na --- does not\\
patati ---  fall\\
vāyuḥ --- the air/prāṇa\\
ca --- also \\
na --- does not \\
prakupyati --- get vitiated
\end{multicols}

\subsection*{Purport}

When Jālaadhara Bandha, that is characterized by constricting the throat is done, the nectar does not fall into the (digestive) fire and the air / prāṇa does not get vitiated.

\subsection*{Inputs from Jyotsnā Commentary}

\begin{enumerate}
\item na ca vāyuḥ prakupyati --- Vitiation of Vāyu/prāṇa refers to flow of prāṇa in the channels in which it should not flow.
\end{enumerate}

\subsection*{Benefits of Jālandhara-bandha…}

\noindent \textbf{Verses 3.73}

\begin{shloka}
kaṇṭha-saṅkocanenaiva dve nāḍyau stambhayed dṛḍham |\\
Madhya-cakram-iḍāṃ jñeyaṃ ṣoḍaśādhāra-bandhanam ||
\end{shloka}

\subsection*{Word Split}

kaṇṭhasaṅkocanena,  eva, dve, nāḍyau, stambhayed, dṛḍham, Madhyacakram, iḍām, jñeyam, ṣoḍaśādhārabandhanam 

\subsection*{Paraphrased Word Meaning}

\begin{multicols}{2}
dṛḍham --- firmly\\
kaṇṭha-saṅkocanena ---  by constricting the throat \\
eva --- only \\
dve --- the two \\
nāḍyau ---  two channels of Prāṇa\\
stambhayed --- (one) should block\\
(ayaṃ jālandharaḥ --- This is jālandhara Bandha)\\
Iḍām --- This \\
Madhyacakram --- middle Cakra\\
jñeyam --- should be known \\
ṣoḍaśādhāra-bandhanam --- that which \\
blocks the 16 ādhāra (centres of Prāṇa)
\end{multicols}

\subsection*{Purport}

Only by constricting the throat the two channels of Prāṇa are blocked. This is jālandhara Bandha. And this (the throat) is the middle Cakra, that blocks (all) the 16 ādhāras (centres of Prāṇa).

\subsection*{Inputs from Jyotsnā Commentary}

\begin{enumerate}
\item dve nāḍyau --- Two nāḍīs are  iḍā and piṅgalā
\item Madhya-cakram --- this refers to  viśuddhi cakra
\item ṣoḍaśādhāra --- The 16 centres of  Prāṇa  are - aṅguṣṭha (toe), gulpha(ankle), jānu(knee), ūru(thigh), sīvanī(perineum), liṅga\break (genital), nābhi(navel), hṛd(chest), grīvā(neck), kaṇṭhadeśa\break (throat region), lambhikā(tongue), nāsikā(nose), bhrūmadhya (place in between eyebrows), lalāṭa(forehead), mūrdhā (the skull), brahmarandhra (aperture in the crown of the head). The benefits of doing Dhāraṇā/ or holding the Prāṇa (visualzing the presence of Prāṇa) in these 16 places has to be learnt from Gorakṣasiddhānta.
\end{enumerate}
\newpage

\section*{5. Additional Inputs on the three Bandhas}

\noindent \textbf{Verses 3.74}

\begin{shloka}
Mūla-sthānaṁ samākuñcya uḍḍiyānaṁ tu kārayet |\\
iḍāṁ ca piṅgalāṁ baddhvā vāhayet paścime pathi ||
\end{shloka}
 
\subsection*{Word Split}

Mūla-sthānam, samākuñcya, uḍḍiyānam, tu, kārayet, iḍām, ca, piṅgalām, baddhvā, vāhayet, paścime, pathi 

\subsection*{Paraphrased Word Meaning}

\begin{multicols}{2}
mūla-sthānam ---  the anus region \\
samākuñcya ---  after constricting \\
uḍḍiyānaṃ ---  uḍḍiyāna (bandha)\\
tu --- (filler) \\
kārayet ---  one should do \\
iḍāṃ ---  iḍā (the left nāḍī)\\
piṅgalāṃ ca --- piṅgalā (the right nāḍī)\\
baddhvā --- after blocking\\
paścime --- posterior (suṣumnā)\\
pathi --- (in the) path (prāṇam)\\
vāhayet --- one should make flow
\end{multicols}

\subsection*{Purport}
\vspace{-5pt}

After constricting the anus region (mūlabandha) one should do uḍḍiyāna-bandha. Block iḍā (the left nāḍī) and piṅgalā (the right nāḍī) (by Jālandhara bandha). Make the  prāṇa flow in the posterior path (suṣumnā). 

\subsection*{Inputs from Jyotsnā Commentary}
\vspace{-5pt}

\begin{enumerate}
\itemsep=0pt
\item The purpose of the three Bandhas that have been stated is presented  in this verse (making the prāṇa flow in suṣumnā). 
\item iḍāṁ ca piṅgalāṁ baddhvā --- This refers to the Jālandhara bandha. 
\item paścime pathi --- This refers to suṣumnā.
\end{enumerate}

\heading{Additional Inputs continued…}

\noindent \textbf{Verses 3.75}

\begin{shloka}
anenaiva vidhānena prayāti pavano layam |\\
tato na jāyate mṛtyuḥ jarā-rogādikaṁ tathā ||
\end{shloka}
 
\subsection*{Word Split}

Anena, eva, vidhānena, prayāti, pavanaḥ, layam, tataḥ, na, jāyate, mṛtyuḥ, jarārogādikam, tathā

\subsection*{Paraphrased Word Meaning}

\begin{multicols}{2}
anena  --- By this \\
vidhānena ---  method \\
eva ---  only\\
pavanaḥ ---  the air/ prāṇa\\
layam ---  steadiness/stillness  \\
prayāti --- attains \\
tataḥ ---  by that \\
mṛtyuḥ ---  death\\
jarā-rogādikam --- aging diseases etc \\
na --- does not   \\
jāyate ---  manifest \\
tathā --- and 
\end{multicols}

\subsection*{Purport}

By this method, the prāṇa attains steadiness/stillness. By this death, aging, disease etc do not manifest. 

\subsection*{Inputs from Jyotsnā Commentary}

\begin{enumerate}
\item pavano layam  --- This refers to steadiness and stillness of prāṇa. The presence of prāṇa without movement in the Brahmarandhra (suṣumnā) is called as Laya. When prāṇa  attains such Laya disease etc do not manifest.  
\end{enumerate}
\newpage

\heading{Additional Inputs continued…}

\noindent \textbf{Verses 3.76}

\begin{shloka}
Bandha-trayam-iḍāṁ śreṣṭhaṁ mahā-siddhaiśca sevitam |\\
sarveṣāṁ haṭha-tantrāṇāṁ sādhanaṁ yogino viduḥ ||
\end{shloka}

\subsection*{Word Split}

bandhatrayam, iḍām, śreṣṭham. mahāsiddhaiḥ, ca, sevitam, sarveṣām, haṭhatantrāṇām, sādhanam yoginaḥ, viduḥ

\subsection*{Paraphrased Word Meaning}

\begin{multicols}{2}
iḍām --- This \\
bandha-trayam ---  set of three Bandhas \\
śreṣṭham ---  (is ) great \\
mahā-siddhaiḥ --- by great siddhas (enlightened souls)\\
ca --- and \\
sevitam ---  practiced \\
sarveṣām --- of all \\
haṭha-tantrāṇām ---  techniques/practies of haṭha\\
sādhanam --- method for accomplishment \\
yoginaḥ --- the Yogins \\
viduḥ --- knew  
\end{multicols}

\subsection*{Purport}
\vspace{-5pt}

This set of three Bandhas is great. It has been practiced by great siddhas. It is considerd by the Yogins as the method of accomplishment of all techniques of Yoga.

\subsection*{Inputs from Jyotsnā Commentary}
\vspace{-5pt}

\begin{enumerate}
\itemsep=0pt
\item mahā-siddhaiḥ --- This refers to the great Yogins such as Matsyendra etc.
\item ca --- here indicates the inclusion of vasiṣṭha and other sages also
\item sādhanam --- \textit{Siddhijanakam} – that which leads to accomplishment (of goals of Yoga)
\end{enumerate}

\section*{6. Three verses on bandhas from Chapter 2}

\noindent \textbf{Verses 2.45}

\begin{shloka}
pūrakānte tu kartavyo bandho jālandharābhidhaḥ |\\
kumbhakānte recakadau kartavyastūḍḍiyanakaḥ || 45 ||
\end{shloka}

\subsection*{Word Split}

pūrakānte, tu ,kartavyaḥ, bandhaḥ, jālandharābhidhaḥ, kumbhakānte, recakādau, kartavyaḥ, tu, uḍḍiyānakaḥ

\subsection*{Paraphrased Word Meaing}

\begin{multicols}{2}
jālandharābhidhaḥ --- that which is known as jālandhara\\
bandhaḥ --- the Bandha \\
pūrakānte --- at the end of inhalation \\
tu --- but  \\
kartavyaḥ ---  should be done \\
uḍḍiyānakaḥ --- the uḍḍiyāna\\
tu --- (filler)\\
kumbhakānte ---  at the end of kumbhaka (hold)\\
recakadau ---  before the beginning exhalation\\
kartavyaḥ --- should be done 
\end{multicols}

\subsection*{Purport}

The jālandhara Bandha has to be done after inhalation and uḍḍiyāna Bandha has to be done at the end of the Kumbhaka (holding the breath within) but before the beginning of exhalation. 

\subsection*{Inputs from Jyotsnā Commentary}

\begin{enumerate}
\item bandhaḥ --- badhnāti prāṇavāyumiti bandhaḥ (the derivation of the term Bandha is given here) --- that which binds/locks the prāṇavāyu is Bandha.
\item Pūrakānte --- jālandhara Bandha has to be done immediately after inhalation 
\item tu --- tuśabdāt kumbhakādau --- but - the word tu used in the first occasion indicates that  jālandhara Bandha has to be done before the start of the holding the breath (but immediately after completing inhalation)
\item Kumbhakānte --- uḍḍiyāna Bandha has to be done when there is some holding of breath still left/just before completing holding the breath
\end{enumerate}
\newpage

\heading{Three verses on bandhas from Chapter 2 continued…}

\noindent \textbf{Verses 2.46}

\begin{shloka}
adhastāt kuñcanenāśu kaṇṭha-saṅkocane kṛte |\\
madhye paścima-tānena syāt prāṇo brahma-nāḍigaḥ || 46 ||
\end{shloka}

\subsection*{Word Split}

adhastāt, kuñcanena, āśu, kaṇṭhasaṅkocane, kṛte, madhye, paścimatānena, syāt , prāṇaḥ, brahmanāḍigaḥ

\subsection*{Paraphrased Word Meaning}

\begin{multicols}{2}
kaṇṭha-saṅkocane ---  When the constricting the throat \\
kṛte --- is done \\
āśu --- that \\
adhastāt ---  in the lower region \\
kuñcanena --- by the constrction \\
madhye --- in the middle \\
paścima-tānena ---  by drawing backwards \\
prāṇaḥ --- the prāṇa\\
brahma-nāḍigaḥ --- entrance into brahma-nāḍi (suṣumnā) \\
syāt --- should be
\end{multicols}

\subsection*{Purport}

After constricting the throat (jālandharabandha) and then quickly constricting the lower region (mūlabandha) and then by drawing (the abdomen) backwards (uḍyānabandha), the prāṇa will flow in the brahma-nāḍi (suṣumnā).

\subsection*{Inputs from Jyotsnā Commentary}

\begin{enumerate}
\itemsep=0pt
\item atredaṃ rahasyam --- The secret herein. Commencing with these words the commentator gives a lot of inputs. He states that if one knows jihvābandha and practices it before jālandharabandha, then - prāṇāyāma will be accomplished. All the benefits of prāṇāyāma such as leanness of the body etc (stated in verse 2.72) can be attained. In this case mūlabandha and uḍyānabandha would not be required. If one does not know jihvābandha then the method of practice of the three bandhas stated in this verse can be practiced. 
\item All the three Bandhas are to be learnt from the Guru. Especially if mūlabandha is not known properly then it leads to various diseases. To explain --- If by the practice of  mūlabandha one experiences depletion in the Dhātus, constipation, lack of appetite, feebleness of the nāda (the unstruct sound?), and (the practitioner’s) feces is formed in the shape of the globules that resembles the feces of the goat – it then means mūlabandha is not learnt and practiced properly. On the contrary, if by the practice of mūlabandha – the Dhātus are nourished, regular bowel movement that removes impurities from the body is experienced, stimulation of the gastric fire and also the manifestation of nāda is also experienced – then it means mūlabandha has been learnt properly.
\end{enumerate}
\vspace{-5pt}

\heading{Three verses on bandhas Contd…}

\noindent \textbf{Verses 2.47}
\vspace{-5pt}

\begin{shloka}
apānam-ūrdhavam-utthāpya prāṇaṁ kaṭhād-adho nayet |\\
yogī jarā vimuktaḥ san ṣoḍaśābda-vayā bhavet ||
\end{shloka}

\subsection*{Word Split}
\vspace{-10pt}

apānam, ūrdham, utthāpya, prāṇam, kaṇṭhāt, adhaḥ, nayet , yogī , jarā , vimuktaḥ, san, ṣoḍaśābdavayāḥ , bhavet

\subsection*{Paraphrased Word Meaning}
\vspace{-10pt}

\begin{multicols}{2}
apānam --- apāna\\
ūrdhavam --- above \\
utthāpya ---  drawing up \\
prāṇam ---  prāṇa\\
kaṭhād --- from the throat \\
adhaḥ ---  below \\
nayet --- one should take \\
yogi ---  a Yogin\\
jarā --- again\\
vimuktaḥ --- freed\\
san ---  being\\
ṣoḍaśābda-vayā --- sixteen years of age\\
bhavet --- shall become
\end{multicols}

\subsection*{Purport}

By drawing the apāna above and taking the prāṇa below, a Yogi shall become free from aging and shall become (a youth of) sixteen years of age.

\subsection*{Inputs from Jyotsnā Commentary}

\begin{enumerate}
\item Though this vese speaks of jālandhara and mūlabandhas only, but when these two are done, naturally uḍyānaandha will happen. Hence it has not been stated separately.
\end{enumerate}
\newpage

\section*{7. Viparīta Karaṇī (5)}

\noindent \textbf{Verses 3.79}

\begin{shloka}
Ūrdhva-nābher-adhaḥ tālvoḥ ūrdhvaṁ bhānur-adhaḥ śaśī |\\
karaṇī viparītākhyā guru-vākyena labhyate || 79 ||
\end{shloka}

\subsection*{Word Split}

ūrdhvanābheḥ, adhaḥ, tālvoḥ, ūrdhvam, bhānuḥ, adhaḥ, śaśī, karaṇī, viparītākhyā, guruvākyena, labhyate

\subsection*{Paraphrased Word Meaning}

\begin{multicols}{2}
ūrdhvanābheḥ --- of the one whose nave is above\\
adhaḥ ---  below \\
tālvoḥ --- of the palates  \\
ūrdhvam ---  above \\
bhānuḥ --- the sun \\
adhaḥ --- below \\
śaśī --- the moon\\ 
(iyam) --- this \\
viparītākhyā ---  known as viparīta (inversion)\\
karaṇī --- the act \\
guruvākyena ---  by the words of the teacher\\
labhyate --- is attained 
\end{multicols}

\subsection*{Purport}

For the one whose navel is above and the palate is below, the sun is above and the moon is below. This called as the practice of inversion which is attained by the words of the Guru. 

\subsection*{Inputs from Jyotsnā Commentary}

\begin{enumerate}
\item Viparītākhyā --- What is inverted here is the postion of the the sun and the moon. That is why it is called as Inversion.  
\end{enumerate}

\subsection*{Viparīta Karaṇī continued…}

\noindent \textbf{Verses 3.80}

\begin{shloka}
nityam-abhyāsa-yuktasya jaṭharāgni-vivardhinī |\\
āhāro bahulas-tasya sampādyaḥ sādhakasya ca || 80 ||
\end{shloka}

\subsection*{Word split}

nityam, abhyāsayuktasya, jaṭharāgnivivardhinī, āhāraḥ, bahulaḥ, tasya, sampādyaḥ, sādhakasya, ca

\subsection*{Paraphrased Word Meaning}

\begin{multicols}{2}
nityam --- always \\
abhyāsa-uktasya ---  endowed with practice \\
jaṭharāgni-vivardhinī ---  enhancer of gastric fire \\
tasya --- of that  \\
sādhakasya ---  practitioner \\
āhāraḥ --- food \\
bahulaḥ --- in plenty \\
sampādyaḥ ---  to be procured \\
ca --- (filler)
\end{multicols}

\subsection*{Purport}

For the one who practices constantly, it enhances gastric fire. Food in plenty has to be arranged for such a practitioner. 

\subsection*{Inputs from Jyotsnā Commentary}

\begin{enumerate}
\item Nityam --- everyday 
\end{enumerate}

\subsection*{Viparīta Karaṇī Cont…}

\noindent \textbf{Verses 3.81}

\begin{shloka}
alpāhāro yadi bhaved-agnir-dahati tat-kṣaṇāt |\\
adhaśśirāś-cordhva-pādaḥ kṣaṇaṁ syāt prathame dine ||
\end{shloka}

\subsection*{Word Split}

alpāhāraḥ, yadi, bhavet, agniḥ, dahati, tatkṣaṇāt, adhaśśirāḥ, ca, ūrdhvapādaḥ, kṣaṇam, syāt, prathame, dine

\subsection*{Paraphrased Word Meaning}

\begin{multicols}{2}
yadi --- if \\
alpāhāraḥ ---  consumer of food in little quantity\\
bhaved --- (one) shall be\\
agniḥ --- the (gastric) fire \\
tatkṣaṇāt ---  immediately\\ 
dahati --- scorches  \\
prathame --- in the first   \\
dine --- day\\
kṣaṇam --- a moment \\
adhaśśirāḥ --- the one who has the head below \\
ūrdhvapādaḥ ---  the one who has the feet above\\ 
ca --- and \\
syāt --- shall be
\end{multicols}

\subsection*{Purport}

If one consumes food scantily the fire scorches immediately. In the first day one should be in (this) head-down-feet-up position for just a moment. 

\subsection*{Inputs from Jyotsnā Commentary}

\begin{enumerate}
\item adhaśśirāḥ --- The palms should be placed on the hip region. The arms till the elbow region, the back of the neck and head should be on the ground.  This is what is meant by keeping the head on the ground.  	
\end{enumerate}

\subsection*{Viparīta-Karaṇī Contd…}

\begin{shloka}
kṣaṇācca kiñcid-adhikam-abhyasecca dine dine |\\
valitaṁ palitaṁ caiva ṣaṇmāsordhvaṁ na dṛśyate |\\
yāma-mātraṁ tu yo nityam-abhyaset sa tu kāla-jit || 82 ||
\end{shloka}

\subsection*{Word Split}

kṣaṇāt, ca, kiñcit, adhikam, abhyaset, ca, dine, dine, valitam, palitam, ca, eva, ṣaṇmāsordhvam, na dṛśyate, yāmamātram, tu, yaḥ nityam, abhyaset, saḥ, tu kālajit

\subsection*{Paraphrased Word Meaning}

\begin{multicols}{2}
dine dine --- daily, daily\\
kṣaṇāt --- a moment \\
ca ---  and \\
kiñcid ---  a little\\
adhikam --- more \\
abhyaset --- one should practice \\
ca  ---  filler \\
valitam --- wrinkles \\
palitam --- greying of hair\\
ca eva --- and also\\
ṣaṇmāsordhvam --- after six months \\
na --- not \\
dṛśyate --- will be seen \\
yaḥ ---  the one \\
tu --- but \\
yāmamātram --- for a period of yama \\
nityam --- regularly\\
abhyaset --- one should practice\\ 
saḥ --- he \\
tu --- filler  \\
kālajit --- conquerer of time
\end{multicols}

\subsection*{Purport}

Everyday one should practice a little more than a moment (than the previous day). A person who regulary practices (Viparīta Karaṇī) for a duration of a yāma (3 hours), he (the practitioner) conquers time.

\subsection*{Inputs from Jyotsnā Commentary}

\begin{enumerate}
\item sa tu kāla-jit (He conquers time) --- This indicates that by the practice of Yoga even the prārabdhakarma (according to Karma theory --- this refers to a set of actions that has led to the commencement of this life and the major experienes therein) (It is to be noted that it is generally held that by the practice of Yoga and other spiritual disciplines anything can be overcome but not that set of karmas which have started yielding effects in this life. It is like an arrow that has left the bow and will stop either on hitting a target or losing its momentum. It cannot be called back. But the commentator here indicates that even such prārabdhakarma can be overcome by Yoga. The commentaror also give references from viṣṇupurāṇa, bhāgavatapurāṇa jīvanmuktiviveka to support his claim). 
\end{enumerate}
\newpage

\section*{8. The other Mudrās}

The other five mudrās in Chapter 3 of Haṭhayogapradīpikā are – 

\begin{enumerate}
\item Mahābandha --- Verses 3.19--25 
\item mahāvedha  --- Verses 3.26--31
\item khecarī  --- 3.32--54
\item vajrolī/Sahajoli/Amaroli  --- 3.83--103 
\item śakticālana ---  3.104--123 
\end{enumerate}

Two more Mudrā s in Chapter - 4 (methods of Prāṇalaya)

\begin{enumerate}
\item śāmbhavī-mudrā --- verses 35--42
\item khecarī- mudrā --- verses 43--53
\end{enumerate}

These mudrās are not elaborated as they are not practiced in the Krishnamacharya Tradition followed in Krishnamacharya Yoga Mandiram.

