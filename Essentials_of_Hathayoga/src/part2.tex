\part{\textit{Āsana-s}, \textit{Prāṇāyāma-s} \& \textit{Mudrā-s}/\textit{Bandha-s}}

\heading{Introduction}

As is evident from Part 1 of this treatise \textit{Haṭhayogapradīpikā} discusses four major limbs of \textit{Haṭhayoga:} 
\begin{enumerate}
\itemsep=0pt
\renewcommand{\theenumi}{\alph{enumi}}
\renewcommand{\labelenumi}{\theenumi)}
\item \textit{Āsana-s}
\item \textit{Prāṇāyāma}
\item \textit{Mudrā-s/Bandha-s} 
\item \textit{Nādānusandhānana}
\end{enumerate}

In this section, with regard to those four limbs, a textual immersion into verses pertaining to those practices and techniques are presented which are in line with \textit{Śrī} Krishnamacharya tradition. The details of the verses taken up by for textual immersion is presented below:

\begin{enumerate}
\itemsep=0pt
\item \textit{Āsana-s} --- Chapter 1 --- Verses 17--54 (15 \textit{Āsana-s})
\item \textit{Prāṇāyāma} --- Chapter 2 ---  Verses 7--10 (\textit{Nāḍīshuddhi}), 35 (\textit{Kapālabhāti}), 44--68 (6 \textit{Kumbhakas} to the \textbf{exclusion of \textit{Plāvinī} and \textit{Mūrcchā}}), 
\item \textit{Mudrā-s} \& \textit{Bandha-s} --- Chapter 3 --- Verses 5--7 (introduction to \textit{Mudrā-s}), 10--18 (\textit{Mahamudrā}), 55--76 (also verses 45--47 --- Chapter 2) (three \textit{Bandha-s - Uḍyānabandha. Mūlabandha} and \textit{Jālandharabandha}), 79--82 (\textit{Viparītakaraṇī})
\end{enumerate}

\heading{Structure of Presentation}

At the outset a summary of all the techiniques of \textit{Āsana-s}, \textit{Prāṇāyāma-s}, \textit{Mudrā-s}/\textit{Bandha-s} will be given to give an overiew of the techniques that are being disussed. This will be followed by a detailed textual immersion. The way in which the verses are treated is as follows:
\begin{enumerate}
\itemsep=0pt
\item Each verse from the text is given (based on the Theosophical society, Adyar edition 1972)\footnote{\textit{Haṭhayogapradīpikā}, with the commentar of Brahmananda, Adyar Library, The Theosophical Society, Madras India (1972)}
\item The word split of the verse is given. This will help identify correct individual components of the verse.   
\item The Paraphrased word meaning of the verses are given. One may ask as to why paraphrasing might be required? To this, it has to be stated that - verses of \textit{Haṭhayogapradīpikā} are bound by meters. Hence, to fit the meters the word order of the sentence may be changed. Paraphrasing (\textit{Anvaya}) puts the words in the correct \textit{Saṁskṛta} grammatical order. When the words of the verses are given in the correct order, their relative position in sentence will become clear and there will be no element of ambiguity in understanding/interpreting the verse.    
\item The purport of the verse is given. 
\item Inputs and insights from \textit{Jyotsnā} commentary if available is added 
%~ \item Image of the technique is presented as applicable.
\end{enumerate}
