\chapter{\textit{Āsana-s}}

\subsection*{Summary}

\section*{Introduction about \textit{āsanas}}

As \textit{Āsana} is the first limb of Haṭha, it is stated first. \textit{Āsana} bestows firmness, health and lightness of limbs.17 

I (svātmārāma) state a few \textit{āsanas} accepted by Munis like vasiṣṭha and others and by Yogins like matsyendra and others.  1.18

\heading{\textit{Āsana} -- 1:  Svastikāsana}

Keeping the two soles of the feet in between the knee and thigh, being seated well with an upright body, is called as Svastikāsana.  1.19

\heading{\textit{Āsana} -- 2:   Gomukhāsana}

The right ankle has to be placed besides the back on the left side. In the same manner the left ankle has to be placed (on the right side beside the back). This is Gomukh\textit{Āsana} that resembles the face of a cow. 1.20

\heading{\textit{Āsana} -- 3:   Vīrāsana}

One (right) foot has to be placed on one thigh firmly and on the other foot the (left) thigh has to be placed. This is stated as Virāsana. 1.21

\heading{\textit{Āsana} -- 4:  Kūrmāsana}

Pressing the anus with both the ankles and being seated in a head down manner is Kuarmāsana. This is known by those who are well versed in Yoga. 1.22

\heading{\textit{Āsana} -- 5:  Kukkuṭāsana}

Onen should assume \textit{Padmāsana}. After that, inserting the hands in between the knees and thighs and placing them on the ground one should raise above (from the ground). This is called Kukkutāsana. (Kukkuta-Cock) 1.23

\heading{\textit{Āsana} -- 6: Uttānakūrmakāsana}

From Kukkut\textit{Āsana} connecting the neck with the hands (that are inserted in between the thighs and knees) and lying on back on the ground is called as Uttanakurmaka \textit{Āsana} (the upturned turtle posture) 1.24

\heading{\textit{Āsana} -- 7:  Dhanurāsana}

Both the big toes are to be held by the hands. Then, till ears one should stretch (one foot)  like drawing a bow. This is called as Dhanurāsana\footnote{This posture is known as Akarnadhanurasana.} (bow posture). 1.25

\heading{\textit{Āsana} -- 8:  Matsyendrāsana}

Hold the right foot that is placed on the base of the left thigh and (hold) the left foot wound outside the right knee with twisted limbs. This is the \textit{Āsana} stated by Sri Matsyanātha. 1.26

\heading{Benefits of Matsyendrāsana}

By practice, Matsyendrāsana, which is the weapon that cuts asunder host of terrible diseases, grants - glow of the digestive fire. It also awakens the  Kuṇḍalinī, and grants firmness is Candra (the vital energy) for the practitioners. 1.27
\medskip

\heading{\textit{Āsana} -- 9:  Paścimatānāsana}

Stretching both legs straight on the ground like stick one should hold the tips of the feet with the arms and place the forehead on the knees and stay in that position. This is stated as Paścimatanāsana.  1.28


\heading{Benefits of Paścimatānāsana}

Paścimatān\textit{Āsana} is the foremost among the \textit{āsanas}. It makes the air (prāṇa) traverse in the Suṣumnā. It increases the glow of the digestive fire. It leads to leanness in the abdomen and freedom from disease.  1.29
\medskip

\heading{\textit{Āsana} -- 10:  Mayūrāsana}

By placing the hands on the ground and supporting the body, and also keeping the elbows besides the navel one should raise the body from the ground like a stick (horizontal to the ground). This posture is called as Mayurāsana.  1.30

\heading{Benefits of Mayūrāsana}

Mayur\textit{Āsana} quickly removes all diseases like enlargement of the spleen, odema. It also helps overcome imbalances in doṣas (Vāta, pitta and kapha). It reduces to ashes (digests) food that has been partaken indiscriminately without any residue. It kindles the digestive fire and also digests vicious poison even. 1.31
\medskip

\heading{\textit{Āsana} -- 11:  Śavāsana}

Lying on the back on the floor like a corpse is called śavāsana. It removes tiredness and grants relaxation to the mind. 1.32

\heading{Prelude to four important postures}

Lord Śiva has stated 84 \textit{āsanas}. From among them I will state four essential \textit{āsanas}.

\heading{The Four Important \textit{āsanas}}

Siddhāsana, \textit{Padmāsana}, Simh\textit{Āsana} and Bhadr\textit{Āsana} are the four best \textit{āsanas}. Even among them, one should always stay in the comfortable Siddhāsana.  1.34

\heading{\textit{Āsana} -- 12: Siddh\textit{Āsana} - 1}

The left heel should press the perineum firmly. Then one should place the right feet (heel) above the genital.  The chin has to be placed firmly in the chest and one should stay still with controlled senses and gaze the place in between the eyebrows. This is Siddh\textit{Āsana} that breaks open the doors of Liberation.  1.35

\heading{Siddh\textit{Āsana} - 2}

The left ankle has to be placed above the genital. The right ankle has to be placed above (the sole of the left foot). This should be Siddhāsana. 1.36

\heading{other nomenclatures of Siddhāsana}

This \textit{Āsana} is stated variously as Siddhāsana, Vajrāsana, Mukt\textit{Āsana} and Guptāsana. 1.37

\heading{Benefits of Siddhāsana}

Like moderate diet  among Yamas and ahimsā in Niyamas, one Siddh\textit{Āsana} is known as important among all \textit{āsanas}.  1.38

Among 84 \textit{āsanas} practice only siddhāsana. It cleanses all the seventy two thousand Nāḍīs.  1.39

With the constant practice of Siddh\textit{Āsana} for twelve years and moderate diet, the one who meditates upon the Ātman,  attains success (attains the goal of Yoga) 1.40

What is the use of many \textit{āsanas} if Siddh\textit{Āsana} is achieved and also if Kevala kumbhaka is attained,? The Samādhi state which is pleasing like the digit of the moon manifests on its own. 1.41

Similarly, if the Siddh\textit{Āsana} is firmly attained, without difficulty the three Bandhas manifest on their own.   1.42

There is no \textit{Āsana} like Siddhāsana. There is no Kumbhaka like Kevalakumbhaka. There is no mudrā like khecarimudrā. And there is no Laya practice comparable to \textit{Nādānusandhānana}. 1.43
\medskip

\heading{\textit{Āsana} -- 13:  \textit{Padmāsana} - 1}

The right foot (sole facing upwards) has to be placed on the left thigh and on the right thigh the left foot has to be placed. The hands crossing from behind should catch hold of the two big toes. Placing the chin on the chest and one should gaze the tip of the nose. This is \textit{Padmāsana}\footnote{This is known as Baddha-padmāsana} that destroys the diseases of the self-controlled practitioners. 1.44

\heading{Padm\textit{Āsana} - 2}

Both the feet with upward facing (sole) has to be placed on (opposite) thighs. The hands with upwards facing palms (placed one above the other) have to be placed in between the thighs. After that, the gaze… (Description continued in the next verse) 1.45

…(gaze) should be fixed on the tip of the nose. The tongue should press against the incisors. The chin has to be placed on the chest and the air has to be raised. (Description continued and completed in next verse) 1.46

This is \textit{Padmāsana} that destroys all diseases. This is difficult to practice and it is attained only by an intelligent person on the earth.  1.47

\heading{Padmāsanam - 3}

Being seated in the \textit{Padmāsana} the hands (palms) are to be placed one above the other firmly. The chin has to be placed firmly on the chest. Meditating upon that (Brahman/ iṣṭadevatā (personal diety)) the apāna has to be drawn up and Prāṇa has to be pushed down repeatedly. Such a practitioner will attain a matchless enlightenment due to efficacy of the śakti (power). 1.48

\heading{Benefits of \textit{Padmāsana}}

The Yogi who being seated in \textit{Padmāsana} holds the breath inhaled through the Nāḍīs (practices \textit{Prāṇāyāma}) attains liberation. There is no doubt about this. 1.49
\medskip

\heading{\textit{Āsana} -- 14:  Siṃhāsana}

Both the ankles of the feet are to be placed below the scrotum on the sides of the perineum. The left ankle has to be placed on the right side and the right ankle on the left side. (Descrption of the \textit{Āsana} is continued in the next verse) 1.50

The hands are to be placed on the knees and the fingers are to be spread. With a wide opened mouth one should gaze at the tip of the nose. 1.51

\heading{Benefits of siṃhāsana}

This best among the \textit{āsanas} is worshipped by great Yogins. This facilitates the practice of /brings together the three Bandhas.  1.52
\medskip

\heading{\textit{Āsana} -- 15: Bhadr\textit{Āsana} \& its Benefits}

Below the scrotum and on the sides of the perineum the two ankles have to be placed. The right ankle has to be placed on the right side and the left ankle on the left. (Description of the \textit{Āsana} is continued and completed in the next verse) 1.53

The sides of the feet (of which the soles are facing each other) are to be held firmly by both the hands. This is Bhadr\textit{Āsana} that destroys all diseases. This is also called as \textit{Gorakṣāsana} by Siddha Yogins. 1.54
\newpage

\thispagestyle{empty}
~
\vfill
\begin{center}
\textbf{\Huge Textual Immersion}
\end{center}
\vfill
\eject

\section*{Introduction about \textit{āsanas}}


\noindent \textbf{Verses 1.17}

\begin{shloka}
\textbf{hāṭhasya prathamāṅgatvādāsanaṃ pūrvamucyate|}\\
\textbf{kuryāt tadāsanaṃ sthairyamārogyaṃ  cāṅgalāghavam||}
\end{shloka}

\subsection*{Word Split}

hāṭhasya, prathamāṅgatvād, \textit{Āsana}m, pūrvam, ucyate, kuryāt,  tad, \textit{Āsana}m, sthairyamm ārogyaṃ,  ca, aṅgalāghavam

\subsection*{Paraphrased word meaning}

\begin{longtable}{>{\noindent\raggedright}p{5cm}>{\noindent\raggedright}p{5cm}}
hāṭhasya --- of hāṭha & \textit{Āsana}ṃ --- posture\tabularnewline
prathamāṅgatvād --- as it is the first limb & sthairyam --- firmness\tabularnewline
āsanaṃ --- posture & ārogyaṃ --- health\tabularnewline
pūrvam --- earlier & aṅgalāghavam --- lightness of limbs\tabularnewline
ucyate --- (is) stated & ca --- and\tabularnewline
tad --- that & kuryāt --- it shall do
\end{longtable}


\subsection*{Purport}


As \textit{Āsana} is the first limb of Haṭha, it is stated first. \textit{Āsana} bestows firmness, health and lightness of limbs

\subsection*{Inputs from Jyotsnā Commentary}

\begin{enumerate}
\item In verse 1.56 --- \textit{āsanas}, \textit{Prāṇāyāma}, \textit{Mudrās} and Nādānusandhāna are presented as the limbs of Haṭha. There \textit{Āsana} is presented as the first limb. 
\item Sthairya --- refers both stability at the level of body and also at the level of mind. This refers to the capabily of the \textit{āsanas} to reduce Rajas.
\item By aṅgalāghavam --- the efficacy of \textit{Āsana} in overcoming heaviness of body is indicated. Heaviness is an attribute of Tamas. Hence the capability of \textit{āsanas} to reduce Tamas is also indicated. 
\end{enumerate}

\newpage
\noindent \textbf{Verses 1.18}

\begin{shloka}
\textbf{vasiṣṭhādyaiśca munibhiḥ matsyendrādyaiśca yogibhiḥ|}\\
\textbf{aṅgīkṛtānyāsanāni kathyante kānicinmayā||}
\end{shloka}


\subsection*{Word Split}

vasiṣṭhādyaiḥ, ca munibhiḥ, matsyendrādyaiḥ, ca yogibhiḥ, aṅgīkṛtāni, āsanāni, kathyante kānicit, mayā

\subsection*{Paraphrased meaning}

\begin{longtable}{>{\noindent\raggedright}p{5cm}>{\noindent\raggedright}p{5cm}}
vasiṣṭhādyaiḥ --- by vasiṣṭha and others & yogibhiḥ - by Yogins\tabularnewline
ca - and & aṅgīkṛtāni  - accepted\tabularnewline
munibhiḥ  --- by Munis & kānicit --- a few\tabularnewline
matsyendrādyaiḥ --- matsyendra and others & āsanāni --- \textit{āsanas}\tabularnewline
ca  --- and & mayā --- by me\tabularnewline
kathyante --- are stated & 
\end{longtable}

\subsection*{Purport}

I (svātmārāma) state a few \textit{āsanas} accepted by Munis like vasiṣṭha and others and by Yogins like matsyendra and others.  

\subsection*{Inputs from Jyotsnā Commentary}

\begin{enumerate}
\item Difference between Munis and Yogins – Though for both Munis and Yogins manana (reflecting/contemplating ) and practice of Haṭha are common – still, for  Munis - manana is important and for Yogins practice of Haṭha is important. These \textit{āsanas} that are stated in this text are acceptable to these two kinds of practitioners of Yoga.
\end{enumerate}

\section*{\textit{Āsana} -- 1: Svastikāsana}

\noindent \textbf{Verses 1.19}

\begin{shloka}
\textbf{jānūrvorantare samyak kṛtvā pādatale ubhe|}\\
\textbf{ṛjukāyaḥ samāsīnaḥ svastikaṃ tat pracakṣate|| 1.19||}
\end{shloka}

\subsection*{Word Split}

jānūrvoḥ, antare, samyak, kṛtvā, pādatale, ubhe, ṛjukāyaḥ, samāsīnaḥ, svastikaṃ, tat, pracakṣate
\vspace{-5pt}

\subsection*{Paraphrased word meaning}
\vspace{-10pt}

\begin{longtable}{>{\noindent\raggedright}p{5cm}>{\noindent\raggedright}p{5cm}}
jānūrvoḥ --- of the knee and the thigh & ṛjukāyaḥ - upright body\tabularnewline
antare --- in  between & samāsīnaḥ --- seated well\tabularnewline
ubhe --- the two & tat --- that\tabularnewline
pādatale --- soles of the feet & svastikaṃ --- svastika\tabularnewline
samyak --- well & pracakṣate --- stated\tabularnewline
kṛtvā --- doing (keeping) & 
\end{longtable}
\vspace{-5pt}

\subsection*{Purport}
\vspace{-10pt}

Keeping the two soles of the feet in between the knee and thigh, being seated well with an upright body, is called as Svastikāsana.

\subsection*{Inputs from Jyotsnā Commentary}

\begin{enumerate}
\item Though it has been stated that the soles of the feet are to be placed inbetween the knees and the thighs – it is to be understood that – the soles of the feet are to be placed in between the thighs and the shanks (the region of the leg from knee to the ankle/ the region of calf muscles). Or rather the reading itself can be taken as jaṅghorvoḥ (the shanks and the thighs) rather than jānūrvoḥ (of the knees and the thighs)
\end{enumerate}
\newpage

\section*{\textit{Āsana} -- 2: Gomukhāsana}

\noindent \textbf{Verses 1.20}

\begin{shloka}
\textbf{savye dakṣiṇagulphaṃ tu pṛṣṭhapārśve niyojayet|}\\
\textbf{dakṣiṇe'pi tathā savyaṃ gomukhaṃ gomukhākṛti||}
\end{shloka}


\subsection*{Word Split}

Savye, dakṣiṇagulphaṃ, tu, pṛṣṭhapārśve, niyojayet, dakṣiṇe, api, tathā, savyaṃ, gomukhaṃ gomukhākṛti

\subsection*{Paraphrased word meaning}

\begin{multicols}{2}
dakṣiṇagulphaṃ --- the right ankle\\ 
savye --- on the left\\ 
pṛṣṭhapārśve --- besides the back\\ 
tu --- indeed\\ 
niyojayet --- place/join\\ 
dakṣiṇe --- on the right (behind the back)\\ 
api --- also\\
tathā --- in the same manner\\
savyaṃ --- the left (ankle)\\
gomukhaṃ --- (this is) Gomukha (āsana)\\
gomukhākṛti --- (which is in the) form of a face of a cow\\
\end{multicols}

\subsection*{Purport}

The right ankle has to be placed besides the back on the left side. In the same manner the left ankle has to be placed (on the right side beside the back). This is Gomukh\textit{Āsana} that resembles the face of a cow.

\subsection*{Inputs from Jyotsnā Commentary}

\begin{enumerate}
\item Though it has been stated that the right ankle has to be placed beside the back, going by sampradāya (tradition), the right ankle has to be placed under the left hip.
\end{enumerate}
\newpage

\section*{\textit{Āsana} -- 3: Vīrāsana}

\noindent \textbf{Verses 1.21}

\begin{shloka}
\textbf{ekaṃ pādaṃ tathaikasmin vinyasedūruṇi sthiram|}\\
\textbf{itarasmiṃstathā coruṃ vīrāsanamitīritam||}
\end{shloka}

\subsection*{Word Split}

ekaṃ, pādaṃ,  tathā, ekasmin, vinyased, ūruṇi, sthiram,itarasmin, tathā, ca, ūruṃ, vīrāsanam, iti, īritam

\subsection*{Paraphrased Word Meaning}

\begin{multicols}{2}
ekaṃ --- one (the right)\\
pādaṃ --- foot\\ 
tatha --- (filler)\\  
ekasmin --- on one\\ 
ūruṇi – on the thigh\\ 
sthiram - firmly\\  
vinyased – one should place\\
itarasmin – on the other (foot)\\
tathā  - similarly (place)\\
ca --- and\\
ūruṃ --- thigh\\
vīrāsanam --- vīrāsana\\
iti --- thus\\
īritam --- stated
\end{multicols}

\subsection*{Purport}

One (right) foot has to be placed on one thigh firmly and on the other foot the (left) thigh has to be placed. This is stated as Virāsana.

\subsection*{Inputs from Jyotsnā Commentary}

\begin{enumerate}
\item Ekam padam refers to the right feet.
\end{enumerate}
\newpage

\section*{\textit{Āsana} -- 4: Kūrmāsana}

\noindent 
\textbf{Verses 22}

\begin{shloka}
\textbf{gudaṃ nirudhya gulphābhyāṃ vyutkrameṇa samāhitaḥ|}\\
\textbf{kūrmāsanaṃ bhavedetaditi yogavido viduḥ||}
\end{shloka}

\subsection*{Word split}

gudaṃ, nirudhya, gulphābhyāṃ, vyutkrameṇa, samāhitaḥ, kūrmā\-sanaṃ, bhaved, etad, iti yogaviḍāḥ viduḥ

\subsection*{Paraphrased Word meaning}

\begin{longtable}{>{\noindent\raggedright}p{5cm}>{\noindent\raggedright}p{5cm}}
gulphābhyāṃ --- by the two ankles  & etad --- this\tabularnewline
gudaṃ --- the anus  & kūrmāsanaṃ --- Kurmāsana\tabularnewline
nirudhya --- after pressing & bhaved --- becomes\tabularnewline
vyutkrameṇa --- in a inverted manner/head down & iti --- thus\tabularnewline
samāhitaḥ --- (one should be) seated  & yogaviḍāḥ --- those well versed in yoga\tabularnewline
viduḥ --- know & 
\end{longtable}

\subsection*{Purport}

Pressing the anus with both the ankles and being seated in a head down manner is Kuarmāsana. This is known by those who are well versed in Yoga. 

%\subsection*{Image}
\newpage


\section*{\textit{Āsana} -- 5: Kukkuṭāsana}

\noindent 
\textbf{Verses 1.23}

\begin{shloka}
\textbf{padmāsanaṃ tu saṃsthāpya jānūrvorantare karau|}\\
\textbf{niveśya bhūmau saṃsthāpya vyomasthaṃ kukkuṭāsanam||23||}
\end{shloka}

\subsection*{Word Split}

padmāsanaṃ, tu, saṃsthāpya, jānūrvoḥ, antare, karau, niveśya, bhūmau, saṃsthāpya, vyomasthaṃ, kukkuṭāsanam

\subsection*{Paraphrased word meaning}
\vspace{-10pt}

\begin{longtable}{>{\noindent\raggedright}p{5cm}>{\noindent\raggedright}p{5cm}}
padmāsanaṃ --- \textit{Padmāsana}ṃ & niveśya – inserting\tabularnewline
tu  --- (filler) & bhūmau --- on the ground\tabularnewline
saṃsthāpya --- assuming  & saṃsthāpya --- placing well\tabularnewline
karau --- the hands  & vyomasthaṃ --- in the space\tabularnewline
jānūrvorḥ --- of the knees and the thighs  & kukkuṭāsanam --- kukkuṭ\textit{Āsana} \tabularnewline
antare --- in between & 
\end{longtable}
\vspace{-10pt}

*(Filler words are used in Verses to fulfil metrical requirments. They do not carry any specific meaning. They are mostly indeclinable forms. They also have various contextual meaning. It will be clarified by the commentators.)
\vspace{-10pt}

\subsection*{Purport}
\vspace{-10pt}

Onen should assume \textit{Padmāsana}. After that, inserting the hands in between the knees and thighs and placing them on the ground one should raise above (from the ground). This is called Kukkutāsana. (Kukkuta-Cock)
\vspace{-10pt}

\subsection*{Input from Jyotsnā Commentary}
\vspace{-10pt}

Here also the word Jānu should be taken to indicate jaṅgha --- the Shank region. 

\section*{\textit{Āsana} -- 6: Uttānakūrmakāsana}

\noindent \textbf{Verses 1.24}

\begin{shloka}
\textbf{kukkuṭāsanabandhastho dorbhyāṃ sambadhya kandharām|}\\
\textbf{bhavet kūrmavaduttāna etaduttānakūrmakam||}
\end{shloka}

\subsection*{Word Split}

kukkuṭāsanabandhasthaḥ,  dorbhyāṃ, sambadhya,  kandharām, bhavet,  kūrmavad, uttānaḥ etad, uttānakūrmakam

\subsection*{Paraphrased word meaning}

\begin{longtable}{>{\noindent\raggedright}p{5cm}>{\noindent\raggedright}p{5cm}}
kukkuṭāsanabandhasthaḥ --- Being in the Kukkut\textit{Āsana} posture  & kūrmavad --- like a tortoise\tabularnewline
dorbhyāṃ --- by the two arms  & uttānaḥ --- lying on the back\tabularnewline
sambadhya --- connecting  & bhavet  --- one should be\tabularnewline
kandharām --- the neck  & etad --- This (is)\tabularnewline
uttānakūrmakam --- uttānakūrm\textit{Āsana} & 
\end{longtable}

\subsection*{Purport}

From Kukkut\textit{Āsana} connecting the neck with the hands (that are inserted in between the thighs and knees) and lying on back on the ground is called as Uttanakurmaka \textit{Āsana} (the upturned turtle posture)

\subsection*{Image}
\newpage

\section*{\textit{Āsana} -- 7: Dhanurāsana}

\noindent 
\textbf{Verses 1.25}

\begin{shloka}
\textbf{pādāṅguṣṭhau tu pāṇibhyāṃ gṛhītvā śravaṇāvadhi|}\\
\textbf{dhanurākarṣaṇaṃ kuryāt dhanurāsanamucyate||}
\end{shloka}

\subsection*{Word Split}

pādāṅguṣṭhau, tu, pāṇibhyāṃ, gṛhītvā, śravaṇāvadhi, dhanurā\-karṣa\-ṇaṃ, kuryāt, dhanurāsanam, ucyate

\subsection*{Paraphrased word Meaning}

\begin{longtable}{>{\noindent\raggedright}p{5cm}>{\noindent\raggedright}p{5cm}}
pādāṅguṣṭhau – the big toes of the feet  & dhanurākarṣaṇaṃ - (like) stretching a bow \tabularnewline
tu  - (filler)  & kuryāt  - one should do\tabularnewline
pāṇibhyāṃ - by both the hands  & dhanurāsanam – (this is) dhanurāsanam\tabularnewline
gṛhītvā  - after holding  & ucyate – stated (as)\tabularnewline
śravaṇāvadhi – till the ears & 
\end{longtable}

\subsection*{Purport}

Both the big toes are to be held by the hands. Then, till ears one should stretch (one foot)  like drawing a bow. This is called as Dhanurāsana\footnote{This posture is known as Akarnadhanurasana.} (bow posture).

\subsection*{Inputs from Jyotsnā Commentary}

\begin{enumerate}
\item One leg has to remain stretched. The toe of it has to be held by the hand.  With the other hand the other leg has to be drawn till the ear. 
\end{enumerate}

\section*{\textit{Āsana} -- 8: Matsyendrāsana}

\noindent \textbf{Verses 1.26}

\begin{shloka}
\textbf{vāmorumūlārpitadakṣapādaṃ jānorbahiveṣṭitavāmapādam|}\\
\textbf{pragṛhya tiṣṭhet parivartitāṅgaḥ śrīmatsyanāthoditamāsanaṃ syāt||26||}
\end{shloka}
\vspace{-10pt}

\subsection*{Word Split}

vāmorumūlārpitadakṣapādaṃ,  jānorbahiveṣṭitavāmapādam, pragṛhya,  tiṣṭhet, parivartitāṅgaḥ śrīmatsyanāthoditam, \textit{Āsana}ṃ ,syāt

\subsection*{Paraphrased word meaning}
\vspace{-10pt}

\begin{longtable}{>{\noindent\raggedright}p{5cm}>{\noindent\raggedright}p{5cm}}
vāmorumūlārpitadakṣapādaṃ --- the right
foot has to be placed at the base of the  
left thigh  & pragṛhya --- holding \tabularnewline
jānoḥ --- of the (right) knee & tiṣṭhet --- stay\tabularnewline
bahiḥ --- outside  & parivartitāṅgaḥ --- twisting the limbs\tabularnewline
veṣṭitavāmapādam --- the left foot has to be placed  & śrīmatsyanāthoditam --- this is stated by śrīmatsyanātha\tabularnewline
āsanaṃ --- posture & \tabularnewline
syāt --- shall be & 
\end{longtable}
\vspace{-10pt}

\subsection*{Purport}
\vspace{-10pt}

Hold the right foot that is placed on the base of the left thigh and (hold) the left foot wound outside the right knee with twisted limbs. This is the \textit{Āsana} stated by Sri Matsyanātha.
\vspace{-10pt}

\subsection*{Inputs from Jyotsnā Commentary}
\vspace{-10pt}

\begin{enumerate}
\itemsep=0pt
\item The right foot that is placed at the base of the left thigh has to be held above the ankle by the left hand from behind. 
\item The left foot, which is wound around the right knee, has to held by the right hand in the big toe. The hand has to wind around outside the left foot. 
\item In this process one has to twist one limbs to face back side on the left side.
\item As this has been stated by Matsyendranātha this is called as Matsyendrāsana. 
\item The same \textit{Āsana} can also be done on the other side also.  
\end{enumerate}

\subsection*{Benefits of Matsyendrāsana}

\noindent 
\textbf{Verses 1.27}

\begin{shloka}
\textbf{matseyendrapīṭhaṃ jaṭharapradīptiṃ pracaṇḍarugmaṇḍalakhaṇḍanāstram|}\\
\textbf{abhyāsataḥ kuṇḍalinīprabodhaṃ candrasthiratvaṃ ca dadāti puṃsām||}
\end{shloka}

\subsection*{Word split}

matseyendrapīṭhaṃ,  jaṭharapradīptiṃ,  pracaṇḍarugmaṇḍalakhaṇḍanāstram, 
abhyāsataḥ, kuṇḍalinīprabodhaṃ,  candrasthiratvam, ca, dadāti, puṃsām

\subsection*{Paraphrased Word Meaning}

\begin{longtable}{>{\noindent\raggedright}p{5cm}>{\noindent\raggedright}p{5cm}}
abhyāsataḥ --- by practice, & kuṇḍalinīprabodhaṃ --- awakening of kuṇḍalinī\tabularnewline
pracaṇḍarugmaṇḍala\-khaṇḍa\-nāstram --- a 
weapon to cut asunder the host  of terrible\tabularnewline 
diseases & candrasthiratvaṃ --- firmness  of Candra\tabularnewline
matseyendrapīṭhaṃ --- Matysendr\textit{Āsana} & ca --- and\tabularnewline
jaṭharapradīptiṃ --- Glow of the  abdominal (digestive) fire & puṃsām --- for human beings\tabularnewline
dadāti --- bestows & 
\end{longtable}
 
\subsection*{Purport}

By practice, Matsyendrāsana, which is the weapon that cuts asunder host of terrible diseases, grants - glow of the digestive fire. It also awakens the  Kuṇḍalinī, and grants firmness is Candra (the vital energy) for the practitioners.
\newpage

\section*{\textit{Āsana} -- 9: Paścimatānāsana}

\noindent 
\textbf{Verses 1.28}

\begin{shloka}
\textbf{prasārya pādau bhuvi daṇḍarūpau dorbhyāṃ padāgradvitayaṃ gṛhītvā|}\\
\textbf{jānūpari nyastalalāṭadeśo vasediḍāṃ paścimatānamāhuḥ||}
\end{shloka}
\vspace{-10pt}

\subsection*{Word Split}
\vspace{-10pt}

Prasārya, pādau, bhuvi, daṇḍarūpau, dorbhyām , padāgradvitayaṃ, gṛhītvā, jānūpari nyastalalāṭadeśaḥ,  vased, iḍām,paścimatānam, āhuḥ

\subsection*{Paraphrased word meaning}
\vspace{-10pt}

\begin{longtable}{>{\noindent\raggedright}p{5cm}>{\noindent\raggedright}p{5cm}}
bhuvi --- on the ground & jānūpari --- on the knees\tabularnewline
pādau --- both the legs  & nyastalalāṭadeśaḥ --- placing the forehead\tabularnewline
daṇḍarūpau --- in the form of sticks  & vased --- one should stay\tabularnewline
prasārya --- after stretching  & iḍāṃ --- this\tabularnewline
dorbhyāṃ --- by the arms  & paścimatānam --- (as) paścimatāna\tabularnewline
padāgradvitayaṃ --- the tip of both the legs  & āhuḥ --- is stated \tabularnewline
gṛhītvā --- after holding & 
\end{longtable}
\vspace{-10pt}

\subsection*{Purport}
\vspace{-10pt}

Stretching both legs straight on the ground like stick one should hold the tips of the feet with the arms and place the forehead on the knees and stay in that position. This is stated as Paścimatanāsana.
\vspace{-10pt}

\subsection*{Inputs from Jyotsnā Commentary}
\vspace{-10pt}

\begin{enumerate}
\itemsep=0pt
\item The ankles of the both the legs that have been stretched on the ground should touch each other. 
\item The respective feet have to be held at the toes with both index fingers which are bent like hook.  
\item While bending forward and touching the knee with the forehead, the lower part of the knee should not raise above the ground.
\end{enumerate}
\vspace{-10pt}

\subsection*{Benefits of Paścimatānāsana}
\vspace{-10pt}

\noindent 
\textbf{Verses 1.29}

\begin{shloka}
\textbf{iti paścimatānamagryaṃ pavanaṃ paścimavāhinaṃ karoti|}\\
\textbf{udayaṃ jaṭharānalasya kuryādudare kārśyamarogatāṃ ca||}
\end{shloka}
\vspace{-10pt}

\subsection*{Word Split}
\vspace{-10pt}

Iti, paścimatānam, agryam, pavanam, paścimavāhinam, karoti, udayam, jaṭharānalasya kuryād, udare, kārśyam, arogatām, ca

\subsection*{Paraphrased word meaning}
\vspace{-10pt}

\begin{longtable}{>{\noindent\raggedright}p{5cm}>{\noindent\raggedright}p{5cm}}
iti --- thus & jaṭharānalasya --- of the fire of the abdomen\tabularnewline
paścimatānam --- paścimatāna (posture) & udayaṃ --- arousal\tabularnewline
agryaṃ  --- the foremost &  kuryād --- makes\tabularnewline
pavanaṃ --- the breath  & udare --- in the abdomen\tabularnewline
paścimavāhinaṃ --- makes it traverse the suṣumnā & kārśyam --- leanness\tabularnewline
karoti --- does & arogatāṃ --- freedom from disease \tabularnewline
ca --- also & 
\end{longtable}
\vspace{-10pt}

\subsection*{Purport}
\vspace{-10pt}

Paścimatān\textit{Āsana} is the foremost among the \textit{āsanas}. It makes the air (prāṇa) traverse in the Suṣumnā. It increases the glow of the digestive fire. It leads to leanness in the abdomen and freedom from disease.
\vspace{-10pt}

\subsection*{Inputs from Jyotsnā Commentary}
\vspace{-10pt}

1. Paścima here refers to Suṣumnā.
\newpage

\section*{\textit{Āsana} -- 10: Mayūrāsana}


\noindent 
\textbf{Verses 1.30}

\begin{shloka}
\textbf{dharāmavaṣṭabhya karadvayena tatkūrparasthāpitanābhipārśvaḥ|}\\
\textbf{uccāsano daṇḍavadutthitaḥ khe māyūrametat pravadanti pīṭham||}
\end{shloka}

\subsection*{Word Split}

Dharām, avaṣṭabhya, karadvayena, tatkūrparasthāpitanābhipārśvaḥ, uccāsanaḥ, daṇḍavad, utthitaḥ, khe, māyūram, etat, pravadanti, pīṭham

\subsection*{Paraphrased Word Meaning}

\begin{longtable}{>{\noindent\raggedright}p{5cm}>{\noindent\raggedright}p{5cm}}
karadvayena --- by both the hands &  daṇḍavad --- like  a stick\tabularnewline
dharām --- the ground  & utthitaḥ --- rising\tabularnewline
avaṣṭabhya --- having supported (the body) & khe --- in the sky\tabularnewline
tatkūrparasthāpitanābhipārśvaḥ --- on the elbow of the hands the sides of the navel  region has to be placed & māyūram --- related to  mayūra (peacock)\tabularnewline
uccāsanaḥ --- in an elevated posture & etat --- this\tabularnewline
pīṭham --- posture  & \tabularnewline
pravadanti --- (they) state & 
\end{longtable}

\subsection*{Purport}

By placing the hands on the ground and supporting the body, and also keeping the elbows besides the navel one should raise the body from the ground like a stick (horizontal to the ground). This posture is called as Mayurāsana.

\subsection*{Inputs from Jyotsnā Commentary}

\begin{enumerate}
\item While placing the hands on the ground the fingers of the hands should be spread. (Probably to give more balance to the posture) 
\end{enumerate}

\subsection*{Benefits of Mayūrāsana}

\noindent 
\textbf{Verses 1.31}

\begin{shloka}
\textbf{harati sakalarogānāśu gulmodarādīn}\\
\textbf{abhibhavati ca doṣānāsanaṃ śrīmayūram|}\\
\textbf{bahu kadaśanabhuktaṃ bhasmakuryādaśeṣam}\\
\textbf{janayati jaṭharāgniṃ jārayet kālakūṭam||}
\end{shloka}

\subsection*{Word Split}

Harati, sakalarogān, āśu, gulmodarādīn, abhibhavati, ca, doṣān, \textit{Āsana}ṃ, śrīmayūram
Bahu, kadaśanabhuktam, bhasmakuryād, aśeṣam, janayati, jaṭharāgnim, jārayet, kālakūṭam

\subsection*{Paraphrased Word Meaning}

\begin{longtable}{>{\noindent\raggedright}p{5cm}>{\noindent\raggedright}p{5cm}}
śrīmayūram --- the respected mayūra  & bahu --- in huge quantity\tabularnewline
āsanaṃ --- posture  & kadaśanabhuktaṃ --- indiscriminately part taken food\tabularnewline
sakalarogān --- all diseases   & aśeṣam --- without residue/completely\tabularnewline
gulmodarādīn --- like enlargement of the spleen, odema (dropsy) etc. &  bhasma --- (to)ashes  \tabularnewline
āśu --- quickly  & kuryād --- renders, \tabularnewline
harati --- removes & janayati --- generates/kindles,\tabularnewline
doṣān --- the (imbalance in) humors  & jaṭharāgniṃ --- the abdominal fire,\tabularnewline
ca --- also & kālakūṭam --- vicious poison \tabularnewline
abhibhavati --- overcomes   & jārayet --- digests
\end{longtable}

\subsection*{Purport}

Mayur\textit{Āsana} quickly removes all diseases like enlargement of the spleen, odema. It also helps overcome imbalances in doṣas (Vāta, pitta and kapha). It reduces to ashes (digests) food that has been partaken indiscriminately without any residue. It kindles the digestive fire and also digests vicious poison even. 

\subsection*{Inputs from Jyotsnā Commentary}

\begin{enumerate}
\item The word doṣa in the verse indicates Vata, pita and Kapha.
\end{enumerate}
\newpage

\section*{\textit{Āsana} -- 11: Śavāsana}

\subsection*{verse 1.32}

\begin{shloka}
\textbf{uttānaṃ śavavadbhūmau śayanaṃ tacchavāsanam|}\\
\textbf{śavāsanaṃ śrāntiharaṃ cittaviśrāntikārakam||}
\end{shloka}

\subsection*{Word Split}

Uttānam, śavavad, bhūmau, śayanam, tat, śavāsanam, śavāsanam, śrāntiharaṃ, cittaviśrāntikārakam

\subsection*{Paraphrased Word Meaning}

\begin{longtable}{>{\noindent\raggedright}p{5cm}>{\noindent\raggedright}p{5cm}}
bhūmau --- on the floor  & śavāsanam --- (is) śav\textit{Āsana} śavāsanaṃ\tabularnewline
śavavad  --- like a cropse  & śrāntiharaṃ --- removes tiredness\tabularnewline
uttānaṃ --- on the back  & cittaviśrāntikārakam --- gives relaxation to the mind\tabularnewline
śayanaṃ --- lying  & \tabularnewline
tat --- that  & 
\end{longtable}

\subsection*{Purport}

Lying on the back on the floor like a corpse is called śavāsana. It removes tiredness and grants relaxation to the mind. 

\subsection*{Inputs from Jyotsnā Commentary}

\begin{enumerate}
\item It removes the tiredness caused by the practice of Haṭha. 
\end{enumerate}

\subsection*{Prelude to four important postures}

\noindent 
\textbf{Verses 1.33}

\begin{shloka}
\textbf{caturaśītyāsanāni śivena kathitāni ca|}\\
\textbf{tebhyaścatuṣkamādāya sārabhūtaṃ bravīmyaham||}
\end{shloka}

\subsection*{Word Split}

Caturaśītyāsanāni, śivena, kathitāni, ca, tebhyaḥ, catuṣkam, ādāya, sārabhūtaṃ, bravīmi, aham||

\subsection*{Paraphrased Word Meaning}

\begin{longtable}{>{\noindent\raggedright}p{5cm}>{\noindent\raggedright}p{5cm}}
śivena --- by Lord Śiva & sārabhūtaṃ --- essential\tabularnewline
caturaśītyāsanāni --- eighty four postures  & catuṣkam --- four\tabularnewline
kathitāni --- have been stated  & ādāya --- taking\tabularnewline
ca --- also & aham  --- I\tabularnewline
tebhyaḥ --- from those & bravīmi --- will state
\end{longtable}

\subsection*{Purport}

Lord Śiva has stated 84 \textit{āsanas}. From among them I will state four essential \textit{āsanas}.

\subsection*{Inputs from Jyotsnā Commentary}

\begin{enumerate}
\item The Ca – also – indicats that Śiva has also stated 84 Lakhs of \textit{āsanas}. He has not merely stated 84 \textit{āsanas}, He has also stated 84 Lakhs of \textit{āsanas}. 
\item There are as many \textit{āsanas} as the number of species of beings. Maheśvara (Śiva) knows all their divisions. 
\end{enumerate}

\subsection*{The Four Important \textit{āsanas}}

\noindent 
\textbf{Verses 1.34}

\begin{shloka}
\textbf{siddhaṃ padmaṃ tathā siṃhaṃ bhadraṃ ceti catuṣṭayam|}\\
\textbf{śreṣṭhaṃ tatrāpi ca sukhe tiṣṭhet siddhāsane sadā||}
\end{shloka}

\subsection*{Word Split}

Siddham, padmam, tathā, siṃham, bhadram, ca, iti, catuṣṭayam, śreṣṭham, tatra, api, ca, sukhe tiṣṭhet, siddhāsane, sadā

\subsection*{Paraphrased Word Meaning}

\begin{longtable}{>{\noindent\raggedright}p{5cm}>{\noindent\raggedright}p{5cm}}
siddhaṃ --- Siddh\textit{Āsana} & śreṣṭhaṃ --- best\tabularnewline
padmaṃ --- \textit{Padmāsana},  & tatra --- among them\tabularnewline
tathā --- and & api --- also\tabularnewline
siṃhaṃ --- Simh\textit{Āsana} & ca --- and\tabularnewline
bhadraṃ --- Bhadr\textit{Āsana} & sukhe --- in the comfortable\tabularnewline
ca, iti --- and thus,  & siddhāsane --- in Siddhāsana\tabularnewline
catuṣṭayam --- the four & sadā --- always\tabularnewline
tiṣṭhet --- one should stay. &
\end{longtable}

\subsection*{Purport}

Siddhāsana, \textit{Padmāsana}, Simh\textit{Āsana} and Bhadr\textit{Āsana} are the four best \textit{āsanas}. Even among them, one should always stay in the comfortable Siddhāsana.

\subsection*{Inputs from Jyotsnā Commentary}

\begin{enumerate}
\item It is clear from this verse that Siddh\textit{Āsana} is the best among the four. 
\end{enumerate}

\section*{\textit{Āsana} -- 12: Siddh\textit{Āsana} - 1}

\noindent 
\textbf{Verses 1.35}

\begin{shloka}
\textbf{yonisthānakamaṅghrimūlaghaṭitaṃ kṛtvā dṛḍhaṃ vinyaset}\\
\textbf{meḍhre pādamathaikameva hṛdaye kṛtvā hanuṃ susthiram}\\
\textbf{sthāṇuḥ saṃyamitendriyo'caladṛśā paśyet bhruvorantaram}\\
\textbf{hyetanmokṣakapāṭabhedajanakaṃ siddhāsanaṃ procyate ||}
\end{shloka}

\subsection*{Word Split}

Yonisthānakam, aṅghrimūlaghaṭitam, kṛtvā, dṛḍham, vinyaset, meḍhre, pādam, atha, ekam, eva, hṛdaye, kṛtvā, hanuṃ, susthiram, sthāṇuḥ, saṃyamitendriyaḥ, acaladṛśā, paśyet, bhruvoḥ, antaram, hi, etat, mokṣa-kapāṭa-bheda-janakaṃ, siddhāsanam, procyate

\subsection*{Paraphrased Word Meaning}

\begin{multicols}{2}
yonisthānakam --- the perineum \\
dṛḍhaṃ --- firmly  \\
aṅghrimūlaghaṭitaṃ --- conjoined with the base (heel) of the (left) feet  \\
kṛtvā --- having made \\
atha --- then\\
ekam --- one ( right)  \\
pādam --- the feet \\
meḍhre --- above the genitals \\
eva --- only \\
vinyaset --- place \\
hṛdaye --- in the chest \\
susthiram --- firmly \\
hanuṃ --- the chin \\
kṛtvā --- placing \\
sthāṇuḥ --- motionless\\
saṃyamitendriyaḥ --- having controlled the senses  \\
acaladṛśā  --- with a  steady gaze  \\
hi --- certainly \\
bhruvoḥ --- of the brows  \\
antaram  --- in between \\
paśyet --- one should see \\
etat --- this  \\
mokṣa-kapāṭa-bheda-janakaṃ --- breaks open the doors of the liberation \\
siddhāsanaṃ --- siddh\textit{Āsana} \\
procyate --- is stated. 
\end{multicols}

\subsection*{Purport}

The left heel should press the perineum firmly. Then one should place the right feet (heel) above the genital.  The chin has to be placed firmly in the chest and one should stay still with controlled senses and gaze the place in between the eyebrows. This is Siddh\textit{Āsana} that breaks open the doors of Liberation.

\subsection*{Inputs from Jyotsnā Commentary}

\begin{enumerate}
\item Yoni-Sthana – refers to the place inbetween the anus and the genital. 
\item The verse merely states one foot has to be placed in the Yoni. It refers to the heel of the left foot. 
\end{enumerate}
\newpage

\section*{Siddh\textit{Āsana} - 2}

\noindent 
\textbf{Verses 1.36}

\begin{shloka}
\textbf{meḍhrādupari vinyasya savyaṃ gulphaṃ tathopari|}\\
\textbf{gulphāntaraṃ ca nikṣipya siddhāsanamiḍāṃ bhavet||}
\end{shloka}

\subsection*{Word Split}

meḍhrād, upari, vinyasya, savyam, gulpham, tathā, upari, gulphāntaram, ca, nikṣipya, siddhāsanam, iḍām, bhavet

\subsection*{Paraphrased Word Meaning}

\begin{multicols}{2}
savyaṃ --- the left  \\
gulphaṃ --- ankle  \\
meḍhrād --- genital  \\ 
upari ---  above  \\
vinyasya --- having placed  \\
tatha --- also  \\
upari --- above  \\
gulphāntaraṃ --- the other (right) ankle  \\
ca --- and  \\
nikṣipya ---  after placing (one should stay) \\
iḍāṃ --- this \\
siddhāsanam --- siddh\textit{Āsana} \\
bhavet ---  should be
\end{multicols}
  
\textbf{Purport}

The left ankle has to be placed above the genital. The right ankle has to be placed above (the sole of the left foot). This should be Siddhāsana.

\subsection*{Inputs from Jyotsnā Commentary}


\begin{enumerate}
\item It has to be noted that –“the right ankle has to be placed above (it)” – does not refer to placing of the right ankle above the left ankle but it refers to placing of the right ankle on the sole of the left foot. 
\end{enumerate}

\subsection*{other nomenclatures of Siddhāsana}

\noindent \textbf{Verses 1.37}

\begin{shloka}
\textbf{etat siddhāsanaṃ prāhuranye vajrāsanaṃ viduḥ|}\\
\textbf{muktāsanaṃ vadantyeke prāhurguptāsanaṃ pare||}
\end{shloka}
\vspace{-10pt}

\subsection*{Word Split}
\vspace{-10pt}

Etat, siddhāsanam, prāhuḥ, anye vajrāsanam, viduḥ, muktāsanaṃ, vadanti, eke, prāhuḥ, guptāsanam, pare
\vspace{-10pt}

\subsection*{Paraphrased Word Meaning}
\vspace{-10pt}

\begin{multicols}{2}
\itemsep=0pt
etat --- this   \\
siddhāsanaṃ --- siddh\textit{Āsana}   \\
prāhuḥ --- (some) state \\
anye --- others   \\
vajrāsanaṃ --- vajr\textit{Āsana}  \\
viduḥ --- know (this as) \\
eke --- some \\
muktāsanaṃ --- mukt\textit{Āsana}   \\
vadanti --- speak \\
pare --- others \\
guptāsanaṃ ---  gupt\textit{Āsana}  \\
prāhuḥ --- state  
\end{multicols}
\vspace{-10pt}

\subsection*{Purport}
\vspace{-10pt}

This \textit{Āsana} is stated variously as Siddhāsana, Vajrāsana, Mukt\textit{Āsana} and Guptāsana.

\subsection*{Inputs from Jyotsnā Commentary}

\begin{enumerate}
\itemsep=0pt
\item Siddh\textit{Āsana} is pressing perineum with the left heel and placing the right heel above the genital. 
\item Vajr\textit{Āsana} is pressing the perineum with the right heel and placing the left heel above the genital.
\item When the right heel is placed above the left and both press against the perineum it is called as Muktāsana.
\item When the right heel is placed above the left and both are placed above the genital it is called as Guptāsana.
\end{enumerate}
\newpage

\subsection*{Benefits of Siddhāsana}


\noindent \textbf{Verses 1.38}

\begin{shloka}
yameṣviva mitāhāramahiṃsāṃ niyameṣviva|\\
mukhyaṃ sarvāsaneṣvekaṃ siddhāḥ siddhāsanaṃ viduḥ||
\end{shloka}

\subsection*{Word Split}

yameṣu, iva, mitāhāramahiṃsāṃ niyameṣu, iva, mukhyaṃ, sarvāsaneṣu, ekam, siddhāḥ, siddhāsanam, viduḥ

\subsection*{Paraphrased Word Meaning}


\begin{multicols}{2}
yameṣu --- amongst yamas \\
mitāhāram --- moderate diet  \\ 
iva --- similarly   \\
niyameṣu ---  amongst the niyamas,  \\
ahiṃsāṃ --- nonviolence  \\
iva ---  like  \\
sarvāsaneṣu --- amongst all \textit{āsanas}   \\
ekaṃ --- one  \\
mukhyaṃ --- important  \\
siddhāsanaṃ --- siddhāsana,  \\
siddhāḥ --- accomplished   \\
viduḥ --- state
\end{multicols}

\subsection*{Purport}

Like moderate diet  among Yamas and ahimsā in Niyamas, one Siddh\textit{Āsana} is known as important among all \textit{āsanas}. 

\subsection*{Inputs from Jyotsnā Commentary}

\begin{enumerate}
\item From this verse and next six verses the benefits of Siddh\textit{Āsana} are stated.
\end{enumerate}

\newpage
\noindent \textbf{Verses 1.39}

\begin{shloka}
caturaśītipīṭheṣu siddhameva sadābhyaset|\\
dvāsaptatisahasrāṇāṃ nāḍīnāṃ malaśodhanam||
\end{shloka}

\subsection*{Word Split}

caturaśītipīṭheṣu, siddham,  eva, sadā, abhyaset, dvāsaptatisahasrāṇām, nāḍīnām, malaśodhanam

\subsection*{Paraphrased Word Meaning}


\begin{multicols}{2}
\itemsep=0pt
caturaśītipīṭheṣu ---  among 84 \textit{āsanas} \\
siddham --- Siddh\textit{Āsana} \\
eva ---  only \\
sadā --- always \\ 		
abhyaset ---  practice	 \\
dvāsaptatisahasrāṇāṃ ---  of seventy two thousand \\
nāḍīnāṃ --- channels of prāṇa \\
malaśodhanam ---  purification (happens)
\end{multicols}

\subsection*{Purport}

Among 84 \textit{āsanas} practice only siddhāsana. It cleanses all the seventy two thousand Nāḍīs. 

\newpage
\noindent \textbf{Verses 1.40}

\begin{shloka}
ātmadhyāyī mitāhārī yāvaddvādaśavatsaram|\\
sadā siddhāsanābhyāsāt yogī niṣpattimāpnuyāt||
\end{shloka}

\subsection*{Word split}

Ātmadhyāyī, mitāhārī, yāvaddvādaśavatsaram, sadā, siddhāsanābhyāsāt, yogī, niṣpattim, āpnuyāt

\subsection*{Paraphrased Word Meaning}

\begin{multicols}{2}
\itemsep=0pt
dvādaśavatsaram --- twelve years \\
yāvad --- till \\
sadā --- constant \\
siddhāsanābhyāsāt --- practice of siddh\textit{Āsana} \\
ātmadhyāyī --- the one who meditates on the consciousness \\
mitāhārī --- partaking moderate diet  \\
yogī --- yoga practitioner \\
niṣpattim ---  success  \\
āpnuyāt --- attains 
\end{multicols}

\subsection*{Purport}

With the constant practice of Siddh\textit{Āsana} for twelve years and moderate diet, the one who meditates upon the Ātman,  attains success (attains the goal of Yoga)

\newpage
\noindent \textbf{Verses 1.41}

\begin{shloka}
kimanyairbahubhiḥ pīṭhaiḥ siddhe siddhāsane sati|\\
prāṇānile sāvadhāne baddhe kevalakumbhake|\\
utpadyate nirāyāsāt svayamevonmanī kalā||
\end{shloka}

\subsection*{Word Split}

Kim, anyaiḥ, bahubhiḥ, pīṭhaiḥ, siddhe, siddhāsane, sati, prāṇānile, sāvadhāne, baddhe kevalakumbhake, utpadyate, nirāyāsāt, svayam, eva, unmanī,  kalā 

\subsection*{Paraphrased Word Meaning}

\begin{multicols}{2}
\itemsep=0pt
siddhāsane ---  on siddh\textit{Āsana}  \\
siddhe sati --- being attained   \\
kim --- what   \\
anyaiḥ --- by others   \\
bahubhiḥ --- by many  \\
pīṭhaiḥ ---  by \textit{āsanas} \\
prāṇānile ---	when the breath\\
sāvadhāne ---  on becoming steady	 \\
kevalakumbhake --- kevelakumbhaka  \\
baddhe ---  on being assumed \\
nirāyāsāt --- without difficulty  \\
unmanī ---  the trans-mental state/Samādhi state \\
kalā --- (pleasing like) the digit of moon  \\
svayam --- on its own \\
eva --- only  \\
utpadyate --- manifests
\end{multicols}

\subsection*{Purport}

What is the use of many \textit{āsanas} if Siddh\textit{Āsana} is achieved and also if Kevala kumbhaka is attained,? The Samādhi state which is pleasing like the digit of the moon manifests on its own. 

\newpage
\noindent \textbf{Verses 1.42}

\begin{shloka}
tathaikasminneva dṛḍhe baddhe siddhāsane sati|\\
bandhatrayamanāyāsāt svayamevopajāyate||
\end{shloka}

\subsection*{Word Split}

Tathā, ekasmin, eva ,dṛḍhe, baddhe, siddhāsane, sati, bandhatrayam, anāyāsāt,  svayam, eva, upajāyate

\subsection*{Paraphrased Word Meaning}

\begin{multicols}{2}
\itemsep=0pt
tathā --- also \\	
ekasmin --- in the one  \\
eva --- only \\
dṛḍhe --- firm \\ 		 
siddhāsane --- siddhāsana	 \\
baddhe sati --- on being attained  \\
anāyāsāt ---  without difficulty  \\
bandhatrayam --- three bandhas	 \\
svayam --- on their own  \\
eva --- certainly  \\
upajāyate --- manifest 
\end{multicols}

\subsection*{Purport}

Similarly, if the Siddh\textit{Āsana} is firmly attained, without difficulty the three Bandhas manifest on their own.   

\newpage
\noindent \textbf{Verses 1.43}

\begin{shloka}
nāsanaṃ siddhasadṛśaṃ na kumbhaḥ kevalopamaḥ|\\
na khecarīsamā mudrā na nādasadṛśo layaḥ||
\end{shloka}

\subsection*{Word Split}

nāsanaṃ, siddhasadṛśaṃ, na, kumbhaḥ, kevalopamaḥ, na, khecarīsamā, mudrā, na, nādasadṛśaḥ, layaḥ 

\subsection*{Paraphrased Word Meaning}

\begin{multicols}{2}
\itemsep=0pt
siddhasadṛśaṃ --- like Siddha \\
na --- no \\
āsanaṃ --- posture  \\
kevalopamaḥ --- similar to Kevala  \\
kumbhaḥ ---  hold  \\
na --- no \\
khecarīsamā --- like khecarī   \\
mudrā --- mudra \\
na ---  no	 \\		
nādasadṛśaḥ --- like nāda	 \\
layaḥ --- absorption  \\
na --- no
\end{multicols}

\subsection*{Purport}

There is no \textit{Āsana} like Siddhāsana. There is no Kumbhaka like Kevalakumbhaka. There is no mudrā like khecarimudrā. And there is no Laya practice comparable to \textit{Nādānusandhānana}.
\newpage

\section*{\textit{Āsana} -- 13: \textit{Padmāsana} - 1}

\noindent \textbf{Verses 1.44}

\begin{shloka}
vāmorūpari dakṣiṇaṃ ca caraṇaṃ saṃsthāpya vāmaṃ tathā\\
dakṣorūpari paścimena vidhinā dhṛtvā karābhyāṃ dṛḍham|\\
aṅguṣṭhau hṛdaye nidhāya cibukaṃ nāsāgramālokayet\\
etat vyādhivināśakāri yamināṃ \textit{Padmāsana}ṃ procyate||
\end{shloka}
\vspace{-10pt}

\subsection*{Word Split}
\vspace{-10pt}

Vāmorūpari, dakṣiṇam, ca, caraṇam, saṃsthāpya, vāmam, tathā, dakṣorūpari, paścimena, vidhinā, dhṛtvā, karābhyām, dṛḍham,aṅguṣṭhau, hṛdaye, nidhāya, cibukam, nāsāgram, ālokayet, etat, vyādhivināśakāri, yaminām, \textit{Padmāsana}m, procyate
\vspace{-10pt}

\subsection*{Paraphrased Word Meaning}
\vspace{-10pt}

\begin{multicols}{2}
\itemsep=0pt
dakṣiṇaṃ --- the right  \\
ca ---  and (filler)  \\
caraṇaṃ ---  foot  \\
vāmorūpari ---  on the left thigh  \\
saṃsthāpya ---  having placed  \\
vāmaṃ ---  the left (foot) \\
tathā ---  also \\
dakṣorūpari --- on the right thigh \\
paścimena ---  by the reverse (from behind)  \\
vidhinā ---  way \\
karābhyāṃ --- by hands \\
aṅguṣṭhau --- two big toes \\
dṛḍham --- firmly \\
dhṛtvā ---  having held  \\
cibukaṃ ---  the chin \\
hṛdaye ---  in the chest \\
nidhāya ---  having placed \\
nāsāgram --- the tip of the nose  \\
ālokayet ---  gaze \\
etat ---  this \\
yamināṃ ---  of the self-controlled practitioners \\
vyādhivināśakāri ---  destroyer of diseases \\
padmāsanaṃ ---  \textit{Padmāsana} \\
procyate --- stated
\end{multicols}

\subsection*{Purport}


The right foot (sole facing upwards) has to be placed on the left thigh and on the right thigh the left foot has to be placed. The hands crossing from behind should catch hold of the two big toes. Placing the chin on the chest and one should gaze the tip of the nose. This is \textit{Padmāsana}\footnote{This is known as Baddha-padmāsana} that destroys the diseases of the self-controlled practitioners.

\subsection*{Inputs from Jyotsnā Commentary}


\begin{enumerate}
\item The right hand has to catch hold of the left big toe from behind and the left hand has to go from the behind and catch hold of the right big toe. 
\item The chin has to be placed on the chest. The measure of its placement is 4 aṅgulas (finger breadths from the starting place of the chest region). This is rahasya – a subtle input.  
\end{enumerate}

\section*{Padm\textit{Āsana} - 2}

\centerline{(The descrption of this asana extends from verse 1.45 to 1.47)}

\noindent \textbf{Verses 1.45}

\begin{shloka}
uttānau caraṇau kṛtvā ūrusaṃsthau prayatnataḥ|\\
ūrumadhye tathottānau pāṇī kṛtvā tato dṛśau||
\end{shloka}

\subsection*{Word Split}


Uttānau, caraṇau, kṛtvā, ūrusaṃsthau, prayatnataḥ, ūrumadhye, tathā, uttānau, pāṇī, kṛtvā, tataḥ, dṛśau

\subsection*{Paraphrased Word Meaning}


\begin{multicols}{2}
caraṇau --- feet  \\
uttānau --- upward facing  \\
prayatnataḥ --- with effort  \\
ūrusaṃsthau ---  being placed on the thighs  \\
kṛtvā --- having made  \\
ūrumadhye --- in the middle of the thigh  \\
tathā ---  also  \\
uttānau ---  upward facing  \\
pāṇī ---  hands  \\
kṛtvā ---  having kept  \\
tataḥ ---  also  \\
dṛśau --- the eyes
\end{multicols}

\subsection*{Purport}

Both the feet with upward facing (sole) has to be placed on (opposite) thighs. The hands with upwards facing palms (placed one above the other) have to be placed in between the thighs. After that, the gaze… (Description continued in the next verse)

\subsection*{Inputs from Jyotsnā Commentary}

\begin{enumerate}
\item This is the \textit{Padmāsana} that is acceptable to Matsyendranātha.
\item The back side of the palms (placed one above the other) should touch both the thighs. The left palm facing upwards touch both the heels of the feet should be placed. Above that the right palm facing upwards has to be placed.
\end{enumerate}

\newpage
\noindent \textbf{Verses 1.46}

\begin{shloka}
nāsāgre vinyased rājadantamūle tu jihvayā\\
uttambhya cibukaṃ vakṣasyutthāpya pavanaṃ śanaiḥ||
\end{shloka}

\subsection*{Word Split}

Nāsāgre, vinyased, rājadantamūle, tu, jihvayā, uttambhya, cibukam, vakṣasi, utthāpya, pavanaṃ, śanaiḥ

\subsection*{Paraphrased Word Meaning}

\begin{multicols}{2}
nāsāgre ---  on the tip of the nose \\
vinyased --- (one) should place  \\
rājadantamūle ---  on the root of the incisors   \\
tu --- (filler)  \\
jihvayā --- with the tongue \\
uttambhya ---  placing/propping \\
vakṣasi --- in the chest  \\
cibukaṃ ---  the chin  \\
śanaiḥ --- slowly  \\
pavanaṃ ---   air  \\
utthāpya --- arousing
\end{multicols}

\subsection*{Purport}

(gaze) should be fixed on the tip of the nose. The tongue should press against the incisors. The chin has to be placed on the chest and the air has to be raised. (Description continued and completed in next verse)

\subsection*{Inputs from Jyotsnā Commentary}

\begin{enumerate}
\item Raising the air here refers to the practice of Mūlabandha.   
\end{enumerate}

\newpage
\noindent \textbf{Verses 1.47}

\begin{shloka}
iḍāṃ \textit{Padmāsana}ṃ proktaṃ sarvavyādhivināśanam|\\
durlabhaṃ yena kenāpi dhīmatā labhyate bhuvi||
\end{shloka}

\subsection*{word split}

iḍām,  \textit{Padmāsana}m, proktam, sarvavyādhivināśanam,  durlabham, yena, kena, api, dhīmatā, labhyate, bhuvi

\subsection*{Paraphrased Word Meaning}

\begin{multicols}{2}
iḍāṃ ---  this  \\
sarvavyādhivināśanam --- destroyer of all diseases  \\
padmāsanaṃ ---  \textit{Padmāsana} \\
durlabhaṃ ---  difficult to obtain \\ 
proktaṃ ---  stated  \\
bhuvi --- on the earth  \\
yena kenāpi ---  by some  \\
dhīmatā ---  intelligent  \\
labhyate --- attained
\end{multicols}

\subsection*{Purport}

This is \textit{Padmāsana} that destroys all diseases. This is difficult to practice and it is attained only by an intelligent person on the earth.

\section*{Padmāsanam - 3}

\noindent \textbf{Verses 1.48}

\begin{shloka}
kṛtvā saṃpuṭitau karau  dṛḍhataraṃ baddhvā tu \textit{Padmāsana}m\\
gāḍhaṃ vakṣasi sannidhāya cibukaṃ dhyāyaṃśca taccetasi|\\
vāraṃ vāramapānamūrdhvamanilaṃ protsārayan pūritam\\
nyañcan prāṇamupaiti bodhamatulaṃ śaktiprabhāvānnaraḥ|| 48 ||
\end{shloka}

\subsection*{Word Split}

kṛtvā, saṃpuṭitau, karau,  dṛḍhataram, baddhvā, tu, \textit{Padmāsana}m, gāḍham, vakṣasi, sannidhāya, cibukam, dhyāyan, ca, tat, cetasi vāram, vāram,  apānam, ūrdhvam, anilam, protsārayan, pūritam, nyañcan, prāṇam, upaiti, bodham, atulam,  śaktiprabhāvāt, naraḥ

\subsection*{Paraphrased Word Meaning}

\begin{multicols}{2}
karau  ---  hands \\
saṃpuṭitau ---  on top of one another \\
kṛtvā ---  making \\
dṛḍhataraṃ ---  firmly \\
tu ---  (filler)  \\
padmāsanam ---  \textit{Padmāsana} \\
baddhvā ---  assuming  \\
cibukaṃ ---  the chin  \\
vakṣasi ---  in the chest \\ 
gāḍhaṃ ---  firmly  \\
sannidhāya ---  having fixed  \\
tat --- that  \\
ca ---   and  \\
cetasi --- in the mind   \\
dhyāyan ---  contemplating \\  
vāraṃ ---  again  \\
vāram ---  again  \\
apānam ---  the apāna \\
anilaṃ ---  the air  \\
ūrdhvam ---  above  \\
protsārayan ---  pulling  \\
pūritam ---  filled  \\
prāṇam ---  prāṇa \\
nyañcan ---  pushing down  \\
naraḥ --- human being  \\
śaktiprabhāvāt ---  by the virtue of the great power  \\
atulaṃ ---  matchless  \\
bodham ---  enlightenment  \\
upaiti ---  attains
\end{multicols}

\subsection*{Purport}

Being seated in the \textit{Padmāsana} the hands (palms) are to be placed one above the other firmly. The chin has to be placed firmly on the chest. Meditating upon that (Brahman/ iṣṭadevatā (personal diety)) the apāna has to be drawn up and Prāṇa has to be pushed down repeatedly. Such a practitioner will attain a matchless enlightenment due to efficacy of the śakti (power).

\subsection*{Inputs from Jyotsnā Commentary}


\begin{enumerate}
\item This type of \textit{Padmāsana} is agreeable to Mahāyogins. 
\item Meditating upon that may refer to Brahman or one’s own iṣṭadevatā (Personal diety)
\item Pulling up apāna refers to mūlabandha and by pushing down the prāṇa (by Jālandhra bandha) attaining the conjoining of Prāṇa and Apāna (by Uḍḍiyāna Bandha) the prāṇa is taken into Suṣumnā. In the process of conjoining Prāṇa and Apāna Kuṇḍalinī is awakenend. When Kuṇḍalinī is awakened Prāṇa enter suṣumnā. When Prāṇa enters Suṣumnā the mind attains stability – citta-sthairya. By doing saṃyama (dhāraṇā dhyāna samādhi) on Citta-sthairya (stability of mind) realisation of the Ātman is attained.
\end{enumerate}

\subsection*{Benefits of \textit{Padmāsana}}


\noindent \textbf{Verses 1.49}

\begin{shloka}
padmāsane sthito yogī nāḍīdvāreṇa pūritam|\\
mārutaṃ dhārayed yastu sa mukto nātra saṃśayaḥ||
\end{shloka}

\subsection*{Word Split}


Padmāsane, sthitaḥ, yogi, nāḍīdvāreṇa, pūritam, mārutam, dhārayed, yaḥ, tu, saḥ, muktaḥ, na, atra saṃśayaḥ

\subsection*{Paraphrased Word Meaning}


\begin{multicols}{2}
yaḥ tu --- the one   \\
yogī --- practitioner of Yoga \\
padmāsane ---  in \textit{Padmāsana} \\
sthitaḥ --- the one who stays  \\
nāḍīdvāreṇa ---  through the nāḍī \\
pūritam --- filled  \\
mārutaṃ ---  air  \\
dhārayed ---  holds  \\
saḥ ---  he  \\
muktaḥ --- is liberated \\
atra --- here  \\
saṃśayaḥ --- doubt\\
na ---  no
\end{multicols}

\subsection*{Purport}

The Yogi who being seated in \textit{Padmāsana} holds the breath inhaled through the Nāḍīs (practices \textit{Prāṇāyāma}) attains liberation. There is no doubt about this.
\newpage

\section*{\textit{Āsana} -- 14: Siṃhāsana}


\noindent \textbf{Verses 1.50}

\begin{shloka}
gulphau ca vṛṣaṇasyādhaḥ sīvanyāḥ pārśvayoḥ kṣipet|\\
dakṣiṇe savyagulphaṃ tu dakṣagulphaṃ tu savyake||
\end{shloka}

\subsection*{Word Split}

Gulphau, ca, vṛṣaṇasya, adhaḥ, sīvanyāḥ, pārśvayoḥ, kṣipet, dakṣiṇe, savyagulpham, tu, dakṣagulpham, tu, savyake

\begin{multicols}{2}
Paraphrased Word Content \\
vṛṣaṇasya -- of the scrotum  \\ 
adhaḥ ---  below  \\
sīvanyāḥ ---  of the perineum  \\
pārśvayoḥ ---  on the sides  \\
ca ---  (filler) \\
gulphau ---  the ankles  \\
kṣipet ---   (one) should place  \\
dakṣiṇe --- on the right  \\
savyagulphaṃ --- the left ankle  \\
tu ---  (filler) \\
dakṣagulphaṃ ---  the right ankle  \\
tu ---  (filler) \\
savyake --- on the left 
\end{multicols}

\subsection*{Purport}


Both the ankles of the feet are to be placed below the scrotum on the sides of the perineum. The left ankle has to be placed on the right side and the right ankle on the left side. (Descrption of the \textit{Āsana} is continued in the next verse)

\newpage
\noindent \textbf{Verses 1.51}

\begin{center}
hastau tu jānvoḥ saṃsthāpya svāṅgulīḥ samprasārya ca |
vyāttavaktro nirīkṣeta nāsāgraṃ susamāhitaḥ||
\end{center}

\subsection*{Word Split}

Hastau, tu, jānvoḥ, saṃsthāpya, svāṅgulīḥ, samprasārya, ca, vyāttavaktraḥ, nirīkṣeta, nāsāgram, susamāhitaḥ

\subsection*{Paraphrased Word Meaning}

\begin{multicols}{2}
hastau ---  hands \\
tu ---  (filler) \\
jānvoḥ ---  on the knees  \\
saṃsthāpya ---  having placed  \\
svāṅgulīḥ ---  one’s fingers  \\
samprasārya ---  having spread \\
ca ---  and \\
susamāhitaḥ --- with a single pointed focus  \\
vyāttavaktraḥ ---  with wide opened mouth  \\
nāsāgraṃ ---  the tip of the nose  \\
nirīkṣeta ---  (one) should gaze
\end{multicols}

\subsection*{Purport}

The hands are to be placed on the knees and the fingers are to be spread. With a wide opened mouth one should gaze at the tip of the nose. 

\subsection*{Inputs from Jyotsnā Commentary}

\begin{enumerate}
\item The tongue should also be extended outside when the mouth is opened.
\item Susamāhita means one should have a single pointed focus.
\end{enumerate}
\newpage

\subsection*{Benefits of siṃhāsana}

\noindent \textbf{Verses 1.52}

\begin{shloka}
siṃhāsanaṃ bhavedetat pūjitaṃ yogipuṅgavaiḥ|\\
bandhatritayasandhānaṃ kurute cāsanottamam||
\end{shloka}

\subsection*{Word Split}

siṃhāsanam,  bhaved, etat, pūjitam, yogipuṅgavaiḥ, bandhatritayasandhānam, kurute, ca āsanottamam

\subsection*{Paraphrased Word Meaning}

\begin{multicols}{2}
\textit{etat} ---  this \\
\textit{yogipuṅgavaiḥ} --- by best among the yogins   \\
\textit{pūjitaṃ} ---  worshipped  \\
\textit{āsanottamam} --- best among the \textit{āsanas} \\
\textit{siṃhāsanaṃ} ---  siṃh\textit{Āsana} \\
\textit{bhaved} ---   shall be   \\
\textit{bandhatritayasandhānaṃ} --- facilitation/brining together of the three bandhas   \\
\textit{kurute} ---  is done  \\
\textit{ca} --- and
\end{multicols}

\subsection*{Purport}

This best among the \textit{āsanas} is worshipped by great Yogins. This facilitates the practice of /brings together the three Bandhas.
\newpage

\section*{\textit{Āsana} -- 15: Bhadr\textit{Āsana} and Its Benefits}

\noindent \textbf{Verses 1.53}

\begin{shloka}
\textit{gulphau ca vṛṣaṇāsyādhaḥ sīvanyāḥ pārśvayoḥ kṣipet|}\\
\textit{savyagulphaṃ tathā savye dakṣagulphaṃ tu dakṣiṇe||}
\end{shloka}

\subsection*{Word Split}

Gulphau, ca, vṛṣaṇāsya, adhaḥ, sīvanyāḥ, pārśvayoḥ, kṣipet,savyagulphaṃ, tathā, savye, dakṣagulpham, tu, dakṣiṇe

\subsection*{Paraphrased Word Meaning}

\begin{multicols}{2}
\textit{vṛṣaṇāsya} ---  of the scrotum \\
\textit{adhaḥ} ---  below  \\
\textit{sīvanyāḥ} --- of the perineum   \\
\textit{pārśvayoḥ} ---  on both sides  \\
\textit{gulphau} ---  the two ankles  \\
\textit{ca} ---   (filler) \\
\textit{kṣipet} --- place  \\
\textit{savyagulphaṃ} --- the left ankle  \\
\textit{tathā} ---  similarly \\
\textit{savye} ---  on the left  \\
\textit{dakṣagulphaṃ} ---  the right ankle  \\
\textit{tu} ---  (filler) \\
\textit{dakṣiṇe} --- on the right 
\end{multicols}

\subsection*{Purport}

Below the scrotum and on the sides of the perineum the two ankles have to be placed. The right ankle has to be placed on the right side and the left ankle on the left. (Description of the \textit{Āsana} is continued and completed in the next verse)

\newpage
\noindent \textbf{Verses 1.54}

\begin{shloka}
pārśvapādau ca pāṇibhyāṃ dṛḍhaṃ baddhvā suniścalam|\\
bhadrāsanaṃ bhavedetat sarvavyādhivināśanam|\\
gorakṣāsanamityāhuriḍāṃ vai siddhayoginaḥ||
\end{shloka}

\subsection*{Word Split}

Pārśvapādau, ca, pāṇibhyām, dṛḍham, baddhvā, suniścalam, bhadrā\-sanaṃ, bhaved, etat sarvavyādhivināśanam, \textit{Gorakṣāsana}m, iti, āhuḥ, iḍām, vai, siddhayoginaḥ
\vspace{-10pt}

\subsection*{Paraphrased Word Meaning}
\vspace{-10pt}

\begin{multicols}{2}
\textit{pārśvapādau} ---  the sides of the feet  \\
\textit{ca} ---  and (filler) \\
\textit{pāṇibhyāṃ} ---  by the hands  \\
\textit{suniścalam} --- motionless  \\
\textit{dṛḍhaṃ} ---  firnly \\
\textit{baddhvā} --- having held  \\
\textit{etat} ---  this  \\
\textit{sarvavyādhivināśanam} --- destroyer of all diseases  \\
\textit{bhadrāsanaṃ} --- bhadr\textit{Āsana} \\
\textit{bhaved} --- should be \\
\textit{siddhayoginaḥ} ---  accomplished Yogins \\
\textit{iḍāṃ} ---  this  \\
\textit{vai} ---  indeed (filler) \\
\textit{gorakṣāsanam} ---  \textit{Āsana} of gorakṣa \\
\textit{iti} --- thus  \\
\textit{āhuḥ}  --- (some) state
\end{multicols}
\vspace{-10pt}

\subsection*{Purport}
\vspace{-10pt}

The sides of the feet (of which the soles are facing each other) are to be held firmly by both the hands. This is Bhadr\textit{Āsana} that destroys all diseases. This is also called as \textit{Gorakṣāsana} by Siddha Yogins.
\vspace{-10pt}

\subsection*{Inputs from Jyotsnā Commentary}
\vspace{-10pt}

\begin{enumerate}
\itemsep=0pt
\item The sides of feet are to be held by interlocking of the fingers and the Tala (the portion of the arm around elbow) should be touching the abdomen. 
\item This \textit{āsanas} might have been called as \textit{Gorakṣāsana} as it might have been practiced (abhyastatvat) by Gorakṣa. 
\end{enumerate}
