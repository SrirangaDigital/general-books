\chapter{\textit{Āsana-s}}

\subsection*{Summary}

\section*{Introduction about \textit{Āsana-s}}

As \textit{Āsana} is the first limb of \textit{Haṭha}, it is stated first. \textit{Āsana} bestows firmness, health and lightness of limbs.17 

I (\textit{Svātmārāma}) state a few \textit{Āsana-s} accepted by \textit{Muni-s} like \textit{Vasiṣṭha} and others and by \textit{Yogins} like \textit{Matsyendra} and others.  1.18

\heading{\textit{Āsana} -- 1:  \textit{Svastikāsana}}

\vspace{-5pt}

Keeping the two soles of the feet in between the knee and thigh, being seated well with an upright body, is called as \textit{Svastikāsana}.  1.19

\heading{\textit{Āsana} -- 2:   \textit{Gomukhāsana}}

\vspace{-5pt}

The right ankle has to be placed besides the back on the left side. In the same manner the left ankle has to be placed (on the right side beside the back). This is \textit{Gomukhāsana} that resembles the face of a cow. 1.20

\heading{\textit{Āsana} -- 3:   \textit{Vīrāsana}}

\vspace{-5pt}

One (right) foot has to be placed on one thigh firmly and on the other foot the (left) thigh has to be placed. This is stated as \textit{Virāsana}. 1.21

\heading{\textit{Āsana} -- 4:  \textit{Kūrmāsana}}

\vspace{-5pt}

Pressing the anus with both the ankles and being seated in a head down manner is \textit{Kuarmāsana}. This is known by those who are well versed in \textit{Yoga}. 1.22

\heading{\textit{Āsana} -- 5:  \textit{Kukkuṭāsana}}

\vspace{-5pt}

Onen should assume \textit{Padmāsana}. After that, inserting the hands in between the knees and thighs and placing them on the ground one should raise above (from the ground). This is called \textit{Kukkutāsana}. (\textit{Kukkuta-Cock}) 1.23

\heading{\textit{Āsana} -- 6: \textit{Uttānakūrmakāsana}}

From \textit{Kukkutāsana} connecting the neck with the hands (that are inserted in between the thighs and knees) and lying on back on the ground is called as \textit{Uttānakūrmaka Āsana} (the upturned turtle posture) 1.24

\heading{\textit{Āsana} -- 7: \textit{ Dhanurāsana}}

Both the big toes are to be held by the hands. Then, till ears one should stretch (one foot)  like drawing a bow. This is called as \textit{Dhanurāsana}\footnote{This posture is known as \textit{Akarnadhanurasana}.} (bow posture). 1.25

\heading{\textit{Āsana} -- 8: \textit{Matsyendrāsana}}

Hold the right foot that is placed on the base of the left thigh and (hold) the left foot wound outside the right knee with twisted limbs. This is the \textit{Āsana} stated by \textit{Sri Matsyanātha}. 1.26

\heading{Benefits of \textit{Matsyendrāsana}}

By practice, \textit{Matsyendrāsana}, which is the weapon that cuts asunder host of terrible diseases, grants glow of the digestive fire. It also awakens the \textit{Kuṇḍalinī}, and grants firmness is \textit{Candra} (the vital energy) for the practitioners. 1.27
\medskip

\heading{\textit{Āsana} -- 9: \textit{ Paścimatānāsana}}

Stretching both legs straight on the ground like stick one should hold the tips of the feet with the arms and place the forehead on the knees and stay in that position. This is stated as \textit{Paścimatanāsana}.  1.28


\heading{Benefits of \textit{Paścimatānāsana}}

\textit{Paścimatānāsana} is the foremost among the \textit{Āsana-s}. It makes the air (\textit{Prāṇa}) traverse in the \textit{Suṣumnā}. It increases the glow of the digestive fire. It leads to leanness in the abdomen and freedom from disease.  1.29
\medskip

\heading{\textit{Āsana} -- 10: \textit{Mayūrāsana}}

By placing the hands on the ground and supporting the body, and also keeping the elbows besides the navel one should raise the body from the ground like a stick (horizontal to the ground). This posture is called as \textit{Mayurāsana}.  1.30

\newpage
\heading{Benefits of \textit{Mayūrāsana}}

\textit{Mayūrāsana} quickly removes all diseases like enlargement of the spleen and odema. It also helps overcome imbalances in Doṣa-s (\textit{Vāta, Pitta} and \textit{Kapha}). It reduces to ashes (digests) food that has been partaken indiscriminately without any residue. It kindles the digestive fire and also digests vicious poison even. 1.31


\heading{\textit{Āsana} -- 11: \textit{ Śavāsana}}

Lying on the back on the floor like a corpse is called \textit{Śavāsana}. It removes tiredness and grants relaxation to the mind. 1.32

\heading{Prelude to four important postures}

Lord \textit{Śiva} has stated 84 \textit{Āsana-s}. From among them I will state four essential \textit{Āsana-s}.

\heading{The Four Important \textit{Āsana-s}}

\textit{Siddhāsana}, \textit{Padmāsana}, \textit{Simhāsana} and \textit{Bhadrāsana} are the four best \textit{Āsana-s}. Even among them, one should always stay in the comfortable \textit{Siddhāsana}.  1.34

\heading{\textit{Āsana} -- 12: \textit{Siddhāsana} - 1}

The left heel should press the perineum firmly. Then one should place the right feet (heel) above the genital.  The chin has to be placed firmly in the chest and one should stay still with controlled senses and gaze the place in between the eyebrows. This is \textit{Siddhāsana} that breaks open the doors of liberation.  1.35

\heading{\textit{Siddhāsana} - 2}

The left ankle has to be placed above the genital. The right ankle has to be placed above (the sole of the left foot). This should be \textit{Siddhāsana}. 1.36

\heading{Other Nomenclatures of \textit{Siddhāsana}}

This \textit{Āsana} is stated variously as \textit{Siddhāsana}, \textit{Vajrāsana}, \textit{Muktāsana} and \textit{Guptāsana}. 1.37

\heading{Benefits of \textit{Siddhāsana}}

Like moderate diet  among \textit{Yamas} and \textit{Ahimsā} in \textit{Niyama-s}, one \textit{Siddhāsana} is known as important among all \textit{Āsana-s}.  1.38

Among 84 \textit{Āsana-s} practice only \textit{Siddhāsana}. It cleanses all the seventy two thousand \textit{Nāḍī-s}.  1.39

With the constant practice of \textit{Siddhāsana} for twelve years and moderate diet, the one who meditates upon the \textit{Ātman},  attains success (attains the goal of \textit{Yoga}) 1.40

What is the use of many \textit{Āsana-s} if \textit{Siddhāsana} is achieved and also if \textit{Kevala Kumbhaka} is attained,? The \textit{Samādhi} state which is pleasing like the digit of the moon manifests on its own. 1.41

Similarly, if the \textit{Siddhāsana} is firmly attained, without difficulty the three \textit{Bandha-s} manifest on their own.   1.42

There is no \textit{Āsana} like \textit{Siddhāsana}. There is no \textit{Kumbhaka} like \textit{Kevalakumbhaka}. There is no mudrā like \textit{Khecarimudrā.} And there is no \textit{Laya} practice comparable to \textit{Nādānusandhānana}. 1.43
\medskip

\heading{\textit{Āsana} -- 13:  \textit{Padmāsana} - 1}

The right foot (sole facing upwards) has to be placed on the left thigh and on the right thigh the left foot has to be placed. The hands crossing from behind should catch hold of the two big toes. Placing the chin on the chest and one should gaze the tip of the nose. This is \textit{Padmāsana}\footnote{This is known as \textit{Baddha-padmāsana}} that destroys the diseases of the self-controlled practitioners. 1.44

\heading{\textit{Padmāsana} - 2}

Both the feet with upward facing (sole) has to be placed on (opposite) thighs. The hands with upwards facing palms (placed one above the other) have to be placed in between the thighs. After that, the gaze… (Description continued in the next verse) 1.45

…(gaze) should be fixed on the tip of the nose. The tongue should press against the incisors. The chin has to be placed on the chest and the air has to be raised. (Description continued and completed in next verse) 1.46

This is \textit{Padmāsana} that destroys all diseases. This is difficult to practice and it is attained only by an intelligent person on the earth.  1.47

\heading{\textit{Padmāsana} - 3}

Being seated in the \textit{Padmāsana} the hands (palms) are to be placed one above the other firmly. The chin has to be placed firmly on the chest. Meditating upon that (\textit{Brahman/Iṣṭadevatā} (personal diety)) the \textit{Apāna} has to be drawn up and \textit{Prāṇa} has to be pushed down repeatedly. Such a practitioner will attain a matchless enlightenment due to efficacy of the \textit{Śakti} (power). 1.48

\heading{Benefits of \textit{Padmāsana}}

The \textit{Yogi} who being seated in \textit{Padmāsana} holds the breath inhaled through the \textit{Nāḍī-s} (practices \textit{Prāṇāyāma}) attains liberation. There is no doubt about this. 1.49
\medskip

\heading{\textit{Āsana} -- 14:  \textit{Siṃhāsana}}

Both the ankles of the feet are to be placed below the scrotum on the sides of the perineum. The left ankle has to be placed on the right side and the right ankle on the left side. (Descrption of the \textit{Āsana} is continued in the next verse) 1.50

The hands are to be placed on the knees and the fingers are to be spread. With a wide opened mouth one should gaze at the tip of the nose. 1.51

\heading{Benefits of \textit{Siṃhāsana}}

This best among the \textit{Āsana-s} is worshipped by great \textit{Yogins}. This facilitates the practice of /brings together the three \textit{Bandha-s}.  1.52
\medskip

\heading{\textit{Āsana} -- 15: \textit{Bhadrāsana} \& its Benefits}

Below the scrotum and on the sides of the perineum the two ankles have to be placed. The right ankle has to be placed on the right side and the left ankle on the left. (Description of the \textit{Āsana} is continued and completed in the next verse) 1.53

The sides of the feet (of which the soles are facing each other) are to be held firmly by both the hands. This is \textit{Bhadrāsana} that destroys all diseases. This is also called as \textit{Gorakṣāsana} by \textit{Siddha Yogins}. 1.54
\newpage

\thispagestyle{empty}
~
\vfill
\begin{center}
\textbf{\Huge Textual Immersion}
\end{center}
\vfill
\eject

\section*{Introduction about \textit{Āsana-s}}


\noindent \textbf{Verses 1.17}

\begin{shloka}
\textit{\textbf{hāṭhasya prathamāṅgatvādāsanaṃ pūrvamucyate|}\\
\textbf{kuryāt tadāsanaṃ sthairyamārogyaṃ  cāṅgalāghavam||}}
\end{shloka}

\subsection*{Word Split}

\textit{Hāṭhasya, Prathamāṅgatvād, Āsanam, Pūrvam, Ucyate, Kuryāt, Tad, Āsanam, Sthairyamm Ārogyaṃ,  Ca, Aṅgalāghavam}

\subsection*{Paraphrased word meaning}

\begin{longtable}{>{\noindent\raggedright}p{5cm}>{\noindent\raggedright}p{5cm}}
\textit{Hāṭhasya} --- of \textit{Hāṭha} & \textit{Āsanaṃ} --- posture\tabularnewline
\textit{Prathamāṅgatvād} --- as it is the first limb & \textit{Sthairyam} --- firmness\tabularnewline
\textit{Asanaṃ} --- posture & \textit{Ārogyaṃ} --- health\tabularnewline
\textit{Pūrvam} --- earlier & \textit{Aṅgalāghavam} --- lightness of limbs\tabularnewline
\textit{Ucyate} --- (is) stated & \textit{Ca} --- and\tabularnewline
\textit{Tad} --- that & \textit{Kuryāt} --- it shall do
\end{longtable}


\subsection*{Purport}


As \textit{Āsana} is the first limb of \textit{Haṭha}, it is stated first. \textit{Āsana} bestows firmness, health and lightness of limbs

\subsection*{Inputs from \textit{Jyotsnā} Commentary}

\begin{enumerate}
\item In verse 1.56 --- \textit{Āsana-s}, \textit{Prāṇāyāma}, \textit{Mudrā-s} and \textit{Nādānusandhāna} are presented as the limbs of \textit{Haṭha}. There \textit{Āsana} is presented as the first limb. 
\item \textit{Sthairya} --- refers both stability at the level of body and also at the level of mind. This refers to the capabily of the \textit{Āsana-s} to reduce \textit{Rajas}.
\item By \textit{Aṅgalāghavam} --- the efficacy of \textit{Āsana} in overcoming heaviness of body is indicated. Heaviness is an attribute of \textit{Tamas}. Hence the capability of \textit{Āsana-s} to reduce \textit{Tamas} is also indicated. 
\end{enumerate}

\noindent \textbf{Verses 1.18}

\begin{shloka}
\textit{\textbf{vasiṣṭhādyaiśca munibhiḥ matsyendrādyaiśca yogibhiḥ|}\\
\textbf{aṅgīkṛtānyāsanāni kathyante kānicinmayā||}}
\end{shloka}


\subsection*{Word Split}

\textit{Vasiṣṭhādyaiḥ, Ca Munibhiḥ, Matsyendrādyaiḥ, Ca Yogibhiḥ, Aṅgīkṛtāni, Āsanāni, Kathyante Kānicit, Mayā}

\subsection*{Paraphrased meaning}

\begin{longtable}{>{\noindent\raggedright}p{5cm}>{\noindent\raggedright}p{5cm}}
\textit{Vasiṣṭhādyaiḥ} --- by \textit{Vasiṣṭha} and others & \textit{Yogibhiḥ} - by \textit{Yogins}\tabularnewline
\textit{Ca} - and & \textit{Aṅgīkṛtāni}  - accepted\tabularnewline
\textit{Munibhiḥ}  --- by \textit{Muni-s} & \textit{Kānicit} --- a few\tabularnewline
\textit{Matsyendrādyaiḥ} --- \textit{Matsyendra} and others & \textit{Āsanāni} --- \textit{Āsana-s}\tabularnewline
\textit{Ca}  --- and & \textit{Mayā} --- by me\tabularnewline
\textit{Kathyante} --- are stated & 
\end{longtable}

\subsection*{Purport}

I (\textit{Svātmārāma}) state a few \textit{Āsana-s} accepted by \textit{Muni-s} like \textit{Vasiṣṭha} and others and by \textit{Yogins} like \textit{Matsyendra} and others.  

\subsection*{Inputs from \textit{Jyotsnā} Commentary}

\begin{enumerate}
\item Difference between \textit{Muni-s} and \textit{Yogins} – Though for both\break \textit{Muni-s} and \textit{Yogins} \textit{Manana} (reflecting/contemplating ) and practice of \textit{Haṭha} are common – still, for  \textit{Muni-s} - \textit{Manana} is important and for \textit{Yogins} practice of \textit{Haṭha} is important. These \textit{Āsana-s} that are stated in this text are acceptable to these two kinds of practitioners of \textit{Yoga}.
\end{enumerate}

\section*{\textit{Āsana} -- 1: \textit{Svastikāsana}}

\noindent \textbf{Verses 1.19}

\begin{shloka}
\textit{\textbf{jānūrvorantare samyak kṛtvā pādatale ubhe|}\\
\textbf{ṛjukāyaḥ samāsīnaḥ svastikaṃ tat pracakṣate|| 1.19||}}
\end{shloka}

\subsection*{Word Split}

\textit{Jānūrvoḥ, Antare, Samyak, Kṛtvā, Pādatale, Ubhe, Ṛjukāyaḥ, Samāsīnaḥ, Svastikaṃ, Tat, Pracakṣate}

\vspace{-5pt}

\subsection*{Paraphrased word meaning}
\vspace{-10pt}

\begin{longtable}{>{\noindent\raggedright}p{5cm}>{\noindent\raggedright}p{5cm}}
\textit{Jānūrvoḥ} --- of the knee and the thigh & \textit{Ṛjukāyaḥ} - upright body\tabularnewline
\textit{Antare} --- in  between & \textit{Samāsīnaḥ} --- seated well\tabularnewline
\textit{Ubhe} --- the two & \textit{Tat} --- that\tabularnewline
\textit{Pādatale} --- soles of the feet & \textit{Svastikaṃ} --- \textit{Svastika}\tabularnewline
\textit{Samyak} --- well & \textit{Pracakṣate} --- stated\tabularnewline
\textit{Kṛtvā} --- doing (keeping) & 
\end{longtable}
\vspace{-5pt}

\subsection*{Purport}

\vspace{-10pt}

Keeping the two soles of the feet in between the knee and thigh, being seated well with an upright body, is called as \textit{Svastikāsana}.

\subsection*{Inputs from \textit{Jyotsnā} Commentary}

\begin{enumerate}
\item Though it has been stated that the soles of the feet are to be placed inbetween the knees and the thighs – it is to be understood that – the soles of the feet are to be placed in between the thighs and the shanks (the region of the leg from knee to the ankle/ the region of calf muscles). Or rather the reading itself can be taken as \textit{Jaṅghorvoḥ} (the shanks and the thighs) rather than \textit{Jānūrvoḥ} (of the knees and the thighs).
\end{enumerate}

\section*{\textit{Āsana} -- 2: \textit{Gomukhāsana}}

\noindent \textbf{Verses 1.20}

\begin{shloka}
\textit{\textbf{savye dakṣiṇagulphaṃ tu pṛṣṭhapārśve niyojayet|}\\
\textbf{dakṣiṇe'pi tathā savyaṃ gomukhaṃ gomukhākṛti||}}
\end{shloka}


\subsection*{Word Split}

\textit{Savye, Dakṣiṇagulphaṃ, Tu, Pṛṣṭhapārśve, Niyojayet, Dakṣiṇe, Api, Tathā, Savyaṃ, Gomukhaṃ Gomukhākṛti}

\subsection*{Paraphrased word meaning}

\begin{multicols}{2}
\textit{Dakṣiṇagulphaṃ} --- the right ankle\\ 
\textit{Savye} --- on the left\\ 
\textit{Pṛṣṭhapārśve} --- besides the back\\ 
\textit{Tu} --- indeed\\ 
\textit{Niyojayet} --- place/join\\ 
\textit{Dakṣiṇe} --- on the right (behind the back)\\ 
\textit{Api} --- also\\
\textit{Tathā} --- in the same manner\\
\textit{Savyaṃ} --- the left (ankle)\\
\textit{Gomukhaṃ} --- (this is) \textit{Gomukha} (\textit{Āsana})\\
\textit{Gomukhākṛti} --- (which is in the) form of a face of a cow\\
\end{multicols}

\subsection*{Purport}

The right ankle has to be placed besides the back on the left side. In the same manner the left ankle has to be placed (on the right side beside the back). This is \textit{Gomukhāsana} that resembles the face of a cow.

\subsection*{Inputs from \textit{Jyotsnā} Commentary}

\begin{enumerate}
\item Though it has been stated that the right ankle has to be placed beside the back, going by \textit{Sampradāya} (tradition), the right ankle has to be placed under the left hip.
\end{enumerate}

\section*{\textit{Āsana} -- 3: \textit{Vīrāsana}}

\noindent \textbf{Verses 1.21}

\begin{shloka}
\textit{\textbf{ekaṃ pādaṃ tathaikasmin vinyasedūruṇi sthiram|}\\
\textbf{itarasmiṃstathā coruṃ vīrāsanamitīritam||}}
\end{shloka}

\subsection*{Word Split}

\textit{Ekaṃ, Pādaṃ,  Tathā, Ekasmin, Vinyased, Ūruṇi, Sthiram,Itarasmin, Tathā, Ca, Ūruṃ, Vīrāsanam, Iti, Īritam}

\subsection*{Paraphrased Word Meaning}

\begin{multicols}{2}
\textit{Ekaṃ} --- one (the right)\\
\textit{Pādaṃ} --- foot\\ 
\textit{Tatha} --- (filler)\\  
\textit{Ekasmin} --- on one\\ 
\textit{Ūruṇi} – on the thigh\\ 
\textit{Sthiram} - firmly\\  
\textit{Vinyased} – one should place\\
\textit{Itarasmin} – on the other (foot)\\
\textit{Tathā}  - similarly (place)\\
\textit{Ca} --- and\\
\textit{Ūruṃ} --- thigh\\
\textit{Vīrāsanam} --- \textit{Vīrāsana}\\
\textit{Iti} --- thus\\
\textit{Īritam} --- stated
\end{multicols}

\subsection*{Purport}

One (right) foot has to be placed on one thigh firmly and on the other foot the (left) thigh has to be placed. This is stated as \textit{Virāsana}.

\subsection*{Inputs from \textit{Jyotsnā} Commentary}

\begin{enumerate}
\item \textit{Ekam padam} refers to the right feet.
\end{enumerate}

\section*{\textit{Āsana} -- 4: \textit{Kūrmāsana}}

\noindent 
\textbf{Verses 22}

\begin{shloka}
\textit{\textbf{gudaṃ nirudhya gulphābhyāṃ vyutkrameṇa samāhitaḥ|}\\
\textbf{kūrmāsanaṃ bhavedetaditi yogavido viduḥ||}}
\end{shloka}

\subsection*{Word split}

\textit{Gudaṃ, Nirudhya, Gulphābhyāṃ, Vyutkrameṇa, Samāhitaḥ, Kūrmā\-Sanaṃ, Bhaved, Etad, Iti Yogaviḍāḥ Viduḥ}

\subsection*{Paraphrased Word meaning}

\begin{longtable}{>{\noindent\raggedright}p{5cm}>{\noindent\raggedright}p{5cm}}
\textit{Gulphābhyāṃ} --- by the two ankles  & \textit{Etad} --- this\tabularnewline
\textit{Gudaṃ} --- the anus  & \textit{Kūrmāsanaṃ} --- \textit{Kurmāsana}\tabularnewline
\textit{Nirudhya} --- after pressing & \textit{Bhaved} --- becomes\tabularnewline
\textit{Vyutkrameṇa} --- in a inverted manner/head down & \textit{Iti} --- thus\tabularnewline
\textit{Samāhitaḥ} --- (one should be) seated  & \textit{Yogaviḍāḥ} --- those well versed in yoga\tabularnewline
\textit{Viduḥ} --- know & 
\end{longtable}

\subsection*{Purport}

Pressing the anus with both the ankles and being seated in a head down manner is \textit{Kuramāsana}. This is known by those who are well versed in \textit{Yoga}. 

%\subsection*{Image}
	
\section*{\textit{Āsana} -- 5: \textit{Kukkuṭāsana}}

\noindent 
\textbf{Verses 1.23}

\begin{shloka}
\textit{\textbf{padmāsanaṃ tu saṃsthāpya jānūrvorantare karau|}\\
\textbf{niveśya bhūmau saṃsthāpya vyomasthaṃ kukkuṭāsanam||23||}}
\end{shloka}

\subsection*{Word Split}

\textit{Padmāsanaṃ, Tu, Saṃsthāpya, Jānūrvoḥ, Antare, Karau, Niveśya, Bhūmau, Saṃsthāpya, Vyomasthaṃ, Kukkuṭāsanam}

\subsection*{Paraphrased word meaning}
\vspace{-10pt}

\begin{longtable}{>{\noindent\raggedright}p{5cm}>{\noindent\raggedright}p{5cm}}
\textit{Padmāsanaṃ} --- \textit{Padmāsanaṃ} & \textit{Niveśya} – inserting\tabularnewline
\textit{Tu}  --- (filler) & \textit{Bhūmau} --- on the ground\tabularnewline
\textit{Saṃsthāpya} --- assuming  & \textit{Saṃsthāpya} --- placing well\tabularnewline
\textit{Karau} --- the hands  & \textit{Vyomasthaṃ} --- in the space\tabularnewline
\textit{Jānūrvorḥ} --- of the knees and the thighs  & \textit{Kukkuṭāsanam} --- \textit{Kukkutāsana} \tabularnewline
\textit{Antare} --- in between & 
\end{longtable}


*(Filler words are used in verses to fulfil metrical requirments. They do not carry any specific meaning. They are mostly indeclinable forms. They also have various contextual meaning. It will be clarified by the commentators.)


\subsection*{Purport}

Onen should assume \textit{Padmāsana}. After that, inserting the hands in between the knees and thighs and placing them on the ground one should raise above (from the ground). This is called \textit{Kukkutāsana} (Kukkuta-Cock).
\vspace{-10pt}

\subsection*{Input from \textit{Jyotsnā} Commentary}

Here also the word \textit{Jānu} should be taken to indicate \textit{Jaṅgha} --- the Shank region. 

\section*{\textit{Āsana} -- 6: \textit{Uttānakūrmakāsana}}

\noindent \textbf{Verses 1.24}

\begin{shloka}
\textit{\textbf{kukkuṭāsanabandha-stho dorbhyāṃ sambadhya kandharām|}\\
\textbf{bhavet kūrmavaduttāna etaduttānakūrmakam||}}
\end{shloka}

\subsection*{Word Split}

\textit{Kukkuṭāsanabandha-Sthaḥ,  Dorbhyāṃ, Sambadhya,  Kandharām, Bhavet,  Kūrmavad, Uttānaḥ Etad, Uttānakūrmakam}

\subsection*{Paraphrased word meaning}

\begin{longtable}{>{\noindent\raggedright}p{5cm}>{\noindent\raggedright}p{5cm}}
\textit{Kukkuṭāsanabandha-sthaḥ} --- Being in the \textit{Kukkutāsana} posture  & \textit{Kūrmavad} --- like a tortoise\tabularnewline
\textit{Dorbhyāṃ} --- by the two arms  & \textit{Uttānaḥ} --- lying on the back\tabularnewline
\textit{Sambadhya} --- connecting  & \textit{Bhavet} --- one should be\tabularnewline
\textit{Kandharām} --- the neck  & \textit{Etad} --- This (is)\tabularnewline
\textit{Uttānakūrmakam} --- \textit{Uttānakūrmāsana} & 
\end{longtable}

\subsection*{Purport}

From \textit{Kukkutāsana} connecting the neck with the hands (that are inserted in between the thighs and knees) and lying on back on the ground is called as \textit{Uttanakurmaka} \textit{Āsana} (the upturned turtle posture).

%~ \subsection*{Image}

\section*{\textit{Āsana} -- 7: \textit{Dhanurāsana}}

\noindent 
\textbf{Verses 1.25}

\begin{shloka}
\textit{\textbf{pādāṅguṣṭhau tu pāṇibhyāṃ gṛhītvā śravaṇāvadhi|}\\
\textbf{dhanurākarṣaṇaṃ kuryāt dhanurāsanamucyate||}}
\end{shloka}

\subsection*{Word Split}

\textit{Pādāṅguṣṭhau, Tu, Pāṇibhyāṃ, Gṛhītvā, Śravaṇāvadhi, Dhanurā\-Karṣa\-Ṇaṃ, Kuryāt, Dhanurāsanam, Ucyate}

\subsection*{Paraphrased word Meaning}

\begin{longtable}{>{\noindent\raggedright}p{5cm}>{\noindent\raggedright}p{5cm}}
\textit{Pādāṅguṣṭhau} – the big toes of the feet  & \textit{Dhanurākarṣaṇaṃ} - (like) stretching a bow \tabularnewline
\textit{Tu}  - (filler)  & \textit{Kuryāt}  - one should do\tabularnewline
\textit{Pāṇibhyāṃ} - by both the hands  & \textit{Dhanurāsanam} – (this is) \textit{Dhanurāsanam}\tabularnewline
\textit{Gṛhītvā}  - after holding  & \textit{Ucyate} – stated (as)\tabularnewline
\textit{Śravaṇāvadhi} – till the ears & 
\end{longtable}

\subsection*{Purport}

Both the big toes are to be held by the hands. Then, till ears one should stretch (one foot)  like drawing a bow. This is called as \textit{Dhanurāsana}\footnote{This posture is known as \textit{Akarnadhanurasana}.} (bow posture).

\subsection*{Inputs from \textit{Jyotsnā} Commentary}

\begin{enumerate}
\item One leg has to remain stretched. The toe of it has to be held by the hand.  With the other hand the other leg has to be drawn till the ear. 
\end{enumerate}

\section*{\textit{Āsana} -- 8: \textit{Matsyendrāsana}}

\noindent \textbf{Verses 1.26}

\begin{shloka}
\textit{\textbf{vāmorumūlārpitadakṣapādaṃ jānorbahiveṣṭitavāmapādam|}\\
\textbf{pragṛhya tiṣṭhet parivartitāṅgaḥ śrīmatsyanāthoditamāsanaṃ syāt||26||}}
\end{shloka}
\vspace{-10pt}

\subsection*{Word Split}

\textit{Vāmorumūlārpitadakṣapādaṃ, Jānorbahiveṣṭitavāmapādam, Pragṛhya,\break Tiṣṭhet, Parivartitāṅgaḥ Śrīmatsyanāthoditam, Āsanaṃ ,Syāt}

\subsection*{Paraphrased word meaning}
\vspace{-10pt}

\begin{longtable}{>{\noindent\raggedright}p{5cm}>{\noindent\raggedright}p{5cm}}
\textit{Vāmorumūlārpitadakṣapādaṃ} --- the right foot has to be placed at the base of the left thigh  & \textit{Pragṛhya} --- holding \tabularnewline
\textit{Jānoḥ} --- of the (right) knee & \textit{Tiṣṭhet} --- stay\tabularnewline
\textit{Bahiḥ} --- outside  & \textit{Parivartitāṅgaḥ} --- twisting the limbs\tabularnewline
\textit{Veṣṭitavāmapādam} --- the left foot has to be placed  & \textit{Śrīmatsyanāthoditam} --- this is stated by \textit{Śrīmatsyanātha}\tabularnewline
\textit{Āsanaṃ} --- posture & \tabularnewline
\textit{Syāt} --- shall be & 
\end{longtable}
\vspace{-5pt}

\subsection*{Purport}
\vspace{-5pt}

Hold the right foot that is placed on the base of the left thigh and (hold) the left foot wound outside the right knee with twisted limbs. This is the \textit{Āsana} stated by Śrī \textit{Matsyanātha}.
\vspace{-10pt}

\subsection*{Inputs from \textit{Jyotsnā} Commentary}
\vspace{-10pt}

\begin{enumerate}
\itemsep=0pt
\item The right foot that is placed at the base of the left thigh has to be held above the ankle by the left hand from behind. 
\item The left foot, which is wound around the right knee, has to held by the right hand in the big toe. The hand has to wind around outside the left foot. 
\item In this process one has to twist one limbs to face back side on the left side.
\item As this has been stated by \textit{Matsyendranātha} this is called as \textit{Matsyendrāsana}. 
\item The same \textit{Āsana} can also be done on the other side also.  
\end{enumerate}

\subsection*{Benefits of \textit{Matsyendrāsana}}

\noindent 
\textbf{Verses 1.27}

\begin{shloka}
\textit{\textbf{matseyendrapīṭhaṃ jaṭharapradīptiṃ pracaṇḍarugmaṇḍalakhaṇḍanāstram|}\\
\textbf{abhyāsataḥ kuṇḍalinīprabodhaṃ candrasthiratvaṃ ca dadāti puṃsām||}}
\end{shloka}

\subsection*{Word split}

\textit{Matseyendrapīṭhaṃ,  Jaṭharapradīptiṃ,  Pracaṇḍarugmaṇḍalakhaṇḍanāstram, 
Abhyāsataḥ, Kuṇḍalinīprabodhaṃ,  Candrasthiratvam, Ca, Dadāti, Puṃsām}
\newpage
\subsection*{Paraphrased Word Meaning}

\begin{longtable}{>{\noindent\raggedright}p{5cm}>{\noindent\raggedright}p{5cm}}
\textit{Abhyāsataḥ} --- by practice, & \textit{Kuṇḍalinīprabodhaṃ} --- awakening of \textit{Kuṇḍalinī}\tabularnewline
\textit{Pracaṇḍarugmaṇḍala\-khaṇḍa\-nāstram} --- a weapon to cut asunder the host  of terrible\tabularnewline 
diseases & \textit{Candrasthiratvaṃ} --- firmness  of \textit{Candra}\tabularnewline
\textit{Matseyendrapīṭhaṃ} --- \textit{Matysendrāsana} & \textit{Ca} --- and\tabularnewline
\textit{Jaṭharapradīptiṃ} --- Glow of the  abdominal (digestive) fire & \textit{Puṃsām} --- for human beings\tabularnewline
\textit{Dadāti} --- bestows & 
\end{longtable}
 
\subsection*{Purport}

By practice, \textit{Matsyendrāsana}, which is the weapon that cuts asunder host of terrible diseases, grants - glow of the digestive fire. It also awakens the  \textit{Kuṇḍalinī}, and grants firmness is \textit{Candra} (the vital energy) for the practitioners.


\section*{\textit{Āsana} -- 9: \textit{Paścimatānāsana}}

\noindent 
\textbf{Verses 1.28}

\begin{shloka}
\textit{\textbf{prasārya pādau bhuvi daṇḍarūpau dorbhyāṃ padāgradvitayaṃ gṛhītvā|}\\
\textbf{jānūpari nyastalalāṭadeśo vasediḍāṃ paścimatānamāhuḥ||}}
\end{shloka}

\subsection*{Word Split}

\textit{Prasārya, Pādau, Bhuvi, Daṇḍarūpau, Dorbhyām , Padāgradvitayaṃ, Gṛhītvā, Jānūpari Nyastalalāṭadeśaḥ,  Vased, Iḍām,Paścimatānam, Āhuḥ}

\subsection*{Paraphrased word meaning}
\vspace{-10pt}

\begin{longtable}{>{\noindent\raggedright}p{5cm}>{\noindent\raggedright}p{5cm}}
\textit{Bhuvi} --- on the ground & \textit{Jānūpari} --- on the knees\tabularnewline
\textit{Pādau} --- both the legs  & \textit{Nyastalalāṭadeśaḥ} --- placing the forehead\tabularnewline
\textit{Daṇḍarūpau} --- in the form of sticks  & vased --- one should stay\tabularnewline
\textit{Prasārya} --- after stretching  & \textit{Iḍāṃ} --- this\tabularnewline
\textit{Dorbhyāṃ} --- by the arms  & \textit{Paścimatānam} --- (as) \textit{Paścimatāna}\tabularnewline
\textit{Padāgradvitayaṃ} --- the tip of both the legs  & \textit{Āhuḥ} --- is stated \tabularnewline
\textit{Gṛhītvā} --- after holding & 
\end{longtable}
\vspace{-10pt}

\subsection*{Purport}

Stretching both legs straight on the ground like stick one should hold the tips of the feet with the arms and place the forehead on the knees and stay in that position. This is stated as \textit{Paścimatanāsana}.
\vspace{-10pt}

\subsection*{Inputs from \textit{Jyotsnā} Commentary}
\vspace{-10pt}

\begin{enumerate}
\itemsep=0pt
\item The ankles of the both the legs that have been stretched on the ground should touch each other. 
\item The respective feet have to be held at the toes with both index fingers which are bent like hook.  
\item While bending forward and touching the knee with the forehead, the lower part of the knee should not raise above the ground.
\end{enumerate}
\vspace{-10pt}

\subsection*{Benefits of \textit{Paścimatānāsana}}
\vspace{-10pt}

\noindent 
\textbf{Verses 1.29}

\begin{shloka}
\textit{\textbf{iti paścimatānamagryaṃ pavanaṃ paścimavāhinaṃ karoti|}\\
\textbf{udayaṃ jaṭharānalasya kuryādudare kārśyamarogatāṃ ca||}}
\end{shloka}
\vspace{-5pt}

\subsection*{Word Split}
\vspace{-5pt}

\textit{Iti, Paścimatānam, Agryam, Pavanam, Paścimavāhinam, Karoti, Udayam, Jaṭharānalasya Kuryād, Udare, Kārśyam, Arogatām, Ca}

\subsection*{Paraphrased word meaning}
\vspace{-10pt}

\begin{longtable}{>{\noindent\raggedright}p{5cm}>{\noindent\raggedright}p{5cm}}
\textit{Iti} --- \textit{Thus} & \textit{Jaṭharānalasya} --- of the fire of the abdomen\tabularnewline
\textit{Paścimatānam} --- \textit{Paścimatāna} (posture) & \textit{Udayaṃ} --- arousal\tabularnewline
\textit{Agryaṃ}  --- the foremost &  \textit{Kuryād} --- makes\tabularnewline
\textit{Pavanaṃ} --- the breath  & udare --- in the abdomen\tabularnewline
\textit{Paścimavāhinaṃ} --- makes it traverse the \textit{Suṣumnā} & \textit{Kārśyam} --- leanness\tabularnewline
\textit{Karoti} --- does & \textit{Arogatāṃ} --- freedom from disease \tabularnewline
\textit{Ca} --- also & 
\end{longtable}
\vspace{-10pt}

\subsection*{Purport}

\textit{Paścimatānāsana} is the foremost among the \textit{Āsana-s}. It makes the air (\textit{Prāṇa}) traverse in the \textit{Suṣumnā}. It increases the glow of the digestive fire. It leads to leanness in the abdomen and freedom from disease.
\vspace{-5pt}

\subsection*{Inputs from \textit{Jyotsnā} Commentary}
\vspace{-5pt}

1. \textit{Paścima} here refers to \textit{Suṣumnā}.

\section*{\textit{Āsana} -- 10: \textit{Mayūrāsana}}


\noindent 
\textbf{Verses 1.30}

\begin{shloka}
\textit{\textbf{dharāmavaṣṭabhya karadvayena tatkūrparasthāpitanābhipārśvaḥ|}\\
\textbf{uccāsano daṇḍavadutthitaḥ khe māyūrametat pravadanti pīṭham||}}
\end{shloka}

\subsection*{Word Split}

\textit{Dharām, Avaṣṭabhya, Karadvayena, Tatkūrparasthāpitanābhipārśvaḥ,\break Uccāsanaḥ, Daṇḍavad, Utthitaḥ, Khe, Māyūram, Etat, Pravadanti, Pīṭham}

\subsection*{Paraphrased Word Meaning}

\begin{longtable}{>{\noindent\raggedright}p{5cm}>{\noindent\raggedright}p{5cm}}
\textit{Karadvayena} --- by both the hands &  \textit{Daṇḍavad} --- like  a stick\tabularnewline
\textit{Dharām} --- the ground  & utthitaḥ --- rising\tabularnewline
\textit{Avaṣṭabhya} --- having supported (the body) & \textit{Khe} --- in the sky\tabularnewline
\textit{Tatkūrparasthāpitanābhipārśvaḥ} --- on the elbow of the hands the sides of the navel  region has to be placed & \textit{Māyūram} --- related to  \textit{Mayūra} (peacock)\tabularnewline
\textit{Uccāsanaḥ} --- in an elevated posture & \textit{Etat} --- this\tabularnewline
\textit{Pīṭham} --- posture  & \tabularnewline
\textit{Pravadanti} --- (they) state & 
\end{longtable}

\subsection*{Purport}

By placing the hands on the ground and supporting the body, and also keeping the elbows besides the navel one should raise the body from the ground like a stick (horizontal to the ground). This posture is called as \textit{Mayurāsana}.

\subsection*{Inputs from \textit{Jyotsnā} Commentary}

\begin{enumerate}
\item While placing the hands on the ground the fingers of the hands should be spread. (Probably to give more balance to the posture) 
\end{enumerate}

\subsection*{Benefits of \textit{Mayūrāsana}}

\noindent 
\textbf{Verses 1.31}

\begin{shloka}
\textit{\textbf{harati sakalarogānāśu gulmodarādīn}\\
\textbf{abhibhavati ca doṣānāsanaṃ śrīmayūram|}\\
\textbf{bahu kadaśanabhuktaṃ bhasmakuryādaśeṣam}\\
\textbf{janayati jaṭharāgniṃ jārayet kālakūṭam||}}
\end{shloka}

\subsection*{Word Split}

\textit{Harati, Sakalarogān, Āśu, Gulmodarādīn, Abhibhavati, Ca, Doṣān, Āsanaṃ, Śrīmayūram
Bahu, Kadaśanabhuktam, Bhasmakuryād, Aśeṣam, Janayati, Jaṭharāgnim, Jārayet, Kālakūṭam}

\subsection*{Paraphrased Word Meaning}

\begin{longtable}{>{\noindent\raggedright}p{5cm}>{\noindent\raggedright}p{5cm}}
\textit{Śrīmayūram} --- the respected \textit{Mayūra}  & \textit{Bahu} --- in huge quantity\tabularnewline
\textit{Āsanaṃ} --- posture  & \textit{Kadaśanabhuktaṃ} --- indiscriminately part taken food\tabularnewline
\textit{Sakalarogān} --- all diseases   & \textit{Aśeṣam} --- without residue/completely\tabularnewline
\textit{Gulmodarādīn} --- like enlargement of the spleen, odema (dropsy) etc. &  \textit{Bhasma} --- (to)ashes  \tabularnewline
\textit{Āśu} --- quickly  & \textit{Kuryād} --- renders, \tabularnewline
\textit{Harati} --- removes & \textit{Janayati} --- generates/kindles,\tabularnewline
\textit{Doṣān} --- the (imbalance in) humors  & \textit{Jaṭharāgniṃ} --- the abdominal fire,\tabularnewline
\textit{Ca} --- also & \textit{Kālakūṭam} --- vicious poison \tabularnewline
\textit{Abhibhavati} --- overcomes   & \textit{Jārayet} --- digests
\end{longtable}

\subsection*{Purport}

\textit{Mayurāsana} quickly removes all diseases like enlargement of the spleen and odema. It also helps overcome imbalances in \textit{Doṣa-s} (\textit{Vāta, Pitta} and \textit{Kapha}). It reduces to ashes (digests) food that has been partaken indiscriminately without any residue. It kindles the digestive fire and also digests vicious poison even. 

\subsection*{Inputs from \textit{Jyotsnā} Commentary}

\begin{enumerate}
\item The word \textit{Doṣa} in the verse indicates \textit{Vāta, Pita} and \textit{Kapha}.
\end{enumerate}


\section*{\textit{Āsana} -- 11: \textit{Śavāsana}}

\subsection*{verse 1.32}

\begin{shloka}
\textit{\textbf{uttānaṃ śavavadbhūmau śayanaṃ tacchavāsanam|}\\
\textbf{śavāsanaṃ śrāntiharaṃ cittaviśrāntikārakam||}}
\end{shloka}

\subsection*{Word Split}

\textit{Uttānam, Śavavad, Bhūmau, Śayanam, Tat, Śavāsanam, Śavāsanam, Śrāntiharaṃ, Cittaviśrāntikārakam}

\subsection*{Paraphrased Word Meaning}

\begin{longtable}{>{\noindent\raggedright}p{5cm}>{\noindent\raggedright}p{5cm}}
\textit{Bhūmau} --- on the floor  & \textit{Śavāsanam} --- (is) \textit{Śavāsana Śavāsanaṃ}\tabularnewline
\textit{Śavavad}  --- like a cropse  & \textit{Śrāntiharaṃ} --- removes tiredness\tabularnewline
\textit{Uttānaṃ} --- on the back  & \textit{Cittaviśrāntikārakam} --- gives relaxation to the mind\tabularnewline
\textit{Śayanaṃ} --- lying  & \tabularnewline
\textit{Tat} --- that  & 
\end{longtable}

\subsection*{Purport}

Lying on the back on the floor like a corpse is called \textit{Śavāsana}. It removes tiredness and grants relaxation to the mind. 

\subsection*{Inputs from \textit{Jyotsnā} Commentary}

\begin{enumerate}
\item It removes the tiredness caused by the practice of \textit{Haṭha}. 
\end{enumerate}

\subsection*{Prelude to Four Important Postures}

\noindent 
\textbf{Verses 1.33}

\begin{shloka}
\textit{\textbf{caturaśītyāsanāni śivena kathitāni ca|}\\
\textbf{tebhyaścatuṣkamādāya sārabhūtaṃ bravīmyaham||}}
\end{shloka}

\subsection*{Word Split}

\textit{Caturaśītyāsanāni, Śivena, Kathitāni, Ca, Tebhyaḥ, Catuṣkam, Ādāya, Sārabhūtaṃ, Bravīmi, Aham||}

\subsection*{Paraphrased Word Meaning}

\begin{longtable}{>{\noindent\raggedright}p{5cm}>{\noindent\raggedright}p{5cm}}
\textit{Śivena} --- by Lord \textit{Śiva} & \textit{Sārabhūtaṃ} --- essential\tabularnewline
\textit{Caturaśītyāsanāni} --- eighty four postures  & \textit{Catuṣkam} --- four\tabularnewline
\textit{Kathitāni} --- have been stated  & \textit{Ādāya} --- taking\tabularnewline
\textit{Ca} --- also & \textit{Aham}  --- I\tabularnewline
\textit{Tebhyaḥ} --- from those & \textit{Bravīmi} --- will state
\end{longtable}

\subsection*{Purport}

Lord \textit{Śiva} has stated 84 \textit{Āsana-s}. From among them I will state four essential \textit{Āsana-s}.

\subsection*{Inputs from \textit{Jyotsnā} Commentary}

\begin{enumerate}
\item The \textit{Ca} – also – indicats that \textit{Śiva} has also stated 84 Lakhs of \textit{Āsana-s}. He has not merely stated 84 \textit{Āsana-s}, He has also stated 84 Lakhs of \textit{Āsana-s}. 
\item There are as many \textit{Āsana-s} as the number of species of beings. \textit{Maheśvara (Śiva)} knows all their divisions. 
\end{enumerate}

\subsection*{The Four Important \textit{Āsana-s}}

\noindent 
\textbf{Verses 1.34}

\begin{shloka}
\textit{\textbf{siddhaṃ padmaṃ tathā siṃhaṃ bhadraṃ ceti catuṣṭayam|}\\
\textbf{śreṣṭhaṃ tatrāpi ca sukhe tiṣṭhet siddhāsane sadā||}}
\end{shloka}

\subsection*{Word Split}

\textit{Siddham, Padmam, Tathā, Siṃham, Bhadram, Ca, Iti, Catuṣṭayam, Śreṣṭham, Tatra, Api, Ca, Sukhe Tiṣṭhet, Siddhāsane, Sadā}

\subsection*{Paraphrased Word Meaning}

\begin{longtable}{>{\noindent\raggedright}p{5cm}>{\noindent\raggedright}p{5cm}}
\textit{Siddhaṃ} --- \textit{Siddhāsana} & \textit{Śreṣṭhaṃ} --- best\tabularnewline
\textit{Padmaṃ} --- \textit{Padmāsana},  & \textit{Tatra} --- among them\tabularnewline
\textit{Tathā} --- and & \textit{Api} --- also\tabularnewline
\textit{Siṃhaṃ} --- \textit{Simhāsana} & \textit{Ca} --- and\tabularnewline
\textit{Bhadraṃ} --- \textit{Bhadrāsana} & \textit{Sukhe} --- in the comfortable\tabularnewline
\textit{Ca, iti} --- and \textit{Thus,}  & \textit{Siddhāsane} --- in \textit{Siddhāsana}\tabularnewline
\textit{Catuṣṭayam} --- the four & \textit{Sadā} --- always\tabularnewline
\textit{Tiṣṭhet} --- one should stay. &
\end{longtable}

\subsection*{Purport}

\textit{Siddhāsana}, \textit{Padmāsana}, \textit{Simhāsana} and \textit{Bhadrāsana} are the four best \textit{Āsana-s}. Even among them, one should always stay in the comfortable \textit{Siddhāsana}.

\subsection*{Inputs from \textit{Jyotsnā} Commentary}

\begin{enumerate}
\item It is clear from this verse that \textit{Siddhāsana} is the best among the four. 
\end{enumerate}

\section*{\textit{Āsana} -- 12: \textit{Siddhāsana} - 1}

\noindent 
\textbf{Verses 1.35}

\begin{shloka}
\textit{\textbf{yonisthānakamaṅghrimūlaghaṭitaṃ kṛtvā dṛḍhaṃ vinyaset}\\
\textbf{meḍhre pādamathaikameva hṛdaye kṛtvā hanuṃ susthiram}\\
\textbf{sthāṇuḥ saṃyamitendriyo'caladṛśā paśyet bhruvorantaram}\\
\textbf{hyetanmokṣakapāṭabhedajanakaṃ siddhāsanaṃ procyate ||}}
\end{shloka}

\subsection*{Word Split}

\textit{Yonisthānakam, Aṅghrimūlaghaṭitam, Kṛtvā, Dṛḍham, Vinyaset, Meḍhre, Pādam, Atha, Ekam, Eva, Hṛdaye, Kṛtvā, Hanuṃ, Susthiram, Sthāṇuḥ, Saṃyamitendriyaḥ, Acaladṛśā, Paśyet, Bhruvoḥ, Antaram, Hi, Etat, Mokṣa-Kapāṭa-Bheda-Janakaṃ, Siddhāsanam, Procyate}
\newpage
\subsection*{Paraphrased Word Meaning}

\begin{multicols}{2}
\textit{Yonisthānakam} --- the perineum \\
\textit{Dṛḍhaṃ} --- firmly  \\
\textit{Aṅghrimūlaghaṭitaṃ} --- conjoined with the base (heel) of the (left) feet  \\
\textit{Kṛtvā} --- having made \\
\textit{Atha} --- then\\
\textit{Ekam} --- one ( right)  \\
\textit{Pādam} --- the feet \\
\textit{Meḍhre} --- above the genitals \\
\textit{Eva} --- only \\
\textit{Vinyaset} --- place \\
\textit{Hṛdaye} --- in the chest \\
\textit{Susthiram} --- firmly \\
\textit{Hanuṃ} --- the chin \\
\textit{Kṛtvā} --- placing \\
\textit{Sthāṇuḥ} --- motionless\\
\textit{Saṃyamitendriyaḥ} --- having controlled the senses  \\
\textit{Acaladṛśā}  --- with a  steady gaze  \\
\textit{Hi} --- certainly \\
\textit{Bhruvoḥ} --- of the brows  \\
\textit{Antaram}  --- in between \\
\textit{Paśyet} --- one should see \\
\textit{Etat} --- this  \\
\textit{Mokṣa-kapāṭa-bheda-janakaṃ} breaks open the doors of the liberation \\
\textit{Siddhāsanaṃ} --- \textit{Siddhāsana} \\
\textit{Procyate} --- is stated. 
\end{multicols}

\subsection*{Purport}

The left heel should press the perineum firmly. Then one should place the right feet (heel) above the genital.  The chin has to be placed firmly in the chest and one should stay still with controlled senses and gaze the place in between the eyebrows. This is \textit{Siddhāsana} that breaks open the doors of Liberation.

\subsection*{Inputs from \textit{Jyotsnā} Commentary}

\begin{enumerate}
\item \textit{Yoni-Sthana} – refers to the place inbetween the anus and the genital. 
\item The verse merely states one foot has to be placed in the Yoni. It refers to the heel of the left foot. 
\end{enumerate}

\section*{\textit{Siddhāsana} - 2}

\noindent 
\textbf{Verses 1.36}

\begin{shloka}
\textit{\textbf{meḍhrādupari vinyasya savyaṃ gulphaṃ tathopari|}\\
\textbf{gulphāntaraṃ ca nikṣipya siddhāsanamiḍāṃ bhavet||}}
\end{shloka}

\subsection*{Word Split}

\textit{Meḍhrād, Upari, Vinyasya, Savyam, Gulpham, Tathā, Upari, Gulphāntaram, Ca, Nikṣipya, Siddhāsanam, Iḍām, Bhavet}

\subsection*{Paraphrased Word Meaning}

\begin{multicols}{2}
\textit{Savyaṃ} --- the left  \\
\textit{Gulphaṃ} --- ankle  \\
\textit{Meḍhrād} --- genital  \\ 
\textit{Upari} ---  above  \\
\textit{Vinyasya} --- having placed  \\
\textit{Tatha} --- also  \\
\textit{Upari} --- above  \\
\textit{Gulphāntaraṃ} --- the other (right) ankle  \\
\textit{Ca} --- and  \\
\textit{Nikṣipya} ---  after placing (one should stay) \\
\textit{Iḍāṃ} --- this \\
\textit{Siddhāsanam} --- \textit{Siddhāsana} \\
\textit{Bhavet} ---  should be
\end{multicols}
  
\textbf{Purport}

The left ankle has to be placed above the genital. The right ankle has to be placed above (the sole of the left foot). This should be \textit{Siddhāsana}.

\subsection*{Inputs from \textit{Jyotsnā} Commentary}


\begin{enumerate}
\item It has to be noted that –“the right ankle has to be placed above (it)” – does not refer to placing of the right ankle above the left ankle but it refers to placing of the right ankle on the sole of the left foot. 
\end{enumerate}

\subsection*{Other Nomenclatures of \textit{Siddhāsana}}

\noindent \textbf{Verses 1.37}

\begin{shloka}
\textit{\textbf{etat siddhāsanaṃ prāhuranye vajrāsanaṃ viduḥ|}\\
\textbf{muktāsanaṃ vadantyeke prāhurguptāsanaṃ pare||}}
\end{shloka}
\vspace{-10pt}

\subsection*{Word Split}
\vspace{-10pt}

\textit{Etat, Siddhāsanam, Prāhuḥ, Anye Vajrāsanam, Viduḥ, Muktāsanaṃ, Vadanti, Eke, Prāhuḥ, Guptāsanam, Pare}
\vspace{-10pt}

\subsection*{Paraphrased Word Meaning}
\vspace{-10pt}

\begin{multicols}{2}
\itemsep=0pt
\textit{Etat} --- this   \\
\textit{Siddhāsanaṃ} --- \textit{Siddhāsana}   \\
\textit{Prāhuḥ} --- (some) state \\
\textit{Anye} --- others   \\
\textit{Vajrāsanaṃ} --- \textit{Vajrāsana}  \\
\textit{Viduḥ} --- know (this as) \\
\textit{Eke} --- some \\
\textit{Muktāsanaṃ} --- \textit{Muktāsana}   \\
\textit{Vadanti} --- speak \\
\textit{Pare} --- others \\
\textit{Guptāsanaṃ} ---  \textit{guptāsana}  \\
\textit{Prāhuḥ} --- state  
\end{multicols}
\vspace{-10pt}

\subsection*{Purport}
\vspace{-5pt}

This \textit{Āsana} is stated variously as \textit{Siddhāsana, Vajrāsana, Muktāsana} and \textit{Guptāsana}.

\subsection*{Inputs from \textit{Jyotsnā} Commentary}

\begin{enumerate}
\itemsep=0pt
\item \textit{Siddhāsana} is pressing perineum with the left heel and placing the right heel above the genital. 
\item \textit{Vajrāsana} is pressing the perineum with the right heel and placing the left heel above the genital.
\item When the right heel is placed above the left and both press against the perineum it is called as \textit{Muktāsana}.
\item When the right heel is placed above the left and both are placed above the genital it is called as \textit{Guptāsana}.
\end{enumerate}

\subsection*{Benefits of \textit{Siddhāsana}}


\noindent \textbf{Verses 1.38}

\begin{shloka}
\textit{Yameṣviva mitāhāramahiṃsāṃ niyameṣviva|\\
mukhyaṃ sarvāsaneṣvekaṃ siddhāḥ siddhāsanaṃ viduḥ||}
\end{shloka}

\subsection*{Word Split}

\textit{Yameṣu, Iva, Mitāhāramahiṃsāṃ Niyameṣu, Iva, Mukhyaṃ, Sarvāsaneṣu, Ekam, Siddhāḥ, Siddhāsanam, Viduḥ}

\subsection*{Paraphrased Word Meaning}


\begin{multicols}{2}
\textit{Yameṣu} --- amongst yamas \\
\textit{Mitāhāram} --- moderate diet  \\ 
\textit{Iva} --- similarly   \\
\textit{Niyameṣu} ---  amongst the niyamas,  \\
\textit{Ahiṃsāṃ} --- nonviolence  \\
\textit{Iva} ---  like  \\
\textit{Sarvāsaneṣu} --- amongst all \textit{Āsana-s}   \\
\textit{Ekaṃ} --- one  \\
\textit{Mukhyaṃ} --- important  \\
\textit{Siddhāsanaṃ} --- siddhāsana,  \\
\textit{Siddhāḥ} --- accomplished   \\
\textit{Viduḥ} --- state
\end{multicols}

\subsection*{Purport}

Like moderate diet  among \textit{Yama-s} and \textit{Ahimsā} in \textit{Niyama-s}, one \textit{Siddhāsana} is known as important among all \textit{Āsana-s}. 

\subsection*{Inputs from \textit{Jyotsnā} Commentary}

\begin{enumerate}
\item From this verse and next six verses the benefits of \textit{Siddhāsana} are stated.
\end{enumerate}

\noindent \textbf{Verses 1.39}

\begin{shloka}
\textit{Caturaśītipīṭheṣu siddhameva sadābhyaset|\\
dvāsaptatisahasrāṇāṃ nāḍīnāṃ malaśodhanam||}
\end{shloka}

\subsection*{Word Split}
\vspace{-5pt}
\textit{Caturaśītipīṭheṣu, Siddham,  Eva, Sadā, Abhyaset, Dvāsaptatisahasrāṇām, Nāḍīnām, Malaśodhanam}

\subsection*{Paraphrased Word Meaning}
\vspace{-5pt}

\begin{multicols}{2}
\itemsep=0pt
\textit{Caturaśītipīṭheṣu} ---  among 84 \textit{Āsana-s} \\
\textit{Siddham} --- \textit{Siddhāsana} \\
\textit{Eva} ---  only \\
\textit{Sadā} --- always \\ 		
\textit{Abhyaset} ---  practice	 \\
\textit{Dvāsaptatisahasrāṇāṃ} ---  of seventy two thousand \\
\textit{Nāḍīnāṃ} --- channels of prāṇa \\
\textit{Malaśodhanam} ---  purification (happens)
\end{multicols}

\subsection*{Purport}

Among 84 \textit{Āsana-s} practice only \textit{Siddhāsana}. It cleanses all the seventy two thousand \textit{Nāḍī-s}. 

\noindent \textbf{Verses 1.40}

\begin{shloka}
\textit{Ātmadhyāyī mitāhārī yāvaddvādaśavatsaram|\\
sadā siddhāsanābhyāsāt yogī niṣpattimāpnuyāt||}
\end{shloka}

\subsection*{Word split}

\textit{Ātmadhyāyī, Mitāhārī, Yāvaddvādaśavatsaram, Sadā, Siddhāsanābhyāsāt, Yogī, Niṣpattim, Āpnuyāt}

\subsection*{Paraphrased Word Meaning}

\begin{multicols}{2}
\itemsep=0pt
\textit{Dvādaśavatsaram} --- twelve years \\
\textit{Yāvad} --- till \\
\textit{Sadā} --- constant \\
\textit{Siddhāsanābhyāsāt} --- practice of \textit{Siddhāsana} \\
\textit{Ātmadhyāyī} --- the one who meditates on the consciousness \\
\textit{Mitāhārī} --- partaking moderate diet  \\
\textit{Yogī} --- yoga practitioner \\
\textit{Niṣpattim} ---  success  \\
\textit{Āpnuyāt} --- attains 
\end{multicols}

\subsection*{Purport}

With the constant practice of \textit{Siddhāsana} for twelve years and moderate diet, the one who meditates upon the \textit{Ātman},  attains success (attains the goal of \textit{Yoga})

\noindent \textbf{Verses 1.41}

\begin{shloka}
\textit{Kimanyairbahubhiḥ pīṭhaiḥ siddhe siddhāsane sati|\\
prāṇānile sāvadhāne baddhe kevalakumbhake|\\
utpadyate nirāyāsāt svayamevonmanī kalā||}
\end{shloka}

\subsection*{Word Split}

\textit{Kim, Anyaiḥ, Bahubhiḥ, Pīṭhaiḥ, Siddhe, Siddhāsane, Sati, Prāṇānile, Sāvadhāne, Baddhe Kevalakumbhake, Utpadyate, Nirāyāsāt, Svayam, Eva, Unmanī,  Kalā}

\subsection*{Paraphrased Word Meaning}

\begin{multicols}{2}
\itemsep=0pt
\textit{Siddhāsane} ---  on \textit{Siddhāsana}  \\
\textit{Siddhe sati} --- being attained   \\
\textit{Kim} --- what   \\
\textit{Anyaiḥ} --- by others   \\
\textit{Bahubhiḥ} --- by many  \\
\textit{Pīṭhaiḥ} ---  by \textit{Āsana-s} \\
\textit{Prāṇānile} ---	when the breath\\
\textit{Sāvadhāne} ---  on becoming steady	 \\
\textit{Kevalakumbhake} --- kevelakumbhaka  \\
\textit{Baddhe} ---  on being assumed \\
\textit{Nirāyāsāt} --- without difficulty  \\
\textit{Unmanī} ---  the trans-mental state/Samādhi state \\
\textit{Kalā} --- (pleasing like) the digit of moon  \\
\textit{Svayam} --- on its own \\
\textit{Eva} --- only  \\
\textit{Utpadyate} --- manifests
\end{multicols}

\subsection*{Purport}

What is the use of many \textit{Āsana-s} if \textit{Siddhāsana} is achieved and also if \textit{Kevala} \textit{Kumbhaka} is attained,? The \textit{Samādhi} state which is pleasing like the digit of the moon manifests on its own. 

\noindent \textbf{Verses 1.42}

\begin{shloka}
\textit{Tathaikasminneva dṛḍhe baddhe siddhāsane sati|\\
bandhatrayamanāyāsāt svayamevopajāyate||}
\end{shloka}

\subsection*{Word Split}

\textit{Tathā, Ekasmin, Eva ,Dṛḍhe, Baddhe, Siddhāsane, Sati, Bandhatrayam, Anāyāsāt,  Svayam, Eva, Upajāyate}

\subsection*{Paraphrased Word Meaning}

\begin{multicols}{2}
\itemsep=0pt
\textit{Tathā} --- also \\	
\textit{Ekasmin} --- in the one  \\
\textit{Eva} --- only \\
\textit{Dṛḍhe} --- firm \\ 		 
\textit{Siddhāsane} --- siddhāsana	 \\
\textit{Baddhe} sati --- on being attained  \\
\textit{Anāyāsāt} ---  without difficulty  \\
\textit{Bandhatrayam} --- three bandha-s	 \\
\textit{Svayam} --- on their own  \\
\textit{Eva} --- certainly  \\
\textit{Upajāyate} --- manifest 
\end{multicols}

\subsection*{Purport}

Similarly, if the \textit{Siddhāsana} is firmly attained, without difficulty the three \textit{Bandha-s} manifest on their own.   

\noindent \textbf{Verses 1.43}

\begin{shloka}
\textit{Nāsanaṃ siddhasadṛśaṃ na kumbhaḥ kevalopamaḥ|\\
na khecarīsamā mudrā na nādasadṛśo layaḥ||}
\end{shloka}

\subsection*{Word Split}

\textit{Nāsanaṃ, Siddhasadṛśaṃ, Na, Kumbhaḥ, Kevalopamaḥ, Na, Khecarīsamā, Mudrā, Na, Nādasadṛśaḥ, Layaḥ} 

\subsection*{Paraphrased Word Meaning}

\begin{multicols}{2}
\itemsep=0pt
\textit{Siddhasadṛśaṃ} --- like Siddha \\
\textit{Na} --- no \\
\textit{Āsanaṃ} --- posture  \\
\textit{Kevalopamaḥ} --- similar to Kevala  \\
\textit{Kumbhaḥ} ---  hold  \\
\textit{Na} --- no \\
\textit{Khecarīsamā} --- like khecarī   \\
\textit{Mudrā} --- mudra \\
\textit{Na} ---  no	 \\		
\textit{Nādasadṛśaḥ} --- like nāda	 \\
\textit{Layaḥ} --- absorption  \\
\textit{Na} --- no
\end{multicols}

\subsection*{Purport}

There is no \textit{Āsana} like \textit{Siddhāsana}. There is no \textit{Kumbhaka} like \textit{Kevalakumbhaka}. There is no \textit{Mudrā} like \textit{Khecarimudrā}. And there is no \textit{Laya} practice comparable to \textit{Nādānusandhānana}.


\section*{\textit{Āsana} -- 13: \textit{Padmāsana} - 1}

\noindent \textbf{Verses 1.44}

\begin{shloka}
\textit{Vāmorūpari dakṣiṇaṃ ca caraṇaṃ saṃsthāpya vāmaṃ tathā\\
dakṣorūpari paścimena vidhinā dhṛtvā karābhyāṃ dṛḍham|\\
aṅguṣṭhau hṛdaye nidhāya cibukaṃ nāsāgramālokayet\\
etat vyādhivināśakāri yamināṃ Padmāsanaṃ procyate||}
\end{shloka}
\vspace{-10pt}

\subsection*{Word Split}
\vspace{-10pt}

\textit{Vāmorūpari, Dakṣiṇam, Ca, Caraṇam, Saṃsthāpya, Vāmam, Tathā, Dakṣorūpari, Paścimena, Vidhinā, Dhṛtvā, Karābhyām, Dṛḍham,Aṅguṣṭhau, Hṛdaye, Nidhāya, Cibukam, Nāsāgram, Ālokayet, Etat, Vyādhivināśakāri, Yaminām, Padmāsanam, Procyate}
\vspace{-10pt}

\newpage
\subsection*{Paraphrased Word Meaning}
\vspace{-10pt}

\begin{multicols}{2}
\itemsep=0pt
\textit{Dakṣiṇaṃ} --- the right  \\
\textit{Ca} ---  and (filler)  \\
\textit{Caraṇaṃ} ---  foot  \\
\textit{Vāmorūpari} ---  on the left thigh  \\
\textit{Saṃsthāpya} ---  having placed  \\
\textit{Vāmaṃ} ---  the left (foot) \\
\textit{Tathā} ---  also \\
\textit{Dakṣorūpari} --- on the right thigh \\
\textit{Paścimena} ---  by the reverse (from behind)  \\
\textit{Vidhinā} ---  way \\
\textit{Karābhyāṃ} --- by hands \\
\textit{Aṅguṣṭhau} --- two big toes \\
\textit{Dṛḍham} --- firmly \\
\textit{Dhṛtvā} ---  having held  \\
\textit{Cibukaṃ} ---  the chin \\
\textit{Hṛdaye} ---  in the chest \\
\textit{Nidhāya} ---  having placed \\
\textit{Nāsāgram} --- the tip of the nose  \\
\textit{Ālokayet} ---  gaze \\
\textit{Etat} ---  this \\
\textit{Yamināṃ} ---  of the self-controlled practitioners \\
\textit{Vyādhivināśakāri} ---  destroyer of diseases \\
\textit{Padmāsanaṃ} ---  \textit{Padmāsana} \\
\textit{Procyate} --- stated
\end{multicols}

\subsection*{Purport}


The right foot (sole facing upwards) has to be placed on the left thigh and on the right thigh the left foot has to be placed. The hands crossing from behind should catch hold of the two big toes. Placing the chin on the chest and one should gaze the tip of the nose. This is \textit{Padmāsana}\footnote{This is known as \textit{Baddha-padmāsana}} that destroys the diseases of the self-controlled practitioners.

\subsection*{Inputs from \textit{Jyotsnā} Commentary}


\begin{enumerate}
\item The right hand has to catch hold of the left big toe from behind and the left hand has to go from the behind and catch hold of the right big toe. 
\item The chin has to be placed on the chest. The measure of its placement is 4 \textit{Aṅgulas} (finger breadths from the starting place of the chest region). This is \textit{Rahasya} – a subtle input.  
\end{enumerate}

\section*{\textit{Padmāsana} - 2}

\centerline{(The descrption of this asana extends from verse 1.45 to 1.47)}

\noindent \textbf{Verses 1.45}

\begin{shloka}
\textit{Uttānau caraṇau kṛtvā ūrusaṃsthau prayatnataḥ|\\
ūrumadhye tathottānau pāṇī kṛtvā tato dṛśau||}
\end{shloka}

\subsection*{Word Split}


\textit{Uttānau, Caraṇau, Kṛtvā, Ūrusaṃsthau, Prayatnataḥ, Ūrumadhye, Tathā, Uttānau, Pāṇī, Kṛtvā, Tataḥ, Dṛśau}

\subsection*{Paraphrased Word Meaning}


\begin{multicols}{2}
\textit{Caraṇau} --- feet  \\
\textit{Uttānau} --- upward facing  \\
\textit{Prayatnataḥ} --- with effort  \\
\textit{Ūrusaṃsthau} ---  being placed on the thighs  \\
\textit{Kṛtvā} --- having made  \\
\textit{Ūrumadhye} --- in the middle of the thigh  \\
\textit{Tathā} ---  also  \\
\textit{Uttānau} ---  upward facing  \\
\textit{Pāṇī} ---  hands  \\
\textit{Kṛtvā} ---  having kept  \\
\textit{Tataḥ} ---  also  \\
\textit{Dṛśau} --- the eyes
\end{multicols}

\subsection*{Purport}

Both the feet with upward facing (sole) has to be placed on (opposite) thighs. The hands with upwards facing palms (placed one above the other) have to be placed in between the thighs. After that, the gaze… (description continued in the next verse)

\subsection*{Inputs from \textit{Jyotsnā} Commentary}

\begin{enumerate}
\item This is the \textit{Padmāsana} that is acceptable to \textit{Matsyendranātha.}
\item The back side of the palms (placed one above the other) should touch both the thighs. The left palm facing upwards touch both the heels of the feet should be placed. Above that the right palm facing upwards has to be placed.
\end{enumerate}

\noindent \textbf{Verses 1.46}

\begin{shloka}
\textit{Nāsāgre vinyased rājadantamūle tu jihvayā\\
uttambhya cibukaṃ vakṣasyutthāpya pavanaṃ śanaiḥ||}
\end{shloka}

\subsection*{Word Split}

\textit{Nāsāgre, Vinyased, Rājadantamūle, Tu, Jihvayā, Uttambhya, Cibukam, Vakṣasi, Utthāpya, Pavanaṃ, Śanaiḥ}

\subsection*{Paraphrased Word Meaning}

\begin{multicols}{2}
\textit{Nāsāgre} ---  on the tip of the nose \\
\textit{Vinyased} --- (one) should place  \\
\textit{Rājadantamūle} ---  on the root of the incisors   \\
\textit{Tu} --- (filler)  \\
\textit{Jihvayā} --- with the tongue \\
\textit{Uttambhya} ---  placing/propping \\
\textit{Vakṣasi} --- in the chest  \\
\textit{Cibukaṃ} ---  the chin  \\
\textit{Śanaiḥ} --- slowly  \\
\textit{Pavanaṃ} ---   air  \\
\textit{Utthāpya} --- arousing
\end{multicols}

\subsection*{Purport}

(Gaze) should be fixed on the tip of the nose. The tongue should press against the incisors. The chin has to be placed on the chest and the air has to be raised (Description continued and completed in next verse).

\subsection*{Inputs from \textit{Jyotsnā} Commentary}

\begin{enumerate}
\item Raising the air here refers to the practice of \textit{Mūlabandha}.   
\end{enumerate}

\noindent \textbf{Verses 1.47}

\begin{shloka}
\textit{Iḍāṃ Padmāsanaṃ proktaṃ sarvavyādhivināśanam|\\
durlabhaṃ yena kenāpi dhīmatā labhyate bhuvi||}
\end{shloka}

\subsection*{word split}

\textit{Iḍām, Padmāsanam, Proktam, Sarvavyādhivināśanam,  Durlabham, Yena, Kena, Api, Dhīmatā, Labhyate, Bhuvi}

\subsection*{Paraphrased Word Meaning}

\begin{multicols}{2}
\textit{Iḍāṃ} ---  this  \\
\textit{Sarvavyādhivināśanam} --- destroyer of all diseases  \\
\textit{Padmāsanaṃ} ---  \textit{Padmāsana} \\
\textit{Durlabhaṃ} ---  difficult to obtain \\ 
\textit{Proktaṃ} ---  stated  \\
\textit{Bhuvi} --- on the earth  \\
\textit{Yena} kenāpi ---  by some  \\
\textit{Dhīmatā} ---  intelligent  \\
\textit{Labhyate} --- attained
\end{multicols}

\subsection*{Purport}

This is \textit{Padmāsana} that destroys all diseases. This is difficult to practice and it is attained only by an intelligent person on the earth.

\section*{\textit{Padmāsana} - 3}

\noindent \textbf{Verses 1.48}

\begin{shloka}
\textit{Kṛtvā saṃpuṭitau karau  dṛḍhataraṃ baddhvā tu Padmāsanam\\
gāḍhaṃ vakṣasi sannidhāya cibukaṃ dhyāyaṃśca taccetasi|\\
vāraṃ vāramapānamūrdhvamanilaṃ protsārayan pūritam\\
nyañcan prāṇamupaiti bodhamatulaṃ śaktiprabhāvānnaraḥ}|| 48 ||
\end{shloka}

\subsection*{Word Split}

\textit{Kṛtvā, Saṃpuṭitau, Karau,  Dṛḍhataram, Baddhvā, Tu, Padmāsanam, Gāḍham, Vakṣasi, Sannidhāya, Cibukam, Dhyāyan, Ca, Tat, Cetasi Vāram, Vāram,  Apānam, Ūrdhvam, Anilam, Protsārayan, Pūritam, Nyañcan, Prāṇam, Upaiti, Bodham, Atulam,  Śaktiprabhāvāt, Naraḥ}
\newpage
\subsection*{Paraphrased Word Meaning}

\begin{multicols}{2}
\textit{Karau}  ---  hands \\
\textit{Saṃpuṭitau} ---  on top of one another \\
\textit{Kṛtvā} ---  making \\
\textit{Dṛḍhataraṃ} ---  firmly \\
\textit{Tu} ---  (filler)  \\
\textit{Padmāsanam} ---  \textit{Padmāsana} \\
\textit{Baddhvā} ---  assuming  \\
\textit{Cibukaṃ} ---  the c	hin  \\
\textit{Vakṣasi} ---  in the chest \\ 
\textit{Gāḍhaṃ} ---  firmly  \\
\textit{Sannidhāya} ---  having fixed  \\
\textit{Tat} --- that  \\
\textit{Ca} ---   and  \\
\textit{Cetasi} --- in the mind   \\
\textit{Dhyāyan} ---  contemplating \\  
\textit{Vāraṃ} ---  again  \\
\textit{Vāram} ---  again  \\
\textit{Apānam} ---  the apāna \\
\textit{Anilaṃ} ---  the air  \\
\textit{Ūrdhvam} ---  above  \\
\textit{Protsārayan} ---  pulling  \\
\textit{Pūritam} ---  filled  \\
\textit{Prāṇam} ---  prāṇa \\
\textit{Nyañcan} ---  pushing down  \\
\textit{Naraḥ} --- human being  \\
\textit{Śaktiprabhāvāt} ---  by the virtue of the great power  \\
\textit{Atulaṃ} ---  matchless  \\
\textit{Bodham} ---  enlightenment  \\
\textit{Upaiti} ---  attains
\end{multicols}

\subsection*{Purport}

Being seated in the \textit{Padmāsana} the hands (palms) are to be placed one above the other firmly. The chin has to be placed firmly on the chest. Meditating upon that (\textit{Brahman/ Iṣṭadevatā} (personal diety)) the Apāna has to be drawn up and \textit{Prāṇa} has to be pushed down repeatedly. Such a practitioner will attain a matchless enlightenment due to efficacy of the \textit{Śakti} (power).

\subsection*{Inputs from \textit{Jyotsnā} Commentary}


\begin{enumerate}
\item This type of \textit{Padmāsana} is agreeable to \textit{Mahāyogins.} 
\item Meditating upon that may refer to \textit{Brahman} or one’s own \textit{Iṣṭadevatā} (Personal diety)
\item Pulling up \textit{Apāna} refers to \textit{Mūlabandha} and by pushing down the prāṇa (by \textit{Jālandhra Bandha}) attaining the conjoining of \textit{Prāṇa} and \textit{Apāna} (by \textit{Uḍḍiyāna Bandha}) the \textit{Prāṇa} is taken into \textit{Suṣumnā}. In the process of conjoining \textit{Prāṇa} and \textit{Apāna Kuṇḍalinī} is awakenend. When \textit{Kuṇḍalinī} is awakened \textit{Prāṇa} enter \textit{Suṣumnā}. When \textit{Prāṇa} enters \textit{Suṣumnā} the mind attains stability – \textit{Citta-sthairya}. By doing \textit{Saṁyama (Dhāraṇā Dhyāna Samādhi)} on \textit{Citta-sthairya} (stability of mind) realisation of the \textit{Ātman} is attained.
\end{enumerate}

\subsection*{Benefits of \textit{Padmāsana}}


\noindent \textbf{Verses 1.49}

\begin{shloka}
\textit{Padmāsane sthito yogī nāḍīdvāreṇa pūritam|\\
mārutaṃ dhārayed yastu sa mukto nātra saṃśayaḥ||}
\end{shloka}

\subsection*{Word Split}


\textit{Padmāsane, Sthitaḥ, Yogi, Nāḍīdvāreṇa, Pūritam, Mārutam, Dhārayed, Yaḥ, Tu, Saḥ, Muktaḥ, Na, Atra Saṃśayaḥ}

\subsection*{Paraphrased Word Meaning}


\begin{multicols}{2}
\textit{Yaḥ tu} --- the one   \\
\textit{Yogī} --- practitioner of Yoga \\
\textit{Padmāsane} ---  in \textit{Padmāsana} \\
\textit{Sthitaḥ} --- the one who stays  \\
\textit{Nāḍīdvāreṇa} ---  through the nāḍī \\
\textit{Pūritam} --- filled  \\
\textit{Mārutaṃ} ---  air  \\
\textit{Dhārayed} ---  holds  \\
\textit{Saḥ} ---  he  \\
\textit{Muktaḥ} --- is liberated \\
\textit{Atra} --- here  \\
\textit{Saṃśayaḥ} --- doubt\\
\textit{Na} ---  no
\end{multicols}

\subsection*{Purport}
\vspace{-8pt}
The \textit{Yogi} who being seated in \textit{Padmāsana} holds the breath inhaled through the \textit{Nāḍī-s} (practices \textit{Prāṇāyāma}) attains liberation. There is no doubt about this.

\section*{\textit{Āsana} -- 14: \textit{Siṃhāsana}}


\noindent \textbf{Verses 1.50}

\begin{shloka}
\textit{gulphau ca vṛṣaṇasyādhaḥ sīvanyāḥ pārśvayoḥ kṣipet|\\
dakṣiṇe savyagulphaṃ tu dakṣagulphaṃ tu savyake||}
\end{shloka}

\subsection*{Word Split}

\textit{Gulphau, Ca, Vṛṣaṇasya, Adhaḥ, Sīvanyāḥ, Pārśvayoḥ, Kṣipet, Dakṣiṇe, Savyagulpham, Tu, Dakṣagulpham, Tu, Savyake}

\subsection*{Paraphrased Word Meaning}

\begin{multicols}{2}
\textit{Vṛṣaṇasya} -- of the scrotum  \\ 
\textit{Adhaḥ} ---  below  \\
\textit{Sīvanyāḥ} ---  of the perineum  \\
\textit{Pārśvayoḥ} ---  on the sides  \\
\textit{Ca} ---  (filler) \\
\textit{Gulphau} ---  the ankles  \\
\textit{Kṣipet} ---   (one) should place  \\
\textit{Dakṣiṇe} --- on the right  \\
\textit{Savyagulphaṃ} --- the left ankle  \\
\textit{Tu} ---  (filler) \\
\textit{Dakṣagulphaṃ} ---  the right ankle  \\
\textit{Tu} ---  (filler) \\
\textit{Savyake} --- on the left 
\end{multicols}

\subsection*{Purport}
\vspace{-5pt}

Both the ankles of the feet are to be placed below the scrotum on the sides of the perineum. The left ankle has to be placed on the right side and the right ankle on the left side. (Descrption of the \textit{Āsana} is continued in the next verse).

\noindent \textbf{Verses 1.51}

\begin{center}
\textit{Hastau tu jānvoḥ saṃsthāpya svāṅgulīḥ samprasārya ca |
vyāttavaktro nirīkṣeta nāsāgraṃ susamāhitaḥ||}
\end{center}

\subsection*{Word Split}

\textit{Hastau, Tu, Jānvoḥ, Saṃsthāpya, Svāṅgulīḥ, Samprasārya, Ca, Vyāttavaktraḥ, Nirīkṣeta, Nāsāgram, Susamāhitaḥ}

\subsection*{Paraphrased Word Meaning}

\begin{multicols}{2}
\textit{Hastau} ---  hands \\
\textit{Tu} ---  (filler) \\
\textit{Jānvoḥ} ---  on the knees  \\
\textit{Saṃsthāpya} ---  having placed  \\
\textit{Svāṅgulīḥ} ---  one’s fingers  \\
\textit{Samprasārya} ---  having spread \\
\textit{Ca} ---  and \\
\textit{Susamāhitaḥ} --- with a single pointed focus  \\
\textit{Vyāttavaktraḥ} ---  with wide opened mouth  \\
\textit{Nāsāgraṃ} ---  the tip of the nose  \\
\textit{Nirīkṣeta} ---  (one) should gaze
\end{multicols}

\subsection*{Purport}

The hands are to be placed on the knees and the fingers are to be spread. With a wide opened mouth one should gaze at the tip of the nose. 

\subsection*{Inputs from \textit{Jyotsnā} Commentary}

\begin{enumerate}
\item The tongue should also be extended outside when the mouth is opened.
\item \textit{Susamāhita} means one should have a single pointed focus.
\end{enumerate}

\subsection*{Benefits of \textit{Siṃhāsana}}
\vspace{-5pt}

\noindent \textbf{Verses 1.52}

\begin{shloka}
\textit{Siṃhāsanaṃ bhavedetat pūjitaṃ yogipuṅgavaiḥ|}\\
\textit{Bandhatritayasandhānaṃ kurute cāsanottamam||}
\end{shloka}

\subsection*{Word Split}
\vspace{-5pt}

\textit{Siṃhāsanam,  Bhaved, Etat, Pūjitam, Yogipuṅgavaiḥ, Bandhatritayasandhānam, Kurute, Ca Āsanottamam}
\vspace{-5pt}
\subsection*{Paraphrased Word Meaning}
\vspace{-5pt}	
\begin{multicols}{2}
\textit{Etat} ---  this \\
\textit{Yogipuṅgavaiḥ} --- by best among the yogins   \\
\textit{Pūjitaṃ} ---  worshipped  \\
\textit{Āsanottamam} --- best among the \textit{Āsana-s} \\
\textit{Siṃhāsanaṃ} ---  \textit{Siṃhāsana} \\
\textit{Bhaved} ---   shall be   \\
\textit{Bandhatritayasandhānaṃ} --- facilitation/brining together of the three bandha-s   \\
\textit{Kurute} ---  is done  \\
\textit{Ca} --- and
\end{multicols}

\subsection*{Purport}

This best among the \textit{Āsana-s} is worshipped by great \textit{Yogins}. This facilitates the practice of /brings together the three \textit{Bandha-s}.


\section*{\textit{Āsana} -- 15: \textit{Bhadrāsana} and its Benefits}

\noindent \textbf{Verses 1.53}

\begin{shloka}
\textit{gulphau ca vṛṣaṇāsyādhaḥ sīvanyāḥ pārśvayoḥ kṣipet|}\\
\textit{Savyagulphaṃ tathā savye dakṣagulphaṃ tu dakṣiṇe||}
\end{shloka}

\subsection*{Word Split}

\textit{Gulphau, Ca, Vṛṣaṇāsya, Adhaḥ, Sīvanyāḥ, Pārśvayoḥ, Kṣipet,Savyagulphaṃ, Tathā, Savye, Dakṣagulpham, Tu, Dakṣiṇe}
\newpage
\subsection*{Paraphrased Word Meaning}

\begin{multicols}{2}
\textit{Vṛṣaṇāsya} ---  of the scrotum \\
\textit{Adhaḥ} ---  below  \\
\textit{Sīvanyāḥ} --- of the perineum   \\
\textit{Pārśvayoḥ} ---  on both sides  \\
\textit{Gulphau} ---  the two ankles  \\
\textit{Ca} ---   (filler) \\
\textit{Kṣipet} --- place  \\
\textit{Savyagulphaṃ} --- the left ankle  \\
\textit{Tathā} ---  similarly \\
\textit{Savye} ---  on the left  \\
\textit{Dakṣagulphaṃ} ---  the right ankle  \\
\textit{Tu} ---  (filler) \\
\textit{Dakṣiṇe} --- on the right 
\end{multicols}

\subsection*{Purport}

Below the scrotum and on the sides of the perineum the two ankles have to be placed. The right ankle has to be placed on the right side and the left ankle on the left (Description of the \textit{Āsana} is continued and completed in the next verse).

\noindent \textbf{Verses 1.54}

\begin{shloka}
\textit{Pārśvapādau ca pāṇibhyāṃ dṛḍhaṃ baddhvā suniścalam|\\
bhadrāsanaṃ bhavedetat sarvavyādhivināśanam|\\
gorakṣāsanamityāhuriḍāṃ vai siddhayoginaḥ||}
\end{shloka}

\subsection*{Word Split}

\textit{Pārśvapādau, Ca, Pāṇibhyām, Dṛḍham, Baddhvā, Suniścalam, Bhadrā\-Sanaṃ, Bhaved, Etat Sarvavyādhivināśanam, Gorakṣāsanam, Iti, Āhuḥ, Iḍām, Vai, Siddhayoginaḥ}

\vspace{-10pt}

\subsection*{Paraphrased Word Meaning}
\vspace{-10pt}

\begin{multicols}{2}
\textit{Pārśvapādau} ---  the sides of the feet  \\
\textit{Ca} ---  and (filler) \\
\textit{Pāṇibhyāṃ} ---  by the hands  \\
\textit{Suniścalam} --- motionless  \\
\textit{Dṛḍhaṃ} ---  firnly \\
\textit{Baddhvā} --- having held  \\
\textit{Etat} ---  this  \\
\textit{Sarvavyādhivināśanam} --- destroyer of all diseases  \\
\textit{Bhadrāsanaṃ} --- \textit{Bhadrāsana} \\
\textit{Bhaved} --- should be \\
\textit{Siddhayoginaḥ} ---  accomplished Yogins \\
\textit{Iḍāṃ} ---  this  \\
\textit{Vai} ---  indeed (filler) \\
\textit{Gorakṣāsanam} ---  \textit{Āsana} of gorakṣa \\
\textit{Iti} --- thus  \\
\textit{Āhuḥ}  --- (some) state
\end{multicols}
\vspace{-10pt}

\subsection*{Purport}

The sides of the feet (of which the soles are facing each other) are to be held firmly by both the hands. This is \textit{Bhadrāsana} that destroys all diseases. This is also called as \textit{Gorakṣāsana} by \textit{Siddha Yogins}.
\vspace{-10pt}

\subsection*{Inputs from \textit{Jyotsnā} Commentary}
\vspace{-5pt}

\begin{enumerate}
\itemsep=0pt
\item The sides of feet are to be held by interlocking of the fingers and the \textit{Tala} (the portion of the arm around elbow) should be touching the abdomen. 
\item This \textit{Āsana-s} might have been called as \textit{Gorakṣāsana} as it might have been practiced (\textit{Abhyastatvat}) by \textit{Gorakṣa}. 
\end{enumerate}
