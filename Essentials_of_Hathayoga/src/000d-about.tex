\thispagestyle{empty}

\chapter*{About}\label{about}

Yoga is popular all over the world. The United Nations announced June 21 as international day of Yoga. It is mentioned in the UN resolution that – “yoga provides a holistic approach to health and well-being” and “wider dissemination of information about the benefits of practising yoga would be beneficial for the health of the world population”. \footnote{\url{http://undocs.org/A/RES/69/131}}Thus we see underlined the value of Yoga and the need to disseminante the teachings of Yoga.
\medskip

Yoga comprises of Āsanas, Prāṇāyāmas and pratices aimed at regulation and focusing the mind. The Patañjali Yogasūtras and the commentaries upon the text by various teachers of the yore and also teachers of contemporary times - provide the philosophical and theoritical frameworks on the aforementioned aspects of Yoga. But more details about those limbs of Yoga like the techniques of practice, outcomes practices and other details and intricacies are dealt in the Haṭhayoga literature and the commentaries upon them.
\medskip

Even among various Haṭhayoga texts written over a millennia\footnote{earliest text of Hatha is from 9th century CE} Haṭhayogapradīpikā by Yogin Svātmārāma (14th Century CE) is considered as a very compact and structured text on the topic and studied world over. A lucid Saṃskṛta (Sanskrit) commentary called Jyotsnā on the text by Yogin Brahmānanda (19th Century) brings out intricate insights and practical details. 
\medskip

There have been many editions and translations and publications on the text in various world languages. This work “Essentials of Haṭhayoga” – is yet another attempt in authentic dissemination of inputs from Haṭhayogapradīpikā. This work is unique because – this is based on inisghts derived from the interactions in a string of textual immersion workshops conducted on Haṭhayogapradīpikā at Krishnamacharya Yoga Mandiram by its Textual Research Division. 
\medskip

These textual immersion workshops were conducted on three important limbs of Haṭhayoga i.e – Āsana, Prāṇāyāma and the Mudrās. Two cycles of workshops on Āsana and prāṇāyāma were conducted and one workshop on Mudrās has been completed thus far, with an average 100+ participation from across the globe. The first in the series of textual immersions commenced in August 4th 2017 and the latest in the series concluded on the March 16, 2021. All these workshops were of the duration of 10 hours.
\medskip

The uniqueness of these workshops lies in its approach indicated by the title – Textual Immersion. Yes, the workshops were “textual immersions” on Haṭhayogapradīpikā, which is a desideratum in the field. Translations and detailed descriptions on Haṭhayogapradīpikā exist. But facilitating direct access to these technical texts for the practitioners is unheard of in the field of Yoga thus far. With regard to Haṭhayogapradīpikā and also other such Yoga texts, all information thus far taught is based on translations.
\smallskip

The method of Textual Immersion adopted in these workshops included - chanting and repeating the veses of Haṭhayogapradīpikā, splitting of the words in the verses, analyzing the meaning of the each of the words of the verses and summarizing the purport of the each verse with additional inputs and clarifications from the Saṃskṛta commentary. Question and answer sessions that followed helped in clarifying doubts that arose in the process of textual immersion.
\smallskip

This approach was well reicieved. A structured feedback form was circulated and the responses received were  analysed and presented as a research paper in the teaching pedagogy track in the 17th World Saṃskṛta conference held at the University of British Columbia, Vancouver, Canāda (June 2019). The full paper is also published as part of the proceedings of the international conferece and is freely accessible for reference.\footnote{The full paper can be seen here - Imparting Yoga Texts in Saṃskṛta : A Teaching Experiment and Its Outcome Mahadevan, Jayaraman 2019 \url{(https://open.library.ubc.ca/cIRcle/collections/70440/items/1.0390880)}}
\smallskip

The research paper cited above analyzes the approach and the impact of the approach. But the content that was disseminated by the approach, in all details is captured in the present work “Essentials of Haṭhayoga”.

The work is in two parts.

The first part presents a Purport/overview of the text Haṭha\-yoga\-pra\-dīpikā. Herein, the whole text in four chapters with 389 Saṃskṛta verses are grouped into meaningful clusters of verses and the purport of each of the cluster is explained to facilitate conceptual clarity and extent of discussion.

The second part is based on the handouts circulated during the workshops. The handout was carefully prepared to facilitate “textual immersion”. This section has three divisions - Āsana, Prāṇāyāma and Mudrās. As the objective of the work is to present the “Essentials of Haṭhayoga” – only those Verses pertaining to specific techniques of Haṭha alone are analysed in this section. Focus is on the verses pertaining to 15 Āsanas, 8 Prāṇāyāma practices (6 Kumbhakas+Nāḍīśodhana+Kapālabhāti) and 5 Mudrās – that are taught and practiced in Krishnamacharya tradition are elaborated. The analysis of the text is done in six steps – for each of the verse – 

\begin{enumerate}
\item Reference number (verse and chapter number)
\item The verse or group of relevant verses pertaining to the technique in Roman script 
\item Word split (this is to recognize the individual words)
\item Paraphrased word meaning (Paraphrasing – As Haṭha\-yoga\-pra\-dīpikā is a versified text, rearranging the word order in the verses is required to form sentences in prose form to understand the correct positioning of the words)
\item Purport of the verse 
\item Inputs from Jyotsnā commentary
\end{enumerate}

As an appendix, to further aid textual immersion – important textual terminologies pertaining to Āsanas Prāṇāyāma and the Mudrās (body parts, sides and directions, words pertatining to actions) are presented. Also, an article on Nādānusandhāna is appened at the end. Nādānusandhāna is stated as a meditative technique which will lead to the state of Samādhi, aimed at practitioners with spiritual goals. This practice is presented in great detail in the fourth chapter of Haṭhayogapradīpikā. The prior three limbs of Haṭha such as Āsana, Prāṇāyāma and Mudrās lead a practitioner to this level. These three limbs also have therapeutic benefits. As the focus of this book is on the means or instruments of Yoga for health and well being – the details of the fourth limb that leads to Samādhi has been briefly introdcued in the appendix through this article.

This book is blessed by an insightful introduction by Sri S Sridharan, Senior Trustee of Krishnamacharya Yoga Mandiram which presents the core ideals and approach of Krishnamaharya Yoga Mandiram towards Yoga and its limbs.

It his hoped that “Essentials of Haṭhayoga”, which is thus based on a systematic textual approach, will give an immersive understanding on essential aspects of Yoga to students, teachers, researchers, therapists and professionals associated with Yoga.

Immersive understanding of authentic texts helps in overcoming uncertanities that may exsit with regard to the core practices of Yoga. Clarity in understanding leads to confidence in practice. Confidence in practice, in turn, leads to achievement of desired outcomes in health and wellbeing through Yoga. 

\textbf{Having said this, it is very important that along with clarity in understanding the text which is achived through this textual immersion, getting trained in an authentic living tradition of Yoga practice is paramount to understand the fullest import of the text and and safely derive the intended wellbeing outcomes.}
\vspace{1cm}

\begin{flushright}
\textbf{Dr M Jayaraman}
\end{flushright}
