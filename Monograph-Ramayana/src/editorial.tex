\chapter*{Volume Editorial}\label{volume_editorial}

\lhead[\small\thepage\quad K S Kannan]{}
\rhead[]{Volume Editorial\quad\small\thepage}


It gives me pleasure to write a few words by way of editorial to this monograph of my dear student Ms. Manjushree Hegde. This monograph problematises Professor Sheldon Pollock's interpretation of the {\sl Rāmāyaṇa}.

\section*{Epic v/s Itihāsa}

It is typical of Western scholars to treat Indian Itihāsa-s and Purāṇa-s on par with their own epics -- the poems of Homer\index{Homer} (dated around 8th c. BCE). ``To place the two Indian epics on a par with Homer or Virgil\index{Virgil} is to ignore how the Indian poems have been adored and how they have moulded the character and faith of the people'' cautioned V. Raghavan. Brockington\index{Brockington, J. L.} is not unaware of the prejudices that alien labels such as ``epics'' may inject into readers' minds. 
 
Already by the time of Plato\index{Plato} (4th c. BCE), the Homeric poems had come to be looked down upon. In vivid contrast, there was/is no Hindu -- through  the length and breadth of this vast country and through the millennia of history - who does not revere the ``epic poems'' of the Hindu heritage viz. the {\sl Rāmāyaṇa}\index{Ramayana@\textsl{Rāmāyaṇa}|passim} and the {\sl Mahābhārata}.\index{Mahabharata@\textsl{Mahābhārata}} (That is not to deny, of course, the new breed of Hindus with their thoroughly colonised minds, which look at the West with admiration if not awe, or are sold to leftist and ``secular'' ideologies).

\section*{Tapas and Yoga}

To the typical Western Indologist the {\sl Rāmāyaṇa} is a ``chaotically structured'' text. The very opening word of the {\sl Rāmāyaṇa} is anathema to him: the {\sl Rāmāyaṇa} starts with {\sl tapas} -  which is a mere mortification of the body (if he professes Christianity), and little else than superstition (if he swears by ``secularism''/science). Yet there is no Sanskrit work which does not revere {\sl tapas} and {\sl yoga}. The {\sl Gītā}\index{Bhagavadgita@\textsl{Bhagavadgītā}} calls itself a Yoga-śāstra. Vyāsa\index{Vyasa@Vyāsa} wrote the {\sl Mahābhārata} with the power of {\sl tapas} and {\sl brahma-carya} says the opening chapter of the epic ({\sl tapasā brahmacaryeṇa  - Mahābhārata}\index{Mahabharata@\textsl{Mahābhārata}} 1.1.54). ``Stationed in yoga'' it was, that Vālmīki\index{Valmiki@Vālmīki}  ``beheld'' all of the {\sl Rāmāyaṇa}. {\sl ``tataḥ paśyati dharmātmā tat sarvaṁ yogam āsthitaḥ''} ({\sl Rāmāyaṇa} 1.3.6), and saw everything in its verity and veracity ({\sl tat sarvaṁ tattvato dṛṣṭvā} 1.3.7) - noted, right again, in the opening chapters.

\section*{Exhume and Censor}

A typical Westerner would love instead some ballad, ``a heroic tale of love, loss and recovery'', which he therefore would suppose to form the ``nucleus of the epic''; even the wise sayings of the epics are but extraneous inanities to him, howsoever integral they are to the story.

His mind runs to “settle” the text, arrive at a “definitive” edition, rid the epic mass of all dross of  “accretions”, given as it is to suspecting and discovering “interpolations” at every step by deploying “Higher Criticism”.\index{Higher Criticism}

As time progresses, the newer breed of Indologists would like to do something smarter: discover something newer, something “deeper”, that is nonsensical with the epics --- discover some evils of patriarchy, or like Pollock, exhume and unpack, and expose the hegemonic and oppressive tactics enshrined in them. And for this, Pollock develops many theoretical frameworks, his 3D-philology\index{philology!threedimensional@3 dimensional} being one positing three meanings-- the authorial, the traditional and the presentist -- “all equally true”. As the last one, mine, is as true as the original author's, I can foist upon a text any interpretation as I choose to. Legitimising nonsense thus can unleash a volley of interpretations and interpretive strategies, which can run easily amuck.

For Pollock and critics of his ilk, whatever is religious/mythological and philosophical/didactic in Sanskrit literature (would smack of “Brahmanism”, and so,) must be purged from the text. The critic has the fullest freedom to be on the look out for censuring, or censoring, whatever sounds superfluous to him, or laborious or incongruous to him - from the epic, branding them as unoriginal. As Manjushree's excursus shows, Higher Criticism was tried on literature from the Bible to Shakespeare, and was finally repudiated by scholars of eminence such as Humboldt and Goethe.\index{Goethe} Vandalising other cultures in the prerogative, nay duty, of the White Man, after all.

\section*{Fratricide/Patricide}

Manjushree has very well shown the hollowness of Pollock's foisting somehow a “fratricidal\index{fratricidal} war” as the recurrent theme in the epics: so he can stoop to the level of showing Kaṁsa\index{Kamsa@Kaṁsa} as the brother of Kṛṣṇa!\index{Krsna@Kṛṣṇa} Even Śakāra of {\sl Mṛcchakaṭika} would not display such high fidelity to Purānic stories.

Pollock seeks to show that dynastic succession marked a major change in the structure of political power in Indian history, towards the beginning of the epic period. To the contrary, there are around 50 references in the {\sl Ṛgveda}\index{Rgveda@\textsl{Ṛgveda}!dynasties mentioned in} alone that speak of the various dynasties. If not fratricide,\index{fratricide} should not patricide be posited somehow as the next step by default as elsewhere (as for example the standard practice that it was with regard to many, if not most, Muslim\index{Muslim!rulers} kings)? :{\sl Vānaprasthāśrama}\index{vanaprastha@\textsl{vānaprastha}} for the king and {\sl yauvarājya}\index{yauvarajya@\textsl{yauvarājya}} for the prince appear to Pollock as an “institutionalized ritual exile of the king”! Manjushree cites P V Kane\index{Kane, P. V.} to show how {\sl vānaprastha} was no new invention in the {\sl Rāmāyaṇa}, but one dealt with already in {\sl Aitareya Brāhmaṇa}.\index{Aitareyabrahmana@\textsl{Aitareyabrāhmaṇa}}\index{Brahmana@\textsl{Brāhmaṇa}!\textsl{Aitareya}}

\section*{Sowing Seeds of Suspicion}

Struggle for power, dynastic conflict, and armed combat are for Pollock, the normal processes of succession! He seeks to show the commonalities between the Ayodhyākāṇḍa\index{Ayodhyakanda@Ayodhyā-kāṇḍa} of the {\sl Rāmāyaṇa} and the Sabhāparvan\index{Sabhaparvan@Sabhā-parvan} of the {\sl Mahābhārata}\index{Mahabharata@\textsl{Mahābhārata}} with an aim to characterise them as the true commencement of the stories therein. He sees similarities in the behaviour of Rāma and Yudhiṣṭhira\index{Yudhisthira@Yudhiṣṭhira} and hints at collapsing the two texts into a single time-frame! Though depictions of fraternal unity incomparably outnumber fratricidal\index{fratricide} wars in ancient Indian literature, Pollock thinks that Daśaratha,\index{Dasaratha@Daśaratha} Kausalyā,\index{Kausalya@Kausalyā} Guha,\index{Guha} Bharadvāja\index{Bharadvaja@Bharadvāja} and even Lakṣmaṇa\index{Laksmana@Lakṣmaṇa} -- all of them suspect Bharata\index{Bharata} would mount a struggle for power! The subtle mistranslations by Pollock of the original {\sl Rāmāyaṇa} passages in these contexts are very well laid bare by the author of this monograph. In certain places, Pollock's own translation betrays his misreading and misinterpretation. Pollock makes much of the (sole) statement in the text on {\sl rājya-śulka},\index{rajyasulka@\textsl{rājya-śulka}} and the author shows how it has no corroboration in the words or deeds of any in the long epic. For the wearer of the political spectacles, the most adventitious can loom as the quintessential. There is no need to read politics into Bharata's staying at his uncle's house: he was sent there by Kaikeyī herself, after all (2.8.28 Mantharā's words to Kaikeyī). {\sl Kṛta-śobhi} in this context can simply be rendered as [the mind which is] full of glow and glee by a [good] deed performed or executed (rather than merely contemplated). {\sl Kṛta} in the positive sense of {\sl sukṛta}, well-done (of a positive, rather than neutral value) shows itself as the first member of at least a dozen compound words in Sanskrit. If Kaikeyī herself rejoiced in prospect - of Rāma's coronation - would not Bharata have, in retrospect?

If there is a dictum that one must act like Rāma (which is to say dharmically), it appears to Pollock as an expedient to “submission to hierarchy” so contrived as to make way for absolute heteronomy! Filial piety is for him a political tool of subjugation! The author points out how filial piety was praised in Chinese and Roman cultures too.\index{Monier-Williams, Monier} Contrast this with the natural ejaculation of Monier Williams (1863): "Nothing can be more beautiful and touching than the pictures of domestic and social happiness in the {\sl Rāmāyaṇa} and the {\sl Mahābhārata}... In England, where national life is strongest, children are ... less respectful to their parents. In this, the Hindus might teach us a good lesson."

The issue of social hierarchy takes our author into the scheme of the {\sl varṇāśrama}\index{varnasrama@\textsl{varṇāśrama}} system which she analyses in the light of the approach of Coomaraswamy\index{Coomaraswamy, Ananda} (who in turn brings in the corrobrative analysis of Plato too). The concepts of {\sl yajña, karman, dharma} and vocation are clearly set forth by her in their inter-relationship.

\section*{Spiritual Authority and Temporal Power}

Pollock speaks of the relationship between “the king and the brahmin”, which is to say {\sl kṣatra}\index{ksatra@\textsl{kṣatra}} and {\sl brahma},\index{brahma@\textsl{brahma}} as an uneasy one. Yet for over three millennia, the system was followed without even a faint hint of cataclysm. The brahmin's monopoly of the source of authority bars, asserts Pollock, kingship from developing its full potential; quite to the contrary, Manu\index{Manu} as well as the {\sl Mahābhārata}\index{Mahabharata@\textsl{Mahābhārata}} bring out the harmony and mutual complementarity and supplementarity of the two very well ({\sl Manusmṛti}\index{Manusmrti@\textsl{Manusmṛti}} 9.322). Manjushree refutes Pollock's claim that Rāma had any contempt for {\sl kṣātra-dharma},\index{ksatra-dharma@\textsl{kṣātra-dharma}} citing half a dozen verses from the {\sl Rāmāyaṇa}, all wilfully ignored by Pollock.

There can be nothing more mischievous than the equation of the Indian king with personal autocracy. The Hindu king\index{Hindu!king} was never considered above law, but always under the Principle of Dharma, the Supreme. There is perfect accord in this issue amongst the {\sl Rāmāyaṇa} and the {\sl Mahābhārata},\index{Mahabharata@\textsl{Mahābhārata}} the {\sl Manusmṛti} and {\sl Arthaśāstra}.\index{Arthasastra@\textsl{Arthaśāstra}} The crowning quote from Coomaraswamy bears out how the healthy development of kingship to its full potential was achieved nowhere else perhaps as in India -- exactly contrary to the contention of Pollock.

The Araṇyakāṇda\index{Aranyakanda@Araṇyakāṇda} of the {\sl Rāmāyaṇa} is very problematic for Pollock, as very many events there are “marvellous and fantastic”. Professing to study the epic from a traditional standpoint, Pollock switches over glibly midway to Northrop Frye's\index{Frye, Northrop} idea of a myth. The monstrous subhuman creatures and beings of superhuman spirituality are items that Pollock cannot digest. Pollock thinks that kingship is the unifying element in the {\sl Rāmāyaṇa} forgetting or ignoring the fact that Rāma acts as but a {\sl kṣatriya},\index{ksatriya@\textsl{kṣatriya}} not as a king himself, in the episodes of the Araṇyakāṇḍa. The sages request Rāma to help, much as the brahmins beseech Yudhiṣṭhira\index{Yudhisthira@Yudhiṣṭhira} for help when he is in exile. The six-fold classification of dharma that the author draws our attention to is inclusive, and well explains the role of Rāma in the forest.

\section*{Divinity: Functional or Ontological?}

Another issue where Pollock expends much of his ingenuity is in the discussion of the question of the divinity of Rāma\index{divinity!of Rama@of Rāma} in the {\sl Rāmāyaṇa}. There is a great deal of difference in referring to kings in general as embodiments of divinity, and speaking of Rāma as an incarnation of Viṣṇu.\index{Visnu@Viṣṇu}\index{Visnu@Viṣṇu!Rama as an incarnation of@Rāma as an incarnation of,} Calling Rāma a god-man, Pollock attempts to conflate the concepts and confuse the readers, but Manjushree lays bare the subterfuge of Pollock.

Pollock has several jibes at Rāma in the wake of the kidnap of Sītā\index{Sita@Sītā} when Rāma is full of grief. Pollock says that Rāma wanders like a madman but Manjushree shows how the poet develops {\sl vipralambha-śṛṅgāra-rasa}\index{vipralambha-srngara@\textsl{vipralambhaśṛṅgāra}} in the poetical work that the {\sl Rāmāyaṇa} is. She is careful enough to notice how even while taking into account though partially, the traditional interpretation, Pollock also subtly disparages them. The clear difference between Rāma as God, and perceiving every king as god, is carefully befuddled by Pollock. Manjushree looks into the examples of Nala\index{Nala} versus the divinities in the {\sl Mahābhārata}\index{Mahabharata@\textsl{Mahābhārata}} as also the episode of Pṛthu Vainya\index{Prthuvainya@Pṛthu Vainya} therein, and Rāma's own claim - that he is human. She brings in relevant statements from {\sl Āpastamba Dharma Sūtra, Gautama Dharma Sūtra}\index{Apastambadharmasutra@\textsl{Āpastamba Dharmasūtra}}\index{Dharmasutra@\textsl{Dharmasūtra}!Apastamba@\textsl{Āpastamba}} and {\sl Vasiṣṭha Dharma Sūtra}\index{Vasisthadharmasutra@\textsl{Vasiṣṭha Dharmasūtra}}\index{Dharmasutra@\textsl{Dharmasūtra}!\textsl{Vasiṣṭha}} in handling this subtle issue. She does not miss the key mistranslation well-wrought by Pollock of a verse in the {\sl Rāmāyaṇa} in this context. She very well establishes the clear cut difference between an ontological divinity\index{divinity!ontological} and a functional divinity\index{divinity!functional} about which even a layman in India may be sensitive, but which even a scholar of our times can be made to get confounded about; and this, Pollock exploits very well.

\section*{Misr(l)ead}

The misinterpretations of Pollock are not limited to literary texts. That he has stakes in the contemporary political developments of India is liable to be missed by many cabined in and confined to literary circles. In his 1993 article he aired his ire against the Rath Yātrā of L K Advani\index{Advani, L K} --- when the call to rebuild the destroyed Ram Mandir at Ayodhya was given. In his zest for playing upto Islamophiles, {\sl a la} the obsequious leftists, Pollock grieves that the epic poem is “invoked to empower and give substance to the politics of the present”.

\section*{Archeology and Epigraphy: Tool to Delude}

Foraying into archaeological and epigraphical “evidence’ too, Pollock tries to demonstrate that Rāma came to occupy a public political space from the 12th Century onwards. He does not accuse politicians of abusing the {\sl Rāmāyaṇa} for political purposes, or that some elements in the {\sl Rāmāyaṇa} have been exploited for that purpose: he lays the blame squarely on the {\sl Rāmāyaṇa} itself as carrying elements and instruments that allow for an easy deployment for dangerous political purposes: “the Other can be fully demonized, categorised, counterposed and condemned”! The message that Pollock gives -- not subtly, but openly, is that 12th century onwards, it is the {\sl Rāmāyaṇa} that has been used for othering Muslims\index{Muslim!demonization of@``demonization'' of} who were demonized, and Hindu\index{Hindu!kind, ``divinisation'' of} kings divinised; and hence the {\sl Rāmāyaṇa} is having a dangerous role in Hindu-Muslim politics. It is only after the 12th century, Pollock argues, that India saw the rise of Rāma temples subsequent, and hence consequent, to the arrival of Muslims.

Our author shows how the archeological and inscriptional evidences\index{evidence!archeological}\index{evidence!inscriptional} furnished by Pollock are either very partial and selective, or are quite inconclusive, and in any case over-interpreted. Pollock quotes from Bakker,\index{Bakker, Hans} but Talbot and Chattopadhyaya show the hollowness of their claims. Typical of Pollock’s academic temerity are these words in the context (but used {\sl passim}): “...my findings have to be regarded as provisional, but again I would be surprised if further work would require fundamental revision of my conclusion”. Our author examines in detail the Dabhoi\index{inscriptions!Dabhoi} inscription (13th c. C.E.) and Hansi\index{inscriptions!Hansi} inscription (12th c. C.E.) which Pollock himself offers as supporting his stand; and she shows how the political mytheme of Rāma v/s Rāvaṇa\index{Ravana@Rāvaṇa} does not figure in them in fact, though Pollock claims they do. Kings are glorified in the context of their vanquishing {\sl turuṣka}-s as not merely Rāma, but as too, Indra,\index{Indra} Viṣṇu,\index{Visnu@Viṣṇu} Śiva,\index{Siva@Śiva} Yudhiṣṭhira\index{Yudhisthira@Yudhiṣṭhira} and many others -- quite contrary to the political imagination of Pollock. Our author offers the evidence\index{evidence!inscriptional} of five inscriptions of the 12th and 13th centuries -- all consistently contradicting the stand taken by Pollock.

\section*{Literary Evidence: Nil}

And when it comes to historiographical or literary evidence\index{evidence!literary}\index{evidence!historiographical} that Pollock seeks to exploit, our author provides both {\sl anvaya} and {\sl vyatireka} evidence in abundance that repulse the tall claims of  the “{\sl Rāmāyaṇa} mytheme” given as some revelation by Pollock. If Pollock points to two {\sl kāvya}-s as supporting his idea, she provides evidence from more to show that there is nothing that even approximates to the “mythopolitical equivalence” that he conjures, even in the {\sl kāvya}-s that he cites himself.

Cherry-picking only convenient facts\index{misinterpretation!techniques of!cherry-picking data} (or rather factoids, or more exactly merely {\sl somewhat suggestive} data), concealing their settings, and stripping them of larger/fuller contexts, and building grand theories out of them -- all this befits or bespeaks of merely overambitious apprentices; not by any means, mature scholars. Perfidy is not compensated by ostentation. It is a tragedy of another order that Pollock considers {\sl dharma} as a mere socio-political issue, whereas in Hindu India, even poetry (or for that matter any art) was considered a yoga of a different format, which is to say, one of universal and transcendent dimensions. 

\section*{Vedic Roots}

The Hindu tradition\index{Hindu!tradition} always looked upon the epics -- the {\sl Rāmāyaṇa} and the {\sl Mahābhārata}\index{Mahabharata@\textsl{Mahābhārata}} -- as elaborations ({\sl upabṛṁhaṇa}) of the Veda-s : the {\sl mitra-sammita}\index{mitra-sammita@\textsl{mitra-sammita}} thus expounded the ideas and ideals  of the {\sl prabhu-sammita};\index{prabhu-sammita@\textsl{prabhu-sammita}} nor did the {\sl kāntā-sammita}\index{kanta-sammita@\textsl{kāntā-sammita}} (poetry/literature in general) lag behind (The {\sl Rāmāyaṇa} could very well be viewed as both {\sl mitra-sammita} and {\sl kāntā-sammita}). The greatest text book on poetics (viz. {\sl Kāvya-prakāśa}) said that {\sl kāvya} sends out its message viz. be like Rāma ({\sl rāmādivad vartitavyaṁ, na rāvaṇādivat}) very subtly; and the greatest {\sl vāda-grantha} on poetics (viz. {\sl Dhvanyāloka}\index{Dhvanyaloka@\textsl{Dhvanyāloka}} 4.5+) showed how {\sl mokṣa} (the {\sl Summum Bonum} of life), and {\sl śānta-rasa}\index{santarasa@\textsl{śānta-rasa}} (the Flavour of Tranquillity), constitute the quintessence of the {\sl Mahābhārata},\index{Mahabharata@\textsl{Mahābhārata}} the twin epic of the {\sl Rāmāyaṇa}.

\section*{Iconoclasts in Academic Cloak}

The “academic” attack suffused with casuistry, upon the spiritual {\sl kāvya} of a yogic poet - by resorting to the ruse of labelling it as but a political poem - is a crime more heinous than the iconoclasm of the Muslim\index{Muslim!invaders} marauders. It is time, then, to wrest Indology from the hands of the haters of the Hindu\index{Hindu!heritage} heritage, and reinstate the insiders’ approach. Naturally secular that Hindus are, they have  tolerated for too long the “academic” Westerners’ intrusion into their cultural and intellectual space. It is time they took custody of their own hoary and hallowed tradition. 

\section*{From Disdain to Disruption}

The words of Matthew Arnold,\index{Arnold, Matthew} carrying a ring of truth, and cited with approval by Coomaraswamy, may not suffice today:

\begin{myquote}
{{\sl ``The brooding East with awe beheld}}\\
{\sl Her impious younger world.}\\
{\sl The Roman tempest swell'd and swell'd,}\\
{\sl And on her head was hurl'd.}
\end{myquote}


\begin{myquote}
{{\sl "The East bow'd low before the blast}}\\
{\sl In patient, deep disdain;}\\
{\sl She let the legions thunder past,}\\
{\sl And plunged in thought again.''}
\end{myquote}

The dictum of Vedānta Deśika\index{Vedantadesika@Vedānta Deśika} (14th century) - that goblins need to be responded to in their own language (lest they understand nothing, nor refrain from their diabolical diatribes) --- {\sl piśācānāṁ piśāca-bhaṣayaiva uttaraṁ deyam} --- is more apposite today than during his own times when the invading hoardes dealt untold destruction upon our opulent temples and innocent populace. What Veṅkaṭādhvarin (17th century) grieved, of the horrendous brutalities and warrantless animosity of the alien tribes is not less true of the contemporary ``elite'':
\begin{quote}
{{\sl niskāruṇyatamais turuṣka-yavanair\index{Yavana} niṣkāraṇa-dveṣibhiḥ}}\index{Venkatadhvarin@Veṅkaṭādhvarin} |
\end{quote}

It is scholars like Manjushree that can rise to the occasion to remedy the grim situation. (With due apologies to Mallinātha commencing his commentary on {\sl Raghuvaṁśa/Kumārasambhava}, we may say:)
\begin{quote}
{{\sl bhāraty ādikaveḥ pollāg(Pollock)-durvyākhyā-viṣa-mūrcchitā}} |\\
{\sl mañjuśrī-mañjulā-vāṇī tām adyojjīvayiṣyati} ||
\end{quote}


\noindent
Makara Sankrānti\hfill	{\bf Dr.~K S Kannan}\\
Hemalamba Saṃvatsara\hfill Academic Director\\
15-Jan-2018\hfill Swadeshi Indology Conference Series

