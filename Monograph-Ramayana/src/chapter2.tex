\chapter{Of Kings and Mortals}\label{chapter2} 
%\Authorline{Sowmya Krishnapur}
%\lhead[\small\thepage\quad Sowmya Krishnapur]{}

In his “Introduction” to the Ayodhyā-kāṇḍa, Pollock reconstructs the literary and historical situation of the “original” {\sl Rāmāyaṇa} to demonstrate kingship as a “new and, in their very nature, urgent” problem of the time --- and therefore a dominant --- if the only--- concern of the text. In his introductory essays to the Araṇya-kāṇḍa, on the other hand, he focuses on the “receptive history” of the text and its “mythical nature” to make more forceful his argument.

\section{Pollock’s “Traditionist–Philology”}\label{sec2.1}
\index{traditionist philology}

According to Pollock, philology is primarily of two types. In an essay titled {\sl The Return to Philology}, Edward Said (in {\sl Humanism and Democratic Criticism}, 2004) drew the distinction between the two by utilizing two intriguing figures: one, Dr. Casaubon from George Eliot’s {\sl Middlemarch} --- a curmudgeonly self-absorbed scholar who symbolized the “sterile, ineffectual, and hopelessly irrelevant to life” (Said 2004:57) variety of philology-; the other, Nietzsche, who stood as the “poster boy for fertile, effective, and relevant philology” (Mallette 2010:6).   In Anthony Grafton’s words, 

\begin{myquote}
“One set of humanists seeks to make the ancient world live again, assuming its undimmed relevance and unproblematic accessibility; another set seeks to put the ancient texts back into their own time, admitting that reconstruction of the past is difficult and that success may reveal the irrelevance of ancient experience and precept to modern problems.” 
\hfill	Grafton (1985: 620)
\end{myquote}

Pollock labeled the first type --- the Casaubon type\index{plane 1 philology} --- of philology as “historicist” or “Plane 1” philology, and the\index{historicist philology} latter --- the Nietzsche type --- as “presentist” or “Plane 3” philology. To this list, he added a Plane- 2 philology\index{plane 2 philology} --- the “traditionist philology”. 

Generally, the phrase “traditionist” philology may conjure to mind a deconstruction/reconstruction of the experience of a poetry by a traditional/original-audience. Richard Martin writes,

\begin{myquote}
“The more we learn about actual oral poetries, from Central Asian to Arabic and African to South American Indian, the more obvious it becomes that the traditional audience of an oral performance, the “native speakers”, as it were, of the poetry, have, all of them, the mental equivalent of a CD-ROM player full of phrases and scenes… I would go further and say that the full “meaning”, and the full enjoyment of traditional poetry, come only when one has heard it all before a hundred times, in a hundred different versions…” 	
\hfill Martin (1993:227-28)
\end{myquote}

But Pollock’s attempt is neither towards a re-construction of “a mental-equivalent of a CD-ROM” of Indian tradition,\index{tradition} nor towards a “full meaning/enjoyment of traditional poetry”. Instead, his effort is towards identifying the cause of this or that traditional-interpretation. According to him, every interpretation is made possible by something, by some property of the text that allows for it. “Every reading of the text is”, therefore, “evidence of human consciousness {\sl activated by the text} in its search to make sense of it” [{\sl italics ours}] (Pollock 2014:407). A philologist, therefore, ought to focus on the text and try to identify what in it allowed for a particular interpretation\endnote{“... this plane of meaning does not, or not only, tell us about the historical consciousness of readers; it also tells us about the text and the properties the text possesses that might have prompted this or that interpretation...” Pollock (2014: 406)}.\index{interpretation} He writes,

\begin{myquote}
“Since all interpretations are embodiments of human consciousness, which have been called into being by certain properties in the text, such forms of consciousness cannot be correct or incorrect {\sl in their historical existence}. What we philologists aim to capture when we read along the plane of tradition --- which, remember, is only one of three --- are those forms themselves, what about the text itself summoned those forms of consciousness into existence…”
\hfill Pollock (2014:406)
\end{myquote}

“Plane 2” philology, therefore, reflects not only the historical consciousness of the original readers but it also mirrors the properties of the text that “prompted this or that interpretation” Pollock (2014:406). Most scholars, according to Pollock, do not take into consideration this plane of philology. In his words,

\begin{myquote} 
“...Most scholars simply ignore these, as my classic teachers always did, for whom no traditional interpretation, whether of Hellenistic scholiast, Roman commentator or medieval scribe, could make any claim to truth. Even those who do not ignore them, like my Indian teachers or Sanskrit colleagues, rarely offer an account of why we should take the meanings, or the truths, of tradition seriously...”
\hfill Pollock (2007b:402)
\end{myquote}

In order to remedy the situation, Pollock offers to employ the “traditional-lens” to read the Araṇya-kāṇḍa, and also to identify the “textual properties” that “prompt” it.  
	
Certainly, Pollock’s call to take a “tradition’s interpretation” into consideration is laudable. Of course, he is not --- as he claims --- the first to propose its importance. Many a renowned Indologists --- as early as Paul Hacker (1913)\index{Hacker, Paul} --- proposed a philology not restricted to “dissection of a text in pursuit of origins/earlier versions”. Halbfrass writes, “[Hacker showed] the positivistic method fails to realize that legends, myths, doctrines, etc., constitute meaningful structures (“Gelstalten”) and totalities, which undergo historical changes and remain meaningful through such changes… this means that “later versions” deserve as much attention as their earlier or “original” elements, and that the changes and transformations themselves have to be explored as meaningful historical processes” (Halbfass 1995:5). 

Unfortunately, Pollock’s specific version of “traditionist-philology”,\index{traditionist philology} and his employment of the word “tradition” is rather ambiguous. In the few references that he makes to tradition, it appears that he explicitly refers only to the medieval academic reception of the text: 

\begin{myquote}
“The traditional readings of the {\sl Rāmāyaṇa} of Vālmīki--- including both the countless literary adaptations and the interpretations of the medieval commentators...”
\hfill	Pollock (2007b: 15)

\smallskip

“...the commentators certainly give no hint of...”\hfill  Pollock (2007b: 1)
	
“...indigenous artistic or scholarly appreciation of the poem...''

\hfill  Pollock (2007b:15)
\end{myquote}

Medieval commentators of the {\sl Rāmāyaṇā} cannot account entirely for its “receptive history”. First of all, the {\sl Rāmāyaṇa} is primarily a composition-in-performance --- performers neither memorize fixed texts nor improvise freely --- they learn stylized diction, performance style, themes, and the outlines of the narratives and recombine them before audiences.  So, the term “tradition” refers here, in Scodel’s words, “diachronically to the history and process of transmission… to the rules of the genre and to such conventions as the poetic dialect, the formulae, and the meter. It also frequently means the themes, and indeed, the actual narratives themselves” (Scodel 2009: 3). In this sense, the “tradition” constitutes a canon --- and in A. K. Ramanujan’s\index{Ramanujan, A.K.} words, “what is contained mirrors the container; the microcosm is both within and like the macrocosm, and paradoxically also contains it. Indian conceptions tend to be such concentric nests” (Ramanujan 1989:51). And in such a world, Ramanujan writes, the systems of meaning are elicited by contexts, by the nature of the listener. Such a tradition, such a context, is still alive in India today. Writing of the {\sl Mahābhārata} in India, V. S. Sukthankar\index{Sukthankar, V.S.} remarks:

\begin{myquote}
“This dateless and deathless poem, which had evoked throughout Indian antiquity such wide interest and which forms the strongest link between India old and new--- what is it, what is this miracle of a book? The learned philologist of the present day feels a deal of hesitation in answering this question, which to the unsophisticated Indian would present no difficulty whatsoever. If questioned, the latter will no doubt promptly and confidently answer that the {\sl Mahābhārata} is a divine work recounting the war-like deeds of his ancestors, the god-like heroes of a past age, the unrighteous Kauravas on the one hand and the righteous Pāṇdavas aided by Lord Śrī Kṛṣṇa on the other --- of the Golden Age when gods use to mingle with men, when the people were much better off, much happier, than they are today. And the illiterate Indian is right, to a very large extent, far more often is than his “educated” brother.” 

\hfill	Sukthankar (1957: 53)
\end{myquote}

Unfortunately, Pollock’s “traditionist-philology” is historicist in its own way; his “tradition” is located in a historical context --- in, arbitrarily, a medieval era --- and thus carries with it a foundational element of historicism that is paradoxical and almost self-cancelling. It is this rationale that allows Pollock to say that the commentarial tradition is “the closest thing we have to an original audience” (Pollock 2007b: 18) or that a “traditional-medieval interpretation” will lead to “trans-historical authentic attitudes” (Pollock 2007b: 232). It is this rationale that allows him to differentiate between a “traditional-philology” and a “presentist-philology”. It is this rationale that pronounced Sanskrit dead --- dead since at least the onset of the colonial era. By equating “tradition” with “medieval”, Pollock, in the end, only substitutes one “historic” lens for another. 

Nevertheless, let us examine how Pollock applies his philological tool to demonstrate the “kingship” as the central concern of the {\sl Rāmāyaṇa}. 

\subsection{“Mythical” nature of the {\sl Rāmāyaṇa}}\label{sec2.2}

For most --- if not every --- Western scholar, the Araṇya-kāṇḍa is, says Pollock, a problematic section of the {\sl Rāmāyaṇa}. Whereas the Ayodhyā-kāṇḍa treats of the “socio-political concerns” of a royal family, and therefore has a touch of reality to its narration, the remainder, to them, he says, is simply fantastic. To quote Jacobi,

\begin{myquote}
“... that [the saga of the {\sl Rāmāyaṇa}] is composed of two utterly different and distinct parts. [In the Ayodhyā-kānda], everything is human, natural, totally free from fantasy… the case is quite otherwise in the second half of the saga [in the Araṇya-kāṇḍa] , where everything is marvelous and fantastic.” 	
\hfill As cited in Pollock (2007b: 4)
\end{myquote}

Contrarily, the traditional/original audience of the {\sl Rāmāyaṇa} has never felt, says Pollock, the said discontinuity between the different sections of the text. Not any of the “countless literary adaptations” or the “medieval commentator’s interpretations” show any doubts regarding the “organic unity” of the text. So, Pollock proposes to start his reading of the {\sl Rāmāyaṇa} from a different point --- from a “traditional” a {\sl priori} that the {\sl Rāmāyaṇa} is not made of two heterogeneous parts, but is, in fact, a unified whole:

\begin{myquote}
“Suppose we were to take seriously what generations of performers and audiences have felt, not to speak of the composer, that the monumental poem is not made up of two heterogeneous and uncombinable narratives, but forms a meaningful whole?”
\hfill Pollock (2007b: 5)
\end{myquote}

From here, a philologist’s task would be to ponder over how the work functions as a unit, {\sl how} its parts fit together to form a whole. 

Pollock’s answer to the problematic is --- by recognizing the Araṇya-kāṇḍa as a myth\index{myth} of the Northrop Frye’s\index{Frye, Northrop} variety. Northrop Frye, an authority on the genre of “romance” in English literature, defined a “myth” in its contrast to the romance:   

\begin{myquote}
“the difference between the mythical and the fabulous is a difference in authority and social function, not in structure. If we were concerned only with structural features we should hardly be able to distinguish them at all… there are only so many effective ways of telling a story, and myths and folktales share them without dividing them. But as a distinctive tendency in the social development of literature, myths have two characteristics that folktales, at least in their earlier stages, do not show, or show much less clearly. First, myths stick together to form a mythology, a large interconnected body of narrative that covers all the religious and historical revelation that its society is concerned with, or concerned about. Second, as part of this sticking-together process, myths take root in a specific culture and it is one of their functions to tell that culture what it is and how it came to be, in their own mythical terms.”
\hfill Pollock (2007b:12)
\end{myquote}

Accordingly, if the {\sl Rāmāyaṇa} is considered as a myth that performs the “social function”\index{social function} of portraying the “nature of kingship”, Pollock says, the Araṇya-kāṇḍa can also be demonstrated to carry the same “social function”, and thus be taken as a myth that forms a “meaningful part” of the whole. So, first of all, he displays the similarities in structure between the Araṇya-kāṇḍa\index{Aranyakanda@Araṇya-kāṇḍa} and the European romance (the only elements that Western scholars purportedly consider), and then demonstrates its “authority and social function” (of “portraying kingship”) to classify it as a “myth”--- thus demonstrating its “unity” with the {\sl Rāmāyaṇa}.

It must be noted at once that Pollock commits a fundamental mistake in his analysis: he commences to study the {\sl Rāmāyaṇa} from a “traditional” perspective but midway, he exchanges it for a Western one.  Indian tradition categorizes the {\sl Rāmāyaṇa} as a {\sl kāvya}\index{kavya@kāvya} or an {\sl itihāsa-purāṇa} --- naturally one would expect Pollock’s “Plane-2” philology to engage in a deeper/subtler understanding of the nature of a {\sl kāvya} or an {\sl “itihāsa-purāṇa”}. On the contrary, it is Frye’s definition of a myth that entirely informs Pollock’s argument --- his reading of the {\sl Rāmāyaṇa} is only in its consideration as a myth, and a myth of the Frye variety at that. Surely, Frye’s “mythical” reading of the {\sl Rāmāyaṇa} cannot be labeled as “Plane 2” or “traditionist” philology.\index{traditionist philology}

Furthermore, the {\sl Rāmāyaṇa} does not even fit properly into Frye’s definition of a myth. According to Frye, a myth\index{myth} is a story ({\sl mythos}), usually about “the acts of gods”. They belong to a society and inform it --- in this way, they primarily carry “a social function”. According to Frye, this “function” is to inform the society (of which it is a part) of their gods, their traditional history, the origins of their customs, structure, etc. There are myths of creation, of exodus and migration, of deluge, of apocalypse, and they discuss why we are here, where we are headed. Such myths outline, as broadly as words can do, a society’s vision of this universe, and its own place in it --- its proclamation is not so much ‘this is true’ as ‘this is what you must know’. Myths are, therefore, regularly used in rituals --- as a commentary, for dramatization, etc. Furthermore, a myth’s poet must be, according to Frye, a vehicle of God, not of memory. In Frye’s words, 

\begin{myquote}
“On the mythical plane there is more legend than evidence, but it is clear that the poet who sings about gods is often considered to be singing as one, or as an instrument of one. His social function is that of an inspired oracle; he is frequently an ecstatic, and we hear strange stories of his powers. Orpheus could draw trees after him; the bards and Ollaves of the Celtic world could kill their enemies with their satire; the prophets of Israel foretold the future. The poet's visionary function, his proper work as a poet, is on this plane to reveal the god for whom he speaks. This usually means that he reveals the god's will in connection with a specific occasion, when he is consulted as an oracle in a state of "enthusiasm" or divine possession. But in time the god in him reveals his nature and history as well as his will, and so a larger pattern of myth and ritual is built up out of a series of oracular pronouncements. We can see this very clearly in the emergence of the Messiah myth from the oracles of the Hebrew prophets. The Koran is one clear historical instance at the beginning of the Western period of the mythical mode in action. Authentic examples of oracular poetry are so largely pre- and extra-literary that they are difficult to isolate. For more recent examples, such as the ecstatic oracles which are said to be an important aspect of the culture of the Plains Indians, we have to depend on anthropologists.”

\hfill Frye (2006: 52)
\end{myquote}

Pollock quotes Frye narrowly by saying that a “myth” must only carry  “a social function” to be called one --- implying that this social function may be of any sort. But Frye’s definition of a myth’s social function is rather rigid --- not “any” social function can inform the “myth”. For Pollock, the {\sl Rāmāyaṇa} is a “sustained and elaborate myth exploring {\sl the nature of a king}, the character and quality of his powers and every domain in which these powers are manifested.” (2007b:14) [{\sl italics ours}] Frye, who classifies the Bible as a myth, would not admit Pollock’s “narrative of kingly discourse” to be called a social function. 

Pollock’s framework is faulty in Frye’s context. It is entirely out of place in a traditional-Indian one.  So, it ultimately stands on a mish-mash of different theories that is shaky at best.  

\section{“Mythical nature” of the Araṇya-kāṇḍa}\label{sec2.3}
\index{Aranyakanda@Araṇya-kāṇḍa}

In accordance with his framework, Pollock demonstrates that the Araṇya-kāṇḍa fits into Frye’s myth --- he shows it is structurally similar to the “romance”, functionally different. First of all, then, he draws attention to a multitude of “fantastic” elements in the Araṇya-kāṇḍa that lend it an appearance of a “romance”. In fact, a “vast inventory” may be found, he says, of the techniques and motifs in the Araṇya-kāṇḍa that are “representative of European romance” --- the piety of the protagonist, an idealized love relationship between himself and the heroine, the loss of a beloved, the hero’s wanderings and the dimension of a quest and the gods’ role in the unfolding adventure, tokens of recognition, the hero’s triumph, etc.  

Yet, the Araṇya-kāṇḍa is informed, says Pollock, with a “social function” which lends it its mythical nature --- that of kingship.\index{kingship} Accordingly, kingship is, he says, the key element that ties the Araṇya-kāṇḍa together with the {\sl Rāmāyaṇa}. 

Let us, then, examine the logic of this conclusion. 

\subsection{“Authority and social function” in the Araṇya-kāṇḍa}\label{sec2.3.1}
\index{social function}

As the title suggests, the Araṇya-kāṇḍa is set in the forest. Pollock, firstly, dwells on what this choice of locale implies for the story. In Western literature, Pollock remarks, wilderness is, 

\begin{myquote}
“...the place where there is no community, just or injust, and no historical change for better or for worse... Therefore the individual (in the wilderness) is free from both the evils and the responsibilities of communal life...”
\hfill Pollock (2007b:13)
\end{myquote}

In Sanskrit literature, too, he says, the forest is viewed in stark opposition to the town or city; it is a place “prior to, or at least exterior to” many of the claims and obligations of the social world. Life in the forest is not bound by the confines of family existence; on the contrary, it is precisely where “those escaping such confines come to find peace and transcendence --- the renouncer, the ascetic, the seer--- and indeed, those who are forced out of collective existence, exiles like Rāma himself”. In his words,

\begin{myquote}
“In this place outside the socialized and the humanized, all that a human is not can be found--- monstrous subhuman creatures as well as beings of an almost superhuman spirituality; it is a place where demons, men, {\sl ṛṣis}, demigods and gods all mingle.”
\hfill	 Pollock (2007b:13)
\end{myquote}

From this {\sl a priori}, Pollock finds it extremely odd that in the very first {\sl sarga} of the Araṇya-kāṇḍa, the ascetics of the forest should approach Rāma and solicit his protection (as if he were still a king/protector in the forest)  (Ascetics to Rāma) ({\sl Rāmāyaṇa} 3.1.19-20): 

\begin{myquote}
{{\sl te vayaṁ bhavatā rakṣyā bhavad-viṣaya-vāsinaḥ}} |

{\sl nagarastho vanastho vā tvaṁ no rājā janeśvaraḥ} ||

{\sl rakṣaṇīyās tvayā śaśvad garbha-bhūtā tapo-dhanāḥ} |

\medskip
“We are residents of your realm and need your protection. Wherever you may find yourself, in city or forest, you are our king, the lord of the people... You must always protect us ascetics for we are your children” [{\sl Trans. Pollock}]
\end{myquote}

Apparently, it is not a one-off instance --- rather it sets, says Pollock, the tone for the rest of the book. From the very beginning, there is a continuous “intrusion” of elements of “kingly duties” into the Araṇya-kāṇḍa, and these elements are “so resolutely anti-romantic in their fundamental significance, so heavily laden with “authority and social function” (Pollock 2007b:13). From here, Pollock proceeds to connect the dots to show “kingship” as the “social function” of the {\sl Rāmāyaṇa}.  It is {\sl impossible} to work from a “traditionalist-plane” towards this end of “kingship” so we’d do better to work like Pollock : backwards!

Accordingly, from the above incident, Pollock concludes that it must be that for an ancient Indian king, “whether he is on the throne or in exile, there is no freedom from the responsibilities of communal life”.  Viewing from this angle, Pollock writes that the Araṇya-kāṇḍa can be understood as an exploration of the functions of the king in a different realm --- in the forest. A forest is more an “extra-social sphere” where “the violence of the kingly warrior could be exhibited” vividly. It is a “realm of {\sl artha} (or {\sl daṇḍa}) complementing that of {\sl dharma}”,\index{dharma} and therefore, the “world outside the settled town” becomes essential to the depiction of the “kingly narrative of power and legitimacy” in India. In his words, 

\begin{myquote}
“One of the most productive ways to think of this unitary product [the {\sl Rāmāyaṇa}]... is as a sustained and elaborate “myth” exploring the nature of king, the character and quality of his powers, and every domain in which these powers are manifested. The forest was one such domain, where a fundamental dimension of kingly function could be illuminated...” 
\hfill Pollock (2007b:14)
\end{myquote}

So, Pollock concludes, kingship must {\sl itself} be the “unifying” element of the {\sl Rāmāyaṇa}--- only from this angle does the text seem a “whole”. Furthermore, he implies, the Indian audience must have always recognized this: 

\begin{myquote}
“Looking at the {\sl Rāmāyaṇā} from this perspective, we can regain a sense of the work as a meaningful whole, which Indian audience has always felt.”

\hfill  Pollock (2007b:54)
\end{myquote}

It must be noted at once that Rāma is technically not a king --- he is not an heir prince, even; he went into exile before his {\sl paṭṭābhiṣekha}. He is but a {\sl kṣatriya}\index{ksatriya@kṣatriya}--- and it is as a {\sl kṣatriya} that he promises, and provides, refuge to the ascetics. 

Grammatically, the word {\sl kṣatra} is derived from the root {\sl kṣaṇu hiṁsāyām}, and etymologically, it is {\sl kṣatāt nāśāt trāyata iti kṣatraṁ}. The word itself refers, therefore, to “protection”.  We know that instances of {\sl kṣatriya’s} duty (of protection) abound in Indian literature. When, for example, in Kālidāsa’s {\sl Raghu-vaṁśa},\index{Raghuvamsa@Raghu-vaṁśa} Nandinī, sage Vaśiṣṭha’s cow, faces threat from Kumbhodara, the lion, Dilīpa offers to sacrifice his own self in order to protect Nandinī. And Dilīpa says (Dilīpa to Kumbhodara):

\begin{myquote}
{{\sl kṣatāt kila trāyata ity udagraḥ kṣatrasya śabdo bhuvaneṣu rūḍhaḥ}} |

{\sl rājyena kiṁ tad-viparīta-vṛtteḥ prāṇair upakrośa-malīmasair vā} ||   

\hfill ({\sl Raghu-vaṁśa} 2.53)

“The word ‘{\sl kṣatraṁ}’ is popularly used in the world for ‘one who protects from danger’. If (I) a king behaves in a manner contrary to such a definition, of what use is a kingdom to him, or even his own tainted life?”
\end{myquote}

A comparable situation arises towards the end of the Vana-parvan of the {\sl Mahābhārata} --- a brahmin’s {\sl araṇi}-s get entangled on a deer’s antlers, and the deer runs away with it. The brahmin approaches Yudhiṣṭhira for help (brahmins to Yudhiṣṭhira) ({\sl Mahābhārata} 3.312.14-15): 

\begin{myquote}
{{\sl tasya gatvā padaṁ rājann āsādya ca mahā-mṛgam}} |

{\sl agnihotraṁ na lupyeta tad ānayata pāṇḍavāḥ} || 14 ||

{\sl brāhmaṇasya vacaḥ śrutvā santapto’tha yudhiṣṭhiraḥ} |

{\sl dhanur ādāya kaunteyaḥ prādravad bhrātṛbhiḥ saha} || 15 ||

\medskip
“Track the great deer, attack it, and bring them back, king, so that the {\sl agnihotra}, Pāṇḍava, may not be destroyed”

“At the brahmin’s words, Yudhiṣṭhira was greatly agitated; the son of Kuntī took up his bow and rushed out together with his brothers.”
\end{myquote}

Here, too, Yudhiṣṭhira\index{Yudhishthira@Yudhiṣṭhira} is not a king officially --- he has lost his kingdom in the great gamble. Yet the brahmin addresses him as {\sl rājan} when he seeks his refuge. Similarly, in the Araṇya-kāṇḍa, the ascetics’ words {\sl tvam naḥ rāja} does not make {\sl official} Rāma’s kingship. It is a technical error on Pollock’s part, then, to say that “kingly” duties continue in the {\sl araṇya} --- it is a {\sl kṣatriya’s} duties that continue into the {\sl araṇya}.

It appears that Pollock presumes that the duties of {\sl varṇāśrama} do not apply “in the forest” (and that only the “kingly duties” continue in this realm) --- an assumption which is completely groundless. In which {\sl dharma-śāstra} is it stated that one may abandon one’s caste or station in life simply because he/she is “in the forest” ? Contrarily, we see that Sītā continues to perform her duties as a loyal wife to Rāma, Rāma performs his duties as a son (he performs the last rites of Daśaratha in the forest), he is still a husband, a brother, {\sl and}, of course, a {\sl kṣatriya}. Until one voluntarily and explicitly formally changes one’s station in life (i.e. into {\sl saṁnyāsa}), one cannot simply adopt the ways of another. 

Indian {\sl dharma-śāstra}-s,\index{dharmasastras@dharma-śāstra} in fact, comprehensively classify {\sl dharma} as six-fold (Kane 1941 Vol 2, Part 1:3), viz. 

{\sl dharma} of {\sl varṇa}-s --- injunctions based on {\sl varṇa} alone, such as “a brahmin should never drink wine” or “a brahmin should not be killed”; 

{\sl āśrama-dharma}\index{ashrama-dharma@āśrama-dharma}  --- such rules as begging and carrying a staff enjoined on a {\sl brahmacārin};
  
{\sl varṇāśrama-dharma} --- the rules of conduct enjoined on a man because he belongs to a particular class and is in a particular stage of life, such as ‘a brahmin {\sl brahmacārin} should carry the staff of a {\sl palāśa} tree; 

{\sl guṇa-dharma} --- such as protection of subjects in the case of a crowned king; 

{\sl naimittika-dharma} --- such as expiation on doing what is forbidden; and finally, 

{\sl sādharaṇa-dharma} --- what is common to all humanity viz, {\sl ahiṁsā} and other virtues.
 
Regarding the abandonment/discontinuation of any of the above, P.V. Kane\index{Kane, P.V.} clarifies, 

\begin{myquote}
“With reference to the four āśramas, there are three different points of view (pakṣas) viz. {\sl samuccaya} (orderly co-ordination), {\sl vikalpa} (option) and {\sl bādha} (annulment or contradiction). Those who hold the first view (samuccaya) say that a person can resort to the four āśramas one after another in order and that he cannot drop any one or more and pass on to the next nor can he resort to the householder's life after becoming a saṁnyāsin... 

The second view is that there is an option after brahmacarya i.e. one may become a {\sl parivrājaka} immediately after he finishes his study or immediately after the householder's way of life. ....
 
The third view of bādha is held by the ancient dharmasūtras of Gautama and Baudhāyana. They hold that there is really one āśrama viz. that of the householder (brahmacarya being only preparatory to it) and that the other āśramas are inferior to that of the householder. ...The Mit. on Yaj. III.56 refutes these three views and says that each is supported by Vedic texts and one may follow any one of the three. .... The word āśrama is derived from 'śram' to exert, to labor and etymologically means 'a stage in which one exerts oneself'.  Commentators like Sarvajna-Narāyana ... endeavor to bring about reconciliation between the three views set out above as follows: the view that a man may pass on to {\sl saṁnyāsa} immediately after the period of student-hood (without being a householder) applies only to those persons who are, owing to the impressions and effects of restrained conduct in past lives, entirely free from desires and whose tongue, sexual appetites, belly and words are thoroughly under control; the prescriptions of Manu enjoining on men not to resort to saṁnyāsa without paying off the three debts are concerned with those whose appetites have not yet thoroughly been brought under control and the words of Gautama that there is only one {\sl āśrama} (that of the house-holder) relate only to those whose appetites for worldly pleasures and pursuits are quite keen”.

\hfill Kane, Vol 2, Part 1 (1941:424)
\end{myquote}

It must be noted also that when those “escaping the confines” of their own station in life adopt a different mode of life “in the forests”, they don’t “escape” anything--- they only exchange one set of duties for another. A person in {\sl vānaprasthāśrama} is also bound by duties: he has to attend to those fires that he had attended to while he was a house-holder, he has to sleep on the bare ground, keep twisted locks, wear deer-skin, perform ablution, worship gods, ancestors, and guests, and live upon food stuffs procurable in forests, etc. If one really wishes to “escape” the confines of the society, it is not enough to “enter the forest”--- one must enter {\sl saṁnyāsa}. Many kings entered {\sl vānaprasthāśrama} after anointing an able heir to the throne --- most of the kings of the Raghu-vamśa, the race of Raghu, for example, Bṛhadaśva, Trayyāruṇa, Viśvāmitra, Kapila, Bali, Manu, Samyāti, Yayāti, Devāpi, etc. But Rāma was not one of them. 

Kingship as a unifying element of the {\sl Rāmāyaṇa} is nowhere corroborated in Indian tradition: it is Pollock’s idea entirely which picks its way through the text seeking confirmation of its existence. While he accepts the “traditional” view of unity (of the different {\sl kāṇḍa}-s), its rationale/explanation is entirely his own --- furthermore, it is thoroughly anti-traditional.

\section{Divinity of Rāma and “other kings”}\label{sec2.4}

Now that the “nature” of the {\sl Rāmāyaṇa} is spelt out, and his definition of a “myth” is in place, Pollock wishes to proceed to explore its role in society. He writes, 

\begin{myquote}
“When I use the term “myth” here, I have in mind a patterned representation of the world, with continuing and vital relevance to the culture, which furnishes a sort of invariable conceptual grid upon which variable and multifarious experience can be plotted and comprehended. It is this essential power to interpret and explain reality--- and I mean social reality in the first instance --- that has gone largely unappreciated in previous mythic interpretations of the {\sl Rāmāyaṇa}....” 

\hfill Pollock (2007:41)
\end{myquote}

Having assembled the essential building blocks, he is now in a position to explore the “mythological map of experience charted by the {\sl Rāmāyaṇa}”--- how the {\sl Rāmāyaṇa} “tells the culture what it is.” 

So, first of all, Pollock --- with his “traditional lens”--- examines the nature of Rāma. Accordingly, he first notices that the traditional audience of the {\sl Rāmāyaṇa} has never had any doubt about the divinity\index{divinity} of the hero and its integral role in the larger narrative. Western scholars have, contrarily, always questioned it. In order to understand this, Pollock proposes to study the “higher-order narrative features” in the logic of the story or within the context of the larger motifs and themes of the Indian tradition. In his words,

\begin{myquote}
“We may come closer to deciding the issue in question if we direct our attention to the poem’s “structured” message residing in certain higher order narratives”
\hfill Pollock (2007b:20)
\end{myquote}

One of these “higher order” logics, according to Pollock, is the boon of Rāvaṇa which is, he says, inextricably linked to the divinity of Rāma. In Vālmīki’s {\sl Rāmāyaṇa}, Rāvaṇa had earned, by means of his ascetic mortifications, a boon that made him invulnerable to all divine and semi-divine beings.  Pollock quotes Prahasta’s words: “Gods, {\sl dānavas, gandharvas, piśācas}, divine birds, and serpents are utterly incapable of harming you in battle --- what of monkeys?” ({\sl Rāmāyaṇa} 6.8.2) [{\sl Trans. Pollock}] And what indeed of men, asks Pollock. Rāvaṇa did not bother to request invulnerability from men and other lower forms of life because to him, it was superfluous --- they were harmless in his eyes, “nothing more than food”. So, a God could not slay Rāvaṇa --- because of the boon. A mere man could not {\sl possibly} kill him. By the logic of the narrative, Pollock says, we are encouraged to believe that Rāma is neither simply a man, nor simply a God, but “an intermediate being who partakes of both existential realms, combining the nature derived from each into a new, superordinated power --- a god-man”\index{god-man} (Pollock 2007b: 28).  

Once Pollock establishes the nature of Rāma as a “god-man”, he proceeds to discover the implications of Frye’s\index{Frye, Northrop} “authoritative social function” of Rāma’s part-mortal, part-divine nature. He writes, 

\begin{myquote}
“Finally, what makes the adaptation of the ancient motif particularly suggestive, complex, and powerful in the {\sl Rāmāyaṇa} is that this second-order being, this divine human or mortal god, is here coupled with a socio-political representation of everyday life in traditional India: such intermediate beings, gods who walk the earth in the form of men, are kings.”
\hfill  Pollock (2007b:43)
\end{myquote}

It is interesting how Pollock takes a gigantic leap from the divinity of Rāma to the divinity of a king in general. In order to do this, he uses the tale of Dhundhumāra as his diving-board. In this tale, the protagonist is an earthly king (like Rāma, Pollock adds). 

\begin{myquote}
“The aged Ikṣvāku king Bṛhadaśva, having set his son Kuvalāśva on the throne, retires to the forest. The sage Uttaṅka tries to stop him, seeking the king’s protection from the {\sl rākṣasa} Dhundhu, who lies beneath the sands of the ocean Ujjanaka practicing austerities in order to destroy the worlds, the thirty gods, and Viṣṇu himself. “For the gods cannot slay him,” Uttaṅka explains, “nor can {\sl daityas} or {\sl rākṣasas}, great serpents, {\sl yakṣas} or {\sl gandharvas}--- no one, for he once received a boon from the  Grandfather of the world.”

The king is asked to slay the demon “for the good of the worlds”, and Uttaṅka tells him further that Viṣṇu shall augment his power by means of his own divine might, thanks to a boon the god once granted the sage. But the aged king, having renounced all violence, declines to do the deed himself and directs the sage to his son. Kuvalāśva and Uttaṅka proceed to the ocean and then, “The Blessed One, Lord Viṣṇu, entered Kuvalāśva with his fiery power at the direction of Uttaṅka, and for the good of the world.” By drinking up the tidal wave caused by the demon’s earthquake, and with the water putting out the fire within, the king, “a great {\sl yogin} by means of Viṣṇu’s yoga,” kills the volcanic Dhundhu (and so receives the name Dhundhumāra)”.\index{Dhundhumara@Dhundhumāra}
\hfill Pollock (2007b: 41-42)
\end{myquote}

Pollock compares this to the motif of Rāvaṇa’s boon --- here, too, the logic of the narrative points, says Pollock, to the incapacity of the gods, or of any other divine, semi-divine being to confront the monster. In Pollock’s words,

\begin{myquote}
“Another creature--- man--- is required; but being naturally powerless man needs the infusion of Viṣṇu’s power. Filled with divine potency, this extraordinary new creature, the earthly king--- and only he, no god or man--- can protect the brahminical world-order by destroying evil”

\hfill Pollock (2007b:42)
\end{myquote}

For Pollock, Dhundhumāra and Rāma are equally a “second-order” of being --- heroes who are men --- but “more than ordinary men, not more than gods”. Furthermore, Pollock uses the stories of Sunda and Upasunda, Tāraka and Skanda, Bali and Vāmana, and Hiraṇyakaśipu and Narasiṁha to draw parallels and demonstrate that it is this adaptation of the “ancient motif” of “divine human, mortal god” that makes the {\sl Rāmāyaṇa} a powerful “myth”. 

From Pollock’s logic, one would have to conclude that: 
\begin{itemize}
\item[(a)] Rāma’s divinity is identical to divinity of Indian kings 
\item[(b)] Both [Rāma, or the kings] are actually men who are “more than ordinary men, not more than gods”.  
\end{itemize}

\newpage

Let us classify the stories Pollock has quoted into two groups to examine closely the motifs he has chosen to analyze: the stories of Rāvaṇa-Rāma, Bali-Vāmana, and Hiraṇyakaśipu into one group, and the other stories (of Sunda-Upasunda and Tāraka-Skanda) into another. We can see that there are many similarities in the two sets of stories --- in both, the villain possesses a unique boon of near- immortality. As a consequence, their actions terrify the three worlds into turmoil, and divine intervention becomes unavoidable. 

Yet, an important difference exists between the two sets of stories --- in the first set, Lord Viṣṇu is directly appealed to, and it is he takes who takes the matter into hand; in the second set, a ruse is created by the different gods to bring the downfall of the villain (in the first story, a celestial woman is created, in the other, a celestial boy). Lord Viṣṇu’s {\sl avatāra}\index{avatara@avatāra} is not equivalent to the “{\sl ruse}” that is “emanating from the gods and yet not one of them” (Pollock 2007b:36). In the {\sl Rāmāyaṇa}, the {\sl Bāla-kāṇḍa}\index{Balakanda@Bāla-kāṇḍa} narrates the story of the gods’ appeal to Lord Viṣṇu and his promise to come down as a human to vanquish Rāvaṇa. 

\begin{myquote}
\textbf{Gods addressing Viṣṇu:} ({\sl Rāmāyaṇa} 1.15.24-25).   

{\sl vadhārthaṁ vayam āyātās tasya vai munibhiḥ saha} | 

{\sl siddha-gandharva-yakṣāś ca tatas tvām śaraṇaṁ gatāḥ} || 24 ||

“We {\sl siddha}-s, {\sl gandharva}-s and {\sl yakṣa}-s, along with ascetics, have hence come here to devise ways of his [Rāvaṇa’s] death. We take refuge in you”

{\sl tvaṁ gatiḥ paramā deva sarveṣāṁ naḥ parantapa} |

{\sl vadhāya deva śatrūṇāṁ nṛṇāṁ loke manaḥ kuru} || 25 ||

“O tormentor of enemies, O Viṣṇu, you are the supreme refuge for all of us. Resolve to be born in the world of men for the destruction of enemies of the gods ({\sl rakṣasa}-s)”
\end{myquote}

\medskip
\begin{myquote}
\textbf{Lord Viṣṇu (to Gods):} ({\sl Rāmāyaṇa} 1.15.28-30)

{\sl bhayaṁ tyajata bhadraṁ vo hitārthaṁ yudhi rāvaṇam} |

{\sl sa-putra-pautraṁ sāmātyaṁ sa-mitra-jñāti-bāndhavam} ||28||

{\sl hatvā krūraṁ durādharṣaṁ devarṣīṇāṁ bhayāvaham} |

{\sl daśa-varṣa-sahasrāṇi daśa-varṣa-śatāni ca} ||29||

{\sl vatsyāmi mānuṣe loke pālayan pṛthivīm imām} | 

“Abandon fear, blessings upon you; for your welfare, for gods and {\sl ṛṣi}-s, I will dwell on this Earth --- in the world of men --- ruling for ten thousand ten hundred years, after killing the cruel and Rāvaṇa in battle along with his sons and grandsons, along with his ministers, along with his friends, relations and allies”
\end{myquote}

{\sl Avataraṇa}, literally a “crossing down”, refers to the “Descent” or “Incarnation” of a divine Person, who is therefore called an “{\sl avatāra}”: it is interesting to note that the same word, {\sl avataraṇa}, is used of the entry of an actor upon the stage, which is an appearance from behind a curtain and a “manifestation” analogous to that of the {\sl avatāra} upon the world-stage. Such “Descents” are explained in the words of Kṛṣṇa\index{Krishna@Kṛṣṇa} spoken to Arjuna in the {\sl Bhagavad-gītā}: \index{Bhagavadgita@Bhagavad-gītā} “Whenever Order fails and Disorder arises, then do I bring forth myself: for guarding the doers-aright and for the destruction of evildoers and to establish Order, I take birth aeon after aeon” (4.6-7). In Aurobindo’s words,  

\begin{myquote}
“It is the descent of God on earth in human form through human birth ... In the West, this belief has never really stamped itself upon the mind because it has been presented through exoteric Christianity as a theological dogma without any roots in the reason and general consciousness and attitude towards life. But in India it has grown up and persisted as a logical outcome of the Vedantic view of life and taken firm root in the consciousness of race.”	

\hfill Aurobindo (1995, 19221: 212-213)\index{Aurobindo, Sri}
\end{myquote}

{\sl Avatāra}-s, in Indian tradition, are of different types: {\sl āveśa} (inspired) (Narasiṁha, for example); {\sl aṁśa} (partial) and {\sl pūrṇa} (complete)\endnote{The {\sl Pāñcarātra} classifies {\sl avatāra}-s as follows: (1) {\sl sākṣād} (Direct) (2) {\sl āveśa} (Entrance or Possessed) (3) {\sl vyūha} (Grouped or Arranged) (4) {\sl antaryāmin} (Inner Controller) (5) {\sl arcā} (Worship).}. They are all “Descents” in play --- {\sl līlāvatāra}. The {\sl purāṇa}-s enumerate 14 {\sl manvantarāvatāra}-s, 25 {\sl kalpāvatāra}-s, 4 {\sl yugāvatāra}-s --- of them, ten have attained greater popularity. In a verse, {\sl Padmapurāṇa} enumerates ten of them in a chronological order:\index{Padmapurana@Padmapurāṇa} 

\begin{myquote}
{{\sl matsyaḥ kūrmo varāhaś ca narasiṁho’tha vāmanaḥ}} |

{\sl rāmo rāmaś ca rāmaś ca buddhaḥ kalkī ca te daśa} || ({\sl Mahābhārata} 12.339)
\end{myquote}

Of the purpose of an avatāra, Aurobindo remarks, 

\begin{myquote}
“There are two aspects of the divine births - one is a descent, the birth of God in humanity, the God-head manifesting itself in the human form and nature, the eternal {\sl avatār}; the other is an ascent, the birth of man into the God-head, man rising into the divine nature and consciousness ({\sl madbhāvam}); it is the being born anew in a second birth of the soul. It is that new birth which {\sl avatār}-hood and the upholding of the {\sl dharma} are intended to serve... If there were not this rising of man into the God-head to be helped by the descent of God into humanity, {\sl avatār}-hood for the sake of the {\sl dharma} would be an Otiose phenomenon, since mere Right, mere justice or standards of virtue can always be upheld by the divine omnipotence through its ordinary means, by great men or great movements, by the life and work of sages and kings and religious teachers without any actual incarnation….”

\hfill Aurobindo (1995, 19221: 214)
\end{myquote}

It is a paradox, this combination of two natures, one divine, one human\endnote{Søren Keirkegaard\index{Keirkegaard, Soren@Keirkegaard, Søren} wrote, “The thesis that God has existed in human form, was born, grew up; is certainly the paradox in the strictest sense, the absolute paradox” (Swenson 1936: 31). }. Different questions it raises are: how can God, the divine unoriginate, be born? How can two complete natures (divine and human) be united in one being? How can God remain unchanged in himself and yet be subject to suffering and death? In order to answer these questions, one can only say that it is only by a leap of faith that it can be understood\endnote{In the words of Cyril of Alexandria who used the Greek term {\sl kath' hypostasin} (hypostasis) to refer to the oneness of divine and human natures, “We must follow these words and teachings, keeping in mind what having been made flesh means; and that it makes clear that the Logos from God\index{God} became man. We do not say that the nature of the Word was altered when he became flesh. Neither do we say that the Word was changed into a complete man of soul and body. We say rather that the Word by having united to himself hypostatically flesh animated by a rational soul, inexplicably and incomprehensibly became man.” McEnerney (1986: 39)}. An {\sl avatāra} is certainly fully God and fully man. In Shelley’s \index{Shelley, Persy Byshe} words, “Within this lies the mystery of the God-man”. So, Rāma is not “more than man, less than god”; {\sl he is fully man and fully god}.\index{god-man} 

Dhundhumāra’s\index{Dhundhumara@Dhundhumāra}  story, on the other hand, is of a king --- born a mere man --- infused with the power of Lord Viṣṇu at the request of the sage Uttaṅka to fight Dhundhu, the {\sl rākṣasa}. If Dhundhumāra was, simply on account of his being a king, as powerful or “divine” as Rāma, clearly there was no need for Uttaṅka’s intervention (or Viṣṇu’s, for that matter). Going further, was his father, Bṛhadaśva, also “divine”? What happened to his divinity when he chose to abandon “kingship” and enter into {\sl vānaprastha}? Did it continue with him?  Was it left behind in his office? Was it passed on to Kuvalāśva? Or was Kuvalāśva also born “divine”? Furthermore, according to Pollock’s logic, the other “kings” of the {\sl Rāmāyaṇa} must also have been considered “divine” --- but this is not the case. Not even Daśaratha --- whose office Rāma must occupy --- is understood as “divine”. Similarly, in the other stories, king Bali would have to be an {\sl avatāra}\index{avatara@avatāra} himself, as too, Hiraṇyakaśipu. 
 
Further, Pollock writes that the divine king is a spiritual redeemer who, “not as an intercessor with the gods, but {\sl directly} secures the spiritual welfare of his people”. Accordingly, “a spiritual function adheres to his person that is symmetrical with and finally indistinguishable from his social function, which the king exercises by reason of the divine substance he incorporated” (Pollock 2007b: 50).  And to Pollock, it seems to be precisely “this power to effect spiritual emancipation” that underpins much of the action of the Araṇya-kāṇḍa.\index{Aranyakanda@Araṇya-kāṇḍa} Pollock uses the logic of Rāma’s divinity to extrapolate and accommodate within it a “king’s” divinity. Consider the following paragraph:   

\begin{myquote}
“At the beginning and end of the volume, {\sl Rāma encounters} the two evil monsters imprisoned in horrific forms as a result of curses, and immediately thereafter two people of extraordinary holiness. {\sl The king slays} the monsters, thereby releasing them from their confinement and allowing them to recover their proper place in heaven. Both Śarabhanga the ascetic and the mendicant woman Śabari commit ritual suicide after their {\sl encounter with Rāma}. Śarabhaṅga had put off departing for the world of Brahmā, which he had won by his asceticism, until he had {\sl experienced Rāma}; but “now that we have met”, he {\sl tells the king}, “I will go to the highest heaven, where the gods reside” (3.4.26), whereupon he immolates himself. Śabari has also been {\sl waiting for Rāma}, having been told by her gurus, “One day Rāma shall come to this holy ashram of yours. You are to receive him. Once you have beheld him, you shall go to the highest imperishable worlds” (3.70.11-12)”. After providing him hospitality she, too, destroys herself in the sacrificial fire. 

By direct intervention, then, or {\sl by his mere presence, Rāma}, “the one to whom all creatures pay homage”, offers freedom from the miseries of this world. For the holy ascetic and mendicant, there is nothing further to live for having once experienced him; {\sl the darśana of the divine king} functions as both the ratification of their holiness and the mechanism of their release. The evil monsters for their part are cleansed by the royal punishment {\sl exacted by Rāma}, and so made fit again for heaven. Punishment as a divine institution and instrument of emancipation is standard doctrine not just for traditional Indian political theology but for Vālmīki himself: “When men who have done evil deeds are {\sl punished by the king}, they are purified and go to heaven, just like men of virtue” (4.18.30).”
\hfill Pollock (2007b: 50) [{\sl italics ours}]
\end{myquote}

It is Rāma who encounters the two evil demons but “the king” who slays them. The demons are purified by the punishment and attain heaven because of “the king’s” punishment. Śarabhaṅga and Śabarī have awaited Rāma’s arrival. They are told they will go to the highest worlds by his {\sl darśana} but finally, they are liberated by the “{\sl darśana} of the divine king”. Furthermore, Pollock writes, “Vālmīki’s poem leaves the impression that the political theology is a doctrine in the making and that its consolidation is a principal objective of the poet.” He accuses Vālmīki of indulging in that exercise of which, ultimately, only he is guilty of. 

Finally, Pollock makes use of another “problematic” situation of the Araṇya-kāṇḍa to establish kingship as the main concern of the {\sl Rāmāyaṇa}: the episode of Rāma’s anguish in the face of Sītā’s loss. In his words, 

\begin{myquote}
“The Ayodhyākāṇḍa seeks to establish an innovative definition of the {\sl dharma}, the code of conduct, of {\sl kṣatriyas}: violence as far as possible is to be eschewed in the realm of sociopolitical action. The Araṇyakāṇḍa shows us a different domain of action where this new valuation of {\sl kṣātradharma} is not always applicable... In this realm, the ideal king is prepared to subordinate every consideration of personal welfare and safety to the duty of protecting the {\sl brahmanical} order of society.”

\hfill	 Pollock (2007b:55)
\end{myquote}

Accordingly, Pollock writes that in the {\sl sarga}-s dealing with this episode, Rāma becomes mad, and wandering like a madman, he explicitly renounces the political ethics to which he has hitherto so tenaciously held. Rāma’s loss of self-possession here is as interesting, says Pollock, as also his potential to destroy the world. 

Here, Pollock writes that the Indian tradition offers one prominent explanation of it:  the reading offered by the {\sl Bhāgavata-purāṇa}\index{Bhagavatapurana@Bhāgavata-purāṇa} is viewed as authoritative by the majority of the medieval commentators. Accepting as an authentic feature of the poem Rāma’s status as an {\sl avatāra} of Viṣṇu, the {\sl purāṇa} explains, “God’s incarnation as a mortal in this world is not simply for slaying {\sl rākṣasa}-s, but it is meant to instruct mortals. How else could it be that the Lord, the Self delighting in Himself, should have suffered so because of Sita? The Blessed One, Vāsudeva, is the Self… without attachment to anything in the three worlds. He would not (except for the purpose of such instruction) have experienced that faintheartedness caused by (his attachment to) a woman” (Pollock 2007b:60)

Pollock says that according to a widespread understanding of the poem, Rāma’s behavior throughout is to be taken “as altogether mimetic”--- it is not “real”, but a representation with “explicit didactic function”. The episode of Rāma’s madness, consequently, is to be viewed as a cautionary tale: “The basest of the {\sl rākṣasa}-s came into the woods stealthily, like a wolf, and abducted the princess of Videha. With his brother in the forest, (Rāma) acted the part of a wretched man when separated from his beloved, thereby to illustrate what happens to all who are too much attached to women.” (Pollock 2007b: 64)

It must at once be noted that the “traditional” interpretation that Pollock cites is not the only one offered. Glossing on 3.60.10: 

\begin{myquote}
{{\sl yatnāt mṛgayamāṇas tu nāsasāda vane priyām}} |

{\sl śoka-raktekṣaṇaḥ śrīmān unmatta iva lakṣyate} || 
\end{myquote}

The {\sl Rāmāyaṇa-śiromaṇi}, for example, says : {\sl śoka-raktekṣaṇaḥ saṁbhoga-śṛṅgāra-poṣaka-sarva-saṁpatti-viśiṣṭo rāmaḥ, yatnāt priyāṁ sītāṁ mṛgayamāṇa anvīkṣan api, nāsasāda prāpa, ata eva unmatta iva lakṣyate} |  

Rāma’s lamentation is understood as Vālmīki’s exquisite portraiture of the tender sentiment of pathos. Rāma’s piteous ravings and lamentations consequent to the unbearable grief of separation from Sītā is {\sl vipralambha-śṛṅgāra} that adds to distill the {\sl karuṇa rasa}. 

Pollock a) does not take into consideration all readings/interpretations offered by the tradition b) does not dismiss it entirely, but subtly disparages it --- it appears he accepts it only when it concurs with the position he wishes to stand for. In this particular case, Pollock proposes another interpretation --- his own --- as a counterpoint, “also based on the central problem of the nature of the king” which he finds more compelling. In his words, 

\begin{myquote}
“If, in the first instance, the {\sl Rāmāyaṇa} is an imaginative inquiry into the nature of kingship and the peculiar transcendent nature of the king, it may be useful to think of the apparent reversal of Rāma’s character in response to the abduction of his wife as an extension of this concern.”

\hfill Pollock (2007b:63)
\end{myquote}

Accordingly, this episode of “Rāma’s madness” shows, to Pollock, the “human face” of the kingly god.  In appropriate circumstances, Pollock writes that the terrestrial king {\sl literally becomes} the one or the other god. Under the compulsion of Rāvaṇa’s\index{Ravana@Rāvaṇa} “egregious evil”, Rāma has become Rudra-Śiva.\index{Rudra-Siva@Rudra-Śiva} Like Śiva, he has gone “mad”; like him, he is bent on and is capable of cosmic destruction. So, when an evil-being commits egregious evil, then the king-god becomes Rudra himself: by their evil acts, evil beings turn Rāma into Rudra, and then he harms all, good and bad alike. 

This interpretation of Rāma’s madness reveals “coherence in an otherwise incoherent image” writes Pollock. Thus, it now appears to be indissolubly linked with a “political theology sustained by the notion of a triune godhead to be fully developed in classical Hinduism: The power of a king is infinite indeed, and as easily as he can preserve the world, he can, if provoked, destroy it” (Pollock 2007b: 67) 

But one is left to wonder how exactly a king “assumes” the form of the different Gods, how Rāma “becomes” Rudra. Also, when Rāma is now Rudra, is he still “part-human, part-God”? 

\section{The King - An Ontological Divinity or Functional Divinity?}\label{sec2.5}

It is needless to say that the divinity of kings is different from an {\sl avatāra}’s divinity. A king’s divinity is bestowed on him through a religious ritual --- unless he is anointed and consecrated, he remains a mortal. A king’s divinity stands on a marriage of spiritual and temporal power --- the king and the {\sl purohita} together uphold the moral order, they take an oath of fidelity towards each other and it is then that the king lords the earth and becomes the guardian of {\sl dharma}.

Mostly, it is for his function that a king is {\sl made} divine\index{divine} --- he is put in a position where he must bear the responsibility of the welfare of his people, and this position makes him “pure” as it were, and trustworthy. In the {\sl Vasiṣṭha Dharma Sūtra} (19.48), for example, it is said that kings remain untainted with impurity for they hold Indra’s place ({\sl aindraṁ sthānam āsīnaḥ}). Yet, despite the formal declaration of a king as god, he is not treated as such in everyday affairs. When he removes his official crown and joins other men --- at festivals, sports, wars --- no one recognizes him as a god.  

We see a clear distinction between a {\sl divine} king,\index{divine king} and a {\sl human} king\index{human king} in the context when Soma, the divine king, comes as a guest, it is said that he must be given what is due to him as a guest; in the same breath, it is said that when a {\sl human} king is received as a guest, an ox must be killed for him.  So we see that “human kings” are seen as antithetic to gods, and as mortal compared with immortals\endnote{{\sl Aitareya-brāhmaṇa} 1.15; 8.23f. For the ritual, see {\sl Aitareya-brāhmaṇa} 7.22f; 8.11f; {\sl Śatapatha-brāhmaṇa} 3.2.1.40;5.3.3.9 and 12; 5.4.4.5; {\sl Taittirīya-saṁhitā} 3.5.3.2. etc and for a thorough discussion of origin of kingship, see U N Ghoshal (1923) }. So it is that the divinity of Pṛthu Vainya is not established in spite of the feats he performs because of his virtuosity (when he wanted to walk over the seas, water solidified and let him walk on water: {\sl āpas tastambhire cāsya samudram abhiyāsyataḥ - Mahābhārata} 12.59.123). On the other hand, it is said, “Viṣṇu entered his (Pṛthu’s) body, and so the world bows to this king as to a god among human gods” 

\begin{myquote}
{\sl tapasā bhagavān viṣṇur āviveśa ca bhūmipaṁ} | 

{\sl devavan naradevānāṁ namate yaṁ jagan nṛpam} || {\sl Mahābhārata} 12.59.130
\end{myquote}

We also see in the {\sl Nalopākhyāna} that Nala, a king, stands in the {\sl svayaṁvara} in contrast to the Gods who have assumed his form. Further, most law-books make a distinction between king, gods, and brahmins\index{brahmins} --- according to Gautama, witnesses must take an oath in the presence of {\sl deva-rāja-brāhmaṇa-saṁsad} ({\sl Gautama Dharmasūtra} 13.13). Āpastamba separates the verbal abuse of gods from that of the king {\sl puruṣaṁ devatānāṁ rājñaś ca} ({\sl Āpastamba Dharmasūtra} 1.31.5). 

In the {\sl Rāmāyaṇa} is the suggestion that a king’s “supernatural goodness” makes him a god: “They say a king is human, {\sl rājānaṁ mānuṣaṁ prāhuḥ} (but you), on account of your more than human conduct seem to me to be godlike ({\sl Rāmāyaṇa} 2.102.4) (Rāma to Bharata). Pollock translates this verse thus:  

\begin{myquote}
{\sl rājānaṁ mānuṣaṁ prāhur devatve sammato mama} |

{\sl yasya dharmārtha-sahitaṁ vṛttam āhur amānuṣam} || {\sl Rāmāyaṇa} 2.102.4

\medskip
“Some say a king is but a mortal; I esteem him a god. {\sl His conduct in matters of righteousness and statecraft}, it is rightly said, is beyond that of mere mortals.” 
\end{myquote}

Here, “{\sl yasya}” ought to be translated as “whose”, not “his”. {\sl Dharmārtha-sahitam} is not “{\sl in} matters of {\sl dharma} and {\sl artha}, but {\sl according} to {\sl dharma} and {\sl artha}. Bharata’s\index{Bharata} words will then be: “A king {\sl whose} conduct is according to {\sl dharmārtha}, him deem they a god”. 

On the other hand, a king who neglects the affairs of the citizen will be roasted in airless hell ({\sl Rāmāyaṇa} 7.53.6). Rāma, on the other hand, is divine as the incarnation of Viṣnu, not as being otherwise a god ({\sl Rāmāyaṇa} 6.35.36) {\sl rāmaṁ manyāmahe viṣṇuṁ mānuṣaṁ rūpam āsthitam}. (Mālyavān to Rāvaṇa) According to Pollock, 

\begin{myquote}
“The king is functionally a god because like a god he saves and protects; he is existentially or ontologically a god because he incorporates the divine essence”
\hfill Pollock (2007b: 47)
\end{myquote}

A king cannot ontologically be a god; it is the Royal Consecration, a Soma sacrifice that is mythically connected with the consecration of Varuṇa, or with Indra, that confers the divinity on him. Pollock writes that it is irrelevant that, technically, Rāma is not yet a consecrated king --- it cannot be if he is arguing for Rāma’s divinity on account of his kingship. In the end, Pollock only fastens his own findings on to the “traditional” narrative to project it as the “traditional” reading of the text. 

If fabrication --- the falsification/misrepresentation of data/information ---  in formal academic exercises is academic dishonesty, Pollock, it is clear, is guilty of several.

\theendnotes
