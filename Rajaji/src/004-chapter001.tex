\chapter{My Scientific Autobiography}

\subsection*{The Beginning:}

A crucial event occured when I was in the Intermediate class of the 
Amercan College, Madurai. A friend of mine took me to observe the night 
sky through the College telescope.


It was the best time to observe the Moon since the shadows of the 
mountains were long. When I saw the deep craters and high mountains, it 
was as if I was looking down at the Moon from a height. It was a 
frightening sight. I realized that


There are more things in heaven and earth than are dreamt of in my 
philosophy.


Science is the key to these other things and that determined my 
trajectory in life.


\vspace{-.3cm}

\subsection*{Preparation:}

\vspace{-.2cm}

I joined TIFR in August 1958 after one year in the AEET (BARC, now) 
Training School.  Since my mind was bent on understanding physics at its 
most fundamental level, I first took up the study of Quantum Mechanics, 
since my knowledge of it was not strong. For any question that I asked, 
the answer was in Quantum Mechanics. I sat with LI Schiff's book on 
Quantum Mechanics for many months and mastered it.

Then I turned to Nuclear Physics since that was the most fundamental 
subject at that time. I read Bethe and Morrison and then Blatt and 
Weisskopf. Went to Kailash Kumar and George Abraham for guidance. The 
former put me in contact with many body theory and the latter in contact 
with few body problems. Abraham even suggested a specific problem. He 
asked me to redo the deuteron and triton structure using the recently 
discovered hard-core repulsion between nucleons.

I was not satisfied. I realized that the force between the nucleons 
comes from a deeper layer of reality which can be understood only from 
the then-new area called particle physics. There was no particle physics 
research in TIFR at that time. B M Udgaonkar (BMU) started studying 
hepernuclei which was in between nuclear and particle physics. 
Hepernuclei are nuclei in which one nucleon is replaced by a lambda 
particle. He introduced me to this subject and also to a few excellent 
reviews by Enrico Fermi on quantum theory of radiation and isospin 
symmetry. Udgaonkar was an excellent teacher. He had taught our batch of 
trainees reactor physics and the second batch quantum mechanics. Bhabha 
had sent him to France to learn about reactors, but BMU shifted to 
particle physics after returning.


Soon SN Biswas and LK Pandit joined and real particle physics started in 
the Theory Group. I started reading particle physics and learnt that the 
real theory of particle physics was Quantum Field Theory (QFT). So 
finally I reached the destination of my ``Inward Bound" journey.


I took up Bethe and Schweber's QFT and Jauch and Rohrlich's Theory of 
Photons and Electrons. I really loved the systematic treatment of QFT in 
Wentzel's book. I took it during my vacation in Kamuthi and read it even 
during the long train journeys.


My learning of QFT was systematized and consolidated only after I 
listened to LK Pandit's course of lectures on QFT. I was so impressed by 
his excellent lectures that I felt I achieved ``Enlightenment".


Soon I began to interact with Biswas.


I will divide the account of my work into two parts; Pre-Standard Model 
and Post-Standard Model. The numbers here refer to the list of 
publications at the end.

\section*{Pre-Standard Model}
%~ \vskip -10pt

\vspace{-.2cm}

This can be subdivided into three parts, Hypernuclear Physics, 
SU(3) and Hadronic Resonances and Current Algebra.

%~ \vspace{-\topsep}

\vspace{-.3cm}

\subsection*{Hypernuclear Physics:}

My first paper [1] was in this field. I had learnt hypernuclear physics 
from BM Udgankar and RH Dalitz's papers. After liste\-ning to SN Biswas's 
excellent lectures on integral equations I was impressed by the fact 
that integral equations can be easily solved if the kernel is separable. 
Using this, with SN Biswas's collaboration I could solve the two-channel 
problem of Lambda-Nucleon scattering. We applied it to Gell-Mann's 
global symmetry and proved that global symmetry that equated all the 
meson-baryon coupling constants does not work.

The next two papers [2], [4] were in collaboration with Dalitz in Chicago. 
The first was on the lifetime of the light hypernuclei such as 
Lambda-$H^3$. The binding here is so weak that the life time is not 
expected to be very different from the lifetime of the free Lambda. 
Experiments did not agree with this. This discre\-pancy exists even now 
and the problem is not yet solved!

The second paper on hypernuclear physics was on the bin\-ding of 
Lambda-Lambda hypernuclei. Here we had to do a three-body problem. We 
did a variational calculation with many para\-meters in the wave function. 
It was done using the new IBM computer in Chicago University. One 
punches the Fortran programme on cards and submits it. After several 
hours you are informed of the error in punching. You repeat the process. 
Finally I succeeded and the paper was written.

\vspace{-\topsep}
\subsection*{SU(3) and hadronic resonances:}

At that time the dominant school of thought was the S matrix philosophy 
of GF Chew. Proving Mandelstam's double dispersion relations was 
considered the biggest challenge. Reinhard Oehme lectured to us on the 
many-sheeted S matrix.

AP Balachandran who had already obtained his PhD in Mad\-ras and joined as 
Dalitz's post-doc was frightening students like me by talking about the 
theory of many complex variables and ``The Edge of the Wedge Theorem". He 
was very mathematically oriented.

Those days you either group or disperse. The former led to SU(3) group 
and the later led to Dispersion Relations. An intere\-sting story about 
the proof of Dispersion Relations from Field Theory is the following:

Feynman: What is Dispersion Relation?

Wigner: What is Field Theory?

Chew: What is Proof?

I had mentioned to Dalitz that I would like to work on a pro\-blem nearer 
to the core of Particle Physics. Dalitz agreed and gave me a recent 
preprint from RJ Oakes and CN Yang that had arrived. They had criticized 
Gell-Mann's SU(3) on two counts:

\begin{enumerate}
\itemsep=0pt
\item The mass differences in the baryonic octet and the same in the mesonic 
octet being very large, the decimet baryons Delta (1238), Sigma (1370), 
cascade (1520) and $\Omega^-$(1670) occur as poles on different Riemann 
sheets. So there is no way in which they can move smoothly to emerge as 
a single pole in the SU(3) limit.
\item Because of the large mass differences again, there is no way by which 
the perturbative Gell-Mann-Okubo mass formula can work.
\end{enumerate}

%~ \vskip 1pt
Since I had learnt about the various Riemann sheets of the S matrix from 
Oehme's papers, I could answer the first objection: there is a retinue 
of poles residing in all the Riemann sheets. These were subsequently 
called ``shadow poles". Because of the existence of shadow poles, one of 
the tenets of S Matrix theory which defines a particle as a pole of the 
S matrix must be modified. The whole retinue of poles define the 
particle!


\newpage

Any typical American physicist would have sent this to the Physical 
Review Letters immediately. Dalitz is more conservative and we sent a 
letter to Oakes and Yang and both of us left for Oxford! Meanwhile many 
others published this result. Later our delayed publications came out 
[3], [5]. The second objection can be answered only by detailed 
calculations and that became my thesis [6]. I showed that if the momentum 
is small compared to the inverse of the range of the interaction, 
perturbation theory is valid.

Dalitz, TC Wong amd myself wrote a paper [7] on the hadron Lambda 
(1405). We used a relativistic mutichannel version of Schrodinger 
equation with potential arising from exchange of rho and omega and could 
generate Lambda (1405).

When I returned to Bombay I began thinking about this pro\-blem. By that 
time quark model had come up. The question was: is Lambda (1405) a bound 
state of three quarks or is it a compo\-site of a baryon and a meson? I 
discovered a way of answering this question.

I showed that the hadron Lambda (1405) cannot be a three-quark bound 
state, but it is a composite of a baryon and meson, the so-called 
``molecular hadron". The test was simply that if it were a quark 
composite, the K matrix for meson-baryon scatte\-ring must have a pole but 
such a pole did not exist for K bar-N, pi-Sigma scattering. I talked 
about this result in two conferences, HEP Symposium at Aligarh [10] and 
Matscience Symposium [13], but did not publish in any journal.

I learnt from my friend Sandip Pakvasa that my teacher Dalitz was not 
happy with me. Since earlier Dalitz, Wong and myself had worked on this 
hadron, perhaps he felt that he should be a coauthor in the K-pole 
paper. I wrote to him apologizing for what I did and explaining the 
circumstances in which this happe\-ned. Then I wrote a detailed paper in 
Physical Review [24] ma\-king due references to Dalitz's work and also 
thanking him. This paper contains a possible extension of the K pole 
text to many other hadrons too.

Much later after QCD came up, it was shown in the paper Phys Rev Let, 
114, 132002 (2015) that QCD also supports the conclusion that Lambda 
(1405) is not a three-quark bound state.

In [8] I showed that in contrast to Lambda (1405) the decimet baryons 
cannot be meson-baryon composites, thus showing that the prevalent 
S-matrix bootstrap philosophy was wrong. With SS Vasan I showed the 
stability of the S matrix pole under various paramatrizations of the 
scattering amplitude.

%~ \vspace{-\topsep}

\vspace{-.3cm}

\subsection*{Current Algebra, K decays etc:}

%~ \vspace{-.2cm}

During 69-71, current algebra became the main focus. I wrote a few 
papers connected to Schwinger terms in collaboration with V Gupta 
[15, 16, 20]. This led to the discovery of a fixed pole in virtual Compton 
amplitude [21]. This paper has an interesting history. Rajaraman and 
Sudendhu Roy Chowdhuri had sent out a preprint pointing out a 
discrepancy between a theoretical sum rule and data on deep inelastic 
electron-nucleon scattering data. I could immediately see that they had 
ignored a possible fixed pole which is indicated by our earlier work on 
Schwinger term in Current Algebra. I pointed this out to Rajaraman who 
was visi\-ting TIFR. On his return to Delhi he corrected the preprint and 
published it with SR Choudhury. But they did not acknowledge me for 
pointing out their error!

I reviewed [11] an important paper of Abers, Dicus and Norton who 
derived the radiative correcton to beta decay using Current Algebra.

The paper [17] was written with KVL Sarma and it addressed the question 
of electron-muon universality in K decays. This is a recurrent topic and 
right now this family universality is an impo\-rtant topic in B decays. 
Papers [18], [19] and [23] written in collaboration with a student SC 
Chhajlani and LK Pandit appli\-ed Current Algebra to K decays.

With PP Divakaran and V Gupta I studied the question whe\-ther the 
electromagnetic current could have an I = 2 compo\-nent [12]. The paper 
[27] with PP Divakaran connected the form of the deep inelastic 
structure function with the asymototic behavior of the elastic form 
factor.

\vspace{-\topsep}
\section*{Post-Standard Model}

\subsection*{Gauge Theory:}
\vskip -6pt
I became aware of Yang-Mills (YM) theory by reading J J Sakurai's paper 
in Annals of Physics in 1959. That was the first paper in which YM was 
used in particle physics. Sakurai constructed a gauge theory of strong 
interactions. I continued to be intere\-sted in YM theory from that time. 
Veltman's lecture at Varenna where he talked about the conserved weak 
current impressed me very much and I felt that weak interaction must be 
described by a YM theory. So, when Weinberg's paper on the SU(2)xU(1) 
electroweak theory came out in 1967 I had no doubt that was the correct 
theory. I read the papers of Goldstone, Higgs and Kibble.

In the subsequent two or three years, I lectured on these at various 
places including TIFR. In particular, in June 1971, I gave a series of 
lectures on the gauge theory of weak interactions including Yang-Mills 
theory, Faddeev-Popov ghosts, Higgs mecha\-nism, electroweak theory, GIM 
mechanism etc. It came out as a SINP report [25]. This was the first 
connected account of what became known as the Standard Model, anywhere 
in the world! It even contained my conjecture that the massless YM gauge 
quantum cannot exist as a particle because of the incurable infrared 
divergences (an early suggestion of what became known later as infrared 
slavery and colour confinement). These lectures were given even before 
t'Hooft's proof of renorma\-lizability appeared!

Nevertheless I failed to make any substantial contribution in gauge 
theory. I will not go over the reasons here.

Then came the discovery of asymtotic freedom of YM theory by Gross, 
Wilczek and Politzer and the construction of SU(3) colour gauge theory 
by Gell-mann, Fritzche and Leutwyler to describe strong interactions.

Renormalizabily of YM with SSB and Asymptotic Freedom are the two most 
important discoveries in Quantum Fild Theory after the discovery of 
renormalizability of Quantum Electrodynamics in 1947-49. I missed the 
boat in both, although I was well-placed with potential to contribute. I 
had already studied path integrals which t'Hooft used in his proof and 
was already giving lectures on Wilson's Renormalization Group and 
Callan-Symanzig equations which are the ingredients in the discovery of 
asymtotic freedom by Politzer, Gross and Wilczek.

Although I missed the stage, I was sitting in the front row. I could 
catch their significance as soon as the discoveries came tumbling one 
after another! The years 1971-73 were truly exi\-ting years. It was the 
watershed in the development of High Ene\-rgy Physics.

In [26], I showed that divergences in the higher-order corrections 
calculated in the SU(2)xU(1) theory cancel.

In the First Symposium on HEP at Bombay in 1972, I reviewed the 
electroweak theory [29]. This was the first review of the electroweak 
theory in the country.
 
Actually, until t'Hooft's proof, as far as I know, nobody except Joe 
Schechter in Syracuse University who added a U(1) to cancel the 
strangeness-changing neutral current and myself had taken Weinberg's 
theory seriously. Even after t'Hooft, only a few theorists took it 
seriously. Situation changed dramatically after the discovery of the 
weak neutral current interaction in the CERN experiment by Perkins and 
others in 1973.
\vskip 1pt
In 1969, TIFR's theoretical physics summer school was held at Nainital. 
Some memorable events took place there. Both Geo\-ffrey Chew and Francis 
Low lectured. Chew lectured on S Matrix Theory. I asked him a question: 
Since S Matrix theory addressed only strong interactions, what happens 
to weak and electromagnetic interactions? Chew gazed at the distant 
Himala\-yan peaks visible through the window for a few minutes and simply 
conti\-nued his lecture.

Low lectured on the divergence problem of the Fermi theory of weak 
interaction and described all the methods proposed to deal with the 
problem. This was two years after Weinberg's paper. I asked Low at the 
end of his lectures why was he ignoring the Yang-Mills theory of weak 
interactions. He merely stared at me and refused to answer my question. 
He described seven or eight unnatural ways of solving the weak 
interaction pro\-blem but left out the one way that turned out to be the 
right way. To this day I have not understood how such a thing is 
possible, Low being a very experienced physicist. Somebody said it was 
because Low did not like Weinberg! Tapas Das and myself took notes of 
Low's lectures and brought it out as a TIFR yellow report.
\vskip 1pt
The evolution of the name starting from Gauge Theory is inte\-resting. As 
soon as t'Hooft showed the renormalizability of electroweak theory I 
calculated the radiative correction to muon decay and sent it to 
Physical Review for publication[26]. I had put the title as ``Radiative 
correction to the muon decay in gauge theory". Physical Review changed 
it to ``Weinberg's gauge theory". In 1972 the HEP Conference was held at 
Chicago that I atten\-ded. That is where gauge theory was presented as a 
Rappor\-teur's talk for the first time. BW Lee gave the talk. He called 
the theory as ``Salam-Weinberg gauge theory". Salam and Gell-Mann were 
sitting in the first row and I happended to be sitting in the second row 
just behind Salam and Gell-Mann. As soon as Lee mentioned 
``Salam-Weinberg gauge theory", Gell-Mann gave a nudge to Salam with his 
elbow. Later when the Nobel Prize was given, Glashow's name was added 
and the theory became Glashow-Salam-Gell-Mann theory. This is certainly 
justi\-fied since Glashow was the first to discover that SU(2)xU(1) was 
the correct gauge group for electroweak theory and also he was one of 
the inventors of the Glashow-Iliopoulos-Maiani (GIM) mechanism to remove 
the strangeness changing neutral current.

However I prefer to call it the SU(2) x U(1) Electroweak Theory.


\vspace{-.3cm}

\subsection*{Neutral Current:}

Weak neutral current (NC) interaction is almost as strong as the usual 
charged current (CC) weak interaction but lay undisco\-vered all those 
years. It could have been discovered many years earlier if only the 
experimenters did not listen to some theorists who said NC cannot exist. 
The theorists thought that NC would lead to strangeness changing NC 
decays which were not seen, but they for\-got that there could be 
strangeness nonchanging NC. So the experi\-menters ignored some data which 
were actually due to NC. But the clinching experimental proof was 
possible only after the huge Gargamelle bubble chamber was constructed 
at CERN. Becau\-se of its size they could clearly distinguish a pion from 
muon and this was crucial for the discovery of NC through the absence of 
muon in neutrino collision.

As soon as the discovery of NC was announced, KVL Sarma and myself 
produced the first model-independant analysis of de\-ep inelastic data 
[30, 32, 33]. JJ Sakurai called our equations ``Master Equations". His 
analysis of elastic scattering using the master equations coupled with 
LM Sehgal's of single pion production led to the complete determination 
of NC coupling constants. Later with Sandip Pakvasa, I generalized the 
analysis to include S, P and T neutral current interactions. With KVL 
Sarma I wrote two more papers on this topic [44, 52]. With SH Patil I 
calculated the contribution of neutral current to the decay $K_L \rightarrow
\mu^+ + \mu^-$ [28].
\vspace{-\topsep}
\subsection*{Integrally charged quarks:}

Our work on Integrally Charged Quarks (ICQ) has a curious history. While 
working on the neutral current paper [35] with Pakvasa, I noticed that 
if there are charged spin one partons, deep inelastic structure 
functions will not scale. Probir Roy and myself pointed this out for the 
neutral current [37]. We noticed that scaling will be restored in a 
unified gauge model. This is how we arrived at the Han-Nambu model which 
we gauged. The results were remarkable:

\begin{enumerate}
\itemsep=0pt
\item Although the Han-Nambu quarks are integrally charged, as observed 
through high $q^2$ probes, they behave like the Gell-Mann-Zweig 
fractionally charged quarks (FCQ).
\item Gluons acquire electrical charge and have weak interactions also.
\end{enumerate}

Papers [40], [42] are on this work. I derived these results in a somewhat 
more general way in [43].

Soon we saw a preprint by JC Pati and Abdus Salam who also had the same 
model with ICQ. But they did not notice the second result, namely the 
gluons have electrical charge. We pointed this out to them in a letter 
and immediately, Salam sent a cable ``We were wrong and you are right." 
They also corrected the published version of their preprint. But they 
did not refer to our work at all!

After I joined Madras University I worked with my collabora\-tors 
S~D~Rindani, T Jayaraman and S Lakshmibala and confronted the\- ICQ model with 
experiments on deep inelastic scatte\-ring and electron-positron 
annihilation and also analyzed other consequences of the model 
[54, 61, 63, 64, 65, 66, 70, 71, 78, 79, 82, 83, 86, 87, 88, 98]. In some of these 
papers there were other collabo\-rators: HS Mani, R Godbole, JC Pati, X.-G 
He, S Pakvasa, and NG Deshpande. To this day, the ICQ model has not been 
disproved.

Some of the other works on the broken colour model are [70], [98].

In an important paper [81] with T Jayaraman and SD Rindani it was shown 
that the time- honoured Equivalent Photon Approxi\-mation does not work 
for massive spin-1 charged particles unless modified suitably. This was 
inspired by our work on ICQ model.

Some time ago, I noticed that the model with broken colour solves the 
problem of strong CP violation from which the standard QCD suffers. I 
have not yet published this result.


\vspace{-.3cm}

\subsection*{How I proved three Nobel Laureates wrong!:}

\begin{enumerate}
\itemsep=0pt
\item Our discovery of shadow poles in disproving CN Yang's objec\-tion to 
SU(3) has been already described.
\item The discovery of CP violation in 1964 by Cronin and Fitch created 
quite a lot of excitement. V Gupta returned to TIFR after a stay at 
Caltech and he showed me a Phys Rev Lett paper in which he had proposed 
what looked like a very elega\-nt model of CP violation. He had discussed 
this with Gell-Mann. I spotted a big error that Gell-Mann did not 
notice! This model violated CPT theorem and hence is untenab\-le! This 
paper [9] is the shotest paper I ever published.
\item While working on ICQ model, we showed Salam was wrong. This is 
described above.
\end{enumerate}

I have to balance the above by the following.


\vspace{-.3cm}

\subsection*{My failed attempts:}

\begin{enumerate}
\itemsep=0pt
\item This was soon after I joined TIFR in 1958. In the primordial nucleo 
synthesis there was a gap. In the successive cooking of nuclei starting 
from proton by abrorption of a neutron, there seemed to be a gap at A = 
5. Because $He^5$ is not bound. When I learnt that the hypernucleus 
Lambda-$He^5$ is bound, I thought that is the solution. I discussed it 
with Udgaonkar, but it did not work.
\item After CP violation was discovered from K decays into two pions in 
1964, I thought the ugly CP violation can be avoi\-ded by recognizing that 
since pions are quark-antiquark bound states, Bose statistics for pions 
is only approxima\-tely valid and without Bose statistics, CP violation 
cannot be inferred. This idea too did not work.
\item During 1964-64, with Arvind Kumar who joined me as a studen\-t, I 
undertook a massive calculation aiming to construct a complete theory of 
hypernuclei. Infact Arvind Kumar managed to do an enormous amount of 
calculation. It did not lead anywhere.
\item I tried to generate the nucleon-pion coupling using the quarks of 
which N and pi were made. This also did not lead anywhere.
\item For quite a long time, I imagined quarks to be leptons. That would have 
been the natural explanation for quarks and leptons satisfying the same 
current algebra. I tried to construct a mechanism by which the weakly 
interacting leptons could sometimes exhibit strong interactions, but I 
failed.
\item During 1967-70, I tried to construct strong interactions fr\-om weak 
interactions. by using the divergences of Fermi's weak interaction 
theory. TD Lee had shown how the qua\-dratic divergences could be turned 
to strong interactions. Gell-Mann, Goldberger, Kroll and Karplus wrote a 
nice paper showing how the quadratic divergences in the diago\-nal parity 
and strangeness conserving sector could be identified as the strong 
interactions. My aim was to redo these calculations using the YM theory 
of weak interactions of Weinberg's 1967 paper. But alas! t'Hooft proved 
the reno\-rmalizability of Weinberg's theory. No divergences were left to 
generate strong interactions!
\end{enumerate}

\subsection*{Psi particle:}

In 1974 I was invited by my friend Sandip Pakvasa to visit the 
University of Hawaii, Honolulu. Sandip and myself worked on the paper 
[35]. Since I had gone with my family, I wanted to show them Hawaii. But 
in October of that year all hell broke loose! A very narrow peak at 3.1 
GeV was seen in electron-positron collisions. San Fu Tuan was constantly 
at the phone pumping out information on the new discovery from Stanford 
and other Centres. We wrote up a dozen explanations [34]. Fortunately it 
included the correct explanation: it was a bound state of the new 
charmed quark and a charmed antiquark. Papers [36, 39, 41] on the charmed 
particles were written in collaboration with J Pasupathy and KVL Sarma.


\vspace{-.3cm}

\subsection*{Madras University:}

In 1976 I shifted to University of Madras. I was very much worried since 
this happened when I was at the peak of my caree\-r and feared that my 
academic performance will be afftected. Fortunately this did not happen, 
mainly because of a brilliant physicist V Srinivasan. Using functional 
methods, Srinivasan and myself could show the equivalence of many field 
theories. In parti\-cular we could show the equivalence of Parisi's model 
of quark confinement with QCD! [46, 47, 48, 49, 80] are the papers written 
in collaboration with Srinivasan; in the last paper MS Sriram also was a 
coauthor.

Soon SD Rindani joined as a faculty member and Sriram and JK Bajaj 
joined as Research Associates under my UGC project ``Ga\-uge Theory". Under 
the enlightened leadership of the Vice Chancellor Malcolm Adiseshaiah, 
MSc teaching started. Two good MSc students T Jayaraman and S 
Lakshmibala joined me for PhD. So we had built up an active Theory group 
in the University.

\vspace{-\topsep}
\section*{Activities in IMSc Phase}

\subsection*{Quantum Statistics:}
%~ \vskip -8pt
After I rejoined Institute of Mathematical Sciences after one year at 
TIFR, AK Mishra joined IMSc. He is very good at long algebraic 
calculations. One day he came to my office and showed me a new algebra 
of creation and destruction operators. He constructed this while 
studying Hubbard model where the Coulomb interaction between electrons 
at the same site is so large that only one electron is allowed at one 
site. This was the starting point of our long and fruitful 
collaboration. We constructed a Generali\-zed Fock Space which allowed us 
to discover many new forms of quantum statistics [115]. These were 
Orthostatistics, Null Stati\-stics, Hubbard Statistics and many others 
[102, 106, 107, 108, 108, 116, 111, 112, 113, 118, 120, 123]. Some related work 
with AK Mishra, A Khare and RP Malik was published in [107, 108, 127].

\vspace{-\topsep}
\subsection*{Neutrinos:}
\vskip -8pt
Neutrino Physics came to centre stage after the discovery of neutrino 
oscillations. Most people worked on the toy model of osci\-llations with 
two neutrinos. The Madras group was one of the earliest to work with the 
full three-neutrino oscillations. In [121] and [122] I worked on solar 
and atmospheric neutrino oscillations with MVN Murthy, S Uma Sankar and 
Mohan Narayan. We were the first to give the correct interpreation of 
the null result that came from the CHOOSE reactor experiment [132]. We 
could give an upper bound on the reactor mixing angle $\theta_{12}$. This 
upper bound of 11 degrees was the only information on this mixing angle 
until Daya Bay experiment determined it to be about 9 degrees, not 
far away from our upper bound.


Actually the CHOOSE preprint concluded wrongly that their result 
contradicts the earlier experimental result on atmosphe\-ric neutrinos. 
This was because they used the wrong toy model of two neutrinos. The 
published version removed this statement since we had written to them, 
but they did not refer to us at all!

Since we had an analytical way of doing neutrino propagation in matter, 
Rahul Sinha, Mohan Narayan and myself could do many calculations more 
easily. We could calculate in detail the time-of-night variation of 
solar neutrinos during their passage through the Earth [125]. In 
collaboration with C Burgess we could calculate the Eclipse Effect in 
which neutrinos get regene\-rated during their passage through the Moon 
[124, 126]. So, as observed through a neutrino telescope, the Sun 
appears brighter during the eclipse!

I worked on neutrinos from Supernovae in collboration with MVN Murthy, D 
Indumathi and G Dutta. [135, 140, 143]

Most of this work was reviewed in [130].

Using RG evolution we showed how the neutrino mixing angles which are 
small at high scales evolve to become large at small scales and match 
the experimental values [151, 156, 158, 164]. My collaborators were RN 
Mohapatra, and MK Parida and later SK Agarwalla. In collaboration with G 
Abbas, S Gupta, R Srivastava, MZ Abyaneh and M Patra these calculations 
were pursued with updated input and in one paper, replacing Majorana with 
Dirac neutrinos [181, 182, 185].

In all the above RG work, the neutrino mixing angles were taken to be 
equal to the quark mixing angles at high scale under the assumption of 
lepton-quark unification. Later I realized that this unification 
hypothesis was unneccessary. What was needed was the Wofenstein 
structure for the mixing matrix. This impor\-tant result is in paper 
[190].

In collaboration with PP Divakaran a new mechanism for the tiny neutrino 
masses was proposed [134]. This will make the Higgs boson a composite 
object.

In the context of the alleged superluminal velocity of neutrinos in the 
CERN-GranSasso experiment, we (D Indumathi, Ro\-mesh Kaul, MVN Murthy and 
myself) calculated the group velocity of the three neutrino flavour 
complex and showed it is not superluminal. Later the experimenters 
withdrew their result on superluminal velocity.

The scale of the dark energy and the neutrino mass are comparable. If 
this is not an accidental coincidence, they must be physically related. 
Such a possibily is realized if the neutrino condensate is the origin of 
dark energy: paper [168] with JR Bha\-tt, Bipin Desai, Ernest Ma and Utpal 
Sarkar.

\subsection*{Model Building:}

In collaboration with the model-builder 'par excellance' Ernest Ma of 
University of California, Riverside I did much work on Model building 
for neutrino masses and mixing and also for other things 
[141, 142, 147, 148, 149, 152, 159, 165, 188]. Paper [146] on $A_4$ turned out to 
be very popular, as evidenced by its large citation index.

Papers [155, 157, 160] on SO(10) and seesaw were written in collaboration 
with Bipin Desai, Utpal Sarkar, K Bhattacharya and CR Das.

\vspace{-\topsep}
\subsection*{String Theory:}

I was aware of String Theory almost from its birth. When I was perusing 
the preprint library in KEK, Japan in 1980 I saw the paper of Scherk and 
Schwartz who liberated String Theory from its hadronic context by 
changing the string tension from 1 GeV to $10^{19}$ GeV. That was the birth 
of String Theory. Then in 1984 I was escorting Tullio Regge from 
Bangalore to Madras and he told me the exciting discovery made by Green 
and Schwartz that all the anomalies cancel in the SO(32) and $E_8 \times E_8$ 
Superstring theories.

From then on I learnt whatever was known in String Theory and gave 
lectures on it in various Conferences, Workshops and the SERC School 
[84, 92, 94]. But because of the heavy work involved in the building up of 
IMSc (1984-88) and the turmoil in the Institute in 1989, I could not 
work in String Theory. But I kept up my interest in it because I believe 
that is the Theory for Future incorporating Standard Model and Quantum 
Gravity.

In 1987, I gave a series of lectures on String Theory at University of 
Hawaii, Honolulu. At that time, a strong criticism of string theory by P 
Ginsparg and S Glashow appeared in Physics Today. I answered that 
criticism in my lecture. San Fu Tuan persuaded me to write that up and 
send it to the journal. I agreed to publish it with him as a coauthor. 
In this I speculated that not only one-dimensional strings but 
consistent theories of multi-dimensional objects also exist. Later, as 
is well known, Polchinski discovered the mutidimensional branes as 
solitons in string theory.


The main problem with String Theory is the lack of experi\-mental support. 
That requires construction of accelerarors going upto Plank energy $10^{19}$ 
GeV. Many regard that as impossible. This is a crisis in Physics. But 
human ingenuity knows no bounds and this energy barrier will be crossed. 
New principles of accele\-ration will be discovered. I have been 
emphasizing this for the last 40 years. Laser Plasma Acceleration (LPA) 
is one such and it has been pursued for some time all over the world. I 
have discussed the importance of starting LPA with experts on lasers at 
TIFR, Centre for Advanced Technology, Institute for Plasma Research and 
BARC. All of them met at the International Centre for Theoretical 
Sciences and they are chalking out a plan of action.
\vspace{-\topsep}
\subsection*{Kolar events:}
\vskip -8pt
One morning, my wife who was looking at the Times of India, exclaimed 
``Look, your friend KVL Sarma's name is in the front page!". I looked and 
found she had missed my name. The news item in the front page said G 
Rajasekaran and KVL Sarma have discovered a new particle. The Kolar 
experiments discovered some events that could not be explained. KVL 
Sarma and myself interpreted those events as due to a new paricle. I 
also described this in an article in Physics News. TOI looked at only 
this po\-pular science article and wrote the story. I was flabbergasrted. 
There was no mention of the experimenters (that included MGK Menon). I 
contacted TOI and asked them to withdraw the story or atleast correct 
it. They refused and said I can send a letter to the editor.

TOI could have verifed the authenticity of their story by pho\-ning TIFR. 
They did't. This is the level of science repor\-ting!

Recently at IMSc, MVN Murthy and myself have interpreted the 40-year old 
Kolar events as due to decaying Dark Matter parti\-cles.


\subsection*{Miscellaneous topics:}

A large-N gauge theory of loops was constructed by B Sakita, but he did 
it only for pure YM theory. Hendrik Bohr and myself extended it to the 
matter sector [69], (Hendrik is the grandson of Niels Bohr's brother 
Harald Bohr).

A unified treatment of Bohm-Aharanov effect for electromagnetic field 
and Collela-Overhauser effect for gravitational field in 
five-dimensional Kaluza-Klein theory was given in [193]. Later it was 
extended to include Berry phase [119].These works were in collaboration 
with R Parthasarathy and R Vasudevan.


The consequences of noncommutative Standard Model were worked out for 
some physical processes: papers [166, 169] with PK Das, NG Deshpande, SK 
Garg and T Shreecharan.


HEP is moving through greater depths down to $10^{-33}$cm in attempting 
to encompass Quantum Gravity. In this venture, will Quantum Mechanics 
remain valid for ever? In what way can it be modified? I discussed this 
question 30 years ago [192].

\vspace{-\topsep}
\subsection*{Reviews:}
\vskip -8pt
I have spent a considerable amount of time and energy in giving review 
talks and writing review articles. Some of these are [51] which are 
Panchgani lectures on Gauge Theory, [91] which is a course of lectures 
on the construction of the Standard Model, [92, 94] on String Theory, 
[100] on electroweak symmetry, and general reviews on the state of HEP 
[90, 104, 114, 128, 129, 175]. Article [162] traces the panoramic history of 
HEP while [163] gives the history of the establishment of scientific 
institutions in South India during the British period.

Some topical reviews are in [171, 172, 174, 176, 177 and 187]

\vspace{-\topsep}
\subsection*{Is God subtle?}
\vskip -8pt
Einstein said: ``Subtle is the Lord; malicious He is not." Let us analyze 
whether the Lord is really subtle.


In the 60's and early 70's, there were many subtle and sophi\-sticated 
ideas on how to solve the problem of hadrons -S Matrix theory, currents 
as coordinates, infinite component wave equations and many more. I have 
already mentioned some of these above, as my failed attempts.


The simplest interpretation of the Sakata-Gell-Mann-Nee\-man SU(3) in 
terms of a triplet of quarks as the building blocks of all hadrons 
turned out to be right although most physicists took a long time to 
realize it.


After the success of the YM paradigm in the electroweak sector a 
mindless repetition of the same in the strong sector appeared naive, but 
that turned out to be the correct solution for the strong interaction. 
That is QCD.


After the neutral current was discovered, I had hoped that Nature would 
spring a surprise. I had thought a more subtle manifestation of SU(2) X 
U(1) symmetry without gauge bosons would be the truth. But W and Z were 
discovered precisely at the masses prdicted by theory.


Finally I had hoped that Higgs boson would not be discove\-red since the 
actual mechanism of spontaneous symmetry brea\-king could be more subtle 
than what Higgs and Kibble had imagi\-ned. Again I was wrong.


Hence the question: Is God subtle?


Standard Model is not the end of the story. Maybe the subtle and 
sophisticated ideas will have their day when we go deeper in our INWARD 
BOUND journey.

\subsection*{Chennai Mathematical Institute (CMI):}

Seshadri founded CMI with Mathematics and Theoretical Computer Science. 
Even before IISERs came, Seshadri admitted into CMI talented students 
after school so that they can pursue their studies in an atmosphere of 
research. He wanted CMI to grow into a full-fledged University and as a 
first step wanted to have Physics. He asked me to help in Physics 
Faculty recruitment and teaching. I have been doing that. We now have a 
Theoretical Physics Group of outstanding young faculty members.

\newpage

Some of the other institutions in whose development I pla\-yed a role as a 
member of their Governing Council or other bo\-dies are Harishchandra 
Research Institute, Institute of Physics, Saha Institute of Nuclear 
Physics, Inter-University Centre for Astro\-nomy and Astrophysics, Indian 
Institute of Astrophysics, SN Bose National Centre for Basic Sciences 
and IISER, Thiruva\-nanthapuram.
\vspace{-\topsep}
\subsection*{Teaching and organization of Schools:}

Apart from teaching full courses first at TIFR and Madras
Univer\-sity and then at IMSc and CMI, I have been involved in
conside\-rable teaching at various other Centres.


Academies-organized Refresher Courses in many Colleges in
Tamil Nadu, Kerala and Karnataka took up a lot of my time and
energy. For many of them I was the Director.


Sunday classes (venue:Department of Nuclear Physics, Mad\-ras University) 
were started by MV Satyanarayana of Pondiche\-rry University with the 
help of Joseph Prabagar of Loyola Colle\-ge. I joined the team and taught 
on every Sunday for many years. The Sunday classes completed twenty 
five years recently. Students came from as far away as Dindigul and 
even from Andhra Pradesh.


I was involved in the running and teaching of the DST-SERC
Schools in Theoretical High Energy Physics (THEP) for more
than 10 years. N Mukunda was in-charge for the first 5-year
cycle and I took over for the next 5 years. Meticulous planning
was done one year before the course. Resource persons who were
selected to give the courses were given detailed instructions
about what is to be taught. As a result, the SERC Schools in
THEP were immensely successful in providing the much-needed
graduate level teaching that was not available in our Universities.


SERC Schools in THEP became the models for SERC Schools in other
subjects. But in other subjects such as Nuclear Physics and
Condensed Matter Physics, no such meticulous planning was do\-ne.
One resource person gave one or two seminar-type lecture and the
next person gave a seminar unrelated to the earlier one. As a result
these were not as successful as the Schools in THEP.


Inspired by the highly successful Mathemamatics Teaching for Talented 
Students (MTTS) conducted by Kumaresan (of Hyderabad University), I 
initiated the Physics Teaching for Talen\-ted Students (PTTS). I selected 
M Sivakumar Of Hyderabad University, MV Satyanarayana of Pondicherry 
University and Raghu Rangarajan of Gujarat University to be in charge 
of PTTS and they are doing it successfully with innovative teaching 
methods.


Courses in High Energy Physics were given by me at IISER-TVM, 
IISER-Mohali, Banares Hindu University, Madurai Kamaraj University and 
many other Universities.


\subsection*{Popular Science in Tamil:}

I strongly believe that Popular Science will succeed only if it is done 
in the mothertongue of the people. My ambition to write Science in Tamil 
fructified through the kindness of my friend Dr Jeyapragasam who was the 
Editor of a Tamil monthly published from Madurai. I wrote every month 
and brought out two volu\-mes containing my articles.

\vspace{-\topsep}
\subsection*{Raju Raghavan:}

Raju Raghavan was a great experimental physicist. Although I knew him 
earlier since he was a second batch trainee, we becam\-e friends only 
after I came to Madras. He used to visit me whenever he came from USA 
and we discussed neutrinos. He had many origina\-l ideas on neutrino 
detection, including Mossbauer resonance absorption and emission of 
neutrinos. If one succeeds in this, neutrino experiments can be done on 
a table-top!


Later, after INO was conceived he was its enthusiastic promotor. He had 
conceived a detector of solar low energy neutrinos, called LENS (Low 
Energy Neutrino Spectrometer) which can revolutionize solar neutrinos 
physics. He wanted to do the experi\-ment in India. I took him to meet the 
secretaries of DAE and DST and they agreed to support him.


But Raghavan passed away suddenly in 2011. I was shocked and took a long 
time to recover. Actually at that moment when I heard the news, I was 
arranging a major meeting of Raghavan with scientists and science 
administrators. His death is a serious loss to Indian Science. India 
must take up the LENS Project.

\vspace{-.2cm}

\subsection*{India-based Neutrino Observatory (INO):}

India was a pioneer in neutrino experiments.\ The very first observation 
of cosmic ray produced neutrinos called atmospheric neutrinos was made 
in India, in the Kolar Gold Field (KGF) mines. That was in 1965. But the 
mines were closed in the 90's. Since there was not much gold, the 
Bharath Gold Mines company decided to close it. We should not have let 
that happen. Science is more precious than gold! It is these atmospheric 
neutrinos whose further study by the Japanese physicists yielded two 
Nobel Prizes, in 1998 and 2002. We clearly missed the boat.

Can we recover this lost initiative? We can and we must. The INO was 
conceived with this aim in view.

It was conceived in IMSc in the year 2001, but it has not still seen the 
light. It was approved by all the Central Government bodies and the 
Government granted Rs 1600 crores for the pro\-ject. This involves the 
construction of a 50,000 ton magnetised iron calorimeter detector for 
atmospheric neutrino studies. This will be installed inside a mountain 
in Theni District. The nerve-centre of INO will be in the outskirts of 
Madurai City and will house R and D of particle detectors with training 
facilities for students. This has been named Inter-Institutional Centre 
for High Energy Physics (IICHEP).

Apart from the study of atmospheric neutrino oscillations, INO lab will 
house experiments searching for Neutrinoless Double Beta Decay (NDBD) 
and Dark Matter (DM). CVK Baba and myself played some role in initiating 
the NDBD activity. As a consequence two or three groups involving 
Vandana Nanal, RG Pillay, PK Raina and PK Rath are involved in 
feasibility studies for the NDBD project. I tried to initiate work on 
Dark Matter Search through Rupak Mahapatra of Texas A and M and 
physicists at SINP. This could have been a major project but it did not 
succeed. Instead a minor Dark Matter project at a shallow depth in the 
Jaduguda mines has been started.

Along with others I have lectured on INO to students in colleges and 
schools and villagers as a part of of the INO's outreach programme. This 
is continuing.

Some ``wise men" of Tamil Nadu blocked INO citing non-exi\-stent 
environmental and other imaginary dangers. This obscurantist propaganda 
must be fought and INO must succeed. Truth has to triumph.

