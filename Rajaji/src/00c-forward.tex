\chapter{Foreword}

For almost sixty years now, Guruswamy Rajasekaran – GR – has been an 
inspiring and leading member of the theoretical high energy physics 
community of India. This autobiographical volume – personal life 
history followed by his scientific work – des\-cribes his career in a 
refreshingly lively, direct and honest way. Born and schooled in a 
small South Indian town, Kamuthi, near Madurai, his family had modest 
means – his father owned a shop selling brass vessels, and he himself 
was the eldest of ten siblings. (As soon as he started earning on his 
own, he began contributing to the family kitty.) The descriptions of 
people and events around him, so well remembered, give a clear picture 
of what life was like in a small town in that part of India in the 
years just before and immediately after India’s independence in 1947. It 
is quite remarkable that his teachers in the school years in Kamuthi 
did such a fine job that he could go on to the American College in 
Madurai and then to the Madras Christian College in Madras, leading 
institutions of those days, for the Intermediate and College years.

After this, he gained admission to the first batch of physics students 
in the Training School of the Atomic Energy Establishment in Bombay, a 
brain child of Homi Bhabha, in 1957-58. Tha\-nks to his innate abilities 
and excellent training, he topped his class by a wide margin. This 
allowed him to choose to join the Tata Institute of Fundamental 
Research, TIFR, another creation of Bha\-bha, as a research physicist. 
His early mentors were Geo\-rge Abraham, S N Biswas, B M Udgaonkar and L 
K Pandit. After three years there, he was sent by TIFR to the 
University of Chicago to work for a PhD under Richard Dalitz’s 
guidance. This was completed in 1964, the last year spent in Oxford 
with Dalitz, and then GR returned to TIFR.


From the latter part of the 1960’s onwards, I was his junior colleague 
at TIFR and watched with both admiration and envy all that he was able 
to accomplish. Younger generations since then can scarce believe that 
we worked in a world without photo\-copying facilities, no phones at home 
or in individual offices, no faxes or email or internet. All these 
would come gradually many years later. And to attend national level 
seminars and confe\-rences we would often travel by long distance trains 
over many days each way.

GR describes his varied and evolving interests in high energy physics 
over the years in some detail. He has always been very perceptive in 
following the most recent advances, and has deve\-loped a keen 
sensitivity to phenomenological aspects of particle physics. In 
addition he has always displayed great imagination and inventiveness in 
his work. The range of areas\- he has worked in is quite astonishing – 
physics of hypernuclei, subtle aspects of flavour SU(3), current 
algebra, neutral currents in weak interactions, the Han Nambu integer 
charge quark model, the Yang Mills idea and the revolution it led to 
via the Standard Model, neutrino physics, string theory, and new forms 
of quantum statistics. If I had to choose just two of his best pieces 
of work, they would be his model independent analysis of neutral 
current processes with K V L Sarma; and the work with Probir Roy on 
the integer charge quark model which gets subtly transmuted to 
observable fractional quark charges.


GR has forged very fruitful collaborations with many physicists in 
India and abroad. Of these I must mention at least K V L Sarma, 
Virendra Gupta, P P Divakaran, Probir Roy, Sandip Pakvasa, V 
Srinivasan, A K Mishra, M V N Murthy, and Ernst Ma. He has given 
lecture courses on current topics any number of times, apart from 
teaching at graduate school level. His lectures on the nonabelian gauge 
theory of electroweak unification at the Saha Institute of Nuclear 
Physics in 1971 is probably the earliest such course anywhere in the 
world.


In later years he moved from the TIFR to the University of Madras in 
1976, then to the Institute of Mathematical Sciences, IMSc, in Madras 
around 1984. In between there was a visit to Hawaii, and a two year 
stay at KEK in Japan. The initial years at IMSc were difficult, indeed 
turbulent, and only after 1990 did things settle down. From about 2001, 
he has been involved with the Chennai Mathematical Institute, and with 
the India-based Neutrino Observatory, INO, in a major way.


All through his career GR has been an exemplary teacher, especia\-lly at 
specialized schools for advanced students and tea\-chers.  


Compared to the days when GR (and I) started on a career in physics, 
there have been enormous changes, not the least the crea\-tion of many 
very high quality institutions devoted to teaching and research. But it 
is important – especially for younger generations – that we remain 
aware of the conditions in which past generations lived and worked, 
and out of which the present has grown. For this we owe thanks to GR 
for having put together his memories and views in such engaging fashion, 
for all that he has achieved, as well as all he has done to help 
others grow.
\vskip 1cm

\begin{flushleft}
\begin{tabular}{l@{\phantom{WWWWWWWWW}}c}
Bengaluru & \quad N Mukunda\\
27 March 2022 & Emeritus Professor, IISc
\end{tabular}
\end{flushleft}
