\documentclass{beamer}

\usepackage{txfonts}
\usepackage{hyperref}
\usepackage{fancybox}
\usepackage{xfrac}
\usepackage{cancel}


\newcommand{\heart}{\ensuremath\heartsuit}

\usepackage{mathtools,amssymb}
\newcommand{\myarrow}{\scalebox{2}[2]{$\mathclap{\curvearrowleft}\mkern2.2mu
                                                 \mathclap{\curvearrowright}$}}

\DeclareMathOperator{\Bin}{\mathrm{Bin}}

\hypersetup{colorlinks=false,linkbordercolor=red,linkcolor=green,pdfborderstyle={/S/U/W 1}}

\addtobeamertemplate{navigation symbols}{}{ \hspace{1em}    \usebeamerfont{footline}%
    \insertframenumber / \inserttotalframenumber}

\geometry{papersize={15cm,13cm}}
\usepackage{lipsum}

\makeatletter
\newenvironment<>{contdproof}[1][\proofname]{%
    \par
    \def\insertproofname{#1\@addpunct{.}}%
    \usebeamertemplate{proof begin}#2}
  {\usebeamertemplate{proof end}}
\makeatother


\setbeamertemplate{theorems}[numbered]

\newtheorem*{nonumdefinition}{Definition}
\newtheorem*{nonumproblem}{Problem}
\newtheorem*{nonumtheorem}{Theorem}
\newtheorem*{nonumproof}{Proof}
\newtheorem*{nonumremark}{Remark}
\newtheorem*{answer}{Answer}
\newtheorem*{nonumremarks}{Remarks}
\newtheorem*{nonumexamples}{Examples}
\newtheorem*{nonumsolution}{Solution}
\newtheorem*{nonumexample}{Example}
\newtheorem*{nonumproposition}{Proposition}
\newtheorem{proposition}[theorem]{Proposition}


\usepackage{tikz}
\newcommand*\mycirc[1]{%
  \tikz[baseline=(C.base)]\node[draw,circle,inner sep=.7pt](C) {#1};\:
}

\newcommand\myheading[1]{%
  \par\bigskip
  {\color{blue}{\large #1}}\par\smallskip}

%\usetheme{Warsaw}
%\usetheme{Berkeley} %sample 1

\usetheme{Berlin} % sample 2
%\usetheme{AnnArbor} % sample 3

\let\otp\titlepage
\renewcommand{\titlepage}{\otp\addtocounter{framenumber}{-1}}

\title{Lecture 29: The confidence interval formulas for the mean in an normal distribution when $\sigma$ is unknown }
\author{}
\date{}

\begin{document}
\begin{frame}[plain]
\titlepage
\end{frame}

\begin{frame}
\myheading{1. Introduction}

In this lecture we will derive the formulas for the symmetric two-sided confidence 
interval and the lower-tailed confidence intervals for the mean in a normal distribution
\textit{when the variance $\sigma^2$ is unknown}. At the end of the lecture I assign the problem
of proving the formula for the upper-tailed confidence interval. We will need the
following theorem from probability theory. Recall that $\overline{X}$ is the sample mean (the
point estimator for the populations mean $\mu$) and $S^2$ is the sample variance, the point
estimator for the unknown population variance $\sigma^2$.

We will need the following theorem from Probability Theory.

\begin{theorem}
$(\overline{X} − \mu)/\dfrac{S}{\sqrt{n}}$ has $t$-distribution with $n - 1$ degrees of freedom.
\end{theorem}
\end{frame}

\begin{frame}
\myheading{2. The two-sided confidence interval formula}

Now we can prove the theorem from statistics giving the required confidence interval
for $\mu$. Note that it is symmetric around $\overline{X}$. There are also asymmetric two-sided
confidence intervals. We will discuss them later. This is one of the basic theorems
that you have to learn how to prove.

\begin{theorem}
The random interval $T = \left(\overline{X} - t_{\alpha/2, n-1} \dfrac{S}{\sqrt{n}}, \overline{X} + t_{\alpha/2, n-1} \dfrac{S}{\sqrt{n}} \right)$ is a $100(1 - \alpha)\%$-confidence interval for $\mu$.
\end{theorem}

\begin{nonumproof}
We are required to prove
$$
P \left(\mu \in \left(\overline{X} - t_{\alpha/2, n-1} \frac{S}{\sqrt{n}}, \overline{X} + t_{\alpha/2, n -1} \frac{S}{\sqrt{n}} \right) \right) = 1 - \alpha.
$$
We have 
\end{nonumproof}
\end{frame}

\begin{frame}
\begin{proof}[Proof {(Cont.)}]
\begin{align*}
\text{LHS } & = P\left(\overline{X} - t_{\alpha/2, n-1} \frac{S}{\sqrt{n}} < \mu, \mu < \overline{X} + t_{\alpha/2, n-1} \frac{S}{\sqrt{n}} \right)\\
& = P \left(\overline{X} - \mu < t_{\alpha/2, n -1} \frac{S}{\sqrt{n}}, -t_{\alpha/2, n-1} \frac{S}{\sqrt{n}} < \overline{X} - \mu \right)\\
& = P \left(\overline{X} - \mu < t_{\alpha/2, n -1} \frac{S}{\sqrt{n}}, \overline{X}-\mu > - t_{\alpha/2,n-1} \frac{S}{\sqrt{n}}\right)\\
& = P \left( (\overline{X} - \mu) / \frac{S}{\sqrt{n}} < t_{\alpha/2, n-1}, (\overline{X} - \mu) / \frac{S}{\sqrt{n}} > - t_{\alpha/2, n-1} \right)\\
& = P\left(T < t_{\alpha/2, n-1}, T> - t_{\alpha/2 , n-1} \right) = P \left(-t_{\alpha/2, n-1} < T <t_{\alpha/2, n-1} \right) = 1 -\alpha
\end{align*}
To prove the last equality draw a picture.
\end{proof}
Once we have an actual sample $x_1, x_2, \ldots, x_n$ we obtain the observed value $\overline{x}$ for
the random variable $\overline{X}$ and the observed value $s$ for the random variable $S$. We
obtain the observed value (an ordinary interval) $\left( \overline{x} - t_{\alpha/2, n-1} \dfrac{s}{\sqrt{n}} , \overline{x} + t_{\alpha/2, n-1} \dfrac{s}{\sqrt{n}} \right)$ 
\end{frame}

\begin{frame}
for the confidence (random) interval $\left(\overline{X} - t_{\alpha/2, n-1} \dfrac{S}{\sqrt{n}}, \overline{X} + t_{\alpha/2, n-1} \dfrac{S}{\sqrt{n}} \right)$ The observed value of the confidence (random) interval is also called the two-sided $100(1 - \alpha)\%$ confidence interval for $\mu$.

\myheading{3. The lower-tailed confidence interval}

In this section we will give the formula for the lower-tailed confidence interval for $\mu$.

\begin{theorem}
The random interval $\left(-\infty, \overline{X}+ t_{\alpha, n-1} \dfrac{S}{\sqrt{n}} \right)$ is a $100(1-\alpha)\%$-confidence
interval for $\mu$.
\end{theorem}

\begin{nonumproof}
We are required to prove
$$
P\left(\mu \in\left(-\infty, \overline{X} + t_{\alpha, n-1} \frac{S}{\sqrt{n}} \right)\right) = 1 -\alpha.
$$
\end{nonumproof}
\end{frame}

\begin{frame}
\begin{proof}[Proof {(Cont.)}]
We have
\begin{align*}
\text{LHS} & = P\left( \mu < \overline{X} + t_{\alpha, n -1} \frac{S}{\sqrt{n}} \right) = P \left( -t_{\alpha, n-1} \frac{S}{\sqrt{n}} < \overline{X} - \mu \right)\\
& = P  \left(-t_{\alpha, n-1} < (\overline{X} - \mu) / \frac{S}{\sqrt{n}} \right)\\
& = P(-t_{\alpha, n -1} <T)\\
& = 1 - \alpha
\end{align*}
To prove the last equality draw a picture - I want \textit{you} to draw the picture on tests
and the homework.
\end{proof}

Once we have an actual sample $x_1, x_2, \ldots,x_n$ we obtain the observed value $\overline{x}$ for the
random variable $\overline{X}$ the observed value $s$ for the random variable $S$ and the observed
value $\left(-\infty, \overline{x} + t_{\alpha, n-1} \dfrac{s}{\sqrt{n}} \right)$ for the confidence (random) interval $\left(-\infty, \overline{X} + t_{\alpha, n-1} \dfrac{S}{\sqrt{n}} \right)$. The observed value of the confidence (random) interval is also called the lower-tailed $100(1 -\alpha)\%$ confidence interval for $\mu$.
\end{frame}

\begin{frame}
The random variable $\overline{X}+ t_{\alpha, n-1} \dfrac{S}{\sqrt{n}}$ or its observed value the number $\overline{x} + t_{\alpha, n-1} \dfrac{s}{\sqrt{n}}$ is often called a confidence \textit{upper bound} for $\mu$ because
$$
P\left(\mu < \overline{X} + t_{\alpha, n-1} \dfrac{S}{\sqrt{n}} \right) = 1 - \alpha.
$$

\myheading{4. The upper-tailed confidence interval for $\mu$}

Problem Prove the following theorem.

\begin{theorem}
The random interval $\left(\overline{X} - t_{\alpha, n -1} \dfrac{S}{\sqrt{n}}, \infty \right)$, is a $100(1- \alpha)\%$ confidence
interval for $\mu$.
\end{theorem}

The random variable $\overline{X} - t_{\alpha, n-1} \dfrac{S}{\sqrt{n}}$ or its observed value the number $\overline{x} - t_{\alpha, n-1} \dfrac{s}{\sqrt{n}}$ is often called a confidence \textit{lower bound} for $\mu$ because
$$
P \left(\mu > \overline{X} - t_{\alpha, n-1} \dfrac{S}{\sqrt{n}} \right) =1-\alpha.
$$
\end{frame}

\end{document}

