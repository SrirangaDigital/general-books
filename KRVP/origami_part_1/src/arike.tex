
%~ {
%~ \makeatletter
%~ \def\@makechapterhead#1{%
  %~ \vspace*{5\p@}%
  %~ {\parindent \z@ \raggedright \normalfont
    %~ \ifnum \c@secnumdepth >\m@ne
      %~ \if@mainmatter
        %~ \LARGE\bfseries \@chapapp\space\thechapter
        %~ \vskip 2pt
        %~ \par\nobreak
        %~ \vskip 5\p@
      %~ \fi
    %~ \fi
    %~ \interlinepenalty\@M
    %~ \LARGE\bfseries #1\par\nobreak
    %~ \vskip 15\p@
  %~ }}
  %~ \makeatother

%~ \chapter*{ಅರಿಕೆ}

\vspace{-1.5cm}

\begin{center}
{\LARGE\textbf{{ಅರಿಕೆ }}}
\end{center}
%~ \vspace{-1cm}

\noindent
ನಮ್ಮ ರಾಜ್ಯದ ಜನರಲ್ಲಿ ವಿಜ್ಞಾನವನ್ನು ಪ್ರಚಾರ ಮಾಡಿ ವೈಜ್ಞಾನಿಕ ಮನೋಭಾವದ\break ಬೆಳವಣಿಗೆಗೆ ಉತ್ತೇಜನ ನೀಡುವುದು  ಕರ್ನಾಟಕ ರಾಜ್ಯ ವಿಜ್ಞಾನ ಪರಿಷತ್ತಿನ ಮುಖ್ಯ ಧ್ಯೇಯ.\break ರಾಜ್ಯದ ವಿವಿಧ ಸ್ಥಳಗಳಲ್ಲಿ ಸ್ವಯಂ ಪ್ರೇರಣೆಯಿಂದ ರೂಪುಗೊಂಡಿರುವ ಪರಿಷತ್ತಿನ  ಘಟಕ\break ಗಳು ಹಾಗೂ ಜಿಲ್ಲಾ ಸಮಿತಿಗಳು ಸ್ಥಳೀಯವಾಗಿ ಈ ಕೆಲಸದಲ್ಲಿ ನಿರತವಾಗಿವೆ.~ಉಪನ್ಯಾಸಗಳು, ವಿಚಾರ ಸಂಕಿರಣಗಳು, ವೈಜ್ಞಾನಿಕ ಪ್ರದರ್ಶನ ಮುಂತಾದವುಗಳನ್ನು ಏರ್ಪಡಿಸುವ ಮೂಲಕ ದಿನನಿತ್ಯದ ಸಮಸ್ಯೆಗಳಿಗೆ ವೈಜ್ಞಾನಿಕ ಪರಿಹಾರಗಳನ್ನು ಹುಡುಕುವಲ್ಲಿ ಜನತೆಗೆ\break ನೆರವು ನೀಡುವ ಮೂಲಕ ಪರಿಷತ್ತಿನ ಧ್ಯೇಯಗಳನ್ನು ಸಫಲಗೊಳಿಸುವ ಪ್ರಯತ್ನ ನಡೆದಿದೆ.\break~ಪರಿಷತ್ತು ಪ್ರಕಟಿಸುವ ನಿಯತಕಾಲಿಕೆಗಳು, ಕಿರುಹೊತ್ತಿಗೆಗಳು ಆ ಪ್ರಯತ್ನಕ್ಕೆ ಬೆಂಬಲ\break ನೀಡಿವೆ. ಈಗಾಗಲೇ 42 ವಸಂತಗಳನ್ನು ಪೂರೈಸಿರುವ ` ಬಾಲವಿಜ್ಞಾನ' ಮಾಸ ಪತ್ರಿಕೆ ಈ\break ದಿಶೆಯಲ್ಲಿ ಸಾಕಷ್ಟು ಯಶಸ್ಸುಗಳಿಸಿ ಜನಪ್ರಿಯವಾಗಿದೆ. ವಿಜ್ಞಾನ ವಿಷಯಗಳ ಕುರಿತ ಕಿರು-\break ಹೊತ್ತಿಗೆಗಳನ್ನು ಪ್ರಕಟಿಸುವ ಕಾರ್ಯವನ್ನು ಪರಿಷತ್ತು ಕೈಗೆತ್ತಿಗೊಂಡು ಈಗಾಗಲೇ 200ಕ್ಕೂ ಹೆಚ್ಚು ಪುಸ್ತಕಗಳನ್ನು ಪ್ರಕಟಿಸಿದೆ.

\noindent
ಈ ದಿಸೆಯಲ್ಲಿ ಗಣಿತ ಲೇಖಕರು ಹಾಗೂ ನಿವೃತ್ತ ಮುಖ್ಯೋಪಾಧ್ಯಾಯರಾಗಿರುವ\break ಶ್ರೀ ವೈ.~ಬಿ.~ಗುರಣ್ಣವರ ಅವರು ಜಪಾನಿನ ಪ್ರಸಿದ್ಧ ಕಲೆಯಾದ ` ಓರಿಗಾಮಿ' ಮೂಲಕ ಶಾಲಾ ಮಕ್ಕಳಿಗೆ, ಶಿಕ್ಷಕರಿಗೆ ಹಾಗೂ ಜನಸಾಮಾನ್ಯರಿಗೆ ಉಪಯೋಗವಾಗುವೆ ನಿಟ್ಟಿನಲ್ಲಿ ` ಓರಿಗಾಮಿ', `ಅಲಂಕಾರಿಕ ವಸ್ತುಗಳ ರಚನೆ ' ಹಾಗೂ ಬಿಲ್ಲೆಗಳ ರಚನೆ ಮುಖೇನ ಗಣಿತವನ್ನು ಮನೋರಂಜನಾತ್ಮಕವಾಗಿ ಕಲಿಸುವ ಪ್ರಯತ್ನ ಮಾಡಿದ್ದಾರೆ. ಕರಾವಿಪವು ಈ ಪುಸ್ತಕವನ್ನು ಹೊರತರುತ್ತಿರುವುದು ಸಂತೋಷದ ವಿಷಯ. 

\noindent
\textbf{ಗಿರೀಶ ಕಡ್ಲೇವಾಡ\hfill ಡಾ॥ ಹುಲಿಕಲ್ ನಟರಾಜ್\\[-.1cm]}
ಅಧ್ಯಕ್ಷರು, ಕರಾವಿಪ \hfill ಅಧ್ಯಕ್ಷರು, ಕರಾವಿಪ ಪುಸ್ತಕ ಪ್ರಕಟಣಾ ಸಮಿತಿ\\[.1cm] 
ಬೆಂಗಳೂರು \\
ಡಿಸೆಂಬರ್ 2021

%~ {
%~ \makeatletter
%~ \def\@makechapterhead#1{%
  %~ \vspace*{5\p@}%
  %~ {\parindent \z@ \raggedright \normalfont
    %~ \ifnum \c@secnumdepth >\m@ne
      %~ \if@mainmatter
        %~ \LARGE\bfseries \@chapapp\space\thechapter
        %~ \vskip 4pt
        %~ \par\nobreak
        %~ \vskip 5\p@
      %~ \fi
    %~ \fi
    %~ \interlinepenalty\@M
    %~ \LARGE\bfseries #1\par\nobreak
    %~ \vskip 15\p@
  %~ }}
  %~ \makeatother
 %\chapter{leVKakara binanxha}

%\lhead[{\footnotesize\fontfamily{txr}\selectfont\thepage}]{{\footnotesize\sl\bfseries leVKakara binanxha}}


\begin{center}
{\LARGE\textbf{{ಲೇಖಕನ ಮಾತು}}}
\end{center}

\noindent
ಗಣಿತ  ವಿಷಯವು ಪ್ರಾರ್ಥಮಿಕ ಮತ್ತು ಪ್ರೌಢ ಶಾಲಾ ಹಂತದಲ್ಲಿ "ಕಬ್ಬಿಣದ ಕಡಲೆ" ಎಂಬ ಮಾತು ಎಲ್ಲರಿಗೂ ತಿಳಿದ ಸಂಗತಿಯಾಗಿದೆ. ಆದರೆ ಈ ಮಾತಿನ ಜೊತೆಗೆ  ಗಣಿತವು `ಕಬ್ಬಿನ ರಸ' ಎಂಬ ಮಾತು ಸಹ ಇತ್ತಿಚೆಗೆ ಸೇರ್ಪಡೆಯಾಗಿದೆ. ಕಬ್ಬಿಣ ಕಡಲೆಯ ಬದಲಾಗಿ ಕಬ್ಬಿನ ರಸ ಬರಲು ಮುಖ್ಯ ಕಾರಣವೆನೆಂದರೆ, ಈಗ ಗಣಿತವನ್ನು ಪ್ರಾಯೋಗಿಕವಾಗಿ ಕಲಿಸುತ್ತಿದ್ದಾರೆ. ಪ್ರಾಯೋಗಿಕ ಗಣಿತವು ಶಾಲೆಯಲ್ಲಿ ಸ್ಥಾಪಿತವಾದ "ಗಣಿತ ಕ್ಲಬ್" [Mathematics Club] ಮೂಲಕ ಸಾಧ್ಯವಾಗಿದೆ. ಗಣಿತ ಕ್ಲಬ್ ನಲ್ಲಿ ಮುಖ್ಯವಾಗಿ "ಓರಿಗಾಮಿ ವಿಧಾನ' ದಿಂದ ಗಣಿತವನ್ನು ಕಲಿಸುತ್ತಿದ್ದಾರೆ. 

\medskip
\noindent
ಓರಿಗಾಮಿ [Origami] ಕಲೆ ಜಪಾನ ಕಲೆಯಾಗಿದ್ದು ಅದನ್ನು ಹೀಗೆ ಹೇಳುತ್ತಾರೆ. "ಕೈ\break ಬೆರಳುಗಳ ಕೈಚಳಕದಲ್ಲಿ ಕಾಗದದಿಂದ ಸುಂದರ ಆಕೃತಿಗಳನ್ನು ಮಾಡುವ ಕಲೆಗೆ ಓರಿಗಾಮಿ ಕಲೆ" ಎನ್ನುತ್ತಾರೆ. ಈ ಕಲೆಯಲ್ಲಿ ಕಾಗದವನ್ನು ಕತ್ತರಿಸದೇ ಕೇವಲ ಮಡಚಿ ವಿವಿಧ ಬಗೆಯ ಆಕೃತಿಗಳನ್ನು ತಯಾರಿಸುತ್ತಾರೆ. ಈ ಕಲೆಯಿಂದ ಪ್ರಾಣಿ, ಪಕ್ಷಿಗಳ, ಎಲೆ ಹೂ, ಹಣ್ಣುಗಳ, ಮನೆ ಗೊಂಬೆಗಳ ಮುಂತಾದ ಆಟಿಕೆಗಳನ್ನು ತಯಾರಿಸುತ್ತಾರೆ. 

\medskip
\noindent
ಕಿೃ.~ಶ.~109 ನೇ ವರ್ಷದಲ್ಲಿ ಚೀನಾ ದೇಶವು ಕಾಗದ ತಯಾರಿಕೆಯನ್ನು ಆರಂಭಿಸಿತು. ಆಗ ಓರಿಗಾಮಿ ಕಲೆ ಮೊಳಕೆ ಒಡೆಯಿತು ಎಂದು ಹೇಳಬಹುದು.  ಕಿೃ.~ಶ.~1983 ರಲ್ಲಿ\break ಟಿ.~ಸುಂದರ ರಾವ್ ಎಂಬವರು ಭಾರತದಲ್ಲಿ ಓರಿಗಾಮಿ ಕಲೆಯನ್ನು ಪರಿಚಯಿಸಿದರು.\break ಇವರು ರಚಿಸಿದ "Geometric excrecise in paper folding" ಪುಸ್ತಕವು ವಿಶ್ವಮಾನ್ಯತೆ ಪಡಿದಿದೆ. 

\medskip
\noindent
ಇಂತಹ ಕಲೆಯನ್ನು ಪ್ರಾರಂಭದಲ್ಲಿ ಅಲಂಕಾರ ವಸ್ತುಗಳ ತಯಾರಿಕೆಗೆ ಉಪಯೋಗಿಸು\break ತ್ತಿದ್ದರು. ಈಗ ಓರಿಗಾಮಿ ಕಲೆಯನ್ನು ಗಣಿತದ ಕಲಿಕೆಗೆ ಮತ್ತು ಕಲಿಸುವಿಕೆಗೆ ಉಪಯೋಗಿ\-ಸುತ್ತಾರೆ. ಆದ್ದರಿಂದ ನಾನು ಶಾಲಾ ಮಕ್ಕಳಿಗೆ, ಶಿಕ್ಷಕರಿಗೆ ಹಾಗೂ ಸಾಮಾನ್ಯ ಜನರು ಸಹ ಉಪಯೋಗವಾಗಲೆಂದು ಈ ಪುಸ್ತಕವನ್ನು ರಚಿಸಿದ್ದೇನೆ. 

\medskip
\noindent
ಈ ಪುಸ್ತಕದ ಪ್ರಕಟಣೆಗೆ ಕರಾವಿಪದ ಅಧ್ಯಕ್ಷರು, ಸದಸ್ಯರು ಮತ್ತು ಪ್ರಕಟಣ ಮಂಡಳಿಯ\break ಅಧ್ಯಕ್ಷರು, ಸದಸ್ಯರಿಗೆ ನಾನು ಚಿರಯಣಿಯಾಗಿದ್ದೆನೆ. ಈ ಪುಸ್ತಕವು ಸುಂದರವಾಗಿ ಮುದ್ರಣ-\break ವಾಗಲು ಸಹಕರಿಸಿದ ಕರಾವಿಪ ಸಿಬ್ಬಂದಿಗಳಿಗೆ  ಮತ್ತು ಪ್ರಕಟಣೆಗೆ ಸಹಾಯ ಮಾಡಿದವರಿಗೆ ನಾನು ಸ್ಮರಿಸುತ್ತೆನೆ. ಮಕ್ಕಳು, ಶಿಕ್ಷಕರು ಹಾಗೂ ಸಾರ್ವಜನಿಕರು ಈ ಪುಸ್ತಕವನ್ನು ಉಪಯೋಗಮಾಡಿ ಗಣಿತ ವಿಷಯವು ಕಬ್ಬಿಣ ಕಡಲೆ ಅಲ್ಲ ಇದೊಂದು ಕಬ್ಬಿನ ರಸ ಎಂಬುದನ್ನು ತೋರಿಸಿ ಕೊಡಬೇಕೆಂದು ವಿನಂತಿ. 

\begin{flushright}
{\bf ವೈ. ಬಿ. ಗುರಣ್ಣವರ}\\
{\bf ನಿವೃತ್ತ ಮುಖ್ಯೋಪಾಧ್ಯಾಯರು,  ಕುಂದಗೋಳ}\\
\end{flushright}


%~ }\relax
