
%~ {
%~ \makeatletter
%~ \def\@makechapterhead#1{%
  %~ \vspace*{5\p@}%
  %~ {\parindent \z@ \raggedright \normalfont
    %~ \ifnum \c@secnumdepth >\m@ne
      %~ \if@mainmatter
        %~ \LARGE\bfseries \@chapapp\space\thechapter
        %~ \vskip 2pt
        %~ \par\nobreak
        %~ \vskip 5\p@
      %~ \fi
    %~ \fi
    %~ \interlinepenalty\@M
    %~ \LARGE\bfseries #1\par\nobreak
    %~ \vskip 15\p@
  %~ }}
  %~ \makeatother

\chapter*{ಅರಿಕೆ}

\vspace{-1cm}

ನಮ್ಮ ರಾಜ್ಯದ ಜನರಲ್ಲಿ ವಿಜ್ಞಾನವನ್ನು ಪ್ರಚಾರ ಮಾಡಿ ವೈಜ್ಞಾನಿಕ ಮನೋಭಾವದ ಬೆಳ\break ವಣಿಗೆಗೆ ಉತ್ತೇಜನ ನೀಡುವುದು  ಕರ್ನಾತಕ ರಾಜ್ಯ ವಿಜ್ಞಾನ ಪರಿಷತ್ತಿನ ಮುಖ್ಯ ಧ್ಯೇಯ.\break ರಾಜ್ಯದ ವಿವಿಧ ಸ್ಥಳಗಳಲ್ಲಿ ಸ್ವಯಂ ಪ್ರೇರಣೆಯಿಂದ ರೂಪುಗೊಂಡಿರುವ ಪರಿಷತ್ತಿನ  ಘಟಕ\break ಗಳು ಹಾಗೂ ಜಿಲ್ಲಾ ಸಮಿತಿಗಳು ಸ್ಥಳೀಯವಾಗಿ ಈ ಕೆಲಸದಲ್ಲಿ ನಿರತವಾಗಿವೆ.~ಉಪನ್ಯಾಸಗಳು, ವಿಚಾರ ಸಂಕಿರಣಗಳು, ವೈಜ್ಞಾನಿಕ ಪ್ರದರ್ಶನ ಮುಂತಾದವುಗಳನ್ನು ಏರ್ಪಡಿಸುವ ಮೂಲಕ ದಿನನಿತ್ಯದ ಸಮಸ್ಯೆಗಳಿಗೆ ವೈಜ್ಞಾನಿಕ ಪರಿಹಾರಗಳನ್ನು ಹುಡುಕುವಲ್ಲಿ ಜನತೆಗೆ ನೆರವು ನೀಡುವ ಮೂಲಕ ಪರಿಷತ್ತಿನ ಧ್ಯೇಯಗಳನ್ನು ಸಫಲಗೊಳಿಸುವ ಪ್ರಯತ್ನ ನಡೆದಿದೆ. ಪರಿಷತ್ತು ಪ್ರಕಟಿಸುವ ನಿಯತಕಾಲಿಕೆಗಳು, ಕಿರುಹೊತ್ತಿಗೆಗಳು ಆ ಪ್ರಯತ್ನಕ್ಕೆ ಬೆಂಬಲ ನೀಡಿವೆ. ಈಗಾ\break ಗಲೇ 42 ವಸಂತಗಳನ್ನು ಪೂರೈಸಿರುವ ` ಬಾಲವಿಜ್ಞಾನ' ಮಾಸ ಪತ್ರಿಕೆ ಈ ದಿಶೆಯಲ್ಲಿ ಸಾಕಷ್ಟು ಯಶಸ್ಸುಗಳಿಸಿ ಜನಪ್ರಿಯವಾಗಿದೆ. ವಿಜ್ಞಾನ ವಿಷಯಗಳ ಕುರಿತ ಕಿರುಹೊತ್ತಿಗೆಗಳನ್ನು ಪ್ರಕಟಿಸುವ ಕಾರ್ಯವನ್ನು ಪರಿಷತ್ತು ಕೈಗೆತ್ತಿಗೊಂಡು ಈಗಾಗಲೇ 200ಕ್ಕೂ ಹೆಚ್ಚು ಪುಸ್ತಕಗಳನ್ನು ಪ್ರಕಟಿಸಿದೆ.

ಪುಸ್ತಕಗಳ ಪ್ರಕಟಣೆಗೆ ಕರಾವಿಪ, ವಿಜ್ಞಾನದ ಎಲ್ಲಾ ಪ್ರಕಾರಗಳನ್ನು ತನ್ನ ವ್ಯಾಪ್ತಿಗೆ ತೆಗೆದುಕೊಂಡಿದ್ದು, ವಿದ್ಯಾರ್ಥಿಗಳು ಮತ್ತು ಜನಸಾಮಾನ್ಯರ ದೈನಂದಿನ ಜೀವನಕ್ಕೆ ಸಂಬಂಧಿಸಿದ ವಿಷಯಗಳಿಗೆ ಆದ್ಯತೆ ನೀಡಿ ಪ್ರಕಟಿಸುತ್ತಿದೆ.

ಈ ದಿಸೆಯಲ್ಲಿ ಗಣಿತ ಲೇಖಕರು ಹಾಗೂ ನಿವೃತ್ತ ಮುಖ್ಯೋಪಾಧ್ಯಾಯರಾಗಿರುವ ಶ್ರೀ ವೈ.~ಬಿ.~ಗುರಣ್ಣವರ ಅವರು ಜಪಾನಿನ ಪ್ರಸಿದ್ಧ ಕಲೆಯಾದ `ಓರಿಗಾಮಿ' ಮೂಲಕ ಶಾಲಾ ಮಕ್ಕಳಿಗೆ, ಶಿಕ್ಷಕರಿಗೆ ಹಾಗೂ ಜನಸಾಮಾನ್ಯರಿಗೆ ಉಪಯೋಗವಾಗುವೆ ನಿಟ್ಟಿನಲ್ಲಿ `ಓರಿಗಾಮಿ' , `ಅಲಂಕಾರಿಕ ವಸ್ತುಗಳ ರಚನೆ ' ಹಾಗೂ ಬಿಲ್ಲೆಗಳ ರಚನೆ ಮುಖೇನ ಗಣಿತವನ್ನು ಮನೋರಂಜನಾತ್ಮಕವಾಗಿ ಕಲಿಸುವ ಪ್ರಯತ್ನ ಮಾಡಿದ್ದಾರೆ. ಕರಾವಿಪವು ಈ ಪುಸ್ತಕವನ್ನು ಹೊರತರುತ್ತಿರುವುದು ಸಂತೋಷದ ವಿಷಯ. 

\noindent
\textbf{ಗಿರೀಶ ಕಡ್ಲೇವಾಗ \hfill ಡಾ ಹುಲಿಕಲ್ ನಟರಾಜ್\\[-.1cm]}
\textbf{ಅಧ್ಯಕ್ಷರು, ಕರಾವಿಪ \hfill ಅಧ್ಯಕ್ಷರು\\[-.1cm]} 
\textbf{\phantom{} \hfill ಕರಾವಿಪ ಪುಸ್ತಕ ಪ್ರಕಟಣಾ ಸಮಿತಿ}
