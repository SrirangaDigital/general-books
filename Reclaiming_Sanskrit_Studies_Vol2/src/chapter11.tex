\chapter{The Science and Nescience of Śāstra}\label{chapter11}
\vskip -10pt

\Authorline{Sudarshan Therani Nadathur}
\lhead[\small\thepage\quad Sudarshan Therani Nadathur]{}
\vskip -10pt

\section*{Abstract}

The interpretations of {\sl śāstra} as done by Sheldon Pollock are critically appraised in this paper with a firm grounding in the traditional perspectives and vocabularies of the {\sl vidyā}-s (poorly translated by Pollock as theory) as practiced ({\sl śāstra}) by actual practitioners of the Vedic tradition. The foundational perspectives and motivations that have driven the theorization of the grammars and metaphors of Indian Knowledge Systems, and the practices derived thereof are examined. It is proposed that the current Western theses on {\sl śāstra} derive from a deep ignorance -- a veritable nescience. The primacy, undilutability, and non-negotiable nature of a sacred perspective ({\sl saṃskāra}) whilst interpreting Sanskrit texts on Indic knowledge systems are established. The flawed and incorrect use of philology by Pollock and its overall nebulous nature is explicated. The limitation of the scientific method in interpreting {\sl śāstra} is discussed. The non-empirical, non-verifiable and unscientific nature of the methods used by Pollock to make his claims, are highlighted. The aim, purpose and science of any {\sl śāstra} is to lead the practitioner on the path to a holistically (nature included) harmonious existence. The scope and role of {\sl śāstra} is beyond that of Science or Religion (as the West currently knows/interprets). Unless this universality of aim is acknowledged and more importantly reinforced and realized by its practice - Western scholarship will continue to provide nebulous and incorrect etici interpretations of Indic knowledge systems driven by nescience.

\section*{Introduction}

This paper aims to highlight fundamental issues in the method chosen, and the fallacy of the assumptions made, in the paper entitled ``The Theory of Practice and the Practice of Theory in Indian Intellectual History'' (Pollock 1985) on Indian {\sl śāstra}. Even after thirty years of the publication of the said paper, there has been no rigorous examination of the interpretation/theses made in this paper. That the basis of Western Indology in general and of Pollock's American school of Orientalism/Indology in particular, is ignorant of the methods of Indian ({\sl śāstra}) science (knowledge systems), and is based on a flawed understanding (nescience) needs to be called out. It is posited here that the entire edifice of the Pollock school is built on questionable claims, is driven by deep underlying institutionalized biases, and is manufactured by insidious methods of philology. There have been numerous erudite discussions, deep analyses and critical perspectives post-Said, i.e. after Said (1979) addressing Western scholarship directed toward the East; but, in the general spirit of inquiry and in the larger context of the problem of Western Indology specific to the motivations in this paper, we refer to the appropriate sections in Sudarshan (2016).  As indicated earlier, we examine the work of Pollock, his methods and his specific theses on Indian {\sl śāstra}, in the spirit of the traditional Indian argumentative method of {\sl pūrvapakṣa}. In the first two sections, we set the stage for a deeper examination of Pollock. S (1985): in the first, comparing the Western and traditional Indian constructs and world-views (epistemology), we highlight the foundational incompatibilities; in the second, very much in the spirit of {\sl pūrvapakṣa} - Western theory ({\sl śāstra}) and Western practice ({\sl prayoga}) are examined with specific focus on Pollock's methods (his {\sl śāstra} and its {\sl prayoga}). In the third section, we `read' and examine ({\sl pūrvapakṣa}) the specific paper in question. Wherever relevant, specific dissonances with Indian perspectives are identified, highlighting foundational issues in method and approach. We conclude with a discussion on the implications of this {\sl pūrvapakṣa}, a brief summary on the Pollockian methods and also identify some possible classes of refutations ({\sl uttarapakṣa}).

\section*{Preliminaries}

Western social sciences freely use and refer to the methods and theories of modern mathematics and science and follow their evolution in an implicit manner.  By themselves, Western social sciences, unlike the (Western) natural sciences, do not have any foundational scientific construct on which their reasoning can be based. We briefly discuss epistemology in Indian Knowledge Systems and attempt to describe Western Indology in its theory and practice in terms of Indian categories.

\section*{Traditional Indian Epistemology}

{\sl Tattvas} - Principles of Reality

For a detailed description of the categories and for a discussion of the salient issues/points and a treatment of the evolution of Indian Epistemology, see Matilal (1971). I quote from the Introduction:
\begin{myquote}
Epistemology, Matilal argues, is the key philosophical discipline in the Indian debate, not metaphysics, a claim that does not preclude the discussion of metaphysical questions, but sees their resolution as lying in an analysis of the structures of knowledge and language. (Matilal 1971:x)
\end{myquote}

Given the variances and distinctness among the different schools and perspectives, there are a set of common presuppositions that can be considered to be ``basis'', whenever one discusses Indian categories in an overarching manner.

{\sl Darśana}-s often describe their epistemology, based on their unique interpretation of the aforementioned common ``basis'' and almost all of them refer to the categories from the Sāṅkhya school or base their variations upon it. Both the {\sl āstika} (orthodox) and {\sl nāstika} (heterodox) {\sl darśana}-s, interpret the basic framework of the Sāṅkhya. Its influence on other schools like Tantra based Kashmir Śaivism and the various other schools of Śaivism is known.

The well-known {\sl Nāsadīya Sūkta} ({\sl Ṛg Veda} 10.129) is considered as the origin for the later development of the Sāṅkhya as a distinct {\sl darśana}. The oft-quoted verse from the Śānti Parvan of the {\sl Mahābhārata} on the Sāṅkhya
\begin{myquote}
There is no knowledge that is equal to this. The knowledge, which is described in the system of the Sāṅkhya, is regarded as the highest.

\hfill (Book 12, Ch.302) (See ``Shanti Parva Part III'')
\end{myquote}

The epistemology in the {\sl Bhagavad Gītā} ({\sl Kṛṣṇa}) uses Sāṅkhya as its basis and uses it to reason about concepts and causalities in and across the categories. The second chapter ({\sl adhyāya}) which summarises the entirety of the text, is sometimes referred to as the Sāṅkhya Yoga or the Yoga of Knowledge. See (Chapter two, {\sl BhaktiVedānta Vedabase}) 

It can safely be assumed that the categories of Sāṅkhya can be used as a basis in a broad sense to reason about perspectives based on Indian systems. For the purposes of this discussion, we can probably also refer to such a perspective as a {\sl Dhārmic worldview} as enunciated in the book {\sl Being Different} (Malhotra 2011).

The Sāṅkhya ontology explicitly acknowledges the existence of the {\sl Puruṣa} or a Supreme consciousness and the manifestation of material existence as the {\sl Prakṛti}. Such an existence of a ``supreme'' is acknowledged in almost all schools of Indian thought excepting a few categorised as the materialist schools. 
\begin{myquote}
The Supreme Good is {{\sl\bfseries mokṣa}\relax} which consists in the permanent impossibility of the incidence of pain... in the realisation of the Self as Self pure and simple.

\hfill ({\sl Sāṁkhyakārikā} 5.1.3)
\end{myquote}

See Heera (2011) for a preliminary treatment of Indian materialism esp. the Cārvāka school for an example of a {\sl darśana} which rejects this.

Given the prevalence of the ``sacred'' dimension in almost all schools of Indian thought, and the `sacred' being the recommended mental disposition when `reading' them, it is apt that  we bring attention to a term called {\bf Sacred Philology} in an Indian context, coined in Malhotra  (2016:362).
\begin{myquote}
A philology rooted in the conviction that Sanskrit cannot be divorced from its matrix in the Vedas and other sacred texts, or from an orientation towards the transcendent realm.
\end{myquote}

{\sl Pramāṇa} -- ``Means to acquire knowledge''

From the Stanford Encyclopedia of Philosophy (SEP) entry on {\sl Epistemology in Classical Indian Philosophy}, we have (Stephen 2015)
\begin{myquote}
Commonalities in the classical Indian approaches to knowledge and justification frame the arguments and refined positions of the major schools. Central is a focus on occurrent knowledge coupled with a theory of ``mental dispositions'' called {\sl saṃskāra}. 
\end{myquote}

We have dealt with the issue of mental disposition in the previous section on {\sl tattva}-s. Unless there is an acknowledgement of a world-view similar to a Sāṅkhya view, such a {\sl saṁskāra} will be for practical purposes, irrelevant. For purposes of interpreting Indian texts that deal with the primary texts and commentaries of the {\sl vidyā}-s and {\sl śāstra}-s, unless there is a rootedness, an awareness of the sacred, the interpretive exercise and the inferences drawn thereby can be considered to be done from a position of {\bf nescience}. The sacred position could thence be considered {\bf scientific}.
\begin{myquote}
Epistemic evaluation of memory, and indeed of all standing belief, is seen to depend upon the epistemic status of the occurrent cognition or awareness or awarenesses that formed the memory, i.e., the mental disposition, in the first place. Occurrent knowledge in turn must have a knowledge source, {\sl pramāṇa}.\hfill (Stephen 2015)
\end{myquote}

The principal means of {\sl pramāṇa}, recognized across most Indian schools of thought are Perception ({\sl Pratyakṣa}), Inference ({\sl Anumāna}), Analogy ({\sl Upamāna}), Postulation ({\sl Arthāpatti}), Cognitive Proof ({\sl Anupalabdhi}) and Testimony ({\sl Śabda}). For purposes of the current discussion (evaluating the validity of the claims made by Western Indology), it will suffice to know that most of Western Indology is based on their own internal {\sl Śabda} and Western {\sl siddhānta} and use the (traditional) sources of knowledge in a dissonant manner,  primarily with a non-sacred {\sl saṃskāra}. The {\sl pramāṇa} need to be constrained by the appropriate {\sl saṁskāra}.
\begin{myquote}
Logic is developed in classical India within the traditions of epistemology. Inference is a second knowledge source, a means whereby we can know things not immediately evident through perception. Oetke (1996) finds three roots to the earliest concerns with logic in India: {\bf (1) common-sense inference, (2) establishment of doctrines in the frame of scientific treatises ({{\sl\bfseries śāstra}\relax}), and (3) justification of tenets in a debate}. The three of these come together (though the latter two are predominant) within the epistemological traditions in an almost universal regard of inference as a knowledge source.\hfill (Stephen 2015)
\end{myquote}

{\sl Mīmāṁsā} - Methods of Investigation

Though Mīmāṁsā is considered to be a {\sl darśana} (school of philosophy) in its own right, it is critical to note that it lays down the rules of interpretation of Sanskrit sentences and the derivation of context. It involves a deep understanding of the techniques of grammar (Vyākaraṇa), prosody (Chandas), gloss (Nirukta). The Pūrvamīmāṁsā describe rules of interpretation, causality, {\sl pramāṇa}-s in the context of the {\sl karma-śāstra} of {\sl yajña}-s. This framework has in general been extended to other contexts too such as the {\sl mokṣa-śāstra} of Vedānta.

The range and scope of Mīmāṁsā is immense and there is no such coherent equivalent in the Western system of knowledge. Given this, we note that when it comes to interpretation of Sanskrit texts, unless based on a solid ground of the {\sl mīmāṁsā-adhikaraṇa}-s, any interpretation of Sanskrit text is decidedly incomplete (even if partially correct), incorrect at best, and at worst irrelevant. Pandurangi (2013) writes,
\begin{myquote}
The semantics, considering the language, autonomy at word and meaning level and sentence level is an important contribution of Pūrvamīmāṁsā. The two theories of sentence meaning: Abhihitānvayavāda (Kumārila's independent meaning) and Anvitābhidhānavāda (Prabhākara's dependent meaning) is another contribution. All systems of Indian philosophy have adopted these two theories with some modifications. There are more than a hundred maxims crystallising the guidelines for interpretation. \hfill (Pandurangi 2013:57-58)
\end{myquote}

Western Indology with its large body of literature and academic work is mostly in non-Sanskrit and non-Indian languages can be summarily dismissed as {\bf irrelevant} if one were to take the orthodox Mīmāṁsaka view. `Readings' of Sanskrit are currently being done with `free-style' techniques like Pollockian philology. For such ``readings'' to be acknowledged as valid by any serious traditionalist/Swadeshi/emic indologist, they should at least be justified in the Mīmāṁsā (however rudimentary) interpretive framework. 

\section*{Epistemology of Western Indology (according to Indian categories)}

Methods, techniques, assumptions, worldviews of the etic (outsider/Western) interpreters do not easily conform to the Indian/Vedic epistemologies. The principles of reality, the sources of knowledge, and the techniques of interpretation do not have much commonality. Most of Western Indology scholarship can be categorised as materialist and anthropocentric. They acknowledge only the gross realities of the world and its relation to man. The approach is closest to the Indian School of Cārvāka or Lokāyata. Categorised interestingly as Cārvāka 2.0 in (Malhotra 2016:90-91), they do not acknowledge the existence of the sacred and are sophisticated materialists.

In summary, the {\sl tattva}-s and {\sl pramāṇa}-s (see section on scientific nature of {\sl śāstra} for further discussion) of Indian systems do not have parallels in Western methods. The inapplicability of the techniques of Mīmāṁsā and the limited logics of the Western interpretive framework make it difficult to view the repository of Western Indological scholarship in serious light. Furthermore, we discuss the limitations of the ``Western scientific method'' on its own and in the context of Western social sciences in the ensuing section.

\section*{{{\sl\bfseries PūrvaPakṣa}\relax} of the Western methods}

As the needs and goals of Western Indology have kept changing, the methods have also evolved and the resulting commentary on India via Indological methods has also been continually changing. American Orientalism, the latest avatar, has been influenced by the policies of dual-use anthropology and is highlighted in a recent book entitled {\sl Cold War Anthropology} (Price 2016). Much of the topics of study and overall direction of research is guided by American policy needs and requirements of the military-industrial complex and aided by the military-politico-academic nexus. Data and fact collections are made to fit the conclusions and inferences that have already been made to conform to existing or new policy decisions. There has been no consistent set of axiomatic assumptions or canonical methods (characterised `Western') in use over the entire period of Western Indology. In the post-Said, postmodernism influenced Pollock school, the overarching consistency has been in the presence of the ``political'' sensibility in the style of `readings' and the sordid manufacture of `political {\bf literarization}' from historical text.

A brief history of the Western method of `science'

According to Western self-hagiographic accounts, humanity owes it to the Greeks for the origins of logical methods. Aristotle's {\sl Organon} is supposed to have influenced the work {\sl Novum Organum} of Francis Bacon. The Baconian method (inductive reasoning via axioms) influenced the development of the scientific method of modern science. See Applebaum (2000) and Applebaum (2005) for a historical account of the methods of modern science. Most modern scientific methods have abandoned the classical methods, and logical empiricism, and depend more on hypothesis formation and its experimental verification (``reproducibility''). The Popperian principle of falsifiability holds sway, wherein every theory is subject to verifiability (Popper 1968) - No theory can be proven to be true. It can only proven to be not false - yet.  Another key principle (via Kuhn 1996) is commensurability wherein, almost always, rival theories are incommensurable. It is not possible to understand one theory in terms of the other leading to relativism and irrationality of theories. To defend this viewpoint and provide a framework to validate theories, Kuhn cited five criteria - accuracy, consistency, broadness, simplicity, fruitfulness - that determine choice of theory. Kuhn states 
\begin{myquote}
``When scientists must choose between competing theories, two men fully committed to the same list of criteria for choice may nevertheless reach different conclusions. I am suggesting, of course, that the criteria of choice with which I began function not as rules, which determine choice, but as {\bf values}, which influence it''\hfill (Kuhn 1977:324)
\end{myquote}

That Western science is not far off from the influence of {{\sl\bfseries saṁskāra}\relax} (values/mental disposition) of the scientist is something that is to be internalised. Populist accounts of the supposed rationalism and objectivity of science gloss over this. Given the variance in the methods of scientific enquiry, there are some basic components of method that the community agrees upon (for natural sciences but {\bf not} for social sciences). The four essential elements are observations, hypotheses, predictions and experiments. The repeated cycle of these four elements when subject to peer review comprise the modern scientific method (in current practice).

In general, the social sciences use much more {\bf subjective} and {\bf fundamentally unverifiable} methods. The misuse and misunderstanding of quantitative analysis/statistics and the lack of statistical rigor in ill-defined ``experiments'' (which only add a veneer of formality) in the social sciences to convince the lay person of its ``scientific'' nature deserves mention.

The scientific nature of {\sl śāstra}

At this juncture, it is essential to understand the deeply scientific nature of {\sl śāstra}. One must closely observe that there is {\bf no dichotomy} of natural-sciences vs social-sciences in the Indian traditional knowledge systems (See Kapoor and Singh 2005). Much of the Indian {\sl śāstra}-s have their origins in the experimental verification (first person empiricism) of the methods by the seers ({\sl ṛṣi}-s). The {\sl ṛṣi}-s could achieve higher states of consciousness by various inner methods of {\sl dhyāna} (conferring {\sl divya-dṛṣṭi}) and experience the realities of the {\sl śāstra}-s first-hand. The {\sl śāstra}-s are not the result of arbitrary theorising and hypothesising. Scientific empiricism by contrast is predominantly third-person and only examines phenomena and its inherent ``causalities'' as an observer, never as a ``subject''.

For the purposes of this paper, it would not be wholly incorrect to state that guesswork driven science (see Feynman 2013) and the first person empirical nature of Indian ``science ({\sl śāstra})'' are incommensurable. On closer examination, one can very well claim that in their genesis, methodology and  evolution, the methods that exemplify  Indian  {\sl śāstra} are {\bf closer} to ``{\bf science}'' in a deeper sense than actual Western science itself - they do not have the intermediate third-party steps of hypothesis and theory-building. The phenomenologies are directly derived from experience (experimental verification) of the actual nature of reality. Though the practice of {\sl śāstra} is traditionally aligned with the overall sacred perspective, it is possible to abstract the inferences and models (as in Yoga, Āyurveda, which can be considered to align with the ``natural sciences''), even if not to their full potential but at least to ``materially'' tractable levels (health/wellness). They can be taught to the actual practitioner who has minimal qualifications or someone who even does not subscribe to the sacred perspective and one can still observe ``visible'' results.

However, such an approach to transfer ``models'' in the context of the social sciences (which deal with the inner states of the individual and thereby collectively of society) fails as it critically depends on ``first person empiricism''. For a novel and probably first-of-a-kind treatment of this perspective, the use of Indian models in applicative social sciences, see Cornelissen (2013). To illustrate - a sacred {\sl saṁskāra} cannot be faked, meditative states cannot be assumed, {\sl cakra} influences cannot be chemically induced etc. There are also other limitations imposed by the (third person empiricism) scientific method, when it comes to the application of abstracted/lifted models of Indian {\sl śāstra} into Western social sciences. The need for an observer, an external reviewer, and also the Western model of peer-review and ``published'' scholarship do not allow for ``primacy'' of individual experience. Western social sciences is mostly a ``social'' and theoretical activity which then supposedly percolates to/affects the individual via the State, unlike the underlying assumptions of the Indian anthropological models, which seamlessly traverse between the individual and society in both directions (see Gurumurthy (2014) Lecture 12)

{\bf A brief history of Western humanities}

Western humanities is deeply rooted in the manufactured history of the West and the attempts to understand the present in terms of this past -
\begin{myquote}
The word ``humanities'' is derived from the Renaissance Latin expression {\sl studia humanitatis}, or ``study of humanitas'' (a classical Latin word meaning--in addition to ``humanity'' -- ``culture, refinement, education'' and, specifically, an ``education befitting a cultivated man'').\hfill (Ref: ``Humanities'')
\end{myquote}

According to the Wikipedia entry quoting Donovan humanities is mostly about study, theories and interpretations and has very little to do with actual realities in terms of practice and experiential knowledge.
\begin{myquote}
A major shift occurred with the Renaissance humanism of the fifteenth century, when the humanities began to be regarded as subjects to {{\sl\bfseries study rather than practice}\relax}, with a corresponding shift away from traditional fields into areas such as literature and history.\hfill (Ref: ``Humanities'')
\end{myquote}

Stanley Fish in his blog in the New York Times, {\sl Think Again}, delivers a devastating critique on the humanities and its overall relevance to modern society. He attempts to place it in perspective and has this to say about its overall utility in general. 
\begin{myquote}
Any attempt to justify the humanities in terms of outside benefits such as social usefulness (say increased productivity) or in terms of ennobling effects on the individual (such as greater wisdom or diminished prejudice) is ungrounded, and simply places impossible demands on the relevant academic departments.\hfill (Fish 2008)
\end{myquote}

Fish concludes
\begin{myquote}
To the question ``of what use are the humanities?'', the only honest answer is {\bf none whatsoever}. And it is an answer that brings honor to its subject. Justification, after all, confers value on an activity from a perspective outside its performance. An activity that cannot be justified is an activity that refuses to regard itself as instrumental to some larger good. The humanities are their own good.\hfill (Fish 2008)
\end{myquote}

{\bf Study of text - Critical Theory}

Scholarship in the Western Humanities use techniques from various schools of Critical Theory to develop arguments and examine a subject in a supposedly unbiased and objective manner (a diluted version of scientific methods that depends on the critic). Critical Theorists including Hegel {\bf rejected} the ``objective'', scientific approach. They sought to frame theories within ideologies of human freedom. There are many schools and approaches to critical theory - See ``Outline of critical theory'' for a list of the various schools. Most humanities (Indology can be seen to be using techniques that originate from both sociology and literary criticism) scholarship can be classified by ``method'' to be using some combination of the methods listed here.

Bohman writes:
\begin{myquote}
``Critical Theory'' in the narrow sense designates several generations of German philosophers and social theorists in the Western European Marxist tradition known as the Frankfurt School. According to these theorists, a ``critical'' theory may be distinguished from a ``traditional'' theory according to a specific practical purpose: a theory is critical to the extent that it seeks human ``emancipation from slavery'', acts as a ``liberating influence'', and works ``to create a world which satisfies the needs and powers'' of human beings\hfill (Bohman 2015)
\end{myquote}

He further states as definition
\begin{myquote}
A critical theory is adequate only if it meets three criteria: it must be explanatory, practical, and normative, all at the same time. That is, it must explain what is wrong with current social reality, identify the actors to change it, and provide both clear norms for criticism and achievable practical goals for social transformation\hfill (Bohman 2015)
\end{myquote}

The hoary aim of the methods of the Critical Theory is to emancipate, in a ``Western'' sense the societies and systems under study. Implicit in the definition is that it is only the ``West'' that gets to comment and critique, based on methods created by the West. In the specific case of Indology - most of the Indologists are {\bf not} practising adherents of Indic lifestyles and do not adhere to or live by {\sl dhārmic} notions of the world and do not have world views which originate from these alternative (non-Western and {\sl dhārmic}) epistemologies. Is not such a social inquiry made via these methods poisoned by existing biases?

He also adds in the conclusion to his entry
\begin{myquote}
Finally, what sort of verification does critical inquiry require? In light of the answers to these questions on the practical, democratic, and multi perspectival interpretation defended here, it is likely that Critical Theory is no longer a unique approach. Methodologically, it becomes more {\sl thoroughly pluralistic}.\hfill (Bohman 2015)
\end{myquote}

Whether these perspectives on actual methods are subscribed to by real-world  academics and departments of Indology is something that could be debated on - but just in case we were in doubt whether a {\sl dhārmic perspective} on Indology would be  not ``allowed'' in a strict sense of ``method'' - it seems, it would be allowed. See Critchley (1992) for detailed discussions on the {{\sl\bfseries ethics}\relax} and {{\sl\bfseries politics}\relax} (Lüdemann 2014) of these Western methods.

The issue of what is considered ``normal'' in the current modern context (circa 2016), the so-called normative is mostly what Malhotra explicitly identifies as {\bf Western Universalism} (Malhotra 2011). Be that as it may, we will still need to examine whether Critical theory with a {\sl dhārmic} perspective is indeed a valid one, if one were to subscribe to the validity of the methods of Western Critical Theory.

It turns out that it could indeed be a possibility, the periodically updated SEP entry by Bohman has this to say on the ``normative''
\begin{myquote}
Critical Theory offers an approach to distinctly normative issues that cooperates with the social sciences in a nonreductive way. Its domain is inquiry into the normative dimension of social activity, in particular how actors employ their practical knowledge and normative attitudes from complex perspectives in various sorts of contexts. It also must consider social facts as problematic situations from the {{\sl\bfseries point of view of variously situated agents}\relax}.
\end{myquote}

{\bf Other methods (deriving from Marx)}

There are many definitions of Marxism, so to be fair, let us use the crowd-edited version from the Wikipedia, 
\begin{myquote}
Marxism is a method of socioeconomic analysis that analyzes class relations and societal conflict that uses a materialist interpretation of historical development, and a dialectical view of social transformation. Marxist methodology uses economic and sociopolitical inquiry and applies that to the critique and analysis of the development of capitalism and the role of class struggle in systemic economic change.
\end{myquote}

Variations of the Marxist theories influenced by the Russian dialectics of Lenin, and Stalin, Marxist Critical Theory of the Frankfurt school are generally the most widespread versions used as the basis in Western (social-science) academia. There exist world-wide variations as in China's Marxism influenced by Mao, the Guevara influenced Latin-American version, North Korean (Juche) not to mention the various regional Indian versions.

Pollock's socio-economic analysis of historical India can be seen to be primarily subscribing to a Marxist-driven framework. Applying the Marxist lenses to the past and also to his (political and liberation) philology, he is able to give novel readings and perspectives of Indian texts that pass for Indology scholarship.

{\bf Postmodernism}

As a movement which is a reaction to the assumptions and values of the Modern West (16th-20th century), many of its characteristics derive from the denial of the enlightenment values and principles. There is no objective natural reality, there is nothing like the truth, no faith in science and technology as instruments of human progress, the relativism of logic and reason and, importantly, the view that language is freely interpretable (via deconstruction). 

The entry on the subject in the Encyclopedia Britannica says:
\begin{myquote}
...postmodernists regard their theoretical position as uniquely inclusive and democratic, because it allows them to recognize the unjust hegemony of Enlightenment discourses over the equally valid perspectives of non elite groups. In the 1980s and '90s, academic advocates on behalf of various ethnic, cultural, racial, and religious groups embraced postmodern critiques of contemporary Western society, and postmodernism became the unofficial philosophy of the new movement of ``identity'' politics.\hfill (Dulgnan 2014)
\end{myquote}

Pollock derives the ``power'' of his philology from the post-modern views of language. The `power discourse' that he attempts to `see' in historical text, is also influenced by these postmodern perspectives.

{\bf Philology - Method to the mischief}

The overall framework and argumentative methods of Critical Theory (described in brief in the previous section) along with the tools and techniques provided by Philology, largely contribute to the bodies of scholarship created by Pollock and others in this new modern school of American Orientalism. A harmless definition of Philology, in Wikipedia reads as follows:
\begin{myquote}
Philology is the study of language in written historical sources; it is a combination of literary criticism, history, and linguistics. It is more commonly defined as the study of literary texts and written records, the establishment of their authenticity and their original form, and the determination of their meaning.\hfill (See ``Philology'')
\end{myquote}

Pollock, with his background of training in the Greek classics considers the use of Philology and its methods (according to his definitions) as being critical to understanding the meaning (hidden or otherwise) of texts.

He says 
\begin{myquote}
...philology is, or should be, the discipline of making sense of texts.

\hfill (Pollock 2009:934)
\end{myquote}

He is vehement in that it must have {{\sl\bfseries nothing to do with meaning or truth}\relax}, as those do not comprise its working definition but one should view it as follows
\begin{myquote}
It is not the theory of language--that's linguistics-- or the theory of meaning or truth--that's philosophy-- but the theory of textuality as well as the history of textualized meaning.\hfill (Pollock 2009:934)
\end{myquote}

Additionally he opines that
\begin{myquote}
...if mathematics is the language of the book of nature, as Galileo taught, philology is the language of the book of humanity.\hfill (Pollock 2009:934)
\end{myquote}

His ``working'' definition of Philology reaches a climax with this rather breathless conclusion
\begin{myquote}
Thus, both in theory and in practice across time and space, philology merits the same centrality among the disciplines as philosophy or mathematics.\hfill (Pollock 2009:934)
\end{myquote}

This begs the question - when will the West have a Nobel for philology? No prizes for guessing its first recipient.

In Section 3 of this paper, Pollock unravels some of the methods of his analysis
\begin{myquote}
I map out three domains of history, or rather of meaning in history, that are pertinent to philology: {\bf textual meaning, contextual meaning, and the philologist's meaning}. I differentiate the first two by a useful analytical distinction drawn in Sanskrit thought between {\sl paramarthika} sat and {\sl vyavaharika sat}-- ultimate and pragmatic truth, perhaps better translated with Vico's {\sl verum} and {\sl certum}.

\hfill (Pollock 2009:950) (Emphasis mine)
\end{myquote}

On textual meaning, he says
\begin{myquote}
People often lie,... and so do texts. It may not be very fashionable to say so these days, but the lies and truths of texts must remain a prime object of any future philology....

We should not throw out the baby of textual truth, however, with the bathwater of Orientalism past or present.\hfill (Pollock 2009:951-952)
\end{myquote}

Pollock thus opens the Pandora's box of ``free'' interpretation that can be assigned to any text, to mean anything independent of context - by ascribing it to {\bf some sort of hidden intent} by the author.

On contextual meaning, he says,
\begin{myquote}
Here what has primacy is ``seeing things their way,''... that is, the meaning of a text for historical actors.\hfill (Pollock 2009:954)
\end{myquote}

Again, strangely, he introduces the method of adding agents into the frame of text, other than the author, to whom the actual text can supposedly refer to. This gives the ``interpreter'' additional degrees of freedom to build context around any text.

On the third and final meaning, the Philologist's meaning, Pollock says
\begin{myquote}
The interpretive circle here can be a virtuous one, and we can tack back and forth between prejudgment and text to achieve real historical understanding....

We somehow assume we can escape our own moment in capturing the moment of historical others, and we elevate the knowledge thereby gained into knowledge that is supposed to be not itself historical, but unconditionally true.\hfill (Pollock 2009:957)
\end{myquote}

The act of interpreting a text to make it mean what we want it to mean based on some pre-existing ideology, personal affiliation can somehow be elevated as historical truth.

It gets murkier:
\begin{myquote}
Discovering the meaning of such texts by understanding and interpreting them and discovering how to apply them ... in relation to one's own life, are not separate actions but a single process. And the principle here holds for all interpretation; {\sl applicatio} is not optional but integral to understanding. Historical objects of inquiry, accordingly, do not exist as natural kinds, but, on the contrary, they only emerge as historical objects from our present-day interests.\hfill (Pollock 2009:958)
\end{myquote}

So now, any kind of meaning that the interpreter (Philologist) wants to see or ascribe, based on personal prejudice or understanding of the subject matter becomes possible. Present day context can be reflected back to historical events. Examples include theories such as the Ramayana being used as a rabble-rousing  political text - then, applying it  to the events in Ayodhya, the conclusions of Deep Orientalism - then linking Sanskrit to the rise of the Nazis etc. See (Pollock 2014), section on {\sl Reading the Sanskrit Tradition} for proof of this method in action).

The most surprising part, however, is the conclusion:
\begin{myquote}
There is, thus, no inherent contradiction between historical truth and application.... It's time we got clear on two things. Historical knowledge does not stand in some sort of fundamental contradiction with truth. Nor does it demand our impartiality; objectivity does not entail neutrality.\hfill (Pollock 2009:958)
\end{myquote}

So, what this means is that using these methods any sort of contextual interpretation and conclusion can be created. Based purely on application of one's own interest and life experiences, any sort of fabricated interpretations of ``text'' becomes somehow valid. The resulting interpretation can now be called the historical truth!

At this point, it must be asserted, the logic governing Pollock's philology and the ``scientific'' method are poles apart. {\bf A serious re-consideration of such fallacious methods by scholars like Sheldon Pollock is necessary}.

That Pollock prefers to read texts, primarily with a political lens is an observation made in Malhotra (2016:205). In a more recent paper (Pollock 2014), Pollock seems to have developed his theories a bit more, and couched them in new vocabulary. He calls it the three dimensions of Philology. His methods and intent are well camouflaged this time around too, but he does rather bravely reveal his motive in the abstract.
\begin{myquote}
Enacting philology in three dimensions requires a delicate balance -- essential if we are to cultivate the important {{\sl\bfseries political--ethical}\relax} values that are only possible by learning to read well.\hfill (Pollock 2014:398)
\end{myquote}

In the conclusion, he aptly summarises the absolutely ``free-style'' intentional nature of his philology
\begin{myquote}
At the same time, and more positively now, philology on Plane 1 (historicism) helps us to better comprehend the nature, or natures, of human existence and the radical differences it has shown over time, that is, the vast variety of ways of being human. Philology on Plane 2 (traditionism) helps us to better understand and to develop patience for the views of others, and so to expand the possibilities of human solidarity ({\bf this is the great value of reading a deep and distant past like India's, since it is precisely the presence of a long and very unfamiliar history of reading and interpretation that lets us exercise so effectively the virtues of the quest for understanding and solidarity}). Philology on Plane 3 (presentism) helps us to come to understand our own historicity and our relationship to all earlier historical interpretations, including the originary, and thereby to gain a new humility for the limits of our capacity to know and a new respect for the importance to keep trying. It may well be there are other intellectual practices that can teach us these lessons both negative and positive, but {\bf I know of none that can do so as consistently and immediately as reading well through the discipline of philology}.

\hfill (Pollock 2014:411) (Emphasis mine)
\end{myquote}

These methods (diagnostic political philology and prescriptive liberation philology) of Pollock are going global and are being used rather ``freely'' to interpret in what could only be called as an {\sl anything-goes} style. For more on the growing global footprint of these rather questionable methods, see Pollock et al. (2015). For a discussion of similar critical methods used in the earlier German school refer to Adluri and Bagchee (2013).

\section*{PūrvaPakṣa}

In the previous section, we briefly examined some of the methods in Western Indology, specifically those of Pollock.  We now examine his methods and work using the lenses of Western methods on the one hand and those of traditional Indian perspectives on the other. We use this new understanding to throw light on Pollock's views on Indian {\sl śāstra} (Pollock 1985).

The aforementioned paper is presented in 3 distinct sections besides the Introduction in which Pollock intentionally misuses a quote by Naipaul, see the blogpost ``Pūrvapaksha of Sheldon Pollock'' (2016). Pollock begins to frame the argument for {\sl śāstra} being theory by referring to bodies of rules and codes in the {\sl dharmaśāstra}-s and {\sl Manusmṛti}. He creates an initial bias in the reader by citing two examples from texts which were meant to be a collection of rules by their very design.

The next three sections mirror the appropriately titled sections in Pollock's paper. By his own admission, it is only the first section that is dealt with in detail wherein he makes some logically valid claims and the remaining two are dealt with cursorily and speculatively.

{\bf 1. The relationship of {{\sl\bfseries śāstra}\relax} to its Object}

To establish that {\sl śāstra} is a set of rules, Pāṇini and Patañjali are quoted selectively - {\sl śāstrārtha-sampratyaya} (intention of a rule), {\sl śāstrato hi nāma vyavasthā} (usage constraints) respectively (Pollock 1985:501). That their domain is {\sl vyākaraṇa} or grammar, which is an encapsulation of language understanding in terms of ({\sl sūtra}-s) rules is conveniently ignored. What does one have in grammar if not a few rules? 

The first formal definition of {\sl śāstra} as theory is supposedly by the Mīmāṁsaka viz. Kumārila-bhaṭṭa
\begin{myquote}
``Śāstra'', we are told by the great eighth-century Mīmāṁsāka Kumārila Bhaṭṭa, ``is that which teaches people what they should and should not do. It does this by means of eternal [words] or those made [by men]. Descriptions of the nature [of things, states] can be embraced by the term {\sl śāstra}, insofar as they are elements subordinate [to injunctions to action].''

{\sl Śāstra} is thus, according to the standard definition, a verbal codification of rules, whether of divine or human provenance, for the positive and negative regulation of some given human practices.\hfill (Pollock 1985:501)
\end{myquote}

Pollock very correctly identifies the Mīmāṁsā interpretation of {\sl śāstra vis-à-vis śruti} - the Veda and also of the relationship to the {\sl upaveda}-s, {\sl vedāṅga}-s and  {\sl vidyāsthāna}-s. He exhaustively lists multiple catalogues of {\sl śāstra} and their evolution.

That {\sl śāstra} is the original source of knowledge is acknowledged but then he digresses on some supposed {\sl bivalency} between rules and revelation. It appears that this is due to Pollock's misunderstanding of revelation of {\sl śruti} being similar to revelation akin to Abrahamic books.

After this point, inexplicably, Pollock assumes that {\sl śāstra} means {\bf theory}, which is incorrect. The {\sl śāstra} have been derived and codified based on actual experience, and are encoded experiences driven by practice, and not some arbitrary theory (like scientific theories) based on guesswork, the principles of falsifiability and third person empiricism.

We need to internalise what Aurobindo has to say about this :
\begin{myquote}
To follow the law of desire is not the true rule of our nature; there is a higher and juster standard of its works. But where is it embodied or how is it to be found ? In the first place, the human race has always been seeking for this just and high Law and {\bf whatever it has discovered is embodied in its {{\sl\bfseries śāstra}\relax}}, its rule of science and knowledge, rule of ethics, rule of religion, rule of best social living, rule of one's right relations with man and God and Nature. {\sl Śāstra} {\bf does not mean a mass of customs, some good, some bad, unintelligently followed by the customary routine mind of the tāmasic man. {{\sl\bfseries Śāstra}\relax} is the knowledge and teaching laid down by intuition, experience and wisdom, the science and art and ethic of life, the best standards available to the race.}\hfill (Aurobindo 1997:229) (Emphasis mine)
\end{myquote}

Also, Gandhi in his lectures on the Gītā, explicitly makes it clear, {\sl śāstra} is very much contextual, and deeply influenced by its practice
\begin{myquote}
He who forsakes the rule of {\sl śāstra} and does but the bidding of his selfish desires, gains neither perfection, nor happiness, nor the highest state.

{\sl Śāstra} does not mean the rites and formulae laid down by so called dharma shastra, but {\bf the path of self-restraint laid down by the seers and the saints.}

Therefore let {\sl śāstra} be thy authority for determining what ought to be done and what ought not to be done; {{\sl\bfseries ascertain}\relax} {\bf thou the role of the} {{\sl\bfseries śāstra}\relax} {\bf and do thy task here} (accordingly).\hfill (Sahadeo 2012:115) (Emphasis mine)
\end{myquote}

Pollock then makes a rather dubious claim that the Indian intellectual history somehow assumes a `finite set of topics of knowledge'. No such claims were made by Rājaśekhara whom he quotes as an authority (cataloguing the types of {\sl śāstra}).

Now that he assumes that {\sl śāstra} means theory, albeit without properly establishing it, Pollock then addresses the relationship between theory and its practice ({\sl prayoga}). He poses the question thus 
\begin{myquote}
If {\sl śāstra} is the systematic exposition of some knowledge, what does the Indian intellectual tradition conceive to be the relationship of this exposition to the actual enactment of the knowledge? How, that is, are theory ({\sl śāstra}) and practice ({\sl prayoga}) viewed as interrelated'? What is the causal -- or more grandly, ontological -- relation that is thought to subsist between the two?\hfill (Pollock 1985:504)
\end{myquote}

He then goes on to examine specific cases and cites selective examples from Vyākaraṇa, Dharmaśāstra, Kāmasūtra which supposedly exhort the practitioner to abide strictly by injunction and rules. He implies that any sort of failure experienced by the practitioner is exclusively ``his'' and not of the {\sl śāstra}.

He then proceeds to make cryptic statements such as 
\begin{myquote}
All knowledge derives from {\sl śāstra}; success in astrology or in the training of horses and elephants, no less than in language use and social intercourse, is achieved only because the rules governing these practices have percolated down to the practitioners - not because they were discovered independently through the creative power of practical consciousness - ``however far removed'' from the practitioners the {\sl śāstra} may be.\hfill (Pollock 1985:507)
\end{myquote}

The point being made is not clear at all - Does Pollock expect every individual to write his/her own {\sl śāstra}? Every civilisation has rules which encode its cultural goals and orients its practitioners to them. In the {\sl dharmic} way of living, human progress via pursuit of {\sl puruṣārtha}-s (the goals of existence) toward possibly a final Mokṣa (liberation) is the overarching goal of all {\sl śāstra}-s. The {\sl kārmic-dhārmic} worldview which includes causal chains across multiple births is fundamental to Vedic cosmology without which, any reading of any {\sl śāstra} will be flawed. The deliberate sidelining of this perspective - which is possibly beyond the scope of Pollock's gaze - seems to be one of the principal causes of the nescience.

He then proceeds to cite Rāmānuja from the {\sl Śrī Bhāṣya} which is a commentary on the {\sl Brahma Sūtra}-s - a {\sl mokṣa śāstra}, where such emphasis is being made.
\begin{myquote}
``{\sl Śāstra} is so called because it instructs; instruction leads to action, and {\sl śāstra} has this capacity to lead to action by reason of its producing knowledge.'' The actual program of spiritual liberation enacts this postulate, since for Rāmānuja, {\sl śāstra} forms the sole means for attaining {\sl mokṣa}.\hfill (Pollock 1985:509)
\end{myquote}

The proverbial elephant in the room is staring right at him - So how does Pollock react? - he does not. He moves on, instead, to cite the rules from Kauṭilya's {\sl Arthaśāstra}.

To juxtapose this new construct of the traditional Indian way of theory over practice with Western thinking, Pollock quotes Ryle and Aristotle, but an actual full reading of the reference implies something else.

He quotes Ryle saying,
\begin{myquote}
But most people today I think would readily accept the commonsense assessment of Ryle, that ``efficient practice precedes the theory of it''.\hfill (Pollock 1985:510)
\end{myquote}

The same reference by Ryle also says this,
\begin{myquote}
The common-sense position has not, however, gone unchallenged. Suggestive is Popper's epistemological conjecture that theories or expectations, logically speaking, must predetermine experience; that our dispositions and in fact senses are ``{\bf theory-impregnated}''.\hfill (Pollock 1985:511) (Emphasis mine)
\end{myquote}

To sum up, Ryle says any practice is a priori influenced by theory.

The fact that theory and practice are a feedback-driven continuum over multiple generations in dhārmic traditions, encoding newer experiences into {\sl śāstra}-s which are rewritten according to the times, is something that Pollock has unfortunately ignored. A similar sentiment is echoed by Ryle.
\begin{myquote}
Still others wonder whether the dichotomy between theory and practice may not itself be more theoretical than practical.... Is it not more intuitive, however, to think that theory evolves out of practice and will itself evolve as practice refines and modifies itself'?\hfill (Pollock 1985:511) (Emphasis mine)
\end{myquote}

Now that he realises that he has possibly argued himself into a corner he concludes the first section by suddenly switching over to {\sl dharma} - 
\begin{myquote}
To simplify a complex argument, we may say that {\sl dharma} in the largest sense connotes the correct way of doing anything. From the Mīmāṁsā perspective, the prevailing one from which the rest of shastric discourse is extrapolated, {\sl dharma} is by definition ``rule-boundedness'' ({\sl codanālakṣaṇa}), and the rules themselves are encoded in {\sl śāstra} ({\sl upadeśa}).\hfill (Pollock 1985:511)
\end{myquote}

He seems to have made a deep inference that Indians follow {\sl dharma} $\to$ {\sl Dharma} is a function of rule-boundedness $\to$ Rules are encoded in {\sl śāstra}. So - what happened to the claim of an underlying operational {\sl theory} ?

This is followed by what could only be called an act of pure reasoning, a hail-maryi$^{2}$ pass to Kant
\begin{myquote}
But rules, as we have known since Kant, are either constitutive or regulative, (the rules of chess and those of dinner table etiquette would be respective examples). Shastric discourse collapses the two, enunciating both in the same injunctive mood.\hfill (Pollock 1985:511)
\end{myquote}

By this sleight of hand, Pollock with some philological magic and invocation of Kant has transformed {\sl śāstra}-s to become objects of political control. So now {\sl śāstra}, somehow because of their supposed regulative nature, make it possible for human social and sexual intercourse to {\bf become amenable to codified legislative control}.

The objective of this section of Pollock's paper was to layout the relationship of {\sl śāstra} to its object - Pollock has argued for interpreting {\sl śāstra} with a political reading and successfully places it in what he calls the larger discourse of power.

Although Pollock has already mentioned that the remaining two sections are dealt with cursorily and speculatively, we shall continue to examine the arguments made.

{\bf 2. The implications of the priority of theory}

Having somehow established the fact that {\sl śāstra} is theory bereft of actual practice (via Pollockian logic), his intent in the second section is to show the effects of prioritizing theory to practice. (Ref. Pollock 1985:512-516)

{\sl Śāstra}-s ascribe their origins to seers and {\sl ṛṣi}-s of the past, individuals with a higher consciousness, those who could access and experience and verify by practice such knowledge; and the {\sl śāstra}-s are now categorised as {\bf myth}!

Myth in the Judeo-Christian (and postmodern) discourse refers to any text/work, that does not have official sanction and that which is not considered true or factual by some central authority or history - controlling ``religious'' body ex: Church, Rabbinic Council etc. Non-Abrahamic traditions (especially dhārmic ones) are not ``centrally'' controlled nor are they history-centric nor do they have centralised institutions. By labeling Indian {\sl śāstra} as {\sl myths}, the credibility of these ``so-labelled'' works are automatically questioned. For a more comprehensive discussion of this label -  ``myth'' - and for more on the current normative on the ``scientific'' view of myth in Western scholarship see Jung (1963).

The modus is generally something like this --
\begin{itemize}
\item[{\bf(1)}] {\bf A people P refer to X (a body of knowledge unknown to the west) as a guideline/reference}

\item[{\bf(2)}] {\bf Call X  a myth} 

\item[{\bf(3)}] {\bf Imply that X is probably untrue}

\item[{\bf(4)}] {\bf People P who refer to X are primitive and regressive}

\item[{\bf(5)}] {\bf The West (which does not use/refer to X) is superior to P}

\item[{\bf(6)}] {\bf In case there is an inkling of something good/monetizable, appropriate X, rename X, call X a Western invention} 
\end{itemize}

(The last step is a phenomenon called digestion and is part of a larger phenomenon of a cultural U-turn. See Malhotra (2011) for definitions and discussion).

This method has been repeatedly used in Indological studies and is also seen as a general phenomenon in the appropriation of traditional knowledge and intellectual property of non-Western civilisations into the larger capitalist Western universal narrative called Science. This is nothing new in the history of the encounter of the West with the non-West. See Raju (2009) for a discussion. The deconstructive reader should be aware that such a mental disposition is in most cases the primary {\sl saṁskāra} of the Western academic/intellectual.

The {\sl Purāṇa}-s encapsulate enormous amount of secular knowledge, which Pollock acknowledges.

On the {\sl Agni Purāṇa}, he says
\begin{myquote}
What Agni goes on to reveal is an encyclopedic synthesis of human knowledge, including what is in fact a vast array of discrete {\sl śāstras} on topics as diverse as {\sl dharma}, architecture and iconology, astronomy, divination, the lapidary's art, the science of weapons, arboriculture, veterinary medicine, metrics, phonetics, grammar, and rhetoric.

\hfill (Pollock 1985:514)
\end{myquote}

%page 25

