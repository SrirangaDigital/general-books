
%%01_kishkindhaakaandam

\chapter{കിഷ്കിന്ധാകാണ്ഡം}

\begin{verse}
ശാരികപ്പൈതലേ! ചാരുശീലേ! വരി-\\
കാരോമലേ! കഥാശേഷവും ചൊല്ലു നീ.\\
ചൊല്ലുവേനെങ്കിലനംഗാരി ശങ്കരന്‍\\
വല്ലഭയോടരുള്‍ചെയ്ത പ്രകാരങ്ങള്‍.\\
കല്യാണശീലന്‍ ദശരഥസൂനു കൗ-\\
സല്യാതനയനവരജന്‍ തന്നോടും\\
പമ്പാസരസ്തടം ലോകമനോഹരം\\
സംപ്രാപ്യ വിസ്മയം പൂണ്ടരുളീടിനാന്‍\\
ക്രോശമാത്രം വിശാലം വിശദാമൃതം\\
ക്ലേശവിനാശനം ജന്തുപൂര്‍ണസ്ഥലം\\
ഉല്‍ഫുല്ലപദ്മകല്ഹാര കുമുദ നീ-\\
ലോല്പലമണ്ഡിതം ഹംസകാരണ്ഡവ\\
ഷഡ്പദകോകില കുക്കുടകോയഷ്ടി\\
സര്‍പസിംഹവ്യാഘ്രസൂകരസേവിതം\\
പുഷ്പലതാപരിവേഷ്ടിതപാദപ്-\\
സല്‍ഫലസേവിതം സന്തുഷ്ടജന്തുകം\\
കണ്ടു കൗതൂഹലം പൂണ്ടു തണ്ണീര്‍ കുടി-\\
ച്ചിണ്ടലും തീര്‍ത്തു മന്ദം നടന്നീടിനാര്‍
\end{verse}

%%02_hanumalsamaagamam

\section{ഹനൂമല്‍ സമാഗമം}

\begin{verse}
കാലേ വസന്തേ സുശീതളേ ഭൂതലേ\\
ഭൂലോകപാലബാലന്മാരിരുവരും\\
ഋശ്യമൂകാദ്രിപാര്‍ശ്വസ്ഥലേ സന്തതം\\
നിശ്വാസമുള്‍ക്കൊണ്ടു വിപ്രലാപത്തോടും\\
സീതാവിരഹം പൊറാഞ്ഞു കരകയും\\
പൂതായുധാര്‍ത്തി മുഴുത്തു പറകയും\\
ചൂതായുധാര്‍ത്തി മുഴുത്തു പറകയും\\
ആധികലര്‍ന്നു നടന്നടുക്കും വിധൗ\\
ഭീതനായ് വന്നു ദിനകരപുത്രനും\\
സത്വരം മന്ത്രികളോടുകുതിച്ചു പാ-\\
ഞ്ഞുത്തുംഗമായ ശൈലാഗ്രമേറീടിനാന്‍\\
മാരുതിയോടു ഭയേന ചൊല്ലീടിനാന്‍:\\
‘ആരീവരുന്നതിരുവര്‍ സന്നദ്ധരായ്?\\
നേരെ ധരിച്ചു വരിക നീ വേഗേന\\
ധീരന്മാരെത്രയുമെന്നു തോന്നും കണ്ടാല്‍\\
അഗ്രജന്‍ ചൊല്കയാലെന്നെബ്ബലാലിങ്ങു\\
നിഗ്രഹിപ്പാനായ് വരുന്നവരല്ലല്ലീ?\\
വിക്രമമുള്ളവരെത്രയും തേജസാ\\
ദിക്കുകളൊക്കെ വിളങ്ങുന്നു കാണ്‍ക നീ.\\
താപസവേഷം ധരിച്ചിരിക്കുന്നിതു\\
ചാപബാണാസിശസ്ത്രങ്ങളുമുണ്ടല്ലോ\\
നീയൊരു വിപ്രവേഷം പൂണ്ടവരോടു\\
വായുസുത! ചെന്നു ചോദിച്ചറിയണം.\\
വക്ത്രനേത്രാലാപഭാവങ്ങള്‍ കൊണ്ടവര്‍\\
ചിത്തമെന്തെന്നതറിഞ്ഞാല്‍ വിരവില്‍ നീ\\
ഹസ്തങ്ങള്‍കൊണ്ടറിയിച്ചീടു നമ്മുടെ\\
ശത്രുക്കളെങ്കി,ലതല്ലെങ്കില്‍ നിന്നുടെ\\
വക്ത്രപ്രസാദമന്ദസ്മേരസംജ്ഞയാ\\
മിത്രമെന്നുള്ളതുമെന്നോടു ചൊല്ലണം.’\\
കര്‍മസാക്ഷീസുതന്‍വാക്കുകള്‍ കേട്ടവന്‍\\
ബ്രഹ്മചാരിവേഷമാലംബ്യ സാദരം\\
അഞ്ജസാ ചെന്നു നമസ്കരിച്ചീടിനാ-\\
നഞ്ജനാപുത്രനും ഭര്‍ത്തൃപാദാംബുജം.\\
കഞ്ജവിലോചനന്മാരായ മാനവ-\\
കുഞ്ജരന്മാരെത്തൊഴുതു വിനീതനായ്:\\
‘അംഗജന്‍ തന്നെ ജയിച്ചോരു കാന്തിപൂ-\\
ണ്ടിങ്ങനെ കാണായ നിങ്ങളിരുവരും\\
ആരെന്നറികയിലാഗ്രഹമുണ്ടതു\\
നേരെ പറയണമെന്നോടു സാദരം\\
ദിക്കുകളാത്മഭാസൈവ ശോഭിപ്പിക്കു-\\
മര്‍ക്കനിശാകരന്മാരെന്നു തോന്നുന്നു.\\
ത്രൈലോക്യകര്‍ത്തൃഭൂതന്മാര്‍ ഭവാന്മാരെ-\\
ന്നാലോക്യ ചേതസി ഭാതി സദൈവ മേ\\
വിശ്വൈകവീരന്മാരായ യുവാക്കളാ-\\
മശ്വിനീദേവകളോ മറ്റതെന്നിയേ\\
വിശ്വൈകകാരണഭൂതന്മാരായോരു\\
വിശ്വരൂപന്മാരാമീശ്വരന്മാര്‍ നിങ്ങള്‍\\
നൂനം പ്രധാനപുരുഷന്മാര്‍ മായയാ\\
മാനുഷാകാരേണ സഞ്ചരിക്കുന്നിതു\\
ലീലയാ ഭൂഭാരനാശനാര്‍ഥം പരി-\\
പാലനത്തിന്നു ഭക്തനാം മഹീതലേ.\\
വന്നു രാജന്യവേഷേണ പിറന്നൊരു\\
പുണ്യപുരുഷന്മാര്‍ പൂര്‍ണഗുണവാന്മാര്‍\\
കര്‍ത്തും ജഗല്‍സ്ഥിതിസംഹാരസര്‍ഗങ്ങ-\\
ളുദ്യതൗ ലീലയാ നിത്യസ്വതന്ത്രന്മാര്‍.\\
മുക്തി നല്കും നരനാരായണന്മാരെ-\\
ന്നുള്‍ത്താരിലിന്നു തോന്നുന്നു നിരന്തരം.’\\
ഇത്ഥം പറഞ്ഞു തൊഴുതുനിന്നീടുന്ന\\
ഭക്തനെക്കണ്ടു പറഞ്ഞു രഘൂത്തമന്‍:\\
‘പശ്യ സഖേ! വടുരൂപിണം ലക്ഷ്മണാ!\\
നിശ്ശേഷശബ്ദശാസ്ത്രമനേനശ്രുതം\\
ഇല്ലൊരപശബ്ദമെങ്ങുമേ വാക്കിങ്കല്‍\\
നല്ല വൈയാകരണന്‍ വടു നിര്‍ണയം.’\\
മാനവവീരനുമപ്പോളരുള്‍ചെയ്തു\\
വാനരശ്രേഷ്ഠനെ നോക്കി ലഘുതരം:\\
‘രാമനെന്നെന്നുടെ നാമം ദശരഥ-\\
ഭൂമിപാലേന്ദ്രതനയ, നിവന്‍ മമ\\
സോദരനാകിയ് ലക്ഷ്മണന്‍, കേള്‍ക്ക നീ\\
ജാതമോദം പരമാര്‍ത്ഥം മഹാമതേ!\\
ജാനകിയാകിയ സീതയെന്നുണ്ടൊരു\\
മാനിനിയെന്നുടെ ഭാമിനി കൂടവേ\\
താതനിയോഗേന കാനനസീമനി\\
യാതന്മാരായി തപസ്സു ചെയ്തീടുവാന്‍\\
ദണ്ഡകാരണ്യേ വസിക്കുന്ന നാളതി-\\
ചണ്ഡനായോരു നിശാചരന്‍ വന്നുടന്‍\\
ജാനകീദേവിയെക്കട്ടുകൊണ്ടീടിനാന്‍\\
കാനനേ ഞങ്ങള്‍ തിരഞ്ഞു നടക്കുന്നു.\\
കണ്ടീലവളെയൊരേടത്തു നിന്നിഹ\\
കണ്ടുകിട്ടീ നിന്നെ, നീയാരെടോ സഖേ!\\
ചൊല്ലീടു’കെന്നതു കേട്ടൊരു മാരുതി\\
ചൊല്ലിനാന്‍ കൂപ്പിത്തൊഴുതു കുതൂഹലാല്‍:\\
‘സുഗ്രീവനാകിയ വാനരേന്ദ്രന്‍ പര്‍വ-\\
താഗ്രേ വസിക്കുന്നിതത്ര രഘുപതേ!\\
മന്ത്രികളായ് ഞങ്ങള്‍ നാലുപേരുണ്ടല്ലോ\\
സന്തതം കൂടെപ്പിരിയാതെ വാഴുന്നു.\\
അഗ്രജനാകിയ ബാലി കപീശ്വര-\\
നുഗ്രനാട്ടിക്കളഞ്ഞീടിനാന്‍ തമ്പിയെ.\\
സുഗ്രീവനുള്ള പരിഗ്രഹംതന്നെയു-\\
മഗ്രജന്‍ തന്നേ പരിഗ്രഹിച്ചീടിനാന്‍.\\
ഋഷ്യമൂകാചലം സങ്കേതമായ് വന്നു\\
വിശ്വാസമോടിരിക്കുന്നിതര്‍ക്കത്മജന്‍\\
ഞാനവന്‍തന്നുടെ ഭൃത്യനായുള്ളോരു\\
വാനരന്‍ വായുതനയന്‍ മഹാമതേ!\\
നാമധേയം ഹനുമാനഞ്ജനാത്മജ-\\
നാമയം തീര്‍ത്തു രക്ഷിച്ചുകൊള്ളേണമേ!\\
സുഗ്രീവനോടു സഖ്യം ഭവാനുണ്ടെങ്കില്‍\\
നിഗ്രഹിക്കാമിരുവര്‍ക്കുമരികളെ.\\
വേലചെയ്യാമതിനാവോളമാശു ഞാ-\\
നാലംബനം മറ്റെനിക്കില്ല ദൈവമേ!\\
ഇത്ഥം തിരുമനസ്സെങ്കിലെഴുന്നള്ളു-\\
കുള്‍ത്താപമെല്ലാമകലും ദയാനിധേ!”\\
എന്നുണര്‍ത്തിച്ചു നിജാകൃതി കൈക്കൊണ്ടു\\
നിന്നു തിരുമുമ്പിലാമ്മാറു മാരുതി.\\
‘പോക മമ സ്കന്ധമേറീടുവിന്‍ നിങ്ങ-\\
ളാകുലഭാവമകലെക്കളഞ്ഞാലും.’\\
അപ്പോള്‍ ശബരിതന്‍ വാക്കുകളോര്‍ത്തുക-\\
ണ്ടുല്പലനേത്രനനുവാദവും ചെയ്തു.
\end{verse}

%%03_sugreevasakhyam

\section{സുഗ്രീവസഖ്യം}

\begin{verse}
ശ്രീരാമലക്ഷ്മണന്മാരെക്കഴുത്തിലാ-\\
മ്മാറങ്ങെടുത്തു നടന്നിതു മാരുതി\\
സുഗ്രീഅവസന്നിധൗ കൊണ്ടു ചെന്നീടിനാന്‍\\
‘വ്യഗ്രം കളക നീ ഭാസ്കരനന്ദന!\\
ഭാഗ്യമഹോ ഭാഗ്യമോര്‍ത്തോളമെത്രയും\\
ഭാസ്കരവംശസമുത്ഭവന്മാരായ\\
രാമനും ലക്ഷ്മണനാകുമനുജനും\\
കാമദാനാര്‍ത്ഥമിവിടേക്കെഴുന്നള്ളി.’\\
സുഗ്രീവനോടിവണ്ണം പറഞ്ഞദ്രീശ്വ-\\
രാഗ്രേ മഹാതരുച്ഛായാതലേ തദാ\\
വിശ്വൈകനായകന്മാരാം കുമാരന്മാര്‍\\
വിശ്രാന്തചേതസാ നിന്നരുളീടിനാര്‍.\\
വാതാത്മജന്‍ പരമാനന്ദമുള്‍ക്കൊണ്ടു\\
നീതിയോടര്‍ക്കാത്മജനോടു ചൊല്ലിനാന്‍:\\
‘ഭീതി കളക നീ മിത്രഗോത്രേ വന്നു\\
ജാതന്മാരായൊരു യോഗേശ്വരന്മാരീ-\\
ശ്രീരാമലക്ഷ്മണന്മാരെഴുന്നള്ളിയ-\\
താരെയും പേടിക്ക വേണ്ട ഭവാനിനി.\\
വേഗേന ചെന്നു വന്ദിച്ചു സഖ്യംചെയ്തു\\
ഭാഗവതപ്രിയനായ് വസിച്ചീടുക.’\\
പ്രീതനായോരു സുഗ്രീവനു മന്നേര-\\
മാദരപൂര്‍വമുത്ഥായ സസംഭ്രമം\\
വിഷ്ടപനാഥനിരുന്നരുളീടുവാന്‍\\
വിഷ്ടരാര്‍ത്ഥം നല്ല പല്ലവജാലങ്ങള്‍\\
പൊട്ടിച്ചവനിയിലിട്ടാ,നതു നേര-\\
മിഷ്ടനാം മാരുതി ലക്ഷ്മണനു മൊടി-\\
ച്ചിട്ടതു കണ്ടു സൗമിത്രി സുഗ്രീവനും\\
പുഷ്ടമോദാലൊടിച്ചിട്ടരുളീടിനാന്‍;\\
തുഷ്ടിപൂണ്ടെല്ലാവരുമിരുന്നീടീനാര്‍;\\
നഷ്ടമായ് വന്നിതു സന്താപസംഘവും.\\
മിത്രാത്മജനോടു ലക്ഷ്മണന്‍ ശ്രീരാമ-\\
വൃത്താന്തമെല്ലാറിയിച്ചതുനേരം\\
ധീരനാമാദിത്യനന്ദനന്‍ മോദേന\\
ശ്രീരാമചന്ദ്രനോടാശു ചൊല്ലീടിനാന്‍:\\
‘നാരീമണിയായ ജാനകീദേവിയെ-\\
യാരാഞ്ഞറിഞ്ഞു തരുന്നുണ്ടുനിര്‍ണയം.\\
ശത്രുവിനാശനത്തിന്നടിയനൊരു\\
മിത്രമായ് വേല ചെയ്യാം തവാജ്ഞാവശാല്‍.\\
ഏതുമിതു നിരൂപിച്ചു ഖേദിക്കരു-\\
താധികളൊക്കെയകറ്റുവന്‍ നിര്‍ണയം.\\
രാവണന്‍തന്നെസ്സകുലംവധം ചെയ്തു\\
ദേവിയേയും കൊണ്ടുപോരുന്നതുണ്ടു ഞാന്‍.\\
ഞാനൊരവസ്ഥ കണ്ടേനൊരുനാളതു\\
മാനവ വീര! തെളിഞ്ഞുകേട്ടീടണം.\\
മന്ത്രിമാര്‍ നാലുപേരും ഞാനുമായച-\\
ലാന്തേ വസിക്കുന്നകാലമൊരുദിനം\\
പുഷ്കരനേത്രയായോരു തരുണിയെ-\\
പ്പുഷ്കരമാര്‍ഗേണ കൊണ്ടുപോയാനൊരു\\
രക്ഷോവരനതുനേരമസ്സുന്ദരി\\
രക്ഷിപ്പതിന്നാരുമില്ലാഞ്ഞു ദീനയായ്\\
രാമരാമേതി മുറയിടുന്നോള്‍, തവ\\
ഭാമിനിതന്നെയവളെന്നതേ വരൂ.\\
‘ഉത്തമയാമവള്‍ ഞങ്ങളെപ്പര്‍വതേ-\\
ന്ദ്രോത്തമാംഗേ കണ്ടനേരം പാരവശാല്‍\\
ഉത്തരീയത്തില്‍ പൊതിഞ്ഞാഭരണങ്ങ-\\
ളദ്രീശ്വരോപരി നിക്ഷേപണംചെയ്താള്‍.\\
ഞാനതു കണ്ടിങ്ങെടുത്തു സൂക്ഷിച്ചു വെ-\\
ച്ചേനതു കാണണമെങ്കിലോ കണ്ടാലും.\\
ജാനകിദേവിതന്നാഭരണങ്ങളോ\\
മാനവവീരാ! ഭവാനറിയാമല്ലോ’\\
എന്നു പറഞ്ഞതെടുത്തു കൊണ്ടുവന്നു\\
മന്നവന്‍ തന്‍ തിരുമുമ്പില്‍ വെച്ചീടിനാന്‍.\\
അര്‍ണോജനേത്രനെടുത്തു നോക്കുന്നേരം\\
കണ്ണുനീര്‍തന്നെ കുശലം വിചാരിച്ചു:\\
‘എന്നെക്കണക്കേ പിരിഞ്ഞിതോ നിങ്ങളും\\
തന്വംഗിയാകിയ വൈദേഹിയോടയ്യോ?\\
സീതേ! ജനകാത്മജേ! മമ വല്ലഭേ!\\
നാഥേ! നളിനദളായതലോചനേ!’\\
രോദനം ചെയ്തു വിഭൂഷണസഞ്ചയ-\\
മാധിപൂര്‍വം തിരുമാറിലമുഴ്ത്തിയും\\
പ്രാകൃതന്മാരാം പുരുഷന്മാരെപ്പോലെ\\
ലോകൈകനാഥന്‍ കരഞ്ഞു തുടങ്ങിനാന്‍.\\
ശോകേന മോഹം കലര്‍ന്നു കിടക്കുന്ന\\
രാഘവനോടു പറഞ്ഞിതു ലക്ഷ്മണന്‍:\\
‘ദുഃഖിയായ്കേതുമേ രാവണന്‍തന്നെയും\\
മര്‍ക്കടശ്രേഷ്ഠസഹായേന വൈകാതെ\\
നിഗ്രഹിച്ചംബുജനേത്രയാം സീതയെ-\\
ക്കൈക്കൊണ്ടുകൊള്ളാം പ്രസീദ പ്രഭോ! ഹരേ!’\\
സുഗ്രീവനും പറഞ്ഞാനതു കേട്ടുടന്‍:\\
‘വ്യഗ്രിയായ്കേതുമേ രാവണന്‍ തന്നെയും\\
നിഗ്രഹിച്ചാശു നല്കീടുവന്‍ ദേവിയെ\\
കൈക്കൊള്‍ക ധൈര്യംധരിത്രീപതേ! വിഭോ!’\\
ലക്ഷ്മണസുഗ്രീവവാക്കുകളിങ്ങനെ\\
തല്‍ക്ഷണം കേട്ടു ദശരഥപുത്രനും\\
ദുഃഖവുമൊട്ടു ചുരുക്കി മരുവിനാന്‍;\\
മര്‍ക്കടശ്രേഷ്ഠനാം മാരുതിയന്നേരം\\
അഗ്നിയേയും ജ്വലിപ്പിച്ചു ശുഭമായ\\
ലഗ്നവും പാര്‍ത്തു ചെയ്യിപ്പിച്ചു സഖ്യവും\\
സുഗ്രീവരാഘവന്മാരഗ്നിസാക്ഷിയായ്\\
സഖ്യവും ചെയ്തു പരസ്പരംകാര്യവും\\
സിദ്ധിക്കുമെന്നുറച്ചാത്മഖേദം കള-\\
ഞ്ഞുത്തുംഗമായ ശൈലാഗ്രേ മരുവിനാര്‍\\
ബാലിയും താനും പിണക്കമുണ്ടായതിന്‍-\\
മൂലമെല്ലാമുണര്‍ത്തിച്ചരുളീടിനാന്‍.
\end{verse}

%%04_baalisugreevakalahakatha

\section{ബാലിസുഗ്രീവ കലഹകഥ}

\begin{verse}
പണ്ടു മായാവിയെന്നൊരസുരേശ്വര-\\
നുണ്ടായിതു മയന്‍ തന്നുടെ പുത്രനായ്\\
യുദ്ധത്തിനാരുമില്ലാഞ്ഞു മദിച്ചവ-\\
നുദ്ധതനായ് നടന്നീടും ദശാന്തരേ\\
കിഷ്കിന്ധയാം പുരി പുക്കു വിളിച്ചിതു\\
മര്‍ക്കടാധീശ്വരനാകിയ ബാലിയെ.\\
യുദ്ധത്തിനായ് വിളിക്കുന്നതു കേട്ടതി-\\
ക്രുദ്ധനാം ബാലി പുറപ്പെട്ടു ചെന്നുടന്‍\\
മുഷ്ടികള്‍കൊണ്ടു താഡിച്ചതു കൊണ്ടതി-\\
ദുഷ്ടനാം ദൈത്യനും പേടിച്ചു മണ്ടിനാന്‍\\
വാനരശ്രേഷ്ഠനുമോടിയെത്തീടിനാന്‍\\
ഞാനുമതു കണ്ടു ചെന്നിതു പിന്നാലെ\\
ദാനവന്‍ ചെന്നു ഗുഹലിലുള്‍പ്പുക്കിതു.\\
വാനരശ്രേഷ്ഠനുമെന്നോടു ചൊല്ലിനാന്‍:\\
ഞാനിതില്‍ പുക്കിവന്‍തന്നെയൊടുക്കുവന്‍\\
നൂനം വിലദ്വാരി നില്ക്ക നീ നിര്‍ഭയം\\
ക്ഷീരം വരികിലസുരന്‍ മരിച്ചീടും\\
ചോരവരികിലടച്ചുപോയ് വാഴ്ക നീ.”\\
ഇത്ഥം പറഞ്ഞതില്‍ പുക്കിതു ബാലിയും\\
തത്ര വിലദ്വാരി നിന്നേനടിയനും.\\
പോയിതുകാലമൊരുമാസമെന്നിട്ടു-\\
മാഗതനായതുമില്ല കപീശ്വരന്‍\\
വന്നിതു ചോര വിലമുഖം തന്നില്‍നി-\\
ന്നെന്നുള്ളില്‍നിന്നു വന്നൂ പാരിതാപവും.\\
അഗ്രജന്‍തന്നെ മായാവി മഹാസുരന്‍\\
നിഗ്രഹിച്ചാനെന്നുറച്ചു ഞാനും തദാ\\
ദുഃഖമുള്‍ക്കൊണ്ടു കിഷ്കിന്ധ പുക്കീടിനാനേന്‍;\\
മര്‍ക്കടവീരരും ദുഃഖിച്ചതുകാലം.\\
വാനരാധീശ്വരനായഭിഷേകവും\\
വാനരേന്ദ്രന്മാരെനിക്കു ചെയ്തീടിനാര്‍.\\
ചെന്നിതുകാലം കുറഞ്ഞൊന്നു പിന്നെയും\\
വന്നിതു ബാലി മഹാബലവാന്‍ തദാ.\\
കല്ലിട്ടു ഞാന്‍ വിലദ്വാരമടച്ചതു\\
കൊല്ലുവാനെന്നോര്‍ത്തു കോപിച്ചു ബാലിയും\\
കൊല്ലുവാനെന്നോടടുത്തു, ഭയേന ഞാ-\\
നെല്ലാടവും പാഞ്ഞിരിക്കരുതാഞ്ഞെങ്ങും.\\
നീളേ നടന്നുഴന്നീടും ദശാന്തരേ\\
ബാലിവരികയില്ലത്ര ശാപത്തിനാല്‍-\\
ഋശ്യമൂകാചലേ വന്നിരുന്നീടിനേന്‍\\
വിശ്വാസമോടു ഞാന്‍ വിശ്വനാഥാ! വിഭോ!\\
മൂഢനാം ബാലി പരിഗ്രഹിച്ചീടിനാ-\\
നൂഢരാഗം മമ വല്ലഭതന്നെയും\\
നാടും നഗരവും പത്നിയുമെന്നുടെ\\
വീടും പിരിഞ്ഞു ദുഃഖിച്ചിരിക്കുന്നു ഞാന്‍.\\
ത്വല്‍പാദപങ്കേരുഹസ്പര്‍ശ കാരണാ-\\
ലിപ്പോളതീവ സുഖവുമുണ്ടായ് വന്നു.\\
മിത്രാത്മജോക്തികള്‍ കേട്ടോരനന്തരം\\
മിത്രദുഃഖേന സന്തപ്തനാം രാഘവന്‍\\
ചിത്തകാരുണ്യം കലര്‍ന്നു ചൊന്നാന്‍, ’തവ\\
ശത്രുവിനെക്കൊന്നു പത്നിയും രാജ്യവും\\
വിത്തമുമെല്ലാമടക്കിത്തരുവന്‍ ഞാന്‍\\
സത്യമിതു രാമഭാഷിതം കേവാലം.’\\
മാനവേന്ദ്രോക്തികള്‍ കേട്ടു തെളിഞ്ഞൊരു\\
ഭാനുതനയനുമിങ്ങനെ ചൊല്ലിനാന്‍:\\
‘സ്വര്‍ലോകനാഥജനാകിയ ബാലിയെ-\\
ക്കൊല്ലുവനേറ്റം പണിയുണ്ടു നിര്‍ണയം,\\
ഇല്ലവനോളം ബലം മറ്റൊരുവനും\\
ചൊല്ലുവന്‍ ബാലിതന്‍ ബാഹുപരാക്രമം.\\
ദുന്ദുഭിയാകും മഹാസുരന്‍ വന്നു കി-\\
ഷ്കിന്ധാപുരദ്വാരി മാഹിഷവേഷമായ്\\
യുദ്ധത്തിനായ് വിളിച്ചോരു നേരത്തതി-\\
ക്രുദ്ധനാം ബാലി പുറപ്പെട്ടു ചെന്നുടന്‍\\
ശൃംഗം പിടിച്ചു പതിപ്പിച്ചു ഭൂമിയില്‍\\
ഭംഗം വരുത്തിച്ചവിട്ടിപ്പറിച്ചുടന്‍\\
ഉത്തമാംഗത്തെച്ചുഴറ്റിയെറിഞ്ഞിതു\\
രക്തവും വീണു മതംഗാശ്രമസ്ഥലേ.\\
‘ആശ്രമദോഷം വരുത്തിയ ബാലിപോ-\\
ന്നൃശ്യമൂകാചലത്തിങ്കല്‍ വരുന്നാകില്‍\\
ബാലിയുടെ തല പൊട്ടിത്തെറിച്ചുടന്‍\\
കാലപുരി പൂക മദ്വാക്യഗൗരവാല്‍.’\\
എന്നു ശപിച്ചതു കേട്ടു കപീന്ദ്രനു-\\
മന്നു തുടങ്ങിയിവിടെ വരുവീല\\
‘ഞാനുമതുകൊണ്ടിവിടെ വസിക്കുന്നു\\
മാനസേ ഭീതികൂടാതെ നിരന്തരം.\\
ദുന്ദുഭിതന്റെ തലയിതു കാണ്‍കൊരു\\
മന്ദരംപോലെ കിടക്കുന്നതു ഭവാന്‍\\
ഇന്നിതെടുത്തെറിഞ്ഞീടുന്ന ശക്തനു\\
കൊന്നുകൂടും കപിവീരനെ നിര്‍ണയം.’\\
എന്നതു കേട്ടു ചിരിച്ചു രഘൂത്തമന്‍\\
തന്നുടെ തൃക്കാല്‍പ്പെരുവിരല്‍കൊണ്ടതു\\
തന്നെയെടുത്തു മേല്‍പോട്ടെറിഞ്ഞീടിനാന്‍\\
ചെന്നു വീണു ദശയോജനപര്യന്തം.\\
എന്നതു കണ്ടു തെളിഞ്ഞു സുഗ്രീവനും\\
തന്നുടെ മന്തികളും വിസ്മയപ്പെട്ടു.\\
നന്നുനന്നെന്നു പുകഴ്ന്നു പുകഴ്ന്നവര്‍\\
നന്നായ്ത്തൊഴുതു തൊഴുതു നിന്നീടിനാര്‍.\\
പിന്നെയുമര്‍ക്കാത്മജന്‍ പറഞ്ഞീടിനാന്‍:\\
‘മന്നവ! സപ്തസാലങ്ങളിവയല്ലോ.\\
ബാലിക്കു മല്‍പിടിച്ചീടുവാനായുള്ള\\
സാലങ്ങളേഴുമിവയെന്നറിഞ്ഞാലും\\
വൃത്രാരിപുത്രന്‍ പിടിച്ചിളക്കുന്നേരം\\
പത്രങ്ങളെല്ലാം കൊഴിഞ്ഞുപോമേഴിനും\\
വട്ടത്തില്‍ നില്ക്കുമിവറ്റെയൊരമ്പെയ്തു\\
പൊട്ടിക്കില്‍ ബാലിയെക്കൊലായ് വരും ദൃഢം.’\\
സൂര്യാത്മജോക്തികളീദൃഷം കേട്ടൊരു\\
സൂര്യാന്വയോത്ഭൂതനാകിയ രാമനും\\
ചാപം കുഴിയെക്കുലച്ചൊരു സായകം\\
ശോഭയോടേ തൊടുത്തെയ്തരുളീടിനാന്‍\\
സാലങ്ങളേഴും പിളര്‍ന്നു പുറപ്പെട്ടു\\
ശൈലവും ഭൂമിയും ഭേദിച്ച് ഉപിന്നെയും\\
ബാണം ജ്വലിച്ചു തിരിച്ചു വന്നാശുതന്‍-\\
തൂണീരമന്‍പോടു പുക്കോരനന്തരം\\
വിസ്മിതനായൊരു ഭാനുതനയനും\\
സസ്മിതം കൂപ്പിത്തൊഴുതു ചൊല്ലീടിനാന്‍:\\
‘സാക്ഷാല്‍ ജഗന്നാഥനാം പരമാത്മാവു\\
സാക്ഷിഭൂതന്‍ നിന്തിരുവടി നിര്‍ണയം.\\
പണ്ടു ഞാന്‍ ചെയ്തൊരു പുണ്യഫലോദയം\\
കൊണ്ടു കാണ്മാനുമെനിക്കു യോഗം വന്നു.\\
ജന്മമരണനിവൃത്തി വരുത്തുവാന്‍\\
നിര്‍മലന്മാര്‍ ഭജിക്കുന്നു ഭവല്‍പദം\\
മോക്ഷദനായ ഭവാനെ ലഭിക്കയാല്‍\\
മോക്ഷമൊഴിഞ്ഞപേക്ഷിക്കുന്നതില്ല ഞാന്‍\\
പുത്രദാരാര്‍ത്ഥരാജ്യാദിസമസ്തവും\\
വ്യര്‍ത്ഥമത്രേ തവ മായാവിരചിതം\\
ആകയാല്‍ മേ മഹാദേവ! ദേവേശ! മ-\\
റ്റാകാംക്ഷയില്ല ലോകേശ! പ്രസീദ മേ.\\
വ്യാപ്തമാനന്ദാനുഭൂതികരം പരം\\
പ്രാപ്തോഹമാഹന്ത ഭാഗ്യഫലോദയാല്‍\\
മണ്ണിനായൂഴികുഴിച്ചനേരം നിധി-\\
തന്നെ ലഭിച്ചതുപോലെ രഘുപതേ!\\
ധര്‍മദാനവ്രതതീര്‍ത്ഥതപഃക്രതു\\
കര്‍മപൂര്‍ത്തേഷ്ട്യാദികള്‍കൊണ്ടൊരുത്തനും\\
വന്നുകൂടാ ബഹു സംസാരനാശനം\\
നിര്‍ണയം ത്വല്‍പാദഭക്തികൊണ്ടെന്നിയേ.\\
ത്വല്‍പാദപത്മാവലോകനം കേവല-\\
മിപ്പോളകപ്പെട്ടതും ത്വല്‍കൃപാബലം.\\
യാതൊരുത്തന്നു ചിത്തം നിന്തിരുവടി-\\
പദാംബുജത്തിലിളകാതുറയ്ക്കുന്നു\\
കാല്‍ക്ഷണംപോലുമെന്നാകിലവന്‍ തനി-\\
ക്കൊക്കെ നീങ്ങീടുമജ്ഞാനമനര്‍ത്ഥദം.\\
ചിത്തം ഭവാങ്കലുറയ്ക്കായ്കിലുമതി-\\
ഭക്തിയോടേ രാമരാമേതി സാദരം\\
ചൊല്ലുന്നവന്നു ദുരിതങ്ങള്‍ വേരറ്റു\\
നല്ലനായേറ്റം വിശുദ്ധനാം നിര്‍ണയം.\\
മദ്യപനെങ്കിലും ബ്രഹ്മഘ്നനെങ്കിലും\\
സദ്യോ വിമുക്തനാം നാമജപത്തിനാല്‍\\
ശത്രുജയത്തിലും ദാരസുഖത്തിലും\\
ചിത്തേയൊരാഗ്രഹമില്ലെനിക്കേതുമേ.\\
ഭക്തിയൊഴിഞ്ഞു മറ്റൊന്നുമേ വേണ്ടീല\\
മുക്തിവരുവാന്‍ മുകുന്ദ! ദയാനിധേ!\\
ത്വല്‍പ്പാദഭക്തിമാര്‍ഗോപദേശംകൊണ്ടു\\
മല്‍പ്പാപമുല്‍പ്പാടായ ത്രിലോകീപതേ!\\
ശത്രുമദ്ധ്യസ്ഥാമിത്രാദിഭേദഭ്രമം\\
ചിത്തത്തില്‍ നഷ്ടമായ് വന്നിതു ഭൂപതേ!\\
ത്വല്‍പ്പാദപത്മാവലോകനംകൊണ്ടെനി-\\
ക്കുല്‍പ്പന്നമായിതു കേവലജ്ഞാനവും\\
പുത്രദാരാദിസംബന്ധമെല്ലാം തവ-\\
ശക്തിയാം മായാപ്രഭാവം ജഗല്‍പ്പതേ!\\
ത്വല്‍പ്പാദപങ്കജത്തിങ്കലുറയ്ക്കണ-\\
മെപ്പോഴുമുള്‍ക്കാമ്പെനിക്കു രമാപതേ!\\
ത്വന്നാമസങ്കീര്‍ത്തനപ്രിയയാകേണ-\\
മെന്നുടെജിഹ്വ സദാ നാണമെന്നിയേ\\
ത്വച്ചരണാംഭോരുഹങ്ങളിലെപ്പോഴു-\\
മര്‍ച്ചനം ചെയ്യായ്വരിക കരങ്ങളാല്‍\\
നിന്നുടെ ചാരുരൂപങ്ങള്‍ കാണായ്വരി-\\
കെന്നുടെ കണ്ണുകള്‍കൊണ്ടു നിരന്തരം\\
കര്‍ണങ്ങള്‍കൊണ്ടു കേള്‍ക്കായ്വരണം സദാ\\
നിന്നുടെ ചാരുചരിതം ധരാപതേ!\\
മച്ചരണദ്വയം സഞ്ചരിച്ചീടണ-\\
മച്യുതക്ഷേത്രങ്ങള്‍തോറും രഘുപതേ!\\
ത്വല്‍പാദപാംസുതീര്‍ഥങ്ങളേല്‍ക്കാകണ-\\
മെപ്പോഴുമംഗങ്ങള്‍കൊണ്ടു ജഗല്‍പതേ!\\
ഭക്ത്യാ നമസ്കരിക്കായ്വരേണം മുഹു-\\
രുത്തമാംഗംകൊണ്ടു നിത്യം ഭവല്‍പദം.’\\
ഇത്ഥം പുകഴ്ന്ന സുഗ്രീവനെ രാഘവന്‍\\
ചിത്തം കുളിര്‍ത്തുപിടിച്ചു പുല്കീടിനാന്‍\\
അംഗസംഗംകൊണ്ടു കല്മഷം വേരറ്റ\\
മംഗലാത്മാവായ സുഗ്രീവനെത്തദാ\\
മായയാ തത്ര മോഹിപ്പിച്ചിതന്നേരം\\
കാര്യസിദ്ധിക്കു കരുണാജലനിധി.
\end{verse}

%%05_baalisugreevayuddham

\section{ബാലിസുഗ്രീവയുദ്ധം}

\begin{verse}
സത്യസ്വരൂപന്‍ ചിരിച്ചരുളിച്ചെയ്തു:\\
‘സത്യമത്രേ നീ പറഞ്ഞതെടോ സഖേ!\\
ബാലിയെച്ചെന്നു വിളിക്ക യുദ്ധത്തിനു\\
കലം കളയരുതേതുമിനിയെടോ!\\
ബാലിയെ കൊന്നു രാജ്യാഭിഷേകം ചെയ്തു\\
പാലനം ചെയ്തുകൊള്‍വന്‍ നിന്നെ നിര്‍ണയം.’\\
അര്‍ക്കാത്മജനതു കേട്ടു നടന്നിതു\\
കിഷ്കിന്ധയാം പുരി നോക്കി നിരാകുലം\\
അര്‍ക്കകുലോത്ഭവന്മാരായ രാമനും\\
ലക്ഷ്മണവീരനും മന്ത്രികള്‍ നാല്‍വരും\\
മിത്രജന്‍ ചെന്നു കിഷ്കിന്ധാപുരദ്വാരി\\
യുദ്ധത്തിനായ് വിളിച്ചീടിനാന്‍ ബാലിയെ.\\
പൃത്ഥ്വീരഹവും മറഞ്ഞു നിന്നീടിനാര്‍\\
മിത്രഭാവേന രാമാദികളന്നേരം.\\
ക്രുദ്ധനാം ബാലിയലറി വന്നീടിനാന്‍\\
മിത്രതനയനും വക്ഷസി കുത്തിനാന്‍\\
വൃത്രാരിപുത്രനും മിത്രതനയനെ-\\
പ്പത്തുനൂറാശു വലിച്ചു കുത്തീടിനാന്‍.\\
ബദ്ധരോഷേണ പരസ്പരം തമ്മിലെ\\
യുദ്ധമതീവ ഭയങ്കരമായിതു\\
രക്തമണിഞ്ഞേകരൂപധരന്മാരായ്\\
ശക്തികലര്‍ന്നവരൊപ്പം പൊരുന്നേരം\\
മിത്രാത്മജനേതു വൃത്രാരിപുത്രനേ-\\
തിത്ഥം തിരിച്ചറിയാവല്ലൊരുത്തനും\\
മിത്രവിനാശനശങ്കയാ രാഘവ-\\
നസ്ത്രപ്രയോഗവും ചെയ്തീലതുനേരം\\
പൃത്രാരിപുത്രമുഷ്ടിപ്രയോഗം കൊണ്ടു\\
രക്തവും ഛര്‍ദിച്ചു ഭിതനായോടിനാന്‍\\
മിത്രതനയനും സത്വരമാര്‍ത്തനായ്;\\
വൃത്രാരിപുത്രനുമാലയം പുക്കിതു.\\
വിത്രസ്തനായ്വന്നു മിത്രതനയനും\\
പൃത്ഥ്വീരഹാന്തികേ നിന്നരുളീടിന\\
മിത്രാന്വയോത്ഭൂതനാകിയ രാമനോ-\\
ടെത്രയുമാര്‍ത്ത്യാ പരുഷങ്ങള്‍ ചൊല്ലിനാന്‍:\\
‘ശത്രുവിനെക്കൊണ്ടു കൊല്ലിക്കയോ തവ\\
ചിത്തത്തിലോര്‍ത്തതറിഞ്ഞീല ഞാനയ്യോ!\\
വധ്യനെന്നാകില്‍ വധിച്ചു കളഞ്ഞാലു-\\
മസ്ത്രേണ മാം നിന്തിരുവടി താന്‍ തന്നെ.\\
സത്യം പ്രമാണമെന്നോര്‍ത്തേ, നാതും പുന-\\
രെത്രയും പാരം, പിഴച്ചു ദയാനിധേ!\\
സത്യസന്ധന്‍ ഭവാനെന്നു ഞാനോര്‍ത്തതും\\
വ്യര്‍ത്ഥമത്രേ ശരണാഗതവത്സല!’\\
മിത്രാത്മജോക്തികളിത്തരമാകുലാല്‍\\
ശ്രുത്വാ രഘൂത്തമനുത്തരം ചൊല്ലിനാന്‍.\\
ബദ്ധാശ്രുനേത്രനായാലിംഗനം ചെയ്തു:\\
‘ചിത്തേ ഭയപ്പേടായ്കേതും മമ സഖേ!\\
അത്യന്തരോഷവേഗങ്ങള്‍ കലര്‍ന്നൊരു\\
യുദ്ധമദ്ധ്യേ ഭവാന്മാരെത്തിരിയാഞ്ഞു\\
മിത്രഘാതിത്വമാശങ്ക്യ ഞാനന്നേരം\\
മുക്തവാനായതില്ലസ്ത്രം ധരിക്ക നീ.\\
ചിത്തഭ്രമം വരായ്വാനൊരടയാളം\\
മിത്രാത്മജ! നിനക്കുണ്ടാക്കുവനിനി\\
ശത്രുവായുള്ളൊരു ബാലിയെസ്സത്വരം\\
യുദ്ധത്തിനായ് വിളിച്ചാലും മടിയാതെ\\
വൃത്രവിനാശനപുത്രനാമഗ്രജന്‍\\
മൃത്രുവശഗതനെന്നുറച്ചീടു നീ.\\
സത്യമിദമഹം രാമനെന്നാകിലോ\\
മിത്ഥ്യയായ് വന്നുകൂടാ രാമഭാഷിതം.’\\
ഇത്ഥം സമാശ്വാസ്യമിത്രാത്മജം രാമ-\\
ഭദ്രന്‍ സുമിത്രാത്മജനോടു ചൊല്ലിനാന്‍:\\
‘മിത്രാത്മജഗളേ പുഷ്പമാല്യത്തെ നീ\\
ബദ്ധ്വാ വിരവോടയയ്ക്ക യുദ്ധത്തിനായ്.’\\
ശത്രുഘ്നപൂര്‍വജന്‍ മാല്യവും ബന്ധിച്ചു\\
മിത്രാത്മജനെ മോദാലയച്ചീടിനാന്‍.
\end{verse}

%%06_baalivadham

\section{ബാലിവധം}

\begin{verse}
വൃത്രാരിപുത്രനെ യുദ്ധത്തിനായ്ക്കൊണ്ടു\\
മിത്രാത്മജന്‍ വിളിച്ചീടിനാന്‍ പിന്നെയും\\
ക്രുദ്ധനായ് നിന്നു കിഷ്കിന്ധാപുരദ്വാരി\\
കൃത്വാ മഹാസംഹനാദം രവിസുതന്‍\\
ബദ്ധരോഷം വിളിക്കുന്ന നാദം തദാ\\
ശ്രൂത്വാതിവിസ്മിതനായൊരു ബാലിയും\\
ബദ്ധ്വാ പരികരം യുദ്ധായ സത്വരം\\
ബദ്ധവൈരം പുറപ്പെട്ടൊരു നേരത്ത്\\
ഭര്‍ത്തുരഗ്രേ ചെന്നു ബദ്ധാശ്രുനേത്രയായ്\\
മദ്ധ്യേ തടുത്തു ചൊല്ലീടീനാള്‍ താരയും.\\
“ശങ്കാവിഹീനം പുറപ്പെട്ടതെന്തൊരു\\
ശങ്കയുണ്ടുള്ളിലെനിക്കതു കേള്‍ക്ക നീ.\\
വിഗ്രഹത്തിങ്കല്‍ പരാജിതനായ്പോയ\\
സുഗ്രീവനാശുവന്നീടുവാന്‍ കാരണം\\
എത്രംയും പാരം പരാക്രമമുള്ളൊരു\\
മിത്രമവനുണ്ടു പിന്‍തുണ നിര്‍ണയം.”\\
ബാലിയും താരയോടാശു ചൊല്ലീടിനാന്‍:\\
“ബാലേ! ബലാലൊരു ശങ്കയുണ്ടാകൊലാ.\\
കൈയയച്ചീടു നീ വൈകരുതേതുമേ\\
നീയൊരു കാര്യം ധരിക്കേണമോമലേ!\\
ബന്ധുവായാരുള്ളതോര്‍ക്ക സുഗ്രീവനു\\
ബന്ധമില്ലെന്നോടു വൈരത്തിനാര്‍ക്കുമേ.\\
ബന്ധുവായുണ്ടവനേകനെന്നാകിലോ\\
ഹന്തവ്യനെന്നാലവനുമറിക നീ.\\
ശത്രുവായുള്ളവന്‍ വന്നു ഗൃഹാന്തികേ\\
യുദ്ധത്തിനായ് വിളിക്കുന്നതും കേട്ടുടന്‍\\
ശൂരനായുള്ള പുരുഷനിരിക്കുമോ\\
ഭിരുവായുള്ളിലടച്ചതു ചൊല്ലു നീ.\\
വൈരിയെക്കൊന്നു വിരവില്‍ വരുവന്‍ ഞാന്‍\\
ധീരത കൈക്കൊണ്ടിരിക്ക നീ വല്ലഭേ!”\\
താരയും ചൊന്നാളതുകേട്ടവനോടു:\\
“വീരശിഖാമണേ! കേട്ടാലുമെങ്കില്‍ നീ.\\
കാനനത്തിങ്കല്‍ നായാട്ടിനു പോയിതു\\
താനേ മമ സുതനംഗദനന്നേരം\\
കേട്ടോരുദന്തമെന്നോടു ചൊന്നാനതു\\
കേട്ടിട്ടു ശേഷം യഥോചിതം പോക നീ.\\
ശ്രീമാന്‍ ദശരഥനാമയോദ്ധ്യാധിപന്‍\\
രാമനെന്നുണ്ടവന്‍ തന്നുടെ നന്ദനന്‍\\
ലക്ഷ്മണനാകുമനുജനോടും നിജ\\
ലക്ഷ്മീസമയായ സീതയോടുമവന്‍\\
വന്നിരുന്നീടിനാന്‍ ദണ്ഡകകാനനേ\\
വന്യാശനനായ് തപസ്സുചെതീടുവാന്‍\\
ദുഷ്ടനായുള്ളൊരു രാവണരാക്ഷസന്‍\\
കട്ടുകൊണ്ടാനവന്‍തന്നുടെ പത്നിയെ.\\
ലക്ഷ്മണനോതുമവളെയന്വേഷിച്ചു\\
തല്‍ക്ഷണമൃശ്യമൂകാചലേ വന്നിതു\\
മിത്രാത്മജനെയും തത്ര കണ്ടീടിനാന്‍\\
മിത്രമായ് വാഴ്കയെന്നന്യോന്യമൊന്നിച്ചു\\
സഖ്യവും ചെയ്തുകൊണ്ടാരഗ്നിസാക്ഷിയായ്.\\
ദുഃഖ ശാന്തിക്കങ്ങിരുവരുമായുടന്‍\\
‘വൃത്രാരിപുത്രനെക്കൊന്നു കിഷ്കിന്ധയില്‍\\
മിത്രാത്മജാ നിന്നെ വാഴിപ്പ’നെന്നൊരു\\
സത്യവും ചെയ്തുകൊടുത്തിതു രാഘവന്‍;\\
സത്വരമര്‍ക്കതനയനുമന്നേരം,\\
‘അന്വേഷണം ചെയ്തറിഞ്ഞു സീതാദേവി-\\
തന്നെയും കാട്ടിത്തരുവ’നെന്നും തമ്മില്‍\\
അന്യോന്യമേവം പ്രതിജ്ഞയും ചെയ്തിതു\\
വന്നതിപ്പോളതുകൊണ്ടുതന്നേയവന്‍\\
വൈരമെല്ലാ കളഞ്ഞാശു സുഗ്രീവനെ\\
സ്വൈരമായ് വാഴിച്ചുകൊള്‍കയിളമയായ്\\
യാഹി രാമം നീ ശരണമായ് വേഗേന\\
പാഹി മാമംഗദം രാജ്യം കുലഞ്ച തേ.’\\
ഇങ്ങനെ ചൊല്ലിക്കരഞ്ഞു കാലും പിടി-\\
ച്ചങ്ങനെ താര നമസ്കരിക്കും വിധൗ\\
വ്യാകുലഹീനം പുണര്‍ന്നു പുണര്‍ന്നനു-\\
രാഗവശേന പറഞ്ഞിതു ബാലിയും:\\
“സ്ര്തീസ്വഭാവംകൊണ്ടു പേടിയായ്കേതുമേ\\
നാസ്തി ഭയം മമ വല്ലഭേ! കേള്‍ക്ക നീ.\\
ശ്രീരാമലക്ഷ്മണന്മാര്‍ വന്നതെങ്കിലോ\\
ചേരുമെന്നോടുമവരെന്നു നിര്‍ണയം.\\
രാമനെ സ്നേഹമെന്നോളമില്ലാര്‍ക്കുമേ\\
രാമനാകുന്നതു സാക്ഷാല്‍ മഹാവിഷ്ണു\\
നാരായണന്‍താനവതരിച്ചു ഭൂമി-\\
ഭാരഹരണാര്‍ഥമെന്നു കേള്‍പ്പുണ്ടു ഞാന്‍\\
പക്ഷഭേദം ഭഗവാനില്ല നിര്‍ണയം\\
നിര്‍ഗുണനേകനാത്മാരാമനീശ്വരന്‍\\
തച്ചരണാംബുജേ വീണു നമസ്കരി-\\
ച്ചിച്ഛയാ ഞാന്‍ കൂട്ടിക്കൊണ്ടിങ്ങു പോരുവന്‍.\\
മല്‍ഗൃഹത്തിങ്കലുപകാരവുമേറും\\
സുഗ്രീവനേക്കാളുമെന്നെക്കൊണ്ടോര്‍ക്ക നീ.\\
തന്നെബ്ഭജിക്കുന്നവനെ ബ്ഭജിച്ചീടു-\\
മന്യഭാവം പരമാത്മാവിനില്ലല്ലോ.\\
ഭക്തിഗമ്യന്‍ പരമേശ്വരന്‍ വല്ലഭേ!\\
ഭക്തിയോപാര്‍ക്കിലെന്നോളമില്ലാര്‍ക്കുമേ\\
ദുഃഖവു നീക്കി വസിക്ക നീ വേശ്മനി\\
പുഷ്കരലോചനേ! പൂര്‍ണഗുണാംബുധേ!\\
ഇത്ഥമാശ്വാസ്യ വൃത്രാരാതിപുത്രനും\\
ക്രുദ്ധനായ് സത്വരം ബദ്ധ്വാ പരികരം\\
നിര്‍ഗമിച്ചീടിനാന്‍ യുദ്ധായ സത്വരം\\
നിഗ്രഹിച്ചീടുവാന്‍ സുഗ്രീവനെ ക്രുധാ\\
താരയുമശ്രുകണങ്ങളും വാര്‍ത്തുവാര്‍-\\
ത്താരൂഢതാപമകത്തുപുക്കീടിനാള്‍.\\
പല്ലും കടിച്ചലറിക്കൊണ്ടു ബാലിയും\\
നില്ലു നില്ലെന്നണഞ്ഞോരു നേരം തദാ\\
മുഷ്ടികള്‍കൊണ്ടു താഡിച്ചിതു ബാലിയെ\\
രുഷ്ടനാം ബാലി സുഗ്രീവനെയും തഥാ.\\
മുഷ്ടിചുരുട്ടി പ്രഹരിച്ചിരിക്കവേ\\
കെട്ടിയും കാല്‍ കൈ പരസ്പരം താഡനം\\
തട്ടിയും മുള്ട്ടുകൊണ്ടും തല തങ്ങളില്‍\\
കൊട്ടിയുമേറ്റം പിടിച്ചും കടിച്ചുമ-\\
ങ്ങൂറ്റത്തില്‍, വീണും പിരണ്ടുമുരുണ്ടുമുള്‍-\\
ച്ചീറ്റം കലര്‍ന്നു നഖംകൊണ്ടു മാന്തിയും\\
ചാടിപ്പതിക്കയും കൂടെക്കുതിക്കയും\\
മാടിത്തടുക്കയും കൂടെക്കൊട്ളുക്കയും\\
ഓടീക്കഴിക്കയും വാടി വിയര്‍ക്കയും\\
മാടി വിളിക്കയും കോപിച്ചടുക്കയും\\
ഊടെ വിയര്‍ക്കയും നാഡികള്‍ ചീര്‍ക്കയും\\
മുഷ്ടിയുദ്ധപ്രയോഗം കണ്ടു നില്പവര്‍\\
ദൃഷ്ടി കുളുര്‍ക്കയും വാഴ്ത്തി സ്തുതിക്കയും\\
കാലനും കാലകാലന്‍ താനുമുള്ളപോര്‍\\
ബാലിസുഗ്രീവയുദ്ധത്തിനൊവ്വാ ദൃഢാ.\\
രണ്ടു സമുദ്രങ്ങള്‍ തമ്മില്‍ പൊരുംപോലെ\\
രണ്ടു ശൈലങ്ങള്‍ തമ്മില്‍ പൊരുംപോലെയും\\
കണ്ടവരാര്‍ത്തുകൊണ്ടാടിപ്പുകഴ്ത്തിയും\\
കണ്ടീലവാടമൊരുത്തനു മേതുമേ.\\
അച്ഛന്‍ കൊണ്ടുത്തോരു മാല ബാലിക്കുമു-\\
ണ്ടച്യുതന്‍ നല്‍കിയ മാല സുഗ്രീവനും\\
ഭേദമില്ലൊന്നുകൊണ്ടും തമ്മിലെങ്കിലും\\
ഭേദിച്ചിതര്‍ക്കതനയനു വിഗ്രയം.\\
സാദവുമേറ്റം കലര്‍ന്നു സുഗ്രീവനും\\
ഖേദമോടേ രഘുനാഥനെ നോക്കിയും\\
അഗ്രജമുഷ്ടിപ്രഹരങ്ങളേല്ക്കയാല്‍\\
സുഗ്രീവനേറ്റം തളര്‍ച്ചയുണ്ടെന്നതു\\
കണ്ടു കാരുണ്യം കലര്‍ന്നു വേഗേന വൈ-\\
കുണ്ഠന്‍ ദശരഥനന്ദനന്‍ ബാലിതന്‍\\
വക്ഷഃപ്രദേശത്തെ ലക്ഷ്യമാക്കിക്കൊണ്ടു\\
വൃക്ഷശണ്ഡം മറഞ്ഞാശു മഹേന്ദ്രമാ-\\
മസ്ത്രം തൊടുത്തു വലിച്ചു നിറച്ചുടന്‍\\
വിദ്രുതമാമ്മാറയച്ചരുളീടിനാന്‍,\\
ചെന്നതു ബാലിതന്‍ മാറില്‍ തറച്ചള-\\
മൊന്നങ്ങലറി വീണീടിനാന്‍ ബാലിയും\\
ഭൂമിയുമൊന്നു വിറച്ചിതന്നേരത്തു\\
രാമനെക്കൂപ്പി സ്തുതിച്ചു മരുല്‍സുതന്‍.\\
മോഹം കലര്‍ന്നു മുഹൂര്‍ത്തമാത്രം പിന്നെ\\
മോഹവും തീര്‍ന്നു നോക്കീടിനാന്‍ ബാലിയും\\
കാണായിതഗ്രേ രഘൂത്തമനെത്തദാ\\
ബാണവും ദക്ഷിണഹസ്തേ ധരിച്ചന്യ-\\
പാണിയില്‍ ചാപവും ചീരവസനവും\\
തൂണീരവും മൃദുസ്മേരവദനവും\\
ചാരുജടാമകുടം പൂണ്ടിടം പെട്ട\\
മാറിടത്തിങ്കല്‍ വനമാലയും പൂണ്ടു\\
ചാര്‍വായതങ്ങളായുള്ള ഭുജങ്ങളും\\
ദുര്‍വാദളച്ഛവിപൂണ്ട ശരീരവും\\
പക്ഷഭാഗേ പരിസേവിതന്മാരായ\\
ലക്ഷ്മണസുഗ്രീവന്മാരെയുമഞ്ജസാ\\
കണ്ടു ഗര്‍ഹിച്ചു പറഞ്ഞിതു ബാലിയു-\\
മുണ്ടായ കോപഖേദാകുലചേതസാ:\\
‘എന്തു ഞാനൊന്നു നിന്നോടു പിഴച്ചതു-\\
മന്തിനെന്നെക്കൊലചെയ്തു വെറുതെ നീ?\\
വ്യാജേന ചോരധര്‍മത്തെയും കൈക്കൊണ്ടു\\
രാജധര്‍മത്തെ വെടിഞ്ഞതെന്തിങ്ങനെ?\\
എന്തൊരു കീര്‍ത്തി ലഭിച്ചതിതുകൊണ്ടു\\
ചിന്തിക്ക രാജകുലോത്ഭവനല്ലോ നീ.\\
വീരധര്‍മം നിരൂപിച്ചു കീര്‍ത്തിക്കെങ്കില്‍\\
നേരേ പൊരുതു ജയിക്കേണമേവനും\\
എന്തോന്നു സുഗ്രീവനാല്‍ കൃതമായതു-\\
മെന്തു മറ്റെന്നാല്‍ കൃതമല്ലയാഞ്ഞതും\\
രക്ഷോവരന്‍ തവ പത്നിയെക്കട്ടതി\\
നര്‍ക്കാത്മജനെശ്ശരണമായ് പ്രാപിച്ചു.\\
നിഗ്രഹിച്ചു ഭവാനെന്നെയെന്നാകിലോ\\
വിക്രമം മാമകംകേട്ടറിയുന്നീലേ?\\
ആരറിയാത്തതു മൂന്നുലോകത്തിലും\\
വീരനാമെന്നുടെ ബാഹുപരാക്രമം?\\
ലങ്കാപുരത്തെ ത്രികൂടാചലത്തൊടും\\
ശങ്കാവിഹീനം ദശാസ്യനോടും കൂടെ\\
ബന്ധിച്ചു ഞാനരനാഴികകൊണ്ടു നി-\\
ന്നന്തികേ വെച്ചു തൊഴുതേനു മാദരാല്‍.\\
ധര്‍മിഷ്ഠനെന്നു ഭവാനെ ലോകത്തിങ്കല്‍\\
നിര്‍മലന്മാര്‍ പറയുന്നു രഘുപതേ!\\
ധര്‍മമെന്തോന്നു ലഭിച്ചതിതുകൊണ്ടു\\
നിര്‍മൂലമിങ്ങനെ കാട്ടാളനെപ്പോലെ\\
വാനരത്തെ ച്ചതിച്ചെയ്തുകൊന്നിട്ടൊരു\\
മാനമുണ്ടായതെന്തെന്നു പറക നീ.\\
വാനരമാംസമഭക്ഷ്യമത്രേ ബത\\
മാനസേ തോന്നിയതെന്തിതു ഭൂപതേ!’\\
ഇത്ഥം ബഹുഭാഷണം ചെയ്ത ബാലിയോ-\\
ടുത്തരമായരുള്‍ചെയ്തു രഘൂത്തമന്‍:\\
‘ധര്‍മത്തെ രക്ഷിപ്പതിന്നായുധവുമായ്\\
നിര്‍മത്സരം നടക്കുന്നിതു നീളെ ഞാന്‍\\
പാപിയായോരധര്‍മിഷ്ഠനാം നിന്നുടെ\\
പാപം കളഞ്ഞു ധര്‍മത്തെ നടത്തുവാന്‍\\
നിന്നെ വധിച്ചിതു ഞാന്‍, മോഹബദ്ധനായ്\\
നിന്നെ നീയേതു മറിയാഞ്ഞതുമെടോ!\\
പുത്രീ ഭഗിനീ സഹോദരഭാര്യയും\\
പുത്രകളത്രവും മാതാവുമേതുമേ\\
ഭേദമില്ലെന്നല്ലോ വേദവാക്യമതു\\
ചേതസി മോഹാല്‍ പരിഗ്രഹിക്കുന്നവന്‍\\
പാപികളില്‍വച്ചുമേറ്റം മഹാപാപി\\
താപമവര്‍ക്കതിനാലെ വരുമല്ലോ.\\
മര്യാദ നീക്കി നടക്കുന്നവര്‍കളെ\\
ശൗര്യമേറും നൃപന്മാര്‍ നിഗ്രഹിച്ചഥ\\
ധര്‍മസ്ഥിതി വരുത്തും ധരണീതലേ\\
നിര്‍മലാത്മ നീ നിരൂപിക്ക മാനസേ.\\
ലോകവിശുദ്ധി വരുത്തുവാനായ്ക്കൊണ്ടു\\
ലോകപാലന്മാര്‍ നടക്കുമെല്ലാടവും.\\
ഏറെപ്പറഞ്ഞു പോകായ്കവരോടതും\\
പാപത്തിനായ് വരും പാപികള്‍ക്കേറ്റവും.’\\
ഇത്ഥമരുള്‍ചെയ്തതൊക്കവേ കേട്ടാശു\\
ചിത്തവിശുദ്ധി ഭവിച്ചു കപീന്ദ്രനും\\
രാമനെ നാരായണനെന്നറിഞ്ഞുടന്‍\\
താമസഭാവമകന്നു സസംഭ്രമം\\
ഭക്ത്യാ നമസ്കൃത്യ വന്ദിച്ചു ചൊല്ലിനാ-\\
നിത്ഥം, ’മമാപരാധം ക്ഷമിക്കേണമേ.\\
ശ്രീരാമ! രാമ! മഹാഭാഗ! രാഘവ!\\
നാരായണന്‍ നിന്തിരുവടി നിര്‍ണയം.\\
ഞാനറിയാതെ പറഞ്ഞതെല്ലാം തവ\\
മാനസേ കാരുണ്യമോടും ക്ഷമിക്കണം.\\
നിന്തിരുമേനിയും കണ്ടു കണ്ടാശു നി-\\
ന്നന്തികേ താവകമായ ശരമേറ്റു\\
ദേഹമുപേക്ഷിപ്പതിന്നു യോഗം വന്ന-\\
താഹന്ത! ഭാഗ്യമെന്തോന്നു ചൊല്ലാവതും!\\
സാക്ഷാല്‍ മഹായോഗിനാമപി ദുര്‍ലഭം\\
മോക്ഷപ്രദം തവ ദര്‍ശനം ശ്രീപതേ!\\
നിന്തിരുനാമം, മരിപ്പാന്‍ തുടങ്ങുമ്പോള്‍\\
സന്താപമുള്‍ക്കൊണ്ടു ചൊല്ലും പുരുഷനു\\
മോക്ഷം ലഭിക്കുന്നതാകയാലിന്നുമേ\\
സാക്ഷാല്‍ പുരഃസ്ഥിതനായ ഭഗവാനെ-\\
ക്കണ്ടു കണ്ടന്‍പോടു നിന്നുടെ സായകം\\
കൊണ്ടു മരിപാനവകാസമിക്കാല-\\
മുണ്ടായതെന്നുടെ ഭാഗ്യാതിരേകമി-\\
തുണ്ടോ പലര്‍ക്കും ലഭിക്കുന്നതീശ്വരാ?\\
നാരായണന്‍ നിന്തിരുവടി, ജാനകി\\
താരില്‍മാതാവായ ലക്ഷ്മീഭഗവതി\\
പംക്തികണ്ഠന്‍തന്നെ നിഗ്രഹിപാനാശു\\
പംക്തിരഥാത്മജനായ് ജനിച്ചു ഭവാന്‍\\
പത്മജന്‍ മുന്നമര്‍ഥിക്കയാലെന്നതും\\
പത്മവിലോചന! ഞാനറിഞ്ഞീടിനേന്‍.\\
നിന്നുടെ ലോകം ഗമിപ്പാന്‍ തുടങ്ങീടു-\\
മെന്നെയനുഗ്രഹിക്കേണം ഭഗവാനേ!\\
എന്നോടുതുല്യബലനാകുമംഗദന്‍-\\
തന്നില്‍ തിരുവുള്ളമുണ്ടായിരിക്കണം.\\
അര്‍ക്കതനയനുമാംഗദബാലനു-\\
മൊക്കുമെനിക്കെന്നു കൈക്കൊള്‍ക വേണമേ!\\
അമ്പും പറിച്ച് തൃക്കൈകൊണ്ടടിയനെ-\\
യന്‍പോടു മെല്ലെത്തലോടുകയും വേണം.’\\
എന്നതു കേട്ടു രഘൂത്തമന്‍ ബാണവും\\
ചെന്നു പറിച്ചു തലോടിനാന്‍ മെല്ലവേ.\\
മാനവവീരന്‍ മുഖാംബുജവും പാര്‍ത്തു\\
വാനരദേഹമുപേക്ഷിച്ചു ബാലിയും.\\
യോഗീന്ദ്രവൃന്ദദുരാപമായുള്ളൊരു\\
ലോകം ഭഗവല്‍പദം ഗമിച്ചീടിനാന്‍.\\
രാമനായോരു പരമാത്മനാ ബാലി\\
രാമപാദം പ്രവേശിച്ചോരനന്തരം\\
മാര്‍ക്കടൗഘം ഭയത്തോടോടി വേഗേന\\
പുക്കിതു കിഷ്കിന്ധയായ പുരാജിരേ.\\
ചൊല്ലിനാര്‍ താരയോടാശു കപികളും,\\
‘സ്വര്‍ല്ലോകവാസിയായ് വന്നു കപീശ്വരന്‍\\
ശ്രീരാമസായകമേറ്റു രണാജിരേ.\\
താരേ! കുമാരനെ വാഴിക്ക വൈകാതെ.\\
ഗോപുരവാതില്‍ നാലും ദൃഢം ബന്ധിച്ചു\\
ഗോപിച്ചുകൊള്‍ക കിഷ്കിന്ധാ മഹാപുരം.\\
മന്ത്രികളോടു നിയോഗിക്ക നീ പരി-\\
പന്ഥികളുള്ളില്‍ കടക്കാതിരിക്കണം.’\\
ബാലി മരിച്ചതു കേട്ടോരു താരയു-\\
മോലോല വീഴുന്ന കണ്ണുനീരും വാര്‍ത്തു\\
ദുഃഖേന വക്ഷസി താഡിച്ചു താഡിച്ചു\\
ഗദ്ഗദവാചാ പറഞ്ഞു പലതരം:\\
എന്തിനെനിക്കിനിപ്പുത്രനും രാജ്യവു-\\
മെന്തിനു ഭൂതലവാസവും മേ വൃഥാ\\
ഭര്‍ത്താവുതന്നോടുകൂടെ മടിയാതെ\\
മൃത്യുലോകം പ്രവേശിക്കുന്നതുണ്ടു ഞാന്‍.’\\
ഇത്ഥം കരഞ്ഞു കരഞ്ഞവള്‍ ചെന്നു തന്‍\\
രക്തപാംസുക്കളണിഞ്ഞു കിടക്കുന്ന\\
ഭര്‍ത്തൃകളേബരം കണ്ടു മോഹം പൂണ്ടു\\
പുത്രനോടുംകൂടെയേറ്റം വിവശയായ്\\
വീണിതു ചെന്നു പാദാന്തികേ താരയും\\
കേണു തുടങ്ങിനാള്‍ പിന്നെപ്പലതരം:\\
‘ബാണമെയ്തെന്നെയും കൊന്നീടു നീ മമ\\
പ്രാണനാഥന്നു പൊറാ പിരിഞ്ഞാലെടോ!\\
എന്നെപ്പതിയോടു കൂടെയയക്കിലോ\\
കന്യകാദാനഫലം നിനക്കുംവരും.\\
ആര്യനാം നിന്നാലനുഭൂതമല്ലയോ\\
ഭാര്യാവിയോഗജദുഃഖം രഘുപതേ!\\
വ്യഗ്രവും തീര്‍ത്തു രുമയുമായ് വാഴ്ക നീ\\
സുഗ്രീവ! രാജ്യഭോഗങ്ങളോടും ചിരം.’\\
ഇത്ഥം പറഞ്ഞു കരയുന്ന താരയോ-\\
ടുത്തരമായരുള്‍ചെയ്തു രഘുവരന്‍\\
തത്ത്വജ്ഞാനോപദേശേന കാരുണ്യേന\\
ഭര്‍ത്തൃവിയോഗദുഃഖം കളഞ്ഞീടുവാന്‍.
\end{verse}

%%07_thaaropadesham

\section{താരോപദേശം}

\begin{verse}
എന്തിനു ശോകം വൃഥാ തവ? കേള്‍ക്ക നീ\\
ബന്ധമില്ലേതുമിതിന്നു മനോഹരേ!\\
നിന്നുടെ ഭര്‍ത്താവ് ദേഹമോ ജീവനോ?\\
ധന്യേ! പരമാര്‍ത്ഥമെന്നോടു ചൊല്ലു നീ\\
പഞ്ചഭൂതാത്മകം ദേഹമേറ്റം ജഡം\\
സഞ്ചിതം ത്വങ്മാംസരക്താസ്ഥികൊണ്ടെടോ!\\
നിശ്ചേഷ്ട കാഷ്ഠതുല്യം ദേഹമോര്‍ക്ക നീ\\
നിശ്ചയമാത്മാവു ജീവന്‍ നിരാമയന്‍.\\
ഇല്ല ജനനം മരണവുമില്ല കേ-\\
ളല്ലലുണ്ടാകായ്കതു നിനച്ചേതുമേ.\\
നില്ക്കയുമില്ല നടക്കയുമില്ല കേള്‍\\
ദുഃഖവിഷയവുമല്ലതു കേവലം.\\
സ്ത്രീപുരുഷക്ലീബഭേദങ്ങളുമില്ല\\
താപശീതാദിയുമില്ലെന്നറിക നീ.\\
സര്‍വഗന്‍ ജീവനേകന്‍ പരനദ്വയ-\\
നവ്യയനാകാശതുല്യനലേപകന്‍\\
ശുദ്ധമായ് നിത്യമായ് ജ്ഞാനാത്മകമായ\\
തത്ത്വമോര്‍ത്തെന്തു ദുഃഖത്തിനു കാരണം?’\\
രാമവാക്യാമൃതം കേട്ടോരു താരയും\\
രാമനോടാശു ചോദിച്ചിതു പിന്നെയും:\\
‘നിശ്ചേഷ്ട കാഷ്ഠതുല്യം ദേഹമായതും\\
സച്ചിദാത്മാ നിത്യമായതു ജീവനും\\
ദുഃഖസുഖാദി സംബന്ധമാര്‍ക്കെന്നുള്ള-\\
തൊക്കെയരുള്‍ചെയ്കവേണം ദയാനിധേ!\\
എന്നതു കേട്ടരുള്‍ചെയ്തു രഘുവരന്‍\\
‘ധന്യേ! രഹസ്യമായുള്ളതു കേള്‍ക്ക നീ.\\
യാതൊരളവു ദേഹേന്ദ്രിയാഹങ്കാര-\\
ഭേദഭാവേന സംബന്ധമുണ്ടായ്വരും\\
അത്ര നാളേക്കുമാത്മാവിനു സംസാര-\\
മെത്തുമവിവേകകാരണാല്‍ നിര്‍ണയം.\\
ഓര്‍ക്കില്‍ മിഥ്യാഭൂതമായ സംസാരവും\\
പാര്‍ക്ക താനേ വിനിവര്‍ത്തിക്കയില്ലെടോ.\\
നാനാവിഷയങ്ങളെ ധ്യായമാനനാം\\
മാനവനെങ്ങനെയെന്നതും കേള്‍ക്ക നീ\\
മിഥ്യാഗമം നിജസ്വപ്നേ യഥാ തഥാ\\
സത്യമായുള്ളതു കേട്ടാലുമെങ്കിലോ.\\
നൂനമനാദ്യവിദ്യാബന്ധഹേതുനാ\\
താനാമഹങ്കൃതിക്കാശു തല്‍ക്കാര്യമായ്\\
സംസാരമുണ്ടാമപാര്‍ഥകമായതും\\
സംസാരമോ രാഗരോഷാദി സങ്കുലം\\
മാനസം സംസാരകാരണമായതും\\
മാനസത്തിന്നു ബന്ധം ഭവിക്കുന്നതും\\
ആത്മമനസ്സമാനത്വം ഭവിക്കയാ-\\
ലാത്മനസ്തല്‍കൃതബന്ധം ഭവിക്കുന്നു.\\
രക്താദിസാന്നിദ്ധ്യമുണ്ടാക കാരണം\\
ശുദ്ധസ്ഫടികവും തദ്വര്‍ണമായ്വരും.\\
വസ്തുതയാ പാര്‍ക്കിലില്ല തദ്രഞ്ജനാ\\
ചിത്തേ നിരൂപിച്ചു കാണ്‍ക നീ സൂക്ഷ്മമായ്\\
ബുദ്ധീന്ദ്രിയാദി സാമീപ്യമുണ്ടാകയാ-\\
ലെത്തുമാത്മാവിനു സംസാരവും ബലാല്‍.\\
ആത്മസ്വലിംഗമായോരു മനസ്സിനെ\\
താത്പര്യമോടു പരിഗ്രഹിച്ചിട്ടല്ലോ\\
തത്സ്വഭാവങ്ങളായുള്ള കാമങ്ങളെ\\
സത്വാദികളാം ഗുണങ്ങളാല്‍ ബദ്ധനായ്\\
സേവിക്കയാലവശത്വം കലര്‍ന്നതു\\
ഭാവിക്കകൊണ്ടു സംസാരേ വലയുന്നു.\\
ആദൗ മനോഗുണാന്‍ സൃഷ്ട്വാ തതസ്തദാ\\
വേദം വിധിക്കും ബഹുവിധകര്‍മങ്ങള്‍\\
ശുക്ലരക്താസിതഭേദഗതികളായ്\\
മിക്കതും തത്സമാനപ്രഭാവങ്ങളായ്\\
ഇങ്ങനെ കര്‍മവശേന ജീവന്‍ ബലാ-\\
ലെങ്ങുമാഭൂതപ്ലവം ഭ്രമിച്ചീടുന്നു.\\
പിന്നെസ്സമസ്തസംഹാരകാലേ ജീവ-\\
നന്നുമനാദ്യവിദ്യാവശം പ്രാപിച്ചു\\
തിഷ്ഠത്യഭിനിവേശത്താല്‍ പുനരഥ\\
സൃഷ്ടികാലേ പൂര്‍വവാസനയാ സമം\\
ജായതേ ഭൂയോഘടീയന്ത്രവല്‍ സദാ\\
മായാബലത്താലതാര്‍ക്കൊഴിക്കാമെടോ?\\
യാതൊരിക്കല്‍ നിജപുണ്യവിശേഷേണ\\
ചേതസി സത്സംഗതി ലഭിച്ചീടുന്നു,\\
മത്ഭക്തനായ ശാന്താത്മാവിനു പുന-\\
രപ്പോളവന്മതി മദ്വിഷയാ ദൃഢം\\
ശ്രദ്ധയുമുണ്ടാം കഥാശ്രവണേ മമ\\
ശുദ്ധസ്വരൂപവിജ്ഞാനവും ജായതേ\\
സല്‍ഗുരുനാഥപ്രസാദേന മാനസേ\\
മുഖ്യവാക്യാര്‍ഥവിജ്ഞാനമുണ്ടായ്വരും.\\
ദേഹേന്ദ്രിയമനഃപ്രാണാദികളില്‍നി-\\
ന്നാഹന്ത! വേറൊന്നു നൂനമാത്മാവിതു.\\
സത്യമാനന്ദമേകം പരമദ്വയം\\
നിത്യം നിരുപമം നിഷ്കളം നിര്‍ഗുണം.\\
ഇത്ഥമറിയുമ്പോള്‍ മുക്തനാമപ്പൊഴേ\\
സത്യം മയോദിതം സത്യം മയോദിതം\\
യാതൊരുത്തന്‍ വിചാരിക്കുന്നതിങ്ങനെ\\
ചേതസി സംസാരദുഃഖമവനില്ല.\\
നീയും മഹാപ്രോക്തമോര്‍ത്തു വിശുദ്ധയായ്\\
മായാവിമോഹം കളക മനോഹരേ!\\
കര്‍മബന്ധത്തിങ്കല്‍ നിന്നുടന്‍ വേര്‍പെട്ടു\\
നിര്‍മലബ്രഹ്മണിതന്നെ ലയിക്ക നീ\\
ചിത്തേ നിനക്കു കഴിഞ്ഞ ജന്മത്തിങ്ക-\\
ലെത്രയും ഭക്തിയുണ്ടെങ്കലതുകൊണ്ടു\\
രൂപവുമേവം നിനക്കു കാട്ടിത്തന്നു\\
താപമിനിക്കളഞ്ഞാലുമശേഷം നീ\\
മദ്രൂപമീദൃശം ധ്യാനിച്ചുകൊള്‍കയും\\
മദ്വചനത്തെ വിചാരിച്ചു കൊള്‍കയും\\
ചെയ്താല്‍ നിനക്കു മോക്ഷംവരും നിര്‍ണയം\\
കൈതവമല്ല പറഞ്ഞതു കേവലം.’\\
ശ്രീരാമവാക്യമാനന്ദേന കേട്ടൊരു\\
താരയും വിസ്മയംപൂണ്ടു വനങ്ങിനാള്‍.\\
മോഹമകന്നു തെളിഞ്ഞിതു ചിത്തവും\\
ദേഹാഭിമാനജദുഃഖവും പോക്കിനാള്‍\\
ആത്മാനുഭൂതികൊണ്ടാശു സന്തുഷ്ടയാ-\\
യാത്മബോധേന ജീവന്മുക്തയായിനാള്‍.\\
മോക്ഷപ്രദനായ രാഘവന്‍തന്നോടു\\
കാല്‍ക്ഷണം സംഗമമാത്രേണ താരയും\\
ഭക്തിമുഴുത്തിട്ടനാദിബന്ധം തീര്‍ന്നു\\
മുക്തയായാളൊരു നാരിയെന്നാകിലും.\\
വ്യഗ്രമെല്ലാമകലെപ്പോയ് തെളിഞ്ഞിതു\\
സുഗ്രീവനുമിവ കേട്ടോരനന്തരം\\
അജ്ഞാനമെല്ലാമകന്നു സൗഖ്യം പൂണ്ടു\\
വിജ്ഞാനമോടതിസ്വസ്ഥനായാന്‍ തുലോം.
\end{verse}

%%08_sugreevaraajyaabhishekam

\section{സുഗ്രീവരാജ്യാഭിഷേകം}

\begin{verse}
സുഗ്രീവനോടരുള്‍ചെയ്താനനന്തര-\\
‘മഗ്രജപുത്രനാമംഗദന്‍തന്നെയും\\
മുന്നിട്ടു സംസ്കാരമാദികര്‍മങ്ങളെ-\\
പ്പുണ്യാഹപര്യന്തമാഹന്ത! ചെയ്ക നീ.’\\
രാമാജ്ഞയാ തെളിഞ്ഞാശു സുഗ്രീവനു-\\
മാമോദപൂവമൊരുക്കിതുടങ്ങിനാന്‍\\
സൗമ്യയായുള്ളോരു താരയും പുത്രനും\\
ബ്രാഹ്മണരുമമാത്യപ്രധാനന്മാരും\\
പൗരജനങ്ങളുമായ് നൃപേന്ദ്രോചിതം\\
ഭേരീമൃദംഗാദി വാദ്യഘോഷത്തൊടും\\
ശാസ്ത്രോക്തമാര്‍ഗേണ കര്‍മം കഴിച്ചഥ\\
സ്നാനാത്വാ ജഗാമ രഘൂത്തമസന്നിധൗ.\\
മന്തികളോടും പ്രണമ്യ പാദാംബുജ-\\
മന്തര്‍മുദാ പറഞ്ഞാന്‍ കപിപുംഗവന്‍:\\
‘രാജ്യത്തെ രക്ഷിച്ചുകൊള്‍കവേണമിനി\\
പൂജ്യനാകും നിന്തിരുവടി സാദരം.\\
ദാസനായുള്ളോരടിയനിനിത്തവ\\
ശാസനയും പരിപാലിച്ചു സന്തതം\\
ദേവദേവേശ! തേ പാദപത്മദ്വയം\\
സേവിച്ചുകൊള്ളൂവന്‍ ലക്ഷ്മണനെപ്പോലെ.’\\
സുഗ്രീവവാക്കുകളിത്തരം കേട്ടുട-\\
നഗ്രേ ചിരിച്ചരുള്‍ചെയ്തു രഘൂത്തമന്‍:\\
‘നീ തന്നെ ഞാനതിനില്ലൊരു സംശയം\\
പ്രീതനായ്പോയാലുമാശു മമാജ്ഞയാ.\\
രാജ്യാധിപത്യം നിനക്കുതന്നേനിനി\\
പൂജ്യനായ് ചെന്നഭിഷേകം കഴിക്ക നീ.\\
“നൂനമൊരു നഗരം പൂകയുമില്ല\\
ഞാനോ പനിന്നാലു സംവത്സരത്തോളം.\\
സൗമിത്രിചെയ്യുമഭിഷേകമാദരാല്‍\\
സാമര്‍ഥ്യമുള്ള കുമാരനെപ്പിന്നെ നീ\\
യൗവരാജ്യാര്‍ഥമഭിഷേചയ പ്രഭോ!\\
സര്‍വമധീനം നിനക്കു രാജ്യം സഖേ!\\
ബാലിയെപ്പോലെ പരിപാലനം ചെയ്തു\\
ബാലനെയും പരിപാലിച്ചുകൊള്‍ക നീ.\\
അദ്രിശിഖരേ വസിക്കുന്നതുണ്ടുഞാ-\\
നദ്യപ്രഭൃതി ചാതുര്‍മാസ്യമാകുലാല്‍.\\
പിന്നെ വരിഷം കഴിഞ്ഞാലനന്തര-\\
മന്വേഷണാര്‍ഥം പ്രയത്നങ്ങള്‍ ചെയ്ക നീ.\\
തന്വംഗിതാനിരിപ്പേടമറിഞ്ഞു വ-\\
ന്നെന്നോടു ചൊല്കയും വേണം മമ സഖേ!\\
അത്രനാളും പുരത്തിങ്കല്‍ വസിക്ക നീ\\
നിത്യസുഖത്തൊടും ദാരാത്മജൈസ്സമം.”\\
രാഘവന്‍ തന്നോടനുജ്ഞയും കൈക്കൊണ്ടു\\
വേഗേന സൗമിത്രിയോടു സുഗ്രീവനും\\
ചെന്നു പുരിപ്പുക്കഭിഷേകവും ചെയ്തു\\
വന്നിതു രാമാന്തികേ സുമിത്രാത്മജന്‍.\\
സോദരനോടും പ്രവര്‍ഷണാഖ്യേ ഗിരൗ\\
സാദരം ചെന്നു കരേറീ രഘൂത്തമന്‍.\\
ഉന്നതമൂര്‍ധ്വശിഖരം പ്രവേശിച്ചു\\
നിന്നനേരമൊരു ഗഹ്വരം കാണായി.\\
സ്ഫടികദീപ്തി കലര്‍ന്നു വിളങ്ങിന\\
ഹാടകദേശം മണിപ്രവരോജ്ജ്വലം\\
വാതവരിഷഹിമാതപവാരണം\\
പാദപവൃന്ദഫലമൂലസഞ്ചിതം\\
തത്രൈവ വാസായ രോചയാമാസ സൗ-\\
മിത്രിണാ ശ്രീരാമഭദ്രന്‍ മനോഹരന്‍.\\
സിദ്ധയോഗീന്ദ്രാദി ഭക്തജനം തദാ\\
മര്‍ത്ത്യവേഷംപൂണ്ട നാരായണന്‍തന്നെ\\
പക്ഷിമൃഗാദിരൂപം ധരിച്ചന്വഹം\\
പക്ഷിധ്വജനെ ബ്ഭജിച്ചു തുടങ്ങിനാര്‍.\\
സ്ഥാവരജംഗമജാതികളേവരും\\
ദേവനെക്കണ്ടു സുഖിച്ചു മരുവിനാര്‍.\\
രാഘവന്‍ തത്ര സമാധിവിരതനാ-\\
യേകാന്തദേശേ മരുവും ദശാന്തരേ\\
ഏകദാ വന്ദിച്ചു സൗമിത്രി സസ്പൃഹം\\
രാഘവനോടു ചോദിച്ചരുളീടിനാന്‍:\\
“കേള്‍ക്കയിലാഗ്രഹം പാരം ക്രിയാമാര്‍ഗ-\\
മാഖ്യാഹി മോക്ഷപ്രദം ത്രിലോകീപതേ!\\
വര്‍ണാശ്രമികള്‍ക്കു മോക്ഷദം പോ,ലതു\\
വര്‍ണിച്ചരുള്‍ചെയ്കവേണം ദയാനിധേ!\\
നാരദവ്യാസവിരിഞ്ചാദികള്‍ സദാ\\
നാരായണപൂജകൊണ്ടു സാധിക്കുന്നു\\
നിത്യം പുരുഷാര്‍ത്ഥമെന്നു യോഗീന്ദ്രന്മാര്‍\\
ഭക്ത്യാ പറയുന്നിതെന്നു കേള്‍പ്പുണ്ടു ഞാന്‍.\\
ഭക്തനായ് ദാസനായുള്ളോരടിയനു\\
മുക്തിപ്രദമുപദേശിച്ചരുളണം.\\
ലോകൈകനാഥ! ഭവാനരുള്‍ചെയ്കിലോ\\
ലോകോപകാരകമാകയുമുണ്ടല്ലോ.”\\
ലക്ഷ്മണനേവമുണര്‍ത്തിച്ച നേരത്തു\\
തല്‍ക്ഷണേ ശ്രീരാമദേവനരുള്‍ചെയ്തു:
\end{verse}

%%09_kriyaamaargopadesham

\section{ക്രിയാമാര്‍ഗോപദേശം}

\begin{verse}
കേള്‍ക്ക നീയെങ്കില്‍ മല്‍പൂജാവിധാനത്തി-\\
നോര്‍ക്കിലവസാനമില്ലെന്നറിക നീ.\\
എങ്കിലും ചൊല്ലുവനൊട്ടു സംക്ഷേപിച്ചു\\
നിങ്കലുള്ളോരു വാത്സല്യം മുഴുക്കയാല്‍.\\
തന്നുടെ തന്നുടെ ഗൃഹ്യോക്തമാര്‍ഗേണ\\
മന്നിടത്തിങ്കല്‍ ദ്വിജത്വമുണ്ടായ് വന്നാല്‍\\
ആചാര്യനോടു മന്ത്രം കേട്ടു സാദര-\\
മാചാര്യപൂര്‍വമാരാധിക്ക മാമെടോ!\\
ഹൃല്‍ക്കമലത്തിങ്കലാകിലുമാം പുന-\\
രഗ്നിഭഗവാങ്കലാകിലുമാമെടോ\\
മുഖ്യ പ്രതിമാദികളിലെന്നാകിലു-\\
മര്‍ക്കങ്കലാകുലുമപ്പിങ്കലാകിലും\\
സ്ഥാണ്ഡിലത്തിങ്കലും നല്ല് സാളഗ്രാമ-\\
മുണ്ടെങിലോ പുനരുത്തമമെത്രയും.\\
വേദതന്ത്രോക്തങ്ങളായ മന്ത്രങ്ങള്‍ കൊ-\\
ണ്ടാദരാല്‍ മൃല്ലേപനാദി വിധിവഴി\\
കാലേ കുളിക്കവേണം ദേഹശുദ്ധയേ\\
മൂലമറിഞ്ഞു സന്ധ്യാവന്ദനാദിയാം\\
നിത്യകര്‍മം ചെയ്തു പിന്നെ സ്വകര്‍മണാ\\
ശുദ്ധ്യര്‍ഥമായ് ചെയ്ക സങ്കല്പമാദിയെ.\\
ആചാര്യനായതു ഞാനെന്നു കല്പിച്ചു\\
പൂജിക്ക ഭക്തിയോടെ ദിവസംപ്രതി.\\
സ്നാപനം ചെയ്ക ശിലായാം പ്രതിമാസു\\
ശോഭനാര്‍ഥം ചെയ്കവേണം പ്രമാര്‍ജനം\\
ഗന്ധപുഷ്പാദ്യങ്ങള്‍കൊണ്ടു പൂജിപ്പവന്‍\\
ചിന്തിച്ചതൊക്കെ ലഭിക്കുമറിക നീ.\\
മുഖ്യപ്രതിമാദികളിലലങ്കാര-\\
മൊക്കെ പ്രസാദമെനിക്കെന്നറിക നീ.\\
അഗ്നൗ യജിക്ക ഹവിസ്സുകൊണ്ടാദരാ-\\
ലര്‍ക്കനെ സ്ഥണ്ഡിലത്തിങ്കലെന്നാകിലോ.\\
മുമ്പിലേ സര്‍വപൂജാദ്രവ്യമായവ\\
സമ്പാദനം ചെയ്തു വേണം തുടങ്ങുവാന്‍.\\
ശ്രദ്ധയോടും കൂടെ വാരിയെന്നാകിലും\\
ഭക്തനായുള്ളവന്‍ തന്നാലതിപ്രിയം,\\
ഗന്ധപുഷ്പാക്ഷതഭക്ഷ്യഭോജ്യാദിക-\\
ളെന്തു പിന്നെപ്പറയേണമോ ഞാനെടോ?\\
വസ്ത്രാജിനകുശാദ്യങ്ങളാലാസന-\\
മുത്തമമായതു കല്പിച്ചുകൊള്ളണം.\\
ദേവസ്യ സമ്മുഖേ ശാന്തനായ് ചെന്നിരി-\\
ന്നാവിര്‍മുദാ ലിപിന്യാസം കഴിക്കണം.\\
ചെയ്ക തത്ത്വന്യാഅസ്വും കേശവാദ്യേന\\
ചെയ്ക മമ മൂര്‍ത്തിപഞ്ജരന്യാസവും\\
പിന്നെ മന്ത്രന്യാസവും ചെയ്തു സാദരം\\
തന്നൂടെ മുമ്പില്‍ വാമേ കലശം വെച്ചു\\
ദക്ഷിണ ഭാഗേ കുസുമാദികളെല്ലാ-\\
മക്ഷതഭക്ത്യൈവ സംഭരിച്ചീടണം.\\
അര്‍ഘ്യപാദ്യപ്രദാനാര്‍ഥമായും മധു-\\
പര്‍ക്കാര്‍ഥമാചമനാര്‍ഥമെന്നിങ്ങനെ\\
പാത്രചതുഷ്ടയവും വെച്ചുകൊള്ളണം\\
പേര്‍ത്തുമറ്റൊന്നും നിരൂപണം കൂടാതെ\\
മല്‍ക്കലാം ജീവസംജ്ഞാം തടിദുജ്ജ്വലാം\\
ഹൃല്‍ക്കമലേ ദൃഢം ധ്യാനിച്ചുകൊള്ളണം.\\
പിന്നെ സ്വദേഹമഖിലം ത്വയാ വ്യാപ്ത-\\
മെന്നുറയ്ക്കേണമിളക്കവും കൂടാതെ.\\
ആവാഹയേല്൬ പ്രതിമാദിഷു മല്‍ക്കലാം\\
ദേവസ്വരൂപമായ് ധ്യാനിക്ക കേവലം.\\
പാദ്യവുമര്‍ഘ്യം തഥാ മധുപര്‍ക്കമി-\\
ത്യാദ്യൈഃ പുനഃ സ്നാനവസ്ത്രവിഭൂഷണൈഃ\\
എത്രയുണ്ടുള്ളതുപചാരമെന്നാലെ-\\
തത്രയുംകൊള്ളാമെനിക്കെന്നതേയുള്ളൂ.\\
ആഗമോക്തപ്രകാരേണ നീരാജനൈര്‍-\\
ദ്ധൂപദീപൈര്‍ന്നിവേദ്യൈര്‍ബഹുവിസ്താരൈഃ\\
ശ്രദ്ധയാ നിത്യമായര്‍ച്ചിച്ചുകൊള്ളൂകില്‍\\
ശ്രദ്ധയാ ഞാനും ഭുജിക്കുമറിക നീ.\\
ഹോമമഗസ്ത്യോക്തമാര്‍ഗകുണ്ഡാനലേ\\
മൂലമന്ത്രം കൊണ്ടുചെയ്യാ, മതെന്നിയേ\\
ഭക്ത്യാ പുരുഷസൂക്തം കൊണ്ടുമാമെടോ\\
ചിത്തതാരിങ്കല്‍ നിനയ്ക്ക കുമാര നീ.\\
ഔപാസനാഗ്നൗ ചരുണാ ഹവിഷാഥ\\
സോപാധിനാ ചെയ്ക ഹോമം മഹാമതേ!\\
തപ്തജാംബൂനന്ദപ്രഖ്യം മഹാപ്രഭം\\
ദീപ്താഭരണവിഭൂഷിതം കേവലം\\
മാമേവ വഹ്നിമദ്ധ്യസ്ഥിതം ധ്യാനിക്ക\\
ഹോമകാലേ ഹൃദി ഭക്ത്യാ ബുധോത്തമന്‍.\\
പാരിഷദാനാം ബലിദാനവും ചെയ്തു\\
ഹോമശേഷത്തെ സമാപയന്മന്ത്രവില്‍\\
ഭക്ത്യാ ജപിച്ചു മാം ധ്യാനിച്ചു മൗനിയായ്\\
വക്ത്രവാസം നാഗവല്ലീദലാദിയും\\
ദത്വാ മദഗ്രേ മഹല്‍പ്രീതി പൂര്‍വകം\\
നൃത്തഗീതസ്തുതിപാഠാദിയും ചെയ്തു\\
പാദാംബുജേ നമസ്കാരവും ചെയ്തുടന്‍\\
ചേതസി മാമുറപ്പിച്ചു വിനീതനായ്\\
മദ്ദത്തമാകും പ്രസാദത്തെയും പുന\\
രുത്തമാഗേ നിധായാനന്ദപൂര്‍വകം\\
‘രക്ഷ മാം ഘോരസംസാരാ’ദിതി മുഹു-\\
രുക്ത്വാ നമസ്കാരവും ചെയ്തനന്തരം\\
ഉദ്വസിപ്പിച്ചുടന്‍ പ്രത്യുങ്മഹസ്സിങ്ക-\\
ലിത്ഥം ദിനമനു പൂജിക്ക മത്സഖേ!\\
ഭക്തിസംയുക്തനായുള്ള മര്‍ത്ത്യന്‍ മുദാ\\
നിത്യമേവം ക്രിയായോഗമനുഷ്ഠിക്കില്‍\\
ദേഹനാശേ മമ സാരൂപ്യവും വരു-\\
മൈഹിക സൗഖ്യങ്ങളെന്തു ചൊല്ലേണമോ?\\
ഇത്ഥം മയോക്തം ക്രിയായോഗമുത്തമം\\
ഭക്ത്യാ പഠിക്ക താന്‍ കേള്‍ക്ക താന്‍ ചെയ്കിലോ\\
നിത്യപൂജാഫലമുണ്ടവനെ“ന്നതും\\
ഭക്തപ്രിയനരുള്‍ചെയ്താനതു നേരം.\\
ശേഷാംശജാതനാം ലക്ഷ്മണന്‍ തന്നോട-\\
ശേഷമിദമരുള്‍ചെയ്തോരനന്തരം\\
മായാമനായ നാരായണന്‍ പരന്‍\\
മായാമവലംബ്യ ദുഃഖം തുടങ്ങിനാന്‍:\\
‘ഹാ ജനകാത്മജേ! സീതേ! മനോഹരേ:\\
ഹാജനമോഹിനീ! നാഥേ! മമ പ്രിയേ!’\\
ഏവമാദി പ്രലാപം ചെയ്തു നിദ്രയും\\
ദേവദേവന്നു വരാതെ ചമഞ്ഞിതു.\\
സൗമിത്രിതന്നുടെ വാക്യാമൃതംകൊണ്ടു\\
സൗമുഖ്യമോടു മരുവൂ ചില നേരം.
\end{verse}

%%10_hanumalsugreevasamvaadam

\section{ഹനൂമല്‍സുഗ്രീവസംവാദം}

\begin{verse}
ഇങ്ങനെ വാഴുന്ന കാലമൊരുദിന-\\
മങ്ങു കിഷ്കിന്ധാപുരത്തിങ്കല്‍ വാഴുന്ന\\
സുഗ്രീവനോടു പറഞ്ഞു പവനജ-\\
നഗ്രേ വണങ്ങിനിന്നേകാന്തമാവണ്ണം:\\
“കേള്‍ക്ക കപീന്ദ്ര! നിനക്കു ഹിതങ്ങളാം\\
വാക്കുകള്‍ ഞാന്‍ പറയുന്നതു സാദരം.\\
നിന്നുടെ കാര്യം വരുത്തി രഘൂത്തമന്‍\\
മുന്നമേ സത്യവ്രതന്‍ പുരുഷോത്തമന്‍\\
പിന്നെ നീയോ നിരൂപിച്ചീലതേതുമെ-\\
ന്നെന്നുടെ മാനസേ തോന്നുന്നിതിന്നഹോ.\\
ബാലിമഹാബലവാന്‍ കപിപുംഗവന്‍\\
ത്രൈലോക്യസമ്മതന്‍ ദേവരാജാത്മജന്‍\\
നിന്നുടെമൂലം മരിച്ചു ബലാലവന്‍\\
മുന്നമേ കാര്യം വരുത്തിക്കൊടുത്തിതു.\\
രാജ്യാഭിഷേകവു ചെയ്തു മഹാജന-\\
പൂജ്യനായ് താരയുമായിരുന്നീടു നീ\\
എത്രനാളുണ്ടിരിപ്പിങ്ങനെയെന്നതും\\
ചിത്തത്തിളുണ്ടു തോന്നുന്നു ധരിക്ക നീ.\\
അദ്യ വാ ശ്വോ വാ പരശ്വോഥവാ തവ\\
മൃത്യു ഭവിക്കുമതിനില്ല സംശയം.\\
പ്രത്യുപകാരം മറക്കുന്ന പൂരുഷന്‍\\
ചത്തതികൊക്കുമേ ജീവിച്ചിരിക്കിലും.\\
പര്‍വതാഗ്രേ നിജസോദരന്‍ തന്നോടു-\\
മുര്‍വീശ്വരന്‍ പരിതാപേന വാഴുന്നു\\
നിന്നെയും പാര്‍ത്തു പറഞ്ഞ സമയവും\\
വന്നതും നീയോ ധരിച്ചതില്ലേതുമേ.\\
വാനരഭാവേന മാനിനീസക്തനായ്\\
പാനവും ചെയ്തു മതിമറന്നന്വഹം\\
രാപ്പകലുമറിയാതെ വസിക്കുന്ന\\
കോപ്പുകളെത്രയും നന്നുനന്നിങ്ങനെ.\\
അഗ്രജനായ ശക്രാത്മജനെപ്പോലെ\\
നിഗ്രഹിച്ചീടും ഭവാനെയും നിര്‍ണയം.”\\
അഞ്ജനാനന്ദനന്‍തന്നുടെ വാക്കുകേ-\\
ട്ടഞ്ജസാ ഭീതനായോരു സുഗ്രീവനും\\
ഉത്തരമായവന്‍തന്നോടു ചൊല്ലിനാന്‍:\\
“സത്യമത്രേ നീ പറഞ്ഞതു നിര്‍ണയം\\
ഇത്തരം ചൊല്ലുമമാത്യനുണ്ടെങ്കിലോ\\
പൃത്ഥ്വീശനാപത്തുമെത്തുകയില്ലല്ലോ.\\
സത്വരമെന്നുടെയാജ്ഞയോടും ഭവാന്‍\\
പത്തുകിക്കിങ്കലേക്കുമയച്ചീടണം\\
സപ്തദ്വീപസ്ഥിതന്മാരായ വാനര-\\
സത്തമന്മാരെ വരുത്തുവാനായ് ദ്രുതം\\
നേരേ പതിനായിരം കപിവീരരെ-\\
പ്പാരാതയയ്ക്ക സന്ദേശപത്രത്തൊടും.\\
പക്ഷത്തിനുള്ളീല്‍ വരണം കപികുലം\\
പക്ഷം കഴിഞ്ഞു വരുത്തതെന്നാകിലോ\\
വധ്യനവന്തിനില്ലൊരു സംശയം\\
സത്യം പറഞ്ഞാലിളക്കമില്ലേതുമേ.’\\
അഞ്ജനാപുത്രനോടിത്ഥം നിയോഗിച്ചു\\
മഞ്ജുളമന്ദിരം പുക്കിരുന്നീടിനാന്‍.\\
ഭര്‍ത്തൃനിയോഗം പുരസ്കൃത്യ മാരുത-\\
പുത്രനും വാനരസത്തമന്മാരെയും\\
പത്തുദിക്കിന്നുമയച്ചാനഭിമത-\\
ദത്തപൂര്‍വം കപീന്ദ്രന്മാരുമന്നേരം\\
വായുവേഗപ്രചാരേണ കപികുല-\\
നായകന്മാരെ വരുത്തുവാനായ് മുദാ\\
പോയിതു ദാനമാനാദി തൃപ്താത്മനാ\\
മായാമനുഷ്യകാര്യാര്‍ത്ഥമിതിദ്രുതം.
\end{verse}

%%11_shreeramantevirahathaapam

\section{ശ്രീരാമന്റെ വിരഹതാപം}

\begin{verse}
രാമനും പര്‍വതമൂര്‍ദ്ധനി ദുഃഖിച്ചു\\
ഭാമിനിയോടും പിരിഞ്ഞുവാഴുംവിധൗ\\
താപേന ലക്ഷ്മണന്‍തന്നോടു ചൊല്ലിനാന്‍:\\
‘പാപമയ്യോ! മമ കാണ്‍ക! കുമാര! നീ.\\
ജാനകീദേവി മരിച്ചിതോ കുത്രചില്‍\\
മാനസതാപേന ജീവിച്ചിരിക്കയോ?\\
നിശ്ചയിച്ചേതുമറിഞ്ഞതുമില്ലല്ലോ\\
കശ്ചില്‍ പുരുഷനെന്നോടു സമ്പ്രീതനായ്\\
ജീവിച്ചിരിക്കുന്നിതെന്നു ചൊല്ലീടുകില്‍\\
കേവലമെത്രയുമിഷ്ടനവന്‍ മമ.\\
എങ്ങാനുമുണ്ടിരിക്കുന്നതെന്നാകില്‍ ഞാ-\\
നിങ്ങു ബലാല്‍ കൊണ്ടുപോരുവന്‍ നിര്‍ണയം.\\
ജാനകീദേവിയെക്കട്ടകള്ളന്‍തന്നെ\\
മാനസകോപേന നഷ്ടമാക്കീടുവന്‍\\
വംശവും കൂടെയൊടുക്കുന്നതുണ്ടൊരു\\
സംശയമേതുമിതിനില്ല നിര്‍ണയം.\\
എന്നെയും കാണാഞ്ഞു ദുഃഖിച്ചിരിക്കുന്ന\\
നിന്നെ ഞാനെന്നിനിക്കണുന്നു വല്ലഭേ!\\
ചന്ദ്രാനനേ നീ പിരിഞ്ഞതു കാരണം\\
ചന്ദ്രനുമാദിത്യനെപ്പോലെയായിതു\\
ചന്ദ്ര! ശീതാംശുക്കളാലവളെച്ചെന്നു\\
മന്ദമന്ദം തലോടിന്നലോടിത്തദാ\\
വന്നു തടവീടുകെന്നെയും സാദരം\\
നിന്നുടെ ഗോത്രജയല്ലോ ജനകജ.\\
സുഗ്രീവനും ദയാഹീനനത്രേ തുലോം\\
ദുഃഖിതനാമെന്നെയും മറന്നാനല്ലോ.\\
നിഷ്കണ്ടകം രാജ്യമാശു ലഭിച്ചവന്‍\\
മൈക്കണ്ണിമാരോടുകൂടെ ദിവാനിശം\\
മദ്യപാനാസക്തചിത്തനാം കാമുകന്‍\\
വ്യക്തം കൃതഘ്നനത്രേ സുമിത്രാത്മജ!\\
വന്നു ശരക്കാലമെന്നതു കണ്ടവന്‍\\
വന്നീലയല്ലോ പറഞ്ഞവണ്ണം സഖേ!\\
അന്വേഷണം ചെയ്തു സീതാധിവാസവു-\\
മിന്നേടമെന്നറിഞ്ഞീടുവാനായവന്‍\\
പൂര്‍വോപകാരിയാമെന്നെ മറക്കയാല്‍\\
പൂര്‍വനവന്‍ കൃതഘ്നന്മാരില്‍ നിര്‍ണയം.\\
ഇഷ്ടരില്‍ മുമ്പുണ്ടു സുഗ്രീവനോര്‍ക്ക നീ\\
കിഷ്കിന്ധയോടും ബന്ധുക്കളോടുംകൂടെ\\
മര്‍ക്കടശ്രേഷ്ഠനെ നിഗ്രഹിച്ചീടുവന്‍\\
അഗ്രജമാര്‍ഗം ഗമിക്കേണമിന്നിനി-\\
സ്സുഗ്രീവനുമതിനില്ലൊരു സംശയം.”\\
ഇത്ഥമരുള്‍ചെയ്ത രാഘവനോടതി-\\
ക്രുദ്ധനായോരു സൗമിത്രി ചൊല്ലീടിനാന്‍:\\
‘വധ്യനായോരു സുഗ്രീവനെസ്സത്വരം\\
ഹത്വാ വിടകൊള്‍വനദ്യ തവാന്തികം\\
ആജ്ഞാപയാശു മാ“മെന്നു പറഞ്ഞിതു\\
പ്രാജ്ഞനായോരു സുമിത്രാതനയനും.\\
ആദായ ചാപതൂണീരഖഡ്ഗങ്ങളും\\
ക്രോധേന ഗന്തുമഭ്യുദ്യതം സോദരം\\
കണ്ടു രഘുപതിചൊല്ലിനാന്‍ പിന്നെയു-\\
“മുണ്ടൊന്നു നിന്നോടിനിയും പറയുന്നു.\\
ഹന്തവ്യനല്ല സുഗ്രീവന്‍ മമ സഖി-\\
കിന്തു ഭയപ്പെടുത്തീടുകെന്നേ വരൂ.\\
‘ബാലിയെപ്പോലെ നിനക്കും വിരവോടു\\
കാലപുരത്തിനു പോകാമറിക നീ”\\
ഇത്ഥമവനോടു ചെന്നു ചൊന്നാലതി-\\
നുത്തരം ചൊല്ലുന്നതും കേട്ടുകൊണ്ടു നീ\\
വേഗേന വന്നാലതിന്നനുരൂപമാ-\\
മാകൂതമോര്‍ത്തു കര്‍ത്തവ്യമനന്തരം.’
\end{verse}

%%12_lakshmanantepurappaadu

\section{ലക്ഷ്മണന്റെ പുറപ്പാട്}

\begin{verse}
അഗ്രജന്മാജ്ഞയാ സൗമിത്രി സത്വരം\\
സുഗ്രീവരാജ്യംപ്രതി നടന്നീടിനാന്‍\\
കിഷ്കിന്ധയോടു ദഹിച്ചുപോമിപ്പോഴേ\\
മര്‍ക്കടജാതികളെന്നുതോന്നും വണ്ണം.\\
വിജ്ഞാനമൂര്‍ത്തി സര്‍വജ്ഞനനാകുല-\\
നജ്ഞാനിയായുള്ള മാനുഷനെപ്പോലെ\\
ദുഃഖസുഖാദികള്‍ കൈക്കൊണ്ടുവര്‍ത്തിച്ചു\\
ദുഷ്കൃതശാന്തി ലോകത്തിനുണ്ടാക്കുവാന്‍.\\
മുന്നം ദശരഥന്‍ ചെയ്ത തപോബലം\\
തന്നുടെ സിദ്ധിവരുത്തിക്കൊടുപ്പാനും\\
പങ്കജസംഭവനാദികള്‍ക്കുണ്ടായ\\
സങ്കടാം തീര്‍ത്തു രക്ഷിച്ചു കൊടുപ്പാനും\\
മാനുഷവേഷം ധരിച്ചു പരാപര-\\
നാനന്ദമൂര്‍ത്തി ജഗന്മയനീശ്വരന്‍.\\
നാനാജനങ്ങളും മായയാ മോഹിച്ചു\\
മാനസമജ്ഞാനസംയുതമാകയാല്‍\\
മോക്ഷം വരുത്തുന്നതെങ്ങനെ ഞാനെന്നു\\
സാക്ഷാല്‍ മഹാവിഷ്ടു ചിന്തിച്ചു കല്പിച്ചു\\
സര്‍വജഗന്മായാനാശിനിയാകിയ\\
ദിവ്യകഥയെ പ്രസിദ്ധയാക്കൂ യഥാ\\
രാമനായ് മാനുഷവ്യാപാരജാതയാം\\
രാമായണാഭിധാമാനന്ദദായിനീം\\
സല്‍ക്കഥാമിപ്രപഞ്ചത്തിങ്കലൊക്കവേ\\
വിഖ്യാതയാക്കുവാനാനന്ദപൂരുഷന്‍\\
ക്രോധവും മോഹവും കാമവും രാഗവും\\
ഖേദാദിയും വ്യവഹാരാര്‍ത്ഥസിദ്ധയേ\\
തത്തല്‍ക്രിയാകാലദേശോചിതം നിജ-\\
ചിത്തേ പരിഗ്രഹിച്ചീടിനാനീശ്വരന്‍.\\
സത്വാദികളാം ഗുണങ്ങളില്‍ത്താനനു-\\
രക്തനെപ്പോലെ ഭവിക്കുന്നു നിര്‍ഗുണന്‍\\
വിജ്ഞാനശക്തിമാനവ്യക്തനന്ദ്വയന്‍\\
കാമാദികളാലവിലിപ്തനവ്യയന്‍\\
വ്യോമവദ്വ്യാപ്തനനന്തനനാമയന്‍\\
ദിവ്യമുനീശ്വരന്മാര്‍ സനകാദികള്‍\\
സര്‍വാത്മകനെച്ചിലരറിഞ്ഞീടുവോര്‍.\\
നിര്‍മലാത്മാക്കളായുള്ള ഭക്തന്മാര്‍ക്കു\\
സമ്യക് പ്രബോധമുണ്ടാമെന്നു ചൊല്ലുന്നു.\\
ഭക്തചിത്താനുസാരേണ സഞ്ജായതേ\\
മുക്തിപ്രദന്‍ മുനിവൃന്ദനിഷേവിതന്‍.\\
കിഷ്കിന്ധയാം നഗരാന്തികം പ്രാപിച്ചു\\
ലക്ഷ്മണനും ചെറുഞാണൊലിയിട്ടിതു.\\
മര്‍ക്കടന്മാരവനെക്കണ്ടു പേടിച്ചു\\
ചക്രുഃ കിലുകിലശബ്ദം പരവശാല്‍.\\
വപ്രോപരിപാഞ്ഞു കല്ലും മരങ്ങളും\\
വിഭ്രമത്തോടു കൈയില്‍ പിടിച്ചേവരും\\
പേടിച്ചു മൂത്രമലങ്ങള്‍ വിസര്‍ജിച്ചു\\
ചാടിത്തുടങ്ങിനാരങ്ങുമിങ്ങും ദ്രുതം.\\
മര്‍ക്കടക്കൂട്ടത്തെയൊക്കെയൊടുക്കുവാ-\\
നുള്‍ക്കാമ്പിലഭ്യുദ്യതനായ സൗമിത്രി\\
വില്ലും കുഴിയെക്കുലച്ചു വലിച്ചിതു.\\
ഭല്ലൂകവൃന്ദവും വല്ലാതെയായിതു.\\
ലക്ഷ്മണനാഗതനായതറിഞ്ഞഥ\\
തല്‍ക്ഷണമംഗദനോടി വന്നീടിനാന്‍.\\
ശാഖാമൃഗങ്ങളെയാട്ടിക്കളഞ്ഞു താ-\\
നേകനായ്ച്ചെന്നു നമസ്കരിച്ചീടിനാന്‍.\\
പ്രീതനായാശ്ലേഷവും ചെയ്തവനോടു\\
ജാതമോദം സുമിത്രാത്മജന്‍ ചൊല്ലിനാന്‍:\\
‘ഗച്ഛ വത്സ! ത്വം പിതൃവ്യനെക്കണ്ടു ചൊ-\\
ല്ലിച്ചെയ്തകാര്യം പിഴയ്ക്കുമെന്നാശു നീ.\\
ഇച്ഛയായുള്ളതു ചെയ്ത മിത്രത്തെ വ-\\
ഞ്ചിച്ചാലനര്‍ത്ഥമവിളംബിതം വരും.\\
ഉഗ്രനാമഗ്രജനെന്നോടരുള്‍ചെയ്തു\\
തിഗ്രഹിച്ചീടുവാന്‍ സുഗ്രീവനെ ക്ഷണാല്‍\\
അഗ്രജമാര്‍ഗം ഗമിക്കണമെന്നുണ്ടു\\
സുഗ്രീവനുള്‍ക്കാമ്പിലെങ്കിലതേ വരു\\
എന്നരുള്‍ചെയ്തതു ചെന്നു പറകെ’ന്നു\\
ചൊന്നതും കേട്ടൊരു ബാലിതനയനും\\
തന്നുള്ളിലുണ്ടായ ഭിതിയോടുമവന്‍\\
ചെന്നു സുഗ്രീവനെ വന്ദിച്ചു ചൊല്ലിനാന്‍:\\
“കോപേന ലക്ഷ്മണന്‍ വന്നിതാ നില്‍ക്കുന്നു\\
ഗോപുരദ്വാരി പുറത്തുഭാഗ, ത്തിനി\\
കാപേയഭാവം കളഞ്ഞു വന്ദിക്ക ചെ-\\
ന്നാപത്തതല്ലായ്കിലുണ്ടായ്വരും ദൃഢം.’\\
സന്ത്രസ്തനായ സുഗ്രീവനതു കേട്ടു\\
മന്ത്രിപ്രവരനാം മാരുതിതന്നോടു\\
ചിന്തിച്ചു ചൊല്ലിനാനംഗദനോടു കൂ-\\
ടന്തികേ ചെന്നു വന്ദിക്ക സൗപിത്രിയെ\\
സാന്ത്വനം ചെയ്തു കൂട്ടിക്കൊണ്ടു പോരിക\\
ശാന്തനായോരു സുമിത്രാതനയനെ.”\\
മാരുതിയെപ്പറഞ്ഞേവമയച്ചഥ\\
താരയോടര്‍ക്കാത്മജന്‍ പറഞ്ഞീടിനാന്‍:\\
“താരാധിപാനനേ! പോകണമാശു നീ\\
താരേ! മനോഹരേ! ലക്ഷ്മണന്‍തന്നുടെ\\
ചാരത്തു ചെന്നു കോപത്തെശ്ശമിപ്പിക്ക\\
സാരസ്യസാരവാക്യങ്ങളാല്‍, പിന്നെ നീ\\
കൂട്ടിക്കൊണ്ടിങ്ങു പോന്നെന്നെയും വേഗേന\\
കാട്ടിക്കലുഷഭാവത്തെയും നീക്കണം.”\\
ഇത്ഥമര്‍ക്കാത്മജന്‍ വാക്കുകള്‍ കേട്ടവള്‍\\
മദ്ധ്യകക്ഷ്യാം പ്രവേശിച്ചു നിന്നീടിനാള്‍.\\
താരാതനയനും മാരുതിയും കൂടി\\
ശ്രീരാമസോദരന്‍ തന്നെ വണങ്ങിനാര്‍.\\
ഭക്ത്യാ കുശലപ്രശ്നങ്ങളും ചെയ്തു സൗ-\\
മിത്രിയോടഞ്ജനാനന്ദനന്‍ ചൊല്ലിനാന്‍:\\
“എന്തു പുറത്തുഭാഗേനിന്നരുളുവാ-\\
നന്തഃപുരത്തിലാമ്മാറെഴുന്നള്ളണം.\\
രാജദാരങ്ങളേയും നഗരാഭയും\\
രാജാവു സുഗ്രീവനേയും കനിവോടു\\
കണ്ടു പറഞ്ഞാലനന്തരം നാഥനെ-\\
ക്കണ്ടുവണങ്ങിയാല്‍ സാദ്ധ്യമെല്ലാം ദ്രുതം.’\\
ഇത്ഥം പറഞ്ഞു കൈയുംപിടിച്ചാശു സൗ-\\
മിത്രിയോടും മന്ദമന്ദം നടന്നിതു\\
യൂഥപന്മാര്‍ മരുവീടും മണിമയ-\\
സൗധങ്ങളും പുരീശോഭയും കണ്ടുക-\\
ണ്ടാനന്ദമുള്‍ക്കൊണ്ടു മദ്ധ്യകക്ഷ്യാം ചെന്നു\\
മാനിച്ചു നിന്ന നേരത്തു കാണായ് വന്നു\\
താരേശതുല്യമുഖിയായ മാനിനീ\\
താരാ ജഗന്മനോമോഹിനി സുന്ദരി\\
ലക്ഷ്മീസമാനയായ് നില്ക്കുന്ന, തന്നേരം\\
ലക്ഷ്മണന്‍ തന്നെ വണങ്ങി വിനീതയായ്\\
മന്ദസ്മിതം പൂണ്ടു ചൊന്നാളഹോ, ’തവ\\
മന്ദിരമായതിതെന്നറിഞ്ഞീലയോ?\\
ഭക്തനായെത്രയുമുത്തമനായ് തവ\\
ഭൃത്യനായോരു കപീന്ദ്രനോടിങ്ങനെ\\
കോപമുണ്ടായാലവനെന്തൊരു ഗതി?\\
ചാപല്യമേറുമിജ്ജാതികള്‍ക്കോര്‍ക്കണം.\\
മര്‍ക്കടവീരന്‍ ബഹുകാലമുണ്ടല്ലോ\\
ദുഃഖമനുഭവിച്ചീടുന്നു ദീനനായ്.\\
ഇക്കാലമാശു ഭവല്‍കൃപയാ പരി-\\
രക്ഷിതനാകയാല്‍ സൗഖ്യം കലര്‍ന്നവന്‍\\
വാണാനതും വിപരീതമാക്കീടായ്ക-\\
വേണം ദയാനിധേ! ഭക്തപരായണ!\\
നാനാദിഗന്തരം തോറും മരുവുന്ന\\
വാനരന്മാരെ വരുത്തുവാനായവന്‍\\
പത്തു സഹസ്രം ദൂതന്മാരെ വിട്ടിതു\\
പത്തു ദിക്കീന്നും കപികുല പ്രൗഢരും\\
വന്നു നിറഞ്ഞതു കാണ്‍കിവിടെ പ്പുന-\\
രൊന്നിനും ദണ്ഡമിനിയില്ല നിര്‍ണയം\\
നക്തഞ്ചരകുലമൊക്കെയൊടുക്കുവാന്‍\\
ശക്തരത്രേ കപിസത്തമന്മാരെല്ലാം\\
പുത്രകളത്രമിത്രാന്വിതനാകിയ\\
ഭൃത്യനാം സുഗ്രീവനെക്കണ്ടവനുമായ്\\
ശ്രീരാമദേവപാദാംബുജം വന്ദിച്ചു\\
കാര്യവുമാശു സാധിക്കാമറിഞ്ഞാലും.”\\
താരാവചനമേവം കേട്ടു ലക്ഷ്മണന്‍\\
പാരാതെ ചെന്നു സുഗ്രീവനെയും കണ്ടു.\\
സത്രപം വിത്രസ്തനായ സുഗ്രീവനും\\
സത്വരമുത്ഥാനവും ചെയ്തു വന്ദിച്ചു\\
മത്തനായ് വിഹ്വലിതേക്ഷണനാം കപി-\\
സത്തമനെക്കണ്ടു കോപേന ലക്ഷ്മണന്‍\\
മിത്രാത്മജനോടു ചൊല്ലിനാന്‍,’നീ രഘു-\\
സത്തമന്‍ തന്നെ മറന്നതെന്തിങ്ങനെ?\\
വൃത്രാരിപുത്രനെക്കൊന്ന ശരമാര്യ-\\
പുത്രന്‍ കരസ്ഥിതമെന്നുമറിക നീ\\
അഗ്രജമാര്‍ഗം ഗമിക്കയിലാഗ്രഹം\\
സുഗ്രീവനുണ്ടെന്നു നാഥനരുള്‍ചെയ്തു.’\\
ഇത്തരം സൗമിത്രി ചൊന്നതു കേട്ടതി-\\
നുത്തരം മാരുതപുത്രനും ചൊല്ലിനാന്‍:\\
“ഇത്ഥമരുള്‍ചെയ്വതിനെന്തു കാരണം\\
ഭക്തനേറ്റം പുരുഷോത്തമങ്കല്‍ കപി-\\
സത്തമനോര്‍ക്കില്‍ സുമിത്രാത്മജനിലും\\
സത്യവും ലംഘിക്കയില്ല കപീശ്വരന്‍\\
രാമകാര്യാര്‍ത്ഥമുണര്‍ന്നിരിക്കുന്നിതു\\
താമസമെന്നിയേ വാനരപുംഗവന്‍\\
വിസ്മൃതനായിരുന്നീടുകയല്ലേതും\\
വിസ്മയമാമ്മാറു കണ്ടീലയോ ഭവാന്‍?\\
വേഗേന നാനാദിഗന്തരത്തിങ്കല്‍ നി-\\
ന്നാഗതന്മാരായ വാനരവീരരെ?\\
ശ്രീരാമകാര്യമശേഷേണ സാധിക്കു-\\
മാമയമെന്നിയേ വാനരനായകന്‍.”\\
മാരുതി ചൊന്നതു കേട്ടു സൗമിത്രിയു-\\
മാരൂഢലജ്ജനായ് നില്ക്കും ദശാന്തരേ\\
സുഗ്രീവനര്‍ഘ്യപാദ്യാദ്യേന പൂജചെ-\\
യ്തഗ്രഭാഗേ വീണു വീണ്ടും വണങ്ങിനാന്‍.\\
‘ശ്രീരാമദാസോഹമാഹന്ത! രാഘവ-\\
കാരുണ്യലേശേന രക്ഷിതനദ്യ ഞാന്‍\\
ലോകത്രയത്തെ ക്ഷണാര്‍ദ്ധമാത്രംകൊണ്ടു\\
രാഘവന്‍ തന്നെ ജയിക്കുമല്ലോ ബലാല്‍.\\
സേവാര്‍ത്ഥമോര്‍ക്കില്‍ സഹായമാത്രം ഞങ്ങ-\\
ളേവരും തന്നിയോഗത്തെ വഹിക്കുന്നു.’\\
അര്‍ക്കാത്മജന്‍മൊഴി കേട്ടു സൗമിത്രിയു-\\
മുള്‍ക്കാമ്പഴിഞ്ഞവനോടു ചൊല്ലീടിനാന്‍:\\
‘ദുഃഖേന ഞാന്‍ പരുഷങ്ങള്‍ പറഞ്ഞതു-\\
മൊക്കെ ക്ഷമിക്ക മഹാഭാഗനല്ലോ നീ\\
നിങ്കല്‍ പ്രണയമധികമുണ്ടാകയാല്‍\\
സങ്കടംകൊണ്ടു പറഞ്ഞിതു ഞാനെടോ!\\
വൈകാതെ പോക വനത്തിനു നാമിനി\\
രാഘവന്‍ താനേ വസിക്കുന്നതുമെടോ!’
\end{verse}

%%13_sugreevanshreeraamasannidhiyil

\section{സുഗ്രീവന്‍ ശ്രീരാമസന്നിധിയില്‍}

\begin{verse}
‘അങ്ങനെ തന്നെ പുറപ്പെടുകെങ്കില്‍ നാ-\\
മിങ്ങിനിപ്പാര്‍ക്കയില്ലെ’ന്നു സുഗ്രീവനും\\
തേരില്‍ കരേറി സുമിത്രാത്മജനുമായ്\\
ഭേരീമൃദംഗശംഖാദിനാദത്തൊടും\\
അഞ്ജനാപുത്ര നീലാംഗദാദ്യൈരല-\\
മഞ്ജസാ വാനരസേനതോടും തദാ\\
ചാമരശ്വേതാതപത്രവ്യജനവാന്‍\\
സാമരസൈന്യനാഖണ്ഡലനെപ്പോലെ\\
രാമന്‍ തിരുവടിയെച്ചെന്നു കാണ്മതി-\\
ന്നാമോദമോടു നടന്നു കപിവരന്‍.\\
ഗഹ്വരദ്വാരി ശിലാതലേ വാഴുന്ന\\
വിഹ്വലമാനസം ചീരാജിനധരം\\
ശ്യാമം ജടാമകുടോജ്ജ്വലം മാനവം\\
രാമം വിശാലവിലോലവിലോചനം\\
ശാന്തം മൃദുസ്മിത ചാരുമുഖാംബുജം\\
കാന്താവിരഹസന്തപ്തം മനോഹരം\\
കാന്തം മൃഗപക്ഷിസഞ്ചയസേവിതം\\
ദാന്തം മുദാ കണ്ടു ദൂരാല്‍ കപിവരന്‍\\
തേരില്‍ നിന്നാശു താഴത്തിറങ്ങീടിനാന്‍\\
വീരനായോരു സൗമിത്രിയോടും തദാ.\\
ശ്രീരാമപാദാരവിന്ദാന്തികേ വീണു\\
പൂരിച്ച ഭക്ത്യാ നമസ്കരിച്ചീടിനാന്ട്.\\
ശ്രീരാമദേവനും വാനര വീരനെ-\\
ക്കാരുണ്യമോടു ഗാഢം പുണര്‍ന്നീടിനാന്‍.\\
‘സൗഖ്യമല്ലീ ഭവാനെ’,ന്നുരചെയ്തുട-\\
നൈക്യഭാവേന പിടിച്ചിരുത്തീടിനാന്‍.\\
ആതിഥ്യമായുള്ള പൂജയും ചെയ്തള-\\
വാദിത്യപുത്രനും പ്രീതിപൂണ്ടാന്‍ തുലോം.
\end{verse}

%%14_seethaanveshanodyogam

\section{സീതാന്വേഷണോദ്യോഗം}

\begin{verse}
ഭക്തിപരവശനായ സുഗ്രീവനും\\
ഭക്തപ്രിയനോടുണര്‍ത്തിച്ചിതന്നേരം:\\
‘വന്നു നില്ക്കുന്ന കപികുലത്തെക്കനി-\\
ഞ്ഞൊന്നു തൃക്കണ്‍പാര്‍ത്തരുളേണമാദരാല്‍.\\
തൃക്കാല്ക്കല്‍ വേലചെയ്തീടുവാന്‍തക്കൊരു\\
മര്‍ക്കടവീരരിക്കാണായതൊക്കവേ.\\
നാനാകുലാചലദംഭവന്‍മാരിവര്‍\\
നാനാസരിദ്ദ്വീപശൈലനിവാസികള്‍\\
പര്‍വതതുല്യശരീരികളേവരു-\\
മുര്‍വീപതേ! കാമരൂപികളെത്രയും.\\
ഗര്‍വം കലര്‍ന്ന നിശാചരന്മാരുടെ\\
ദുര്‍വീര്യമെല്ലാമുടക്കുവാന്‍ പോന്നവര്‍\\
ദേവാംശസംഭവന്മാരിവരാകയാല്‍\\
ദേവാരികളെയൊടുക്കുമിവരിനി.\\
കേചില്‍ ഗജബലന്മാരതിലുണ്ടു താന്‍\\
കേചില്‍ ദശഗജശക്തിയുള്ളോരുണ്ട്\\
കേചിദമിതപരാക്രമമുള്ളവര്‍\\
കേചിന്‍മൃഗേന്ദ്രസമന്മാരറിഞ്ഞാലും\\
കേചിന്മഹേന്ദ്രനീലോപലരൂപികള്‍\\
കേചില്‍ കനകസമാനശരീരികള്‍\\
കേചന രക്താന്തനേത്രം ധരിച്ചവര്‍\\
കേചന ദീര്‍ഘവാലന്മാരഥാപരേ\\
ശുദ്ധസ്ഫടികസങ്കാശശരീരികള്‍\\
യുദ്ധവൈദഗ്ദ്ധ്യമിവരോളമില്ലാര്‍ക്കും.\\
നിങ്കഴല്‍പ്പങ്കജത്തിങ്കലുറച്ചവര്‍\\
സംഖ്യയില്ലാതോളമുണ്ടു കപിബലം\\
മൂലഫലദലപക്വാശനന്മാരായ്\\
ശിലഗുണമുള്ള വാനരന്മാരിവര്‍\\
താവകാജ്ഞാകാരികളെന്നു നിര്‍ണയം\\
ദേവദേവേശ! രഘുകുലപുംഗവ!\\
ഋക്ഷകുലാധിപനായുള്ള ജാംബവാന്‍\\
പുഷ്കരസംഭവപുത്രനിവനല്ലോ.\\
കോടി ഭല്ലൂകവൃന്ദാധിപതി മഹാ-\\
പ്രൗഢമതി ഹനൂമാനിവനെന്നുടെ\\
മന്തിവരന്‍ മഹാസത്വപരാക്രമന്‍\\
ഗന്ധവാഹാത്മജനീശാംശസംഭവന്‍.\\
നീലന്‍ ഗജന്‍ ഗവയന്‍ ഗവാക്ഷന്‍ ദീര്‍ഘ-\\
വാലധിപൂണ്ടവന്‍ മൈന്ദന്‍ വിവിദനും\\
കേസരി മാരുതിതാതന്‍ മഹാബലി\\
വീരന്‍ പ്രമാഥി ശരഭന്‍ സുഷേണനും\\
ശൂരന്‍ സുമുഖന്‍ ദദ്ധിമുഖന്‍ ദുര്‍മുഖന്‍\\
ശ്വേതന്‍ വലീമുഖനും ഗന്ധമാദനന്‍\\
താരന്‍ വൃഷഭന്‍ നളന്‍ വിനതന്‍ മമ\\
താരാതനയനാമംഗദനിങ്ങനെ\\
ചൊല്ലുള്ള വാനരവംശരാജാക്കന്മാര്‍\\
ചൊല്ലുവാനാവതല്ലാതോളമുണ്ടല്ലോ.\\
വേണ്ടുന്നതെന്തെന്നിവരോടരുള്‍ചെയ്ക\\
വേണമെന്നാലിവര്‍ സാധിക്കുമൊക്കവേ.”\\
സുഗ്രീവവാക്യമിത്ഥം കേട്ടു രാഘവന്‍\\
സുഗ്രീവനെപ്പിടിച്ചാലിംഗനം ചെയ്തു\\
സന്തോഷപൂര്‍ണാശ്രുനേത്രാംബുജത്തോടു-\\
മന്തര്‍ഗതമരുള്‍ചെയ്തിതു സാദരം:\\
‘മല്‍ക്കാര്യഗൗരവം നിങ്കലേ നിര്‍ണയ-\\
മുള്‍ക്കാമ്പിലോര്‍ത്തു കര്‍ത്തവ്യം കുരുഷ്വ നീ.\\
ജാനകീമാര്‍ഗണാര്‍ത്ഥം നിയോഗിക്ക നീ\\
വാനരവീരരെ നാനാദിശി സഖേ!’\\
ശ്രീരാംഅവാക്യാമൃതം കേട്ടു വാനര-\\
വീരനയച്ചിതു നാലുദിക്കിങ്കലും.\\
“നൂറായിരം കപിവീരന്മാര്‍പോകണ-\\
മോരോ ദിഷി പടനായകന്മാരൊടും\\
പിന്നെ വിശേഷിച്ചു ദക്ഷിണദിക്കിന-\\
ത്യുന്നതന്മാര്‍ പലരും പോയ്ത്തിരയണം\\
അംഗദന്‍ ജാംബവാന്‍ മൈന്ദന്‍ വിവിദനും\\
തുംഗന്‍ നളനും ശരഭന്‍ സുഷേണനും\\
വാതാത്മജന്‍ ശ്രീഹനൂമാനുമായ് ചെന്നു\\
ബാധയൊഴിഞ്ഞുടന്‍ കണ്ടു വന്നീടണം.\\
അത്ഭുതഗാത്രിയെ നീളെത്തിരഞ്ഞിങ്ങു\\
മുപ്പതുനാളിനകത്തു വന്നീടണം\\
ഉല്പലപത്രാക്ഷിതന്നെയും കാണാതെ\\
മുപ്പതുനാള്‍ കഴിഞ്ഞിങ്ങു വരുന്നവന്‍\\
പ്രാണാന്തികം ദണ്ഡമാശു ഭുജിക്കണ-\\
മേണാങ്കശേഖരന്‍തന്നാണെ നിര്‍ണയം.”\\
നാലുകൂട്ടത്തോടു മിത്ഥംനിയോഗിച്ചു\\
കാലമേ പോയാലുമെന്നയച്ചീടിനാന്‍.\\
രാഘവന്‍തന്നെത്തൊഴുതരികേ ചെന്നു\\
ഭാഗവതൊത്തമനുമിരുന്നീടിനാന്‍.\\
ഇത്ഥം കപികള്‍ പുറപ്പെട്ട നേരത്തു\\
ഭക്ത്യാ തൊഴുതിതു വായുതനയനും\\
അപ്പോളവനെ വേറേ വിളിച്ചാദരാ-\\
ലത്ഭുതവിക്രമന്‍താനുമരുള്‍ചെയ്തു:\\
‘മാനസേ വിശ്വാസമുണ്ടാവതിന്നു നീ\\
ജാനകികൈയില്‍ കൊടുത്തീടിതു സഖേ!\\
രാമനാമാങ്കിതമാമംഗുലീയകം\\
ഭാമിനിക്കുള്ളില്‍ വികല്പം കളവാനായ്.\\
എന്നുടെ കാര്യത്തിനോര്‍ക്കില്‍ പ്രമാണം നീ-\\
യെന്നിയേ മറ്റാരുമില്ലെന്നു നിര്‍ണയം.’\\
പിന്നെയടയാളവാക്കുമരുള്‍ചെയ്തു\\
മന്നവന്‍, പോയാലുമെന്നയച്ചീടിനാന്‍.\\
ലക്ഷ്മീഭഗവതിയാകിയ സീതയാം\\
പുഷ്കരപത്രാക്ഷിയെക്കൊണ്ടുപോയൊരു\\
രക്ഷോവരനായ രാവണന്‍ വാഴുന്ന\\
ദക്ഷിണദിക്കുനോക്കിക്കപിസഞ്ചയം\\
ലക്ഷവും വൃത്രാരിപുത്രതനയനും\\
പുശ്കരസംഭവപുത്രനും നീലനും\\
പുഷകരബാന്ധവശിഷ്യനും മറ്റുള്ള\\
മര്‍ക്കടസേനാപതികളുമായ് ദ്രുതം\\
നാനാ നഗനഗരഗ്രാമദേശങ്ങള്‍\\
കാനനരാജ്യപുരങ്ങളിലും തഥാ\\
തത്ര തത്രൈവ തിരഞ്ഞു തിരഞ്ഞതി-\\
സത്വരം നീളെ നടക്കും ദശാന്തരേ\\
ഗന്ധവാഹാത്മജനാദികളൊക്കവേ\\
വിന്ധ്യാചലാടവി പുക്കു തിരയുമ്പോള്‍\\
ഘോരമൃഗങ്ങളേയും കൊന്നുതിന്നതി-\\
ക്രൂരനായോരു നിശാചരവീരനെ-\\
ക്കണ്ടു വേഘത്തോടടുത്താരിതു ദശ-\\
കണ്ഠനെന്നോര്‍ത്തു കപിവരന്മാരെല്ലാം\\
നിഷ്ഠുരമായുള്ള മുഷ്ടിപ്രഹാരേണ\\
ദുഷ്ടനെപ്പെട്ടെന്നു നഷ്ടമാക്കീടിനാര്‍.\\
പംക്തിമുഖനല്ലിവനെന്നു മാനസേ\\
ചിന്തിച്ചു പിന്നെയും വേഗേന പോയവര്‍.
\end{verse}

%%15_svayamprabhaagathi

\section{സ്വയംപ്രഭാഗതി}

\begin{verse}
അന്ധകാരാരണ്യമാശു പുക്കീടിനാ-\\
രന്തരാ ദാഹവും വര്‍ധിച്ചിതേറ്റവും\\
ശുഷ്കകണ്ഠോഷ്ഠതാലുപ്രദേശത്തൊടും\\
മര്‍ക്കടവീരരുണങ്ങിവരണ്ടൊരു\\
ജിഹ്വയോടും നടക്കുന്ന നേരത്തൊരു\\
ഗഹ്വരം തത്ര കാണായി വിധിവശാല്‍.\\
വല്ലീതൃണഗണച്ഛന്നമായൊന്നതി-\\
ലില്ലയല്ലീജലമെന്നോര്‍ത്തു നില്ക്കുമ്പോള്‍\\
ആര്‍ദ്രപക്ഷ ക്രൗഞ്ചഹംസാദി പക്ഷിക-\\
ളൂര്‍ദ്ധ്വദേശേ പറന്നാരതില്‍നിന്നുടന്‍\\
പക്ഷങ്ങളില്‍ നിന്നു വീടു ജലകണം\\
മര്‍ക്കടന്മാരുമതുകണ്ടു കല്പിച്ചാര്‍.\\
“നല്ല ജലമതിലുണ്ടെന്നു നിര്‍ണയ-\\
മെല്ലാവരും നാമിതിലിറങ്ങീടുക.’\\
എന്നു പറഞ്ഞോരു നേരത്തു മാരുതി\\
മുന്നിലിറങ്ങിനാന്‍ മറ്റുള്ളവര്‍കളും\\
പിന്നാലെ തന്നിലിറങ്ങി നടക്കുമ്പോള്‍\\
കണ്ണുകാണാഞ്ഞിതിരുട്ടുകൊണ്ടന്നേര-\\
മന്യോന്യമൊത്തുകൈയും പിടിച്ചാകുലാല്‍\\
ഖിന്നതയോടും നടന്നു നടന്നുപോയ്-\\
ച്ചെന്നാരതീവ ദൂരം തത്ര കണ്ടിതു\\
മുന്നിലാമ്മാറതി ധന്യദേശസ്ഥലം.\\
സ്വര്‍ണമയം മനോമോഹനം കാണ്മവര്‍-\\
കണ്ണിനുമേറ്റ മാനന്ദകരം പരം\\
വാപികളുണ്ടു മണിമയവാരിയാ-\\
ലാപൂര്‍ണകളായതീവ വിശദമായ്\\
പക്വഫലങ്ങളാല്‍ നമ്രങ്ങളായുള്ള\\
വൃക്ഷങ്ങളുണ്ടു കല്പദ്രുമതുല്യമായ്\\
പീയൂഷസാമ്യമധുദ്രോണസംയുത\\
പേയഭക്ഷ്യാന്നസഹിതങ്ങളായുള്ള\\
വസ്ത്യങ്ങളുണ്ടു പലതരം തത്രൈവ\\
വസ്ത്രരത്നാദി പരിഭൂഷിതങ്ങളായ്\\
മാനസമോഹനമായ ദിവ്യസ്ഥലം\\
മാനുഷവര്‍ജിതം ദേവഗേഹോപമം\\
തത്ര ഗേഹേ മണികാഞ്ചനവിഷ്ടരേ\\
ചിത്രാകൃതിപൂണ്ടു കണ്ടാരൊരുത്തിയെ.\\
യോഗം ധരിച്ചു ജടാവല്ക്കലം പൂണ്ടു\\
യോഗിനി നിശ്ചലധ്യാനനിരതയായ്\\
പാവകജ്വാലാസമാഭകലര്‍ത്തതി-\\
പാവനയായ മഹാഭാഗയെക്കണ്ടു.\\
തല്‍ക്ഷണേ സന്തോഷപൂര്‍ണമനസ്സോടു\\
ഭക്തിയും ഭീതിയും പൂണ്ടു വണങ്ങിനാര്‍.\\
ശാഖാമൃഗങ്ങളെക്കണ്ടു മോദം പൂണ്ടു\\
യോഗിനിതാനുമവരോടു ചൊല്ലിനാള്‍:\\
“നിങ്ങളാരാകുന്നതെന്നു പറയണ-\\
മിങ്ങുവന്നീടുവാന്‍ മൂലവും ചൊല്ലണം.\\
എങ്ങനെ മാര്‍ഗമറിഞ്ഞവാറെന്നതു-\\
മെങ്ങിനിപ്പോകുന്നതെന്നും പറയണം.”\\
എന്നിവ കേട്ടൊരു വായുതനയനും\\
നന്നായ് വണങ്ങി വിനീതനായ് ചൊല്ലിനാന്‍:\\
“വൃത്താന്തമൊക്കവേ കേട്ടാലുമെങ്കിലോ\\
സത്യമൊഴിഞ്ഞു പറയുമാറില്ല ഞാന്‍\\
ഉത്തരകോസലത്തിങ്കലയോദ്ധ്യയെ-\\
ന്നുത്തമയായുണ്ടൊരു പുരി ഭൂതലേ\\
തത്രൈവ വാണു ദശരഥനാം നൃപന്‍\\
പുത്രരുമുണ്ടായ് ചമഞ്ഞിതു നാലുപേര്‍.\\
നാരായണസമന്‍ ജ്യേഷ്ഠനവര്‍കളില്‍\\
ശ്രീരാമനാകുന്നതെന്നുമറിഞ്ഞാലും\\
താതാജ്ഞയാ വനവാസാര്‍ത്ഥമായവന്‍\\
ഭ്രാതാവിനോടും ജനകാത്മജയായ\\
സീതയാം പത്നിയോടും വിപിനസ്ഥലേ\\
മോദേന വാഴുന്നകാലമൊരുദിനം\\
ദുഷ്ടനായുള്ള ദശാസ്യനിശാചരന്‍\\
കട്ടുകൊണ്ടാശു പോയീടിനാന്‍ പത്നിയെ.\\
രാമനും ലക്ഷ്മണനാകുമനുജനും\\
ഭാമിനി തന്നെത്തിരഞ്ഞു നടക്കുമ്പോള്‍\\
അര്‍ക്കാത്മജനായ സുഗ്രീവനെക്കണ്ടു\\
സഖ്യവും ചെയ്തിതു തമ്മിലന്യോന്യമായ്\\
എന്നതിന്നഗ്രജനാകിയ ബാലിയെ-\\
ക്കൊന്നു സുഗ്രീവനു രാജ്യവും നല്കിനാന്‍\\
ശ്രീരാമനുമതിന്‍ പ്രത്യുപകാരമാ-\\
യാരാഞ്ഞു സീതയെക്കണ്ടു വരികെന്നു\\
വാനരനായകനായ സുഗ്രീവനും\\
വാനരന്മാരെയയച്ചിതെല്ലാടവും\\
ദക്ഷിണദിക്കിലന്വേഷിപ്പതിനൊരു\\
ലക്ഷം കപിവരന്മാരുണ്ടു ഞങ്ങളും\\
ദാഹം പൊറാഞ്ഞു ജലകാംക്ഷയാ വന്നു\\
മോഹേന ഗഹ്വരം പുക്കിതറിയാതെ.\\
ദൈവവസാലിവിടെപ്പോന്നു വന്നിഹ\\
ദേവിയെക്കാണായതും ഭാഗ്യമെത്രയും.\\
ആരെന്നതും ഞങ്ങളേതുമറിഞ്ഞീല\\
നേരേയരുള്‍ചെയ്കവേണമതും ശുഭേ!”\\
യോഗിനിതാനുമതുകേട്ടവരോടു\\
വേഗേന മന്ദസ്മിതം പൂണ്ടു ചൊല്ലിനാള്‍:\\
“പക്വഫലമൂല ജാലങ്ങളൊക്കവേ\\
ഭക്ഷിച്ചമൃതപാനം ചെയ്തു തൃപ്തരായ്\\
ബുദ്ധിതെളിഞ്ഞു വരുവിനെന്നാല്‍ മമ\\
വൃത്താന്തമാദിയേ ചൊല്ലിത്തരുവന്‍ ഞാന്‍.”\\
എന്നതു കേട്ടവര്‍ മൂലഫലങ്ങളും\\
നന്നായ് ഭുജിച്ചു മധുപാനവും ചെയ്തു\\
ചിത്തം തെളിഞ്ഞു ദേവീസമീപം പുക്കു\\
ബദ്ധാഞ്ജലി പൂണ്ടു നിന്നോരനന്തരം\\
ചാരുസ്മിതപൂര്‍വമഞ്ജസാ യോഗിനി\\
മാരുതിയോടു പറഞ്ഞു തുടങ്ങിനാള്‍:\\
“വിശ്വവിമോഹനമൂപിണിയാകിയ\\
വിശ്വകര്‍മാത്മജാ ഹേമാ മനോഹരീ\\
നൃത്തഭേദംകൊണ്ടു സന്തുഷ്ടനാക്കിനാള്‍\\
മുഗ്ദ്ധേന്ദുശേഖരന്‍ തന്നെ, യതുമൂലം\\
ദിവ്യപുരമിദം നല്കിനാനീശ്വരന്‍\\
ദിവ്യസംവത്സരാണാമയുതായുതം\\
ഉത്സവം പൂണ്ടു വസിച്ചാളിഹ പുരാ\\
തത്സഖി ഞാനിഹ നാമ്നാ സ്വയംപ്രഭാ\\
സന്തതം മോക്ഷമപേക്ഷിച്ചിരിപ്പൊരു\\
ഗന്ധര്‍വപുത്രി സദാ വിഷ്ണുതല്പരാ.\\
ബ്രഹ്മലോകം പ്രവേശിച്ചിതു ഹേമയും\\
നിര്‍മലഗാത്രിയുമെന്നോടു ചൊല്ലിനാള്‍:\\
സന്തതം നീ തപസ്സും ചെയ്തിരിക്കെടോ\\
ജന്തുക്കളത്ര വരികയുമില്ലല്ലോ\\
ത്രേതായുഗേ വിഷ്ണു നാരായണന്‍ ഭുവി\\
ജാതനായീടും ദശരഥപുത്രനായ്\\
ഭൂഭാരനാശനാര്‍ഥം വിപിനസ്ഥലേ\\
ഭുപതി സഞ്ചരിച്ചീടും ദശാന്തരേ\\
ശ്രീരാമപതിനിയെക്കട്ടുകൊള്ളൂമതി-\\
ക്രൂരനായീടും ദശാനനനക്കാലം.\\
ജാനകീദേവിയെയന്വേഷണത്തിനായ്\\
വാനരന്മാര്‍ വരും നിന്‍ ഗുഹാമന്ദിരേ\\
സല്‍ക്കരിച്ചീടവരെ പ്രീതിപൂണ്ടു നീ\\
മര്‍ക്കടാന്മാര്‍ക്കുപകാരവും ചെയ്തുപോയ്\\
ശ്രീരാമദേവനെക്കണ്ടു വണങ്ങുക\\
നാരായണസ്വാമിതന്നെ രഘൂത്തമന്‍\\
ഭക്ത്യാ പരനെ സ്തുതിച്ചാല്‍ വരും തവ\\
മുക്തിപദം യോഗിഗമ്യം സനാതനം\\
ആകയാല്‍ ഞാനിനി ശ്രീരാമദേവനെ\\
വേഗേന കാണ്മതിന്നായ്ക്കൊണ്ടു പോകുന്നു\\
നിങ്ങളെ നേരേ പെരുവഴി കൂട്ടുവന്‍\\
നിങ്ങളെല്ലാവരും കണ്ണടച്ചീടുവിന്‍,”\\
ചിത്തം തെളിഞ്ഞവര്‍ കണ്ണടച്ചീടിനാര്‍\\
സത്വരം പൂര്‍വസ്ഥിതാടവി പുക്കിതു.\\
ചിത്രം വിചിത്രം വിചിത്രമെന്നോര്‍ത്തവര്‍\\
പദ്ധതിയൂടെ നടന്നു തുടങ്ങിനാര്‍.
\end{verse}

%%16_svayamprabhaasthuthi

\section{സ്വയംപ്രഭാസ്തുതി}

\begin{verse}
യോഗിനിയും ഗുഹാവാസമുപേക്ഷിച്ചു\\
യോഗേശസന്നിധി പുക്കാളതിദ്രുതം.\\
ലക്ഷ്മണസുഗ്രീവസേവിതനാകിയ\\
ലക്ഷ്മീശനെക്കണ്ടു കൃത്വാ പ്രദക്ഷിണം\\
ഭക്ത്യാ സഗദ്ഗദം രോമാഞ്ചസംയുതം\\
നത്വാ മുഹുര്‍മുഹുസ്തുത്വാ ബഹുവിധം:\\
“ദാസീ തവാഹം രഘുപതേ! രാജേന്ദ്ര!\\
വാസുദേവ! പ്രഭോ! രാമ! ദഹാനിധേ!\\
കാണ്മതിന്നായ്ക്കൊണ്ടു വന്നേനിവിടെ ഞാന്‍\\
സാമ്യമില്ലാത ജഗല്‍പ്പതേ! ശ്രീപതേ!\\
ഞാനനേകായിരം സംവത്സരം തവ\\
ധ്യാനേന നിത്യം തപസ്സു ചെയ്തീടിനേന്‍\\
ത്വദ്രൂപസന്ദര്‍ശനാര്‍ത്ഥം തപോബല-\\
മദ്യൈവ നൂനം ഫലിതം തപോനിധേ!\\
ആദ്യനായോരു ഭവന്തം നമസ്യാമി\\
വേദ്യനല്ലാരാലുമേ ഭവാന്‍ നിര്‍ണയം.\\
അന്തരാ‍ബഹിഃ സ്ഥിതം സര്‍വഭൂതേഷ്വപി\\
സന്തമലക്ഷ്യമാദ്യന്തഹീനം പരം\\
മായായവൈകാച്ഛന്നനായ് വാഴുന്ന\\
മായാമയനായ മാനുഷവിഗ്രഹന്‍\\
അജ്ഞാനികളാലറിഞ്ഞുകൂടാതൊരു\\
വിജ്ഞാനമൂര്‍ത്തിയല്ലോ ഭവാന്‍ കേവലം.\\
ഭാഗവതന്മാര്‍ക്കു ഭക്തിയോഗാര്‍ത്ഥമായ്\\
ലോകേശമുഖ്യാമരൗഘമര്‍ത്ഥിക്കയാല്‍\\
ഭൂമിയില്‍ വന്നവതീര്‍ണനാം നാഥനെ-\\
ത്താമസിയായ ഞാനെന്തറിയുന്നതും!\\
സച്ചിന്മയം തവ തത്ത്വം ജഗത്രയേ\\
കശ്ചില്‍ പുരുഷനറിയും സുകൃതിനാം\\
രൂപം തവേദം സദാ ഭാതു മാനസേ\\
താപസാന്തഃ സ്ഥിതം താപത്രയാപഹം\\
നാരായണ! തവ ശ്രീ പാദദര്‍ശനം\\
ശ്രീരാമ! മോക്ഷൈകദര്‍ശനം കേവലം.\\
ജന്മമരണഭീതാനാമദര്‍ശനം\\
സന്മാര്‍ഗദര്‍ശനം വേദാന്തദര്‍ശനം\\
പുത്രകളത്രമിത്രാര്‍ത്ഥവിഭൂതികൊ-\\
ണ്ടെത്രയും ദര്‍പ്പിതരായുള്ള മാനുഷര്‍\\
രാമരാമേതി ജപിക്കയില്ലെന്നുമേ\\
രാമനാമം മേ ജപിക്കായ്വരേണമേ!\\
നിത്യം നിവൃത്തഗുണത്രയമാര്‍ഗായ\\
നിത്യായ നിഷ്കിഞ്ചനാര്‍ത്ഥായ തേ നമഃ\\
സ്വാത്മാഭിരാമായ നിര്‍ഗുണായ ത്രിഗു-\\
ണാത്മനേ സീതാഭിരാമായ തേ നമഃ\\
വേദാത്മകം കാമരൂപിണമീശാന-\\
മാദിമദ്ധ്യാന്തവിവര്‍ജിതം സര്‍വത്ര\\
മന്യേ സമം ചരന്തം പൂരുഷം പരം\\
നിന്നെ നിനക്കൊഴിഞ്ഞാര്‍ക്കറിഞ്ഞീടാവൂ?\\
മര്‍ത്ത്യവിഡംബനം ദേവ! തേ ചേഷ്ടിതം\\
ചിത്തേ നിരൂപിക്കിലെന്തറിയാവതും?\\
ത്വന്മായയാ പിഹിതാത്മാക്കള്‍ കാണുന്നു\\
ചിന്മയനായ ഭവാനെബ്ബഹുവിധം.\\
ജന്മവും കര്‍ത്തൃത്വവും ചെറുതില്ലാത\\
നിര്‍മലാത്മാവാം ഭവാനവസ്ഥാന്തരേ\\
ദേവതിര്യങ്മനുജാദികളില്‍ ജനി-\\
ച്ചേവമാദ്യങ്ങളാം കര്‍മങ്ങള്‍ ചെയ്വതും\\
നിന്മഹാമായാവിഡംബനം നിര്‍ണയം\\
കല്മഷഹീന! കരുണാനിധേ! വിഭോ!\\
മേദിനിതന്നില്‍ വിചിത്രവേഷത്തൊടും\\
ജാതനായ് കര്‍മങ്ങള്‍ ചെയ്യുന്നതും ഭവാന്‍\\
ഭക്തരായുള്ള ജനങ്ങള്‍ക്കു നിത്യവും\\
ത്വല്‍ക്കഥാപീയൂഷപാനസിദ്ധിക്കെന്നു\\
ചൊല്ലുന്നിതു ചിലര്‍; മറ്റുചിലരിഹ\\
ചൊല്ലുന്നിതു ഭുവി കോസലഭൂപതി-\\
തന്നുടെ ഘോരതപോബലസിദ്ധയേ\\
നിര്‍ണയമെന്നു; ചിലര്‍ പറയുന്നിതു\\
കൗസല്യയാല്‍ പ്രാര്‍ത്ഥ്യമാനനായിട്ടിഹ;\\
മൈഥിലീഭാഗ്യസിദ്ധിക്കെന്നതു ചിലര്‍;\\
സ്രഷ്ടാവുതാനപേക്ഷിക്കയാല്‍ വന്നിഹ\\
ദുഷ്ടനിശാചരവംശമൊടുക്കുവാന്‍\\
മര്‍ത്ത്യനായ് വന്നു പിറന്നിതു നിര്‍ണയം\\
പൃത്ഥ്വീയിലെന്നു ചിലര്‍ പറയുന്നിതു\\
ഭൂപാലപുത്രനായ് വന്നു പിറന്നിതു\\
ഭൂഭാരനാശനത്തിന്നെന്നിതു ചിലര്‍;\\
ധര്‍മത്തെ രക്ഷിച്ചധര്‍മത്തെ നീക്കുവാന്‍\\
കര്‍മസാക്ഷീകുലത്തിങ്കല്‍ പിറന്നിതു\\
ദേവശത്രുക്കളെ നിഗ്രഹിച്ചന്‍പോടു\\
ദേവകളെപ്പരിപാലിച്ചു കൊള്ളുവാന്‍\\
എന്നു ചൊല്ലുന്നിതു ദിവ്യമുനിജന-\\
മൊന്നും തിരിച്ചറിയാവതുമല്ല മേ.\\
യാതൊരുത്തന്‍ ത്വല്‍ക്കഥകള്‍ ചൊല്ലുന്നതു-\\
മാദരവോടു കേള്‍ക്കുന്നതും നിത്യമായ്\\
നൂനം ഭവാര്‍ണവത്തെക്കടന്നീടുവോന്‍\\
കാണാമവനു നിന്‍ പാദപങ്കേരുഹം.\\
ത്വന്മഹാമായാഗുണബദ്ധയാകയാല്‍\\
ചിന്മയമായ ഭവത്സ്വരൂപത്തെ ഞാന്‍\\
എങ്ങനെയുള്ളവണ്ണമറിഞ്ഞീടുന്ന-\\
തെങ്ങനെ ചൊല്ലി സ്തുതിക്കുന്നതുമഹം.\\
ശ്യാമളം കോമളം ബാണധനുര്‍ദ്ധരം\\
രാമം സഹോദരസേവിതം രാഘവം\\
സുഗ്രീവമുഖ്യകപികുലസേവിത-\\
മഗ്രേ ഭവന്തം നമസ്യാമി സാമ്പ്രതം.\\
രാമായ രാമഭദ്രായ നമോനമോ\\
രാമചന്ദ്രായ നമസ്തേ നമോനമഃ\\
ഇങ്ങനെചൊല്ലി സ്വയംപ്രഭയും വീണു\\
മംഗലവാചാ നമസ്കരിച്ചീടിനാള്‍.\\
മുക്തിപ്രദനായ രാമന്‍ പ്രസന്നനായ്\\
ഭക്തയാം യോഗിനിയോടരുളിച്ചെയ്തു:\\
‘സന്തുഷ്ടനായേനഹം തവ ഭക്തികൊ-\\
ണ്ടെന്തോന്നു മാനസേ കാംക്ഷിതം ചൊല്ലുനീ.’\\
എന്നതു കേട്ടവളും പറഞ്ഞീടിനാള്‍:\\
‘ഇന്നുവന്നു മമ കാംക്ഷിതമൊക്കവേ.\\
യത്രകുത്രാപി വസിക്കിലും ത്വല്‍പാദ-\\
ഭക്തിക്കിളക്കമുണ്ടാകാതിരിക്കണം.\\
ത്വല്‍പാദഭക്തഭൃത്യേഷു സംഗം പുന-\\
രുള്‍പ്പൂവിലെപ്പോഴുമുണ്ടാകയും വേണം.\\
പ്രാകൃതന്മാരാം ജനങ്ങളില്‍ സംഗമ-\\
മേകദാ സംഭവിച്ചീടായ്ക മാനസേ.\\
രാമരാമേതി ജപിക്കായ് വരേണമേ\\
രാമപാദേ രമിക്കേണമെന്മാനസം.\\
സീതാ സുമിത്രാത്മജാന്വിതം രാഘവം\\
പീതവസ്ത്രം ചാപബാണാസനധരം\\
ചാരുമകുടകടകകടിസൂത്ര-\\
ഹാരമകരമണിമയകുണ്ഡല-\\
നൂപുര ഹേമാംഗദാദി വിഭൂഷണ-\\
ശോഭിതരൂപം വസിക്ക മേ മാനസേ.\\
മറ്റെനിക്കേതുമേ വേണ്ടാ വരം വിഭോ!\\
പറ്റായ്ക ദുസ്സംഗമുള്ളിലൊരിക്കലും.”\\
ശ്രീരാമദേവനതു കേട്ടവളോടു\\
ചാരുമന്ദസ്മിതം പൂണ്ടരുളിച്ചെയ്തു:\\
“ഏവം ഭവിക്ക നിനക്കു മഹാഭാഗേ!\\
ദേവി! നീ പോക ബദര്യാശ്രമസ്ഥലേ\\
തത്രൈവ നിത്യമെന്നെ ധ്യാനവും ചെയ്തു\\
മുക്ത്വാ കളേബരം പഞ്ചഭൂതാത്മകം\\
ചേരുമെങ്കല്‍ പരമാത്മനി കേവലേ\\
തീരും ജനനമരണദുഃഖങ്ങളും.”\\
ശ്രുത്വാ രഘൂത്തമവാക്യാമൃതം മുദാ\\
ഗത്വാ തദൈവ ബദര്യാശ്രമസ്ഥലേ\\
ശ്രീരാമദേവനെ ധ്യാനിച്ചിരുന്നുടന്‍\\
നാരായണപദം പ്രാപിച്ചിതവ്യയം.
\end{verse}

%%17_angadaadikaludesamshayam

\section{അംഗദാദികളുടെ സംശയം}

\begin{verse}
മര്‍ക്കടസഞ്ചയം ദേവിയെയാരാഞ്ഞു\\
വൃക്ഷഷണ്ഡേഷു വസിക്കും ദശാന്തരേ\\
എത്ര ദിവസം കഴിഞ്ഞിതെന്നും ധരാ-\\
പുത്രിയെയെങ്ങുമേ കണ്ടുകിട്ടായ്കയും\\
ചിന്തിച്ചു ഖേദിച്ചു താരാസുതന്‍ നിജ-\\
ബന്ധുക്കളായുള്ളവരോടു ചൊല്ലിനാന്‍:\\
“പാതാളമുള്‍പ്പുക്കുഴന്നു നടന്നു നാ-\\
മേതുമറിഞ്ഞീല വാസരം പോയതും\\
മാസമതീതമായ് വന്നിതു നിര്‍ണയം\\
ഭൂസുതയെ കണ്ടറിഞ്ഞതുമില്ല നാം\\
രാജനിയോഗമനുഷ്ഠിയാതെ വൃഥാ\\
രാജധാനിക്കു നാം ചെല്ലുകിലന്നുതാന്‍\\
നിഗ്രഹിച്ചീടുമതിനില്ല സംശയം\\
സുഗ്രീവസാസനം നിഷ്ഫലമായ് വരാ.\\
പിന്നെ വിശേഷിച്ചു ശത്രുതനയനാ-\\
മെന്നെ വധിക്കുമതിനില്ലൊരന്തരം\\
എന്നിലവന്നൊരു സമ്മതമെന്തുള്ള\\
തെന്നെ രക്ഷിച്ചതു രാമന്‍ തിരുവടി\\
രാമകാര്യത്തെയും സാധിയാതെ ചെല്കില്‍\\
മാമകജീവനം രക്ഷിക്കയില്ലവന്‍.\\
മാതാവിനോടു സമാനയാകും നിജ\\
ഭ്രാതാവുതന്നുടെ ഭാര്യയെ നിസ്ത്രപം\\
പ്രാപിച്ചുവാഴുന്ന വാനരപുംഗവന്‍\\
പാപി, ദുരാത്മാവിനെന്തരുതാത്തതും?\\
തല്‍പാര്‍ശ്വദേശേ ഗമിക്കുന്നതില്ല ഞാ-\\
നിപ്പോളിവിടെ മരിക്കുന്നതേയുള്ളൂ.\\
“വല്ല പ്രകാരവും നിങ്ങള്‍ പൊയ്ക്കൊള്‍കെ“ന്നു\\
ചൊല്ലിക്കരയുന്ന നേരം കപികളും\\
തുല്യദുഃഖേന ബാഷ്പം തുടച്ചന്‍പോടു\\
ചൊല്ലിനാര്‍ മിത്രഭാവത്തോടു സത്വരം:\\
ദുഃഖിക്കരുതൊരുജാതിയുമിങ്ങനെ\\
രക്ഷിപ്പതിനുണ്ടു ഞങ്ങളറിക നീ.\\
ഇന്നു നാം പോന്ന ഗുഹയിലകംപുക്കു\\
നന്നായ് സുഖിച്ചു വസിക്കാം വയം ചിരം\\
സര്‍വസൗഭാഗ്യസമന്വിതമായൊരു\\
ദിവ്യപുരമതു ദേവലോകോപമം\\
ആരാലുമില്ലൊരുനാളും ഭയം സഖേ!\\
താരേയ! പോക നാം വൈകരുതേതുമേ.”\\
അംഗദന്‍ തന്നോടിവണ്ണം കപികുല-\\
പുംഗവന്മാര്‍ പറയുന്നതു കേള്‍ക്കയാല്‍\\
ഇംഗിതജ്ഞന്‍ നയകോവിദന്‍ വാതജ-\\
നംഗദനെത്തഴുകിപ്പറഞ്ഞീടിനാന്‍:\\
“എന്തൊരു ദുര്‍വിചാരം? യോഗ്യമല്ലിദ-\\
മന്ധകാരങ്ങള്‍ നിനയായ്വിനാരുമേ.\\
ശ്രീരാമനേറ്റം പ്രിയന്‍ ഭവാനെന്നുടെ\\
താരാസുതനെന്നു തന്മാനസേ സദാ\\
പാരം വളര്‍ന്നൊരു വാത്സല്യമുണ്ടതു\\
നേരേ ധരിച്ചീല ഞാനൊഴിഞ്ഞാരുമേ.\\
സൗമിത്രിയെക്കാളതിപ്രിയന്‍ നീ തവ\\
സാമര്‍ഥ്യവും തിരുവുള്ളത്തിലുണ്ടെടോ!\\
പ്രേമത്തിനേതുമിളക്കമുണ്ടായ്വരാ\\
ഹേമത്തിനുണ്ടോ നിറക്കേടകപ്പെടൂ?\\
ആകയാല്‍ ഭീതി ഭവാനൊരു നാളുമേ\\
രാഘവന്‍ പക്കല്‍നിന്നുണ്ടായ്വരാ സഖേ!\\
ശാഖാമൃഗധിപനായ സുഗ്രീവനും\\
ഭാഗവതോത്തമന്‍ വൈരമില്ലാരിലും\\
വ്യാകുലമുള്ളിലുണ്ടാകരുതേതുമേ\\
നാകാധിപാത്മജനന്ദന! കേളിദം.\\
ഞാനും തവ ഹിതത്തിങ്കല്‍ പ്രസക്തന-\\
ജ്ഞാനികള്‍ വാക്കുകേട്ടേതും ഭ്രമിക്കൊലാ\\
ഹാനി വരായ്വാന്‍ ഗുഹയില്‍ വസിക്കെന്നു\\
വാനരൗഘം പറഞ്ഞീലയോ ചൊല്ലുനീ\\
രാഘവാസ്ത്രത്തിന്നഭേദ്യമായൊന്നുമേ\\
ലോകത്രയത്തിങ്കലില്ലെന്നറിക നീ\\
അല്പമതികള്‍ പറഞ്ഞുബോധിപ്പിച്ചു\\
ദുര്‍ബോധമുണ്ടായ് ചമയരുതേതുമേ.\\
ആപത്തുവന്നടുത്തീടുന്ന കാലത്തു\\
ശോഭിക്കയില്ലെടോ സജ്ജനഭാഷിതം.\\
ദുര്‍ജനത്തെക്കുറിച്ചുള്ള വിശ്വാസവും\\
സജ്ജനത്തോടു വിപരീതഭാവവും\\
ദേവദ്വിജകുല ധര്‍മ വിദ്വേഷവും\\
പൂര്‍വബന്ധുക്കളില്‍ വാച്ചൊരു വൈരവും\\
വര്‍ദ്ധിച്ചുവര്‍ദ്ധിച്ചു വംശനാശത്തിനു\\
കര്‍ത്തൃത്വവും തനിക്കായ് വന്നുകൂടുമേ.\\
അത്യന്തഗുഹ്യം രഹസ്യമായുള്ളൊരു\\
വൃത്താന്തമമ്പോടു ചൊല്ലുവന്‍ കേള്‍ക്ക നീ\\
ശ്രീരാമദേവന്‍ മനുഷ്യനല്ലോര്‍ക്കെടോ!\\
നാരായണന്‍ പരമാത്മാ ജഗന്മയന്‍\\
മായാഭഗവതി സാക്ഷാല്‍ മഹാവിഷ്ണു-\\
ജായാ സകലജഗന്മോഹകാരിണി\\
സീതയാകുന്നതു, ലക്ഷ്മണനും ജഗ-\\
ദാധാരഭൂതനായുള്ള ഫണീശ്വരന്‍\\
ശേഷന്‍, ജഗത്സ്വരൂപന്‍ ഭുവി മാനുഷ-\\
വേഷമായ് വന്നു പിറന്നിതയോദ്ധ്യയില്‍\\
രക്ഷോഗണത്തെയൊടുക്കി ജഗത്ത്രയ-\\
രക്ഷവരുത്തുവാന്‍ പണ്ടു വിരിഞ്ചനാല്‍\\
പ്രാര്‍ത്ഥിതനാകയാല്‍ പാര്‍ത്ഥിവപുത്രനായ്\\
മാര്‍ത്താണ്ഡഗോത്രത്തിലാര്‍ത്ത പരായണന്‍\\
ശ്രീകണ്ഠസേവ്യന്‍ ജനാര്‍ദനന്‍ മാധവന്‍\\
വൈകുണ്ഠവാസി മുകുന്ദന്‍ ദയാപരന്‍\\
മര്‍ത്ത്യനായ് വന്നിങ്ങവതരിച്ചീടിനാന്‍\\
ഭൃത്യവര്‍ഗം നാം പരിചരിച്ചീടുവാന്‍\\
ഭര്‍ത്തൃനിയോഗേന വാനരവേഷമായ്\\
പൃത്ഥ്വിയില്‍വന്നു പിറന്നിരിക്കുന്നതും.\\
പണ്ടു നാമേറ്റം തപസ്സുചെയ്തീശനെ-\\
ക്കണ്ടു വണങ്ങി പ്രസാദിച്ചു മാധവന്‍\\
തന്നുടെ പാരിഷദന്മാരുടെ പദം\\
തന്നതിപ്പോഴും പരിചരിച്ചിന്നിയും\\
വൈകുണ്ഠലോകം ഗമിച്ചു വാണീടുവാന്‍\\
വൈകേണ്ടതേതുമില്ലെന്നറിഞ്ഞീടു നീ.”\\
അംഗദനോടിവണ്ണം പവനാത്മജന്‍\\
മംഗലവാക്കുകള്‍ ചൊല്ലിപ്പലതരം\\
ആശ്വസിപ്പിച്ചുടന്‍ വിന്ധ്യാചലം പുക്കു\\
കാശ്യപീപുത്രിയെ നോക്കിനോക്കിദ്രുതം\\
ദക്ഷിണവാരിധിതീരം മനോഹരം\\
പുക്കുമഹേന്ദ്രാചലേന്ദ്രപദം മുദാ.\\
ദുസ്തരമേറ്റമഗാധം ഭയങ്കരം\\
ദുഷ്പ്രാപമാലോക്യ മര്‍ക്കടസഞ്ചയം\\
വൃത്രാരിപുത്രാത്മജാദികളൊക്കെയും\\
ത്രസ്തരായത്യാകുലം പൂണ്ടിരുന്നുടന്‍\\
ചിന്തിച്ചു ചിന്തിച്ചു മന്ത്രിച്ചിതന്യോന്യ-\\
“മെന്തിനിച്ചെയ്വതു സന്തതമോര്‍ക്ക നാം\\
ഗഹ്വരം പുക്കു പരിഭ്രമിച്ചെത്രയും\\
വിഹ്വലന്മാരായ്, കഴിഞ്ഞിതു മാസവും,\\
തണ്ടാരില്‍മാതിനെ കണ്ടീല നാം ദശ-\\
കണ്ഠനേയും കണ്ടുകിട്ടീല കുത്രചില്‍\\
സുഗ്രീവനും തീക്ഷ്ണദണ്ഡനത്രേ തുലോം\\
നിഗ്രഹിച്ചീടുവമന്‍ നമ്മെ നിര്‍ണയം\\
ക്രുദ്ധനായുള്ള സുഗ്രീവന്‍ വധിക്കയില്‍\\
നിത്യോപവാസേന മൃത്യുഭവിപ്പതു\\
മുക്തിക്കു നല്ലൂ നമുക്കു പാര്‍ത്തോള’മെ-\\
ന്നിത്ഥം നിരൂപിച്ചുറച്ചു കപികുലം\\
ദര്‍ഭവിരിച്ചു കിടന്നിതെല്ലാവരും\\
കല്പിച്ചതിങ്ങനെ നമ്മെയെന്നോര്‍ത്തവര്‍.
\end{verse}

%%18_sampathivaakyam

\section{സമ്പാതിവാക്യം}

\begin{verse}
അപ്പോള്‍ മഹേന്ദ്രാചലേന്ദ്രഗുഹാന്തരാല്‍\\
ഗൃദ്ധ്രം പുറത്തു പതുക്കെപ്പുറപ്പെട്ടു\\
വൃദ്ധനായുള്ളോരു ഗൃദ്ധ്രപ്രവരനും\\
പൃത്ഥ്വീധരപ്രവരോത്തുംഗ രൂപനായ്\\
ദൃഷ്ട്വാ പരക്കെക്കിടക്കും കപികളെ\\
തുഷ്ട്യാ പറഞ്ഞിതു ഗൃദ്ധ്രകുലാധിപന്‍.\\
“പക്ഷമില്ലാതോരെനിക്കു ദൈവം ബഹു-\\
ഭക്ഷണം തന്നതു ഭാഗ്യമല്ലോ ബലാല്‍\\
മുമ്പില്‍ മുമ്പില്‍ പ്രാണഹാനി വരുന്നതു\\
സമ്പ്രീതിപൂണ്ടു ഭക്ഷിക്കാമനുദിനം.”\\
ഗൃദ്ധ്രവാക്യം കേട്ടു മര്‍ക്കടൗഘം പരി-\\
ത്രസ്തരായന്യോന്യമാശു ചൊല്ലീടിനാര്‍:\\
അദ്രീന്ദ്രതുല്യനായോരു ഗൃദ്ധ്രാധിപന്‍\\
സത്വരം കൊത്തി വിഴുങ്ങുമെല്ലാരെയും\\
നിഷ്ഫലം നാം മരിച്ചീടുമാറായിതു\\
കല്പിതമാര്‍ക്കും തടുക്കരുതേതുമേ.\\
നമ്മാലൊരു കാര്യവും കൃതമായീല\\
കര്‍മദോഷങ്ങള്‍ പറയാവതെന്തഹോ!\\
രാമകാര്യത്തെയും സാധിച്ചതില്ല നാം\\
സ്വാമിയുടെ ഹിതവും വന്നതില്ലല്ലോ.\\
വ്യര്‍ത്ഥമിവനാല്‍ മരിക്കെന്നു വന്നതുജ്-\\
മെത്രയും പാപികളാകതന്നേ വയം.\\
നിര്‍മലനായ ധര്‍മാത്മാ ജടായുതന്‍\\
നന്മയോര്‍ത്തോളം പറയാവതല്ലല്ലോ\\
വര്‍ണിപ്പതിന്നു പണിയുണ്ടവനുടെ\\
പുണ്യമോര്‍ത്താല്‍ മറ്റൊരുത്തര്‍ക്കു കിട്ടുമോ?\\
ശ്രീരാമകാര്യാര്‍ഥമാശുമരിച്ചവന്‍\\
ചേരുമാറായിതു രാമപാദാംബുജേ.\\
പക്ഷിയെന്നാകിലും മോക്ഷം ലഭിച്ചിതു\\
പക്ഷീന്ദ്രവാഹനാനുഗ്രഹം വിസ്മയം.”\\
വാനരഭാഷിതം കേട്ടു സമ്പാതിയും\\
മാനസാനന്ദം കലര്‍ന്നു ചോദിച്ചിതു:\\
“കര്‍ണപീയൂഷസമാനമാം വാക്കുകള്‍\\
ചൊന്നതാരിന്നു ജടായുവെന്നിങ്ങനെ?\\
നിങ്ങളാരെന്തു പറയുന്നിതന്യോന്യ-\\
മിങ്ങുവരുവിന്‍ ഭയപ്പെടായ്കേതുമേ.”\\
ഉമ്പര്‍കോന്‍ പൗത്രനുമന്‍പോടതു കേട്ടു\\
സമ്പാതിതന്നുടെ മുമ്പിലാമ്മാറു ചെ-\\
ന്നംഭോജലോചനന്‍തന്‍ പാദപങ്കജം\\
സംഭാവ്യ സമ്മോദമുള്‍ക്കൊണ്ടു ചൊല്ലിനാന്‍:\\
“സൂര്യകുലജാതനായ ദശരഥ-\\
നാര്യപുതന്‍ മഹാവിഷ്ണു നാരായണന്‍\\
പുഷ്കരനേത്രനാം രാമന്‍ തിരുവടി\\
ലക്ഷ്മണനായ സഹോദരനും നിജ-\\
ലക്ഷ്മിയാം ജാനകിയോടും തപസ്സിനായ്\\
പുക്കിതു കാനനം താതാജ്ഞയാ പുരാ.\\
കട്ടുകൊണ്ടീടിനാന്‍ തല്‍കാലമെത്രയും\\
ദുഷ്ടനായുള്ള ദശമുഖന്‍ സീതയെ\\
ലക്ഷ്മണനും കമലേക്ഷണനും പിരി-\\
ഞ്ഞക്ഷോണിപുത്രി മുറയിട്ടതു കേട്ടു\\
തല്‍ക്ഷണം ചെന്നു തടഞ്ഞു യുദ്ധം ചെയ്താ-\\
നക്ഷണദാചരനോടു ജടായുവാം\\
പക്ഷിപ്രവരനതിനാല്‍ വലഞ്ഞൊരു\\
രക്ഷോവരന്‍ നിജചന്ദ്രഹാസം കൊണ്ടു\\
പക്ഷവും വെട്ടിയറുത്താനതു നേരം\\
പക്ഷീന്ദ്രനും പതിച്ചാന്‍ ധരണീതലേ.\\
ഭര്‍ത്താവിനെക്കണ്ടു വൃത്താന്തമൊക്കവേ\\
സത്യം പറഞ്ഞൊഴിഞ്ഞെന്നുമേ നിന്നുടെ\\
മൃത്യുവരായെന്നനുഗ്രഹിച്ചാള്‍ ധരാ\\
പുത്രിയും, തല്‍പ്രസാദേന പക്ഷീന്ദ്രനും\\
രാമനെക്കണ്ടു വൃത്താന്തമറിയിച്ചു\\
രാമസായുജ്യം ലഭിച്ചിതു ഭാഗ്യവാന്‍.\\
അര്‍ക്കകുലോത്ഭവനാകിയ രാമനു-\\
മര്‍ക്കജനോടഗ്നിസാക്ഷികമാം വണ്ണം\\
സഖ്യവും ചെയ്തുടന്‍ കൊന്നിതു ബാലിയെ\\
സുഗ്രീവനായ്ക്കൊണ്ടു രാജ്യവും നലികിനാന്‍.\\
വാനരധീശ്വരനായ സുഗ്രീവനൂം\\
ജാനകിയെത്തിരഞ്ഞാശു കണ്ടീടുവാന്‍\\
ദിക്കുകള്‍ നാലിലും പോകെന്നയച്ചിതു\\
ലക്ഷം കപിവരന്മാരെയോരോ ദിശി.\\
ദക്ഷിണദിക്കിനു പോന്നിതു ഞങ്ങളും\\
രക്ഷോവരനെയും കണ്ടതില്ലെങ്ങുമേ.\\
മുപ്പതുനാളിനകത്തു ചെന്നീടായ്കി-\\
ലപ്പോളവരെ വധിക്കും കപിവരന്‍.\\
പാതാളമുള്‍പ്പുക്കു വാസരം പോയതു-\\
മേതുമറിഞ്ഞീല ഞങ്ങളതുകൊണ്ടു\\
ദര്‍ഭവിരിച്ചു കിടന്നു മരിപ്പതി-\\
ന്നപ്പോള്‍ ഭവാനെയും കണ്ടുകിട്ടീ ബലാല്‍\\
ഏതാനുമുണ്ടറിഞ്ഞിട്ടു നീയെങ്കിലോ\\
സീതാവിശേഷം പറഞ്ഞുതരേണമേ\\
ഞങ്ങളുടേ പരമാര്‍ത്ഥവൃത്താന്തങ്ങ-\\
ളിങ്ങനെയുള്ളോന്നു നീയറിഞ്ഞീടെടോ!”\\
താരേയവാക്കുകള്‍ കേട്ടു സമ്പാതിയു-\\
മാരൂഢമോദമവനോടു ചൊല്ലിനാന്‍:\\
“ഇഷ്ടനാം ഭ്രാതാവെനിക്കു ജടായു ഞ-\\
നൊട്ടുനാളുണ്ടവനോടു പിരിഞ്ഞതും\\
ഇന്നനേകായിരം വത്സരം കൂടി ഞാ-\\
നെന്നുടെ സോദരന്‍വാര്‍ത്ത കേട്ടീടിനേന്‍.\\
എന്നുടെ സോദരനായുദകക്രിയ-\\
യ്ക്കെന്നെയെടുത്തു ജലാന്തികേ കൊണ്ടുപോയ്\\
നിങ്ങള്‍ ചെയ്യിപ്പിനുദകകര്‍മാദികള്‍\\
നിങ്ങള്‍ക്കു വാക്സഹായം ചെയ്വനാശു ഞാന്‍.”\\
അപ്പോളവനെയെടുത്തു; കപികളു-\\
മബ്ധിതീരത്തുവെച്ചീടിനാരാദരാല്‍.\\
തത്സലിലേ കുളിച്ചഞ്ജലിയും നല്കി\\
വത്സനാം ഭ്രാതാവിനായ്ക്കൊണ്ടു സാദരം\\
സ്വസ്ഥനദേശത്തിരുത്തിനാര്‍ പിന്നെയു-\\
മുത്തമന്മാരായ വാനരസഞ്ചയം.\\
സ്വസ്ഥനായ് സമ്പാതിജാനകി തന്നുടെ\\
വൃത്താന്തമാശു പറഞ്ഞു തുടങ്ങിനാന്‍:\\
“തുംഗമായീടും ത്രികൂടാചലോപരി\\
ലങ്കാപുരിയുണ്ടു മദ്ധ്യേ സമുദ്രമായ്\\
തത്ര മഹാശോകകാനനേ ജാനകി\\
നക്തഞ്ചരീജനമദ്ധ്യേ വസിക്കുന്നു.\\
ദൂരമൊരുനൂറുയോജനയുണ്ടതു\\
നേരേ നമുക്കു കാണം ഗൃദ്ധ്രനാകയാല്‍.\\
സാമര്‍ത്ഥ്യമാര്‍ക്കതു ലംഘിപ്പതിന്നവന്‍\\
ഭൂമീതനൂജയെക്കണ്ടുവരും ധ്രുവം.\\
“സോദരനെക്കൊന്ന ദുഷ്ടനെക്കൊല്ലണ-\\
മേതൊരുജാതിയും പക്ഷവുമില്ല മേ.\\
യത്നേന നിങ്ങള്‍ കടക്കണമാശുപോയ്\\
രത്നാകരം, പിന്നെ വന്നു രഘൂത്തമന്‍\\
രാവണന്‍ തന്നെയും നിഗ്രഹിക്കും ക്ഷണാ-\\
ലേവമിതിന്നു വഴിയെന്നു നിര്‍ണയം.’\\
“രത്നാകരം ശതയോജനവിസ്തൃതം\\
യത്നേന ചാടിക്കടന്നു ലങ്കാപുരം\\
പുക്കു വൈദേഹിയെക്കണ്ടു പറഞ്ഞുട-\\
നിക്കരെച്ചാടിക്കടന്നു വരുന്നതും\\
തമ്മില്‍ നിരൂപിക്ക നാ“,മെന്നൊരുമിച്ചു\\
തമ്മിലന്യോന്യം പറഞ്ഞു തുടങ്ങിനാര്‍.\\
സമ്പാതിതന്നുടെ പൂര്‍വവൃത്താന്തങ്ങ-\\
ളമ്പോടു വാനരന്മാരോടു ചൊല്ലിനാന്‍:\\
“ഞാനും ജടായുവാം ഭ്രാതാവുമായ് പുരാ\\
മാനേന ദര്‍പ്പിതമാനസന്മാരുമായ്\\
വേഗബലങ്ങള്‍ പരീക്ഷിപ്പതിന്നതി-\\
വേഗം പറന്നിതു മേല്പോട്ടു ഞങ്ങളും.\\
മാര്‍ത്താണ്ഡമണ്ഡലപര്യന്തമുല്‍പ്പതി-\\
ച്ചാര്‍ത്തരായ് വന്നു ദിനകരരശ്മിയാല്‍.\\
തല്‍ക്ഷണേ തീയും പിടിച്ചിതനുജനു\\
പക്ഷപുടങ്ങളി, ലപ്പോളവനെ ഞാന്‍\\
രക്ഷിപ്പതിനുടന്‍ പിന്നിലാക്കീടിനേന്‍\\
പക്ഷം കരിഞ്ഞു ഞാന്‍ വീണിതു ഭൂമിയില്‍.\\
പക്ഷദ്വയത്തോടു വീണാനനുജനും\\
പക്ഷികള്‍ക്കാശ്രയം പക്ഷമല്ലോ നിജം.\\
വിന്ധ്യാചലേന്ദ്രശിരസി വീണീടിനേ-\\
നന്ധനായ് മൂന്നുദിനം കിടന്നീടിനേന്‍\\
പ്രാണശേഷത്താലുണര്‍ന്നോരു നേരത്തു\\
കാണായിതു ചിറകും കരിഞ്ഞിങ്ങനെ.\\
ദിഗ്ഭ്രമം പൂണ്ടു ദേശങ്ങളറിയാഞ്ഞു\\
വിഭ്രാന്തമാനസനായുഴന്നങ്ങനെ\\
ചെന്നേന്‍ നിശാകരതാപസന്‍ തന്നുടെ\\
പുണ്യാശ്രമത്തിന്നു പൂര്‍ണഭാഗ്യോദയാല്‍\\
കണ്ടു മഹാമുനി ചൊല്ലിനാനെന്നോടു\\
പണ്ടു കണ്ടുള്ളോരറിവുനിമിത്തമായ്:\\
‘എന്തു സമ്പാതേ! വിരൂപനായ് വന്നതി-\\
നെന്തുമൂലമിതാരാലകപ്പെട്ടതും?\\
എത്രയും ശക്തനായോരു നിനക്കിന്നു\\
ദഗ്ദ്ധമാവാനെന്തു പക്ഷം, പറക നീ.’\\
എന്നതു കേട്ടു ഞാനെന്നുടെ വൃത്താന്ത-\\
മൊന്നൊഴിയാതെ മുനിയോടു ചൊല്ലിനേന്‍.\\
പിന്നെയും കൂപ്പിത്തൊഴുതു ചോദിച്ചിതു:\\
‘സന്നമായ് വന്നു ചിറകും ദയാനിധേ!\\
ജീവനത്തെദ്ധരിക്കേണ്ടുമുപായമി-\\
ന്നേവമെന്നെന്നോടു ചൊല്ലിത്തരേണമേ!’\\
എന്നതു കേട്ടു ചിരിച്ചു മഹാമുനി\\
പിന്നെദ്ദയാവശനായരുളിച്ചെയ്തു:\\
‘സത്യമായുള്ളതു ചൊല്ലുന്നതുണ്ടു ഞാന്‍\\
കൃത്യം നിനക്കൊത്തവണ്ണം കുരുഷ്വ നീ\\
ദേഹം നിമിത്തമീ ദുഃഖമറിക നീ\\
ദേഹമോര്‍ക്കില്‍ കര്‍മസംഭവം നിര്‍ണയം\\
ദേഹത്തിലുള്ളോരഹംബുദ്ധികൈക്കൊണ്ടു\\
മോഹാദഹംകൃതി കര്‍മങ്ങള്‍ ചെയ്യുന്നു.\\
മിഥ്യയായുള്ളോരനിദ്യാസമുത്ഭവ-\\
വസ്തുവായുള്ളോന്നഹങ്കാരമോര്‍ക്ക നീ\\
ചിച്ഛായയോടു സംയുക്തമായ് വര്‍ത്തതേ\\
തപ്തമായുള്ളോരയഃ പിണ്ഡവല്‍ സദാ\\
തേന ദേഹത്തിന്നു താദാത്മ്യയോഗേന\\
താനൊരു ചേതനവാനായ് ഭവിക്കുന്നു\\
ദേഹോഹമെന്നുള്ള ബുദ്ധിയുണ്ടായ് വരു-\\
മാഹന്ത! നൂനമാത്മാവിനു മായയാ\\
ദേഹോഹമദ്യൈവ കര്‍മകര്‍ത്താഹമി-\\
ത്യാഹന്ത! സങ്കല്പ്യ സര്‍വദാ ജീവനുമ്\\
കര്‍മങ്ങള്‍ ചെയ്തു ഫലങ്ങളാല്‍ ബദ്ധനായ്\\
സമ്മോഹമാര്‍ന്നു ജനനമരണമാം\\
സംസാരസൗഖ്യദുഃഖാദികള്‍ സാധിച്ചു\\
ഹംസപദങ്ങള്‍ മറന്നു ചമയുന്നു.\\
മേല്‍പോട്ടുമാശു കീഴ്പോട്ടും ഭ്രമിച്ചതി\\
താത്പര്യവാന്‍ പുണ്യപാപാത്മകഃ സ്വയം\\
എത്രയും പുണ്യങ്ങള്‍ ചെയ്തേന്‍ വളരെ ഞാന്‍\\
വിത്താനുരൂപേണ യജ്ഞദാനാദികള്‍\\
ദുര്‍ഗതി നീക്കിസ്സുഖിച്ചു വസിക്കണം\\
സ്വര്‍ഗം ഗമിച്ചെന്നു കല്പിച്ചിരിക്കവേ\\
മൃത്യുഭവിച്ചു സുഖിച്ചു വാഴുംവിധൗ\\
ഉത്തമാംഗം കൊള്ളവീഴുമധോഭൂവി\\
പുണ്യമൊടുങ്ങിയാലിന്ദുതന്‍ മണ്ഡലേ\\
ചെന്നു പതിച്ചു നീഹാരസമേതനായ്\\
ഭൂമൗ പതിച്ചു ശാല്യാദികളായ് ഭവി-\\
ച്ചാമോദമുള്‍ക്കൊണ്ടു വാഴും ചിരതരം\\
പിന്നെപ്പുരുഷന്‍ ഭുജിക്കുന്ന ഭോജ്യങ്ങള്‍-\\
തന്നെ ചതുര്‍വിധമായ് ഭവിക്കും ബലാല്‍\\
എന്നതിലൊന്നു രേതസ്സായ് ചമഞ്ഞതു\\
ചെന്നു സീമന്തിനീയോനിയിലായ് വരും.\\
യോനിരക്തത്തോടു സംയുക്തമായ് വന്നു\\
താനേ ജരായുപരിവേഷ്ടിതവുമം.\\
ഏകദിനേന കലര്‍ന്നു കലലമാ-\\
മേകീഭവിച്ചാലതും പിന്നെ മെല്ലവേ\\
പഞ്ചരാത്രംകൊണ്ടു ബുദ്ബുദാകാരമാം\\
പഞ്ചദിനംകൊണ്ടു പിന്നെയഥാക്രമം\\
മംസപേശിത്വം ഭവിക്കുമതിന്നതു\\
മാസാര്‍ദ്ധകാലേന പിന്നെയും മെല്ലവേ\\
പേശിരുധിര പരിപ്ലുതമായ് വരു-\\
മാശു തസ്യാമങ്കുരോല്പത്തിയും വരും\\
പിന്നെയൊരു പഞ്ചവിംശതി രാത്രിയാല്‍;\\
പിന്നെയൊരു മൂന്നു മാസേന സന്ധിക-\\
ളംഗങ്ങള്‍ തോറും ക്രമേണ ഭവിച്ചീടു-\\
മംഗുലീ ജാലവുംനാലുമാസത്തിനാല്‍.\\
ദന്തങ്ങളും നഖപംക്തിയും ഗുഹ്യവും\\
സന്ധിക്കും നാസികാകര്‍ണനേത്രങ്ങളും\\
പഞ്ചമാസംകൊണ്ടു,ഷഷ്ഠമാസേ പുനഃ\\
കിഞ്ചനപോലും പിഴയാതെ ദേഹിനാം\\
കര്‍ണയോഃഛിദ്രം ഭവിക്കുമതിസ്ഫുടം\\
പിന്നെ മേഡ്രോപസ്ഥനാഭിപായുക്കളും\\
സപ്തമേ മാസി ഭവിക്കും, പുനരുടന്‍\\
ഗുപ്തമായോരു ശിരഃകേശരോമങ്ങള്‍\\
അഷ്ടമേ മാസി ഭവിക്കും, പുനരപി\\
പുഷ്ടമായീടും ജഠരസ്ഥലാന്തരേ.\\
ഒന്‍പതാം മാസേ വളരും ദിനംപ്രതി\\
കമ്പം കരചരണാദികള്‍ക്കും വരും\\
പഞ്ചമേമാസി ചൈതന്യവാനായ് വരു-\\
മഞ്ജസാ ജീവന്‍ ക്രമേണ ദിനേ ദിനേ\\
നാഭിസൂത്രാല്പരന്ധ്രേണ മാതാവിനാല്‍\\
സാപേക്ഷമായ ഭുക്താന്നരസത്തിനാല്‍\\
വര്‍ധതേ ഗര്‍ഭഗമായ പിണ്ഡം മുഹുര്‍-\\
മൃത്യുവരാ നിജകര്‍മ്മബലത്തിനാല്‍\\
പൂര്‍വജന്മങ്ങളും കാമങ്ങളും നിജം\\
സര്‍വകാലം നിരൂപിച്ചു നിരൂപിച്ചു\\
ദുഃഖിച്ചു ജാഠരവഹ്നി പ്രതപ്തനായ്-\\
ത്തല്‍ക്കാരണങ്ങള്‍ പറഞ്ഞുതുടങ്ങിനാന്‍:\\
‘പത്തുനൂറായിരം യോനികളില്‍ ജനി-\\
ച്ചെത്ര കര്‍മങ്ങളനുഭവിച്ചേനഹം\\
പുത്രദാരാര്‍ത്ഥബന്ധുക്കള്‍ സംബന്ധവു-\\
മെത്ര നൂറായിരം കോടി കഴിഞ്ഞിതു.\\
നിത്യകുടുംബഭരണൈകസക്തനായ്\\
വിത്തമന്യായമായാര്‍ജിച്ചിതന്വഹം\\
വിഷ്ണുസ്മരണവും ചെയ്തുകൊണ്ടീല ഞാന്‍\\
കൃഷ്ണകൃഷ്ണേതി ജപിച്ചീലൊരിക്കലും\\
തല്‍ഫലമെല്ലാമനുഭവിച്ചീടുന്നി-\\
തിപ്പോളിവിടെക്കിടന്നു ഞാനിങ്ങനെ.\\
ഗര്‍ഭപാത്രത്തില്‍നിന്നെന്നു ബാഹ്യസ്ഥലേ\\
കെല്‍പോടെനിക്കു പുറപ്പെട്ടു കൊള്ളാവു?\\
ദുഷ്കര്‍മമൊന്നുമേ ചെയ്യുന്നതില്ല ഞാന്‍\\
സല്‍കര്‍മജാലങ്ങള്‍ ചെയ്യുന്നതേയുള്ളൂ\\
നാരായണസ്വാമിതന്നെയൊഴിഞ്ഞുമ-\\
മറ്റാരെയും പൂജിക്കയില്ല ഞാനെന്നുമേ.’\\
ഇത്യാദിചിന്തിച്ചു ചിന്തിച്ചു ജീവനും\\
ഭക്ത്യാ ഭഗവല്‍സ്തുതി തുടങ്ങീടിനാന്‍\\
പത്തുമാസം തികയും വിധൗ ഭൂതലേ\\
ചിത്തതാപേന പിറക്കും വിധിവശാല്‍\\
സൂതിവാതത്തിന്‍ ബലത്തിനാല്‍ ജീവനും\\
ജാതനാം യോനിരന്ധ്രേണ പീഡാന്വിതം\\
പാല്യമാനോപി മാതാപിതാക്കന്മാരാല്‍\\
ബാല്യാദി ദുഃഖങ്ങളെന്തു ചൊല്ലാവതും?\\
യൗവനദുഃഖവും വാര്‍ദ്ധക്യദുഃഖവും\\
സര്‍വവുമോര്‍ത്തോളമേതും പൊറാ സഖേ!\\
നിന്നാലനുഭൂതമായുള്ളതെന്തിനു\\
വര്‍ണിച്ചു ഞാന്‍ പറയുന്നു വൃഥാ ബലാല്‍?\\
ദേഹോഹമെന്നുള്ള ഭാവനയാ മഹാ-\\
മോഹേന സൗഖ്യദുഃഖങ്ങളുണ്ടാകുന്നു.\\
ഗര്‍ഭവാസാദി ദുഃഖങ്ങളും ജന്തുവര്‍-\\
ഗോത്ഭവനാശവും ദേഹമൂലം സഖേ!\\
സ്ഥൂലസൂക്ഷ്മാത്മക ദേഹദ്വയാല്‍ പരം\\
മേലേയിരിപ്പതാത്മാപരന്‍ കേവലന്‍\\
ദേഹാദികളില്‍ മമത്വമുപേക്ഷിച്ചു\\
മോഹമകന്നാത്മജ്ഞാനിയായ് വാഴ്ക നീ.\\
ശുദ്ധം സദാ സാന്തമാത്മാനമവ്യയം\\
ബുദ്ധം പരബ്രഹ്മമാനന്ദമദ്വയം\\
സത്യം സനാതനം നിത്യം നിരുപമം\\
തത്ത്വമേകം പരം നിര്‍ഗുണം നിഷ്കളം\\
സച്ചിന്മയം സകലാത്മകമീശ്വര-\\
മച്യുതം സര്‍വജഗന്മയം ശാശ്വതം\\
മായാവിനിര്‍മുക്തമെന്നറിയുന്നേരം\\
മായാവിമോഹമകലുമെല്ലാവനും\\
പ്രാരബ്ധകര്‍മവേഗാനുരൂപം ഭുവി\\
പാരമാര്‍ത്ഥ്യാത്മനാ വാഴുക നീ സഖേ!\\
മറ്റൊരുപദേശവും പറയാം തവ\\
ചെറ്റു ദുഃഖം മനക്കാമ്പിലുണ്ടാകൊലാ\\
ത്രേതായുഗേവന്നു നാരായണന്‍ ഭുവി\\
ജാതനായീടും ദശരഥപുത്രനായ്\\
നക്തഞ്ചരേന്ദ്രനെ നിഗ്രഹിച്ചന്‍പോടു\\
ഭക്തജനത്തിനു മുക്തി വരുത്തുവാന്‍\\
ദണ്ഡകാരണ്യത്തില്‍ വാഴുംവിധൗ ബലാല്‍\\
ചണ്ഡനായുള്ള ദശാസ്യനാം രാവണന്‍\\
പുണ്ഡരീകോല്‍ഭൂതയാകിയ സീതയെ\\
പണ്ഡിതന്മാരായ രാമസൗമിത്രികള്‍\\
വേര്‍പ്പെട്ടിരിക്കുന്ന നേരത്തു വന്നു ത-\\
ന്നാപത്തിനായ്ക്കട്ടു കൊണ്ടുപോം മായയാ.\\
ലങ്കയില്‍ കൊണ്ടുവെച്ചീടും ദശാന്തരേ\\
പങ്കജലോചനയെത്തിരഞ്ഞീടുവാന്‍\\
മര്‍ക്കടരാജനിയോഗാല്‍ കപികുലം\\
ദക്ഷിണവാരിധി തീരദേശേ വരും\\
തത്ര സമാഗമം നിന്നോടു വാനരര്‍-\\
ക്കെത്തുമൊരു നിമിത്തേന നിസ്സംശയം\\
എന്നാലവരോടു ചൊലിക്കൊടുക്ക നീ\\
തന്വംഗി വാഴുന്ന ദേശം ദയാവശാല്‍\\
അപ്പോള്‍ നിനക്കു പക്ഷങ്ങള്‍ നവങ്ങളാ-\\
യുത്ഭവിച്ചീടുമതിനില്ല സംശയം.’\\
‘എന്നെപ്പറഞ്ഞു ബോധിപ്പിച്ചിതിങ്ങനെ\\
മുന്നം നിശാകരനായ മഹാമുനി.\\
വന്നതുകാണ്മിന്‍ ചിറകുകള്‍ പുത്തനാ-\\
യെന്നേ വിചിത്രമേ! നന്നുനന്നെത്രയും.\\
ഉത്തമതാപസന്മാരുടെ വാക്യവും\\
സത്യമല്ലാതെ വരികയില്ലെന്നുമേ.\\
ശ്രീരാമദേവകഥാമൃതമാഹാത്മ്യ-\\
മാരാലുമോര്‍ത്താലറിയാവതല്ലേതും.\\
രാമനാമാമൃതത്തിന്നു സമാനമായ്\\
മാമകേ മാനസേ മറ്റു തോന്നീലഹോ\\
നല്ലതു മേന്മേല്‍ വരേണമേ നിങ്ങള്‍ക്കു\\
കല്യാണഗാത്രിയെക്കണ്ടുകിട്ടേണമേ!\\
നന്നായതിപ്രയത്നം ചെയ്കിലര്‍ണവ-\\
മിന്നുതന്നെ കടക്കായ് വരും നിര്‍ണയം.\\
ശ്രീരാമനാമസ്മൃതികൊണ്ടു സംസാര-\\
വാരാന്നിധിയെക്കടക്കുന്നിതേവരും\\
രാമഭാര്യാലോകനാര്‍ഥമായ് പോകുന്ന\\
രാമഭക്തന്മാരാം നിങ്ങള്‍ക്കൊരിക്കലും\\
സാഗരത്തെക്കടന്നീടുവാനേതുമൊ-\\
രാകുലമുണ്ടാകയില്ലൊരുജാതിയും’\\
എന്നു പറഞ്ഞു പറന്നു മറഞ്ഞിത-\\
ത്യുന്നതനായ സമ്പാതി വിഹായസാ.
\end{verse}

%%19_samudralanghanachintha

\section{സമുദ്രലംഘനചിന്ത}

\begin{verse}
പിന്നെക്കപിവരന്മാര്‍ കൗതുകത്തോടു-\\
മന്യോന്യമാശു പറഞ്ഞു തുടങ്ങിനാര്‍\\
ഉഗ്രം മഹാനക്രചക്രഭയങ്കര-\\
മഗ്രേ സമുദ്രമാലോക്യ കപികുലം\\
എങ്ങനെ നാമിതിനെക്കടക്കുന്നവാ-\\
റെങ്ങും മറുകര കാണ്മാനുമില്ലല്ലോ.\\
ആവതല്ലാത്തതു ചിന്തിച്ചു ഖേദിച്ചു\\
ചാവതൈനെന്തവകാശം കപികളേ!’\\
ശക്രതനയതനൂജനാ മംഗദന്‍\\
മര്‍ക്കടനായകന്മാരോടു ചൊല്ലിനാന്‍:\\
‘എത്രയും വേഗബലമുള്ള ശൂരന്മാര്‍\\
ശക്തിയും വിക്രമവുംപാരമുണ്ടല്ലോ\\
നിങ്ങളെല്ലാവര്‍ക്കു,മെന്നാലിവരില്‍ വെ-\\
ച്ചിങ്ങുവന്നെന്നോടൊരുത്തന്‍ പറയണം\\
ഞാനിതിനാളെന്നവനല്ലോ നമ്മുടെ\\
പ്രാണനെ രക്ഷിച്ചുകൊള്ളൂന്നതും ദൃഢം.\\
സുഗ്രീവരാമസൗമിത്രികള്‍ക്കും ബഹു-\\
വ്യഗ്രം കളഞ്ഞു രക്ഷിക്കുന്നതുമവന്‍.’\\
അംഗദനിങ്ങനെ ചൊന്നതു കേട്ടവര്‍\\
തങ്ങളില്‍ത്തങ്ങളില്‍ നോക്കിനാരേവരും.\\
ഒന്നും പറഞ്ഞീലൊരുത്തരുമംഗദന്‍\\
പിന്നെയും വാനരന്മാരോടു ചൊല്ലിനാന്‍:\\
‘ചിത്തേ നിരൂപിച്ചു നിങ്ങളുടെ ബലം\\
പ്രത്യേഗമുച്യതാമുദ്യോഗപൂര്‍വകം.’\\
ചാടാമെനിക്കു ദശയോജനവഴി\\
ചാടാമിരുപതെനിക്കെന്നൊരു കപി\\
മുപ്പതു ചാടാമെനിക്കെന്നപരനു-\\
മപ്പടി നാല്പതാമെന്നു മറ്റേവനും\\
അന്‍പതറുപതെഴുപതുമാമെന്നു-\\
മെണ്‍പതു ചാടാമെനിക്കെന്നൊരുവനും\\
തൊണ്ണൂറു ചാടുവാന്‍ ദണ്ഡമില്ലേകനെ-\\
ന്നര്‍ണവമോ, നൂറുയോജനയുണ്ടല്ലോ.\\
‘ഇക്കണ്ട നമ്മിലാര്‍ക്കും കടക്കാവത-\\
ല്ലിക്കടല്‍ മര്‍ക്കടവീരരേ! നിര്‍ണയം.\\
മുന്നം ത്രിവിക്രമന്‍ മൂന്നു ലോകങ്ങളും\\
ഛന്നനായ് മൂന്നടിയായളക്കും വിധൗ\\
യൗവനകാലേ പിരുമ്പറയും കൊട്ടി\\
മൂവേഴുവട്ടം വലത്തുവെച്ചീടിനേന്‍\\
വാര്‍ധകഗ്രസ്തനായേനിദാനീം ലവ-\\
ണാബ്ധി കടപ്പാനുമില്ല വേഗം മമ\\
ഞാനിരുപത്തൊന്നു വട്ടാം പ്രദക്ഷിണം\\
ദാനവാരിക്കു ചെയ്തേന്‍ ദശമാത്രയാ\\
കാലസ്വരൂപനാമീശ്വരന്‍ തന്നുടെ\\
ലീലകളോര്‍ത്തോളമത്ഭുതമെത്രയും.’\\
ഇത്ഥമജാത്മജന്‍ ചൊന്നതു കേട്ടതി-\\
നുത്തരം വൃത്രാരിപൗത്രനും ചൊല്ലിനാന്‍:\\
‘അങ്ങോ‌ട്ടുചാടാമെനിക്കെന്നു നിര്‍ണയ-\\
മിങ്ങോട്ടു പോരുവാന്‍ ദണ്ഡമുണ്ടാകിലാം.’\\
സാമര്‍ത്ഥ്യമില്ല മറ്റാര്‍ക്കുമെന്നാകിലും\\
സാമര്‍ത്ഥ്യമുണ്ടു ഭവാനിതിനെങ്കിലും\\
ഭൃത്യജനങ്ങളയയ്ക്കയില്ലെന്നുമേ\\
ഭൃത്യരിലേകനുണ്ടാമെന്നതേ വരൂ.\\
ആര്‍ക്കുമേയില്ല സാമര്‍ത്ഥ്യമനശനം\\
ദീക്ഷിച്ചുതന്നെ മരിക്കനല്ലൂ വയം.’\\
താരേയനേവം പറഞ്ഞോരനന്തരം\\
സാരസസംഭവനന്ദനന്‍ ചൊല്ലിനാന്‍:\\
‘എന്തു ജഗല്‍പ്രാണനന്ദനനിങ്ങനെ\\
ചിന്തിച്ചിരിക്കുന്നതേതും പറയാതെ?\\
കുണ്ഠനായ്ത്തന്നെയിരുന്ന കളകയോ?\\
കണ്ടീല നിന്നെയൊഴിഞ്ഞു മറ്റാരെയും.\\
ദാക്ഷായണീഗര്‍ഭപാത്രസ്ഥനായൊരു\\
സാക്ഷാല്‍ മഹാദേവബീജമല്ലോ ഭവാന്‍.\\
പിന്നെ വാതാത്മജനാകയുമു,ണ്ടവന്‍-\\
തന്നോടു തുല്യന്‍ ബലവേഗമോര്‍ക്കിലോ.\\
കേസരിയെക്കൊന്നു താപം കളഞ്ഞൊരു\\
കേസരിയാകിയ വാനരനാഥനു\\
പുത്രനായഞ്ജന പെറ്റുളവായൊരു\\
സത്വഗുണപ്രധാനന്‍ ഭവാന്‍ കേവലം\\
അഞ്ജനാഗര്‍ഭച്യുതനായവനിയി-\\
ലഞ്ജസാ ജാതനായ് വീണനേരം ഭവാന്‍\\
അഞ്ഞൂറുയോജന മേല്പോട്ടു ചാടിയ-\\
തും ഞാനറിഞ്ഞിരിക്കുന്നിതു മാനസേ.\\
ചണ്ഡകിരണനുദിച്ചുപൊങ്ങുന്നേരം\\
മണ്ഡലംതന്നെത്തുടുതുടെക്കണ്ടു നീ\\
പക്വമെന്നോര്‍ത്തു ഭക്ഷിപ്പാനടുക്കയാല്‍\\
ശക്രനുടെ വജ്രമേറ്റു പതിച്ചതും\\
ദുഃഖിച്ചു മാരുതന്‍ നിന്നെയും കൊണ്ടുപോയ്-\\
പുക്കിതു പാതാള,മപ്പോള്‍ ത്രിമൂര്‍ത്തികള്‍\\
മുപ്പത്തുമുക്കോടി വാനവര്‍ തമ്മൊടും\\
ഉല്പലസംഭവപുത്രവര്‍ഗത്തൊടും\\
പ്രത്യക്ഷമായ് വന്നനുഗ്രഹിച്ചീടിനാര്‍\\
മൃത്യുവരാ ലോകനാശം വരുമ്പോഴും\\
കല്പാന്തകാലത്തുമില്ല മൃതിയെന്നു\\
കല്പിച്ചതിന്നിളക്കം വരാ നിര്‍ണയം\\
ആമ്നായസാരാര്‍ത്ഥമൂര്‍ത്തികള്‍ ചൊല്ലിനാര്‍\\
നാമ്നാ ഹനുമാനിവനെന്നു സാദരം.\\
വജ്രം ഹനുവിങ്കലേറ്റു മുറികയാ-\\
ലച്ചരിത്രങ്ങള്‍ മറന്നിതോ മാനസേ?\\
നിന്‍ കൈയിലല്ലയോ തന്നതു രാഘവ-\\
നംഗുലീയമതുമെന്തിനെന്നോര്‍ക്ക നീ.\\
ത്വല്‍ബലവീര്യവേഗങ്ങള്‍ വര്‍ണിപ്പതി-\\
നിപ്രപഞ്ചത്തിങ്കലാര്‍ക്കുമാമല്ലെടോ!’\\
ഇത്ഥം വിധിസുതന്‍ ചൊന്നനേരം വായു-\\
പുത്രനുമുത്ഥായ സത്വരം പ്രീതനായ്\\
ബ്രഹ്മാണ്ഡമാശു കുലുങ്ങുമാറൊന്നവന്‍\\
സമ്മദാല്‍ സിംഹനാദം ചെയ്തരുളിനാന്‍.\\
വാമനമൂര്‍ത്തിയെപ്പോലെ വളര്‍ന്നവന്‍\\
ഭിധരാകാരനായ് നിന്നു ചൊല്ലിനാന്‍:\\
‘ലംഘനം ചെയ്തു സമുദ്രത്തെയും പിന്നെ\\
ലങ്കാപുരത്തെയും ഭസ്മമാക്കി ക്ഷണാല്‍\\
രാവണനെക്കുലത്തോടുമൊടുക്കി ഞാന്‍\\
ദേവിയേയുംകൊണ്ടു പോരുവനിപ്പൊഴേ.\\
അല്ലായ്കിലോ ദശകണ്ഠനെബ്ബന്ധിച്ചു\\
മെല്ലവേ വാമകരത്തിലെടുത്തുടന്‍\\
കൂടത്രയത്തോടു ലങ്കാപുരത്തെയും\\
കൂടെ വലത്തു കരത്തിലാക്കിക്കൊണ്ടു\\
രാമാന്തികേവെച്ചു കൈതൊഴുതീടുവന്‍\\
രാമാംഗുലീയമെന്‍ കൈയിലുണ്ടാകയാല്‍.’\\
മാരുതിവാക്കു കേട്ടോരു വിധിസുത-\\
നാരൂഢകൗതുകം ചൊല്ലിനാന്‍ പിന്നെയും:\\
‘ദേവിയെക്കണ്ടു തിരിയേ വരിക നീ\\
രാവണനോടെതിര്‍ത്തീടുവാന്‍ പിന്നെയാം.\\
നിഗ്രഹിച്ചീടും ദശാസ്യനെ രാഘവന്‍\\
വിക്രമം കാട്ടുവാനന്നേരമാമല്ലോ.\\
പുഷ്കരമാര്‍ഗേണ പോകും നിനക്കൊരു\\
വിഘ്നം വരായ്ക! കല്യാണം ഭവിക്ക! തേ.\\
മാരുതദേവനുമുണ്ടരികേ തവ\\
ശ്രീരാമകാര്യാര്‍ത്ഥമായല്ലോ പോകുന്നു.’\\
ആശീര്‍വചനവും ചെയ്തു കപികുല-\\
മാശു പോകെന്നു വിധിച്ചോരനന്തരം\\
വേഗേന പോയ് മഹേന്ദ്രത്തിന്‍ മുകളേറി\\
നാഗാരിയെപ്പോലെ നിന്നു വിളങ്ങിനാന്‍.\\
ഇത്ഥം പറഞ്ഞറിയിച്ചൊരു തത്തയും\\
ബദ്ധമോദത്തോടിരുന്നിതക്കാലമേ.
\end{verse}

\begin{verse}
ഇത്യദ്ധ്യാത്മരാമായണേ ഉമാമഹേശ്വരസംവാദേ\\
കിഷ്കിന്ധകാണ്ഡം സമാപ്തം
\end{verse}
