
\chapter{ಮಾಯಾ ಆಕೃತಿಗಳು (Magic Figures) ಮತ್ತು ಮಾಯಾ ನಕ್ಷತ್ರಗಳು}

ನಮಗೆ ಗೊತ್ತೇ ಇದೆ. ಮಾಯಾಚೌಕವೆಂದರೆ ಒಂದು ಚೌಕ. ಚಚ್ಚೌಕ (Square). ಉದ್ದ, ಅಗಲಗಳನ್ನು ಸಮಾನ ಸಂಖ್ಯೆಯ ಭಾಗಗಳಾಗಿ ಮಾಡಿ ರಚಿಸಿರುವ ಹಂದರದಲ್ಲಿ ಸಂಖ್ಯೆಗಳನ್ನು ತುಂಬಿಸಿರುವುದು. ಅಡ್ಡಸಾಲು, ಕಂಭಸಾಲು ಮತ್ತು ಕರ್ಣಗಳಲ್ಲಿನ ಸಂಖ್ಯೆಗಳ ಮೊತ್ತ ಒಂದು ಚೌಕಕ್ಕೆ ನಿರ್ದಿಷ್ಟ. ಅನೇಕ ಗಣಿತಜ್ಞರು ಈ ದಿಶೆಯಲ್ಲಿ ಕೆಲಸಮಾಡಿ ಸಾವಿರಾರು ವಿನೂತನ, ವಿಶಿಷ್ಟ ಮಾಯಾಚೌಕಗಳನ್ನು ಪಡೆದಿರುವುದು ನಮಗೇ ವೇದ್ಯ.

ಕೆಲವು ಗಣಿತಜ್ಞರು ಹಾಗೂ ಮಾಯಾಚೌಕಾಸಕ್ತರು ಈ ಚೌಕಟ್ಟನ್ನು ದಾಟಿ ಬೇರೆ ಜ್ಯಾಮಿತೀಯ ಆಕೃತಿಗಳ ಬಾಹುಗಳ ಮೇಲೆ ಸಂಖ್ಯೆ ಹಾಕಿ, ಮೊತ್ತ ಒಂದು ನಿರ್ದಿಷ್ಟ ಸಂಖ್ಯೆ ಯಾಗುವುದನ್ನು ಪ್ರಯತ್ನಿಸಿದ್ದಾರೆ. ಅದರಲ್ಲಿ ಯಶಸ್ಸನ್ನೂ ಗಳಿಸಿದ್ದಾರೆ. ಇವುಗಳಿಗೆ \textbf{‘‘ಮಾಯಾ ಆಕೃತಿ’’} ಗಳೆಂದು ಹೆಸರಿಸಲಾಗಿದೆ. ಅಂತಹ ಕೆಲವನ್ನು ಗಮನಿಸೋಣ.

\section*{* ಮಾಯಾ ತ್ರಿಭುಜಗಳು :}

ಸರಳ ರೇಖೆಗಳಿಂದಾಗಬಹುದಾದ ಆಕೃತಿಗಳಲ್ಲಿ ತ್ರಿಭುಜವೇ ಅತಿ ಸರಳ. ಕನಿಷ್ಠ ಸಂಖ್ಯೆಯ ಸರಳ ರೇಖೆಗಳಿಂದ ರಚಿಸಬಹುದಾದುದು. ತ್ರಿಭುಜದ ಶೃಂಗ ಮತ್ತು ಬಾಹುಗಳ ಮೇಲೆ ಸಂಖ್ಯೆಗಳನ್ನು ಬರೆದು, ಒಂದೊಂದು ಬಾಹುವಿನ ಮೇಲೆ ಬರುವ ಸಂಖ್ಯೆಗಳ ಮೊತ್ತ ಒಂದು ತ್ರಿಭುಜಕ್ಕೆ ಸ್ಥಿರಾಂಕವಾದರೆ ಅದನ್ನು ಮಾಯಾತ್ರಿಭುಜವೆನ್ನಬಹುದು. ಇವುಗಳನ್ನು ರಚಿಸುವುದು ಸುಲಭವೇ. ಇವುಗಳಲ್ಲಿ ಅಂತಹ ಕುತೂಹಲಕಾರಿ ಅಂಶಗಳೇನೂ ಇಲ್ಲ. ಕೆಲವು ಮಾಯಾ ತ್ರಿಭುಜಗಳನ್ನು ನೋಡಿ.
\begin{figure}[H]
\includegraphics{src/figures/chap8/fig8-1.jpg}
\end{figure}

ತ್ರಿಭುಜಗಳು ಸಮಬಾಹು ತ್ರಿಭುಜಗಳೇ ಆಗಬೇಕೆಂದಿಲ್ಲ. ಉದಾಹರಣೆಯಲ್ಲಿ 1 ರಿಂದ 6ರವರೆಗಿನ ಸಂಖ್ಯೆಗಳನ್ನು ಮಾತ್ರ ಬಳಸಿದೆ. ಮೊತ್ತ ಬೇರೆ ಬೇರೆಯಾಗಿರುವುದನ್ನು ಗಮನಿಸಿ.

\section*{* ಮಾಯಾ ಪಂಚಭುಜ :}

4 ಭುಜಗಳ ಆಕೃತಿಗಳನ್ನು ಪರಿಗಣಿಸಿಲ್ಲ. ಎಲ್ಲ ಮಾಯಾಚೌಕಗಳು 4 ಭುಜಗಳ ಆಕೃತಿಗಳೇ. ಮಾಯಾಪಂಚಭುಜಾಕೃತಿಗೆ ಒಂದು ಮಾದರಿ.
\begin{figure}[H]
\includegraphics{src/figures/chap8/fig8-2.jpg}
\end{figure}
\begin{itemize}
	\item 1 ರಿಂದ 10ರವರೆಗೆ ಕ್ರಮಾಗತ ಸಂಖ್ಯೆಗಳು.
	\item ಯಾವುದೇ ಬಾಹುವಿನ ಮೇಲೆ ಇರುವ ಸಂಖ್ಯೆಗಳ ಮೊತ್ತ 16.
\end{itemize}

ಇಂತಹ ಆಕೃತಿಗಳಲ್ಲಿ ಹೆಚ್ಚು ವಿಸ್ಮಯ ಕಂಡುಬರುವುದಿಲ್ಲ. ಆಸಕ್ತರು ಸ್ವಲ್ಪ ತಾಳ್ಮೆ ಮತ್ತು ಪರಿಶ್ರಮ ಹಾಕಿದರೆ ರಚಿಸಲು ಸಾಧ್ಯ. ಇನ್ನೂ ಹೆಚ್ಚಿನ ಬಾಹುಗಳಿರುವ ಆಕೃತಿಗಳನ್ನು ರಚಿಸಬಹುದು.

ಇವುಗಳಿಗಿಂತ ಮಾಯಾನಕ್ಷತ್ರಗಳು ಭಿನ್ನ. ಅವು ಸರಳ ರೇಖೆಗಳಿಂದಾದ ನಕ್ಷತ್ರ ಆಕೃತಿಯವು. 5 ಮತ್ತು ಮೇಲ್ಪಟ್ಟ ಸಂಖ್ಯೆಗಳಷ್ಟು ಶೃಂಗಗಳನ್ನು ಹೊಂದಿರುವ ನಕ್ಷತ್ರಗಳನ್ನು ರಚಿಸಲಾಗಿದೆ. ಇವುಗಳಲ್ಲಿ ಸರಳ ರೇಖೆಗಳ ಮೇಲೆ ಬರುವ ಶೃಂಗ ಮತ್ತು ಛೇದನ ಬಿಂದುಗಳಲ್ಲಿ ಸಂಖ್ಯೆಗಳನ್ನು ಬರೆಯಲಾಗುತ್ತದೆ. ಯಾವುದೇ ಸರಳ ರೇಖೆಯ ಮೇಲೆ ಬರುವ ಸಂಖ್ಯೆಗಳ ಮೊತ್ತ ನಿರ್ದಿಷ್ಟ.

ಕೆಲವು ಮಾಯಾನಕ್ಷತ್ರಗಳನ್ನು ಪರಿಶೀಲಿಸೋಣ.

\section*{* 5. ಶೃಂಗಗಳ ನಕ್ಷತ್ರ:} 

ಯಾವುದೇಬಾಹುವಿನಮೇಲಿನ ಸಂಖ್ಯೆಗಳ ಮೊತ್ತ 24.
\begin{figure}[H]
\includegraphics{src/figures/chap8/fig8-3.jpg}
\end{figure}

\section*{* 6. ಶೃಂಗಗಳ ನಕ್ಷತ್ರ:} 

\begin{figure}[H]
\includegraphics{src/figures/chap8/fig8-4.jpg}
\end{figure}

\section*{* 7. ಶೃಂಗಗಳ ನಕ್ಷತ್ರ:} 

1ರಿಂದ 14ರವರೆಗೆ ಕ್ರಮಾಗತ ಸಂಖ್ಯೆಗಳು. ಮೊತ್ತ : 30
\begin{figure}[H]
\includegraphics{src/figures/chap8/fig8-5.jpg}
\end{figure}

\section*{* 8 ಶೃಂಗಗಳ ನಕ್ಷತ್ರ :}

\begin{figure}[H]
\includegraphics{src/figures/chap8/fig8-6.jpg}
\end{figure}
\begin{itemize}
	\item ಸಂಖ್ಯೆಗಳು ಕ್ರಮಾಗತವಾಗಿಲ್ಲ. ಮೊತ್ತ : 39
	\begin{figure}[H]
	\includegraphics{src/figures/chap8/fig8-7.jpg}
	\end{figure}
	\item ಸಂಖ್ಯೆಗಳು ಕ್ರಮಾಗತವಾಗಿಲ್ಲ. ಮೊತ್ತ : 40
\end{itemize}

\section*{IX- 2. ಮಾಯಾ ಜೇನುಗೂಡು (Magic Honey Comb)}

\begin{figure}[H]
\includegraphics{src/figures/chap8/fig8-8.jpg}
\end{figure}
\begin{itemize}
	\item 1ರಿಂದ 19ರವರೆಗಿನ ಕ್ರಮಾಗತ ಸಂಖ್ಯೆಗಳನ್ನು ಬಳಸಲಾಗಿದೆ.
	\item ಎಲ್ಲ ಮನೆಗಳೂ ಷಡ್ಭುಜಾಕೃತಿಗಳು.
	\item ಬಾಣದ ಗುರುತಿನ ನೇರದಲ್ಲಿ ಬರುವ ಷಡ್ಭುಜಗಳೊಳಗಿನ ಸಂಖ್ಯೆಗಳ ಮೊತ್ತ $\boxed{38}$
\end{itemize}

\section*{* ಮಾಯಾ ಷಡ್ಭುಜ :}

\begin{figure}[H]
\includegraphics{src/figures/chap8/fig8-9.jpg}
\end{figure}
\begin{itemize}
	\item 1ರಿಂದ 30 ರವರೆಗಿನ ಕ್ರಮಾಗತ ಸಂಖ್ಯೆಗಳನ್ನು ಬಳಸಲಾಗಿದೆ.
	\item ಒಂದು ಕೇಂದ್ರ ಷಡ್ಭುಜಾಕೃತಿಗೆ ಲಗತ್ತಾದಂತೆ 6 ಷಡ್ಭುಜಗಳಿವೆ.
	\item ಯಾವುದೇ ಷಡ್ಭುಜದ ಶೃಂಗ ಸಂಖ್ಯೆಗಳ ಮೊತ್ತ 93
	\item ಗೀಟು ರೇಖೆಗಳನ್ನೊಳಗೊಂಡ ಒಂದು ದೊಡ್ಡ ಷಡ್ಭುಜವಿದೆ. ಇದರ ಶೃಂಗ ಸಂಖ್ಯೆಗಳು 17,16,14,21,6,19. ಇವುಗಳ ಮೊತ್ತವೂ 93.
\end{itemize}
\begin{center}
*****
\end{center}

\section*{IX. 3. ಪೈಥಾಗೊರಾಸ್ ತ್ರಿವಳಿಗಳಿಂದಾದ ಮಾಯಾಚೌಕ :}

ಪೈಥಾಗೊರಾಸ್ ಪ್ರಮೇಯ ಹೀಗೆ ಹೇಳುತ್ತದೆ. ಒಂದು ಲಂಬ ಕೋನ ತ್ರಿಭುಜದಲ್ಲಿ ಕರ್ಣರೇಖೆಯ ಮೇಲಿನ ವರ್ಗದ ವಿಸ್ತೀರ್ಣವು ಉಳಿದೆರಡು ಬಾಹುಗಳ ಮೇಲೆ ರಚಿಸಿದ ವರ್ಗಗಳ ವಿಸ್ತೀರ್ಣದ ಮೊತ್ತಕ್ಕೆ ಸಮ. $ABC$ ತ್ರಿಭುಜವು ಅಯಲ್ಲಿ ಲಂಬಕೋನ ಹೊಂದಿರಲಿ. $BC$ ಕರ್ಣ, ಆಗ $BC^2 = AB^2 +AC^2$. 

ಒಂದು ಲಂಬಕೋನ, ತ್ರಿಭುಜ $ABC$ಯಲ್ಲಿ $BC$ ಕರ್ಣವಾಗಿರಲಿ. $BC$, $AB$ ಮತ್ತು $AC$ಗಳ ಮೇಲೆ ಚೌಕಗಳನ್ನು ರಚಿಸಿ. ಮೂರು ಚೌಕಗಳನ್ನೂ $3 \times 3$ಇರುವಂತೆ ವಿಭಾಗಿಸಿ. ಚಿತ್ರದಲ್ಲಿರುವಂತೆ ಸಂಖ್ಯೆಗಳನ್ನು 3 ಚೌಕಗಳ ಮನೆಗಳಲ್ಲೂ ತುಂಬಿಸಿ.
\begin{figure}[H]
\includegraphics{src/figures/chap8/fig8-10.jpg}
\end{figure}
\begin{itemize}
	\item ಪ್ರತಿ ಬಾಹುವಿನ ಮೇಲಿರುವ 3 ಕ್ರಮವರ್ಗದ ಮಾಯಾಚೌಕಗಳಲ್ಲಿನ ಅನುರೂಪ ಮನೆಗಳಲ್ಲಿರುವ ಸಂಖ್ಯೆಗಳು ಪೈಥಾಗೊರಾಸ್ ತ್ರಿವಳಿಗಳು.

	ಉದಾ: $40^2 = 24^2 +32^2 ; 25^2 =15^2 +20^2 ; 20^2 =16^2+12^2$
	\item ಮಾಯಾ ಚೌಕಗಳ ಮೊತ್ತಗಳೂ ಪೈಥಾಗೊರಾಸ್ ತ್ರಿವಳಿಗಳೇ 752 =452 +602
\end{itemize}

\section*{IX- 4. ಮಾಯಾ ಘನ (Magic Cube)}

ಪುಟ 83ರಲ್ಲಿನ 8 ಕ್ರಮವರ್ಗದ ಮಾಯಾಚೌಕವನ್ನು \textbf{‘‘ಮಾಯಾಘನ’’} ವೂ ಎಂದು ಹೇಳಿದೆವಲ್ಲವೆ. ಅದನ್ನು ಇಲ್ಲಿ ವಿನ್ಯಾಸಗೊಳಿಸಲಾಗಿದೆ.

ಮಾಯಾಚೌಕವನ್ನು 4 ಸಮಭಾಗಗಳಾಗಿ ಮಾಡಿದೆ. ಎಡ ಮೇಲ್ತುದಿಯ $4 \times 4$ ಚೌಕವನ್ನು ಒಂದು ಘನದ ಮೇಲಿನ ಮುಖದಲ್ಲಿ ಹೊಂದಿಸಿದೆ. ಉಳಿದ ಮೂರು $4 \times 4$ ಚೌಕಗಳನ್ನು ಪ್ರದಕ್ಷಿಣ ಕ್ರಮವಾಗಿ ತೆಗೆದುಕೊಂಡು ಘನದ ಮಧ್ಯಭಾಗದಿಂದ ತಳಭಾಗದವರೆಗೆ ಜೋಡಿಸಿದೆ. ಒಂದರ ಕೆಳಗೆ ಒಂದು ಬರುವಂತೆ.

ಈ ಆಕೃತಿಯಲ್ಲಿ ಪ್ರತಿ $4 \times 4$ಚೌಕದ ಅಡ್ಡಸಾಲು, ಕಂಭಸಾಲು ಮತ್ತು ಕರ್ಣಗಳ ಸಂಖ್ಯೆಗಳ ಮೊತ್ತ 130. ಹಾಗೆಯೇ ಮೇಲಿನಿಂದ ಕೆಳಕ್ಕೆ 4 ಅನುರೂಪ ಮನೆಗಳ ಸಂಖ್ಯೆತೆಗೆದುಕೊಂಡರೂ ಮೊತ್ತ 130.

\begin{minipage}{4cm}
\begin{figure}[H]
\includegraphics{src/figures/chap8/fig8-11.jpg}
\end{figure}
\end{minipage}
\quad
\begin{minipage}{5.5cm}
ಉದಾ : 15+34+63+18=130 53+28+5+44=130

ಮೇಲಿನ ಚೌಕದ ಯಾವುದೇ ಮೂಲೆಯ ಮನೆಯಿಂದ ಪ್ರಾರಂಭಿಸಿ ಅತಿ ಕೆಳಗಿನ ಎದುರು ಮೂಲೆಯ ಮನೆಗೆ ಎಳೆದ ಕರ್ಣದಲ್ಲಿ ಬರುವ ಸಂಖ್ಯೆಗಳ ಮೊತ್ತವೂ 130.

ಉದಾ : 1+22+64+43=130

ಉಳಿದ ಕರ್ಣಗಳ ಸಂಖ್ಯೆಗಳ ಮೊತ್ತವೂ 130.
\end{minipage}

\section*{IX- 5. ಜಾನಪದ ಒಗಟಿಗೆ ಮಾಯಾಚೌಕ ಪರಿಹಾರ :}

ಜಾನಪದ ಸಾಹಿತ್ಯದಲ್ಲಿ ಸಮಸ್ಯೆಗಳು/ಒಗಟುಗಳು ವಿಪುಲ. ಇವು ಸಾಮಾನ್ಯವಾಗಿ ಪದ್ಯರೂಪದಲ್ಲಿರುತ್ತವೆ. ಇಂತಹವುಗಳನ್ನು ಬಿಡುವಿನ ವೇಳೆಯ ಕಾಲಕ್ಷೇಪಕ್ಕೆ ಬಳಸುತ್ತಿದ್ದುದು ಈಗ್ಗೆ 60-70 ವರ್ಷಗಳ ಹಿಂದೆ ಕಾಣಸಿಗುತ್ತಿತ್ತು. ಇಂತಹ ಒಂದು ಸಮಸ್ಯೆಯನ್ನು ನೋಡೋಣ.

\textbf{ಒಗಟು :} 
\begin{quote}
ಒಂಭತ್ತು ರಂಭೆಯರಿಗವನೊಬ್ಬ ಗಂಡ\\
ಎಂಭತ್ತೊಂದು ಎಮ್ಮೆಗಳ ತಾ ಕೊಂಡು ತಂದ\\
ಕುಂಭ ಕುಂಭಕೆ ಹೆಚ್ಚು ಹಾಲು ಕರಕೊಂಡ\\
ರಂಭೆಯರಿಗೆ ಇದ ಸಮ ಮಾಡಿಕೊಳ್ಳಿರೆಂದ
\end{quote}
(ಒಬ್ಬ ಕವಾಡಿಗ. ಅವನಿಗೆ 9 ಹೆಂಡತಿಯರು. ಅವನು 81 ಎಮ್ಮೆಗಳನ್ನು ಕೊಂಡು ತರುತ್ತಾನೆ. ಅವುಗಳು 1 ನೇ ಎಮ್ಮೆ 1 ಅಳತೆ ಹಾಲುಕೊಟ್ಟರೆ, 2ನೆಯದು 2 ಅಳತೆ, 3ನೆಯದು 3ಅಳತೆ, 4ನೆಯದು 4 ಅಳತೆ.........81ನೆಯದು 81 ಅಳತೆ ಹಾಲು ಕೊಡುತ್ತವೆ. ಎಮ್ಮೆಗಳ ಸಂಖ್ಯೆ, ಹಾಲಿನ ಪರಿಮಾಣ ಸಮವಾಗಿ ಬರುವಂತೆ 9 ಹೆಂಡತಿಯರಿಗೆ ಹಂಚಿ)

\textbf{ಪರಿಹಾರ :}
\begin{itemize}
	\item ಒಟ್ಟು ಎಮ್ಮೆಗಳು 81; 9 ಜನರಿಗೆ ಸಮನಾಗಿ ಹಂಚಿದರೆ ಪ್ರತಿಯೊಬ್ಬರಿಗೂ 9 ಎಮ್ಮೆ ಬರುತ್ತದೆ.
	\item ಎಮ್ಮೆಗಳು ಕೊಡುವ ಹಾಲಿನ ಪರಿಮಾಣ 1+2+3+4.......+81 ಅಳತೆಗಳು ಅಂದರೆ $81 \times 82\div 2 = 3321$ ಅಳತೆಗಳು 9 ಜನರಿಗೆ ಸಮನಾಗಿ ಹಂಚಿದರೆ ಒಬ್ಬರಿಗೆ ಬರುವ ಹಾಲಿನ ಪರಿಮಾಣ $3321 \div 9 = 369$ ಅಳತೆಗಳು
\end{itemize}

\textbf{ಪರಿಹಾರ :}
\begin{itemize}
	\item 1ರಿಂದ 81 ರವರೆಗಿನ ಸಂಖ್ಯೆಗಳನ್ನು ಬಳಸಿ 9 ಕ್ರಮವರ್ಗದ ಮಾಯಾಚೌಕ ರಚಿಸಿ.
	\item ಎಮ್ಮೆಗಳನ್ನು 1,2,3.....81ಎಂದು ಗುರ್ತಿಸಿದರೆ ಪ್ರತಿ ಹೆಂಡತಿಗೂ ಅಡ್ಡಸಾಲಿನ/ಕಂಭಸಾಲಿನ ಸಂಖ್ಯೆಗಳ 9 ಎಮ್ಮೆ ಹಂಚಬಹುದು.
	\item ಆಗ ಆ 9 ಎಮ್ಮೆಗಳು ಕೊಡುವ ಹಾಲಿನ ಪರಿಮಾಣವೂ ಸಮ. ಪ್ರತಿಯೊಂದೂ 369 ಅಳತೆಗಳು.
\end{itemize}
ಈ ಒಗಟಿಗೆ ಬೇರೆ ಬೇರೆ ರೀತಿಯ ಮಾಯಾಚೌಕಗಳನ್ನು ರಚಿಸಿ ಪರಿಹಾರ ನೀಡಬಹುದು. ಪ್ರಯತ್ನಿಸಿ.
\begin{figure}[H]
\includegraphics{src/figures/chap8/fig8-12.jpg}
\end{figure}

\section*{IX- 6. ಮೈಸೂರು ಸಂಸ್ಥಾನದ ಮಹಾರಾಜರಾಗಿದ್ದ ಮುಮ್ಮಡಿ ಕೃಷ್ಣರಾಜ ಒಡೆಯರ್ (ಕ್ರಿ.ಶ. 1794-1868) ವಿರಚಿತ ‘‘ಚತುರಂಗ ಸಾರ ಸರ್ವಸ್ವ’’ ಕೃತಿಯಿಂದ,}

1 ರಿಂದ 88 ರ ವರೆಗೆ ಸಂಖ್ಯೆಗಳು ಕುದುರೆ ನಡಿಗೆಯಲ್ಲಿವೆ.
\begin{figure}[H]
\includegraphics{src/figures/chap8/fig8-13.jpg}
\end{figure}
