\chapter*{ಲೇಖಕನ ಮಾತು}

\phantom{a}

\vskip  -1.2cm

\quad 
ಗಣಿತ ಶಾಸ್ತ್ರವು ಒಂದು ಮಹಾಸಾಗರದಂತೆ ನಿಗೂಢ, ವಿಶಾಲ, ವಿಚಿತ್ರ, ವಿಶಿಷ್ಟ ಅಂಶಗಳನ್ನೊಳಗೊಂಡಿದೆ. ಸಾಗರದಲ್ಲಿ ಈಸಿ, ಮುಳುಗಿ ಪರಿಶೀಲಿಸುವ ಸಾಹಸಿಗಳಿಗೆ ಮುತ್ತು, ಹವಳ ಮುಂತಾದ ಅನರ್ಘ್ಯ ವಸ್ತುಗಳು ದೊರೆಯುವ ಸಾಧ್ಯತೆಗಳಿವೆ. ಅಂತೆಯೇ ಗಣಿತಶಾಸ್ತ್ರದಲ್ಲಿಯೂ ಸಹ. ಈ ಶಾಸ್ತ್ರದ ವಿವಿಧ ಪ್ರಕಾರಗಳಲ್ಲಿ ಮುಳುಗಿ ಅವನ್ನು ಜಾಲಾಡಿ ವಿಶಿಷ್ಟ ಫಲಿತ\-ಗಳನ್ನು ಸಾಧಿಸಿದ್ದಾರೆ ಗಣಿತ ವಿದ್ವಾಂಸರು. ಸಾವಿರಾರು ವರ್ಷಗಳಿಂದ ಈ ಪ್ರಕ್ರಿಯೆ ನಿರಂತರ\-ವಾಗಿ ನಡೆಯುತ್ತಲೇ ಬಂದಿದೆ. ಬಹುಶ: ಇದಕ್ಕೆ ಕೊನೆಯೆಂಬುದಿಲ್ಲ. ಈ ಸಾಧನೆಗಳಿಂದ ಮಾನವ ಕುಲಕ್ಕೆ ಆಗಿರುವ ಉಪಕಾರ ಊಹಿಸಲಸಾಧ್ಯ.ಹಲವು ಸಾಧನೆಗಳು ವಿಜ್ಞಾನ ಕ್ಷೇತ್ರಕ್ಕೆ, ತಂತ್ರ ವಿದ್ಯೆಗೆ, ಸಮಾಜ ವಿಜ್ಞಾನಕ್ಕೆ, ಔಷಧ ಪ್ರಪಂಚಕ್ಕೆ ಒಟ್ಟಿನಲ್ಲಿ ಮಾನವನ ಸರ್ವತೋಮುಖ \linebreak ಬೆಳವಣಿಗೆಗೆ ಬಹಳಷ್ಟು ಪೂರಕವಾಗಿವೆ.

ಇಂತಹ ಅನೇಕ ಪ್ರಕಾರಗಳಲ್ಲಿ “ಮಾಯಾಚೌಕ” ಕೂಡ ಒಂದು. ಇದಕ್ಕೆ ಎರಡು ಸಾವಿರ ವರ್ಷಕ್ಕೂ ಹೆಚ್ಚಿನ ಇತಿಹಾಸವಿದೆ. ಜಗತ್ತಿನಾದ್ಯಂತ ಗಣಿತ ವಿದ್ವಾಂಸರು ತಮ್ಮ ತೀಕ್ಷ್ಣ ಬುದ್ಧಿ\-ಮತ್ತೆ, ರಚನಾ ಕೌಶಲ್ಯ, ಕಲ್ಪನೆಗಳನ್ನು ಬಳಸಿ ಹಲವಾರು ವಿಧಗಳ ಮಾಯಾಚೌಕ\-ಗಳನ್ನು ರಚಿಸಿ\-ದ್ದಾರೆ, ರಚಿಸುತ್ತಲೇ ಇದ್ದಾರೆ. ಕೆಲವರು ಗಣಿತ ಶಾಸ್ತ್ರದ ಅನ್ಯ ಕ್ಷೇತ್ರಗಳಲ್ಲಿ ಇವನ್ನು ಬಳಸಿರುವುದೂ ಇದೆ. ಇಂತಹ ವಿಷಯದ ಕಿರು ಪರಿಚಯ ಮಾಡಿಕೊಡುವುದು ಈ ಪುಸ್ತಕದ ಆಶಯ.

ಇದನ್ನು ರಚಿಸಲು ನನ್ನನ್ನು ಪ್ರೋತ್ಸಾಹಿಸಿದವರು  ಹಾಸನದ \hbox{ಪ್ರೊ. ವಿ.ನರಹರಿ ಹಾಗೂ} ಬೆಂಗಳೂರಿನ ಪ್ರೊ.ಎಂ.ಆರ್​.ನಾಗರಾಜು ಮೊದಲ ಹಸ್ತಪ್ರತಿಯನ್ನು ಓದಿ ಪರಿಶೀಲಿಸಿ, \linebreak ಕ್ರಿಯಾತ್ಮಕ ಸಲಹೆಗಳನ್ನು ಇವರು ನೀಡಿದ್ದಾರೆ. ಆತ್ಮೀಯರಾದ ಭ.ರಾ.ವಿಜಯಕುಮಾರ್​, ವಿ.ಎಸ್​.ಎಸ್​.ಶಾಸ್ತ್ರಿ ಮತ್ತು ನನ್ನ ತಮ್ಮ ಬಿ.ಕೆ.ಸುಬ್ಬರಾವ್​ ಇವರೂ ಹಸ್ತಪ್ರತಿಯನ್ನು \linebreak ಅವಲೋಕಿಸಿ ಸಾಮಾನ್ಯ ಓದುಗರ ದೃಷ್ಟಿಯಿಂದ ಕೆಲವು ಮಾರ್ಪಾಡುಗಳನ್ನು ಸೂಚಿಸಿದರು. \linebreak ಇವರೆಲ್ಲರಿಗೆ ನಾನು ಆಭಾರಿ.
\eject

ಈ ಪುಸ್ತಕದಲ್ಲಿನ ಹಲವಾರು ಉದಾಹರಣೆಗಳು ಅನ್ಯ ಗ್ರಂಥಗಳಿಂದ ಹಾಗೂ \linebreak ಅಂತರ್ಜಾಲದಿಂದ ತೆಗೆದುಕೊಳ್ಳಲ್ಪಟ್ಟಿವೆ. ಆಗ್ರಂಥಗಳ ಲೇಖಕರಿಗೆ ಮತ್ತು ಅಂತರ್ಜಾಲ \linebreak ಕರ್ತೃಗಳಿಗೆ ಕೃತಜ್ಞತಾ ಪೂರ್ವಕ ನಮನ.

ಈ ಪುಸ್ತಕವು ಇಂದಿನ ಈ ರೂಪಕ್ಕೆ ಬರಲು ಕಾರಣರಾದವರು ಕ.ರಾ.ವಿ.ಪ ಅಧ್ಯಕ್ಷರಾದ ಡಾ॥ ಎಚ್​.ಎಸ್​.ನಿರಂಜನ ಆರಾಧ್ಯರು. ನನ್ನ ಕುರಿತ ಅವರ ಅಭಿಮಾನಕ್ಕಾಗಿ ಅವರಿಗೆ ನನ್ನ ಗೌರವಪೂರ್ವಕ ಪ್ರಣಾಮ. ಕರಾವಿಪ ಆಡಳಿತ ಮಂಡಳಿ, ಅಧಿಕಾರಿ ವರ್ಗ ಹಾಗೂ ನೌಕರ ವರ್ಗ ಇವರುಗಳ ಸಹಕಾರಕ್ಕಾಗಿ ನನ್ನ ನಮನ.

ಓದುಗರು ಈ ಪುಸ್ತಕದ ಉತ್ತಮಿಕೆಗೆ ಸೂಚಿಸುವ ಸಲಹೆಗಳಿಗೆ ಸದಾ ಸ್ವಾಗತ.

\begin{flushright}
\begin{tabular}{c}
{\bf ಬಿ. ಕೆ. ವಿಶ್ವನಾಥರಾವ್}\\
ಗಣಿತ ಸಂವಹನಕಾರ\\
ನ:{\rm 94}, {\rm 30}ನೇ ಅಡ್ಡ ರಸ್ತೆ, \\
ಬನಶಂಕರಿ {\rm 2}ನೇ ಹಂತ, ಬೆಂಗಳೂರು: {\rm 70}\\
ದೂರವಾಣಿ: {\rm 080-26739273}
\end{tabular}
\end{flushright}
