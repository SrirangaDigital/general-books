\chapter{ON WEAKLY SEMI CLOSED SETS IN TOPOLOGICAL SPACES}
\graphicspath{{Chapter1/Chapter1Figs/EPS/}{Chapter1/Chapter1Figs/}}

 Semi-closed sets, w-closed sets, semi pre-closed sets and ws-closed sets.\\\\\textbf{1.Introduction}\\\indent In 1970 N. Levine [18], first introduced the concept of generalized closed sets were defined and investigated. In 2000 M. Sheik John [33], introduced and studied w-closed sets in topological space X. Throughout this paper X or  $(X,\tau)$, represent non-empty topological space. Let A be subset of a topological space X. cl(A), int(A), scl(A),  cl(A) and spcl(A)) denote the closure of A, the interior of A, the semi-closure of A, the  -closure of A and the semi pre closure of A  in X respectively.\\
	\\\textbf{2. Preliminaries}\\Definition 2.1:  A subset A of a topological space $(X,\tau)$ is called a
	\begin{enumerate}[1.2.1]
		\item{} Regular open set [32] if A=int(cl(A)) and regular closed if A=cl(int(A))
		\item{}  Semi-open set [19] if A  $\subseteq$ cl(int(A)) and a semi-closed set if int(cl(A))  $\subseteq$ A.
		\item[{}  $\alpha$-open set [20] if A  $\subseteq$int(cl(int(A))) and $\alpha$  -closed set if cl(int(cl(A))) $\subseteq$ A.
		\item{}  Generalized semi pre closed set (gsp-closed) [8] if spcl(A)  $\subseteq$ U whenever A $\subseteq$  U  and U is open in $(X,\tau)$.
		\item{}  w-closed set[33] if cl(A) $\subseteq$  U whenever A $\subseteq$  U and U   is semi -open in $(X,\tau)$ .
		\item{}gspr-closed set[10] if spcl(A)  $\subseteq$ U whenever A $\subseteq$  U and U   is regular -open in $(X,\tau)$.
		\item[vii.]  $\alpha$gp-closed set[11] if cl(A)  $\subseteq$ U whenever A $\subseteq$  U and U   is pre-open in $(X,\tau)$.
		\item[viii.]  *g$\alpha$ -closed set[41] if cl(A) $\subseteq$  U whenever A $\subseteq$  U   and U  is  g  - open in $(X,\tau)$.
		\item[ix.]  g$\#$s-closed set[40] if scl(A) $\subseteq$  U whenever A $\subseteq$  U and U   is $\alpha$g -open in $(X,\tau)$.
		\item[x.]  rb-closed set[24] if cl(A) $\subseteq$  U whenever A   U and U   is  b-open in$(X,\tau)$.
		\item[xi.]  g$\xi$* -closed set[17] if   cl(A) $\subseteq$   U whenever A $\subseteq$  U and U  is  $\#$g - open in $(X,\tau)$.
	\end{enumerate}
	\textbf{3.  Basic properties of ws-closed sets in topological space :  }
	\\Definition 3.1: A subset A of a topological space $(X,\tau)$ is called weakly semi closed (ws-closed) set if scl(A)$\subseteq$ U , whenever A $\subseteq$  U and   U is w-open set in $(X,\tau)$. The family of all ws -closed sets X is denoted by WSC(X). the compliment of ws -closed set is called ws-open set in $(X,\tau)$.  The family  of all ws-open  sets in X is denoted by WSO(X).
	\\\textbf{Example 3.2:}Let X = \{a, b, c, d\},   $\tau$= \{X, $\phi$ ,\{a\}, \{b\}, \{a, b\}, \{a, b, c\}\}.
	\\\textbf{i)}	closed sets in (X, $\tau$ ) are $\tau$ = \{X, $\phi$ ,\{d\},\{c, d\},\{a, c, d\}, \{b, c, d\}.
	\\\textbf{ii)}	semi-closed sets in (X, $\tau$ ) are $\tau$ = \{X, $\phi$ , \{a\},\{b\},\{c\},\{d\},\{a, c\},\{a, d\},\{b, c\},\{b, d\},\{c, d\}, \{a, c, d\},\{b, c, d\}
	\\\textbf{iii)}w-closed sets in (X, $\tau$ ) are $\tau$ = \{X, $\phi$ ,\{d\},\{c, d\},\{a,b\}, \{a,b,c\}.
	\\\textbf{iv)}w-open sets in (X, $\tau$ ) are $\tau$ = \{X, $\phi$ ,\{a\},\{b\},\{a, c, d\}, \{b, c, d\}.
	\\\textbf{v)} ws-closed sets in (X, $\tau$ ) are $\tau$ = \{X, $\phi$ ,\{a\},\{b\},\{c\},\{d\},\{a, c\},\{a, d\},\{b, c\},\{b, d\},\{c, d\},\{a, b, d\}, \{a, c, d\},\{b, c, d\}.
	\\\textbf{vi)} ws-open sets in (X, $\tau$ ) are $\tau$ = \{X, $\phi$ ,\{a\},\{b\},\{c\},\{a, c\},\{a, d\},\{b, c\},\{b, d\},\{c, d\},\{a, b, c\},\{a, b, d\}, \{a, c, d\},\{b, c, d\}.
	\\\indent We prove that the class of ws-closed sets are properly lies between the class of all semi-closed sets and generalised semi-pre regular closed sets in topological spaces
	\\\textbf{Theorem 3.3:} Every semi-closed [19] set in X is ws-closed set in X.
	\\\textbf{Proof:} Let A be a semi-closed set in X. Let U be any w-open set in X  such that  A $\subseteq$ U. Since A is semi-closed, we have scl(A) = A $\subseteq$ U, we have  scl(A) $\subseteq$ U.   Hence A is ws-closed set in X.
	\\\textbf{Remark 3.4:} The converse of the above Theorem 3.3 need not be true as seen from the following Example 3.5.
	\\\textbf{Example 3.5:} Let X = \{a, b, c, d\},   $\tau$= \{X, $\phi$ ,\ {a\}, \{b\}, \{a, b\}, \{a, b, c\}\}.  then the set  A= \{a,b,d\} is ws-closed set but not semi-closed in X.
		\\\textbf{Corollary 3.6:} In a topological space $(X,\tau)$,
		\\i)	Every regular closed [32] set in X  is ws-closed set in X.
		\\ii)	Every  closed  set in X  is ws-closed set in X.
		\\iii)	Every  $\alpha$- closed [20] set in X  is ws-closed set in X.
		\\iv)	Every g$\#$-closed[37] set in X  is ws-closed set in X.
		\\v)	Every *g$\alpha$ -closed[ 41] set in X  is ws-closed set in X.
		\\vi)	Every g$\#$s -closed[40] set in X  is ws-closed set in X.
		\\vii)	Every rb -closed[24]  set in X  is ws-closed set in X.
		\\viii)	Every  $\ddot{g}$-closed set in X  is ws-closed set in X.
		\\ix)	Every g$\xi$* -closed[17]] set in X  is ws-closed set in X.
		\\x)	Every $\alpha$gp -closed[ 11]  set in X  is ws-closed set in X.
		\\\textbf{Proof i: } In view of the fact that every regular closed is  semi-closed, therefore  by 3.3 every regular closed is ws-closed set.
		\\\textbf{Proof ii:} In view of the fact that every  closed set is  semi -closed, therefore  by 3.3  every closed set  is ws-closed set.
		\\\textbf{Proof iii:}In view of the fact that  every $\alpha$ - closed is  semi -closed, therefore  by 3.3 every$\alpha$ - closed is ws-closed set.
		\\\textbf{Proof iv:}Let A be g$\#$-closed set in X. Let U be any w-open set in X such that A $\subseteq$ U. Since A is g$\#$-closed, we have scl(A) = A $\subseteq$ U, we have  scl(A) $\subseteq$ U.   Hence A is ws-closed set in X.
		\\\textbf{Proof v:}  Let A be *g$\alpha$ -closed set in X. Let U be any w-open set in X such that A$\subseteq$ U. Since A is *g$\alpha$ -closed, we have scl(A) = A$\subseteq$ U, we have  scl(A) $\subseteq$ U.   Hence A is ws-closed set in X.
		\\\textbf{Proof vi: } Let A be g$\#$s -closed set in X. Let U be any w-open set in X such that  A $\subseteq$ U. Since A is g$\#$s -closed, we have scl(A) = A $\subseteq$ U, we have  scl(A)$\subseteq$ U.   Hence A is ws-closed   set in X.
		\\\textbf{Proof vii: } Let A be rb -closed set in X. Let U be any w-open set in X such that  A $\subseteq$ U. Since A is rb-closed, we have scl(A) = A $\subseteq$ U, we have  scl(A) $\subseteq$ U.   Hence A is ws-closed set in X.
		\\\textbf{Proof viii:} Let A be   $\ddot{g}$-closed set in X. Let U be any w-open set in X such that  A $\subseteq$ U. Since A is  $\ddot{g}$-closed, we have scl(A) = A$\subseteq$ U, we have  scl(A) $\subseteq$ U.   Hence A is ws-closed set in X.
		\\\textbf{Proof ix: } Let A be g$\xi$* -closed set in X. Let U be any w-open set in X such that  A $\subseteq$ U. Since A is g$\xi$* -closed, we have scl(A) = A $\subseteq$ U, we have  scl(A) $\subseteq$ U.   Hence A is ws-closed set in X.
		\\\textbf{Proof x: }Let A be$\alpha$gp -closed set in X. Let U be any w-open set in X such that  A $\subseteq$ U. Since A is $\alpha$gp-closed, we have scl(A) = A$\subseteq$ U, we have  scl(A) $\subseteq$ U.   Hence A is ws-closed  set in  X.
		\\\textbf{Remark 3.7:} The converse of the above Corollary 3.6 need not be true as seen from the following Example 3.8.
		\\\textbf{Example 3.8:} Let X = \{a, b, c, d\},   $\tau$= \{X, $\phi$ ,\{a\}, \{b\}, \{a, b\}, \{a, b, c\}\}.then the sets
		\\i.	regular-closed sets in $(X,\tau)$ are X, $\phi$ ,\{a, c, d\}, \{b, c, d\}.
		\\ii.	closed sets in $(X,\tau)$ are X, $\phi$ ,\{d\},\{c, d\},\{a, c, d\}, \{b, c, d\}
		\\iii.	 $\alpha$ -closed sets in $(X,\tau)$ are X, $\phi$ ,\{c\},\{d\},\{c, d\}, \{a, c, d\},\{b, c, d\}
		\\iv.	g$\#$ -closed sets in $(X,\tau)$ are X, $\phi$ ,\{d\},\{c, d\}, \{a, c, d\},\{b, c, d\}
		\\v.	*g$\alpha$ -closed sets in $(X,\tau)$ are X, $\phi$ ,\{d\},\{c, d\}, \{a, c, d\},\{b, c, d\}
		\\vi.	g$\#$s-closed sets in $(X,\tau)$ are X, $\phi$ ,\{a\},\{b\},\{c\},\{d\},\{a, c\},\{a, d\},\{b, c\},\{b, d\},\{c, d\}, \{a, c, d\},\{b, c, d\}.
		\\vii.	rb -closed sets in $(X,\tau)$ are X, $\phi$,\{c, d\},\{a, c, d\}, \{b, c, d\}.
		\\viii.	  $\ddot{g}$ -closed sets in $(X,\tau)$ areX, $\phi$ ,\{d\},\{c, d\},\{a, c, d\}, \{b, c, d\}.
		\\ix.	 g$\xi$*  -closed sets in $(X,\tau)$ are X, $\phi$ ,\{d\},\{c, d\},\{a, c, d\}, \{b, c, d\}
		\\x.	$\alpha$gp -closed sets in $(X,\tau)$ areX, $\phi$ ,\{c\},\{d\},\{c, d\}, \{a, c, d\},\{b, c, d\}
		\\ws-closed sets in  $(X,\tau)$ are  X, $\phi$ ,\{a\},\{b\},\{c\},\{d\},\{a, c\},\{a, d\},\{b, c\},\{b, d\},\{c, d\},\{a, b, d\}, \{a, c, d\},\{b, c, d\}.
		\\\indent It is observed that set   A= \{a,b,d\} is ws-closed set but not regular closed (closed,  $\alpha$ - closed, g$\#$-closed, *g$\alpha$ -closed, g$\#$s -closed ,rb -closed,    $\ddot{g}$-closed,   g$\xi$*-closed, $\alpha$gp -closed  sets) in X .
		\\\textbf{Theorem 3.9:} Every ws-closed set in X is gspr-closed[10] set in X.
		\\\textbf{Proof:}  Let A be a  ws-closed set in X. Let U be any regular open set in X  such that  A $\subseteq$ U. Since every regular  open set is w- open set and A is  ws-closed set, we have scl(A)$\subseteq$  U. Therefore scl(A)$\subseteq$  U. Therefore U is regular open in X. Hence A is gspr -closed in X.
		\\\textbf{Remark 3.10:} The converse of the above Theorem 3.9 need not be true as seen from the following Example 3.11.
		\\\textbf{Example 3.11:} Let X = \{a, b, c, d\},   $\tau$= \{X, $\phi$ ,\{a\}, \{b\}, \{a, b\}, \{a, b, c\}\}.    Then the set A= \{b\} is gspr -closed set but not ws-closed set in X.
		\\\textbf{Corollary 3.12:}
		\\i)	Every ws-closed set is gsp-closed[8] set in X.
		\\ii)	 Every ws-closed set is rgb-closed[22] set in X.
		\\\textbf{Proof:}
		\\i) Follow from Govindappa Navalagi et all[8], every gspr-closed set is gsp-closed set and then follows from Theorem 3.9
		\\ii) Follow from K. Mariappa, S. Sekar[22], every rgb-closed is gsp-closedset, then follows from Corollary 3.12(i).
		The converse of the Corollary 3.12 is need not be true in general as seen from the following Example 3.13.
		\\\textbf{Example 3.13:} Let X = \{a, b, c, d\},   $\tau$= \{X, $\phi$ ,\ {a\}, \{b\}, \{a, b\}, \{a, b, c\}\}.  Then the set A= \{b\} is gsp (rgb) -closed set but not ws-closed set in X.
			\\\textbf{Remark 3.14:} The following Example 3.15, shows that ws-closed sets are independent of gpr-closed[9] sets, wgr$\alpha$-closed[16] sets, pgr$\alpha$-closed[5 ] sets, $\widehat{rg}$ -closed sets[31], gp closed[30] sets, rgw-closed[29] sets, rw-closed[2] sets, rg$\alpha$-closed[36] sets, $\beta$wg**-closed[35] sets.
			\\\textbf{Example 3.15:} Let X = \{a, b, c, d\}, $\tau$= \{X, $\phi$ ,\{a\}, \{b\}, \{a, b\}, \{a, b, c\}\}.  Then
			\\\textbf{i)} closed sets in (X, $\tau$ ) are $\tau$ = \{X, $\phi$ ,\{d\},\{c, d\},\{a, c, d\}, \{b, c, d\}.
			\\\textbf{ii)}	ws-closed sets in (X, $\tau$ ) are $\tau$ = \{X, $\phi$ , \{a\},\{b\},\{c\},\{d\},\{a, c\},\{a, d\},\{b, c\},\{b, d\},\{c, d\}, \{a, b, d\}, \{a, c, d\},\{b, c, d\}.
			\\\textbf{iii}	gpr -closed sets in (X, $\tau$ ) are $\tau$ = \{ X,$\phi$  ,\{c\},\{d\},\{a, b\},\{a, c\},\{a, d\},\{b, c\},\{b, d\},\{c, d\},\{a, b, c\}, \{a, b, d\},\{a, c, d\},\{b, c, d\}.
			\\\textbf{iv)} wgr$\alpha$ -closed sets in  (X, $\tau$ ) are $\tau$ = \{X, $\phi$ ,\{c\},\{d\},\{a, b\},\{a, c\},\{a, d\},\{b, c\},\{b, d\},\{c, d\},\{a, b, c\},\{a, b, d\}, \{a, c, d\},\{b, c, d\}.
			\\\textbf{v)} pgr$\alpha$ -closed sets in (X, $\tau$ ) are $\tau$ = \{X, $\phi$ ,\{c\},\{d\},\{a, b\},\{a, c\},\{a, d\},\{b, c\},\{b, d\},\{c, d\}, \{a, b, c\},\{a, b, d\},\{a, c, d\},\{b, c, d\}.
			\\\textbf{vi)} $\hat{rg}$-closed sets in (X, $\tau$ ) are $\tau$ = \{ X, $\phi$ ,\{c\},\{d\},\{a, b\},\{a, c\},\{a, d\},\{b, c\},\{b, d\},\{c, d\},\{a, b, c\}, \{a, b, d\},\{a, c, d\},\{b, c, d\}.
			\\\textbf{vii)}gprw-closed sets in (X, $\tau$ ) are $\tau$ = \{X, $\phi$ ,\{c\},\{d\},\{a, b\},\{c, d\},\{a, b, c\},\{a, b, d\}, \{a, c, d\},\{b, c, d\}.
			\\\textbf{x)} rgw-closed sets in (X, $\tau$ ) are $\tau$ = \{X, $\phi$ ,\{c\},\{d\},\{a, b\},\{c, d\},\{a, b, c\},\{a, b, d\}, \{a, c, d\},\{b, c, d\}.
			\\\textbf{xi)} rw-closed sets in (X,  $\tau$) are $\tau$ = \{X, $\phi$ ,\{d\},\{a, b\},\{c, d\},\{a, b, c\},\{a, b, d\},\{a, c, d\}, \{b, c, d\}.
			\\\textbf{xiv)} rg$\alpha$-closed sets in (X,$\tau$) are \{X,$\phi$ ,\{c\},\{d\},\{a, b\},\{a, c\},\{a, d\},\{b, c\},\{b, d\},\{c, d\}, \{a, b, c\},\{a, b, d\},\{a, c, d\},\{b, c, d\}\}.
			\\\textbf{xvi)}	$\beta$wg**-closed sets in (X,$\tau$) are \{X, $\phi$ ,\{c\},\{d\},\{a, b\},\{a, c\},\{a, d\},\{b, c\},\{b, d\},\{c, d\},\{a, b, c\},\{a, b, d\},\{a, c, d\},\{b, c, d\}\}.
			\\\indent Therefore \{a\} is ws-closed in X  but not gpr-closed[9] (wgr$\alpha$-closed[16] sets, pgr$\alpha$-closed[5 ] sets, $\widehat{rg}$ -closed sets[31], gp closed[30] sets, rgw-closed[29] sets, rw-closed[2] sets, rg$\alpha$-closed[36] sets, $\beta$wg**-closed[35] sets.) set  in X.
			\\\textbf{Remark 3.16:}  The following example shows that ws-closed sets are independent of  sets   wg-closed[23] sets, gw$\alpha$-closed[3] sets, g*p-closed[39] sets,$\beta$wg*-closed[7] sets,**g$\alpha$-closed[41] sets , $ \hat{g}$-closed[38] sets,$\tilde{g}$ -closed[14] sets, $\#$g$\alpha$-closed[6] sets,  g*-preclosed
			and g$\#$p$\#$-closed[28] sets.
			\\\textbf{Example 3.17:} Let X = \{a, b, c \}, $ \tau_{1}$ = { X, $\phi$ , \{a\}, \{b\},\{a, b\}} and  $ \tau_{2}$ = { X,$\phi$ ,\{a\}, \{b, c\}}. Then
			\\\textbf{i:}closed sets in (X,  $ \tau_{1}$) are X, $\phi$ ,\{c\},\{a, c\},\{b, c\}.
			\\\textbf{ii:}	ws-closed sets in (X,  $ \tau_{1}$) are X, $\phi$ ,\{a\},\ {b\},\{c\},\{a, c\},\{b, c\}.
				\\\textbf{v:}	wg-closed sets in (X,  $ \tau_{1}$) are X, $\phi$ , \{c\},\{a, c\},\{b, c\}.
				\\\textbf{vi:}	gw$\alpha$-closed sets in (X,  $ \tau_{1}$) are X, $\phi$ ,\{c\},\{a, c\},\{b, c\}.
				\\\textbf{vii:}	 g*p-closed sets in (X,  $ \tau_{1}$) are X,  $\phi$, \{c\},\{a, c\},\{b, c\}.
				\\\textbf{x:}$\beta$wg*-closed sets in (X,  $ \tau_{1}$) are X, $\phi$ , \{c\},\{a, c\},\{b, c\}.
				\\\textbf{xi:}**g$\alpha$-closed sets in (X,  $ \tau_{1}$) are X,  $\phi$, \{c\},\{a, c\},\{b, c\}.
				\\\textbf{xii:}$\hat{g}$-closed sets in (X,  $ \tau_{1}$) are X,  $\phi$, \{c\},\{a, c\},\{b, c\}.
				\\\textbf{xiii:}$\tilde{g}$-closed sets in (X,  $ \tau_{1}$) are X,  $\phi$, \{c\},\{a, c\},\{b, c\}.
				\\\textbf{xv:}	$\#g$$\alpha$-closed sets  in (X,  $ \tau_{1}$) are X,  $\phi$, \{c\},\{a, c\},\{b, c\}.
				\\\textbf{xvi:}	g*-preclosed sets in (X,  $ \tau_{1}$) are X, $\phi$ , \{c\},\{a, c\},\{b, c\}.
				\\\textbf{xvii:}g$\#$p$\#$--closed sets in (X,  $ \tau_{1}$) are X, $\phi$ , \{c\},\{a, c\},\{b, c\}.
				\\and also
				\\\textbf{i:}	closed sets in (X,  $ \tau_{2}$) are X, $\phi$ ,\{a\},\{b, c\}.
				\\\textbf{ii:}	ws-closed sets in ( X,  $ \tau_{2}$) are X,$\phi$,\{a\} ,\{b, c\}.
				\\\textbf{ii:}	wg-closed set  in  (X,  $ \tau_{2}$) are X,$\phi$,\{a\}, \{b\},\{c\}\{a, b\},\{a, c\},\{b, c\}.
				\\\textbf{iv:}	gw$\alpha$-closed sets in ( X,$ \tau_{2}$) are X,  $\phi$, \{a\},\ {b\},\{c\},\{a, b\},\{a, c\},\{b, c\}.
					\\\textbf{v:} g*p-closed sets in (X,  $ \tau_{2}$) are X, $\phi$, \{a\}, \{b\},\{c\},\{a, b\},\{a, c\},\{b, c\}.
					\\\textbf{vi:}	$\beta$wg*-closed sets in (X,  $ \tau_{2}$) are X, $\phi$,\{a\}, \{b\},\{c\},\{a, b\},\{a, c\},\{b, c\}.
					\\\textbf{vii:}**g$\alpha$-closed sets in (X,  $ \tau_{2}$) are X,$\phi$ ,\{a\},\{b\},\{c\},\{a, b\},\{a, c\},\{b, c\}.
					\\\textbf{viii:}$\hat{g}$-closed sets in (X,  $ \tau_{2}$) are X, $\phi$,\{a\}, \{b\},\{c},\{a, b\},\{a, c\},\{b, c\}.
				\\\textbf{ix:}	$\tilde{g}$-closed sets in (X,  $ \tau_{2}$) are X, $\phi$ , \{a\}, \{b\},\{c\},\{a, b\},\{a, c\},\{b, c\}.
				\\\textbf{x:}$\#g$$\alpha$-closed sets in (X,  $ \tau_{2}$) are X,  $\phi$,\{a\}, \{b\},\{c\},\{a, b\},\{a, c\},\{b, c\}.
				\\\textbf{xi:}	g*-preclosed sets in (X,  $ \tau_{2}$) are X,$\phi$  ,\{a\}, \{b\},\{c\},\{a, b\},\{a, c\},\{b, c\}.
				\\\textbf{xii:}g$\#$p$\#$--closed sets in (X,  $ \tau_{2}$) are X, $\phi$ ,\{a\}, \{b\}\{c\},\{a, b\},\{a, c\},\{b, c\}.
				\\\indent Therefore \{b\} is ws-closed set in (X, $ \tau_{1}$) but not in wg-closed( gw$\alpha$-closed sets, g*p-closed sets,$\beta$wg*-closed sets,**g$\alpha$-closed sets, $ \hat{g}$-closed sets,$\tilde{g}$ -closed sets, $\#$g$\alpha$-closed sets,  g*-pre closed and g$\#$p$\#$-closed sets)set  in (X,  $ \tau_{1}$).
				\\\indent Meanwhile \{b\} in wg-closed( gw$\alpha$-closed sets, g*p-closed sets,$\beta$wg*-closed sets,**g$\alpha$-closed sets , $ \hat{g}$-closed sets,$\tilde{g}$ -closed sets, $\#$g$\alpha$-closed sets, g*-pre closed  and g$\#$p$\#$-closed sets.) in  (X, $ \tau_{2}$) but not ws-closed  set in (X, $ \tau_{2}$).
				\\\textbf{Remark 3.18:}  The following example 3.19 shows that ws-closed sets are independent of sets g-closed[18] sets , sg-closed[14] sets, g$\alpha$-closed[21] sets, sgb-closed[13] sets,rg*b-closed[12] sets,pgpr-closed[1] sets, g$\alpha$b-closed[42] sets and rps-closed[34] sets.
				\\\textbf{Example 3.19:}Let X = \{a, b, c, d\}, $\tau_{1}$ =\{X, $\phi$,\{a\},\{a, b\},\{a, b, c\}\} and  $\tau_{2}$ =\{ X, $\phi$,\{a, b\},\{c, d\}\}.
				\\ i.  closed sets in (X,$ \tau_{1}$) are X, $\phi$,\{d\},\{c, d\},\{b, c, d\}.
				\\ii.  ws-closed sets in (X,$ \tau_{1}$) are X, $\phi$ ,\{a\},\{b\},\{c\},\{d\},\{a, c\},\{a, d\},\{b, c\},\{b, d\},\{c, d\},\{a, b, d\},\{a, c, d\},\{b, c, d\}.
				\\iii. g-closed sets in (X,  $ \tau_{1}$) are X, $\phi$ ,\{d\},\{a, d\},\{b, d\},\{c, d\},\{a, b, d\},\{a, c, d\},\{b, c, d\}.
				\\iv.	sg-closed sets in (X,$ \tau_{1}$) are X, $\phi$ ,\{b\},\{c\},\{d\},\{b, c\},\{b, d\},\{c, d\},\{b, c, d\}.
				\\v. 	g$\alpha$- closed sets in (X,$ \tau_{1}$) are X, $\phi$ ,\{b\},\{c\},\{d\},\{b, c\},\{b, d\},\{c, d\},\{b, c, d\}.
				\\vi.	sgb -closed sets in (X,$ \tau_{1}$) are X,  $\phi$,\{b\},\{c\},\{d\},\{b, c\},\{b, d\},\{c, d\},\{b, c, d\}.
				\\vii.	rg*b- closed sets in (X,$ \tau_{1}$) are X, $\phi$ ,\{b\},\{c\},\{d\},\{b, c\},\{b, d\},\{c, d\},\{b, c, d\}.
				\\viii.	pgpr- closed sets in (X,$ \tau_{1}$) are X,$\phi$  ,\{b\},\{c\},\{d\},\{b, c\},\{b, d\},\{c, d\},\{b, c, d\}.
				\\ix.	g$\alpha$b- closed sets in (X,$ \tau_{1}$) are X, $\phi$ ,{b},{c},{d},{b, c},{b, d},{c, d},{b, c, d}.
				\\x.	rps- closed sets in (X,$ \tau_{1}$) are X,  $\phi$,{b},{c},{d},{b, c},{b, d},{c, d},{b, c, d}.
				\\ and also
				\\i. closed sets in (X,  $ \tau_{2}$) are X,$\phi$,\{c, d\},\{a, b\}.
				\\ii.	ws-closed sets in (X,$ \tau_{2}$) are X,$\phi$,\{a, b\},\{c, d\}.
				\\iii.	g-closed sets in (X,$ \tau_{2}$) are X, $\phi$,\{a\},\{b\},\{c\},\{d\},\{a,b\},\{a,c\},\{a,d\},\{b,c\},\{b,d\},\{c,d\},\{a,b,c\},
				\{a,b,d\},\{a,c,d\},\{b,c,d\}.
				\\iv.	sg- closed sets in (X,$ \tau_{2}$) are X,$\phi$,\{a\},\{b\},\{c\},\{d\},\{a,b\},\{a,c\},\{a,d}\,\{b,c\},\{b,d\},\{c,d\},\{a,b,c\},\{a,b,d\},\{a,c,d\},\{b,c,d\}.
			\\v.	g$\alpha$- closed sets in  (X, $ \tau_{2}$) are X, $\phi$,\{a\},\{b\},\{c\},\{d\},\{a,b\},\{a,c\},\{a,d\},\{b,c\}, \{b,d\},\{c,d\}, \{a,b,c\},\{a,b,d\},\{a,c,d\},\{b,c,d\}.
			\\vi.	sgb -closed sets in  (X, $ \tau_{2}$) are X, $\phi$,\{a\},\{b\},\{c\},\{d\},\{a,b\},\{a,c\},\{a,d\},\{b,c\},\{b,d\},\{c,d\},\{a,b,c\},\{a,b,d\},\{a,c,d\},\{b,c,d\}.
			\\vii.	rg*b- closed sets in  (X, $ \tau_{2}$) are X, $\phi$,\{a\},\{b\},\{c\},\{d\},\{a,b\},\{a,c\},\{a,d\},\{b,c\},\{b,d\},\{c,d\},\{a,b,c\},\{a,b,d\},\{a,c, d\},\{b,c,d\}.
			\\viii.	pgpr- closed sets in  (X, $ \tau_{2}$) are X, $\phi$,\{a\},\{b\},\{c\},\{d\},\{a,b\},\{a,c\},\{a,d\},\{b,c\},\{b,d\},\{c,d\},\{a,b,c\},\{a,b,d\},\{a,c,d\},\{b,c,d\}.
			\\ix.	g$\alpha$b- closed sets in  (X, $ \tau_{2}$) are X,$\phi$,\{a\},\{b\},\{c\},\{d\},\{a,b\},\{a,c\},\{a,d\},\{b,c\},\{b,d\},\{c,d\}, \{a,b,c\},\{a,b,d\},\{a,c,d\},\{b,c,d\}.
			\\x.	rps- closed sets in  (X, $ \tau_{2}$) are X,$\phi$ ,{a},{b},{c}{d},{a, b},{a, c},{a, d},{b, c},{b, d}, {c, d}, {a, b, c},{a, b, d},{a, c, d},{b, c, d}.
			\\\indent Therefore \{a\} is ws-closed set in (X, $ \tau_{1}$) but not g-closed (resp. sg-closed, g$\alpha$-closed, sgb-closed sets, rg*b-closed, pgpr-closed, g$\alpha$b-closed,rps-closed) set in (X,  $ \tau_{1}$).
			\\\indent Meanwhile \{a\} is g-closed (resp.  sg-closed, g$\alpha$-closed, sgb-closed, rg*b-closed, pgpr-closed, g$\alpha$b-closed, rps-closed) set in (X,  $ \tau_{2}$) but not ws-closed set in (X, $ \tau_{2}$).
			\\\textbf{Remark 3.20:}The following example shows that ws-closed sets are independent of R*-closed[15]sets, rg$\beta$- closed[26]sets, pgr$\alpha$-closed [5]sets, rgw-closed[29]sets and gprw-closed[30]sets.
			\\\textbf{Example 3.21:}Let X =\{a,b,c\},  $\tau$ =\{X,$\phi$,\{a\},\{b\},\{a,b\}\}. Then
			\\i.closed sets in(X,$\tau$) are X,$\phi$,\{c\},\{a,c\},\{b,c\}.
			\\ii.ws-closed sets in(X, $\tau$) are X, $\phi$,\{a\},\{b\},\{c\},\{b,c\},\{a,c\}.
			\\iii.R* -closed sets in(X,$\tau$) are X,$\phi$,\{c\},\{a,b\},\{b,c\},\{a,c\}.
			\\iv.rg$\beta$ -closed sets in(X, $\tau$) are X,$\phi$,\{c\},\{a,b\},\{b,c\},\{a,c\}.
			\\v.pgr$\alpha$- closed sets in(X,$\tau$) are X,$\phi$,\{c\},\{a,b\},{\b,c\},\{a,c\}.
				\\vi.rgw- closed sets in(X, $\tau$ ) are X, $\phi$,\{c\},\{a,b\},\{b,c\},\{a,c\}.
				\\vii.gprw -closed sets in(X, $\tau$ ) are X,$\phi$,\{c\},\{a,b\},\{b,c\},\{a,c\}.
				\\\indent Therefore {a} is ws-closed set in X but not R*-closed (resp. rg$\beta$- closed, pgr$\alpha$-closed, rgw-closed, gprw-closed) set in X.
				\\\textbf{Remark 3.22:} From the above discussion and results,  we have the following implications.	
				\begin{center}
					\includegraphics[width=8in]{ws-closedsetpicture.jpg}
					\\Fig. 1
				\end{center}
				\textbf{Theorem 3.23:} The intersection of two ws-closed subsets of X is ws-closed set in X.
				\\\textbf{Proof:} Let A and B be are ws-closed sets in X. Let U be any semiopen set in X such that (A$\cap$B)$\subseteq$ U that is A $\subseteq$U and B $\subseteq$ U. Since A and B are ws-closed sets then scl(A)  $\subseteq$ U and scl(B)$\subseteq$ U and we know that (scl(A) $\cap$scl(B)) = scl(A $\cap$B)$\subseteq$ U. Therefore A$\cap$ B is ws-closed set in X.
				\\\textbf{Remark 3.24:} The union of two ws-closed sets in X is generally not a ws-closed set in X.
				\\\textbf{Example 3.25:} Let X = \{a, b, c\} and  $\tau$ = \{ X,$\phi$, \{a\}, \{b\},\{a, b\}\} then the sets A=\{a\} and B=\{b\}are ws-closed sets in X but A$\cup$ B =\{a,b \} is not a  ws-closed set in X.
				\\\textbf{Theorem 3.26:} If a subset A of a topological space X is ws-closed set in X then scl(A)-A does not contain any non-empty open set in X but converse is not true.
				\\\textbf{Proof:} Let A is an ws-closed set in X and suppose F be an non empty w-closed subset of scl(A)-A.
				\\F$\subseteq$ scl(A)-A $\Longrightarrow$F $\subseteq$scl(A)$\cap$ (X-A) $\Longrightarrow$ F $\subseteq$scl(A) $\longrightarrow$(1) and F $\subseteq$ X-A$\Longrightarrow$
				A$\subseteq$ X-F and X-F is w-open set and A is a ws-closed set, scl(A)$\subseteq$ X-F,
				F $\subseteq$X-scl(A) $\longrightarrow$ (2)
				\\ from equations (1) and (2) we get F$\subseteq$ scl(A)$\cap$ (X-scl(A))=$\emptyset$
				\\$\Longrightarrow$ F=$\emptyset$, thus scl(A)-A does not contain any non-empty w-closed set in X.
				\\\textbf{Remark 3.27:} The converse of the above Theorem need not be true as seen from the following Example 3.28.
				\\\textbf{Example 3.28:} Let X=\{a,b,c,d\}, $\tau$ = \{X,$\phi$,\{a\},\{a,b\},\{a,b,c\}\} then the set A=\{b\}, scl\{b\}=\{b\}, scl\{A\}-A=\{b\} does not contain any non-empty w-closed set in X but A is not ws-closed set.
				\\\textbf{Theorem 3.29:} If A is a  ws-closed set in X and A$\subseteq$ B$\subseteq$ scl(A) then B is also ws-closed set in X.
				\\\textbf{Proof:}Let A be a  ws-closed set in X such that B$\subseteq$ scl(A). Let U be a w-open set of X such that B$\subseteq$ U then A$\subseteq$ U. Since A is ws-closed set, we have scl(A)$\subseteq$ U and A$\subseteq$ U. Now B$\subseteq$ scl(A)$\Longrightarrow$   scl(B)$\subseteq$ scl(scl(A))=scl(A)$\subseteq$ U. That is scl(B)$\subseteq$ U. Therefore B is a  ws-closed set in X.
				\\\textbf{Remark 3.30:} The converse of the above Theorem 3.29 is need not be true as seen from the following Example 3.31.
				\\\textbf{Example 3.31:} Let X = \{a, b, c \},   $\tau$= \{ X,$\phi$, \{a\}, \{b\}, \{a, b\}\}, then the set A=\{a\}, B=\{a,c\} such that A and B are  ws-closed sets in X but A$\subseteq$ B$\subseteq$ scl(A)  because scl(A)=\{a\}.
				\\\textbf{Theorem 3.32:} Let (X, $\tau$) be a topological space then for each x $\varepsilon$ X the set X-{x} is ws-closed or semi open.
				\\\textbf{Proof:} Let x$\varepsilon$ X. Suppose X-{x} is not a semiopen set. Then X is the only  semiopen set containing X-{x}, that is X-{x}$\subseteq$  X $\Longrightarrow$  cl(X-{x})$\subseteq$  cl(X)$\Longrightarrow$   cl(X-{x})$\subseteq$ X. Therefore X-{x} is ws-closed set in X.
				\\\textbf{Theorem 3.33:} Let X and Y are topological spces and  A$\subseteq$ Y$\subseteq$ X . Suppose that A is ws-closed set in X then A is  ws-closed relative to Y.
				\\\textbf{Proof:} Let A$\subseteq$ Y$\cap$ G, where G is a w-open. Since A is a  ws-closed set in X, then A$\subseteq$ G and scl(A)$\subseteq$ G. This implies that Y$\cap$ scl(A)$\subseteq$  Y$\cap$ G where  Y $\cap$ scl(A) is closed set of A in Y. Thus A is a ws-closed relative to Y.
				\\\textbf{Theorem 3.34:} In a topological space X if SO(X) =\{X, $\tau$\} then every subset of X is a ws-closed set.
				\\\textbf{Proof:} Let X be a topological space and SO(X) = \{X,$\tau$ \}. Let A be any subset of X. Suppose A= $\phi$. Then  $\phi$   is ws-closed set. Suppose A$\neq$ $\phi$.  Then X is the only semiopen set containing A and so scl(A)$\subseteq$ X. Hence A is a ws-closed set in X.
				\\\textbf{Remark 3.35:} The converse of the above Theorem need not be true in general as seen from the following Example 3.36..
				\\\textbf{Example 3.36:} Let X = \{a, b, c\},  $\tau$ = \{X,$\phi$,\{a\},\{b,c\}\}. Then every subset of (X, $\tau$) is a  ws-closed set in X but SO=\{X,$\phi$, \{a\},\{b,c\}\}.
				\\\textbf{Theorem 3.37:} If A is regular open and gspr-closed set in X then A is ws-closed set in X.
				\\\textbf{Proof:} Let A be a regular open and gspr-closed set  in X. Let U be any w-open set in X such that A$\subseteq$ U. Since A is regular open and gspr-closed set in X, by definition, scl(A)$\subseteq$ A then scl(A)$\subseteq$ A$\subseteq$ U. Hence A is  ws-closed set in X.
				\\\textbf{Theorem 3.38:} If A is regular open and rgb-closed set then A is ws-closed set in X.
				\\\textbf{Proof:} Let A be a regular open and rgb-closed in X. Let U be any w-open set in X such that A U. Since A is regular open and rgb-closed in X, by definition, scl(A)$\subseteq$ A then scl(A)$\subseteq$ A$\subseteq$ U. Hence A is ws-closed set in X.
				\\\textbf{Theorem 3.39:} If A is semiopen and swg*-closed then A is  ws-closed set in X.
				\\\textbf{Proof:} Let A be a semiopen and swg*-closed in X. Let U be any w- open set in X such that A$\subseteq$ U. Since A is semiopen and swg*-closed in X, by definition, scl(A)$\subseteq$ A then scl(A)$\subseteq$ A$\subseteq$ U. Hence A is  ws-closed set in X.
				\\\textbf{Theorem 3.40:} If A is semiopen and swg-closed then A is  ws-closed set in X.
				\\\textbf{Proof:} Let A be a semiopen and swg-closed in X. Let U be any w- open set in X such that A$\subseteq$ U. Since A is semiopen and swg-closed in X, by definition, scl(A)$\subseteq$ A then scl(A)$\subseteq$ A$\subseteq$ U. Hence A is  ws-closed set in X.
				\\\textbf{Theorem 3.41:} If A is semiopen and sg-closed then A is ws-closed set in X.
				\\\textbf{Proof:} Let A be a semiopen and sg-closed in X. Let U be any w- open set in X such that A$\subseteq$ U. Since A is semiopen and sg-closed in X, by definition, scl(A)$\subseteq$ A then scl(A)$\subseteq$ A$\subseteq$ U. Hence A is ws-closed set in X.
				\\\textbf{Theorem 3.42:} If A is semiopen and sgb-closed then A is ws-closed set in X.
				\\\textbf{Proof:} Let A be a semiopen and sgb-closed in X. Let U be any w- open set in X such that A U. Since A is semiopen and sgb-closed in X, by definition, scl(A)$\subseteq$ A then scl(A)$\subseteq$ A$\subseteq$ U. Hence A is ws-closed set in X.
				\\\textbf{Theorem 3.43:} If A is semiopen and$\alpha$gs-closed then A is  ws-closed set in X.
				\\\textbf{Proof:} Let A be a semiopen and $\alpha$gs -closed in X. Let U be any w- open set in X such that A$\subseteq$ U. Since A is semiopen and $\alpha$gs -closed in X, by definition, scl(A)$\subseteq$ A then scl(A)$\subseteq$ A$\subseteq$ U. Hence A is ws-closed set in X.
				\\\textbf{Theorem 3.44:} If A is $\beta$-open and $\beta$wg*-closed then A is ws-closed set in X.
				\\\textbf{Proof:} Let A be a  $\beta$-open and $\beta$wg*-closed in X. Let U be any regular semiopen set in X such that A$\subseteq$ U. Since A is $\beta$-open and $\beta$wg*-closed in X, by definition, scl(A)$\subseteq$ A then scl(A)$\subseteq$ A$\subseteq$ U. Hence A is ws-closed set in X.
				\\\textbf{Theorem 3.45:} If A is both open and g-closed then A is ws-closed set in X.
				\\\textbf{Proof:} Let A be open and g-closed set in X. Let U be any regular open set in X such that A$\subseteq$ U. By definition, cl(A)$\subseteq$ A$\subseteq$ U and scl(A)=A. This implies that cl(A)$\subseteq$ scl(A)$\subseteq$  A $\subseteq$U $\subseteq$scl(A)$\subseteq$ U. Hence A is ws-closed set.
				\\\textbf{Theorem 3.46:} If A is regular semiopen and rw-closed then A is ws-closed set in X.
				\\\textbf{Proof:} Let A be a regular semiopen and rw-closed set in X. Let U be any w-open set in X such that A$\subseteq$ U. Now A$\subseteq$ A by hypothesis cl(A)$\subseteq$ A then we know that cl(A)$\subseteq$  scl(A) A. Hence scl(A)$\subseteq$ U therefore A is  ws-closed set in X.
				\\\textbf{Theorem 3.47:} If A is regular semiopen and R*-closed then A is ws-closed set in X.
				\\\textbf{Proof:} Let A be a regular semiopen and R*-closed set in X. Let U be any w-open set in X such that A$\subseteq$ U. Now A$\subseteq$ A by hypothesis cl(A)$\subseteq$ A then we know that cl(A)$\subseteq$  scl(A)$\subseteq$  A. Hence scl(A)$\subseteq$ U therefore A is  ws-closed set in X.
				\\\textbf{Theorem 3.48:} If A is regular semiopen and gprw-closed then A is ws-closed set in X.
				\\\textbf{Proof:} Let A be a regular semiopen and gprw -closed set in X. Let U be any w-open set in X such that A$\subseteq$ U. Now A$\subseteq$ A by hypothesis cl(A)$\subseteq$ A then we know that cl(A)$\subseteq$  scl(A)$\subseteq$  A. Hence scl(A)$\subseteq$ U therefore A is  ws-closed set in X.
				\\\textbf{Theorem 3.49:} If A is regular semiopen and rgw-closed then A is ws-closed set in X.
				\\\textbf{Proof:} Let A be a regular semiopen and rgw -closed set in X. Let U be any w-open set in X such that A$\subseteq$ U. Now A$\subseteq$ A by hypothesis cl(A)$\subseteq$ A then we know that cl(A)$\subseteq$  scl(A)$\subseteq$  A. Hence scl(A)$\subseteq$ U therefore A is  ws-closed set in X.

			\begin{center}
				\includegraphics[width=3in]{ws-closed set picture.jpg}
				\\Fig. 1
			\end{center}
			

%------------------------------------------------------------------------
