\chapter{Citing References}
\graphicspath{{Chapter5/Chapter5Figs/EPS/}{Chapter5/Chapter5Figs/}}

%
Citations are an important part of every document. Tables, graphs, figures, images have already been cited using the $backslash$ref command. It is also required to quote works found in literature. This is done using the $backslash$cite command. References to be cited fall into many categories - articles, conference/workshop proceedings, technical reports, conference papers, books, thesis and other miscellneous documents. Citing these works is the same but the difference lies in creating the key to the citation which describes the nature of the document. 

\section{Citing references from the .bib file}

Minsky \& Papert \cite{Minsky-Papert:Neuron} showed that perceptrons cannot solve a simple XOR problem and hence have severe limitations. \\

After a longtime following Minsky's paper \cite{Minsky-Papert:Neuron}, Rumelhart and others \cite{Rumelhart-Hinton-Williams:MLP} proposed the famous back propagation algorithm to train MLPs. \\ 

Moody \& Darken \cite{Moody-Darken:RBF} in 1989 introduced Radial Basis Function (RBF) networks and showed that they are better alternatives to MLP. \\

Simon Haykin \cite{Haykin:NN} is one of the standard books in the study of Neural Networks. \\

Benchmark datasets to test classification algorithms are available at the UCI machine learning repository \cite{UCI:Database}.  \\

Auto-adjustable method for Gaussian width optimization on RBF neural network, Application to face authentication on a mono-chip system \cite{Lionel-Jacques-Cecile:RBF} is an paper presented in the conference IECON 2006.  \\

Papers \cite{Slade-Gedeon:AL} which appear in proceedings that are published can also be cited.  \\

Entries like \cite{Orr:RBF} which fit into the miscellaneous category of references could also be quoted. \\

Previous work in the form of thesis also find place in references \cite{Simon:LA}.  \\

%------------------------------------------------------------------------

