
\chapter{ಜಗತ್ತಿಗೆ ಮಹಮ್ಮದನ ಸಂದೇಶ}

(೧೯೦೦ರ ಮಾರ್ಚ್ ೨೫ರಂದು ಸ್ಯಾನ್ಫ್ರಾಂಸಿಸ್ಕೊದಲ್ಲಿ ನೀಡಿದ ಉಪನ್ಯಾಸದ ಟಿಪ್ಪಣಿ - ಇಡಾ ಆಂಸೆಲ್ಳಿಂದ ತೆಗೆದುಕೊಂಡುದು.)

ಪ್ರತಿಯೊಬ್ಬ ಶ್ರೇಷ್ಠ ಪ್ರವಾದಿಯೂ ಹೊಸ ಪಂಥವನ್ನು ಸ್ಥಾಪಿಸುವುದು ಮಾತ್ರ ವಲ್ಲದೆ, ಅವನೇ ಒಂದು ಹೊಸ ಪಂಥದ ಸೃಷ್ಟಿಯಾಗಿರುತ್ತಾನೆ. ಸ್ವತಂತ್ರ ಕಾರಣ ಎಂಬುದಿಲ್ಲ. ಪ್ರತಿಯೊಂದು ಕಾರಣವೂ ಒಮ್ಮೆಯೇ ಕಾರಣ ಮತ್ತು ಕಾರ್ಯ ಎರಡೂ ಆಗಿರುತ್ತದೆ. ತಂದೆಯು ತಂದೆ ಮತ್ತು ಮಗ ಎರಡೂ ಆಗಿರುತ್ತಾನೆ. ತಾಯಿಯು ತಾಯಿ ಮತ್ತು ಮಗಳು ಎರಡೂ ಆಗಿರುತ್ತಾಳೆ. ಶ್ರೇಷ್ಠ ಪ್ರವಾದಿಗಳು ಹುಟ್ಟಿದ ಪರಿಸರ ಮತ್ತು ಪರಿಸ್ಥಿತಿ ಎರಡನ್ನೂ ತಿಳಿದುಕೊಳ್ಳುವ ಆವಶ್ಯಕತೆಯಿದೆ....

ಇದೇ ನಾಗರಿಕತೆಯ ವೈಶಿಷ್ಟ್ಯ. ಒಂದು ಜನಾಂಗದ ಅಲೆಯು ತನ್ನ ಮೂಲ ಸ್ಥಾನ ದಿಂದ ಬೇರೆ ಸ್ಥಳಗಳಿಗೆ ವಲಸೆ ಹೋಗಿ ಅಲ್ಲಿ ಅದ್ಭುತ ನಾಗರಿಕತೆಯನ್ನು ನಿರ್ಮಿಸು ತ್ತದೆ. ಉಳಿದವರು ಅನಾಗರಿಕರಾಗಿ ಉಳಿಯುತ್ತಾರೆ. ಹಿಂದೂಗಳು ಭಾರತಕ್ಕೆ ವಲಸೆ ಬಂದರು, ಅವರ ಮೂಲಸ್ಥಾನವಾದ ಮಧ್ಯ ಏಷ್ಯಾವು ಅನಾಗರಿಕವಾಗಿ ಉಳಿಯಿತು. ಉಳಿದವರು ಏಷ್ಯ ಮೈನರ್ ಮತ್ತು ಯೂರೋಪಿಗೆ ಬಂದರು. ಯಹೂದ್ಯರು ಈಜಿಪ್ಟಿ ನಿಂದ ಹೊರಬಂದ ಸಂಗತಿಯನ್ನು ನೆನಪಿಸಿಕೊಳ್ಳಿ. ಅವರ ವಾಸಸ್ಥಾನವು ಅರೇಬಿಯಾ ಮರುಭೂಮಿಯಾಗಿತ್ತು. ಅದರಿಂದ ಹೊಸ ಸಂಸ್ಕೃತಿಯು ಪ್ರಾರಂಭವಾಯಿತು.

..... ಎಲ್ಲ ನಾಗರಿಕತೆಯೂ ಹೀಗೆಯೇ ಬೆಳೆಯುತ್ತವೆ. ಯಾವುದೊ ಒಂದು ಜನಾಂಗವು ನಾಗರಿಕವಾಗುತ್ತದೆ. ಅನಂತರ ಅಲೆಮಾರಿ ಜನಾಂಗವು ಬರುತ್ತದೆ. ಅಲೆಮಾರಿ ಜನಾಂಗದವರು ಯಾವಾಗಲೂ ಯುದ್ಧಾಕಾಂಕ್ಷಿಗಳು. ಅವರು ಬಂದು ಒಂದು ಜನಾಂಗವನ್ನು ಗೆಲ್ಲುತ್ತಾರೆ. ಅವರು ಉತ್ತಮ ರಕ್ತವನ್ನೂ ಸಬಲ ಮಾಂಸಖಂಡವನ್ನೂ ತರುತ್ತಾರೆ. ಅವರು ತಾವು ಗೆದ್ದ ಜನಾಂಗದ ಮನಸ್ಸನ್ನು ತೆಗೆದುಕೊಂಡು, ಅದನ್ನು ತಮ್ಮ ದೇಹಶಕ್ತಿ ಯೊಂದಿಗೆ ಸೇರಿಸಿ ನಾಗರಿಕತೆಯು ಮುಂದುವರಿಯುವಂತೆ ಮಾಡುತ್ತಾರೆ. ಒಂದು ಜನಾಂಗವು ದೇಹ ಕ್ಷೀಣವಾಗುವವರೆಗೂ ನಾಗರಿಕತೆ ಮತ್ತು ಸಂಸ್ಕೃತಿಯಲ್ಲಿ ಮುಂದು ವರಿಯುತ್ತದೆ. ಅನಂತರ ಭೌತಿಕವಾಗಿ ಬಲಯುತವಾಗಿರುವ ಜನಾಂಗವೊಂದು ಬಂದು, ಅವರನ್ನು ಗೆದ್ದು, ಅವರ ಕಲೆ ವಿಜ್ಞಾನಗಳನ್ನು ತಾನು ತೆಗೆದುಕೊಂಡು ನಾಗರಿ ಕತೆಯು ಮುಂದುವರಿಯುವಂತೆ ಮಾಡುತ್ತದೆ. ಇದು ಹೀಗೆಯೇ ಇರಬೇಕು. ಇಲ್ಲದಿದ್ದರೆ ಜಗತ್ತು ಉಳಿಯುವುದೇ ಇಲ್ಲ.

\delimiter

ಒಬ್ಬ ಶ್ರೇಷ್ಠ ವ್ಯಕ್ತಿಯು ಜನಿಸಿದರೆ ಜನರು ಅವನ ಸುತ್ತ ಸುಂದರ ಪುರಾಣ ಕಥೆಗಳನ್ನು ಹೆಣೆಯುತ್ತಾರೆ. ವಿಜ್ಞಾನ ಮತ್ತು ಸತ್ಯ ಇವಿಷ್ಟೇ ಧರ್ಮ. ಸತ್ಯವು ಯಾವುದೇ ಪುರಾಣಕಥೆಗಿಂತಲೂ ಹೆಚ್ಚು ಸುಂದರವಾದುದು.....

ಹಳೆಯ ಗ್ರೀಕರು ಆಗಲೇ ಕಣ್ಮರೆಯಾಗಿರುವರು. ಇಡೀ ದೇಶವು ಅವರ ಕಲೆ ಮತ್ತು ವಿಜ್ಞಾನವನ್ನು ಕಲಿತ ರೋಮನರ ಕೈವಶವಾಗಿತ್ತು. ರೋಮನರು ಅನಾಗರಿಕ ರಾಗಿದ್ದರು, ಆಕ್ರಮಣಕಾರಿ ಜನಾಂಗವಾಗಿದ್ದರು. ಅವರು ಕಲಾಪ್ರೇಮಿಗಳಾಗಲಿ, ಸಾಹಿತ್ಯ ಪ್ರೇಮಿಗಳಾಗಲಿ ಆಗಿರಲಿಲ್ಲ. ಹೇಗೆ ಆಳಬೇಕು, ಅಧಿಕಾರ ಶಕ್ತಿಯನ್ನು ಹೇಗೆ ಕೇಂದ್ರೀ ಕರಿಸಿ ಅದನ್ನು ಅನುಭವಿಸಬೇಕು ಎಂಬುದಷ್ಟೇ ಅವರಿಗೆ ಗೊತ್ತಿತ್ತು. ಅದು ಅವರಿಗೆ ಪ್ರಿಯವಾಗಿತ್ತು. ಆದರೂ ರೋಮನ್ ಚಕ್ರಾಧಿಪತ್ಯವು ಕಣ್ಮರೆಯಾಯಿತು, ಎಲ್ಲ ರೀತಿಯ ವಿಲಾಸಜೀವನ, ಕಷ್ಟ ಪರಂಪರೆಗಳು ಮತ್ತು ಹೊಸಧರ್ಮದ ಧಾಳಿ - ಇವುಗಳಿಂದ ಅದು ನಾಶವಾಯಿತು. ಕ್ರೈಸ್ತಧರ್ಮವು ಆಗಲೇ ಆರು ಶತಮಾನಗಳಿಂದ ರೋಮನ್ ಚಕ್ರಾಧಿಪತ್ಯದಲ್ಲಿ ಇತ್ತು.....

ಒಂದು ಧರ್ಮವು ಇನ್ನೊಂದು ಜನಾಂಗವನ್ನು ತನ್ನ ತೆಕ್ಕೆಯೊಳಗೆ ತರಲು ಪ್ರಯತ್ನಿ ಸಿದಾಗ, ಆ ಜನಾಂಗವು ದುರ್ಬಲವಾಗಿದ್ದರೆ ಮಾತ್ರ ಅದು ಯಶಸ್ವಿಯಾಗುತ್ತದೆ. ಆ ಜನಾಂಗವು ಸುಸಂಸ್ಕೃತವಾಗಿದ್ದರೆ ಆ ಧರ್ಮವನ್ನೇ ನಾಶಮಾಡುತ್ತದೆ.... ರೋಮನ್ ಚಕ್ರಾಧಿಪತ್ಯವು ಇದಕ್ಕೆ ಒಂದು ಉದಾಹರಣೆ. ಪರ್ಶಿಯನರು ಇದನ್ನು ಪ್ರತ್ಯಕ್ಷವಾಗಿ ಕಂಡರು. ಅನಾಗರಿಕರಿದ್ದ ಉತ್ತರ ಯೂರೋಪಿನಲ್ಲಿ ಕ್ರೈಸ್ತ ಧರ್ಮದ ಸ್ವರೂಪವೇ ಬೇರೆ ಯಾಗಿತ್ತು. ಆದರೆ ರೋಮನ್ ಚಕ್ರಾಧಿಪತ್ಯದಲ್ಲಿದ್ದ ಕ್ರೈಸ್ತ ಧರ್ಮವು ಪರ್ಶಿಯ, ಜ್ಯೂಸ್, ಇಂಡಿಯ ಮತ್ತು ಗ್ರೀಸ್ - ಈ ಸಂಸ್ಕೃತಿಗಳ ಮಿಶ್ರವಾಗಿತ್ತು.

\delimiter

ಜನಾಂಗವು ಯಾವಾಗಲೂ ಯುದ್ಧದಿಂದ ನಾಶವಾಗುತ್ತದೆ. ಯುದ್ಧವು ಉತ್ತಮ ವ್ಯಕ್ತಿಗಳನ್ನು ನಾಶಮಾಡಿ ಹೇಡಿಗಳು ಮಾತ್ರ ಉಳಿಯುವಂತೆ ಮಾಡುತ್ತದೆ. ಹೀಗೆ ಜನಾಂಗದ ಅವನತಿ ಉಂಟಾಗುತ್ತದೆ..... ಜನರು ಸಣ್ಣವರಾದರು. ಏಕೆ? ಎಲ್ಲ ಶ್ರೇಷ್ಠ ವ್ಯಕ್ತಿಗಳೂ ಯೋಧರಾದರು. ಹೀಗೆ ಯುದ್ಧವು ಜನಾಂಗವನ್ನು ನಾಶಮಾಡು ತ್ತದೆ, ಅತ್ಯುತ್ತಮರನ್ನು ರಣರಂಗಕ್ಕೆ ಒಯ್ಯುತ್ತದೆ.

ಅನಂತರ ಸಂನ್ಯಾಸೀ ಮಠಗಳು. ಅವರೆಲ್ಲ ಧ್ಯಾನ ಮಾಡುವುದಕ್ಕಾಗಿ ಮರು ಭೂಮಿಗೆ, ಗುಹೆಗಳಿಗೆ ಹೋದರು. ಮಠಗಳು ಕ್ರಮೇಣ ಐಶ್ವರ್ಯ ಮತ್ತು ವಿಲಾಸ ಗಳ ಕೇಂದ್ರವಾದವು.....

ಈ ಮಠಗಳು ಇಲ್ಲದಿದ್ದಿದ್ದರೆ ಆಂಗ್ಲೋ - ಸ್ಯಾಕ್ಸನ್ ಜನಾಂಗಗಳು ಈಗಿರುವಂತೆ ಇರುತ್ತಿರಲಿಲ್ಲ. ಪ್ರತಿಯೊಬ್ಬ ದುರ್ಬಲನೂ ಗುಲಾಮನಿಗಿಂತಲೂ ಕೀಳು.

......ಆ ಅರಾಜಕ ಸ್ಥಿತಿಯಲ್ಲಿ ಈ ಮಠಗಳು ಬೆಳಕಿನ ಮತ್ತು ರಕ್ಷಣೆಯ ಕೇಂದ್ರಗಳಾಗಿದ್ದವು.

ಸಂಸ್ಕೃತಿಯಲ್ಲಿ ತುಂಬ ಭೇದವಿದ್ದರೆ ಆ ಜನಾಂಗಗಳು ಜಗಳವಾಡುವುದಿಲ್ಲ. ಈ ಎಲ್ಲ ಕದನ ಕೋಲಾಹಲಗಳಲ್ಲಿ ನಿರತವಾಗಿರುವವರೆಲ್ಲರೂ ಮೂಲತಃ ಒಂದೇ ಜನಾಂಗದವರಾಗಿದ್ದರು.

ಈ ಅರಾಜಕತೆಯ ಸ್ಥಿತಿಯಲ್ಲಿ ಪ್ರವಾದಿಯೊಬ್ಬನು ಜನಿಸಿದನು....

(ಇಲ್ಲಿಗೆ ಸ್ವಾಮೀಜಿಯವರ ಮಹಮದನ ಮೇಲಿನ ಉಪನ್ಯಾಸದ ಮೊದಲನೆಯ ಭಾಗವು ಮುಗಿಯುತ್ತದೆ. ಉಳಿದ ಭಾಗಕ್ಕಾಗಿ \enginline{"Complete works" vol 1} ನೋಡಿ)

