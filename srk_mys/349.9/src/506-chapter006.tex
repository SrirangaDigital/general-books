
\chapter{ಅಧ್ಯಾಯ ೫: ಬಾರಾಮುಲ್ಲಾದ ಹಾದಿಯಲ್ಲಿ}

ವ್ಯಕ್ತಿಗಳು: ಸ್ವಾಮಿ ವಿವೇಕಾನಂದರು, ಗುರುಭಾಯಿಗಳು, ಧೀರಮಾತಾ, ಜಯಾ ಎಂಬ ಹೆಸರಿನವಳು ಮತ್ತು ಸೋದರಿ ನಿವೇದಿತಾಳನ್ನೊಳಗೊಂಡ ಯೂರೋಪಿಯನ್ ಶಿಷ್ಯರುಗಳ ಮತ್ತು ಅತಿಥಿಗಳ ಗುಂಪು.

ಸ್ಥಳ: ಬರೇಲಿಯಿಂದ ಕಾಶ್ಮೀರದ ಬಾರಾಮುಲ್ಲಾಕ್ಕೆ.

ಕಾಲ: ೧೮೯೮ರ ಜೂನ್ ೧೪ರಿಂದ ಜೂನ್ ೨೦ರವರೆಗೆ.

\textbf{ಜೂನ್ ೧೪.}

ಮಾರನೆಯ ದಿನ ನಾವು ಪಂಜಾಬ್ನ್ನು ಪ್ರವೇಶಿಸಿದೆವು; ಸ್ವಾಮಿಗಳ ಸಂಭ್ರಮ ಹೇಳತೀರದು. ಅವರು ಅಲ್ಲೇ ಹುಟ್ಟಿದವರೇನೋ ಎಂಬಂತಿತ್ತು ಆ ಪ್ರಾಂತದ ಬಗ್ಗೆ ಅವರ ವಿಶೇಷ ಪ್ರೇಮ, ಆತ್ಮೀಯತೆ. ಚರಕದಿಂದ “ಸೋಣಿಹಮ್​! ಸೋಣಿಹಮ್​!” ಶಬ್ದವನ್ನು ಕೇಳುತ್ತ ನೂಲು ತೆಗೆಯುವ ಅಲ್ಲಿಯ ಹುಡುಗಿಯರನ್ನು ಕುರಿತು ಮಾತನಾಡಿದರು. ಅನಂತರ, ಇದ್ದಕ್ಕಿದ್ದಂತೆ ಪ್ರಾಚೀನ ಕಾಲಕ್ಕೆ ಸ್ಥಿತ್ಯಂತರಗೊಂಡು, ಇಂಡಸ್ ನದಿಯ ಈ ನಾಡಿನ ಮೇಲೆ ಗ್ರೀಕರು ದಂಡೆತ್ತಿ ಬಂದುದು, ಚಂದ್ರಗುಪ್ತನ ಉಗಮ, ಬೌದ್ಧಸಾಮ್ರಾಜ್ಯದ ಸ್ಥಾಪನೆ ಇತ್ಯಾದಿಯಾಗಿ ಚರಿತ್ರೆಯ ವಿಹಂಗಮ ದೃಶ್ಯಗಳನ್ನು ನಮ್ಮ ಬಗೆಗಣ್ಣ ಮುಂದೆ ಚಿತ್ರಿಸತೊಡಗಿದರು. ಈ ಬೇಸಿಗೆಯಲ್ಲಿ ನಾವು ಹೇಗಾದರೂ ಮಾಡಿ ಅಟ್ಟಾಕ್ ತಲುಪಿ, ಅಲೆಗ್ಸಾಂಡರ್ನನ್ನು ಹಿಂದಿರುಗುವಂತೆ ಮಾಡಿದ ಜಾಗವನ್ನು ತಮ್ಮ ಕಣ್ಣುಗಳಿಂದಲೇ ನೋಡಬೇಕು ಎಂದು ಅವರು ಸಂಕಲ್ಪಿಸಿದ್ದರು. ಗಾಂಧಾರ ಶಿಲ್ಪಕಲಾವೈಭವವನ್ನು ನಮಗೆ ಬಣ್ಣಿಸತೊಡಗಿದರು - ಬಹುಶಃ ಹಿಂದಿನ ವರ್ಷ ಲಾಹೋರ್ ವಸ್ತುಸಂಗ್ರಹಾಲಯದಲ್ಲಿ ನೋಡಿದ್ದರೆಂದು ಕಾಣುತ್ತದೆ. ಮತ್ತು ಭಾರತವು ಕಲೆಯ ವಿಚಾರದಲ್ಲಿ ಎಂದೆಂದಿಗೂ ಗ್ರೀಸ್ನ ಪದತಲದಲ್ಲಿ ಕುಳಿತಿರುವುದೆಂಬ ಅಸಂಬದ್ಧ ಯೂರೋಪಿಯನ್ ಅಭಿಪ್ರಾಯವನ್ನು ಸಾತ್ವಿಕ ಕ್ರೋಧಯುಕ್ತರಾಗಿ ಖಂಡಿಸುವುದರಲ್ಲಿ ಮಗ್ನರಾದರು.

ಅನಂತರ, ಬಹುನಿರೀಕ್ಷಿತವಾದ ನಗರಗಳ ಪಕ್ಷಿನೋಟ - ತಮ್ಮ ಕೆಲವು ನೆಚ್ಚಿನ ಆಂಗ್ಲ ಶಿಷ್ಯರು ಬಾಲ್ಯದಲ್ಲಿ ವಾಸವಾಗಿದ್ದ ಲೂಧಿಯಾನಾ; ಭಾರತದಲ್ಲಿಯ ತಮ್ಮ ಉಪನ್ಯಾಸ ಮಾಲೆ ಕೊನೆಗೊಂಡ ಲಾಹೋರ್ - ಇತ್ಯಾದಿಯಾಗಿ. ಇಂಡಸ್ ನದಿಯ ಅನೇಕ ಸೀಳುಗಳ ನಡುವಣ ಶುಷ್ಕ ಮರಳನ್ನು ಹಾದು ಹೋಗುತ್ತಿರುವಾಗ, ಎರಡು ಸೀಳುಗಳ ನಡುವಣ ಜಾಗವನ್ನು ದೋಬ್ ಎನ್ನುವರು, ಎಲ್ಲ ಸೀಳುಗಳ ನಡುವಣ ಒಟ್ಟು ಜಾಗವನ್ನು ಪಂಜಾಬ್ ಎನ್ನುವರು ಎಂದು ನಮಗೆ ವಿವರಿಸಿದರು.

ಒಂದು ದಿನ ಸಂಜೆಗತ್ತಲು. ಇಂಥದೊಂದು ಕಡಿದಾದ ಶಿಲಾಹಾದಿಯನ್ನು ನಾವು ದಾಟುತ್ತಿದ್ದೆವು. ಆಗ ಅವರು - ವರ್ಷಗಳ ಹಿಂದೆ ತಾವಿನ್ನೂ ಸಂನ್ಯಾಸಿ ಜೀವನಕ್ಕೆ ಹೊಸದಾಗಿ ಕಾಲಿರಿಸುತ್ತಿದ್ದಾಗ ಪ್ರಾಚೀನಕಾಲದ ಸಂಸ್ಕೃತ ವೇದಪಠನಕ್ರಮವು ಹೇಗಿದ್ದಿರಬಹು ದೆಂಬ ಕಲ್ಪನೆ ತಮಗೆ ದರ್ಶನವೊಂದರ ಮೂಲಕ ಹೇಗೆ ಪ್ರಾಪ್ತವಾಯಿತು - ಎಂಬುದನ್ನು ಕುರಿತು ಹೇಳಿದರು.

“ಆರ್ಯರು ಇನ್ನೂ ಇಂಡಸ್ ನದೀತೀರಕ್ಕೆ ಬಂದು ತಲುಪಿದ್ದ ಆ ಕಾಲದ ಒಂದು ಸಂಜೆ. ಆ ಮಹಾ ನದಿಯ ದಡದ ಮೇಲೆ ಕುಳಿತಿದ್ದ ವೃದ್ಧನೊಬ್ಬನನ್ನು ನಾನು ಕಂಡೆ. ಅವನು ಋಗ್ವೇದವನ್ನು ರಾಗವಾಗಿ ಪಠಿಸುತ್ತಿದ್ದಂತೆ ಕತ್ತಲೆಯ ಅಲೆಗಳು ಒಂದೊಂದಾಗಿ ಬಂದು ಅವನನ್ನು ಆವರಿಸಿಕೊಳ್ಳತೊಡಗಿದವು. ಆಗ ನಾನು ಎಚ್ಚರಗೊಂಡೆ; ವೇದಪಠನವನ್ನು ಮುಂದುವರೆಸಿದೆ. ಪ್ರಾಚೀನ ಕಾಲದಲ್ಲಿ ನಾವು ಬಳಸುತ್ತಿದ್ದ ವೇದಗಾಯನದ ಕ್ರಮವೇ ಆದಾಗಿರಬೇಕು”.

...ತಮ್ಮದೇ ಭೂತಕಾಲದ ನೆರಳುಗಳಲ್ಲಿಯೇ ಯಾವಾಗಲೂ ಮುಳುಗಿಹೋಗಿರುವವರನ್ನು ಕಂಡರೆ ಅವರು ಸಾಮಾನ್ಯವಾಗಿ ಸಿಟ್ಟಿಗೇಳುತ್ತಿದ್ದರು. ಆದರೆ ಕಥೆಯನ್ನು ಹೇಳುತ್ತಿರುವ ಈ ಸಂದರ್ಭದಲ್ಲಿ, ಬೇರೆಯೇ ದೃಷ್ಟಿಕೋನದಿಂದ ಅದರದೊಂದು ಮಿನುಗುನೋಟವನ್ನು ನಮಗೆ ಆಗಮಾಡಿಕೊಟ್ಟರು.

“ನಿಜಕ್ಕೂ ಶಂಕರಾಚಾರ್ಯರಿಗೆ ವೇದೋಚ್ಚಾರಣೆಯ ದೇಶೀಯ ಸ್ವರಗಳ - ಜ್ಞಾನವಿದ್ದಿತೆನ್ನುವುದು ನನ್ನ ಕಲ್ಪನೆ”, ಎಂದರವರು. ಇದ್ದಕ್ಕಿದ್ದಂತೆ, ದಿಗಂತದಲ್ಲಿ ಏನನ್ನೋ ನೋಡುತ್ತಿರುವವರ ಹಾಗೆ, ಅವರೂ ನನ್ನಂತೆಯೇ ದರ್ಶನವೊಂದನ್ನು ಕಂಡಿರಬೇಕು, ಮತ್ತು ಆ ಮೂಲಕ ಆ ಪುರಾತನ ವೇದಗಾಯನಕ್ರಮವನ್ನು ಕಲಿತು ಅದನ್ನು ಪುನರು ಜ್ಜೀವನಗೊಳಿಸಿರಬೇಕು - ಎಂದು ನಾನು ಯಾವಾಗಲೂ ಕಲ್ಪಿಸಿಕೊಳ್ಳುತ್ತೇನೆ. ಏನೇ ಆಗಲಿ, ಅವರ ಇಡೀ ಜೀವಮಾನದ ಕಾರ್ಯವೇ ಅದಾಗಿತ್ತು - ವೇದೋಪನಿಷತ್ತುಗಳ ದಿವ್ಯ ಸೌಂದರ್ಯವನ್ನು ಪುಟಪುಟಿಸುವುದು”...

ರಾವಲ್ಪಿಂಡಿಯಿಂದ ಮುರ್ರೀವರೆಗೆ ನಾವು ಟಾಂಗಾದಲ್ಲಿ ಹೋದೆವು; ಕಾಶ್ಮೀರಕ್ಕೆ ಹೊರಡುವ ಮುನ್ನ ಅಲ್ಲಿ ಕೆಲವು ದಿನಗಳನ್ನು ಕಳೆದೆವು. ಯೂರೋಪಿಯನ್ ಒಬ್ಬನನ್ನು ಶಿಷ್ಯನನ್ನಾಗಿ ತೆಗೆದುಕೊಳ್ಳುವುದಕ್ಕಾಗಿ ಸಂಪ್ರದಾಯಸ್ಥರ ಮನವೊಲಿಸುವ ಪ್ರಯತ್ನ ಮಾಡುವುದು, ಸ್ತ್ರೀವಿದ್ಯಾಭ್ಯಾಸದ ದಿಕ್ಕಿನಲ್ಲಿ ಏನನ್ನಾದರೂ ಮಾಡುವುದು - ಮುಂತಾದವುಗಳನ್ನು ಬಂಗಾಳದಲ್ಲಿ ಪ್ರಾರಂಭಿಸುವುದೊಳ್ಳೆಯದು ಎಂಬ ತೀರ್ಮಾನಕ್ಕೆ ಇಲ್ಲಿ ಅವರು ಬಂದರು. ಅನ್ಯದೇಶೀಯರ ಮೇಲಣ ಅಪನಂಬಿಕೆ ಪಂಜಾಬ್ನಲ್ಲಿ ಬಹಳ ಹೆಚ್ಚಾಗಿರುವುದರಿಂದ ಅಂತಹ ಕೆಲಸ ಇಲ್ಲಿ ಯಶಸ್ವಿಯಾಗುವುದು ಕಷ್ಟ. ಆಗಿಂದಾಗ್ಯೆ ಈ ಪ್ರಶ್ನೆಯನ್ನೇ ಕುರಿತು ಅವರು ಯೋಚಿಸುತ್ತಿದ್ದರು; ಕೆಲವೊಮ್ಮೆ ಆಂಗ್ಲರನ್ನು ಕುರಿತಾದ ರಾಜಕೀಯ ವಿರೋಧ ಮತ್ತು ಅದೇ ಸಮಯಕ್ಕೆ ಅವರನ್ನು ನಂಬಿ ಪ್ರೀತಿಸುವ ಮನೋ ಭಾವ - ಇವುಗಳು ಒಟ್ಟಿಗೇ ಬಂಗಾಳದಲ್ಲಿರುವ ವಿರೋಧಾಭಾಸವನ್ನು ಕುರಿತು ಪಡಿ ಪ್ರತಿಕ್ರಿಯಿಸುತ್ತಿದ್ದರು...

\textbf{ಜೂನ್ ೧೮.}

ಮಧ್ಯಾಹ್ನವೆಲ್ಲಾ ಬಿರುಗಾಳಿಯಿಂದಾಗಿ ನಾವು ಒಳಗೇ ಕುಳಿತಿರಬೇಕಾಯಿತು; ಹಾಗೂ ದುಲಾಯ್​ನಲ್ಲಿ ಹಿಂದೂಧರ್ಮದ ಬಗೆಗಿನ ನಮ್ಮ ತಿಳುವಳಿಕೆಯಲ್ಲಿ ಒಂದು ಹೊಸ ಅಧ್ಯಾಯವೇ ಪ್ರಾರಂಭವಾದಂತಾಯಿತು. ಏಕೆಂದರೆ ಸ್ವಾಮಿಗಳು ಭಾರವಾದ ಮನಸ್ಸಿನಿಂದ, ಆದರೆ ನೇರವಾಗಿ, ಅದರ ಇಂದಿನ ಕೆಡಕುಗಳನ್ನು ವಿವರಿಸಿದರು; ಅಲ್ಲದೆ ತಮಗೆ ವಾಮಾಚಾರವೆಂಬ ಹೆಸರಿನಿಂದ ಹಿಂದೂಧರ್ಮದೊಳಗೆ ಸೇರಿಹೋಗಿರುವ ಕೆಟ್ಟ ಆಚರಣೆಗಳ ಬಗ್ಗೆ ಎಂದಿಗೂ ರಾಜಿಮಾಡಿಕೊಳ್ಳಲು ಆಗದಂತಹ ದ್ವೇಷವಿರುವುದನ್ನೂ ತಿಳಿಸಿದರು. ಯಾರೊಬ್ಬರ ಭರವಸೆಯನ್ನೂ ಕೆಡಿಸುವುದನ್ನು ಸಹಿಸಲಾರದ ಶ‍್ರೀರಾಮಕೃಷ್ಣರು ಇಂತಹ ವಿಚಾರಗಳನ್ನು ಹೇಗೆ ಪರಿಗಣಿಸುತ್ತಿದ್ದರು ಎಂದು ಯಾರೋ ಕೇಳಿದ್ದಕ್ಕೆ ಸ್ವಾಮಿಗಳೆಂದರು: “ಆ ಮುದುಕರು ‘ಒಳ್ಳೆಯದು, ಒಳ್ಳೆಯದು! ಆದರೆ ಪ್ರತಿಯೊಂದು ಮನೆಗೂ ಝಾಡಮಾಲಿ ಬಂದು ಹೋಗುವುದಕ್ಕೊಂದು ಹಿಂಬಾಗಿಲು ಇರಬೇಕಲ್ಲ!’ ಎನ್ನು ತ್ತಿದ್ದರು; ಯಾವ ದೇಶದ ವಾಮಾಚಾರದ ಗುಂಪುಗಳಾದರೂ ಇರುವುದು ಹೀಗೆಯೇ!”

\textbf{ಜೂನ್ ೧೯}

ಸ್ವಾಮಿಗಳೊಂದಿಗೆ ಟಾಂಗಾದಲ್ಲಿ ನಾವು ಸರದಿ ಪ್ರಕಾರ ಸಾಗಿದೆವು. ಅವರ ಈ ಮರುದಿನ ನೆನಪುಗಳಿಂದಲೇ ತುಂಬಿರುವಂತೆ ತೋರಿತು.

ಅವರು ಬ್ರಹ್ಮವಿದ್ಯೆಯ ಬಗ್ಗೆ ಮಾತನಾಡಿದರು - ಏಕತ್ವದ ದರ್ಶನದ ಬಗ್ಗೆ, ಏಕೈಕ ಸತ್ಯದ ಬಗ್ಗೆ. ಕೆಡುಕೆಂಬುದಕ್ಕೆ ಹೇಗೆ ಪ್ರೇಮದ ಹೊರತು ಬೇರಾವ ಪರಿಹಾರವಿಲ್ಲ ಎಂಬುದನ್ನು ತಿಳಿಸಿದರು. ಅವರಿಗೆ ಶ‍್ರೀಮಂತನಾಗಿ ಬೆಳೆಯುತ್ತಲೇ ನಡೆದ, ಆದರೆ ತನ್ನ ಆರೋಗ್ಯವನ್ನೇ ಕಳೆದುಕೊಂಡ ಶಾಲಾಸಹಪಾಠಿಯೊಬ್ಬನಿದ್ದ. ಅದೊಂದು ನಿಗೂಢ ಖಾಯಿಲೆ - ಅವನ ಶಕ್ತಿಯನ್ನು, ಜೀವಂತಿಕೆಯನ್ನು ನಿತ್ಯವೂ ಅದು ಹೀರುತ್ತಿತ್ತು, ಆದರೂ ವೈದ್ಯರ ಪ್ರತಿಭೆಗೆ ಹೇಗೆಹೇಗೂ ನಿಲುಕದ ಒಂದು ಸವಾಲಾಗಿತ್ತು. ಸ್ವಾಮಿಗಳು ಮೊದಲಿನಿಂದಲೂ ಧರ್ಮಮಾರ್ಗದಲ್ಲಿರುವುದನ್ನು ಅರಿತಿದ್ದುದರಿಂದ, ಯಾವುದರಿಂದಲೂ ಪರಿಹಾರ ಕಾಣದಾಗ ಜನರು ಧರ್ಮದ ಕಡೆಗೆ ತಿರುಗುವ ಹಾಗೆ, ಅವನು ಅವರನ್ನು ತನ್ನ ಬಳಿಗೆ ಬರುವಂತೆ ಬೇಡಿಕೊಂಡ, ಹೇಳಿಕಳುಹಿಸಿದ. ಗುರುದೇವರು ಅವನ ಬಳಿಗೆ ಹೋಗು ತ್ತಿದ್ದಂತೆ ಅವರಿಗೊಂದು ಸ್ಮೃತಿ ಜ್ಞಾಪಕಕ್ಕೆ ಬಂದಿತಂತೆ: “ಬ್ರಾಹ್ಮಣನಿಂದ ತಾನು ಭಿನ್ನ ನೆಂದು ತಿಳಿದುಕೊಂಡವನನ್ನು ಬ್ರಾಹ್ಮಣನು ಗೆಲ್ಲುತ್ತಾನೆ. ಕ್ಷತ್ರಿಯನಿಂದ ತಾನು ಭಿನ್ನನೆಂದು ತಿಳಿದುಕೊಂಡವನನ್ನು ಕ್ಷತ್ರಿಯನು ಗೆಲ್ಲುತ್ತಾನೆ. ವಿಶ್ವದಿಂದ ತಾನು ಭಿನ್ನನೆಂದು ತಿಳಿದುಕೊಂಡವನನ್ನು ವಿಶ್ವವು ಗೆಲ್ಲುತ್ತದೆ”, ಸ್ವಾಮಿಗಳೆಂದರು, “ಆಗಾಗ ನಾನು ವಿಚಿತ್ರವಾಗಿ ಮಾತನಾಡಬಹುದು, ಕೋಪದಿಂದ ಕೂಗಾಡಬಹುದು, ಆದರೂ ಜ್ಞಾಪಕದಲ್ಲಿ ಟ್ಟುಕೊಳ್ಳಿ, ನನ್ನ ಹೃದಯದಲ್ಲಿ ಪ್ರೀತಿಯ ಹೊರತು ಬೇರೇನೂ ಇರುವುದಿಲ್ಲವೆಂದು! ನಾವು ಪರಸ್ಪರ ಪ್ರೀತಿಸುತ್ತಿರುವೆವು ಎಂಬುದನ್ನು ಮನವರಿಕೆ ಮಾಡಿಕೊಂಡಿದ್ದರೆ ಎಲ್ಲವೂ ಸರಿಹೋಗುವುದು!”

ಆ ಹೊತ್ತೋ, ಅದರ ಹಿಂದಿನ ದಿನವೋ ಎಂದು ಕಾಣುತ್ತದೆ - ದೇವರ ವಿಚಾರವಾಗಿ ಮಾತನಾಡುತ್ತಿರುವಾಗ ಸ್ವಾಮಿಗಳು ತಮ್ಮ ಬಾಲ್ಯವನ್ನು ಸ್ಮರಿಸಿಕೊಂಡರು. ತಮ್ಮ ತುಂಟತನವನ್ನು ತಡೆಯಲಾರದೆ ಹೇಗೆ ತಮ್ಮ ತಾಯಿ “ಒಳ್ಳೆಯ ಮಗನನ್ನು ಕೊಡು ಎಂದು ಶಿವನನ್ನು ಬೇಡಿಕೊಂಡೆ, ತಪಸ್ಸು ಮಾಡಿದೆ. ಆದರೆ ಅವನು ತನ್ನ ಗಣಗಳಲ್ಲೊಂದನ್ನು ಕಳು ಹಿಸಿದ!” ಎನ್ನುತ್ತಿದ್ದಳು, ಹೇಗೆ ತಾನು ಶಿವನ ಗಣಗಳಲ್ಲೊಬ್ಬನೆಂದೇ ತಿಳಿದುಕೊಂಡು ಬಿಟ್ಟಿದ್ದೆ ಎಂಬುದನ್ನು ಹೇಳಿದರು. ಯಾವುದೋ ತಪ್ಪಿಗೆ ಶಿಕ್ಷೆಯೆಂದು ಶಿವನು ತಮ್ಮನ್ನು ಕೈಲಾಸದಿಂದ ತಾತ್ಕಾಲಿಕವಾಗಿ ಹೊರದೊಡಿರುವನು, ಪ್ರಯತ್ನಪಟ್ಟು ಅಲ್ಲಿಗೆ ಹಿಂದಿ ರುಗುವುದೇ ತಮ್ಮ ಜೀವನದ ಧ್ಯೇಯ ಎಂದು ಅವರು ತಿಳಿದುಕೊಂಡಿದ್ದರಂತೆ!

ತಾವು ಮೊಟ್ಟಮೊದಲು ಅಪಚಾರವೆಸಗಿದ್ದು ತಮಗೆ ಐದು ವರ್ಷವಾಗಿದ್ದಾಗ ಎಂದವರು ಒಮ್ಮೆ ಹೇಳಿದರು. ಊಟಮಾಡುತ್ತ ಬಲಗೈ ಎಂಜಲಾಗಿರುವಾಗ ಎಡಗೈಯಿಂದ ನೀರಿನ ಲೋಟವನ್ನೆ ತ್ತಿ ಕುಡಿಯುವುದೇ ಹೆಚ್ಚು ಶುಚಿ ಎಂದು ತಾಯಿಯೊಡನೆ ಅವರವಾದ - ಬಿರುಗಾಳಿಯ ಹಾಗೆ! ಆಗ ಮತ್ತು ಅಂಥವೇ ಇನ್ನಿತರ ಸಂದರ್ಭಗಳಲ್ಲಿ ತಾಯಿ ಒತ್ತಾಯದಿಂದ ಅವರನ್ನು ಕರೆದೊಯ್ದು, ತಣ್ಣೀರಿನ ಧಾರೆಯ ಕೆಳಗೆ ಕೂರಿಸಿ, “ಶಿವ! ಶಿವ!” ಎನ್ನುತ್ತಿದ್ದಳಂತೆ! ತಪ್ಪದೇ ಇದರಿಂದ ನಿರೀಕ್ಷಿತ ಪರಿಣಾಮವಾಗುತ್ತಿತ್ತಂತೆ. ಶಿವ ತಮ್ಮನ್ನು ಹೊರದೂಡಿರುವುದು ಅವರಿಗೆ ನೆನಪಾಗುತ್ತಿತ್ತಂತೆ; “ಇಲ್ಲ, ಇಲ್ಲ, ಇನ್ನೆಂದೂ ಹೀಗೆ ಮಾಡುವುದಿಲ್ಲ!” ಎಂದುಕೊಂಡು ತಾಯಿಗೆ ವಿಧೇಯರಾಗುತ್ತಿ ದ್ದರಂತೆ, ಸುಮ್ಮನಾಗುತ್ತಿದ್ದರಂತೆ.

ಮಹಾದೇವನೆಂದರೆ ಅವರಿಗೆ ಇನ್ನಿಲ್ಲದ ಪ್ರೀತಿ. ಭಾರತದ ಸ್ತ್ರೀಯರ ಭವಿಷ್ಯದ ಬಗ್ಗೆ ಮಾತನಾಡುತ್ತ ಅವರೊಮ್ಮೆ - ಕೆಲಸದ ಜಂಜಡದ ನಡುವೆ ಅವರು ಆಗಿಂದಾಗ “ಶಿವ! ಶಿವ!” ಎನ್ನುತ್ತಿದ್ದರೆ ಸಾಕು, ಅದೇ ಪೂಜೆಯಾಗುತ್ತದೆ - ಎಂದರು. ಅವರ ಪಾಲಿಗೆ ಹಿಮಾಲಯದ ಹವೆಯಲ್ಲಿಯೇ ಯಾವ ಸುಖದ ಯೋಚನೆಯೂ ಮುರಿಯಲಾರ ದಂತಹ ನಿರಂತರ ಧ್ಯಾನವಿತ್ತು. ಶಿವನ ತಲೆಯ ಮೇಲೆ ಮೊದಲು ಬಿದ್ದ ಗಂಗೆ, ಅವನ ಜಟೆಯಲ್ಲಿಯೇ ಅಲ್ಲಿಲ್ಲಿ ಅಲೆದು ಕೊನೆಗೆ ಕೆಳಗಿನ ಬಯಲಿಗೆ ಧುಮ್ಮಿಕ್ಕುವುದಕ್ಕೊಂದು ಕಿಂಡಿಯನ್ನು ಕಂಡುಕೊಂಡಳಂತೆ ಎನ್ನುವ ಸಹಜ ಕಥೆಯ ಅಂತರಾರ್ಥ ತಮಗೆ ಈ ಬೇಸಿಗೆಯಲ್ಲಿ ತಿಳಿಯಿತು ಎಂದವರು ಹೇಳಿದರು. ಪರ್ವತದ ಬಂಡೆಗಳ ನಡುವೆ ಜಲಪಾತವಾಗಿ ಧುಮುಕುವ ನದಿಗಳ ಕಲರವಕ್ಕೆ ಅರ್ಥವೇನು ಎಂದು ಅವರು ಬಹುಕಾಲದಿಂದ ಯೋಚಿಸುತ್ತಿದ್ದರಂತೆ; ಅದೇ “ಬಂ ಬಂ ಹರ ಹರ!” ಎಂಬ ಚಿರಂತನ ಶಿವಧ್ಯಾನವೆಂದು ಅವರಿಗೆ ಹೊಳೆಯಿತಂತೆ!

ಶಿವನನ್ನು ಕುರಿತು ಒಂದು ದಿನ ಮಾತನಾಡುತ್ತ “ನಿಜ, ಅವನು ಮಹಾದೇವ - ಶಾಂತ, ಸುಂದರ, ಮೌನಿ! ನಾನವನ ಮಹಾಭಕ್ತ!” ಎಂದು ಉದ್ಗರಿಸಿದರು.

ಅನಂತರ, ಮದುವೆಯೆಂಬುದು ಜೀವವು ದೇವರೊಂದಿಗೆ ಪಡುವ ಸಂಬಂಧ ಒಂದು ವಿಧಾನ ಎಂಬ ವಿಷಯ ಬಂದಿತು. “ಅದಕ್ಕೇ - ತಾಯಿ ಪ್ರೇಮ ಅನೇಕ ವಿಧಗಳಲ್ಲಿ ಹೆಚ್ಚಿನ ದಾದರೂ - ಲೋಕವು ಪ್ರೇಮವೆಂದರೆ ಸ್ತ್ರೀಪುರುಷರದ್ದೇ ಎಂದುಕೊಳ್ಳುವುದು. ಇನ್ನಾವ ಬಗೆಯ ಪ್ರೇಮಕ್ಕೂ ಅಂತಹ ಅಮೋಘವಾದ ಆದರ್ಶೀಕರಣದ ಶಕ್ತಿ ಇಲ್ಲ. ಪ್ರೇಮಿಯು ನಿಜವಾಗಿ ಕಲ್ಪಿಸಿಕೊಂಡಂತೆಯೇ ಆಗಿಬಿಡುತ್ತಾನೆ. ಈ ಪ್ರೇಮವು ಪ್ರೇಮಿಸಲ್ಪಟ್ಟವನನ್ನು ನಿಜವಾಗಿ ಪರಿವರ್ತಿಸಿಬಿಡುತ್ತದೆ!” ಎಂದರು.

ನಂತರ ಮಾತು ಅನೇಕ ಬಗೆಯ ದೇಶಗಳಿಗೆ ತಿರುಗಿತು; ಪರದೇಶದಿಂದ ಹಿಂದಿರು ಗುತ್ತಿರುವ ಪ್ರವಾಸಿಯು ಮತ್ತೊಮ್ಮೆ ತನ್ನ ದೇಶದ ಸ್ತ್ರೀಪುರುಷರನ್ನು ನೋಡಿದಾಗ ಪಡುವ ಸಂತೋಷವನ್ನು ಅವರು ಬಣ್ಣಿಸಿದರು. ಇಡಿಯ ಜೀವನವೇ ಮುಖದಲ್ಲಿ ಹಾಗೂ ರೂಪದಲ್ಲಿನ ಅಭಿವ್ಯಕ್ತಿಯ ಅತ್ಯಂತ ಸೂಕ್ಷ್ಮತರಂಗವನ್ನೂ ಗ್ರಹಿಸಿ ಅರ್ಥಮಾಡಿಕೊಳ್ಳಬಲ್ಲ ಸಾಮರ್ಥ್ಯವನ್ನು ತಂದುಕೊಡುವ ಸುಪ್ತಪ್ರಜ್ಞೆಯ ಒಂದು ಶಿಕ್ಷಣವಾಗಿದೆ ಎಂದರು.

ಆಗ ನಮ್ಮೆದರು ಬರಿಗಾಲಿನಲ್ಲಿ ನಡೆಯುತ್ತಿದ್ದ ಸಂನ್ಯಾಸಿಗಳ ಗುಂಪೊಂದು ಹಾದು ಹೋಯಿತು. ಸ್ವಾಮಿಗಳಾಗ ದೇಹದಂಡನೆಯ ತಪಸ್ಸನ್ನು ಖಂಡಿಸುತ್ತ, ಅದು ಶುದ್ಧ ಅನಾಗರಿಕತೆ, ಕಾಡುತನವಲ್ಲದೆ ಇನ್ನೇನಲ್ಲ ಎಂದು ಭರ್ತ್ಸನೆಮಾಡತೊಡಗಿದರು... ಆದರೆ ಆ ಹಾದಿಹೋಕರು ತಾವು ನಂಬಿದ ಆದರ್ಶದ ಹೆಸರಿನಲ್ಲಿ ನಿಧಾನವಾಗಿ ಕಾಲ್ನಡೆಯಲ್ಲೇ ಮೈಲಿಗಳನ್ನು ಸವೆಸುತ್ತಿರುವ ಆ ದೃಶ್ಯವು ಅವರ ಮನಸ್ಸಿನ ಯಾವುದೋ ನೋವನ್ನು ಬಡಿದೆಬ್ಬಿಸುತ್ತಿದೆಯೆಂದೆನಿಸಿತು; ಸಮಸ್ತ ಮಾನವಜನಾಂಗವೇ “ಧರ್ಮದ ಚಿತ್ರಹಿಂಸೆ” ಗೆ ತುತ್ತಾಗಿಬಿಟ್ಟಿದೆ ಎನ್ನುತ್ತ ತಮ್ಮ ಅಸಹನೆಯನ್ನು ವ್ಯಕ್ತಪಡಿಸಲಾರಂಭಿಸಿದರು. ಅನಂತರ ಆ ಭಾವವು ಮೇಲೆದ್ದು ಬಂದ ಹಾಗೆಯೇ ತಾನಾಗಿ ಶಾಂತವಾಯಿತು; ಅದರ ಜಾಗದಲ್ಲಿ ಅಷ್ಟೇ ಶಕ್ತಿಯುತವಾದ ಇನ್ನೊಂದೇ ಭಾವ - ಅಂತಹ “ಕಾಡುತನ” ಇಲ್ಲದೆ ಹೋಗಿದ್ದರೆ ಮಾನವನು ಸೃಜಿಸಿಕೊಂಡಿರುವ ಸುಖಸೌಲಭ್ಯಗಳು ಅವನ ಪೌರುಷವನ್ನೆಲ್ಲ ಸೂರೆಮಾಡಿಬಿಟ್ಟಿರುತ್ತಿದ್ದವು ಎಂದು ಅವರು ನುಡಿಯುವಂತೆ ಮಾಡಿತು.

ಆ ಸಂಜೆ ನಾವು ‘ಊರಿ ಡಾಕ್’ ಬಂಗಲೆಯಲ್ಲಿ ತಂಗಿದೆವು. ಸಂಜೆಗತ್ತಲಿನ ಮಬ್ಬು ಬೆಳಕಿನಲ್ಲಿ ನಾವು ಪೇಟೆಯಲ್ಲಿ ಹಾಗೂಸುತ್ತಮುತ್ತಣ ಹುಲ್ಲುಗಾವಲಿನಲ್ಲಿ ನಡೆದಾಡಿ ದೆವು. ಆ ಸ್ಥಳವು ನಿಜವಾಗಿ ಅದೆಷ್ಟು ಸುಂದರವಾಗಿತ್ತು! ಮಣ್ಣಿನಿಂದ ನಿರ್ಮಿತವಾದ ಪುಟ್ಟದೊಂದು ಕೋಟೆ - ಯೂರೋಪಿಯನ್ ಊಳಿಗಮಾನ್ಯ ಮಾದರಿಯನ್ನೇ ಹೋಲು ವಂಥದು -ಪರ್ವತಾಳಿ ಹಾಗೂ ಗದ್ದೆಗಳ ಅದ್ಭುತ ದೃಶ್ಯವೊಂದರೆಡೆಗೆ ಕರೆದೊಯ್ಯುತ್ತಿದ್ದ ನಮ್ಮ ಕಾಲುದಾರಿಯ ಪಕ್ಕದಲ್ಲಿ ಎತ್ತರಕ್ಕೆ ಧುತ್ತೆಂದು ಎದ್ದು ನಿಂತಿತ್ತು. ನದಿಯ ಮೇಲೆ ಎತ್ತರದಲ್ಲಿದ್ದ ಹಾದಿಯುದ್ದಕ್ಕೂ ಪೇಟೆ. ನಾವು ಗದ್ದೆಯ ಕಾಲುದಾರಿಯಲ್ಲಿ ಸಾಗಿ, ಕೈ ತೋಟಗಳಲ್ಲಿ ಗುಲಾಬಿ ಹೂಗಳು ಅರಳಿ ನಿಂತಿದ್ದ ರೈತರ ಗುಡಿಸಲುಗಳನ್ನು ಹಾದು, ಬಂಗಲೆಗೆ ಹಿಂದಿರುಗಿದೆವು. ಹಾಗೆ ನಾವು ಸಾಗಿ ಬರುತ್ತಿದ್ದಾಗ, ಅಲ್ಲಲ್ಲಿ ಒಂದಲ್ಲ ಒಂದು ತುಂಟ ಮಗು, ಇತರ ಮಕ್ಕಳಂತಲ್ಲದೆ ಕುತೂಹಲದಿಂದ ನಮ್ಮ ಬಳಿಗೆ ಬಂದು, ನಮ್ಮೊಡನೆ ಆಟವಾಡುತ್ತಿತ್ತು.

\textbf{ಜೂನ್ ೨೦.}

ಮಾರನೆಯ ದಿನ, ಹಾದಿಯಲ್ಲಿ ಸಿಕ್ಕಿದ ಅತ್ಯಂತ ಸುಂದರ ದೃಶ್ಯಗಳನ್ನು ಆಸ್ವಾದಿಸುತ್ತ, ಬಂಡೆಗಳ ಮೇಲಿದ್ದ ಹಳೆಯದೊಂದು ಸೂರ್ಯ ದೇವಸ್ಥಾನದ ಅವಶೇಷಗಳ ಮೂಲಕ ಹಾದು, ನಾವು ಬಾರಾಮುಲ್ಲಾ ತಲುಪಿದೆವು. ಕಾಶ್ಮೀರ ಕಣಿವೆಯು ಒಂದು ಕಾಲದಲ್ಲಿ ಸರೋವರವಾಗಿತ್ತು, ಈ ಸ್ಥಳದಲ್ಲಿ ವಿಷ್ಣುವು ವರಾಹವತಾರದಲ್ಲಿ ತನ್ನ ಕೋರೆ ದಾಡೆಗಳಿಂದ ಪರ್ವತವನ್ನು ಮೇಲಕ್ಕೆತ್ತಿ ಝೀಲಮ್​ ನದಿಯನ್ನು ಸ್ವತಂತ್ರವಾಗಿ ಹೋಗಲು ಬಿಟ್ಟನು ಎಂದು ಸ್ಥಳಪುರಾಣ. ಪುರಾಣದ ರೂಪದಲ್ಲಿರುವ ಭೂಗೋಳ ಶಾಸ್ತ್ರದ ಒಂದು ಭಾಗವೋ, ಅಥವಾ ಅದು ಇತಿಹಾಸಪೂರ್ವದ ಒಂದು ಚರಿತ್ರೆಯೋ?

\textbf{ಅಧ್ಯಾಯ ೬: ಕಾಶ್ಮೀರದ ಕಣಿವೆ}

ವ್ಯಕ್ತಿಗಳು: ಸ್ವಾಮಿ ವಿವೇಕಾನಂದರು, ಗುರುಭಾಯಿಗಳು, ಧೀರಮಾತಾ, ಜಯಾ ಎಂಬ ಹೆಸರಿನವಳು ಮತ್ತು ಸೋದರಿ ನಿವೇದಿತಾಳನ್ನೊಳಗೊಂಡ ಯೂರೋಪಿಯನ್ ಶಿಷ್ಯರುಗಳ ಮತ್ತು ಅತಿಥಿಗಳು ಗುಂಪು.

ಸ್ಥಳ: ಝೀಲಮ್​ ನದಿ - ಬಾರಾಮುಲ್ಲಾದಿಂದ ಶ‍್ರೀನಗರಕ್ಕೆ.

ಕಾಲ: ೧೮೯೮ರ ಜೂನ್ ೨೦ರಿಂದ ಜೂನ್ ೨೨ರವರೆಗೆ.

ಡಾಕ್ ಬಂಗಲೆಯ ನಮ್ಮ ಕೊಠಡಿಗೆ ಹಿಂದಿರುಗಿ ಬಂದ ಸ್ವಾಮಿಗಳು ಮೊಳ ಕಾಲುಗಳ ಮೇಲೆ ಛತ್ರಿಯನ್ನಿಟ್ಟುಕೊಂಡು ಕುಳಿತು ತುಂಬ ಸಂತಸದಿಂದ “ಅದೃಷ್ಟ ವಂತರ ಕಡೆಗೇ ಭಗವಂತನೂ ಇರುತ್ತಾನೆ ಎಂಬ ಮಾತೊಂದಿದೆ!” ಎಂದು ಕೂಗಿ ಹೇಳಿದರು. ಅನುಚರರನ್ನೇನೂ ಕರೆದುಕೊಂಡು ಬಂದಿರಲಿಲ್ಲವಾದ್ದರಿಂದ, ಗಂಡಸರೇ ಮಾಡಬೇಕಾದಂತಹ ಕೆಲವು ಸಣ್ಣಪುಟ್ಟ ಕೆಲಸಗಳನ್ನು ತಾವೇ ನಿರ್ವಹಿಸಬೇಕಾಗಿ ದ್ದಿತು; ಹಾಗೆಂದೇ ಅವರು ಡಂಗಾ (ದೋಣಿಮನೆ)ಗಳನ್ನು ಬಾಡಿಗೆಗೆ ಗೊತ್ತುಮಾಡಿಕೊಂಡು ಬರಲು ಹೋಗಿದ್ದರು. ಆದರೆ ಎದುರಿಗೆ ಸಿಕ್ಕಿದ ಯಾರೋ ಒಬ್ಬರು, ಅವರ ಹೆಸರನ್ನು ಕೇಳಿದ ತಕ್ಷಣ ಆ ಜವಾಬ್ದಾರಿಯನ್ನು ತಾನೇ ತೆಗೆದುಕೊಂಡು, ಅವರನ್ನು ಹಿಂದಕ್ಕೆ ಕಳುಹಿಸಿದ್ದರು.

ಹಾಗೆ ನಾವು ಆ ದಿನ ಮುಕ್ತ ಮನಸ್ಸಿನಿಂದ ಸಂತೋಷವಾಗಿ ಕಾಲ ಕಳೆಯುವಂತಾಯಿತು. ಸಾಮೋವರ್ ಒಂದರಲ್ಲಿ ಕಾಶ್ಮೀರಿ ಟೀಯನ್ನು ಕುಡಿದೆವು; ಸ್ಥಳೀಯವಾಗಿ ತಯಾರಿಸಿದ ಹಣ್ಣಿನ ರಸಾಯನವನ್ನು ತಿಂದೆವು. ಸುಮಾರು ನಾಲ್ಕು ಘಂಟೆಯ ಹೊತ್ತಿಗೆ ನಮಗೆ ದೊರಕಿದ ಮೂರು ತೇಲುದೋಣಿಗಳನ್ನು ವಶಕ್ಕೆ ತೆಗೆದುಕೊಂಡು, ಅವುಗಳಲ್ಲಿ ಕುಳಿತು ಶ‍್ರೀನಗರದ ಕಡೆಗೆ ಹೊರಟೆವು. ಮೊದಲ ಸಂಜೆಗೆ ಸ್ವಾಮಿಗಳ ಸ್ನೇಹಿತರೊಬ್ಬರ ತೋಟದಲ್ಲಿ ಲಂಗರು ಹಾಕಿ ನಿಂತೆವು...

ಮಾರನೆಯ ಬೆಳಗ್ಗೆ ನಾವಿದ್ದುದುಸುತ್ತಲೂ ಮಂಜುಕವಿದ ಪರ್ವತಗಳಿಂದ ಆವೃತವಾದ ಸುಂದರವಾದೊಂದು ಕಣಿವೆಯ ನಟ್ಟನಡುವೆ - ಇದೇ ಕಾಶ್ಮೀರದ ಕಣಿವೆ; ಬಹುಶಃ ಇದನ್ನು ಶ‍್ರೀನಗರದ ಕಣಿವೆ ಎಂದು ಕರೆಯುವುದೇ ಹೆಚ್ಚು ಸರಿಯೇನೋ...

ಅದು ಮೊದಲನೆಯ ದಿನ. ಬೆಳಗ್ಗೆ ಹೊಲಗಳಲ್ಲಿ ನಡೆದಾಡಿ ಬರಲು ಹೊರಟೆವು. ಅಲ್ಲೊಂದು ವಿಶಾಲವಾದ ಹುಲ್ಲುಗಾವಲು, ಅದರ ನಡುವೆ ಒಂದು ಬೃಹತ್ ಚಿನಾರ್ ಮರ. ಅದರಲ್ಲಿ ಗಾದೆ ಮಾತಿನಂತೆ ಇಪ್ಪತ್ತು ದನಗಳು ಹಾದುಹೋಗಬಹುದೇನೋ ಎಂಬಂತೆ ತೋರುವ ಒಂದು ಪೊಟರೆ. ಅದೊಂದು ಋಷಿಕುಟೀರವಾದರೆ ಹೇಗೆ? ಎಂದು ಸ್ವಾಮಿಗಳು ಕನಸುಕಾಣತೊಡಗಿದರು. ಆ ಪೊಟರೆಯೊಳಗೆ ನಿಜವಾಗಿಯೂ ಒಂದು ಸಣ್ಣ ಕುಟೀರವನ್ನು ಕಟ್ಟಬಹುದಾದಷ್ಟು ದೊಡ್ಡ ಜಾಗವಿತ್ತು. ಅನಂತರ, ಇನ್ನು ಮುಂದೆ ಕಾಣಬಹುದಾದ ಪ್ರತಿಯೊಂದು ಚಿನಾರ್ ಮರವನ್ನೂ ನಾವು ಪವಿತ್ರವೆಂದು ಭಾವಿಸುವ ಹಾಗೆ, ಅವರು ಧ್ಯಾನವನ್ನು ಕುರಿತು ಮಾತನಾಡಿದರು.

ಅವರೊಂದಿಗೆ ನಾವು ಪಕ್ಕದಲ್ಲಿದ್ದ ಹೊಲದ ಕಣವೊಂದಕ್ಕೆ ತಿರುಗಿದೆವು. ಅಲ್ಲೊಂದು ಮರದ ಕೆಳಗೆ ಅನುಪಮ ಸುಂದರಿಯಾಗಿದ್ದ ವೃದ್ಧೆಯೊಬ್ಬಳು ಕುಳಿತಿದ್ದಳು. ಕಾಶ್ಮೀರಿ ಹೆಂಗಸಿನ ಕೆಂಪು ತಲೆಯುಡಿಗೆ, ಬಿಳಿಯ ಅವಕುಂಠನ ಧರಿಸಿದ್ದ ಅವಳು ಉಣ್ಣೆಯನ್ನು ನೂಲುತ್ತ ಕುಳಿತಿದ್ದಳು. ಇಬ್ಬರು ಸೊಸೆಯಂದಿರು ಮತ್ತು ಮಕ್ಕಳು ಅವಳಿಗೆ ಸಹಾಯ ಮಾಡುತ್ತಿದ್ದರು. ಸ್ವಾಮಿಗಳು ಹಿಂದೊಮ್ಮೆ ವಸಂತಕಾಲದಲ್ಲಿ ಇಲ್ಲಿಗೆ ಬಂದಿದ್ದರು; ಆಗಿನಿಂದಲೂ ಅವರು ಇದೇ ಹೆಂಗಸಿನ ಶ್ರದ್ಧಾವಂತಿಕೆ ಹಾಗೂ ಆತ್ಮಾ ಭಿಮಾನಗಳನ್ನು ಕುರಿತಾಗಿ ಆಗಾಗ ಹೇಳುತ್ತಿದ್ದರು. ನೀರು ಬೇಕೆಂದು ಸ್ವಾಮಿಗಳು ಕೇಳಿದಾಗ ತಕ್ಷಣವೇ ಅವಳು ಕೊಟ್ಟಿದ್ದಳು. ಅನಂತರ, ಅಲ್ಲಿಂದ ಹೊರಡುವುದಕ್ಕೆ ಮುನ್ನ, “ತಾಯಿ, ನಿಮ್ಮ ಧರ್ಮ ಯಾವುದು?” ಎಂದು ಸ್ವಾಮಿಗಳು ಕೇಳಿದಾಗ, “ಸ್ವಾಮಿ ದಯಾಮಯನಾದ ಆ ಭಗವಂತನಿಗೆ ನಾನು ಕೃತಜ್ಞಳು. ಆತನ ಕೃಪೆಯಿಂದ ನಾನೊಬ್ಬ ಮುಸಲ್ಮಾನರವಳು!” ಎಂದಿದ್ದಳು ಅವಳು. ಈಗ ಅವಳ ಕುಟುಂಬದವ ರೆಲ್ಲರೂ ಸ್ವಾಮಿಗಳನ್ನು ತಮ್ಮ ಹಳೆಯ ಸ್ನೇಹಿತರೆಂಬಂತೆ ಬರಮಾಡಿಕೊಂಡರು; ಸ್ವಾಮಿಗಳೊಡನೆ ಬಂದಿದ್ದ ನಮ್ಮೆಲ್ಲರಿಗೂ ಸಹ ಎಲ್ಲ ರೀತಿಯಲ್ಲೂ ಆದರವನ್ನು ತೋರಿಸಿದರು.

ಶ‍್ರೀನಗರದವರೆಗಿನ ನಮ್ಮ ಪ್ರಯಾಣ ಎರಡು - ಮೂರು ದಿನ ತೆಗೆದುಕೊಂಡಿತು. ಒಂದು ಸಂಜೆ ರಾತ್ರಿಯೂಟಕ್ಕಿಂತ ಮೊದಲು ನಾವು ಹೊಲಗಳಲ್ಲಿ ನಡೆಯುತ್ತಿರುವಾಗ, ಕಾಳೀಘಾಟನ್ನು ನೋಡಿ ಬಂದಿದ್ದ ಒಬ್ಬಳು ಅಲ್ಲಿಯ ಅತಿರೇಕದ ಶರಣಾ ಗತಿಯ ಭಾವನೆಯ ಬಗ್ಗೆ ತನಗಾದ ಆಘಾತವನ್ನು ಗುರುದೇವರಿಗೆ ನಿವೇದಿಸಿಕೊಂಡಳು. “ಮೂರ್ತಿಯ ಮುಂದಿರುವ ನೆಲವನ್ನು ಅವರೇಕೆ ಮುತ್ತಿಕ್ಕುತ್ತಾರೆ?” ಎಂದು ಅಚ್ಚರಿಯಿಂದ ಕೇಳಿದವಳು. ಸ್ವಾಮಿಗಳು ಆಗ ಅಲ್ಲಿ ಬೆಳೆದು ನಿಂತಿದ್ದ ತಿಲ್(ಎಳ್ಳು)ನ್ನು ತೋರಿಸಿ - ಇಂಗ್ಲಿಷಿನ ಡಿಲ್ (ಸಬ್ಬಸಿಗೆ) ಎಂಬ ಪದವು ಅದರಿಂದ ಬಂದಿರಬೇಕು ಎಂಬುದು ಅವರ ಅಭಿಪ್ರಾಯವಾಗಿತ್ತು - “ಆರ್ಯರ ಅತ್ಯಂತ ಪ್ರಾಚೀನ ಎಣ್ಣೆ ಬೀಜವಿದು” ಎಂದು ವಿವರಿಸುತ್ತಿದ್ದರು. ಆದರೆ ಈ ಪ್ರಶ್ನೆಯನ್ನು ಕೇಳಿದಾಕ್ಷಣ ಅವರ ಕೈಯಲ್ಲಿದ್ದ ಪುಟ್ಟ ನೀಲಿ ಹೂ ಕೆಳಕ್ಕೆ ಬಿತ್ತು; ಮುಖ ಗಂಭೀರವಾಯಿತು; ಧ್ವನಿ ನಿಷ್ಪಂದ ನೀರವವಾಯಿತು; ನಡೆಯುತ್ತಿದ್ದವರು ನಿಶ್ಚಲರಾಗಿ ನಿಂತು “ಈ ಪರ್ವತಗಳೆದುರು ಭೂಚುಂಬನ ಮಾಡು ವುದೂ ಆ ಮೂರ್ತಿಯೆದುರು ನೆಲವನ್ನು ಮುತ್ತಿಕ್ಕುವುದೂ ಒಂದೇ ಅಲ್ಲವೇ?” ಎಂದರು.

ಬೇಸಗೆ ಕೊನೆಗೊಳ್ಳುವುದರೊಳಗೆ ನಮ್ಮನ್ನೆಲ್ಲ ವಿವಿಕ್ತವಾಸಕ್ಕೆ ಕರೆದೊಯ್ದು ಧ್ಯಾನ ಮಾಡುವುದನ್ನು ಕಲಿಸುವುದಾಗಿ ಗುರುದೇವರು ಮಾತು ಕೊಟ್ಟಿದ್ದರು... ದೇಶವನ್ನೆಲ್ಲ ತಿರುಗಿ ಆದ ಮೇಲೆ ಏಕಾಂತವಾಸಕ್ಕೆ ತೆರಳುವುದು ಎಂದು ನಿರ್ಧಾರವಾಗಿತ್ತು.

ಶ‍್ರೀನಗರಕ್ಕೆ ತಲುಪಿದ ಮೊದಲ ರಾತ್ರಿ ನಾವು ಕೆಲವು ಬಂಗಾಳಿ ಅಧಿಕಾರಿಗಳ ಜೊತೆಯಲ್ಲಿ ಊಟ ಮಾಡಿದೆವು. ಅವರೊಂದಿಗೆ ಸರಸ ಸಂಭಾಷಣೆ ನಡೆಯುತ್ತಿರುವಾಗ, ಪಾಶ್ಚಾತ್ಯ ಅತಿಥಿಗಳಾದ ನಮ್ಮಲ್ಲೊಬ್ಬರು - ಪ್ರತಿಯೊಂದು ದೇಶದ ಚರಿತ್ರೆಯೂ ವಿಶದಪಡಿಸುವ ಕೆಲವು ಆದರ್ಶಗಳನ್ನು, ಬೆಳೆಸಿಕೊಂಡು ಬಂದಿರುವ ಆದರ್ಶಗಳನ್ನು, ಆ ದೇಶದ ಜನರು ಶ್ರದ್ಧೆಯಿಂದ ನಂಬಿ ನಡೆದುಕೊಳ್ಳಬೇಕು - ಎಂದು ತಮ್ಮವಾದವನ್ನು ಮಂಡಿಸಿದರು. ಅಲ್ಲಿದ್ದ ಹಿಂದೂಗಳು ಇದಕ್ಕೆ ಹೇಗೆ ವಿರೋಧ ವ್ಯಕ್ತಪಡಿಸಿದರು ಎಂಬುದೇ ಕುತೂಹಲ ಕಾರಿಯಾಗಿತ್ತು. ಮನುಷ್ಯನ ಮನಸ್ಸು ಶಾಶ್ವತವಾಗಿ ಇಂತಹ ಬಂಧನಕ್ಕೆ ಸಿಲುಕಿಕೊಳ್ಳುವುದು ತರವಲ್ಲ ಎಂದವರ ಸ್ಪಷ್ಟಾಭಿಪ್ರಾಯ. ಸಿದ್ಧಾಂತವು ಬಂಧನವನ್ನುಂಟುಮಾಡು ವಂಥದು ಎಂಬುದನ್ನು ವಿರೋಧಿಸುವ ಭರಾಟೆಯಲ್ಲಿ ಸ್ವಾಮಿಗಳು ಮಧ್ಯೆ ಪ್ರವೇಶಿಸಿ, “ಇದರಲ್ಲಿ ಕೊಟ್ಟ ಕೊನೆಯ ಅಂಶವು ಮಾನಸಿಕವಾದುದು ಎಂಬುದನ್ನು ನೀವೆಲ್ಲರೂ ಒಪ್ಪಬೇಕು. ಅದು ಭೌಗೋಳಿಕ ಅಂಶಕ್ಕಿಂತ ಹೆಚ್ಚು ಶಾಶ್ವತವಾದುದಲ್ಲವೆ?” ಎಂದರು. ಅನಂತರ, ನಮಗೆಲ್ಲರಿಗೂ ಗೊತ್ತಿರುವ ಪ್ರಕರಣಗಳನ್ನು ಪ್ರಸ್ತಾಪಿಸುತ್ತ, ತಾವು ನೋಡಿದವರೆಲ್ಲರಿಗಿಂತ ಹೆಚ್ಚು “ಕ್ರಿಶ್ಚಿಯನ್” ಆದವಳೊಬ್ಬ ಬಂಗಾಳಿ ಮಹಿಳೆ ಎಂದೂ, ಇನ್ನೊಬ್ಬಳು ಪಶ್ಚಿಮದೇಶದಲ್ಲಿ ಜನ್ಮ ತಳೆದಿದ್ದರೂ ತಮಗಿಂತಲೂ ಹೆಚ್ಚು “ಹಿಂದೂ” ಆಗಿರುವಳು ಎಂದೂ ಉದಾಹರಣೆ ಕೊಟ್ಟರು. ಇಷ್ಟು ಮಾತ್ರವಲ್ಲ, ಆದರ್ಶದ ಸ್ಥಿತಿ ಎಂದರೆ ಒಂದು ಆದರ್ಶವನ್ನು ಸಾಧ್ಯವಾದಷ್ಟೂ ದೂರಕ್ಕೆ ಪ್ರಸರಿಸುವುದಕ್ಕಾಗಿ ನಮ್ಮಲ್ಲಿ ಒಬ್ಬೊಬ್ಬರೂ ಇನ್ನೊಬ್ಬರ ದೇಶದಲ್ಲಿ ಹುಟ್ಟಿ ಬರಬೇಕಾಗುತ್ತದೆ ಅಲ್ಲವೆ?

