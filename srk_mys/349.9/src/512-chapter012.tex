
\chapter{ಅಧ್ಯಾಯ ೧೨: ಚೆನ್ನಾರ್ ಮರಗಳ ಕೆಳಗೆ ತಂಗಿದ್ದಾಗ}

ವ್ಯಕ್ತಿಗಳು: ಸ್ವಾಮಿ ವಿವೇಕಾನಂದರು, ಗುರುಭಾಯಿಗಳು, ಧೀರಮಾತಾ, ಜಯಾ ಎಂಬ ಹೆಸರಿನವಳು ಮತ್ತು ಸೋದರಿ ನಿವೇದಿತಾಳನ್ನೊಳಗೊಂಡ ಯೂರೋಪಿಯನ್ ಶಿಷ್ಯರುಗಳ ಮತ್ತು ಅತಿಥಿಗಳ ಗುಂಪು.

ಸ್ಥಳ: ಕಾಶ್ಮೀರ - ಶ‍್ರೀನಗರ.

ಕಾಲ: ೧೮೯೮ರ ಆಗಸ್ಟ್ ೧೪ರಿಂದ ಆಗಸ್ಟ್ ೨೦ರವರೆಗೆ.

\textbf{ಆಗಸ್ಟ್ ೧೪.}

ಈ ಹೊತ್ತು ಭಾನುವಾರ ಬೆಳಗ್ಗೆ. ವೇದಾಂತದಲ್ಲಿ ಆಸಕ್ತರೆಂದು ಕಂಡುಬರು ತ್ತಿದ್ದ ಒಬ್ಬ ಯೂರೋಪಿಯನ್ ಅತಿಥಿಯನ್ನು ಭೇಟಿಮಾಡಿಸುವುದಕ್ಕಾಗಿ ಮಧ್ಯಾಹ್ನ ಸ್ವಾಮೀಜಿಯವರನ್ನು ನಮ್ಮ ಜೊತೆಗೆ ಟೀ ತೆಗೆದುಕೊಳ್ಳಲು ಬನ್ನಿ ಎಂದು ಮನವೊಲಿಸಿದ್ದೆವು. ಅವರಿಗೆ ಈ ವಿಷಯದಲ್ಲಿ ತಮ್ಮನ್ನು ತಾವು ತೊಡಗಿಸಿಕೊಳ್ಳಲು ಇಷ್ಟ ವಿರಲಿಲ್ಲ; ನನಗನ್ನಿಸುವ ಮಟ್ಟಿಗೆ, ಅವರು ಬರುವುದಕ್ಕೆ ಒಪ್ಪಿಕೊಂಡಿದ್ದು ಕಾತರರಾಗಿದ್ದ ತಮ್ಮ ಶಿಷ್ಯರಿಗೆ ಇಂತಹ ಎಲ್ಲ ಪ್ರಯತ್ನಗಳ ನಿರರ್ಥಕತೆಯನ್ನು ಮನವರಿಕೆ ಮಾಡಿ ಕೊಡುವ ಒಂದು ಅವಕಾಶ ಎಂಬುದಕ್ಕಾಗಿ. ಆ ಅತಿಥಿಗಾಗಿ ಅವರು ಇನ್ನಿಲ್ಲದ ಹಾಗೆ ಶ್ರಮಿಸಿದರಾದರೂ, ಮೊದಲೇ ಅಂದುಕೊಂಡಂತೆ ಅವರ ಶ್ರಮವೆಲ್ಲವೂ ವ್ಯರ್ಥವಾಯಿತು.

ಇನ್ನಿತರ ಸಂಗತಿಗಳ ನಡುವೆ ಅವರು ಹೇಳಿದ ಮಾತು ನನಗೆ ನೆನಪಿದೆ: “ನಿಯಮವೊಂದನ್ನು ಮುರಿಯುವುದಾಗಲಿ ಎಂದು ನಾನೆಷ್ಟು ಹಾರೈಸುತ್ತೇನೆ! ಹಾಗೆ ನಾವು ನಿಯಮವನ್ನು ನಿಜವಾಗಿಯೂ ಮುರಿದದ್ದೇ ಆದರೆ ಸ್ವತಂತ್ರರಾಗುತ್ತೇವೆ. ನಿಯಮವನ್ನು ಮುರಿಯುವುದು ಎಂದು ಯಾವುದನ್ನು ನೀವು ಕರೆಯುತ್ತೀರೋ ಅದು ನಿಜಕ್ಕೂ ಅದರ ಅನುಕರಣೆಯ ಇನ್ನೊಂದು ವಿಧಾನವಷ್ಟೇ”. ಆ ನಂತರ ಅವರು ಅತೀತ ಪ್ರಜ್ಞೆಯ ಜೀವನವನ್ನು ಕುರಿತು ಸ್ವಲ್ಪ ವಿವರಿಸಲೆತ್ನಿಸಿದರು; ಆದರೆ ಅವರ ಮಾತುಗಳು ಕೇಳಲಾರದ ಕಿವಿಗಳ ಮೇಲೆ ಬಿದ್ದವು.

\textbf{ಆಗಸ್ಟ್ ೧೬.}

ಮಂಗಳವಾರ ಅವರು ಇನ್ನೊಂದು ಸಲ ಮಧ್ಯಾಹ್ನದ ಊಟಕ್ಕಾಗಿ ನಮ್ಮ ಬೀಡಿಗೆ ಬಂದರು. ಊಟ ಮುಗಿಯುವ ಹೊತ್ತಿಗೆ ಅವರು ಹಿಂದಿರುಗಲು ಆಗದ ಹಾಗೆ ಕುಂಭದ್ರೋಣ ಮಳೆ ಪ್ರಾರಂಭವಾಯಿತು; ಆಗ ಅವರು ಅಲ್ಲೇ ಇದ್ದ ಟಾಡ್ ಅವರ ರಾಜಾಸ್ಥಾನದ ಚರಿತ್ರೆಯನ್ನು ಕೈಗೆತ್ತಿಕೊಂಡರು. ಮಾತು ಮೀರಾಬಾಯಿಯ ಕಡೆಗೆ ಹೊರಳಿತು. “ಇಂದು ಬಂಗಾಳದಲ್ಲಿರುವ ರಾಷ್ಟ್ರೀಯ ಕಲ್ಪನೆಗಳಲ್ಲಿ ಮೂರರಲ್ಲೆರಡು ಭಾಗ ಈ ಪುಸ್ತಕದಿಂದ ಸಂಗ್ರಹಿಸಿರುವುದಾಗಿದೆ” ಎಂದರು.

ಆದರೆ ಟಾಡ್ ನ ಪುಸ್ತಕದಲ್ಲಿಯೂ ಸಹ, ಅವರಿಗೆ ಇಷ್ಟವಾದುದು ರಾಣಿಯಲ್ಲದ ರಾಣಿ, ಕೃಷ್ಣಪ್ರೇಮಿಗಳ ಜೊತೆಗೆ ಪ್ರಪಂಚವನ್ನೆಲ್ಲಾಸುತ್ತಬಯಸಿದ ರಾಣಿ ಮೀರಾ ಬಾಯಿಯ ಪ್ರಸಂಗ. ದೇವರ ಹೆಸರಿನಲ್ಲಿ ಪ್ರೇಮವನ್ನು ಹಾಗೂ ಸಮಸ್ತರಲ್ಲಿ ದಯೆಯನ್ನು ಪ್ರತಿಪಾದಿಸಿದ ಚೈತನ್ಯರ ಜೊತೆಗೆ ಸರ್ವಜನರಲ್ಲಿ ಸೇವೆ ಸಮರ್ಪಣೆ ಪ್ರಾರ್ಥನೆಗಳನ್ನು ಬೋಧಿಸಿದ ಅವಳನ್ನು ಹೋಲಿಸಿ ಮಾತನಾಡಿದರು.

ಅವರು ಯಾವಾಗಲೂ ಎತ್ತಿಹಿಡಿಯುತ್ತಿದ್ದವರಲ್ಲಿ ಮೀರಾಬಾಯಿಯೂ ಒಬ್ಬಳು. ಅವಳ ಜೀವನಗಾಥೆಗೆ ಅವರು ಅನೇಕಕೊಂಡಿಗಳನ್ನು ಜೋಡಿಸುತ್ತಿದ್ದರು. ಇನ್ನಿತರ ಸಂದರ್ಭಗಳಲ್ಲಿ ನಮಗೆ ಈಗಾಗಲೇ ಗೊತ್ತಿರುವ, ಇಬ್ಬರು ಕಳ್ಳರ ನಡುವೆ ನಡೆದ ಸಂಭಾಷಣೆಯ ಕಥೆ, ಕೊನೆಯಲ್ಲಿ ಕೃಷ್ಣನ ಪ್ರತಿಮೆಯೊಂದು ಬಾಯ್ದೆರೆದು ಅವಳನ್ನು ನುಂಗಿಬಿಟ್ಟುದು ಇತ್ಯಾದಿ. ಒಮ್ಮೆ ಸ್ತ್ರೀಯೊಬ್ಬಳಿಗೆ ಅವರು ಮೀರಾಬಾಯಿಯ ಹಾಡೊಂದನ್ನು ಉದ್ಧರಿಸಿ ಅದರ ಅನುವಾದವನ್ನೂ ಶ್ರುತಪಡಿಸಿದ್ದನ್ನು ಕೇಳಿಸಿಕೊಂಡಿರುವೆನು. ಇಡಿಯಾಗಿ ನನಗೆ ನೆನಪಿದ್ದಿದ್ದರೆ ಚೆನ್ನಾಗಿತ್ತು; ಆದರೆ ಅದು ಪ್ರಾರಂಭವಾಗುವುದು “ಅಂಟಿಕೋ ಸೋದರನೇ, ಅಂಟಿಕೋ ಅದಕಂಟಿಕೋ” ಎಂದು; ಕೊನೆಗೊಳ್ಳುವುದು “ಅಂಕಾ ಬಂಕಾ ಎಂಬ ಚೋರ ಸೋದರರು, ನಿರ್ದಯ ಕಟುಕನಾದ ಸುಜನ, ಆಡುತ್ತಾಡುತ್ತಲೇ ತನ್ನ ಗಿಳಿಗೆ ಕೃಷ್ಣನಾಮವನ್ನು ಹೇಳಿಕೊಟ್ಟ ರಾಜನರ್ತಕಿ ಇವರೆಲ್ಲರೂ ಉದ್ಧಾರವಾಗಿರುವರಾದರೆ, ಭರವಸೆ ಎಲ್ಲರಿಗೂ ಇದ್ದೇ ಇದೆ” ಎಂಬುದಾಗಿ.

ಮತ್ತೆಯೂ ಅವರು ಮೀರಾಬಾಯಿಯ ಕಥಾಕೌತುಕವೊಂದನ್ನು ಹೇಳಿದುದನ್ನು ನಾನು ಕೇಳಿದೆನು. ವೃಂದಾವನವನ್ನು ತಲುಪಿದ ಅವಳು ಅಲ್ಲಿದ್ದ ಒಬ್ಬ ಪ್ರಖ್ಯಾತ ಸಾಧುವಿಗಾಗಿ\footnote{1. ಬಂಗಾಳದ ಶ‍್ರೀ ಚೈತನ್ಯರ ಸಂನ್ಯಾಸಿ ಶಿಷ್ಯನಾದ ಸನಾತನ; ಭಕ್ತನಾಗುವುದಕ್ಕಾಗಿ ಬಂಗಾಳದ ನವಾಬನ ಆಸ್ಥಾನದಲ್ಲಿನ ಸಚಿವಪದವಿಯನ್ನು ತ್ಯಜಿಸಿದವನು.} ಹೇಳಿ ಕಳುಹಿಸಿದಳು. ವೃಂದಾವನದಲ್ಲಿ ಸ್ತ್ರೀಯರು ಪುರುಷರನ್ನು ಸಂದರ್ಶಿಸ ತಕ್ಕದ್ದಲ್ಲ ಎಂಬ ನಿಲುವಿನ ಮೇಲೆ ಅವನು ಅವಳ ಕರೆಯನ್ನು ನಿರಾಕರಿಸಿದನು. ಇದು ಮೂರು ಬಾರಿ ನಡೆಯಿತು. ಆ ನಂತರ ಮೀರಾಬಾಯಿಯೇ ಅವನ ಬಳಿಗೆ ಹೋಗಿ, ವೃಂದಾವನದಲ್ಲಿ ಪುರುಷರು ಇರುವರೆಂದು ತನಗೆ ತಿಳಿದಿರಲಿಲ್ಲ, ಇರುವುದು ಕೃಷ್ಣ ನೊಬ್ಬನೇ ಎಂದುಕೊಂಡಿದ್ದೆ ಎಂದಳು. ಆಶ್ಚರ್ಯಚಕಿತನಾದ ಆ ಸಾಧುವಿನ ಎದುರಿಗೆ ತನ್ನನ್ನು ತಾನು ಸಂಪೂರ್ಣವಾಗಿ ಅನಾವರಣ ಮಾಡಿಕೊಂಡ ಮೀರಾ, “ಮೂರ್ಖ, ನಿನ್ನನ್ನು ನೀನು ಪುರುಷನೆಂದು ಕರೆದುಕೊಳ್ಳುತ್ತೀಯಾ?” ಎಂದು ಛೇಡಿಸಿದಳು. ಭಯವಿಸ್ಮಿತನಾದ ಆ ಸಾಧು ಅವಳ ಕಾಲಿಗೆ ಬೀಳಲು, ಅವಳು ಅವನನ್ನು ತಾಯಿ ಮಗುವನ್ನು ಸಂತೈಸುವಂತೆ ಸಂತೈಸಿದಳು.

ಈ ಹೊತ್ತು ಸ್ವಾಮಿಗಳ ಮಾತು ಅಕ್ಬರನ ಕಡೆಗೆ ತಿರುಗಿತು. ಸಾಮ್ರಾಟನ ಆಸ್ಥಾನಕವಿಯಾಗಿದ್ದ ತಾನ್ ಸೇನನ ಹಾಡೊಂದನ್ನು ಅವರು ನಮಗಾಗಿ ಹಾಡಿದರು:

\begin{myquote}
ಪಟ್ಟವೇರಿದ ದೊರೆಯೆ, ಹೇ ನರೋತ್ತಮನೆ\\ನೀನೀ ದೆಹಲಿಯ ಸಾಮ್ರಾಟ.\\ಪುಣ್ಯದ ಕ್ಷಣವದು ನೀ ನೃಪನಾದುದು\\ಗಂಟೆ-ನಿಮಿಷಗಳವು ಪುಣ್ಯದವು.\\ರಾಜಲಾಂಛನವು ಘನಸಿಂಹಾಸನ\\ಚಿರವಾಗುಳಿಯಲಿ ಕಿರೀಟ ನಿನ್ನಯ\\ಉಳಿಯಲಿ ರಾಜ್ಯ ಹುಮಾಯೂನ್ ಪುತ್ರನೆ\\ಜಾಗೃತನಾಗಿರು ಚಿರಕಾಲ;\\ರವಿನಲಿವಿನ ನೀ ನರರೊಳ ದೇವನು\\ನೀನೀ ದೆಹಲಿಯ ಸಾಮ್ರಾಟ!
\end{myquote}

ಅನಂತರ ಮಾತು ಎಂದಿಗೂ ಶರಣಾಗತನಾಗಲೊಲ್ಲದ “ನಮ್ಮ ರಾಷ್ಟ್ರವೀರ”ನಾದ ಪ್ರತಾಪಸಿಂಹನ ಕಡೆಗೆ ತಿರುಗಿತು. ಒಂದು ಸಲ ಅವನು ನಿಜವಾಗಿಯೂ ಕೈಚೆಲ್ಲಿ ಶರಣಾಗುವ ಹಂತ ತಲುಪಿದ್ದ - ಚಿತ್ತೂರಿನಿಂದ ಓಡಿಹೋದಾಗ. ರಾಣಿ ಅವನ ಸಂಜೆಯ ಊಟಕ್ಕಾಗಿ ತನ್ನ ಕೈಯಿಂದಲೇ ಏನೋ ಒಂದು ಅಡುಗೆ ಮಾಡಿದ್ದಳು. ಮಕ್ಕಳಿಗೆಂದು ತೆಗೆದಿರಿಸಿದ ಭಾಗದ ಮೇಲೆ ಹಸಿದ ಬೆಕ್ಕೊಂದು ಎಗರಿ ಬಿತ್ತು. ಮಕ್ಕಳು ಹಸಿವೆ ಎಂದು ಅಳಲಾರಂಭಸಿದವು. ಉಕ್ಕಿನ ಹೃದಯದ ಮೇವಾರದ ರಾಜನಿಗೆ ಸಹ ಆಗ ನಿಜಕ್ಕೂ ತತ್ತರಿಸುವಂತಾಯಿತು. ಶರಣಾದರೆ ತನ್ನೆಲ್ಲ ಕಷ್ಟಗಳೂ ಕೊನೆಗೊಳ್ಳುವ ಪ್ರಲೋಭನೆ ಅವನ ಮನಸ್ಸನ್ನು ತುಂಬಿಕೊಂಡಿತು. ಆ ಕ್ಷಣದಲ್ಲಿ ತನ್ನ ಅಸಮ ಪ್ರತಿರೋಧವನ್ನು ಬಿಟ್ಟುಕೊಟ್ಟು ಅಕ್ಬರನ ಬಳಿಗೆ ರಾಜಿಗಾಗಿ ತನ್ನ ಮಂತ್ರಿಯನ್ನು ಕಳಿಸುವ ಯೋಚನೆ ಮಾಡಿದ - ಕ್ಷಣಕಾಲ ಮಾತ್ರ. ದೈವೇಚ್ಛೆ ತನ್ನವರನ್ನು ರಕ್ಷಿಸಿಯೇ ರಕ್ಷಿಸುತ್ತದೆ. ಈ ಯೋಚನೆಯ ಚಿತ್ರ ಅವನ ಮನಸ್ಸನ್ನು ಹಾದುಹೋಗುತ್ತಿರುವಾಗಲೇ, ಪ್ರಖ್ಯಾತನಾದ ರಜಪೂತ ನಾಯಕನೊಬ್ಬನ ದೂತ ಒಂದು ಸಂದೇಶವನ್ನು ತಂದ: “ಅನ್ಯ ರೊಡನೆ ಮಿಶ್ರವಾಗದ ರಕ್ತದ ಒಬ್ಬನೇ ಒಬ್ಬ ನಮ್ಮವನು ಉಳಿದಿದ್ದಾನೆ. ಅವನ ಶಿರವೂ ಸಹ ನೆಲದ ಧೂಳನ್ನು ಮುಟ್ಟಿತು ಎಂದಾಗದಿರಲಿ!” ತಕ್ಷಣವೇ ಪ್ರತಾಪನ ಜೀವ ಧೈರ್ಯದ ನಿಟ್ಟುಸಿರನ್ನು ಎಳೆದುಕೊಂಡಿತು; ಅವನ ಶ್ರದ್ಧೆ ಜಾಗೃತವಾಯಿತು; ಮೇಲೆದ್ದ ಅವನು ದೇಶದಲ್ಲಿದ್ದ ಶತ್ರುಗಳನ್ನು ಸಂಪೂರ್ಣವಾಗಿ ನಿರ್ನಾಮ ಮಾಡಿ ಉದಯಪುರಕ್ಕೆ ಹಿಂದಿರುಗಿದ.

ಅನಂತರ ರಾಜಕುವರಿ ಕೃಷ್ಣಕುಮಾರಿಯ ಸ್ವಯಂವರದ ಕಥೆ. ಅನೇಕ ಯುವ ರಾಜರುಗಳು ಏಕಕಾಲದಲ್ಲಿ ಅವಳ ಮೇಲೆ ಮನಸ್ಸನ್ನಿಟ್ಟು ಅವಳನ್ನು ವರಿಸಲು ಬಯಸಿದರು. ರಾಜದ್ವಾರದಲ್ಲಿ ಮೂರು ಸೈನ್ಯಗಳು ಬಂದು ಮುತ್ತಿಗೆ ಹಾಕಿದಾಗ, ಮಗಳಿಗೆ ವಿಷ ಕೊಡುವುದರ ಹೊರತು ಅವಳ ತಂದೆಗೆ ಬೇರೇನೂ ತೋರಲಿಲ್ಲ. ಅವಳ ಚಿಕ್ಕಪ್ಪನಿಗೆ ಆ ಕೆಲಸವನ್ನು ವಹಿಸಲಾಯಿತು. ನಿದ್ರಿಸುತ್ತಿರುವ ಅವಳಿಗೆ ವಿಷಪ್ರಾಶನ ಮಾಡಿಸುವುದಕ್ಕೆಂದು ಕೊಠಡಿಯನ್ನು ಪ್ರವೇಶಿಸಿದ ಅವನಿಗೆ ತಾನು ಕೈಯಾರೆ ಬೆಳೆಸಿದ ಆ ಮಗುವಿನ ಯೌವನ ಸೌಂದರ್ಯಗಳನ್ನು ನೋಡಿ ಹಾಗೆ ಮಾಡಲು ಮನಸ್ಸು ಬರಲಿಲ್ಲ. ಅಷ್ಟರಲ್ಲಿ ಸಪ್ಪಳದಿಂದ ರಾಜಕುಮಾರಿಗೆ ಎಚ್ಚರವಾಯಿತು. ಚಿಕ್ಕಪ್ಪನು ತಾನೇಕೆ ಬಂದಿರುವೆನೆಂಬುದನ್ನು ಹೇಳಲು, ಅವಳು ನಗುನಗುತ್ತ ತಾನೇ ಕೈಯೊಡ್ಡಿ ವಿಷದ ಬಟ್ಟಲನ್ನು ತೆಗೆದುಕೊಂಡು ಕುಡಿದುಬಿಟ್ಟಳು. ಇಂತಹ ರಜಪೂತ ವೀರರ ಕಥೆಗಳಿಗೆ ಕೊನೆಯೆಂಬುದಿಲ್ಲ.

\textbf{ಆಗಸ್ಟ್ ೨೦.}

ಶನಿವಾರದ ದಿನ ಸ್ವಾಮಿಗಳು ಮತ್ತು ಸೂಂಗ್ ಎಂಬುವರು ಒಂದೆರಡು ದಿನಗಳ ಮಟ್ಟಿಗೆ ಅಮೆರಿಕಾದ ರಾಯಭಾರಿ ದಂಪತಿಗಳ ಅತಿಥಿಗಳಾಗಿರುವುದಕ್ಕೆಂದು ಡಾಲ್ ಸರೋವರಕ್ಕೆ ಹೋದರು; ಹೋದವರು ಸೋಮವಾರ ಹಿಂದಿರುಗಿದರು. ಮಂಗಳ ವಾರದ ದಿನ ಸ್ವಾಮಿಗಳು ಹೊಸ ಮಠವೆಂದು ನಾವು ಕರೆಯುತ್ತಿದ್ದಲ್ಲಿಗೆ ಬಂದರು; ಗಂಡೇರ್ಬಾಲ್ಗೆ ಹೊರಡುವ ಮುನ್ನ ಕೆಲವು ದಿನಗಳು ನಮ್ಮ ಜೊತೆಯಲ್ಲಿರುವುದಕ್ಕೆ ಸಾಧ್ಯವಾಗುವ ಹಾಗೆ ಅವರ ದೋಣಿಯು ನಮ್ಮ ದೋಣಿಯ ಜೊತೆಗೆ ಹತ್ತಿರದಲ್ಲಿಯೇ ಚಲಿಸುತ್ತಿತ್ತು.

\begin{center}
\textbf{ಸಂಪಾದಕರ ಕೊನೆಯ ನುಡಿ}
\end{center}

ಸ್ವಾಮಿಗಳು ಗಂಡೇರ್ ಬಾಲ್ದಿಂದ ಅಕ್ಟೋಬರ್ ಮೊದಲನೆಯ ವಾರದ ಹೊತ್ತಿಗೆ ಹಿಂದಿರುಗಿದರು; ತುರ್ತು ಕಾರಣಗಳಿಗಾಗಿ ಕೆಲವು ದಿನಗಳು ಬಯಲು ಪ್ರದೇಶಕ್ಕೆ ಇಳಿದು ಹೋಗಬೇಕಾಗಿರುವ ತಮ್ಮ ಉದ್ದೇಶವನ್ನು ಅರುಹಿದರು. ಯೂರೋಪಿಯ ನ್ನರ ಗುಂಪು ಚಳಿಗಾಲ ಬರುತ್ತಲೂ ಲಾಹೋರ್, ದೆಹಲಿ, ಆಗ್ರಾ ಮುಂತಾದ ಉತ್ತರ ಭಾರತದ ಮುಖ್ಯವಾದ ನಗರಗಳನ್ನು ನೋಡುವುದಕ್ಕೆಂದು ಅದಾಗಲೇ ಯೋಜನೆ ಹಾಕಿತ್ತು. ಆದಕಾರಣ ಎರಡು ಗುಂಪುಗಳೂ ಒಟ್ಟಿಗೆ ಲಾಹೋರಿಗೆ ಹಿಂದಿರುಗುವು ದೆಂದು ನಿರ್ಧರಿಸಿದವು. ಉಳಿದವರಿಗೆ ಅವರ ಯೋಜನೆಯಂತೆ ಉತ್ತರ ಭಾರತ ಪ್ರವಾಸ ಕೈಕೊಳ್ಳಲು ಬಿಟ್ಟು, ಇಲ್ಲಿಂದ ಸ್ವಾಮಿಗಳು ಮತ್ತು ಅವರ ಗುಂಪು ಕಲ್ಕತ್ತಕ್ಕೆ ಹಿಂದಿರುಗಿದರು.

