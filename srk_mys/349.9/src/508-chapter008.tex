
\chapter{ಅಧ್ಯಾಯ ೮: ಪಾಂಡ್ರೆಂಥನದ ದೇಗುಲ}

ವ್ಯಕ್ತಿಗಳು: ಸ್ವಾಮಿ ವಿವೇಕಾನಂದರು, ಗುರುಭಾಯಿಗಳು, ಧೀರಮಾತಾ, ಜಯಾ ಎಂಬ ಹೆಸರಿನವಳು ಮತ್ತು ಸೋದರಿ ನಿವೇದಿತಾಳನ್ನೊಳಗೊಂಡ ಯೂರೋಪಿ ಯನ್ ಶಿಷ್ಯರುಗಳ ಮತ್ತು ಅತಿಥಿಗಳ ಗುಂಪು.

ಸ್ಥಳ: ಕಾಶ್ಮೀರ.

ಕಾಲ: ೧೮೯೮ರ ಜುಲೈ ೧೬ರಿಂದ ೧೯ರವರೆಗೆ.

\textbf{ಜುಲೈ ೧೬.}

ಶಿಷ್ಯರಲ್ಲಿ ಒಬ್ಬರಿಗೆ ಮಾರನೆಯ ದಿನ ಸ್ವಾಮಿಗಳ ಜೊತೆಗೆ ಸಣ್ಣದೊಂದು ದೋಣಿಯಲ್ಲಿ ನದೀಪ್ರವಾಹದೊಂದಿಗೆ ಸ್ವಲ್ಪ ದೂರ ಹೋಗಿ ಬರುವ ಅವಕಾಶ ದೊರಕಿತು. ದೋಣಿ ಸಾಗುತ್ತಿದ್ದ ಹಾಗೆ, ಅವರು ಒಂದಾದ ಮೇಲೆ ಒಂದರಂತೆ ರಾಮಪ್ರಸಾದನ ಕವನಗಳನ್ನು ಹಾಡುವುದಕ್ಕೆ ಆರಂಭಿಸಿದರು; ಆಗೊಮ್ಮೆ ಈಗೊಮ್ಮೆ ಒಂದೊಂದು ಕವನವನ್ನು ಭಾಷಾಂತರಿಸಿಯೂ ಹೇಳುತ್ತಿದ್ದರು:

\begin{myquote}
ಹೇ ತಾಯಿ, ನಾ ನಿನ್ನ ಕೂಗಿ ಕರೆವೆ.\\ತಾಯಿ ಕೋಪದಿ ಮಗುವ ಥಳಿಸುತಿದ್ದರು ಕೂಡ\\ಮಗು ಕೂಗಿ ಕರೆಯುವುದು “ಅಮ್ಮ!” ಎಂದೇ, ಅದಕೆ.\\ನನ್ನ ಕಣ್ಣಿಗೆ ನೀನು ಕಾಣದಿದ್ದರು ಕೂಡ\\ಕಳೆದುಹೋಗಿರುವ ಮಗು ನಾನಲ್ಲವಮ್ಮ!\\ಹಾಗಿದ್ದರೂ ನಾನು ನಿನ್ನನ್ನು ಕೂಗುವುದು\\“ಹೇ ತಾಯಿ! ಅಮ್ಮಾ! ಎಲ್ಲಿರುವಿ?” ಎಂದೇ.
\end{myquote}

ಅನಂತರ, ಖತಿಗೊಂಡ ಮಗುವಿನ ದುಮ್ಮಾನದಿಂದ ಏನೋ ಒಂದನ್ನು ಹೇಳಿ, ಕೊನೆಗೆ ಕವನವು

“ಇನ್ನೊಬ್ಬಳನು ತಾಯಿ ಎಂದು ಕರೆಯುವ ಮಗುವು ಖಂಡಿತಕು ನಾನಾಗಲಾ ರೆನಮ್ಮ!” ಎಂದು ಮುಗಿಯುತ್ತದೆ.

\textbf{ಜುಲೈ ೧೭.}

ಅವರು ಧೀರಮಾತಾಳ ಡುಂಗಾಕ್ಕೆ ಬಂದು ಭಕ್ತಿಯ ವಿಚಾರ ಮಾತನಾಡಿದ್ದು ಅದರ ಮಾರನೆಯ ದಿನವಿರಬೇಕು. ಮೊದಲು ಶಿವ ಉಮೆ ಇಬ್ಬರನ್ನೂ ಒಂದೆಂದು ಭಾವಿಸುವ ಆ ವಿಶಿಷ್ಟ ಕುತೂಹಲದ ಹಿಂದೂ ಚಿಂತನೆ. ಪದಗಳಲ್ಲಿ ನುಡಿಯುವು ದೇನೋ ಸುಲಭ, ಆದರೆ ಅವುಗಳ ಹಿಂದೆ ಧ್ವನಿ ಇಲ್ಲದೆ ಹೋದರೆ, ಹೇಗೆ ಅವು ನಿರ್ಜೀವವೆನ್ನಿಸಿಬಿಡುವುವು! ನಂತರ, ಜೊತೆಗೆ ಅಲ್ಲಿನ ಸುಂದರ ಪರಿಸರ - ಕಣ್ಣಿಗೆ ಮೆತ್ತಿದಂತೆ ಕಾಣುತ್ತಿದ್ದ ಶ‍್ರೀನಗರ, ಎತ್ತರವಾದ ಲಾಂಬರ್ಡಿ ಪಾಪ್ಲರ್ ವೃಕ್ಷಗಳು, ದೂರದಲ್ಲಿ ಕಾಣುವ ಮಂಜುಶಿಖರಗಳು. ಅಲ್ಲಿ, ಆ ನದಿಯ ಕಣಿವೆಯ ಪ್ರಪಾತದಲ್ಲಿ, ಎರಡು ಕಡೆಯೂ ಇದ್ದ ಉನ್ನತ ಪರ್ವತಗಳ ನಡುವಣ ತಾಣದಲ್ಲಿ, “ಜಗದೀಶ್ವರನು ಅರ್ಧನಾರೀಶ್ವರನ ರೂಪ ತಳೆದುದನ್ನು” ಹಾಡಲಾರಂಭಿಸಿದರು. “ಒಂದು ಕಡೆ ಸುಂದರ ಪುಷ್ಪಹಾರಗಳು; ಇನ್ನೊಂದು ಕಡೆ ಸರ್ಪಾಭರಣಗಳು, ಮೂಳೆ ಕುಂಡಲಗಳು. ಒಂದು ಕಡೆ ಕಪ್ಪಾಗಿ, ಸುರುಳಿಸುರುಳಿಯಾಗಿ ಇಳಿಬಿದ್ದಿರುವ ವೇಣಿ; ಇನ್ನೊಂದು ಕಡೆ ಜಡ್ಡುಗಟ್ಟಿದ ಜಟೆಗಳು”. ಆ ನಂತರ, ಇದ್ದಕ್ಕಿದ್ದಂತೆ ಅದೇ ಕಲ್ಪನೆಯ ಇನ್ನೊಂದು ರೂಪಕ್ಕೆ ನೆಗೆದು, ಹಾಡಿದರು:

\begin{myquote}
ದೇವನೇ ತಳೆದಿಹನು ಕೃಷ್ಣ-ರಾಧೆಯ ರೂಪ-\\ಹರಿಯುತಿದೆ ಪ್ರೇಮವದು ಸುಳಿಯು ಸಾಸಿರವಾಗಿ.\\ಯಾರು ಬೇಕೆನ್ನುವರೊ ಅವರುಕೊಳ್ಳುವರಿದನು.
\end{myquote}

\begin{myquote}
ಹರಿಯುತಿದೆ ಪ್ರೇಮವದು ಸುಳಿಯು ಸಾಸಿರವಾಗಿ-\\ಪ್ರೇಮಿಸಿದ, ಪ್ರೇಮದಲೆ ನುಗ್ಗಿ ಸಾಗಿದ ಮೇಲೆ\\ಚೆಲ್ಲಿ ಸೂಸಿತು ಜೀವ - ಸಂತೋಷ, ಆನಂದ!
\end{myquote}

ಭಾವದಲ್ಲಿ ಅವರು ಎಷ್ಟೊಂದು ಲೀನವಾಗಿಬಿಟ್ಟಿದ್ದರೆಂದರೆ, ಸಿದ್ಧವಾಗಿದ್ದ ತಿಂಡಿ ಅವರನ್ನು ಬಹು ಹೊತ್ತಿನಿಂದ ಕಾಯುತ್ತಿತ್ತು. ಕೊನೆಗೊಮ್ಮೆ, ಮನಸ್ಸಿಲ್ಲದ ಮನಸ್ಸಿನಿಂದ, “ಇಂತಹ ಭಕ್ತಿ ಮನಸ್ಸನ್ನು ತುಂಬಿರುವಾಗ ತಿಂಡಿ ಯಾರಿಗೆ ಬೇಕೆನಿಸುತ್ತದೆ?” ಎನ್ನು ತ್ತ ತಿಂಡಿ ತಿಂದರು; ಪುನಃ ಬೇಗನೆ ವಿಷಯಕ್ಕೆ ಬಂದರು.

ಈಗಲೋ ಅಥವಾ ಇನ್ನೊಂದು ಸಂದರ್ಭದಲ್ಲಿಯೋ - ಕಾರ್ಯಮಗ್ನರಾಗಿರುವಾಗ ತಾವು ಕೃಷ್ಣ-ರಾಧೆಯರ ಬಗ್ಗೆ ಮಾತನಾಡುವುದಿಲ್ಲ ಎಂದವರು ನುಡಿದದ್ದು. ಕುಶಲಕರ್ಮಿಗಳನ್ನು ನಿರ್ಮಿಸುವವನು ಶಿವ; ಕರ್ಮದಲ್ಲಿ ತೊಡಗಿಕೊಂಡಿರುವವನು ಶಿವನನ್ನು ನೆಮ್ಮಿಕೊಳ್ಳಬೇಕು.

ಮಾರನೆಯ ದಿನ ಶ‍್ರೀರಾಮಕೃಷ್ಣರ ವಿಲಕ್ಷಣವಾದ ವಚನವೊಂದನ್ನು ಹೇಳಿದರು - ಇತರರನ್ನು ವಿಮರ್ಶಿಸುವವರು ಮಧುವನ್ನಾರಿಸಿಕೊಳ್ಳುವ ಜೇನಿನಂತೆಯೂ ಇರಬಹುದು, ಕೀವನ್ನು ಆರಿಸಿಕೊಳ್ಳವ ನೊಣದಂತೆಯೂ ಇರಬಹುದು ಎಂದು.

ಆ ನಂತರ ನಾವು ಇಸ್ಲಾಮಾಬಾದ್ಗೆಂದು ಅಲ್ಲಿಂದ ಹೊರಟೆವು; ನಿಜವಾಗಿ, ಕೊನೆಗೆ ಹೋಗಿ ತಲುಪಿದ್ದು ಅಮರನಾಥಕ್ಕೆ.

\textbf{ಜುಲೈ ೧೯.}

ಮೊದಲ ದಿನದ ಮಧ್ಯಾಹ್ನ ಝೀಲಮ್​ ನದಿಯ ದಡದ ಕಾಡಿನಲ್ಲಿದ್ದ ಪಾಂಡ್ರೆಂಥನ ದೇಗುಲವನ್ನು ಪತ್ತೆಮಾಡಿದೆವು - ನಾವು ಬಹುಕಾಲದಿಂದ ಹುಡುಕುತ್ತಿದ್ದ ಪಾಂಡ್ರೆಸ್ಥಾನ, ಅಥವಾ ಪಾಂಡವರ ಸ್ಥಾನ ಇದೇಯೆ?

ಅದೊಂದು ಕೊಳದಲ್ಲಿ ಮುಳುಗಿತ್ತು; ಕೊಳವನ್ನು ಅಂತರಗಂಗೆ ಆವರಿಸಿಕೊಂಡಿತ್ತು; ಅದರ ನಡುವಿನಿಂದ ದೇಗುಲ ತಲೆಯೆತ್ತಿತ್ತು. ಪುರಾತನ ದೇಗುಲ; ಬೂದುಬಣ್ಣದ ಗಟ್ಟಿಯಾದ ಸುಣ್ಣಗಲ್ಲಿನಿಂದ ಕಟ್ಟಲ್ಪಟ್ಟದ್ದು. ದೇಗುಲವೆಂದರೆ ಸಣ್ಣದೊಂದು ಕೊಠಡಿ; ಅದಕ್ಕೆ ನಾಲ್ಕು ದಿಕ್ಕುಗಳಿಗೆ ನಾಲ್ಕು ಬಾಗಿಲುಗಳು. ಹೊರಗೆ ಕಾಣುವಂತೆ ಅದೊಂದು ತುದಿ ಚೂಪಾಗುತ್ತ ನಡೆದ ಪಿರಮಿಡ್ - ಅದರ ಕತ್ತರಿಸಿದಂತಿದ್ದ ತುದಿಯ ಮೇಲೊಂದು ಪೊದೆ ಬೆಳೆದು ನಿಂತಿತ್ತು. ದೇಗುಲದ ಕಂಭಗಳು ಹೊಸ್ತಿಲಿನ ನಾಲ್ಕು ತೋಡುಗುಂಡಿಗಳ ಮೇಲೆ ಆಧಾರಿತವಾಗಿದ್ದವು. ತ್ರಿಕೋನ ಹಾಗೂ ತ್ರಿದಳಾಕೃತಿಗಳ ಅಪರೂಪ ಸೇರ್ಪಡೆ ಯುಳ್ಳ ಕಮಾನುಗಳು ನೇರವಾದ ಉತ್ತರಾಸಿನ ಮೇಲೆ ಕುಳಿತ ಶಿಲ್ಪ; ಅದ್ಭುತವೆನಿಸು ವಷ್ಟು ಘನವಾಗಿ ಕಟ್ಟಿದ ಆ ದೇಗುಲದ ಆವಶ್ಯಕ ರೇಖೆಗಳನ್ನೇ ಮರೆಯಾಗಿಸುವಂತಹ ಅಲಂಕರಣಗಳು...

ಸ್ವಾಮಿಗಳೊಬ್ಬರನ್ನು ಬಿಟ್ಟು ಉಳಿದ ನಮ್ಮೆಲ್ಲರ ಪಾಲಿಗೂ ಇದು ಭಾರತೀಯ ಪುರಾ ತತ್ತ್ವದ ಕ್ಷೇತ್ರಕ್ಕೆ ಮೊಟ್ಟಮೊದಲ ಇಣಕು ನೋಟವಾಗಿತ್ತು. ಆದಕಾರಣ, ಅವರು ತಾವೆಲ್ಲವನ್ನೂ ನೋಡಿ ಆದ ಮೇಲೆ ನಮಗೆ ದೇಗುಲದ ಒಳಭಾಗದ ವೀಕ್ಷಣೆ ಹೇಗೆ ಮಾಡಬೇಕು ಎಂಬುದನ್ನು ತಿಳಿಸಿಕೊಟ್ಟರು.

ಛಾವಣಿಯ ಮಧ್ಯದಲ್ಲಿ ಉತ್ತರಮುಖಿಯ ನಾಲ್ಕು ದಿಕ್ಕುಗಳ ಕಡೆಗೆ ಶೃಂಗಗಳನ್ನುಳ್ಳ ಚೌಕಾಕೃತಿಯಲ್ಲಿ ಅಳವಡಿಸಲಾಗಿದ್ದ ಒಂದು ದೊಡ್ಡ ಸೂರ್ಯಫಲಕವಿದ್ದಿತು. ಇದು ಛಾವಣಿಯ ನಾಲ್ಕು ಮೂಲೆಗಳಲ್ಲಿ ನಾಲ್ಕು ಸಮತ್ರಿಭುಜಾಕೃತಿಗಳನ್ನುಂಟುಮಾಡಿತ್ತು. ಇವುಗಳ ತುಂಬ ಉಬ್ಬುಗೆತ್ತನೆಯ ಪದ್ಧತಿಯಲ್ಲಿ ಕೆತ್ತಲ್ಪಟ್ಟ ಶಿಲ್ಪಗಳು. ಸರ್ಪಗಳಿಂದಸುತ್ತಲ್ಪಟ್ಟ ಸುಂದರ ಸ್ತ್ರೀ-ಪುರುಷ ಆಕೃತಿಗಳು. ಗೋಡೆಗಳ ಮೇಲೆ ಸ್ತೂಪಗಳಿದ್ದಂತೆ ಕಾಣುವ ಖಾಲಿ ಜಾಗಗಳು.

ಹೊರಭಾಗದಲ್ಲಿಯೂ ಅಂತೆಯೇ ಶಿಲ್ಪಾಕೃತಿಗಳನ್ನು ಕೆತ್ತಲಾಗಿದ್ದಿತು. ಒಂದು ತ್ರಿದಳಾ ಕೃತಿಯ ಕಮಾನಿನಲ್ಲಿ - ಪೂರ್ವದ್ವಾರದ ಮೇಲೆ ಎಂದು ನನ್ನ ನೆನಪು- ಉಪದೇಶದ ಭಂಗಿಯಲ್ಲಿ ಕೈ ಮೇಲಕ್ಕೆತ್ತಿ ನಿಂತಿರುವ ಬುದ್ಧನ ವಿಗ್ರಹ. ಗೋಡೆಗೆ ಆಸರೆಯಾಗಿ ಕಟ್ಟಿದ ಬುಡ ಚಾಚಿನಸುತ್ತ, ಬೋದಿಗೆಗೆ ತಗುಲಿದಂತೆ ಕೆತ್ತಲ್ಪಟ್ಟ, ಮುಖ ಸ್ಪಷ್ಟವಾಗಿ ಕಾಣಿಸದಷ್ಟು ನಾಶವಾಗಿರುವ, ವೃಕ್ಷವೊಂದರ ಕೆಳಗೆ ಕುಳಿತಂತಿರುವ ಸ್ತ್ರೀ -ಬುದ್ಧನ ತಾಯಿ ಮಾಯಾದೇವಿಯ ವಿಗ್ರಹ. ಉಳಿದ ಮೂರು ದ್ವಾರಗಳ ಮೇಲ್ಭಾಗದ ಕಮಾನುಗಳು ಖಾಲಿ; ಆದರೆ ಕೊಳದ ಕಡೆಯ ಕಮಾನಿನಿಂದ ಬಿದ್ದಂತೆ ತೋರುತ್ತಿದ್ದ ಚಪ್ಪಡಿಯೊಂದಿತ್ತು; ಅದರ ಮೇಲೊಂದು ಕೆಟ್ಟದಾಗಿ ಕೆತ್ತಿದ ರಾಜನ ವಿಗ್ರಹ; ಇದು ಸೂರ್ಯನನ್ನು ಪ್ರತಿನಿಧಿಸುತ್ತದೆಂದು ಸ್ಥಳೀಯ ಜನರೆನ್ನುವರು.

ಈ ಪುಟ್ಟ ದೇಗುಲದ ಗಾರೆ ಕೆಲಸವಂತೂ ಅತ್ಯುತ್ಕೃಷ್ಟವಾಗಿತ್ತು-ಬಹುಶಃ ಇಷ್ಟು ಕಾಲ ಉಳಿದು ಬಂದಿರುವುದಕ್ಕೆ ಅದೇ ಕಾರಣವೆನ್ನಬಹುದು. ಒಂದೊಂದು ಕಲ್ಲಿನ ಭಾಗವನ್ನೂ ಗೋಡೆಯ ಒಂದು ಇಟ್ಟಿಗೆಯಾಗುವಂತೆ ಅಲ್ಲ, ಶಿಲ್ಪಿಯ ಕಲ್ಪನೆಯ ಒಂದಂಶ ವನ್ನೇ ಸಾಕ್ಷಾತ್ಕರಿಸುವಂತೆ ಕೆತ್ತಲಾಗಿತ್ತು. ಮೂಲೆಯನ್ನು ತಿರುಗಿಸಿ ಎರಡು ಗೋಡೆಗಳ ಭಾಗವಾಗುವಂತೆ, ಅಥವಾ ಮೂರು ಗೋಡೆಗಳಿಗೂ ಸಾಮಾನ್ಯವಾಗುವಂತೆ! ಇದರಿಂದಾಗಿ ಕಟ್ಟಡವು ತುಂಬ ಪ್ರಾಚೀನ ಎಂದು- ಬಹುಶಃ ಮಾರ್ತಾಂಡನಿಗಿಂತಲೂ ಹಿಂದಿನದೆಂದು- ಊಹಿಸಬಹುದಾಗಿದ್ದಿತು. ಕೆಲಸಗಾರರು ಕಲ್ಲಿನ ಕಟ್ಟಡದಲ್ಲಿ ಮರ ಗೆಲಸದ ಕೌಶಲವನ್ನು ಪ್ರದರ್ಶಿಸಿದಂತೆ ಇತ್ತು! ದೇಗುಲದಸುತ್ತ ಇದ್ದ ನೀರು, ಒಳಭಾಗದಲ್ಲಿದ್ದ ಪವಿತ್ರ ಚಿಲುಮೆಯೊಂದರ ಹೊರ ಹರಿವಾಗಿದ್ದಿತು. ಸ್ವಾಮಿಗಳ ಅಭಿಪ್ರಾಯದಂತೆ ದೇಗುಲವನ್ನೇ ಪವಿತ್ರ ಚಿಲುಮೆಯನ್ನು ಗರ್ಭಮಂದಿರದಲ್ಲಿ ಒಳಗೊಳ್ಳುವುದಕ್ಕಾಗಿ ಕಟ್ಟ ಲಾಗಿತ್ತು.

ಸ್ವಾಮಿಗಳಿಗೋ, ಈ ಸ್ಥಳವು ಅನೇಕ ಸಂಗತಿಗಳನ್ನು ಸೂಚಿಸುತ್ತಿದ್ದಂತಹ ಸಂತಸ. ಸ್ಪಷ್ಟವಾಗಿ ಇದು ಬೌದ್ಧಧರ್ಮದ ಸ್ಮಾರಕ. ಕಾಶ್ಮೀರದ ಇತಿಹಾಸವನ್ನು ಅವರೀಗಾಗಲೇ ನಾಲ್ಕು ಕಾಲಗಳಾಗಿ ವಿಂಗಡಿಸಿದ್ದರು; (೧) ವೃಕ್ಷ -ನಾಗಗಳ ಆರಾಧನೆಯ ಕಾಲ - ಅದರಿಂದಾಗಿಯೇ ಇಲ್ಲಿನ ಚಿಲುಮೆಗಳಿಗೆಲ್ಲ ನಾಗಗಳೆಂದೇ ಹೆಸರು (ವೇರಿನಾಗ, ಅನಂತನಾಗ ಇತ್ಯಾದಿ); (೨) ಬೌದ್ಧಧರ್ಮದ ಕಾಲ; (೩) ಸೂರ್ಯಾರಾಧನಾ ರೂಪದ ಹಿಂದೂ ಧರ್ಮದ ಕಾಲ ಮತ್ತು (೪) ಇಸ್ಲಾಂ ಧರ್ಮದ ಕಾಲ. ಸ್ವಾಮಿಗಳ ಅಭಿಪ್ರಾಯದಲ್ಲಿ ಈ ದೇಗುಲವು ಇವುಗಳಲ್ಲಿ ಒಂದು ಕಾಲವನ್ನು ಪ್ರತಿನಿಧಿಸುತ್ತಿತ್ತು.

ಶಿಲ್ಪಕಲೆಯು ಬೌದ್ಧಧರ್ಮದ ಒಂದು ವೈಶಿಷ್ಟ್ಯ ಎಂದರವರು. ಸೂರ್ಯಫಲಕ, ಅಥವಾ ಕಮಲ, ಅವರ ತೀರ ಸಾಮಾನ್ಯ ಅಲಂಕರಣದ ಪ್ರತೀಕ. ಸರ್ಪಗಳನ್ನುಳ್ಳ ಕೆತ್ತನೆಗಳು ಬೌದ್ಧಧರ್ಮಕ್ಕಿಂತ ಹಿಂದಿನವಾಗಿರಬಹುದು. ಆದರೆ ಸೂರ್ಯರಾಧನೆಯ ಕಾಲಕ್ಕೆ ಬರುವಷ್ಟರಲ್ಲಿ ಶಿಲ್ಪಕೌಶಲ ಖಿಲವಾಗಿತ್ತು - ಹಾಗಾಗಿ ಸೂರ್ಯವಿಗ್ರಹದ ಆ ಅವಸ್ಥೆ...

ಸೂರ್ಯಾಸ್ತದ ಸಮಯವಾಗಿತ್ತು. ಅದೆಂತಹ ಸೂರ್ಯಾಸ್ತ! ಪಶ್ಚಿಮದ ಪರ್ವತಗಳೆಲ್ಲವೂ ನೇರಳೆ ವರ್ಣದಲ್ಲಿ ನಳನಳಿಸುತ್ತಿದ್ದವು. ಉತ್ತರಕ್ಕೆ ಹೋದ ಹಾಗೆ ಅವು ಮಂಜಿನಿಂದಾಗಿ, ಮೋಡಗಳಿಂದಾಗಿ, ನೀಲಿಯಾಗಿ ಕಾಣುತ್ತಿದ್ದವು. ಹಸಿರು-ಹಳದಿಯಾಗಿದ್ದ ಆಗಸಕ್ಕೆ ಅರುಣರಾಗದ ಲೇಪನವಾಗುತ್ತಿರುವಂತಿತ್ತು - ಜ್ವಾಲೆಯಂತಹ ತಿಳಿಹಳದಿಯ ಪ್ರಕಾಶ, ಒಂದು ಬಗೆಯ ನೀಲಿ ಹಾಲುಬಿಳುಪಿನ ಹಿನ್ನೆಲೆಯಲ್ಲಿ. ನಾವು ನಿಬ್ಬೆರಗಾಗಿ ನಿಂತು ನೋಡುತ್ತಿದ್ದೆವು. ಆಗ ಗುರುಗಳು, ದೂರದಲ್ಲಿ ಕಾಣುತ್ತಿದ್ದ ಸುಲೇಮಾನ್ ಪೀಠ - ಈಗಾಗಲೇ ನಾವು ನೋಡಿ ಬಂದಿದ್ದ, ಇಷ್ಟಪಟ್ಟಿದ್ದ ದೇಗುಲವನ್ನು ಗುರುತಿಸಿದ್ದ ವರೇ ಉದ್ಗರಿಸಿದರು: “ದೇಗುಲ ನಿರ್ಮಾಣಕ್ಕೆ ಸ್ಥಳವನ್ನಾಯ್ದುಕೊಳ್ಳುವುದರಲ್ಲಿ ಹಿಂದೂವಿನದು ಅದೇನು ಪ್ರತಿಭೆ! ಅದ್ಭುತ ಪ್ರಕೃತಿಸೌಂದರ್ಯವುಳ್ಳ ಪರಿಸರವನ್ನೇ ಯಾವಾಗಲೂ ಆರಿಸುತ್ತಾನೆ! ತಖ್ತ್ ಇಡೀ ಕಾಶ್ಮೀರವನ್ನೇ ವಿಹಂಗಮನ ಮಾಡುತ್ತಿರುವಂತೆ ಇದೆ. ಹರಿ ಪರಬತ್ ಬಂಡೆ ನೀಲಿ ನೀರಿನಿಂದ ಮೇಲಕ್ಕೆದ್ದು ನಿಂತಿದೆ -ಮಲಗಿರುವ ಕೆಂಪು ಕೇಸರಿಯಂತೆ, ಕಿರೀಟಪ್ರಾಯವಾಗಿ. ಮಾರ್ತಾಂಡ ದೇಗುಲವಂತೂ ಇಡೀ ಕಣಿವೆಯನ್ನೇ ತನ್ನ ಪಾದದಡಿ ಬಿಟ್ಟಿದೆ!”

ಕಾಡಿನ ಅಂಚಿನ ಬಳಿ ನಮ್ಮ ದೋಣಿಗಳನ್ನು ಕಟ್ಟಲಾಗಿತ್ತು. ಅಲ್ಲಿನ ಆ ಮೌನವಾಂತ ಬುದ್ಧದೇಗುಲ - ನಾವು ಈಗತಾನೆ ನೋಡಿ ಬಂದುದು - ಸ್ವಾಮಿಗಳ ಮನಸ್ಸನ್ನು ಗಾಢವಾಗಿ ಕಲಕಿತ್ತು. ಆ ಸಂಜೆ ಮುಂಚೆಯೇ ನಾವೆಲ್ಲರೂ ಧೀರಮಾತಾಳ ದೋಣಿಮನೆಯಲ್ಲಿ ಸೇರಿದ್ದೆವು; ಸಂಭಾಷಣೆಯ ಸ್ವಲ್ಪಭಾಗವನ್ನು ಟಿಪ್ಪಣಿ ಮಾಡಿಕೊಳ್ಳಲಾಯಿತು.

ಕ್ರೈಸ್ತರ ಕ್ರಿಯಾವಿಧಿಗಳು ಬೌದ್ಧ ಕ್ರಿಯಾವಿಧಿಗಳಿಂದ ವ್ಯುತ್ಪನ್ನವಾದವು ಎಂದು ಗುರುದೇವರು ನುಡಿಯುತ್ತಿದ್ದಂತೆ, ನಮ್ಮಲ್ಲಿ ಒಬ್ಬಳಿಗೆ ಆವಾದ ಸರಿಬೀಳಲಿಲ್ಲ. “ಹಾಗಾದರೆ ಬೌದ್ಧ ಕ್ರಿಯಾವಿಧಿಗಳು ಬಂದುದೆಲ್ಲಿಂದ?” ಎಂದು ಅವಳು ಕೇಳಿದಳು.

“ವೈದಿಕ ಕ್ರಿಯಾವಿಧಿಗಳಿಂದ” ಎಂದರು ಸ್ವಾಮಿಗಳು.

“ಅಥವಾ ದಕ್ಷಿಣ ಯೂರೋಪಿನಲ್ಲೂ ಸಹ ಅವು ಇದ್ದುದರಿಂದ, ಅವುಗಳಿಗೂ ಕ್ರೈಸ್ತ ಹಾಗೂ ವೈದಿಕ ಕ್ರಿಯಾವಿಧಿಗಳಿಗೂ ಒಂದೇ ಮೂಲವಿರಬೇಕೆಂದು ಕಲ್ಪಿಸುವುದು ಹೆಚ್ಚು ಸೂಕ್ತವಲ್ಲವೆ?”

“ಇಲ್ಲ, ಇಲ್ಲ!” ಎಂದರವರು. “ಬೌದ್ಧಧರ್ಮ ಸಂಪೂರ್ಣವಾಗಿ ಹಿಂದೂಧರ್ಮದಲ್ಲೇ ಇದ್ದಿತೆಂಬುದನ್ನು ನೀವು ಮರೆತಿರಿ! ಜಾತಿಪದ್ಧತಿಯ ಮೇಲೂ ಆಕ್ರಮಣವಾಗಲಿಲ್ಲ - ಏಕೆ, ಅದಿನ್ನೂ ರೂಪುಗೊಂಡೇ ಇರಲಿಲ್ಲ! -ಬುದ್ಧ ಪ್ರಯತ್ನಿಸಿದ್ದು ಕೇವಲ ಆದರ್ಶದ ಪುನರುತ್ಥಾನಕ್ಕೆ. ಮನು ಹೇಳುವಂತೆ, ಈ ಜನ್ಮದಲ್ಲಿ ಭಗವಂತನನ್ನು ಸಾಕ್ಷಾತ್ಕರಿಸಿಕೊಂಡಿರುವವನೇ ಬ್ರಾಹ್ಮಣ. ಬೇಕಿದ್ದರೆ ಬುದ್ಧನೂ ಅದನ್ನುಗಳಿಸಿಕೊಳ್ಳಬಹುದಾಗಿತ್ತು”.

“ಆದರೆ ವೈದಿಕ ಹಾಗೂ ಕ್ರೈಸ್ತ ಕ್ರಿಯಾವಿಧಿಗಳು ಹೇಗೆ ಪರಸ್ಪರ ಸಂಬಂಧಿಸಿವೆ?” - ವಿರೋಧಿಸಿದ ಆ ಶಿಷ್ಯೆ ಪಟ್ಟು ಹಿಡಿದಳು. “ಅವರೆಡೂ ಒಂದೇ ಆಗಿರುವುದೆಂತು? ನಮ್ಮ ಪೂಜಾಕ್ರಮದ ಕೇಂದ್ರ ವಿಧಿಗೆ ಸಾಟಿಯಾಗುವಂತಹ ಯಾವುದೂ ನಿಮ್ಮಲ್ಲಿಲ್ಲ!”

“ಏಕೆ, ಇದ್ದೇ ಇದೆ” ಎಂದರು ಸ್ವಾಮಿಗಳು. “ವೈದಿಕ ಕ್ರಿಯಾಕರ್ಮಗಳಲ್ಲಿ ದಿವ್ಯ ಪೂಜೆಯೂ ಇದೆ, ಭಗವಂತನಿಗೆ ಆಹಾರ ನಿವೇದನವೂ ಇದೆ; ಇದೇ ನಿಮ್ಮ ಪ್ರಭು ಭೋಜನ ಸಂಸ್ಕಾರ, ನಮ್ಮ ಪ್ರಸಾದ ವಿನಿಯೋಗ. ನೈವೇದ್ಯವನ್ನು ನಿಮ್ಮಲ್ಲಿಯ ಹಾಗೆ ಮೊಣಕಾಲೂರಿ ಮಾಡುವ ಬದಲು ಉಷ್ಣದೇಶಗಳಲ್ಲಿ ಸಾಮಾನ್ಯವಾಗಿರುವಂತೆ ಕುಳಿತುಕೊಂಡು ಮಾಡುವರು. ಬೆಳಕು, ಧೂಪ, ಸಂಗೀತ ಎಲ್ಲವೂ ವೈದಿಕ ಕ್ರಿಯೆಯಲ್ಲಿಯೂ ಸಹ ಇವೆ”.

“ಆದರೆ” ಸ್ವಲ್ಪ ಸೌಜನ್ಯವಿಲ್ಲದಂತಹವಾದ ಇನ್ನೂ ಮುಂದುವರೆಯಿತು, “ಸಮುದಾಯ ಪ್ರಾರ್ಥನೆಯಂಥದು ಏನಾದರೂ ನಿಮ್ಮಲ್ಲಿದೆಯೆ?” ಹೀಗೆ ಹಿಂದುಮುಂದಿಲ್ಲದೆ ಮಾಡುವ ಆಕ್ಷೇಪಣೆ ಯಾವಾಗಲೂ ಏನಾದರೊಂದು ಹೊಸದಾದ ಅನಿರೀಕ್ಷಿತ ಅಸಂಬದ್ಧತೆಯನ್ನು ಒಳಗೊಂಡಿರುತ್ತದೆ.

ಅವರು ಪ್ರಶ್ನೆಯನ್ನು ಝಳಪಿಸಿ ಹಾಕಿಬಿಟ್ಟರು. “ಇಲ್ಲ! ಕ್ರೈಸ್ತಧರ್ಮದಲ್ಲಿಯೂ ಸಹ ಅದು ಇಲ್ಲ! ಅದು ಅಪ್ಪಟ ಪ್ರಾಟೆಸ್ಟೆಂಟ್ ಧರ್ಮ, ಪ್ರಾಟೆಸ್ಟೆಂಟರು ಅದನ್ನು ತೆಗೆದುಕೊಂಡದ್ದು ಮುಸ್ಲಿಮರಿಂದ, ಬಹುಶಃ ಮೂರ್ ಜನಾಂಗದ ಪ್ರಭಾವದಿಂದಾಗಿ!

“ಪುರೋಹಿತ ಕಲ್ಪನೆಯನ್ನು ಸಂಪೂರ್ಣವಾಗಿ ಕತ್ತರಿಸಿ ಹಾಕಿರುವ ಏಕಮಾತ್ರ ಧರ್ಮ ಎಂದರೆ ಇಸ್ಲಾಮ್​. ಪ್ರಾರ್ಥನಾ ಸಮಯದಲ್ಲಿ ಮುಂದಾಳುವಾದವನು ಜನರಿಗೆ ಬೆನ್ನು ಹಾಕಿ ನಿಂತುಕೊಳ್ಳುತ್ತಾನೆ; ವೇದಿಕೆಯಿಂದ ಕೇವಲ ಕುರಾನ್ ಪಠಣ ಮಾತ್ರವೇ ನಡೆಯಬಹುದು. ಪ್ರಾಟೆಸ್ಟೆಂಟ್ ವಿಧಾನ ಇದನ್ನೇ ಅನುಸರಿಸಿರುವಂಥದು.

“ಕೇಶಮುಂಡನ ಕ್ರಿಯೆಯೂ ಭಾರತದಲ್ಲಿತ್ತು. ಇಬ್ಬರು ನುಣ್ಣಗೆ ತಲೆ ಬೋಳಿಸಿಕೊಂಡಿದ್ದ ಕೈಸ್ತಸಂನ್ಯಾಸಿಗಳಿಂದ ನ್ಯಾಯದೀಕ್ಷೆ ತೆಗೆದುಕೊಳ್ಳುತ್ತಿರುವ ಜಸ್ಟೀನಿಯನ್ ಒಬ್ಬರ ಚಿತ್ರವೊಂದನ್ನು ನಾನು ನೋಡಿದ್ದೇನೆ. ಬುದ್ಧಪೂರ್ವ ಹಿಂದೂಧರ್ಮದಲ್ಲಿ ಸಂನ್ಯಾಸಿ ಸಂನ್ಯಾಸಿನಿ ಇಬ್ಬರೂ ಇದ್ದರು. ಯೂರೋಪ್ ಇರುವುದೇ ಥೀಬ್ಸ್ ಆಜ್ಞಾನು ವರ್ತಿಯಾಗಿ.

“ಹಾಗಿದ್ದರೆ, ನೀವು ಕ್ಯಾಥೊಲಿಕ್ ಕ್ರಿಯಾವಿಧಿಗಳನ್ನು ಆರ್ಯನ್ ಎಂದು ಒಪ್ಪಿಕೊಳ್ಳುತ್ತೀರಿ ಎಂದಾಯಿತು!”

“ಹೌದು, ಹೆಚ್ಚುಕಡಿಮೆ ಕ್ರೈಸ್ತಧರ್ಮವೆಲ್ಲವೂ ಆರ್ಯನ್ ಎಂದೇ ನನ್ನ ನಂಬಿಕೆ. ಕ್ರಿಸ್ತನೆಂಬಾತ ಎಂದೂ ಅಸ್ತಿತ್ವದಲ್ಲೇ ಇರಲಿಲ್ಲವೆಂದು ನನ್ನ ಎಣಿಕೆ. ನಾನು ನನ್ನ ಆ “ಕ್ರೀಟ್” ಕನಸನ್ನು\footnote{1. ೧೮೯೭ರ ಜನವರಿಯಲ್ಲಿ ಭಾರತಕ್ಕೆ ಹಿಂದಿರುಗುತ್ತಿರುವಾಗ, ನೇಪಲ್ಸ್ನಿಂದ ಸೇಡ್ ಬಂದರಿಗೆ ಪ್ರಯಾಣ ಸಾಗುತ್ತಿರುವ ಸ್ಮಯದಲ್ಲಿ, ಸ್ವಾಮಿಗಳಿಗೆ ಒಂದು ಕನಸಾಯಿತಂತೆ. ಅದರಲ್ಲಿ ವೃದ್ಧನಾದ ಒಬ್ಬ ಗಡ್ಡಧಾರಿ ಪ್ರತ್ಯಕ್ಷನಾಗಿ, “ಇದೇ ದ್ವೀಪ” ಎನ್ನುತ್ತ ದ್ವೀಪದಲ್ಲಿನ ಒಂದು ಸ್ಥಳವನ್ನು ತೋರಿಸಿ, ಇದನ್ನು ನೀನು ಅನಂತರ ಗುರುತಿಸಬಹುದು ಎಂದನಂತೆ. ಮುಂದುವರೆದು, ಕ್ರೈಸ್ತಧರ್ಮವು ಅಂಕುರಿಸಿದ್ದು ಈ ದ್ವೀಪದಲ್ಲೇ ಎಂದು ಹೇಳಿ, ಅದಕ್ಕೆ ಸಂಬಂಧಿಸಿದಂತೆ ಎರಡು ಯೂರೋಪಿಯನ್ ಪದಗಳನ್ನ್ನು ಉಸುರಿದನಂತೆ -ಅದರಲ್ಲಿ ಒಂದು ಪದ ಸಂಸ್ಕೃತದಿಂದ ವ್ಯುತ್ಪನ್ನವಾದ ಥೆರಪೀತೆ (ಖ್ಜ್ಛ್ಟಿಚ್ಟ್ಛ್ಠಿಠಿಚ್ಛಿ) ಎಂಬುದಾಗಿತ್ತಂತೆ. ಥೆರಪೀತೆ (ಎಂದರೆ ಮಕ್ಕಳು - ಸಂಸ್ಕೃತದ ಪುತ್ರ) ಎಂದು ಅರ್ಥ. ಕ್ರೈಸ್ತಧರ್ಮ ಅಂಕುರಿಸಿದ್ದೇ ಬೌದ್ಧ ಧರ್ಮಪ್ರಚಾರವೊಂದರಲ್ಲಿ ಎಂದು ಸ್ವಾಮಿಗಳು ಇದರಿಂದ ಅರಿತುಕೊಳ್ಳಬೇಕಾಗಿದ್ದು ಎಂದ ಆತನು, ನೆಲವನ್ನು ತೋರಿಸುತ್ತ, “ಪುರಾವೆಗಳೆಲ್ಲ ಇಲ್ಲಿವೆ - ಇಲ್ಲಿ ಅಗೆದು ನೋಡು, ನಿನಗೆ ದೊರಕುತ್ತವೆ!” ಎಂದೂ ಸೇರಿಸಿದನಂತೆ. ಎಚ್ಚರವಾಗುತ್ತಿದ್ದಂತೆ, ಇದು ಸಾಮಾನ್ಯ ಕನಸಲ್ಲ ಎಂದೆನಿಸಿದ ಸ್ವಾಮಿಗಳು, ತಕ್ಷಣವೇ ತತ್ತರಿಸುತ್ತ ಹಡಗಿನ ಕಟ್ಟೆಯ ಮೇಲಕ್ಕೆ ಬಂದರಂತೆ. ಇಲ್ಲಿ ಎದುರಾದ ಹಡಗಿನ ಅಧಿಕಾರಿಯೊಬ್ಬರನ್ನು “ಈಗ ಗಂಟೆ ಎಷ್ಟು?” ಎಂದು ವಿಚಾರಿಸಿದರಂತೆ. “ಮಧ್ಯ ರಾತ್ರಿ” ಎಂದು ಉತ್ತರ ಬಂದಿತಂತೆ. “ನಾವೆಲ್ಲಿದ್ದೇವೆ ಈಗ?” ಎಂದು ಕೇಳಿದ್ದಕ್ಕೆ, ಅಚ್ಚರಿಯೆಂಬಂತೆ, “ಕ್ರೀಟ್ನಿಂದ ಸುಮಾರು ಐವತ್ತು ಮೈಲಿ ದೂರದಲ್ಲಿ” ಎಂದು ಉತ್ತರ ಬಂದಿತಂತೆ!

ಈ ಕನಸು ತಮ್ಮ ಮೇಲೆ ಅದೆಂತಹ ಪ್ರಭಾವವನ್ನುಂಟುಮಾಡಿತು ಎಂಬುದನ್ನು ನೆನಪಿಸಿಕೊಂಡು ಗುರುಗಳು ತಮ್ಮನ್ನು ತಾವೇ ಹಾಸ್ಯಮಾಡಿಕೊಂಡು ನಗುತ್ತಿದ್ದರು. ಆದರೆ ಆ ಕನಸನ್ನು ಮರೆಯಲು ಅವರಿಗೆ ಸಾಧ್ಯವಾಗಲಿಲ್ಲ. ಕನಸಿನಲ್ಲಿ ಆ ಎರಡನೆಯ ಪದ ಮರೆತು ಹೋದುದರ ಬಗ್ಗೆ ಅವರಿಗೆ ಗಾಢವಾದ ವಿಷಾದವಿತ್ತು. ಈ ಕನಸು ಬೀಳುವುದಕ್ಕೆ ಮುಂಚೆ ಸ್ವಾಮಿಗಳಿಗೆ ಎಂದೂ ಕ್ರಿಸ್ತನ ವ್ಯಕ್ತಿತ್ವ ಚಾರಿತ್ರಿಕವೇ ಎಂಬ ಅನುಮಾನ ಬಂದಿರಲಿಲ್ಲವಂತೆ. ಆದರೆ, ಹಿಂದೂ ತತ್ತ್ವಶಾಸ್ತ್ರದ ಪ್ರಕಾರ ಕಲ್ಪನೆ ಯೊಂದರ ಪರಿಪೂರ್ಣತೆ ಮುಖ್ಯವೇ ಹೊರತು, ಅದಕ್ಕೆ ಚಾರಿತ್ರಿಕ ವಿಶ್ವಾಸಾರ್ಹತೆ ಇದೆಯೇ ಎಂಬ ಪ್ರಶ್ನೆಯಲ್ಲ ಎಂಬುದನ್ನು ನಾವು ನೆನಪಿಡಬೇಕು. ಸ್ವಾಮಿಗಳು ತಾವು ಬಾಲಕರಾಗಿದ್ದಾಗ ಒಮ್ಮೆ ಶ‍್ರೀರಾಮಕೃಷ್ಣರನ್ನು ಇದೇ ವಿಚಾರವಾಗಿ ಕೇಳಿದರಂತೆ.ಅದಕ್ಕೆ ಅವರ ಗುರುಗಳು, “ಅಂತಹ ಕಲ್ಪನೆಯನ್ನು ಯಾರು ಮಾಡಬಲ್ಲರೋ ಅವರು ಅದೇ ಆಗಿರುತ್ತಾರೆ ಎಂದು ನಿನಗನ್ನಿಸುವುದಿಲ್ಲವೆ?” ಎಂದರಂತೆ.} ಕಂಡಾಗಿನಿಂದಲೂ ನನಗೆ ಆ ಸಂದೇಹವಿದೆ. ಭಾರತೀಯ ಹಾಗೂ ಈಜಿಪ್ಟಿನ ಕಲ್ಪನೆಗಳು ಅಲೆಗ್ಸಾಂಡ್ರಿಯಾದಲ್ಲಿ ಪರಸ್ಪರ ಒಂದುಗೂಡಿ, ಯಹೂದಿ ಧರ್ಮದ ಹಾಗೂ ಗ್ರೀಕ್ ಸಂಸ್ಕೃತಿಯ ಲೇಪನವನ್ನು ಪಡೆದು, ಕ್ರೈಸ್ತಧರ್ಮವೆಂಬ ಹೆಸರಿನಲ್ಲಿ ಪ್ರಪಂಚದಲ್ಲಿ ಪ್ರಚಾರವಾದುವು. ಅಲೆಗ್ಸಾಂಡ್ರಿಯಾದಲ್ಲಿ ಪರಸ್ಪರ ಒಂದು ಗೂಡಿ, ಯಹೂದಿ ಧರ್ಮದ ಹೂಗೂ ಗ್ರೀಕ್ ಸಂಸ್ಕೃತಿಯ ಲೇಪನವನ್ನು ಪಡೆದು, ಕ್ರೈಸ್ತಧರ್ಮವೆಂಬ ಹೆಸರಿನಲ್ಲಿ ಪ್ರಪಂಚದಲ್ಲಿ ಪ್ರಚಾರವಾದುವು.

“ನಿಮಗೆ ತಿಳಿದಿರುವಂತೆ, ಕ್ರೈಸ್ತಧರ್ಮದಸುವಾರ್ತೆಗಳಿಗಿಂತ ಆದ್ಯಪ್ರವರ್ತಕರ ಅಧ್ಯಾಯಗಳು ಹಿಂದಿನವು; ಸಂತ ಜಾನನಸುವಾರ್ತೆಯಂತೂ ಸುಳ್ಳಿನ ಕಂತೆ. ನೈಜವೆಂದು ನಾವು ನಂಬಬಹುದಾದದ್ದು ಸಂತ ಪಾಲನೊಬ್ಬನನ್ನೇ - ಅವನೂ ಸಹ ಪ್ರತ್ಯಕ್ಷ ಸಾಕ್ಷಿಯೇನಲ್ಲ; ಅವನೇ ತೋರಿಸಿಕೊಂಡಿರುವಂತೆ ಅವನು ಇಬ್ಬಂದಿ ಮಾತುಗಾರ- ‘ಸರ್ವ ಪ್ರಕಾರದಿಂದಲೂ ನಿಮ್ಮನ್ನು ನೀವು ವಿಮೋಚನೆ ಮಾಡಿಕೊಳ್ಳಿರಿ’ -ಹೌದಲ್ಲವೆ?

“ಇಲ್ಲ! ಧಾರ್ಮಿಕ ಪ್ರಬೋಧಕರುಗಳಲ್ಲಿ ಬುದ್ಧ ಹಾಗೂ ಮಹಮ್ಮದ್ ಇಬ್ಬರೇ ಚಾರಿತ್ರಿಕವಾಗಿ ನಮ್ಮೆದುರು ಪ್ರತ್ಯೇಕವಾಗಿ ನಿಲ್ಲುತ್ತಾರೆ - ತಾವು ಬದುಕಿದ್ದಾಗಲೇ ಶತ್ರುಮಿತ್ರರಿಬ್ಬರನ್ನೂ ಪಡೆದ ತಮ್ಮ ವಿಶೇಷ ಅದೃಷ್ಟದಿಂದಾಗಿ. ಇನ್ನು ಕೃಷ್ಣ -ನನಗೆ ಅನುಮಾನ; ಒಬ್ಬ ಯೋಗಿ, ಒಬ್ಬ ಗೋಪಾಲಕ; ಒಬ್ಬ ಮಹಾರಾಜ ಎಲ್ಲವನ್ನೂ ಒಂದಾಗಿ ಬೆಸೆದು ಗೀತೆಯನ್ನು ಕೈಯಲ್ಲಿ ಹಿಡಿದ ಒಬ್ಬ ಸುಂದರ ವ್ಯಕ್ತಿಯನ್ನಾಗಿಸಿದ್ದಾರೆ.

“ರೇನನ್ ಬರೆದ ಜೀಸಸ್ನ ಜೀವನ ಬರಿಯ ಬುದ್ಬುದನ, ಬರಿಯ ನೊರೆ. ನಿಜವಾದ ಪುರಾವಸ್ತು ಶೋಧಕ ಸ್ಟ್ರಾಸ್ನ ಸಮೀಪಕ್ಕೂ ಅವನ ಕೃತಿ ಬರುವುದಿಲ್ಲ. ಕ್ರಿಸ್ತ ಜೀವನದಲ್ಲಿನ ಎರಡು ಸಂಗತಿಗಳು.”

ವೈಯಕ್ತಿಕವಾಗಿ ಮನಮುಟ್ಟುವಂತಿದ್ದು ವಿಶಿಷ್ಟವೆನಿಸಿವೆ -ಲೋಕಸಾಹಿತ್ಯದಲ್ಲಿನ ಅತ್ಯಂತ ಸುಂದರ ಕಥೆಯಾಗಿರುವ, ವ್ಯಭಿಚಾರ ಮಾಡಿದಳೆಂದು ಹೇಳಲಾದ ಹೆಂಗಸಿನದು ಹಾಗೂ ಬಾವಿಯ ಬಳಿಯ ಹೆಂಗಸಿನದು. ಅಚ್ಚರಿ ಎಂಬಂತೆ ಈ ಎರಡನೆಯದು ಭಾರತೀಯ ಜೀವನಕ್ಕೆ ಅದೆಷ್ಟು ಸತ್ಯವಾಗಿ ತೋರುತ್ತದೆ! ನೀರು ಸೇದಲೆಂದು ಬಾವಿಯ ಬಳಿಗೆ ಬಂದ ಸ್ತ್ರೀ ಒಬ್ಬಳು ಅಲ್ಲೇ ಪಕ್ಕದಲ್ಲಿ ಕುಳಿತಿದ್ದ ಪೀತವಸ್ತ್ರಧಾರಿ ಸಂನ್ಯಾಸಿಯನ್ನು ನೋಡುತ್ತಾಳೆ. ಅವನು ನೀರು ಕೊಡುವಂತೆ ಅವಳನ್ನು ಕೇಳುತ್ತಾನೆ. ಆ ನಂತರ ಅವಳ ಮನಸ್ಸನ್ನು ಸ್ವಲ್ಪ ಓದಿನೋಡುತ್ತಾನೆ, ಉಪದೇಶ ಕೊಡುತ್ತಾನೆ, ಇತ್ಯಾದಿ. ಭಾರತೀಯ ಕಥೆಯಾಗಿದ್ದರೆ ಮಾತ್ರ, ಅವಳು ಬನ್ನಿ, ಉಪದೇಶವನ್ನು ಕೇಳಿ, ಎಂದು ಗ್ರಾಮಸ್ಥರನ್ನು ಕರೆತರಲು ಹೋದಾಗ, ಆ ಅವಕಾಶವನ್ನು ಉಪಯೋಗಿಸಿಕೊಂಡು ಸಂನ್ಯಾಸಿ ಕಾಡಿನ ಕಡೆಗೆ ಪರಾರಿಯಾಗಿರುತ್ತಿದ್ದ!

“ಒಟ್ಟಿನಲ್ಲಿ, ಜೀಸಸ್ನ ಬೋಧನೆಗಳಿಗೆ ಕಾರಣ ಹಳೆಯ ಯಹೂದಿ ಧರ್ಮಗುರು ರ್ಯಾಬೈ ಹಿಲ್ಲೆಲ್ ಎಂದು ನನಗನ್ನಿಸುತ್ತದೆ. ಅಲ್ಲದೆ, ಸಂತ ಪಾಲನಿಂದ ಇದ್ದಕ್ಕಿದ್ದಂತೆ ಪ್ರೇರೇಪಿಸಲ್ಪಟ್ಟು ಕಾರ್ಯೋನ್ಮುಖರಾದ ನಜರೇನರೆಂಬ ಅಪ್ರಸಿದ್ಧ ಯಹೂದಿ ಪಂಗಡ ದವರು - ತುಂಬ ಪುರಾತನರಾದವರು - ಪುರಾಣಪ್ರಸಿದ್ಧ ಕ್ರೈಸ್ತನೆಂಬ ವ್ಯಕ್ತಿಯನ್ನು ಆರಾಧನೆಯ ಕೇಂದ್ರವನ್ನಾಗಿ ಪ್ರಚುರಗೊಳಿಸಿದರು.

“ಪುನರುತ್ಥಾನವಂತೂ, ಸರಳವಾಗಿ ಚಳಿಗಾಲದ ಶವದಹನವಷ್ಟೇ. ಹೇಗೇ ಇರಲಿ, ಕೇವಲ ಶ‍್ರೀಮಂತ ಗ್ರೀಕರು ಮತ್ತು ರೋಮನರು ಶವದಹನ ಮಾಡುತ್ತಿದ್ದರು; ನವೀನ ಸೂರ್ಯಕಲ್ಪನೆಯು ಎಲ್ಲೋ ಕೆಲವರಲ್ಲಿ ಮಾತ್ರ ಅದನ್ನು ನಿಲ್ಲಿಸುವಂತೆ ಮಾಡಿತ್ತು.

“ಆದರೆ ಬುದ್ಧ! ಬುದ್ಧ! ಖಂಡಿತವಾಗಿಯೂ ಅವನು ಈ ಭೂಮಿಯ ಮೇಲೆ ಬಾಳಿ ಬದುಕಿದ ಮಾನವರಲ್ಲೆಲ್ಲ ಅತ್ಯಂತ ಶ್ರೇಷ್ಠನಾಗಿದ್ದವನು. ಅವನು ತನಗಾಗಿ ಒಂದು ಉಸಿ ರನ್ನು ಸಹ ತೆಗೆದುಕೊಳ್ಳಲಿಲ್ಲ. ಎಲ್ಲಕ್ಕಿಂತ ಹೆಚ್ಚಾಗಿ, ಅವನು ತನ್ನ ಆರಾಧನೆಯ ನ್ನೆಂದಿಗೂ ಒಪ್ಪಲಿಲ್ಲ. ‘ಬುದ್ಧನೆಂದರೆ ಒಬ್ಬ ಮನುಷ್ಯನಲ್ಲ, ಒಂದು ಸ್ಥಿತಿ. ನಾನು ನಿರ್ವಾಣಪಥವನ್ನು ಕಂಡಿರುವೆ. ಬನ್ನಿ, ನೀವೆಲ್ಲರೂ ಅದನ್ನು ಪ್ರವೇಶಿಸಿ!’ ಎಂದನವನು.

“ಪಾಪಿಯೆನ್ನಿಸಿಕೊಂಡಿದ್ದ ಅಂಬಾಪಾಲಿಯ ಹಬ್ಬದೂಟಕ್ಕೆ ಹೋದನವನು. ತನ್ನನ್ನದು ಕೊಲ್ಲಬಹುದು ಎಂದು ಗೊತ್ತಿದ್ದರೂ ಪರಯನ ಮನೆಯಲ್ಲಿ ಉಂಡನವನು. ಸಾಯುವ ಸ್ಥಿತಿಯಲ್ಲಿ ಇರುವಾಗ ಆ ಉಣಿಸನ್ನು ತನಗೆ ನೀಡಿದ ಆ ಪರಯನಿಗೆ ತನಗೆ ದೇಹಮುಕ್ತಿಯನ್ನು ದೊರಕಿಸಿಕೊಟ್ಟದ್ದಕ್ಕಾಗಿ ಕೃತಜ್ಞತೆಯ ಸಂದೇಶವನ್ನು ಕಳುಹಿಸಿ ದನವನು. ಜ್ಞಾನೋದಯವಾಗುವುದಕ್ಕಿಂತ ಮುಂಚೆಯೇ ಆಡಿನ ಮರಿಯೊಂದರ ಮೇಲೆ ದಯಾಪೂರ್ಣನಾಗಿದ್ದನವನು, ಪ್ರೀತಿ ತುಂಬಿದವನಾಗಿದ್ದನವನು! ಶಿಶುಮೇಧ ಮಾಡಲು ಹೊರಟಿದ್ದ ರಾಜನಿಗೆ, ಆ ಶಿಶುವನ್ನು ಉಳಿಸುವುದಾದರೆ ಸ್ವಯಂ ರಾಜಕು ವರನೂ ಸಂನ್ಯಾಸಿಯೂ ಆಗಿದ್ದ ತನ್ನ ತಲೆಯ ದಂಡವನ್ನೇ ಕೊಡಲು ಹೇಗೆ ಸಿದ್ಧನಾದ, ಆತನ ಜೀವಕಾರುಣ್ಯ ರಾಜನ ಮನಸ್ಸನ್ನು ಹೇಗೆ ತಟ್ಟಿತು, ಹೇಗೆ ಅವನು ಶಿಶುವಿನ ಜೀವವನ್ನು ಉಳಿಸಿದ, ನಿಮಗೆ ನೆನಪಿದೆಯೆ? ಅಂತಹ ವೈಚಾರಿಕತೆ ಭಾವನಾತ್ಮಕತೆಗಳ ಸಮ್ಮಿಶ್ರಣವನ್ನು ಈ ಲೋಕ ಕಂಡುದಿಲ್ಲ! ಖಂಡಿತವಾಗಿ, ಖಂಡಿತವಾಗಿ, ಅವನಂಥವನು ಇನ್ನೊಬ್ಬನಿರಲಿಲ್ಲ!”

