
\chapter{ಉಪನ್ಯಾಸ ಟಿಪ್ಪಣಿ}

(ಈ ಶತಮಾನದ ಪ್ರಾರಂಭದಲ್ಲಿ ಮಿಸ್ ಅಲೆನ್ ವಾಲ್ಡೋಳು ಸ್ವಾಮಿ ವಿವೇಕಾನಂದರ ಸ್ವಹಸ್ತದಿಂದ ಬರೆಯಲ್ಪಟ್ಟ ಈ ಟಿಪ್ಪಣಿಯನ್ನು ತನ್ನ ಸ್ನೇಹಿತಳಾದ ಸೋದರಿ ದೇವ ಮಾತಾಳಿಗೆ ಕೊಟ್ಟಳು.)

\begin{myquote}
ಮಾನವನು ಈಗಿರುವ ಸ್ಥಿತಿಯಲ್ಲಿ ಇರುವವರೆಗೆ ಅವನಿಗೆ ಧರ್ಮದ ಆವಶ್ಯಕತೆ ಇದೆ.\\ರೂಪಗಳು ಕಾಲಾನುಸಾರವಾಗಿ ಬದಲಾಗುತ್ತವೆ.\\ಇಂದ್ರಿಯ ವಿಷಯಗಳಲ್ಲಿ ಅತೃಪ್ತಿ\\ಅತೀತದ ಹಂಬಲ\\ಭೌತಿಕ ವಿಜ್ಞಾನದ ಕ್ಷೇತ್ರವನ್ನು ಧರ್ಮವು ಪ್ರವೇಶಿಸುತ್ತಿತ್ತು. ಆದರೆ ಈಗ ಅದರ\\ವಾಪ್ತಿ ಕಿರಿದಾಗುತ್ತಿದೆ.\\ಆದರೂ ವಿಜ್ಞಾನವು ಪ್ರವೇಶಿಸಲಾಗದ ವಿಶಾಲವಾದ ಧಾರ್ಮಿಕ ಕ್ಷೇತ್ರವಿದೆ.\\ಮನುಷ್ಯನನ್ನು ಇಂದ್ರಿಯ ಮಿತಿಗೆ ನಿರ್ಬಂಧಿಸುವ ಪ್ರಯತ್ನ ವ್ಯರ್ಥವಾದುದು.\\ಏಕೆಂದರೆ, ಆಗಾಗ ಅತೀತ, ಅನಂತ ಸತ್ಯದ ಹೊಳಹನ್ನು ಪಡೆದವರಿರುವರು.\\ವಿವಿಧ ಬಗೆಯ ಜನರು\\ಕರ್ಮಯೋಗಿ, ರಾಜಯೋಗಿ, ಭಕ್ತಿಯೋಗಿ ಮತ್ತು ಜ್ಞಾನಿ\\ಸಮಾಜದ ಒಳಿತಿಗೆ ಎಲ್ಲ ಬಗೆಯವರೂ ಆವಶ್ಯಕ\\ಪ್ರತಿಯೊಂದು ಬಗೆಯವರ ಅಪಾಯ -\\ಬೆರಕೆಯು ಅಪಾಯವನ್ನು ಹ್ರಸ್ವಗೊಳಿಸುತ್ತದೆ\\ಪ್ರಾಚ್ಯದೇಶಗಳು ಧ್ಯಾನಪರ ವ್ಯಕ್ತಿಗಳಿಂದ ತುಂಬಿದೆ, ಪಶ್ಚಿಮವು ಕ್ರಿಯಾಶೀಲರಿಂದ\\ತುಂಬಿದೆ.\\ಪರಸ್ಪರ ವಿನಿಮಯ ಎರಡಕ್ಕೂ ಒಳ್ಳೆಯದು\\ಧರ್ಮದ ಆವಶ್ಯಕತೆ -\\ಧರ್ಮದ ಕಡೆಗೆ ಬರುವ ನಾಲ್ಕು ಬಗೆಯ ಜನರು.\\ಏಕತೆಯ ತಳಹದಿ - ಮಾನವನ ದಿವ್ಯತೆ. ಈ ಶಬ್ದವನ್ನು ಏಕೆ ಬಳಸುವುದು?\\ಪಾಶ್ಚಾತ್ಯ ಸಮಾಜವು ಕರ್ಮ ಮತ್ತು ಬೌದ್ಧಿಕತತ್ತ್ವವನ್ನು ಹೊಂದಿದೆ - ಆದರೆ\\ಕರ್ಮವು ಇತರರ ನಾಶಕ್ಕಾಗಿರಬಾರದು.\\ತತ್ತ್ವಶಾಸ್ತ್ರವು ಕೇವಲ ಒಣ ಬೌದ್ಧಿಕತೆಯಾಗಿರಬಾರದು.
\end{myquote}

