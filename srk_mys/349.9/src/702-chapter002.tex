
\chapter{: ಯೂರೋಪಿನ ಪತ್ರಿಕಾವರದಿಗಳು}

\begin{center}
\textbf{ಪ್ರೇಮದ ಮೇಲೆ ಸ್ವಾಮಿ ವಿವೇಕಾನಂದರು\supskpt{\footnote{\enginline{1. New Discoveries, Vol. 3, pp.237-40}}}}
\end{center}

\begin{center}
(ಮೇಡನ್ಹೆಡ್ ಅಡ್ವೈಸರ್, ೨೩ ಅಕ್ಟೋಬರ್ ೧೮೯೫)
\end{center}

ಗುರುವಾರ ಸ್ವಾಮಿ ವಿವೇಕಾನಂದರು “ಪ್ರೇಮದ ಪೌರ್ವಾತ್ಯ ಸಿದ್ಧಾಂತ”\footnote{2. ೧೮೯೫ರ ಅಕ್ಟೋಬರ್ ೧೭ರಂದು ನೀಡಿದ ಈ ಉಪನ್ಯಾಸದ ಪದಶಃ ವರದಿ ಲಭ್ಯವಿಲ್ಲ.} ಎಂಬ ವಿಷಯವನ್ನು ತೆಗೆದುಕೊಂಡು ಮೇಡನ್ಹೆಡ್ನ ಪುರಭವನದಲ್ಲಿ ಒಂದು ಉಪನ್ಯಾಸವನ್ನು ಕೊಟ್ಟರು. ನಗರದಲ್ಲಿ ಇನ್ನಿತರ ಆಕರ್ಷಣೆಗಳಿದ್ದುದರಿಂದ, ಜನರು ಹೆಚ್ಚಾಗಿ ಸೇರಿರಲಿಲ್ಲ. ಸಾರ್ವಜನಿಕರಲ್ಲಿ ಅನೇಕರು, ಭಾಷಣಕಾರರು ಥಿಯಸಾಫಿಕಲ್ ಸೊಸೈಟಿಗೆ ಸೇರಿದವರೆಂದು ಭಾವಿಸಿದ್ದರು; ಆದರೆ ನಮಗೆ ತಿಳಿದುಬಂದಂತೆ ಅವರು ಥಿಯಸಾಫಿಕಲ್ ಸೊಸೈಟಿಗಾಗಲಿ ಇನ್ನಾವ ಸೊಸೈಟಿಗಾಗಲಿ ಸೇರಿದವರಲ್ಲ; ಅಲ್ಲದೆ ತಮ್ಮದೇ ಒಂದು ಸೊಸೈಟಿಯನ್ನು ಮಾಡಲು ಇಚ್ಛಿಸುವವರೂ ಅಲ್ಲ. ಯಾರು ಕೇಳಲಿಚ್ಛಿಸುವರೋ ಅವರೆ ದುರು ತಮ್ಮ ಅಭಿಪ್ರಾಯಗಳನ್ನು ಪ್ರತಿಪಾದಿಸುವುದರಲ್ಲಿ ನಂಬಿಕೆಯುಳ್ಳವರು; ಅವರು ಅವುಗಳನ್ನು ಒಟ್ಟಾರೆಯಾಗಿ ಎತ್ತಿ ಹಿಡಿಯಲಿ, ಅಥವಾ ಅವರಿಗೆ ತೋರಿದ ಹಾಗೆ ಪರಿಷ್ಕ ರಿಸಿಕೊಳ್ಳಲಿ, ಅಥವಾ ಪೂರ್ಣವಾಗಿ ನಿರಾಕರಿಸಲಿ; ಅಭಿಪ್ರಾಯಗಳ ಸೆಣೆಸಾಟದ ನಡುವಿನಿಂದಲೇ ಕೊನೆಗೆ ಸತ್ಯವು ಹೊರಬರುವುದು ಎಂದವರ ನಂಬಿಕೆ.

ರಾತ್ರಿ ಎಂಟು ಗಂಟೆಗೆ ಅಧ್ಯಕ್ಷಪೀಠವನ್ನು ಮಿ. ಈ. ಗಾರ್ಡ್ನರ್, ಜೆ.ಪಿ., ಸಿ. ಸಿ., ಅವರು ಅಲಂಕರಿಸಿದರು; ತಮ್ಮ ದೇಶೀಯ ಉಡುಗೆಯಲ್ಲಿದ್ದ ಉಪನ್ಯಾಸಕರನ್ನು ಚುಟುಕಾಗಿ ಪರಿಚಯಿಸಿದರು. ಸ್ವಾಮಿಗಳು ಅನಂತರ ದೇವರ ಮೇಲಣ ಭಕ್ತಿಯ ಬಗ್ಗೆ ತಮ್ಮ ದೃಷ್ಟಿಕೋನವನ್ನು ಪ್ರಾಚ್ಯದಲ್ಲಿ ಸಾಮಾನ್ಯವಾಗಿ - ಪ್ರೇಮ (ಭಕ್ತಿ) - ಅಭಿವ್ಯಕ್ತ ಗೊಳಿಸುವ ಹಾಗೆ ಈ ರೀತಿ ತಿಳಿಸಿದರು:- ಧರ್ಮವನ್ನು ಎರಡು ರೂಪಗಳಲ್ಲಿರುವಂತೆ ವಿಂಗಡಿಸಬಹುದು; ಮೊದಲನೆಯದು ಬಹಳಷ್ಟು ಮಟ್ಟಿಗೆ ಅಂಧಶ್ರದ್ಧೆ ಯದು, ಎರಡನೆಯದು ಕೇವಲ ತತ್ತ್ವ ಜಿಜ್ಞಾಸೆಯದು; ಆದರೆ ಇವೆರಡರಲ್ಲಿ ಯಾವುದ ಕ್ಕಾದರೂ ಬಲ ಬರಬೇಕಾದರೆ ಅದು ಪ್ರೇಮದಿಂದೊಡಗೂಡಿರಬೇಕು. ಈ ಅಂಶವಿಲ್ಲದ ಕೇವಲ ಕರ್ಮವು ತೃಪ್ತಿಕರವಾಗಿರದು. ದೇಶವು ಒಳ್ಳೆಯ ರಸ್ತೆಗಳಿಂದ ಕೂಡಿರಬಹುದು; ದೇಶದ ಎಲ್ಲೆಡೆ ವೈದ್ಯಶಾಲೆಗಳಿರಬಹುದು.ಚೆನ್ನಾಗಿ ನಡೆಯು ತ್ತಿರುವ ಬೃಹತ್ ಸಾಮಾಜಿಕ ಸಂಸ್ಥೆಗಳಿರಬಹುದು, ಒಳ್ಳೆಯ ನೈರ್ಮಲ್ಯವಿರಬಹುದು; ಆದರೆ ಇವೆಲ್ಲ ಬಾಹ್ಯ ಭೌತಿಕ ನಿರ್ವಹಣೆಗಳು; ಅವುಗಳಷ್ಟೇ ಮಾನವನನ್ನು ದಿವ್ಯತೆಯ ಸಮೀಪಕ್ಕೆಕೊಂಡೊಯ್ಯಲಾರವು. ಆದರ್ಶವಾದಿ, ವಾಸ್ತವವಾದಿ ಇಬ್ಬರೂ ಆವಶ್ಯಕವಷ್ಟೇ ಅಲ್ಲ, ಪರಸ್ಪರ ಪೂರಕ ಕೂಡ. ಆದರ್ಶವಾದಿ ಧೈರ್ಯವಾಗಿ ಮಹತ್ವಾಕಾಂಕ್ಷೆಯನ್ನು ನೆಲಮಟ್ಟಕ್ಕೆ ತರುವನು, ವಾಸ್ತವವಾದಿ ಕಾರ್ಯನಿರ್ವಹಣೆಯ ಮೂಲಕ ಅದು ರೂಪು ತಳೆಯುವಂತೆ ಮಾಡುವನು. ಪ್ರೇಮವನ್ನು ವಿಧಾಯಕವಾಗಿ ಹೀಗೇ ಎಂದು ಹೇಳಲಾಗದು; ನಿಷೇಧಾತ್ಮಕವಾಗಿ ಹೀಗಲ್ಲ ಎಂದು ಹೇಳಬಹುದಷ್ಟೆ. ಅದರ ಸ್ವಭಾವ ತ್ಯಾಗರೂಪದ್ದು. ಸಾಮಾನ್ಯಾರ್ಥದಲ್ಲಿ ಅದನ್ನು ಮೂರು ಸ್ತರಗಳಲ್ಲಿ ಗುರುತಿಸಬಹುದು: (೧) ಇತರರಿಗೆ ನೋವನ್ನುಂಟುಮಾಡುವುದೋ ನಲಿವನ್ನುಂಟುಮಾಡುವುದೋ ಎಂಬ ಗಮನವೇ ಇಲ್ಲದ ಕೇವಲ ತನ್ನ ಸಂತೋಷಕ್ಕಾಗಿ ಇರುವ ಪ್ರೇಮ - ಅತ್ಯಂತ ಸ್ವಾರ್ಥಪರವಾದದ್ದು, ಕೀಳುಸ್ತರದ್ದು; (೨) ಕೊಟ್ಟು ತೆಗೆದುಕೊಳ್ಳುವ, ಭಾಗಶಃ ಸ್ವಾರ್ಥದ, ಬಹಳಷ್ಟು ಜನ ಅನುಸರಿಸುವ ಮಧ್ಯಮ ಪಂಥದ ಪ್ರೇಮ; (೩) ಎಲ್ಲವನ್ನೂ ಕೊಟ್ಟುಬಿಡುವ, ಪ್ರತಿಯಾಗಿ ಯಾವುದನ್ನೂ ಕೇಳದ, ಪೂರ್ವಯೋಜಿತವಲ್ಲದ, ಪಶ್ಚಾತ್ತಾಪವಿಲ್ಲದ ಪ್ರೇಮ; ಯಾರಿಂದ ಹೊರ ಹೊಮ್ಮುತ್ತಿರುವುದೋ ಅವರಿಗೆ ಯಾವ ಕೇಡನ್ನು ಬಗೆದರೂ ನಿಲ್ಲದ ಪ್ರೇಮ- ಇದು ಅತ್ಯಂತ ಉನ್ನತವಾದುದ್ದು, ದಿವ್ಯವಾದದ್ದು. ನಮಗೀಗ ಪ್ರಸ್ತುತವಾಗಿರುವುದು ಈ ಕೊನೆಯ ರೀತಿಯ ಪ್ರೇಮ. ಮೊದಲನೆಯದು ಮೃಗೀಯ, ಇಂದ್ರಿಯಲೋಲುಪನ ಮಾರ್ಗ; ಎರಡನೆಯದು ಒಳಿತಿಗಾಗಿ ಹೋರಾಡುತ್ತಿರುವ ಸಮಸ್ತ ಮಾನವ ಕೋಟಿಯದು; ಮೂರನೆಯದೇ ಪ್ರಪಂಚವನ್ನು ತ್ಯಾಗಮಾಡಿ ಚಿರಶಾಂತಿಯೆಡೆಗೆ ನಡೆಯುತ್ತಿರುವವರು ಅನುಸರಿಸುವ ನಿಜವಾದ ಪ್ರೇಮದ ಮಾರ್ಗ. ಈ ಪ್ರೇಮದಲ್ಲಿ ಭಯವೆಂಬುದಿಲ್ಲ. ಪ್ರೇಮವು ಭಯವನ್ನು ನಿವಾರಿಸುವುದು. ಒಂದು ಸಿಂಹವು ಮಗು ವೊಂದರ ಮೇಲೆ ನಿಂತು ಕೊಲ್ಲುವೆನೆಂದು ಹೆದರಿಸುತ್ತಿರಬಹುದು; ತಾಯಿಗೆ ಆಗ ಭಯವಾಗದು, ಅವಳು ಅಲ್ಲಿಂದ ಓಡುವ ಬದಲಿಗೆ ಪ್ರತಿರೋಧಿಸುವಳು. ಆ ಕ್ಷಣದಲ್ಲಿ ಪ್ರೇಮವು ಭಯವನ್ನು ನಿವಾರಿಸಿಬಿಡುವುದು; ಬೇರೆ ಸಮಯದಲ್ಲಿ ಅದೇ ಹೆಂಗಸು ಒಂದು ನಾಯಿಗೂ ಹೆದರಿ ಓಡಬಹುದು. ಕ್ರೂರ ಮಹಮ್ಮದೀಯ ಯೋಧನೊಬ್ಬ ಪ್ರಾರ್ಥಿಸುವುದಕ್ಕೆಂದು ಒಂದು ಉದ್ಯಾನವನಕ್ಕೆ ಹೋದನು. ಅದೇ ತೋಟದಲ್ಲಿ ಹುಡುಗಿಯೊಬ್ಬಳು ತನ್ನಿಯನ ಭೇಟಿಗಾಗಿ ಬಂದಿದ್ದಳು. ಯೋಧನು ತನ್ನ ಧರ್ಮದಲ್ಲಿ ನಿಯುಕ್ತವಾದ ರೀತಿಯಲ್ಲಿ ತಲೆಬಗ್ಗಿಸಿ ನಮಸ್ಕರಿಸುತ್ತಿದ್ದನು. ಅದೇ ಕ್ಷಣಕ್ಕೆ ತನ್ನ ಇನಿಯನನ್ನು ಕಂಡ ಹುಡುಗಿ ಸಂತೋಷಾತಿರೇಕದಿಂದ ಅವನೆಡೆಗೆ ಓಡುವ ಭರದಲ್ಲಿ ನಮಸ್ಕರಿಸುತ್ತಿದ್ದ ಆ ಯೋಧನನ್ನು ತುಳಿದುಬಿಟ್ಟಳು. ಸಿಟ್ಟಿನಿಂದ ಭುಗಿಲೆದ್ದ ಅವನು ತನ್ನ ಕತ್ತಿಯನ್ನು ಹಿರಿದು ಹುಡುಗಿಯನ್ನು ಇನ್ನೇನು ಸಿಗಿದುಬಿಡುತ್ತಿದ್ದನು. “ಎಲಾ ಕೀಳು ಬೆಲೆವೆಣ್ಣೆ, ನನ್ನ ಪೂಜೆಯನ್ನು, ನನ್ನ ದೈವಭಕ್ತಿಯನ್ನು, ನಿನ್ನ ಕಾಲ್ತುಳಿತದಿಂದ ಕೆಡಿಸುವಷ್ಟು ಸೊಕ್ಕೆ ನಿನಗೆ!” ಎಂದು ಅವನು ಗರ್ಜಿಸಿದನು. ಅದಕ್ಕೆ ಆ ಹುಡುಗಿ “ಪೂಜೆ! ಭಕ್ತಿ! ನಿನಗೆ ಅವುಗಳೇನೆಂಬುದೇ ತಿಳಿಯದು. ಬಾಗಿ ಮಲಗಿದ್ದ ನಿನಗೆ ಪೂಜೆಯ ಭಾವವೂ ಇರಲಿಲ್ಲ, ಭಕ್ತಿಯೂ ಇರಲಿಲ್ಲ. ಮುಗ್ಧ ಬಾಲಕಿಯಾದ ನಾನು, ನನ್ನ ಇಹಲೋಕದ ಪ್ರೇಮಿಯ ಮೇಲಿನ ಭಕ್ತಿಯಿಂದ, ಆರಾಧನಾ ಭಾವದಿಂದ ನಿನ್ನಂಥ ಭಯಂಕರ ವ್ಯಕ್ತಿ ಇರುವುದನ್ನೇ ಮರೆತು ನಿನ್ನ ಮೇಲೆ ಕಾಲಿರಿಸಬಹುದಾದರೆ, ನೀನು ನಿನ್ನ ಹೃದಯದಲ್ಲಿ ದೈವಭಕ್ತಿಯಿದ್ದು ಪೂಜಾಭಾವದಲ್ಲಿ ಒಂದು ವೇಳೆ ಮೈಮರೆತಿದ್ದಿದ್ದರೆ, ನನ್ನ ಸ್ಪರ್ಶ ಹೇಗೆ ತಾನೆ ನಿನಗೆ ಗೊತ್ತಾಗುತ್ತಿತ್ತು?” ಎಂದು ಛೇಡಿಸಿದಳು. ಯೋಧನಿಗೆ ಹೌದೆನಿಸಿ, ಅವನು ವಿನಯದಿಂದ ತಲೆಬಾಗಿ ಅಲ್ಲಿಂದ ಹೊರಟುಹೋದನು. ನಮ್ಮ ಪ್ರೇಮದ ಅತ್ಯುನ್ನತ ಆದರ್ಶವು ನಮ್ಮೊಳಗೆ ದೇವರ ಬಗ್ಗೆ ನಾವು ಮಾಡಿಕೊಂಡಿರುವ ಕಲ್ಪನೆಯಂತೆ ಇರುತ್ತದೆ. ರೂಕ್ಷರಾದವರ ದೇವರು ಕ್ರೂರಿಯೂ ಹಿಂಸಕನೂ ಆಗಿರುತ್ತಾನೆ. ವಿವೇಕಿಗಳೂ ಸುಸಂಸ್ಕೃ ತರೂ ಆದವರು ದೇವರನ್ನು ಹೆಚ್ಚು ಹೆಚ್ಚು ವಿಸ್ತರಿಸುತ್ತಿರುವ ಸಾಧ್ಯತೆಗಳೊಂದಿಗೆ ನೋಡುತ್ತಾರೆ. ದೇವರು ಯಾವಾಗಲೂ ದೇವರೇ; ಆದರೆ ಜನರು, ದೇಶಗಳು ಅವನ ಬಗ್ಗೆ ತಳೆಯುವ ದೃಷ್ಟಿ ಭಿನ್ನವಾಗಿರಬಹುದು. ಪ್ರೇಮಕ್ಕಿಂತಲೂ ಉನ್ನತವಾದ ದೃಷ್ಟಿ ಮನುಷ್ಯನಿಗೆ ತಿಳಿದಿಲ್ಲ. ಯಾರ ಹೃದಯದಲ್ಲಿ ಪ್ರತಿಯೊಂದು ಜೀವಿಯ ಬಗ್ಗೆ ಸಡಿಲ ಗೊಳ್ಳದಂತಹ ಪ್ರೇಮವಿದೆಯೋ, ಆ ಜೀವಿಗಳಲ್ಲಿ ನಿಜವಾಗಿಯೂ ದೇವರು ಇರುವುದನ್ನು ಅವನು ಅರಿತಿದ್ದರೂ, ಅಥವಾ ಅದು ಕೇವಲ ದೇವರ ಸೃಷ್ಟಿ ಎಂದು ಭಾವಿಸಿ ದ್ದರೂ, ಅವನು ಭಕ್ತಿಯ ಹಾಗೂ ತ್ಯಾಗದ ಮಹತ್ಪಥದಲ್ಲಿರುವುದು ಖಂಡಿತ. ಅವನ ಸಂಕುಚಿತ ದೃಷ್ಟಿಯಲ್ಲಿ ಅದು ಹೀಗಿರಬೇಕು ಹಾಗಿರಬೇಕು ಎಂಬ ವಿಕರ್ಷಣೆಯಿದ್ದರೂ, ಅವನು ದೇವರ ಸೃಷ್ಟಿಯಾದ ಆ ಜೀವವನ್ನು ಹಿಂಸಿಸಲಾರ. ಅವನು ಕೊಡುವುದನ್ನು ಪ್ರೇಮದಿಂದ ಕೊಡುತ್ತಾನೆಯೇ ಹೊರತು ಹೆಮ್ಮೆಯಿಂದಲ್ಲ; ದೇವರನ್ನು ಪ್ರೀತಿಸುವ ಮೂಲಕ ಅವನು ದೇವರ ಆವಿರ್ಭಾವಗಳನ್ನೂ ಪ್ರೀತಿಸುತ್ತಾನೆ, ಅವುಗ ಳೊಂದಿಗೆ ಇರುತ್ತಾನೆ, ಅವುಗಳೊಂದಿಗೆ ಕೆಲಸ ಮಾಡುತ್ತಾನೆ.

ಉಪನ್ಯಾಸವು ತುಂಬ ಪ್ರಭಾವಪೂರ್ಣವಾಗಿದ್ದಿತು; ಮುಗಿದ ಮೇಲೆ (ಕೇವರ್ ಷಾಮ್​ನ ಮಿ. ಇ. ಟಿ. ಸ್ಟರ್ಡಿಯವರ ಸೂಚನೆಯಂತೆ) ಅಧ್ಯಕ್ಷರು ವಂದನಾರ್ಪಣೆ ಮಾಡಿದರು. ಕಾರ್ಯಕ್ರಮಕ್ಕೆ ಅರ್ಧಗಂಟೆಗಿಂತ ಸ್ವಲ್ಪ ಹೆಚ್ಚು ಕಾಲ ಮಾತ್ರ ತಗುಲಿತು.

\begin{center}
\textbf{ಒಬ್ಬ ಭಾರತೀಯ ತಪಸ್ವಿ\supskpt{\footnote{\enginline{1. New Discoveries, Vol. 3, pp.267-269}}}}
\end{center}

\begin{center}
(ಸ್ಟ್ಯಾಂಡರ್ಡ್, ೨೩ ಅಕ್ಟೋಬರ್ ೧೮೯೫)
\end{center}

ರಾಮಮೋಹನ ರಾಯ್​ \enginline{(Ramahoun Roy)} ಕಾಲದಿಂದೀಚೆಗೆ, ಕೇಶವಚಂದ್ರ \enginline{(Keshub Chunder)} ಸೇನ್ ಒಬ್ಬರ ಹೊರತಾಗಿ, ಕಳೆದ ರಾತ್ರಿ ಪ್ರಿನ್ಸೆಸ್ ಹಾಲ್ನಲ್ಲಿ ಉಪನ್ಯಾಸ ಮಾಡಿದ ಬ್ರಾಹ್ಮಣನಿಗಿಂತ ಹೆಚ್ಚು ಆಸಕ್ತಿ ಮೂಡಿಸಿದ ಇನ್ನೊಬ್ಬ ಭಾರತೀಯ ಆಂಗ್ಲ ವೇದಿಕೆಯ ಮೇಲೆ ಕಾಣಿಸಿರಲಿಲ್ಲ ಎಂದು ಸ್ಟ್ಯಾಂಡರ್ಡ್ ಉದ್ಘೋಷಿಸುತ್ತದೆ....

ಉಪನ್ಯಾಸವು\footnote{2. ‘ಆತ್ಮಜ್ಞಾನ’ ದ ಕುರಿತಾದ ಈ ಉಪನ್ಯಾಸದ ಪದಶಃ ವರದಿ ಲಭ್ಯವಿಲ್ಲ.} ವೇದಾಂತ ಪಂಥಕ್ಕೆ ಸೇರಿದ ಸರ್ವಬ್ರಹ್ಮವಾದದ ಅತ್ಯಂತ ನಿರ್ಭೀತವಾದ ಹಾಗೂ ಅಸ್ಖಲಿತವಾಗ್ಮಿತೆಯಿಂದ ಕೂಡಿದ ಪ್ರತಿಪಾದನೆಯಾಗಿದ್ದಿತು. ಸ್ವಾಮಿಗಳು ಯೋಗಪಂಥದ ನೈತಿಕ ಅಂಶವನ್ನೂ ಸಹ ತಮ್ಮ ಚಿಂತನೆಯಲ್ಲಿ ಒಳಗೊಂಡಂತೆ ಕಂಡಿತು; ಏಕೆಂದರೆ ಉಪನ್ಯಾಸದ ಕೊನೆಯಲ್ಲಿ ಆ ಪಂಥದ ಪ್ರಮುಖ ವೈಶಿಷ್ಟ್ಯವಾದ ದೈಹಿಕತಪಸ್ಸನ್ನು ಎತ್ತಿಹಿಡಿಯುವ ಬದಲು ಐಹಿಕ ಸುಖಸೌಲಭ್ಯಗಳೆನ್ನಿಸಿಕೊಳ್ಳುವ ಎಲ್ಲವನ್ನೂ ತ್ಯಾಗ ಮಾಡುವುದೊಂದೇ ಪರಮ ನಿರಪೇಕ್ಷ ಆತ್ಮದಲ್ಲಿ ಒಂದಾಗುವ ವಿಧಾನವೆಂಬ ಪರಿಷ್ಕೃತ ನಿಲುವನ್ನು ಮಂಡಿಸಿದರು. ಉಪನ್ಯಾಸದ ಮೊದಲ ಭಾಗವು ಈ ಶತಮಾನದ ಪ್ರಾರಂಭದಲ್ಲಿ ಭೌತಿಕವಾದ ತನ್ನ ಸ್ಥೂಲರೂಪದಲ್ಲಿ ಹೇಗೆ ಅಭಿವೃದ್ಧಿಯಾಯಿತು, ಅನಂತರ ಅಮೂರ್ತ ಚಿಂತನೆಯ ಅನೇಕ ಛಾಯೆಗಳು ಹೇಗೆ ವೃದ್ಧಿಯಾದವು, ಭೌತಿಕವಾದವನ್ನು ತತ್ಕಾಲಕ್ಕೆ ಹೇಗೆ ಹಿಮ್ಮೆಟ್ಟಿಸಿದವು ಎಂಬುದರ ಸಮೀಕ್ಷೆಯಾಗಿತ್ತು. ಇದರ ನಂತರ ಅವರು ಜ್ಞಾನದ ಉಗಮ ಮತ್ತು ಅದರ ಸ್ವರೂಪದ ಚರ್ಚೆಯನ್ನು ಕೈಗೆತ್ತಿಕೊಂಡರು. ಈ ಒಂದು ಅಂಶದ ಮೇಲೆ ಅವರ ದೃಷ್ಟಿಯು ಹೆಚ್ಚುಕಡಿಮೆ ಶುದ್ಧ ಆದರ್ಶವಾದ \enginline{(Fichteism-German Idealism)}ದಂತೆಯೇ ಇತ್ತು; ಆದರೆ ಭಾಷೆಯಲ್ಲಿ ಅಭಿವ್ಯಕ್ತಿಸಿದ ರೀತಿ, ಉದಾಹರಣೆಗಳ ಆಯ್ಕೆ, ಅಂಗೀಕರಿಸುವುದರಲ್ಲಿ ತೋರಿದ ಔದಾರ್ಯ ಮುಂತಾದವುಗಳನ್ನು ಯಾವ ಜರ್ಮನ್ ಅತೀತವಾದಿಯೂ ಮೆರೆದಿರಲಾರ. ಹೊರಗಡೆ ಸ್ಥೂಲವಾದ ದ್ರವ್ಯಜಗತ್ತು ಇದೆ ಎಂದು ಅವರು ಒಪ್ಪಿಕೊಂಡರಾದರೂ, ದ್ರವ್ಯ ಎಂದರೆ ಏನು ಎಂಬುದು ತಮಗೆ ಗೊತ್ತಿಲ್ಲ ಎಂದರು. ಮನಸ್ಸು ಎನ್ನುವುದು ಸೂಕ್ಷ್ಮತರವಾದ ದ್ರವ್ಯ ಎನ್ನುವುದನ್ನು ಒತ್ತಿ ಹೇಳಿದರು; ಅದರ ಹಿಂದಿರುವುದು ಮನುಷ್ಯನ ಅಚಲಪ್ರತಿಷ್ಠವಾದ, ಅನಾದಿಯೂ ಅನಂತವೂ ಆದ ಆತ್ಮ, ಅದರ ಮುಂದೆ ಬಾಹ್ಯ ಜಗತ್ತಿನ ವಸ್ತುಗಳ ಮೆರವಣಿಗೆಯೇ ನಡೆದಿದೆ; ಚಿರಂತನವಾದ ಆ ಆತ್ಮವೇ ಕೊಟ್ಟಕೊನೆಗೆ ದೇವರು ಎಂದರು. ಮನುಷ್ಯನ ವ್ಯಕ್ತಿಗತ ಆತ್ಮವೂ ದೇವರೂ ಒಂದೇ ಎಂಬ ಈ ಸುಂದರ ಏಕದೇವತಾವಾದದ- ಮಾನ ವನೂ ದೇವರೂ ಒಂದೇ ಎಂಬ - ಕಲ್ಪನೆಯನ್ನು ಪಾಂಕ್ತವಾಗಿ, ಎಳೆಎಳೆಯಾಗಿ, ಅರ್ಥವತ್ತಾಗಿ, ಶ‍್ರೀಮಂತ ದೃಷ್ಟಾಂತಗಳೊಂದಿಗೆ ಒಂದಾದ ಮೇಲೊಂದು ಅನುಪಮ ಸುಂದರ ಶ‍್ರೀಮದ್ಗಾಂಭೀರ್ಯದ ಹಾಗೂ ಅವಿಚ್ಛಿನ್ನ ಶ್ರದ್ಧೆಯ ಮಾತುಗಾರಿಕೆಯ ಮೋಡಿಗಳ ಮೂಲಕ ತೆರೆದಿಟ್ಟರು. ಅವರೆಂದರು:

“ವಿಶ್ವದಲ್ಲಿರುವುದು ಕೇವಲ ಒಂದೇ ಆತ್ಮ; ನೀವು ಅಥವಾ ನಾನು ಎಂಬುದು ಇಲ್ಲವೇ ಇಲ್ಲ; ವೈವಿಧ್ಯತೆಯೆಲ್ಲವೂ ಒಂದು ನಿರಪೇಕ್ಷ ಏಕದಲ್ಲಿ, ದೇವರೆಂಬ ಒಂದು ಅನಂತ ಅಸ್ತಿತ್ವದಲ್ಲಿ ಅಡಗಿದೆ”.

ಇದರಿಂದ ಆತ್ಮದ ಅಮೃತತ್ವ ಹಾಗೂ ಪೂರ್ಣತೆಯ ಉನ್ನತ ಅಭಿವ್ಯಕ್ತಿಯತ್ತ ಆತ್ಮಗಳ ಜನ್ಮಾಂತರ ಪ್ರಯಾಣಂತಹ ಅಭಿಪ್ರಾಯಗಳು ಮೂಡಿದವು. ಆಗಲೇ ಹೇಳಿದಂತೆ, ಅವರ ಇಪ್ಪತ್ತು ನಿಮಿಷಗಳ ಭಾಷಣ ಸಮಾರೋಪ ತ್ಯಾಗದ ನಿರೂಪಣೆಗೆ ಮೀಸಲಾಗಿತ್ತು. ಇದರಲ್ಲಿ ಅವರು ಬುದ್ಧ ಅಥವಾ ಜೀಸಸ್ನ ಬೋಧನೆಯ ಹಲವು ಪದಗಳು ಬೀರುವ ಪ್ರಭಾವಕ್ಕಿಂತಲೂ ಕಾರ್ಖಾನೆಗಳು, ಯಂತ್ರಗಳೇ ಮುಂತಾದ ಆವಿಷ್ಕರಣಗಳ ಬಗೆಗೆ ಹಾಗೂ ಪುಸ್ತಕಗಳು ಮನುಷ್ಯನಿಗಾಗಿ ಮಾಡುತ್ತಿರುವ ಕೆಲಸ ಏನೇನೂ ಇಲ್ಲ ಎಂದು ಜರೆದು ಮಾತನಾಡಿದರು. ನಿಶ್ಚಿತವಾಗಿಯೂ, ಉಪನ್ಯಾಸವು ಯಾವುದೇ ರೀತಿಯ ಹಿಂಜರಿಕೆ ಯಿಲ್ಲದೆ, ಮುದಗೊಳಿಸುವ ಧ್ವನಿಯಲ್ಲಿ ಕೊಟ್ಟ ಒಂದು ಉತ್ತಮ ಆಶುಭಾಷಣವಾಗಿತ್ತು.

\begin{center}
\textbf{ಪ್ರಿನ್ಸೆಸ್ ಹಾಲ್ನಲ್ಲಿ ದೇಶೀ ಭಾರತೀಯ ಉಪನ್ಯಾಸಕರು\supskpt{\footnote{\enginline{1. New Discoveries, Vol. 3, pp.248}}}}
\end{center}

\begin{center}
(ಲಂಡನ್ ಮಾರ್ನಿಂಗ್ ಪೋಸ್ಟ್, ೨೩ ಅಕ್ಟೋಬರ್ ೧೮೯೫)
\end{center}

- ಕಳೆದ ರಾತ್ರಿ ಪಿಕಡಿಲ್ಲಿಯ ಪ್ರಿನ್ಸೆಸ್ ಹಾಲ್ನಲ್ಲಿ ಪ್ರಸಕ್ತ ಈ ದೇಶದ ಪ್ರವಾಸದ ಲ್ಲಿರುವ ಭಾರತೀಯ ಯೋಗಿ ಸ್ವಾಮಿ ವಿವೇಕಾನಂದರು “ಆತ್ಮಜ್ಞಾನ”\footnote{2. ಇದರ ಪದಶಃ ವರದಿ ಲಭ್ಯವಿಲ್ಲ. ಇದೇ ವಿಷಯದ ಮೇಲೆ ೧೮೯೫ರ ಅಕ್ಟೋಬರ್ ೨೨ರಂದು ನೀಡಿದ ಉಪನ್ಯಾಸದ ವರದಿಗಾಗಿ, ಹಿಂದಿನ ಭಾರತೀಯ ತಪಸ್ವಿಯ ಪತ್ರಿಕಾ ವರದಿ ನೋಡಿ.} ಎಂಬ ವಿಷಯದ ಮೇಲೆ ‘ಪ್ರವಚನ’ ಎನ್ನಲಾದುದನ್ನು ನೀಡಿದರು. ವಿಧಿವತ್ತಾಗಿ ಪ್ರಪಂಚತ್ಯಾಗ ಮಾಡಿ ಅಧ್ಯಯನ ಹಾಗೂ ಭಕ್ತಿಯಲ್ಲಿ ತನ್ನನ್ನು ತಾನು ತೊಡಗಿಸಿಕೊಂಡವನನ್ನು ಯೋಗಿ ಎಂದು ವಿವರಿಸಲಾಯಿತು. ಸ್ವಾಮಿ ವಿವೇಕಾನಂದರು ಮೊದಲು ತಮ್ಮ ದೇಶವನ್ನು ಬಿಟ್ಟು ಹೊರಟದ್ದು ಎರಡು ವರ್ಷಗಳ ಹಿಂದೆ ನಡೆದ ಚಿಕಾಗೋ ಸರ್ವಧರ್ಮ ಸಮ್ಮೇಳನದಲ್ಲಿ ವೇದಾಂತ ತತ್ತ್ವದ ಮೇಲಣ ತಮ್ಮ ಪ್ರತಿಪಾದನೆಯನ್ನು ನೀಡುವ ಉದ್ದೇಶದಿಂದ. ಆಗಿನಿಂದಲೂ ಅವರು ಅಮೆರಿಕಾದಲ್ಲಿ ಅದೇ ವಿಷಯದ ಮೇಲೆ ಉಪನ್ಯಾಸಗಳನ್ನು ಕೊಡುತ್ತಲೇ ಬಂದಿರುತ್ತಾರೆ. ಕಳೆದ ರಾತ್ರಿಯ ತಮ್ಮ ಉಪನ್ಯಾಸದಲ್ಲಿ ಅವರು ಹತ್ತೊಂಭತ್ತನೆಯ ಶತಮಾನದ ಕೊನೆಯ ದಿನಗಳಲ್ಲಿ ವೈಜ್ಞಾನಿಕ ಚಿಂತನೆಯ ಲೋಲಕ ಹಿಂದಕ್ಕೆ ಬರುವಂತೆ ತೋರುತ್ತಿದೆ, ಏಕೆಂದರೆ ಪ್ರಪಂಚದಾದ್ಯಂತ ಜನರು ಪ್ರಾಚೀನ ಗ್ರಂಥಗಳನ್ನು ತಿರುವಿ ಹಾಕುತ್ತಿದ್ದು ಪುರಾತನ ಧಾರ್ಮಿಕ ರೂಪಗಳು ಪುನಃ ಪ್ರಾಮುಖ್ಯತೆಗೆ ಬರುತ್ತಿವೆ ಎಂದರು. ಅನೇಕರಿಗೆ ಇದು ಅವನತಿಯತ್ತ ಸಾಗುವಿಕೆ ಎನ್ನಿಸಬಹುದು; ಇನ್ನು ಕೆಲವರಿಗೆ ಸಮಾಜವನ್ನು ಆಗಾಗ ಬಂದು ಕಾಡುವ ಮೂಢನಂಬಿಕೆಯ ಉನ್ಮೇಷಗಳಲ್ಲೊಂದಾಗಿ ತೋರಬಹುದು; ಆದರೆ ವಿಜ್ಞಾನದ ವಿದ್ಯಾರ್ಥಿಗೆ ಈ ಪ್ರಚಲಿತ ಸ್ಥಿತಿಯು ಭವಿಷ್ಯದಲ್ಲಿ ಭವ್ಯವಾದ ಲಾಭವನ್ನು ತರಲಿರುವ ಮುನ್ಸೂಚನೆಯಾಗಿ ಕಾಣುತ್ತದೆ. ಅನಂತರ ಭಾಷಣಕಾರರು ತಾವು ಬೋಧಿಸುವ ವಿಚಿತ್ರವಾದ ತಾತ್ತ್ವಿಕ ಪ್ರಕಾರವನ್ನು ದೀರ್ಘವಾಗಿ ವಿವರಿಸತೊಡಗಿದರು; ಅದರಿಂದ ಹೊರಹೊಮ್ಮಿದ ಮೂರು ವಿಭಿನ್ನ ಧಾರ್ಮಿಕ ಹಂತಗಳನ್ನು ಚಿತ್ರಿಸಿದರು. ಅವರು ತುಂಬ ನಿರರ್ಗಳವಾಗಿ ಮಾತನಾಡಿದರು; ಕೆಲಮಟ್ಟಿಗೆ ಸಣ್ಣದೆನ್ನಬಹುದಾದ ಸಭೆಯು ಗಮನವಿಟ್ಟು ಅವರ ವ್ಯಾಖ್ಯಾನವನ್ನು ಆಲಿಸಿತು.

\begin{center}
\textbf{ಸ್ವತಂತ್ರ ಕ್ರೈಸ್ತ ಸಮುದಾಯ\supskpt{\footnote{\enginline{1. New Discoveries, Vol. 3, pp.267-69}}}}
\end{center}

\begin{center}
(ಕ್ರಿಶ್ಚಿಯನ್ ಕಾಮನ್ವೆಲ್ತ್, ೧೪ ನವೆಂಬರ್ ೧೮೯೫)
\end{center}

\begin{center}
\textbf{ಸೌತ್ ಪ್ಲೇಸ್ ಚಾಪೆಲ್ ಉಪನ್ಯಾಸ}
\end{center}

ಸ್ವಾಮಿ ವಿವೇಕಾನಂದರು ಸೌತ್ ಪ್ಲೇಸ್ ಚಾಪೆಲ್ನಲ್ಲಿ ನೆರೆದಿದ್ದ ಭಕ್ತಜನರನ್ನು ಕಳೆದ ಭಾನುವಾರ ಬೆಳಗ್ಗೆ “ವೇದಾಂತ ನೈತಿಕತೆಯ ತಳಹದಿ”\footnote{2.ಇಂಗ್ಲೆಂಡಿನ ಲಂಡನ್ನಿನಲ್ಲಿ ೧೮೯೫ ೧೦ರಂದು ಕೊಟ್ಟ ಉಪನ್ಯಾಸದ ಪದಶಃ ವರದಿಲಭ್ಯವಿಲ್ಲ.} ಎಂಬ ವಿಷಯವಾಗಿ ಮಾತಾನಾಡಿ ರಂಜಿಸಿದರು....

ತಾವು ಪ್ರತಿಪಾದಿಸುತ್ತಿರುವ ನೈತಿಕತೆಯ ಪದ್ಧತಿಯಲ್ಲಿ ಕರ್ಮಗಳನ್ನು ಈಗ ಅಥವಾ ಮುಂದೆ ಬರಬಹುದಾದ, ಯಾವುದೋ ಪ್ರತಿಫಲಾಪೇಕ್ಷೆಯಿಂದ ಅಥವಾ ಇಹಲೋಕದಲ್ಲಿಯೋ ಪರಲೋಕದಲ್ಲಿಯೋ ಆಗಬಹುದಾದ ಶಿಕ್ಷೆಗೆ ಭಯಪಟ್ಟು ಮಾಡಲಾಗುವುದಿಲ್ಲ; “ನಾವು ಕೇವಲ ಅಂತಃಪ್ರೇರಣೆಯನ್ನನುಸರಿಸಿ ಕರ್ಮಕ್ಕಾಗಿ ಕರ್ಮ, ಕರ್ತವ್ಯಕ್ಕಾಗಿ ಕರ್ತವ್ಯ ಎನ್ನುವ ಹಾಗೆ ಕೆಲಸ ಮಾಡಬೇಕು”. ಈ ನೈತಿಕತೆಯ ಕಲ್ಪನೆಯು ಜೀಸಸ್ನ ಧರ್ಮಕ್ಕಿಂತ ಉನ್ನತವಾದದ್ದೆಂದು ಅನ್ನಿಸಿದ್ದರಿಂದಲೇ ಕ್ರೈಸ್ತರಾಗಿದ್ದ ಕೆಲವರು ಬೌದ್ಧ ಧರ್ಮಕ್ಕೋ ಇನ್ನಿತರ ಪೌರ್ವಾತ್ಯ ತತ್ತ್ವಗಳಿಗೋ ಆಕರ್ಷಿತರಾದರು. ಆದರೆ ನಿಜವಾದ ಕ್ರೈಸ್ತಧರ್ಮದ ಸಾರವೆಂದರೆ, ನೀವು ಮಾಡುವ ಕೆಲಸಗಳು ನಿಮ್ಮೊಳಗಿನ ಸ್ವರ್ಗೀಯ ಅಂತರಂಗದಿಂದ ಪ್ರೇರಿತವಾಗಿದ್ದರೆ, ದಿವ್ಯ ಸ್ವರ್ಗದ ಪ್ರತಿಫಲ ನಿಮಗೆ ದೊರೆ ಯುವುದು; ಬದಲಾಗಿ, ಹೊರಗಿನ ಸೈತಾನರಾಜ್ಯದೊಂದಿಗೆ ಸಾಮರಸ್ಯದಿಂದ ಕೆಲಸ ಮಾಡಿದರೆ, ನೀವು ಅಧೋಗತಿಗಿಳಿಯುವಿರಿ. ಸ್ವಾಮಿಗಳು ಹೇಳುವಂತೆ ನಿಜವಾದ ಕ್ರೈಸ್ತನಾದವನು ಶಿಕ್ಷೆಯಿಂದ ತಪ್ಪಿಸಿಕೊಳ್ಳುವ ಉದ್ದೇಶದಿಂದ ಕೆಲಸ ಮಾಡುವುದಿಲ್ಲ; ಅವನು ತನ್ನ ಕರ್ಮಗಳಿಂದ ಕೊನೆಗೆ ಏನಾಗುವುದೆಂದು ನೋಡಿಕೊಳ್ಳುತ್ತಾನೆ....

\begin{center}
\textbf{ಒಂದು ಸಾರ್ವತ್ರಿಕ ಧರ್ಮ\supskpt{\footnote{\enginline{1. New Discoveries, Vol. 3, pp.276-77.}}}}
\end{center}

\begin{center}
(ದಿ ಕ್ವೀನ್, ದಿ ಲೇಡೀಸ್ ನ್ಯೂಸ್ಪೇಪರ್, ೨೩ ನವೆಂಬರ್ ೧೮೯೫)
\end{center}

ಮಿಸೆಸ್ ಹ್ಯಾವೀಸ್ ಅವರು ಕಳೆದ ಶನಿವಾರ ಕ್ವೀನ್ಸ್ ಹೌಸ್ನಲ್ಲಿ ಏರ್ಪಡಿಸಿದ್ದ ಶರತ್ಕಾಲದ ಮೊದಲ ಅಟ್ ಹೋಮ್​ ಗೋಷ್ಠಿಯಲ್ಲಿ ಭಾರತೀಯ ಯೋಗಿ, ಅಥವಾ ತಪಸ್ವಿ, ಸ್ವಾಮಿ ವಿವೇಕಾನಂದರು (೧೮೯೩ರ ಚಿಕಾಗೋ ಸರ್ವಧರ್ಮ ಸಮ್ಮೇಳನದಲ್ಲಿ ಬೌದ್ಧ ಪ್ರತಿನಿಧಿಯಾಗಿದ್ದವರು) ಉದಾರ ಭಾವದಲ್ಲಿ, ಹಾಸ್ಯವನ್ನೂ ಇಲ್ಲವೆನಿಸದೆ, ಸಾರ್ವತ್ರಿಕ ಧರ್ಮವೊಂದರ ಸೊಬಗು ಸಂಭಾವ್ಯತೆಗಳನ್ನು ಕುರಿತು ಚರ್ಚಿಸಿದರು.\footnote{2. ೧೮೯೫ರ ನವೆಂಬರ್ ೧೯ರಂದು ನೀಡಿದ ಈ ಉಪನ್ಯಾಸದ ಪದಶಃ ವರದಿ ಲಭ್ಯವಿಲ್ಲ.} ಲೋಕದ ಎಲ್ಲ ಮಹಾಧರ್ಮಗಳ ತಳಹದಿಯಾಗಿರುವ ಮೂಲಭೂತ ತತ್ತ್ವಗಳು ಒಂದನ್ನೊಂದು ಹೋಲುವಂತಿರುವುದನ್ನು ಮನವರಿಕೆ ಮಾಡಿಕೊಟ್ಟರು; ಮಹಾಪ್ರವಾದಿಗಳಲ್ಲಿ ಕ್ರೈಸ್ತ ವಿಮೋಚಕನನ್ನು ಉನ್ನತ ಸ್ಥಾನದಲ್ಲಿರಿಸಿದರಾದರೂ, ಕೆಲವೊಮ್ಮೆ ಅವನ ಬೋಧನೆಗಳನ್ನು ಅವನ ಅನುಯಾಯಿಗಳೆನಿಸಿಕೊಂಡವರೇ ಅನುಸರಿಸುವುದು ಕಡಿಮೆ ಎನ್ನುವುದನ್ನು ಸೂಚ್ಯವಾಗಿ ಹೇಳಿದರು. ಮಂಟಪಕ್ಕೆ, ದೇಗುಲಕ್ಕೆ, ಚರ್ಚಿಗೆ ಔದಾರ್ಯವನ್ನೂ ಸಹಾನುಭೂತಿಯನ್ನೂಕೊಂಡೊಯ್ಯುವುದಾದರೆ, ಈಗ ಅತ್ಯುತ್ತಮ ಉದ್ದೇಶದಿಂದಲೇ ಪರಸ್ಪರರನ್ನು ಕಚ್ಚಿ ಹರಿದುಕೊಳ್ಳುತ್ತಿರುವ ಪಂಥಗಳಲ್ಲಿ ಸಾಮರಸ್ಯವನ್ನುಂಟು ಮಾಡುವುದು ತೀರ ಅಸಾಧ್ಯವೇನಲ್ಲ. ಕ್ಯಾನನ್ ಬಸಿಲ್ ವಿಲ್ಬರ್ಫೋರ್ಸ್ ಮತ್ತು ರೆವರೆಂಡ್ ಹೆಚ್.ಆರ್. ಹ್ಯಾವೀಸ್ ಇಬ್ಬರೂ ಸಹ ಸ್ವಾಮಿಗಳಿಗೆ ಉತ್ತರವಾಗಿ ಕುತೂಹಲಕರ ಭಾಷಣಗಳನ್ನು ಮಾಡಿದರು.... ಬಂದಿದ್ದ ಅತಿಥಿಗಳ ಸಂಖ್ಯೆ ಸುಮಾರು ನೂರೈವತ್ತರಷ್ಟಿತ್ತು.

\begin{center}
\textbf{ವಿದ್ಯಾಭ್ಯಾಸ *\supskpt{\footnote{\enginline{1. New Discoveries, Vol. 4, pp.157}}}}
\end{center}

\begin{center}
(ಡೈಲಿ ಕ್ರಾನಿಕಲ್, ೧೪ ಮೇ ೧೮೯೬)
\end{center}

ಸಿಸೇಮ್​ ಕ್ಲಬ್ - ಮಂಗಳವಾರ ರಾತ್ರಿ (ಮೇ ೧೨) ಸಿಸೇಮ್​ ಕ್ಲಬ್ನ ಒಂದು ಗೋಷ್ಠಿಯಲ್ಲಿ ಮೊದಲೇ ಘೋಷಿಸಿದ್ದಂತೆ” ನಾಡಿಗೆ ನಾವು ಹಿಂದಿರುಗಬೇಕೆ?” ಎಂಬ ಚರ್ಚೆಯನ್ನು ಉದ್ಘಾಟಿಸಲು ಮಿಸೆಸ್ ನಾರ್ಮನ್ ಅವರಿಗೆ ತೀವ್ರ ಅನಾರೋಗ್ಯದಿಂದ ಸಾಧ್ಯವಾಗುತ್ತಿಲ್ಲವೆಂದು ಅಧ್ಯಕ್ಷರಾದ ಮಿ. ಆ್ಯಷ್ಟನ್ ಜಾನ್ಸನ್ ಅವರು ವಿಷಾದದಿಂದ ತಿಳಿಸಿದರು. ಆದಕಾರಣವೆ ಸ್ವಾಮಿ ವಿವೇಕಾನಂದರು ವಿದ್ಯಾಭ್ಯಾಸವನ್ನು ಕುರಿತು ಒಂದು ಉಪನ್ಯಾಸವನ್ನು ಕೊಟ್ಟರು\footnote{4. ಪದಶಃ ವರದಿ ಲಭ್ಯವಿಲ್ಲ. ‘ವಿದ್ಯಾಭ್ಯಾಸವನ್ನು ಕುರಿತು’ ಎಂಬ ವಿಷಯದ ಮೇಲಣ ಭಾರತೀಯ ಪತ್ರಿಕಾ ವರದಿಯನ್ನು ನೋಡಿ.}; ಅದರಲ್ಲಿ ಅವರು ದೈಹಿಕ ಶುದ್ಧತೆಯನ್ನುಗಳಿಸುವ ಮುನ್ನ ಯಾರೊಬ್ಬರೂ ಬೌದ್ಧಿಕ ಉನ್ನತಿಯನ್ನು ಪಡೆಯಲಾರರು ಎಂಬುದನ್ನು ಒತ್ತಿ ಹೇಳಿದರು. ನೈತಿಕತೆ ಎಂಬುದು ಶಕ್ತಿಯನ್ನು ಕೊಡುವುದು; ಅನೈತಿಕರು ಯಾವಾಗಲೂ ದುರ್ಬ ಲರೇ, ಅವರು ಬೌದ್ಧಿಕ ಉನ್ನತಿಗೇರಲಾರರು; ಆಧ್ಯಾತ್ಮಿಕ ಉನ್ನತಿಯಂತೂ ಅವರಿಗೆ ತುಂಬ ಕಷ್ಟವೇ. ರಾಷ್ಟ್ರೀಯ ಜೀವನದಲ್ಲಿ ಅನೈತಿಕತೆ ನೇರವಾಗಿ ಪ್ರವೇಶಿಸುತ್ತಿರುವಂತೆಯೇ ಅದರ ತಳಪಾಯ ಕೊಳೆಯಲಾರಂಭಿಸುತ್ತದೆ. ಪ್ರತಿಯೊಂದು ದೇಶದ ಜೀವನಾಡಿಯು ಬಾಲಕ ಬಾಲಕಿಯರು ವಿದ್ಯಾಭ್ಯಾಸವನ್ನು ಪಡೆದುಕೊಳ್ಳುತ್ತಿರುವ ಶಾಲೆಗಳಲ್ಲಿರುವುದರಿಂದ, ಎಳೆಯ ವಿದ್ಯಾರ್ಥಿಗಳು ಶುದ್ಧರಾಗಿರುವುದು ಅತ್ಯಂತ ಅಗತ್ಯ; ಈ ಶುದ್ಧತೆಯನ್ನು ಅವರಿಗೆ ಬೋಧಿಸತಕ್ಕುದು.

\begin{center}
\textbf{ಅಧ್ಯಾತ್ಮವಾದ ವತ್ತು ವೇದಾಂತ ತತ್ತ್ವ\supskpt{\footnote{\enginline{1. New Discoveries, Vol. 4, pp.229-30}}}}
\end{center}

\begin{center}
(ಲೈಟ್, ೪ ಜುಲೈ ೧೮೯೬)
\end{center}

ಸ್ವಾಮಿ ವಿವೇಕಾನಂದರು ವೇದಾಂತ ತತ್ತ್ವ ಪ್ರತಿಪಾದನೆಗಾಗಿ ಲಂಡನ್ನಿಗೆ ಬರು ತ್ತಿರುವರು ಎಂದು ಮೊದಲು ನಾವು ಕೇಳಿದಾಗ, ಅವರ ಬೋಧನೆಯು ಅಧ್ಯಾತ್ಮವಾದಿಗಳ ನಂಬಿಕೆಯನ್ನು ಸ್ಥಿರಪಡಿಸುವುದು ಮಾತ್ರವಲ್ಲದೆ ಅವರ ಸಂಖ್ಯೆಯನ್ನೂ ಹೆಚ್ಚಿಸುವುದು ಎಂದು ಊಹಿಸಿದ್ದೆವು. ಹಿಂದೂ ತತ್ತ್ವಜ್ಞಾನದ ಸಾರವೇ ಮನುಷ್ಯನು ದೇಹವನ್ನುಳ್ಳ ಒಂದು ಚೇತನ ಎಂಬುದು ಹೊರತು, ಚೇತನವನ್ನು ಹೊಂದಿದ ಒಂದು ಶರೀರ ಎಂಬ ಪಾಶ್ಚಾತ್ಯರ ನಿಲುವಿನಂತೆ ಅಲ್ಲ ಎನ್ನು ವುದರ ಮೇಲೆ ನಮ್ಮ ಈ ಊಹೆ ನಿಂತಿತ್ತು...

ಬಾಹ್ಯ ಶರೀರಕ್ಕಿಂತ ಭಿನ್ನವಾದ ಚೈತನ್ಯದ ಅಸ್ತಿತ್ವವನ್ನು ನೈಜ ಪ್ರಾತ್ಯಕ್ಷೀಕರಣದಿಂದ ಸಾಬೀತುಪಡಿಸುವುದು ನಮ್ಮ ನವಯುಗದ ಆಧ್ಯಾತ್ಮಿಕತೆಯ ಮಹಿಮಾನ್ವಿತ ಹಕ್ಕು. ಆದುದರಿಂದ ಈ ಹಿನ್ನಲೆಯಲ್ಲಿ ವೇದಾಂತತತ್ತ್ವ ಪ್ರತಿಪಾದಕರ ಹಾಗೂ ಆಧ್ಯಾತ್ಮವಾದದ ಬೆಂಬಲಿಗರ ಸಹಕಾರವನ್ನು ನೀರಿಕ್ಷೀಸುವುದು ಉಚಿತವಾಗಿದೆ. ತೀರ ಇತ್ತೀಚೆಗೆ ಸ್ವಾಮಿಗಳು ವ್ಯಕ್ತ ಪಡಿಸಿದ ಅಭಿಪ್ರಾಯಗಳು\footnote{2. ಸೇಂಟ್ ಜಾರ್ಜ್ ರಸ್ತೆಯಲ್ಲಿ ೧೮೯೬ರ ಬೇಸಿಗೆಯಲ್ಲಿ ಗುರುವಾರದ ತರಗತಿಯಲ್ಲಿ ನೀಡಿದ ಉಪನ್ಯಾಸ.. ಇದರ ಪದಶಃ ವರದಿ ಲಭ್ಯವಿಲ್ಲ.} ಎರಡು ಪಂಥಗಳನ್ನು ವಿಭಾಗಿಸುವಂತೆ ಇರುವುದರಿಂದ, ಈ ಆಶೋತ್ತರವು ಈಡೇರುವುದೆಂಬುದು ಸಂದೇಹಾಸ್ಪದ. ವೇದಾಂತ ತತ್ತ್ವವು ವಿದ್ಯಾರ್ಥಿಯೆದುರು ಒಂದು ಆದರ್ಶ ಧ್ಯೇಯವನ್ನಿರಿಸುತ್ತದೆ! ಅದು ವಾಸ್ತವವಾಗಿ ತನ್ನೊಳಗಿರುವ ದೇವರನ್ನು ಅನಾವರಣಗೊಳಿಸುವುದಕ್ಕಿಂತ ಕಡಿಮೆಯಾದದ್ದೇನಲ್ಲ; ಸ್ವಾಮಿಗಳಂತಹ ಅಪ್ರತಿಮವಾಗ್ಮಿ ಹಾಗೂ ಶಕ್ತಿಯುತ ಭಾಷಣಕಾರರ ಮೂಲಕ ಈ ಕಲ್ಪನೆಯ ಮಂಡನೆಗಿಂತ ಹೆಚ್ಚು ಸ್ಫೂರ್ತಿಯುತವಾದದ್ದೂ, ಮನಮೆಚ್ಚುವಂಥದ್ದೂ ಇನ್ನೊಂದಿಲ್ಲ. ಅವರು ಆಧುನಿಕ ಅಧ್ಯಾತ್ಮವಾದವನ್ನು ಪ್ರಾಸಂಗಿಕವಾಗಿ ಪ್ರಸ್ತಾಪಿಸುವವರೆಗೆ ನಮ್ಮಲ್ಲಿ ಕೇವಲ ಗೌರವ ಮೆಚ್ಚುಗೆಗಳಿದ್ದವು. ಈ ವಿದ್ಯಮಾನದಲ್ಲಿ ತೊಡಗಿಕೊಂಡಿರುವವರೆಲ್ಲರನ್ನೂ ಎಗ್ಗಿಲ್ಲದೆ ಸ್ವಾಮಿಗಳು ಖಂಡಿಸಿದರೆಂಬ ಭಾವನೆಯೇ ನಮ್ಮ ಮನಸ್ಸಿನಲ್ಲಿ ಉಂಟಾಯಿತು. ಅವಲೋಕನಕ್ಕಾಗಿ ತಾವು ಪರಿಣತ ಮಾಧ್ಯಮಗಳೊಟ್ಟಿಗೆ ಕುಳಿತಿ ದ್ದುದಾಗಿಯೂ, ಅದರಲ್ಲಿನ ಎಲ್ಲರೂ ಆಚರಿಸಿದ್ದು ವಂಚನೆಯನ್ನು ಮಾತ್ರವೇ ಎಂದು ಅವರು ಹೇಳಿದರು. ಸ್ವಾಮಿಗಳ ಪ್ರಕಾರ, ಚೇತನದ ಧ್ವನಿಗಳು ಎಂದೂ ಪರಸ್ಪರ ವಿರೋಧವಾಗಿರುವಂತೆ ಕೇಳಿಸುವುದಿಲ್ಲ! “ಗುಹೆಯಿಂದ ಹೊರಟ ಶೂನ್ಯಧ್ವನಿ ತಣ್ಣ ಗಾಗುತ್ತಲೂ ಎಳೆಯ ಮಗುವಿನ ಧ್ವನಿ ಏರಿಬರುತ್ತದೆ”; ಎಲ್ಲಿಂದಲೋ ಮೂಡಿ ಬರು ತ್ತಿರುವಂತೆ ನುಡಿಯುವ ಗಾರುಡಿವಾಣಿಯ ಕಲೆಯೇ ಈ ಚೇತನವಾಣಿಯ ಮೂಲ ಎಂದು ಇದು ಸೂಚಿಸುತ್ತದೆ.

“ಚೇತನಧ್ವನಿಗಳು ಯಾವ ಪ್ರಯೋಜನಕ್ಕೂ ಬಾರವು, ಏಕೆಂದರೆ ಅವು ‘ನಾನು ಇಲ್ಲಿ ಚೆನ್ನಾಗಿರುವೆ, ಸುಖವಾಗಿರುವೆ’ ಅಥವಾ ‘ಜಾನನಿಗೊಂದು ತುಂಡು ರೊಟ್ಟಿ ಕೊಡು’ ಎನ್ನುವ ಮಟ್ಟಕ್ಕಿಂತ ಮೇಲಕ್ಕೇರಿದ್ದೇ ಇಲ್ಲ” ಎಂದೂ ಅವರು ಹೇಳಿದರು.

ಸ್ವಾಮಿ ವಿವೇಕಾನಂದರ ಅತ್ಯಂತ ಉದಾತ್ತವಾದ ಬೋಧನೆಯೊಂದಿಗೆ ಸರಿಸಮವಾಗಿ ನಿಲ್ಲಬಲ್ಲ “ಚೇತನ ಬೋಧೆಗಳು” ಎಂಬ ಪುಸ್ತಕದೊಳಗೆ ಏನಿದೆ ಎಂಬುದರ ಬಗ್ಗೆ ಅಜ್ಞಾನವಿದ್ದರೆ ಮಾತ್ರವೇ ಈ ಮಾತನ್ನು ಹೇಳಬಹುದು. ಅಣಕು ಚೇತನಪ್ರತ್ಯಕ್ಷೀ ಕರಣದ ವಿಧಾನವನ್ನೂ, ತಂತಿಯೊಂದರ ಕೊನೆಯಲ್ಲಿ ಚಿತ್ರವೊಂದನ್ನು ಮೂಡಿಸುವ ವಿಧಾನವನ್ನೂ ಸಹ ಅವರು ಎಳೆಎಳೆಯಾಗಿ ವಿವರಿಸಿದರು.

ನಾವು ಮಾರನೆಯ ದಿನ ಸಂಜೆ\footnote{1. ಸೇಂಟ್ ಜಾರ್ಜ್ ರಸ್ತೆಯಲ್ಲಿ ಶುಕ್ರವಾರ ಸಂಜೆ ನಡೆದ ತರಗತಿಯಲ್ಲಿ ನೀಡಿದ ಉಪನ್ಯಾಸ ಇದರ ಪದಶಃ ವರದಿ ಲಭ್ಯವಿಲ್ಲ.} ಯೂ ಹಾಜರಿದ್ದೆವು; ಆಗ ಗೋಷ್ಠಿಯಲ್ಲಿ ಸ್ವಾಮಿಗಳ ತೀವ್ರವಾದ ಟೀಕೆಗಳ ಬಗ್ಗೆ ಪ್ರಶ್ನೆಗಳನ್ನುಳ್ಳ ಮಸೂದೆಯೊಂದನ್ನು ಓದಲಾಯಿತು. ಕಳೆದ ರಾತ್ರಿಯ ತಮ್ಮ ಹೇಳಿಕೆಗಳ ಸ್ಪಷ್ಟೀಕರಣದಲ್ಲಿ, ಅವನ್ನು ವಿವರಿಸುವುದರಲ್ಲಿ ಸುಮಾರು ಅರ್ಧಗಂಟೆ ಕಾಲ ಕಳೆಯಿತು; ನಮ್ಮ ಆಂತರ್ಯಕ್ಕೆ ತೃಪ್ತಿಯಾಗುವ ಹಾಗೆ ಸ್ವಾಮಿಗಳು ಚೇತನಗಳು ಮನುಷ್ಯರೊಂದಿಗೆ ವಿನಿಮಯಿಸುವ ಸಾಧ್ಯತೆಯ ಬಗ್ಗೆ ತಮಗೆ ನಂಬಿಕೆ ಯಿರುವುದೆಂದೂ, ಕೆಲವೊಮ್ಮೆ ಉನ್ನತಸ್ತರದ ಚೇತನಗಳು ಮಾನವರಿಗೆ ಸಹಾಯ ಮಾಡುವುದಕ್ಕೆಂದು ಭೂಮಿಗೆ ಬರುವುವೆಂದೂ ಅಭಿಪ್ರಾಯಪಟ್ಟರು. ವೇದಾಂತ ತತ್ತ್ವದ ಯಾವ ಅಂಶವೂ ಚೇತನಗಳೊಂದಿಗೆ ಅಂತಹ ವಿನಿಮಯವನ್ನು, ಅದರಲ್ಲಿ ಅಡಕವಾಗಿರುವ ಅಪಾಯಗಳ ಕಾರಣಕ್ಕಾಗಿ ಶಿಫಾರಸು ಮಾಡುವುದಿಲ್ಲ ಎಂಬುದು ನಮಗೆ ಮನ ದಟ್ಟಾಯಿತು. ಪ್ರಚಲಿತ ನಂಬಿಕೆಯಂತೆ ಮುಂಬರಿಯದ ಕೆಲವು ಚೇತನಗಳು ಮನುಷ್ಯ ನೊಂದಿಗೆ ಅತ್ಯಂತ ಸುಲಭವಾಗಿ ಭಾವವಿನಿಮಯ ಮಾಡಿಕೊಳ್ಳಬಲ್ಲವು; ಆದಕಾರಣ ಸ್ವಾಮಿಗಳು ತಮ್ಮ ಎಚ್ಚರಿಕೆಯನ್ನು ನೀಡುವುದರೊಂದಿಗೆ, ಆ ದಿಶೆಯಲ್ಲಿ ಪ್ರೋತ್ಸಾಹದ ಯಾವ ಮಾತನ್ನೂ ಆಡಲಿಲ್ಲ....

\begin{center}
\textbf{ಒಂದು ಅಕ್ಟೋಬರ್ ತರಗತಿಯ ಪರಾಮರ್ಶೆ *\supskpt{\footnote{\enginline{1. New Discoveries, Vol. 4, pp.370-71}}}}
\end{center}

\begin{center}
(ಲೈಟ್, ೨೮ ಅಕ್ಟೋಬರ್, ೧೮೯೬)
\end{center}

ವಿಕ್ಟೋರಿಯಾ ರಸ್ತೆಯ ಒಂದು ಮಂಕುಕವಿದ ಆದರೆ ಅನುಕೂಲವಾದ ಮನೆಯ ಆರನೆಯ ಮಹಡಿಯಲ್ಲಿ ಇತ್ತೀಚೆಗೆ ನಾವು ಸ್ವಾಮಿ ವಿವೇಕಾನಂದರ ಒಂದು ಉಪನ್ಯಾಸವನ್ನು ಕೇಳಿದೆವು. ಇದು ಹಿಂದೂಧರ್ಮ ಹಾಗೂ ತತ್ತ್ವಜ್ಞಾನದ ಮೇಲಣ ದೀರ್ಘ ಉಪನ್ಯಾಸ ಸರಮಾಲೆಯಲ್ಲಿ ಒಂದು ಉಪನ್ಯಾಸವಾಗಿತ್ತು\footnote{2. ಬಹುಶಃ ಇಗ್ಲೆಂಡಿನ ಲಂಡನ್ನಿನಲ್ಲಿ ೧೮೯೬ರ ಅಕ್ಟೋಬರ್ ೨೮ರಂದು ಕೊಟ್ಟ “ವೈದಿಕ ಧರ್ಮದ ಆದರ್ಶಗಳು” ಎಂಬ ಉಪನ್ಯಾಸ. ಇದರ ಪದಶಃ ವರದಿ ಲಭ್ಯವಿಲ್ಲ.}.... ಅವರು ಒಂದೂವರೆ ಗಂಟೆಯ ಕಾಲ ಯಾವುದೇ ಟಿಪ್ಪಣಿಯಿಲ್ಲದೆ ಮಾತನಾಡಿದರು. ಉಪನ್ಯಾಸವು ಒಂದು ಪ್ರಬಂಧವೆನ್ನಬಹುದಾದ ಅಧ್ಯಯನ ಎನ್ನುವುದಕ್ಕಿಂತ ಉಕ್ತಿಗಳ ಪ್ರವಾಹ ಎನ್ನಬಹುದಾದರೂ, ಎಲ್ಲವೂ ತುಂಬಾ ಆಸಕ್ತಿ ಮೂಡಿಸುವಂಥದಾಗಿತ್ತು.

ವೇದಗಳೇ ಮುಖ್ಯ ವಿಷಯವಾಗಿದ್ದರೂ, ಜೀವವಿಕಾಸ, ನವಯುಗದ ವಿಜ್ಞಾನ, ಆದರ್ಶ ಹಾಗೂ ವಾಸ್ತವಿಕತೆ, ಚೈತನ್ಯದ ಪಾರಮ್ಯ ಇತ್ಯಾದಿಗಳ ಮೇಲೆ ಭಾಷಣಕಾರರು ನಮ್ಮನ್ನು ಪ್ರವಾಸಕ್ಕೆ ಕರೆದೊಯ್ದಂತಿತ್ತು. ಒಟ್ಟಿನಲ್ಲಿ, ಭಾಷಣಕಾರರು ಆಧ್ಯಾತ್ಮಿಕ ಉನ್ನತಿ ಹಾಗೂ ಆಧ್ಯಾತ್ಮಿಕ ಸಾಮರಸ್ಯಗಳ ಒಂದು ವಿಶ್ವಧರ್ಮದ ಪ್ರಬೋಧಕರು ಎಂಬುದನ್ನು ಗ್ರಹಿಸಿದೆವು. ನಡುನಡುವೆ ವೇದಗಳಿಂದ ಸುಂದರವಾಗಿ ಅನುವಾದಿಸಿ ಪಠಿಸಲಾದ, ಮಾನವೀಯತೆಯ ಗುಣದಿಂದ ಕೂಡಿದ ಹಾಗೂ ಅವಗುಂಠನದಿಂದಾಚೆಗಿನ ಬದುಕಿನ ದಾರುಣ ವಾಸ್ತವಿಕತೆಯನ್ನೊಳಗೊಂಡ ಕೆಲವು ಭಾಗಗಳು ಅತ್ಯಂತ ಮನಮೋಹಕವಾಗಿದ್ದವು. ಯಾರೂ ಇಂಥವನ್ನು ಇನ್ನೂ ಹೆಚ್ಚು ಬೇಡುವಂತಿತ್ತು.

ವೇದಗಳಲ್ಲಿ ಅನೇಕ ವಿರೋಧಾಭಾಸಗಳಿವೆ ಎಂದು ಒಪ್ಪಿಕೊಂಡದ್ದು ಮತ್ತು ಶ್ರದ್ಧಾ ವಂತ ಹಿಂದೂಗಳು ಅವುಗಳನ್ನು ಅಲ್ಲಗಳೆಯಲಾಗಲಿ ಅವುಗಳೊಂದಿಗೆ ಹೊಂದಿಕೊಳ್ಳ ಲಾಗಲಿ ಹೊರಡುವುದಿಲ್ಲ ಎಂದದ್ದು ನಮಗೆ ಬಹು ಮೆಚ್ಚಿಕೆಯಾಯಿತು. ಪ್ರತಿಯೊಬ್ಬರೂ ತಮಗೆ ಬೇಕಾದ್ದನ್ನು ತೆಗೆದುಕೊಳ್ಳಲು ಸ್ವತಂತ್ರರು. ಬದುಕಿನ ಒಂದಲ್ಲ ಒಂದು ಹಂತದಲ್ಲಿ, ಒಂದಲ್ಲ ಒಂದು ಸ್ತರದಲ್ಲಿ ಎಲ್ಲವೂ ಸತ್ಯವಾಗಿಯೇ ತೋರುವುವು. ಆದ್ದರಿಂದ ಹಿಂದೂಗಳು ಯಾರನ್ನೂ ಹಿಂಸಿಸುವುದೂ ಇಲ್ಲ, ಹೊರಗಿಡುವುದೂ ಇಲ್ಲ. ವೇದಗಳಲ್ಲಿನ ವಿರೋಧಾಭಾಸಗಳು ಜೀವನದಲ್ಲಿಯ ವಿರೋಧಾಭಾಸಗಳಂತೆಯೇ - ಅವು ತುಂಬ ವಾಸ್ತವ, ಆದರೂ ಅವು ಎಲ್ಲವೂ ಸತ್ಯ. ಇದು ಅಸಾಧ್ಯವೆಂದು ಕಾಣುವುದಾದರೂ, ಅದರಲ್ಲಿ ಗಾಢವಾದ ಅರ್ಥವಿದೆ. ಎಲ್ಲ ಸಂದರ್ಭಗಳಲ್ಲಿಯೂ, ಹಿಂಸಿಸುವ ಹಾಗೂ ಹೊರಗಿಡುವ ವಿಚಾರದಲ್ಲಿ, ಹಿಂದೂಗಳ ಹಕ್ಕುಗಳು ಕ್ರೈಸ್ತರದೂ ಆಗಲಿ ಎಂದು ಮಾತ್ರ ನಾವು ಹಾರೈಸಲು ಸಾಧ್ಯ.

