
\addtocontents{toc}{\protect\vspace{-0.7cm}}

\part{ಸಂಭಾಷಣೆಗಳು ಮತ್ತು ಸಂದರ್ಶನಗಳು}

\chapter{ಮೇಡಂ ಎಮ್ಮಾ ಕಾಲ್ವಿಯೊಡನೆ ಮೊದಲ ಭೇಟಿ}

(ಕಾಲ್ವಿಯ ಆತ್ಮಕಥೆಯಾದ ‘My life ‘ಎಂಬ ಪುಸ್ತಕದಲ್ಲಿ ಚಿತ್ರಿಸಿರುವಂತೆ ಸ್ವಾಮಿ ವಿವೇಕಾನಂದರೊಡನೆ ಮೇಡಂ ಕಾಲ್ವಿಯ ಪ್ರಥಮ ಭೇಟಿಯ ಕಥೆ)

..... ಸ್ವಾಮಿ ವಿವೇಕಾನಂದರು ಚಿಕಾಗೊದಲ್ಲಿ ಒಂದು ವರ್ಷ ಉಪನ್ಯಾಸ ನೀಡು ತ್ತಿದ್ದ ಸಂದರ್ಭದಲ್ಲಿ ನಾನು ಅಲ್ಲಿದ್ದೆ. ನಾನು ಆಗ ಮಾನಸಿಕವಾಗಿಯೂ ದೈಹಿಕವಾಗಿಯೂ ಅತ್ಯಂತ ಖಿನ್ನಳಾಗಿದ್ದುದರಿಂದ ಅವರನ್ನು ಸಂಧಿಸುವ ನಿರ್ಧಾರಮಾಡಿದೆ.

..... ಅವರೇ ನನ್ನನ್ನು ಕೇಳುವವರೆಗೆ ನಾನು ಮಾತನಾಡಬಾರದೆಂದು ಇತರರು ನಾನು ಹೋಗುವುದಕ್ಕೆ ಮುಂಚೆ ನನಗೆ ಸಲಹೆ ನೀಡಿದ್ದರು. ನಾನು ಕೋಣೆಯನ್ನು ಪ್ರವೇಶಿಸಿ ಮೌನವಾಗಿ ಸ್ವಲ್ಪ ಹೊತ್ತು ಅವರ ಮುಂದೆ ನಿಂತುಕೊಂಡೆ. ಅವರು ಧ್ಯಾನ ಸ್ಥರಾಗಿ ಗಂಭೀರವಾಗಿ ಕುಳಿತಿದ್ದರು. ಅವರು ಕಾವಿ ಬಟ್ಟೆ ನೆಲದ ಮೇಲೆ ಹಾಸಿಕೊಂಡಿತ್ತು. ಪೇಟದಿಂದ ಕೂಡಿದ ಅವರ ತಲೆಯು ಮುಂದಕ್ಕೆ ಬಾಗಿಕೊಂಡಿತ್ತು ಮತ್ತು ಕಣ್ಣುಗಳು ನೆಲದ ಕಡೆಗೆ ನೆಟ್ಟಿದ್ದವು. ಸ್ವಲ್ಪ ಹೊತ್ತಿನ ನಂತರ ಅವರು ತಲೆಯನ್ನು ಎತ್ತದೆಯೇ ಹೇಳ ತೊಡಗಿದರು:

“ಮಗು, ಎಂತಹ ದುಃಖದಿಂದ ಕೂಡಿರುವೆ ನೀನು! ಶಾಂತಳಾಗು, ಇದು ಆವಶ್ಯಕ.” ಅನಂತರ ಶಾಂತ ಧ್ವನಿಯಲ್ಲಿ ನಿರ್ಲಿಪ್ತಭಾವದಲ್ಲಿ, ನನ್ನ ಹೆಸರನ್ನೂ ಅರಿತಿರದ ಅವರು ನನ್ನ ರಹಸ್ಯ ಸಮಸ್ಯೆಗಳು ಮತ್ತು ಕಾತರತೆಯ ಬಗ್ಗೆ ಹೇಳತೊಡಗಿದರು. ನನ್ನ ನಿಕಟ ಸ್ನೇಹಿತರಿಗೂ ತಿಳಿದಿರದ ವಿಷಯಗಳನ್ನು ಅವರು ಹೇಳಿದರು. ಇದು ಪವಾಡ, ಅಪ್ರಾಕೃತ ಎಂದು ನನಗನಿಸಿತು.

ಕೊನೆಗೆ ನಾನು ಕೇಳಿದೆ: “ಇದೆಲ್ಲ ನಿಮಗೆ ಹೇಗೆ ತಿಳಿಯಿತು? ಯಾರು ನಿಮಗೆ ನನ್ನ ಬಗ್ಗೆ ಹೇಳಿದರು?”

ಒಂದು ಮೂರ್ಖ ಪ್ರಶ್ನೆಯನ್ನು ಕೇಳಿದ ಮಗುವನ್ನು ನೋಡುವಂತೆ ಅವರು ನನ್ನನ್ನು ನೋಡಿ ಮುಗುಳ್ನಕ್ಕರು.

ಅವರು ಹೇಳಿದರು: “ಯಾರೂ ಹೇಳಲಿಲ್ಲ. ಅದು ಆವಶ್ಯಕ ಎಂದು ಭಾವಿಸು ವೆಯಾ? ನಾನು ತೆರೆದ ಪುಸ್ತಕವನ್ನು ಓದುವಂತೆ ನಾನು ನಿನ್ನೊಳಗಿರುವುದನ್ನು ತಿಳಿಯುತ್ತೇನೆ.”

ನಾನು ಹಿಂತಿರುಗುವ ಸಮಯ ಬಂತು.

ನಾನು ಮೇಲೇಳುತ್ತಿದ್ದಂತೆ, “ನೀನು ಎಲ್ಲವನ್ನೂ ಮರೆಯಬೇಕು” ಎಂದರವರು.

“ಸಂತೋಷದಿಂದಲೂ ನೆಮ್ಮದಿಯಿಂದಲೂ ಇರು. ನಿನ್ನ ಆರೋಗ್ಯವನ್ನು ಸುಧಾರಿಸಿಕೊ. ಮೌನವಾಗಿ ನಿನ್ನ ದುಃಖವನ್ನು ಕುರಿತು ಚಿಂತಿಸಬೇಡ. ನಿನ್ನ ಭಾವೋ ದ್ವೇಗವನ್ನು ಯಾವುದಾದರೂ ರೂಪದಲ್ಲಿ ಹೊರಗೆ ವ್ಯಕ್ತಪಡಿಸು. ನಿನ್ನ ಆಧ್ಯಾತ್ಮಿಕ ಆರೋಗ್ಯಕ್ಕೆ ಇದು ಆವಶ್ಯಕ. ನಿನ್ನ ಕಲೆಗೆ ಇದು ಅತಿಮುಖ್ಯ,” ಎಂದು ಅವರು ಹೇಳಿದರು.

ಅವರ ಮಾತುಗಳು ಮತ್ತು ವ್ಯಕ್ತಿತ್ವದಿಂದ ತುಂಬ ಪ್ರಭಾವಿತಳಾಗಿ ನಾನು ಅವರಿಂದ ಬೀಳ್ಕೊಂಡೆ. ನನ್ನ ಮಿದುಳಿನಲ್ಲಿದ್ದ ಎಲ್ಲ ಉದ್ವೇಗಗಳನ್ನೂ ಹೋಗಲಾಡಿಸಿ ಅಲ್ಲಿ ತಮ್ಮ ಸ್ಪಷ್ಟ ಮತ್ತು ಶಾಂತ ಭಾವನೆಗಳನ್ನು ತುಂಬಿದರು. ನಾನು ಪುನಃ ಉತ್ಸಾಹ ಭರಿತಳಾದೆ ಮತ್ತು ಆನಂದಭರಿತಳಾದೆ. ಇದೆಲ್ಲ ಅವರ ಶಕ್ತಿಯುತ ಇಚ್ಛೆಯ ಪರಿಣಾಮ. ಅವರು ಯಾವ ಸಂಮೋಹಿನಿ ವಿದ್ಯೆಯನ್ನೂ ಪ್ರಯೋಗಿಸಲಿಲ್ಲ. ಅವರ ಚಾರಿತ್ರ್ಯಶಕ್ತಿ, ಪಾವಿತ್ರ್ಯ, ಉದ್ದೇಶದ ತೀವ್ರತೆ – ಇವು ಭರವಸೆಯನ್ನು ನೀಡಿದವು. ಅವರ ನಿಕಟ ಪರಿಚಯವಾದ ಮೇಲೆ ತಿಳಿದುಬಂತು: ಅವರು ವ್ಯಕ್ತಿಯ ಪ್ರಕ್ಷುಬ್ಧ ಮನಸ್ಸನ್ನು ಶಾಂತಗೊಳಿಸಿ ತಮ್ಮ ಮಾತುಗಳನ್ನು ಅವನು ಪೂರ್ಣಗಮನವಿಟ್ಟು ಕೇಳುವಂತೆ ಮಾಡುತ್ತಿದ್ದರು.

\begin{center}
\textbf{ಜಾನ್ ಡಿ. ರಾಕ್ಫೆಲರ್ನೊಡನೆ ಮೊದಲ ಭೇಟಿ}
\end{center}

\begin{center}
(ಡ್ರಿನೆಟ್ ವಾರ್ಡೀರ್ಳಿಗೆ ಎಮ್ಮಾ ಕಾಲ್ವಿ ಹೇಳಿದಂತೆ.)
\end{center}

ಸ್ವಾಮಿ ವಿವೇಕಾನಂದರು ಚಿಕಾಗೊದಲ್ಲಿ ತಂಗಿದ್ದ ಮನೆಯ ಯಜಮಾನನು ರಾಕ್ಫೆಲರ್ನೊಡನೆ ವ್ಯಾಪಾರ ಸಂಬಂಧವನ್ನು ಹೊಂದಿದ್ದನು. ತನ್ನ ಸ್ನೇಹಿತನು ಅವನ ಮನೆಯಲ್ಲಿ ತಂಗಿದ್ದ ಅಸಾಮಾನ್ಯ ಸಂನ್ಯಾಸಿಯ ಬಗ್ಗೆ ಹೇಳುತ್ತಿದ್ದುದನ್ನು ರಾಕ್ ಫೆಲರ್ ಕೇಳಿದ್ದ ಮತ್ತು ಸ್ವಾಮೀಜಿಯವರನ್ನು ಬಂದು ಭೇಟಿಯಾಗಬೇಕೆಂದೂ ಸ್ನೇಹಿತನು ಅವನನ್ನು ಕೇಳಿಕೊಂಡಿದ್ದನು. ಆದರೆ ಯಾವುದಾದರೂ ಕಾರಣದಿಂದ ಹೋಗುವುದನ್ನು ನಿರಾಕರಿಸುತ್ತಿದ್ದನು. ಆ ಸಂದರ್ಭದಲ್ಲಿ ರಾಕ್ಫೆಲರ್ ಇನ್ನೂ ತನ್ನ ಶ‍್ರೀಮಂತಿಕೆಯ ಅತ್ಯುನ್ನತ ಸ್ಥಿತಿಗೇರಿರಲಿಲ್ಲ. ಆದರೆ ಅವನು ಪ್ರಬಲ ಇಚ್ಛಾಶಕ್ತಿಯುಳ್ಳವನಾಗಿದ್ದನು; ಅವನೊಡನೆ ವ್ಯವಹರಿಸುವುದು ಕಷ್ಟವಾಗಿತ್ತು ಮತ್ತು ಅವನಿಗೆ ಏನಾದರೆ ಸಲಹೆ ನೀಡುವುದಂತೂ ಕಷ್ಟಸಾಧ್ಯ.

ಆದರೆ, ಅವನಿಗೆ ಸ್ವಾಮೀಜಿಯವರನ್ನು ಭೇಟಿಮಾಡಲು ಇಷ್ಟವಿಲ್ಲದಿದ್ದರೂ, ಒಂದು ದಿನ ಯಾವುದೊ ಪ್ರೇರಣೆಗೆ ಒಳಗಾಗಿ ಅವರನ್ನು ನೋಡಲು ಸ್ನೇಹಿತನ ಮನೆಗೆ ಹೋದನು. ಬಾಗಿಲನ್ನು ತೆರೆದ ಮುಖ್ಯ ಸೇವಕನನ್ನು ಗಮನಿಸದೆಯೇ ತಾನು ಹಿಂದೂ ಸಂನ್ಯಾಸಿಯನ್ನು ನೋಡಬೇಕಾಗಿದೆ ಎಂದು ಹೇಳಿದನು.

ಸೇವಕನು ಅವನನ್ನು ವಸತಿ ಕೊಠಡಿಗೆ ಕರೆದುಕೊಂಡು ಹೋದನು. ತನ್ನ ಆಗಮನವನ್ನು ತಿಳಿಸುವುದಕ್ಕೆ ಮುಂಚೆಯೇ ಅವನು ಸ್ವಾಮಿ ವಿವೇಕಾನಂದರ ಅಧ್ಯಯನ ಕೊಠಡಿಯನ್ನು ಪ್ರವೇಶಿಸಿದನು. ಯಾವುದೊ ಪುಸ್ತಕವನ್ನು ಓದುತ್ತಿದ್ದ ಸ್ವಾಮೀಜಿ ಯವರು ತಲೆಯನ್ನು ಎತ್ತಿ ನೋಡದಿರುವುದು ಅವನಿಗೆ ಅಚ್ಚರಿಯನ್ನುಂಟುಮಾಡಿತು.

ಸ್ವಲ್ಪ ಹೊತ್ತಿನ ನಂತರ, ಕಾಲ್ವಿ ವಿಷಯದಲ್ಲಿಯಂತೆಯೆ, ಸ್ವಾಮೀಜಿಯವರು ರಾಕ್ಫೆಲರ್ಗೆ ಮಾತ್ರ ತಿಳಿದಿದ್ದ ಅವನ ಹಿಂದಿನ ವಿಷಯಗಳನ್ನು ಹೇಳತೊಡಗಿದರು; ಅವನು ಆಗಲೇ ಸಂಗ್ರಹಿಸಿದ ಹಣವು ಅವನದಲ್ಲ, ಅವನು ಕೇವಲ ಒಂದು ಪ್ರಣಾಳಿಕೆ, ಅವನು ತನ್ನ ಹಣದ ಮೂಲಕ ಜಗತ್ತಿನ ಹಿತವನ್ನು ಸಾಧಿಸಬೇಕು ಎಂದು ಅವನಿಗೆ ತಿಳಿವಳಿಕೆ ನೀಡಿದರು. ಜನರಿಗೆ ಸಹಾಯ ಮಾಡಲೆಂದು ದೇವರೇ ಅವನಿಗೆ ಅಷ್ಟೊಂದು ಐಶ್ವರ್ಯವನ್ನು ನೀಡಿರುವನು ಎಂದು ಅವನಿಗೆ ತಿಳಿಸಿದರು.

ತನ್ನೊಡನೆ ಈ ರೀತಿ ಮಾತನಾಡಿ, ತನಗೆ ಸಲಹೆ ನೀಡುವ ಸಾಹಸವನ್ನು ನೋಡಿ ರಾಕ್ಫೆಲರ್ ಕುಪಿತನಾದನು. ಅವನು ವ್ಯಗ್ರನಾಗಿ ಅಭಿನಂದನೆಯನ್ನೂ ಸೂಚಿಸದೆ ಕೊಠಡಿಯಿಂದ ಹೊರ ಹೊರಟನು. ಆದರೆ ಕೆಲವು ದಿನಗಳನಂತರ ಮತ್ತೆ ಪೂರ್ವ ಮಾಹಿತಿಯಿಲ್ಲದೆ ಸ್ವಾಮೀಜಿಯ ಕೊಠಡಿಯನ್ನು ಪ್ರವೇಶಿಸಿದನು. ಅವರು ಮೊದಲಿ ನಂತೆಯೇ ಓದುತ್ತ ಕುಳಿತಿದ್ದರು. ಒಂದು ಸಾರ್ವಜನಿಕ ಸಂಸ್ಥೆಗೆ ಅಪಾರ ಧನಸಹಾಯ ಮಾಡುವ ತನ್ನ ಯೋಜನೆಯನ್ನೊಳಗೊಂಡ ಒಂದು ಪತ್ರವನ್ನು ಅವರ ಡೆಸ್ಕಿನ ಮೇಲೆ ಎಸೆದನು.

“ಇಲ್ಲಿ ನೋಡಿ, ನಿಮಗೀಗ ತೃಪ್ತಿಯಾಗಿರಬಹುದು. ಈಗ ನೀವು ನನಗೆ ಕೃತಜ್ಞತೆ ಅರ್ಪಿಸಬೇಕು”, ಎಂದನು.

ಸ್ವಾಮೀಜಿ ಕಣ್ಣನ್ನೂ ಎತ್ತಲಿಲ್ಲ, ಸ್ವಲ್ಪವೂ ಚಲಿಸಲಿಲ್ಲ. ಆ ಪತ್ರವನ್ನು ಕೈಯಲ್ಲಿ ತೆಗೆದುಕೊಂಡು ಶಾಂತವಾಗಿ ಓದಿದರು. “ನೀನೇ ನನಗೆ ಕೃತಜ್ಞತೆಯನ್ನು ಸಲ್ಲಿಸಬೇಕು” ಎಂದರು. ಇಲ್ಲಿಗೆ ಈ ಘಟನೆ ಮುಗಿಯಿತು. ಇದು ರಾಕ್ಫೆಲರ್ ಪ್ರಪ್ರಥಮವಾಗಿ ನೀಡಿದ ಸಾರ್ವಜನಿಕ ಬೃಹತ್ ದಾನ.

\begin{center}
\textbf{ಭಾರತದ ಕಂದು ಬಣ್ಣದ ತತ್ತ್ವಜ್ಞಾನಿ}
\end{center}

(ಮಾರ್ಚ್ ೧೮, ೧೯೦೦ರ \enginline{‘San Francisco Chronicle ‘}ನಲ್ಲಿ ಪ್ರಕಟವಾದ, ಬ್ಲೆಂಚ್ ಪಾರ್ಟಿಂಗ್ಟನ್ರವರು ಸ್ವಾಮಿ ವಿವೇಕಾನಂದರೊಡನೆ ನಡೆಸಿದ ಸಂದರ್ಶನ.)

..... ‘ಡೈಲಿ ಪ್ರೆಸ್’ ಎಂಬ ಪೂರ್ಣಪಾಶ್ಚಾತ್ಯವಾದ ಸಂಸ್ಥೆಯ ಪ್ರತಿನಿಧಿಯನ್ನು ಗಂಗಾತೀರದಿಂದ ಬಂದಿರುವ ಕಂದುಬಣ್ಣದ ತತ್ತ್ವಜ್ಞಾನಿಗಳು ಪೌರ್ವಾತ್ಯ ಶೈಲಿಯಲ್ಲಿ ತಲೆಬಾಗಿ, ಕೈ ನೀಡಿ ಆದರದಿಂದ ಬರಮಾಡಿಕೊಂಡರು.

...... ಈ ಲೇಖನದಲ್ಲಿ ಅಚ್ಚುಹಾಕಿಸುವುದಕ್ಕಾಗಿ ಒಂದು ಚಿತ್ರವನ್ನು ಕೇಳಿದೆ. ಯಾರೋ ಒಬ್ಬರು ಆಗಲೇ ಉಪನ್ಯಾಸದ ಜಾಹಿರಾತಿಗೆ ಬಹಳ ಬಾರಿ ಉಪಯೋಗಿ ಸಲ್ಪಟ್ಟಿದ್ದ ಒಂದು ಚಿತ್ರವನ್ನು ನೀಡಿದಾಗ, ಅದರ ಬಳಕೆಯನ್ನು ಮೃದುವಾಗಿ ವಿರೋಧಿಸಿದರು.

“ಇದು ನಿಮಗೆ ತರವಲ್ಲವೆಂದೆನಿಸುತ್ತದೆ”, – ನಾನು ಹೇಳಿದೆ.

“ಇಲ್ಲ, ಆ ಚಿತ್ರದಲ್ಲಿ ನಾನು ಯಾರನ್ನೋ ಕೊಲ್ಲಲು ಉದ್ಯುಕ್ತನಾಗಿರುವಂತೆ ಕಾಣುತ್ತೇನೆ” – ಎಂದು ಅವರು ಮುಗುಳ್ನಗುತ್ತ ಹೇಳಿ, “ಅವನಂತೆ.....” ಎಂದು ಅವರು ಹೇಳುತ್ತಿದ್ದಂತೆಯೇ “ಒಥಲೊನಂತೆಯೇ?” ಎಂದು ನಾನು ಮುಂದು ವರಿಸಿದೆ. ಅಲ್ಲಿ ನೆರೆದಿದ್ದ ಅಲ್ಪ ಮಂದಿ ಶ್ರೋತೃಗಳು ಸ್ವಾಮೀಜಿಯವರ ಮಾತನ್ನು ಕೇಳಿ ಮುಗುಳ್ನಕ್ಕರು. ಆದರೆ ನಾನು ಆ ಚಿತ್ರವನ್ನು ಉಪಯೋಗಿಸಲಿಲ್ಲ.

ನಾನು ಕೇಳಿದೆ: “ಸ್ವಾಮೀಜಿ, ನೀವು ವಿಶ್ವಧರ್ಮ ಸಮ್ಮೇಳನದ ನಂತರ ಸ್ವದೇಶಕ್ಕೆ ಹಿಂದಿರುಗಿದಾಗ ರಾಜರುಗಳು ನಿಮ್ಮ ಪಾದಕ್ಕೆರಗಿದರಂತೆ, ನೀವು ಕುಳಿತಿದ್ದ ಗಾಡಿಯನ್ನು ಐದಾರು ಜನ ಆಳುವ ರಾಜರುಗಳು ಬೀದಿಯಲ್ಲಿ ಎಳೆದರಂತೆ – ಇದು ನಿಜವೆ? ನಾವು ಧರ್ಮಾಧಿಕಾರಿಗಳನ್ನು ಹೀಗೆ ನೋಡಿಕೊಳ್ಳುವುದಿಲ್ಲ.”

ಸ್ವಾಮೀಜಿ ಹೇಳಿದರು, “ಇದು ಮಾತನಾಡುವುದಕ್ಕೆ ಯೋಗ್ಯವಾದ ವಿಷಯವಲ್ಲ. ಆದರೆ ಅಲ್ಲಿ ಧರ್ಮವೇ ಆಳುತ್ತಿರುವುದು, ಹಣವಲ್ಲ ಎಂಬುದಂತೂ ನಿಜ.”

“ಹಾಗಾದರೆ ಜಾತಿಯ ಬಗ್ಗೆ ಏನು ಹೇಳುತ್ತೀರಿ?”

ಅವರು ಮುಗುಳ್ನಕ್ಕು ನುಡಿದರು: “ನಿಮ್ಮ ವರಿಷ್ಠ ವರ್ಗದವರ ವಿಷಯವೇನು? ಭಾರತೀಯ ವರ್ಣವ್ಯವಸ್ಥೆಯನ್ನು ಅರ್ಥಮಾಡಿಕೊಳ್ಳುವುದು ಪಾಶ್ಚಿಮಾತ್ಯ ಮನಸ್ಸಿಗೆ ಕಷ್ಟ. ಅದೊಂದು ಅಪೂರ್ಣ ವ್ಯವಸ್ಥೆಯೆಂದು ಗುರುತಿಸಲಾಗಿದೆ, ಆದರೆ ನಿಮ್ಮಲ್ಲಿಯ ವರ್ಗಭೇದಗಳಿಂದಲೂ ಹೆಚ್ಚಿನ ಪ್ರಯೋಜನವಿದೆಯೆಂಬುದು ಕಂಡುಬರುವುದಿಲ್ಲ. ತನ್ನ ಜನರ ಮೇಲೆ ಶಾಶ್ವತವಾದ ವರ್ಣವ್ಯವಸ್ಥೆಯನ್ನು ವಿಧಿಸುವುದರಲ್ಲಿ ಭಾರತವು ಮಾತ್ರ ಯಶಸ್ವಿಯಾಗಿದೆ. ಪಾಶ್ಚಿಮಾತ್ಯ ಮೂಢನಂಬಿಕೆ ಮತ್ತು ದೋಷಗಳಿಂದ ಆ ದೇಶಕ್ಕೆ ಒಳಿತಾಗುವುದೆಂಬುದು ಸಂದೇಹಾಸ್ಪದ.

“ಆದರೆ ಅಂತಹ ವ್ಯವಸ್ಥೆಯಲ್ಲಿ ಯಾರೂ ತಾವಿಷ್ಟಪಟ್ಟಿದ್ದನ್ನು ತಿನ್ನುವಂತಿಲ್ಲ, ಕುಡಿಯುವಂತಿಲ್ಲ; ತಮಗಿಷ್ಟವಾದವರನ್ನು ವಿವಾಹವಾಗುವಂತಿಲ್ಲ. ನೀವು ಬೋಧಿಸುವ ಸ್ವಾತಂತ್ರ್ಯ ಅಲ್ಲಿ ಅಸಾಧ್ಯ” – ಎಂದು ನಾನು ಹೇಳುವ ಧೈರ್ಯಮಾಡಿದೆ.

ಸ್ವಾಮೀಜಿ ಒತ್ತಿಹೇಳಿದರು: “ಹೌದು, ಅದು ಅಸಾಧ್ಯ. ಆದರೆ ಭಾರತವು ವರ್ಣ ವ್ಯವಸ್ಥೆಯ ನಿಯಮವನ್ನು ಮೀರಿಹೋಗುವವರೆಗೆ ವರ್ಣವ್ಯವಸ್ಥೆ ಉಳಿಯುತ್ತದೆ.”

ನಾನು ಕೇಳಿದೆ: “ನಾಸ್ತಿಕನಾದ ವಿದೇಶೀಯನು – ತಯಾರಿಸಿದ ಆಹಾರಪದಾರ್ಥವನ್ನು ನೀವು ಸ್ವೀಕರಿಸುವುದಿಲ್ಲವೆ?”

“ಭಾರತದಲ್ಲಿ ಅಡುಗೆಯವನು, ಅವನು ಯಾರಿಗಾಗಿ ಅಡುಗೆ ಮಾಡುತ್ತಾನೋ ಅವರಿಗಿಂತ ಉನ್ನತ ಜಾತಿಯವನಾಗಿರಬೇಕು. ಮಾನವನು ಯಾವುದನ್ನು ಸ್ಪರ್ಶಿಸುತ್ತಾನೊ ಅದು ಅವನ ವ್ಯಕ್ತಿತ್ವದ ಗುಣವನ್ನು ಪಡೆಯುತ್ತದೆ ಎಂಬ ಭಾವನೆ ಭಾರತದಲ್ಲಿ ಪ್ರಚಲಿತವಿದೆ. ಯಾವುದರ ಮೂಲಕ ಮನುಷ್ಯನು ತನ್ನ ದೇಹವನ್ನು ನಿರ್ಮಿಸು ತ್ತಾನೊ ಮತ್ತು ಯಾವುದರ ಮೂಲಕ ತನ್ನನ್ನು ವ್ಯಕ್ತಪಡಿಸುತ್ತಾನೊ ಅಂತ ಆಹಾರ ಪದಾರ್ಥಗಳೂ ವ್ಯಕ್ತಿಯ ಗುಣಗಳಿಂದ ಪ್ರಭಾವಿತವಾಗಿರುತ್ತವೆ. ಕೆಲವು ಪದಾರ್ಥಗಳು ನಮ್ಮಲ್ಲಿ ಸದ್ಗುಣಗಳನ್ನು ವರ್ಧಿಸುತ್ತವೆ, ಮತ್ತೆ ಕೆಲವು ನಮ್ಮ ಆಧ್ಯಾತ್ಮಿಕ ಪ್ರಗತಿಯನ್ನು ಕುಂಠಿಸುತ್ತವೆ ಎಂಬ ನಂಬಿಕೆ ಇದೆ. ಉದಾಹರಣೆಗೆ ನಾವು ತಿನ್ನುವುದಕ್ಕಾಗಿ ಕೊಲ್ಲುವುದಿಲ್ಲ. ಅಂತಹ ಆಹಾರವು ನಮ್ಮಲ್ಲಿ ಆಧ್ಯಾತ್ಮಿಕ ಸ್ವಭಾವಕ್ಕೆ ಬದಲಾಗಿ ಪ್ರಾಣಿ ಸ್ವಭಾವವನ್ನು ವೃದ್ಧಿಸುತ್ತವೆ. ಇಷ್ಟೇ ಅಲ್ಲದೆ ಕಟುಕನ ಮೇಲೆ ಕೊಂದ ಪಾಪವನ್ನು ಹೊರಿಸಿದಂತಾಗುತ್ತದೆ.”

ನಾನು ಇದ್ದಕ್ಕಿದ್ದಂತೆಯೇ “ಊಹ್” ಎಂದು ಉದ್ಗರಿಸಿದೆ. ಚಿಕ್ಕ ಕುರಿಗಳ ವಿಕೃತ ರೂಪಗಳು, ಚಿಕ್ಕ ಕೋಳಿಯ ಭೂತಗಳು, ಹಸುವಿನ ಭೂತಗಳು (ನನಗಂತೂ ಹಸುಗಳನ್ನು ಕಂಡರೆ ಭಯ) – ನನ್ನ ಮುಂದೆ ಕಾಣಿಸಿಕೊಂಡವು.

ಸ್ವಾಮೀಜಿ ಹೇಳಿದರು: “ಒಂದು ಕೀಟದಿಂದ ಹಿಡಿದು, ಶ್ರೇಷ್ಠ ಯೋಗಿಯವರೆಗೆ ಇಡೀ ವಿಶ್ವ ಒಂದು. ಇದೆಲ್ಲ ಒಂದೇ. ನಾವೆಲ್ಲ ಒಂದೇ. ಮತ್ತು ನೀವು ಮತ್ತು ನಾನು ಎಲ್ಲ ಒಂದೇ.” ಅಲ್ಲಿದ್ದ ಪಾಶ್ಚಾತ್ಯ ಶ್ರೋತೃಗಳು ಇದನ್ನು ಕೇಳಿ ಮುಗುಳ್ನಕ್ಕರು. ಇದರ ಕಡೆಗೆ ಗಮನ ಹರಿಸದೆ ಆ ಸಂನ್ಯಾಸಿಗಳು ಏಕತೆಯನ್ನೂ ಇತರ ಜೀವಿಗಳನ್ನು ಕೊಲ್ಲುವುದರಿಂದ ಉಂಟಾಗುವ ಪಾಪವನ್ನೂ ಪ್ರತಿಪಾದಿಸುವ ಸಂಸ್ಕೃತ ಶ್ಲೋಕಗಳನ್ನು ಪಠಸಿದರು.

..... ಅವರು ನಮ್ಮ ಮಾತಿನ ಸಂದರ್ಭದಲ್ಲಿ ಅತ್ತಿಂದಿತ್ತ ಓಡಾಡುತ್ತಿದ್ದರು, ಮಧ್ಯೆ ಮಧ್ಯೆ ಅವರು \enginline{Register} ಮುಂದೆ ನಿಂತುಕೊಳ್ಳುತ್ತಿದ್ದರು – ಏಕೆಂದರೆ, ಉಷ್ಣಪ್ರದೇಶದಿಂದ ಬಂದಿದ್ದ ಅವರಿಗೆ ಆ ಬೆಳಗಿನ ಜಾವದಲ್ಲಿ ಚಳಿಯಾಗುತ್ತಿತ್ತು – ಮತ್ತು ತಮಗೆ ಬೇಕಾದುದನ್ನು ಘನತೆಯಿಂದ, ಮತ್ತು ಮುಕ್ತವಾಗಿ ಮಾಡುತ್ತಿದ್ದರು– ಆಗಾಗ ಧೂಮಪಾನವನ್ನು ಕೂಡ.

ನಾನು ಈ ರೀತಿ ಹೇಳುವ ಸಾಹಸ ಮಾಡಿದೆ: “ನೀವೇ ಎಲ್ಲ ಆಸೆಗಳನ್ನು ಇನ್ನೂ ಗೆದ್ದಿಲ್ಲ!” ಸ್ವಾಮೀಜಿಯವರ ಋಜತ್ವ ಎಲ್ಲರಿಗೂ ಹರಡುವಂತಹುದು.

ಅವರು ಮಗುವಿನಂತೆ ಮೋಹಕವಾಗಿ ನಕ್ಕು ಹೇಳಿದರು: “ಇಲ್ಲ, ನಾನು ಹಾಗೆ ಕಾಣಿಸುವೆನೆ?” ಹಶೀಷ್ ಮತ್ತು ಕನಸಿನ ದೇಶದಿಂದ ಬಂದಿದ್ದ ಅವರು ನನ್ನ ಪ್ರಶ್ನೆಯನ್ನು ಧೂಮಪಾನದ ವಿಷಯಕ್ಕೆ ಸಂಬಂಧಿಸಲಿಲ್ಲ.

“ಹಿಂದೂ ಸಂನ್ಯಾಸಿಗಳು ಮದುವೆಯಾಗುವ ವಾಡಿಕೆಯಿದಯೆ?” ಎಂದು ನಾನು ಕೇಳಿದೆ.

“ಇದು ಅವರವರ ಆಯ್ಕೆಗೆ ಬಿಟ್ಟಿದ್ದು. ಮಡದಿ ಮಕ್ಕಳಿಗೆ ಗುಲಾಮನಾಗ ದಿರಲು ಅಥವಾ ಮಹಿಳೆಯೊಬ್ಬಳು ತನಗೆ ಗುಲಾಮಳಾಗದಿರಲು ಅವನು ವಿವಾಹವಾಗುವುದಿಲ್ಲ”.

“ಹಾಗಾದರೆ ಜನಸಂಖ್ಯೆಯು ವೃದ್ಧಿಯಾಗುವುದು ಹೇಗೆ?” ಎಂದು ಮಾಲ್ಥೂಸಿ ಯನ್ ಸಿದ್ಧಾಂತದ ವಿರೋಧಿಯಾದ ನಾನು ಕೇಳಿದೆ.

ಅವರ ವಿಶಾಲ ಕಣ್ಣುಗಳಿಂದ ಭರ್ತ್ಸನವು ಹೊರಚಿಮ್ಮಿತು. ಅವರು ಹೇಳಿದರು: “ನೀವು ಹುಟ್ಟಿದುದಕ್ಕಾಗಿ ತುಂಬ ಸಂತೋಷಪಡುತ್ತೀರೇನು? ಈ ಕಲಹಾತ್ಮಕವಾದ, ಹಸಿವು ಬಾಯಾರಿಕೆಗಳಿಂದ ತುಂಬಿರುವ ಮತ್ತು ಅಜ್ಞಾನಮೂಲವಾದ ಪ್ರಪಂಚ ಕ್ಕಿಂತ ಉನ್ನತವಾದುದೇನನ್ನೂ ಚಿಂತಿಸಲಾರಿರಾ? ಈಗಿನ ಈ ಅಲ್ಪ ಹೊರಟು ಹೋಗಬಹುದು. ಆದರೆ ‘ನೀನು’ ಹೊರಟು ಹೋಗುವೆನೆಂದು ಭಯಪಡಬೇಡಿ. ಈ ಅಲ್ಪ ಪ್ರಜ್ಞೆ ಮತ್ತು ದುಃಖಮಯ ಪ್ರಜ್ಞೆ ಹೋದರೆ ತಾನೆ ಏನಂತೆ?

“ಮಗುವು ಅಳುತ್ತ ಈ ಪ್ರಪಂಚಕ್ಕೆ ಬರುತ್ತದೆ. ಆಗಲಿ, ಅದು ಅಳಲಿ, ಆದರೆ ಅದು ಈ ಪ್ರಪಂಚವನ್ನು ಬಿಟ್ಟುಹೋಗುವಾಗ ನಾವೇಕೆ ಅಳುವುದು?”

ಈಗ ಅವರ ಮುಖದಲ್ಲಿ ಪ್ರಕಾಶಮಯ ಮುಗುಳ್ನಗೆ ಮತ್ತೆ ಮೂಡಿಬಂತು. ಅವರು ಹೇಳಿದರು: “ಸಾವನ್ನು ಕುರಿತ ಪಾಶ್ಚಾತ್ಯ ಮತ್ತು ಪೌರ್ವಾತ್ಯರ ನಡುವೆ ಇರುವ ಅಭಿಪ್ರಾಯ ಭೇದವನ್ನು ಕುರಿತು ನೀವು ಯೋಚಿಸಿದ್ದೀರಾ? ನಾವು ಸತ್ತವನ ಬಗ್ಗೆ ‘ಅವನು ದೇಹವನ್ನು ತ್ಯಜಿಸಿದ’ ಎನ್ನುತ್ತೇವೆ; ನೀವು ‘ಅವನು ಚೈತನ್ಯವನ್ನು ತ್ಯಜಿಸಿದ’ ಎನ್ನುತೀರಿ. ಅದು ಹೇಗೆ? ಮೃತ ದೇಹವು ಚೈತನ್ಯವು ಹೊರಟುಹೋಗುವಂತೆ ಮಾಡುತ್ತದೆಯೆ? ಎಂಥ ವಿಕೃತ ಭಾವನೆಯಿದು?”

ಜನಸಂಖ್ಯೆ ವೃದ್ಧಿಯನ್ನು ಒಪ್ಪುವವನು ಪುನಃವಾದಿಸಿದ: “ಆದರೆ, ಸ್ವಾಮೀಜಿ, ಜೀವಂತ ಸಿಂಹಕ್ಕಿಂತ ಮೃತಪ್ರಾಯರಾಗಿರುವುದೇ ಒಳ್ಳೆಯದು ಎಂಬುದು ನಿಮ್ಮ ಅಭಿಪ್ರಾಯವೆ?”

“ಸ್ವಾಹಾ, ಸ್ವಾಹಾ, ಹಾಗೆಯೇ ಆಗಲಿ” ಎಂದು ಸ್ವಾಮೀಜಿ ಕೂಗಿದರು.

“ಇಂಥ ತತ್ತ್ವವನ್ನು ಆಧರಿಸಿ ಜನರು ಬದುಕುವುದಕ್ಕೆ ಹೇಗೆ ಒಪ್ಪುತ್ತಾರೆ?”

“ಏಕೆಂದರೆ, ಮನುಷ್ಯ ಜೀವನವು ಎಲ್ಲ ಜೀವನದಂತೆಯೇ ಪವಿತ್ರವಾದುದು. ಈ ಜೀವನಾಧ್ಯಾಯವನ್ನು ಕಲಿಯದೆ ಹಿಂತಿರುಗಲು ಅವನು ಬಯಸುವುದಿಲ್ಲ” – ಎಂದು ಸ್ವಾಮೀಜಿ ಹೇಳಿದರು.

“ಶಕ್ತಿಯನ್ನು ವೃದ್ಧಿಗೊಳಿಸಿ ಸಮಯವನ್ನು ಹ್ರಸ್ವಗೊಳಿಸಬಹುದು, ಕಲಿಯುವ ಕಾಲವನ್ನು ಕಡಿಮೆ ಮಾಡಬಹುದು. ಪ್ರಕೃತಿಗೆ ಅಮೃತ ಶಿಲೆಯನ್ನು ರೂಪಿಸಲು ಸಾವಿ ರಾರು ವರ್ಷಗಳು ಬೇಕಾದರೆ ವಿದ್ವಾಂಸನು ಹನ್ನೆರಡು ವರ್ಷಗಳಲ್ಲಿ ಅದನ್ನು ಮಾಡುತ್ತಾನೆ. ಇದೆಲ್ಲ ಸಮಯದ ಪ್ರಶ್ನೆ ಅಷ್ಟೆ.”

“ಭಾರತವು ಬಹುಕಾಲದಿಂದ ಈ ಬೋಧನೆಯನ್ನು ಹೊಂದಿದ್ದರೂ ಅದು ಇದರಿಂದ ಯಾವ ಪಾಠವನ್ನೂ ಕಲಿಯಲಿಲ್ಲ.”

“ಇಲ್ಲಿ, ದಯೆಯ ವಿಚಾರದಲ್ಲಿ ಬಹುಶಃ ಅದು ಬೇರೆ ಎಲ್ಲ ದೇಶಗಳನ್ನೂ ಮೀರಿಸುತ್ತದೆ.”

“ಭಾರತವನ್ನು ಇಂಗ್ಲೆಂಡ್ ಆಳುತ್ತಿರುವುದರ ಬಗ್ಗೆ ನೀವೇನು ಹೇಳುತ್ತೀರಿ?” ನಾನು ಕೇಳಿದೆ.

“ಆಂಗ್ಲ ಆಳ್ವಿಕೆ ಇಲ್ಲದಿದ್ದರೆ ನಾನು ಈಗ ಇಲ್ಲಿ ಇರುತ್ತಿರಲಿಲ್ಲ. ಆದರೆ ನಿಮ್ಮಲ್ಲಿ ಅತ್ಯಂತ ಕೆಳಮಟ್ಟದ ನೀಗ್ರೋ ಕೂಡ ರಾಜಕೀಯವಾಗಿ ಭಾರತದಲ್ಲಿ ನನ್ನ ಸ್ಥಾನಕ್ಕಿಂತ ಉತ್ತಮ ಸ್ಥಾನದಲ್ಲಿದ್ದಾನೆ. ಬ್ರಾಹ್ಮಣನಾಗಲಿ ಕೂಲಿಯಾಗಲಿ – ನಾವೆಲ್ಲ ‘ನೇಟಿವ್’ಗಳು (ದೇಶೀಯರು). ತಪ್ಪು ಕಲ್ಪನೆ, ದಬ್ಬಾಳಿಕೆಗಳಿದ್ದರೂ ಆಳ್ವಿಕೆ ಚೆನ್ನಾಗಿಯೇ ಇದೆ. ಇಂಗ್ಲೆಂಡ್ ಭಾರತದ ಕರ್ಮ ಎನ್ನಬಹುದು – ಅಂದರೆ ತನ್ನ ಪೂರ್ವ ದೋಷಗಳ, ದೌರ್ಬಲ್ಯಗಳ ಫಲವಾಗಿಯೇ ಭಾರತವು ಆಂಗ್ಲ ಆಡಳಿತದಲ್ಲಿದೆ. ಆದರೆ ಅದರ ಅಂತಃ ಸತ್ವದಿಂದ ಭವ್ಯ ಭವಿಷ್ಯ ಭಾರತವು ಹೊರಮೂಡಿ ಬರುವ ಭರವಸೆಯಿದೆ. ನಾನು ವಿಕ್ಟೋರಿಯಾ ರಾಣಿಯ ವಿಧೇಯ ಪ್ರಜೆ” ಎಂದು ಹೇಳಿ ಸ್ವಾಮೀಜಿಯವರು ಅತಿ ವಿಧೇಯತೆಯಿಂದ ಬಾಗಿ ಕೈಮುಗಿದರು.

“ಅಷ್ಟೊಂದು ಸ್ವಾತಂತ್ರ್ಯವನ್ನು ಪ್ರೀತಿಸುವ ನಿಮಗೆ ಈ ವಿಧೇಯತೆ ತರವೆ?” ಎಂದು ನಾನು ಮೆತ್ತಗೆ ಕೇಳಿದೆ.

ಸ್ವಾಮೀಜಿ ಗಂಭೀರವಾಗಿ ಹೇಳಿದರು: “ಆದರೆ ರಾಣಿಯು ಬಹಳ ಕಾಲದಿಂದ ವಿಧವೆಯಾಗಿರುವಳು. ಭಾರತದಲ್ಲಿ ಇಂತಹ ವಿಧವೆಯರು ತುಂಬು ಗೌರವಕ್ಕೆ ಅರ್ಹರು. ಸ್ವಾತಂತ್ರ್ಯದ ವಿಚಾರವಾಗಿ ಹೇಳಬೇಕಾದರೆ ಎಲ್ಲ ಬೆಳವಣಿಗೆಯ ಗುರಿಯೇ ಸ್ವಾತಂತ್ರ್ಯ – ನಿಯಮ ಮತ್ತು ವ್ಯವಸ್ಥೆಗಳಲ್ಲ. ಸ್ಮಶಾನದಲ್ಲಿ ಎಲ್ಲ ಕಡೆಗಿಂತ ಹೆಚ್ಚು ನಿಯಮ ವ್ಯವಸ್ಥೆಗಳಿರುತ್ತವೆ. ಇದನ್ನು ಪರೀಕ್ಷಿಸಿ ನೋಡಿ.”

ನಾನು ಹೇಳಿದೆ: “ನಾನು ಈಗ ಹೋಗಬೇಕು, ರೈಲು ಹಿಡಿಯಬೇಕಾಗಿದೆ.”

“ಇದೇ ಎಲ್ಲ ಅಮೆರಿಕನರ ಗೀಳು,” ಎಂದು ಸ್ವಾಮೀಜಿ ಮುಗುಳ್ನಕ್ಕರು. ಅವರ ಆ ಶಾಂತ ಮುಖಮುದ್ರೆಯಲ್ಲಿ ಪೂರ್ಣತ್ವದ ಹೊಳಹನ್ನು ಕಂಡೆ. ಅವರಂದರು: “ನೀವು ಯಾವಾಗಲೂ ಈ ಕಾರನ್ನು ಹಿಡಿಯಬೇಕಾಗಿರುತ್ತದೆ ಅಥವಾ ಆ ರೈಲನ್ನು ಹಿಡಿಯಬೇಕಾಗಿರುತ್ತದೆ. ಮುಂದೆ ಬೇರೆ ವಾಹನ ಸಿಕ್ಕುವುದಿಲ್ಲವೆ?”

ಆದರೆ ನಾನು ಈ ಪ್ರಾಚ್ಯ ಸಂಜಾತರಿಗೆ ಪಾಶ್ಚಾತ್ಯ ಸಮಯಪ್ರಜ್ಞೆಯ ಮೌಲ್ಯವನ್ನು ವಿವರಿಸುವ ಪ್ರಯತ್ನ ಮಾಡಲಿಲ್ಲ. ಏಕೆಂದರೆ ಅದರಿಂದ ಏನೂ ಪ್ರಯೋಜನವಿಲ್ಲವೆಂದು ನನಗನಿಸಿತು \enginline{(see p. ೩೨೭).} ಸಮಯಾಭಾವವನ್ನೇ ಭಾವಿಸದ ಸ್ಥಳದಲ್ಲಿ ವಾಸಿಸುವುದು ಎಷ್ಟೊಂದು ಆನಂದದಾಯಕವಾಗಿರಬೇಕು! ಪ್ರಾಚ್ಯ ದೇಶದಲ್ಲಿ, ಸ್ವಾಮೀಜಿ ಹೇಳುವಂತೆ, ಜನರಿಗೆ ಉಸಿರಾಡುವುದಕ್ಕೆ ಸಮಯವಿದೆ, ಚಿಂತಿಸುವುದಕ್ಕೆ ಸಮಯವಿದೆ, ಬಾಳುವುದಕ್ಕೆ ಸಮಯವಿದೆ. ಇದಕ್ಕೆ ಬದಲಾಗಿ ನಮ್ಮಲ್ಲೇನಿದೆ? ನಾವು ಕಾಲದಲ್ಲಿ ಬಾಳುತ್ತೇವೆ, ಅವರು ಅನಂತತೆಯಲ್ಲಿ – ಬಾಳುತ್ತಾರೆ.

\begin{center}
\textbf{“ನಮ್ಮ ಪರಿಸರದಿಂದ ನಾವು ಮೋಹಿತರಾಗಿದ್ದೇವೆ”}
\end{center}

(\enginline{"San Francisco Examiner"} ಪತ್ರಿಕೆಯ ೧೮–೩–೧೯೦೦ನೆ ಸಂಚಿಕೆಯಲ್ಲಿ ಪ್ರಕಟವಾದ ಸಂದರ್ಶನ.)

ಹಿಂದೂ ದಾರ್ಶನಿಕರು (ಸ್ವಾಮಿ ವಿವೇಕಾನಂದರು) ಕೆಲವು ಪಾಶ್ಚಾತ್ಯ ದೋಷಗಳ ಮೂಲಕ್ಕೇ ಕುಠಾರ ಪ್ರಹಾರವನ್ನು ನೀಡುತ್ತಾ ನಾವು ಹೇಗೆ ದೇವರನ್ನು ಪೂಜಿಸಬೇಕು, ವ್ಯರ್ಥ ಪ್ರಾರ್ಥನೆ ಮಾಡುವುದಲ್ಲ ಎಂಬುದನ್ನು ತಿಳಿಸಿಕೊಡುತ್ತಾರೆ.

ಸ್ವಾಮೀಜಿಯವರು ಅಮೆರಿಕನರ ಸ್ನೇಹಿತರಂತಿದ್ದರು. ಅವರಲ್ಲಿ ಸಂದರ್ಶನಕ್ಕೆ ಯೋಗ್ಯವಾದ ಮೋಹಕ ವ್ಯಕ್ತಿತ್ವವಿತ್ತು.

ಅವರು ತಾವಿದ್ದ ಕೊಠಡಿಯಲ್ಲಿ ಅತ್ತಿಂದಿತ್ತ ಓಡಾಡುತ್ತ ಅಲ್ಲಿ ನೆರೆದಿದ್ದ ಸಂದರ್ಶಕರು ಮತ್ತು ಸ್ನೇಹಿತರಿಗೆ ಎರಡು ಗಂಟೆಗಳವರೆಗೆ ಮನರಂಜನೆಯನ್ನು ನೀಡಿದರು.

“ಭಾರತದಲ್ಲಿರುವ ಇಂಗ್ಲಿಷರ ಬಗ್ಗೆ ಹೇಳಲೆ? ಆದರೆ ರಾಜಕೀಯದ ಬಗ್ಗೆ ಮಾತ ನಾಡಲು ನನಗೆ ಇಷ್ಟವಿಲ್ಲ. ಆದರೆ ಉನ್ನತ ದೃಷ್ಟಿಯಿಂದ ನೋಡಿದರೆ, ಅವರು ಇಲ್ಲದಿದ್ದಿದ್ದರೆ ಈಗ ನಾನು ಇಲ್ಲಿರುತ್ತಿರಲಿಲ್ಲ. ದೇಶೀಯರಾದ ನಾವು, ಇಂಗ್ಲೀಷ್ ರಕ್ತ ಮತ್ತು ಭಾವನೆಗಳ ಪರಸ್ಪರ ಮಿಶ್ರಣದಿಂದ ಭಾರತದ ಉದ್ಧಾರವಾಗುತ್ತದೆ ಎಂದು ನಂಬುತ್ತೇವೆ. ಐವತ್ತು ವರ್ಷಗಳ ಹಿಂದೆ ನಮ್ಮ ಜನಾಂಗದ ಸಾಹಿತ್ಯ ಮತ್ತು ಧರ್ಮಗಳು ಸಂಸ್ಕೃತ ಭಾಷೆಯಲ್ಲಿ ಬೀಗಮುದ್ರಿತವಾಗಿತ್ತು. ಇಂದು ನಾಟಕಗಳು ಮತ್ತು ಕಾದಂಬರಿಗಳು ದೇಶೀಯ ಭಾಷೆಯಲ್ಲಿ ಬರೆಯಲಾಗುತ್ತಿವೆ ಮತ್ತು ಧಾರ್ಮಿಕ ಸಾಹಿತ್ಯವು ಅನುವಾದವಾಗುತ್ತಿವೆ. ಇದು ಇಂಗ್ಲೀಷರ ಕಾರ್ಯ. ಅಮೆರಿಕದಲ್ಲಿ ಜನಸಾಮಾನ್ಯರ ಶಿಕ್ಷಣದ ಬಗ್ಗೆ ಮಾತನಾಡುವುದು ಅನವಶ್ಯಕ.”

“ಇಂಗ್ಲೆಂಡಿನ ಬೊಯರ್ಸ್ ಯುದ್ಧದ ಬಗ್ಗೆ ನಿಮ್ಮ ಅಭಿಪ್ರಾಯವೇನು?”

ಸ್ವಾಮೀಜಿಯವರ ಉತ್ತರ: “ಓಹ್, ಬೆಳಗಿನ ಪತ್ರಿಕೆಯನ್ನು ಓದಿರುವಿರೇನು? ಆದರೆ ನನಗೆ ರಾಜಕೀಯದ ಬಗ್ಗೆ ಮಾತನಾಡಲು ಇಷ್ಟವಿಲ್ಲ. ಇಂಗ್ಲಿಷ್ ಮತ್ತು ಬೋಯರ್ಸ್ ಇಬ್ಬರದೂ ತಪ್ಪಿದೆ. ಈ ರಕ್ತಪಾತ ಭಯಂಕರ! ಇಂಗ್ಲಿಷರು ಗೆಲ್ಲುತ್ತಾರೆ, ಆದರೆ ಅನಾಹುತ ಭಯಂಕರ. ಇಂಗ್ಲೆಂಡಿನ ಭಾಗ್ಯ ಒಳ್ಳೆಯದಿರುವಂತೆ ಕಾಣುತ್ತದೆ.”

ರಾಜಕೀಯವನ್ನು ಕುರಿತು ಮಾತನಾಡಲು ತಮಗಿಷ್ಟವಿಲ್ಲವೆಂಬುದನ್ನು ಸೂಚಿಸಲು ಸ್ವಾಮೀಜಿಯವರು ಸಂಸ್ಕೃತ ಶ್ಲೋಕಗಳನ್ನು ಪಠಿಸತೊಡಗಿದರು.

ಅನಂತರ ಅವರು ರಷ್ಯಾದ ಪುರಾತನ ಇತಿಹಾಸವನ್ನು ಕುರಿತೂ, ಅಲೆಮಾರಿ ಜನಾಂಗವಾದ ಟಾರ್ಟರರ ಬಗ್ಗೆಯೂ, ಸ್ಪೈನ್ನಲ್ಲಿ ಮೂರರ ಆಳ್ವಿಕೆಯನ್ನು ಕುರಿತೂ ಬಹಳ ಹೊತ್ತು ಮಾತನಾಡತೊಡಗಿದರು. ಇದರ ಮೂಲಕ ತಮ್ಮ ಅತ್ಯಾಶ್ಚರ್ಯಕರ ಜ್ಞಾಪಕ. ಶಕ್ತಿಯನ್ನೂ ಸಂಶೋಧನೆಯನ್ನೂ ತೋರ್ಪಡಿಸಿದರು. ತಮ್ಮ ಸಂಪರ್ಕಕ್ಕೆ ಬರುವ ಪ್ರತಿಯೊಂದರಲ್ಲಿಯೂ ಅವರು ತೋರಿಸುತ್ತಿದ್ದ ಶಿಶುಸಹಜ ಆಸಕ್ತಿಯು ನಿಜವಾಗಿಯೂ ಅವರು ಹೊಂದಿದ್ದ ಕುತೂಹಲಯುಕ್ತವೂ ಸಾರ್ವತ್ರಿಕವೂ ಆದ ಜ್ಞಾನದ ಪರಿಣಾಮವೆಂದು ಹೇಳಬಹುದು.

\begin{center}
\textbf{ವಿವಾಹ\supskpt{\footnote{\enginline{1. New Discoveries, Vol.5, p.138}}}}
\end{center}

ಜೋಸಫಿನ್ ಮೆಕ್ಲೋಡ್ ಮಿಸ್ ಮೇರಿ ಹೇಲ್ಳಿಗೆ ೧೯೦೮ರಲ್ಲಿ ಬರೆದ ಪತ್ರದಲ್ಲಿ ಸ್ವಾಮೀಜಿಯವರು ಆಲ್ಬೆರ್ಟಾ ಸ್ಟರ್ಜೀಸ್ಳ ಈ ಕೆಳಗಿನ ಪ್ರಶ್ನೆಗೆ ನೀಡಿದ್ದ ಉತ್ತರವನ್ನು ಪ್ರಸ್ತಾಪಿಸುತ್ತಾಳೆ;

ವಿವಾಹ ಜೀವನದಲ್ಲಿ ಸುಖವಿಲ್ಲವೇನು?

ಸ್ವಾಮೀಜಿ: ಇದೆ, ಆದರೆ ವೈವಾಹಿಕ ಜೀವನದೊಂದಿಗೆ ತಪಸ್ಸು ಸೇರಿರಬೇಕು; ಎಲ್ಲವನ್ನೂ, ತತ್ತ್ವವನ್ನು ಕೂಡ ತ್ಯಜಿಸಬೇಕು. \enginline{(p.} ೩೨೮)

\begin{center}
\textbf{ಸೀಮಾರೇಖೆ\supskpt{\footnote{\enginline{1. New Discoveries Vol. 5.p. 225}}}}
\end{center}

ಅಲೈಸ್ ಹ್ಯಾಂಸ್ ಬ್ರೋಳ ನೆನಪಿನ ಕತೆಯಲ್ಲಿ ಬರುವ ಪ್ರಶ್ನೋತ್ತರ ಮಾಲೆ.

ಸ್ವಾಮೀಜಿ, ಎಲ್ಲವೂ ಒಂದೇ ಆಗಿದ್ದರೆ, ಕೋಸುಗೆಡ್ಡೆಗೂ ಮನುಷ್ಯನಿಗೂ ಇರುವ ವ್ಯತ್ಯಾಸವೇನು?

ನಿಮ್ಮ ಕಾಲನ್ನು ಚಾಕುವಿನಿಂದ ಕೊಯ್ದುಕೊಳ್ಳಿ. ಆಗ ನಿಮಗೆ ವ್ಯತ್ಯಾಸ ಗೊತ್ತಾಗುತ್ತದೆ.

\begin{center}
\textbf{ದೇವರಿದ್ದಾನೆ!\supskpt{\footnote{2. ಅಲ್ಲೇ, ಪು. \enginline{276}}}}
\end{center}

ಉಪನ್ಯಾಸದನಂತರ ನಡೆದ ಪ್ರಶ್ನೋತ್ತರಗಳ ವರದಿ – ಅರ್ಲೆಸ್ ಹ್ಯಾಂಸ್ ಬ್ರೋಳಿಂದ.

ಸ್ವಾಮೀಜಿ, ಎಲ್ಲವೂ ಒಳಿತು ಎಂಬುದು ನಿಮ್ಮ ಅಭಿಪ್ರಾಯವೆ?

ಉತ್ತರ: ಖಂಡಿತ ಅಲ್ಲ. ನನ್ನ ಅಭಿಪ್ರಾಯವೇನೆಂದರೆ ಯಾವುದೂ ಇಲ್ಲ, ದೇವರೊಬ್ಬನೇ ಇರುವುದು. ಇಷ್ಟೇ ವ್ಯತ್ಯಾಸ.

\begin{center}
\textbf{ತ್ಯಾಗ\supskpt{\footnote{\enginline{New Discoveries, Vol. 6. p. 11–12}}}}
\end{center}

ವಿದ್ಯಾರ್ಥಿನಿ: ಎಲ್ಲರೂ ತ್ಯಾಗಮಾಡಿದರೆ ಪ್ರಪಂಚದ ಗತಿಯೇನು?

ಸ್ವಾಮೀಜಿ: ಆ ಸುಳ್ಳಿನೊಡನೆ ನನ್ನೆಡೆಗೆ ಏಕೆ ಬರುತ್ತಿರುವಿರಿ? ಈ ಪ್ರಪಂಚದಲ್ಲಿ ನಿಮ್ಮ ಸುಖವನ್ನಲ್ಲದೆ ಮತ್ತೇನನ್ನೂ ನೀವು ಭಾವಿಸಲೇ ಇಲ್ಲ.

\begin{center}
\textbf{ಶ‍್ರೀ ರಾಮಕೃಷ್ಣರ ಶಿಷ್ಯರು\supskpt{\footnote{4. ಅಲ್ಲೇ, ಪು. \enginline{12}}}}
\end{center}

ಮಿಸೆಸ್ ಎಡಿತ್ ಅಲೆನ್, ಸ್ಯಾನ್ ಪ್ರಾಂಸಿಸ್ಕೋದಲ್ಲಿ ವಿವೇಕಾನಂದರ ಉಪನ್ಯಾಸದ ನಂತರ ನಡೆದ ಪ್ರಶ್ನೋತ್ತರವನ್ನು ವರದಿ ಮಾಡುತ್ತಾರೆ.

ಸ್ವಾಮೀಜಿ: ತಮ್ಮ ಹೆಸರನ್ನು ಕೂಡ ಬರೆಯಲು ಬರದ ವ್ಯಕ್ತಿಯೊಬ್ಬರ ಶಿಷ್ಯನಾನು. ಅವರ ಪಾದರಕ್ಷೆಯನ್ನು ಸ್ಪರ್ಶಿಸುವುದಕ್ಕೂ ನಾನು ಅರ್ಹನಲ್ಲ. ನನ್ನ ಬುದ್ಧಿ ಶಕ್ತಿಯನ್ನು ತೆಗೆದು ಗಂಗೆಗೆ ಎಸೆಯ ಬೇಕೆಂದು ಎಷ್ಟೊಂದು ಬಾರಿ ಯೋಚಿಸಿರುವೆನು!

ವಿದ್ಯಾರ್ಥಿನಿ: ಆದರೆ ಸ್ವಾಮೀಜಿ, ನಿಮ್ಮಲ್ಲಿ ಅದನ್ನೇ ನಾನು ಹೆಚ್ಚು ಮೆಚ್ಚುವುದು.

ಸ್ವಾಮೀಜಿ: ಅದಕ್ಕೆ ಕಾರಣ ನೀವು ನನ್ನಂತೆಯೇ ಮೂರ್ಖರಾಗಿರುವುದು.

\begin{center}
\textbf{ದಿವ್ಯಾವತಾರ ಶ‍್ರೀ ರಾಮಕೃಷ್ಣ\supskpt{\footnote{1. ಅಲ್ಲೇ, ಪು. \enginline{17.}}}}
\end{center}

ಸ್ವಾಮೀಜಿ: ನಾನು ಮತ್ತೊಮ್ಮೆ ಬರಬೇಕಾಗುತ್ತದೆ. ನನ್ನ ಗುರು ಹೇಳಿರುವರು, ಅವರೊಡನೆ ನಾನು ಮತ್ತೊಮ್ಮೆ ಬರಬೇಕಾಗುತ್ತದೆ ಎಂದು.

ಮಿಸೆಸ್ ಅಲೆನ್: ಶ‍್ರೀ ರಾಮಕೃಷ್ಣರು ಹಾಗೆ ಹೇಳುತ್ತಾರೆಂದು ನೀವು ಮತ್ತೆ ಬರ ಬೇಕೇನು?

ಸ್ವಾಮೀಜಿ: ಹೌದಮ್ಮ, ಅಂತಹ ಜೀವಿಗಳಲ್ಲಿ ವಿಶೇಷ ಶಕ್ತಿ ಇರುತ್ತದೆ.

\begin{center}
\textbf{ಖಾಸಗಿ ಪ್ರವೇಶ\supskpt{\footnote{\enginline{2. New Discoveries, Vol. 6. p. 12}}}}
\end{center}

೧೯೦೦ರಲ್ಲಿ ವಿವೇಕಾನಂದರು ಉತ್ತರ ಕ್ಯಾಲಿಫೋರ್ನಿಯದಲ್ಲಿ ಇದ್ದಾಗಿನ ಸ್ಮೃತಿಕಥೆಯಿಂದ. ಇದನ್ನು ಬರೆದವರು ಮಿಸೆಸ್ ಎಡಿತ್ ಅಲೆನ್.

ವಿದ್ಯಾರ್ಥಿನಿ: ನಾನು ಈ ಹಿಂದೆ ಇದ್ದಿದ್ದರೆ ಶ‍್ರೀರಾಮಕೃಷ್ಣರನ್ನು ನೋಡಿರಬಹುದಿತ್ತು!

ಸ್ವಾಮೀಜಿ: (ಆಕೆಯ ಕಡೆ ನಿಧಾನವಾಗಿ ತಿರುಗಿ) ನೀವು ಹಾಗೆ ಹೇಳುತ್ತೀರಾ? ನೀವು ನನ್ನನ್ನು ನೋಡಿರುವಿರಲ್ಲ?

\begin{center}
\textbf{ಶುಭಾಶಯ\supskpt{\footnote{3. ಅಲ್ಲೇ, ಪು. \enginline{136}.}}}
\end{center}

ಮಿ. ಥಾಮಸ್ ಅಲೆನ್ ಬರೆದಿರುವ ವಿವೇಕಾನಂದರ ಅಲಮೇಡಾ (ಕ್ಯಾಲಿಫೋರ್ನಿಯ) ಭೇಟಿಯನ್ನು ಕುರಿತ ಸ್ಮೃತಿಕಥೆಯಿಂದ.

ಮಿ. ಅಲೆನ್: ಸ್ವಾಮೀಜಿ, ನೀವು ಈಗ ಅಲಮೇಡಾದಲ್ಲಿದ್ದೀರಿ.

ಸ್ವಾಮೀಜಿ: ಇಲ್ಲ ಅಲೆನ್, ಅಲಮೇಡಾ ನನ್ನಲ್ಲಿದೆ.

\begin{center}
\textbf{“ಈ ಪ್ರಪಂಚವು ಒಂದು ಸರ್ಕಸ್ಕೊಂಡಿ”\supskpt{\footnote{4. ಅಲ್ಲೇ, ಪು. \enginline{156}}}}
\end{center}

ಕ್ಯಾಂಪ್ ಟೈಲರ್ನಲ್ಲಿ ಮೇ ೧೯೦೦ನಲ್ಲಿ ಮಿಸ್ ಬೆಲ್ ಒಡನೆ ನಡೆದ ಸಂಭಾಷಣೆಯನ್ನು ಕುರಿತ ಹ್ಯಾಂಸ್ಬ್ರೋಳ ಸ್ಮೃತಿಕಥೆಯಿಂದ.

ಮಿಸ್ ಬೆಲ್: ಈ ಪ್ರಪಂಚವು ಹಳೆಯ ಶಾಲೆ, ನಾವಿಲ್ಲಿಗೆ ಪಾಠ ಕಲಿಯಲು ಬರುತ್ತೇವೆ.

ಸ್ವಾಮೀಜಿ: ನಿಮಗೆ ಇದನ್ನು ಯಾರು ಹೇಳಿದರು? (ಮಿಸ್ ಬೆಲ್ಳಿಗೆ ಇದು ನೆನಪಾಗಲಿಲ್ಲ) ನನಗೆ ಹಾಗೆ ಅನಿಸುವುದಿಲ್ಲ. ಈ ಪ್ರಪಂಚವು ಒಂದು ಸರ್ಕಸ್ ರಿಂಗ್, ಇದರಲ್ಲಿ ನಾವೆಲ್ಲ ಕುಣಿಯುವ ಕೋಡಂಗಿಗಳು.

ಮಿಸ್ ಬೆಲ್: ನಾವೇಕೆ ಕುಣಿಯುತ್ತೇವೆ?

ಸ್ವಾಮೀಜಿ: ಏಕೆಂದರೆ, ನಮಗೆ ಕುಣಿಯುವುದು ಇಷ್ಟ. ನಮಗೆ ಸುಸ್ತಾದಾಗ ಶಾಂತರಾಗುತ್ತೇವೆ.

\begin{center}
\textbf{ಕಾಳಿಯನ್ನು ಕುರಿತು\supskpt{\footnote{\enginline{1. New Discoveries Vol1.p.118}}}}
\end{center}

ಕಾಳಿಪೂಜೆಯ ಬಗ್ಗೆ ತಿಳಿಯುತ್ತಿದ್ದಾಗ ಸೋದರಿ ನಿವೇದಿತಾ ಸ್ವಾಮೀಜಿಯವರೊಡನೆ ನಡೆಸಿದ ಸಂಭಾಷಣೆಯಿಂದ.

ನಿವೇದಿತಾ: ಸ್ವಾಮೀಜಿ, ಬಹುಶಃ ಕಾಳಿಯು ಶಿವನಿಗಾದ ದರ್ಶನವಿರಬಹುದು, ಹೌದೆ?

ಸ್ವಾಮೀಜಿ: ಸರಿ, ಸರಿ, ನಿನ್ನದೇ ರೀತಿಯಲ್ಲಿ ಇದನ್ನು ವ್ಯಕ್ತಪಡಿಸು.

\begin{center}
\textbf{ಶ‍್ರೀ ರಾಮಕೃಷ್ಣರ ಕೈಕೆಳಗೆ ತರಬೇತಿ\supskpt{\footnote{\enginline{2 New The Complete Works of Sister Nivedita, Vol. 1. pp. 159–60}}}}
\end{center}

ಸ್ವಾಮಿ ವಿವೇಕಾನಂದರು ಇಂಗ್ಲೆಂಡಿಗೆ ಹಡಗಿನಲ್ಲಿ ಹೋಗುತ್ತಿದ್ದಾಗ ಅಲ್ಲಿಯ ಒಬ್ಬ ಸೇವಕನ ಶಿಶು ಸಹಜ ಸರಳತೆ ಅವರಿಗೆ ಬಹಳ ಮೆಚ್ಚಿಕೆಯಾಯಿತು.

ಸ್ವಾಮೀಜಿ: ನೋಡಿ, ನಾನು ನಮ್ಮ ಮಹಮ್ಮದೀಯರನ್ನು ಪ್ರೀತಿಸುತ್ತೇನೆ.

ನಿವೇದಿತಾ: ಹೌದು, ಪ್ರತಿಯೊಬ್ಬರನ್ನೂ ಅವರ ಅತ್ಯುತ್ತಮ ಗುಣವನ್ನು ನೋಡಿ ಅರ್ಥಮಾಡಿಕೊಳ್ಳುವ ಅಭ್ಯಾಸವನ್ನು ನಾನು ಅರ್ಥಮಾಡಿಕೊಳ್ಳಬೇಕಾಗಿದೆ. ಇದು ಯಾವುದರಿಂದ ಬರುತ್ತದೆ? ನೀವು ಇದನ್ನು ಯಾವುದಾದರೂ ಚಾರಿತ್ರಿಕ ವ್ಯಕ್ತಿಯಲ್ಲಿ ಗುರುತಿಸುತ್ತೀರಾ? ಅಥವಾ ಇದನ್ನು ನೀವು ಶ‍್ರೀರಾಮಕೃಷ್ಣರಿಂದ ಪಡೆದುದೆ?

ವಿವೇಕಾನಂದ: ಇದು ಖಂಡಿತ ಶ‍್ರೀರಾಮಕೃಷ್ಣ ಪರಮಹಂಸರಿಂದ ಪಡೆದ ತರಬೇತಿಯ ಫಲ. ನಾವೆಲ್ಲರೂ ಅವರ ಮಾರ್ಗವನ್ನು ಬಹುಮಟ್ಟಿಗೆ ಅನುಸರಿಸುತ್ತೇವೆ. ಆದರೆ ಅವರಿಗೆ ಅದು ಎಷ್ಟು ಸುಲಭವಾಗಿತ್ತೊ ಅಷ್ಟರಮಟ್ಟಿಗೆ ನಮಗಾಗಿಲ್ಲ. ತಾವು ಅರ್ಥಮಾಡಿಕೊಳ್ಳಬೇಕೆಂದು ಬಯಸಿದ ಧರ್ಮದವರಂತೆಯೇ ಉಣ್ಣು ತ್ತಿದ್ದರು, ಅವರಂತೆಯೇ ಬಟ್ಟೆ ಹಾಕಿಕೊಳ್ಳುತ್ತಿದ್ದರು; ಅವರಿಂದ ದೀಕ್ಷೆ ಪಡೆದು ಅವರ ಭಾಷೆ ಯನ್ನೇ ಬಳಸುತ್ತಿದ್ದರು. ಅವರು ಹೇಳುತ್ತಿದ್ದರು: ‘ನಾವು ಕಲಿಯಬೇಕಾದರೆ ನಾವು ಇತರರ ಹೃದಯವನ್ನು ಪ್ರವೇಶಿಸಬೇಕು’. ಈ ವಿಧಾನ ಅವರದೇ ಆಗಿತ್ತು. ಈ ಹಿಂದೆ ಭಾರತದಲ್ಲಿ ಯಾವ ಒಬ್ಬ ವ್ಯಕ್ತಿಯೂ ಕ್ರೈಸ್ತನೂ, ಮಹಮ್ಮದನೂ, ವೈಷ್ಣವನೂ ಆಗಿರಲಿಲ್ಲ.

