
\addtocontents{toc}{\protect\vspace{-0.7cm}}

\part{ಸ್ವಾಮಿ ವಿವೇಕಾನಂದರ ಮಾತುಗಳ ಪತ್ರಿಕಾವರದಿಗಳು}
%~ \chapter*{ಸ್ವಾಮಿ ವಿವೇಕಾನಂದರ ಮಾತುಗಳ ಪತ್ರಿಕಾವರದಿಗಳು}

%~ \begin{center}
%~ (ಅಮೆರಿಕನ್, ಯೂರೋಪಿಯನ್ ಹಾಗೂ ಭಾರತೀಯ ಪತ್ರಿಕೆಗಳಲ್ಲಿ)
%~ \end{center}

%~ ಈ ಪತ್ರಿಕಾವರದಿಗಳ ಚಾರಿತ್ರಿಕ ಅಧಿಕೃತತೆಯನ್ನು ಉಳಿಸಿಕೊಳ್ಳುವುದಕ್ಕಾಗಿ, ಅವುಗಳ ವ್ಯಾಕರಣ, ಕಾಗುಣಿತ, ಪಂಕ್ಚ್ಯುಯೇಶನ್ಗಳನ್ನು ಹಾಗೆಯೇ ಉಳಿಸಿಕೊಳ್ಳಲಾಗಿದೆ. ಸ್ಪಷ್ಟತೆಗಾಗಿ ಸ್ವಾಮಿ ವಿವೇಕಾನಂದರ ಮಾತುಗಳನ್ನು ಉದ್ಧರಣ ಚಿಹ್ನೆಗಳಲ್ಲಿಡಲಾಗಿದೆ. ಪ್ರಕಾಶಕರು ಕೊಟ್ಟಿರುವ ಶೀರ್ಷಿಕೆಗಳನ್ನು ನಕ್ಷತ್ರ ಗುರುತಿನಿಂದ ಸೂಚಿಸಿದೆ. ಸಾಧ್ಯ ವಿದ್ದೆಡೆಯಲ್ಲೆಲ್ಲ ಅವುಗಳ ನವೀನ ಮಾದರಿಗಳನ್ನು ಬಿಟ್ಟು ಮೂಲ ವರ್ತಮಾನ ಪತ್ರಿಕೆಗಳ ಅಕ್ಷರಶೈಲಿಯನ್ನೇ ಆರಿಸಿಕೊಳ್ಳಲಾಗಿದೆ.

%~ \begin{flushright}
%~ ಪ್ರಕಾಶಕರು
%~ \end{flushright}




\chapter{ಅಮೆರಿಕಾದ ಪತ್ರಿಕಾವರದಿಗಳು}

\begin{center}
\textbf{ಸ್ವಾಗತಕ್ಕೆ ಉತ್ತರ}\footnote{\enginline{1. New Discoveries, Vol. 1. pp. 83-84}}
\end{center}

(೧೮೯೩ ಸೆಪ್ಟೆಂಬರ್ ೧೧ರ ಸಮೀಪದ ದಿನಾಂಕಗಳ ಹೆರಾಲ್ಡ್, ಇಂಟರ್ ಓಷನ್, ಟ್ರಿಬ್ಯೂನ್ ಮತ್ತು ರೆಕಾರ್ಡ್ ಎಂಬ ನಾಲ್ಕು ಪತ್ರಿಕೆಗಳ ವರದಿಗಳನ್ನಾಧರಿಸಿ ಸಂಪಾದಕರು ಮಾಡಿದ ಸಂಯೋಜನೆ)\footnote{೨.ಸ್ವಲ್ಪ ಮಟ್ಟಿಗೆ ಭಿನ್ನವಾದ ವರದಿಗೆ ನೋಡಿ: \enginline{‘Response to Welcome ‘, complete Works, I:3-4}}

\begin{center}
(ಅಮೆರಿಕಾದ ಸಹೋದರಿಯರೆ ಮತ್ತು ಸಹೋದರರೆ)
\end{center}

ನಮಗೆ ನೀವು ನೀಡಿರುವ ಭವ್ಯವಾದ ಸ್ವಾಗತಕ್ಕೆ ಉತ್ತರಿಸಲು ಎದ್ದು ನಿಲ್ಲು ತ್ತಿರುವ ನನ್ನ ಹೃದಯವು ಅವರ್ಣನೀಯ ಆನಂದದಿಂದ ತುಂಬಿಹೋಗಿದೆ. ಈ ಪ್ರಪಂಚ ಇದುವರೆಗೆ ಕಂಡಿರುವ ಅತ್ಯಂತ ಪ್ರಾಚೀನ ಸಂನ್ಯಾಸೀ ಸಂಘದ ಹೆಸರಿನಲ್ಲಿ ನಿಮ್ಮನ್ನು ನಾನು ವಂದಿಸುತ್ತೇನೆ. ಗೌತಮನು ಅದರ ಒಬ್ಬ ಸದಸ್ಯ ಮಾತ್ರವೇ ಆಗಿದ್ದ. ಬೌದ್ಧಧರ್ಮ ಜೈನ ಧರ್ಮಗಳು ಯಾವ ಧರ್ಮದ ಕೇವಲ ಶಾಖೆಗಳಾಗಿರುವುವೋ ಆ ಮಾತೃಧರ್ಮದ ಹೆಸರಿನಲ್ಲಿ ನಿಮ್ಮನ್ನು ನಾನು ವಂದಿಸುತ್ತೇನೆ. ಕೊನೆಯದಾಗಿ ಎಲ್ಲಾ ಜಾತಿ ಪಂಥಗಳ ಕೋಟ್ಯನುಕೋಟಿ ಹಿಂದೂ ಜನರ ಹೆಸರಿನಲ್ಲಿ ನಿಮ್ಮನ್ನು ವಂದಿಸುತ್ತೇನೆ. ದೂರದೇಶಗಳಿಂದ ಬಂದಿರುವ ಈ ವಿವಿಧ ಜನರು ಬೇರೆ ಬೇರೆ ನಾಡುಗಳಿಗೆ ತಂದಿರುವ ಸಹನೆಯ ಭಾವನೆಯನ್ನು ನೀವಿಲ್ಲಿ ನೋಡಬಹುದೆಂದು ಹೇಳಿರುವ ವೇದಿಕೆಯ ಮೇಲಿನ ಕೆಲವು ಭಾಷಣಕಾರರಿಗೂ ಸಹ ನನ್ನ ವಂದನೆಗಳು ಸಲ್ಲುತ್ತವೆ. ಈ ಭಾವನೆಗಾಗಿ ನಾನು ಅವರನ್ನು ವಂದಿಸುತ್ತೇನೆ.

ಸಾರ್ವತ್ರಿಕ ಸ್ವೀಕಾರ ಮತ್ತು ಸಹನೆ ಎರಡನ್ನೂ ಲೋಕಕ್ಕೆ ಬೋಧಿಸಿದ ಒಂದು ಧರ್ಮಕ್ಕೆ ಸೇರಿದವನು ನಾನೆಂದು ನನಗೆ ಹೆಮ್ಮೆಯಾಗುತ್ತಿದೆ. ಸರ್ವಧರ್ಮ ಸಹಿಷ್ಣುತೆಯಲ್ಲಿ ನಂಬಿಕೆ ಇರುವುದು ಮಾತ್ರವಲ್ಲ, ಎಲ್ಲ ಧರ್ಮಗಳೂ ಸತ್ಯವೆಂದು ನಾವು ಒಪ್ಪುತ್ತೇವೆ. ಯಾವ ಧರ್ಮದ ಪವಿತ್ರ ಭಾಷೆಯಾದ ಸಂಸ್ಕೃತದಲ್ಲಿ ಹೊರಗಿಡುವುದು ಎಂಬ ಪದಕ್ಕೆ ಸರಿಸಾಟಿಯಾದ ಪದ ಇಲ್ಲವೋ ಅಂತಹ ಧರ್ಮಕ್ಕೆ ಸೇರಿದವನು ನಾನೆಂದು ನನಗೆ ಹೆಮ್ಮೆಯಾಗುತ್ತಿದೆ. (ಚಪ್ಪಾಳೆ) ಈ ಭೂಮಿಯ ಸಮಸ್ತ ದೇಶಗಳ ಸಮಸ್ತ ಧರ್ಮಗಳ ಹಿಂಸಿತರಿಗೂ ನಿರಾಶ್ರಿತರಿಗೂ ಆಶ್ರಯವನ್ನು ನೀಡಿದ ದೇಶಕ್ಕೆ ಸೇರಿದವನು ನಾನೆಂಬ ಹೆಮ್ಮೆ ನನ್ನದು. ರೋಮನ್ ದೌರ್ಜನ್ಯದಿಂದಾಗಿ ತಮ್ಮ ಪವಿತ್ರ ದೇಗುಲವು ಪುಡಿ ಪುಡಿಯಾಗಿ ಹೋದ ವರ್ಷಗಳಲ್ಲಿಯೇ, ಅಳಿದುಳಿದ ಪರಿಶುದ್ಧ ಯಹೂದಿಗಳು ದಕ್ಷಿಣ ಭಾರತಕ್ಕೆ ಬಂದು ನಮ್ಮಲ್ಲಿ ಆಶ್ರಯ ಪಡೆದರು; ನಾವು ಅವರನ್ನು ಮಡಿಲಿನ ಲ್ಲಿಟ್ಟುಕೊಂಡು ಸಾಕುತ್ತಿರುವೆವು ಎಂದು ನಿಮಗೆ ಹೇಳುವುದಕ್ಕೆ ನನಗೆ ಹೆಮ್ಮೆಯಾಗುತ್ತದೆ. ಶ್ರೇಷ್ಠ ಜರತುಷ್ಟ್ರ ದೇಶದ ಪಳೆಯುಳಿಕೆಗೂ ಆಶ್ರಯ ಕೊಟ್ಟು ಇಂದಿಗೂ ಸಾಕು ತ್ತಿರುವ ಧರ್ಮಕ್ಕೆ ಸೇರಿದವನು ನಾನೆಂದು ನನಗೆ ಹೆಮ್ಮೆಯಾಗುತ್ತಿದೆ.

ಸೋದರರೆ, ಈಗ ನಿಮಗೆ ನಾನು ಪ್ರತಿಯೊಂದು ಹಿಂದೂ ಮಗುವೂ ಸಹ ದಿನ ನಿತ್ಯವೂ ಹೇಳಿಕೊಳ್ಳುವ ಸ್ತೋತ್ರವೊಂದರಿಂದ ಕೆಲವು ಸಾಲುಗಳನ್ನು ಉದ್ಧರಿಸಿ ಹೇಳುತ್ತೇನೆ. ನಾನು ಹುಡುಗನಾಗಿದ್ದಾಗಿನಿಂದಲೂ ಬಾಯಿಪಾಠ ಹೇಳಿಕೊಳ್ಳುತ್ತಿದ್ದ, ಭಾರತದಲ್ಲಿನ ಕೋಟಿ ಕೋಟಿ ಜನರು ದಿನನಿತ್ಯ ಹೇಳಿಕೊಳ್ಳುವ ಸ್ತೋತ್ರದ ಅಂತರಾರ್ಥವನ್ನು ಕೊನೆಗೂ ಮನಗಾಣುವಂತೆ ಆಗಿದೆ ಎಂದು ನಾನು ಭಾವಿಸುತ್ತೇನೆ. “ವಿಭಿನ್ನ ಸ್ಥಳಗಳಲ್ಲಿ ಹುಟ್ಟಿದ ಬೇರೆ ಬೇರೆ ನದಿಗಳು ಸಮುದ್ರದಲ್ಲಿ ಒಂದಾಗಿ ಸೇರಿಹೋಗುವ ಹಾಗೆ, ಹೇ ಭಗವಂತ, ವಿಭಿನ್ನ ಮನೋಧರ್ಮಗಳನ್ನುಳ್ಳ ಜನರ ಬೇರೆ ಬೇರೆ ಪಥಗಳು, ನಾನಾ ರೀತಿಯ ಋಜು ಕುಟಿಲ ಪಥಗಳಂತೆ ತೋರಿದರೂ, ಕೊನೆಗೆ ಬಂದು ಸೇರುವುದು ನಿನ್ನಲ್ಲಿಯೇ”.

ಈವರೆಗೆ ನೆರೆದ ಸಭೆಗಳಲ್ಲೇ ಅತ್ಯಂತ ಮಹತ್ವಪೂರ್ಣವಾದದ್ದು ಇಂದಿನ ಈ ಸಭೆ. ಇಂದು ನೆರೆದಿರುವ ಈ ಸಭೆಯು, ಗೀತೆಯಲ್ಲಿ ಬೋಧಿಸಲ್ಪಟ್ಟಿರುವ “ಯಾರೇ ನನ್ನೆಡೆಗೆ ಬರಲಿ, ಯಾವ ರೂಪದಲ್ಲೇ ನಾನು ಅವರನ್ನು ಬಳಿಸಾರಲಿ, ಅವರೆಲ್ಲರೂ ಕೊನೆಗೆ ಯಾವಾಗಲೂ ನನ್ನನ್ನೇ ಬಂದು ಸೇರುವ ಪಥಗಳಲ್ಲಿ ಹೋರಾಡುತ್ತಿರುವರು” ಎಂಬ ಸಿದ್ಧಾಂತವನ್ನು ಪ್ರಪಂಚಕ್ಕೆ ಸಾರುತ್ತಿರುವುದರ ಸೂಚನೆಯಾಗಿದೆ. ಪಂಥೀಯ ಮನೋಭಾವ, ಸ್ವಮತಾಭಿಮಾನ ಹಾಗೂ ಅವುಗಳಿಂದ ಹುಟ್ಟಿದ ಭೀಕರ ಮತಾಂಧತೆ- ಇವುಗಳು ಬಹುಕಾಲದಿಂದಲೂ ಈ ಸುಂದರ ಭೂಮಿಯನ್ನು ಹಿಡಿದಿಟ್ಟುಕೊಂಡಿವೆ. ಇದು ಈ ಭೂಮಿಯನ್ನು ಹಿಂಸೆಯಿಂದ ತುಂಬಿಸಿದೆ; ನೆಲವನ್ನು ಮತ್ತೆ ಮಾನವ ರಕ್ತದಿಂದ ನೆನೆಸಿದೆ; ನಾಗರಿಕತೆಗಳನ್ನು ನಾಶಮಾಡಿದೆ; ರಾಷ್ಟ್ರ ರಾಷ್ಟ್ರಗಳನ್ನೇ ದುಃಖ ವಿದೀರ್ಣವನ್ನಾಗಿ ಮಾಡಿದೆ. ಆದರೆ ಅದರ ಕಾಲ ಮುಗಿಯುತ್ತ ಬಂದಿದೆ; ಈ ಸಭೆಯಲ್ಲಿ ಬಂದು ನೆರೆದಿರುವ ಈ ಭೂಮಿಯ ವಿಭಿನ್ನ ಧರ್ಮಗಳ ಪ್ರತಿನಿಧಿಗಳ ಗೌರವಾರ್ಥವಾಗಿ ಇಂದು ಬೆಳಗ್ಗೆ ಮೊಳಗಿದ ಘಂಟಾನಾದವು ಎಲ್ಲ ಧರ್ಮಾಂಧತೆಯ ಅಂತ್ಯಕ್ರಿಯೆಯ ಘಂಟಾನಾದವಾಗಲಿ (ಚಪ್ಪಾಳೆ), ಲೇಖನಿಯ ಅಥವಾ ಖಡ್ಗದ ಮೂಲಕ ನಡೆಯುವ ಎಲ್ಲ ಹಿಂಸೆ-ಶೋಷಣೆಗಳ, ಹಾಗೂ ಒಂದೇ ಧ್ಯೇಯದೆಡೆಗೆ ಆದರೂ ವಿಭಿನ್ನ ಪಥಗಳಲ್ಲಿ ಸಾಗುತ್ತಿರುವ ಸೋದರರ ನಡುವಣ ಎಲ್ಲ ಅನುದಾರ ಭಾವನೆಗಳ ಅಂತ್ಯಕ್ರಿಯೆಯ ಘಂಟಾನಾದವಾಗಲಿ ಎಂದು ಮನಸಾರೆ ಭಾವಿಸುತ್ತೇನೆ.

\begin{center}
\textbf{ಒಳಾಂಗಣ ಮಾತುಕಥೆ *}\footnote{\enginline{1. New Discoveries, Vol. 1. pp. 60-61.}}
\end{center}

\begin{center}
(ಚಿಕಾಗೊ ರೆಕಾರ್ಡ್, ೧೮೯೩ರ ಸೆಪ್ಟೆಂಬರ್ ೧೧)
\end{center}

ಡಾ. ಬರೋಸ್ \enginline{[Barrow's]} ಅವರ ಒಳಾಂಗಣದಲ್ಲಿ ಧಾರ್ಮಿಕ ಚಿಂತನೆಯ ನಾಲ್ಕು ಮುಂದಾಳುಗಳು ಕುಳಿತಿದ್ದರು - ಜೈನ್, ಚೈನಾದಲ್ಲಿ ಹದಿನಾರು ವರ್ಷಗಳನ್ನು ಕಳೆದಿರುವ ಪ್ರಚಾರಕ ಜಾರ್ಜ್ ಕಾಂಡ್ಲಿನ್ \enginline{[Candline]}, ವಿದ್ಯಾವಂತ ಬ್ರಾಹ್ಮಣ\footnote{2. ಸ್ವಾಮಿಗಳು ಕ್ಷತ್ರಿಯರು, ಬ್ರಾಹ್ಮಣರಲ್ಲ.} ಹಿಂದೂ ಆದ ಸ್ವಾಮಿ ವಿವೇಕಾನಂದ ಮತ್ತು ಚಿಕಾಗೊ ಪ್ರೆಸ್ಬೆಟೆರಿಯನ್ ಆದ ಡಾ. ಜಾನ್ ಹೆಚ್. ಬರೋಸ್. ಈ ನಾಲ್ಕು ಜನರು ಒಂದೇ ಮತದ ಸೋದರರೋ ಎಂಬಂತೆ ಮಾತನಾಡಿದರು\footnote{3. ಈ ಅಜ್ಞಾತ ಮಾತುಕಥೆಯ (ಬಹುಶಃ ಸೆಪ್ಟೆಂಬರ್ ೧೦ ಭಾನುವಾರ ನಡೆದಿರಬಹುದು) ಪದಶಃ ವರದಿ ಲಭ್ಯವಿಲ್ಲ.}.

ಆ ಹಿಂದೂ ಮೃದು ಮುಖಭಾವದವರು. ಅವರ ಪರಿಪುಷ್ಟ ಮುಖವು ಬೌದ್ಧಿಕವೂ ತೇಜಸ್ವಿಯೂ ಆಗಿದ್ದಿತು. ಕಿತ್ತಳೆ ವರ್ಣದ ರುಮಾಲು ಧರಿಸಿ ಅದೇ ವರ್ಣದ ಉಡುಪನ್ನೂ ಧರಿಸಿದ್ದರು. ಅವರ ಇಂಗ್ಲಿಷ್ ತುಂಬ ಚೆನ್ನಾಗಿದೆ. “ನನಗೆ ಮನೆಯೆಂಬುದಿಲ್ಲ” ಎಂದರು ಅವರು. ನಾನು ಭಾರತದಲ್ಲಿ ಒಂದು ಕಾಲೇಜಿನಿಂದ ಇನ್ನೊಂದಕ್ಕೆ ಪ್ರವಾಸ ಮಾಡುತ್ತ ವಿದ್ಯಾರ್ಥಿಗಳಿಗೆ ಉಪನ್ಯಾಸ ಮಾಡುತ್ತೇನೆ. ಅಮೆರಿಕಾಗೆ ಹೊರಡುವ ಮುನ್ನ ನಾನು ಕೆಲವು ಕಾಲ ಮದರಾಸಿನಲ್ಲಿದ್ದೆ. ಈ ದೇಶಕ್ಕೆ ಬಂದಾಗಿನಿಂದ ನನ್ನನ್ನು ತುಂಬ ಮರ್ಯಾದೆಯಿಂದ, ಸೌಜನ್ಯಪೂರ್ವಕವಾಗಿ ನಡೆಸಿಕೊಂಡಿದ್ದಾರೆ. ಲೋಕದ ಧಾರ್ಮಿಕ ಚರಿತ್ರೆಯಲ್ಲಿ ಬಹು ಮುಖ್ಯವಾಗಬಹುದಾದ ಈ ಮಹಾಸಭೆಯಲ್ಲಿ ಮನ್ನಣೆ ಪಡೆದಿರುವುದು ನಮಗೆ ತುಂಬ ಸಂತೋಷವನ್ನು ಕೊಟ್ಟಿದೆ. ಇಲ್ಲಿ ನಮ್ಮ ಅರಿವನ್ನು ಹೆಚ್ಚಿಸಿಕೊಂಡು, ಹಿಂದಿರುಗುವಾಗ ಅನೇಕ ಮಹತ್ವದ ಸತ್ಯಗಳನ್ನು ನಮ್ಮ ಒಂದೂವರೆ ಕೋಟಿ ಬ್ರಾಹ್ಮಣ ಸಮುದಾಯಕ್ಕಾಗಿಕೊಂಡೊಯ್ಯುವುದನ್ನು ಎದುರು ನೋಡುತ್ತಿದ್ದೇವೆ.

\begin{center}
\textbf{ಧರ್ಮವು ಭಾರತದ ಆದ್ಯಾವಶ್ಯಕತೆಯಾಗಿಲ್ಲ *}\footnote{\enginline{1. New Discoveries, Vol. 1. pp. 123-126.}}
\end{center}

\begin{center}
(೧೮೯೩ ಸೆಪ್ಟೆಂಬರ್ ೨೦ರಂದು ಸರ್ವಧರ್ಮಸಮ್ಮೇಳನದಲ್ಲಿ ಕೊಟ್ಟ ಉಪನ್ಯಾಸದ ಶಬ್ದಶಃ ಪ್ರತಿಲಿಪಿ-೧೮೯೩ರ ಸೆಪ್ಟೆಂಬರ್ ೨೧ರ ‘ಚಿಕಾಗೊ ಇಂಟರ್ ಓಷನ್’ ಪತ್ರಿಕೆಯಲ್ಲಿ)\footnote{೨. ಪೂರ್ಣ ಉಪನ್ಯಾಸದಿಂದ ಆಯ್ದ ಉಲ್ಲೇಖಗಳಿಗಾಗಿ ನೋಡಿ, ಭಾರತದ ತೀವ್ರವಾದ ಆಗತ್ಯ ಮತಧರ್ಮವಲ್ಲ’, ಸ್ವಾಮಿ ವಿವೇಕಾನಂದರ ಕೃತಿಶ್ರೇಣಿ, ೧, ಪುಟ ೨೨.}
\end{center}

\begin{center}
\enginline{Suami Vivekananda}
\end{center}

ಮಿ. ಹೆಡ್ ಲ್ಯಾಂಡ್ ಅವರ “ಪೀಕಿಂಗ್ನಲ್ಲಿನ ಧರ್ಮ” ಎಂಬ ಪ್ರಬಂಧವನ್ನು ಓದಿ ಆದ ಮೇಲೆ ಡಾ. ಮೊಮೇರಿ ಅವರು ಸಂಜೆಗೆ ಪಟ್ಟಿಮಾಡಲ್ಪಟ್ಟ ಇನ್ನಿತರ ಭಾಷಣ ಕಾರರು ಬಂದಿಲ್ಲ ಎಂದು ಉದ್ಘೋಷಿಸಿದರು. ಆಗ ಇನ್ನೂ ಒಂಭತ್ತು ಗಂಟೆ; ಮುಖ್ಯ ಸಭಾಂಗಣ ಹಾಗೂ ಗ್ಯಾಲರಿಗಳು ಸಭಿಕರಿಂದ ಭರ್ತಿಯಾಗಿದ್ದವು. ವೇದಿಕೆಯ ಮೇಲೆ ತಮ್ಮ ಕಿತ್ತಳೆ ಬಣ್ಣದ ಉಡುಪನ್ನೂ ಕೆಂಬಣ್ಣದ ರುಮಾಲನ್ನೂ ಧರಿಸಿ ಕುಳಿತಿದ್ದ ಹಿಂದೂ ಸಂನ್ಯಾಸಿ ವಿವೇಕಾನಂದರು ಕಣ್ಣಿಗೆ ಬಿದ್ದೊಡನೆ ಸಭಿಕರ ಚಪ್ಪಾಳೆ ಸದ್ದು ಮುಗಿಲು ಮುಟ್ಟಿತು.

ಈ ಜನಪ್ರಿಯ ಹಿಂದೂ ತಾನು ಆ ರಾತ್ರಿ ಮಾತನಾಡುವುದಕ್ಕೆ ಬಂದಿಲ್ಲವೆಂದು ಜನರ ಉದಾರ ಚಪ್ಪಾಳೆಗೆ ಪ್ರತಿಸ್ಪಂದಿಸಿದರು. ಆದರೂ, ಆ ಸಂದರ್ಭವನ್ನು ಅವರು ಮಿ.ಹೆಡ್ಲ್ಯಾಂಡ್ ಅವರ ಪ್ರಬಂಧದಲ್ಲಿನ ಅನೇಕ ಸಂಗತಿಗಳ ಬಗ್ಗೆ ಟೀಕೆ ಮಾಡಲು ಬಳಸಿಕೊಂಡರು. ಚೀನಾದಲ್ಲಿ ಇರುವ ಬಡತನವನ್ನು ಉದಾಹರಿಸಿ, ಪ್ರಚಾರಕರು ಚೀನೀಯರನ್ನು ಶತಮಾನಗಳಿಂದ ಬಂದ ತಮ್ಮ ಮತಧರ್ಮವನ್ನು ತ್ಯಜಿಸಿ, ತಾವಿತ್ತ ಆಹಾರಕ್ಕೆ (ಪ್ರತಿಯಾಗಿ) ಕ್ರೈಸ್ತಧರ್ಮವನ್ನಪ್ಪಿಕೊಳ್ಳುವಂತೆ ಒತ್ತಾಯಿಸುವ ಬದಲು ಅವರ ಹಸಿವನ್ನು ನೀಗಿಸಲು ಪ್ರಯತ್ನಿಸುವುದು ಒಳ್ಳೆಯದು ಎಂದರು. ಅನಂತರ ಹಿಂದೂವು ವೇದಿಕೆಯ ಮೇಲೆ ಎರಡು ಹೆಜ್ಜೆ ಹಿಂದಕ್ಕೆ ಹೋಗಿ ಕ್ಯಾಥೊಲಿಕ್ ಚರ್ಚಿನ ಬಿಷಪ್ ಕಿಯನೆ ಅವರ ಕಿವಿಯಲ್ಲಿ ಕ್ಷಣಕಾಲ ಏನನ್ನೋ ಪಿಸುನುಡಿದರು.

ಅನಂತರ, ಅಮೆರಿಕನ್ನರಿಗೆ ಪ್ರಾಮಾಣಿಕ ಟೀಕೆಯಿಂದ ಮನನೋಯಲಾರ ದೆಂದು ಬಿಷಪ್ ಕಿಯನೆಯವರು ತಮಗೆ ಹೇಳಿದ್ದಾರೆ ಎನ್ನುತ್ತ ಭಾಷಣವನ್ನು ಮುಂದುವರೆಸಿದರು. ಚೀನಾದಲ್ಲಿರುವ ಎಲ್ಲ ಭೀಕರ ಸಂಗತಿಗಳನ್ನೂ ಭಯಾನಕ ಪರಿಸ್ಥಿತಿಯನ್ನೂ ತಾವು ಕೇಳಿರುವೆವು, ಆದರೆ ಈ ಕಷ್ಟಗಳನ್ನು ನೀಗಿಸುವುದಕ್ಕಾಗಿ ಕ್ರೈಸ್ತರು ಯಾವುದೇ ಅನಾಥಾಲಯಗಳನ್ನು ಸ್ಥಾಪಿಸಿರುವ ಬಗ್ಗೆ ಕೇಳಿಲ್ಲವೆಂದರು. ಅನಂತರ ಹೀಗೆ ನುಡಿದರು:

ಅಮೆರಿಕಾದ ಕ್ರೈಸ್ತ ಸೋದರರೆ, ನಿಮಗೆ ವಿಧರ್ಮೀಯರ ಆತ್ಮಗಳನ್ನು ಸಂರಕ್ಷಿಸಲು ಪ್ರಚಾರಕರನ್ನು ಕಳಿಹಿಸುವುದೆಂದರೆ ಬಹಳ ಆಸಕ್ತಿ. ನಾನು ನಿಮ್ಮನ್ನು ಕೇಳುತ್ತಿದ್ದೇನೆ, ಅವರ ದೇಹಗಳನ್ನು ಹಸಿವೆಯಿಂದ ಸಂರಕ್ಷಿಸುವುದಕ್ಕೆ ನೀವೇನು ಮಾಡಿದ್ದೀರಿ, ಏನು ಮಾಡಲಿದ್ದೀರಿ? (ಚಪ್ಪಾಳೆ). ಭಾರತದಲ್ಲಿ ತಿಂಗಳಿಗೆ ಸರಾಸರಿ ಐವತ್ತು ಸೆಂಟ್ನಷ್ಟು ಹಣದ ಮೇಲೆ ಬದುಕುತ್ತಿರುವ ಮೂವತ್ತು ಕೋಟಿ ಸ್ತ್ರೀಪುರುಷರಿದ್ದಾರೆ. ವರ್ಷಗಟ್ಟಲೆ ಕಾಡುಹೂಗಳನ್ನು ತಿಂದುಕೊಂಡು ಅವರು ಬದುಕುತ್ತಿರುವುದನ್ನು ನಾನು ನೋಡಿದ್ದೇನೆ. ಸ್ವಲ್ಪ ಬರ ಪರಿಸ್ಥಿತಿಯುಂಟಾಯಿತೋ, ನೂರುಗಟ್ಟಲೆ, ಸಾವಿರಗಟ್ಟಲೆ ಜನರು ಹಸಿವಿನಿಂದ ಪ್ರಾಣಬಿಡುತ್ತಾರೆ. ಕ್ರೈಸ್ತ ಪ್ರಚಾರಕರು ಬಂದು ಅವರಿಗೆ ಜೀವನಕ್ಕೊಂದು ದಾರಿ ಮಾಡಿಕೊಡುತ್ತಾರೆಯಾದರೂ, ಅದಕ್ಕೆ ಪ್ರತಿಯಾಗಿ ಹಿಂದೂಗಳು ತಮ್ಮ ಪಿತೃಗಳಿಂದ ಪರಂಪರಾಗತವಾಗಿ ಬಂದ ಧರ್ಮಶ್ರದ್ಧೆಯನ್ನು ತ್ಯಜಿಸಿ ಕ್ರೈಸ್ತರಾಗಬೇಕು ಎಂಬ ನಿಬಂಧನೆ ಹಾಕುತ್ತಾರೆ. ಇದು ಸರಿಯೆ? ನೂರಾರು ಅನಾಥಲಯಗಳಿವೆ. ಆದರೆ ಮುಸ್ಲಿಮರೋ ಹಿಂದೂಗಳೋ ಅಲ್ಲಿಗೆ ಹೋದರೆ ಒದ್ದೋಡಿಸುತ್ತಾರೆ. ಹಿಂದೂಗಳು ಕಟ್ಟಿರುವ ಸಾವಿ ರಾರು ಅನಾಥಾಲಯಗಳಲ್ಲಿ ಯಾರು ಹೋದರೂ ಸೇರಿಸಿಕೊಳ್ಳುತ್ತಾರೆ. ಹಿಂದೂಗಳ ಸಹಕಾರದಿಂದ ಕಟ್ಟಿರುವ ನಾರಾರು ಚರ್ಚುಗಳಿವೆಯಾದರೂ, ಕ್ರೈಸ್ತನೊಬ್ಬ ಒಂದು ಕಾಸುಕೊಟ್ಟಿರುವ ಯಾವ ಹಿಂದೂ ದೇಗುಲವೂ ಇಲ್ಲ.

\begin{center}
\textbf{ಪ್ರಾಚ್ಯಕ್ಕೆ ಅಗತ್ಯವಾಗಿರುವುದೇನು?}
\end{center}

ಅಮೆರಿಕಾದ ಸೋದರರೆ, ಪ್ರಾಚ್ಯದಲ್ಲಿ ತುರ್ತಾಗಿ ಬೇಕಾಗಿರುವುದು ಧರ್ಮವಲ್ಲ. ಧರ್ಮವು ನಮ್ಮಲ್ಲಿ ಅಗತ್ಯಕ್ಕಿಂತ ಹೆಚ್ಚಾಗಿಯೇ ಇದೆ; ಅವರಿಗೆ ಬೇಕಾದುದು ರೊಟ್ಟಿ, ಆದರೆ ಅವರಿಗೆ ಕೊಟ್ಟಿರುವುದು ಕಲ್ಲು (ಚಪ್ಪಾಳೆ). ಹಸಿವಿನಿಂದ ಸಂಕಟಪಡುತ್ತ ಸಾಯುತ್ತಿರುವವನಿಗೆ ತತ್ತ್ವವನ್ನು ಬೋಧಿಸುವುದು ಅವನಿಗೆ ಮಾಡುವ ಅವಮಾನ. ಆದ್ದರಿಂದ, “ಸಹೋದರತ್ವ” ಎಂಬ ಮಾತಿನ ಅರ್ಥವನ್ನು ವಿಶದಪಡಿಸಬೇಕೆಂದು ನೀವು ಇಚ್ಛಿಸುವುದಾದರೆ, ಹಿಂದೂವನ್ನು ಅನುಕಂಪದಿಂದ ನಡೆಸಿಕೊಳ್ಳಿರಿ- ಅವನು ಹಿಂದೂ ಆಗಿದ್ದರೂ ತನ್ನ ಧರ್ಮದ ಬಗ್ಗೆ ಶ್ರದ್ಧೆಯಿಟ್ಟುಕೊಂಡಿದ್ದರೂ ಸಹ, ಇನ್ನೂ ಒಳ್ಳೆಯ ಒಂದು ರೊಟ್ಟಿಯನ್ನು ಸಂಪಾದಿಸುವುದು ಹೇಗೆ ಎಂದು ಅವರಿಗೆ ತೋರಿಸಿಕೊಡುವುದಕ್ಕೆ ಬೇಕಾದರೆ ಪ್ರಚಾರಕರನ್ನು ಕಳುಹಿಸಿ; ಅರ್ಥಹೀನ ತತ್ತ್ವವನ್ನು ಬೋಧಿಸುವುದಕ್ಕಲ್ಲ (ದೀರ್ಘ ಕರ ತಾಡನ).

ಅನಂತರ ಸಂನ್ಯಾಸಿ ಈ ಹೊತ್ತು ಅನಾರೋಗ್ಯದ ಕಾರಣದಿಂದಾಗಿ ಭಾಷಣ ನಿಲ್ಲಿಸಬೇಕಾಗಿರುವುದರಿಂದ ತಮ್ಮನ್ನು ಕ್ಷಮಿಸಬೇಕು ಎಂದರು. ಆದರೆ ಸಿಡಿಲಿನಂತಹ ಚಪ್ಪಾಳೆಯೂ, “ಮುಂದುವರೆಸಿ” ಎಂಬ ಕೂಗುಗಳೂ ಕೇಳಿಸತೊಡಗಿದವು; ಆಗ ಮಿ. ವಿವೇಕಾನಂದರು ಮುಂದುವರೆಸಿದರು.

ಈಗತಾನೆ ಓದಿದ ಪ್ರಬಂಧವು ದಯನೀಯ, ಅಜ್ಞಾನಿ, ಧರ್ಮಬೋಧಕನ ಬಗ್ಗೆ ಕೆಲವು ಸಂಗತಿಗಳನ್ನು ಹೇಳುತ್ತದೆ. ಅದನ್ನೇ ಭಾರತದ ಬಗೆಗೂ ಹೇಳಬಹುದು. ಭಿಕ್ಷುಕ ಎಂದು ವರ್ಣಿಸಲ್ಪಟ್ಟ ಸಂನ್ಯಾಸಿಗಳಲ್ಲೊಬ್ಬನು ನಾನು. ಅದೇ ನನ್ನ ಜೀವನದ ಹೆಮ್ಮೆ (ಚಪ್ಪಾಳೆ). ಕ್ರಿಸ್ತನ ಹಾಗಿರುವವನು ಎಂಬರ್ಥದಲ್ಲಿ ನನಗೆ ಹೆಮ್ಮೆ. ಈ ಹೊತ್ತು ನನಗೆ ಸಿಕ್ಕಿದ್ದನ್ನು ನಾನು ತಿನ್ನುತ್ತೇನೆ; ನಾಳೆಯ ಚಿಂತೆ ನನಗಿಲ್ಲ. “ನೋಡಾ ಬಯಲಲ್ಲಿ ಪುಷ್ಪಗಳ; ಬಸವಳಿಯಲೊಲ್ಲದವು, ತಿರುಗಾಡಲೊಲ್ಲದವು”. ಹಿಂದೂ ಇದನ್ನು ಅಕ್ಷರಶಃ ಪಾಲಿಸು ವನು. ಕಳೆದ ಹನ್ನೆರಡು ವರ್ಷಗಳಿಂದ ನನ್ನ ಮುಂದಿನ ಊಟ ಎಲ್ಲಿಂದ ಬರಬಹುದೆಂದು ನಾನು ತಿಳಿದಿರಲಿಲ್ಲ ಎಂಬುದಕ್ಕೆ ಈ ವೇದಿಕೆಯ ಮೇಲೆ ಕುಳಿತಿರುವ ಚಿಕಾಗೋದ ಅನೇಕ ಸಭ್ಯಜನರೇ ಸಾಕ್ಷಿ. ದೇವರಿಗೋಸ್ಕರ ನಾನು ಭಿಕ್ಷುಕನಾಗಿರುವುದಕ್ಕೆ ನನಗೆ ಹೆಮ್ಮೆ ಇದೆ. ಪ್ರಾಚ್ಯದಲ್ಲಿರುವ ಕಲ್ಪನೆಯಂತೆ, ಹಣಕ್ಕಾಗಿ ಏನನ್ನೇ ಆದರೂ ಭೋಧಿಸುವುದು ಕೀಳು, ಅಸಭ್ಯ; ಹಣಕ್ಕಾಗಿ ದೇವರ ಹೆಸರನ್ನು ಬೋಧಿಸುವುದಂತೂ ಅದೆಂತಹ ಹೀನಾಯವೆಂದರೆ, ಹಾಗೆ ಮಾಡುವವನು ಜಾತಿಭ್ರಷ್ಟನಾಗುವನು; ಜನರು ಅವನ ಮೇಲೆ ಉಗುಳುವರು. ಪ್ರಬಂಧದಲ್ಲಿ ಒಂದು ಸಲಹೆ ಇದೆ, ಅದು ನಿಜ: ಭಾರತದ ಮತ್ತು ಚೀನಾದ ಧರ್ಮ ಬೋಧಕರು ಸಂಘಟಿತರಾದರೆ, ಸುಪ್ತವಾಗಿರುವ ಆಗಾಧ ಚೈತನ್ಯವನ್ನು ಸಮಾಜದ, ಜನಾಂಗದ ಪುನರುತ್ಥಾನಕ್ಕೆ ಬಳಸಬಹುದು. ನಾನು ಅದನ್ನು ಭಾರತದಲ್ಲಿ ಸಂಘಟಿಸಲು ಯತ್ನಿಸಿದೆ; ಹಣದ ಅಭಾವದಿಂದ ವಿಫಲನಾದೆ. ನನಗೆ ಬೇಕಾದ ಸಹಾಯವನ್ನು ನಾನು ಅಮೆರಿಕಾದಿಂದ ಪಡೆಯಬಹುದೇನೋ.

ಆದರೆ ನಮಗೆ ಗೊತ್ತು, ವಿಧರ್ಮೀಯನೊಬ್ಬನು “ಕ್ರೈಸ್ತ ಜನ”ರಿಂದ ಏನಾದರೂ ಸಹಾಯ ಪಡೆಯುವುದು ಬಹಳ ಕಷ್ಟ ಎಂದು (ದೀರ್ಘ ಕರತಾಡನ). ಸ್ವಾತಂತ್ರ್ಯದ, ಮುಕ್ತತೆಯ, ಅಭಿಪ್ರಾಯಸ್ವಾತಂತ್ರ್ಯದ ಈ ನಾಡಿನ ವಿಚಾರವಾಗಿ ನಾನು ಬಹಳಷ್ಟು ಕೇಳಿದ್ದೇನೆ; ಆದ್ದರಿಂದ ನನಗೆ ಧೈರ್ಯ. ಮಹಿಳೆಯರೆ, ಮಾನ್ಯರೆ, ನಿಮಗೆ ನಾನು ವಂದಿಸುತ್ತೇನೆ.

ಅನಂತರ ಈ ಜನಪ್ರಿಯ ಸಂದರ್ಶಕರು ತಲೆಬಾಗಿ ಮಂದಹಾಸದೊಂದಿಗೆ ಮುಗಿ ಸಲೆತ್ನಿಸಿದಾಗ, ಸಭಿಕರು ಮುಂದುವರೆಸಿರೆಂದು ಕೂಗಿದರು. ಸಜ್ಜನಿಕೆಯ ಭಾವದೊಂದಿಗೆ ಮಿ. ವಿವೇಕಾನಂದರು ಅನಂತರ ಹಿಂದೂ ಪುನರ್ಜನ್ಮ ಸಿದ್ಧಾಂತವನ್ನು ವಿವರಿಸಿದರು. ಭಾಷಣದ ಕೊನೆಯಲ್ಲಿ ಡಾ. ಮೊಮೇರಿ (ಇಂಗ್ಲೆಂಡಿನಿಂದ ಬಂದ ಒಬ್ಬ ಪ್ರತಿನಿಧಿ)ಈ ಸಮ್ಮೇಳನವನ್ನು ಪತ್ರಿಕೆಗಳು ಸಹಸ್ರಮಾನದ ಪ್ರವೇಶವೆಂದು ಏಕೆ ಕರೆದಿವೆ ಎನ್ನುವುದು ತಮಗೀಗ ಅರ್ಥವಾಯಿತು ಎಂದರು.....

\begin{center}
\textbf{ಚಿಕಾಗೋ ಪತ್ರ *\supskpt{\footnote{\enginline{1. New Discoveries, Vol. 1. pp. 162-63}}}}
\end{center}

\begin{center}
(ನ್ಯೂಯಾರ್ಕ್ ಕ್ರಿಟಿಕ್, ೧೧ ನವೆಂಬರ್ ೧೮೯೩)
\end{center}

....ಪುರಾತನ ಮತಗಳಲ್ಲಿನ ತತ್ತ್ವವು ಆಧುನಿಕರಿಗೆ ಹೆಚ್ಚಿನ ಸೌಂದರ್ಯವನ್ನು ತುಂಬಿಕೊಂಡದ್ದಾಗಿದೆ ಎಂಬ ವಾಸ್ತವಕ್ಕೆ ನಮ್ಮ ಕಣ್ಣುಗಳನ್ನು ತೆರೆದಿದ್ದು ಸರ್ವಧರ್ಮ ಸಮ್ಮೇಳನದ ಸಹಜವಾದೊಂದು ಪರಿಣಾಮ. ನಾವಿದನ್ನು ಸ್ಪಷ್ಟವಾಗಿ ಗ್ರಹಿಸುತ್ತಲೂ, ಅವುಗಳ ಪ್ರಬೋಧಕರಲ್ಲಿ ನಮ್ಮ ಆಸಕ್ತಿ ಜಾಗೃತವಾಯಿತು; ವಿಶಿಷ್ಟವಾದ ಕಾತರದಿಂದ ನಾವು ಜ್ಞಾನಕ್ಕಾಗಿ ಹಂಬಲಿಸತೊಡಗಿದೆವು. ಸರ್ವಧರ್ಮಸಮ್ಮೇಳನ ಮುಗಿದ ಮೇಲೆ ಅದನ್ನು ಪಡೆಯಲು ಅತ್ಯಂತ ಲಭ್ಯವೆನಿಸಿದ್ದು, ಈ ನಗರದಲ್ಲೇ ಇನ್ನೂ ಉಳಿದಿರುವ ಸ್ವಾಮಿ \enginline{Suami} ವಿವೇಕಾನಂದರ ಉಪನ್ಯಾಸಗಳ ಮೂಲಕ. ಈ ದೇಶಕ್ಕೆ ಬರುವ ಮೂಲ ಉದ್ದೇಶ ಹಿಂದೂಗಳ ನಡುವೆ ಹೊಸ ಕೈಗಾರಿಕೆಗಳನ್ನು ಪ್ರಾರಂಭಿಸುವುದಕ್ಕೆ ಅಮೆರಿಕನ್ನ ರನ್ನು ಆಸಕ್ತರನ್ನಾಗಿ ಮಾಡುವುದಾಗಿತ್ತು. ಆದರೆ ಅವರು ಅದನ್ನು ಸದ್ಯದ ಮಟ್ಟಿಗೆ ಬಿಟ್ಟುಕೊಟ್ಟಿರುವರು- ಏಕೆಂದರೆ, ಅವರು ಕಂಡುಕೊಂಡಿರುವಂತೆ “ಪ್ರಪಂಚದಲ್ಲೇ ಅಮೆರಿಕನ್ನರು ಅತ್ಯಂತ ಉದಾರಿಗಳಾದ್ದರಿಂದ,” ಯಾವುದಾದರೊಂದು ಉದ್ದೇಶವಿಟ್ಟುಕೊಂಡಿರುವ ಪ್ರತಿಯೊಬ್ಬನೂ ಅದನ್ನು ಈಡೇರಿಸಲು ಸಹಾಯ ಕೋರಿ ಇಲ್ಲಿಗೆ ಬರುತ್ತಾನೆ. ಇಲ್ಲಿನ ಬಡವರಿಗೆ ಹೋಲಿಸುತ್ತ ಭಾರತದಲ್ಲಿಯ ಬಡತನದ ಬಗ್ಗೆ ಕೇಳಿದಾಗ, ಇಲ್ಲಿನ ಬಡವರೆಂದರೆ ಅಲ್ಲಿನ ರಾಜಕುಮಾರರ ಹಾಗೆ ಎಂದರು; ತಮ್ಮನ್ನು ನಗರದ ಕಳಪೆ ಪ್ರದೇಶದ ಮೂಲಕಕೊಂಡೊಯ್ದಿದ್ದರೂ, ತಮ್ಮ ಅರಿವಿನ ದೃಷ್ಟಿಯಿಂದ ಅದೂ ಸಹ ಸುಖಕರ ಮಾತ್ರವಲ್ಲ ಸಂತೋಷದಾಯಕವೂ ಆಗಿ ಕಂಡಿದೆ ಎಂದರು.

ಬ್ರಾಹ್ಮಣರಲ್ಲಿ ಬ್ರಾಹ್ಮಣರಾದ ವಿವೇಕಾನಂದರು, ಇಚ್ಛಾಪೂರ್ವಕವಾಗಿ ಜಾತಿಯ ಅಹಂಕಾರವನ್ನು ತ್ಯಜಿಸಿಬಿಡುವ ಸಂನ್ಯಾಸಿಗಳ ಸಂಘಕ್ಕೆ ಸೇರಿದರು. ಆದರೂ ಅವರ ಮೈ ಮೇಲೆ ಅವರ ಜಾತಿಯ ಲಕ್ಷಣಗಳು ಢಾಳವಾಗಿವೆ. ಅವರ ಸಂಸ್ಕೃತಿ,ವಾಗ್ವೈಖರಿ ಮತ್ತು ಆಕರ್ಷಕ ವ್ಯಕ್ತಿತ್ವಗಳು ನಮಗೆ ಹಿಂದೂ ನಾಗರಿಕತೆಯ ಬಗ್ಗೆ ಹೊಸ ಕಲ್ಪನೆಯನ್ನು ತಂದುಕೊಟ್ಟಿವೆ. ಅವರ ಕುತೂಹಲ ಕೆರಳಿಸುವ ಭಾವಭಂಗಿ, ಒಳ್ಳೆಯ ಬುದ್ಧಿವಂತ ಹಾಗೂ ಚಲನಶೀಲ ಮುಖ, ಅದಕ್ಕೆ ಕಿರೀಟವಿಟ್ಟಂತಿರುವ ಹಳದಿಯುಡುಪು, ಆಳವಾದ ಸಂಗೀತದಂತಹ ಧ್ವನಿ - ಒಮ್ಮೆಲೇ ಯಾರನ್ನಾದರೂ ಸೆಳೆದು ತನ್ನವರನ್ನಾಗಿಸಿಕೊಂಡು ಬಿಡುವಂಥವು\footnote{2. ಸ್ವಾಮಿಗಳು ಇದನ್ನು ೧೮೯೩ರ ನವೆಂಬರ್ ೧೫ರಂದು ಶ‍್ರೀ ಹರಿದಾಸ್ ವಿಹಾರಿದಾಸ್ ದೇಸಾಯಿಯವರಿಗೆ ಬರೆದ ಪತ್ರದಲ್ಲಿ ಉಲ್ಲೇಖಿಸುತ್ತಾರೆ. ನೋಡಿ, ಕೃತಿಶ್ರೇಣಿ, ೪, ಪುಟ ೧೯೯.}. ಆದ್ದರಿಂದ, ಸಾಹಿತ್ಯಸಂಘಗಳು ಅವರನ್ನು ಕರೆದೊಯ್ದಿರುವುದರಲ್ಲಿ ಅಚ್ಚರಿಯೇನಿಲ್ಲ; ಅವರೂ ಬುದ್ಧನ ಜೀವನ ಮತ್ತು ಆತನ ನಂಬಿಕೆಯ ಸಿದ್ಧಾಂತಗಳು ನಮಗೆ ಚೆನ್ನಾಗಿ ಪರಿಚಯವಾಗುವ ಹಾಗೆ ಚರ್ಚುಗಳಲ್ಲಿ ಉಪನ್ಯಾಸ ಮಾಡಿದರು, ಬೋಧಿಸಿದರು. ಟಿಪ್ಪಣಿಯಿಲ್ಲದೆ ಮಾತನಾಡುವ ಅವರು, ಹೇಳಬೇಕಾಗಿರುವ ವಾಸ್ತವ ಸಂಗತಿಗಳನ್ನೂ ತಮ್ಮ ತೀರ್ಮಾನಗಳನ್ನೂ ಅತ್ಯಂತ ಕಲಾತ್ಮಕವಾಗಿ, ಅತ್ಯಂತ ಮನಮುಟ್ಟುವಂತೆ ಪ್ರಾಮಾಣಿಕತೆಯಿಂದ, ಕೆಲವೊಮ್ಮೆ ಸ್ಫೂರ್ತಿಯಿಂದ ಪುಟಿದೇಳುವಂತೆ ನಮ್ಮ ಮುಂದಿಡುವರು. ಅತ್ಯಂತ ಕುಶಲ ಜೆಸುಯಿಟರಷ್ಟೇ ವಿದ್ಯೆ ಪರಿಣತಿಗಳನ್ನು ಪಡೆದಿರುವ ಅವರ ಮನಃಪರಿಪಾಕದಲ್ಲಿ ಸ್ಪಷ್ಟವಾಗಿ ಒಂದಷ್ಟು ಜೆಸುಯಿಟತನವಿದೆ; ಆದರೆ ಅವರು ಉಪನ್ಯಾಸಗಳಲ್ಲಿ ತುಂತುರಿಸುವ ಸಣ್ಣಪುಟ್ಟ ಕಟಕಿಗಳು ಇರಿಗತ್ತಿಗಳಂತೆ ತಾಕಿದರೂ, ಅವು ಕೇಳುಗರಲ್ಲಿ ಕಳೆದುಹೋಗಿ ಬಿಡುವಷ್ಟು ಸೂಕ್ಷ್ಮವಾಗಿರುತ್ತವೆ. ಅಲ್ಲದೆ ಎಂದೂ ಸೋಲದ ಸೌಜನ್ಯ ಅವರದ್ದು; ಏಕೆಂದರೆ ಆ ಇರಿತಗಳು ನಮ್ಮ ಸಂಪ್ರದಾಯಗಳನ್ನು ನೇರ ತಾಕಿ ಅಸಭ್ಯವೆನಿಸುವುದೇ ಇಲ್ಲ. ಸದ್ಯದಲ್ಲಿ ಅವರು ತಮ್ಮ ಧರ್ಮದ ಬಗ್ಗೆ, ಅದರ ತಾತ್ತ್ವಿಕರ ಉಕ್ತಿಗಳ ಬಗ್ಗೆ ನಮಗೆ ಅರಿವನ್ನುಂಟುಮಾಡುವುದರಲ್ಲಿ ತೃಪ್ತರಾಗಿದ್ದಾರೆ. ನಾವು ವಿಗ್ರಹಾರಾಧನೆಯನ್ನು ಮೀರಿ ಮುಂದೆ ಹೋಗುವ ದಿನವನ್ನು ಅವರು ಎದುರು ನೋಡುತ್ತಿದ್ದಾರೆ. ಅವರ ಅಭಿಪ್ರಾಯದಲ್ಲಿ ಈಗ ಅದು ಅಜ್ಞರಾದವರಿಗೆ ಆವಶ್ಯಕವೇ. ಆರಾಧನೆಯನ್ನೂ ಮೀರಿ, ಪ್ರಕೃತಿಯಲ್ಲೇ ಭಗವಂತನಿರುವುದರ ಜ್ಞಾನವನ್ನು ಪಡೆದು, ಮಾನವನ ದಿವ್ಯತೆ ಹಾಗೂ ಜವಾಬ್ದಾರಿಗಳನ್ನು ಅರಿಯುವುದನ್ನು ಎದುರುನೋಡುತ್ತಿದ್ದಾರೆ. ಸಾಯುತ್ತಿರುವ ಬುದ್ಧನೊಡನೆ ಅವರು ಹೇಳುತ್ತಾರೆ - “ನಿಮ್ಮ ಮುಕ್ತಿಗಾಗಿ ನೀವೇ ಶ್ರಮಿಸಬೇಕು; ನಾನು ನಿಮಗೆ ಸಹಾಯ ಮಾಡಲಾರೆ. ಯಾರೊಬ್ಬರೂ ನಿಮಗೆ ಸಹಾಯ ಮಾಡಲಾರರು. ನಿಮಗೆ ನೀವೇ ಸಹಾಯ ಮಾಡಿಕೊಳ್ಳಬೇಕು”.

\begin{center}
\textbf{ಭಾರತದ ಧರ್ಮಗಳು}\footnote{\enginline{1. New Discoveries, Vol. 1. pp. 191}}
\end{center}

\begin{center}
ಹಿಂದೂವಾಗ್ಮಿ \enginline{Viva Kannada} ಅವರ ಆಸಕ್ತಿ ಮೂಡಿಸುವ ಉಪನ್ಯಾಸ\footnote{2. ಇದರ ಪದಶಃ ವರದಿ ಲಭ್ಯವಿಲ್ಲ. ಸಾರಾಂಶಕ್ಕೆ ನೋಡಿ, \enginline{Complete Works, III: 481}}
\end{center}

(ಡೇಲಿ ಕಾರ್ಡಿನಲ್, ಮಡಿಸನ್ನ ವಿಸ್ಕಾನ್ಸಿನ್ ವಿಶ್ವವಿದ್ಯಾನಿಲಯ, ೨೧ ನವೆಂಬರ್ ೧೮೯೩)

ಕಳೆದ ಸಂಜೆ ಭಕ್ತಸಮುದಾಯ ಚರ್ಚ್​ನಲ್ಲಿ \enginline{Viva Kannada} ಅವರನ್ನು ತುಂಬಿದ ಸಭೆ ಜಯಘೋಷದೊಂದಿಗೆ ಎದುರುಗೊಂಡಿತು. ಭಾಷಣಕಾರರು ಕೆನೆಬಣ್ಣದ ರುಮಾಲು, ಹಳದಿ ಗೌನ್ ಮತ್ತು ಕಡುಕೆಂಪು ಕಟಿಬಂಧಗಳ ದೇಶೀಯ ಉಡುಪಿನಿಂದ ಅಲಂಕೃತರಾಗಿದ್ದರು.

ಉಪನ್ಯಾಸದ ಮೊದಲ ಭಾಗವು ಹಿಂದೂಗಳ ಭಾಷೆಯಾದ ಸಂಸ್ಕೃತಕ್ಕೂ ಇಂಗ್ಲಿಷಿಗೂ ಇರುವ ಸಾಮ್ಯತೆಯನ್ನು ವಿಶದಪಡಿಸುವುದಕ್ಕೆ ಮುಡಿಪಾಗಿತ್ತು. \enginline{Salvation} ಎನ್ನುವುದಕ್ಕೆ ಅವರ ಭಾಷೆಯಲ್ಲಿ ಪದವಿಲ್ಲ; ಅವರಿಗೆ ಅದು ಬಂಧನದಿಂದ ಮುಕ್ತಿ. ಮಾನವನ ನಿಜ ಸ್ವರೂಪವೇ ಪರಿಪೂರ್ಣತೆ, ಕಾರ್ಯಕಾರಣಗಳೇ ದೇವರೊಬ್ಬನನ್ನು ಬಿಟ್ಟು ಉಳಿದೆಲ್ಲವನ್ನೂ ನಿಯಂತ್ರಿಸುವಂಥವು ಎಂದು ಅವರು ನಂಬುತ್ತಾರೆ. ದೊಡ್ಡ ಆನೆಯೊಂದರ ಭಾಗಗಳನ್ನು ಮುಟ್ಟಿ ನೋಡಿದ ಕುರುಡರ ಕಥೆ ಧರ್ಮವನ್ನು ವಿಶದಪಡಿಸಲು ಯೋಗ್ಯವಾಗಿದೆ. ಒಬ್ಬೊಬ್ಬ ಕುರುಡನೂ ಆನೆಯು ತಾನು ಮುಟ್ಟಿದ ಭಾಗದಂತೆಯೇ ಇದೆ ಎಂದುಕೊಂಡ ಹಾಗೆಯೇ, ಧರ್ಮದಲ್ಲೂ ಸಹ ವಿವಿಧ ಮತಪಂಥಗಳ ಜನರು ಭಾಗಶಃ ಸತ್ಯವನ್ನು ಕಂಡುಕೊಂಡಿರುವರು; ಸತ್ಯವಾದರೋ ಅನಂತವಾದುದು; ಯಾವನೊಬ್ಬನೂ “ನಾನದನ್ನು ಪೂರ್ಣವಾಗಿ ಕಂಡಿರುವೆ” ಎನ್ನುವ ಹಾಗಿಲ್ಲ.

ಧರ್ಮಶ್ರದ್ಧೆಗಳಲ್ಲೆಲ್ಲ ಹಿಂದೂ ಶ್ರದ್ಧೆಯು ಅತ್ಯಂತ ಉದಾರವಾದುದೆಂದು ತೋರಿಸಿ ಕೊಡಲಾಯಿತು. ಪೀಡನೆ ಎಂಬುದು ಭಾರತದಲ್ಲಿಲ್ಲ; ಅವರ ಭಾಷೆಯಲ್ಲಿ ಅಂತಹ ಪದವೇ ಇಲ್ಲ. ಹಿಂದೂಗಳು ನಡೆದು ಬಂದ ಹಾದಿಯಲ್ಲಿ ಕ್ರೈಸ್ತ ಪ್ರಚಾರಕನೊಬ್ಬರನ್ನು ಪೀಡಿಸಿದ ಉದಾಹರಣೆಯನ್ನು ತೋರಿಸಿ ಎಂದು ಭಾಷಣಕಾರರು ಪ್ರಪಂಚಕ್ಕೇ ಸವಾಲೆಸೆದರು. ಹಿಂದೂಗಳ ಬಗ್ಗೆ ಬರೆಯುತ್ತ ಗ್ರೀಕ್ ಚರಿತ್ರೆಕಾರನೊಬ್ಬನು “ಯಾವೊಬ್ಬ ಹಿಂದೂ ಪುರುಷನೂ ಅಪ್ರಾಮಾಣಿಕನಲ್ಲ; ಯಾವೊಬ್ಬ ಹಿಂದೂ ಸ್ತ್ರೀಯೂ ಶೀಲಗೆಟ್ಟವ ಳಲ್ಲ” ಎಂದು ಹೇಳಿರುವನು.

\enginline{Viva Kannada} ಈ ದೇಶಕ್ಕೆ ಬಂದುದು ವಿಶ್ವ ಧರ್ಮಸಮ್ಮೇಳನದ ಕುತೂಹಲದಿಂದಾಗಿ. ಕಳೆದ ಸಂಜೆಯ ಅವರ “ಭಾರತದ ಧರ್ಮಗಳು” ಎಂಬ ಉಪನ್ಯಾಸವು ಕೇಳಿದವರಿಗೆಲ್ಲ ಸ್ಫೂರ್ತಿದಾಯಕವಾಗಿತ್ತು. ಮುದ ನೀಡುವ, ಸುಂದರವಾದ, ನಸುಗಪ್ಪು ಬಣ್ಣದ ಮುಖ ಹಾಗೂ ನಿಜಕ್ಕೂ ಮನಮೆಚ್ಚುವಂಥ ಭಾವಭಂಗಿ ಅವರದ್ದು. ಅವರ ಧ್ವನಿ ಆನಂದದಾಯಕ; ಪ್ರಾರಂಭದಲ್ಲೇ ನಿಮ್ಮ ಗಮನವನ್ನು ಸೆಳೆಯುವ ಮೃದು ಹಾಗೂ ಗೂಢವಾದ ಏನೋ ಒಂದು ಅದರಲ್ಲಿದೆ.

\begin{center}
\textbf{ಎಲ್ಲ ಧರ್ಮಗಳೂ ಸತ್ಯ}
\end{center}

\begin{center}
ಹಿಂದೂ ಸಂನ್ಯಾಸಿಯಿಂದ
\end{center}

\begin{center}
\textbf{ಭಾರತದಿಂದ ತಂದ ಸಂದೇಶ}\footnote{\enginline{1. New Discoveries, Vol. 1. pp. 200-202}}
\end{center}

\begin{center}
(ಡೈಲಿ ಅಯೋವಾ ಕ್ಯಾಪಿಟಲ್, ೨೮ ನವೆಂಬರ್ ೧೮೯೩)
\end{center}

ಪ್ರಾಚೀನ ಧರ್ಮಶ್ರದ್ಧೆಯನ್ನು ಕುರಿತು ಸ್ವಾಮಿ ವಿವೇಕಾನಂದರಿಂದ ಈ ರಾತ್ರಿ ಮತ್ತೆ ಭಾಷಣ\footnote{2. ೧೮೯೨ರ ನವೆಂಬರ್ ೨೭ರಂದು ಕೊಟ್ಟ “ಹಿಂದೂ ಧರ್ಮ’ ಎಂಬ ಈ ಭಾಷಣದ ವರದಿ ಪದಶಃ ಲಭ್ಯವಿಲ್ಲ. ವಿಭಿನ್ನ ಮುಖ್ಯಾಂಶಗಳಿಗೆ ನೋಡಿ, \enginline{Complete Works, III: 48-484}}

ಕಳೆದ ಸಂಜೆ ಸೆಂಟ್ರಲ್ ಚರ್ಚ್ ಆಫ್ ಕ್ರೈಸ್ಟ್ನಲ್ಲಿದೆ ಮೋಯ್ನ್ಸ್ ನಗರದ ಜನರಿ ಗೊಂದು ಅಪರೂಪದ ಹಾಗೂ ಅಸಾಧಾರಣ ಸಂತೋಷದ ಅನುಭವ. ಪ್ರಾಚೀನ ಬ್ರಹ್ಮ ಶ್ರದ್ಧೆಯ ಸಂನ್ಯಾಸಿಯೊಬ್ಬರು ಆ ಶ್ರದ್ಧೆಯ ಪ್ರಮುಖ ತತ್ತ್ವಗಳನ್ನು - ಕೊಂಚ ಅದರ ವೈಚಿತ್ರ್ಯಗಳೊಡನೆ - ಸಂತೋಷವಾಗುವಂತೆ ಪ್ರಸ್ತುತಪಡಿಸಿದರು. ಸುಮಾರು ಐನೂರು ಅಥವಾ ಆರು ನೂರು ಶ್ರೋತೃಗಳ ದೊಡ್ಡ ಸಭೆಯೇ ಸೇರಿತ್ತು. ಮುಖ್ಯ ಸಭಾಂಗಣ ತುಂಬಿ, ಉಪ್ಪರಿಗೆಯ ಕೈಸಾಲೆಯಲ್ಲಿಯೂ ನೂರಿನ್ನೂರು ಜನರಿದ್ದರು.

ಎಲ್ಲ ಧರ್ಮಗಳೂ ‘ನಾನಾರು?’ ಎಂಬ ಪ್ರಶ್ನೆಗೆ ಉತ್ತರಿಸುವ ಪ್ರಯತ್ನವಷ್ಟೇ ಎಂದು ಹೇಳುತ್ತ ಭಾಷಣಕಾರರು ಪ್ರಾರಂಭಿಸಿದರು. ಇದರೊಂದಿಗೆ ಉಪಪ್ರಶ್ನೆಗಳಾದ ನಾನೆಲ್ಲಿಂದ ಬಂದೆ?ನಾನೆಲ್ಲಿಗೆ ಹೋಗುತ್ತಿರುವೆ? ಮುಂತಾದುವುಗಳು ಯಾವಾಗಲೂ ಮತ್ತೆ ಮತ್ತೆ ಎದುರಾಗುತ್ತವೆ. ಉಪನ್ಯಾಸವನ್ನು ಭಾಷಣಕಾರ ಸಂಪೂರ್ಣ ಅನುಸರಿಸದೆ ಇದ್ದರೂ, ಅವರ ಪ್ರಕಾರ “ನಾವೆಲ್ಲರೂ ದಿವ್ಯರು” ಎಂಬ ನಂಬಿಕೆಯೇ ಹಿಂದೂ ಧರ್ಮದ ತಳಹದಿ ಎಂದು ಹೇಳಿದರೆ ಸಾಕು. ಪ್ರತಿಯೊಬ್ಬನಲ್ಲಿಯೂ ಒಂದು ಪ್ರಜ್ಞಾಪೂರ್ವಕ ಚೇತನವಿರುತ್ತದೆ; ನಿರಪೇಕ್ಷದ ಭಾಗವಾಗಿರುವ ಈ ಚೇತನವೇ ದೇಹವನ್ನೂ ಮನಸ್ಸನ್ನೂ ಹಿಡಿದಿಟ್ಟುಕೊಂಡಿರುವುದು. ಭಾಷಣಕಾರರು ವಿಜ್ಞಾನದ ಆಕ್ರಮಣದೆದುರು ಧರ್ಮವನ್ನು ಬಹು ಸಮರ್ಥವಾಗಿ ಪ್ರತಿಪಾದಿಸಿದರು. ವಿಜ್ಞಾನ ಪಂಚೇಂದ್ರಿಯಗಳನ್ನು ಮಾತ್ರ ಬಳಸುತ್ತಿದೆ; ಅವುಗಳಿಂದ ಯಾವುದನ್ನು ಸಾಧಿಸಲಾಗುವುದಿಲ್ಲವೋ ಅದರ ಅಸ್ತಿತ್ವವನ್ನೇ ಅನುಮಾನಿಸಿ ಬಿಟ್ಟುಬಿಡುತ್ತಾರೆ. ಇಂದ್ರಿಯಗಳು ಐದು ಮಾತ್ರ ಎಂದು ವಿಜ್ಞಾನಕ್ಕೆ ಗೊತ್ತೆ? ಈ ಪಂಚೇಂದ್ರಿಯಗಳನ್ನು ಮೀರಿದ ಇಂದ್ರಿಯವೊಂದಿದೆ, ಅದರ ಮೂಲಕ ಮಾನವನು ಆಧ್ಯಾತ್ಮಿಕ ಸತ್ಯಗಳನ್ನು ಅರಿತುಕೊಳ್ಳುತ್ತಾನೆ ಎಂದು ಭಾಷಣಕಾರರುವಾದಿ ಸಿದರು. ಈ ದಿವ್ಯಜ್ಞಾನಕ್ಕೆ ಹಿಂದೂಗಳು ಬಳಸಿರುವ ಪದವೇ “ವೇದ”. ಆದ್ದರಿಂದ “ವೇದಗಳು” ಎಂದರೆ ದಿವ್ಯಜ್ಞಾನದ ಅಭಿವ್ಯಕ್ತಿಗಳು. ಇಂಥ ಬರಹಗಳು ಹಿಂದೂಗಳಿಗೆ ಮಾತ್ರವೇ ಅಲ್ಲ, ಎಲ್ಲರಿಗೂ ಸಲ್ಲುವಂಥವು; ಏಕೆಂದರೆ ಎಲ್ಲ ಧರ್ಮಗಳೂ ಸತ್ಯ, ಎಂದರು ಭಾಷಣಕಾರರು.

ಈ “ದಿವ್ಯಜ್ಞಾನ”ಗಳು ವಸ್ತು ಪ್ರಪಂಚದ ಸಂಗತಿಗಳನ್ನು ಹೇಳತೊಡಗಿದಾಗ, ವಿಜ್ಞಾನ ಕ್ಷೇತ್ರವನ್ನು ಪ್ರವೇಶಿಸುತ್ತವೆ, ಹಾಗಾಗಿ ಅವು ಸ್ವೀಕಾರಾರ್ಹವಲ್ಲ. ಮೋಸಸ್ ದೇವರಿಚ್ಛೆಯ ದಿವ್ಯಜ್ಞಾನವನ್ನು ಕೊಟ್ಟನು ಎಂಬ ಕಾರಣಕ್ಕಾಗಿ ಮೋಸಸ್ ಬರೆದಿದ್ದೆಲ್ಲವೂ ಸತ್ಯವಾಗಿರಲೇಬೇಕು ಎಂದು ಪ್ರಾಚೀನ ಮೂಢನಂಬಿಕೆಯಾಗಿತ್ತು. ಆಧುನಿಕಯುಗದ ಮೂಢನಂಬಿಕೆ ಎಂದರೆ ಮೋಸಸ್ ಬರೆದಿದ್ದರಲ್ಲಿ ದೋಷಗಳಿರುವ ಕಾರಣ, ಮೋಸಸ್ ಬರೆದಿದ್ದು ಯಾವುದೂ ಸತ್ಯವಲ್ಲ ಎನ್ನುವುದು. ಮೋಸಸ್ ಜಗನ್ನಿಯಾಮಕನ ಕೋಷ್ಟಕ ಬರೆದಾಗ ಸ್ಫೂರ್ತಿಗೊಂಡಿದ್ದ. ಸೃಷ್ಟಿಯ ಬಗ್ಗೆ ಅವನು ಹೇಳಿದಾಗ, ಅದು ಕೇವಲ ಮೋಸಸ್ ಎಂಬ ಯಹೂದಿಯ ಊಹೆ ಮಾತ್ರವಾಗಿತ್ತು.

ಭಾಷಣಕಾರರು ಹಿಂದೂಗಳನ್ನು ಕ್ರೈಸ್ತರನ್ನಾಗಿ ಪರಿವರ್ತಿಸುವ ಪ್ರಯತ್ನಗಳ ನ್ನಾಗಲಿ ಅದರ ವಿಲೋಮ ಪ್ರಯತ್ನಗಳನ್ನಾಗಲಿ ಒಪ್ಪುವ ಹಾಗೆ ತೋರಲಿಲ್ಲ. ಅಂಥವ ರನ್ನು ಅವರು ಧರ್ಮಭ್ರಷ್ಟರು ಎಂದರು. ಎಲ್ಲ ಧರ್ಮಗಳೂ ಒಂದೇ ಆದ್ದರಿಂದ, ಅಂತಹ ಧರ್ಮಭ್ರಷ್ಟತನದಿಂದ ಯಾವ ಒಳ್ಳೆಯದೂ ಆಗಲಾರದು. ಭಾಷಣಕಾರರು ಹೇಳಿದಂತೆ, ಹಿಂದೂ ಧರ್ಮವು ಯಾವ ಮತಶ್ರದ್ಧೆಯನ್ನೂ ವಿರೋಧಿಸದು; ಬದಲಿಗೆ ಅದು ಎಲ್ಲವನ್ನೂ ಒಳಗೊಳ್ಳುವುದು. ಇತರ ಧರ್ಮಗಳ ಬಗ್ಗೆ ಸಹಿಷ್ಣುತೆಯ ಬಗ್ಗೆ ಹೇಳುವು ದಾದರೆ, ಹಿಂದೂವಿನ ಭಾಷೆಯಲ್ಲಿ ಇಂಗ್ಲಿಷಿನ ‘ಜ್ಞಿಠಿಟ್ಝಛ್ಟಿಚ್ಞ್ಚಛಿ ‘ಎಂಬುದಕ್ಕೆ ಸಮಾನ ಪದವೇ ಇಲ್ಲ. ಆ ಭಾಷೆಯಲ್ಲಿ ಧರ್ಮವೆಂಬುದಕ್ಕೆ ಒಂದು ಪದವಿದೆ, ಮತ ಎನ್ನುವುದಕ್ಕೆ ಒಂದು ಪದವಿದೆ. ಮೊದಲನೆಯದರಲ್ಲೇ ಎಲ್ಲಾ ಮತಶ್ರದ್ಧೆಗಳೂ ಬರುತ್ತವೆ. ಎರಡನೆಯದರ ಒಂದು ಕಲ್ಪನೆ ಮಾಡಿಕೊಡುವುದಕ್ಕೆ ಭಾಷಣಕಾರರು ತಾನು ವಾಸಿಸುವ ಬಾವಿಯ ಹೊರಗಡೆ ಇನ್ನಾವ ಪ್ರಪಂಚವೂ ಇಲ್ಲ ಎಂದುಕೊಂಡಿದ್ದ ಕಪ್ಪೆಯ ಕಥೆಯನ್ನು ಹೇಳಿದರು.

ಭಾಷಣಕಾರರು ಅಂತರಂಗದ ದಿವ್ಯತೆಯನ್ನು ಬೆಳೆಸಿಕೊಳ್ಳಿ, “ಅರ್ಥಹೀನ” ಮತಪಂಥಗಳನ್ನು ತ್ಯಜಿಸಿ ಎಂದು ತಮ್ಮ ಶ್ರೋತೃಗಳನ್ನು ಒತ್ತಾಯಿಸಿದರು.

ಭಾಷಣಕಾರರು ಬಹು ಸಮರ್ಥರೂ, ಸಭ್ಯರೂ, ಶಕ್ತಿಶಾಲಿವಾಗ್ಮಿಯೂ ಆಗಿದ್ದಾರೆ. ಅವರ ಇಂಗ್ಲಿಷ್ ಪ್ರಭುತ್ವ ಪರಿಪೂರ್ಣವಾದದ್ದು - ವಿದೇಶೀ ಉಚ್ಚಾರಣದೋಷ ಅಲ್ಪವಾಗಿತ್ತು. ಸಭಿಕರು ಅತ್ಯಂತ ಗಮನವಿಟ್ಟು ಭಾಷಣವನ್ನು ಕೇಳಿದರು. ಅನಂತರ, ಸಭಿಕರಲ್ಲಿ ಅದಕ್ಕಾಗಿಯೇ ಕುಳಿತಿದ್ದ ಕೆಲವರ ಪ್ರಶ್ನೆಗಳನ್ನು ಉತ್ತರಿಸಲು ಒಪ್ಪಿಕೊಂಡರು. ಯಾವುದೇ ಪ್ರಾಣಿಯ ಜೀವ ತೆಗೆಯುವುದನ್ನು ಹಿಂದೂಗಳು ಸಂಪೂರ್ಣ ವಿರೋಧಿಸುತ್ತಾರೆ ಎಂದು ಉತ್ತರಿಸಿದರು. ಪವಿತ್ರ ಗೋ ಪೂಜೆಯನ್ನು ಅವರು ಒಪ್ಪಿಕೊಂಡರು. ನಮ್ಮ ಚರ್ಚ್ ಸಂಘಟನೆಗಳಿಗೆ ಉತ್ತರವಾಗಿ ಹಿಂದೂಗಳಲ್ಲಿ ಏನೂ ಇಲ್ಲವೆಂದೂ ಅವರು ಹೇಳಿದರು. ಅವರಿಗೆ ಅವರೇ ಪಾದ್ರಿ, ಬಿಷಪ್ ಹಾಗೂ ಪೋಪ್ ಎಲ್ಲವೂ ಆಗಿರುವರು....

\begin{center}
\textbf{ಭಾರತದಿಂದ ಒಂದು ಸಂದೇಶ}\footnote{\enginline{1. New Discoveries, Vol. 1. pp. 204-206}}
\end{center}

\begin{center}
ಪ್ರಖ್ಯಾತ ಹಿಂದೂ ಸಂನ್ಯಾಸಿ ಹಾಗೂ ವಿದ್ವಾಂಸ, \enginline{Vive Kananda}ದೆ ಮೋಯ್ನ್ಸ್ನಲ್ಲಿ ಪ್ರತ್ಯಕ್ಷ \\(ದಿ ಅಯೋವಾ ಸ್ಟೇಟ್ ರಿಜಿಸ್ಟರ್, ೨೮ ನವೆಂಬರ್ ೧೮೯೩)\\ಮೂವತ್ತು ವರ್ಷದ ತರುಣನ ಚುರುಕಾದ ಉದಾರಬುದ್ಧಿ ಮತ್ತು ನಿಜಹೃದಯ
\end{center}

ದೆ ಮೋಯ್ನ್ಸ್ ನಗರದ ಜನರು ನೆನ್ನೆ ಪ್ರಖ್ಯಾತ ಹಿಂದೂ ಸಂನ್ಯಾಸಿ ಸ್ವಾಮಿ ವಿವೇಕಾ ನಂದ ಅವರಿಂದ ಪೌರಸ್ತ್ಯ ಜೀವನ ಮತ್ತು ಚಿಂತನೆಯ ಒಂದು ಅತ್ಯುತ್ತಮ ಮಿನುಗು ನೋಟವನ್ನು ಪಡೆದರು. ಈ ಬೇಸಿಗೆಯಲ್ಲಿ ನಡೆದ ಚಿಕಾಗೋ ಸರ್ವಧರ್ಮಸಮ್ಮೇಳನದ ಮಹಾಧಿವೇಶನದಲ್ಲಿ ಕೇಂದ್ರವ್ಯಕ್ತಿಯಾಗಿದ್ದು, ಈ ದೇಶದ ಹಲವಾರು ಮಹಾಚೇತನಗಳೊಂದಿಗೆ ತಮಗೂ ತಮ್ಮ ಜನರಿಗೂ ಗೌರವ ಹೆಚ್ಚುವಂತೆ ಒಡನಾಡಿದ ಅವರು, ತಮ್ಮ ಶ್ರೋತೃಗಳಿಗೆ - ಅದರಲ್ಲೂ ಡಾ. ಬ್ರೀಡನ್ ಅವರ ನಿವಾಸದಲ್ಲಿ ತಮ್ಮನ್ನು ಭೇಟಿಯಾ ದವರಿಗೆ - ಹೊಸ ಚಿಂತನೆಯೊಂದನ್ನು ಕೊಟ್ಟರು. ಬೇರೆಯೇ ಪರಿಸರದ, ಬೇರೆಯೇ ಶಿಕ್ಷಣದ, ಬೇರೆಯೇ ರೂಢಿಸಂಪ್ರದಾಯಗಳನ್ನುಳ್ಳ ಬೇರೆಯೇ ಜನರ ಸಾಗರದಾಚೆಯ ಸಂದೇಶವಾಗಿದ್ದರೂ, ಸಂನ್ಯಾಸಿಯು ಹೇಳುವಂತೆ, ಎಲ್ಲ ಧರ್ಮಗಳಲ್ಲೂ ಮೂಲ ತತ್ತ್ವಗಳು ಒಂದೇ ಆಗಿರುವುವು. ಎಲ್ಲ ಧರ್ಮಗಳಲ್ಲಿಯೂ ಒಳ್ಳೆಯ ಅಂಶಗಳಿವೆ ಎನ್ನುವುದು ಅವರ ಸಿದ್ಧಾಂತ; ಅದನ್ನವರು ಬಹು ಸಮರ್ಥವಾಗಿ ಬೋಧಿಸುತ್ತಾರೆ....

ನೆನ್ನೆ ಮಧ್ಯಾಹ್ನದ ನಂತರ ಅವರು ದೆ ಮೋಯ್ನ್ಸ್ ನಗರದ ತೀಕ್ಷ್ಣಮತಿಗಳಾದ ಮಹಿಳೆಯರನ್ನು ಬಹುಸಂಖ್ಯೆಯಲ್ಲಿ ಭೇಟಿಯಾದರು. ವಿಭಿನ್ನ ಸಾಹಿತ್ಯಸಂಘಗಳ ಸದಸ್ಯೆರಾದ ಇವರುಗಳು ಮಿಸೆಸ್ ಹೆಚ್.ಓ.ಬ್ರೀಡನ್ ಅವರ ೧೩೧೮ ವುಡ್ಲ್ಯಾಂಡ್ ಅವೆನ್ಯೂ\footnote{1. ಒಂದು ಅನೌಪಚಾರಿಕ ಸಂಭಾಷಣೆಯಾದ ಇದರ ಪದಶಃ ವರದಿ ಲಭ್ಯವಿಲ್ಲ.}ದಲ್ಲಿರುವ ನಿವಾಸದಲ್ಲಿ ಆಮಂತ್ರಿತರಾಗಿದ್ದರು. ತಮ್ಮ ಧರ್ಮ, ಕ್ರೈಸ್ತ ಧರ್ಮದ ಬಗ್ಗೆ ತಮ್ಮ ದೃಷ್ಟಿ (ಇದರಲ್ಲಿ ಅವರು ಹೃತ್ಪೂರ್ವಕ ಏಕಾಭಿಪ್ರಾಯವುಳ್ಳವರು) ಹಾಗೂ ತಮ್ಮ ಜನರ ರೂಢಿಯ ನಡಾವಳಿಗಳು, ಸಂಪ್ರದಾಯಗಳು ಇತ್ಯಾದಿಗಳನ್ನು ಕುರಿತು ಎರಡು - ಮೂರು ಗಂಟೆಗಳ ಕಾಲ ಅವರನ್ನುದ್ದೇಶಿಸಿ ಮಾತನಾಡಿದ ವಿವೇಕಾ ನಂದರು ಒತ್ತಿ ಹೇಳುವುದೇನೆಂದರೆ, ಅಮೆರಿಕಾದಲ್ಲಿನ ಕೇಡುಗಳಿಗಾಗಿ ಕ್ರೈಸ್ತಧರ್ಮವನ್ನು ತೆಗಳಬೇಕಾದದ್ದಕ್ಕಿಂತ ಹೆಚ್ಚಾಗಿ ಭಾರತದಲ್ಲಿರುವ ಕೇಡುಗಳಿಗಾಗಿ ಹಿಂದೂ ಧರ್ಮವನ್ನು ತೆಗಳತಕ್ಕದ್ದಲ್ಲ. ಅನುಯಾಯಿಗಳಾದವರು ಕೈಕೊಂಡು ಸಾಧಿಸಿದ ಅಚ್ಚರಿಯ ಯಶಸ್ಸನ್ನು ಕ್ರೈಸ್ತಧರ್ಮದ ಯಶಸ್ಸೆನ್ನುವುದು ಅಸಂಬದ್ಧ ಎಂದೂ ಅವರು ಒತ್ತಿ ಹೇಳುತ್ತಾರೆ. ಬೈಬಲ್ನಲ್ಲಿರುವ ಉದಾತ್ತ ಸಂಗತಿಗಳನ್ನು ಪ್ರಶಂಸಿಸುವ ಅವರು, ಮೋಸಸ್ ಪ್ರಪಂಚದ ಸೃಷ್ಟಿಯನ್ನು ಕುರಿತು ಹೇಳುವಾಗ ಅವನು ಯಹೂದಿ ಮೋಸಸ್ ಮಾತ್ರವೇ ಆಗಿದ್ದ, ಅದಕ್ಕಿಂತ ಹೆಚ್ಚೇನಲ್ಲ ಎನ್ನುತ್ತಾರೆ.

ಅವರ ಕಡೆಯ ಈ ಸಹಾನುಭೂತಿಯ ದೃಷ್ಟಿ ಅತ್ಯಂತ ಪ್ರಯೋಜನಕಾರಿಯೂ ಕುತೂಹಲ ಕೆರಳಿಸುವಂಥದೂ ಆಗಿದೆ. ವಿವೇಕಾನಂದರು ಬಳಸುವುದು ಅತ್ಯಂತ ಶುದ್ಧವಾದ ಇಂಗ್ಲಿಷ್ - ಏಕೆಂದರೆ ಅವರು ಕೊಲ್ಕತ್ತದ ಇಂಗ್ಲಿಷ್ ವಿಶ್ವವಿದ್ಯಾನಿಲಯದಲ್ಲಿ ಶಿಕ್ಷಣ ಪಡೆದವರು. ಅಮೆರಿಕಾದ ಸ್ತ್ರೀಯರನ್ನು ಅವರು ತುಂಬ ಉತ್ಸಾಹದಿಂದ ಪ್ರಶಂಸಿಸುತ್ತಾರೆ. ಕಳೆದ ರಾತ್ರಿ ‘ದಿ ರಿಜಿಸ್ಟರ್’ ಪತ್ರಿಕೆಯ ವರದಿಗಾರರಿಗೆ (ಸ್ವಾಮಿಗಳು) ನಿಮ್ಮ ಸ್ತ್ರೀಯರಿಲ್ಲದೆ ಹೋಗಿದ್ದರೆ ನನಗೆ ಏನಾಗುತ್ತಿತ್ತೋ ತಿಳಿಯದು. ಅವರು ನನಗೆ ಮನ್ನಣೆ ಕೊಟ್ಟರು, ಯೋಗಕ್ಷೇಮ ನೋಡಿಕೊಂಡರು, ಆವಶ್ಯಕವಾದ ಎಲ್ಲ ಏರ್ಪಾಡುಗಳನ್ನೂ ನನಗಾಗಿ ಮಾಡಿದರು. ಪ್ರಪಂಚದಲ್ಲಿರುವ ಅತ್ಯುತ್ತಮ ಸ್ತ್ರೀಯರಿವರು. ನನಗೆ ಅಷ್ಟೊಂದು ಸೌಜನ್ಯವನ್ನು ತೋರಿಸಿರುವರು ಎಂದು ಕೃತಜ್ಞತಾಪೂರ್ವಕವಾದ ಮಂದಹಾಸ ದೊಂದಿಗೆ ಹೇಳಿದರು....

\begin{center}
\textbf{ಪುನರ್ಜನ್ಮ\supskpt{\footnote{\enginline{1. New Discoveries, Vol. 1, pp. 206-207}}}}
\end{center}

\begin{center}
(ಡೈಲಿ ಅಯೋವಾ ಕ್ಯಾಪಿಟಲ್, ೨೯ ನವೆಂಬರ್ ೧೮೯೩)
\end{center}

ಕಳೆದ ರಾತ್ರಿ ಸ್ವಾಮಿ ವಿವೇಕಾನಂದರು ಪುನರ್ಜನ್ಮವನ್ನು ಕುರಿತು ಮಾತನಾಡಿದರು\footnote{2. ಇದರ ಪದಶಃ ವರದಿ ಲಭ್ಯವಿಲ್ಲ.}. ಅನಾದಿಯಿಂದಲೂ ಸೃಷ್ಟಿಯು ದೇವರೊಂದಿಗೆ ಸಮಕಾಲಿನಕವಾಗಿಯೇ ಅಸ್ತಿತ್ವದಲ್ಲಿದೆ; ಹೊಸದಾಗಿ ಯಾವ ಸೃಷ್ಟಿಯೂ ಆಗಿಲ್ಲ ಎಂಬ ವಾಸ್ತವವನ್ನು ಅದು ಆಧರಿಸಿದೆ ಎಂದು ಅವರುವಾದಿಸಿದರು. ಸತ್ತ ಮೇಲೆ ಆತ್ಮಗಳು ಮೊದಲಿದ್ದುದಕ್ಕಿಂತ ಒಳ್ಳೆಯ ಅಥವಾ ಕೆಟ್ಟ ಶರೀರಗಳನ್ನು - ತಮ್ಮನ್ನು ತಾವು ಯಾವುದಕ್ಕೆ ಯೋಗ್ಯರ ನ್ನಾಗಿಸಿಕೊಂಡಿರುವುವೋ ಅಂಥವುಗಳನ್ನು - ಆಯ್ದುಕೊಳ್ಳುತ್ತವೆ. ಭಾಷಣಕಾರರು ವಂದನಾರ್ಪಣೆಯ ದಿನ \enginline{(Thanksgiving Day)} ದಂದು ಸಂಜೆ ಇದೇ ಸ್ಥಳದಲ್ಲಿ ಭಾರತದ ರೂಢಿಗಳು ಹಾಗೂ ನಡಾವಳಿಗಳ ಮೇಲೆ ಮಾತನಾಡುತ್ತಾರೆ.

\begin{center}
\textbf{ಒಂದು ಬೌದ್ಧಿಕ ಹಬ್ಬ}\footnote{\enginline{1. New Discoveries, Vol. 1, pp. 208}}
\end{center}

(ಅಯೋವಾ ಸ್ಟೇಟ್ ರಿಜಿಸ್ಟರ್, ೩೦ ನವೆಂಬರ್ ೧೮೯೩)

ಪ್ರಖ್ಯಾತ ಹಿಂದೂ ಸಂನ್ಯಾಸಿ ವಿವೇಕಾನಂದರು ನೆನ್ನೆ ಪ್ರಾರಂಭಿಸಿದ ಚರ್ಚೆಯು ಬೌದ್ಧಿಕ ವಲಯಗಳಲ್ಲಿ ಬಹು ಮೆಚ್ಚುಗೆಯ ವಿಷಯವಾಗಿದ್ದಿತು\footnote{4. ಅಯೋವಾದ ದೆ ಮೋಯ್ನ್ಸ್ ನಗರದಲ್ಲಿ ನವೆಂಬರ್ ೨೭ರಿಂದ ಡಿಸೆಂಬರ್ ೧ರವರೆಗಿನ ಸ್ವಾಮಿಗಳ ಸಂಭಾಷಣೆಗಳ, ಉಪನ್ಯಾಸಗಳ, ಚರ್ಚೆಗಳ ಪದಶಃ ವರದಿ ಲಭ್ಯವಿಲ್ಲ.}. ಅದರಲ್ಲೂ ಭಾರತದಲ್ಲಿನ ಅಮೆರಿಕನ್ ಪ್ರಚಾರಕರ ಮೇಲಿನ ಅವರ ಟೀಕೆಗಳು ಮತ್ತು ಅವರ ಜನರ ಧರ್ಮ - ನೈತಿಕತೆಗಳನ್ನು ಶಕ್ತಿಯುತವಾಗಿ ಎತ್ತಿ ಹಿಡಿದ ರೀತಿ. ಭಾರತದ ಜನರಿಗೆ ಬೇಕಾದುದು ಧರ್ಮವಲ್ಲ, ಆದರೆ ಭಾರತವನ್ನು ಆಕ್ರಮಿಸಿಕೊಂಡಿರುವ ಬ್ರಿಟಿಷರಿಗೆ ಸಾಟಿಯಾಗುವಂತೆ ಬದುಕಿನ ವಾಸ್ತವಾಂಶಗಳಲ್ಲಿ ತರಬೇತಿ ಎಂಬುದು ಅವರ ನಿಲುವು. ವಿವೇಕಾನಂದರು ನೆನ್ನೆ ಮಿ. ಎಫ್. ಡಬ್ಲೂ. ಲೆಹ್ಮನ್ ಮತ್ತು ಓ.ಹೆಚ್. ಪರ್ಕಿನ್ಸ್ ಅವರುಗಳ ಅತಿಥಿಯಾಗಿದ್ದರು; ಅವರುಗಳ ಜೊತೆಯಲ್ಲಿ ಸ್ಟೇಟ್ ಭವನಕ್ಕೆ ಭೇಟಿ ಕೊಟ್ಟ ಅವರು ಅದನ್ನು ತುಂಬ ಪ್ರಶಂಸಿಸಿದರು. ಅಲ್ಲಿ ಅವರು ನೋಡಿದ ಅಮೆರಿಕನ್ ಇಂಡಿಯನ್ಗಳ ಭಾವಚಿತ್ರಗಳಲ್ಲಿ ವಿಶೇಷ ಆಸಕ್ತಿಯನ್ನು ತೋರಿಸಿದರು.....

\begin{center}
\textbf{ಒಂದು ಪ್ರಾರ್ಥನಾಸಭೆ}\footnote{\enginline{1. New Discoveries, Vol. 1, pp. 207}}
\end{center}

\begin{center}
(ದೆ ಮೋಯ್ನ್ಸ್ ಡೈಲಿ ನ್ಯೂಸ್, ೩೦ ನವೆಂಬರ್ ೧೮೯೩)
\end{center}

ಬುಧವಾರ ವಿವೇಕಾನಂದರು ಒಂದು ಪ್ರಾರ್ಥನಾಸಭೆಯಲ್ಲಿ ಭಾಗವಹಿಸಿದ್ದರು; ಇಬ್ಬರು ತರುಣಿಯರಿಗೆ ದೀಕ್ಷಾಸ್ನಾನ ಮಾಡಿಸಿದುದನ್ನು ವೀಕ್ಷಿಸಿದರು. ಈ ಕಾರ್ಯ ಕ್ರಮವು ಅವರ ಮೆಚ್ಚುಗೆಗೆ ಪಾತ್ರವಾಯಿತು. ಅವರೆಂದರು:

ಭಾವವು ಉದಾತ್ತವಾಗಿದೆ; ಆಚರಣೆಯು ಸುಂದರವಾಗಿದೆ. ಪಾದ್ರಿಗಳ ಪ್ರಾಮಾಣಿ ಕತೆ, ತಾವು ಹೇಳುವುದರಲ್ಲಿ ಅವರಿಗಿರುವ ನಂಬಿಕೆ ಹಾಗೂ ಮಾಡುವ ಕಾರ್ಯದಲ್ಲಿನ ಶ್ರದ್ಧೆ ಇವು ವಿಶೇಷವಾಗಿ ಮೆಚ್ಚುವಂಥವು.

\begin{center}
\textbf{ಅಮೆರಿಕನ್ ಸ್ತ್ರೀಯರನ್ನು ಕುರಿತು}\footnote{\enginline{2. New Discoveries, Vol. 1, pp. 208}}
\end{center}

\begin{center}
(ಡೈಲಿ ಅಯೋವಾ ಕ್ಯಾಪಿಟಲ್, ೩೦ ನವೆಂಬರ್ ೧೮೯೩)
\end{center}

ಈಗ ಪ್ರಸಿದ್ಧರಾಗಿರುವ ಹಿಂದೂ ಸಂನ್ಯಾಸಿ ಸ್ವಾಮಿ ವಿವೇಕಾನಂದರು ಈ ರಾತ್ರಿ ಕೊನೆಯ ಬಾರಿಗೆ ದೆ ಮೋಯ್ನ್ಸ್ ನಗರದಲ್ಲಿ ಭಾಷಣ ಮಾಡುವರು. “ಭಾರತದಲ್ಲಿನ ಜನಜೀವನ” (“ಭಾರತದಲ್ಲಿನ ರೂಢಿಗಳು ಹಾಗೂ ರೀತಿರಿವಾಜುಗಳು”) ಎಂಬ ಅತ್ಯಂತ ಆಸಕ್ತಿಯ ವಿಷಯ ಅವರದು. ಖ್ಯಾತಿವೆತ್ತ ಈ ಹಿಂದೂವು ಸುಮಾರು ಮೂವತ್ತು ವಯಸ್ಸಿನ ಉಜ್ವಲ ತೇಜಸ್ವಿ. ಅಮೆರಿಕನ್ ಸ್ತ್ರೀಯರು ಶ್ರೇಷ್ಠರೆಂದೂ, ಆದರೆ ಅಮೆರಿಕನ್ ಪುರುಷರೆಲ್ಲ ತೀರ ವ್ಯಾವಹಾರಿಕರು ಎಂದೂ ಅವರೆನ್ನುತ್ತಾರೆ.

\begin{center}
\textbf{ಬ್ರಹ್ಮಸಮಾಜವನ್ನು ಕುರಿತು}\footnote{3. Ibid., p. 215}
\end{center}

\begin{center}
(ಅಯೋವಾ ಸ್ಟೇಟ್ ರಿಜಿಸ್ಟರ್, ೩೧ ನವೆಂಬರ್ ೧೮೯೩)
\end{center}

ದೆ ಮೋಯ್ನ್ಸ್ ನಗರವನ್ನು ಬಿಡುವುದಕ್ಕೆ ಮುಂಚೆ ವಿವೇಕಾನಂದರು ಬ್ರಹ್ಮ ಸಮಾಜವನ್ನು, ಭಾರತದಲ್ಲಿ ಅದು ವಿಶೇಷವಾಗಿ ಸ್ತ್ರೀಯರಿಗಾಗಿ ಮಾಡುತ್ತಿರುವ ಕೆಲಸವನ್ನು, ಮತ್ತು ಈ ದೇಶಕ್ಕೆ ಬಂದಿರುವ ಅದರ ಪ್ರತಿನಿಧಿಯನ್ನು ಪ್ರಶಂಸಿಸುವ ಒಂದು ಸೌಹಾರ್ದದ ಮಾತನ್ನಾಡಿದರು. ನಗರದ ಬೌದ್ಧಿಕ ಕೇಂದ್ರಗಳನ್ನು ಆಳವಾಗಿ ಅಲ್ಲಾ ಡಿಸಿಬಿಟ್ಟಿರುವ ಹಾಗೂ ಸ್ವಾರಸ್ಯಕರವಾದ ಧಾರ್ಮಿಕ ಚರ್ಚೆಯನ್ನು ಮೊದಲುಮಾಡಿರುವ ವಿವೇಕಾನಂದರ ಭೇಟಿಯು, ಈಗ ಆಗಮಿಸಿರುವ ಪ್ರಾಚ್ಯ ಭೇಟಿಕಾರ (ನಗರ್ ಕರ್)ರಿಗೆ ಮಾರ್ಗವನ್ನು ಸುಗಮ ಗೊಳಿಸಿರುವುದಲ್ಲದೆ, ಅವರು ಏನು ಹೇಳಲಿರು ವರೋ ಅದರ ಕಡೆಗೆ ಸಾರ್ವಜನಿಕ ಆಸಕ್ತಿಯನ್ನು ಹೆಚ್ಚಿಸಿದೆ.

\begin{center}
\textbf{ಚತುರೋಕ್ತಿಯ ಹಿಂದೂ\supskpt{\footnote{\enginline{1. New Discoveries, Vol. 1, pp. 216-217}}}}
\end{center}

\begin{center}
(ಮಿನ್ನಿಯಾಪೋಲಿಸ್ ಜರ್ನಲ್, ೧೫ ಡಿಸೆಂಬರ್ ೧೮೯೩)
\end{center}

\begin{center}
ಇನ್ನೊಂದು ಬೃಹತ್ ಸಭೆಯನ್ನು ರಂಜಿಸಿದ ಸ್ವಾಮಿ ವಿವೇಕಾನಂದರು
\end{center}

ಭಾರತದ ಸ್ವಾಮಿ ವಿವೇಕಾನಂದರ ಮಾತನ್ನು ಕೇಳುವ ಉದ್ದೇಶದಿಂದ ಕಳೆದ ಸಂಜೆ ಯೂನಿಟೇರಿಯನ್ ಚರ್ಚ್ನಲ್ಲಿ ಭಾರೀ ಜನಜಂಗುಳಿ ನೆರೆದಿತ್ತು. ಆ ದೇಶದ ರೀತಿ - ರಿವಾಜುಗಳನ್ನು ವಿವರಿಸಿದ\footnote{2. “ಭಾರತದ ರೀತಿ - ರಿವಾಜುಗಳು” ಎಂಬ ಈ ಉಪನ್ಯಾಸದ ಪದಶಃ ವರದಿ ಲಭ್ಯವಿಲ್ಲ. ಮುಖ್ಯಾಂಶಗಳಿಗಾಗಿ ಇದರ ಮುಂದಿನ ಅಮೆರಿಕನ್ ಪತ್ರಿಕಾವರದಿಯನ್ನು ನೋಡಿ.} ಆ ಬ್ರಾಹ್ಮಣ, ತಮ್ಮ ಉಪನ್ಯಾಸದಲ್ಲಿ ಸಂದರ್ಭೋಚಿತವಾಗಿ ಅಮೆರಿಕಾದ ಕೆಲವು ಅಹಿತಕರ ಸಂಗತಿಗಳನ್ನೂ ತೋರಿಸಿಕೊಟ್ಟರು. ಚತುರ ಹಾಸ್ಯದ ಅವರ ತತ್ಕ್ಷಣದ ಉತ್ತರಗಳು, ಚಮತ್ಕಾರದ ನುಡಿಗಳು ಸಭಿಕರ ಚಪ್ಪಾಳೆ ಗಿಟ್ಟಿಸಿಕೊಳ್ಳದ ಸಂದರ್ಭಗಳೇ ಅಪರೂಪ. ತಮ್ಮ ದೇಶದ ಜನರು ಪ್ರತಿಯೊಂದರಲ್ಲೂ ತಪ್ಪಿರುವವರೆಂದು ಒಪ್ಪಿಕೊಳ್ಳದ ಅವರು, ಅಮೆರಿಕನ್ನರು ಒಪ್ಪದ, ಆದರೂ ಒಪ್ಪವಾಗಿರಬಹುದಾದ, ಭಾರತದ “ವಿಚಿತ್ರ” ಸಂಗತಿಗಳು ಬೇಕಾದಷ್ಟಿವೆ ಎಂದರು. ಗಂಡ - ಹೆಂಡತಿ ಇಬ್ಬರೂ ತಮ್ಮ ತೊಂದರೆಗಳನ್ನು ಹೇಳಿಕೊಳ್ಳುವುದಕ್ಕಾಗಿ ಒಬ್ಬ ನ್ಯಾಯಾ ಧೀಶನ ಬಳಿಗೆ ಹೋಗುವುದನ್ನು ತಾವೆಂದೂ ಕಂಡಿಲ್ಲ ಎಂದರು. ತಾವು ಮುಂದೆ ಮದುವೆಯಾಗಲಿರುವವರು ಎಂಬ ಕಲ್ಪನೆಯೊಂದಿಗೆ ಬೆಳೆಯುವ ಅವರುಗಳು ಅಣ್ಣ ತಂಗಿಯರ ಹಾಗೆಯೇ ಪರಸ್ಪರರನ್ನು ಪ್ರೀತಿಸುವರು.

ಅವರು ತಮ್ಮ ದೇಶದ ಜನರ ರೀತಿ - ರಿವಾಜುಗಳನ್ನು, ದೇಗುಲಗಳನ್ನು, ಯಕ್ಷಿಣಿ ವಿದ್ಯೆ ಕಲಿತವರ ಕಲೆಯನ್ನು, ಪ್ರಾಚ್ಯ ದೇಶಗಳ ಇನ್ನಿತರ ವೈಚಿತ್ರ್ಯಗಳನ್ನು ಬಹು ಸ್ವಾರಸ್ಯವಾಗಿ ವಿವರಿಸಿದರು. ಉಪನ್ಯಾಸವಾದ ನಂತರ ಸಭಿಕರು ಅನೇಕಾನೇಕ ಪ್ರಶ್ನೆಗಳನ್ನು ಕೇಳಿದರು.

\begin{center}
\textbf{ಭಾರತದ ರೀತಿ - ರಿವಾಜುಗಳು\supskpt{\footnote{\enginline{3. New Discoveries, Vol. 1, pp. 217-219}}}}
\end{center}

\begin{center}
(ಮಿನ್ನಿಯಾಪೋಲಿಸ್ ಟ್ರಿಬ್ಯೂನ್, ೧೫ ಡಿಸೆಂಬರ್ ೧೮೯೩)
\end{center}

ಕಳೆದ ಸಂಜೆ ಯೂನಿಟೇರಿಯನ್ ಚರ್ಚ್ನಲ್ಲಿ ತಮ್ಮ ಎರಡನೆಯ ಶ್ರೋತೃ ವೃಂದವನ್ನು ಎದುರಿಸಿದ ಬ್ರಾಹ್ಮಣ ಪೂಜಾರಿ ಸ್ವಾಮಿ ವಿವೇಕಾನಂದರರನ್ನು ಕಿಕ್ಕಿರಿದ ಸಭೆಯು ಉತ್ಸಾಹದಿಂದ ಬರಮಾಡಿಕೊಂಡಿತು. ಎಲ್ಲ ವಿಚಾರಗಳಲ್ಲೂ ಆಕ್ರಮಣದ ಇಲ್ಲವೇ ರಕ್ಷಣಾತ್ಮಕ ಸಿದ್ಧ ನಿಲುವು ತಳೆಯುವ ವಿವೇಕಾನಂದರು ಬಹು ಉಜ್ವಲ ಚತುರೋಕ್ತಿಯ ಮಾತುಗಾರರು; ಭಾಷಣದಲ್ಲಿ ತರುವ ಅವರ ಹಾಸ್ಯ ವಿಡಂಬನೆಗಳು ಸಭಿಕರ ಮೇಲೆ ವ್ಯರ್ಥವಾದದ್ದೇ ಇಲ್ಲ. ಕಳೆದ ಸಂಜೆ ಅವರು ವಿಶ್ವ ವಿದ್ಯಾನಿಲಯದ ‘ಕಪ್ಪಕಪ್ಪ ಗಾಮ’ಗಳ ಆಶ್ರಯದಲ್ಲಿ ಮಾತನಾಡಿದರು; ಸಭೆಯ ಲ್ಲಿದ್ದ ಶ್ರದ್ಧಾವಂತ ಚಿಂತನೆಯ ಬಹುಸಂಖ್ಯಾತ ಸ್ತ್ರೀಪುರುಷರು ಭಾಷಣದ ಆಯ್ಕೆಯ ವಿಷಯವಾದ “ಭಾರತದ ರೀತಿ-ರಿವಾಜುಗಳ” ಬಗ್ಗೆ ಅರಿತುಕೊಂಡರು; ತುಷ್ಟರಾದರು.\footnote{1. ಇದರ ಬಗ್ಗೆ ಪದಶಃ ವರದಿ ಲಭ್ಯವಿಲ್ಲ. ಇನ್ನಿತರ ವಿವರಗಳಿಗಾಗಿ ‘ಚತುರೋಕ್ತಿಯ ಹಿಂದೂ’ ಎಂಬ ಅಮೆರಿಕ ವೃತ್ತ ಪತ್ರಿಕಾವರದಿಯನ್ನು ನೋಡಿ.}

ದೇಶೀಯ ಉಡುಪು ಧರಿಸಿರುವ ಕಾನಂದರು, ಹೆಚ್ಚು ಕಾಲ ಕೈಗಳನ್ನು ಬೆನ್ನ ಹಿಂದೆ ಕಟ್ಟಿಕೊಂಡೇ ಇರುತ್ತ, ಪುಟ್ಟ ವೇದಿಕೆಯ ಮೇಲೆಯೇ ಹಿಂದಕ್ಕೂ ಮುಂದಕ್ಕೂ ಶತಪಥ ಹಾಕುತ್ತ, ತಮ್ಮ ಮಾತು ಶೋತೃಗಳ ಮನಸ್ಸಿನಲ್ಲಿ ಆಳವಾಗಿ ಇಳಿಯಲಿ ಎಂಬಂತೆ ವಾಕ್ಯಗಳ ಮಧ್ಯೆ ದೀರ್ಘ ವಿರಾಮ ಕೊಡುತ್ತ, ಹೆಜ್ಜೆಹಾಕುತ್ತಲೇ ಮಾತನಾಡುವರು. ಹುಡುಗ ತನದ ಮನಸ್ಸುಗಳು ಕೆಲವು ಉಕ್ತಿಗಳನ್ನು ಮೆಚ್ಚದಿರುವಷ್ಟು ತೂಕವಾದ ಮಾತುಗಳೇ ನಲ್ಲ ಅವರವು; ಆದರೂ ಅವರು ಅತ್ಯಂತ ಗಂಭೀರ, ನಿರಾಡಂಬರ ಸತ್ಯವನ್ನೊಳ ಗೊಂಡೇ ತತ್ತ್ವವನ್ನೂ ಸಹ ವಿವರಿಸುವರು. ಭಾರತದ ರೀತಿ - ರಿವಾಜುಗಳನ್ನು, ಬದುಕನ್ನು ಸ್ತ್ರೀ ಪುರುಷರು ಹೇಗೆ ಹಂಚಿಕೊಳ್ಳುವರೆಂಬುದನ್ನು, ಸ್ತ್ರೀಯರ ಪಾವಿತ್ರ್ಯವನ್ನು, ಅವರಿ ಗಿರುವ ಗೌರವವನ್ನು, ಜೊತೆಗೇ ಅವರ ಅವನತಿಯನ್ನು ಕುರಿತು ಹೇಳುವರು; ಶಾಂತಿ ಸಮಾಧಾನಗಳಿರುವ ದಿನನಿತ್ಯದ ಬದುಕು, ಆದರೂ ಸ್ವಾತಂತ್ರ್ಯವಿಲ್ಲದಿರುವುದರಿಂದ ಅದು ನಿಜವಾದ ಬದುಕಲ್ಲವೆಂಬುದನ್ನು ವಿವರಿಸುವರು; ಭಾರತದ ಜನಸಂಖ್ಯೆಯ ಐದನೆಯ ಒಂದು ಭಾಗದಷ್ಟಿರುವ ಮುಸಲ್ಮಾನರ ಬಗ್ಗೆ ಮಾತನಾಡುವರು; ಆ ಅರವತ್ತೈದು ಮಿಲಿಯ ಎಂದರೆ ಸಂಯುಕ್ತ ಸಂಸ್ಥಾನಗಳ ಇಡಿಯ ಜನಸಂಖ್ಯೆಯಾಗುವುದು ಎನ್ನುವರು. ದೇವಾಲಯಗಳ ಔನ್ನತ್ಯವನ್ನು, ಭಾರತ ಜನಾಂಗದ ಜಿಪ್ಸಿಗಳೆನ್ನ ಬಹುದಾದ ಅಲೆಮಾರಿಗಳ ಯಕ್ಷಿಣೀವಿದ್ಯೆಯ ಕಲೆಯನ್ನು, ವಿವರಿಸುವರು; ಜನರು ಪ್ರಯಾಣ ಹೊರಡುವ ಮೊದಲು ಹೇಗೆ ನೀರುತುಂಬಿದ ಪಾತ್ರೆಯನ್ನು ಹೊಸ್ತಿಲ ಮೇಲಿಟ್ಟು ಸ್ವಲ್ಪಹೊತ್ತು ಕಾಯುವರು ಎನ್ನುತ್ತ ಅಲ್ಲಿಲ್ಲಿ ಮೂಢನಂಬಿಕೆಗಳನ್ನೂ ತರುವರು; ನೇಗಿಲುಹೊತ್ತ ರೈತನಿಗಿರುವ ತಾತ್ತ್ವಿಕ ಜ್ಞಾನ, ಆದರೂ, “ಸರ್ಕಾರಕ್ಕೆ ತೆರಿಗೆ ಕಟ್ಟುವುದು” ಮಾತ್ರವೇ ಅವನಿಗೆ ಗೊತ್ತಿರುವುದು, ಗಂಗಾನದಿಯ ಮೇಲಿರುವ ಹಿಂದೂವಿನ ಭಕ್ತಿ, ತಮಗೂ ಹೇಗೆ ಅವರು ಮೌನಶಾಂತದ ಬಲು ಉಪೇಕ್ಷೆಯೆಂಬಂತಹ ದನಿಯಲ್ಲಿ ನುಡಿಯುತ್ತ, ಅಮೆರಿಕನ್ ಕಾರ್ಯವಿಧಾನದ ಮೇಲೆ ಏನಾದರೊಂದು ಟೀಕೆಯನ್ನು ಉಸುರಿದಾಗ ಸಭೆಯ ತುಂಬ ಪುಂಖಾನು ಪುಂಖವಾಗಿ ನಗೆಯ ಅಲೆಗಳೇಳುವುವು; ಚಪ್ಪಾಳೆಯಿಂದ ಇಡೀ ಸಭೆ ನಡುಗುವುದು; ಮುಸಿನಗುತ್ತಲೇ ಜನರು ಅವರ ಕಟುನುಡಿಯನ್ನೊಪ್ಪಿಕೊಳ್ಳುವರು....

ಅವರ ಉಪನ್ಯಾಸ ಮುಗಿಯುತ್ತಿದ್ದಂತೆ ಯಾರೋ ಒಬ್ಬರು “ಪ್ರಚಾರಕರು ಯಾವ ವರ್ಗದ ಜನರನ್ನು ಬಳಿಸಾರಿ ಪರಿವರ್ತಿಸಿರುವರು?” ಎಂದು ಕೇಳಿದಾಗ, ತತ್ಕ್ಷಣವೇ ಅವರು “ಅದರ ಬಗ್ಗೆ ನಿಮಗೇ ಚೆನ್ನಾಗಿ ಗೊತ್ತು; ಅಮೆರಿಕನ್ ಜನರು ವರದಿಗಳನ್ನು ನೋಡುತ್ತಾರೆ, ಆದರೆ ನಾವು ನೋಡುವುದಿಲ್ಲ” ಎಂದು ಉತ್ತರಿಸಿದರು; ಪ್ರಶ್ನೆಯನ್ನು ಒಂದು ನಗೆಚಾಟಿಕೆಯನ್ನಾಗಿಸಿಬಿಟ್ಟರು. ಸಭೆ ಮತ್ತೊಮ್ಮೆ ಶಾಂತವಾಗುತ್ತಿದ್ದಂತೆ ಮೌನವಾಗಿ ತಮ್ಮ ಶತಪಥ ಮುಂದುವರೆಸುವರು. ಉಪನ್ಯಾಸವನ್ನು ಜನರು ತುಂಬ ಗಮನವಿಟ್ಟು ಕೇಳಿದರು; ಆ ನಂತರ ಪೂರಕವಾಗಿ ಸಭಿಕರಿಂದ ಅನೇಕ ಪ್ರಶ್ನೋತ್ತರಗಳಿದ್ದವು; ಸಭಿಕರು ಪ್ರಶ್ನೆ ಕೇಳುವುದನ್ನು ಅವರು ಸ್ವಾಗತಿಸಿದರು.

\begin{center}
\textbf{ಹಿಂದೂ ತತ್ತ್ವಶಾಸ್ತ್ರ\supskpt{\footnote{\enginline{3. New Discoveries, Vol. 1, pp. 366-369}}}}
\end{center}

\begin{center}
(ಡೆಟ್ರಾಯಿಟ್ ಟ್ರಿಬ್ಯೂನ್, ೧೮ ಫೆಬ್ರವರಿ ೧೮೯೪)
\end{center}

\begin{center}
ವಿವೇಕಾನಂದರಿಂದ ಅದರ ಇತ್ತೀಚಿನ ಅಭಿವ್ಯಕ್ತಿ
\end{center}

\begin{center}
ಅವರ ಸಂದೇಶ ಅಮೆರಿಕನ್ನರ ಗಂಭೀರ ಅವಗಾಹನೆಗೆ ಯೋಗ್ಯವಾಗಿದೆ.
\end{center}

ಪ್ರಖ್ಯಾತ ಕ್ರೈಸ್ತೇತರನಿಗೆ ತೃಪ್ತಿಯಿತ್ತಿರುವ ಸಂಯುಕ್ತ ಸಂಸ್ಥಾನಗಳಲ್ಲಿನ ಎರಡು ಸಂಗತಿಗಳು - ಪರಿಸರ ಜನಗಳಿಗೇನು ಮಾಡಬಲ್ಲದು - ಪ್ರಚಾರಕರಿಗೆ ಛೀಮಾರಿ.

ವಿದ್ಯಾವಂತ ಹಿಂದೂ ಸಂನ್ಯಾಸಿ ಸ್ವಾಮಿ ವಿವೇಕಾನಂದರ ಆಗಮನವು ಡೆಟ್ರಾಯಿಟ್ನ ಸುಸಂಸ್ಕೃತ ವಲಯಗಳಲ್ಲಿ ಹಿಂದೆಂದೂ ಇದ್ದಿರದಷ್ಟು ಭಾವೋದ್ರೇಕವನ್ನು ಉಂಟು ಮಾಡಿದೆ. ನಮ್ಮ ಭಾಷೆಯ ಮೇಲೆ ಅವರಿಗೆ ಇರುವ ಅಸಾಧಾರಣ ಪ್ರಭುತ್ವದಿಂದಾಗಿ, ಪ್ರಾಚ್ಯ ದೃಷ್ಟಿಕೋನದಿಂದ ನಮ್ಮನ್ನು ನಾವೇ ನೋಡಿಕೊಳ್ಳುವುದಕ್ಕೆ ಸಾಧ್ಯವಾಗಿದೆ; ಮತ್ತು ವಿಚಿತ್ರ ನಾಗರಿಕತೆಯವರೆಂದು, ವಿಚಿತ್ರ ಜೀವನದೃಷ್ಟಿಯವರೆಂದು ನಾವು ಅಷ್ಟೊಂದು ಕೇಳಿದ್ದ ಜನತೆಯ ಬಗೆಗೆ ಅರಿವನ್ನು ಪಡೆದುಕೊಳ್ಳುವಂತಾಗಿದೆ.

ಸಾರ್ವಜನಿಕವಾಗಿ ಹಾಗೂ ಖಾಸಗಿಯಾಗಿ ಹಿಂದೂ ಸೋದರ ಮುಕ್ತವಾಗಿ, ಮನ ಬಿಚ್ಚಿ ಮಾತನಾಡಿರುವರು. ಭಾರತದ ಜನಸ್ತೋಮವು ತುಂಬ ಬಡವರೆಂದು, ತುಂಬ ಅಜ್ಞಾನಿಗಳು ಎಂದು, ವಿವಿಧ ಮತಪಂಥಗಳಾಗಿ ವಿಭಾಗಿಸಲ್ಪಟ್ಟಿರುವರೆಂದು, ಅವರ ಆರಾಧನಾಕ್ರಮವು ಸಾಮಾನ್ಯ ಮೂರ್ತಿಪೂಜೆಯಿಂದ ಹಿಡಿದು, ಮಾನವರ ಪರಸ್ಪರ ಸೋದರತ್ವ ಮತ್ತು ದೇವರಲ್ಲಿ ಒಂದಾಗಿರುವಿಕೆಯ ವಿಶಾಲಭಾವನೆಯ ತಳಹದಿಯ ನ್ನುಳ್ಳ ಅತ್ಯಂತ ಉದಾರವಾದ ದೈವಕಲ್ಪನೆಯವರೆಗೂ ಇರುವುದೆಂದು ಅವರು ಒಪ್ಪಿಕೊಳ್ಳುತ್ತಾರೆ. ಅವರ ಉದ್ದೇಶ ನಮ್ಮನ್ನು ತಮ್ಮಂತೆಯೇ ಯೋಚಿಸುವ ಹಾಗೆ ಮಾಡಲು ಪ್ರಯತ್ನಿಸುವುದಲ್ಲ, ಮತಾಂತರಗೊಳಿಸುವುದಲ್ಲ; ಬದಲಿಗೆ ಭಾರತದಲ್ಲಿ ಒಂದು ಕಾಲೇಜನ್ನು ಸ್ಥಾಪಿಸಿ, ಅಲ್ಲಿ ಪರಿಣತಿ ಪಡೆದ ಪ್ರಬೋಧಕರುಗಳು ಜನಸಾಮಾನ್ಯರ ನಡುವೆ ಇರುವ ಅನೇಕಾನೇಕ ಕೇಡುಗಳನ್ನು ಸುಧಾರಿಸಲು ಕೆಲಸಮಾಡುವಂತೆ ಮಾಡುವುದು; ಹಾಗೂ ಅದಕ್ಕೆ ಅಗತ್ಯವಾದ ದ್ರವ್ಯವನ್ನು ಸಂಪಾದಿಸುವುದು ಎನ್ನುತ್ತಾರೆ ಅವರು. ಭಾರತವು ಪುರೋಹಿತಶಾಹಿಯಿಂದ ಜರ್ಜರಿತವಾಗಿದೆ; ಈ ಪುರೋಹಿತಶಾಹಿಯೇ ಸತ್ಯವನ್ನು ವಿಕೃತಗೊಳಿಸಿ ಅಜ್ಞಾನವನ್ನು ಚಿರಂತನವಾಗಿಸಿರುವುದು; ಸತ್ಯಕ್ಕೆ ತನ್ನದೇ ಆದ ಅಸಂಸ್ಕೃತ ಸಂಕುಚಿತ ವ್ಯಾಖ್ಯಾನವನ್ನಿತ್ತು ಜನರ ಸಹಜವಾದ ನೈತಿಕ ಪದೋನ್ನತಿಯನ್ನು ತಡೆದು ಅವರ ಮನಸ್ಸನ್ನು ವಿಕೃತವನ್ನಾಗಿಸುವುದು ಎನ್ನುತ್ತಾರೆ ಅವರು. ಸ್ವಾಮಿಗಳು ಎಲ್ಲ ಮತಪಂಥಗಳನ್ನು, ಪಂಗಡಗಳನ್ನು, ಬುಡಕಟ್ಟುಗಳನ್ನು, ವಿಶಾಲವಾದ ತಳಹದಿಯ ಮೇಲೆ ಪರಿಗಣಿಸುತ್ತಾರೆ. ವಿಗ್ರಹಾರಾಧನೆಯಲ್ಲೂ ಅವರು ಒಳ್ಳೆಯದನ್ನೇ ಕಾಣುತ್ತಾರೆ. ನಿರಪೇಕ್ಷ ಕಲ್ಪನೆಗಳನ್ನು ಗ್ರಹಿಸುವ ಮಾನಸಿಕ ಸಾಮರ್ಥ್ಯ ಸಾಲದ, ವಾಸ್ತವಿಕವಾದ ಏನಾದರೊಂದು ರೂಪದಲ್ಲಿ ನಿರಪೇಕ್ಷದ ಆವಿರ್ಭಾವ ಅಗತ್ಯವಾಗಿರುವ ಅಜ್ಞಾನಿಗಳಿಗೆ ವಿಗ್ರಹಾರಾಧನೆ ಒಂದು ಆದರ್ಶ ಎಂದು ಅವರು ಭಾವಿಸುತ್ತಾರೆ. ಅತಿಯಾದ ಪುರೋಹಿತ ಶಾಹಿಯಿಂದಾಗಿ ಪಾಶ್ಚಾತ್ಯರಾದ ನಾವೂ ಸಹ ಅಭಿವೃದ್ಧಿಯಿಂದ ಕುಂಠಿತರಾಗಿದ್ದೇವೆ, ನಾವೂ ಸಹ ವಿಗ್ರಹಾರಾಧನೆಯ ಅಭ್ಯಾಸದಿಂದ ಮುಕ್ತರಾಗಿಲ್ಲ, ಏಕೆಂದರೆ ನಮ್ಮ ಕೆಲವು ಪಂಥಗಳ ನೂತನ ಚರ್ಚುಗಳಲ್ಲಿ ಪೂಜೆಗೊಳ್ಳುವ ಗರ್ಭಗೃಹಗಳು, ಶಿಲ್ಪಗಳು, ಚಿತ್ರಗಳು; ವೇದಿಕೆ, ಪೀಠ ಇತ್ಯಾದಿಗಳನ್ನು ಪವಿತ್ರವೆಂದು ಪರಿಗಣಿಸುವ ರೀತಿ - ಇವೆಲ್ಲವೂ ಆದರ್ಶ ಮೂರ್ತಿಪೂಜೆಯಲ್ಲದೆ ಬೇರೆಯಲ್ಲ ಎಂದು ಅವರು ನಿರ್ಭಿಡೆಯಿಂದ ಹೇಳಿದರು.

\begin{center}
\textbf{ಈ ದೇಶದಲ್ಲಿರುವ ಎರಡು ಅಸಾಧಾರಣ ಸಂಗತಿಗಳು}
\end{center}

ನಮ್ಮ ಬಗ್ಗೆ ನಿಮ್ಮ ನಿರ್ಭಿಡೆಯ ಅಭಿಪ್ರಾಯವೇನು ಎಂದು ಕೇಳಿದಾಗ, ಸ್ವಾಮಿಗಳು ಎರಡು ಅತ್ಯಂತ ಗಮನಾರ್ಹವಾದ ಸಂಗತಿಗಳಿವೆ ಎಂದರು: ಮೊದಲನೆಯದಾಗಿ, ಸ್ಥಾನ ಮತ್ತು ಬೌದ್ಧಿಕತೆಯಲ್ಲಿ ನಮ್ಮ ಸ್ತ್ರೀಯರ ಶ್ರೇಷ್ಠತೆ ಹಾಗೂ ಪ್ರಭಾವ; ಎರಡನೆಯದಾಗಿ, ನಮ್ಮ ಮಾನವ ಪ್ರೇಮ ಹಾಗೂ ಬಡವರನ್ನು ನೋಡಿಕೊಳ್ಳುವಿಕೆ -ಇವುಗಳಲ್ಲಿ ನಾವು ಮಾಡಬೇಕಾದದ್ದನ್ನೆಲ್ಲ ಮಾಡಿ ಸಮಸ್ಯೆಯನ್ನು ಬಹಳಮಟ್ಟಿಗೆ ಪರಿಹರಿಸಿಕೊಂಡಿದ್ದೇವೆ. ಇಷ್ಟು ಮಾತ್ರವೇ ಅಲ್ಲ, ಆಸ್ಪತ್ರೆಗಳು ಮತ್ತಿತರ ಧರ್ಮಾರ್ಥ ಸಂಸ್ಥೆಗಳ ವಿಚಾರದಲ್ಲಿ, ಹಾಗೂ ಮಾನವ ಪರಿಶ್ರಮವನ್ನು ಕಡಿಮೆಮಾಡುವ ಯಂತ್ರಗಳ ಅಭಿವೃದ್ಧಿಯಲ್ಲಿ ಅದ್ಭುತ ಎನ್ನಬಹುದಾದ ಸಾಧನೆ ನಮ್ಮದಾಗಿದೆ. ನಮ್ಮ ಭೌತಿಕ ಮುನ್ನಡೆಯನ್ನು ಅವರು ಪ್ರಶಂಸಿಸಲಿಲ್ಲ; ಏಕೆಂದರೆ ಅದು ಮನುಷ್ಯನನ್ನು ಉತ್ತಮ ಪಡಿಸುವುದಿಲ್ಲ. ನಾವು ಬಹುವಾಗಿ ಹೆಮ್ಮೆಪಟ್ಟುಕೊಳ್ಳುವ ನಾಗರಿಕತೆಯನ್ನೂ ಅವರು ಮೆಚ್ಚಿಕೊಳ್ಳಲಿಲ್ಲ; ನಾವು ಇಂಗ್ಲಿಷರ ರೀತಿ - ರಿವಾಜುಗಳನ್ನು ಕೇವಲ ಅನುಕರಿಸುತ್ತೇವೆ - ಕೆಲವೊಮ್ಮೆ ತೀರ ಹಾಸ್ಯಾಸ್ಪದ ಎನ್ನುವಮಟ್ಟಿಗೆ - ಅಷ್ಟೇ ಎಂದರು. ವಿಶಿಷ್ಟವೆನ್ನಿಸಬಹುದಾದ ನಾಗರಿಕತೆ ನಮ್ಮದಾಗಲು ನಾವಿನ್ನೂ ತುಂಬ ಎಳೆವಯಸ್ಸಿನವರು; ನಮ್ಮದೇ ಅಮೆರಿಕನ್ ವೈಶಿಷ್ಟ್ಯವನ್ನು ಸೃಜಿಸುವ ಮುನ್ನ, ನಮ್ಮ ಮೇಲೆ ಬಂದು ಸುರಿಯುವುದಕ್ಕೆ ನಾವೇ ಅನುವುಮಾಡಿಕೊಟ್ಟಿರುವ ಯೂರೋಪಿನ ಕೊಳಚೆಯನ್ನು ನಾವಿನ್ನೂ ಜೀರ್ಣಿಸಿಕೊಳ್ಳಬೇಕಾಗಿದೆ ಎಂದರು.

(ಇಲ್ಲಿ ಲೇಖಕರು ಹೇಳುತ್ತಾರೆ -ಸ್ವಾಮಿಗಳ ಭಾರತೀಯ ಹಿನ್ನೆಲೆಯಿಂದಾಗಿ ಅವರಿಗೆ ಪಾಶ್ಚಾತ್ಯ ಸ್ಪರ್ಧಾತ್ಮಕತೆ ಅನಪೇಕ್ಷಣೀಯವೇನೂ ಅಲ್ಲ, ಅದು ಕೇವಲ ಸಮರ್ಥರ ಉಳಿವು ಎಂಬ ಒಂದು ಮೂಲಭೂತ ಪ್ರಕೃತಿನಿಯಮ ಎಂಬುದನ್ನು ಅರ್ಥಮಾಡಿಕೊಳ್ಳಲು ಕಷ್ಟವಾಗಿದೆ; ಅಲ್ಲದೆ, “ಹಿಂದೂಗಳ ಕಲ್ಪನಾವಿಲಾಸದ, ಭಾವಾತಿರೇಕದ ಜೀವನತತ್ತ್ವ” ಎಷ್ಟರಮಟ್ಟಿಗೆ ಬಡತನ, ಅವನತಿ ಹಾಗೂ “ಕೈಬೆರಳೆಣಿಕೆಯಷ್ಟು ಬ್ರಿಟಿಷರಿಂದ” ಆಳಲ್ಪಡಬೇಕಾಗಿರುವ ಸಂದರ್ಭಗಳಿಗೆ ಕಾರಣವಾಗಿದೆ ಎಂಬುದನ್ನು ಗಮನಿಸಿ ಸ್ವಾಮಿಗಳು ಪಶ್ಚಿಮದ ಭೌತಿಕವಾದವನ್ನು ಉಪೇಕ್ಷಿಸುವುದಾಗಲಿ ಹೀಗಳೆಯುವುದಾಗಲಿ ಮಾಡದಿದ್ದರೆ ಒಳ್ಳೆಯದು ಎಂದು ತಮ್ಮ ಸಂಪಾದಕೀಯ ಟೀಕೆಯನ್ನು ಬರೆದು, ಅನಂತರ ಮುಂದುವರೆಸುತ್ತಾರೆ:)

\begin{center}
\textbf{ಪ್ರಚಾರಕರ ಮೇಲಣ ಅವರ ಟೀಕೆ}
\end{center}

ಭಾರತದಲ್ಲಿ ಅನ್ಯದೇಶಿಯ ಪ್ರಚಾರಕರು ಸಾಧಿಸಿರುವ ಫಲಿತಾಂಶಗಳ ಬಗ್ಗೆ ಅವರು ಹೇಳುವುದು ನಿಜವಾಗಿದ್ದ ಪಕ್ಷದಲ್ಲಿ, ಈ ವಿವಿಧ ಸಂಘಟನೆಗಳ ಆಡಳಿತವರ್ಗದವರು ಅವರೊಡನೆ ಸಮಾಲೋಚಿಸಿ ಅವರ ಸಲಹೆಯಂತೆ ನಡೆಯುವುದೊಳ್ಳೆಯದು. ಅವರು ಇಲ್ಲಿಗೆ ಬಂದಿರುವುದು ತಮ್ಮ ಜನರ ಸ್ಥಿತಿಯನ್ನು ಸುಧಾರಿಸುವುದಕ್ಕೆ. ಆದರೆ ಅವರು ಹೇಳುತ್ತಾರೆ - ಪ್ರಚಾರಕರ ಕೆಲಸ ಒಳ್ಳೆಯದನ್ನು ಮಾಡುತ್ತಿಲ್ಲ; ಈಗಾಗಲೇ ವಿಭಿನ್ನ ಮತಪಂಥಗಳಿಂದ ತುಂಬಿರುವ ದೇಶದಲ್ಲಿ ಇನ್ನೂ ಹೆಚ್ಚು ಮತಪಂಥಗಳನ್ನು ಸೃಷ್ಟಿಸುತ್ತದೆ; ಪ್ರತಿಯೊಬ್ಬ ಹಿಂದೂವಿಗೂ ಗೊತ್ತಿರುವ ವೇದಗಳಲ್ಲಿನ ಬೋಧನೆಗಳಿಗೂ ಕ್ರಿಸ್ತನ ಬೋಧನೆಗಳಿಗೂ ವ್ಯತ್ಯಾಸವಿಲ್ಲ - ಎನ್ನುತ್ತಾರೆ. ಅವರ ಪರಂಪರಾಗತ ಪ್ರವಣತೆಗಳು ಅಥವಾ ನಾಗರಿಕತೆಯೊಂದಿಗೆ ಅನ್ಯದೇಶೀಯ ಮತಪಂಥಗಳ ಸಿದ್ಧಾಂತಗಳು ಹೊಂದಾ ಣಿಕೆಯಾಗುವುದಿಲ್ಲ, ಪರಿಣಾಮವಾಗಿ ಅವುಗಳ ಪ್ರಸರಣ ಕಷ್ಟವಾಗುತ್ತದೆ ಎನ್ನುವ ವಿವೇಚನೆಯಿಂದ ಕೂಡಿದ ಪ್ರಸಾರವಾದ ಅವರದು.

ಹೇಗೇ ಇರಲಿ, ಕಾನಂದ ಅವರ ನಿಯೋಗವು ಪ್ರತಿಯೊಬ್ಬ ಮಾನವತಾ ಪ್ರೇಮಿಯ ಮೆಚ್ಚುಗೆಗೆ ಪಾತ್ರವಾಗುವಂಥದು. ಅವರು ನಮ್ಮ ದ್ರವ್ಯಸತ್ತಾವಾದ ಹಾಗೂ ಅಭಿವೃದ್ಧಿ ಸಾಧನೆಯ ಉತ್ತಮಾಂಶಗಳನ್ನು ಹಿಂದೂ ನಾಗರಿಕತೆಯಲ್ಲಿ ಅಳವಡಿಸಿಕೊಳ್ಳುವ ನಿರೀಕ್ಷೆಯನ್ನು ಹೊಂದಿದ್ದಾರೆ; ಯುಗಗಳ ಹಿಂದೆ ನಾವೆಲ್ಲರೂ ಸಮಾನ ನಾಗರಿಕತೆಯನ್ನು ಹೊಂದಿದ್ದ ಆರ್ಯ ಸೋದರರಾಗಿದ್ದಂತೆ ಮತ್ತೆ ಒಂದುಗೂಡುವವರೆಗೆ - ಭಗವಂತನಲ್ಲಿ ಯಾವ ವ್ಯತ್ಯಾಸವಿಲ್ಲದೆ, ಯಾವ ಮತಪಂಥಗಳಿಲ್ಲದೆ, ಪ್ರತ್ಯೇಕತೆಯಿಲ್ಲದೆ, ಭಗವಂತನಲ್ಲಿ ಒಂದಾಗಿರುವ ಎಲ್ಲವೂ ಏಕವೆನ್ನುವ ಉದಾತ್ತ ತತ್ತ್ವ ನಮ್ಮದಾಗುವವರೆಗೆ- ನಾವೂ ಅವರಿಂದ ಕಲಿಯಬೇಕಾಗಿದೆ.

\begin{flushright}
ಫ್ರೆಡ್ ಹೆಚ್. ಸೇಮರ್\footnote{1. ೧೮೯೪ರ ಫೆಬ್ರವರಿ ೧೨ ಶನಿವಾರದಂದು ಸ್ವಾಮಿ ವಿವೇಕಾನಂದರ ಗೌರವಾರ್ಥ ಚರ್ಲ್ಸ್ ಎಲ್. ಫ್ರೀರ್ ಅವರ ನಿವಾಸದಲ್ಲಿ; ಔತಣಕೂಟಕ್ಕೆ ಆಗಮಿಸಿದ್ದ ಅತಿಥಿಗಳಲ್ಲೊಬ್ಬರು.}
\end{flushright}

\begin{center}
\textbf{ಪ್ರತಿದಿನದ ದೇವರು\supskpt{\footnote{1. ನೋಡಿ, \enginline{New Deiscoveries, Vol 1, pp, 348}. ಈ ಉದ್ಧರಣೆಯನ್ನು ೧೮೯೪ ಫೆಬ್ರವರಿ ೧೮ ರಂದು ರಬ್ಬಿ ಗ್ರಾಸ್ಮನ್ ಅವರು ಕೊಟ್ಟ” \enginline{"What Vive Kannada Has Taught Us"}” ಎಂಬ ಪ್ರವಚನದಿಂದ ತೆಗೆದುಕೊಳ್ಳಲಾಗಿದೆ.}}}
\end{center}

\begin{center}
(ಡೆಟ್ರಾಯಿಟ್ ಟ್ರಿಬ್ಯೂನ್, ೧೯ ಫೆಬ್ರವರಿ ೧೮೯೪)
\end{center}

\begin{center}
\textbf{ರಬ್ಬಿ ಗ್ರಾಸ್ಮನ್ ಅವರು ಸ್ವಾಮಿ ವಿವೇಕಾನಂದರಿಂದ ಉಲ್ಲಸಿತರಾಗಿರುವರು}
\end{center}

... “ನಾನು ನಿಮ್ಮ ಜೀಸಸ್ನನ್ನು ಸ್ವೀಕರಿಸುತ್ತೇನೆ” ಎಂದು ವಿವೇಕಾನಂದ ಅವರು ಕಳೆದ ಶನಿವಾರ ಸಂಜೆ (ಫೆಬ್ರವರಿ ೧೨) ಹೇಳಿದರು.

ನಾನು ಎಲ್ಲ ದೇಶಗಳ, ಎಲ್ಲ ಕಾಲಗಳ, ಒಳಿತನ್ನು ಹಾಗೂ ಮಹತ್ವವನ್ನು ಸ್ವೀಕರಿಸುವ ಹಾಗೆಯೇ ಕ್ರಿಸ್ತನನ್ನೂ ಸ್ವೀಕರಿಸುತ್ತೇನೆ. ಆದರೆ ನೀವು, ನೀವು ನನ್ನ ಕೃಷ್ಣನನ್ನು ನಿಮ್ಮ ಹೃದಯದಲ್ಲಿ ಸ್ವೀಕರಿಸುವಿರಾ? ಇಲ್ಲ - ಸ್ವೀಕರಿಸಲಾರಿರಿ, ಸ್ವೀಕರಿಸುವ ಧೈರ್ಯ ಮಾಡಲಾರಿರಿ - ಆದರೂ ಸಹ ನೀವು ಸುಸಂಸ್ಕೃತರು, ನಾನು ಅನಾಗರಿಕ ಕ್ರೈಸ್ತೇತರ...\footnote{2. ನೋಡಿ,” ಮಾನವನ ದಿವ್ಯತೆ”, ಕೃತಿಶ್ರೇಣಿ ೩, ೪೩೭ ಹಾಗೂ” ಭರತಖಂಡ ಅಜ್ಞಾನಕೂಪದಲ್ಲಿಯೇ?”, ಕೃತಿಶ್ರೇಣಿ ೭, ೧೭೧.}

\begin{center}
\textbf{ವಿವೇಕಾನಂದರ ನಿರ್ಗಮನ\supskpt{\footnote{\enginline{3. New Discoveries, Vol. 1, pp. 380}}}}
\end{center}

\begin{center}
(ಡೆಟ್ರಾಯಿಟ್ ಜರ್ನಲ್, ೨೩ ಫೆಬ್ರವರಿ ೧೮೯೪)\\ಹಿಂದೂ ಕಾರ್ಮಿಕರ ಸ್ಥಿತಿಗತಿಗಳ ಬಗ್ಗೆ ಅವರು ಹೇಳಿದ್ದು
\end{center}

ಸ್ವಾಮಿ ವಿವೇಕಾನಂದರು ತಮ್ಮ ಪರಿಚಯದ ಸ್ತ್ರೀಯರ ಪ್ರಶಂಸೆಗೆ ಪ್ರತಿಯಾಗಿ ನಿನ್ನೆ ಮಧ್ಯಾಹ್ನ ಧಾರ್ಮಿಕವೂ ಅರೆಭಾವುಕವೂ ಆದ ಕವನಗಳನ್ನು ಬರೆದು ಸಮರ್ಪಿಸಿದರು\footnote{4. ಹೊರಟು ಹೋಗುತ್ತಿರುವ ತಮ್ಮ ಅತಿಥಿಗೆಂದು ಮಿಸೆಸ್ ಜೆ. ಬ್ಯಾಗ್ಲೆ ಅವರು ಏರ್ಪಡಿಸಿದ್ದ ಪುಟ್ಟ ಬೀಳ್ಕೊಡಿಗೆಯ ಟೀ ಸಮಾರಂಭದಲ್ಲಿ.}. ಈ ದಿನ ಬೆಳಿಗ್ಗೆ ಅವರು ಅಡಙ (ಓಹಿಯೋ) ಎಂಬ ಊರಿಗೆ ಹೊರಟರು.

ಹಿಂದೂ ಕೆಲಸಗಾರರ ಭೌತಿಕ ಸ್ಥಿತಿಗತಿಗಳ ಬಗ್ಗೆ ನಡೆದ ಒಂದು ಸಂಭಾಷಣೆಯಲ್ಲಿ ಜ್ಞಾ ನಿ, ಸಂನ್ಯಾಸಿ ವಿವೇಕಾನಂದರು, ಬಡವರು ಕೇವಲ ಗಂಜಿಯ ಮೇಲೆ ಬದುಕಿರುವರು ಎಂದರು. ಕಾರ್ಮಿಕ ಬೆಳಗ್ಗೆ ಒಮ್ಮೆ ಗಂಜಿ ಕುಡಿದು ದಿನನಿತ್ಯದ ಕೆಲಸಕ್ಕೆ ಹೋಗುವನು; ಸಂಜೆ ಹಿಂದಿರುಗಿದ ಮೇಲೆ ಮತ್ತೊಮ್ಮೆ ಗಂಜಿ ಕುಡಿದು ಅದನ್ನೇ ಊಟವೆನ್ನುವನು. ದೇಶದ ಹೆಚ್ಚು ಭಾಗದ ರೈತರು ಎಷ್ಟು ಬಡವರೆಂದರೆ, ಅವರು ಬೆಳೆದ ಗೋಧಿಯನ್ನೇ ಅವರು ತಿನ್ನಲಾರರು. ಸಾಮಾನ್ಯವಾಗಿ ಹೊಲದಲ್ಲಿ ದುಡಿಯುವ ಕೆಲಸಗಾರ ದಿನಕ್ಕೆ ಕೇವಲ ಹನ್ನೆರಡು ಪೆನ್ಸ್ ಮಾತ್ರ ಪಡೆಯುವನು; ಆದರೆ ಒಂದು ಡಾಲರ್ ಈ ದೇಶದಲ್ಲಿ ಕೊಳ್ಳಬಹುದಾದ್ದಕ್ಕಿಂತ ಹತ್ತರಷ್ಟನ್ನು ಭಾರತದಲ್ಲಿಕೊಳ್ಳಬಲ್ಲದು. ಹತ್ತಿಯನ್ನು ಬೆಳೆಯುವರು, ಆದರೆ ಅದರ ಎಳೆಗಳು ಎಷ್ಟು ಚಿಕ್ಕವೆಂದರೆ ಅದನ್ನು ಕೈಯಿಂದ ನೇಯ ಬೇಕಾಗುವುದು; ಆದರೂ ಸಹ ಅದರೊಂದಿಗೆ ಬೆರೆಸಲು ಅಮೆರಿಕನ್ ಮತ್ತು ಈಜಿಪ್ಷಿ ಯನ್ ಹತ್ತಿಯನ್ನು ಆಮದು ಮಾಡಿಕೊಳ್ಳುವುದು ಅಗತ್ಯವೆನಿಸುವುದು.

\begin{center}
\textbf{ಸ್ವದೇಶದಲ್ಲಿನ ಸಂಸ್ಕೃತಿ\supskpt{\footnote{\enginline{1. New Discoveries, Vol. 1, pp. 365}}}}
\end{center}

\begin{center}
(ಡೆಟ್ರಾಯಿಟ್ ಈವಿನಿಂಗ್ ನ್ಯೂಸ್, ೨೫ ಫೆಬ್ರವರಿ ೧೮೯೪)\\ಸ್ವಾಮಿ ವಿವೇಕಾನಂದರ ಡೆಟ್ರಾಯಿಟ್ ಭೇಟಿಯ ಕಥೆಗಳು
\end{center}

ಸ್ವಾಮಿ ವಿವೇಕಾನಂದರ ಭೇಟಿಯನ್ನು ಕುರಿತ ಅನೇಕ ಮುದಗೊಳಿಸುವ ಕಥೆಗಳಿವೆ. ಅಮೆರಿಕನ್ ಆತ್ಮ ಪ್ರೇಮಕ್ಕೆ ಇವು ಸ್ವಲ್ಪ ಕುಂದನ್ನುಂಟುಮಾಡಬಹುದಾದರೂ, ಅವರಿಗಂತೂ ಇವು ನಗು ತರಿಸುವುದು ಖಂಡಿತ. ಮಹಿಳೆಯೊಬ್ಬರು ಹೇಳಿದರು:

“ತಮ್ಮನ್ನು ತಾವು ಸಭ್ಯ-ಸುಸಂಸ್ಕೃತರೆಂದು ತಿಳಿದುಕೊಂಡಿರುವ ನಮ್ಮ ಕೆಲವು ಡೆಟ್ರಾಯಿಟ್ ಮಹನೀಯರುಗಳ ಜ್ಞಾನಕ್ಕೂ ಅವರ ಜ್ಞಾನಭಂಡಾರಕ್ಕೂ ಇರುವ ವ್ಯತ್ಯಾಸ ಕಂಡಾಗ ನನಗೆ ನಿಜವಾಗಿಯೂ ನಾಚಿಕೆ ಎನಿಸಿತು. ಭೋಜನಕೂಟವೊಂದರಲ್ಲಿ ಸಭ್ಯರೊಬ್ಬರು ಕಾನಂದರನ್ನು ತಮಗೆ ರಸಾಯನವಿಜ್ಞಾನದಲ್ಲಿ ಓದುವುದಕ್ಕೆ ನೀವು ಸಲಹೆ ಮಾಡುವ ಗ್ರಂಥಗಳಾವುವು ಎಂದು ಕೇಳಿದರು. ಯಾರಾದರೂ ನಿರೀಕ್ಷಿಸ ಬಹುದಾದಂತೆ, ಈ ವಿಷಯದಲ್ಲಿ ಒಬ್ಬ ಹಿಂದೂವಿಗಿಂತ ಅಮೆರಿಕನ್ನರಿಗೆ ತಿಳಿವಳಿಕೆ ಹೆಚ್ಚು; ಆದರೆ ಕಾನಂದರು ಆ ಕ್ಷೇತ್ರದ ಗ್ರಂಥಗಳ ಇನ್ನೊಂದು ದೊಡ್ಡ ಪಟ್ಟಿಯನ್ನು ಕೊಟ್ಟರು. ಒಬ್ಬ ಮಹಿಳೆ “ಕ್ರಿಸ್ತ - ಹಾಗೆಂದರೇನು?” ಎಂದು ಕೇಳಿದಾಗ ಅವರ ಅಚ್ಚರಿ ಮುಗಿಲುಮುಟ್ಟಿತು; ಮತ್ತೊಮ್ಮೆ ಅವರು ಅಪೇಕ್ಷಿತ ಮಾಹಿತಿಯನ್ನು ಕೊಟ್ಟರು, ಆದರೆ ಸ್ವಲ್ಪವೇ ಸ್ವಲ್ಪ ವ್ಯಂಗ್ಯ ಬೆರೆತ ದನಿಯಲ್ಲಿ”.

ಬಹುಶಃ ಹತ್ತೊಂಭತ್ತನೆಯ ಶತಮಾನದ ಸಂಸ್ಕೃತಿ-ನಾಗರಿಕತೆಗಳ ಆರಿಸಿ ತೆಗೆದ ಉದಾಹರಣೆ ಮಹಿಳೆಯೊಬ್ಬಳಿಂದ ಬಂದಿತು. ಕಾನಂದರನ್ನು ಅವಳು ನೀವು ಇಂಗ್ಲಿಷ ರನ್ನು ಇಷ್ಟಪಡುವಿರಾ ಎಂದು ಕೇಳಿದಳು. ಸಹಜವಾಗಿಯೆ ಅವರು ಇಲ್ಲ ಎಂದರು. ಅನಂತರ ಅವಳು ಬಹು ನಯವಾಗಿ ವಿಷಯವನ್ನು ಮುಂದುವರೆಸುತ್ತ, ತನಗೆ ಪ್ರಿಯವಾದ ಸಿಪಾಯಿ ದಂಗೆಯ ಘಟನೆಯ ಬಗ್ಗೆ ಕೇಳಿದಳು. ಹಿಂದೂವು ಉದ್ರಿಕ್ತರಾಗುತ್ತಲೂ, ಅವಳು ಅವರೆಡೆಗೆ ನಗುತ್ತ, ವ್ಯಂಗ್ಯವಾಗಿ “ನಿಮ್ಮ ಪ್ರಾಚ್ಯ ತಾತ್ತ್ವಿಕ ಶಾಂತಿಯನ್ನು ನಾನು ಕದಡಬಹುದೇನೋ ಅಂದುಕೊಂಡೆ” ಎಂದಳು!

\begin{center}
\textbf{ಅಕ್ರೈಸ್ತ ಕಾನಂದರು}\footnote{\enginline{1. New Discoveries, Vol. 1, pp. 410-416}}
\end{center}

\begin{center}
(ಡೆಟ್ರಾಯಿಟ್ ಟ್ರಿಬ್ಯೂನ್, ೧೧ ಮಾರ್ಚ್ ೧೮೯೪)
\end{center}

\begin{center}
ಕಳೆದ ರಾತ್ರಿಯ ಉಪನ್ಯಾಸದಲ್ಲಿ ಕ್ರೈಸ್ತ ಪ್ರಚಾರದ ಮೇಲೆ ಆಕ್ರಮಣ.
\end{center}

\begin{center}
-----
\end{center}

\begin{center}
ಸಭಿಕರಿಂದ ಅವರ ಮಾತಿಗೆ ಪ್ರಚಂಡ ಕರತಾಡನ.
\end{center}

\begin{center}
-----
\end{center}

ಕ್ರೈಸ್ತ ದೇಶಗಳು ಕೊಲೆ ಮಾಡುವುವು, ಅನ್ಯದೇಶಗಳಲ್ಲಿ ರೋಗಗಳನ್ನು ಹರಡು ವುವು, ಅಲ್ಲದೆ ಶಿಲುಬೆಗೇರಿದ ಕ್ರಿಸ್ತನನ್ನು ಬೋಧಿಸುವ ಮೂಲಕ ಪೆಟ್ಟಿನ ಜೊತೆಗೆ ಅಪಮಾನವನ್ನೂ ಸೇರಿಸುವುವು ಎಂದರರವರು.

ಸ್ವಾಮಿ ವಿವೇಕಾನಂದರು ನೆನ್ನೆ ರಾತ್ರಿ ಡೆಟ್ರಾಯಿಟ್ನ ಒಪೇರಾ ಹೌಸನಲ್ಲಿ “ಭಾರತದಲ್ಲಿ ಕ್ರೈಸ್ತ ಪ್ರಚಾರ\footnote{2. ಸ್ವಲ್ಪ ಭಿನ್ನವಾದ ಸಾರಾಂಶ ಸಂಗ್ರಹಕ್ಕಾಗಿ ನೋಡಿ, “ಭರತಖಂಡದಲ್ಲಿ ಕ್ರೈಸ್ತಧರ್ಮ”, ಕೃತಿಶ್ರೇಣಿ, ೭, ೩೭೫.} ” ಎಂಬ ವಿಷಯವಾಗಿ ಬಹು ದೊಡ್ಡ ಸಭೆಯನ್ನು ದ್ದೇಶಿಸಿ ಉಪನ್ಯಾಸ ಮಾಡಿದರು. ಈ ನಗರದಲ್ಲಿ ಕಳೆದ ಎರಡು ವಾರಗಳಿಂದ ಕಾನಂದ ಅವರನ್ನುದ್ದೇಶಿಸಿ ಕ್ರೈಸ್ತ ಪ್ರಚಾರಕರು ಕೊಟ್ಟಿರುವ ಹೇಳಿಕೆಗಳಿಗೆ ಈ ಉಪನ್ಯಾಸವು ಉತ್ತರವಾಗಿದೆಯೆಂದು ಯಾರಿಗಾದರೂ ಅನ್ನಿಸುವಂತೆ ಇತ್ತು.

ಕಾನಂದ ಅವರನ್ನು ಕಳೆದ ರಾತ್ರಿ ಮಾನ್ಯ ಥಾಮಸ್ ಡಬ್ಲೂ. ಪಾಮರ್ ಅವರು ಪರಿಚಯ ಮಾಡಿಕೊಟ್ಟರು. ಮುನ್ನುಡಿಯಾಗುವಂತೆ ಅವರೊಂದು ಕಟ್ಟುಕಥೆಯನ್ನು ಹೇಳಿದರು. “ಇಬ್ಬರು ನೈಟ್ ಗೌರವಾನ್ವಿತ ಗಂಡುಗಲಿಗಳು ಹೊಲವೊಂದರಲ್ಲಿ ಒಮ್ಮೆ ಭೇಟಿಯಾದರು. ಮರದ ಮೇಲೊಂದು ಗುರಾಣಿ ನೇತಾಡುತ್ತಿದ್ದುದನ್ನು ಕಂಡು ನಿಂತರು. ಒಬ್ಬನು, ‘ಎಂತಹ ಒಳ್ಳೆಯ ಬೆಳ್ಳಿ ಗುರಾಣಿ’ ಎಂದನು. ಅದಕ್ಕೆ ಇನ್ನೊಬ್ಬನು ಅದು ಬೆಳ್ಳಿಯಲ್ಲ, ತಾಮ್ರದ್ದು ಎಂದನು. ಒಬ್ಬರ ಮಾತನ್ನು ಇನ್ನೊಬ್ಬರು ಒಪ್ಪದ ಕಾರಣ ಇಬ್ಬರೂ ಕುದುರೆಯಿಂದಿಳಿದು, ಕುದುರೆಗಳನ್ನು ಮರಕ್ಕೆ ಕಟ್ಟಿಹಾಕಿ, ಖಡ್ಗಗಳನ್ನು ಹೊರಸೆಳೆದು, ಅನೇಕ ಗಂಟೆಗಳ ಕಾಲ ಹೋರಾಡಿದರು. ತುಂಬ ರಕ್ತ ಕಳೆದುಕೊಂಡು ಆಯಾಸಗೊಂಡ ಅವರು, ತಾವು ಹೋರಾಡುತ್ತಿದ್ದಲ್ಲಿಂದ ಒಬ್ಬೊಬ್ಬರು ಒಂದೊಂದು ಕಡೆ ಬಿದ್ದರು. ಆಗ ಒಬ್ಬನು ಮೇಲೆ ನೇತಾಡುತ್ತಿದ್ದ ಗುರಾಣಿಯನ್ನು ನೋಡಿ, ‘ಗೆಳೆಯ, ನೀನು ಹೇಳುವುದು ಸರಿ. ಗುರಾಣಿ ತಾಮ್ರದ್ದು’ ಎಂದನು. ಅವರಿಬ್ಬರೂ ಗುರಾಣಿಯ ಎರಡೂ ಬದಿಗಳನ್ನು ಈ ಮೊದಲೇ ನೋಡಿದ್ದಿದ್ದರೆ, ರಕ್ತ ಸುರಿಸುವುದಾದರೂ ಉಳಿಯುತ್ತಿತ್ತು. ಹಾಗೆಯೇ ನಾವು ಒಂದೊಂದು ಪ್ರಶ್ನೆಯ ಎರಡೂ ಬದಿಗಳನ್ನು ನೋಡಿದ್ದೇ ಆದರೆವಾದವಿವಾದಗಳು, ಹೋರಾಟ ಕಡಿಮೆಯಾಗಬಹುದು.

“ಕ್ರೈಸ್ತ ದೃಷ್ಟಿಕೋನದಿಂದ ಕ್ರೈಸ್ತೇತರ, ಅನಾಗರಿಕರೆನ್ನಬಹುದಾದ ಸಭ್ಯರೊಬ್ಬರು ಈ ರಾತ್ರಿ ನಮ್ಮೊಂದಿಗೆ ಇದ್ದಾರೆ. ಆದರೆ ಜನರು ನಮ್ಮ ಧರ್ಮಕ್ಕೆ ಕಲ್ಪಿಸಿರುವ ಕಾಲಕ್ಕಿಂತಲೂ ಬಹು ಹಿಂದಿನ ಧರ್ಮವೊಂದಕ್ಕೆ ಸೇರಿದವರು ಅವರು. ಗುರಾಣಿಯ ತಾಮ್ರದ ಕಡೆಯಿಂದ ಕೇಳುವುದು ಖಂಡಿತವಾಗಿಯೂ ಹಿತವಾಗಿರುತ್ತದೆ ಎಂದುಕೊಳ್ಳುತ್ತೇನೆ. ನಾವು ಬೆಳ್ಳಿಯ ಕಡೆಯಿಂದ ಮಾತ್ರ ನೋಡಿರುವೆವು. ಮಹಿಳೆಯರೆ, ಮಹನೀ ಯರೆ, ಇದೋ ಸ್ವಾಮಿ ವಿವೇಕಾನಂದರು.”

ಪಾಮರ್ ಅವರು ಮಾತನಾಡುವ ತನಕವೂ ವೇದಿಕೆಯ ಮೇಲೆ ಕುಳಿತಿದ್ದ ಕಾನಂದ ಅವರು ಮುಂದಕ್ಕೆ ಬಂದರು. ಬ್ರಾಹ್ಮಣ ಪೂಜಾರಿಯ ವಿಶಿಷ್ಟ ಪೇಟ ಹಾಗೂ ಕಿತ್ತಳೆ ವರ್ಣದ ಉಡುಪು ಧರಿಸಿದ್ದ ಅವರು ಸ್ವಾಗತದ ಕರತಾಡನಕ್ಕೆ ವಂದಿಸಿ, ತತ್ಕ್ಷಣ ತಮ್ಮ ಉಪನ್ಯಾಸದ ವಿಷಯವನ್ನು ಪ್ರಾರಂಭಿಸಿದರು.

\begin{center}
\textbf{ಭಾರತವೆಂದರೆ ಏನು?}
\end{center}

(ಸ್ವಾಮಿಗಳೆಂದರು:)

ಪುಸ್ತಕಗಳನ್ನು ಹಾಗೂ ಲಭ್ಯ ಸಾಹಿತ್ಯವನ್ನು ಓದಿದ್ದನ್ನು ಬಿಟ್ಟರೆ ನನಗೆ ಚೈನಾ ಹಾಗೂ ಜಪಾನ್ಗಳಲ್ಲಿನ ಕ್ರೈಸ್ತ ಪ್ರಚಾರಕರ ಕೆಲಸದ ಬಗ್ಗೆ ತಿಳಿಯದು; ಆದರೆ ನಾನು ಭಾರತವನ್ನು ಕ್ರೈಸ್ತದೇಶವನ್ನಾಗಿಸುವ ಪ್ರಯತ್ನಗಳ ಬಗ್ಗೆ ಮಾತನಾಡಬಲ್ಲೆ. ಆದರೆ ಈ ವಿಷಯಕ್ಕೆ ಹೋಗುವ ಮೊದಲು ಭಾರತವೆಂದರೆ ಏನು ಎನ್ನುವುದರ ಕಲ್ಪನೆಯನ್ನು ನಿಮ್ಮ ಮುಂದಿಡಲು ಅಪೇಕ್ಷಿಸುತ್ತೇನೆ.

ಅನಂತರ ಅವರು ಭಾರತದ ಮೂವತ್ತು ಕೋಟಿ ಜನರು ಹೇಗೆ ಪರಸ್ಪರ ಹೊಂದಾಣಿಕೆಯಿಲ್ಲದ ಎರಡು ಜಾತಿಗಳಾಗಿ ಭಾಗವಾಗಿದ್ದಾರೆ, ದಕ್ಷಿಣದ ಸ್ಥಳೀಯರು ಉತ್ತರದವರ ಭಾಷೆಯನ್ನೂ ಉತ್ತರದವರು ಇವರ ಭಾಷೆಯನ್ನೂ ಹೇಗೆ ಅರಿತುಕೊಳ್ಳಲಾರರು ಎಂದು ವಿವರವಾಗಿ ವರ್ಣಿಸಿದರು. ಕೆಳಜಾತಿಯವರು ಸತ್ತ ಪ್ರಾಣಿಗಳನ್ನು ತಿಂದು ಬದುಕುವರು, ಎಂದೂ ಸ್ನಾನಮಾಡುವುದಿಲ್ಲ, ಅವೇ ನಿಯಮಗಳು ಈರ್ವರಿಗೂ ಅನ್ವಯಿಸುವುವಾದರೂ ಮೇಲ್ಜಾತಿಯವರಿಗೆ ಅವರೊಡನೆ ಬೆರೆಯುವುದು ಅಸಾಧ್ಯವಾಗುತ್ತದೆ, ಎಂಬುದನ್ನೆಲ್ಲ ಹೇಳಿದರು.

ಬುದ್ಧನ ಅನುಯಾಯಿಗಳನ್ನು ಕ್ರೈಸ್ತ ಮತಕ್ಕೆ ಒಲಿಸಿಕೊಳ್ಳುವುದಕ್ಕಾಗಿ ಮೊಟ್ಟ ಮೊದಲ ಕ್ರೈಸ್ತ ಪದಾರ್ಪಣೆಯನ್ನು ಉಲ್ಲೇಖಿಸುತ್ತ, ಸ್ಪೇಯಿನ್ ದೇಶದವರಾದ ಅವರು ಸಿಲೋನ್ ಬಳಿ ಬುದ್ಧನ ಹಲ್ಲೊಂದು ಇದ್ದ ಒಂದು ದೇಗುಲವನ್ನು ಕಂಡುಹಿಡಿದರು ಎಂದರು.

“ಸ್ಪೇಯಿನ್ ಕ್ರೈಸ್ತರು ತಮ್ಮ ದೇವರು- ಹೋಗಿ, ಹೋರಾಡಿ, ಕೊಲ್ಲಿ, ಸಾಯಿಸಿ ಎಂದು ಆಜ್ಞಾ ಪಿಸಿದನೆಂದು ಯೋಚಿಸಿ, ಬುದ್ಧನ ಹಲ್ಲನ್ನು ಕಿತ್ತುಕೊಂಡು ನಾಶಮಾಡಿದರು. ಆದರೆ ಅದು ಬುದ್ಧನ ಹಲ್ಲು ಆಗಿರಲೇ ಇಲ್ಲ, ಪುರೋಹಿತರುಗಳೇ ಮಾಡಿ ಇರಿಸಿದ್ದ ಪಳೆಯುಳಿಕೆಯಾದ ಅದು ಒಂದು ಅಡಿಯಷ್ಟು ಉದ್ದವಿತ್ತು (ನಗು). ಪ್ರತಿಯೊಂದು ಧರ್ಮವೂ ತನ್ನದೇ ಆದ ಪವಾಡಗಳನ್ನು ಹೊಂದಿರುತ್ತದೆ; ಹಲ್ಲು ಒಂದು ಅಡಿ ಉದ್ದವಿತ್ತು ಎಂದು ನೀವು ನಗಬೇಕಾಗಿಲ್ಲ. ಒಳ್ಳೇದು, ಸ್ಪೇಯಿನ್ ಕ್ರೈಸ್ತರು ಹಲ್ಲನ್ನು ಕಿತ್ತುಕೊಂಡಾದ ಮೇಲೆ ಕೆಲವು ನೂರು ಜನರನ್ನು ಮತಾಂತರಗೊಳಿಸಿದರು, ಕೆಲವು ಸಾವಿರ ಜನರನ್ನು ಸಾಯಿಸಿದರು; ಬೌದ್ಧರ ನಡುವೆ ಪ್ರಚಾರಕರ ಪ್ರಯತ್ನದ ಚರಿತ್ರೆಯಲ್ಲಿ ಸ್ಪೇಯಿನ್ ಪಾತ್ರ ಅಲ್ಲಿಗೆ ನಿಂತಿತು.”

ಅನಂತರ, ಪೋರ್ಚುಗೀಸ್ಕ್ರೈಸ್ತರು ಮುಂಬಯಿಯಲ್ಲಿ ಮೂರು ಶಿರಸ್ಸುಗಳ ರೂಪದ ಒಂದು ಮಹಾ ದೇವಾಲಯವನ್ನು ಕಂಡುಹಿಡಿದರು ಎಂದು ಅವರು ಹೇಳಿದರು. ಈ ದೇಗುಲ ಹಿಂದೂಗಳು ನಂಬುವ ತ್ರಿಮೂರ್ತಿಗಳನ್ನು ಪ್ರತಿನಿಧಿಸುತ್ತಿತ್ತು ಎಂದ ಕಾನಂದರು, ಸ್ವಲ್ಪ ವ್ಯಂಗ್ಯ ಬೆರೆತ ಧ್ವನಿಯಲ್ಲಿ ಹೀಗೆಂದರು:

ಪೋರ್ಚುಗೀಸರು ನೋಡಿದರಾದರೂ ಅವರಿಗೆ ಅದನ್ನು ವಿವರಿಸಲಾಗಲಿಲ್ಲ; ಅದ್ದರಿಂದ ಅದು ಸೈತಾನನದ್ದಾಗಿರಬೇಕೆಂದು ತೀರ್ಮಾನಿಸಿದರು. ತಮ್ಮ ಬಲವನ್ನು ಒಗ್ಗೂಡಿಸಿ ತಂದು, ದೇಗುಲದ ಮೂರು ತಲೆಗಳನ್ನೂ ಒಡೆದು ಹಾಕಿದರು. ಸೈತಾನ ಅದೆಷ್ಟು ಸುಲಭವಾಗಿ ಕೈಗೆ ಸಿಕ್ಕುವನು - ಅಷ್ಟು ಜಾಗ್ರತೆಯಾಗಿ ಅವನು ಅದೃಶ್ಯನಾದ ನಲ್ಲಾ ಎಂದು ನನಗೆ ಖೇದವಾಗುತ್ತಿದೆ.”

ಅನಂತರ ಕಾನಂದರು ಭಾರತದಲ್ಲಿ ಕ್ರೈಸ್ತ ಮತಾಂತರದ ಇನ್ನೂ ಅನೇಕ ಹಂತಗಳನ್ನು ಸಂಕ್ಷೇಪವಾಗಿ ವಿವರಿಸಿದರು. ಇಬ್ಬರು-ಮೂವರು ಕ್ರೈಸ್ತ ಪ್ರಚಾರಕರಿಗೆ ಮಹತ್ತರ ಗೌರವವನ್ನು ಸಲ್ಲಿಸಿ, ಅವರು ನಿಯಮಕ್ಕೊಂದು ಅಪವಾದವೆಂಬಂತೆ ಜನರ ನಡುವೆಯೇ ಅವರ ಯೋಗಕ್ಷೇಮವನ್ನು ನೋಡಿಕೊಳ್ಳುತ್ತ ಬದುಕಿ ಅವರನ್ನು ಉನ್ನತಿಗೇರುವಂತೆ ಮಾಡಿದರು ಎಂದರು.

\begin{center}
\textbf{ಸ್ಥಳೀಯರ ಹಿತಾಸಕ್ತಿಗಳಿಗೆ ವಿರೋಧವಾಗಿದ್ದರು}
\end{center}

ಹಿಂದೂ ಪಾದ್ರಿಯು ಮುಂದುವರೆದು, ನಾಡು ಇಂಗ್ಲಿಷ್ ಜನರ ಕೈಗೆ ಬಂದೊಡನೆ ಪ್ರತಿ ಹಳ್ಳಿಯಲ್ಲಿಯೂ ಹೇಗೆ ಒಂದು ಬಿಳಿಯರ ಕಾಲನಿ ಹುಟ್ಟಿಕೊಂಡಿತು, ಹೇಗೆ ಅವರುಗಳು ಮಾತ್ರವೇ ಒಗ್ಗೂಡಿ ಬದುಕುತ್ತ ಸ್ಥಳೀಯರೊಂದಿಗೆ ಯಾವ ಒಡನಾಟವಿಲ್ಲದಂತಿದ್ದರು ಎಂಬುದನ್ನು ವಿವರಿಸಿದರು. ನಂತರ ಕ್ರೈಸ್ತ ಪ್ರಚಾರಕರು ದೇಶಕ್ಕೆ ಬರುತ್ತಲೂ ಅವರು ಹೇಗೆ ಸಹಜವಾಗಿಯೆ ತಮ್ಮನ್ನು ಸಹಾನುಭೂತಿಯಿಂದ ಕಾಣಬಹುದಾದ, ತಮ್ಮೊಂದಿಗೆ ಸಂಭಾಷಿಸಬಹುದಾದ ಇಂಗ್ಲಿಷ್ ಜನರ ನಡುವೆ ಹೋಗಿ ಇರುತ್ತಿದ್ದರು ಎಂದು ಹೇಳಿದರು. ಪ್ರಚಾರಕರಿಗೆ ಸ್ಥಳೀಯರ ಭಾಷೆ ಸ್ವಲ್ಪವೂ ತಿಳಿಯದು; ಆದ್ದರಿಂದ ಅವರೊಂದಿಗೆ ಬೆರೆಯಲಾರರು. ಪ್ರಚಾರಕರಲ್ಲಿ ಹೆಚ್ಚು ಜನ ಮದುವೆಯಾದವರು; ಹೆಂಡಂದಿರಿಗೆ ಇಂಗ್ಲಿಷ್ ಸಮಾಜದಲ್ಲಿ ಬೆರೆಯುವ ಅವಕಾಶ ದೊರಕಿಸುವಸಲುವಾಗಿ ಅವರು ಇಂಗ್ಲಿಷರ ಹಿತಾಸಕ್ತಿಗಳೊಡನೆ ಒಂದಾಗುತ್ತಿದ್ದರು; ಹಾಗೆ ಮಾಡುವಾಗ ನೇರವಾಗಿ ಸ್ಥಳೀಯರ ಹಿತಾಸಕ್ತಿಗಳಿಗೆ ವಿರೋಧವಾಗಿರಬೇಕಾಗುತ್ತಿತ್ತು; ಆದಕಾರಣ ಸ್ಥಳೀಯರೊಂದಿಗೆ ಬೆರೆಯುವುದು ಅಸಾಧ್ಯವಾಗುವಂತೆ ಮಾಡಿಕೊಳ್ಳುತ್ತಿದ್ದರು. ಹೀಗೆಂದ ಕಾನಂದರು ಮುಂದುವರೆಸಿದರು:

“ಭಾರತದಲ್ಲಿ ಕೆಲವೊಮ್ಮೆ ನಾವು ಕ್ಷಾಮಗಳಿಗೆ ತುತ್ತಾಗುವುದುಂಟು. ಹಾಗಾದಾಗ ತರುಣ ಪ್ರಚಾರಕ ಕ್ಷಾಮ ಇನ್ನೇನು ಮುಗಿಯಿತು ಎನ್ನುವವರೆಗೆ ಕಾಯುತ್ತಾನೆ; ನಂತರ ಹಸಿವೆಯಿಂದ ಕಂಗೆಟ್ಟ ಸ್ಥಳೀಯನಿಗೆ ಐದು ಷಿಲಿಂಗ್ ಕೊಡುತ್ತಾನೆ. ಆಗ ಒಂದು ಸಿದ್ಧ ಮತಾಂತರ ನಡೆಯುತ್ತದೆ. ಹೊಸ ಕ್ರೈಸ್ತನನ್ನು ಬರಮಾಡಿಕೊಳ್ಳುತ್ತಾರೆ. ಅವನು ಬಹುಶಃ ಬ್ಯಾಪ್ಟಿಸ್ಟ್ ಪ್ರಚಾರಕನಿರಬೇಕು. ಆ ನಂತರ ಮೆಥಾಡಿಸ್ಟ್ ಪ್ರಚಾರಕ ಬರುತ್ತಾನೆ. ಅದೇ ಸ್ಥಳೀಯನಿಗೆ ಐದು ಷಿಲಿಂಗ್ ಕೊಡುತ್ತಾನೆ. ಮತಾಂತರವಾದವನೆಂದು ಅವನ ಹೆಸರನ್ನು ಮತ್ತೊಮ್ಮೆ ದಾಖಲಿಸಿಕೊಳ್ಳಲಾಗುತ್ತದೆ. ಪ್ರತಿಯೊಬ್ಬ ಪ್ರಚಾರಕನಸುತ್ತಲಿರುವ ಮತಾಂತರವಾದವರೆಲ್ಲ ಜೀವನಕ್ಕಾಗಿ ಅವನನ್ನು ಅವಲಂಬಿಸಿರುವವರೇ ಆಗಿರುತ್ತಾರೆ. ಅವರು ಕ್ರೈಸ್ತ ರಾಗಬೇಕು ಇಲ್ಲವೇ ಉಪವಾಸವಿರಬೇಕು. ಪ್ರಚಾರಕನಲ್ಲಿ ಹಣ ಕಡಿಮೆಯಾಗುತ್ತಿದ್ದಂತೆ ಮತಾಂತರಗೊಂಡವರ ಸಂಖ್ಯೆಯೂ ಕಡಿಮೆಯಾಗುತ್ತದೆ. ಭಾರತದಲ್ಲಿ ನೀವು ಬಡವರಿಗೆ ಮಾಡುವುದಕ್ಕೆ ಕೆಲಸವನ್ನೂ ತಿನ್ನುವುದಕ್ಕೆ ಅನ್ನವನ್ನೂ ಕೊಟ್ಟು ಅವರನ್ನು ಕ್ರೈಸ್ತರನ್ನಾಗಿಸುವುದಾದರೆ ನನಗೆ ಸಂತೋಷ. ವೇಗವಾಗಿ ಆಗಲಿ ಆ ಕೆಲಸ. ಕ್ರೈಸ್ತ ಪ್ರಚಾರಕರ ಚಳುವಳಿಯನ್ನು ಒಂದು ಕಾರಣಕ್ಕಾಗಿ ಮೆಚ್ಚಬೇಕು. ಅದರಿಂದ ವಿದ್ಯಾಭ್ಯಾಸ ಅಗ್ಗವಾಗಿದೆ. ಪ್ರಚಾರಕರು ತಮ್ಮನ್ನು ಕಳುಹಿಸಿದವರಿಂದ ಸ್ವಲ್ಪ ಹಣವನ್ನು ತಂದಿರುತ್ತಾರೆ, ಭಾರತ ಸರ್ಕಾರವೂ ಒಂದಷ್ಟನ್ನು ಕೊಡುತ್ತದೆ; ಹಾಗೆ ಕೆಲವು ಬಹಳ ಒಳ್ಳೆಯ ಕಾಲೇಜುಗಳೂ, ಸ್ಕೂಲ್ಗಳೂ ದೇಶಿಯರಿಗೆ ಪ್ರಚಾರಕರ ಮೂಲಕ ಲಭ್ಯವಾಗಿವೆ. ಆದರೆ ನಾನು ನಿಮಗೆ ನೇರವಾಗಿ ಹೇಳುತ್ತೇನೆ, ಈ ಶಾಲಾ ಕಾಲೇಜುಗಳಿಂದ ಕ್ರೈಸ್ತಧರ್ಮಕ್ಕೆ ಮತಾಂತರಗಳಾಗಿಲ್ಲ. ಹಿಂದೂ ಹುಡುಗ ಬಹಳ ಬುದ್ಧಿವಂತ. ಅವನು ತಿಂಡಿಯನ್ನು ಕಚ್ಚಿಕೊಳ್ಳುತ್ತಾನೆ, ಆದರೆ ಕೊಕ್ಕೆಗೆ ಸಿಕ್ಕುವುದಿಲ್ಲ.”

ಭಾಷಣಕಾರರೆಂದರು, ಮಹಿಳಾ ಪ್ರಚಾರಕಿ ಕೆಲವು ಮನೆಗಳಿಗೆ ಹೋಗುತ್ತಾಳೆ, ತಿಂಗಳಿಗೆ ನಾಲ್ಕು ಷಿಲಿಂಗ್ ಪಡೆಯುತ್ತಾಳೆ, ಬೈಬಲ್ ಓದುತ್ತಾಳೆ, ಆಗ ದೇಶೀಯ ಹುಡುಗಿಯರು ಉಪೇಕ್ಷೆಯಿಂದಲೇ ಕೇಳಿಸಿಕೊಳ್ಳುತ್ತಾರೆ; ಮತ್ತು ಸ್ವೆಟರ್ ಹೆಣೆಯುವುದನ್ನು ಹೇಳಿಕೊಡುತ್ತಾಳೆ, ಆಗ ಹುಡುಗಿಯರು ಗಮನವಿಟ್ಟು ಕಲಿತುಕೊಳ್ಳುತ್ತಾರೆ. ಹುಡುಗಿಯರೂ ಹುಡುಗರ ಹಾಗೆಯೇ ವಾಸ್ತವ ಸಂಗತಿಗಳನ್ನು ಕಲಿತುಕೊಳ್ಳುವುದರಲ್ಲಿ ಎಚ್ಚರಿಕೆ ವಹಿಸುತ್ತಾರೆ; ಆದರೆ ಒಂದುವೇಳೆ ಲಾಭವಾಗುವುದಾದರೆ ಮತಾಂತರ ಗೊಂಡರೂ, ಕ್ರೈಸ್ತಧರ್ಮಕ್ಕೆ ಗಮನ ಕೊಡುವುದಿಲ್ಲ.

\begin{center}
\textbf{ಬಹುಮಟ್ಟಿಗೆ ಪ್ರಚಾರಕರೆಲ್ಲ ಅಸಮರ್ಥರು}
\end{center}

“ನೀವು ಪ್ರಚಾರಕ್ಕೆಂದು ಕಳುಹಿಸುವ ಬಹಳಷ್ಟು ಮಂದಿ ಅಸಮರ್ಥರು. ಪ್ರಚಾರಕನಾಗಿ ಭಾರತಕ್ಕೆ ಹೋಗುವ ಮೊದಲು ಅಲ್ಲಿಯ ಪುಸ್ತಕ, ಸಾಹಿತ್ಯವೆಲ್ಲ ಯಾವ ಭಾಷೆಯಲ್ಲಿ ಇದೆಯೋ ಆ ಸಂಸ್ಕೃತ ಭಾಷೆಯನ್ನು ಕಲಿತ ಒಬ್ಬನೇ ಒಬ್ಬನನ್ನು ನಾನು ನೋಡಲಿಲ್ಲ” -

ಎಂದರು ವಿವೇಕಾನಂದರು. ಪ್ರಚಾರಕರ ಭೇಟಿಗಳಿಗೆ ವಿವರಣೆಯಾಗಿ ಅವರು “ಬಹುಶಃ ತಮ್ಮ ದೇಶದಲ್ಲಿ ತಾಂಡವವಾಡುತ್ತಿರುವ ನಾಸ್ತಿಕತೆ ಹಾಗೂ ಸಂದೇಹವಾದಗಳು ಪ್ರಚಾರಕರನ್ನು ಪ್ರಪಂಚದೆಲ್ಲೆಡೆಗೆ ದೂಡುತ್ತಿರಬೇಕು” ಎಂದು ಸಲಹೆಯಿತ್ತರು. ಭಾರತದಲ್ಲಿದ್ದಾಗ ನಾನು ಕ್ರೈಸ್ತಧರ್ಮದ ಕೆಲಸವೇ ಎಲ್ಲ ಜನರನ್ನೂ ನರಕದ ಬೇಗೆಗೆ ಕಳುಹಿಸುವುದು ಮಾತ್ರ ಎಂದುಕೊಂಡಿದ್ದೆ, ಆದರೆ ಅಮೆರಿಕಾಕ್ಕೆ ಬಂದ ಮೇಲೆ ಬೇಕಾ ದಷ್ಟು ಮಂದಿ ಉದಾರಹೃದಯಿಗಳನ್ನು ಕಂಡಿದ್ದೇನೆ ಎಂದರು. ಅವರು ಸರ್ವಧರ್ಮ ಸಮ್ಮೇಳನವನ್ನು ಉದ್ಧರಿಸಿ, ಹೇಗೆ ಒಂದು ಪ್ರೆಸ್ಬಿಟೇರಿಯನ್ ಪತ್ರಿಕೆಯ ಸಂಪಾದಕರು ಸಮ್ಮೇಳನ ಮುಗಿಯುತ್ತಲೇ “ಸುಳ್ಳುಗಾರ ಹಿಂದೂ” ಎಂಬ ಲೇಖನದಲ್ಲಿ ತಮ್ಮನ್ನು ತೀವ್ರವಾಗಿನಿಂದಿಸಿರುವರು ಎಂಬುದನ್ನು ಹೇಳಿದರು.

ಆ ಲೇಖನದಲ್ಲಿ ಸಂಪಾದಕರು “ಸಮ್ಮೇಳನ ನಡೆಯುತ್ತಿರುವಷ್ಟು ದಿನ ಅವರು ನಮ್ಮ ಅತಿಥಿಯಾಗಿದ್ದರು; ಆದರೆ ಈಗ ಅದು ಮುಗಿದಿರುವುದರಿಂದ ನಾವು ಅವರ ಹಾಗೂ ಅವರ ಸುಳ್ಳು ಸಿದ್ಧಾಂತಗಳ ವಿರುದ್ಧ ಹುಮ್ಮಸ್ಸಿನಿಂದ ಆಕ್ರಮಣ ಮಾಡಬೇಕು” ಎಂದು ಬರೆದಿರುವರು.

ಭಾರತದಲ್ಲಿರುವ ವೈದ್ಯಕೀಯ ಪ್ರಚಾರಕರನ್ನು ಉದ್ದೇಶಿಸಿ ಕಾನಂದರು ಹೇಳಿದರು:

“ಭಾರತಕ್ಕೆ ಆರೋಗ್ಯಬೇಕು, ಆದರೆ ಅದು ಜನಗಳಿಗೆ ದೊರಕುವ ಆರೋಗ್ಯವಾಗಿರಬೇಕು. ಜನಗಳ ಜೊತೆಗೆ ಸಂಪರ್ಕನವೇ ಇಲ್ಲದೆ ಇದ್ದರೆ, ನೀವು ಅವರಿಗೆ ಹೇಗೆ ಸಹಾಯ ಮಾಡಬಲ್ಲಿರಿ? ನೀವು ನಮ್ಮ ನಡುವೆ ಪ್ರಚಾರಕರಾಗಿ ಬಂದಿರುವಾಗ, ರಾಷ್ಟ್ರೀಯತೆಯ ಕಲ್ಪನೆಯೆಲ್ಲವನ್ನೂ ದೂರವಿರಿಸಬೇಕು. ಜೀಸಸ್ ಕೇವಲ ಶಾಂಪೇನ್ ಭೋಜನಕೂಟಗಳಲ್ಲಿ ಭಾಗವಹಿಸುತ್ತ ಇಂಗ್ಲಿಷ್ ಅಧಿಕಾರವರ್ಗದ ನಡುವೆಯೇ ಓಡಾಡಿಕೊಂಡಿರಲಿಲ್ಲ. ಅವನು ತನ್ನ ಹೆಂಡತಿ ಯೂರೋಪಿಯನ್ ಸಮಾಜದ ಉನ್ನತ ವಲಯಗಳಲ್ಲಿ ಒಡನಾಟವಿಟ್ಟುಕೊಳ್ಳಬೇಕೆಂದು ಬಯಸಲಿಲ್ಲ. ನಿಮ್ಮ ಪ್ರಚಾರಕರು ಕ್ರಿಸ್ತನನ್ನು ಅನು ಸರಿಸದೇ ಇದ್ದರೆ, ಅವರಿಗೆ ತಮ್ಮನ್ನು ತಾವು ಕ್ರೈಸ್ತರೆಂದು ಕರೆದುಕೊಳ್ಳಲು ಹಕ್ಕಾದರೂ ಎಲ್ಲಿದೆ? ನಮಗೆ ಕ್ರಿಸ್ತನ ನಿಜವಾದ ಪ್ರಚಾರಕರುಬೇಕು. ಅಂಥವರು ಭಾರತಕ್ಕೆ ನೂರು ಗಟ್ಟಲೆ, ಸಾವಿರಗಟ್ಟಲೆ ಬರಲಿ. ಕ್ರಿಸ್ತನ ಜೀವನವನ್ನು ನಮಗೆ ತಂದುಕೊಡಿ, ಅದು ನಮ್ಮ ಸಮಾಜದ ಅಂತರಾಳದೊಳಕ್ಕೆ ಇಳಿಯಲಿ. ಅವನನ್ನು ಭಾರತದ ಮೂಲೆ ಮೂಲೆಗಳಲ್ಲೂ, ಪ್ರತಿಯೊಂದು ಹಳ್ಳಿಯಲ್ಲಿಯೂ ಪ್ರಚಾರಮಾಡಿ. ಆದರೆ ನಿಮ್ಮ ಪ್ರಚಾರಕರು ತಮ್ಮ ಕೆಲಸವನ್ನು ಅನ್ನಸಂಪಾದನೆಯ ಮಾರ್ಗವಾಗಿ ಭಾವಿಸದಿರಲಿ. ಅವರಲ್ಲಿ ಕ್ರಿಸ್ತನ ಕರೆ ಮೂಡಲಿ, ತಮ್ಮ ಮನಸ್ಸಿನಲ್ಲಿ ಅವರು ತಾವು ಆ ಕೆಲಸಕ್ಕಾಗಿಯೇ ಹುಟ್ಟಿದವರು ಎಂದು ಭಾವಿಸಲಿ.

“ಭಾರತವನ್ನು ಕ್ರೈಸ್ತಧರ್ಮಕ್ಕೆ ಪರಿವರ್ತಿಸುವ ವಿಚಾರದಲ್ಲಿಯಾದರೋ, ಯಾವ ಭರವಸೆ ಇದ್ದಂತೆ ಕಾಣುವುದಿಲ್ಲ. ಅದು ಒಂದು ವೇಳೆ ಸಾಧ್ಯವಿದ್ದರೂ, ಅದನ್ನು ಮಾಡತಕ್ಕದ್ದಲ್ಲ. ಅದು ಅಪಾಯಕಾರಿಯಾಗಬಹುದು; ಅದು ಎಲ್ಲ ಧರ್ಮಗಳ ನಾಶಕ್ಕೆ ನಾಂದಿಯಾಗಬಹುದು. ಇಡಿಯ ವಿಶ್ವವೇ ಒಂದು ಪರಿಪಾಕಕ್ಕೆ ಬಂದುಬಿಟ್ಟರೆ - ಅದು ಭೌತಿಕವಾಗಲಿ, ಮಾನಸಿಕವಾಗಲಿ, -ತಕ್ಷಣವೇ ಸರ್ವನಾಶವಾಗುವುದರಲ್ಲಿ ಸಂದೇಹವಿಲ್ಲ. ನಿಮಗೆ ಯಹೂದಿಗಳನ್ನು ಪರಿವರ್ತಿಸುವುದಕ್ಕೆ ಸಾಧ್ಯವಾಗಲಿಲ್ಲವೇಕೆ? ಪರ್ಷಿ ಯನ್ನರನ್ನು ಕ್ರೈಸ್ತರನ್ನಾಗಿ ಮಾಡುವುದಕ್ಕೆ ಆಗಲಿಲ್ಲವೇಕೆ? ಕ್ರೈಸ್ತನಾಗಿ ಪರಿವರ್ತಿತನಾಗು ತ್ತಿರುವ ಪ್ರತಿ ಒಬ್ಬ ಆಫ್ರಿಕನ್ನನಿಗೆ ನೂರು ಜನರಂತೆ ಮಹಮ್ಮದನ ಅನುಯಾಯಿಗಳಾಗುತ್ತಿದ್ದಾರೆ ಏಕೆ? ಭರತಖಂಡದಲ್ಲಿ, ಚೀನಾದಲ್ಲಿ ಮತ್ತು ಜಪಾನಿನಲ್ಲಿ ನಿಮಗೇಕೆ ಪ್ರಭಾವ ಬೀರಲಾಗುತ್ತಿಲ್ಲ? ಏಕೆಂದರೆ, ಪ್ರಪಂಚದಾದ್ಯಂತ ಏಕಪ್ರಕಾರವಾದ ಮನಃ ಪರಿಪಾಕ ಎಂದರೆ ಸಾವಲ್ಲದೆ ಬೇರೆಯಲ್ಲ. ಅಂಥದನ್ನು ಅಸಾಧ್ಯವಾಗಿಸುವ ವಿವೇಚನೆ ಪ್ರಕೃತಿಗಿದೆ.”

\begin{center}
\textbf{ಪ್ರಪಂಚದಾದ್ಯಂತ ರಕ್ತಪಾತವನ್ನು ತುಂಬಿರುವಿರಿ}
\end{center}

(ಸ್ವಾಮಿಗಳು ಹೇಳಿದರು:)

“ಕ್ರೈಸ್ತ ದೇಶಗಳು ಪ್ರಪಂಚವನ್ನು ರಕ್ತಪಾತದಿಂದ, ದಬ್ಬಾಳಿಕೆಯಿಂದ ತುಂಬಿಬಿಟ್ಟಿವೆ. ಈಗ ಅವರ ಕಾಲ ನಡೆಯುತ್ತಿದೆ. ನೀವು ಸಾಯಿಸುವಿರಿ, ಕೊಲೆ ಮಾಡುವಿರಿ, ಮದಿ ರಾಮತ್ತತೆಯನ್ನೂ ರೋಗವನ್ನೂ ನಮ್ಮ ದೇಶಕ್ಕೆ ತಂದು ಹರಡುವಿರಿ, ಆ ನಂತರ ಕ್ರಿಸ್ತನನ್ನು ಬೋಧಿಸುವ ಮೂಲಕ, ಅವನನ್ನು ಶಿಲುಬೆಗೇರಿಸುವ ಮೂಲಕ ಗಾಯದ ಜೊತೆಗೆ ಅಪಮಾನವನ್ನೂ ಬೆರೆಸುವಿರಿ. ಇಂತಹ ಘೋರಾತಿಘೋರಗಳನ್ನು ವಿರೋಧಿಸುವ ಯಾವ ಕ್ರೈಸ್ತ ದನಿ ನಾಡಿನಲ್ಲಿ ಕೇಳಿಬರುತ್ತಿದೆ? ನಾನಂತೂ ಕೇಳಿಲ್ಲ. ನೀವು ದೇವದೂತರು, ನಾವೆಲ್ಲ ರಾಕ್ಷಸರು ಎಂಬ ಕಲ್ಪನೆಯನ್ನು ನೀವು ತಾಯ ಹಾಲಿನೊಡನೆಯೇ ಕುಡಿದು ಬೆಳೆದಿರುವಿರಿ. ಕೇವಲ ಸೂರ್ಯನ ಬೆಳಕು ಇದ್ದರೆ ಸಾಲದು; ಅದನ್ನು ನೋಡುವುದಕ್ಕೆ ನಿಮಗೆ ಕಣ್ಣುಗಳಿರಬೇಕು. ಜನರಲ್ಲಿ ಒಳ್ಳೆಯತನ ಇದ್ದರೆ ಸಾಲದು; ಅದನ್ನು ಗುರುತಿಸುವುದಕ್ಕೆ ನಿಮ್ಮೊಳಗೆ ಒಳ್ಳೆಯತನವನ್ನು ಮೆಚ್ಚುವ ಸಾಮರ್ಥ್ಯವೂ ಇರಬೇಕು. ಮೂಢ ನಂಬಿಕೆಯೂ ಭಯಂಕರ ಪಾಷಂಡತನವೂ ಅದನ್ನು ಕೊಲೆಗೈಯುವ ತನಕ ಈ ಸಾಮರ್ಥ್ಯ ಪ್ರತಿಯೊಂದು ಹೃದಯದಲ್ಲಿಯೂ ಇರುತ್ತದೆ.”

ಅನಂತರ ಕಾನಂದರು ತುಂಬ ಸುಂದರವಾದೊಂದು ಉಪಮಾನವನ್ನು ಬಳಸಿ ಎಲ್ಲ ಧರ್ಮಗಳಲ್ಲಿಯೂ ಮೂಲಭೂತ ಸತ್ಯಗಳು ಒಂದೇ, ಉಳಿದುದೆಲ್ಲ ಅಮುಖ್ಯವಾದದ್ದು ಹಾಗೂ ಗೌಣ ಎಂಬುದನ್ನು ಶ್ರುತಪಡಿಸಿದರು. ಕಾಡುಮನುಷ್ಯನೊಬ್ಬನಿಗೆ ಕೆಲವು ಒಡವೆಗಳು ದೊರಕುತ್ತವೆ. ಅವನ್ನು ಇಷ್ಟಪಟ್ಟ ಅವನು ಒಂದು ನಾರಿನಿಂದ ಅವನ್ನೆಲ್ಲ ಪೋಣಿಸಿಕೊಂಡು ಕೊರಳಿಗೆ ಹಾಕಿಕೊಳ್ಳುತ್ತಾನೆ. ಸ್ವಲ್ಪ ನಾಗರಿಕನಾಗುತ್ತಲೂ ಅವನು ನಾರನ್ನು ಬಿಟ್ಟು ದಾರವನ್ನು ಬಳಸಬಹುದು. ಇನ್ನೂ ಸ್ವಲ್ಪ ತಿಳಿದವನಾದ ಮೇಲೆ ರೇಷ್ಮೆ ದಾರದಲ್ಲಿ ಪೋಣಿಸಿಕೊಳ್ಳಬಹುದು; ನಾಗರಿಕತೆಯ ಉತ್ತುಂಗವನ್ನೇರಿದ ಮೇಲೆ ಒಳ್ಳೆಯದೊಂದು ಬಂಗಾರದ ಎಳೆಯನ್ನೇ ಮಾಡಿಸಿಕೊಳ್ಳಬಹುದು. ಆದರೆ ಈ ಎಲ್ಲ ಬದಲಾವಣೆಗಳ ಉದ್ದಕ್ಕೂ ಅವನಿಗೆ ಮೊದಲು ದೊರಕಿದ್ದ ಒಡವೆಗಳು - ಎಂದರೆ ಮೂಲಭೂತ ಸತ್ಯಗಳು - ಏನೂ ವ್ಯತ್ಯಾಸವಾಗದೆ ಹಾಗೇ ಉಳಿದಿರುತ್ತವೆ.

“ಹಿಂದೂವು ಕ್ರೈಸ್ತಧರ್ಮವನ್ನು ಟೀಕಿಸಲು ಬಯಸಿದ್ದೇ ಆದರೆ ದಂತಕಥೆ ಪವಾಡ ಮತ್ತಿತರ ಬೈಬಲ್ನ ಅರ್ಥರಹಿತ ಅಂಶಗಳನ್ನು ಕುರಿತು ಮಾತನಾಡಬಹುದು; ಆದರೆ ಅವನು ಬೆಟ್ಟದ ಮೇಲಿನ ಉಪದೇಶವನ್ನು ಕುರಿತೋ, ಜೀಸಸ್ನ ಸುಂದರವಾದ ಜೀವನವನ್ನು ಕುರಿತೋ ಒಂದೇ ಮಾತನ್ನೂ ಆಡುವುದಿಲ್ಲ. ಹಾಗೆಯೇ ಕ್ರೈಸ್ತನು ಹಿಂದೂ ಧರ್ಮವನ್ನು ಟೀಕಿಸುವುದಾದರೆ ದೇಗುಲಗಳನ್ನೋ ಶಾಸ್ತ್ರಾಧಾರಗಳನ್ನೋ ಟೀಕಿಸಬಹುದು; ಆದರೆ ಅವನು ಹಿಂದೂವಿನ ನೈತಿಕತೆಯನ್ನು, ತತ್ತ್ವವನ್ನು ಕುರಿತು ಏನನ್ನೂ ಹೇಳುವುದಿಲ್ಲ (ಹೇಳಕೂಡದು). ಯಹೂದಿಗೆ ನೆರವಾಗಿರಿ, ಅವನು ನಿಮಗೆ ನೆರವಾಗಲಿ. ಹಿಂದೂವಿಗೆ ನೆರವಾಗಿರಿ, ಅವನು ನಿಮಗೆ ನೆರವಾಗಲಿ. ಯಾವ ಮನುಷ್ಯನಿಗೇ ಆಗಲಿ, ಎಲ್ಲೆಡೆ ಒಳ್ಳೆಯತನವನ್ನು ನೋಡಲಾರದ ಮಾತ್ರಕ್ಕೆ ಅವನಿಗೆ ಒಳ್ಳೆಯತನವನ್ನು ನೋಡುವ ಸಾಮರ್ಥ್ಯವೇ ಇಲ್ಲವೆಂದು ನಾನು ಹೇಳಲಾರೆ. ಕ್ರಿಸ್ತನ ನೈತಿಕ ಸತ್ವದ ಸೌಂದರ್ಯವೇ ಬುದ್ಧನ ನೈತಿಕ ಸತ್ವದಲ್ಲಿಯೂ ಇರುವುದು. ನಮಗೆ ಅಗತ್ಯವಾಗಿರುವುದು ಜೀರ್ಣಿಸಿಕೊಳ್ಳುವುದೇನೂ ಅಲ್ಲ, ಕೇವಲ ಹೊಂದಾಣಿಕೆ ಮತ್ತು ಸಾಮರಸ್ಯ. ಪ್ರಬೋಧಕರನ್ನು ನಾನು ಕೇಳುವುದು - ಮೊದಲು ನಿಮ್ಮ ರಾಷ್ಟ್ರೀಯತೆಯ ಭಾವನೆಯನ್ನು ಬಿಟ್ಟುಬಿಡಿ; ಎರಡನೆಯದಾಗಿ, ಮತಪಂಥದ ಕಲ್ಪನೆಯನ್ನು ತೆಗೆದುಹಾಕಿ. ದೇವರ ಮಕ್ಕಳು ಯಾವ ಪಂಥಕ್ಕೂ ಸೇರಿದವರಲ್ಲ.

“ಭಾರತೀಯ ನಾರಿಯರ ಬಗ್ಗೆ, ಅವರ ದೋಷಗಳ ಬಗ್ಗೆ, ಪರಿಸ್ಥಿತಿಯ ಬಗ್ಗೆ ಬಹಳಷ್ಟು ಹೇಳಲಾಗಿದೆ. ದೋಷಗಳಿವೆ; ಅವನ್ನು ಸರಿಪಡಿಸಲು ದೇವರು ನಮಗೆ ಸಹಾಯ ಮಾಡಲಿ. ನಮ್ಮ ಸ್ತ್ರೀಯರ ಬಗ್ಗೆ ನಿಮ್ಮ ಟೀಕೆಗಳಿಗೆ ನಾವು ಕೃತಜ್ಞರಾಗಿದ್ದೇವೆ. ಆದರೆ ಅವರ ಬಗ್ಗೆ ನೀವು ಮಾತನಾಡುತ್ತಿರುವಾಗ, ಅಮೆರಿಕಾದಲ್ಲಿ ಒಂದು ಡಜನ್ ಆಧ್ಯಾತ್ಮಿಕ ಮಹಿಳೆಯರನ್ನು ನೋಡಿದರೆ ನನಗೆ ಸಂತೋಷವಾದೀತು ಎಂದು ನಾನು ಹೇಳದಿರಲಾರೆ. ಒಳ್ಳೆಯ ಉಡುಪು, ಶ‍್ರೀಮಂತಿಕೆ, ಪ್ರದೀಪ್ತ ಸಮಾಜ, ಒಪೆರಾಗಳು, ಕಾದಂಬರಿಗಳು - ಎಲ್ಲವೂ ಸರಿ; ಪುರುಷನಿಗೆ ಅಥವಾ ಮಹಿಳೆಗೆ ಬೌದ್ಧಿಕತೆಯೇ ಸರ್ವಸ್ವವಲ್ಲ. ಆಧ್ಯಾತ್ಮಿಕತೆಯೂ ಇರಬೇಕಾಗುತ್ತದೆ, ಆದರೆ ಈ ಭಾಗವು ಕ್ರೈಸ್ತ ದೇಶಗಳಲ್ಲಿ ಸಂಪೂರ್ಣವಾಗಿ ಕೊರೆಯಾಗಿಯೇ ಕಾಣುತ್ತಿದೆ. ಅದು ಜೀವಂತವಾಗಿರುವುದು ಭಾರತದಲ್ಲಿ”. ಕಳೆದ ರಾತ್ರಿ ವಿವೇಕಾನಂದರ ದೊಡ್ಡ ಸಭೆಯ ಸಭಿಕರೆಲ್ಲ ಬಹು ಗೌರವದಿಂದ ಅವರ ಮಾತುಗಳನ್ನು ಕೇಳಿದರು; ಹಾಗೂ ಒಂದೆರೆಡು ಸಲ ಹೃತ್ಪೂರ್ವಕವಾಗಿ ಕರತಾಡನ ಮಾಡಿದರು.

\begin{center}
\textbf{ಅಲೆ ಅಲೆಯನ್ನು ಅನುಸರಿಸುವಂತೆ}\footnote{\enginline{1. New Discoveries, Vol. 2, pp. 441-443}}
\end{center}

\begin{center}
(ಡೆಟ್ರಾಯಿಟ್ ಟ್ರಿಬ್ಯೂನ್, ೨೦ ಮಾರ್ಚ್ ೧೮೯೪)\\ಆತ್ಮವು ಆತ್ಮವನ್ನು ಅನುಸರಿಸುವುದು, ಕಾನಂದರ ಪ್ರಕಾರ.
\end{center}

ಕಳೆದ ರಾತ್ರಿ ಸಭಾಂಗಣದಲ್ಲಿ ವಿವೇಕಾನಂದರು ಸುಮಾರು ನೂರೈವತ್ತು ಜನರ ಸಭೆಯನ್ನು (ಜರ್ನಲ್ ಪ್ರಕಾರ ಐನೂರು) ಉದ್ದೇಶಿಸಿ “ಏಷ್ಯಾದ ಜ್ಯೋತಿಯ ಧರ್ಮ - ಬೌದ್ಧಧರ್ಮ”\footnote{2. ಈ ಉಪನ್ಯಾಸದ ಪದಶಃ ವರದಿ ಲಭ್ಯವಿಲ್ಲ.} ಎಂಬ ವಿಷಯವಾಗಿ ಉಪನ್ಯಾಸ ಮಾಡಿದರು. ಗೌರವಾನ್ವಿತ ವಿದ್ಯಾ ಲಂಕಾರ ಎಮ್​. ಡಿಕಿನ್ಸನ್ ಅವರನ್ನು ಸಭಿಕರಿಗೆ ಪರಿಚಯ ಮಾಡಿಕೊಟ್ಟರು.

ಮಿ. ಡಿಕಿಸ್ಸನ್ ತಮ್ಮ ಮೊದಲ ಮಾತಿನಲ್ಲಿ “ಈ ಧರ್ಮದ ವ್ಯವಸ್ಥೆ ದಿವ್ಯವಾದದ್ದು, ಅದು ದುರ್ಗತಿಗೆ ಹೋಗುವಂಥದ್ದು ಎಂದು ಯಾರು ತಾನೆ ಹೇಳಬಲ್ಲರು? ಅನುಭಾವದ ಗೆರೆಯನ್ನು ಯಾರು ಎಳೆಯಬೇಕು?” ಎಂದು ಪ್ರಶ್ನಿಸಿದರು.

ಅಲ್ಲದೆ, ಒಂದು ಕಾಲದಲ್ಲಿ ಬುದ್ಧನ ಅನುಯಾಯಿಗಳು ಇಷ್ಟವಿಲ್ಲದಿದ್ದರೂ ಕ್ರೈಸ್ತ ಧರ್ಮದ ಅನುಯಾಯಿಗಳಾಗಿದ್ದರು ಎಂದೂ ಹೇಳಿದರು. ವಿವೇಕಾನಂದರು ಕಿತ್ತಳೆ ಹಳದಿ ಉಡುಪಿನಲ್ಲಿದ್ದರು; ಕಟಿಬಂಧವನ್ನೂ, ಯಾವುದೋ ರೇಷ್ಮೆ ನವಿರಿನ ಪ್ರಾಚ್ಯ ವಸ್ತ್ರದ ರುಮಾಲನ್ನೂ ಧರಿಸಿದ್ದ ಅವರು ಅದರ ಇಳಿಬಿದ್ದ ತುದಿಯನ್ನು ಒಂದು ಭುಜದ ಮುಂದಕ್ಕೆ ಬರುವಂತೆ ಇಟ್ಟುಕೊಂಡಿದ್ದರು.

ವಿವೇಕಾನಂದರು ಭಾರತದ ಪುರಾತನ ಧರ್ಮಗಳನ್ನು ದೀರ್ಘವಾಗಿ ಪರಾಮರ್ಶೆ ಮಾಡಿದರು. ಯಜ್ಞವೇದಿಕೆಯಲ್ಲಿ ನಡೆಯುತ್ತಿದ್ದ ಪ್ರಾಣಿವಧೆಯ ಬಗ್ಗೆ; ಬುದ್ಧನ ಜನ್ಮ ಹಾಗೂ ಜೀವನದ ಬಗ್ಗೆ; ಈ ಸೃಷ್ಟಿಗೆ ಕಾರಣಗಳೇನು, ಅದು ಅಸ್ತಿತ್ವದಲ್ಲಿರುವುದೇಕೆ ಎಂಬ ಗೊಂದಲದ ಪ್ರಶ್ನೆಗಳನ್ನು ತನಗೆ ತಾನು ಹಾಕಿಕೊಂಡದ್ದು; ಅವುಗಳಿಗೆ ಪರಿಹಾರ ಕಂಡುಹಿಡಿಯಲು ಪರಮ ಶ್ರದ್ಧೆಯಿಂದ ತನ್ನೊಳಗೇ ಹೋರಾಟ ನಡೆಸಿದ್ದು; ಕೊನೆಯ ಫಲಿತಾಂಶ - ಎಲ್ಲವನ್ನೂ ಕುರಿತು ಹೇಳಿದರು.

ಬುದ್ಧನು ಉಳಿದೆಲ್ಲ ಮಾನವರಿಗಿಂತ ಮಿಗಿಲಾಗಿ ಮೇಲೆದ್ದು ನಿಲ್ಲುವನು ಎಂದರವರು. ಅವನೊಬ್ಬನ ವಿಚಾರವಾಗಿ, ಅವನ ಮಿತ್ರರಾಗಲಿ ಶತ್ರುಗಳಾಗಲಿ, ಎಲ್ಲರ ಹಿತಕ್ಕಾಗಿಯಲ್ಲದೆ ತನಗೆಂದೇ ಒಂದು ಉಸಿರನ್ನು ಎಳೆದನೆಂದಾಗಲಿ ಒಂದು ತುತ್ತು ಅನ್ನವನ್ನು ತಿಂದನೆಂದಾಗಲಿ ಎಂದಿಗೂ ಹೇಳಲಾರರು ಎಂದರು. ಕಾನಂದರು ಮುಂದುವರೆದು,

“ಅವನೆಂದೂ ಆತ್ಮದ ದೇಹಾಂತರಪ್ರಾಪ್ತಿಯನ್ನು ಬೋಧಿಸಲಿಲ್ಲ. ಸಮುದ್ರದ ಅಲೆಯು ಮೇಲೇರಿ ಬೆಳೆದು, ಮುಂದಿನ ಅಲೆಗೆ ತನ್ನ ಬಲವನ್ನಲ್ಲದೆ ಇನ್ನೇನನ್ನೂ ಬಿಡದೆ ನಾಶವಾಗುವುದೋ ಹಾಗೆಯೇ ಒಂದು ಆತ್ಮವು ತನ್ನ ಮುಂದಿನ ಆತ್ಮಕ್ಕೆ ಸಂಬಂಧಿಸಿದೆ ಎಂದು ತಾನು ನಂಬುವುದಾಗಿ ಅವನು ಹೇಳಿದನು. ದೇವರಿದ್ದಾನೆ ಎಂದೂ ಅವನು ಬೋಧಿಸಲಿಲ್ಲ, ದೇವರಿಲ್ಲ ಎಂದೂ ಹೇಳಲಿಲ್ಲ.

‘ನಾವೇಕೆ ಒಳ್ಳೆಯವರಾಗಿರಬೇಕು?’ ಎಂದು ಅವನ ಶಿಷ್ಯರು ಅವನನ್ನು ಕೇಳಿದರು. ‘ಏಕೆಂದರೆ ನೀವು ಪರಂಪರಾಗತವಾಗಿ ಒಳ್ಳೆಯತನವನ್ನು ಪಡೆದಿರುವಿರಿ. ನಿಮ್ಮ ಸರದಿಯಲ್ಲಿ ನೀವು ನಿಮ್ಮ ಮುಂದಿನವರಿಗೆ ಒಳ್ಳೆಯತನದ ಪರಂಪರೆಯನ್ನು ಬಿಟ್ಟು ಹೋಗಬೇಕು. ನಾವೆಲ್ಲರೂ ಒಳ್ಳೆಯತನಕ್ಕಾಗಿ ಒಳ್ಳೆಯತನವನ್ನು ಒಗ್ಗೂಡಿಸಿ ಬೆಳೆಸುತ್ತ ಮುಂದುವರೆಯೋಣ’ ಎಂದು ಅವನು ಉತ್ತರಿಸಿದನು. “ಅವನೇ ಆದಿಮ ಪ್ರವಾದಿ. ಅವನು ಯಾರಿಗೂ ಕಟೂಕ್ತಿಯನ್ನಾಡಲಿಲ್ಲ, ಯಾರಿಂದಲೂ ಕಟೂಕ್ತಿಯನ್ನು ತನ್ನ ಮೇಲೆ ಆವಾಹಿಸಿಕೊಳ್ಳಲಿಲ್ಲ. ಧರ್ಮದ ವಿಚಾರದಲ್ಲಿ, ನಮ್ಮ ಮುಕ್ತಿಗಾಗಿ ನಾವೇ ಶ್ರಮಿಸಬೇಕು ಎನ್ನುವುದನ್ನು ಅವನು ನಂಬಿಕೊಂಡಿದ್ದ.

“ಮೃತ್ಯುಶಯ್ಯೆಯಿಂದ ಅವನು ‘ನಾನು ನಿಮಗೆ ಹೇಳಲಾರೆ. ಯಾರೂ ಹೇಳಲಾರರು. ಯಾರ ಮೇಲೂ ಅವಲಂಬಿಸದಿರಿ. ನಿಮ್ಮ ಧರ್ಮಕ್ಕಾಗಿ (ಮುಕ್ತಿಗಾಗಿ) ನೀವೇ ಶ್ರಮಿಸಿರಿ’ ಎಂದು ಹೇಳಿದನು. “ಮಾನವ-ಮಾನವರ ನಡುವೆ, ಮಾನವ-ಪ್ರಾಣಿಗಳ ನಡುವೆ ಅಸಮಾನತೆಯನ್ನು ಅವನು ವಿರೋಧಿಸಿದನು. ಎಲ್ಲ ಜೀವಿಗಳೂ ಸಮಾನರು ಎಂದು ಅವನು ಬೋಧಿಸಿದನು. ಮದ್ಯಪಾನವನ್ನು ನಿಷೇಧಿಸುವ ಸಿದ್ಧಾಂತವನ್ನು ಎತ್ತಿ ಹಿಡಿದ ಮೊದಲಿಗನವನು. ‘ಒಳ್ಳೆಯವರಾಗಿರಿ, ಒಳ್ಳೆಯದನ್ನು ಮಾಡಿರಿ. ದೇವರೆಂಬು ವನೊಬ್ಬನು ಇರುವುದಾದರೆ, ನೀವು ಒಳ್ಳೆಯವರಾಗುವ ಮೂಲಕ ಅವನನ್ನು ಪಡೆದು ಕೊಳ್ಳಿರಿ. ದೇವರಿಲ್ಲವೆಂದಾದರೂ, ಒಳ್ಳೆಯವರಾಗಿರುವುದು ಒಳ್ಳೆಯದೇ.ಯಾತನೆ ಯನ್ನನುಭವಿಸುತ್ತಿರುವುದಕ್ಕೆ ವ್ಯಕ್ತಿಯೇನಿಂದ್ಯನು, ಒಳ್ಳೆಯವನಾಗಿರುವುದಕ್ಕೆ ವ್ಯಕ್ತಿಯೇ ಪ್ರಶಂಸಾರ್ಹನು’.

“ಧರ್ಮಪ್ರಚಾರಕರನ್ನು ಮೊಟ್ಟಮೊದಲು ಅಸ್ತಿತ್ವಕ್ಕೆ ತಂದವನೇ ಅವನು. ಭಾರತದ ತುಳಿಯಲ್ಪಟ್ಟ ಮಿಲಿಯಗಟ್ಟಲೆ ದಲಿತರನ್ನು ಉದ್ಧಾರಮಾಡುವುದಕ್ಕಾಗಿ ಬಂದವನವನು. ಅವರಿಗೆ ಅವನ ತತ್ತ್ವ ಅರ್ಥವಾಗದಿದ್ದರೂ, ಮನುಷ್ಯ ಎಂಥವನೆಂದು ಅರ್ಥಮಾಡಿಕೊಂಡರು; ಅವನ ಬೋಧನೆಯನ್ನು ಅನುಸರಿಸಿದರು.”

ಉಪಸಂಹಾರವಾಗಿ ವಿವೇಕಾನಂದರು ಕ್ರೈಸ್ತಧರ್ಮದ ತಳಹದಿಯೇ ಬೌದ್ಧಧರ್ಮ ಎಂದರು; ಕ್ಯಾಥೊಲಿಕ್ ಚರ್ಚ್ ಬಂದದ್ದು ಬೌದ್ಧಧರ್ಮದಿಂದಲೇ ಎಂದು ಹೇಳಿದರು.

\begin{center}
\textbf{ದಾರಿಬದಿಯ ಕಥೆಗಳು\supskpt{\footnote{\enginline{1. New Discoveries, Vol. 2, pp. 436}}}}
\end{center}

\begin{center}
(ಡೆಟ್ರಾಯಿಟ್ ಈವಿನಿಂಗ್ ನ್ಯೂಸ್, ೨೧ ಮಾರ್ಚ್ ೧೮೯೪)
\end{center}

ಕುತೂಹಲವೇ ಅಮೆರಿಕನ್ನರ ಎದ್ದು ತೋರುವ ಗುಣಲಕ್ಷಣ ಎಂದ ನಮ್ಮ ಹಿಂದೂ ಭೇಟಿಕಾರರು, ಜ್ಞಾನಸಂಪಾದನೆಗೆ ಅದೇ ಹಾದಿ ಎಂದೂ ಸೇರಿಸಿದರು. ಇದು ಅಮೆರಿಕನ್ನರ ಬಗ್ಗೆ ಬಹುಕಾಲದಿಂದಲೂ ಇರುವ ಯೂರೋಪಿಯನ್ನರ ಅಭಿಪ್ರಾಯ - ಸ್ಪಷ್ಟವಾಗಿ ಹೇಳುವುದಾದರೆ ಯಾಂಕೀ ಗುಣಲಕ್ಷಣ; ಬಹುಶಃ ಹಿಂದೂವಿನ ಮಾತು ಅವರು ಭಾರತದಲ್ಲಿ ಇಂಗ್ಲಿಷ್ ಜನರಿಂದ “ಯಾಂಕೀ” ಬಗ್ಗೆ ಕೇಳಿದ್ದರ ಪ್ರತಿಧ್ವನಿಯಾಗಿರಬಹುದು.

\begin{center}
\textbf{ಹಿಂದೂ ಸಂನ್ಯಾಸಿ\supskpt{\footnote{\enginline{1. New Discoveries, Vol. 2, pp. 6-7. ನೋಡಿ "Swami Vivekananda on India", Completed works, 11: 479-481}}}}
\end{center}

\begin{center}
(ಬೇ ಸಿಟಿ ಟೈಮ್ಸ್ ಪ್ರೆಸ್, ೨೧ ಮಾರ್ಚ್ ೧೮೯೪)
\end{center}

ಕಳೆದ ಸಂಜೆ ಒಪೇರಾ ಹೌಸನಲ್ಲಿ ಅವರು ಒಂದು ಆಸಕ್ತಿಯ ಉಪನ್ಯಾಸವನ್ನು ನೀಡಿದರು. ಬೇ ನಗರದ ಜನರಿಗೆ ಕಳೆದ ರಾತ್ರಿ ಸ್ವಾಮಿ ವಿವೇಕಾನಂದರು ಕೊಟ್ಟ ಉಪನ್ಯಾಸದಂಥದನ್ನು ಕೇಳುವ ಅವಕಾಶ ಅಪರೂವೇ. ಕೊಲ್ಕತ್ತದಲ್ಲಿ ಸುಮಾರು ಮೂವತ್ತು ವರ್ಷಗಳ ಹಿಂದೆ ಜನ್ಮ ತಳೆದ ಮಾನ್ಯರು ಭಾರತ ದೇಶೀಯರು. ಭಾಷಣ ಕಾರರನ್ನು ಡಾ. ಸಿ. ಟಿ. ನ್ಯೂಕಿರ್ಕ್ ಅವರು ಪರಿಚಯ ಮಾಡಿಕೊಟ್ಟಾಗ ಒಪೇರಾ ಹೌಸ್ನ ಕೆಳ ಸಭಾಂಗಣ ಅರ್ಧದಷ್ಟು ಭರ್ತಿಯಾಗಿತ್ತು. ಭಾಷಣದ ನಡುವೆ ಅವರು ಅಮೆರಿಕಾದ ಜನರನ್ನು ಸರ್ವಶಕ್ತ ಡಾಲರ್ನ ಉಪಾಸಕರಾಗಿರುವುದಕ್ಕಾಗಿ ಛೇಡಿಸಿದರು. ಹೌದು, ಭಾರತದಲ್ಲಿ ಜಾತಿ ಇರುವುದು ನಿಜ. ಅಲ್ಲಿ ಕೊಲೆಗಾರನೊಬ್ಬ ಎಲ್ಲರಿಗಿಂತ ಮೇಲಿನವ ನಾಗಲಾರ. ಇಲ್ಲಿ, ಅವನಿಗೆ ಒಂದು ಮಿಲಿಯ ಡಾಲರ್ ಸಿಕ್ಕಿಬಿಟ್ಟರೆ ಅವನು ಎಲ್ಲರಂತೆಯೇ. ಭಾರತದಲ್ಲಿ, ಒಬ್ಬನು ಒಂದು ಸಲ ಕೇಡಿಗನೆನ್ನಿಸಿಕೊಂಡುಬಿಟ್ಟರೆ, ಅವನನ್ನು ಶಾಶ್ವತವಾಗಿ ಕೆಳಕ್ಕೆ ಹಾಕುವರು. ಹಿಂದೂ ಧರ್ಮದ ಒಂದು ಮಹತ್ವದ ಅಂಶವೆಂದರೆ, ಇತರ ಧರ್ಮಗಳ, ನಂಬಿಕೆಗಳ ಬಗ್ಗೆ ಇರುವ ಸಹಿಷ್ಣುತೆ. ಇನ್ನಿತರ ಪ್ರಾಚ್ಯ ದೇಶಗಳಲ್ಲಿಗಿಂತ ಭಾರತದಲ್ಲಿ ಪ್ರಚಾರಕರು ಧರ್ಮಗಳನ್ನು ಹೀಗಳೆಯುವುದು ಹೆಚ್ಚು - ಏಕೆಂದರೆ ಅವರನ್ನು ಹಾಗೆ ಮಾಡಲು ಬಿಡುತ್ತಾರೆ, ಆ ಮೂಲಕ ತಮ್ಮ ಮೂಲಭೂತ ನಂಬಿಕೆಯಂತೆ ನಡೆದುಕೊಳ್ಳುತ್ತಾರೆ. ಕಾನಂದರು ತುಂಬ ವಿದ್ಯಾವಂತರು, ನಯನಾಜೂಕಿನ ಸಭ್ಯಮನುಷ್ಯ. ಡೆಟ್ರಾಯಿಟ್ನಲ್ಲಿ ಅವರನ್ನು ಹಿಂದೂಗಳು ತಮ್ಮ ಮಕ್ಕಳನ್ನು ನದಿಗೆಸೆಯುತ್ತಾರಂತೆ ನಿಜವೇ ಎಂದು ಯಾರೋ ಕೇಳಿದರಂತೆ. ಅದಕ್ಕೆ ಅವರು ಹಾಗೇನೂ ಮಾಡುವುದಿಲ್ಲ, ಅಲ್ಲದೆ ಮುದುಕಿಯರನ್ನು ಸುಡುಗಂಬಕ್ಕೆ ಕಟ್ಟಿ ಸುಡುವುದೂ ಇಲ್ಲ ಎಂದು ಉತ್ತರಿಸಿದ ರಂತೆ. ಭಾಷಣಕಾರರು ಈ ರಾತ್ರಿ ಸ್ಯಾಗಿನಾದಲ್ಲಿ ಉಪನ್ಯಾಸ ಮಾಡುವರು.

\begin{center}
\textbf{ಕಾನಂದರ ಆಗಮನ\supskpt{\footnote{\enginline{1. New Discoveries, Vol. 2, pp. 11}}}}
\end{center}

\begin{center}
(ಸ್ಯಾಗಿನಾ ಈವಿನಿಂಗ್ ನ್ಯೂಸ್, ೨೧ ಮಾರ್ಚ್ ೧೮೯೪)
\end{center}

ಬೇ ನಗರದಿಂದ ಇಂದು ಮಧ್ಯಾಹ್ನ ಆಗಮಿಸಿದ ಹಿಂದೂ ಸಂನ್ಯಾಸಿ ಸ್ವಾಮಿ ವಿವೇಕಾನಂದರು ವಿನ್ಸೆಂಟ್ನಲ್ಲಿ ಇಳಿದುಕೊಂಡಿದ್ದಾರೆ. ಶ‍್ರೀಮಂತ ಅಮೆರಿಕಾದವರ ಹಾಗೆ ಉಡುಪು ಧರಿಸುವ ಅವರು ಅತ್ಯುತ್ತಮ ಇಂಗ್ಲಿಷ್ ಮಾತನಾಡುತ್ತಾರೆ. ಮಧ್ಯಮ ಎತ್ತರ ಕ್ಕಿಂತ ಸ್ವಲ್ಪ ಹೆಚ್ಚಾಗಿರುವ ಅವರು ಕಟ್ಟುಮಸ್ತಾಗಿದ್ದಾರೆ ಹಾಗೂ ಅವರ ಮೈಬಣ್ಣ ಭಾರತೀ ಯರ ಮೈಬಣ್ಣವನ್ನು ಹೋಲುತ್ತದೆ. ‘ನ್ಯೂಸ್’ ಪ್ರತಿನಿಧಿಯ ಪ್ರಶ್ನೆಯೊಂದಕ್ಕೆ ಉತ್ತರಿಸುತ್ತ ಅವರು ತಾವು ಇಂಗ್ಲಿಷ್ ಕಲಿತದ್ದು ಖಾಸಗಿ ಶಿಕ್ಷಕರಿಂದ ಹಾಗೂ ಹಿಂದೂಸ್ಥಾನಕ್ಕೆ ಭೇಟಿ ಕೊಟ್ಟ ಯೂರೋಪಿಯನ್ನರೊಂದಿಗಿನ ಒಡನಾಟದಿಂದ ಎಂದರು. ಮುಂದುವರೆದು, ಈ ರಾತ್ರಿಯ ತಮ್ಮ ಮಾತು ಹಿಂದೂಧರ್ಮವನ್ನು ವಿವರಿಸುತ್ತದೆ ಎಂದೂ, ಹಿಂದೂಗಳು ಅಕ್ರೈಸ್ತ ಅನಾಗರಿಕರೇನೂ ಅಲ್ಲ, ಆದರೆ ತಮ್ಮ ಭವ್ಯ ಭವಿಷ್ಯದಲ್ಲಿ ನಂಬಿಕೆ ಉಳ್ಳವರು ಎಂದು ತೋರಿಸುತ್ತದೆ ಎಂದೂ ಹೇಳಿದರು.

\begin{center}
\textbf{ಭಾರತದ ರೀತಿ-ರಿವಾಜುಗಳು\supskpt{\footnote{\enginline{2. Vedanta Kesari, 1987 anual Issue, pp. 446 ಮತ್ತು New Discoveries, Vol. 2, pp. 37-39}}}}
\end{center}

\begin{center}
(ದಿ ಲಿನ್ ಡೈಲಿ ಈವಿನಿಂಗ್ ಐಟಮ್​, (ದಿನಾಂಕ?)
\end{center}

\begin{center}
ನಾರ್ತ್ ಷೋರ್ ಕ್ಲಬ್
\end{center}

ಭಾರತದ ಸುಶಿಕ್ಷಿತ ಸಂನ್ಯಾಸಿ ಸ್ವಾಮಿ ವಿವೇಕಾನಂದರಿಂದ ಮಂಗಳವಾರ ಮಧ್ಯಾಹ್ನದ ಗೋಷ್ಠಿಯನ್ನುದ್ದೇಶಿಸಿ ಭಾಷಣ - ತಮ್ಮ ದೇಶದ ರೀತಿ - ರಿವಾಜುಗಳ ವರ್ಣನೆ\footnote{3. ೧೮೯೪ರ ಏಪ್ರಿಲ್ ೧೭ರಂದು ಕೊಡಲಾದ ಈ ಉಪನ್ಯಾಸದ ಪದಶಃ ವರದಿ ಲಭ್ಯವಿಲ್ಲ.}

ನಾರ್ತ್ ಷೋರ್ ಕ್ಲಬ್ನಲ್ಲಿ ಮಂಗಳವಾರ ನಡೆದ ಮಧ್ಯಾಹ್ನದ ಗೋಷ್ಠಿಯಸಭೆ ದೊಡ್ಡದಾಗಿತ್ತು, ಪ್ರದೀಪ್ತವಾಗಿತ್ತು, ಅನೇಕಾನೇಕ ಬಹುಮಾನ್ಯ ಅತಿಥಿಗಳಿಂದ ಕೂಡಿ ಅತ್ಯುನ್ನತ ಸಂಸ್ಕೃತಿಯನ್ನು ಪ್ರತಿನಿಧಿಸುತ್ತಿತ್ತು. ನಿರರ್ಗಳ ಸುಭಗಸ್ರೋತವಾಗಿ ಇಂಗ್ಲಿಷ್ ಮಾತನಾಡುವ ಭಾರತದ ಸುಶಿಕ್ಷಿತ ಸಂನ್ಯಾಸಿ ಸ್ವಾಮಿ ವಿವೇಕಾನಂದರು ತಮ್ಮ ದೇಶದ ರೀತಿ-ರಿವಾಜುಗಳ ಬಗ್ಗೆ ತುಂಬ ಆಸಕ್ತಿ ಮೂಡಿಸುವಂತಹ ವಿವರಣೆಯನ್ನು ಕೊಟ್ಟರು. ತಮ್ಮ ವಿಶಿಷ್ಟ ರುಮಾಲು ಹಾಗೂ ಹಳದಿ ಉಡುಪನ್ನು ಧರಿಸಿದ್ದ ಸ್ವಾಮಿ (Suami) ವಿವೇಕಾನಂದರು ಭಾರತವು ಉತ್ತರ, ದಕ್ಷಿಣ ಎಂಬ ಎರಡು ಭಾಗವಾಗಿ ವಿಂಗಡಿಸಲ್ಪಟ್ಟಿದೆ ಎನ್ನು ತ್ತ ಪ್ರಾರಂಭಿಸಿದರು. ಇವುಗಳಲ್ಲಿನ ಭಾಷೆ ಹಾಗೂ ರೂಢಿಗಳು ಅದೆಷ್ಟು ವಿಭಿನ್ನವೆಂದರೆ, ಉತ್ತರಭಾರತದಿಂದ ಬಂದಿದ್ದ ಒಬ್ಬ ಭಾಷಣಕಾರರು, ದಕ್ಷಿಣಭಾರತದಿಂದ ಬಂದಿದ್ದ ತಮ್ಮದೇ ದೇಶೀಯರೊಬ್ಬರನ್ನು ಸರ್ವಧರ್ಮಸಮ್ಮೇಳನದಲ್ಲಿ ಇಂಗ್ಲಿಷಿನಲ್ಲಿ ಮಾತನಾಡಿಸುವಂತಾಯಿತು; ಏಕೆಂದರೆ ಇಬ್ಬರಿಗೂ ಪರಸ್ಪರರ ದೇಶೀಯ ಭಾಷೆ ಗೊತ್ತಿ ರಲಿಲ್ಲ. ಇಡೀ ದೇಶದಲ್ಲಿ ಒಂಭತ್ತು ಭಾಷೆಗಳಿವೆ ಹಾಗೂ ನೂರು ಪ್ರಾಂತೀಯ ಆಡುಭಾಷೆಗಳಿವೆ.

ಧರ್ಮದಲ್ಲಿ ತಕ್ಕಮಟ್ಟಿನ ಏಕರೂಪತೆಯಿದೆಯಾದರೂ, ಪ್ರತಿಯೊಂದು ಮತಪಂಥವೂ ತನಗೆ ತಾನೇ ಒಂದು ಧರ್ಮ, ಒಂದು ನಿಯಮ ಎಂದುಕೊಂಡಿದೆ. ಭಾರತದ ಬಗ್ಗೆ ಅಪರಿಪೂರ್ಣ ಗ್ರಹಿಕೆಯಿಂದ ಬರೆಯಲ್ಪಟ್ಟ ಅನೇಕ ತಪ್ಪು ವಿವರಣೆಗಳಿವೆ; ಇವುಗಳನ್ನಾ ಧರಿಸಿ ಅತ್ಯಂತ ಘಾತುಕ ನಿರ್ಣಯಗಳನ್ನು ಮಾಡಲಾಗಿದೆ. ಹಿಂದೂವಿಗೆ ಪ್ರತಿಯೊಂದೂ ಧರ್ಮಕ್ಕೆ ಅಧೀನವಾಗಿರುವಂಥದು; ಧರ್ಮಕ್ಕೆ ವಿರೋಧವಾಗಿರುವುದನ್ನು ಅವನು ತ್ಯಜಿಸುವನು. ತನ್ನ ಬದುಕು ಇರುವುದು ಸುಖಪಡುವುದಕ್ಕಾಗಿ ಅಲ್ಲ, ಅದನ್ನು ಜಯಿಸುವುದಕ್ಕಾಗಿ; ಜಯಿಸಿ ಆತ್ಮಸಂಯಮ ಪಡೆಯುವುದಕ್ಕಾಗಿ; ಆತ್ಮಸಂಯಮವೇ ಅತ್ಯುನ್ನತ ನಾಗರಿಕತೆ ಎಂದು ಅವನ ಅಭಿಮತ. ಅಳಿದುಹೋಗುತ್ತಿರುವ ಜಾತಿಯ ವ್ಯತ್ಯಾಸಗಳನ್ನು ಸರಳವಾಗಿ ಆರ್ಯ ಅನಾರ್ಯ ಅಥವಾ ಬ್ರಾಹ್ಮಣ, ಶೂದ್ರ ಎನ್ನಬಹುದು. ಸಾವಿರ ವರ್ಷಗಳ ಸಂಸ್ಕೃತಿಯ ಶಿಶುವಾದ ಬ್ರಾಹ್ಮಣನು ಬಲು ಶಿಸ್ತಿನ ಜೀವನವನ್ನು ನಡೆಸಬೇಕು; ಅಜ್ಞಾನಿ ಎನಿಸಿಕೊಳ್ಳುವ ಶೂದ್ರನಿಗೆ ಇದರಲ್ಲಿ ಬಹಳ ಔದಾರ್ಯವನ್ನು ತೋರಿಸಲಾಗಿದೆ.

ತಾಯಿಯ ಸ್ಧಾನದಲ್ಲಿರುವ ಹೆಂಗಸಿಗೆ ಭಾರತದಲ್ಲಿ ಸಾರ್ವತ್ರಿಕವಾಗಿ ಗೌರವದ ಸ್ಥಾನವಿದೆ. ಸಂನ್ಯಾಸಿಯಾಗಿರುವ ಮಗನು ಮನೆಗೆ ಬಂದರೆ, ತಂದೆಯಾದವನು ತಲೆಬಗ್ಗಿ ನಮಸ್ಕರಿಸಿ ಹಣೆಯನ್ನು ಭೂಮಿಗೆ ಮುಟ್ಟಿಸಬೇಕು; ಆದರೆ ಸಂನ್ಯಾಸಿ ತಾಯಿಗೆ ನಮಸ್ಕರಿಸಬೇಕು. ಭಾರತದ ಸ್ತ್ರೀಯರು ತಮ್ಮ ಮಕ್ಕಳನ್ನು ಮೊಸಳೆಗಳು ತಿನ್ನಲೆಂದು ನದಿಗೆ ಎಸೆಯುವುದಿಲ್ಲ. ಸ್ವಂತ ಇಚ್ಛೆಯಿಂದ ಪ್ರಾಣತ್ಯಾಗ ಮಾಡುವ ಇರಾದೆಯಿರದ ಹೊರತು ವಿಧವೆಯನ್ನು ಅವಳ ಪತಿಯ ಚಿತೆಯಲ್ಲೇ ಸುಡುವುದಿಲ್ಲ.

ಉತ್ತಮ ಜಾತಿಯವರಲ್ಲಿ ವಿವಾಹವಿಚ್ಛೇದನಕ್ಕೆ ಅವಕಾಶವಿಲ್ಲ; ಗಂಡನನ್ನು ಬಿಟ್ಟು ಬಂದ ಹೆಂಗಸಿಗೆ, ಅವಳೆಷ್ಟೇ ಕೆಟ್ಟುಹೋಗಿದ್ದರೂ, ಅವನ ಆಸ್ತಿಯಲ್ಲಿ ಹಕ್ಕಿದೆ. ಸ್ವಾಮಿ (Suami) ವಿವೇಕಾನಂದರು ಪತಿಯ ಮೇಲೆ ಪತ್ನಿಯಾದವಳ ಪ್ರೇಮ ಹೇಗಿರಬೇಕು ಎಂಬುದಕ್ಕೆ ಭಾರತದ ಮಹೋನ್ನತ ಕಾವ್ಯಗಳಲ್ಲೊಂದಾದ ರಾಮಾಯಣದ ಕಥೆಯಿಂದ ಸುಂದರವಾದ ಭಾಗವೊಂದನ್ನು (ರಾಮನ ಮೇಲೆ ಸೀತೆಯ ಪ್ರೇಮ) ಉದ್ಧರಿಸಿ ಹೇಳಿದರು. ಅನಂತರ “ಇಂದಿನ ದಿನಗಳಲ್ಲಿ ‘ಬಲಾಢ್ಯರದ್ದೇ ಉಳಿವು’ ಎಂಬುದರ ಬಗ್ಗೆ ಬಹಳವಾಗಿ ಹೇಳಲಾಗುತ್ತಿದೆ” ಎಂದೂ ಸೇರಿಸಿದರು. ಪಶ್ಚಿಮ ದೇಶಗಳು ಇದನ್ನು ಭಾರತವಿರೋಧಿವಾದವಾಗಿ ಬಳಸಿಕೊಳ್ಳುತ್ತಿವೆ ಎಂದರು. ಇದನ್ನು ತಮ್ಮ ಶ‍್ರೀಮಂತಿಕೆ, ಪುರೋಭಿವೃದ್ಧಿ, ಅಧಿಕಾರ ಎಲ್ಲವೂ ತಾವು ಹೆಚ್ಚು, ತಮ್ಮ ಧರ್ಮವೂ ಉನ್ನತವಾದದ್ದು, ಶುದ್ಧವಾದದ್ದು ಎಂಬುದನ್ನು ತೋರಿಸುತ್ತದೆ ಎಂಬುದು ಅವರವಾದ.

ಆದರೆ ಭಾರತವು ವಿಜಯದ ಬಲ ಮತ್ತು ಐಹಿಕ ಜೀವನದ ವೈಭವಗಳನ್ನೇ ಗುರಿಯಾಗಿರಿಸಿಕೊಂಡು ಬಲಿಷ್ಠ ದೇಶಗಳ ಏಳುಬೀಳುಗಳನ್ನು ಬೇಕಾದಷ್ಟು ಕಂಡಿದೆ. ಇತರ ದೇಶಗಳು ಭಾರತವನ್ನು ಪುನಃ ಪುನಃ ಕೊಳ್ಳೆಹೊಡೆದಿವೆ; ಜಯಿಸಿದ ದೇಶಗಳಿಗಾಗಿ ಭಾರತ ದುಡಿದಿದೆ; ತುಳಿತದ ಭಾರವನ್ನು ಅದಮ್ಯ ಕ್ಷಮತೆಯಿಂದ ಅದು ಹೊತ್ತಿದೆ; ಎಲ್ಲರಿಗೂ ಸಹನೆಯನ್ನೇ ಪ್ರದರ್ಶಿಸಿದೆ-ಏಕೆಂದರೆ ತನ್ನ ಜನರು ಉನ್ನತ ಆಧ್ಯಾತ್ಮಿಕತೆಯ ಮೇಲೆ ಗಟ್ಟಿಯಾಗಿ ನಿಂತಿರುವ ಧರ್ಮವನ್ನು ಆಧರಿಸುವರು, ವರ್ತಮಾನದ ಸುಖ ವನ್ನಲ್ಲ ಎನ್ನುವುದು ಅದಕ್ಕೆ ಗೊತ್ತಿದೆ.

\begin{center}
\textbf{“ಭಾರತ ಮತ್ತು ಹಿಂದೂಧರ್ಮ” ದ ಮೇಲಣ ಉಪನ್ಯಾಸ\supskpt{\footnote{\enginline{1. New Discoveries, Vol. 2, pp. 42}}}}
\end{center}

\begin{center}
(ನ್ಯೂಯಾರ್ಕ್ ಡೈಲಿ ಟ್ರಿಬ್ಯೂನ್, ೨೫ ಏಪ್ರಿಲ್ ೧೮೯೪)
\end{center}

ನೆನ್ನೆ ಸಂಜೆ ವಾಲ್ಡಾರ್ಫ್ನ ಮಿಸೆಸ್ ಆರ್ಥರ್ ಸ್ಮಿತ್ ಅವರ ಸಂಭಾಷಣಾ ವಲಯದೆದುರು ಸ್ವಾಮಿ ವಿವೇಕಾನಂದರು “ಭಾರತ ಮತ್ತು ಹಿಂದೂಧರ್ಮ\footnote{2. ಈ ಉಪನ್ಯಾಸದ ಪದಶಃ ವರದಿ ಲಭ್ಯವಿಲ್ಲ.} ” ದ ಬಗ್ಗೆ ಉಪನ್ಯಾಸ ಮಾಡಿದರು. ಮಂದ್ರಸ್ಥಾಯಿಯಲ್ಲಿ ಮಿಸ್ ಸಾರಾ ಹಂಬರ್ಟ್ ಹಾಗೂ ತಾರಸ್ಥಾಯಿಯಲ್ಲಿ ಮಿಸ್ ಆ್ಯನ್ ವಿಲ್ಸನ್ ಅನೇಕ ಆಯ್ಕೆಯ ಗಾಯನಗಳನ್ನು ಶ್ರುತ ಪಡಿಸಿದರು. ಭಾಷಣಕಾರರು ಒಂದು ಕಿತ್ತಳೆ ವರ್ಣದ ಕೋಟನ್ನು ಧರಿಸಿದ್ದರು; ಜೊತೆಗೆ ಹಳದಿಯ ರುಮಾಲೊಂದಿತ್ತು; ಇವನ್ನು ಭಿಕ್ಷುಕರ ಉಡುಪೆಂದು ಕರೆಯುವರಂತೆ. ಬೌದ್ಧಧರ್ಮೀಯನೊಬ್ಬ “ಎಲ್ಲವನ್ನೂ ದೇವರಿಗಾಗಿ ಹಾಗೂ ಮಾನವರಿಗಾಗಿ” ತ್ಯಾಗ ಮಾಡಿದಾಗ ಧರಿಸುವ ಉಡುಪು ಇದು. ಪುನರ್ಜನ್ಮದವಾದವನ್ನು ಕುರಿತು ಚರ್ಚೆ ಮಾಡಲಾಯಿತು. ಪಾಂಡಿತ್ಯಕ್ಕಿಂತ ಜೋರೇ ಹೆಚ್ಚಾದ ಪಾದ್ರಿಗಳನೇಕರು “ಅಂಥದು ಸತ್ಯವಾಗಿದ್ದ ಪಕ್ಷದಲ್ಲಿ ಪೂರ್ವಜನ್ಮದ ನೆನಪು ಇರುವುದಿಲ್ಲವೇಕೆ?” ಎಂದು ಕೇಳುವರು; ಈ ಜನ್ಮದ ತನ್ನ ಹುಟ್ಟಿನ ಬಗ್ಗೆ, ನಡೆದ ಎಷ್ಟೋ ಸಂಗತಿಗಳ ಬಗ್ಗೆ ಮನುಷ್ಯನಿಗೆ ನೆನಪು ಇಲ್ಲದಿರುವಾಗ, ನೆನಪನ್ನೇ ಒಂದು ತಳಹದಿಯನ್ನಾಗಿ ಇಟ್ಟುಕೊಳ್ಳುವುದು ಬಾಲಿಶವಾಗುತ್ತದೆ” ಎನ್ನುವುದೇ ಇದಕ್ಕೆ ಉತ್ತರ ಎಂದು ಭಾಷಣಕಾರರು ಹೇಳಿದರು.

“ನ್ಯಾಯನಿರ್ಣಯದ ದಿನ ಎನ್ನುವಂಥ ಯಾವುದೂ ತಮ್ಮ ಧರ್ಮದಲ್ಲಿ ಇಲ್ಲ” ಎಂದ ಅವರು, ತಮ್ಮ ದೇವರು ಶಿಕ್ಷಿಸುವುದೂ ಇಲ್ಲ, ಪುರಸ್ಕಾರ ನೀಡುವುದೂ ಇಲ್ಲ ಎಂದು ಹೇಳಿದರು. ಯಾವ ರೀತಿಯಲ್ಲಾದರೂ ತಪ್ಪು ಮಾಡಿದ್ದೇ ಆದರೆ, ಸಹಜವಾದ ಶಿಕ್ಷೆ ತಕ್ಷಣವೇ ಆಗುತ್ತದೆ. ಆತ್ಮವು ಪರಿಪೂರ್ಣತೆಯನ್ನು ಪಡೆಯುವವರೆಗೂ, ದೇಹದ ಮಿತಿಗಳನ್ನು ಮೀರಿ ನಿಲ್ಲುವವರೆಗೂ ಒಂದು ಶರೀರದಿಂದ ಇನ್ನೊಂದು ಶರೀರಕ್ಕೆ ಹೋಗುತ್ತಿರುತ್ತದೆ ಎಂದೂ ಸೇರಿಸಿದರು...

\begin{center}
\textbf{ನಾರ್ಥಾಂಪ್ಪನ್, ಮೆಸಾಚುಸೆಟ್ಸ್ ನ ಸ್ಮಿತ್ ಕಾಲೇಜಿನಲ್ಲಿ\supskpt{\footnote{\enginline{1. New Discoveries, Vol. 2, pp. 36-37}}}}
\end{center}

\begin{center}
(ಸ್ಮಿತ್ ಕಾಲೇಜ್ ಮಾಸಿಕ, ಮೇ ೧೮೯೪)
\end{center}

ಬ್ರಾಹ್ಮಣತ್ವದ ಮೇಲಣ ಯಾರ ಪಾಂಡಿತ್ಯಪೂರ್ಣ ಪ್ರವಚನ ಸರ್ವಧರ್ಮ ಸಮ್ಮೇಳನದಲ್ಲಿ ಅತಂಹ ಉತ್ತಮ ಅನುಮೋದನೆಯನ್ನುಗಳಿಸಿಕೊಂಡಿತೋ ಆ ಹಿಂದೂ ಸಂನ್ಯಾಸಿ ಸ್ವಾಮಿ ವಿವೇಕಾನಂದರು ಏಪ್ರಿಲ್ ೧೫ ಭಾನುವಾರದಂದು ಸಂಜೆ ಪ್ರಾರ್ಥನಾವೇಳೆಯಲ್ಲಿ ಮಾತನಾಡಿದರು.\footnote{2. ಈ ಉಪನ್ಯಾಸದ ಪದಶಃ ವರದಿ ಲಭ್ಯವಿಲ್ಲ.} ನಾವು ಮಾನವ ಸೋದರತ್ವ ಮತ್ತು ದೇವರ ಪಿತೃತ್ವ ಎಂದು ಬಹಳವಾಗಿ ಮಾತನಾಡುತ್ತೇವೆಯಾದರೂ ಈ ಪದಗಳ ಅರ್ಥವನ್ನು ಅರಿತಿಲ್ಲ. ನಿಜವಾದ ಸಹೋದರತ್ವ ಸಾಧ್ಯವಾಗುವುದು ನಮ್ಮ ಜೀವನವನ್ನು ಸರ್ವಪಿತೃವಾದ ಆತನು ತನ್ನ ಅತ್ಯಂತ ಸಮೀಪಕ್ಕೆ ಸೆಳೆದುಕೊಂದಾಗ ಮಾತ್ರವೇ- ಆಗ ನಾವು ದೇವರಿಗೆ ಎಷ್ಟು ಹತ್ತಿರವಾಗಿರುತ್ತೇವೆ ಎಂದರೆ ನಮ್ಮ ಸಣ್ಣಪುಟ್ಟ ಹೆಚ್ಚುಗಾರಿಕೆ ಹೊಟ್ಟೆಕಿಚ್ಚುಗಳೆಲ್ಲ ನಾಶವಾಗಿರುತ್ತವೆ - ಏಕೆಂದರೆ ನಾವು ಅವುಗಳಿಗಿಂತ ತುಂಬ ಮೇಲಿನ ಸ್ತರದಲ್ಲಿರುತ್ತೇವೆ. ದೀರ್ಘಕಾಲದವರೆಗೆ ಒಂದು ಸಣ್ಣ ಸ್ಥಳದಲ್ಲೇ ಇದ್ದು ಕೊನೆಗೆ ಅದಕ್ಕಿಂತ ವಿಶಾಲವಾದ ಜಾಗದ ಅಸ್ತಿತ್ವವನ್ನೇ ನಿರಾಕರಿಸಿದ ಆ ಹಿಂದೂ ಕಥೆಯಲ್ಲಿನ ಬಾವಿಯ ಕಪ್ಪೆಯ ಹಾಗೆ ಆಗದಂತೆ ನಮ್ಮ ಬಗ್ಗೆ ನಾವು ಎಚ್ಚರ ವಹಿಸಬೇಕು.

\begin{center}
\textbf{ಭಾರತ ಮತ್ತು ಪುನರ್ಜನ್ಮದ ಮೇಲೆ ಒಂದು ಉಪನ್ಯಾಸ\supskpt{\footnote{\enginline{1. New Discoveries, Vol. 2, pp. 45}}}}
\end{center}

\begin{center}
(ನ್ಯೂಯಾರ್ಕ್ ಡೈಲಿ ಟ್ರಿಬ್ಯೂನ್, ೩ ಮೇ ೧೮೯೪)
\end{center}

ಕಳೆದ ಸಂಜೆ ಮಿಸ್ ಮೇರಿ ಫಿಲಿಪ್ಸ್ರವರ ಪಶ್ಚಿಮ, ೩೮ನೇ ಬೀದಿ, ನಂ.೧೯ನೆಯ ಮನೆಯಲ್ಲಿ ಸ್ವಾಮಿ ವಿವೇಕಾನಂದರು \enginline{(Virekanmda)} “ಭಾರತ ಮತ್ತು ಪುನರ್ಜನ್ಮ\footnote{4. ಈ ಉಪನ್ಯಾಸದ ಪದಶಃ ವರದಿ ಲಭ್ಯವಿಲ್ಲ.} ” ದ ಮೇಲೆ ಉಪನ್ಯಾಸ ಮಾಡಿದರು. ಹಿಂದೂತ್ವದ, ಅಥವಾ ಬ್ರಾಹ್ಮಣತ್ವದ, ಇನ್ನಿತರ ಎದ್ದು ತೋರುವ ಅಂಶಗಳ ನಡುವೆ, ತಮ್ಮ ಧರ್ಮವು ಯಾವ ವಿಶಿಷ್ಟ ನಾಮಧೇಯವನ್ನೂ ಹೊಂದಿಲ್ಲ ಎಂದರು; ಎಲ್ಲ ಮತಪಂಥಗಳಲ್ಲಿ ಅಡಕವಾಗಿರುವ ಸತ್ಯದಲ್ಲಿನ ನಂಬಿಕೆಯೇ ಧರ್ಮವೆಂದು ಪರಿಗಣಿಸಲಾಗುತ್ತದೆ ಎಂದರು; ಯಾವುದೋ ಒಂದು ಶಾಸ್ತ್ರಾಧಾರವೇ ನಿಜವಾದ ಏಕಮಾತ್ರ ಧರ್ಮ ಎಂದು ನಂಬುವುದೇ ಮತ ಎಂದರು. ಕಾರಣ ಪರಿಣಾಮಗಳ ಕರ್ಮದ ನಿಯಮವನ್ನು ವಿವರಿಸಿದರು; ಬಾಹ್ಯ ಮತ್ತು ಆಂತರಿಕ ಸ್ವರೂಪಗಳು ಮತ್ತು ಅವುಗಳ ಪರಸ್ಪರ ಹತ್ತಿರದ ಸಂಬಂಧವನ್ನೂ ವಿವರಿಸಿದರು. ಹಿಂದಿನ ಜನ್ಮದಿಂದ ನಿಯಂತ್ರಿತವಾದ ಈ ಪ್ರಪಂಚದಲ್ಲಿನ ನಮ್ಮ ಕ್ರಿಯೆಗಳು ಮುಂದಿನ ಇನ್ನೂ ಒಂದು ಜನ್ಮಕ್ಕೆ ಹೇಗೆ ಬದಲಾವಣೆಯನ್ನು ತರುತ್ತವೆ ಎಂಬುದರ ವಿವರಗಳನ್ನೂ ಶ್ರುತಪಡಿಸಿದರು...

\begin{center}
\textbf{ಹಿಂದೂ ಸಂನ್ಯಾಸಿಯಿಂದ ಉಪನ್ಯಾಸ\supskpt{\footnote{\enginline{1. New Discoveries, Vol. 2, pp. 65-68}}}}
\end{center}

\begin{center}
\textbf{ಮೇಲ್ಜಾತಿಯ ಭಾರತೀಯರ ಧರ್ಮದ ಬಗ್ಗೆ ಸ್ವಾಮಿ ವಿವೇಕಾನಂದರು\supskpt{\footnote{2. ಈ ಉಪನ್ಯಾಸದ ಪದಶಃ ವರದಿ ಲಭ್ಯವಿಲ್ಲ.}}}
\end{center}

\begin{center}
(ಲಾರೆನ್ಸ್, ಮೆಸಾಚುಸೆಟ್ಸ್, ಈವಿನಿಂಗ್ ಟ್ರಿಬ್ಯೂನ್, ೧೬ ಮೇ ೧೮೯೪)
\end{center}

ಖ್ಯಾತ ಬ್ರಾಹ್ಮಣ ಸಂನ್ಯಾಸಿ ಸ್ವಾಮಿ ವಿವೇಕಾನಂದರ ಉಪನ್ಯಾಸದ ಸಂದರ್ಭದಲ್ಲಿ ಕಳೆದ ಸಂಜೆ ಲಿಬರ್ಟಿ ಹಾಲ್ ಸಾಕಷ್ಟು ಭರ್ತಿಯಾಗಿತ್ತು. ಚಿಕಾಗೊದಲ್ಲಿ ಕಳೆದ ಬೇಸಿಗೆಯಲ್ಲಿ ನಡೆದ ವಿಶ್ವ ಸರ್ವಧರ್ಮಸಮ್ಮೇಳನದಲ್ಲಿ ಅವರು ಪ್ರಮುಖ ವ್ಯಕ್ತಿಯಾಗಿದ್ದರು. ಈ ದೇಶದ ರೀತಿ-ರಿವಾಜುಗಳನ್ನು ಅಭ್ಯಸಿಸುವುದಕ್ಕೆಂದು ಅವರು ಕೆಲಕಾಲ ಇಲ್ಲಿರುವರು. ಸ್ತ್ರೀಯರು ಕ್ಲಬ್ನ ಆಶ್ರಯದಲ್ಲಿ ನಡೆದ ಈ ಉಪನ್ಯಾಸವು ಒಂದು ನವೀನವೂ ಆಸಕ್ತಿ ಮೂಡಿಸುವಂಥದೂ ಆದ ಸಂದರ್ಭವಾಗಿತ್ತು. ಖ್ಯಾತ ಹಿಂದೂವನ್ನು ಕ್ಲಬ್ನ ಅಧ್ಯಕ್ಷಿಣಿಯಾದ ಮಿಸ್ ವಿದರ್ಬೀ ಅವರು ಸುಮುಖವಾಗಿ ಪರಿಚಯಿಸುತ್ತ, ಭರತ ಖಂಡದ ಪುರಾತತ್ವದ ಮಹತ್ವ, ಅದರ ಅಚ್ಚರಿಯ ಇತಿಹಾಸ ಮತ್ತು ಹಿಂದೂ ಜನಾಂಗದ ಉನ್ನತ ಬೌದ್ಧಿಕ ಗುಣಗಳನ್ನು ಕುರಿತು ಪ್ರಾಸಂಗಿಕವಾಗಿ ಪ್ರಸ್ತಾಪಿಸಿದರು.

ಸಂಜೆಯ ಭಾಷಣಕಾರರು ತಮ್ಮ ದೇಶೀಯ ಉಡುಪಿನಲ್ಲಿ ಭೂಷಿತರಾಗಿದ್ದರು. ಅದೆಂದರೆ ಒಂದು ಪ್ರದೀಪ್ತ ಕೆಂಬಣ್ಣದ ಸೊಂಟಪಟ್ಟಿಯಿಂದ ಕಟ್ಟಲ್ಪಟ್ಟ ಅದೇ ಬಣ್ಣದ ನಿಲುವಂಗಿ; ಮತ್ತು ಆಕರ್ಷಕವಾಗಿ ತಲೆಗೆಸುತ್ತಿದ ಒಂದು ಬಿಳಿಯ ರೇಷ್ಮೆಯ ರುಮಾಲು, ಮೇಲ್ನೋಟಕ್ಕೆ ಶ್ಯಾಮಲ ಲಾವಣ್ಯ, ಕಪ್ಪಾದ ಕಲ್ಪನಾ ವಿಹಾರಿ ಕಣ್ಣುಗಳು, ಬ್ರಹ್ಮಚಾರಿಯೂ ಧಾರ್ಮಿಕ ಜೀವನಕ್ಕೆ ಅರ್ಪಿತನಾದವನೂ ಆದ ಮೇಲ್ಜಾತಿಯ ಬ್ರಾಹ್ಮಣನ ಆತ್ಮಾವಲೋಕನದ ಭಂಗಿ, ಎಂದು ಕಾಣುತ್ತಿದ್ದವು. ಅವರು ಬಹು ಚೆನ್ನಾಗಿ ಶಿಕ್ಷಣ ಪಡೆದವರೆಂಬುದು, ಅವರ ಅಚ್ಚರಿಯ ಇಂಗ್ಲಿಷ್ ಪ್ರಭುತ್ವ, ಅವರ ತರ್ಕಬದ್ಧವಾದಸಾಮರ್ಥ್ಯಗಳಿಂದ ಗೊತ್ತಾಗುತ್ತಿತ್ತು. ಸಾಂದರ್ಭಿಕವಾದ ಮಿಲ್ಟನ್ ಮತ್ತು ಡಿಕೆನ್ಸ್ ಉದ್ಧರಣೆಗಳು ಅವರು ಶ್ರೇಷ್ಠ ಇಂಗ್ಲಿಷ್ ಸಾಹಿತ್ಯದಲ್ಲೂ ಪ್ರೌಢಿಮೆ ಯುಳ್ಳವರೆಂಬುದನ್ನು ತೋರಿಸಿಕೊಡುತ್ತಿದ್ದವು.

ಅವರು ಮೊದಲು ಹಿಂದೂವಿನ ಜಾತಿಯ ಸಾಮಾಜಿಕ ಪರಿಸ್ಥಿತಿಯ ಎದ್ದು ತೋರುವ ವೈಚಿತ್ರ್ಯದ ಬಗ್ಗೆ ಮಾತನಾಡಿದರು; ಹಿಂದೆ ಇದ್ದಷ್ಟು ಕಟ್ಟುನಿಟ್ಟಾದ ಪರಿಸ್ಥಿತಿ ಈಗಿಲ್ಲವಾದರೂ, ಪ್ರತಿಯೊಂದೂ ವಂಶಪಾರಂಪರ್ಯವಾಗಿಯೇ ಈಗಲೂ ನಡೆ ಯುತ್ತದೆ ಎಂದರು. ಜಾತಿಗಳ ಕಲಬೆರಕೆ ತೀರ ನಿಷಿದ್ಧವಲ್ಲವಾದರೂ, ಮಕ್ಕಳಿಗೆ ಪ್ರತಿ ಕೂಲ ಪರಿಸ್ಥಿತಿಯನ್ನುಂಟುಮಾಡುತ್ತದೆ. ಬ್ರಾಹ್ಮಣ ಅಥವಾ ಮೇಲ್ಜಾತಿಯ ವ್ಯಕ್ತಿ ತನ್ನ ಜೀವನದ ಮೊದಲ ಭಾಗವನ್ನು ವೇದಗಳ ಅಥವಾ ಪವಿತ್ರ ಗ್ರಂಥಗಳ ಅಧ್ಯಯನಕ್ಕೆ ಹಾಗೂ ಉಳಿದ ಭಾಗವನ್ನು ದಿವ್ಯತೆಯ ಉಪಾಸನೆಗಾಗಿ ಮುಡಿಪಾಗಿಡುತ್ತಾನೆ; ತನ್ನಲ್ಲಿಯ ಮನುಷ್ಯ ಪ್ರಕೃತಿಯನ್ನು ಕಳೆದುಕೊಂಡು ತನ್ನ ಆತ್ಮಗುಣವನ್ನು ಪಡೆದು ಕೊಳ್ಳಬೇಕೆಂದು ಅವನಿಂದ ನಿರೀಕ್ಷಿಸಲಾಗುತ್ತದೆ.

ಭಾಷಣಕಾರರು ಕೆಲವು ಪಾಶ್ಚಿಮಾತ್ಯ ಪದ್ಧತಿಗಳನ್ನು ವ್ಯತಿರಿಕ್ತವಾಗಿ ಟೀಕಿಸುವುದಕ್ಕೆ ಹಿಂದೆ ಮುಂದೆ ನೋಡಲಿಲ್ಲ - ಅದರಲ್ಲೂ ಸ್ತ್ರೀಯರ ಸ್ಥಾನಮಾನವನ್ನು ಕುರಿತಾದ ಕೆಲವನ್ನು. ನಾವು ಸ್ತ್ರೀಯರನ್ನು ಹೆಂಡತಿಯಾಗಿ ಆರಾಧಿಸುವುವೆಂದೂ, ಆದರೆ ಹಿಂದೂ ವಿಗೆ ಎಲ್ಲ ಸ್ತ್ರೀಯರೂ ತಾಯಿಯ ಅಂಶವನ್ನೇ ಪ್ರತಿನಿಧಿಸುವಂತೆ ತೋರುತ್ತಾರೆ ಎಂದೂ ಖಚಿತವಾಗಿ ನುಡಿದರು. ಅಮೆರಿಕಾದಲ್ಲಿ ಸ್ತ್ರೀಯರು ತಮ್ಮ ತಾರುಣ್ಯವನ್ನೂ ಸೌಂದರ್ಯವನ್ನೂ ಕಳೆದುಕೊಂಡ ಮೇಲೆ ಕಷ್ಟಪರಿಸ್ಥಿತಿಯನ್ನೆದುರಿಸಬೇಕಾಗುತ್ತದೆ, ಆದರೆ ಭಾರತದಲ್ಲಿ ವಯಸ್ಸಾದ ಹೆಂಗಸೊಬ್ಬಳು ಬರುತ್ತಿದ್ದರೆ ರಾಜನೂ ಸಹ ದಾರಿ ಬಿಟ್ಟು ನಿಲ್ಲಬೇಕು- ಅಷ್ಟೊಂದು ಗೌರವದಿಂದ ಸ್ತ್ರೀಯರನ್ನು ಕಾಣಲಾಗುತ್ತದೆ. ಹಿಂದೂ ಬೈಬಲ್ ಆದ ವೇದಗಳ ಅತ್ಯಂತ ಸುಂದರವಾದ ಭಾಗಗಳಲ್ಲಿ ಕೆಲವನ್ನು ಸ್ತ್ರೀ ಯರೇ ಬರೆದಿರುವರು, ಆದರೆ ಪ್ರಪಂಚದಲ್ಲಿ ಸ್ತ್ರೀಯರಿಗೆ ಸ್ಥಾನವಿರುವ ಇನ್ನಾವುದೇ ಬೈಬಲ್ ಇಲ್ಲ ಎಂದು ಅವರು ಖಚಿತಪಡಿಸಿದರು.

ಭಾರತದಲ್ಲಿ ವಿಧವೆಯರನ್ನು ಕ್ರೂರವಾಗಿ ನಡೆಸಿಕೊಳ್ಳಲಾಗುತ್ತಿದೆ ಎಂಬ ಮಾತನ್ನು ಸತ್ಯಕ್ಕೆ ದೂರವಾದುದೆಂದು ಖಂಡಿಸುವುದಕ್ಕೆ ಸಾಕಷ್ಟು ಸಮಯ ವ್ಯಯವಾಯಿತು. ಅನ್ಯದೇಶೀಯ ಕ್ರೈಸ್ತ ಪ್ರಚಾರಕರು ಅದೆಷ್ಟೋ ಕಾಲದಿಂದ ಟೀಕಿಸುವುದಕ್ಕೆ ಪಟ್ಟಾಗಿ ಹಿಡಿದುಕೊಂಡಿದ್ದ ಜನಾನಾದ ವಿಧವೆಯರ ವಿಚಾರವನ್ನೆತ್ತಿಕೊಂಡು ಭಾಷಣಕಾರರು ಈ ಪ್ರಸ್ತಾಪವನ್ನು ಮಾಡಿದರು. ವಿವಾಹವನ್ನು ಭಾರತದಲ್ಲಿ ಭದ್ರವಾಗಿ ಕಾಪಾಡಿಕೊಳ್ಳಲಾಗುತ್ತದೆ; ಬ್ರಾಹ್ಮಣನು ತನ್ನ ಸಂಬಂಧಿಯನ್ನು ಮದುವೆಯಾಗುಂತಿಲ್ಲ ಎಂಬ ನಿಯಮದ ಜೊತೆಗೇ ಕ್ಷಯ ಮುಂತಾದ ಗುಣಪಡಿಸಲಾಗದ ಕಾಯಿಲೆ ಇರುವ ವರನ್ನು ಯಾರೂ ಮದುವೆಯಾಗುವಂತಿಲ್ಲ. ಒಬ್ಬರು ಕುಡಿದ ಲೋಟದಲ್ಲಿಯೇ ಇನ್ನೊಬ್ಬರು ಕುಡಿಯಬಾರದೆಂಬ ಜಾತಿಯ ಕಟ್ಟುನಿಟ್ಟಾದ ನಿಯಮದ ಜೊತೆಗೆ ನೆಂಟಸ್ತನದ ನಿಯಮಗಳು ಧರ್ಮದ ಭಾಗವಾಗಿಲ್ಲದೆ ಇದ್ದರೂ, ಇಪ್ಪತ್ತೆಂಟೂವರೆ ಕೋಟಿ ಜನರಿರುವ ದೇಶದಲ್ಲಿ ಸೋಂಕು ರೋಗಗಳು ಹರಡದಂತೆ ತಡೆದು ಭೌತಿಕ ಪರಿಸ್ಥಿತಿಯನ್ನು ನಿಭಾಯಿಸುವುದರಲ್ಲಿ ಅತ್ಯುತ್ತಮ ಫಲಿತಾಂಶವನ್ನೇ ಕೊಟ್ಟಿವೆ. ಈ ದೇಶದ ರೈಲ್ವೆ ಬೋಗಿಗಳಲ್ಲಿ, ನಿಲ್ದಾಣಗಳಲ್ಲಿ ಜನರು ಸ್ವಚ್ಛಂದವಾಗಿ ನೀರು ಕುಡಿ ಯುವುದನ್ನು ನೋಡಿ ತಮಗೆ ಭಯವಾಯಿತೆಂದು ಭಾಷಣಕಾರರು ಹೇಳಿದರು. ಮಕ್ಕಳಿಗೆ ಎಲ್ಲಕ್ಕಿಂತ ಮೊದಲು ಸಮಸ್ತ ಜೀವಿಗಳ ಮೇಲೆ ದಯೆ ತೋರಬೇಕೆನ್ನುವುದನ್ನು ಕಲಿಸುವರು. ಇದನ್ನು ಎಷ್ಟು ಚೆನ್ನಾಗಿ ಕಲಿಸುವರೆಂದರೆ ತೀರ ಚಿಕ್ಕ ಮಗು ಕೂಡ ಒಂದು ಹುಳುವನ್ನು ಸಹ ತುಳಿಯುವುದಿಲ್ಲ. ಒಂದು ಅಚ್ಚರಿಯ ಯೋಚನೆ ಯೆಂದರೆ ಈ ಅಕ್ರೈಸ್ತ ಅನಾಗರಿಕರೆಂದು ಕರೆಸಿಕೊಳ್ಳುವವರಿಗೆ ಕ್ರೈಸ್ತ ದೇಶಗಳಲ್ಲೇ ಪದೇ ಪದೇ ಸೋಲುತ್ತಿರುವ ಉದ್ದ ಹೆಸರಿನ ಸೊಸೈಟಿ\footnote{1. ಬಹುಶಃ ಸ್ವಾಮೀಜಿ \enginline{American Society fro the Prevention of Cruelty of Animal} ನ್ನು ಕುರಿತು ಹೇಳುತ್ತಿರಬೇಕು.} ಯ ಅಗತ್ಯ ಇಲ್ಲ ಎನ್ನುವುದು. ಮನೆ ಬಾಗಿಲಿಗೆ ಬಂದು ನಾನು ಹಸಿದಿದ್ದೇನೆ ಎನ್ನುವ ಅಭ್ಯಾಗತನು ಹಿಂದೂವಿಗೆ ದೇವರ ಪ್ರತಿರೂಪ; ಅವನನ್ನು ಅತ್ಯಂತ ಪ್ರೀತ್ಯಾದರಗಳಿಂದ ಸತ್ಕರಿಸಲಾಗುವುದು; ಮನೆಯ ಯಜಮಾನ ದಂಪತಿಗಳಿಗಿಂತ ಮೊದಲೇ ಅವನಿಗೆ ಭೋಜನ ನೀಡಲಾಗುವುದು.

ಭಾಷಣಕಾರರು ದುಃಖದಿಂದಲೇ ತಮ್ಮ ದೇಶದ ಬಡತನವನ್ನು ಪ್ರಾಸಂಗಿಕವಾಗಿ ಪ್ರಸ್ತಾಪಿಸಿದರು. ಏಕೆಂದರೆ, ಮೇಲ್ಜಾತಿಯವರು ಸುಖವಾಗಿ ಜೀವನವನ್ನು ನಡೆಸು ತ್ತಿರುವಂತೆಯೇ, ಒಣಗಿಸಿದ ಹೂಗಳನ್ನು ತಿಂದುಕೊಂಡು ಬದುಕಿರುವ ಕೋಟ್ಯಂತರ ಜನರಿರುವರು; ಅವರ ಅಸ್ತಿತ್ವ ಅದೆಷ್ಟು ಕೀಳುಮಟ್ಟದ್ದೆಂದರೆ ಗುರುತಿಸಲ್ಪಡದಷ್ಟು ಅವರು ದಯನೀಯವಾಗಿ ಬದುಕುತ್ತಿರುವರು. ಕಳೆದ ನೂರಾರು ವರ್ಷಗಳಿಂದ ಕ್ರೈಸ್ತರೂ ಮಹಮದೀಯರೂ ಅವರೆಡೆಗೆ ಎಸೆಯುತ್ತಿರುವ ಧರ್ಮೋಪದೇಶಗಳಿಗಿಂತ ಅವರಿಗೆ ಶಿಕ್ಷಣ ಮತ್ತು ತಿನ್ನಲು ಆಹಾರ ಕೊಡುವುದು ಮೇಲು ಎಂದು ಭಾಷಣ ಕಾರರು ಕ್ರುದ್ಧರಾಗಿ ನುಡಿದರು. ಈ ವಿಚಿತ್ರ ಜನರ ಅದೆಷ್ಟೋ ಸರಳವಾದ ಪುರಾತನ ಪದ್ಧತಿಗಳನ್ನು ಅದೆಷ್ಟು ಮುಗ್ಧತೆ ಮತ್ತು ನಿಷ್ಕಾಪಟ್ಯದಿಂದ ಹೇಳಿದರೆಂದರೆ, ಮನಸ್ಸಿನಲ್ಲಿರುವುದನ್ನು ಅಡಗಿಸುವುದಕ್ಕಾಗಿ ಪದಗಳನ್ನು ಬಳಸುವ ಈ ಯುಗದಲ್ಲಿ ಆ ಮಾತುಗಳು ಕೇಳುವವರಿಗೆ ಚೇತೋಹಾರಿಯಾಗಿದ್ದವು.ಅವರ ತರುಣ ತರುಣಿಯರಲ್ಲಿ ಒನಪು ಒಯ್ಯಾರ ಚೆಲ್ಲಾಟ ಯಾವುದೂ ಇಲ್ಲ, ಅವರ ತರುಣಿಯರು ತಮಗೆ ಗಂಡ ನಾಗುವವನೊಬ್ಬನನ್ನು ಪಡೆದುಕೊಳ್ಳುವ ಉದ್ದೇಶದಿಂದ ಬಿಂಕವನ್ನು ಪ್ರದರ್ಶಿಸುತ್ತ ತಮ್ಮೆಲ್ಲ ಧೈರ್ಯವನ್ನು (ಬಿನ್ನಾಣವನ್ನು?) ಒಗ್ಗೂಡಿಸಿ ಸಾರ್ವಜನಿಕ ಪ್ರದೇಶಗಳಿಗೆ ನುಗ್ಗುವುದಿಲ್ಲ ಎಂದರು. ಇವೆಲ್ಲ ನಮ್ಮ ಮಹಾವೈಭವದ ಗಣರಾಜ್ಯದ ಪ್ರಜೆಗಳನ್ನು - ಈ ಡೆನ್ಮಾರ್ಕ್ ರಾಜ್ಯದವರಲ್ಲಿ ಏನಾದರೂ ಸ್ವಲ್ಪ ಹೆಚ್ಚುಕಡಿಮೆಯಾಗಿರಬಹುದೇ ಎಂದು ಆಶ್ಚರ್ಯಪಡುವ ಹಾಗೆ ಮಾಡಿತು. ಪೂರ್ವಗ್ರಹವಿಲ್ಲದಂತೆ ತೀರ್ಮಾನಿಸಲು ಸಾಧ್ಯವಾಗಬೇಕಾದರೆ ಗುರಾಣಿಯ ಎರಡೂ ಬದಿಗಳನ್ನು ನೋಡುವುದು ಒಳ್ಳೆಯದು; ಆದರೆ ಕೆಲವು ಮಂದಿ ಶ್ರೋತೃಗಳು ತಮಗೆ ಆಪ್ಯಾಯಮಾನವಾಗಿದ್ದ ಅಮೆರಿಕನ್ ಪದ್ಧತಿಗಳನ್ನು ಹಿಂದೂ ಹಾಗೂ ಅಕ್ರೈಸ್ತ- ಅನಾಗರಿಕವನೊಬ್ಬ ದೂರುವುದನ್ನು ಕಂಡು ಮನಸ್ಸಿನಲ್ಲಿ ತಬ್ಬಿಬ್ಬುಗೊಂಡು ಹೊರಟುಹೋದರು.

ಉಪನ್ಯಾಸವು ಅತ್ಯಂತ ಕುತೂಹಲಕಾರಿಯಾಗಿದ್ದು, ಸಭಿಕರೆಲ್ಲರೂ ತುಂಬ ಗಮನ ವಿಟ್ಟು ಆಲಿಸಿದರು. ಆ ನಂತರ ಕೇಳಲಾದ ಅನೇಕ (ಪ್ರಶ್ನೆಗಳನ್ನು) ಚಿಂತನಶೀಲ ಸಂನ್ಯಾಸಿಯು ಯಾವುದೇ ಆಡಂಬರ ಅಟಾಟೋಪಗಳಿಲ್ಲದೆ ಉತ್ತರಿಸಿದರು. ಅವರಿಗೆ ಈ ಗಣರಾಜ್ಯ ಉದಯಿಸುವುದಕ್ಕಿಂತ ಶತಮಾನಗಳಷ್ಟು ಹಳೆಯದಾದ ಆ ಅಚ್ಚರಿಯ ನಾಡಿಗೆ ಹಿಂದೆಂದೋ ಭೇಟಿಕೊಟ್ಟಿದ್ದ ಏಕಮಾತ್ರ ಸಭಿಕರಾದ ಡಾ. ಬೋಕರ್ ಅವರಲ್ಲಿ ತುಂಬ ಆಸಕ್ತಿ ಮೂಡಿದಂತೆ ಕಂಡುಬಂದಿತು.

\begin{center}
\textbf{ಬ್ರಾಹ್ಮಣ ಸಂನ್ಯಾಸಿ\supskpt{\footnote{\enginline{1. New Discoveries, Vol. 2, pp. 68-71}}}}
\end{center}

\begin{center}
\textbf{ಮಹಿಳಾ ಕ್ಲಬ್ ನ ಅತಿಥಿಯಾಗಿ ಸ್ವಾಮಿ ವಿವೇಕಾನಂದರು\supskpt{\footnote{2. ಈ ಉಪನ್ಯಾಸದ ಪದಶಃ ವರದಿ ಲಭ್ಯವಿಲ್ಲ. ಇದೇ ಉಪನ್ಯಾಸದ ಇನ್ನಿತರ ಮುಖ್ಯಾಂಶಗಳಿಗೆ ಇದರ ಹಿಂದಿನ ‘ಹಿಂದೂ ಸಂನ್ಯಾಸಿಯಿಂದ ಉಪನ್ಯಾಸ’ ಎಂಬ ವರದಿಯನ್ನು ನೋಡಿ.}}}
\end{center}

\begin{center}
(ಲಾರೆನ್ಸ್ ಅಮೆರಿಕನ್ ಮತ್ತು ಆ್ಯಂಡೋವರ್ ಅಡ್ವರ್ಟೈಸರ್, ೧೮ ಮೇ ೧೮೯೪)
\end{center}

\begin{center}
ಅವರು ಬ್ರಾಹ್ಮಣತ್ವದ ಮುಖ್ಯಾಂಶಗಳನ್ನು ತೋರಿಸಿಕೊಟ್ಟರಲ್ಲದೆ\\ಕ್ರೈಸ್ತರಿಗೆ ಒಂದು ಸೂಚ್ಯವಾದ ಸಂದೇಶವನ್ನೂ ಕೊಟ್ಟರು.
\end{center}

ಮಂಗಳವಾರ ರಾತ್ರಿ ಲೈಬ್ರರಿ ಹಾಲ್ನಲ್ಲಿಲಾರೆನ್ಸ್ ಮಹಿಳಾ ಕ್ಲಬ್ ಆಶ್ರಯದಲ್ಲಿ ಬ್ರಾಹ್ಮಣ ಸಂನ್ಯಾಸಿ ಸ್ವಾಮಿ ವಿವೇಕಾನಂದರು ಅತ್ಯಂತ ಕುತೂಹಲಿಗಳಾಗಿದ್ದ ಸಭಿಕ ರನ್ನುದ್ದೇಶಿಸಿ ಮಾತನಾಡಿದರು.

ಮಿಸ್ ವಿದರ್ಬೀ ಅವರು ಭಾಷಣಕಾರರನ್ನು ಸಭೆಗೆ ಪರಿಚಯಮಾಡಿಕೊಡುತ್ತ ಹಾರ್ದಿಕ ಸುಸ್ವಾಗತಕ್ಕೆ ನಾಂದಿ ಹಾಡಿದರು. ಇನ್ನೊಂದು ದೇಶದಿಂದ ಆಗಮಿಸುವ ಗೌರವಾನ್ವಿತರಿಗೆ ಅಮೆರಿಕನ್ನರು ಸೌಜನ್ಯ ಮರ್ಯಾದೆಗಳನ್ನು ಸಲ್ಲಿಸದೆ ಇರುವುದು ತೀರ ಅಪರೂಪ.

ಮಿಸ್ ವಿದರ್ಬೀ ಅವರು ಜಾಣತನದಿಂದ ಅವರನ್ನು ವಿಶ್ವಸರ್ವಧರ್ಮ ಸಮ್ಮೇಳನದಲ್ಲಿನ ಪ್ರಮುಖ ವ್ಯಕ್ತಿಯೆಂದು ಪರಿಚಯಿಸಿ, ವರ್ಲ್ಡ್ ಫೇರ್ನಲ್ಲಿಯೂ ತಮ್ಮ ವಿಶಿಷ್ಟ ಛಾಪನ್ನು ಮೂಡಿಸಿದವರೆಂದು ಹೇಳಿದರು......

\begin{center}
\textbf{ಅವರು ಒತ್ತಿ ಹೇಳಿದ್ದು}
\end{center}

...ಅವರದೇ ದೇಶದಲ್ಲಿ, ಅವರದೇ ವರ್ಗದಲ್ಲಿ ಅವರು ಎಲ್ಲ ಹೆಂಗಸನ್ನೂ ಕರೆಯುವುದು ತಾಯಿ ಎಂದೇ. ಸ್ತ್ರೀಯರನ್ನು ಹೀಗೆ ತಾಯಿಯೆಂದು ಭಾವಿಸುವುದನ್ನು ಬ್ರಾಹ್ಮಣನು ಶಿಕ್ಷಣವಾಗಿ ಪಡೆಯುವನು; ಒಬ್ಬನು ತಾಯಿಯನ್ನು ಮದುವೆಯಾಗಲಾರ. ಆ ದೇಶದ ಹೆಂಗಸರಲ್ಲಿ ತಾಯಿಯ ಭಾವ ಸಹಜವಾಗಿಯೇ ಅಭಿವೃದ್ಧಿ ಹೊಂದುವುದು. ಇಲ್ಲಿ, ಅವರಿಗೆ ಅನ್ನಿಸಿದಂತೆ ಪತ್ನಿಯ ಭಾವವನ್ನು ಬೆಳೆಸುವರು. ಅವರ ಉಪನ್ಯಾಸದಲ್ಲಿ ಅತ್ಯಂತ ಚೆನ್ನಾಗಿದ್ದುದೆಂದರೆ ಅವರು ತಾಯಿಗೆ ಗೌರವ ಸಲ್ಲಿಸಿದ ರೀತಿ. ಪುಟ್ಟ ಹಿಂದೂ ಮಗುವಿನಲ್ಲಿ ಸಹ ಒಂದು ಹುಳುವನ್ನು ತುಳಿಯಲಾರ ದಷ್ಟು ಮಟ್ಟಿಗೆ ಪ್ರಾಣಿದಯೆ ಇರುವುದೆಂಬುದೂ ಗಮನ ಸೆಳೆಯದೆ ಹೋಗಲಿಲ್ಲ.

\begin{center}
\textbf{ಮದುವೆಯ ವಿಷಯ}
\end{center}

ಇದು ಅವರ ಉಪನ್ಯಾಸದ ದೊಡ್ಡ ಭಾಗವೇ ಆಗಿತ್ತು. ಆರ್ಯರೆಂದು ಕರೆಯ ಲ್ಪಡುವ ಮೇಲ್ಜಾತಿಯವರಲ್ಲಿ ಹೆಣ್ಣುಮಕ್ಕಳು ಮದುವೆಯೆಂಬುದನ್ನು ಅಸಭ್ (?) ಎಂದುಕೊಳ್ಳುವರು. ವಿಧವೆಯಾದವಳಂತೂ ಎಂದೂ ಮದುವೆಯಾಗುವಂತೆಯೇ ಇಲ್ಲ. ಮದುವೆಯಾಗದೆ ಇದ್ದವನನ್ನು ಎಲ್ಲರೂ ಹೊಗಳುವರು, ನಿಜಕ್ಕೂ ಆರಾಧಿಸುವರು; ಆದರೆ ಅವನು ಮದುವೆಯಾದದ್ದೇ ಆದರೆ ನಿಮಿಷಗಳಲ್ಲಿ ಎಲ್ಲವೂ ಬದಲಾಗಿ ಬಿಡುವುದು. ಮದುವೆಯಾಗದಿದ್ದವನನ್ನು ಆಧ್ಯಾತ್ಮಿಕತೆ ಉಳ್ಳವನು, ಉತ್ತಮ ಮನಸ್ಸಿ ನವನು, ಪವಿತ್ರನು ಎಂದು ಪರಿಗಣಿಸುವರು.

ಆರ್ಯರ ಮದುವೆಯಲ್ಲಿ ಹಣವನ್ನು ಸಲ್ಲಿಸಲಾಗುವುದಿಲ್ಲ (?); ಹೆಣ್ಣು ಮಕ್ಕಳು ಹೆಚ್ಚು ಸಂಖ್ಯೆಯಲ್ಲಿರುವುದರಿಂದ ತಂದೆಯಾದವನಿಗೆ ತನ್ನ ಮಗಳನ್ನು ಮದುವೆ ಮಾಡಿಕೊಡುವುದು ಅತ್ಯಂತ ಕಷ್ಟವಾದ ಸಂಗತಿಗಳಲ್ಲೊಂದು; ಅವಳು ಹುಟ್ಟಿದಾಗಿನಿಂದಲೇ ಅವನು ಮಗಳಿಗೆ ಗಂಡನ್ನು ಹೇಗೆ ಹುಡುಕುವುದೆಂಬ ಚಿಂತೆಯನ್ನು ಹಚ್ಚಿಕೊಂಡಿರುತ್ತಾನೆ.

ಕೆಳಸ್ತರದ ಎರಡು ಜಾತಿಗಳಲ್ಲಿ ವಿವಾಹದ ನಿಯಮಗಳು ಬೇರೆಯೇ ಆಗುತ್ತವೆ. ವಿಧವೆಯಾದವಳು ಪುನಃ ಮದುವೆಯಾಗುತ್ತಾಳೆ; ಗಂಡ ಹೆಂಡತಿ ತಮಗೆ ಇಷ್ಟವಾದರೆ ವಿವಾಹವಿಚ್ಛೇದನ ಮಾಡಿಕೊಳ್ಳಬಹುದು. ಮಗು ಹುಟ್ಟಿದಾಗ ಜ್ಯೋತಿಷಿಯೊಬ್ಬನು ಬಂದು ಮಗುವಿನ ಜಾತಕವನ್ನು ಬರೆಯುವ ಮೂಲಕ ಆ ಹುಡುಗ ಅಥವಾ ಹುಡುಗಿಯ ಭವಿಷ್ಯದ ಗುಣವನ್ನು ನಿರೂಪಿಸುತ್ತಾನೆ. ಅದು ಮಾನುಷಗಣದ ಅಥವಾ ರಾಕ್ಷಸಗಣದ ಮಗುವೆಂದು ನಿರ್ಧಾರವಾಗುತ್ತದೆ; ರಾಕ್ಷಸ ಗಣದ್ದಾದರೆ ಅದನ್ನು ಮುಂದಿನ ಕೆಳಸ್ತರದ ಜಾತಿಗೆ ಕೊಟ್ಟು ಮದುವೆ ಮಾಡುತ್ತಾರೆ - ಹಾಗೆ ಮಾಡುವ ಮೂಲಕ ಅದರ ಪರಿಸ್ಥಿತಿ ಸ್ವಲ್ಪವಾದರೂ ಉತ್ತಮವಾಗಬಹುದಾದ ಸಂಭವ ಇರಬಹುದೆಂದು ತಿಳಿಯಲಾಗುತ್ತದೆ.

ಮದುವೆಯ ವಿಚಾರವನ್ನು ಮಗುವಿನ ತೀರ್ಮಾನಕ್ಕೆ ಬಿಡುವುದಿಲ್ಲ; ಹಾಗೆ ಮಾಡಿದರೆ ಅದು ಮೂಗೋ ಕಣ್ಣೋ ಚೆಲುವಾಗಿ ತೋರಿತೆಂದು ಪ್ರೇಮ (ದಲ್ಲಿ ಬಿದ್ದು) ಮದುವೆಯಾಗುಬಹುದು; ಅವನ ಹಾದಿಯಲ್ಲಿ ಅವನು ಹೋಗಿ ಎಲ್ಲವನ್ನೂ ಹಾಳುಮಾಡಬಹುದು. ಪ್ರಾಧಾನ್ಯತೆ ನೀಡಿದ ಅಂಶವೆಂದರೆ ಕೇವಲ ಮೇಲ್ಜಾತಿ ಯವರು ಮಾತ್ರವೇ

\begin{center}
\textbf{ನಿಜವಾದ ಆಧ್ಯಾತ್ಮಿಕ ಜೀವನ}
\end{center}

ಹಾಗೂ ಭಗವದಾರಾಧನೆಯ ಬಗ್ಗೆ ಯೋಚಿಸಬಹುದು. ಕೆಳಜಾತಿಯವರ ದಯನೀಯ ಪರಿಸ್ಥಿತಿಯ ಬಗ್ಗೆ, ಅವರ ಬಡತನ ಅಜ್ಞಾನಗಳ ಬಗ್ಗೆ ಹೇಳಿದ ಭಾಷಣ ಕಾರರು, ಕೋಟ್ಯಾಂತರ ಜನರಿಗೆ ತಮ್ಮ ಹೆಸರನ್ನೂ ಬರೆ (ಲಾಗದು) ಎಂದರು. ಅವರು ಮತ್ತೂ ಹೇಳಿದರು:

ತುತ್ತು ಕೂಳಿಗಾಗಿ ಕೈಚಾಚಿರುವ ಅವರಿಗೆ ನಾವೆಲ್ಲರೂ ಉಪದೇಶ ಕೊಡುತ್ತಿ ದ್ದೇವೆ. ಕೆಳವರ್ಗದವರಲ್ಲಿ ಬಡತನ ಅದೆಷ್ಟು ಅತಿಯಾಗಿದೆಯೆಂದರೆ ಹಿಂದೂವೊಬ್ಬನ ಸರಾಸರಿ ಮಾಸಿಕ ವರಮಾನ ಐವತ್ತು ಸೆಂಟ್ಗಳಷ್ಟು. ಕೋಟ್ಯಂತರ ಜನರು ಒಪ್ಪತ್ತು ಊಟದ ಮೇಲೆ ಬದುಕಿರುವರು; ಕೋಟ್ಯಂತರ ಜನರು ಆಹಾರಕ್ಕಾಗಿ ಕಾಡು ಹೂವನ್ನು ತಿಂದು ಬದುಕಿರುವರು.

\begin{center}
\textbf{ಯಾವ ಬೈಬಲ್ನಲ್ಲಿಯೂ}
\end{center}

ಸಹ ಮಹಿಳೆಯರು ಬರೆದ ಒಂದು ಸಾಲು ಸಿಕ್ಕಲಾರದು. ಆದರೆ ತಮ್ಮ ದೇಶದ ಬೈಬಲ್ನಲ್ಲಿ ಅನೇಕ ಸುಂದರ ಸಂಗತಿಗಳನ್ನು ಸ್ತ್ರೀಯರು ಬರೆದಿರುವುದು ಎಂದವರು ಹೆಮ್ಮೆಯಿಂದ ಹೇಳಿದರು.

ತಮ್ಮ ಜನರಿಗೆ ಕ್ರೈಸ್ತಧರ್ಮವನ್ನು ಬೋಧಿಸುವ ಮೂಲಕ ಅವರನ್ನು ಮೇಲೆತ್ತುವ ಪ್ರಯತ್ನ ವ್ಯರ್ಥವೇ ಸರಿ ಎಂದು ಸ್ಪಷ್ಟ ಆಂಗ್ಲ ಭಾಷೆಯಲ್ಲಿ ಸ್ವಾಮಿ ವಿವೇಕಾನಂದರು ಸಭಿಕರಿಗೆ ಹೇಳಿದರು. ಅವರೆಂದರು:

ಗ್ರೀಕರು, ಪರ್ಷಿಯನ್ನರು ನಮ್ಮಲ್ಲಿಗೆ ಬಂದುದನ್ನು ನಾವು ನೋಡಿರುವೆವು- ಸ್ಪೇಯಿನ್ ದೇಶದವರು ನಮ್ಮನ್ನು ಕ್ರೈಸ್ತರನ್ನಾಗಿಸಲು ಬಂದೂಕಿನೊಂದಿಗೆ ನಮ್ಮಲ್ಲಿಗೆ ಬಂದರು; ಆದರೂ ಸಹ ನಾವಿನ್ನೂ ಹಿಂದೂಗಳಾಗಿಯೇ ಇದ್ದೇವೆ. ಮುಂದೆಯೂ ಹಾಗೆಯೇ ಇರುತ್ತೇವೆ.

ತಮ್ಮ ಪ್ರಖರ ತೇಜಸ್ಸಿನ ಕಣ್ಣುಗಳ ಸರ್ವಶಕ್ತಿಯನ್ನೂ, ಕಂಠದ ಅಭಿವ್ಯಕ್ತಿಯನ್ನೂ ಬಳಸಿ ಅತ್ಯಂತ ನಾಟಕೀಯವಾಗಿ ಭಾಷಣಮಾಡುತ್ತ ಅವರೆಂದರು:

ನಾವು ಭಾರತೀಯರು ನಮ್ಮ ಧರ್ಮದಲ್ಲಿಯೇ ಇರುತ್ತೇವೆ ಎಂದು ಇಲ್ಲಿ ಅಮೆರಿಕಾದಲ್ಲಿ ನಿಂತು ನಾನು ಧೈರ್ಯವಾಗಿ ಹೇಳುತ್ತೇನೆ.

ನಮ್ಮ ಪದ್ಧತಿಗಳು ನಮಗೆ ಸರಿ, ಅವುಗಳನ್ನು ಇಟ್ಟುಕೊಂಡಿರುವುದು ಸಾಧು ಎಂದು ಅವರು ನಮಗೆ ಹೇಳಿದರು. ಹಿಂದೆ ಅನೇಕ ಸುಸಂಸ್ಕೃತ ಅಮೆರಿಕನ್ ಸಭೆಗಳಲ್ಲಿ ನಿಂತಿ ದ್ದಂತೆಯೇ ನಮ್ಮ ಮುಂದೆಯೂ ನಿಂತಿದ್ದರು- ಬ್ರಾಹ್ಮಣಧರ್ಮದ ಪಾಂಡಿತ್ಯಪೂರ್ಣ ಭಾಷಣಕಾರ, ಈ ದೇಶದವರೆಗೆ ಬಂದು ನಮಗೆ ಹೇಳುವುದಕ್ಕೆ ಬಂದ ಏಕೈಕ ಹಿಂದೂ -ಸಾಧ್ಯವಿದ್ದಷ್ಟೂ ಮೃದುವಾಗಿ, ಧೈರ್ಯವಿದ್ದಷ್ಟೂ ಶಕ್ತಿಯುತವಾಗಿ ಹೇಳಿದರು-ಬಡ ಹಿಂದೂವಿಗೆ ಇನ್ನೇನೂ ಹೇಳಬೇಡಿ, ದಯವಿಟ್ಟು ನಿಮ್ಮ ಕೆಲಸವನ್ನು ನೀವು ನೋಡಿಕೊಂಡು ನಿಮ್ಮ ಪಾಡಿಗೆ ನೀವಿರಿ ಎಂದು.

ಉಪನ್ಯಾಸ ಮುಗಿದ ಮೇಲೆ ಸಭಿಕರಲ್ಲಿ ಅನೇಕರು ವಿವೇಕಾನಂದರನ್ನು ಮಿ. ಹಾಗೂ ಮಿಸೆಸ್ ಯಂಗ್ ಅವರ ಮನೆಯಲ್ಲಿ ಭೇಟಿಮಾಡುವ ಅವಕಾಶವನ್ನು ಉಪಯೋಗಿಸಿಕೊಂಡರು. ಅಲ್ಲಿ ಅವರು ಅತ್ಯಂತ ಸಂತೋಷದಾಯಕ ಅತಿಥಿ ಯೆನ್ನಿಸಿಕೊಂಡರು; ನೆರೆದವರನ್ನು ರಂಜಿಸಿದರು.

\begin{center}
\textbf{ಸ್ವಾಮಿ ವಿವೇಕಾನಂದರು\supskpt{\footnote{\enginline{1. New Discoveries, Vol. 2, pp. 144-145}}}}
\end{center}

(೧೮೯೪ರ ಆಗಸ್ಟ್ ೩ ಶುಕ್ರವಾರ ಸ್ವಾಮಿ ವಿವೇಕಾನಂದರು ಮೇಯ್ನಿಯ ಗ್ರೀನೆಕರ್ನಲ್ಲಿ ಕೊಟ್ಟ ಸಾರ್ವಜನಿಕ ಉಪನ್ಯಾಸದ ಈ ವರದಿಯನ್ನು ಮಿಸೆಸ್ ಓಲ್ಬುಲ್ ಅವರು ಬೋಸ್ಟನ್ ಈವಿನಿಂಗ್ ಟ್ರಾನ್ಸ್ಕ್ರಿಪ್ಟ್ಗೆ ಸಲ್ಲಿಸಿದ್ದರು. ಈ ಉಪನ್ಯಾಸದ ಪದಶಃ ವರದಿ ಲಭ್ಯವಿಲ್ಲ. ನೋಡಿ, ಮೆಯ್ನಿಯ ಗ್ರೀನೆಕರ್ನಲ್ಲಿ ಕೊಟ್ಟ ಉಪನ್ಯಾಸಗಳ ಮೇಲಣ ಟಿಪ್ಪಣಿಗಳು, “ಭಾರತದ ಧರ್ಮ” ಎಂಬ ತಲೆಬರಹದಡಿಯಲ್ಲಿ, ಕೃತಿಶ್ರೇಣಿಯ ಇದೇ ಸಂಪುಟದ ೨೯೨-೨೯೬ ಪುಟಗಳಲ್ಲಿ.)

\begin{center}
(ಬೋಸ್ಟನ್ ಈವಿನಿಂಗ್ ಟ್ರಾನ್ಸ್ಕ್ರಿಪ್ಟ್, ೧೧ ಆಗಸ್ಟ್ ೧೮೯೪)
\end{center}

ಹಿಂದೂ ಒಬ್ಬರಿಂದ ಕ್ರೈಸ್ತ ಸಭಿಕರೆದುರಿಗೆ ಮಹಮ್ಮದನ ಸಮರ್ಥನೆ; ಎಲ್ಲ ಪ್ರವಾದಿಗಳನ್ನೂ ಗೌರವಿಸತಕ್ಕದ್ದು ಮತ್ತು ಅವರ ಬೋಧನೆಗಳನ್ನು ಗೌರವಬುದ್ಧಿಯಿಂದ ಅಧ್ಯಯನ ಮಾಡತಕ್ಕದ್ದು ಎಂಬ ಪಾಠ; ಮನುಷ್ಯನಲ್ಲಿ ಭಗವಂತನ ಆವಿರ್ಭಾವವನ್ನು ಈ ಪ್ರಬೋಧಕರ ಅನುಯಾಯಿಗಳು ತಮ್ಮ ನಡತೆಯಿಂದ ದಿಕ್ಕುಗೆಡಿಸಲು ಬಿಡತಕ್ಕದ್ದಲ್ಲ - ಎಂಬ ವಿಷಯಗಳು ಗ್ರೀನೆಕರ್ನಲ್ಲಿ ಉಪನ್ಯಾಸದ ಅಂಶಗಳಾಗಿದ್ದವು.

ಸ್ಪಷ್ಟವಾದ ಚಿಂತನೆ ಹಾಗೂ ಹೇಳಿಕೆಗಳು ಪುನರ್ಜನ್ಮವೆಂಬ ಪ್ರಾಚ್ಯ ನಂಬಿಕೆಯ ಬಗ್ಗೆ ಮಾಡಲಾಗುವ ಒರಟೂ ಅಸಂಸ್ಕೃತವೂ ಆದ ಮೇಲುಮೇಲಿನ ಕೆಟ್ಟ ಟೀಕೆಗಳನ್ನು ತಾಳ್ಮೆಯಿಂದ ಸರಿಪಡಿಸಿದವು. ಹೇಳಿಕೆಯು ಕುಶಲತೆಯಿಂದ ಕೂಡಿತ್ತು; ಏಕೆಂದರೆ ಸರಳವಾಗಿತ್ತು, ಪರಿಚಿತ ಹಾಗೂ ನಿತ್ಯಜೀವನದ ಉದಾಹರಣೆಗಳಿಂದಾಗಿ ಸುಲಭವಾಗಿ ಅರ್ಥವಾಗುತ್ತಿತ್ತು. ಇದಾದ ನಂತರ ಉದಾತ್ತವೂ ಸುಙಟವೂ ಆದ ಒಂದು ಮನವಿ-ಆ ಕಾಲದ ಚರಿತ್ರೆಯನ್ನು, ಮಹಮ್ಮದನ ನಂಬಿಕೆಯನ್ನು, ಆತನ ಪ್ರವಾದಿತನದ ಮೂಲ ಭೂತ ಅಂಶಗಳಿಂದಾಗಿ ಮಾನವ ಜನಾಂಗಕ್ಕಾಗಿರುವ ಉಪಕಾರಗಳನ್ನು ನ್ಯಾಯ ದೃಷ್ಟಿಯಿಂದ ನಿರ್ಣಯಿಸಬೇಕು ಎಂದು. ಅಕ್ರೈಸ್ತರಿಗೆ ಅಂಜುವ ಅಲ್ಲಿದ್ದ ಸ್ತ್ರೀಪುರುಷರಲ್ಲಿ ಎಷ್ಟೋ ಜನರ ಮನ ಕಲಕಿತು; ಮಾಮೂಲಿನಂತೆ ಜೀತಪದ್ಧತಿಯ ಪಾಪ ಕಠಿಣ ಹೃದಯಗಳ ಮನಸ್ಸಿಗೆ ಬರುವುದಿಲ್ಲವೇ, ಇದನ್ನು ಪರಿಗಣಿಸಬೇಕು ಎಂದು ವೆಂಡೆಲ್ ಫಿಲಿಪ್ಸ್\footnote{1. ಅಮೆರಿಕನ್ ಸುಧಾರಕ ಹಾಗೂವಾಗ್ಮಿ (೧೮೧೧-೧೮೮೪).} ಹೇಳಿದುದನ್ನು ತಿಳಿಸಿದರು.

ಈ ಮನವಿಯಲ್ಲಿ ನಿರ್ಲಕ್ಷ್ಯ, ಚಮತ್ಕಾರ, ಬುದ್ಧಿವಂತಿಕೆ ಎಲ್ಲವೂ ಎಷ್ಟು ಮೃದುವಾಗಿ, ಗೌರವಯುಕ್ತವಾಗಿ, ಉದಾತ್ತಭಾವದಿಂದ ಕೆಲಸಮಾಡಿದವೆಂದರೆ ಪ್ರತಿ ಯೊಂದು ಧರ್ಮದ ದೋಷಗಳನ್ನು, ಭಯಾನಕ ಸಂಗತಿಗಳನ್ನು ಬದಿಗಿಟ್ಟು ಎಲ್ಲ ಧರ್ಮಗಳಿಗೂ ಸಾಮಾನ್ಯವಾಗಿರುವ ಆತ್ಮದ ಅವಿನಾಶಿತ್ವ, ದೇವರು ಒಬ್ಬನೇ ಎನ್ನುವುದು, ಮಾನವಜನಾಂಗದ ಕೆಲ ಭಾಗಕ್ಕಾದರೂ ಸಲ್ಲುವ ದೇವದೂತರ ಪಾವಿತ್ರ್ಯ; ನೀಡಬಹುದಾದ, ಎಲ್ಲರಿಗೂ ಸಲ್ಲುವ ಸತ್ಯವು ಒಂದಲ್ಲ ಒಂದು ರೂಪದಲ್ಲಿ ಅವರ ಬಳಿ ಇರುವುದು - ಇವುಗಳನ್ನು ಪರಿಗಣಿಸ ತಕ್ಕದ್ದು, ಅವರು ಬೋಧಿಸಿದ ಮುಕ್ತಿಮಾರ್ಗದ ಮೇಲೆ ಭಕ್ತಿಯಿರತಕ್ಕದ್ದು - ಎಂದು ನಿರ್ಣಯಿಸಿದೆವು.

ಮಹಾತ್ಮನಾದವನು ಕೊಡಬಹುದಾದುದನ್ನೇ ಭಾಷಣಕಾರ ಸ್ವಾಮಿ ವಿವೇಕಾ ನಂದರು ಕೊಟ್ಟರು. ಆ ಒಂದು ಘಳಿಗೆ ಎಂದಿಗೂ ಮರೆಯುವಂಥದಲ್ಲ. ಫಿಲಿಪ್ಸ್ ಬ್ರೂಕ್ಸ್ ಏಕಮೂರ್ತಿವಾದವನ್ನೂ ಬಿಷಪ್ ಗಣಪ್ರಭುತ್ವವನ್ನೂ ಒಂದುಗೂಡಿಸಿದ ಹಾಗೆ, ಈ ಮಹಾಪುರುಷನು ಅಲ್ಲಿದ್ದವರನ್ನೆಲ್ಲ - ಅವರವರ ಪೂರ್ವಶಿಕ್ಷಣ ಪೂರ್ವ ಗ್ರಹಗಳು ಏನೇ ಆಗಿದ್ದಿರಲಿ - ಸತ್ಯದ ಬೆಳಕಿನೆಡೆಗೆ ಕರೆತಂದನು. ಸತ್ಯವನ್ನೂ ಒಳಿತನ್ನೂ ಪ್ರೀತಿಸುವವರೆಲ್ಲರೂ ಈತನಲ್ಲಿ ತಮ್ಮ ಬಿಷಪ್ನನ್ನು ಕಂಡುಕೊಂಡರು. ಹೀಗೆ, ಈ ಹಿಂದೂವು ತನ್ನ ರಚನಾತ್ಮಕ ಚಿಂತನೆಯ ದೆಸೆಯಿಂದ, ಅದನ್ನು ಇತರರಿಗೆ ಕೊಡುವಾಗ, ಪ್ರವಾದಿಗಳ ಶಕ್ತಿಯನ್ನು ತನ್ನ ಸಾನಿಧ್ಯ ಮಾತ್ರದಿಂದಲೇ ತಿಳಿಯಪಡಿಸುವನು.

\begin{center}
\textbf{ನಿರ್ವಾಣಷಟ್ಕಮ್​*\supskpt{\footnote{\enginline{1. New Discoveries, Vol. 2, pp. 149-150 (Arena, October 1899, p.499)}}}}
\end{center}

(ಮೇಯ್ನಿ, ಗ್ರೀನೆಕರ್ನಲ್ಲಿ ಸ್ವಾಮಿ ವಿವೇಕಾನಂದರಿಂದ ಓದಲ್ಪಟ್ಟು ೧೮೯೪ರ ಗ್ರೀನೆಕರ್ ವಾಯ್ಸ್\footnote{2. ನೋಡಿ, ಮೇಯ್ನಿಯ ಗ್ರೀನೆಕರ್ನಲ್ಲಿ ಕೊಟ್ಟ ಉಪನ್ಯಾಸಗಳ ಮೇಲಣ ಟಿಪ್ಪಣಿಗಳು, “ಭಾರತದ ಧರ್ಮ” ಎಂಬ ತಲೆಬರಹದಡಿಯಲ್ಲಿ, ಕೃತಿಶ್ರೇಣಿಯ ಇದೇ ಸಂಪುಟದ ೨೯೨-೨೯೬ ಪುಟಗಳಲ್ಲಿ.}ನಲ್ಲಿ ವರದಿಯಾದ ಶಂಕರಾಚಾರ್ಯಕೃತ “ನಿರ್ವಾಣಷಟ್ಕಮ್​” ನ ಭಾಗಶಃ ಅನುವಾದ.)

ಗ್ರೀನೆಕರ್ನಲ್ಲಿ ಸ್ವಾಮಿಗಳ ಪ್ರಸಿದ್ಧವಾದ ಪೈನ್ ಮರದ ಅಡಿಯಲ್ಲಿ ವಿವೇಕಾನಂದರು ಹೇಳಿದರು:

\begin{center}
\textbf{ದೇಶಗಳೆಂಬ ಅಸಂಬದ್ಧತೆ\supskpt{\footnote{\enginline{1. New Discoveries, Vol. 2, p. 154-155}}}}
\end{center}

\begin{center}
(ಬೋಸ್ಟನ್ ಈವಿನಿಂಗ್ ಟ್ರಾನ್ಸ್ಕ್ರಿಪ್ಟ್, ೧೫ ಆಗಸ್ಟ್ ೧೮೯೪)
\end{center}

ಇಲಿಯಟ್ನಲ್ಲಿನ ಪೈನ್ ಮರಗಳ ಕೆಳಗೆ ಮಾಡಿದ ವಿವೇಕಾನಂದರ ಕೊನೆಯ ಭಾಷಣದ\footnote{4. ಈ ಉಪನ್ಯಾಸದ ಪದಶಃ ವರದಿ ಲಭ್ಯವಿಲ್ಲ. ನೋಡಿ, ಮೇಯ್ನಿಯ ಗ್ರೀನೆಕರ್ನಲ್ಲಿ ಕೊಟ್ಟ ಉಪನ್ಯಾಸಗಳ ಮೇಲಣ ಟಿಪ್ಪಣಿಗಳು, “ಭಾರತದ ಧರ್ಮ” ಎಂಬ ತಲೆಬರಹದಡಿಯಲ್ಲಿ, ಕೃತಿಶ್ರೇಣಿಯ ಇದೇ ಸಂಪುಟದ ೨೯೨-೨೯೬ ಪುಟಗಳಲ್ಲಿ.} ಸಣ್ಣದೊಂದು ಸಾರಾಂಶವನ್ನು ಈ ಕೆಳಗೆ ಕೊಡಲಾಗಿದೆ. ಬ್ರ್ಯಾಂಟನ\footnote{5. ವಿಲಿಯಂ ಕ್ಯೂಲೆನ್ ಬ್ರ್ಯಾಂಟ್ (೧೭೯೪ -೧೮೭೮).} “ತೋಪುಗಳೆ ದೇವನಿದ್ದ ಮೊದಲ ತಾಣ” ಎಂಬ ಒಂದು ಸಾಲನ್ನು ವಿವರಿಸುವಂತೆ ಇದ್ದ ದೇವರುಗಳ ನಿವಾಸದಂತಹ ಜಾಗ ಅದು.

“ರಾಷ್ಟ್ರವೆಂದರೆ ಏನು? ನಿಯಮವೆಂದರೆ ಏನು? ನಾವು ನಿಯಮಗಳನ್ನು ಮಾಡಿಕೊಂಡಿರುವುದೇ ಅವನ್ನು ಮೀರಿ ಮೇಲೇರುವುದಕ್ಕಾಗಿ.

“ಆತ್ಮನ ಮುಕ್ತತೆ ಎಂಬುದು ಇದೆ; ಇದರ ಮೂಲಕ ನಾವು ನಿಯಮದ ಸ್ವಾತಂತ್ರ್ಯವನ್ನು ಅರಿಯುತ್ತೇವೆ. ಆತ್ಮನ ಮುಕ್ತತೆಯನ್ನು ಅರಸುವವರ ದೇಶಕ್ಕೆ ಸೇರಿದವನು ನಾನು. ದೇವರನ್ನು ಆರಾಧಿಸುವವರ ನಾಡಿನವನು ನಾನು.

“ದೇವರ ದಿವ್ಯತೆಯುಳ್ಳವರೆಲ್ಲ ನನ್ನ ಗುರುಗಳು. ಕೃಷ್ಣನನ್ನು ಅರಿಯುವಾಗ, ಬುದ್ಧನನ್ನು ಅರಿಯುವಾಗ, ಮಹಮ್ಮದನನ್ನು ಅರಿಯುವಾಗ, ನಿಮ್ಮ ಕ್ರಿಸ್ತನನ್ನೂ ಅರಿ ಯುವೆ ನಾನು. ದೇವರನ್ನು ಮಾತ್ರವೇ ನಾನು ಪೂಜಿಸುವುದು. “ಸತ್-ಚಿತ್- ಆನಂದ ರೂಪ ಶಿವನೆ ನಾನು; ನಾನೆ ಅದು, ನಾನೆ ಅದು” ದೇವರನ್ನು ಎಲ್ಲದರಲ್ಲಿಯೂ ಕಾಣುವುದರಿಂದ, ನಾನು ರಾಷ್ಟ್ರದಲ್ಲಿ, ರಾಜ್ಯದಲ್ಲಿ, ಧರ್ಮದಲ್ಲಿ ಏನೆಲ್ಲವನ್ನೂ ಕಾಣುವೆನೋ ಅದೊಂದನ್ನೂ ಹಳಿಯಲಾರೆ. ನಾವು ಬೆಳೆಯುವುದು ಕೇಡಿನಿಂದ ಒಳಿತಿಗಲ್ಲ, ಆದರೆ ಒಳ್ಳೆಯದರಿಂದ ಇನ್ನೂ ಒಳ್ಳೆಯದಕ್ಕೆ ಇತ್ಯಾದಿ, ಇತ್ಯಾದಿ. ನಾನು ಕೇಡು ಒಳಿತು ಎಂದು ಕರೆಯಲ್ಪಡುವ ಎಲ್ಲದರಿಂದಲೂ ಕಲಿಯುತ್ತೇನೆ. ದೇಶ, ರಾಷ್ಟ್ರ ಮುಂತಾದ ಅಸಂಬದ್ಧತೆಯೆಲ್ಲ ಹೋಗುವುದೊಳ್ಳೆಯದು. ಪ್ರೇಮ, ಪ್ರೇಮ, ಭಗವತ್ಪ್ರೇಮ ಮತ್ತು ಸೋದರಪ್ರೇಮ ಮಾತ್ರವೇ ಉಳಿಯಲಿ.”

\begin{center}
\textbf{ಭಾರತದ ಒಬ್ಬ ಶ್ರೇಷ್ಠ ಪೂಜಾರಿ\supskpt{\footnote{\enginline{1. New Discoveries, Vol. 2, p. 191-192}}}}
\end{center}

\begin{center}
(ಬಾಲ್ಟಿಮೋರ್ ಅಮೆರಿಕನ್, ೧೩ ಅಕ್ಟೋಬರ್ ೧೮೯೪)
\end{center}

\begin{center}
ಬಾಲ್ಟಿಮೋರ್ ಗೆ ಸ್ವಾಮಿ ವಿವೇಕಾನಂದರ ಆಗಮನ\\ಧರ್ಮವನ್ನು ಕುರಿತಾದ ಅವರ ದೃಷ್ಟಿ
\end{center}

ಭಾರತದ ಶ್ರೇಷ್ಠ ಬ್ರಾಹ್ಮಣ ಪೂಜಾರಿಯಾದ ಸ್ವಾಮಿ ವಿವೇಕಾನಂದರು ಕಳೆದ ರಾತ್ರಿ ಬಾಲ್ಟಿಮೋರ್ಗೆ ಬಂದು ತಲುಪಿದರು; ರೆವರೆಂಡ್ ವಾಲ್ಟರ್ ಮ್ರಾಮನ್ ಅವರ ಅತಿಥಿಯಾಗಿ ಇರುವರು...

ಕಳೆದ ರಾತ್ರಿ ಅಮೆರಿಕನ್ ಪತ್ರಿಕಾ ವರದಿಗಾರರೊಬ್ಬರಿಗೆ ಸ್ವಾಮಿ ವಿವೇಕಾನಂದರು ಹೀಗೆಂದರು:

“ಈ ದೇಶದಲ್ಲಿರುವ ಅವಧಿಯಲ್ಲಿ ನನಗೆ ಅಮೆರಿಕನ್ ಸಂಸ್ಥೆಗಳ ಬಗ್ಗೆ ತುಂಬ ಸದಭಿಪ್ರಾಯ ಉಂಟಾಗಿದೆ. ಚಿಕಾಗೋ, ನ್ಯೂಯಾರ್ಕ್, ಬೋಸ್ಟನ್ ಮತ್ತು ಡೆಟ್ರಾಯಿಟ್ - ಈ ನಾಲ್ಕು ಪಟ್ಟಣಗಳಲ್ಲಿ ನನ್ನ ಕಾಲ ಹಂಚಿಹೋಗಿದೆ. ಭಾರತದಲ್ಲಿದ್ದಾಗ ನಾನು ಚಿಕಾಗೋ ನಗರದ ವಿಚಾರ ಕೇಳಿರಲಿಲ್ಲ; ಆದರೆ ಅಗಿಂದಾಗ್ಯೆ ಬಾಲ್ಟಿಮೋರ್ ಬಗ್ಗೆ ಕೇಳುತ್ತಿದ್ದೆ. ನಾನು ಅಮೆರಿಕಾದ ಮೇಲೆ ಮಾಡಬೇಕೆಂದಿರುವ ಮುಖ್ಯವಾದ ಟೀಕೆ ಎಂದರೆ ಇಲ್ಲಿ ಧರ್ಮ ತೀರ ಅಲ್ಪವಾಗಿದೆ. ಭಾರತದಲ್ಲಿ ಅದು ತೀರ ಹೆಚ್ಚಾಗಿದೆ. ಭಾರತದಲ್ಲಿ ಧಾರಾಳವಾಗಿರುವ ಧರ್ಮದಲ್ಲಿ ಸ್ವಲ್ಪವನ್ನು ಇಲ್ಲಿಗೆ ಕಳುಹಿಸಿದರೆ ಲೋಕಕ್ಕೆ ಒಳಿ ತಾಗಬಹುದೆಂದು, ಹಾಗೆಯೇ ಅಮೆರಿಕಾದ ಕೈಗಾರಿಕಾ ಮುನ್ನಡೆಯ ಮತ್ತು ನಾಗರಿಕತೆಯ ಸ್ವಲ್ಪವನ್ನು ಭಾರತೀಯ ಜನರು ಪಡೆಯುವಂತಾದರೆ ಅವರಿಗೆ ಅದರಿಂದ ಲಾಭವಾಗುತ್ತದೆ ಎಂದು, ನನ್ನ ಅನ್ನಿಸಿಕೆ. ನಾನು ಎಲ್ಲ ಧರ್ಮಗಳಲ್ಲಿಯೂ ನಂಬಿಕೆ ಇರುವವನು. ಧರ್ಮಗಳಲ್ಲಿರುವ ಒಂದೇ ಸತ್ಯ ಅನೇಕ ರೀತಿಗಳಲ್ಲಿ ಪ್ರವಹಿಸುತ್ತಿರುವುದು; ನನ್ನ ಧರ್ಮದಲ್ಲಿ ಸತ್ಯವಿದೆ; ಹಾಗೆಯೇ ನಿಮ್ಮ ಧರ್ಮದಲ್ಲಿಯೂ ಸತ್ಯವಿದೆ. ಸಮಸ್ತವೂ ಕೊನೆಗೆ ಹೋಗಿ ಸೇರುವುದೂ ಒಂದೇ ಗುರಿಗೆ. ಪ್ರಪಂಚಕ್ಕೆ ಬೇಕಿರುವುದು ಕಡಿಮೆ ಶಾಸನ ಮತ್ತು ಹೆಚ್ಚು ಹೆಚ್ಚು ದಿವ್ಯತೆಯ ಸ್ತ್ರೀಪುರುಷರು ಎಂದು ನನಗನಿಸುತ್ತದೆ...”

\begin{center}
\textbf{ನಗರದಲ್ಲಿ ಪೂಜಾರಿ ಸ್ವಾಮಿ\supskpt{\footnote{\enginline{1. New Discoveries, Vol. 2, p. 196-200}}}}
\end{center}

\begin{center}
(ಬಾಲ್ಟಿಮೋರ್ ನ್ಯೂಸ್, ೧೩ ಅಕ್ಟೋಬರ್ ೧೮೯೪)
\end{center}

\begin{center}
\textbf{ಬಾಲ್ಟಿಮೋರ್ ಮೇಲ್ಜಾತಿಯ ಹಿಂದೂ ಒಬ್ಬರ ಭೇಟಿ}
\end{center}

\begin{center}
ರೆನ್ನರ್ಟ್ ಪ್ರಾಂಗಣದಲ್ಲಿ ಜನರನ್ನು ಆಕರ್ಷಿಸಿದ ಅವರ ಉಜ್ವಲ ವಸನ\\ಸಿಳ್ಳು ಹಾಕಿದ ಅವರ ಪೂರ್ವಭಾರತದ ಕುಚೋದ್ಯ\\ದೇಶದಲ್ಲಿ ಪ್ರವಾಸದಲ್ಲಿರುವ ಅವರಿಂದ ನಾಳೆ ರಾತ್ರಿ ಲೈಸಿಯಂನಲ್ಲಿ ಭಾಷಣ
\end{center}

ಒಬ್ಬ ಉನ್ನತ ಹಿಂದೂ ಪೂಜಾರಿಗಳಾದ ಸ್ವಾಮಿ ವಿವೇಕಾನಂದರು ಈ ಹೊತ್ತು ಮಧ್ಯಾಹ್ನ ಹೋಟೆಲ್ ರೆನ್ನರ್ಟ್ನ ಪ್ರಾಂಗಣದೊಳಕ್ಕೆ ನಡೆದು ಬಂದರು. ಅವರ ಜ್ವಲಂತ ಕೆಂಪು ನಿಲುವಂಗಿ ಮತ್ತು ಗಾಢ ಹಳದಿ ರುಮಾಲುಗಳಿಂದಾಗಿ ಅವರು ಎಲ್ಲರ ದೃಷ್ಟಿಯ ಕೇಂದ್ರವಾದರು...

\begin{center}
\textbf{ಅವರ ಹಾಸ್ಯದ ಕಲ್ಪನೆ}
\end{center}

ಸ್ವಾಮಿ ವಿವೇಕಾನಂದರಿಗೆ ಒಳ್ಳೆಯ ಹಾಸ್ಯಪ್ರಜ್ಞೆಯಿದೆ. ಈ ಹೊತ್ತು ಬೆಳಗ್ಗೆ ಅವರು ಇಂದು ತಾವು ಭೇಟಿ ಕೊಡಬೇಕೆಂದಿರುವ ಫುಡ್ ಶೋ ಬಗ್ಗೆ ಮಾತನಾಡುತ್ತಿದ್ದರು. ತಮಗೆ ಫುಡ್ (ಆಹಾರದ) ಬಗ್ಗೆ ಅದನ್ನು ನುಂಗುವುದರ ಹೊರತಾಗಿ ಬೇರೇನೂ ತಿಳಿಯದೆಂದು ಅವರು ಹೇಳಿದರು. ಇದು ಆರ್ಮಸ್\footnote{2. ಹಾರ್ಮಜ್ ಅಥವಾ ಆರ್ಮಜ್ ಒಂದು ಇರಾನಿಯನ್ ನಗರ.} ಮತ್ತು ಭಾರತದ ಚಾಟೂಕ್ತಿಗೆ ಒಳ್ಳೆಯ ಉದಾಹರಣೆ.

ಇನ್ನೊಂದು ಸಲ ಅವರು ಸ್ತ್ರೀಯರ ಹಕ್ಕುಗಳ ಬಗ್ಗೆ ಮಾತನಾಡುವಾಗ ನಗುನಗುತ್ತ - ಸ್ತ್ರೀಯರಿಗೆ ಪ್ರಪಂಚದಾದ್ಯಂತ ನಾವಂದುಕೊಂಡಿರುವುದಕ್ಕಿಂತ ಹೆಚ್ಚಾಗಿಯೇ ಹಕ್ಕುಗಳಿವೆ ಎಂದರು. ರೆನ್ನರ್ಟ್ ಹೋಟೆಲಿಗೆ ಹೋಗುವ ಮುನ್ನ ತಮ್ಮ ಕರಿಯ ಕೋಟನ್ನು ಬದಲಾಯಿಸಿ ಕಡುಕೆಂಪು ನಿಲುವಂಗಿಯನ್ನೂ ಹಳದಿ ರುಮಾಲನ್ನೂ ಧರಿಸಿ ತಮ್ಮ ಕೊಠಡಿಯಿಂದ ಹೊರಬಂದ ಅವರು ನಗುತ್ತ ಹೇಳಿದರು: “ಪರಿವರ್ತನೆ!”

ಅವರಿಗೆ ಸಿಳ್ಳು ಹಾಕಲೂ ಬರುತ್ತದೆ; ಹಿಂದೂ ಆಗಿರುವ ಬದಲು ಮೆಥಾಡಿಸ್ಟ್ ಆಗಿದ್ದಿದ್ದರೆ ತಮ್ಮ ತರಗತಿಗಳಲ್ಲಿ ಸಂಗೀತವನ್ನು ಪ್ರಾರಂಭಮಾಡಲು ಸಾಲುವಷ್ಟು ಸಂಗೀತ ಅವರ ಜೀವದಲ್ಲಿ ಇದೆ. ತಮ್ಮ ಕೊಠಡಿಯಲ್ಲಿ ‘ದಿ ನ್ಯೂಸ್’ ವರದಿಗಾರರೊಬ್ಬರಿಗಾಗಿ ಅವರು ಒಂದೆರಡು ಗೀತೆಗಳನ್ನು ಸಿಳ್ಳುಹಾಕಿ ತೋರಿಸಿದರು. ಅವು “ಡೈಸಿ ಬೆಲ್” ಅಥವಾ “ಸ್ವೀಟ್ ಮಾರೀ” ಆಗಿರಲಿಲ್ಲ; ಯಾವುದೋ ಕ್ರೈ ಸ್ತೇತರ ಹಿಂದೂ ಲಾವಣಿಯಾಗಿರಬೇಕು...

\begin{center}
\textbf{ಪ್ರಚಲಿತ ವಿಷಯಗಳ ಬಗೆಗಿನ ದೃಷ್ಟಿ}
\end{center}

ಅವರೇ ಹೇಳುವಂತೆ, ಸ್ವಾಮಿಗಳು ಅಮೆರಿಕನ್ ಸಂಸ್ಥೆಗಳನ್ನು ಅಧ್ಯಯನ ಮಾಡುತ್ತ, ಉಪನ್ಯಾಸ ಮಾಡುತ್ತ, ದೇಶದಾದ್ಯಂತ ಪ್ರವಾಸ ಮಾಡುತ್ತಿದ್ದಾರೆ; ಆದರೆ ಅಮೆರಿಕಾ ಸಮಾಜದ ತಿರುಳನ್ನು ಅವರಿನ್ನೂ ಅಷ್ಟಾಗಿ ಅರಿತಿರುವಂತಿಲ್ಲ - ಏಕೆಂದರೆ ನಾಡಿನ ಆರ್ಥಿಕ ತಜ್ಞರಿಗೆ ಆತಂಕವನ್ನುಂಟುಮಾಡುತ್ತಿರುವ ಯೂರೋಪಿಯನ್ನರ ವಲಸೆ, ವಿವಾಹವಿಚ್ಛೇದನ, ಬುಡಕಟ್ಟು ಸಮಸ್ಯೆ ಮುಂತಾದುವುಗಳ ಬಗ್ಗೆ ಅವರಿಗೆ ಏನೂ ತಿಳಿದಿಲ್ಲ.

ಆದರೆ ಅವರು ಪೌರಸ್ತ್ಯರ ವಲಸೆಯ ಬಗ್ಗೆ ವಿಚಲಿತರಾದಂತಿದೆ; ಏಕೆಂದರೆ ಚೀನೀ ಯರು ವಲಸೆ ಬರುವುದನ್ನು ತಡೆಯಲು ಅಮೆರಿಕಾಕ್ಕೆ ಹಕ್ಕಿಲ್ಲ ಎಂದೆನ್ನುತ್ತಾರೆ. ಪ್ರೇಮದ ನಿಯಮ ಉಳಿಯಬೇಕು, ಬಲಪ್ರಯೋಗ ಹಿಮ್ಮೆಟ್ಟಬೇಕು ಎನ್ನುತ್ತಾರೆ. ಬಲ ಪ್ರಯೋಗವನ್ನು ಬಳಸುವ ಯಾವುದೇ ದೇಶವಾಗಲಿ ಅವನತಿ ಹೊಂದುವುದು ಖಂಡಿತ ಎಂದವರು ಭವಿಷ್ಯ ನುಡಿಯುತ್ತಾರೆ. ಸಂಯುಕ್ತ ಸಂಸ್ಥಾನಗಳು ಇಡೀ ಲೋಕಕ್ಕೆ ತಮ್ಮ ಮಹಾದ್ವಾರವನ್ನು ತೆರೆದಿಡಬೇಕು ಎನ್ನುತ್ತಾರೆ. ಈ ದೇಶದ ದಕ್ಷಿಣ ಭಾಗವೆಲ್ಲ ಚೀನೀಯರಿಂದಲೂ ಹಿಂದೂಗಳಿಂದಲೂ ತುಂಬಿ ಹೋಗಬೇಕು ಎನ್ನುವುದು ಅವರ ನಂಬಿಕೆ. ಅವರೆಂದರು:

“ಭಾರತದಲ್ಲಿ ವಿವಾಹವಿಚ್ಛೇದನ ಎನ್ನುವುದೇ ಇಲ್ಲ; ನಮ್ಮ ಶಾಸನಗಳು ಅದಕ್ಕೆ ಅವಕಾಶ ಕೊಡುವುದಿಲ್ಲ. ನಮ್ಮ ಸ್ತ್ರೀಯರು ಅಮೆರಿಕನ್ ಸ್ತ್ರೀಯರಂತಲ್ಲದೆ ತಮ್ಮ ವಲಯಗಳ ಮಿತಿಯಲ್ಲಿರುತ್ತಾರೆ. ಅವರಲ್ಲಿ ಕೆಲವರು ತುಂಬ ವಿದ್ಯಾವಂತರು. ಈಗ ಅವರು ಸ್ವಲ್ಪಮಟ್ಟಿಗೆ ವೈದ್ಯಕೀಯ ಕ್ಷೇತ್ರವನ್ನು ಪ್ರವೇಶಿಸುತ್ತಿದ್ದಾರೆ. ಅಮೆರಿಕನ್ ಸ್ತ್ರೀ ಯರು ಏಕೆ ಮತ ಚಲಾವಣೆ ಮಾಡಬಾರದು ಎನ್ನುವುದು ನನಗೆ ತಿಳಿಯುತ್ತಿಲ್ಲ”.

ಹಿಂದೂ ಸ್ತ್ರೀಯರಿಗೆ ತಮ್ಮ ಮನೆಗಳಲ್ಲಿ ಯಾವ ಸ್ಥಾನವಿದೆ, ಗಂಡಂದಿರು ಅವರನ್ನು ಹೇಗೆ ನೋಡಿಕೊಳ್ಳುತ್ತಿದ್ದಾರೆ ಎಂಬ ಪ್ರಶ್ನೆಗೆ ಅವರು ಉತ್ತರಿಸದೆ ನುಣುಚಿಕೊಂಡರು. ಬಹುಶಃ ಅದು ಅವರಿಗೆ ಅಷ್ಟಾಗಿ ಗೊತ್ತಿಲ್ಲದಿರಬಹುದು. ಅವರು ಮದುವೆಯಾದವರಲ್ಲ. ಅವರ ಜಾತಿಯ ಪೂಜಾರಿಗಳು ಮದುವೆಯಾಗುವುದಿಲ್ಲ.

ಅಮೆರಿಕಾದಲ್ಲಿ ಎರಡು ಸಂಗತಿಗಳು ತನ್ನ ಮೇಲೆ ಪರಿಣಾಮ ಮಾಡಿವೆ ಎಂದು ಅವರು ಹೇಳಿದರು. ಒಂದು ಇಡೀ ದೇಶದಲ್ಲಿ ಬಡತನವೇ ಇಲ್ಲದಿರುವುದು, ಇನ್ನೊಂದು ದಕ್ಷಿಣ ಭಾಗದಲ್ಲಿ ಅಸಾಧಾರಣವಾದ ಅಜ್ಞಾನವಿರುವುದು.

\begin{center}
\textbf{ಎಲಿವೇಟರ್ ಇಷ್ಟವಾದದ್ದು}
\end{center}

ರೆನ್ನರ್ಟ್ ಹೋಟಲಿನಲ್ಲಿ ಎಲಿವೇಟರ್ ಬಳಿ ಹೋದಾಗ ಅವರೆಂದರು:

“ನಮ್ಮ ಭಾರತದಲ್ಲಿ ಎಷ್ಟು ಮಾತ್ರಕ್ಕೂ ಇಲ್ಲದ ಒಂದು ಸಂಸ್ಥಾಪನೆ ಇದು. ನನಗೆ ಇದು ಬಹಳ ಇಷ್ಟವಾಗಿದೆ.”

ಮಹಿಳೆಯೊಬ್ಬರು ಎಲಿವೇಟರ್ನಿಂದ ಆಗತಾನೆ ಬರುತ್ತಿದ್ದರು. ಇವರ ಕಡುಗೆಂಪು ಮತ್ತು ಹಳದಿ ಉಡುಪನ್ನು ನೋಡಿ ಅವರು ದಿಗ್ಭ್ರಾಂತರಾದರು; ಆದರೆ ಎಳ್ಳಷ್ಟೂ ವ್ಯತ್ಯಸ್ತವಾಗದ ಇವರ ಮುಖಭಾವ ತಾವು ಅವರ ಗಮನ ಸೆಳೆದಿರುವುದರ ಪರಿಜ್ಞಾನವೇ ಇಲ್ಲದಿರುವಂತೆ ಇತ್ತು.

ನಾಳೆ ರಾತ್ರಿ ಲೈಸಿಯಂನಲ್ಲಿ ಅವರ ಭಾಷಣ ಮುಖ್ಯವಾಗಿ ತಮ್ಮನ್ನು ತಾವು ಪರಿಚಯಿಸಿಕೊಳ್ಳುವುದು ಮತ್ತು ಹಿಂದೂ ದೇಶದ ಬಗ್ಗೆ ವಿವರಣೆ ಕೊಡುವುದು. ಸ್ವಲ್ಪಹೊತ್ತು ಮಾತ್ರವೇ ಮಾತನಾಡುವರು; ಆದರೆ ಬಾಲ್ಟಿಮೋರ್ನಲ್ಲೇ ಉಳಿದು ನಾಳೆ ರಾತ್ರಿಯಿಂದ ಒಂದು ವಾರ ಸುದೀರ್ಘವಾಗಿ ಮಾತನಾಡಲಿರುವರು.

\begin{center}
\textbf{ನಮ್ಮ ನಡುವೆ ಒಬ್ಬ ವಿವೇಕಿ\supskpt{\footnote{\enginline{1. New Discoveries, Vol. 2, p. 191-192}}}}
\end{center}

\begin{center}
(ಬಾಲ್ಟಿಮೋರ್ ಸಂಡೇ ಹೆರಾಲ್ಡ್, ೧೪ ಅಕ್ಟೋಬರ್ ೧೮೯೪)
\end{center}

\begin{center}
\textbf{ಈ ನಗರಕ್ಕೆ ಪ್ರಸಿದ್ಧ ಹಿಂದೂ ಭಗವದಾರಾಧಕರೊಬ್ಬರ ಭೇಟಿ}
\end{center}

ವ್ರೂಮನ್ ಸೋದರರ ಅತಿಥಿಯಾಗಿರುವ ಅವರು ಧರ್ಮಗಳ ಅಂತರರಾಷ್ಟ್ರೀಯ ವಿಶ್ವವಿದ್ಯಾನಿಲಯವೊಂದನ್ನು ಸ್ಥಾಪಿಸುವ ಆಸಕ್ತಿ ಹೊಂದಿರುವರು-ಅವರು ಉಜ್ಜಲ ವಸನ

..... ಮಿ. ವಿವೇಕಾನಂದರು \enginline{(Vivecananda)} ಸಂಡೇ ಹೆರಾಲ್ಡ್ ವರದಿಗಾರ ರೊಬ್ಬರೊಡನೆ ಸಂಭಾಷಿಸಿದರು. ಅವರು ಸುಶಿಕ್ಷಿತ ಇಟಾಲಿಯನ್ನರಂತೆ ಉಚ್ಚಾರಣೆ ಯುಳ್ಳ ಇಂಗ್ಲಿಷನ್ನು ಕಷ್ಟವಿಲ್ಲದೆ ಮಾತನಾಡುತ್ತಾರೆ. ಅವರಿಗೆ ಈ ದೇಶದ ಧಾರ್ಮಿಕ, ರಾಜಕೀಯ ಮತ್ತು ಸಾಮಾಜಿಕ ಸಂಸ್ಥೆಗಳ ಪರಿಚಯ ತುಂಬ ಚೆನ್ನಾಗಿರುವಂತೆ ಕಂಡುಬಂತು.

ವ್ರೂಮನ್ ಸೋದರರ- ಹಿರಮ್​, ಕಾರ್ಲ್ ಮತ್ತು ವಾಲ್ಟರ್ - ಆಹ್ವಾನದ ಮೇರೆಗೆ ಮಿ. ವಿವೇಕಾನಂದರು ಬಾಲ್ಟಿಮೋರ್ಗೆ ಆಗಮಿಸಿರುವರು; ಈ ನಗರದಲ್ಲಿರುವಾಗ ಆವರ ಅತಿಥಿಯಾಗಿರುವರು. ರೆವರೆಂಡ್ ಹಿರಮ್​ ಅವರನ್ನು ಅವರ ೧೧೨೨, ಉತ್ತರ ಕ್ಯಾಲ್ವರ್ಟ್ ಬೀದಿಯ ನಿವಾಸದಲ್ಲಿ ಸಂದರ್ಶಿಸಲಾಯಿತು. ಅವರು ಪ್ರಖ್ಯಾತ ಅತಿಥಿಯ ಭೇಟಿಯನ್ನು ಕುರಿತು ಮುಕ್ತವಾಗಿ ಮಾತನಾಡಿದರು.

“ಮಿ. ವಿವೇಕಾನಂದರು ನಾನು ಭೇಟಿಮಾಡಿದ ಅತ್ಯಂತ ಬುದ್ಧಿವಂತರಲ್ಲಿ ಒಬ್ಬರು. ನಮ್ಮ ಆಹ್ವಾನದ ಮೇರೆಗೆ ಅವರು ಈ ನಗರಕ್ಕೆ ಬಂದಿರುವರು; ಇಲ್ಲಿರುವಾಗ ನಮ್ಮೊಂದಿಗೆ ಅವರು ಅಂತರರಾಷ್ಟ್ರೀಯ ವಿಶ್ವವಿದ್ಯಾನಿಲಯದ ಸ್ಥಾಪನೆಯನ್ನು ಕುರಿತು ಚರ್ಚಿಸುವರು. ವಿಶ್ವ ಸಮ್ಮೇಳನ ಅತ್ಯಂತ ಆಸಕ್ತಿಯ ಅಂಶವಾಗಿದ್ದ ವಿಶ್ವ ಧಾರ್ಮಿಕ ಅಧಿವೇಶನದ ಒಂದು ಸತಙಲವಾಗಿ ಇದನ್ನು ಸ್ಥಾಪಿಸುವುದೆಂದು ಯೋಜಿಸಲಾಗಿದೆ. ಈ ವಿಶ್ವವಿದ್ಯಾನಿಲಯವು ಮಿ. ವಿವೇಕಾನಂದರ ಇಷ್ಟವಾದ ಕಲ್ಪನೆಗಳಲ್ಲೊಂದು; ಇದಕ್ಕೆ ನನ್ನ ಮತ್ತು ನನ್ನ ಸೋದರರ ಪೂರ್ಣ ಸಹಾನುಭೂತಿ ಇದೆ; ಅನೇಕ ಧರ್ಮಗಳ ಉತ್ತಮ ಸ್ಥಾನವಂತರೂ ಶ‍್ರೀಮಂತರೂ ಆದ ಅನೇಕಾನೇಕ ಸಭ್ಯ ಮಹನೀಯರ ಸಹಸ್ಪಂದ ನವೂ ಇದೆ. ಹಿಬ್ರೂ ಧರ್ಮಗಳವರೂ, ರೋಮನ್ ಕ್ಯಾಥೊಲಿಕ್ ಸದಸ್ಯರೂ ಸಹ ಇದರ ಪ್ರೇರಕರಾಗಿರುವರು. ವಿಶ್ವವಿದ್ಯಾನಿಲಯದ ಕಲ್ಪನೆಯೆಂದರೆ ಸಾಮಾನ್ಯ ಧರ್ಮದಲ್ಲಿ ಶಿಕ್ಷಣ ಕೊಡುವುದು...

“ವಿಶ್ವವಿದ್ಯಾನಿಲಯವನ್ನು ಸ್ಥಾಪಿಸುವ ಹಿಂದಿನ ಮಿ. ವಿವೇಕಾನಂದರ ಒಂದು ಕಲ್ಪನೆ ಎಂದರೆ ಭಾರತದಲ್ಲಿ ಕೆಲಸ ಮಾಡುವುದಕ್ಕಾಗಿ ಉತ್ತಮ ರೀತಿಯ ಪ್ರಚಾರಕರನ್ನು ಶಿಕ್ಷಣ ಕೊಟ್ಟು ತಯಾರು ಮಾಡುವುದು. ತಾವು ತಮ್ಮ ಧಾರ್ಮಿಕ ನಂಬಿಕೆಯಲ್ಲಿ ಸ್ಥಿರವಾಗಿರುವರಾದರೂ, ಭಾರತಕ್ಕೆ ಅಜ್ಞಾನಿಗಳಾದವರನ್ನು ಪ್ರಚಾರಕರೆಂದು ಕಳುಹಿಸು ತ್ತಿರುವ ಈಗಿನ ಪದ್ಧತಿ ನಿಲ್ಲಬೇಕು, ಬದಲಿಗೆ ಕ್ರೈಸ್ತಧರ್ಮವನ್ನೇ ಉನ್ನತಸ್ತರದಲ್ಲಿ ಬೋಧಿಸುವಂಥವರನ್ನು ಕಳುಹಿಸಬೇಕು ಎಂದು ಅವರು ಬಯಸುತ್ತಾರೆ. ಸಾಮಾನ್ಯ ಧರ್ಮದ ಒಳಿತಿಗಾಗಿ ಮಾತ್ರವೇ ಅವರು ಹೀಗೆ ಬಯಸುತ್ತಾರೆ.

“ತಮ್ಮ ತಂದೆಲಾರ್ಡ್ ಜೀಸಸ್ (ಅವರು ಹಾಗೆ ಕರೆಯುತ್ತಿದ್ದರು)ನಲ್ಲಿ ತುಂಬ ನಂಬಿಕೆಯಿಟ್ಟುಕೊಂಡಿದ್ದವರು; ತಾವು ಬಾಲಕರಾಗಿದ್ದಾಗ ಸಂತ ಜಾನನಸುವಾರ್ತೆಯಲ್ಲಿರುವ ಮುಕ್ತಿಪ್ರದಾಯಕನನ್ನು ಶಿಲುಬೆಗೇರಿಸಿದ ಮೈನವಿರೇಳಿಸುವ ವಿವರನ್ನು ಓದಿ ಅತ್ತು ಬಿಟ್ಟಿದ್ದರೆಂದೂ ಮಿ. ವಿವೇಕಾನಂದರು ನನಗೆ ಹೇಳಿದರು. ಅವರು ಈ ನಗರದಲ್ಲಿ ಅನೇಕ ವಾರಗಳ ಕಾಲ ಇರುತ್ತಾರೆ. ನಾಳೆ ಸಂಜೆ ಅವರು ನಾವು ಸಭೆ ಸೇರುವ ಲೈಸಿಯಂನಲ್ಲಿ ಚಿಕ್ಕದೊಂದು ಭಾಷಣ ಮಾಡಲಿದ್ದಾರೆ; ಭಾನುವಾರದಂದು, ನಾವು ಎರಡನೆಯ ಬಾರಿಗೆ ಸಭೆ ಸೇರುವಾಗ ವಿಶ್ವ ವಿದ್ಯಾನಿಲಯದ ಯೋಜನೆಯನ್ನು ಕುರಿತು ದೀರ್ಘವಾಗಿ ಮಾತನಾಡುತ್ತಾರೆ.”

\begin{center}
\textbf{ಧರ್ಮದ ಸಾರವೇ ಪ್ರೇಮ\supskpt{\footnote{\enginline{Ray and Wanda Ellis, "Swami Vivekananda in Washungton, D.C.", The Vesdanta Kesari, 1991, pp. 370-379}}}}
\end{center}

\begin{center}
\textbf{ಬ್ರಾಹ್ಮಣ ಸಂನ್ಯಾಸಿ ಸ್ವಾಮಿ ವಿವೇಕಾನಂದರು ಪೀಪಲ್ಸ್ ಚರ್ಚ್ನಲ್ಲಿ ಮಾಡಿದ ಬೋಧನೆ}
\end{center}

(ವಾಷಿಂಗ್ಟನ್ ಟೈಮ್​, ಸೋಮವಾರ, ೨೯ ಅಕ್ಟೋಬರ್ ೧೮೯೪)

ಬ್ರಾಹ್ಮಣ ಸಂನ್ಯಾಸಿ ವಿವೇಕಾನಂದರು ನೆನ್ನೆ ಪ್ರಾತಃಕಾಲ\footnote{3. ಸ್ವಾಮಿ ವಿವೇಕಾನಂದರು ಭಾನುವಾರ ೨೮ ಅಕ್ಟೋಬರ್ ೧೮೯೪ರಂದು ಪೀಪಲ್ಸ್ ಚರ್ಚ್ನಲ್ಲಿ ಎರಡು ಭಾಷಣಗಳನ್ನು ಮಾಡಿದ್ದಾರೆ. ಈ ಉಪನ್ಯಾಸಗಳ ಪದಶಃ ವರದಿ ಲಭ್ಯವಿಲ್ಲ. ಈ ಬೆಳಗಿನ ಉಪನ್ಯಾಸವಾದ ಮೇಲೆ ನಡೆದ ಸಂದರ್ಶನವೊಂದಕ್ಕಾಗಿ ನೋಡಿ, \enginline{Complete Works, II: 497-499}} ಹನ್ನೊಂದು ಗಂಟೆಗೆ ನಂ.೪೨೩, ವಾಯುವ್ಯ ಜಿ ರಸ್ತೆಯಲ್ಲಿರುವ ಪೀಪಲ್ಸ್ ಚರ್ಚ್ನಲ್ಲಿ ಸೇರಿದ್ದ ಭಕ್ತಮಂಡಲಿಯನ್ನುದ್ದೇಶಿಸಿ ಮಾತನಾಡಿದರು... ಯತಿಗಳನ್ನು ಡಾ. ಕೆಂಟ್ ಅವರು ಪರಿಚಯ ಮಾಡಿಕೊಟ್ಟರು...

ಮುಂದೆ ಬಂದ ವಿವೇಕಾನಂದರು, ತಾವು ಹುಡುಗನಾಗಿದ್ದಾಗ ವಿಶ್ವವಿದ್ಯಾನಿಲಯದಲ್ಲಿ ತೌಲನಿಕ ಧರ್ಮವನ್ನು ಅಧ್ಯಯನ ಮಾಡಿರುವುದಾಗಿ ಹೇಳಿದರು. ಮುಂದುವರೆದು ಹೇಳಿದರು - ಭಾರತದಲ್ಲಿ ಅನೇಕ ಧರ್ಮಗಳು ಇವೆ. ಐದನೆಯ ಒಂದು ಭಾಗದಷ್ಟು ಜನರು ಮಹಮ್ಮದೀಯರು. ಹತ್ತು ಲಕ್ಷದಷ್ಟು ಜನರು ಕ್ರೈಸ್ತರು. ಅವರನ್ನೆಲ್ಲ ಅಭ್ಯಸಿಸಿದ್ದೇನೆ ಎಂದರು. ದೊಡ್ಡ ಹಿಂದೂ ಪ್ರಬೋಧಕರು ಹೇಳಿದುದನ್ನೆಲ್ಲ ಕೇಳಿದ ನಂತರ ಅವರು ಕೇಳಿದರಂತೆ:

“ಸೋದರನೆ, ನೀನು ದೇವರನ್ನು ನೋಡಿರುವೆಯಾ?”

ಪ್ರಬೋಧಕರು ಅಚ್ಚರಿಯಿಂದ ತಲೆಯೆತ್ತಿ ನೋಡಿದರು.

“ಇಲ್ಲ.”

“ಹಾಗಿದ್ದ ಮೇಲೆ, ಇವೆಲ್ಲ ನಿಜವೆಂದು ನಿನಗೆ ಹೇಗೆ ಗೊತ್ತು?”

“ನನ್ನ ತಂದೆಯವರು ನನಗೆ ಹೇಳಿದರು.”

“ನಿಮ್ಮ ತಂದೆಗೆ ಯಾರು ಹೇಳಿದರು?”

“ಅವರ ತಂದೆ”, ಇತ್ಯಾದಿಯಾಗಿ ಪಿತೃಗಳ ಮೂಲಕ ಮೇಘಗಳವರೆಗೂ ಹೋಯಿತು.

ಪ್ರಚಂಡವಾಗ್ಮಿಯಾಗಿದ್ದ ಕ್ರೈಸ್ತ ಪ್ರಬೋಧಕರೊಬ್ಬರ ಭಾಷಣವನ್ನು ಕೇಳಿದರಂತೆ. ಈ ಮನುಷ್ಯ ಸತ್ಯಾನ್ವೇಷಿಯಾಗಿದ್ದವನಿಗೆ ಹೇಳಿದನಂತೆ - ನಿನ್ನನ್ನೀಗಲೇ ನೀರಿನಲ್ಲಿ ಮುಳುಗಿಸದೇ ಇದ್ದರೆ ಜೀವಂತ ಸುಡುವ ಅಪಾಯ ಕಾದಿದೆ ಎಂದು. ಮುಂದು ಮುಂದಕ್ಕೆ ಪ್ರಶ್ನಿಸುತ್ತ ಹೋದಂತೆ ಈ ಕ್ರೈಸ್ತನೂ ಸಹ, ತನ್ನ ಶಾಸ್ತ್ರಗಂಥಗಳ ಮೂಲಕ, ಪಿತೃಗಳಿಗೆ, ಆ ನಂತರ ಮೇಘಗಳಿಗೆ ಹೋದನಂತೆ.

\begin{center}
\textbf{ವಿದ್ಯಾರ್ಥಿಗೆ ತೃಪ್ತಿಯಾಗಲಿಲ್ಲ}
\end{center}

ಇದರಿಂದ ವಿದ್ಯಾರ್ಥಿಗೆ ತೃಪ್ತಿಯಾಗಲಿಲ್ಲ. ಅವನು ಪ್ರಾರ್ಥಿಸುತ್ತಲೇ ಹೋದ. ಮೂರು ಹಗಲು ರಾತ್ರಿ ಅಳುತ್ತ ಆಹಾರವಿಲ್ಲದೆ ಪ್ರಾರ್ಥಿಸುತ್ತಲೇ ಇದ್ದ. ಕೊನೆಗೆ ಅವನಿಗೆ ಪುಸ್ತಕಗಳನ್ನೋದುವುದಿರಲಿ, ತನ್ನ ಹೆಸರನ್ನೇ ಬರೆಯಲು ಬಾರದ ಒಬ್ಬ ಮನುಷ್ಯ ಸಿಕ್ಕಿದ ನಂತೆ. ಈ ತಪಸ್ವಿಯೂ ಧರ್ಮವನ್ನು ಬೋಧಿಸುತ್ತಿದ್ದನಂತೆ. ಅದೇ ಹಳೆಯ ಪ್ರಶ್ನೆಯನ್ನು ಕೇಳಿದಾಗ, ಆ ತಪಸ್ವಿ

“ಹೌದು, ನಾನು ದೇವರನ್ನು ಕಂಡಿರುವೆನು, ನಿನಗೂ ಅವನನ್ನ್ನು ಕಾಣುವುದು ಹೇಗೆ ಹೇಳಿಕೊಡುವೆನು” ಎಂದು ಉತ್ತರಿಸಿದರಂತೆ.

ಅವರ ಚಹರೆಯೆಲ್ಲವೂ ಭಗವಂತನ ಮುದ್ರೆಯಿಂದ ತುಂಬಿಹೋಗಿತ್ತು. ಆ ನಜರತ್ದ ಮನುಷ್ಯನಿಗೆ ಜೋರ್ಡಾನ್ನಲ್ಲಿ ಪಾರಿವಾಳ ಅವನ ಬಳಿಗೆ ಇಳಿದು ಬಂದಾಗ ದೊರಕಿದ್ದೂ ಇದೇ ಪ್ರಮಾಣಪತ್ರವೇ. ದೇವರು ಇರುವನೆಂದೂ ಧರ್ಮವೆಂದರೆ ಅನುಕರಣೆಯಲ್ಲವೆಂದೂ ಅವರು ತನ್ನ ಶ್ರೋತೃಗಳನ್ನು ನಂಬಿಸಿದರು.

ವಿವೇಕಾನಂದರು ಈ ಮನುಷ್ಯನ ಪದತಲದಲ್ಲಿ ಹನ್ನೆರಡು ವರ್ಷಗಳ ಕಾಲ ಕುಳಿತರು. ಅವರೇ ಗುರು. ಒಂದು ದಿನ ಅವರು “ಈ ಪುಸ್ತಕ ತೆಗೆದುಕೋ” ಎಂದರಂತೆ. ವಿವೇಕಾನಂದರು ಅದನ್ನು ತೆಗೆದುಕೊಂಡು ಓದಿದರು. ಅದೊಂದು ಕ್ಯಾಲೆಂಡರ್. ಅದನ್ನೋದುತ್ತ ಅವರು ಈ ವರ್ಷ ಎಷ್ಟು ಮಳೆಯಾಗುತ್ತದೆ ಎಂದು ಭವಿಷ್ಯ ನುಡಿ ಯುವ ಭಾಗಕ್ಕೆ ಬಂದರು. ಒಂದು ಅವಧಿಯಲ್ಲಿ ಒಂದಾನೊಂದು ಜಿಲ್ಲೆಯಲ್ಲಿ ಇಷ್ಟು ಮಳೆಯಾಗುವುದು ಎಂದು ಅದರಲ್ಲಿ ಇತ್ತಂತೆ. ಗುರುಗಳು “ಈಗ ಪುಸ್ತಕವನ್ನು ಚೆನ್ನಾಗಿ ಹಿಂಡು” ಎಂದರಂತೆ. ಇವರೂ ವಿಧೇಯರಾಗಿ ಹಾಗೆಯೇ ಮಾಡಿದರಂತೆ, “ಪುಸ್ತಕದಿಂದ ನೀರೇನಾದರೂ ಬಂದಿತೆ?” “ಇಲ್ಲ.” ಗುರುಗಳು, “ಅಂತೆಯೇ ಎಲ್ಲ ಪುಸ್ತಕಗಳೂ ಸಹ. ನಿಜವಾದ ಧರ್ಮ ಇಲ್ಲಿದೆ, ಹೃದಯದಲ್ಲಿ” ಎಂದು ನುಡಿದರಂತೆ.

ನಿಜ ಸಂಗತಿ ಎಂದರೆ ಜನರಿಗೆ ದೇವರ ಅಪೇಕ್ಷೆಯೇ ಇಲ್ಲ. ಖಂಡಿತವಾಗಿಯೂ ಇಲ್ಲ. ಧರ್ಮ ಎಂದರೆ ಹೆಚ್ಚೆಂದರೆ ಒಂದು ಫ್ಯಾಶನ್. ನನ್ನ ಮಹಿಳೆಗೆ ಒಳ್ಳೆಯದೊಂದು ಪ್ರಾಂಗಣವಿದೆ, ಒಪ್ಪವಾಗಿ ಸಜ್ಜಾದ ಪೀಠೋಪಕರಣಗಳಿವೆ, ಪಿಯಾನೋ ಇದೆ, ಅಂದವಾದ ಆಭರಣಗಳಿವೆ, ಮೈಗೆ ಚೆನ್ನಾಗಿ ಒಪ್ಪುವ ದುಬಾರಿ ಉಡುಪುಗಳಿವೆ, ಅತ್ಯಾಧುನಿಕ ಹ್ಯಾಟ್ ಸಹ ಇದೆ. ಈ ಸೆಟ್ ಜೊತೆಗೆ ಸರಿಹೊಂದುವಂತಹ ಧರ್ಮದ ಒಗ್ಗರಣೆ ಯೊಂದು ಇಲ್ಲದೆ ಅವಳಿಗೆ ನಿತ್ಯಗಟ್ಟಲೆಯ ವ್ಯವಹಾರಲ್ಲಿ ಏನೋ ಕೊರೆ, ಏನೋ ಚಡಪಡಿಕೆ. ಇಂತಹ ಧರ್ಮ ಬೇಕಾದಷ್ಟಿದೆ, ಆದರೆ ಅದು ಆಷಾಢ ಭೂತಿತನ; ಅಲ್ಲದೆ ಈ ಡಬಡ್ಡಾಳಿಕೆಯೇ ಎಲ್ಲ ಕೇಡಿನ ಮೂಲ. ಈ ತರಹದ ಧರ್ಮ ದೇವರ ಧರ್ಮವಲ್ಲ. ಅದು ಕೇವಲ ನೆರಳು. ಜನರು ಕೆಲವು ವೇಳೆ ಇಂತಹ ಧರ್ಮವನ್ನಿಟ್ಟುಕೊಂಡೇ ಕಾತರರಾಗಿಬಿಡುತ್ತಾರೆ, ತಮ್ಮೊಳಗೆ ಸತ್ಯ ಇದ್ದರೆ ಹೇಗೋ ಹಾಗೆ ಧರ್ಮದ ಬಗ್ಗೆ ಮಾತನಾಡಲಾರಂಭಿಸುತ್ತಾರೆ. ಒಳಗೆ ಸತ್ವ ಇಲ್ಲದೆ ಹೀಗೆ ಧರ್ಮದ ಬಗ್ಗೆ ಮಾತ ನಾಡತೊಡಗಿದಾಗ ಜನರು ಜಗಳಕ್ಕೆ ಬೀಳುತ್ತಾರೆ, ಹೊಡೆದಾಡಲಾರಂಭಿಸುತ್ತಾರೆ. “ನನ್ನದು, ನನ್ನದು”. “ಇಲ್ಲ, ನನ್ನದು”. ಹೀಗೇ ಜಗಳವಾಡುತ್ತಾರೆ, ಅನಾಗರಿಕ ಕಾಡು ಜನರ ಬುಡಕಟ್ಟುಗಳವರು ಮಂಬೋ ಜಂಬೋ ಎಂಬ ತಮ್ಮ ಸ್ಪರ್ಧಿ ದೇವತೆಗಳ ಬಗ್ಗೆ ಜಗಳವಾಡಿದ ಹಾಗೆ. ಸ್ಪರ್ಧೆಯೇ ಎಲ್ಲ ಬೇನೆಗಳಿಗೂ ಮೂಲ, ವ್ಯಾಪಾರದಲ್ಲಿರುವ ಹಾಗೆ, ಧರ್ಮದಲ್ಲೂ ಸಹ.

\begin{center}
\textbf{ಪ್ರೇಮ ಮಾತ್ರವೇ ಉಳಿಯುವುದು}
\end{center}

ನಿಮ್ಮ ಸಂತ ಪಾಲರೇ ಹೇಳುತ್ತಾರೆ “ಎಲ್ಲವೂ ನಾಶವಾಗುತ್ತದೆ, ಆದರೆ ಪ್ರೇಮವೊಂದೇ ಉಳಿಯುತ್ತದೆ” ಎಂದು. ಇದೇ ಮಹಾ ಸತ್ಯ. ಉಳಿದೆಲ್ಲ ದೇಶಗಳನ್ನೂ ಹಿಂದಕ್ಕೆ ಹಾಕಿ ತನ್ನ ದೇಶವೊಂದನ್ನೇ ವೈಭವೀಕರಿಸಬೇಕೆನ್ನುವ ಸಿದ್ಧಾಂತ -ಅದು ದೇವರಿಗೆ ಸಂಬಂಧಿಸಿದ್ದಲ್ಲ.

ತರುಣನೊಬ್ಬ ತನ್ನ ಗುರುವಿನ ಬಳಿಗೆ ಹೋಗಿ “ನಾನು ದೇವರನ್ನು ಅರಿತು ಕೊಳ್ಳಬೇಕೆಂದಿರುವೆ” ಎಂದನು. ಗುರುವು ಇದಕ್ಕೆ ಗಮನವನ್ನೇ ಕೊಡಲಿಲ್ಲ; ಆದರೆ ತರುಣನದು ಒಂದೇ ಹಟ - ಏನು ಮಾಡಿದರೂ ಬಿಡಲೊಲ್ಲ. ಕೊನೆಗೆ ಒಂದು ದಿನ ಗುರು “ನದಿಗೆ ಹೋಗಿ ಸ್ನಾನ ಮಾಡಿ ಬರೋಣ, ಬಾ” ಎಂದು ಕರೆದ. ಇಬ್ಬರೂ ನದಿಗೆ ಹೋದರು; ತರುಣ ಮುಳುಗು ಹಾಕತೊಡಗಿದ. ಗುರು ಅವನ ಮೇಲೆ ಬಿದ್ದು ಅವನನ್ನು ನೀರಿನೊಳಗೆ ಮುಳುಗಿರುವಂತೆಯೇ ಹಿಡಿದುಕೊಂಡ. ತರುಣ ಬಿಡಿಸಿಕೊಳ್ಳಲು ಹೆಣ ಗಾಡಿದರೂ, ಗುರು ಬಿಡಲೊಲ್ಲ. ಕೊನೆಗೆ, ತರುಣ ಇನ್ನೇನು ಸತ್ತೇ ಹೋಗಬಹುದು ಎಂದೆನ್ನಿಸುವಂತಾದಾಗ, ನೀರಿನಿಂದ ಮೇಲಕ್ಕೆತ್ತಿ ಬದುಕಿಸಿದ. ಆಮೇಲೆ ಗುರು “ನೀನು ನೀರಿನೊಳಗೆ ಇರುವಾಗ ಯಾವುದಕ್ಕಾಗಿ ಹಂಬಲಿಸುತ್ತಿದ್ದೆ?” ಎಂದು ಕೇಳಿದಾಗ ತರುಣ “ಉಸಿರಿಗಾಗಿ” ಎಂದು ಉತ್ತರ ಕೊಟ್ಟ. “ಹಾಗಿದ್ದರೆ ನಿನಗೆ ದೇವರು ಬೇಕಾಗಿಲ್ಲ” ಎಂದು ಬಿಟ್ಟ ಗುರು.

ಎಲ್ಲ ಮನುಷ್ಯರೂ ಹಾಗೆಯೇ; ನಿಮಗೆ ಏನುಬೇಕು? ನಿಮಗೆ ಉಸಿರುಬೇಕು, ಅದಿಲ್ಲದೆ ಬದುಕಲಾರಿರಿ; ನಿಮಗೆ ಆಹಾರಬೇಕು, ಅದಿಲ್ಲದೆ ಬದುಕಲಾರಿರಿ; ಇರ ಲೊಂದು ಮನೆಬೇಕು, ಅದಿಲ್ಲದೆ ಬದುಕಲಾರಿರಿ. ಈ ವಸ್ತುಗಳನ್ನು ನೀವು ಬಯಸುವ ಹಾಗೆಯೇ ದೇವರನ್ನು ಬಯಸುವುದಾದರೆ, ಆಗ ದೇವರು ನಿಮ್ಮೊಳಗೆ ಆವಿರ್ಭವಿಸುತ್ತಾನೆ. ದೇವರು ಬೇಕೆಂದು ಅಪೇಕ್ಷಿಸುವುದಿದೆಯಲ್ಲ, ಅದೇ ದೊಡ್ಡದು.

ಈ ಪ್ರಪಂಚದಲ್ಲಿರುವ ಹೆಚ್ಚು ಜನ ಸ್ತ್ರೀಪುರುಷರಿಗೆ ಇಂದ್ರಿಯಗಳ ಆನಂದಬೇಕು. ದೂರದಲ್ಲೆಲ್ಲೋ ದೇವರೊಬ್ಬನಿದ್ದಾನೆ ಎಂದು ಅವರಿಗೆ ಹೇಳಿರುವರು. ಒಂದು ಗಾಡಿ ತುಂಬ ಶಬ್ದಗಳನ್ನು ಅವನಿಗೆ ಕಳುಹಿಸಿಕೊಟ್ಟರೆ, ಅವನು ಇಲ್ಲಿ ಈ ಪ್ರಪಂಚದ ಈ ‘ಒಳ್ಳೆಯ’ ಸಂಗತಿಗಳನ್ನು ಪಡೆದುಕೊಳ್ಳುವುದಕ್ಕೆ ಸಹಾಯ ಮಾಡುತ್ತಾನೆ. ಆದರೆ, ಪ್ರತಿಯೊಂದು ದೇಶದಲ್ಲಿಯೂ ದೇವರನ್ನು ಅಪೇಕ್ಷಿಸುವವರು ಕೆಲವರಾದರೂ ಇರುತ್ತಾರೆ. ಅವರು ಸತ್ಯವೆಂಬುದರ, ಒಳಿತು ಎಂಬುದರ ಸಾರದೊಡನೆ ಒಂದಾಗಿರುತ್ತಾರೆ. ಧರ್ಮ ಎಂದರೆ ಅಂಗಡಿಯಿಟ್ಟು ವ್ಯಾಪಾರಮಾಡುವುದಲ್ಲ. ಪ್ರೇಮವು ಪ್ರತಿಯಾಗಿ ಏನನ್ನೂ ಅಪೇಕ್ಷಿಸುವುದಿಲ್ಲ; ಪ್ರೇಮ ಬೇಡುವುದಿಲ್ಲ; ಪ್ರೇಮ ಕೊಡುತ್ತದೆ.

ಧರ್ಮ ಎನ್ನುವುದು ಭಯದಿಂದ ಬಂದಿರುವ ಬೆಳವಣಿಗೆಯಲ್ಲ; ಧರ್ಮ ಎನ್ನುವುದು ಆನಂದಮಯ. ಅದು ಹಕ್ಕಿಯ ಹಾಡಿನಂತೆ, ಬೆಳಗಿನ ಸುಂದರ ದೃಶ್ಯದಂತೆ,

ತಂತಾನೆ ಹೊರ ಹೊಮ್ಮುವಂಥದು. ಅದು ಚೇತನದ ಅಭಿವ್ಯಕ್ತಿ. ಅಂತರಂಗದಿಂದ ಹೊರಡುವ ಮುಕ್ತವೂ ಉನ್ನತೋನ್ನತ ಅನ್ಯಾದೃಶವೂ ಆದ ಚೇತನದ ಅಭಿವ್ಯಕ್ತಿ.

ಯಾತನೆಯೇ ಧರ್ಮ ಎಂದಾದರೆ, ನರಕ ಎನ್ನುವುದು ಏನು? ಯಾರಿಗೂ ತನ್ನನ್ನು ತಾನು ದುಃಖಿಯನ್ನಾಗಿ ಮಾಡಿಕೊಳ್ಳುವ ಹಕ್ಕಿಲ್ಲ. ಹಾಗೆ ಮಾಡುವುದು ತಪ್ಪು; ಹಾಗೆ ಮಾಡುವುದು ಪಾಪ. ಪ್ರತಿಯೊಂದು ನಗೆಯ ಬುಗ್ಗೆಯೂ ದೇವರಿಗೆ ಕಳುಹಿಸಿದ ಪ್ರಾರ್ಥನೆ.

ಹಿಂತಿರುಗಿ ನೋಡಿದಾಗ, ನಾನು ಕಲಿತಿರುವುದು ಇದು: ಧರ್ಮ ಪುಸ್ತಕಗಳಲ್ಲಿ ಇಲ್ಲ, ರೂಪಗಳಲ್ಲಿ ಇಲ್ಲ, ಮತಪಂಥಗಳಲ್ಲಿ ಇಲ್ಲ, ದೇಶಗಳಲ್ಲಿ ಇಲ್ಲ; ಧರ್ಮವಿರುವುದು ಮನುಷ್ಯನ ಹೃದಯದಲ್ಲಿ. ಅದು ಅಲ್ಲಿಯೇ ನೆಲೆನಿಂತಿದೆ. ಇದಕ್ಕೆ ತಾರ್ಕಣೆ ನಮ್ಮೊಳಗೇ ಇದೆ.

ನಾನು ಎರಡು ವಿಷಯಗಳನ್ನು ಹೇಳಬಯಸುತ್ತೇನೆ. ಈ ಮತಪಂಥಗಳಿವೆಯಲ್ಲಾ, ಅವು ಹೆಚ್ಚಾಗುತ್ತಲೇ ಹೋಗಲಿ - ಪ್ರತಿಯೊಬ್ಬ ವ್ಯಕ್ತಿಯದೂ ಒಂದು ಮತಪಂಥವಾಗುವವರೆಗೆ. ಯಾರೂ ದೇವರನ್ನು ನಿಷ್ಕೃಷ್ಟವಾಗಿ ಇನ್ನೊಬ್ಬನು ನೋಡಿದ ಹಾಗೇ ನೋಡಲಾರ. ಪ್ರತಿಯೊಬ್ಬನೂ ದೇವನಲ್ಲಿ ನಂಬಿಕೆಯಿಟ್ಟು ತನಗೆ ಅವನು ಕಾಣಿಸುವಂತೆಯೇ ಅವನನ್ನು ಆರಾಧಿಸಬೇಕು. ಅನಂತರ, ಮತಪಂಥಗಳಲ್ಲಿ ಸಾಮ ರಸ್ಯವಿರಬೇಕೆಂದು ನಾನು ಬಯಸುತ್ತೇನೆ. ವೈಯಕ್ತಿಕತೆ ಎನ್ನುವುದು ಸಾರ್ವತ್ರಿಕತೆ ಎನ್ನುವುದರೊಂದಿಗಿನ ಯುದ್ಧದಲ್ಲಿ ಇಲ್ಲ.

ಪ್ರತಿಯೊಬ್ಬನೂ ತನ್ನ ಮಟ್ಟಿಗೆ ತಾನು ಹಾಗೂ ಎಲ್ಲರೂ ಒಗ್ಗೂಡಿ ಕೇಡುಗಳೊಂದಿಗೆ ಹೋರಾಡುವಂತೆ ಆಗಲಿ. ನಿಮ್ಮ ಬಳಿ ಎಂಟರ ಶಕ್ತಿ ಇದ್ದು, ನನ್ನ ಬಳಿ ನಾಲ್ಕರ ಶಕ್ತಿ ಇದ್ದು, ನೀವು ಬಂದು ನನ್ನನ್ನು ನಾಶ ಮಾಡಿದರೆ, ನಿಮಗೆ ನಾಲ್ಕರ ಶಕ್ತಿಯಾದರೂ ನಷ್ಟವಾದಂತಾಯಿತು. ಕೇಡಿನೊಂದಿಗೆ ಹೋರುವುದಕ್ಕೆ ನಿಮಗೆ ಉಳಿಯುವುದು ನಾಲ್ಕರ ಶಕ್ತಿ ಮಾತ್ರವೇ. ಪ್ರೇಮ ಮಾತ್ರವೇ ದ್ವೇಷವನ್ನು ಗೆಲ್ಲಬಲ್ಲದು. ದ್ವೇಷದಲ್ಲಿ ಶಕ್ತಿ ಇದೆ ಎನ್ನುವುದಾದರೆ, ಪ್ರೇಮದಲ್ಲಿ ಅಪರಿಮಿತವಾದ ಶಕ್ತಿ ಇದೆ.

\begin{center}
\textbf{ಹಿಂದೂ ಆಶಾವಾದಿ\supskpt{\footnote{\enginline{Ray and Wanda Ellis, "Swami Vivekananda in Washungton, D.C.", The Vesdanta Kesari, 1991, pp. 369-370}}}}
\end{center}

\begin{center}
(ವಾಷಿಂಗ್ ಟನ್ ಟೈಮ್ಸ್, ಸೋಮವಾರ, ೨ ನವೆಂಬರ್ ೧೮೯೪)
\end{center}

\begin{center}
\textbf{ವಿವೇಕಾನಂದರು ತೌಲನಿಕ ಧರ್ಮ ಮತ್ತು ಪುನರ್ಜನ್ಮದ ಮೇಲೆ ಮಾಡಿದ ಭಾಷಣ\supskpt{\footnote{2. ಸ್ವಾಮಿ ವಿವೇಕಾನಂದರು ಭಾನುವಾರ ೨೮ ಅಕ್ಟೋಬರ್ ೧೮೯೪ರಂದು ಪೀಪಲ್ಸ್ ಚರ್ಚ್ನಲ್ಲಿ ಮಾಡಿದ “ಕರ್ಮ ಮತ್ತು ಪುನರ್ಜನ್ಮ ಎಂಬ ಪದಶಃ ವರದಿ ಲಭ್ಯವಿಲ್ಲದ ಭಾಷಣ.}}}
\end{center}

ಕಳೆದ ರಾತ್ರಿ ಮೆಟ್ಸೆರಾಟ್ ಹಾಲ್ನಲ್ಲಿ ಸಾಧಾರಣ ಸಭೆಯನ್ನುದ್ದೇಶಿ ಮಾತನಾಡಿದ ಬ್ರಾಹ್ಮಣ ಸಂನ್ಯಾಸಿ ವಿವೇಕಾನಂದರ ಪ್ರಕಾರ, ಪಾಶ್ಚಾತ್ಯ ಧರ್ಮಗಳಿಂದ ಭಿನ್ನವಾದ ಆರ್ಯರ ಹಾಗೂ ಹಿಂದೂಗಳ ನಂಬಿಕೆಯ ಒಂದಂಶವೆಂದರೆ ಆಶಾವಾದ. ಅವರ ಭಾಷಣದ ವಿಷಯ ಪುನರ್ಜನ್ಮ ಎಂಬುದಾಗಿತ್ತು. ಭಾಷಣದ ಹೆಚ್ಚು ಭಾಗವನ್ನು ಹಿಂದೂ ಮತ್ತು ಕ್ರೈಸ್ತ ಸಿದ್ಧಾಂತಗಳ ತುಲನೆಗೆ ವಿನಿಯೋಗಿಸಿದರು.

ಪುನರ್ಜನ್ಮ ಸಿದ್ಧಾಂತವನ್ನು ಪ್ರತಿಪಾದನೆ ಮಾಡುವುದಕ್ಕೋಸ್ಕರ ಅವರು ಮಾನವ ಶರೀರವನ್ನು ಒಂದು ನದಿಗೆ ಹೋಲಿಸಿದರು. ಪ್ರತಿಯೊಂದು ಹನಿ ನೀರೂ ಮುಂದೆ ಹೋಗುತ್ತಿರುತ್ತದೆ; ಅದರ ಜಾಗಕ್ಕೆ ಇನ್ನೊಂದು ಹನಿ ಬರುತ್ತಿರುತ್ತದೆ. ನದಿಯಲ್ಲಿನ ಎಲ್ಲ ನೀರೂ ಕೆಲವೇ ಕ್ಷಣಗಳಲ್ಲಿ ಮುಂದೆ ಹೋಗಿರುತ್ತದೆಯಾದರೂ, ನಾವು ಅದೇ ನದಿ ಎಂದು ಕರೆಯುತ್ತೇವೆ. ಇದೇ ರೀತಿ ದೇಹದ ಕಣಗಳೆಲ್ಲವೂ ಹೊಸ ಕಣಗಳಿಂದ ಬದಲಾಗುತ್ತಲೇ ಇರುತ್ತವೆ; ಯಾವ ಎರಡು ದಿನಗಳೂ ನಮಗೆ ಅದೇ ಶರೀರ ಇರುವುದಿಲ್ಲ; ಆದರೂ ನಮ್ಮ ವಿಶಿಷ್ಟತೆಯನ್ನು ನಾವು ಕಾಪಾಡಿಕೊಂಡಿರುತ್ತೇವೆ.

ಹಾಗೆಯೇ ಚೇತನವೂ ಉಳಿದಿರುತ್ತದೆ ಎಂಬುದು ಹಿಂದೂಗಳ ನಂಬಿಕೆ. ಮನುಷ್ಯನು ಸಾವಿನಲ್ಲಿ ಇದ್ದಕ್ಕಿದ್ದಂತೆ ಬೇರೆಯೇ ಆದ ಆಘಾತದ ಬದಲಾವಣೆಯನ್ನು ಹೊಂದಬಹುದು; ತನ್ನ ಅಸ್ತಿತ್ವವನ್ನು ವಿಶ್ವದಲ್ಲಿ ಇನ್ನಾವುದಾದರೂ ಸ್ಥಳಕ್ಕೆ - ನಕ್ಷತ್ರಕ್ಕೋ, ಗ್ರಹಕ್ಕೋ - ಬದಲಾಯಿಸಬಹುದು; ಆ ನಂತರ ಮಾಂಸಲವಾದ ಬೇರೊಂದು ಬಗೆಯ ಶರೀರವನ್ನು ಪಡೆಯಬಹುದು.

ಪಾಪ ಎನ್ನುವುದರ ಬಗ್ಗೆ ಮಾತನ್ನಾಡುವುದೇ ಕೂಡದು ಎಂದವರು ಹೇಳಿದರು. ಕಳೆದು ಹೋದ ಕಾಲದ ತಪ್ಪುಗಳನ್ನು ಭವಿಷ್ಯದ ಬಗೆಗಿನ ಮಾರ್ಗದರ್ಶನಕ್ಕೆ ಬಳಸಬೇಕೇ ಹೊರತು, ಅವುಗಳ ಬಗ್ಗೆ ದುಃಖಿಸುತ್ತ ಕುಳಿತಿರುವುದಲ್ಲ. ಅವುಗಳಿಂದ ಪಾಠ ಕಲಿ ಯುತ್ತಲೂ, ಅವುಗಳನ್ನು ಮರೆತುಬಿಡಬೇಕು.

“ಜ್ಯೋತಿಯೊಂದನ್ನು ಹಚ್ಚಿಕೊಳ್ಳಿರಿ. ಕಗ್ಗತ್ತಲಿನಲ್ಲಿ ಕುಳಿತು ದುಃಖಿಸುತ್ತಿರಬೇಡಿ. ಯಾವಾಗಲೂ ಒಳಿತನ್ನು ಮಾಡುತ್ತ ಸುಖವಾಗಿರಿ” ಎಂದು ಅವರು ಹೇಳಿದರು.....

\begin{center}
\textbf{ವಿವೇಕಾನಂದರ ಉಪನ್ಯಾಸ\supskpt{\footnote{\enginline{1. New Discoveries, Vol. 2, p. 223}}}}
\end{center}

\begin{center}
(ಬಾಲ್ಟಿಮೋರ್ ನ್ಯೂಸ್, ೩ನವೆಂಬರ್ ೧೮೯೪)
\end{center}

ಕಳೆದ ರಾತ್ರಿ ಹ್ಯಾರಿಸ್ ಅಕಾಡೆಮಿ ಆಫ್ ಮ್ಯೂಸಿಕ್ ಕನ್ಸಾರ್ಟ್ ಹಾಲ್ನಲ್ಲಿ ಉನ್ನತ ಹಿಂದೂ ಭಗವದಾರಾಧಕರಾದ ಸ್ವಾಮಿ ವಿವೇಕಾನಂದರು ಉಪನ್ಯಾಸ ಮಾಡಿದರು. ಉಪನ್ಯಾಸದ ವಿಷಯ “ಭಾರತ ಮತ್ತು ಅದರ ಧರ್ಮ\footnote{2. ಈ ಉಪನ್ಯಾಸದ ಪದಶಃ ವರದಿ ಲಭ್ಯವಿಲ್ಲ.} ” ಎಂಬುದಾಗಿದ್ದಿತು. ಬ್ರಾಹ್ಮಣಧರ್ಮವಾದ ತಮ್ಮದನ್ನೂ ಸೇರಿಸಿ, ವಿವಿಧ ಪ್ರಾಚ್ಯ ಧರ್ಮಗಳ ನಂಬಿಕೆಗಳನ್ನು ಅವರು ವಿವರಿಸಿದರು. ಅಷ್ಟೊಂದು ಬೇರೆ ಬೇರೆ ಶ್ರದ್ಧೆಗಳನ್ನುಳ್ಳ ಪ್ರಚಾರಕರನ್ನು ಅಕ್ರೈಸ್ತ - ಅನಾಗರಿಕ ದೇಶಗಳಿಗೆ ಕಳುಹಿಸುವ ಯೋಜನೆಯನ್ನು ಅವರು ಲೇವಡಿ ಮಾಡಿದರು; ಪ್ರಚಾರ ಕೆಲಸದಲ್ಲಿ ನಿರತವಾಗಿರುವ ಬೇರೆ ಬೇರೆ ಧರ್ಮಗಳೆಲ್ಲ ಒಂದಾಗಬೇಕು ಎಂದು ಹೇಳಿದರು. ಮಿ. ವಿವೇಕಾನಂದರು ಹಿಂದೂಧರ್ಮವು ನಿರಾಶಾವಾದಿ ಧರ್ಮವಲ್ಲ, ಆಶಾವಾದಿ ಎಂಬುದನ್ನು ವಿವರಿಸಿದರು. ಅವರ ಮುಖ್ಯ ವಿಷಯ ಎಲ್ಲರೂ ಹಿಂದೆಯೂ ಇದ್ದರು, ಮುಂದೆಯೂ ಇನ್ನಿತರ ರೂಪಗಳನ್ನು ತಳೆದು ಬದುಕಿರುತ್ತಾರೆ ಎಂಬ ಅರ್ಥವುಳ್ಳ ಪುನರ್ಜನ್ಮ ಎಂಬುದಾಗಿತ್ತು. ಉಪನ್ಯಾಸದ ದ್ರವ್ಯಾರ್ಜನೆಯನ್ನು ಒಂದು ಅಂತರರಾಷ್ಟ್ರೀಯ ಕಾಲೇಜನ್ನು ಸ್ಥಾಪಿಸುವ ಕೆಲಸಕ್ಕೆ ಬಳಸಲಾಗುವುದು.

\begin{center}
\textbf{ಭಾರತವನ್ನು ಅದರಷ್ಟಕ್ಕೆ ಬಿಡಿ\supskpt{\footnote{\enginline{1. New Discoveries, Vol. 2, p. 340-341}}}}
\end{center}

\begin{center}
\textbf{(ಡೈಲಿ ಈಗಲ್, ೮ ಏಪ್ರಿಲ್ ೧೮೯೫)}
\end{center}

\textbf{ಆಗ ಅದು ಸರಿಯಾಗಿ ಹೊರಹೊಮ್ಮುವುದು, ಎನ್ನುತ್ತಾರೆ ಸ್ವಾಮಿ ವಿವೇಕಾನಂದರು}

ಕಳೆದ ರಾತ್ರಿ ಪೌಚ್ ಅರಮನೆಯಲ್ಲಿ ನೆರೆದ ಕಿಕ್ಕಿರಿದ ಜನಸಂದಣಿಯನ್ನುದ್ದೇಶಿಸಿ ಉಪನ್ಯಾಸ ಮಾಡಿದ ಭಾರತದ ಸ್ವಾಮಿ ವಿವೇಕಾನಂದರು ಇಂಗ್ಲಿಷ್ ಜನರನ್ನು ಗೋರಿಬಿಟ್ಟರು\footnote{2. ಬರಹ ರೂಪದಲ್ಲಿರುವ ಹೇಳಿಕೆಯೊಂದರಿಂದ ಸ್ವಾಮಿಗಳ ಬೋಧನೆಯ ಸಾರವನ್ನು ತೆಗೆದುಕೊಂಡಿರುವುದು ಎಂದು ತೋರುತ್ತದೆ.}. ಭಾರತವನ್ನು ನಾಗರಿಕವನ್ನಾಗಿ ಮಾಡುವುದಕ್ಕೆ ಇಂಗ್ಲಿಷ್ ಜನರು ಬಳಸುವ ಮೂರು ‘ಬಿ’ಗಳೆಂದರೆ - ಬೈಬಲ್, ಬ್ರಾಂಡಿ ಮತ್ತು ಬಯೊನೆಟ್ಗಳು - ಎಂದು ಅವರು ಹೇಳಿದರು. ತಮ್ಮ ಭದ್ರಕೋಟೆಯನ್ನು ನಿರ್ಮಾಣ ಮಾಡುವುದಕ್ಕೆ ಪ್ರಬೋಧಕರು ಬೈಬಲ್ನ್ನು ಧಾರಾಳವಾಗಿ ಬಳಸಿದರು. ಇಂಗ್ಲಿಷರು ತಮ್ಮ ಬರಹಗಳಲ್ಲಿ ಭಾರತದ ಸಾಮಾಜಿಕ ಪರಿಸ್ಥಿತಿಯನ್ನು ಉತ್ಪ್ರೇಕ್ಷೆ ಮಾಡಿರುವರು ಎಂದವರು ಹೇಳಿದರು. ಅವರು ತಮ್ಮ ಕಲ್ಪನೆಯನ್ನು ಮಾಡಿಕೊಂಡಿರುವುದು ಪರಯರಿಂದ; ಪರಯರೆಂದರೆ ಒಂದು ಬಗೆಯ ಮಾನವ ಝಾಡಮಾಲಿಗಳು. ಆತ್ಮಗೌರವವುಳ್ಳ ಯಾವ ಹಿಂದೂವೂ ಇಂಗ್ಲಿಷ್ ಜನರ ಸಹವಾಸ ಮಾಡಲಾರ ಎಂದವರು ಉದ್ಘೋಷಿಸಿದರು. ವಿಧವೆಯರು ಜಗನ್ನಾಥನ ತೇರಿನ ಗಾಲಿಗಳಡಿಯಲ್ಲಿ ತಾವಾಗಿ ಹೋಗಿ ಬೀಳುವ ಕಥೆಯೆಲ್ಲ ಸುಳ್ಳು ಎಂದೂ ಅವರು ಘೋಷಿಸಿದರು. ಜಾತಿ ಹಾಗೂ ಬಾಲ್ಯವಿವಾಹ ಕೆಟ್ಟ ಪದ್ಧತಿಗಳು ಎಂದು ಅವರು ಒಪ್ಪಿಕೊಂಡರು. ಜಾತಿಪದ್ಧತಿಯ ಮೂಲ (ಮಧ್ಯಯುಗೀನ) ಕಾರ್ಮಿಕರ ಸಂಘಗಳು ಎಂದವರು ಹೇಳಿದರು. ಭಾರತಕ್ಕೆ ಈಗ ಅಗತ್ಯವಾಗಿರುವುದೆಂದರೆ ಅದನ್ನು ಅದರಷ್ಟಕ್ಕೆ ಬಿಡುವುದು; ಆಗ ಅದು ತಂತಾನೆ ಸರಿಯಾಗಿ ಹೊರ ಹೊಮ್ಮುವುದು ಎಂದರು.

\textbf{ಅಬು ಬೆನ್ ಆದೆಮ್​ನ ಆದರ್ಶ\supskpt{\footnote{\enginline{1. New Discoveries, Vol. 3, pp. 316-18}
ಲೀ ಹಂಟ್ ಅವರ ಪ್ರಸಿದ್ಧ ಕವನದ ನಾಯಕ ಅಬು ಬೆನ್ ಆದೆಮ್​ ತಾನು ಸಹಮಾನವರನ್ನು ಪ್ರೀತಿಸುತ್ತೇನೆ ಎಂದು ಬರೆದುಕೊಳ್ಳುವಂತೆ ದೇವದೂತನೊಬ್ಬನನ್ನು ಕೇಳಿದನಂತೆ.}}}

\begin{center}
(ನ್ಯೂಯಾರ್ಕ್ ವರ್ಲ್ಡ್, ೮ ಡಿಸೆಂಬರ್ ೧೮೯೫)
\end{center}

\textbf{ಮುಂಬಯಿಯಿಂದ ಬಂದಿರುವ ಯೋಗಿ ಸ್ವಾಮಿ ವಿವೇಕಾನಂದರು ತಮ್ಮ ಸಹಮಾನವರಿಗೆ ಬೋಧಿಸುವುದು ಪ್ರೇಮವನ್ನು}

ನಿಚ್ಚಳವಾಗಿಯೂ ಅಮೆರಿಕನ್ ಆಗಿರುವ ಷರಾಯಿ ಮೇಲೆ ಇಳಿಬಿದ್ದಿರುವ ಹಿಂದೂ ಕಡುಕೆಂಪು ನಿಲುವಂಗಿಯಲ್ಲಿ ಅತ್ಯುಚ್ಚ ಪೌರ್ವಾತ್ಯ ಮಾದರಿಯ ತಪಸ್ವಿಯೊಬ್ಬರನ್ನು ಕಾಣುವುದು ಖಂಡಿತವಾಗಿಯೂ ಒಂದು ಅಚ್ಚರಿಯೇ. ಆದರೆ, ಉಡುಪು ಹೊರತಾಗಿ ಇನ್ನುಳಿದ ಸಂಗತಿಗಳಲ್ಲೂ ಸ್ವಾಮಿ ವಿವೇಕಾನಂದರು ನಿಬ್ಬೆರಗಾಗಿಸುತ್ತಾರೆ. ಮೊಟ್ಟ ಮೊದಲನೆಯದಾಗಿ, ಅವರು ನಿಮ್ಮ ಅಥವಾ ಯಾರೊಬ್ಬರ ಧರ್ಮ ತನ್ನ ಧರ್ಮ ದಷ್ಟೇ ಸಾಧುವಾದದ್ದು ಎಂದು ಉದ್ಘೋಷಿಸುತ್ತಾರೆ; ನೀವು ಕ್ರೈಸ್ತ, ಮುಸಲ್ಮಾನ, ಬ್ಯಾಪ್ಟಿಸ್ಟ್, ಬ್ರಾಹ್ಮಣ, ನಾಸ್ತಿಕ, ಅಜ್ಞೇಯತಾವಾದಿ, ಕ್ಯಾಥೊಲಿಕ್ ಅಥವಾ ಇನ್ನೇನು ಬೇಕಾದರೂ ಆಗಿರಿ, ಅದು ಅವರಿಗೆ ಯಾವ ವ್ಯತ್ಯಾಸವನ್ನೂ ಉಂಟುಮಾಡುವುದಿಲ್ಲ. ಅವರು ನಿಮ್ಮನ್ನು ಕೇಳುವುದು ನಿಮ್ಮ ಧರ್ಮದ ಬೆಳಕಿನಲ್ಲಿ ನೀವು ಋಜುತ್ವವುಳ್ಳವರಾಗಿರಿ ಎಂದು ಮಾತ್ರವಷ್ಟೇ.

ಆರಾಧನೆಯ ಹಾಗೂ ಉಡುಪಿನ ಬಗೆಗಿನ ತಮ್ಮದೇ ವಿಚಿತ್ರ ಕಲ್ಪನೆಗಳನ್ನುಳ್ಳ ಈ ಯೋಗಿ ಶುಕ್ರವಾರ ಬ್ರಿಟಾನಿಕ್ ಹಡಗಿನಲ್ಲಿ ಬಂದು ತಲುಪಿದರು. ಪಶ್ಚಿಮ ಮೂವ ತ್ತೊಂಭತ್ತನೆಯ ರಸ್ತೆಯ ನಂ. ೨೨೮ ನಿವಾಸಕ್ಕೆ ತೆರಳಿದರು. ನ್ಯೂಯಾರ್ಕ್ನಲ್ಲಿ ಇರುವಾಗ ಅವರು ಮನಶಾಸ್ತ್ರ ಹಾಗೂ ತತ್ತ್ವಮೀಮಾಂಸೆಯ ಬಗ್ಗೆ ಉಪನ್ಯಾಸ ಕೊಡುವರು; ಅಲ್ಲದೆ, ಮತಪಂಥ ಬೇರೆ ಎಂಬ ಕಾರಣಕ್ಕೆ ಯಾರೂ ಇನ್ನೊಬ್ಬರ ಕುತ್ತಿಗೆ ಹಿಸುಕಬೇಕಾ ಗಿಲ್ಲ ಎನ್ನುವ ತಮ್ಮ ವಿಶ್ವಧರ್ಮದ ಕಲ್ಪನೆಯನ್ನೂ ಸರಳಸಾಮಾನ್ಯವಾಗಿ ಪ್ರಚುರ ಗೊಳಿಸುವರು. “ಸಹಮಾನವರಿಗೆ ಸಹಾಯ ಮಾಡುವುದಷ್ಟನ್ನೇ ನಾನು ಬಯಸುವುದು” ಎಂದವರು ಹೇಳುತ್ತಾರೆ. ಅಲ್ಲದೆ, ಅವರೆನ್ನುತ್ತಾರೆ: “ನಾಲ್ಕು ಬಗೆಯ ಜನರು ಸಾಮಾನ್ಯವಾಗಿ ಕಾಣಸಿಗುತ್ತಾರೆ - ವಿಚಾರವಂತ, ಭಾವುಕ, ಮುಮುಕ್ಷು ಮತ್ತು ಕಾರ್ಯಕುಶಲಿ. ಅವರುಗಳಿಗೆ ಸರಿಹೊಂದುವ ಆರಾಧನಾಕ್ರಮವಿರಬೇಕು. ವಿಚಾರ ವಂತನು ಬಂದು ‘ಈ ರೀತಿಯ ಆರಾಧನೆಯನ್ನು ನಾನೊಪ್ಪುವುದಿಲ್ಲ. ನನಗಿಷ್ಟವಾಗುವ ತಾತ್ತ್ವಿಕವಾದ, ವೈಚಾರಿಕವಾದ ಏನನ್ನಾದರೂ ಕೊಡಿ’ ಎನ್ನುತ್ತಾನೆ. ಹೀಗೆ, ವಿಚಾರವಂತನಿಗೆ ವೈಚಾರಿಕ, ತಾತ್ತ್ವಿಕ ಆರಾಧನೆ.

“ಕಾರ್ಯಕುಶಲಿಯು ಬಂದು ‘ತಾತ್ತ್ವಿಕನ ಆರಾಧನೆಯನ್ನು ನಾನೊಪ್ಪುವುದಿಲ್ಲ. ನನ್ನ ಸಹಮಾನವರಿಗಾಗಿ ಮಾಡಬಹುದಾದಂತಹ ಕೆಲಸವನ್ನು ನನಗೆ ಕೊಡಿ’ ಎನ್ನುತ್ತಾನೆ. ಹೀಗೆ, ಅವನಿಗಾಗಿ ಒಂದು ಆರಾಧನಾಕ್ರಮವನ್ನು ಮಾಡಬೇಕು; ಅಂತೆಯೇ ಮುಮುಕ್ಷು ಹಾಗೂ ಭಾವುಕರುಗಳಿಗೆ. ಈ ಎಲ್ಲ ಜನರಿಗಾಗಿ ಇರುವ ಧರ್ಮದಲ್ಲಿ ಅವರುಗಳು ಒಪ್ಪು ವಂತಹ ಈ ಎಲ್ಲ ಅಂಶಗಳೂ ಇರಬೇಕು.”

ಅಷ್ಟರಲ್ಲಿ ಪ್ರಶ್ನೆಯೊಂದಕ್ಕೆ ಉತ್ತರವಾಗಿ ಸ್ವಾಮಿಗಳು ತುಂಬ ಮೃದುವಾಗಿ ಹೇಳಿದರು: “ಇಲ್ಲ, ಇಂದ್ರಜಾಲದಲ್ಲಿ ನನಗೆ ನಂಬಿಕೆಯಿಲ್ಲ. ಯಾವುದು ನಿಜವಲ್ಲವೋ ಅದು ಅಸ್ತಿತ್ವದಲ್ಲಿ ಇರುವುದಕ್ಕೆ ಸಾಧ್ಯವಿಲ್ಲ. ನಿಜವಲ್ಲದ್ದೆಂದಾದ ಮೇಲೆ ಅದು ಇರಲಾರದು. ಎಷ್ಟೋ ವಿಚಿತ್ರ ಸಂಗತಿಗಳು ಪ್ರಕೃತಿಯಲ್ಲೇ ಇವೆ. ಅವೆಲ್ಲವೂ ವೈಜ್ಞಾನಿಕ ವಿಷಯಗಳೆಂದು ನನಗೆ ಗೊತ್ತಿದೆ. ಎಂದ ಮೇಲೆ ಅವು ನನಗೆ ಇಂದ್ರಜಾಲದಂತೆ ಕಾಣುವುದಿಲ್ಲ. ಮಾಂತ್ರಿಕತೆಯ ಸಂಘಟನೆಗಳನ್ನು ನಾನು ನಂಬಲಾರೆ. ಅವು ಒಳ್ಳೆಯದನ್ನು ಮಾಡುವುದಿಲ್ಲ; ಎಂದಿಗೂ ಅವು ಒಳಿತನ್ನು ಮಾಡಲಾರವು.

ವಾಸ್ತವವಾಗಿ ಸ್ವಾಮಿಗಳು ಯಾವ ಸಂಘಟನೆಗಾಗಲಿ, ಮತಪಂಥಗಳಿಗಾಗಲಿ ಸೇರಿದವರಲ್ಲ. ಅವರ ಧರ್ಮವು ಎಲ್ಲ ವರ್ಗಗಳವರ ಎಲ್ಲ ಬಗೆಯ ಆರಾಧನೆಗಳನ್ನೂ ಎಲ್ಲ ಬಗೆಯ ನಂಬಿಕೆಗಳನ್ನೂ ಒಳಗೊಂಡಿರುವಂಥದು.

ಗಾಢ ಕಂದುಗಪ್ಪು ಮುಖಲಕ್ಷಣವಾದರೂ ಸುರೂಪಿ ತರುಣರಾದ ಸ್ವಾಮಿಗಳು ನೆನ್ನೆ ವಿಶಿಷ್ಟವಾದ ಶುದ್ಧ ಇಂಗ್ಲಿಷ್ನಲ್ಲಿ ತಮ್ಮ ಧರ್ಮವನ್ನು ವಿವರಿಸಿದರು. ಅವರು ಮಾತನಾಡುತ್ತಿದ್ದಾಗ - ಅಮೆರಿಕನ್ ಷರಾಯಿಯನ್ನು ಅರೆಬರೆಯಾಗಿ ಮುಚ್ಚು ತ್ತಿದ್ದ ಮುಂಬಯಿ ನಿಲುವಂಗಿ, ಅದರ ಮೇಲೆ ಸರಸವಾಡುತ್ತಿದ್ದ ಸಾಂಪ್ರದಾಯಿಕ ಮೇಲ್ವಸ್ತ್ರ - ಎಲ್ಲವೂ ಮರೆತುಹೋಗುತ್ತಿತ್ತು. ಬದಲಿಗೆ ಅವರ ಸಮ್ಮೋಹಕ ಮುಗುಳುನಗೆ ಹಾಗೂ ಅವರ ಆಳವಾದ, ದೇದೀಪ್ಯಮಾನವಾದ ಕಪ್ಪು ಕಣ್ಣುಗಳು ಮಾತ್ರವೇ ಢಳಾಯಿಸುತ್ತಿದ್ದವು.

ಸ್ವಾಮಿಗಳು ಪುನರ್ಜನ್ಮವನ್ನು ನಂಬುತ್ತಾರೆ. ದೇಹವು ಶುದ್ಧವಾಗುತ್ತ ನಡೆದಂತೆ ಜೀವವು ಮೇಲ್ತರದ ಸ್ಥಿತಿಗೇರುತ್ತದೆ; ಹೀಗೆ ದ್ರವ್ಯದ ಮಟ್ಟದ ಶುದ್ಧೀಕರಣ ಮುಂದುವರೆದಂತೆ ಜೀವವು ಮೇಲೇರುತ್ತ ಸಾಗುತ್ತದೆ, ದೇಹಾಂತರದ ಅಗತ್ಯವನ್ನೂ ಕೊನೆಗೆ ಮೀರಿ, ವಿಶ್ವಚೇತನದೊಂದಿಗೆ ಒಂದಾಗಿ ಸೇರಿಹೋಗುತ್ತದೆ ಎಂದು ಅವರ ನಂಬಿಕೆ.

ಈಗತಾನೆ ಈ ದೇಶಕ್ಕೆ ಬಂದಿರುವ (ಹರ್ಮನ್?) ಅಲ್ವಾರ್ಟ್ರಂತಹ ಯಹೂದಿ - ಹೊಂಚುಕ ಮನುಷ್ಯ ಅವರಿಗೆ ಅರ್ಥವಾಗಲಾರ.

“ತನ್ನ ಸಹಮಾನವನನ್ನು ದ್ವೇಷಿಸಬೇಕು ಎಂದು ಬೋಧಿಸುವುದಕ್ಕಾಗಿ ಆತ ಬಂದಿರು ವನು ಎಂದು ನೀವು ಹೇಳುತ್ತಿರುವಿರಿ. ಆತನ ಮನಸ್ಸು ದಾರಿತಪ್ಪಿಲ್ಲವಷ್ಟೆ? ಈ ದ್ವೇಷವನ್ನು ಹರಡುವುದಕ್ಕೆ ಆತನಿಗೆ ಅವಕಾಶ ಮಾಡಿಕೊಡಲಾಗುವುದೆ? ವೈದ್ಯರುಗಳು ಅವನ ಮೆದುಳನ್ನು ಪರೀಕ್ಷಿಸಿ ನೋಡಿ ತಪ್ಪೆಲ್ಲಿದೆ ಎಂಬುದನ್ನು ಕಂಡುಹಿಡಿಯುವುದೊಳಿತು” ಎಂದು ಅವರು ಹೇಳಿದರು.

ಈ ಯೋಗಿಯ ವಿಚಿತ್ರ ಹೆಸರು ಅಕ್ಷರಶಃ “ವಿವೇಚನೆಯ ಆನಂದ”ವನ್ನು ಸೂಚಿಸುವುದು. ಅವರು ಈ ದೇಶಕ್ಕೆ ಬಂದಿರುವ ಪ್ರಪ್ರಥಮ ಭಾರತೀಯ ಯೋಗಿ. ಅವರು ಬಂದಿರುವುದು ಮುಂಬಯಿಯಿಂದ.

\begin{center}
\textbf{ಸ್ವಾಮಿಗಳ ಸಿದ್ಧಾಂತ\supskpt{\footnote{\enginline{1. New Discoveries, Vol. 2, p. 340-341}}}}
\end{center}

\begin{center}
(ನ್ಯೂಯಾರ್ಕ್ ಹೆರಾಲ್ಡ್, ೧೯ ಜನವರಿ ೧೮೯೬)
\end{center}

ಈ ಕೆಳಗೆ ಕಾಣಿಸಿರುವುದು ಸ್ವಾಮಿಗಳ ಮುಖ್ಯವಾದ ಬೋಧನೆಗಳ ಸಂಕ್ಷಿಪ್ತ ಟಿಪ್ಪಣಿ:\footnote{2. ಬರಹ ರೂಪದಲ್ಲಿರುವ ಹೇಳಿಕೆಯಿಂದ ಆಯ್ದ ಸ್ವಾಮೀಜಿಯವರ ಉಪದೇಶದ ಸಾರಾಂಶ.}

ಪ್ರತಿಯೊಬ್ಬ ಮನುಷ್ಯನೂ ಅವನವನ ಸಹಜಪ್ರವೃತ್ತಿಗನುಗುಣವಾಗಿಯೇ ಮೇಲೇರಬೇಕು. ಹೇಗೆ ಪ್ರತಿಯೊಂದು ವಿಜ್ಞಾನವೂ ಅದರದ್ದೇ ವಿಧಾನವನ್ನು ಹೊಂದಿರುವುದೋ ಪ್ರತಿಯೊಂದು ಧರ್ಮವೂ ಹಾಗೆಯೇ. ನಮ್ಮ ಧರ್ಮದ ಕೊನೆಯನ್ನು ತಲುಪುವ ವಿಧಾನಗಳಿಗೆ ಯೋಗ ಎಂದು ಹೆಸರು; ಮಾನವರ ವಿವಿಧ ಸ್ವಭಾವಗಳಿಗೆ, ಮನಃ ಪರಿಪಾಕಗಳಿಗೆ ಹೊಂದುವಂತಹ ಬೇರೆ ಬೇರೆ ರೀತಿಯ ಯೋಗಗಳನ್ನು ನಾವು ಹೇಳಿಕೊಡುತ್ತೇವೆ. ನಾವು ಅವುಗಳನ್ನು ನಾಲ್ಕು ತಲೆಬರಹಗಳಡಿಯಲ್ಲಿ ಹೀಗೆ ವಿಂಗಡಿಸಬಹುದು:

(೧) ಕರ್ಮಯೋಗ - ತಾನು ಮಾಡುವ ಕೆಲಸ ಮತ್ತು ಕರ್ತವ್ಯ ಪರಿಪಾಲನೆಗಳ ಮೂಲಕ ತನ್ನ ದಿವ್ಯತೆಯನ್ನು ಒಬ್ಬ ಮನುಷ್ಯ ಸಾಕ್ಷಾತ್ಕಾರ ಮಾಡಿಕೊಳ್ಳುವ ವಿಧಾನ.

(೨) ಭಕ್ತಿಯೋಗ - ತನ್ನ ಇಷ್ಟದೇವರ ಮೇಲಿನ ಪ್ರೀತಿ ಹಾಗೂ ಭಕ್ತಿಗಳ ಮೂಲಕ ದಿವ್ಯತೆಯನ್ನು ಸಾಕ್ಷಾತ್ಕಾರ ಮಾಡಿಕೊಳ್ಳುವುದು.

(೩) ರಾಜಯೋಗ - ಮನಸ್ಸನ್ನು ನಿಯಂತ್ರಣ ಮಾಡಿಕೊಳ್ಳುವ ಮೂಲಕ ದಿವ್ಯತೆಯನ್ನು ಸಾಕ್ಷಾತ್ಕರಿಸಿಕೊಳ್ಳುವುದು.

(೪) ಜ್ಞಾನಯೋಗ - ಜ್ಞಾನದ ಮೂಲಕವಾಗಿ ಮನುಷ್ಯನು ತನ್ನ ದಿವ್ಯತೆಯನ್ನು ಸಾಕ್ಷಾತ್ಕಾರ ಮಾಡಿಕೊಳ್ಳುವುದು.

ಇವೆಲ್ಲ ಒಂದೇ ಕೇಂದ್ರಕ್ಕೆ - ಎಂದರೆ ದೇವರೆಡೆಗೆ -ಕೊಂಡೊಯ್ಯುವ ಬೇರೆ ಬೇರೆ ಪಥಗಳಿದ್ದ ಹಾಗೆ. ನಿಜವಾಗಿ ಹೇಳಬೇಕೆಂದರೆ, ಧಾರ್ಮಿಕ ನಂಬಿಕೆಗಳ ವೈವಿಧ್ಯತೆಯೇ ಒಂದು ಅನುಕೂಲ; ಏಕೆಂದರೆ ಮಾನವನ ಧಾರ್ಮಿಕ ಜೀವನವನ್ನು ಬೆಂಬಲಿಸುವ ಎಲ್ಲ ನಂಬಿಕೆಗಳೂ ಒಳ್ಳೆಯವೇ. ಮತಪಂಥಗಳು ಹೆಚ್ಚಾದಂತೆಲ್ಲ ಎಲ್ಲ ಮನುಷ್ಯರಲ್ಲಿ ಹುದುಗಿರುವ ದಿವ್ಯತೆಯೆಡೆಗೆ ಸಾಗಲೆತ್ನಿಸುವ ಪ್ರಯತ್ನಗಳು ಯಶಸ್ವಿಯಾಗುವ ಅವಕಾಶಗಳೂ ಹೆಚ್ಚಾಗುತ್ತವೆ.

\begin{center}
\textbf{“ವಿಶ್ವಧರ್ಮ”\supskpt{\footnote{New Discoveries, Vol. 3,pp. 475-79}}}
\end{center}

\begin{center}
\textbf{(ಲೋಕದ ಮತಸಿದ್ಧಾಂತಗಳ ಮೇಲೆ ವಿವೇಕಾನಂದರ ಉಪನ್ಯಾಸ}
\end{center}

\begin{center}
ಹಾರ್ಟ್ಫರ್ಡ್ ಡೈಲಿ ಟೈಮ್ಸ್, ೧ ಫೆಬ್ರವರಿ ೧೮೯೬)
\end{center}

ಕಳೆದ ರಾತ್ರಿ ತುಂಬಿದ ಸಭೆ ಹಿಂದೂ ಸಂನ್ಯಾಸಿ ವಿವೇಕಾನಂದರನ್ನು ಸಂತೋಷದಿಂದ ಬರಮಾಡಿಕೊಂಡಿತು..... ಅವರನ್ನು ಮಿ.ಸಿ.ಬಿ. ಪ್ಯಾಟರ್ಸನ್ ಅವರು ಯೋಗ್ಯವಾದ ಮಾತುಗಳಿಂದ ಪರಿಚಯಮಾಡಿದರು.... ಕಳೆದ ರಾತ್ರಿಯ ಅವರ ವಿಷಯ “ಆದರ್ಶ ಅಥವಾ ವಿಶ್ವಧರ್ಮ”\footnote{೩೧ ಜನವರಿ ೧೮೯೬ರಂದು ಮಾಡಿದ ‘ವಿಶ್ವಧರ್ಮದ ಆದರ್ಶ’ ಎಂಬ ಉಪನ್ಯಾಸದ ಪದಶಃ ವರದಿ ಲಭ್ಯವಿಲ್ಲ.} ಎಂಬುದಾಗಿತ್ತು.

ವಿಶ್ವದಾದ್ಯಂತ ಎರಡು ಬಲಗಳು ನಿರಂತರ ಕಾರ್ಯಪ್ರವೃತ್ತವಾಗಿರುವುದನ್ನು ನೋಡಬಹುದು - ಕೇಂದ್ರತ್ಯಾಗಿ ಹಾಗೂ ಕೇಂದ್ರಾಭಿಮುಖ, ಧನಾತ್ಮಕ ಹಾಗೂ ಋಣಾತ್ಮಕ, ಕ್ರಿಯೆ ಹಾಗೂ ಪ್ರತಿಕ್ರಿಯೆ, ಆಕರ್ಷಣೆ ಹಾಗೂ ವಿಕರ್ಷಣೆ, ಒಳಿತು - ಕೆಡುಕುಗಳನ್ನೂ ಪ್ರೀತಿ - ದ್ವೇಷಗಳನ್ನೂ ಸಹ ನಾವು ನೋಡುತ್ತೇವೆ. ಧರ್ಮದ ಪಾತಳಿಗಿಂತ, ಆಧ್ಯಾತ್ಮಿಕತೆಯ ಪಾತಳಿಗಿಂತ ಬೇರಾವ ಪಾತಳಿ ತಾನೇ ಶಕ್ತಿಯುತವಾಗಿದೆ? ಧರ್ಮದಿಂದ ಉದ್ಭವಿಸಿದ ದ್ವೇಷಕ್ಕಿಂತ ಬಲವಾದ ದ್ವೇಷವನ್ನಾಗಲಿ ಧರ್ಮದ ಕಾರಣವಾಗಿ ಜನಿಸಿದ ಪ್ರೀತಿಗಿಂತ ಗಾಢವಾದ ಪ್ರೀತಿಯನ್ನಾಗಲಿ ಲೋಕ ಕಂಡುದಿಲ್ಲ. ಬೇರಾವ ಬೋಧನೆಗಳೂ ಈ ಪ್ರಪಂಚಕ್ಕೆ ಹೆಚ್ಚಿನ ಸುಖವನ್ನಾಗಲಿ ದುಃಖವನ್ನಾಗಲಿ ಕೊಟ್ಟಿಲ್ಲ. ಬುದ್ಧನ ಸುಂದರ ಪ್ರಬೋಧನೆಗಳನ್ನು ಆತನ ಶಿಷ್ಯರುಗಳು ಇಪ್ಪತ್ತು ಸಾವಿರ ಅಡಿ ಎತ್ತರದ ಹಿಮಾಲಯದ ಮೇಲೆ ಸಾಗಿಸಿ ತಂದರು. ಐನೂರು ವರ್ಷಗಳ ನಂತರ ನಿಮ್ಮ ಸುಂದರ ಕ್ರಿಸ್ತನ ಬೋಧನೆಗಳು ಬಂದವು; ಇವು ಗಾಳಿಯ ರೆಕ್ಕೆಗಳ ಮೇಲೆ ಹಾರಿ ಹರಡಿದವು. ಇನ್ನೊಂದು ಕಡೆ, ನಿಮ್ಮೀ ಸುಂದರ ಭೂಮಿ ಧರ್ಮ ಹಾಗೂ ಅದರ ಪ್ರಚಾರದ ಕಾರಣವಾಗಿ ರಕ್ತಪಾತದಲ್ಲಿ ನೆನೆಯುತ್ತಿರುವುದನ್ನು ನೋಡಿ. ಮನುಷ್ಯನೊಬ್ಬನು ತನ್ನಂತೆಯೇ ನಂಬದ ಇನ್ನೊಬ್ಬನ ಹತ್ತಿರ ಬರುತ್ತಲೇ ಆತನ ಸ್ವರೂಪವೇ ಬದಲಾಗಿಬಿಡುವುದು. ಅವನು ತನ್ನ ಧರ್ಮಕ್ಕಾಗಿಯಲ್ಲ ಹೋರಾಡುವುದು, ತನ್ನ ಸ್ವಂತದ ಅಭಿಪ್ರಾಯಗಳಿಗಾಗಿ. ಆಗ ಅವನು ಮೂರ್ತೀಭವಿಸಿದ ಮತಾಂಧತೆಯಾಗಿ ಬಿಡುತ್ತಾನೆ, ಕ್ರೌರ್ಯವಾಗಿ ಬಿಡುತ್ತಾನೆ. ಅವನ ಧರ್ಮವೇನೋ ಸರಿಯಾದದ್ದೇ; ಆದರೆ ಅವನು ತನ್ನದೇ ಸ್ವಾರ್ಥದ ಅಭಿಪ್ರಾಯಗಳಿಗಾಗಿ ಯಾವಾಗ ಹೋರಾಡಲು ಪ್ರಾರಂಭಿಸುತ್ತಾನೆಯೋ ಆಗ ಅವನದೆಲ್ಲವೂ ತಪ್ಪೇ ಆಗುತ್ತದೆ. ಅರ್ಮೇನಿಯನ್ ಮತ್ತು ಟರ್ಕಿಶ್ ಕಗ್ಗೊಲೆಗಳನ್ನು ವಿರೋಧಿಸಿ ಜನರು ದಂಡೆತ್ತಿ ಹೊರ ಡಲು ಉಪಕ್ರಮಿಸುತ್ತಾರೆ; ಆದರೆ ಅಂಥವೇ ಕಗ್ಗೊಲೆಗಳು ಅವರ ಧರ್ಮದ ಹೆಸರಿನಲ್ಲಿ ನಡೆದಾಗ ಅವರ ಅಂತಃಪ್ರಜ್ಞೆ ಯಾವ ಮಾತನ್ನೂ ಆಡುವುದಿಲ್ಲ. ಮನುಷ್ಯನಲ್ಲಿ ನಾವು ದೇವ, ಮಾನವ ಮತ್ತು ದಾನವರುಗಳ ಒಂದು ಕುತೂಹಲಕಾರಿ ಮಿಶ್ರಣವನ್ನು ನೋಡುತ್ತೇವೆ; ಧರ್ಮ ಬೇರಾವುದಕ್ಕಿಂತ ಈ ದಾನವನನ್ನು ಬಡಿದೆಬ್ಬಿಸುತ್ತದೆ. ಎಲ್ಲರೂ ಒಟ್ಟಾಗಿ ಏಕಪ್ರಕಾರವಾಗಿ ಯೋಚಿಸುವಾಗ ನಮ್ಮ ಸ್ವಭಾವದಲ್ಲಿನ ದೇವ ಹೊರಬರುತ್ತಾನೆ; ಆದರೆ ಸ್ವಲ್ಪ ಅಭಿಪ್ರಾಯಭೇದ ಉಂಟಾಗಲಿ, ನೋಡುತ್ತಿರಿ ಬದಲಾವಣೆಯನ್ನು! ದಾನವನದೇ ರಾಜ್ಯಭಾರವಾಗಿಬಿಡುತ್ತದೆ. ಅನಾದಿಕಾಲದಿಂದಲೂ ಇದು ಹೀಗೇ ಆಗಿ ಬಂದಿರುವಂಥದು; ಮುಂದೆಯೂ ಹೀಗೇ ಇರುತ್ತದೆ. ಧರ್ಮಾಂಧತೆಯೆಂದರೇನೆಂಬುದು ಭಾರತದವರಾದ ನಮಗೆ ಗೊತ್ತು. ಏಕೆಂದರೆ ಸಾವಿರಾರು ವರ್ಷಗಳಿಂದ ಅದು ಪ್ರಚಾರಕರ ಕ್ಷೇತ್ರವೇ ಆಗಿಬಿಟ್ಟಿದೆ. ಆದರೆ, ಅಭಿಪ್ರಾಯಭೇದಕ್ಕಾಗಿ ತಿಕ್ಕಾಟ, ಧರ್ಮಗಳಿಗಾಗಿ ಹೋರಾಟ ಇವುಗಳನ್ನು ಮೀರಿ ಬರುವುದು ಶಾಂತಿಯ ದನಿ. ಮೂರು ಸಾವಿರ ವರ್ಷಗಳಿಂದ ವಿಭಿನ್ನ ಧರ್ಮಗಳನ್ನು ಸಾಮರಸ್ಯಕ್ಕೆ ತರುವ ಪ್ರಯತ್ನ ನಡೆದೇ ಇದೆ. ಆದರೆ ನಮಗೆ ಗೊತ್ತು ಈ ಎಲ್ಲ ಪ್ರಯತ್ನಗಳೂ ಹೇಗೆ ಸೋತವೆಂದು. ಈ ಪ್ರಯತ್ನಗಳು ಸೋಲುವುವೇ ಯಾವಾಗಲೂ, ಸೋಲತಕ್ಕವುಗಳೇ. ನಮ್ಮಲ್ಲಿ ಪ್ರೇಮದ ಬಗ್ಗೆ, ಶಾಂತಿಯ ಬಗ್ಗೆ, ವಿಶ್ವ ಭ್ರಾತೃತ್ವದ ಬಗ್ಗೆ ಕೆಲವು ಸಿದ್ಧ ಪದಗುಚ್ಛಗಳಿವೆ; ಅವು ಮೊದಲು ಹುಟ್ಟಿದ್ದು ಸದುದ್ದೇಶದಿಂದಲೇ ಎನ್ನುವುದು ನಿಜವಾದರೂ, ನಾವು ಅವನ್ನು ಗಿಣಿಗಳ ಹಾಗೆ ಪಾಠ ಒಪ್ಪಿಸುತ್ತೇವೆಯಾಗಿ ಅವುಗಳು ನಮಗೆ ಏನೂ ಅನ್ನಿಸುವುದೇ ಇಲ್ಲ. ಈ ಪ್ರಪಂಚಕ್ಕೊಂದು ವಿಶ್ವ ಮಟ್ಟದ ತತ್ತ್ವಜ್ಞಾನವಿದೆಯೆ? ಇನ್ನೂ ಇಲ್ಲ. ಪ್ರತಿಯೊಂದು ಧರ್ಮವೂ ತನ್ನದೇ ಆದ ಮತಪಂಥ ಸಿದ್ಧಾಂತಗಳನ್ನು ಹೊಂದಿದೆ; ಅವನ್ನು ಹರಡುವುದಕ್ಕಾಗಿ ಪ್ರಯತ್ನಿಸುತ್ತಿದೆ. ನೀವು ಇಡೀ ಪ್ರಪಂಚಕ್ಕೆ ಒಂದೇ ಧರ್ಮ ಇರುವ ಹಾಗೆ ಮಾಡಲಾರಿರಿ. ಅದು ಹಾಗೆ ಆಗಲೂ ಕೂಡದು. ಅರ್ಮೇನಿಯನ್ನರು ನೀವೆಲ್ಲರೂ ಅರ್ಮೇನಿಯನ್ನರಾಗಿಬಿಟ್ಟರೆ ಎಲ್ಲವೂ ಸರಿಹೋಗುತ್ತದೆ ಎನ್ನುತ್ತಾರೆ. ರೋಮ್​ ನಗರದ ಪೋಪರು ‘ಹೌದು, ಅದು ಬಹಳ ಸುಲಭ. ನೀವೆಲ್ಲರೂ ರೋಮನ್ ಕ್ಯಾಥೊಲಿಕರುಗಳಾಗಿಬಿಟ್ಟರೆ ಎಲ್ಲವೂ ಸರಿಯಾಗಿಬಿಡುತ್ತದೆ’ ಎನ್ನುತ್ತಾರೆ. ಅಂತೆಯೇ ಗ್ರೀಕ್ ಚರ್ಚ್ನವರು, ಅಂತೆಯೇ ಪ್ರಾಟೆಸ್ಪೆಂಟ್ ಚರ್ಚ್ನವರು, ಇನ್ನಿತರ ಎಲ್ಲರೂ. ಒಂದೇ ಧರ್ಮ ಉಳಿಯುವುದು ಎಂದಿಗೂ ಸಾಧ್ಯವಿಲ್ಲ; ಅದು ಉಳಿದೆಲ್ಲ ಧರ್ಮಗಳ ಸಾವಿಗೆ ಕಾರಣವಾಗುತ್ತದೆ. ಎಲ್ಲರೂ ಒಂದೇ ಚಿಂತನೆಯನ್ನು ಹೊಂದುವರಾದರೆ, ಆಗ ಚಿಂತಿಸುವುದಕ್ಕೆ ಯಾವ ಚಿಂತನೆಯೂ ಉಳಿದಿರುವುದಿಲ್ಲ. ಪ್ರತಿಯೊಬ್ಬನೂ ಒಂದೇ ತರಹ ಕಾಣಿಸುವುದಾದರೆ, ಎಂತಹ ಏಕತಾನ!ನೋಡುವುದಕ್ಕೆ ಒಂದೇ ತರಹ, ಯೋಚನೆಯೂ ಒಂದೇ ತರಹ - ಆಗ ನಾವೇನು ತಾನೆ ಮಾಡಬಲ್ಲೆವು, ಸುಮ್ಮನೆ ಕುಳಿತು ನಿರಾಶೆಯ ಬೇಸರ ತಾಳಲಾರದೆ ಸಾಯುವುದರ ಹೊರತು? ನೆಲ ಅಳಿಲುಗಳ ಒಂದು ಸಾಲಿನ ಹಾಗೆ ನಾವು ಬದುಕುವುದಕ್ಕಾಗುವುದಿಲ್ಲ; ವೈವಿಧ್ಯತೆ ಮನುಷ್ಯ ಜೀವನದ ವೈಶಿಷ್ಟ್ಯ. ಒಬ್ಬ ದೇವರು, ಒಂದು ಧರ್ಮ - ಇದು ಹಳೆಯ ಲಾವಣಿ ನಿಜ, ಆದರೆ ಅದರಲ್ಲಿ ಅಪಾಯವಿದೆ. ಆದರೆ ಅದು ಹಾಗಿರಲು ಸಾಧ್ಯವಿಲ್ಲ - ಇದಕ್ಕೆ ನಾವು ದೇವರಿಗೆ ಕೃತಜ್ಞರಾಗಿರಬೇಕು. ನೀವು ದೇಶದ ಕೋಶವನ್ನು ಹೊತ್ತುಕೊಂಡು ಹೊರಡುತ್ತೀರಿ, ಬಂದೂಕು ಫಿರಂಗಿ ಸಮೇತ, ನಿಮ್ಮ ಪ್ರಚಾರವನ್ನು ನುಗ್ಗಿಸುವುದಕ್ಕೆ. ಮೊದಲು ಸ್ವಲ್ಪ ಕಾಲ ಯಶಸ್ವಿಯೂ ಆಗುತ್ತೀರಿ ಎನ್ನೋಣ. ಹತ್ತುವರ್ಷಗಳಲ್ಲಿ ನಿಮ್ಮ ಐಕ್ಯವೆಂಬುದು ಮುರಿದುಬೀಳುತ್ತದೆ - ಅದಕ್ಕೇ ಅಷ್ಟೊಂದು ಮತಪಂಥಗಳು ಹುಟ್ಟಿಕೊಂಡಿರುವುದು. ವಿಶಾಲವಾಗಿ ಹರಡಿದ್ದ ಬೌದ್ಧಧರ್ಮವನ್ನೇ ತೆಗೆದುಕೊಳ್ಳೋಣ. ಅವರು ಲೋಕದ ಒಳಿತಿಗಾಗಿ ಸಹಾಯ ಮಾಡಲು ಪ್ರಯತ್ನಿಸುತ್ತಾರೆ. ಅನಂತರ ಕ್ರೈಸ್ತರು ಬರುತ್ತಾರೆ, (ಅದೆಷ್ಟೋ) ಒಳ್ಳೆಯ ಸಂಗತಿಗಳನ್ನು ಹೇಳಿಕೊಡುವುದಕ್ಕೆ. ಅವರಲ್ಲಿ, ಒಂದರಲ್ಲಿ ಮೂರು ದೇವರುಗಳಿವೆ, ಮೂವರಲ್ಲಿ ಒಬ್ಬ ದೇವನಿದ್ದಾನೆ; ಈ ಮೂವರಲ್ಲೊಬ್ಬನು ಲೋಕದ ಪಾಪಗಳನ್ನೆಲ್ಲ ಮೈಮೇಲೆ ಹೊತ್ತುಕೊಂಡ; ಆದರೆ ಅವನನ್ನು ಸಾಯಿಸಿದರು. ಅವನನ್ನು ಯಾರು ನಂಬುವುದಿಲ್ಲವೋ ಅವರು ಹೋಗುವುದು ಬೆಂಕಿ ಬೇಯುವ ಸ್ಥಳಕ್ಕೆ. ಇನ್ನು ಮಹಮ್ಮದ್, ಯಾರು ಅವನಲ್ಲಿ ನಂಬಿಕೆ ಇರಿಸುವುದಿಲ್ಲವೋ ಅವನ ಚರ್ಮವನ್ನು ಸೀದುಹೋಗುವ ಹಾಗೆ ಸುಡಲಾಗುತ್ತದೆ; ಹೊಸದಾಗಿ ಬೆಳೆಯುವ ಚರ್ಮವನ್ನೂ ಹಾಗೆಯೇ ಸುಡಲಾಗುತ್ತದೆ - ಅಲ್ಲಾ ಮಾತ್ರವೇ ಸರ್ವಶಕ್ತ ಎಂದು ಅವನಿಗೆ ಮನವರಿಕೆಯಾಗುವವರೆಗೆ. ಎಲ್ಲಾ ಧರ್ಮಗಳೂ ಮೂಲತಃ ಬಂದಿರುವುದು ಪ್ರಾಚ್ಯದಿಂದಲೇ. ಈ ಮಹಾ ಪ್ರಬೋಧಕರುಗಳು ಅಥವಾ ಅವತಾರಗಳು ಬೇರೆ ಬೇರೆ ರೂಪಗಳಲ್ಲಿ ಬಂದಿರುವರು. ಹಿಂದೂಗಳಿಗೆ ಹತ್ತು ಅವತಾರಗಳಿವೆ; ಮೊದಲನೆಯದು ಮೀನು ಇತ್ಯಾದಿ, ಐದರವರೆಗೆ; ಅಲ್ಲಿಂದ ಮುಂದಕ್ಕೆ ಎಲ್ಲರೂ ಮನುಷ್ಯರು. ಬೌದ್ಧರೆನ್ನುತ್ತಾರೆ, ‘ನಮಗೆ ಅಷ್ಟೊಂದು ಅವತಾರಗಳು ಬೇಡ, ಒಬ್ಬನೇ ಸಾಕು’ ಎಂದು. ಕ್ರೈಸ್ತರೆನ್ನುತ್ತಾರೆ, ‘ನಮಗೂ ಒಬ್ಬನೇ ಸಾಕು, ಆದರೆ ಅವನು ಈ ಕ್ರಿಸ್ತ’ ಎಂದು. ಅವನೊಬ್ಬನೇ ದೇವರು, ಅವನೊಬ್ಬನೇ ಅವತಾರ. ಆದರೆ ಬೌದ್ಧರು ಕಾಲದ ಆರಂಭವೇ ನಮ್ಮದು, ನಮ್ಮ ಮಹಾಗುರು ಐನೂರು ವರ್ಷ ಹಿಂದೆಯೇ ಇದ್ದವನು ಎನ್ನುತ್ತಾರೆ. ಪ್ರತಿ ಯೊಬ್ಬನೂ ತನ್ನದನ್ನೇ ಪ್ರೀತಿಸುತ್ತಾನೆ, ತಾಯಿ ತನ್ನ ಮಗುವನ್ನು ಪ್ರೀತಿಸುವಂತೆ. ಬೌದ್ಧನಿಗೆ ಬುದ್ಧನಲ್ಲಿ ಯಾವ ತಪ್ಪೂ ಕಾಣಿಸುವುದಿಲ್ಲ; ಕ್ರೈಸ್ತನಿಗೆ ಕ್ರಿಸ್ತನಲ್ಲಿ ಯಾವ ತಪ್ಪೂ ಕಾಣಿ ಸದು; ಮುಸ್ಲಿಮನಿಗೆ ಮಹಮ್ಮದನಲ್ಲಿ ಯಾವ ತಪ್ಪೂ ಕಾಣಿಸುವುದಿಲ್ಲ. ಕ್ರೈಸ್ತನೆನ್ನುತ್ತಾನೆ, ತನ್ನ ದೇವರು ಪಾರಿವಾಳದ ರೂಪದಲ್ಲಿ ಬಂದಿಳಿದಿರುವನು, ಅದೇನೂ ದಂತಕಥೆಯಲ್ಲ, ಚರಿತ್ರೆ ಎಂದು. ಹಿಂದೂವು ತನ್ನ ದೇವರು ಆಕಳಿನಲ್ಲಿ ಆವಿರ್ಭವಿಸಿರುತ್ತಾನೆ, ಇದು ಖಂಡಿತ ಮೂಢನಂಬಿಕೆಯಲ್ಲ, ಸತ್ಯ ಎಂದು ಹೇಳುತ್ತಾನೆ. ಪವಿತ್ರರಲ್ಲಿ ಪವಿತ್ರನಾದ ದೇವರನ್ನು ಅಕ್ಕಪಕ್ಕಗಳಲ್ಲಿ ಅಂಗ ರಕ್ಷಕ ದೇವದೂತರೊಡನೆ ಕರಂಡಕವೊಂದರಲ್ಲಿ ಹಾಕಿಡಬಹುದು ಎಂಬುದು ಯಹೂದಿಯ ಯೋಚನೆ. ಅವನ ಪಾಲಿಗೆ ಸುಂದರ ಸ್ತ್ರೀಯ ಅಥವಾ ಪುರುಷನ ರೂಪದಲ್ಲಿರುವ ಕ್ರೈಸ್ತನ ದೇವರು ಯೋಚಿಸಿಕೊಳ್ಳುವುದಕ್ಕೇ ಭಯಂಕರನಾದ ಮೂರ್ತಿ. ‘ಒಡೆದು ಹಾಕಿರಿ ಅದನ್ನು!’ ಎನ್ನುತ್ತಾರವರು. ಒಬ್ಬನ ಪ್ರವಾದಿಯು ಇಂತಿಂತಹ ಅಚ್ಚರಿಯ ಪವಾಡಗಳನ್ನು ಮಾಡಿರುವನು ಎಂದಾದರೆ, ಇನ್ನಿತರರಿಗೆ ಅದೆಲ್ಲಾ ಮೂಢನಂಬಿಕೆ. ಎಂದಮೇಲೆ, ಎಲ್ಲಿದೆ ನಿಮ್ಮ ಐಕ್ಯ? ಇನ್ನು ನಿಮ್ಮ ಕ್ರಿಯಾಕರ್ಮಗಳು ಬೇರೆ ಇವೆ. ನಾನು ನನ್ನ ದಿರಿಸನ್ನು ಹಾಕಿಕೊಳ್ಳುವ ಹಾಗೆಯೇ ರೋಮನ್ ಕ್ಯಾಥೊಲಿಕ್ ತನ್ನ ದಿರಿಸನ್ನು ಧರಿಸುತ್ತಾನೆ. ಅವನ ಗಂಟೆಗಳು, ಮೇಣದ ಬತ್ತಿಗಳು, ಪವಿತ್ರಜಲ ಎಲ್ಲವೂ ಒಳ್ಳೆಯವು, ಆವಶ್ಯಕ, ಆದರೆ ನೀವು ಮಾಡುತ್ತಿರುವುದೆಲ್ಲ ಮೂಢನಂಬಿಕೆಯಲ್ಲದೆ ಇನ್ನೇನೂ ಅಲ್ಲ ಎಂದು ಅವನು ಹೇಳುತ್ತಾನೆ. ನಾವು ಇದನ್ನೆಲ್ಲ ಏರುಪೇರುಗೊಳಿಸಿ ಒಂದೇ ಧರ್ಮವನ್ನು ಇಟ್ಟುಕೊಳ್ಳುವುದಕ್ಕಾಗುವುದಿಲ್ಲ; ಏಕೆಂದರೆ ಚಿಂತನೆಯ ಜೀವವಿರುವುದೇ ಚಿಂತನೆಯ ವೈವಿಧ್ಯತೆಯಲ್ಲಿ. ನಾವು ಚಿಂತಿಸುವುದರ ವಿಲೋಮವೈರುಧ್ಯಗಳನ್ನು ಚಿಂತಿಸುವವರನ್ನು ಪ್ರೀತಿಸಲು ನಾವು ಕಲಿಯಬೇಕು. ನಮಗೆಲ್ಲರಿಗೂ ಹಿನ್ನೆಲೆಯಾಗಿ ಮಾನವೀಯತೆಯಿದೆ; ಆದರೆ ಪ್ರತಿಯೊಬ್ಬನಿಗೂ ತನ್ನ ವೈಯಕ್ತಿಕತೆ, ತನ್ನದೇ ವಿಶಿಷ್ಟ ಚಿಂತನೆ ಇರಬೇಕು. ಮತಪಂಥಗಳನ್ನು ಇನ್ನೂ ಮುಂದು ಮುಂದಕ್ಕೆ ಹೆಚ್ಚಿಸುತ್ತಾ ನಡೆಯಿರಿ - ಪ್ರತಿಯೊಬ್ಬ ಸ್ತ್ರೀಯೂ ಪುರುಷನೂ ಸಹ ಒಂದೊಂದು ಮತದವರಾಗುವವರೆಗೆ. ನಮ್ಮ ಅಭಿಪ್ರಾಯಕ್ಕೆ ಭಿನ್ನವಾದವರನ್ನು ಪ್ರೀತಿಸಲು ನಾವು ಕಲಿಯಬೇಕು. ಭಿನ್ನತೆಯೇ ಚಿಂತನೆಯ ಜೀವಾಳ ಎಂಬುದನ್ನು ತಿಳಿದುಕೊಳ್ಳಬೇಕು. ನಮ್ಮೆಲ್ಲರಿಗೂ ಸಾಮಾನ್ಯವಾದ ಧ್ಯೇಯವೆಂದರೆ ಮಾನವ ಜೀವದ ಪರಿಪೂರ್ಣತೆ; ಅದೇ ನಮ್ಮಲ್ಲಿರುವ ದೇವರು. ಧರ್ಮ ಎನ್ನುವುದು ಮಾನವನಲ್ಲಿ ಹುದುಗಿರುವ ದೇವ ರನ್ನು ಆವಿರ್ಭವಿಸುವಂತೆ ಮಾಡುವುದಕ್ಕೆ ಇರುವ ಮಹತ್ತರವಾದ ಬಲ. ಆದರೆ ಈ ಅನಾವರಣವನ್ನು ನಮ್ಮದೇ ಆದ ರೀತಿಯಲ್ಲಿ ಮಾಡಬೇಕು. ನಾವೆಲ್ಲರೂ ಒಂದೇ ರೀತಿಯ ಆಹಾರವನ್ನು ಜೀರ್ಣಿಸಿಕೊಳ್ಳಲಾರೆವು. ನಿಮ್ಮ ಆಶೋತ್ತರಗಳು ಅತ್ಯುನ್ನತವಾಗಿರಲಿ, ಆಗ ನಿಮ್ಮ ಸ್ಫೂರ್ತಿಚೇತನವು ವಿವೇಚನೆಯೊಂದಿಗೆ ಮತ್ತು ಜ್ಞಾತ ನಿಯಮಗಳೊಂದಿಗೆ ಸಾಮರಸ್ಯದಿಂದಿರುವುದು; ದೇವರು ಯಾವಾಗಲೂ ನಿಮ್ಮೊಂದಿಗಿರುವನು.

\begin{center}
\textbf{ವಿವೇಕಾನಂದರ ತತ್ತ್ವಜ್ಞಾನ\supskpt{\footnote{\enginline{1. New Discoveries, Vol. 4, p.20}}}}
\end{center}

\begin{center}
(ಟ್ರಿಬ್ಯೂನ್, ೫ ಮಾರ್ಚ್ ೧೮೯೬)
\end{center}

\begin{center}
\textbf{ಅವರು ಅನೇಕ ಬಗೆಯ ಧರ್ಮಗಳನ್ನು ಇಟ್ಟುಕೊಂಡಿರಬಲ್ಲರು}
\end{center}

ಕಳೆದ ರಾತ್ರಿ ರಿಕೆಲಿಯು ಹೋಟೆಲಿನಲ್ಲಿ ಹಿಂದೂ ಪ್ರಚಾರಕ ವಿವೇಕಾನಂದರು ಉಪನ್ಯಾಸ\footnote{2. ‘ವಿಶ್ವ ಧರ್ಮದ ಆದರ್ಶ’ ಈ ತರಗತಿಯ ಪದಶಃ ವರದಿ ಲಭ್ಯವಿಲ್ಲ.} ಮಾಡಿದರು. ಈ ಖಾಸಗಿ ಹೋಟೆಲಿನ ಪ್ರಾಂಗಣಗಳೆಲ್ಲ ಮಹಿಳೆಯರಿಂದ ತುಂಬಿ ತುಳುಕುತ್ತಿದ್ದವು. ವಿವೇಕಾನಂದರು ಹೋಟೆಲಿಗೆ ಆಗಮಿಸಿದಾಗ ಅವರು ಬಹು ಕಷ್ಟಪಟ್ಟು ಹಾದಿ ಮಾಡಿಕೊಂಡು ಒಳಕ್ಕೆ ಹೋಗಬೇಕಾಯಿತು. ಮಹಡಿಯ ಮೇಲಕ್ಕೆ ಹತ್ತಿಹೋದ ಅವರು ಕೆಂಪು ಬಣ್ಣದ ನಿಲುವಂಗಿ ಮತ್ತು ಕೆಂಪು ಬಣ್ಣದ ಸೊಂಟ ಪಟ್ಟಿಯ ಉಡುಪಿನಲ್ಲಿ ಬೇಗನೆ ಕೆಳಗಿಳಿದು ಬಂದರು.

ತಮ್ಮ ಭಾಷಣದಲ್ಲಿ ವಿವೇಕಾನಂದರು ಅನೇಕ ಧರ್ಮಗಳಿವೆ. ಪ್ರತಿಯೊಬ್ಬ ಶ್ರದ್ಧಾವಂತನೂ ತನ್ನ ಧರ್ಮವೊಂದೇ ನಿಜವಾದ ಧರ್ಮ ಎಂದುಕೊಳ್ಳುತ್ತಾನೆ ಎಂದರು. ಎಲ್ಲರಿಗೂ ಒಂದೇ ಧರ್ಮವಿರಬೇಕು ಎಂದು ಯೋಚಿಸುವುದು ತಪ್ಪು ಎಂದರು. ನಂತರ ಅವರೆಂದರು:

“ಎಲ್ಲರೂ ಒಂದೇ ಧಾರ್ಮಿಕ ಅಭಿಪ್ರಾಯವುಳ್ಳವರಾದರೆ ಧರ್ಮ ಎಂಬುದೇ ಇರುವುದಿಲ್ಲ. ಒಂದು ಧರ್ಮವು ಪ್ರಾರಂಭವಾಗುತ್ತಿದ್ದ ಹಾಗೆಯೇ ಅನೇಕ ಭಾಗಗಳಾಗಿ ಒಡೆಯುತ್ತದೆ. ಧರ್ಮವು ಹೀಗೆ ಒಡೆಯುವ ಕ್ರಿಯೆ ಸಾಗುತ್ತ ನಡೆದು, ಕೊನೆಗೆ ಪ್ರತಿ ಯೊಬ್ಬನೂ ತನ್ನ ಚಿಂತನೆಯನ್ನು ತಾನೇ ಸ್ವಂತವಾಗಿ ಮಾಡಿಕೊಂಡು ತನಗಾಗಿ ತನ್ನದೇ ಧರ್ಮವನ್ನು ಕಂಡುಕೊಳ್ಳುವವರೆಗೂ ಮುಂದುವರೆಯುತ್ತದೆ.”

ವಿವೇಕಾನಂದರು ಡೆಟ್ರಾಯಿಟ್ನಲ್ಲಿ ಸುಮಾರು ಎರಡು ವಾರಗಳ ಕಾಲ ಉಳಿದಿರುತ್ತಾರೆ ಮತ್ತು ಪ್ರತಿದಿನ ಬೆಳಗ್ಗೆ ಹನ್ನೊಂದು ಗಂಟೆಗೆ ಮತ್ತು ಸಂಜೆ ಎಂಟು ಗಂಟೆಗೆ ಹೋಟೆಲಿನಲ್ಲಿ ತರಗತಿಗಳನ್ನು ತೆಗೆದುಕೊಳ್ಳುತ್ತಾರೆ....

\begin{center}
\textbf{ಸ್ವಾಮಿಗಳ ಮಾತನ್ನು ಕೇಳಿದೆವು\supskpt{\footnote{\enginline{1. New Discoveries, Vol. 4, p.41}}}}
\end{center}

\begin{center}
(ನ್ಯೂಸ್ ಟ್ರಿಬ್ಯೂನ್, ೧೬ ಮಾರ್ಚ್ ೧೮೯೬)
\end{center}

\begin{center}
\textbf{ಬೆತ್ ಎಲ್ ಟೆಂಪಲ್ನಲ್ಲಿ ವಿವೇಕಾನಂದರು ಉಪನ್ಯಾಸ ಮಾಡಿದರು}
\end{center}

\begin{myquote}
ವಿಶ್ವಧರ್ಮದ ಆದರ್ಶವೊಂದರ ಮೇಲೆ ಮಾತನಾಡಿದರು\\ಬಹುಶಃ ಮಂಗಳವಾರ ಅವರು ಹೊರಡಲಿರುವರು
\end{myquote}

ಕಳೆದ ರಾತ್ರಿ ಸ್ವಾಮಿ ವಿವೇಕಾನಂದರು ತಮ್ಮ ‘ವಿಶ್ವಧರ್ಮದ ಆದರ್ಶ’ವೆಂಬ ಉಪನ್ಯಾಸ\footnote{2. ಇದರ ಪದಶಃ ವರದಿ ಲಭ್ಯವಿಲ್ಲ.} ಕೊಟ್ಟಾಗ ಟೆಂಪಲ್ ಬೆತ್ ಎಲ್ನಲ್ಲಿ ಜನಗಳ ಗುಂಪು ಬಾಗಿಲವರೆಗೂ ದಟ್ಟೈಸಿತ್ತು. ಕಾರ್ಯಕ್ರಮ ಎಂಟು ಗಂಟೆಗೆ ಎಂದು ಘೋಷಿಸಿದ್ದರೂ, ಸಂಜೆ ಬಹು ಮುಂಚೆಯೇ ಜನದಟ್ಟಣೆ ಬಹಳವಾದ್ದರಿಂದ ದೇಗುಲದ ಬಾಗಿಲುಗಳನ್ನು ಆರು ಗಂಟೆ ಇಪ್ಪತ್ತೈದು ನಿಮಿಷಕ್ಕೇ ತೆರೆಯಬೇಕಾಯಿತು. ಏಳು ಗಂಟೆಗೆ ಬಾಗಿಲುಗಳನ್ನು ಮುಚ್ಚಿದ್ದರಿಂದ ಆ ನಂತರ ಬಂದ ನೂರಾರು ಜನರನ್ನು ಹಾಗೆಯೇ ವಾಪಸು ಕಳುಹಿಸುವಂತಾಯಿತು. ಸ್ವಾಮಿಗಳು ಇಂತೆಂದರು:

“ನಾವೆಲ್ಲರೂ ವಿಶ್ವಭ್ರಾತೃತ್ವದ ಬಗ್ಗೆ ಕೇಳುತ್ತಿರುತ್ತೇವೆ, ಹೇಗೆ ಸಂಘಟನೆಗಳು ಎದ್ದು ನಿಂತು ಇದನ್ನು ಬೋಧಿಸಲಪೇಕ್ಷಿಸುತ್ತವೆ ಎಂಬುದನ್ನು ನೋಡುತ್ತೇವೆ. ಆದರೆ ಇದರ ಪರಿಣಾಮ ಎಲ್ಲಿಯವರೆಗೆ? ನೀವು ಒಂದು ಮತಪಂಥವನ್ನು ಮಾಡಿಕೊಳ್ಳುತ್ತಲೇ, ಸಮಾನತೆಯನ್ನು ವಿರೋಧಿಸುತ್ತೀರಿ, ಹಾಗಾಗಿ ಇದು ಉಳಿಯುವುದಿಲ್ಲ.

“ವೈವಿಧ್ಯತೆಯಲ್ಲಿ ಏಕತೆ ಎನ್ನುವುದೇ ವಿಶ್ವದಲ್ಲಿರುವ ಯೋಜನೆ. ನಾವೆಲ್ಲರೂ ಮನುಷ್ಯರಾಗಿರುವಂತೆಯೇ, ನಾವೆಲ್ಲರೂ ಬೇರೆ ಬೇರೆಯೂ ಹೌದು. ಹಾಗಾಗಿ, ವಿಶ್ವಧರ್ಮದ ಕಲ್ಪನೆ ಎಂದರೆ ಸಿದ್ಧಾಂತಗಳ ಸಂಚಯವೊಂದನ್ನು ಮಾನವಜನಾಂಗವೆಲ್ಲಾ ನಂಬಬೇಕು ಎಂದು ಅರ್ಥೈಸುವುದಾದರೆ, ಅದು ಅಸಾಧ್ಯ; ಅದು ಎಂದೆಂದಿಗೂ ಸಾಧ್ಯವಾಗಲಾರದು ಎಂಬುದನ್ನು ನಾವು ಮನಗಾಣಬೇಕು. ಎಲ್ಲರ ಮುಖವೂ ಒಂದೇ ರೀತಿ ಆಗಬಹುದಾದ ಕಾಲ ಬರಬಹುದು ಎಂದು ಊಹಿಸಿದಂತೆ ಇದು. ನಾವೆಲ್ಲರೂ ಒಂದೇ ರೀತಿ ಯೋಚಿಸಬೇಕು - ಮ್ಯೂಸಿಯಂನಲ್ಲಿರುವ ಈಜಿಪ್ಷಿಯನ್ ಮಮ್ಮಿಗಳ ಹಾಗೆ, ಯೋಚಿಸಲು ಯೋಚನೆಗಳೇ ಇಲ್ಲದೆ ಒಂದನ್ನೊಂದು ನೋಡುತ್ತಿರುವ ಹಾಗೆ - ಎನ್ನುವುದನ್ನು ನಾವು ನಿರೀಕ್ಷಿಸಬಾರದು. ಚಿಂತನೆಯ ಈ ವ್ಯತ್ಯಾಸವೇ, ಈ ವೈವಿಧ್ಯವೇ, ಯೋಚನೆಯ ಸಮತೋಲವನ್ನು ಕಳೆದುಕೊಳ್ಳುತ್ತಿರುವುದೇ, ನಮ್ಮ ಮುನ್ನಡೆಯ ಜೀವಾಳ, ಚಿಂತನೆಯ ಜೀವಾಳ.”

ಸ್ವಾಮಿಗಳು ಬಹುಶಃ ಮಂಗಳವಾರ (ಮಾರ್ಚ್ ೧೭ರಂದು) ಹೊರಡುತ್ತಾರೆ. ಕಳೆದ ರಾತ್ರಿ ಉಪನ್ಯಾಸವನ್ನು ಮುಗಿಸುವ ಮೊದಲು ಅವರು ತಮಗೆ ಹಾಗೂ ತಮ್ಮ ತತ್ತ್ವ ಜ್ಞಾನಕ್ಕೆ ಇತ್ತಿ ಸ್ವಾಗತಕ್ಕಾಗಿ ಡೆಟ್ರಾಯಿಟ್ ಜನತೆಯನ್ನು ವಂದಿಸಿದರು.

\begin{center}
\textbf{ಮುಕ್ತತೆಯ ತತ್ತ್ವ\supskpt{\footnote{\enginline{1. New Discoveries, Vol. 4, p.56-58}}}}
\end{center}

\begin{center}
(ಬೋಸ್ಟನ್ ಈವಿನಿಂಗ್ ಟ್ರಾನ್ಸ್ಕ್ರಿಪ್ಟ್, ೨೧ ಮಾರ್ಚ್, ೧೮೯೬)
\end{center}

\begin{center}
\textbf{ಸ್ವಾಮಿ ವಿವೇಕಾನಂದರು ಹಿಂದೂ ವಿವೇಚನಾತ್ಮಕ ಬೋಧನೆಯನ್ನೂ ಪಾಶ್ಚಾತ್ಯ ಧರ್ಮಗಳನ್ನೂ ತುಲನೆಮಾಡಿದರು}
\end{center}

ವಿಶ್ವ ಸರ್ವಧರ್ಮ ಸಮ್ಮೇಳದನಲ್ಲಿ ಹಿಂದೂ ಪ್ರತಿನಿಧಿ ಎಂದು ಸ್ಮರಣೆಯಲ್ಲಿ ಉಳಿಯುವ ಸ್ವಾಮಿ ವಿವೇಕಾನಂದರು ಪ್ರೋಕೋಪಿಯ, ೪೫ ಸಂತ ಬೋಟೋಲ್ಙ ರಸ್ತೆಯಲ್ಲಿಯ ಮಾರ್ಚ್ ತರಗತಿಗಳಿಗೆ ಉಪನ್ಯಾಸಕರಾಗಿ ನಗರದಲ್ಲಿರುವರು\footnote{2. ಈ ತರಗತಿಯ ಪದಶಃ ವರದಿ ಲಭ್ಯವಿಲ್ಲ.}. ನ್ಯೂಯಾರ್ಕ್ನಲ್ಲಿ ವ್ಯವಸ್ಥಿತವಾಗಿ ತರಗತಿ ಉಪನ್ಯಾಸಗಳನ್ನು ನಡೆಸುವುದರಲ್ಲಿ ಯಶಸ್ವಿಯಾಗಿ ಅತ್ಯಂತ ಮೌಲಿಕವಾದ ಕೆಲಸವನ್ನು ಸ್ವಾಮಿಗಳು ಮಾಡುತ್ತಿರುವರು. ಅವರ ತರಗತಿಗಳಿಗೆ ಬರುವ ವರ ಸಂಖ್ಯೆ ಕಳೆದ ಎರಡು ಚಳಿಗಾಲಗಳಲ್ಲಿ ಹೆಚ್ಚುತ್ತಲೇ ಇದ್ದಿತು. ಅವರೀಗ ಬೋಸ್ಟನ್ ನಗರಕ್ಕೆ ಅತ್ಯಂತ ಸಮಯೋಚಿತವಾಗಿ ಬಂದಿರುವರು.

ಸ್ವಾಮಿಗಳು ತಮ್ಮ ಕೆಲಸದ ಬಗ್ಗೆ ಈ ಮುಂದೆ ಕಾಣಿಸಿದ ವಿವರಣೆಯನ್ನು ಕೊಟ್ಟಿರುವರು. ‘ಸಂನ್ಯಾಸಿ’ ಎಂಬ ಪದವನ್ನು ವಿವರಿಸುತ್ತ ಅವರೆಂದರು (ನೋಡಿ, ‘ಸಂನ್ಯಾಸಿ’, ಕೃತಿ ಶ್ರೇಣಿ, ೭, ೩೪೫).

ಸ್ವಾಮಿಗಳು ತಾವು ಮಾಡುತ್ತಿರುವ ಕೆಲಸ ಮತ್ತು ಅದರ ವಿಧಾನದ ಬಗ್ಗೆ ನಮಗೆ ಒಂದು ಕಲ್ಪನೆಯನ್ನು ಮಾಡಿಕೊಡುತ್ತ, ತಮಗೆ ಚಿಕ್ಕಂದಿನಿಂದಲೂ ಧರ್ಮ ಮತ್ತು ತತ್ತ್ವಜ್ಞಾನ ಎಂದರೆ ಬಹಳ ಆಸಕ್ತಿಯಿದ್ದುದರಿಂದ ತಾವು ಪ್ರಪಂಚವನ್ನು ಪರಿತ್ಯಾಗ ಮಾಡಿರುವರೆಂದೂ, ಭಾರತೀಯ ಶಾಸ್ತ್ರಗ್ರಂಥಗಳು ತ್ಯಾಗವೇ ಮನುಷ್ಯನೊಬ್ಬನ ಅತ್ಯುನ್ನತ ಆದರ್ಶ ಎಂದು ಬೋಧಿಸುತ್ತವೆ ಎಂದೂ ಹೇಳಿದರು. ಸ್ವಾಮಿಗಳ ಬೋಧನೆಯನ್ನು ಅವರ ಮಾತಿನಲ್ಲೇ ಹೇಳುವುದಾದರೆ, “ನನ್ನ ಗುರುಗಳು (ಒಬ್ಬ ಹೆಸರಾಂತ ಹಿಂದೂ ಸಂತ) ತೋರಿದ ಬೆಳಕಿನಲ್ಲಿ ನಮ್ಮ ಪುರಾತನ ಶಾಸ್ತ್ರಗಳ ಬಗ್ಗೆ ನಾನು ವ್ಯಾಖ್ಯಾನ ಮಾಡುತ್ತಿರುವೆ. ಪ್ರಕೃತ್ಯತೀತವೆನ್ನ ಬಹುದಾದ ಯಾವುದೇ ಅಧಿಕೃತತೆ ನನಗಿದೆ ಎಂದು ನಾನು ಹೇಳುವುದಿಲ್ಲ. ನನ್ನ ಬೋಧನೆಗಳಲ್ಲಿ ಯಾವುದನ್ನಾದರೂ ಅತ್ಯುನ್ನತ ಚಿಂತಕ ಮನಸ್ಸುಗಳು ಸ್ವೀಕಾರಾರ್ಹವೆಂದು ತೆಗೆದುಕೊಂಡರೆ, ಅಂತಹ ಸ್ವೀಕಾರವನ್ನೇ ನಾನು ದೊಡ್ಡ ಪ್ರತಿಫಲವೆಂದುಕೊಳ್ಳುತ್ತೇನೆ. ಎಲ್ಲ ಧರ್ಮಗಳೂ ತಮ್ಮ ಬೋಧನೆಯಲ್ಲಿ ಭಕ್ತಿ, ಜ್ಞಾನ, ಕಾರ್ಯಕ್ಷಮತೆಗಳನ್ನು ಮೂರ್ತಿಮತ್ತಾದ ರೂಪದಲ್ಲಿ ಧ್ಯೇಯವಾಗಿರಿಸಿಕೊಂಡಿರುತ್ತವೆ. ವೇದಾಂತದ ತತ್ತ್ವವು ಈ ಮೂರೂ ವಿಧಾನಗಳನ್ನು ಮೈಗೂಡಿಸಿಕೊಂಡಿರುವ ಒಂದು ಅಮೂರ್ತ ವಿಜ್ಞಾನ; ಇದನ್ನೇ ನಾನು ಬೋಧಿಸುವುದು. ಪ್ರತಿಯೊಬ್ಬರನ್ನೂ ತಮ್ಮದೇ ಆದ ವಾಸ್ತವಿಕ ರೀತಿಯಲ್ಲಿ ಇದನ್ನು ಗ್ರಹಿಸಿ ಸ್ವೀಕರಿಸಿಕೊಳ್ಳಲು ಬಿಡುತ್ತೇನೆ. ಪ್ರತಿಯೊಬ್ಬ ವ್ಯಕ್ತಿಯನ್ನೂ ನಾನು ಅವನವನ ಅಂತರಂಗದ ಅನುಭವದ ಆಧಾರದ ಮೇಲೇ ಪರ್ಯಾಲೋಚಿಸುವಂತೆ ಬಿಡುತ್ತೇನೆ; ಗ್ರಂಥಗಳ ಆಧಾರ ಅಗತ್ಯವಾದಾಗ ಅಂತಹ ಗ್ರಂಥಗಳು ಲಭ್ಯವಿರುತ್ತವೆ; ಪ್ರತಿಯೊಬ್ಬರೂ ಅವುಗಳನ್ನು ತಾವೇ ಸ್ವಂತವಾಗಿ ಅಧ್ಯಯನ ಮಾಡಬಹುದು.”

ಸ್ವಾಮಿಗಳು ಗುಪ್ತವಾದ ಯಾವುದರಿಂದಲಾದರೂ - ಗುಪ್ತ ವ್ಯಕ್ತಿಗಳಾಗಲಿ, ಗ್ರಂಥಗಳಾಗಲಿ, ಕೈಬರಹಗಳಾಗಲಿ - ಪಡೆಯುವ ಅಧಿಕೃತತೆಯನ್ನು ಬೋಧಿಸುವುದಿಲ್ಲ. ಗುಪ್ತ ಸಂಘಟನೆಗಳಿಂದ ಯಾವ ಒಳಿತೂ ಆಗಲಾರದೆಂದು ಅವರ ನಂಬಿಕೆ.

“ಸತ್ಯವು ನಿಲ್ಲುವುದು ಸ್ವಯಂಸಿದ್ಧ ಅಧಿಕೃತತೆಯ ಮೇಲೆ, ಸತ್ಯವು ಪ್ರಚಲಿತ ದಿನದ ಪ್ರಖರ ಬೆಳಕನ್ನು ಸಹಿಸಿ ನಿಲ್ಲಬಲ್ಲುದು.”

ಅವರು ಬೋಧಿಸುವುದು ಎಲ್ಲರಿಗೂ ಸಾಮಾನ್ಯವಾದ, ಎಲ್ಲರ ಹೃದಯದಲ್ಲಿ ಹುದುಗಿರುವ ಆತ್ಮನನ್ನು ಮಾತ್ರ. ಆತ್ಮನನ್ನು ತಿಳಿದಿರುವ, ಅದರ ಬೆಳಕಿನಲ್ಲಿಯೇ ಬದುಕುತ್ತಿರುವ, ಒಂದು ಕೈಬೆರಳೆಣಿಕೆಯಷ್ಟು ಸಶಕ್ತ ಜನರು ಈ ಹೊತ್ತಿಗೂ ಇಡೀ ಪ್ರಪಂಚದಲ್ಲಿ ಕ್ರಾಂತಿಯನ್ನುಂಟುಮಾಡಿ ಬಿಡಬಲ್ಲರು - ಹಿಂದೆ ಏಕಾಂಗಿಗಳಾದ ಅಂತಹ ಮಹಾ ಸಮರ್ಥರು ತಮ್ಮದೇ ಕಾಲಗಳಲ್ಲಿ ಮಾಡಿರುವುದೂ ಹಾಗೆ.

ಪಾಶ್ಚಾತ್ಯ ಧರ್ಮಗಳ ಬಗ್ಗೆ ಅವರ ಧೋರಣೆ ಸಂಕ್ಷಿಪ್ತವಾಗಿ ಹೇಳುವುದಾದರೆ ಇಷ್ಟು. ಅವರು ಪ್ರಸ್ತುತಪಡಿಸುವ ತತ್ತ್ವವು ಲೋಕದಲ್ಲಿ ಸಾಧ್ಯವಿರುವ ಪ್ರತಿಯೊಂದು ಧರ್ಮಕ್ಕೂ ತಳಹದಿಯಾಗುವಂಥದು; ಅವುಗಳೆಲ್ಲದರ ಬಗ್ಗೆ ಅವರ ಧೋರಣೆ ಅತ್ಯಂತ ಸಹಾನು ಭೂತಿಯದು. ಅವರ ಬೋಧನೆ ಯಾರಿಗೂ ವಿರೋಧವಾಗಿರುವುದಲ್ಲ. ಅವರು ವ್ಯಕ್ತಿಯ ಕಡೆಗೆ ಗಮನ ಹರಿಸುತ್ತಾರೆ - ಅವನನ್ನು ಬಲಪಡಿಸುವುದಕ್ಕಾಗಿ, ಅವನು ತಾನೇ ದಿವ್ಯತೆಯನ್ನು ಹೊಂದಿರುವನು ಎಂಬುದನ್ನು ಅವನಿಗೆ ಬೋಧಿಸುವುದಕ್ಕಾಗಿ; ನಿಮ್ಮ ದಿವ್ಯತೆಯ ಪ್ರಜ್ಞೆಯನ್ನು ತಂದುಕೊಳ್ಳಿರಿ ಎಂದು ಎಲ್ಲರಿಗೂ ಕರೆ ಕೊಡುತ್ತಾರೆ. ಅವರದನ್ನು ಅವರದೇ ಆದ ರೀತಿಯಲ್ಲಿ ಮಾರ್ಪಡಿಸಿಕೊಳ್ಳಬಹುದು; ಅದೇನೂ ಮತ ಸಿದ್ಧಾಂತವಲ್ಲ; ಕೊನೆಗೆ ಗತ್ಯಂತರವಿಲ್ಲದೆ ಉಳಿದೇ ಉಳಿಯುವುದು ಸತ್ಯ ಮಾತ್ರವೇ....

\begin{center}
\textbf{ಪ್ರಾಚ್ಯದಿಂದ ಹೊರಕ್ಕೆ\supskpt{\footnote{\enginline{1. New Discoveries, Vol. 4, p.60-62}}}}
\end{center}

\begin{center}
(ಬೋಸ್ಟನ್ ಡೈಲಿ ಗ್ಲೋಬ್, ೨೪ ಮಾರ್ಚ್ ೧೮೯೬)
\end{center}

\begin{center}
\textbf{ಸ್ವಾಮಿ ವಿವೇಕಾನಂದರು ತಂದಿರುವ ಸಂದೇಶ - ಅವರ ದೇಶದ ದೇವರುಗಳು ನೆರವಾಗುವ ‘ಜ್ಯೋತಿ’ಗಳು}
\end{center}

ಈ ಹಿಂದೆ ಭೇಟಿಕೊಟ್ಟಾಗ ಸೊಗಸುಗಾರರು, ಬೌದ್ಧಿಕರು ಹಾಗೂ ಸ್ವಾಭಿಮಾನದ ಜನರು ಮುಗಿಬಿದ್ದು ಅವರ ಹಿಂದೆ ಹೋಗಿದ್ದಂತೆಯೇ, ಈ ಸಲದ ಭೇಟಿಯಲ್ಲಿಯೂ ಸಹ ಸ್ವಾಮಿ ವಿವೇಕಾನಂದರು ಜನಪ್ರಿಯತೆಯ ಉತ್ತುಂಗದಲ್ಲಿದ್ದಾರೆ....

.... ನ್ಯೂಯಾರ್ಕ್ ಪತ್ರಿಕೆಯೊಂದು ಸ್ವಾಮಿಗಳನ್ನು ಸಂದರ್ಶಿಸಿ ಪ್ರಕಟಿಸಿರುವಂತೆ, ಅವರ ಅಭಿಪ್ರಾಯದಲ್ಲಿ “ಬೋಸ್ಟನ್ನಿನ ಮಹಿಳೆಯರೆಲ್ಲ ಸ್ವಾಭಿಮಾನದವರು, ಚಂಚಲರು, ಕೇವಲ ಅಚ್ಚರಿಯ ಹೊಸತನ್ನು ಅನುಸರಿಸಲಷ್ಟೇ ಬಯಸುವವರು.”\footnote{2. ನೋಡಿ, ಕೃತಿ ಶ್ರೇಣಿ ಭಾಗ \enginline{8,} ಪುಟ \enginline{408}} ಆದರೆ ಸ್ವಾಮಿ ವಿವೇಕಾನಂದರು ತಾವು ಎಲ್ಲ ಅಮೆರಿಕನ್ ಮಹಿಳೆಯರ ಬಗ್ಗೆ ಅವರು ತೀರ ಮೇಲುಮೇಲಿನ ಗ್ರಹಿಕೆಯವರು, ತೀರ ಭಾವುಕತೆಯ ಬೆನ್ನುಬೀಳುವವರು, ತೀರ ಚಂಚಲರು ಎಂದು ಹೇಳಿರುವುದನ್ನು ಕುರಿತು ಇದೊಂದು ಉತ್ಪ್ರೇಕ್ಷಿತ, ವಿಕೃತಗೊಳಿಸಿದ ವರದಿ ಎಂದು ನುಡಿದರು. ಈ ಟೀಕೆಯನ್ನು ಮಾಡಲೇಬೇಕಾದ ಸನ್ನಿವೇಶವಿತ್ತು ಎನ್ನುತ್ತಾರೆ ಅವರು. ಅಮೆರಿಕನ್ ಮಹಿಳೆಯರು ಒಳ್ಳೆಯ ಬೌದ್ಧಿಕರು, ಆದರೆ ಗಂಭೀರ ಗ್ರಹಿಕೆ, ಪ್ರಾಮಾಣಿಕತೆ ಹಾಗೂ ಏಕಪ್ರಕಾರತೆ ಅವರಲ್ಲಿ ಕಡಿಮೆ ಎಂದರು.

ಸ್ವಾಮಿಗಳ ಮೊದಲನೆಯ ಉಪನ್ಯಾಸ “ಕಾರ್ಯಕ್ಷಮತೆಯ ವಿಜ್ಞಾನ” ನಡೆದದ್ದು ಶನಿವಾರ ಸಂಜೆ ಅಲೆನ್ ಜಿಮ್ನ್ಯಾಷಿಯಮ್​ನಲ್ಲಿ ನೆರೆದಿದ್ದ ಸುಮಾರು ನಾಲ್ಕು ನೂರು ಜನರ ಸಭೆಯಲ್ಲಿ. ಮುಂಬರಿದ ಎರಡನೆಯ ಉಪನ್ಯಾಸ “ಭಕ್ತಿ”\footnote{1. “೧೮೯೬ರ ಮಾರ್ಚ್೨೧ ಮತ್ತು ೨೩ರಂದು ಮಾಡಲಾದ ಈ ಉಪನ್ಯಾಸಗಳ ಪದಶಃ ವರದಿ ಲಭ್ಯವಿಲ್ಲ.}ಯನ್ನೂ ಅದೇ ಸ್ಥಳದಲ್ಲಿ ಕೊಟ್ಟರು; ಸಭಾಂಗಣ ಕಿಕ್ಕಿರಿದು ತುಂಬಿತ್ತು; ಕೆಲವರಂತೂ ಒಳಕ್ಕೆ ಪ್ರವೇಶ ಪಡೆಯಲಾರದೆ ಹಿಂದಿರುಗುವಂತಾಯಿತು.

ಉಪನ್ಯಾಸವು ಅತ್ಯಂತ ಆಸಕ್ತಿದಾಯಕವಾಗಿತ್ತು; ಭಾಷಣಕಾರರ ನಡಾವಳಿ ಯಂತೂ ತುಂಬ ಮನಮೋಹಕವಾಗಿತ್ತು. ತಮ್ಮ ದೇಶದಲ್ಲಿ ದೇವತೆಗಳೆಂದರೆ ಜ್ಯೋತಿ ಸ್ವರೂಪಿಗಳು, ಅವರು ಜನರಿಗೆ ನೆರವನ್ನು ನೀಡುವರಲ್ಲದೆ ಜನರಿಂದ ನೆರವನ್ನು ಪಡೆಯುವರು ಎಂದು ಸ್ವಾಮಿಗಳು ಹೇಳಿದರು. ದೇವತೆಗಳೂ ಮನುಷ್ಯರೇ, ಮರಣಾನಂತರ ಸ್ವಲ್ಪಮಟ್ಟಿಗೆ ಉನ್ನತಿಗೇರಿರುವವರು. ಆದರೆ ಪರಮಾತ್ಮನೆಂಬ ಅತ್ಯುನ್ನತನಿಗೆ ಪ್ರಾರ್ಥನೆಯನ್ನೂ ಸಲ್ಲಿಸುವುದಿಲ್ಲ, ಸಹಾಯವನ್ನು ಕೇಳುವುದೂ ಇಲ್ಲ. ಅವನನ್ನು ಕೇವಲ ಪ್ರೀತಿಸಿ ಪೂಜಿಸುವರು; ಅದಕ್ಕೆ ಪ್ರತಿಯಾಗಿ ಏನನ್ನೂ ಬೇಡುವುದಿಲ್ಲ. ಈ ಪರಮಾತ್ಮನಲ್ಲಿ ಎರಡು ಹಂತಗಳಿವೆ; ವಿಶ್ವಸಾರದ ಹಿಂದಿರುವ ಅಮೂರ್ತನು ಹಾಗೂ ಮಾನವ ಬುದ್ಧಿಗೆ ಗಮ್ಯನಾಗಿರುವ, ಗುಣಗಳನ್ನು ಆರೋಪಿಸಲ್ಪಟ್ಟ ಇಷ್ಟದೇವತೆ.

ಪರಮಾತ್ಮನಿಗೆ ಕೊಟ್ಟ ಪ್ರೇಮವು ಕಳೆಯುವುದೇನೂ ಇಲ್ಲ; ಬದಲಿಗೆ ಯಾವಾ ಗಲೂ ಹಿಂದಿರುಗಿ ಕೊಡುವಂಥದು; ಅದು ಯಾವುದನ್ನೂ ಅವಲಂಬಿಸಿರುವುದಿಲ್ಲ. ಆರಾಧಕರು ಹಣ, ಆರೋಗ್ಯ ಅಥವಾ ಇನ್ನಾವುದನ್ನಾದರೂ ಬಯಸಿ ಅದಕ್ಕಾಗಿ ಪ್ರಾರ್ಥಿಸುವುದಿಲ್ಲ; ತನ್ನ ಭಾಗಕ್ಕೆ ಬಂದಿರುವುದರಲ್ಲೇ ತೃಪ್ತನಾಗಿರುವನು.

ಕೇವಲ ಕುತೂಹಲದಿಂದ ಧರ್ಮದ ಬಗ್ಗೆ ಯಾರು ಕೇಳುತ್ತಾರೆಯೋ ಅವರಿಗೆ ಅದೇ ಒಂದು ಗೀಳಾಗಿಬಿಡುತ್ತದೆ; ಅವರು ಯಾವಾಗಲೂ ಏನಾದರೊಂದು ಹೊಸತನ್ನು ಹುಡುಕುತ್ತಲೇ ಇರುತ್ತಾರೆ; ಇಂಥವರ ಮೆದುಳು ಕ್ರಮೇಣ ನಶಿಸಿಹೋಗಿ ಅವರು ಕೊನೆಗೆ ಕೆಲಸಕ್ಕೆ ಬಾರದ ವೃದ್ಧರಾಗುತ್ತಾರೆ. ಧರ್ಮದಿಂದಾಗಿಯೆ ಅವರು ವಿನಾಶದ ಹಾದಿ ಹಿಡಿಯುತ್ತಾರೆ.

ಸ್ವರ್ಗ ನರಕಗಳನ್ನು ಸೃಷ್ಟಿಸುವುದು ಸ್ಥಳವಲ್ಲ, ಮನಸ್ಸು. ಪ್ರೇಮಕ್ಕೆ ಭಯವಿಲ್ಲ; ಭಯ ಎಲ್ಲಿರುವುದೋ ಅಲ್ಲಿ ಪ್ರೇಮವಿರಲಾರದು. ಯಾವುದೇ ರೀತಿಯ ಪ್ರೇಮದ ಲ್ಲಾಗಲಿ, ಹೊರಗಿನ ವಸ್ತುವಿಗೆ ಸೂಚನೆ ನೀಡುವ ಯಾವುದೋ ಒಳಗಿನ ಒಂದು ವಸ್ತು ಬೇಕೆನಿಸುತ್ತದೆ - ಅದು ಹೊರಹೊಮ್ಮಿಸಲ್ಪಟ್ಟ ನಮ್ಮದೇ ಆದರ್ಶವಾಗಿರಬಹುದು. ಕಲ್ಪಿಸಿಕೊಳ್ಳಬಹುದಾದ ಆದರ್ಶಗಳಲ್ಲೆಲ್ಲ ಪರಮಾತ್ಮನೇ ಅತ್ಯುನ್ನತವಾದ ಆದರ್ಶ.

ಪ್ರಪಂಚದ ಮೇಲಣ ಅರುಚಿ ಸತ್ಪುರುಷರನ್ನು ಅದರಿಂದ ಹೊರದೂಡುವುದಿಲ್ಲ; ಬದಲಿಗೆ ಸಂತರಾಗಿರುವವರಿಂದ, ಮಹಾತ್ಮರಾಗಿರುವವರಿಂದ ಪ್ರಪಂಚವೇ ಜಾರಿ ಹೋಗುತ್ತದೆ. ಲೋಕ, ಸಂಸಾರ, ಸಮಾಜಜೀವನ ಇವೆಲ್ಲ ಪರಿಣತಿ ಪಡೆಯುವ ಕ್ಷೇತ್ರಗಳಷ್ಟೇ.

ಪರಮಾತ್ಮನೇ ಪ್ರೇಮ ಎಂಬುದನ್ನು ಸಾಕ್ಷಾತ್ಕರಿಸಿಕೊಂಡಿರುವವನಿಗೆ ಆತನ ಇನ್ನಿತರ ಗುಣಗಳು ಗೌಣವಾಗುತ್ತವೆ; ಪ್ರೇಮವೊಂದೇ ಸಾಕಾಗುತ್ತದೆ.

ಒಬ್ಬನು ತನ್ನನ್ನು ತಾನು ಎಷ್ಟರಮಟ್ಟಿಗೆ ಹೊರಹಾಕುವನೋ ಅಷ್ಟೇಮಟ್ಟಿಗೆ ಪರ ಮಾತ್ಮನ ಆವಿರ್ಭಾವವಾಗುತ್ತದೆ. ಹೀಗೆ, ಸ್ವಾರ್ಥತ್ಯಾಗವೇ ಧರ್ಮ, ನೈತಿಕತೆ, ಎಲ್ಲದರ ರಹಸ್ಯ.

ಬಹಳಷ್ಟು ಜನರು ತಮ್ಮ ಆದರ್ಶಗಳನ್ನು ಕೆಳಕ್ಕಿಳಿಸಿಕೊಳ್ಳುತ್ತಾರೆ. ಅವರಿಗೆ ಸುಖಸೌಲಭ್ಯಗಳನ್ನುಳ್ಳ ಅಪ್ಯಾಯಮಾನ ಧರ್ಮಬೇಕು; ಆದರೆ ಅಂಥದು ಇರಲು ಸಾಧ್ಯವಿಲ್ಲ. ಇರುವುದೆಲ್ಲ ಆತ್ಮ ಬಲಿದಾನ, ಏರುಹಾದಿಯ ಪ್ರಯತ್ನ.

\begin{center}
\textbf{ವಿಶ್ವಧರ್ಮ ಅಸಾಧ್ಯವೆಂದರು\supskpt{\footnote{\enginline{1. New Discoveries, Vol. 4, p.64-65}}}}
\end{center}

\begin{center}
\textbf{(ಬೋಸ್ಟನ್ ಈವಿನಿಂಗ್ ಟ್ರಾನ್ಸ್ಕ್ರಿಪ್ಟ್, ೨೭ ಮಾರ್ಚ್ ೧೮೯೬)}
\end{center}

ಕಳೆದ ರಾತ್ರಿ ಅಲೆನ್ ಜಿಮ್ನ್ಯಾಷಿಯಮ್​ನಲ್ಲಿ “ವಿಶ್ವಧರ್ಮದ ಆದರ್ಶ”\footnote{2. ಉಪನ್ಯಾಸದ ಪದಶಃ ವರದಿ ಲಭ್ಯವಿಲ್ಲ.}ವೆಂಬ ಉಪನ್ಯಾಸವನ್ನು ಕೇಳುವುದಕ್ಕೆ ಕಿಕ್ಕಿರಿದಿದ್ದ ದೊಡ್ಡ ಸಭೆಯನ್ನುದ್ದೇಶಿಸಿ ಸ್ವಾಮಿ ವಿವೇಕಾನಂದರು, ಇತ್ತೀಚೆಗೆ ಚಿಕಾಗೋದಲ್ಲಿ ನಡೆದ ಸರ್ವಧರ್ಮಸಮ್ಮೇಳನವು ಈ ಹೊತ್ತಿನವರೆಗೂ ವಿಶ್ವಧರ್ಮವೆಂಬುದು ಅಸಾಧ್ಯವೆಂದೇ ತೋರಿಸಿಕೊಟ್ಟಿದೆ ಎಂದು ನುಡಿದರು. ಆ ನಂತರ ಮುಂದುವರೆದು ಹೇಳಿದರು;

“ಪ್ರಕೃತಿಮಾತೆ ನಾವು ಅಂದುಕೊಳ್ಳುವುದಕ್ಕಿಂತ ಹೆಚ್ಚು ಬುದ್ಧಿವಂತಳು. ಕಲ್ಪನೆಗಳಲ್ಲಿ ಪೈಪೋಟಿ, ಚಿಂತನೆಗಳಲ್ಲಿ ಸಂಘರ್ಷ, ಇವೇ ನಮ್ಮ ಯೋಚನೆಗಳನ್ನು ಜೀವಂತವಾಗಿ ಇಟ್ಟಿರುವುದು. ಮತಪಂಥಗಳು ಯಾವಾಗಲೂ ಪರಸ್ಪರ ವೈದೃಶ್ಯದವು, ಯಾವಾಗಲೂ ತಮ್ಮಲ್ಲಿಯೇ ಸಣ್ಣಪುಟ್ಟ ಭಿನ್ನತೆಗಳಿಂದಾಗಿ ಒಡೆಯುತ್ತಲೇ ಇರುವಂಥವು. ಧರ್ಮಗಳ ಈ ಹೋರಾಟದಿಂದ ಹೊರಗೆ ಬರುವ ಮಾರ್ಗವೆಂದರೆ, ಮತಪಂಥಗಳು ಒಡೆಯುತ್ತ ಹೋಗುವುದಕ್ಕೆ ಮುಕ್ತ ಅವಕಾಶವನ್ನು ಕೊಟ್ಟುಬಿಡುವುದು.

“ತತ್ತ್ವಶಾಸ್ತ್ರ (ದೇವತಾಶಾಸ್ತ್ರ?), ಪುರಾಣ ಮತ್ತು ಕ್ರಿಯಾಕರ್ಮಗಳೆಂಬ ಧರ್ಮದ ಈ ಮೂರು ಅಂಶಗಳಲ್ಲಿ ಐಕ್ಯತೆ ಇಲ್ಲ. ಪ್ರತಿಯೊಬ್ಬ ದೇವತಾಶಾಸ್ತ್ರಜ್ಜನಿಗೂ ಐಕ್ಯತೆಬೇಕು, ಆದರೆ ಅವನ ಐಕ್ಯತೆಯ ಕಲ್ಪನೆ ಎಂದರೆ ಇನ್ನಿತರ ಮತಪಂಥಗಳವರು ತನ್ನ ಜೊತೆಗೆ ಸರಿ ಹೊಂದಿಸಿಕೊಳ್ಳಬೇಕು ಎನ್ನುವುದು. ನನ್ನ ಜೊತೆಗೆ ಏಕಾಭಿಪ್ರಾಯ ಇರುವವರೆಗೆ ನಾನೂ ಹಿಂದಿನ ಪ್ರವಾದಿಗಳನ್ನೆಲ್ಲ ಒಪ್ಪುತ್ತೇನೆ. ಆದರೆ ಧರ್ಮದ ಒಂದು ಅಂಶ ಉಳಿದೆಲ್ಲಕ್ಕಿಂತ ಉತ್ತುಂಗದಲ್ಲಿದೆ - ಅದೇ ತತ್ತ್ವಜ್ಞಾನ, ತತ್ತ್ವಜ್ಞಾನಿಯು ಸತ್ಯವನ್ನು ಅನ್ವೇಷಿಸುತ್ತಿರುತ್ತಾನೆ. ಆ ಸತ್ಯವು ಯಾವಾಗಲೂ ಒಂದೇ ಆಗಿರುತ್ತದೆ; ಅಲ್ಲದೆ ಅದು ಪ್ರತಿಯೊಂದು ಧಾರ್ಮಿಕ ಸ್ವಭಾವದ ನಾಲ್ಕು ಪಕ್ಷಗಳಿಗೂ - ಭಾವನಾತ್ಮಕ, ಧ್ಯಾನಶೀಲ, ಕ್ರಿಯಾಶೀಲ ಹಾಗೂ ತಾತ್ತ್ವಿಕ - ಸ್ವೀಕಾರಾರ್ಹವಾಗಿರುತ್ತದೆ. ಸತ್ಯವನ್ನು ಸತ್ಯಕ್ಕಾಗಿಯೇ ಅನ್ವೇಷಿಸುವವನೇ ಮಾನವರಲ್ಲೆಲ್ಲ ಅತ್ಯುನ್ನತನಾಗಿರುವನು.

\begin{center}
\textbf{ವಿಶ್ವಧರ್ಮಕ್ಕಾಗಿಯೇ\supskpt{\footnote{\enginline{1. New Discoveries, Vol. 4, pp.81-86}}}}
\end{center}

\begin{center}
(ಬೋಸ್ಟನ್ ಈವಿನಿಂಗ್ ಟ್ರಾನ್ಸ್ಕ್ರಿಪ್ಟ್, ೩೦ ಮಾರ್ಚ್ ೧೮೯೬)
\end{center}

\textbf{ಹಿಂದೂ ಸ್ವಾಮಿಗಳು ಅನೇಕ ಸಂಘಟನೆಗಳನ್ನು ಕುರಿತ ಉಪನ್ಯಾಸ ಮಾಡಿದರು}

ಕಳೆದ ಕೆಲವು ದಿನಗಳಿಂದ ಸ್ವಾಮಿ ವಿವೇಕಾನಂದರು ಪ್ರೋಕೋಪಿಯಾಕ್ಕೆ ಸಂಬಂಧಪಟ್ಟಂತೆ ಅತ್ಯಂತ ಯಶಸ್ವೀ ಕಾರ್ಯವನ್ನು ನಿರ್ವಹಿಸಿರುವರು. ಈ ಅವಧಿಯಲ್ಲಿ ಅವರು ಕ್ಲಬ್ಗಾಗಿಯೇ ೪೪, ಸಂತ ಬೋಟೋಲ್ಙ ರಸ್ತೆಯ ಅಲೆನ್ ಜಿಮ್ನ್ಯಾಷಿಯಮ್​ನಲ್ಲಿ ನಾಲ್ಕರಿಂದ ಐದು ನೂರು ಸ್ಥಿರಸಂಖ್ಯೆಯ ಸಭಿಕರಿಗಾಗಿ ನಾಲ್ಕು ತರಗತಿ ಉಪನ್ಯಾಸಗಳನ್ನು, ಕೇಂಬ್ರಿಡ್ಜ್ನ ಮಿಸೆಸ್ ಓಲ್ ಬುಲ್ ಅವರ ನಿವಾಸದಲ್ಲಿ ಎರಡು ಉಪನ್ಯಾಸಗಳನ್ನು ಮತ್ತು ಹಾರ್ವರ್ಡ್ ವಿಶ್ವವಿದ್ಯಾನಿಲಯದ ತತ್ತ್ವಶಾಸ್ತ್ರವಿಭಾಗದ ಸ್ನಾತಕ ವಿದ್ಯಾರ್ಥಿಗಳು ಹಾಗೂ ಪ್ರಾಧ್ಯಾಪಕರೆದುರು ಒಂದು ಉಪನ್ಯಾಸವನ್ನು ಕೊಟ್ಟಿರುವರು.

ಸ್ವಾಮಿಗಳನ್ನು ಮೂರು ವರ್ಷಗಳ ಹಿಂದೆ ಸರ್ವಧರ್ಮ ಸಮ್ಮೇಳನಕ್ಕೆ ಹಿಂದೂ ಪ್ರತಿನಿಧಿಯಾಗಿ ಕರೆತಂದ ಯೋಜನೆ ಹಾಗೂ ಆ ನಂತರ ಅವರು ಅಮೆರಿಕಾ ಮತ್ತು ಇಂಗ್ಲೆಂಡ್ಗಳೆರಡರಲ್ಲಿಯೂ ಮುಂದುವರೆಸಿಕೊಂಡು ಬಂದಿರುವ ಕಾರ್ಯದ ಹಿಂದಿರುವ ಮಾರ್ಗದರ್ಶಕ ಸೂತ್ರ ಹಾಗೂ ಪ್ರೇರಣೆ - ಇವು ಸರ್ವಧರ್ಮಸಮ್ಮೇಳನವನ್ನು ಸೃಜಿಸಿದ ನಿಯೋಜಕರಿಗೆ ತುಂಬ ಚೆನ್ನಾಗಿ ಒಪ್ಪಿಗೆಯಾಗುವಂಥವುಗಳೇ; ಆದರೂ ಅದನ್ನು ಸಾಧ್ಯವಾಗಿಸುವುದಕ್ಕಾಗಿ ಅವರು ಸೂಚಿಸುತ್ತಿರುವ ವಿಚಿತ್ರ ವಿಧಾನಗಳು ಮಾತ್ರ ಅವರವೇ ಆಗಿವೆ. ಇದೇ ವಾರ ಅವರು “ವಿಶ್ವಧರ್ಮವೊಂದರ ಆದರ್ಶ”\footnote{1. ಇದು ಸ್ವಾಮಿ ವಿವೇಕಾನಂದರು ಮತ್ತೆ ಮತ್ತೆ ಉಪನ್ಯಾಸ ಮಾಡುತ್ತಿದ್ದ ವಿಷಯಗಳಲ್ಲೊಂದಾದರೂ ಈ ೨೬ ಮಾರ್ಚ್ ೧೮೯೬ರ ಉಪನ್ಯಾಸದ ಪದಶಃ ವರದಿ ಲಭ್ಯವಿಲ್ಲ. ನೋಡಿ ಕೃತಿಶ್ರೇಣಿ ೨ ಪು. ೨೧೮} ಎಂಬ ಉಪನ್ಯಾಸ ಮಾಡಿರುವರು. ಆದರೆ ಅವರು ಶ್ರಮಿಸುತ್ತಿರುವುದಕ್ಕೆ ಅದು ನಿಜವಾಗಿಯೂ ಇನ್ನೂ ಹೆಚ್ಚು ಸಮರ್ಪಕವೆಂದಾಗದಿದ್ದಲ್ಲಿ, “ಸಾಮರಸ್ಯದ ಧರ್ಮ” ಎಂಬುದು ಬಹುಶಃ ಅದೇ ಉದ್ದೇಶಕ್ಕೆ ಸೂಕ್ತವಾಗಬಹುದೇನೋ ಎಂದೆನಿಸುತ್ತದೆ. ಸ್ವಾಮಿಗಳು ಕೇವಲ ಸಿದ್ಧಾಂತದ ಪ್ರಬೋಧಕರೇನೂ ಅಲ್ಲ. ಅವರು ಪ್ರತಿಪಾದಿಸುವ ವೇದಾಂತ ತತ್ತ್ವದಲ್ಲಿ ವಿಶೇಷವಾಗಿ ಮುದಗೊಳಿಸುವಂತೆ ಕಾಣುವ ಅಂಶವೊಂದು ಇರುವುದೇ ಆದರೆ, ಅದು ತನ್ನ ಆಮೂಲಾಗ್ರ ಪ್ರಾತ್ಯಕ್ಷೀಕರಣಕ್ಕೊಳಪಡಬಹುದಾದ ಸಾಮರ್ಥ್ಯದಿಂದಾಗಿ. ಧರ್ಮವೆಂದರೆ ಯಾವುದೋ ಭವ್ಯವಾದ ಸಿದ್ಧಾಂತ, ಅದು ನಮ್ಮ ಅನುಷ್ಠಾನಕ್ಕೆ ದಕ್ಕಬೇಕಾ ದರೆ ಇನ್ನೊಂದು ಜನ್ಮದಲ್ಲೇ ಸರಿ ಎನ್ನುವಂತಹ ಕಲ್ಪನೆಗೆ ಮನಸೋತುಬಿಟ್ಟಿರು ವುದೇ ಅಭ್ಯಾಸವಾಗಿರುವ ನಮಗೆ ಸ್ವಾಮಿಗಳು ಆ ಕಲ್ಪನೆಯ ಮೂರ್ಖತನವನ್ನು ತೋರಿಸಿಕೊಡುತ್ತಾರೆ. ಮಾನವನ ದಿವ್ಯತೆಯನ್ನು ಅವರು ಬೋಧಿಸುವಾಗ ನಮ್ಮಲ್ಲಿ ಅವರು ಅದೆಂತಹ ಶಕ್ತಿಚೈತನ್ಯಗಳನ್ನು ತುಂಬುತ್ತಾರೆಂದರೆ, ಸಾಮಾನ್ಯ ಮನುಷ್ಯನಿಗೆ ನಮ್ಮ ಈ ಬದುಕಿಗೂ ಧರ್ಮದ ಭವ್ಯತೆಗೂ ನಡುವೆ ಇದ್ದಂತೆನ್ನಿಸುವ, ದಾಟಲಸಾಧ್ಯವೆಂದು ಭಾಸವಾಗುವ ತಡೆಗೋಡೆಯು ಇಲ್ಲದಂತಾಗುತ್ತದೆ.

ವಿಶ್ವಧರ್ಮ ಮಾತ್ರವೇ ಸ್ಥಾಪನಯೋಗ್ಯವೆಂದು ಅವರಿಗನ್ನಿಸುವುದು ಹೇಗೆ ಎಂಬುದನ್ನು ಸಾಮಾನ್ಯ ಚರ್ಚೆಗೆ ತೆಗೆದುಕೊಳ್ಳುವಾಗ ಅವರು ಅದಕ್ಕಾಗಿ ಯಾವ ಅಧಿಕೃತತೆಯನ್ನೂ ತಮ್ಮ ಮೇಲೆ ಆರೋಪಿಸಿಕೊಳ್ಳುವುದಿಲ್ಲ. ಅವರೇ ಹೇಳುವಂತೆ:

“ನನ್ನ ಬಳಿಯೂ ಸಣ್ಣದೊಂದು ಯೋಜನೆ ಇದೆ. ಅದು ಕಾರ್ಯಸಾಧುವೋ ಅಲ್ಲವೋ ನನಗೆ ತಿಳಿಯದು. ನಾನದನ್ನು ಚರ್ಚೆಗಾಗಿ ನಿಮ್ಮ ಮುಂದೆ ಇಡುತ್ತೇನೆ. ಮೊದಲನೆಯದಾಗಿ, ‘ನಾಶಮಾಡುವುದಕ್ಕೆ ಹೊರಡಬೇಡಿ’ ಎನ್ನುವ ಸೂತ್ರವಾಕ್ಯಕ್ಕೆ ಮನ್ನಣೆ ಕೊಡಿ ಎಂದು ಮನುಜಕುಲವನ್ನು ಕೇಳಿಕೊಳ್ಳುತ್ತೇನೆ. ಶ್ರದ್ಧೆಯನ್ನು ಹಾಳುಮಾಡುವ ಸುಧಾರಕರಿಂದ ಲೋಕಕ್ಕೆ ಒಳ್ಳೆಯದಾಗುವುದಿಲ್ಲ. ನಿಮಗೆ ಸಾಧ್ಯವಿದ್ದರೆ ಸಹಾಯ ಮಾಡಿ; ಇಲ್ಲವಾದರೆ ಕೈಕಟ್ಟಿ ದೂರ ನಿಂತುಕೊಂಡು ಆಗುವುದನ್ನು ನೋಡುತ್ತಿರಿ. ಆದ್ದರಿಂದ, ಯಾವೊ ಬ್ಬನ ಶ್ರದ್ಧೆಯ ವಿರುದ್ಧವಾಗಿ - ಅದು ಪ್ರಾಮಾಣಿಕವಾಗಿ ಇರುವವರೆಗೆ - ಒಂದೇ ಒಂದು ಅಪಶಬ್ದವನ್ನು ನುಡಿಯದಿರಿ, ಎರಡನೆಯದಾಗಿ, ಮಾನವನು ಎಲ್ಲಿ ನಿಂತಿರುವನೋ ಅಲ್ಲಿಯೇ ಅವನನ್ನು ಪರಿಗ್ರಹಿಸಿ, ಅಲ್ಲಿಂದಲೇ ಅವನನ್ನು ಮೇಲಕ್ಕೆತ್ತಿ.\footnote{2. ನೋಡಿ ಕೃತಿಶ್ರೇಣಿ ೨ ಪು. ೨೨೫}

“ವೈವಿಧ್ಯತೆಯಲ್ಲಿ ಏಕತೆ ಎನ್ನುವುದೇ ವಿಶ್ವದಲ್ಲಿರುವ ಯೋಜನೆ. ನಾವೆಲ್ಲರೂ ಮನುಷ್ಯರು ಎಂದಾಗಿರುವಾಗಲೇ ನಾವೆಲ್ಲರೂ ಪ್ರತ್ಯೇಕ ವ್ಯಕ್ತಿಗಳೂ ಹೌದು. ಮಾನವೀಯತೆ ಎಂದಾದಾಗ, ನಾನೂ ನೀವೂ ಒಂದೇ; ಶ‍್ರೀಮಾನ್ ಇಂಥವನು ಎನ್ನುವಾಗ ನಾನು ನಿಮ್ಮಿಂದ ಬೇರೆ. ಪುರುಷನೆನ್ನುವಾಗ ನೀವು ಸ್ತ್ರೀಯರಿಗಿಂತ ಭಿನ್ನ; ಆದರೆ ಮಾನವರು ಎನ್ನುವಾಗ ನೀವೆಲ್ಲರೂ ಒಂದೇ; ಜೀವಿ ಎಂದಾದಾಗ ನೀವು ಪ್ರಾಣಿಗಳೊಡನೆ, ಏಕೆ ಜೀವಿಸಿರುವ ಎಲ್ಲವುಗಳೊಡನೆ ಒಂದೇ, ಆದರೆ ಮಾನವ ಎನ್ನುವಾಗ ನೀವು ಬೇರೆ. ಅಸ್ತಿತ್ವವೇ ಈ ವಿಶ್ವದ ಕೊಟ್ಟಕೊನೆಯ ಏಕತೆ, ಅದೇ ದೇವರು; ಅವನಲ್ಲಿ ನಾವೆಲ್ಲರೂ ಒಂದೇ ಎಂದಮೇಲೆ, ವಿಶ್ವಧರ್ಮದ ಕಲ್ಪನೆ ಎಂದರೆ ಒಂದಷ್ಟು ಸಿದ್ಧಾಂತಗಳನ್ನು ಎಲ್ಲರೂ ನಂಬಬೇಕೆಂದು ಅರ್ಥಮಾಡಿಕೊಂಡರೆ ಅದು ಎಂದೆಂದಿಗೂ ಅಸಾಧ್ಯ; ಎಲ್ಲರ ಮುಖಗಳೂ ಒಂದೇ ತರಹ ಆಗಬೇಕು ಎನ್ನುವ ಹಾಗೆ ಅದು. ಇನ್ನು ಎಲ್ಲರಿಗೂ ಒಂದು ಪುರಾಣಕಥೆ ಇರಬೇಕು ಎನ್ನುವುದಾದರೆ, ಅದೂ ಅಸಾಧ್ಯ; ಅದೆಂದಿಗೂ ಸಾಧ್ಯವಾಗದು. ಅಂತೆಯೇ ಎಲ್ಲರಿಗೂ ಸಮಾನವಾದ ಕ್ರಿಯಾವಿಧಿಗಳೂ ಇರುವುದು ಅಸಾಧ್ಯ. ಇಂಥ ಕಾಲವೇನಾದರೂ ಬಂದರೆ, ಪ್ರಪಂಚ ನಾಶವಾಗುವುದು; ಏಕೆಂದರೆ ವೈವಿಧ್ಯತೆಯೇ ಜೀವನದ ಮೊಟ್ಟ ಮೊದಲ ತತ್ತ್ವ. ನಾವೇಕೆ ರೂಪ ತಳೆದ ಜೀವಿಗಳಾಗಿ ದ್ದೇವೆ? ಭಿನ್ನತೆಯಿಂದ. ಪರಿಪೂರ್ಣ ಸಮತೋಲ ಎಂದರೆ ನಾಶವೇ ಸರಿ.\footnote{1. ನೋಡಿ ಕೃತಿಶ್ರೇಣಿ ೨ ಪು. ೨೨೩-೨೨೪}

“ಹಾಗಾದರೆ ವಿಶ್ವಧರ್ಮದ ಆದರ್ಶ ಎಂದರೆ ಅರ್ಥವೇನು? ಯಾವುದೋ ಒಂದು ವಿಶ್ವ ತತ್ತ್ವವಲ್ಲ, ಒಂದು ವಿಶ್ವ ಪುರಾಣವಲ್ಲ, ಒಂದು ಸಾರ್ವತ್ರಿಕ ಕ್ರಿಯಾಕರ್ಮವಲ್ಲ; ನಾನು ಹೇಳುವುದೇನೆಂದರೆ, ಚಕ್ರದೊಳಗೊಂದು ಚಕ್ರವಿರುವ ಯಂತ್ರದ ಹಾಗೆ, ಈ ಪ್ರಪಂಚ ಮುಂದುವರೆಯಬೇಕು. ನಾವೇನು ಮಾಡಬಲ್ಲೆವು? ನಾವು ಅದನ್ನು ನಯವಾಗಿ ನಡೆಯುವಂತೆ ಮಾಡಬಹುದು. ಚಕ್ರಗಳಿಗೆ ಎಣ್ಣೆ ಸವರಿ ಘರ್ಷಣೆ ಕಡಿಮೆಯಾಗುವಂತೆ ಮಾಡಬಹುದು ಹೇಗೆ? ವೈವಿಧ್ಯತೆಯನ್ನು ಸ್ವೀಕರಿಸುವ ಮೂಲಕ. ನಮ್ಮ ಸ್ವಭಾವದಲ್ಲಿರುವ ಏಕತೆಯನ್ನು ನಾವು ಹೇಗೆ ಪರಿಗಣನೆಗೆ ತಂದುಕೊಂಡಿರುವೆವೋ, ಹಾಗೆಯೇ ವೈವಿಧ್ಯತೆಗೂ ಮನ್ನಣೆ ಕೊಡಬೇಕು. ಸತ್ಯವನ್ನು ಸಾವಿರ ರೀತಿಗಳಲ್ಲಿ ಅಭಿವ್ಯಕ್ತಿಗೊಳಿಸಬಹುದಾದರೂ, ಅದರಲ್ಲಿ ಪ್ರತಿಯೊಂದೂ ನಿಜ ಎನ್ನುವುದನ್ನು ಕಲಿತುಕೊಳ್ಳಬೇಕು. ಒಂದನ್ನೇ ನೂರಾರು ದೃಷ್ಟಿಗಳಿಂದ ನೋಡ ಬಹುದಾದರೂ, ಅದು ಒಂದೇ ಆಗಿ ಉಳಿದಿರುವುದು ಎನ್ನುವುದನ್ನು ತಿಳಿಯಬೇಕು.\footnote{2. ನೋಡಿ ಕೃತಿಶ್ರೇಣಿ ೨ ಪು. ೨೨೪-೨೨೫}

“ಸಮಾಜದಲ್ಲಿ ಅನೇಕ ರೀತಿಯ ಮನುಷ್ಯಸ್ವಭಾವಗಳನ್ನು ನಾವು ನೋಡುವೆವು. ವಾಸ್ತವಿಕ ಸಾಮಾನ್ಯೀಕರಣ ಅಸಾಧ್ಯವಾದರೂ, ನನ್ನ ಉದ್ದೇಶಕ್ಕಾಗಿ ನಾನು ಅವುಗಳನ್ನು ನಾಲ್ಕಾಗಿ ವಿಂಗಡಿಸಿರುವೆನು - ಕ್ರಿಯಾಶೀಲ, ಭಾವುಕ, ಧ್ಯಾನಿ, ಹಾಗೂ ಜ್ಞಾನಿ ಎಂಬುದಾಗಿ. “ವಿಶ್ವಧರ್ಮವೆಂದಾಗಬೇಕಾದರೆ, ಅದು ಈ ಯಾವುದೇ ಬಗೆಯ ಮನಸ್ಸುಗಳಿಗೆ ಸಾಧ್ಯವಾಗಬಹುದಾದ ಸತ್ಯಸಾಕ್ಷಾತ್ಕಾರದ ಮಾರ್ಗಗಳನ್ನು ಹಾಕಿಕೊಡಬೇಕು. ಒಂದೇ ಮಾರ್ಗದಲ್ಲಿ ಎಲ್ಲ ಮಾನವರೂ ಶ್ರಮಪಟ್ಟು ಮುಂದುವರೆಯಬೇಕು - ಅವರ ಮನಸ್ಸುಗಳಿಗೆ ಹೀಗೆ ಹೋರಾಡಲು ಸಾಮರ್ಥ್ಯವಿರಲಿ ಇಲ್ಲದಿರಲಿ - ಎನ್ನುವಂತಹ ಧರ್ಮ ಕೊನೆಗಾಣುವುದು ಅಜ್ಞೇಯತಾವಾದದಲ್ಲಿಯೇ.”

ಸ್ವಾಮಿಗಳು ತಮ್ಮ ಕರ್ಮಯೋಗದ ಮೇಲಣ ಉಪನ್ಯಾಸ\footnote{1. ಬಹುಶಃ “ಕಾರ್ಯಶೀಲತೆಯ ವಿಜ್ಞಾನ” ಎಂಬ ಮಾರ್ಚ್ ೨೧ರ ತರಗತಿ ಇರಬೇಕು. ಇದರ ಪದಶಃ ವರದಿ ಲಭ್ಯವಿಲ್ಲ.}ದಲ್ಲಿ ಕಾರ್ಯಶೀಲತೆಯ ವಿಜ್ಞಾನವನ್ನು ವಿವರಿಸಿದರು. ಉಪನ್ಯಾಸದ ಹೆಚ್ಚು ಭಾಗ ಮನುಷ್ಯರ ಕಾರ್ಯಗಳ ಹಿಂದಿನ ಉದ್ದೇಶವನ್ನು - ಅದರಲ್ಲೂ ಈ ಭೂಮಿಯ ಮೇಲೆ ಮಾಡುವ ಒಳ್ಳೆಯ ಕೆಲಸಗಳಿಗೆ ಪ್ರತಿಫಲವಾಗಿ ಸ್ವರ್ಗವನ್ನು ಪಡೆಯುವ ಉದ್ದೇಶವನ್ನು - ವಿಶ್ಲೇಷಿಸುವುದಕ್ಕೆ ವಿನಿಯೋಗವಾಯಿತು. ಸ್ವಾಮಿಗಳು ಇದನ್ನು ಅಂಗಡಿವ್ಯಾಪಾರದ ಧರ್ಮ ಎಂದು ಕರೆದರು. ಯಾವುದೇ ಪ್ರತಿಫಲದ ಆಸೆ ಲವಲೇಶವೂ ಇಲ್ಲದೆ, ಪರಿಣಾಮ ಏನೇ ಆಗಲಿ ಅದನ್ನು ಲೆಕ್ಕಿಸದೆ, ಕೇವಲ ಕೆಲಸಕ್ಕಾಗಿಯೇ ಕೆಲಸ ಎನ್ನುವ ಹಾಗೆ ಮಾಡಿದಾಗ ಮಾತ್ರ ಕರ್ಮವು ತನ್ನ ಪರಾಕಾಷ್ಠೆಗೇರುತ್ತದೆ.

ಭಕ್ತಿಯೋಗ\footnote{2. ಬಹುಶಃ “ಭಕ್ತಿಯೋಗ” ಎಂಬ ಮಾರ್ಚ್ ೨೩ರ ತರಗತಿ ಇರಬೇಕು. ಇದರು ಪದಶಃ ವರದಿ ಲಭ್ಯವಿಲ್ಲ.}ವನ್ನು ಚರ್ಚಿಸುತ್ತ ಸ್ವಾಮಿಗಳು ಇಷ್ಟದೇವತಾಕಲ್ಪನೆಯ ತಾರ್ಕಿಕ ವಿವರಣೆಯನ್ನು ಕೊಟ್ಟರು. ಕಲ್ಪನೆಯ ಯಾವುದೋ ಒಂದನ್ನು ಪ್ರೀತಿಸಿ, ಅದರಿಂದ ಪ್ರೀತಿಯು ಫ್ರತಿಫಲಿತವಾಗಿ ನಮಗೂ ಹಿಂದಿರುಗಿ ಬರುತ್ತದೆ ಎಂದುಕೊಳ್ಳುವುದು, ಆ ನಮ್ಮ ಇಷ್ಟವನ್ನು ಪೂಜಿಸಿ ಆರಾಧಿಸುವುದು ಪ್ರಪಂಚಾದ್ಯಂತ ಇರತಕ್ಕಂಥದೇ. ಪ್ರೇಮ ಭಕ್ತಿಗಳ ಅಭಿವ್ಯಕ್ತಿಯ ಪ್ರಾಥಮಿಕ ಸ್ತರವೇ ಕ್ರಿಯಾಕರ್ಮಗಳನ್ನು ನಡೆಸುವುದು; ಇಲ್ಲಿ ಮನುಷ್ಯನಿಗೆ ಬೇಕಾಗಿರುವುದು ದ್ರವ್ಯರೂಪದ ವಸ್ತುಗಳು; ಅಮೂರ್ತ ಕಲ್ಪನೆಗಳೇ ಅವನಿಗೆ ಅಸಾಧ್ಯ. ಲೋಕದ ಚರಿತ್ರೆಯ ಉದ್ದಕ್ಕೂ ಮನುಷ್ಯನು ಅಮೂರ್ತವಾದದ್ದನ್ನು ಚಿಂತನಾರೂಪಗಳ, ಚಿಹ್ನೆಗಳ ಹಾಗೂ ಧರ್ಮದ ಬಾಹ್ಯ ಅಭಿವ್ಯಕ್ತಿಗಳ ಮೂಲಕ ಗ್ರಹಿಸಲು ಯತ್ನಿಸುತ್ತ ಬಂದಿರುವುದನ್ನು ಕಾಣಬಹುದು. ಘಂಟೆಗಳು, ಸಂಗೀತ, ಕ್ರಿಯಾಕರ್ಮಗಳು, ಗ್ರಂಥಗಳು, ಪ್ರತಿಮೆಗಳು ಮುಂತಾದವುಗಳೆಲ್ಲ ಇಂಥವೇ. ಮನುಷ್ಯನು ರೂಪಗಳ ಹಾಗೂ ಶಬ್ದದ ಜೊತೆಯಾಗಿ ಮಾತ್ರ ಚಿಂತಿಸಬಲ್ಲ. ಯೋಚನೆ ಬಂದೊಡನೆಯೇ ಅದರೊಂದಿಗೆ ನಾಮರೂಪಗಳೂ ಮನಸ್ಸಿನಲ್ಲಿ ಮಿಂಚುತ್ತವೆ; ಆದಕಾರಣ ದೇವರನ್ನು ಕುರಿತು ನಾವು ಯೋಚಿಸಿದಾಗ - ಅದು ಮಾನವರೂಪಿನ ಇಷ್ಟದೇವತೆಯೇ ಆಗಲಿ, ದಿವ್ಯತತ್ತ್ವವೆಂದೇ ಆಗಲಿ ಅಥವಾ ಇನ್ನಾವುದಾದರೂ ಆಗಲಿ - ಯಾವಾಗಲೂ ನಾವು ನಮ್ಮ ಅತ್ಯುನ್ನತ ಆದರ್ಶವನ್ನು ಯಾವುದಾದರೊಂದು ರೂಪದೊಂದಿಗೇ ಯೋಚಿಸುತ್ತಿರುವೆವು; ನಾವು ಕಲ್ಪಿಸಿಕೊಳ್ಳಬಲ್ಲ ಅತ್ಯುತ್ತಮ ರೂಪ ಮಾನವನದೇ ಆದ್ದರಿಂದ ಈ ರೂಪ ಸಾಮಾನ್ಯವಾಗಿ ಮಾನವನದ್ದೇ ಆಗಿರುವುದು. ಆದರೆ, ಇದು ಮಾನವನ ದೌರ್ಬಲ್ಯವೆಂದು ಇಟ್ಟುಕೊಂಡಾಗಲೂ, ಆ ಕಾರಣವಾಗಿ ಕ್ರಿಯಾವಿಧಿಗಳ, ಚಿಹ್ನೆಗಳ, ಗ್ರಂಥಗಳ ಮತ್ತು ಚರ್ಚುಗಳ ಒಂದು ಪ್ರಮಾಣದ ಬಳಕೆಯನ್ನು ಮಾಡುತ್ತಿರುವಾಗಲೂ, ಚರ್ಚಿನಲ್ಲಿ ಹುಟ್ಟುವು ದೇನೋ ಮಹಾಭಾಗ್ಯವೆ-ಆದರೂ ಚರ್ಚಿನಲ್ಲಿ ಸಾಯುವುದೆಂದರೆ ಅದೊಂದು ದೌರ್ಭಾಗ್ಯವೆನ್ನುವುದನ್ನು ನಾವು ಯಾವಾಗಲೂ ಜ್ಞಾಪಕದಲ್ಲಿಟ್ಟುಕೊಂಡಿರಬೇಕು. ಈ ರೂಪಗಳ ಪರಿಮಿತಿಯಲ್ಲಿಯೇ ಒಬ್ಬ ಮನುಷ್ಯ ಸಾಯುವಂತಾದರೆ, ಅದು ಸೂಚಿಸುವುದು ಅವನು ಬೆಳೆದಿಲ್ಲ, ಅವನಲ್ಲಿ ಸತ್ಯ ಅನಾವೃತವಾಗಿಲ್ಲ, ದಿವ್ಯತೆ ಅರಳಿಲ್ಲ ಎನ್ನುವುದನ್ನೇ.

ನಿಜವಾದ ಪ್ರೇಮವನ್ನು ಒಂದು ತ್ರಿಕೋನವನ್ನಾಗಿ ಭಾವಿಸಬಹುದು. ಮೊದಲನೆಯ ಕೋನವೇ, ಪ್ರೇಮಕ್ಕೆ ಚೌಕಾಶಿ ಗೊತ್ತಿಲ್ಲ ಎನ್ನುವುದು. ಆದ್ದರಿಂದ, ‘ನನಗೆ ಅದನ್ನು ಕೊಡು, ಇದನ್ನು ಕೊಡು’ ಎಂದು ಮನುಷ್ಯನೊಬ್ಬ ಪ್ರಾರ್ಥಿಸುತ್ತಿದ್ದರೆ, ಅದು ಪ್ರೇಮವಲ್ಲ. ಹೇಗಾದೀತು? ‘ನಾನು ನನ್ನ ಈ ಸಣ್ಣ ಪ್ರಾರ್ಥನೆಯನ್ನು ಸಲ್ಲಿಸುತ್ತಿರುವೆ, ಪ್ರತಿಯಾಗಿ ನೀನು ನನಗೆ ಇಂಥದನ್ನು ಕೊಡು’ - ಇದು ಕೇವಲ ಅಂಗಡಿ ವ್ಯಾಪಾರ. ಎರಡ ನೆಯ ಕೋನವೇ, ಪ್ರೇಮಕ್ಕೆ ಭಯವೆಂಬುದಿಲ್ಲ ಎನ್ನುವುದು. ದೇವರನ್ನು ಎಲ್ಲಿಯವರೆಗೆ ಶಿಕ್ಷೆ ಕೊಡುವವನು ಎಂದೋ ಭಾಗ್ಯ ನೀಡುವವನು ಎಂದೋ ಭಾವಿಸುವೆವೋ ಅಲ್ಲಿಯವರೆಗೆ ಅವನ ಮೇಲೆ ಪ್ರೇಮವಿರಲು ಸಾಧ್ಯವಿಲ್ಲ. ಮೂರನೆಯ ಕೋನವೇ ಪ್ರೇಮ ಯಾವಾಗಲೂ ಉತ್ತಮೋತ್ತಮ ಆದರ್ಶ ಎನ್ನುವುದು. ಆದರ್ಶವನ್ನು ಆದರ್ಶವನ್ನಾ ಗಿಯೆ ಆರಾಧಿಸಬಲ್ಲ ಈ ಸ್ತರವನ್ನು ನಾವು ತಲುಪಿದಾಗ, ಎಲ್ಲವಾದವಿವಾದಗಳೂ ಸಂದೇಹಗಳೂ ಮತ್ತೆಂದೂ ಹಿಂದಿರುಗದಂತೆ ಮಾಯವಾಗಿರುವುವು. ಆದರೆ ನಮ್ಮ ಸ್ವಭಾವದಲ್ಲೇ ಹಾಸುಹೊಕ್ಕಾಗಿರುವ ಆದರ್ಶವು ಮಾತ್ರ ಎಂದಿಗೂ ತಪ್ಪಿಸಿಕೊಳ್ಳಲಾರದು.

ಹಾರ್ವರ್ಡ್ ವಿಶ್ವವಿದ್ಯಾನಿಲಯದಲ್ಲಿ ಕೊಟ್ಟ ತಮ್ಮ ಉಪನ್ಯಾಸದಲ್ಲಿ ಸ್ವಾಮಿಗಳು ವೇದಾಂತ ತತ್ತ್ವದ ಇತಿಹಾಸವನ್ನು ತಿಳಿದಿರುವಷ್ಟೂ ಚಿತ್ರಿಸಿದರು; ವೇದಗಳು (ಅಥವಾ ಹಿಂದೂ ಶಾಸ್ತ್ರಗ್ರಂಥಗಳು) ಎಲ್ಲಿಯವರೆಗೆ ಅಧಿಕೃತವೆಂದು ಪರಿಗಣಿಸಲ್ಪಟ್ಟಿವೆ, ಎಷ್ಟರ ಮಟ್ಟಿಗೆ ವಿವೇಚನೆಗೆ ಹೊಂದುತ್ತವೆಯೋ ಅಷ್ಟರಮಟ್ಟಿಗೆ ಮಾತ್ರ ಹೇಗೆ ಅವುಗಳನ್ನು ತತ್ತ್ವದ ತಳಹದಿಯನ್ನಾಗಿ ಪರಿಗಣಿಸಲಾಗುತ್ತದೆ ಎನ್ನುವುದನ್ನು ವಿವರಿಸಿದರು. ವೇದಾಂತದ ಮೂರು ಮತಗಳನ್ನು ಅವರು ಹೋಲಿಸಿದರು - ಈಶ್ವರ ಹಾಗೂ ಮಾನವನಲ್ಲಿ ಅಭಿವ್ಯಕ್ತಿಗೊಳ್ಳುವ ಆದರೆ ಚಿರಂತನವಾಗಿ ಈ ಈಶ್ವರನಿಂದ ಬೇರೆಯಾಗಿರುವ ಜೀವ ಇವೆರಡನ್ನು ಸಮರ್ಥಿಸುವವನನ್ನು ದ್ವೈತಿಗಳು ಎಂದರು. ಅನಂತರ - ದೇವರು ಮತ್ತು ಪ್ರಕೃತಿ ಎಂಬ ಎರಡು ಇವೆಯಾದರೂ, ಜೀವ ಹಾಗೂ ಪ್ರಕೃತಿಗಳು ಈ ದೇವರ ವಿಕಸಿತ ಶರೀರ, ನಮ್ಮ ಜೀವಕ್ಕೊಂದು ಭೌತಿಕ ಶರೀರವಿದ್ದಹಾಗೆ - ಎನ್ನುವ ವಿಶಿಷ್ಟಾದ್ವೈತಿಗಳ ಮತವನ್ನು ವಿವರಿಸಿದರು. ಈವಾದಕ್ಕೆ ಸಮರ್ಥನೆಯಾಗಿ ಅವರು ಪರಿಣಾಮವು ಕಾರಣದಿಂದ ಭಿನ್ನವಾಗಿರುವುದಿಲ್ಲ, ಕಾರಣವೇ ಪರಿಣಾಮವೆಂಬ ಇನ್ನೊಂದು ರೂಪದಿಂದ ಅಭಿವ್ಯಕ್ತವಾಗುವುದು ಎಂದುವಾದಿಸುವರು. ಹೀಗೆ, ದೇವರು ಈ ವಿಶ್ವಕ್ಕೆ ಕಾರಣನಾದ್ದರಿಂದ, ಅವನೇ ಪರಿಣಾಮವಾದ ಈ ಪ್ರಪಂಚವೂ ಆಗಿರುವನು ಎನ್ನುವರು. ಅದ್ವೈತಿಗಳು... ದೇವರು ಇರುವುದೇ ಹೌದಾದರೆ, ಅವನು ಈ ವಿಶ್ವಕ್ಕೆ ಉಪಾದಾನ ಕಾರಣನಾಗಿರು ವುದಲ್ಲದೆ ನಿಮಿತ್ತ ಕಾರಣನೂ ಆಗಿರಬೇಕು ಎಂದು ಉದ್ಘೋಷಿಸುವರು. ಅವನು ಸೃಷ್ಟಿಕರ್ತ ಮಾತ್ರವಲ್ಲ, ಸೃಷ್ಟಿಯೂ ಸಹ ಆಗಿರುವನು. ಈ ವಿಶ್ವವೆಲ್ಲವೂ ಅವನೇ; ಆದರೆ ವಾಸ್ತವವಾಗಿ ಈ ವಿಶ್ವ ಅಸ್ತಿತ್ವದಲ್ಲಿಲ್ಲ - ಅದು ಕೇವಲ ಭ್ರಮೆ ಮಾತ್ರ. ಭೇದ ಇರುವುದು ನಾಮರೂಪಗಳಲ್ಲಿ ಮಾತ್ರ. ಇಡಿಯ ವಿಶ್ವದಲ್ಲಿರುವುದು ಒಂದೇ ಆತ್ಮ, ಎರಡಲ್ಲ; ಏಕೆಂದರೆ ಯಾವುದು ದ್ರವ್ಯವಸ್ತುವಲ್ಲವೋ ಅದಕ್ಕೆ ಪರಿಮಿತಿಯಿರದು, ಅದು ಅನಂತವಾಗಿರಬೇಕು. ಎರಡು ಅನಂತಗಳಿರಲು ಸಾಧ್ಯವಿಲ್ಲ; ಏಕೆಂದರೆ ಒಂದು ಇನ್ನೊಂದನ್ನು ಮಿತಿಗೊಳಪಡಿಸುವುದು. ಆತ್ಮವು ತನ್ನಲ್ಲಿ ತಾನು ಶುದ್ಧವಾದದ್ದು; ಆದರೆ ಕೇಡು ಇರುವಂತೆ ಕಾಣುವುದು. ಶುದ್ಧವೂ ನಿಷ್ಕಲ್ಮಷವೂ ಆದ ಸ್ಘಟಿಕವು ಅದರ ಮುಂದೆ ಪುಷ್ಪವನ್ನಿಟ್ಟಾಗ ಪುಷ್ಪದ ಬಣ್ಣಗಳನ್ನು ತೋರಿಸುವ ಹಾಗೆ.

ಮನಶ್ಶಾಸ್ತ್ರೀಯವಾಗಿ ದೇವರೊಂದಿಗೆ ಐಕ್ಯಹೊಂದುವ ಮಾರ್ಗವಾದ ರಾಜಯೋಗ\footnote{1. ವಾಸ್ತವವಾಗಿ ‘ಸಾಕ್ಷಾತ್ಕಾರ, ಅಥವಾ ಧರ್ಮದ ಆತ್ಯಂತಿಕತೆ’ ಎಂಬ ವಿಷಯದ ಕುರಿತಾದ ಉಪನ್ಯಾಸ ವಿದು. ಇದರ ಪದಶಃ ವರದಿ ಲಭ್ಯವಿಲ್ಲ.}ವನ್ನು ಚರ್ಚಿಸುವಾಗ, ಸ್ವಾಮಿಗಳು ಮನಸ್ಸನ್ನು ಏಕಾಗ್ರಗೊಳಿಸುವ ಮೂಲಕ ಅದಕ್ಕೆ ಭೌತಿಕ ಮತ್ತು ಆಧ್ಯಾತ್ಮಿಕ ಪ್ರಪಂಚಗಳೆರಡರಲ್ಲೂ ಒದಗಿಬರುವ ಶಕ್ತಿಯನ್ನು ಕುರಿತು ದೀರ್ಘವಾಗಿ ವ್ಯಾಖ್ಯಾನ ಮಾಡಿದರು. ನಮ್ಮೆಲ್ಲ ಜ್ಞಾನದಲ್ಲಿಯೂ ನಮಗೆ ಲಭ್ಯವಿರುವುದು ಇದೊಂದೇ ವಿಧಾನ. ಸೂಕ್ಷ್ಮ ಕ್ರಿಮಿಯಿಂದ ಹಿಡಿದು ಅತ್ಯುನ್ನತ ಸಾಧಕನಾದ ಋಷಿಯವರೆಗೆ, ನೀಚಸ್ತರಿಂದ ಉಚ್ಚ ಸ್ತರದವರೆಗೆ, ಪ್ರತಿಯೊಬ್ಬರೂ ಈ ಒಂದು ವಿಧಾನವನ್ನೇ ಬಳಸಬೇಕಾಗುವುದು. ಅಂತರಿಕ್ಷದಲ್ಲಿನ ನಿಗೂಢ ರಹಸ್ಯಗಳನ್ನು ಆವಿಷ್ಕರಿಸುವುದಕ್ಕೆ ಖಗೋಳವಿಜ್ಞಾನಿ, ಪ್ರಯೋಗಾಲಯದಲ್ಲಿ ರಸಾಯನ ವಿಜ್ಞಾನಿ, ಪೀಠದಲ್ಲಿ ಕುಳಿತಿರುವ ಪ್ರಾಧ್ಯಾಪಕ - ಎಲ್ಲರೂ ಬಳಸುವುದು ಈ ವಿಧಾನವನ್ನೇ. ಪ್ರಕೃತಿ ತನ್ನ ಬಾಗಿಲನ್ನು ತೆರೆದು ತನ್ನೊಳಗಿರುವ ಜ್ಯೋತಿ ಪ್ರವಾಹವನ್ನು ಹೊರಗೆ ಬಿಡಬೇಕಾದರೆ ಬೇಕಾಗಿರುವುದು ಇದೊಂದೇ ಕರೆ, ಇದೊಂದೇ ಬಡಿತ. ಇದೊಂದೇ ಒಂದು ಬೀಗದ ಕೈ, ಏಕಮಾತ್ರ ಶಕ್ತಿ - ಈ ಏಕಾಗ್ರತೆ. ಪ್ರಸ್ತುತ ನಮ್ಮ ಶರೀರಗಳು ಇರುವ ಸ್ಥಿತಿಯಲ್ಲಿ ನಾವೆಷ್ಟು ವಿಚಲಿತರಾಗಿರುವೆವು ಎಂದರೆ ನಮ್ಮ ಮನಸ್ಸಿನ ಶಕ್ತಿ ನೂರಾರು ರೀತಿಯ ಸಂಗತಿಗಳ ಮೇಲೆ ಹರಿದು ಹಂಚಿಹೋಗಿರುತ್ತದೆ. ದೇಹವನ್ನು ದುಡಿಸುವ ಬಲಗಳ ವೈಜ್ಞಾನಿಕ ನಿಯಂತ್ರಣದ ಮೂಲಕ ಏಕಾಗ್ರತೆಯನ್ನು ಸಾಧಿಸಬಹುದು; ಮತ್ತು ಅದರ ಕೊಟ್ಟಕೊನೆಯ ಫಲವೇ ಸಾಕ್ಷಾತ್ಕಾರ. ಧರ್ಮ ಎಂದರೆ ಕೇವಲ ಮಾತಲ್ಲ. ಅದು ಧರ್ಮ ಎನ್ನಿಸಿಕೊಳ್ಳುವುದು ತಲಸ್ಪರ್ಶಿಯಾದಾಗ; ನಾವು ಅಷ್ಟೊಂದು ಮಾತನಾಡುವ ಸಂಗತಿಯನ್ನು ನಮ್ಮ ಸಂವೇದನೆಯಾಗಿಸಿಕೊಳ್ಳಲು ಹೆಣಗುವವರೆಗೆ ನಾವು ಅಜ್ಞೇಯತಾವಾದಿಗಳಿಗಿಂತ ಹೆಚ್ಚೇನಲ್ಲ; ಏಕೆಂದರೆ ಅವರು ಪ್ರಾಮಾಣಿಕವಾಗಿಯಾದರೂ ಇರುತ್ತಾರೆ, ನಾವು ಅಷ್ಟೂ ಇರುವುದಿಲ್ಲ.

ಶನಿವಾರ (ಮಾರ್ಚ್ ೨೮) ಸ್ವಾಮಿಗಳು ಟ್ವೆಂಟೀಯಥ್ ಸೆಂಚುರಿ ಕ್ಲಬ್ನವರ ಅತಿಥಿಯಾಗಿದ್ದರು; ಅವರಿಂದ “ವೇದಾಂತ ತತ್ತ್ವದ ಪ್ರಾಯೋಗಿಕ ಭಾಗ”\footnote{1. ಈ ಉಪನ್ಯಾಸದ ವಿವರ ಹಾಗೂ ಚರ್ಚೆಗಾಗಿ ನೋಡಿ ಕೃತಿಶ್ರೇಣಿ ೬, ಪು. ೩೫೩} ಎಂಬ ಉಪನ್ಯಾಸವನ್ನು ಕ್ಲಬ್ ಆಲಿಸಿತು. ಅವರು ಇಂದು ಬೋಸ್ಟನ್ನಿಂದ ನಿರ್ಗಮಿಸುವರು; ಮತ್ತು ಕೆಲವೇ ದಿನಗಳಲ್ಲಿ ಇಂಗ್ಲೆಂಡಿನ ಮಾರ್ಗವಾಗಿ ಭಾರತಕ್ಕೆ ಪಯಣಿಸುವರು.

\begin{center}
\textbf{ಸ್ವಾಮಿ ವಿವೇಕಾನಂದರು\supskpt{\footnote{\enginline{1. New Discoveries, Vol. 5, pp.184-186}}}}
\end{center}

\begin{center}
\textbf{ಹಿಂದೂ ಧರ್ಮ ಮತ್ತು ತತ್ತ್ವಗಳನ್ನು ಕುರಿತು ಉಪನ್ಯಾಸ ಮಾಡಿದರು}
\end{center}

\begin{center}
(ಲಾಸ್ ಏಂಜಲಿಸ್ ಟೈಮ್ಸ್, ೯ ಡಿಸೆಂಬರ್ ೧೮೯೯)
\end{center}

.... ಬ್ರಾಹ್ಮಣ ಜಾತಿಯ ಹಳದಿ ಉಡುಪಿನಲ್ಲಿದ್ದ ಪ್ರಖ್ಯಾತ ಹಿಂದೂ ತತ್ತ್ವಜ್ಞಾನದ ಪ್ರತಿಪಾದಕರು, ಭಾಗಶಃ ಹೀಗೆಂದು ಮಾತನಾಡಿದರು\footnote{3. ಇದು ಸ್ವಾಮಿ ವಿವೇಕಾನಂದರು ‘ವೇದಾಂತ ತತ್ತ್ವ ಅಥವಾ ಹಿಂದೂ ತತ್ತ್ವ ಧರ್ಮವಾಗಿ’ ಎಂಬ ವಿಷಯದ ಕುರಿತು ಕ್ಯಾಲಿಫೋರ್ನಿಯಾದಲ್ಲಿ ಮಾಡಿದ ಮೊಟ್ಟ ಮೊದಲನೆಯ ಉಪನ್ಯಾಸ. ಇದರ ಪದಶಃ ವರದಿ ಲಭ್ಯವಿಲ್ಲ.}:

“ಮಹಿಳೆಯರೆ ಮತ್ತು ಮಹನೀಯರೆ, ನಾನು ನಿಮ್ಮ ಮುಂದೆ ಬಂದಿರುವುದು ಯಾವುದೋ ಹೊಸ ಧರ್ಮವನ್ನು ಬೋಧಿಸುವುದಕ್ಕಲ್ಲ. ನಾನು ಎಲ್ಲ ಧರ್ಮಗಳನ್ನೂ ಒಂದಾಗಿಸುವ ಕೆಲವು ಅಂಶಗಳನ್ನು ನಿಮಗೆ ತಿಳಿಸಲು ಅಪೇಕ್ಷಿಸುತ್ತೇನೆ ಅಷ್ಟೆ. ನಿಮಗೆ ವಿಚಿತ್ರವೆಂದು ಕಾಣಬಹುದಾದ ಪ್ರಾಚ್ಯ ನಾಗರಿಕತೆಯಲ್ಲಿನ ಕೆಲವು ಚಿಂತನೆಗಳನ್ನು ಸಂಕ್ಷೇಪವಾಗಿ ನಿಮ್ಮ ಮುಂದಿರಿಸುತ್ತೇನೆ; ನಿಮಗೆ ಪ್ರಿಯವಾಗಬಹುದಾದ ಕೆಲವು ಅಂಶಗಳನ್ನೂ ಹೇಳುತ್ತೇನೆ. ಲೋಕದ ಎಲ್ಲ ಧರ್ಮಗಳೂ ಐಕ್ಯತೆಯ ಒಂದು ಬೆನ್ನೆಲುಬನ್ನು ಹೊಂದಿವೆ. ಇದೇ ತತ್ತ್ವದ ಹಾಗೂ ಸಹಿಷ್ಣುತೆಯ ಮೂಲಭೂತ ಅಂಶ.

ಭಾರತ ಎಂದರೇನೆಂಬುದು ಈ ದೇಶದ ಎಲ್ಲೋ ಕೆಲವರಿಗೆ ಮಾತ್ರವೇ ಅರ್ಥವಾಗುತ್ತದೆ. ಅದು ಅನೇಕ ಭಾಷೆಗಳನ್ನಾಡುವ, ಆದರೆ ಒಂದು ಸಾಮಾನ್ಯ ಧರ್ಮದ ಕಲ್ಪನೆಗಳಿಂದಾಗಿ ಒಂದಾಗಿರುವ, ಮೂವತ್ತು ಕೋಟಿ ಜನರು ವಾಸಿಸುತ್ತಿರುವ, ಸಂಯುಕ್ತ ಸಂಸ್ಥಾನಗಳ ಕೇವಲ ಅರ್ಧದಷ್ಟು ವಿಸ್ತಾರವಾಗಿರುವ ಒಂದು ದೇಶ. ಪಾಶ್ಚಾತ್ಯ ನಾಗರಿಕತೆ ಶಸ್ತ್ರಾಸ್ತ್ರ ಬಲದಿಂದ ಜಯಿಸುತ್ತಿರುವಾಗ, ಹಿಂದೂಗಳು ಈ ಕಲ್ಪನೆಗಳಿಂದಾ ಗಿಯೇ ಯುಗಯುಗಗಳಿಂದ ಮೌನವಾಗಿ, ಸೌಮ್ಯವಾಗಿ, ತಾಳ್ಮೆಯಿಂದ, ತಮ್ಮ ಪ್ರಭಾವವನ್ನು ಬೀರುತ್ತ ಬಂದಿರುವರು. ಯಾವುದು ಹೆಚ್ಚು ಶಕ್ತಿಶಾಲಿಯಾದುದು - ಭೌತಿಕ ಬಲವೋ, ಚಿಂತನೆಯ ಬಲವೋ - ಎಂಬುದನ್ನು ಭವಿಷ್ಯವು ತೋರಿಸಿಕೊಡಲಿರುವುದು. ಅವರ ಅಂಕಿಗಳು, ಗಣಿತ ಚಿಂತನೆ, ಅವರ ನೈತಿಕತೆ, ಹಿಂದೂಗಳ ವಿಜ್ಞಾನ ಹಾಗೂ ಕಲೆಗಳು ಪ್ರಪಂಚದಾದ್ಯಂತ ಹರಡಿರುವುವು - ಭಾರತದಲ್ಲಿ ಮಾತ್ರವೇ, ಅಲ್ಲಿ ಎಂದರೆ ಅಲ್ಲಿ ಮಾತ್ರವೇ, ಅಲ್ಲವೆ ಪ್ರೇಮವೆಂಬ ಸಿದ್ಧಾಂತವನ್ನು ಮೊಟ್ಟ ಮೊದಲು ಬೋಧಿಸಿದ್ದು - ಸಹಮಾನವರ ಮೇಲಿನ ಪ್ರೇಮವನ್ನು ಮಾತ್ರವಲ್ಲ, ಎಲ್ಲ ಜೀವಿಗಳ ಮೇಲೂ ಪ್ರೇಮವನ್ನು, ಹೌದು, ನಿಮ್ಮ ಪಾದದ ಅಡಿಯಲ್ಲಿ ತೆವಳುತ್ತಿರುವ ಅತ್ಯಂತ ಕೀಳು ಕ್ರಿಮಿಯ ಮೇಲೂ. ಭಾರತದ ಕಲೆಯನ್ನು ಹಾಗೂ ಸಂಸ್ಥೆಗಳನ್ನು ಅಧ್ಯಯನ ಮಾಡಲು ಎಂದು ನೀವು ಪ್ರಾರಂಭಿಸುವಿರೋ, ಅಂದು ನೀವು ಮೋಡಿಗೊಳಗಾಗುವಿರಿ, ಆಕರ್ಷಿತರಾಗುವಿರಿ. ಅದರಿಂದ ನೀವು ಖಂಡಿತ ತಪ್ಪಿಸಿಕೊಳ್ಳಲಾರಿರಿ.

“ಇತರೆಡೆಗಳಲ್ಲಿರುವ ಹಾಗೆಯೇ, ಭಾರತದಲ್ಲಿಯೂ ಜನರು ಹಿಂದಿನಿಂದ ಅನೇಕ ಪುಟ್ಟ ಪುಟ್ಟ ಬುಡಕಟ್ಟುಗಳಾಗಿ ವಿಂಗಡವಾಗಿರುವುದನ್ನು ನಾವು ನೋಡುತ್ತೇವೆ. ಈ ಬೇರೆ ಬೇರೆ ಬುಡಕಟ್ಟಿನ ಜನರು ತಮ್ಮದೇ ಬೇರೆಯ ದೇವರುಗಳನ್ನು, ತಮ್ಮದೇ ಬೇರೆಯ ಕ್ರಿಯಾವಿಧಿಗಳನ್ನು ಹೊಂದಿದ್ದರು. ಆದರೆ ಪರಸ್ಪರ ಸಂಪರ್ಕ ಏರ್ಪಟ್ಟಾಗ ಈ ಬುಡಕಟ್ಟುಗಳು ಪಾಶ್ಚಿಮಾತ್ಯ ನಾಗರಿಕತೆ ತುಳಿದ ಹಾದಿಯನ್ನು ತುಳಿಯಲಿಲ್ಲ - ಈ ವ್ಯತ್ಯಾಸಗಳಿಂದಾಗಿ ಅವರು ಒಬ್ಬರನ್ನೊಬ್ಬರು ಹಿಂಸಿಸಿ ಕೊಲ್ಲಲಿಲ್ಲ. ಎಲ್ಲ ಧರ್ಮಗಳಲ್ಲಿಯೂ ಇರುವ ಸಾಮಾನ್ಯ ಕಲ್ಪನೆಗಳ ಬೀಜಗಳನ್ನು ಹುಡುಕಲು ಪ್ರಯತ್ನಿಸಿದರು. ಈ ಪ್ರಯತ್ನಗಳಿಂದಾಗಿ ಸಹಿಷ್ಣುತೆಯ ಅಭ್ಯಾಸ ಬೆಳೆಯಿತು; ಇದೇ ಭರತ ಖಂಡದ ಧರ್ಮದ ಮೂಲಮಂತ್ರ. ಸತ್ಯವು ಒಂದೇ, ಬೇರೆ ಬೇರೆ ಭಾಷೆಗಳಲ್ಲಿ ಅದನ್ನು ಅಭಿವ್ಯಕ್ತಿಸುವುದಾದರೂ ಅದು ಒಂದೇ ಅಲ್ಲದೆ ಬೇರೆಯಲ್ಲ.

“ಪ್ರಾಚ್ಯ ಹಾಗೂ ಪಾಶ್ಚಾತ್ಯ ಧರ್ಮಗಳಲ್ಲಿನ ಇನ್ನೊಂದು ಮಹತ್ತರವಾದ ವ್ಯತ್ಯಾಸವಿರುವುದು ವಿಶ್ವದ ವೈಜ್ಞಾನಿಕ ಮತ್ತು ತಾತ್ತ್ವಿಕ ದೃಷ್ಟಿಯನ್ನು ಒಪ್ಪಿಕೊಳ್ಳುವುದರಲ್ಲಿ. ಇತ್ತೀಚಿನ ವರ್ಷಗಳಲ್ಲಿ ಪಾಶ್ಚಿಮಾತ್ಯರಲ್ಲಿ ಅಜ್ಞೇಯತಾವಾದ ಬೆಳೆಯುತ್ತಿದೆ; ತಾವು ಬಯಸುವ ಹಾಗೂ ಹುಡುಕುತ್ತಿರುವ ವೈಯಕ್ತಿಕ ಅಮೃತತ್ವದ ಭರವಸೆ ಕಾಣದೆ ಹೋದಾಗ, ಒಂದು ನಿರಾಸೆಯ ಧ್ವನಿ ಅವರ ಚಿಂತನೆಯಲ್ಲಿ ಹೊರಹೊಮ್ಮಿದೆ. ವಿಶ್ವವು ಒಂದು ನಿಯಮಕ್ಕನುಗುಣವಾಗಿ ನಡೆಯುತ್ತಿದೆ, ಎಲ್ಲವೂ ಬದಲಾಗತಕ್ಕುದು ಎಂಬುದೇ ಈ ನಿಯಮ ಎಂಬುದನ್ನು ಹಿಂದೂವು ಯುಗಗಳ ಹಿಂದೆಯೇ ಅರಿತುಕೊಂಡಿರುವನು. ಆದ್ದರಿಂದ, ಅವಿನಾಶಿಯಾದ ವೈಯಕ್ತಿಕತೆ ಎಂಬುದು ಒಂದು ಅಸಾಧ್ಯವಾದ ಸಂಗತಿ. ಆದರೆ ಈ ಚಿಂತನೆ ಹಿಂದೂವಿಗೆ ನಿರಾಸೆಯನ್ನುಂಟುಮಾಡದು. ಬದಲಿಗೆ - ಇದುವೇ ಪೌರ್ವಾತ್ಯ ಚಿಂತನೆಯಲ್ಲಿ ಪಾಶ್ಚಾತ್ಯರಿಗೆ ಅರ್ಥವಾಗದ್ದು - ಅವನು ಇಂದ್ರಿಯಗಳ ಜಂಜಡದಿಂದ, ನೋವು ನಲಿವುಗಳ ಜಂಜಡದಿಂದ ಮುಕ್ತನಾಗುವುದಕ್ಕೆ ಬಯಸುತ್ತಾನೆ.

“ಪಾಶ್ಚಾತ್ಯ ನಾಗರಿಕತೆ ಒಂದು ಇಷ್ಟದೇವತೆಯನ್ನು ಇಟ್ಟುಕೊಳ್ಳಲು ಬಯಸಿತು; ಅದರಲ್ಲಿ ನಂಬಿಕೆ ಕಳೆದುಕೊಂಡಾಗ ನಿರಾಸೆಯನ್ನು ಅನುಭವಿಸಿತು. ಹಿಂದೂ ಕೂಡ ಅಂಥದನ್ನು ಹುಡುಕಿದ್ದಾನೆ. ಆದರೆ ದೇವರು ಬಾಹ್ಯ ಇಂದ್ರಿಯಗಳ ಅರಿವಿಗೆ ನಿಲುಕುವಂಥದಲ್ಲ. ಅನಂತವನ್ನು, ನಿರಪೇಕ್ಷವನ್ನು ಗ್ರಹಿಸುವುದಕ್ಕೆ ಸಾಧ್ಯವಿಲ್ಲ. ಅದು ನಮ್ಮಿಂದ ನುಣುಚಿಕೊಳ್ಳುವುದಾದರೂ, ಅದು ಅಸ್ತಿತ್ವದಲ್ಲಿಲ್ಲ ಎಂಬ ನಿರ್ಣಯಕ್ಕೆ ಬರಬೇಕಾಗಿಲ್ಲ. ಅದು ಇದ್ದೇ ಇದೆ. ನಮ್ಮ ಹೊರಗಿನ ಕಣ್ಣಿಗೆ ನೋಡಲು ಸಿಕ್ಕದಿರುವುದು ಯಾವುದು? ಅದು ನಮ್ಮ ಕಣ್ಣೇ ಹೊರತು ಇನ್ನೇನೂ ಅಲ್ಲ. ಅದು ಎಲ್ಲವನ್ನೂ ಗ್ರಹಿಸಬಲ್ಲುದಾದರೂ, ತನ್ನನ್ನೇ ತಾನು ಪ್ರತಿಫಲಿಸಿಕೊಳ್ಳಲಾರದು. ಎಂದ ಮೇಲೆ, ಸಮಸ್ಯೆಗೆ ಇದೇ ಪರಿಹಾರ. ಭಗವಂತನು ಬಾಹ್ಯ ಇಂದ್ರಿಯಗಳಿಗೆ ಸಿಕ್ಕದೇ ಹೋದರೆ, ನಿಮ್ಮ ಚಕ್ಷುವನ್ನು ಅಂತರಂಗದ ಕಡೆಗೆ ತಿರುಗಿಸಿ, ಅಲ್ಲಿ ನಿಮ್ಮೊಳಗೇ ಸಮಸ್ತ ಆತ್ಮಗಳಿಗೂ ಆತ್ಮನಾಗಿರುವವನು ಇರುವುದನ್ನು ಕಾಣುವಿರಿ. ಮಾನವನಲ್ಲಿಯೇ ಎಲ್ಲವೂ ಇದೆ. ಮೂಲಭೂತ ಸತ್ಯವನ್ನು ನಾನು ತಿಳಿದುಕೊಳ್ಳಲಾರೆ ಏಕೆಂದರೆ ನಾನೇ ಮೂಲಭೂತ ಸತ್ಯವಾಗಿರುವೆ. ಇಲ್ಲಿ ದ್ವೈತವೆನ್ನುವುದು ಇಲ್ಲವೇ ಇಲ್ಲ. ಇದೇ ಎಲ್ಲ ತಾತ್ತ್ವಿಕ ಹಾಗೂ ನೈತಿಕ ಪ್ರಶ್ನೆಗಳಿಗೂ ಉತ್ತರ. ನಮ್ಮಲ್ಲಿರುವ ಪರಹಿತಚಿಂತನೆಗೆ ಕಾರಣವಾದರೂ ಏನು ಎಂಬುದನ್ನು ಅರಿಯುವುದಕ್ಕೆ ಪಾಶ್ಚಾತ್ಯ ನಾಗರಿ ಕತೆ ವ್ಯರ್ಥವಾಗಿ ಪ್ರಯತ್ನಿಸಿದೆ. ಇಲ್ಲಿದೆ ಅದು. ನನ್ನ ಸೋದರ ನಾನೇ, ಅವನ ನೋವು ನನ್ನ ನೋವೇ. ನನ್ನನ್ನು ನಾನು ನೋಯಿಸಿಕೊಳ್ಳದೆ ಅವನನ್ನು ನೋಯಿಸಲಾರೆ; ನನ್ನ ಆತ್ಮಕ್ಕೆ ಹಾನಿ ಮಾಡಿಕೊಳ್ಳದೆ ಇತರ ಜೀವಿಗಳಿಗೆ ನಾನು ಹಾನಿಯುಂಟುಮಾಡಲಾರೆ. ನಾನೇ ನಿರಪೇಕ್ಷನಾಗಿರುವೆ ಎನ್ನುವುದನ್ನು ಅರಿತುಕೊಂಡಾಗ, ನನಗೆ ಬದುಕಾಗಲಿ ಸಾವಾಗಲಿ ಇಲ್ಲ, ನೋವಾಗಲಿ ನಲಿವಾಗಲಿ ಇಲ್ಲ, ಜಾತಿಯಾಗಲಿ ಲಿಂಗವಾಗಲಿ ಇಲ್ಲ. ನಿರಪೇಕ್ಷವಾಗಿರುವಂಥದು ಹುಟ್ಟುವುದು ಹೇಗೆ, ಸಾಯುವುದು ಹೇಗೆ? ಒಂದು ಪುಸ್ತಕದ ಪುಟಗಳಂತೆ ಪ್ರಕೃತಿಯ ಪುಟಗಳು ನಮ್ಮೆದುರು ತೆರೆದುಕೊಳ್ಳುತ್ತಿರುತ್ತವೆ; ನಾವು ನಾವೇ ತಿರುಗುತ್ತಿರುವೆವು ಎಂದುಕೊಳ್ಳುತ್ತೇವೆ; ನಿಜ ಏನೆಂದರೆ ನಾವು ಚಿರಂತನವಾಗಿ ಇದ್ದ ಹಾಗೇ ಇರುತ್ತೇವೆ.”

\begin{center}
\textbf{ಹಿಂದೂ ತತ್ತ್ವಜ್ಞಾನ\supskpt{\footnote{\enginline{1. New Discoveries, Vol. 5, pp.194-95}}}\\ದೂರದ ಭಾರತದಲ್ಲಿ ವಿಶ್ವದ ಬಗೆಗಿನ ಕಲ್ಪನೆ }
\end{center}

\begin{center}
(ಲಾಸ್ ಏಂಜಲಿಸ್ ಟೈಮ್ಸ್, ೧೩ ಡಿಸೆಂಬರ್ ೧೮೯೯)
\end{center}

ಕಳೆದ ಸಂಜೆ ಯೂನಿಟಿ ಚರ್ಚ್ನಲ್ಲಿ ದಕ್ಷಿಣ ಕ್ಯಾಲಿಫೋರ್ನಿಯಾ ಅಕಾಡೆಮಿ ಆಫ್ ಸೈನ್ಸಸ್ನವರ ಮಾಸಿಕ ಸಭೆಯನ್ನು ಉದ್ದೇಶಿಸಿ ಹಿಂದೂ ತತ್ತ್ವಶಾಸ್ತ್ರಜ್ಞ ಸ್ವಾಮಿ ವಿವೇಕಾನಂದರು ಮಾತನಾಡಿದರು\footnote{2. ಲಾಸ್ ಏಂಜಲಿಸ್ ಹೆರಾಲ್ಡ್ ಪತ್ರಿಕೆಯಲ್ಲಿನ ಇದೇ ಉಪನ್ಯಾಸದ ಡಿಸೆಂಬರ್ ೧೩. ೧೮೯೯ರ ಪತ್ರಿಕಾ ವರದಿಯನ್ನು ನೋಡಿ. (ಮುಂದಿನ ಪುಟ ನೋಡಿ)}. ಮೆಚ್ಚುಗೆಯಿಂದ ಸೇರಿದ್ದ ಸಭೆ ದೊಡ್ಡದಾಗಿಯೇ ಇತ್ತು; ಉಪನ್ಯಾಸವಾದ ಮೇಲೆ ಸಭಿಕರಿಂದ ಅನೇಕಾನೇಕ ಪ್ರಶ್ನೆಗಳು ಕೇಳಲ್ಪಟ್ಟವು ಹಾಗೂ ಉಪನ್ಯಾಸಕರು ಅವುಗಳನ್ನೆಲ್ಲ ಉತ್ತರಿಸಿದರು....

ಹಿಂದೂಗಳು ಹೇಗೆ ವಿಶ್ವದ ಉಗಮವನ್ನು ವಿವರಿಸಲು ಪ್ರಯತ್ನಿಸಿದರು ಎನ್ನುವುದಕ್ಕೆ ಕೆಲವು ಪುರಾಣಕಥೆಗಳನ್ನು ಉದ್ಧರಿಸುತ್ತ ಭಾಷಣಕಾರರು ಉಪನ್ಯಾಸವನ್ನು ಪ್ರಾರಂಭಿಸಿದರು. ಪುರಾತನ ಹಿಂದೂಗಳು ತಮ್ಮಸುತ್ತಮುತ್ತಲಿನ ನಿಗೂಢ ಅಚ್ಚರಿಗಳನ್ನು ವಿವರಿಸಲು ಹೇಗೆ ಪ್ರಯತ್ನಿಸಿದರು ಎನ್ನುವುದನ್ನೂ ಹೇಳಿದರು.

ಅವರ ನಂಬಿಕೆಯ ಪ್ರಕಾರ, ಮನುಷ್ಯನ ಮೊಟ್ಟಮೊದಲ ಕಲ್ಪನೆ ಎಂದರೆ ಅವನೇ. ಅವನ ಇಚ್ಛೆಯೇ ಉಳಿದವರೆಲ್ಲರಲ್ಲಿ ಸಂಚಲನವನ್ನು ಉಂಟುಮಾಡುವುದು. ಒಂದು ಮಗುವಿನ ಶಕ್ತಿಯ ಕಲ್ಪನೆಯೇ ಅದರ ಇಚ್ಛೆ. ವಿಶ್ವದಲ್ಲಿನ ಸಮಸ್ತ ಚಲನೆಯ ಹಿಂದೆಯೂ ಒಂದು ಇಚ್ಛೆ ಇದೆ. ಭಾಷಣಕಾರರು ಹೇಳಿದಂತೆ, ಇರುವುದೊಬ್ಬನೇ ದೇವರು, ಅವನು ತಮ್ಮ ಹಾಗೆಯೇ ಇರುವ ವ್ಯಕ್ತಿ. ಅದರೆ ಅವನು ಅಪರಿಮಿತ ಶಕ್ತಿಯುಳ್ಳವನು ಎನ್ನುವುದು ಹಿಂದೂಗಳ ನಂಬಿಕೆ. ಒಬ್ಬ ಒಳ್ಳೆಯವನು, ಇನ್ನೊಬ್ಬ ಕೇಡಿಗ ಹೀಗೆ ಎರಡು ದೇವರುಗಳನ್ನು ನಂಬದಿರುವಷ್ಟರಮಟ್ಟಿಗೆ ಅವರ ಮನಸ್ಸು ತಾತ್ತ್ವಿಕವಾದುದು. ಅವರ ಕಲ್ಪನೆಯಂತೆ ಪ್ರಕೃತಿಯೇ ಒಂದು ಸಂಯೋಜಕ ತತ್ತ್ವ. ಸಮಸ್ತ ಅಸ್ತಿತ್ವದ ಏಕತೆಯೇ ವಿಶ್ವ; ದೇವರು ಎಂದರೆ ಪ್ರಕೃತಿಯೇ ಹೊರತು ಬೇರೆಯಲ್ಲ.

“ಪುರಾತನ ಈಜಿಪ್ಷಿಯನ್ನರಿಂದ ಹಿಡಿದು ರೋಮನ್ ಕ್ಯಾಥೊಲಿಕ್ ಚರ್ಚ್ನವರೆಗೂ ಈ ತತ್ತ್ವವನ್ನು ಆಂಶಿಕವಾಗಿಯಾದರೂ ಅಭಿವ್ಯಕ್ತಿಸದ ತತ್ತ್ವಜ್ಞಾನವೇ ಇಲ್ಲ” ಎಂದರು ಭಾಷಣಕಾರರು. “ಭೌತಿಕ ಹಾಗೂ ಮಾನಸಿಕ ಲೋಕಗಳಲ್ಲಿರುವ ಸಮಸ್ತ ಬಲಗಳ ಮೂಲಾಂಶಗಳನ್ನೂ ‘ಪಿತೃ’ (ಪ್ರಾಣ?) ಎಂಬ ಒಂದೇ ಪದವನ್ನು ಬಳಸಿ ಭಾರತದಲ್ಲಿ ಅಭಿವ್ಯಕ್ತಿಸಿರುವರು. ಏನಿದೆಯೋ ಅದೆಲ್ಲವೂ ಅವನಿಂದಲೇ ಹೊರಹೊಮ್ಮಿರುವುದು.”

ಮಾನ್ಯ ತಾತ್ತ್ವಿಕರು ಮುಗಿಸುವ ಮುನ್ನ, ಭಾರತದ ಪುರಾತನ ಧ್ವನಿಯು ಹತ್ತೊಂಭತ್ತನೆಯ ಶತಮಾನದ ಹರ್ಬರ್ಟ್ ಸ್ಪೆನ್ಸರ್ನ ಬರಹಗಳಲ್ಲಿ ಪ್ರತಿಧ್ವನಿತವಾಗಿದೆ ಎಂದು ನುಡಿದರು.

\begin{center}
\textbf{ವಿಶ್ವದ ಪರಿಕಲ್ಪನೆ\supskpt{\footnote{\enginline{1. New Discoveries, Vol. 5, pp.192-94}}}}
\end{center}

\begin{center}
(ಲಾಸ್ ಏಂಜಲಿಸ್ ಹೆರಾಲ್ಡ್, ೧೩ ಡಿಸೆಂಬರ್ ೧೮೯೯)
\end{center}

\begin{center}
\textbf{ಅಕಾಡೆಮಿ ಆಫ್ ಸೈನ್ಸಸ್ ಎದುರಿಗೆ ಸ್ವಾಮಿ ವಿವೇಕಾನಂದರ ಭಾಷಣ}
\end{center}

ಕಳೆದ ಸಂಜೆ ದಕ್ಷಿಣ ಕ್ಯಾಲಿಫೋರ್ನಿಯಾ ಅಕಾಡೆಮಿ ಆಫ್ ಸೈನ್ಸಸ್ನ ಆಶ್ರಯದಲ್ಲಿ ಭರತಖಂಡದ ನಿವಾಸಿಯಾದ ಸ್ವಾಮಿ ವಿವೇಕಾನಂದರ ಬ್ರಹ್ಮಾಂಡ ಅಥವಾ ವೇದದಲ್ಲಿನ ವಿಶ್ವದ ಪರಿಕಲ್ಪನೆ ಎಂಬ ಭಾಷಣ\footnote{2. ಕ್ಯಾಲಿಫೋರ್ನಿಯಾದಲ್ಲಿ ಸ್ವಾಮಿ ವಿವೇಕಾನಂದರು ಎರಡನೆ ಭಾಷಣವಿದು. ವಿಷಯ: ಬ್ರಹ್ಮಾಂಡ ಅಥವಾ ವೇದದಲ್ಲಿ ವಿಶ್ವದ ಪರಿಕಲ್ಪನೆ? ಇದರ ಪದಶಃ ವರದಿ ಲಭ್ಯವಿಲ್ಲ. ವಿವರಗಳಿಗೆ ಸ್ವಾಮೀಜಿಯವರು ನ್ಯೂಯಾರ್ಕಿ ನ ಲ್ಲಿ ‘ಬ್ರಹ್ಮಾಂಡ ಮತ್ತು ಜೀವ’ ದ ಕುರಿತು ಮಾಡಿದ ಎರಡು ಉಪನ್ಯಾಸಗಳ ವರದಿ ನೋಡಿ - ೧೮೬೬ರಲ್ಲಿ ಕೃತಿಶ್ರೇಣಿ ೨ ಪುಟ ೧೩೩-೧೫೨.}ವನ್ನು ಆಲಿಸಲು ಯೂನಿಟಿ ಚರ್ಚ್ನಲ್ಲಿ ಸಭೆ ಜನರಿಂದ ತುಂಬಿ ತುಳುಕುತ್ತಿತ್ತು.....

ತಮ್ಮ ವಿಷಯಕ್ಕೆ ಪೀಠಿಕೆಯಾಗಿ ಭಾಷಣಕಾರರು ವಿಶ್ವದ ಸೃಷ್ಟಿಯ ಬಗ್ಗೆ ಎಲ್ಲರ ಲ್ಲಿಯೂ ಸಮಾನವಾದ ಕಲ್ಪನೆಯಿದ್ದಿತು ಎನ್ನಲು ಹಿಬ್ರೂ ಶಾಸ್ತ್ರಗ್ರಂಥಗಳಲ್ಲಿ ಬರುವ ಪ್ರವಾಹದ ಪುರಾಣಕಥೆಯನ್ನು ಹೇಳಿ, ಬ್ಯಾಬಿಲೋನಿಯನ್ನರಲ್ಲಿ, ಈಜಿಪ್ಷಿಯನ್ನರಲ್ಲಿ, ಅಸ್ಸೀರಿಯನ್ನರಲ್ಲಿ ಹಾಗೂ ಮತ್ತಿತರ ಬುಡಕಟ್ಟುಗಳ ಜನರಲ್ಲಿ ಅದೇ ತೆರನಾದ ನಂಬಿಕೆಗಳಿದ್ದವು ಎಂಬುದನ್ನು ಪರಾಮರ್ಶಿಸಿದರು.

“ಪ್ರಾಚೀನ ಜನರು ತಮ್ಮಸುತ್ತಮುತ್ತಲಿನ ನಿಗೂಢತೆ ಅಚ್ಚರಿಗಳ ವಿವರಣೆಗೆ ಪ್ರಯತ್ನಿಸಿರುವುದನ್ನು ನಾವು ಸೂರ್ಯನ ಮತ್ತು ಇತರ ಪ್ರಕೃತಿಶಕ್ತಿಗಳ ಉಪಾಸನೆಯಲ್ಲಿ ನೋಡಬಹುದು. ಬಲದ ಕಲ್ಪನೆ ಮಾನವನಿಗೆ ಮೊದಲು ಬಂದದ್ದೇ ತನ್ನ ಬಲದ ದೆಸೆಯಿಂದ. ಒಂದು ಕಲ್ಲು ಬೀಳುವಾಗ ಅವನು ಅದರಲ್ಲಿ ಬಲವನ್ನು ಕಾಣುವುದರ ಬದಲಾಗಿ ಅದರ ಹಿಂದಿರುವ ಇಚ್ಛೆಯನ್ನು ಕಂಡ; ಇಡಿಯ ವಿಶ್ವವೇ ಚಲಿಸುತ್ತಿರುವುದು ಇಚ್ಛಾಶಕ್ತಿಗಳಿಂದಾಗಿ ಎಂಬ ಕಲ್ಪನೆ ಅವನಿಗೆ ಬಂದಿತು. ಕಾಲಕ್ರಮೇಣ ಈ ಇಚ್ಛಾಶಕ್ತಿಗಳೆಲ್ಲ ಮೇಳೈಸಿ ಒಂದಾದವು; ವಿಜ್ಞಾನವು ವರ್ಧಿಸಲು ಆರಂಭವಾಯಿತು. ದೇವರುಗಳು ನಶಿಸಲಾರಂಭಿಸಿದರು, ಅವನ ಜಾಗದಲ್ಲಿ ಏಕತ್ವ ಬಂದಿತು; ಈಗ ನವವಿಜ್ಞಾನದಿಂದಾಗಿ ದೇವರೂ ತನ್ನ ಸಿಂಹಾಸನವನ್ನು ಕಳೆದುಕೊಳ್ಳಬೇಕಾಗಿ ಬಂದಿದೆ. ವಿಜ್ಞಾನವು ಎಲ್ಲವನ್ನೂ ಅವುಗಳದ್ದೇ ಪ್ರಕೃತಿಯನ್ನಾಧರಿಸಿ ವಿವರಿಸಲು ಅಪೇಕ್ಷಿಸುತ್ತದೆ ಮತ್ತು ವಿಶ್ವವನ್ನು ಆತ್ಮಸಂತೃಪ್ತವನ್ನಾಗಿಸಲು ಬಯಸುತ್ತದೆ.

“ಇಚ್ಛಾಶಕ್ತಿಗಳು ಕಾಲಕ್ರಮೇಣ ಹೊರಟುಹೋಗಿ ಅವುಗಳ ಜಾಗದಲ್ಲಿ ಒಂದು ಇಚ್ಛೆ ಬಂದಿತು. ಇದು ಪ್ರಪಂಚದ ಎಲ್ಲ ದೇಶಗಳಲ್ಲಿಯೂ ನಡೆದಿರುವ ಮುನ್ನಡೆ; ಅಂತೆಯೇ ಭರತಖಂಡದಲ್ಲಿಯೂ ಸಹ. ಅವರ ಕಲ್ಪನೆಗಳು ಹಾಗೂ ದೇವರುಗಳು ಬಹುಮಟ್ಟಿಗೆ ಇನ್ನಿತರ ದೇಶಗಳಲ್ಲಿನಂತೆಯೇ ಇದ್ದುವು; ಆದರೆ ಭಾರತದಲ್ಲಿ ಮಾತ್ರ ಅವರು ಅಲ್ಲಿಗೇ ನಿಲ್ಲಿಸಲಿಲ್ಲ. ಜೀವಾಂಕುರವಾಗುವುದು ಜೀವದಿಂದಲೇ ಹೊರತು ಮೃತ್ಯುವಿನಿಂದ ಸಾಧ್ಯವಿಲ್ಲ ಎನ್ನುವುದನ್ನು ಅವರು ಕಲಿತುಕೊಂಡರು. ದೇವರ ಬಗೆಗಿನ ನಮ್ಮ ಊಹೆಗಳಲ್ಲಿ ನಾವು ಏಕದೇವತಾವಾದ ದವರೆಗೆ ಬಂದಿರುವೆವು. ಬೇರೆ ಎಲ್ಲ ಕಡೆಯೂ ಊಹೆ ಅಲ್ಲಿಗೆ ನಿಲ್ಲುತ್ತದೆ; ಎಲ್ಲದರ ಕೊಟ್ಟಕೊನೆ ಹಾಗೂ ಅಂತಿಮ ಅಸ್ತಿತ್ವ ಅಷ್ಟೇ ಎನ್ನುವೆವು; ಆದರೆ ಭಾರತದಲ್ಲಿ ಅದು ಅಲ್ಲಿಗೆ ನಿಲ್ಲುವುದಿಲ್ಲ. ನಾವು ನಮ್ಮಸುತ್ತಮುತ್ತ ನೋಡುತ್ತಿರುವ ಒಂದು ಬೃಹತ್ ಭೂಮ ಇಚ್ಛಾಶಕ್ತಿ ಈ ಎಲ್ಲ ವಿದ್ಯಮಾನಗಳನ್ನೂ ವಿವರಿಸಲಾರದು. ಮನುಷ್ಯನಲ್ಲೂ ಸಹ ಅವನ ಇಚ್ಛೆಯ ಹಿಂದೆ ಏನೋ ಒಂದು ಇದೆ. ರಕ್ತಪರಿಚಲನೆಯಂತಹ ನಿತ್ಯದ ವಿದ್ಯ ಮಾನದಲ್ಲಿ ಸಹ ಅದರ ಹಿಂದಿರುವ ಶಕ್ತಿ ಇಚ್ಛಾಶಕ್ತಿ ಅಲ್ಲ ಎಂಬುದು ನಮಗೆ ಗೊತ್ತು.

“ನಾವು ದೇವರನ್ನು ನಮ್ಮಂತಹ ವ್ಯಕ್ತಿಯೆಂದೂ, ಆದರೆ ಅಪರಿಮಿತ ಶಕ್ತಿಯುಳ್ಳವನೆಂದೂ ಕಲ್ಪಿಸಿಕೊಂಡಿದ್ದೇವೆ. ಪ್ರಪಂಚದಲ್ಲಿ ಸುಖ, ಒಳ್ಳೆಯತನ ಹಾಗೂ ಕರುಣೆ ಇರುವುದರಿಂದ ಈ ಗುಣಗಳನ್ನು ಅಪರಿಮಿತವಾಗಿ ಪಡೆದುಕೊಂಡಿರುವವನೊಬ್ಬನು ಇರಬೇಕು. ಆದರೆ ಕೆಡಕು ಎನ್ನುವುದೂ ಇದೆ. ಒಳ್ಳೆಯ ಹಾಗೂ ಕೆಟ್ಟ ಎಂದು ಎರಡು ದೇವರುಗಳ ಅಸ್ತಿತ್ವವನ್ನು ಒಪ್ಪದೆ ಇರುವಷ್ಟು ತಾತ್ತ್ವಿಕವಾದದ್ದು ಹಿಂದೂ ಮನಸ್ಸು. ಒಬ್ಬನೇ ದೇವರೆಂಬ ಕಲ್ಪನೆಗೆ ಭಾರತ ನಿಷ್ಠವಾಗಿ ಉಳಿಯಿತು. ನನಗೆ ಕೇಡಾಗಿ ಕಾಣುವುದು ಇನ್ನಾರಿಗೋ ಒಳ್ಳೆಯದೇ ಆಗಿರಬಹುದು; ನನಗೆ ಒಳಿತಾಗಿ ಕಾಣುವುದು ಇತರರಿಗೆ ಕೆಟ್ಟದ್ದಾಗಿರಬಹುದು. ನಾವೆಲ್ಲರೂ ಒಂದು ಸರಪಳಿಯ ಉಂಗುರಗಳಿದ್ದಂತೆ. ಮಾನವ ಜನಾಂಗದ ಪೈಕಿ ಮೂವತ್ತು ಕೋಟಿ ಜನರ ಧರ್ಮವಾದ ಉಪನಿಷತ್ತುಗಳಲ್ಲಿ ಊಹೆ ಬರುವುದು ಹಾಗೆ. ಪ್ರಕೃತಿಯೇ ಒಂದು ಮೂಲಮಾನ ವಿದ್ದಂತೆ; ಈ ಏಕತೆ ಅಸ್ತಿತ್ವದಲ್ಲಿರುವ ಎಲ್ಲದರಲ್ಲಿಯೂ ಇದೆ. ಮತ್ತು ದೇವರೂ ಪ್ರಕೃತಿಯೂ ಒಂದೇ. ಭಾರತದ ಹೊರಗಿನ ಪ್ರಪಂಚಕ್ಕೆಲ್ಲ ತಿಳಿದಿರುವ ಭಾರತೀಯ ಊಹೆ ಯೆಂದರೆ ಇದೊಂದೇ.

“ಇಂದಿನ ಕ್ಯಾಥೊಲಿಕ್ ಚರ್ಚ್ನ ತನಕವೂ, ಪ್ರಪಂಚದಲ್ಲಿ ಭಾರತೀಯ ಊಹೆಯ ಪ್ರಭಾವವನ್ನು ತೋರಿಸದಿರುವ ಧರ್ಮವಾಗಲಿ ತತ್ತ್ವಜ್ಞಾನವಾಗಲಿ ಇಲ್ಲ. ಹೊಸ ಆವಿಷ್ಕರಣವೆಂದು ಪರಿಗಣಿಸಲಾಗುವ ಶಕ್ತಿನಿತ್ಯತೆಯ ಪರಿಕಲ್ಪನೆಯು ಅಲ್ಲಿ ಪಿತೃ (ಪ್ರಾಣ?) ಎಂಬ ಹೆಸರಿನಲ್ಲಿ ಮೊದಲಿನಿಂದಲೂ ಇತ್ತು. ಏನಿರುವುದೋ ಅದೆಲ್ಲವೂ ಬಂದಿರುವುದು ಪಿತೃವಿನಿಂದ. ಬ್ರಹ್ಮ (ಪ್ರಾಣ?) ಯಾವುದಕ್ಕೋ ಶಕ್ತಿ ಯೀಯುವನು; ಅದನ್ನವರು ಅದೃಶ್ಯ ಈಥರ್ ಎಂದು ಕರೆಯುವರು. ಈಥರ್ ನೊಂದಿಗೆ ಕಂಪಿಸುವ ಬ್ರಹ್ಮ (ಪ್ರಾಣ?)ನೇ ಘನ, ದ್ರವ, ಹಾಗೂ ತೇಜಸ್ಸು; ಎಲ್ಲವೂ ಅದೇ ಈಥರ್. ಪ್ರತಿಯೊಂದರ ಪ್ರಚ್ಛನ್ನತೆಯೂ ಇರುವುದು ಅಲ್ಲಿಯೇ. ಮುಂದಿನ ಕಲ್ಪದ ಪ್ರಾರಂಭದಲ್ಲಿ ಬ್ರಹ್ಮ (ಪ್ರಾಣ?)ನು ಇನ್ನೂ ಹೆಚ್ಚುಹೆಚ್ಚಾಗಿ ಕಂಪಿಸತೊಡಗುವನು.

“ಹೀಗೆ, ಭಾರತೀಯ ಶಾಸ್ತ್ರಗಳಲ್ಲಿನ ಊಹೆಗಳು ನವವಿಜ್ಞಾನಕ್ಕೆ ಹೆಚ್ಚು ಸಮೀಪವಾಗಿವೆ. ಇಂದಿನ ವಿಕಾಸವಾದದಲ್ಲಿಯೂ ತೆಗೆದುಕೊಂಡಿರುವುದು ಇದೇ ಕಲ್ಪನೆಯನ್ನೇ. ಘನತೆಯಲ್ಲಿ ಮಾತ್ರ ವಿಭಿನ್ನವಾದ ನಮ್ಮ ಶರೀರಗಳೂ ಸಹ ಅದೇ ಸರಪಳಿಯ ಉಂಗುರಗಳು. ಒಬ್ಬ ವ್ಯಕ್ತಿಯಲ್ಲಿ ಇನ್ನಿತರ ಪ್ರತಿಯೊಬ್ಬ ವ್ಯಕ್ತಿಯ ಸಾಧ್ಯತೆಗಳೂ ಸಹ ಇರುತ್ತವೆ. ಒಂದು ಜೀವಂತ ವಸ್ತುವಿನಲ್ಲಿ ಎಲ್ಲ ಜೀವಿಗಳ ಸಾಧ್ಯತೆಗಳೂ ಇರುವುದಾದರೂ, ಅಭಿವ್ಯಕ್ತಿ ಕೊಡುವುದು ಪರಿಸರವು ಬೇಡುವಂಥದನ್ನು ಮಾತ್ರ. ನವೀನ ವಿಜ್ಞಾನದಲ್ಲಿ ಅತ್ಯಂತ ಅಚ್ಚರಿಯ ಊಹೆಗಳು ರೂಪುಗೊಂಡಿವೆ. ಧರ್ಮಪ್ರಬೋಧಕನಾಗಿರುವ ನನಗೆ ಆಸಕ್ತಿ ಮೂಡಿಸುವುದು ಎಲ್ಲ ಧರ್ಮಗಳ (ಜೀವವೆಲ್ಲದರ?) ಏಕತೆ. ಸಸ್ಯದಲ್ಲಿ ಉಗಮಿಸುತ್ತಿರುವ ಜೀವಶಕ್ತಿಯೇ ವ್ಯಕ್ತಿಯಲ್ಲಿ ಉಗಮಿಸುತ್ತಿರುವುದೂ ಸಹ ಎನ್ನುವ ಹರ್ಬರ್ಟ್ ಸ್ಪೆನ್ಸರ್ನ ದನಿಯಲ್ಲಿ ಭಾರತೀಯ ಧರ್ಮವೂ ಹತ್ತೊಂಭತ್ತನೆಯ ಶತಮಾನದಲ್ಲಿ ತನ್ನ ಅಭಿವ್ಯಕ್ತಿಯನ್ನು ಕಂಡುಕೊಂಡಿದೆ.”

\begin{center}
\textbf{ಭರತಖಂಡದ ಬಗ್ಗೆ ಹೇಳಿದುದು\supskpt{\footnote{\enginline{1. New Discoveries, Vol. 5, p.269}}}}
\end{center}

\begin{center}
(ಲಾಸ್ ಏಂಜಲಿಸ್ ಹೆರಾಲ್ಡ್, ೩ ಜನವರಿ ೧೯೦೦)
\end{center}

\begin{center}
\textbf{ಕಳೆದ ರಾತ್ರಿ ಬ್ಲಾಂಕರ್ಡ್ ಹಾಲ್ನಲ್ಲಿ ಸ್ವಾಮಿ ವಿವೇಕಾನಂದರ ಭಾಷಣ}
\end{center}

ಈ ನಗರದಲ್ಲಿ ಉಪನ್ಯಾಸಗಳ ಮತ್ತು ತರಗತಿಗಳ ಸರಣಿಯನ್ನೇ ನಡೆಸುತ್ತಿರುವ ಪುರಾತನ ಹಿಂದೂ ಸಂನ್ಯಾಸಿಗಳ ಪಂಥಕ್ಕೆ ಸೇರಿದ ಸ್ವಾಮಿ ವಿವೇಕಾನಂದರು ಕಳೆದ ರಾತ್ರಿ ಬ್ಲಾಂಕರ್ಡ್ ಹಾಲ್ನಲ್ಲಿ ‘ಭಾರತದ ಇತಿಹಾಸ’ (‘ಭಾರತೀಯ ಜನತೆ’)\footnote{2. ಇದರ ಪದಶಃ ವರದಿ ಲಭ್ಯವಿಲ್ಲ.} ಎಂಬ ಉಪನ್ಯಾಸವನ್ನು ಕೊಟ್ಟರು. ಸ್ವಾಮಿಗಳು ಸಭೆಯ ಮುಂದೆ ಅಮೆರಿಕನ್ ಉಡುಪಿನಲ್ಲಿ ಬಂದುದರಿಂದ, ಅವರ ಪಂಥದವರ ವೇಷ ಸುಂದರ ರೇಷ್ಮೆ ವಸ್ತ್ರ, ರುಮಾಲುಗಳಿಂದ ಬರುತ್ತಿದ್ದ ವಿಶಿಷ್ಟ ಹಾಗೂ ವಿಚಿತ್ರವಾದ ಶೋಭೆಯು ಬಹುಮಟ್ಟಿಗೆ ಇಲ್ಲದಂತಾಗಿತ್ತು.

ಭಾಷಣಕಾರರು ಭಾರತವೆಂದರೆ ಒಂದು ದೇಶವಲ್ಲ, ಅನೇಕ ಬುಡಕಟ್ಟುಗಳಿಗೆ ಸೇರಿದ, ಆದರೆ ಧರ್ಮದ ದೆಸೆಯಿಂದ ಒಂದಾಗಿರುವ ‘ಬಹು ದೊಡ್ಡ ಜನಸಮುದಾಯ’ ಎಂದು ಹೇಳಿದರು. ಭಾರತ ಬಹು ಪುರಾತನವಾದದ್ದು, ಕೊಲಂಬಸ್ ಭಾರತವನ್ನು ತಲುಪುವುದಕ್ಕೆ ಹತ್ತಿರದ ಹಾದಿಯನ್ನು ಅನ್ವೇಷಿಸುತ್ತ ಅಕಸ್ಮಾತ್ತಾಗಿ ಅಮೆರಿಕಾವನ್ನು ಕಂಡುಹಿಡಿದ ಕಾಲದಲ್ಲೇ ಭಾರತದಲ್ಲಿ ಜನರು ವಾಸಿಸುತ್ತಿದ್ದರು; ಅಲ್ಲಿಯ ಉತ್ಪಾದನೆಗಳಾದ ಹತ್ತಿ, ಸಕ್ಕರೆ, ನೀಲಿ ಮತ್ತು ಸಂಬಾರ ಪದಾರ್ಥಗಳು ಪ್ರಪಂಚವನ್ನು ಶ‍್ರೀಮಂತ ಗೊಳಿಸಿದ್ದವು. ಇಪ್ಪತ್ತು ಕೋಟಿ ಜನರು ವಾಸಿಸುವ ಈ ದೇಶದಲ್ಲಿಯ ಎಲ್ಲ ಕಣಿವೆಗಳಲ್ಲಿಯೂ ಮತ್ತು ಸಮುದ್ರಮಟ್ಟದಿಂದ ಸಾವಿರಾರು ಅಡಿಗಳ ಎತ್ತರದವರೆಗಿನ ಪರ್ವತಪ್ರದೇಶಗಳಲ್ಲಿಯೂ ಸಹ ಪುಟ್ಟಪುಟ್ಟ ಹಳ್ಳಿಗಳೇ ತುಂಬಿವೆ. ಮಣ್ಣಿನ ಅತ್ಯಂತ ಹೆಚ್ಚಿನ ಫಲವತ್ತತೆಗೆ ಬಹುಮಟ್ಟಿಗೆ ಧಾರಾಕಾರ ಮಳೆಯೇ ಕಾರಣ; ವರ್ಷಋತು ವೊಂದರಲ್ಲಿ ಕೆಲವೊಮ್ಮೆ ೧೮೦೦ ಅಂಗುಲಗಳಷ್ಟಾಗುವ ಈ ಮಳೆ ಸರಾಸರಿ ಬಹುಶಃ ೬೦೦ ಅಂಗುಲಗಳಷ್ಟಿರಬಹುದು. ಬೇಕಾದಷ್ಟು ಉತ್ಪಾದನೆ ಇದ್ದರೂ ಬಹಳ ಜನರುರಾಗಿ, ನವಣೆಯಂಥ ಆಹಾರಧಾನ್ಯಗಳ ಮೇಲೆ ಬದುಕಿದ್ದಾರೆ; ಮಾಂಸ ಮೊಟ್ಟೆ ಮೀನಿನಂಥ ಪ್ರಾಣಿಜನ್ಯ ಆಹಾರವನ್ನು ತಿನ್ನುವುದಿಲ್ಲ.

ಬಹಳ ಹಿಂದಿನ ಕಾಲದಿಂದಲೂ ದೇಶವು ತನ್ನದೇ ಭಾಷೆಗಳನ್ನು, ಜಾತಿಪದ್ಧತಿಗಳನ್ನು, ಸಂಪ್ರದಾಯಗಳನ್ನು ಉಳಿಸಿಕೊಂಡು ಬಂದಿದೆ. ಬೇರೆ ಭಾಗಗಳ (ದೇಶಗಳ) ಉಚ್ಛ್ರಾಯ ಅವನತಿಗಳಿಗೆ ಸಾಕ್ಷಿಯಾಗಿದ್ದರೂ ಸಹ, ತನ್ನ ಧರ್ಮದಿಂದಾಗಿ ಅದು ತನ್ನನ್ನು ತಾನು ಉಳಿಸಿಕೊಂಡು ಬಂದಿದೆ. ಬ್ಯಾಬಿಲೋನಿಯನ್ ನಾಗರಿಕತೆಯೇನೂ ಇತ್ತೀಚಿನದಲ್ಲ; ಆದರೆ ಭಾರತದ ಚರಿತ್ರೆ ಅದರ ಉಚ್ಛ್ರಾಯ - ಅವನತಿಗಳಿಗಿಂತಲೂ ತುಂಬ ಹಿಂದಿನದು. ಅತ್ಯಂತ ಪುರಾತನ ಭಾಷೆಯಾದ ಸಂಸ್ಕೃತವನ್ನು ಒಂದು ಕಾಲದಲ್ಲಿ ಎಲ್ಲ ಜನಾಂಗದವರೂ ಆಡುಭಾಷೆಯಾಗಿ ಬಳಸುತ್ತಿದ್ದರು; ಈಗ ಪುರೋಹಿತರು ಬಳಸುತ್ತಿರುವರು. ನಮ್ಮ ಅನೇಕ ಸಾಮಾನ್ಯ ಬಳಕೆಯ ಇಂಗ್ಲಿಷ್ ಪದಗಳು ಮೂಲತಃ ಹೇಗೆ ಸಂಸ್ಕೃತದಿಂದ ಬಂದಿರುವುವು ಎಂಬುದಕ್ಕೆ ಭಾಷಣಕಾರರು ಅನೇಕ ಉದಾಹರಣೆಗಳನ್ನು ಕೊಟ್ಟರು; ನಮ್ಮ ಅನೇಕ ಹಳೆಯ ಧಾರ್ಮಿಕ ಕಲ್ಪನೆಗಳು, ಪುರಾಣಕಥೆಗಳು ಸಹ ಹೇಗೆ ಪುರಾತನ ಆರ್ಯನ್ ಜನಾಂಗದಿಂದಲೇ ಬಂದವುಗಳು ಎಂಬುದನ್ನು ತೋರಿಸಿಕೊಟ್ಟರು.

ದೇಶದ ಅನೇಕ ರೂಢಿಗಳ ಸಂಪ್ರದಾಯಗಳನ್ನು ಚಿತ್ರಿಸಿದರಲ್ಲದೆ, ಈ ದೇಶವು ಹೇಗೆ ಪ್ರಪಂಚದ ನಾಗರಿಕತೆಯ ತವರಾಗಿತ್ತು, ಕಲೆಗಳ, ವಿಜ್ಞಾನಗಳ ಮತ್ತು ತಾತ್ತ್ವಿಕ ಚಿಂತನೆಯ ಕೇಂದ್ರವಾಗಿತ್ತು ಎಂಬುದನ್ನು ವಿಶದಪಡಿಸಿದರು.

ಜಾತಿಪದ್ಧತಿಯನ್ನು ಸ್ವಯಂಪರಿಪೂರ್ಣವನ್ನಾಗಿಸಿಕೊಳ್ಳುವ ಮೂಲಕ ಭಾರತೀ ಯರು ತಮ್ಮಸುತ್ತಲೂ ಒಂದು ಗೋಡೆಯನ್ನೇ ನಿರ್ಮಿಸಿಕೊಂಡು ತಮ್ಮನ್ನು ತಾವು ಉಳಿಸಿಕೊಂಡರು. ಭಾರತದ ಸಾಮ್ರಾಟನೊಬ್ಬ ತನಗಿಂತ ಮೇಲ್ಜಾತಿಯ ಪುರೋಹಿತರೊಬ್ಬರ ವಂಶದವನು ಎಂದು ಹೇಳಿಕೊಳ್ಳಲು ಹೆಮ್ಮೆಪಡುತ್ತಾನೆ. ಈ ಜಾತಿಗಳು ಒಂದು ಕಾಲದಲ್ಲಿದ್ದಂತೆ ಉಳಿದು ಬಂದಿಲ್ಲ; ಅನೇಕಾನೇಕ ಭಾಗಗಳಾಗಿ, ನೂರಾರು ಉಪವರ್ಗಗಳಾಗಿ ಒಡೆದಿವೆ. ಬೇರೆ ಬೇರೆ ಜಾತಿಗೆ ಸೇರಿದ ಜನರು ಜೊತೆಗೆ ಊಟ ಮಾಡುವುದಿಲ್ಲ, ಒಟ್ಟಿಗೆ ಅಡುಗೆಯನ್ನೂ ಮಾಡುವುದಿಲ್ಲ. ತನ್ನ ಜಾತಿಯಿಂದ ಹೊರಗೆ ಮದುವೆಯಾದರೆ ಅದು ಮಾನ್ಯವೆಂದೆನಿಸಿಕೊಳ್ಳುವುದಿಲ್ಲ. ಜಾತಿಯ ನಿಯಮಗಳು ತುಂಬ ತೊಡಕಿನವು; ಆಚರಣೆಯ ಸೂಕ್ಷ್ಮ ವಿವರಗಳವರೆಗೂ ಕವಲುಕವಲಾಗಿ ವ್ಯಾಪಿಸಿರುವ ಅವು ತೀರ ಸಂಕೀರ್ಣವಾದವುಗಳು. ಭಾರತದಲ್ಲಿ ವೈಸರಾಯ್​ ಆದವನೂ ಗರೀಬನಾದ ಭಿಕ್ಷುಕನೊಬ್ಬನೂ ಒಂದೇ ಜಾತಿಯವರಾಗಿರಬಹುದು.

ಪ್ರಾಣಿಚರ್ಮದಿಂದ ಮಾಡಲ್ಪಟ್ಟಿರುವುದರಿಂದ ಶೂ ಧರಿಸುವುದನ್ನು ನಿಷೇಧಿಸಿರುವರು. ಈ ವಿವರಗಳಿಗೆ ಗಂಡಸರಿಗಿಂತಲೂ ಹೆಚ್ಚಾಗಿ ಹೆಂಗಸರು ಗಮನ ಕೊಡುವರು. ಈ ಎಲ್ಲ ರೂಢಿಗಳಿಗೂ ಅವುಗಳದ್ದೇ ಆದ ತತ್ತ್ವದ ಹಿನ್ನೆಲೆಯಿದೆ. ಇದೇ ನಿಜವಾದ ಪ್ರಜಾಪ್ರಭುತ್ವ, ಇದೇ ಸಾಮಾಜಿಕ ಕಲ್ಪನೆ; ಇದು ಜನಾಂಗಗಳ ಅಭಿವೃದ್ಧಿಯೇ ಹೊರತು ವ್ಯಕ್ತಿಯದಲ್ಲ.

ಭಾಷಣಕಾರರು ಈ ದೇಶದ ಸ್ತ್ರೀಯರ ಸ್ಥಾನಮಾನಗಳೊಂದಿಗೆ ಭಾರತದ ಸ್ತ್ರೀಯರ ಸ್ಥಾನಮಾನಗಳನ್ನು ಹೋಲಿಸುವ ಮೂಲಕ ತಮ್ಮ ಮಾತನ್ನು ಮುಗಿಸಿದರು. ಭಾರತದಲ್ಲಿ ಸ್ತ್ರೀತ್ವ ಎಂಬುದರ ಇಡಿಯ ಕಲ್ಪನೆಯೇ ತಾಯಿ. ತಾಯಿಗೆ ಪೂಜ್ಯಸ್ಥಾನ, ಅವಳೇ ಜೀವ ದಾತೆ, ಜನಾಂಗವನ್ನು ಸೃಜಿಸುವವಳು.

\begin{center}
\textbf{ಭಾರತದಲ್ಲಿಯ ಧಾರ್ಮಿಕ ದಂತಕಥೆಗಳು *\supskpt{\footnote{\enginline{1. New Discoveries, Vol. 5, p.269}}}}
\end{center}

\begin{center}
(ಲಾಸ್ ಏಂಜಲಿಸ್ ಟೈಮ್ಸ್, ೧೭ ಜನವರಿ ೧೯೦೦)
\end{center}

\begin{center}
\textbf{ಸ್ವಾಮಿಗಳು}
\end{center}

ಈ ಸಂಜೆ (ಜನವರಿ ೧೬) ಷೇಕ್ಸ್ಪಿಯರ್ ಕ್ಲಬ್ನಲ್ಲಿ ಹೆಚ್ಚಾಗಿ ಮಹಿಳೆಯರಿಂದಲೇ ಕೂಡಿದ್ದ ಸಣ್ಣ ಸಭೆಯನ್ನುದ್ದೇಶಿಸಿ ತಮ್ಮ ಕಡುಗೆಂಪು ಉಡುಪಿನಲ್ಲಿದ್ದ ಸ್ವಾಮಿ ವಿವೇಕಾನಂದರು ಮಾತನಾಡಿದರು.\footnote{2. ಇದರ ಪದಶಃ ವರದಿ ಲಭ್ಯವಿಲ್ಲ.} ಹಿಂದೂಗಳ ದೈನಂದಿನ ಜೀವನದಲ್ಲಿ ಹಾಸುಹೊಕ್ಕಾಗಿರುವ ಬ್ರಾಹ್ಮಣತ್ವದ ಅನೇಕ ಧಾರ್ಮಿಕ ದಂತಕಥೆಗಳನ್ನು ಅವರು ಹೇಳಿದರು. ಶಿವನ ಮೂಲವನ್ನು, ಅವನು ತನ್ನ ಪತ್ನಿಯ ಶುದ್ಧಚೇತನಕ್ಕೆ ಹೇಗೆ ಶರಣಾದನು ಆಕೆ ಹೇಗೆ ಇಂದು ಇಡೀ ಭಾರತಕ್ಕೇ ಮಾತೆಯೆನಿಸಿರುವಳು ಎಂಬುದನ್ನು ಹೇಳಿದರು. ಈಕೆಯ ಉಪಾಸನೆ ಎಷ್ಟರಮಟ್ಟಿಗೆ ನಡೆಯುತ್ತಿದೆಯೆಂದರೆ, ಒಂದು ಹೆಣ್ಣು ಪ್ರಾಣಿಯನ್ನೂ ಸಹ ಕೊಲ್ಲುವಂತಿಲ್ಲ. ವಿವೇಕಾನಂದರು ನಿರರ್ಗಳವಾಗಿ ಸಂಸ್ಕೃತದಿಂದ ಉದ್ಧರಿಸುತ್ತ, ಅವುಗಳನ್ನು ಲೀಲಾಜಾಲವಾಗಿ ಅನುವಾದಿಸುತ್ತ ಮುಂದುಮುಂದಕ್ಕೆ ನಡೆದರು....

\begin{center}
\textbf{ಯೋಗವಿಜ್ಞಾನ *\supskpt{\footnote{\enginline{1. New Discoveries, Vol. 5, p.276}}}}
\end{center}

\begin{center}
(ಲಾಸ್ ಏಂಜಲಿಸ್ ಹೆರಾಲ್ಡ್, ೨೬ ಜನವರಿ ೧೯೦೦)
\end{center}

ಪೌರ್ವಾತ್ಯ ದಾರ್ಶನಿಕರಾದ ಸ್ವಾಮಿ ವಿವೇಕಾನಂದರು ಈ ಹೊತ್ತು (ಗುರುವಾರ, ಜನವರಿ ೨೫) ಬೆಳಗ್ಗೆ ಷೇಕ್ಸ್ಪಿಯರ್ ಕ್ಲಬ್ನಲ್ಲಿ ‘ಯೋಗವಿಜ್ಞಾನ’\footnote{2. ಇದರ ಪದಶಃ ವರದಿ ಲಭ್ಯವಿಲ್ಲ.} ಎಂಬ ವಿಷಯದ ಮೇಲೆ ಉಪನ್ಯಾಸ ಮಾಡಿದರು. ಪ್ರಕೃತಿಯಲ್ಲಿರವ ವ್ಯತ್ಯಾಸಗಳೆಲ್ಲ ಕೇವಲ ತರತಮ ದವೇ ಹೊರತು ಗುಣಾತ್ಮಕವಾದ ಯಾವ ವ್ಯತ್ಯಾಸವೂ ಇಲ್ಲ ಎಂದು ಅವರು ಹೇಳಿದರು. ಲೋಕದ ಚಾಲನ ಶಕ್ತಿಯೇ ಮನಸ್ಸು, ಅದೇ ಪರಮ ಎಂದರು.

\begin{center}
\textbf{ಲಾಸ್ ಏಂಜಲೀಸ್ ಹೋಮ್​ನಲ್ಲಿ ಸ್ವಾಮಿ ವಿವೇಕಾನಂದರು\supskpt{\footnote{\enginline{1. New Discoveries, Vol. 5, p.128-120}}}}
\end{center}

\begin{center}
(ಯೂನಿಟಿ, ಫೆಬ್ರವರಿ (?) ೧೯೦೦)
\end{center}

.... ಹೋಮ್​ನಲ್ಲಿ ಸ್ವಾಮಿಗಳಿಂದ\footnote{4. ಹೋಮ್​ ಆಫ್ ಟ್ರೂತ್ ಎಂಬಲ್ಲಿ ೧೮೯೯ ಡಿಸೆಂಬರ್ ೧೯೦೦ ಜನವರಿಯಲ್ಲಿ ಸ್ವಾಮೀಜಿಯವರು ಮಾಡಿದ ಎಂಟು ತರಗತಿಗಳ ಒಟ್ಟು ಸಾರಾಂಶವನ್ನು ಪತ್ರಿಕೆ ವರದಿ ಮಾಡಿದೆ. ಈ ತರಗತಿಗಳ ಪೈಕಿ ಆಧ್ಯಾತ್ಮಿಕ ಸಾಧನೆಗೆ ಸಲಹೆಗಳು’ ಎಂಬುದರ ಪದಶಃ ವರದಿ ಕೃತಿಶ್ರೇಣಿ ೬ ಪುಟ ೧೭೨ರಲ್ಲಿ ಪ್ರಕಟವಾಗಿದೆ.} ನಮಗೆ ದೊರಕಿದ ಎಂಟು ಉಪನ್ಯಾಸಗಳಲ್ಲಿ ಎಲ್ಲವೂ ಅತ್ಯಂತ ಆಸಕ್ತಿಕರವಾಗಿದ್ದುವು; ಆದರೆ ಅವರು (ಸ್ವರ್ಗದ) ರಾಜ್ಯಕ್ಕೆ ಹತ್ತಿರದ ದಾರಿಯನ್ನು ತೋರಿಸಲಿಲ್ಲ, ಮಾನಸಿಕ ಶಕ್ತಿಗಳನ್ನು ಪಡೆಯಲು ಸುಲಭಮಾರ್ಗವನ್ನು ತೋರಿಸಲಿಲ್ಲ ಎಂದು ಕೆಲವು ಅಸಂತೃಪ್ತರು ಅಸಮಾಧಾನವನ್ನು ವ್ಯಕ್ತಪಡಿಸಿದರು. ಬದಲಿಗೆ ಸ್ವಾಮಿಗಳು “ಮನೆಗೆ ಹೋಗಿ, ಚಾಕರಿಯ ಹೆಂಗಸು ನಿಮ್ಮೆಲ್ಲ ಅತ್ಯುತ್ತಮ ಚೈನಾ ಪಾತ್ರೆಗಳನ್ನು ಒಡೆದು ಹಾಕಿದರೂ ಒಂದು ತಿಂಗಳ ಕಾಲ ವಿಷಾದಿಸುವುದಿಲ್ಲವೆಂದು ನಿಮಗೆ ನೀವೇ ವಚನವಿತ್ತುಕೊಳ್ಳಿ” ಎಂದರು.

ಸ್ವಾಮಿ ವಿವೇಕಾನಂದರಲ್ಲಿ ವಿಶ್ವವಿದ್ಯಾನಿಲಯದ ಅಧ್ಯಕ್ಷರ ವಿದ್ವತ್ತು, ಆರ್ಚ್ ಬಿಷಪ್ರ ಘನತೆ, ಮುಕ್ತಸಹಜತೆಯುಳ್ಳ ಮಗುವೊಂದರ ಮನ ಗೆದ್ದುಕೊಳ್ಳುವ ಮುಗ್ಧ ದಿವ್ಯತೆ ಎಲ್ಲವೂ ಸಮ್ಮಿಳಿತವಾಗಿವೆ. ಕ್ಷಣಮಾತ್ರದ ಪೂರ್ವತಯಾರಿಯೂ ಇಲ್ಲದೆ ವೇದಿಕೆಯ ಮೇಲೆ ಬರುವ ಅವರು ಬಹು ಬೇಗನೆ ತಮ್ಮ ವಿಷಯದಾಳಕ್ಕೆ ಧುಮ್ಮಿಕ್ಕಿ ಬಿಡುತ್ತಾರೆ; ಕೆಲವೊಮ್ಮೆ ಅವರ ಮನಸ್ಸು ಗಾಢವಾದ ತತ್ತ್ವವಿಚಾರಗಳನ್ನು ಬಿಟ್ಟು ಕ್ರೈಸ್ತ ದೇಶಗಳಲ್ಲಿರುವ ಇಂದಿನ ಪರಿಸ್ಥಿತಿಯ ಕಡೆಗೆ ಹೊರಳಿದಾಗ ವಿಷಣ್ಣರಾಗುತ್ತಾರೆ - ಒಂದು ಕೈಯಲ್ಲಿ ಖಡ್ಗವನ್ನೂ ಇನ್ನೊಂದು ಕೈಯಲ್ಲಿ ಬೈಬಲ್ನ್ನೂ ಹಿಡಿದುಕೊಂಡು ಫಿಲಿಪಿನೋಗಳ ಸುಧಾರಣೆಗೆಂದು ಹೊರಟವರು, ಅಥವಾ ದಕ್ಷಿಣ ಆಫ್ರಿಕಾದಲ್ಲಿ ಒಬ್ಬನೇ ತಂದೆಯ ಮಕ್ಕಳನ್ನು ಪರಸ್ಪರರನ್ನು ಕತ್ತರಿಸಿಕೊಳ್ಳಲು ಬಿಡುವವರು - ಹೀಗೆ. ಇಂತಹ ಪ್ರಸಕ್ತ ಪರಿಸ್ಥಿತಿಯೊಂದಿಗೆ ಹೋಲಿಸುವುದಕ್ಕೆಂದು ಭಾರತದಲ್ಲಿ ಈ ಹಿಂದೆ ಕ್ಷಾಮ ಬಂದಾಗ ಹಸಿವಿನಿಂದ ಕಂಗೆಟ್ಟ ಜನರು ತಾವು ಸಾಕಿದ ಜಾನುವಾರುಗಳ ಪಕ್ಕದಲ್ಲೇ ಮಲಗಿ ಸತ್ತರೇ ಹೊರತು ಕೊಲ್ಲುವುದಕ್ಕೆಂದು ಒಂದು ಕೈಯನ್ನೂ ಮುಂದಕ್ಕೆ ತರಲಿಲ್ಲ ಎಂಬುದನ್ನು ಅವರು ವಿವರಿಸಿದರು. (ಇಂದು ಹಸಿವಿನಿಂದ ನರಳುತ್ತಿರುವ ಐವತ್ತು ಮಿಲಿಯ ಹಿಂದೂಗಳನ್ನು ನೆನಪಿಸಿಕೊಂಡು ಯೂನಿಟಿ ಓದುಗರು ಒಂದು ಶುಭಾ ಕಾಂಕ್ಷೆಯನ್ನು ಕಳುಹಿಸುವರೇ?)

ನಾವು ನೇರವಾಗಿ ಸ್ವಾಮಿಗಳಿಂದ ಏನನ್ನು ಕೇಳಿರುವೆವೋ ಅದನ್ನೇ ಕೊಡಲೆತ್ನಿಸು ವುದರ ಬದಲಾಗಿ, ಅವರ ಗುರುವಾದ ರಾಮಕೃಷ್ಣರು ಹೇಳಿದ ಕೆಲವು ಮಾತುಗಳನ್ನು ಅನುಬಂಧವಾಗಿ ಕೊಡುವೆನು; ಅವು ಸ್ವಾಮಿಗಳ ಬೋಧನೆಯ ಸ್ವರೂಪವನ್ನು ಇನ್ನೂ ಚೆನ್ನಾಗಿ ನಿರ್ದೇಶಿಸಬಲ್ಲವು. ಅವರ ಮುಖ್ಯ ಧ್ಯೇಯವು ಜನರನ್ನು ಸರಳವಾದ, ಶಾಂತವಾದ ಪೂರ್ಣಜೀವನವನ್ನು ಬಾಳುವಂತೆ ಪ್ರೋತ್ಸಾಹಿಸುವುದೆಂದು ತೋರುತ್ತದೆ; ಅಂತಹ ಜೀವನವೇ ಧರ್ಮವೇ ಹೊರತು, ಯಾವುದೋ ಪ್ರತ್ಯೇಕವಾಗಿರುವಂಥದಲ್ಲ.

ನಿಜವಾದ ತಾಯಿಗೆ ಅವರು ಅತ್ಯುನ್ನತ ಸ್ಥಾನವನ್ನು ಕೊಡುತ್ತಾರೆ; ನಾನೂ ಬೋಧಿಸುವೆನೆಂದು ಅಲ್ಲಿಂದಿಲ್ಲಿಗೆ ತಿರುಗುತ್ತಿರುವವರಿಗಿಂತ ಹೆಚ್ಚಿನ ಗೌರವಾನ್ವಿತೆಯೆಂದು ಭಾವಿಸುತ್ತಾರೆ. “ಯಾರು ಬೇಕಾದರೂ ಮಾತನಾಡಬಹುದು; ಆದರೆ ಒಂದು ಮಗುವನ್ನು ಲಾಲಿಸುವ ಕೆಲಸ ನನ್ನ ಮೇಲೆ ಬಿದ್ದರೆ, ನಾನು ಅದರ ಇರುವಿಕೆಯನ್ನು ಮೂರು ದಿನಗಳಿಗಿಂತ ಹೆಚ್ಚು ತಾಳಿಕೊಳ್ಳಲಾರೆ” ಎಂದು ಅವರು ಹೇಳಿದರು.

ನಾವು “ತಂದೆ” ಎನ್ನುವ ಹಾಗೆಯೇ, ಅವರು ಆಗಿಂದಾಗ್ಯೆ “ತಾಯಿ” ಎನ್ನುವರು; “ತಾಯಿ ನಮ್ಮನ್ನೆಲ್ಲ ಕಾಪಾಡುವಳು” ಎಂದೋ, “ತಾಯಿ ಎಲ್ಲವನ್ನೂ ನೋಡಿಕೊಳ್ಳುವಳು” ಎಂದೋ ಹೇಳುತ್ತಿರುವರು.

ಕ್ರಿಸ್ ಮಸ್ ದಿನದಂದು ಸ್ವಾಮಿಗಳಿಂದ “ಲೋಕಕ್ಕೆ ಕ್ರಿಸ್ತನ ಸಂದೇಶ” ಎಂಬ ಉಪನ್ಯಾಸವಿದ್ದಿತು; ಈ ವಿಷಯದ ಮೇಲೆ ಅದಕ್ಕಿಂತ ಮಿಗಿಲಾದ ಇನ್ನೊಂದನ್ನು ನಾನೆಂದೂ ಕೇಳಿಲ್ಲ. ಯಾವ ಕ್ರೈಸ್ತಪಾದ್ರಿಯೂ ಈ ಹಿಂದೂ ಪ್ರಬೋಧಕರ ಹಾಗೆ ಜೀಸಸ್ನ ವ್ಯಕ್ತಿತ್ವವನ್ನು ಅಷ್ಟೊಂದು ನಿರರ್ಗಳವಾಗಿ, ಅಷ್ಟೊಂದು ಶಕ್ಯಿಯುತವಾಗಿ, ನಮ್ಮ ಮನಸ್ಸಿನಲ್ಲಿ ಆತನ ಬಗ್ಗೆ ಅತ್ಯುನ್ನತ ಗೌರವ ಮೂಡುವಂತೆ ಚಿತ್ರಿಸಿರಲಿಲ್ಲ. ಈ ದೇಶದಲ್ಲಿ ತನ್ನ ಕಂದು ಬಣ್ಣದ ಚರ್ಮದಿಂದಾಗಿ ಹೋಟೆಲುಗಳಲ್ಲಿ ಪ್ರವೇಶವನ್ನು ನಿರಾಕರಿಸಿದ್ದನ್ನು, ಕ್ಷೌರಿಕರು ಸಹ ಕೆಲವೊಮ್ಮೆ ಕ್ಷೌರಮಾಡಲು ನಿರಾಕರಿಸಿದ್ದನ್ನು ಹೇಳಿದ ಅವರು - ಈ ನಮ್ಮ ಅಕ್ರೈಸ್ತ ಅನಾಗರಿಕ ಸೋದರ - ನಮ್ಮ ಗುರುವಾದ ಜೀಸಸ್ ಸಹ ಪೌರ್ವಾತ್ಯ ಎಂಬ ವಾಸ್ತವವನ್ನು ಮರೆಯದೆ ನಮಗೆ ಆಗಾಗ್ಯೆ ನೆನಪಿಸುವುದರಲ್ಲಿ ಅಚ್ಚರಿಯೇನಿದೆ?

\begin{center}
\textbf{ಹಿಂದೂ ಸಂನ್ಯಾಸಿ ಉಪನ್ಯಾಸ ಮಾಡಿದರು\supskpt{\footnote{\enginline{1. New Discoveries, Vol. 5, pp.315-16}}}}
\end{center}

\begin{center}
(ಸ್ಯಾನ್ ಫ್ರಾನ್ಸಿಸ್ಕೋ ಕ್ರಾನಿಕಲ್, ೨೪ ಫೆಬ್ರವರಿ ೧೯೦೦)
\end{center}

\begin{center}
\textbf{“ವಿಶ್ವಧರ್ಮದ ಪರಿಕಲ್ಪನೆ” ಎಂಬುದು ಸ್ವಾಮಿ ವಿವೇಕಾನಂದರ ವಿಷಯ}
\end{center}

ಕಳೆದ ಸಂಜೆ ಗೋಲ್ಡನ್ ಗೇಟ್ ಹಾಲ್ನಲ್ಲಿ ಹಿಂದೂ ಸಂನ್ಯಾಸಿ ಸ್ವಾಮಿ ವಿವೇಕಾನಂದರು ತಮ್ಮ “ವಿಶ್ವಧರ್ಮದ ಪರಿಕಲ್ಪನೆ” ಎಂಬ ಉಪನ್ಯಾಸ\footnote{2. ಇದರ ಪದಶಃ ವರದಿ ಲಭ್ಯವಿಲ್ಲ} ದ ಮೂಲಕ ಸಭೆಯೊಂದನ್ನು ಒಂದೂವರೆ ಗಂಟೆಗಳ ಕಾಲ ರಂಜಿಸಿದರು....

ಇತಿಹಾಸದ ಪ್ರಾರಂಭದಿಂದಲೂ ಧರ್ಮವು ನಡೆದು ಬಂದ ದಾರಿಯನ್ನು ಚಿತ್ರಿಸಿದ ಅವರು, ಮತಪಂಥಗಳ ಅಸ್ತಿತ್ವವನ್ನು ಕುರಿತು ಹೇಳಿದರು. ವಿಭಿನ್ನ ಪಂಥಗಳು ಪುರಾತನ ಕಾಲದಿಂದಲೂ ಇದ್ದದ್ದೇ ಎಂದರು. ಕಾಲ ಸರಿದಂತೆ, ಪಂಥಗಳ ನಡುವೆ ಹೆಚ್ಚುಗಾರಿಕೆಗಾಗಿ ಬಗೆಬಗೆಯ ಸ್ಪರ್ಧೆಗಳು ಏರ್ಪಟ್ಟವು. ಚರಿತ್ರೆ ಎಂದರೆ ಧರ್ಮದ ಸೋಗಿನಡಿಯಲ್ಲಿ ಪುನರಾವರ್ತಿಸಿದ ಕಗ್ಗೊಲೆಗಳಲ್ಲದೆ ಬೇರೆಯಲ್ಲ ಎಂದವರು ಉದ್ಘೋಷಿಸಿದರು. ಮಾನವ ಮನಸ್ಸುಗಳು ವಿಕಾಸ ಹೊಂದಿದಂತೆಲ್ಲ ಮೂಢನಂಬಿಕೆಗಳು ಬಹು ಬೇಗನೆ ಗತಕಾಲದ ಸಂಗತಿಗಳಾಗುತ್ತಿವೆ ಎಂದು ಅವರು ಅಭಿಪ್ರಾಯಪಟ್ಟರು. ಮಾನವರಲ್ಲಿ ಚಿಂತನೆಯ ಔದಾರ್ಯ ಈಗ ಹೆಚ್ಚು ತ್ತಿದೆ. ತತ್ತ್ವದ ಆಳ ಅಧ್ಯಯನ ಪ್ರವೃತ್ತಿ ಹೆಚ್ಚು ತ್ತಿದೆ; ನಿಜವಾದ ತತ್ತ್ವದ ಮೂಲಾಂಶಗಳ ಮೂಲಕವಷ್ಟೇ ಧರ್ಮದ ನಿಗೂಢ ಆಂತರ್ಯವನ್ನರಿಯುವುದು ಸಾಧ್ಯ. ಎಲ್ಲಿಯವರೆಗೆ ಮನುಷ್ಯ ಇತರರಿಗೆ ಎಲ್ಲ ವಿಷಯಗಳ ಬಗ್ಗೆ ಮುಕ್ತ ನಂಬಿಕೆಯ ಹಕ್ಕನ್ನು ಕೊಡಮಾಡುವುದಿಲ್ಲವೋ, ಸತ್ಯವು ಯಾವ ರೂಪದಲ್ಲೇ ಪ್ರತ್ಯಕ್ಷವಾದರೂ ನಂಬುವುದಕ್ಕೆ ಹಿಂಜರಿಯುವುದಿಲ್ಲವೋ ಅಲ್ಲಿಯವರೆಗೆ ಪ್ರಪಂಚದಲ್ಲಿ ಯಾವುದೇ ರೀತಿಯ ವಿಶ್ವಧರ್ಮದ ಆವಿರ್ಭಾವ ಸಾಧ್ಯವಿಲ್ಲವೆಂದು ಅವರು ಉದ್ಘೋಷಿಸಿದರು. ಯಾವುದಾದರೊಂದು ಸಂಘಟನೆ ಅದನ್ನು ಪ್ರವರ್ತಿಸಲಾರದು; ಮಾನವನ ಬುದ್ಧಿಶಕ್ತಿ ಅಭಿವೃದ್ಧಿಯಾದಂತೆಲ್ಲ ಅದು ಸಹಜಪ್ರೇರಣೆಯಿಂದ ಮಾತ್ರವೇ ಬೆಳೆಯುತಕ್ಕುದು.

\begin{center}
\textbf{ವೇದಾಂತ - ಅದು ಏನು, ಏನಲ್ಲ\supskpt{\footnote{\enginline{1. New Discoveries, Vol. 5, pp.329-31}}}}
\end{center}

\begin{center}
\textbf{ಹಿಂದೂಗಳ ಧರ್ಮದ ಬಗ್ಗೆ ಸ್ವಾಮಿ ವಿವೇಕಾನಂದರ ಉಪನ್ಯಾಸ}
\end{center}

\begin{center}
(ಓಕ್ಲ್ಯಾಂಡ್ ಟ್ರಿಬ್ಯೂನ್, ೨೬ ಫೆಬ್ರವರಿ ೧೯೦೦)
\end{center}

\begin{center}
\textbf{ಸುಳ್ಳು ಹೇಳದೆ, ರಾಜಿಮಾಡಿಕೊಳ್ಳದೆ ಬೋಧಿಸಬಹುದಾದ ಪಂಥ ಅದೊಂದೇ ಎನ್ನುತ್ತಾರವರು}
\end{center}

ಇಂದು ಸಂಜೆ ಪ್ರಥಮ ಯೂನಿಟೇರಿಯನ್ ಚರ್ಚ್ನ ಧರ್ಮಗಳ ಸಮ್ಮೇಳನದಲ್ಲಿ ನವಯುಗದ ಪ್ರಪಂಚದಲ್ಲಿ ಬ್ರಾಹ್ಮಣ ಧರ್ಮದ ಅಥವಾ ವೇದಾಂತದ ಹಕ್ಕುಗಳನ್ನು ಆ ಧರ್ಮದ ವಿಶಿಷ್ಟ ನಿರರ್ಗಳ ಪ್ರತಿಪಾದಕರಾದ ಸ್ವಾಮಿ ವಿವೇಕಾನದಂದರು\footnote{2. ಆಧುನಿಕ ಜಗತ್ತಿಗೆ ವೇದಾಂತದ ಉದ್ವೋಷ ಎಂಬ ಈ ಉಪನ್ಯಾಸದ ಪದಶಃ ವರದಿ ಲಭ್ಯವಿಲ್ಲ. ಓ ಕ್ಲ್ಯಾಂಡ್ ಟ್ರಿಬ್ಯೂನ್ನಲ್ಲಿ ಪ್ರಕಟವಾಗಿರುವ ಸ್ವಾಮಿಗಳ ನೇರ ಉದ್ಧರಣೆಗಳಲ್ಲಿ ಕೆಲವನ್ನು ಮಾತ್ರ ಒಳಗೊಂಡಿರುವ ಸ್ವಲ್ಪ ಭಿನ್ನವಾದ ವರದಿಗಾಗಿ ಸ್ವಾಮಿಗಳ ನೇರ ಉದ್ಧರಣೆಗಳಲ್ಲಿ ಕೆಲವನ್ನು ಮಾತ್ರ ಒಳಗೊಂಡಿರುವ ಸ್ವಲ್ಪ ಭಿನ್ನವಾದ ವರದಿಗಾಗಿ ಕೃತಿಶ್ರೇಣಿ ೭ ಪುಟ ೩೮೫ ನೋಡಿ.} ಎತ್ತಿಹಿಡಿದರು...

ಪ್ರಾಚೀನ ಹಿಂದೂ ಶಾಸ್ತ್ರಗ್ರಂಥಗಳಾದ ವೇದಗಳ ಧರ್ಮವೇ ವೇದಾಂತ ಎಂದು ಅವರು ತಮ್ಮ ಶ್ರೋತೃಗಳಿಗೆ ವಿವರಿಸುತ್ತ, “ಅದು ಧರ್ಮಗಳ ತಾಯಿ” ಎಂದು ಒತ್ತಿ ಹೇಳಿದರು. “ಗ್ರಂಥವೊಂದು ಆದಿ ಅಂತ್ಯಗಳಿಲ್ಲದೆ ಇರುವುದಾದರೂ ಹೇಗೆ, ಇದು ಹಾಸ್ಯಾಸ್ಪದ ಎಂದು ನಿಮಗೆ ಅನ್ನಿಸಬಹುದು; ಆದರೆ ವೇದಗಳೆಂದರೆ ಪುಸ್ತಕಗಳು ಎಂದರ್ಥವಲ್ಲ. ಬೇರೆ ಬೇರೆ ವ್ಯಕ್ತಿಗಳಿಂದ ಬೇರೆ ಬೇರೆ ಕಾಲಗಳಲ್ಲಿ ಆವಿಷ್ಕರಿಸಲ್ಪಟ್ಟು ಕ್ರೋಢೀಕೃತವಾದ ಆಧ್ಯಾತ್ಮಿಕ ನಿಯಮಗಳ ಸಮುದಾಯ ಸಂಪತ್ತು ಅವುಗಳು. ತಾನೊಂದು ಚೇತನ ಎನ್ನುವುದು ಹಿಂದೂವಿನ ನಂಬಿಕೆ. ಅವನನ್ನು ಖಡ್ಗವು ಕತ್ತರಿಸಲಾರದು, ಬೆಂಕಿ ಸುಡಲಾರದು, ನೀರು ಕರಗಿಸಲಾರದು, ಗಾಳಿ ಶೋಷಿಸಲಾರದು. ಪ್ರತಿಯೊಂದು ಆತ್ಮವೂ ಪರಿಧಿ ಎಲ್ಲಿದೆಯೆಂದು ಗೊತ್ತಿಲ್ಲದ ವೃತ್ತ; ದೇಹದಲ್ಲಿರುವುದು ಅದರ ಕೇಂದ್ರ ಮಾತ್ರ ಎಂದವನು ನಂಬುತ್ತಾನೆ. ಮೃತ್ಯುವೆಂದರೆ ಈ ಕೇಂದ್ರವು ಒಂದು ದೇಹದಿಂದ ಇನ್ನೊಂದು ದೇಹಕ್ಕೆ ಆಗುವ ವರ್ಗಾವಣೆ. ನಾವೆಲ್ಲರೂ ದೇವರ ಮಕ್ಕಳು. ದ್ರವ್ಯವಸ್ತು ಏನಿದ್ದರೂ ನಮ್ಮ ಸೇವಕ.

“ಹಿಂದಿನ ಅಣಕಕ್ಕೆ ಒಂದು ರೀತಿಯ ಪ್ರತಿರೋಧವಾಗಿರುವುದು ವೇದಾಂತ. ಕೆಲವರಂತೂ ಅದೆಷ್ಟು ಪ್ರಾಯೋಗಿಕರಾಗಿರುತ್ತಾರೆಂದರೆ, ತಮ್ಮ ತಲೆ ಕಡಿದುಕೊಂಡುಬಿಟ್ಟರೆ ಮುಕ್ತಿ ಸಿಗುತ್ತದೆ ಎಂದಾದರೆ, ಅದನ್ನು ಮಾಡಲು ಎಷ್ಟೋ ಜನ ಸಿದ್ಧರಾಗಿರುತ್ತಾರೆ. ಇದೆಲ್ಲ ಬಾಹ್ಯದ್ದು; ನೀವು ನಿಮ್ಮ ದೃಷ್ಟಿಯನ್ನು ಅಂತರ್ಮುಖವಾಗಿಸಿ ನಿಮ್ಮ ಅಂತರಂಗವನ್ನು

ಅರಿತುಕೊಳ್ಳಬೇಕು. ಆತ್ಮವೆಂದರೆ ಸ್ವಯಂಸ್ಥಾಯಿಯಾದ ಚೇತನ. ಸತ್ತ ಮೇಲೆ ಆತ್ಮ ಎಲ್ಲಿಗೆ ಹೋಗುತ್ತದೆ? ಭೂಮಿ ಎಲ್ಲಿಗೆ ಬೀಳುತ್ತದೆ? ಆತ್ಮ ಎಲ್ಲಿಗೆ ತಾನೆ ಹೋದೀತು? ಈಗಾಗಲೇ ಅದಿಲ್ಲದೆ ಇರುವ ಜಾಗ ಯಾವುದು? ವೇದಾಂತದ ಮಹತ್ತರ ಸಾಧನೆಯೇ ಈ ಆತ್ಮವನ್ನು ಪರಿಗಣಿಸಿದ್ದು. ಮನುಷ್ಯನೇ, ನಿನ್ನಲ್ಲಿ ನೀನು ನಂಬಿಕೆ ಇಡು. ಎಲ್ಲರಲ್ಲಿರುವ ಆತ್ಮ ಒಂದೇ. ಅದೆಂದರೆ ಸಂಪೂರ್ಣ ಪರಿಶುದ್ಧತೆ ಹಾಗೂ ಪರಿಪೂರ್ಣತೆ; ನಾವು ಹೆಚ್ಚು ಹೆಚ್ಚು ಶುದ್ಧರಾದಂತೆಲ್ಲ, ಪರಿಪೂರ್ಣರಾದಂತೆಲ್ಲ ಹೆಚ್ಚು ಹೆಚ್ಚು ಶುದ್ಧತೆಯನ್ನು, ಪರಿಪೂರ್ಣತೆಯನ್ನು ನೋಡುವೆವು “ಪ್ರಬೋಧಕ ಪೋರನೊಬ್ಬ ‘ಹೇ ದೇವ, ನಾನೊಂದು ತೆವಳುತ್ತಿರುವ ಕ್ರಿಮಿ ಅಷ್ಟೆ!’ ಎಂದು ಕೂಗುತ್ತಿದ್ದರೆ, ಅವನು ಅಲ್ಲೇ ಇದ್ದುಕೊಂಡು ತನ್ನ ಗೂಡೊಳಕ್ಕೆ ತೆವಳಬೇಕಾಗುತ್ತದೆ. ಅವನ ಕೂಗು ಲೋಕದಲ್ಲಿರುವಯಾತನೆಯನ್ನು ಇನ್ನಷ್ಟು ಹೆಚ್ಚಿಸಬಹುದು. ನಿಮ್ಮ ಪತ್ರಿಕೆಯೊಂದರಲ್ಲಿ ನಾನು ‘ಕ್ರಿಸ್ತನು ಹೇಗೆ ಪತ್ರಿಕೆಯ ಸಂಪಾದನೆ ಕೆಲಸ ಮಾಡಬಹುದು!’ ಎಂದು ಓದಿ ನಗುವಂತಾಯಿತು. ಎಂತಹ ಮೂರ್ಖತನ, ಕ್ರಿಸ್ತ ಹೇಗೆ ಅಡುಗೆ ಮಾಡಬಹದು? ಹೀಗಿದ್ದರೂ ನೀವು ಪಶ್ಚಿಮದ ಮುಂದುವರೆದ ಜನರಲ್ಲವೆ! ಕ್ರಿಸ್ತ ಇಲ್ಲಿಗೆ ಬಂದರೆ ನೀವು ಅಂಗಡಿ ಮುಚ್ಚಿ ಅವನೊಂದಿಗೆ ಬೀದಿಗಿಳಿದು ಬಡವರಿಗೆ ಮತ್ತು ತುಳಿತಕ್ಕೊಳಗಾದವರಿಗೆ ಸಹಾಯ ಮಾಡುವಿರಿ. ಸುಳ್ಳುಗಳಿಲ್ಲದೆ, ಶಾಸ್ತ್ರಗಳನ್ನವಲಂಬಿಸದೆ, ರಾಜಿ ಮಾಡಿಕೊಳ್ಳದೆ ಕಲಿಸ ಬಹುದಾದ ಧರ್ಮವೆಂದರೆ ವೇದಾಂತ ಒಂದೇ.”

\begin{center}
\textbf{ನಿಜವಾದ ಧರ್ಮ\supskpt{\footnote{\enginline{1. New Discoveries, Vol. 6, pp.405-6}}}}
\end{center}

\begin{center}
(ಅಲಮೇಡ ಎನ್ಸಿನಲ್, ೫ ಏಪ್ರಿಲ್ ೧೯೦೦)
\end{center}

\begin{center}
\textbf{ಹಿಂದೂ ತತ್ತ್ವಜ್ಞಾನಿ ನೀಡಿದ ಕಲ್ಪನೆಗಳು}
\end{center}

ಕಳೆದ ಸಂಜೆ ಸ್ವಾಮಿ ವಿವೇಕಾನಂದರು ಮೂರು ಉಪನ್ಯಾಸಗಳ ಸರಮಾಲೆಯಲ್ಲಿ “ಧಾರ್ಮಿಕ ಕಲ್ಪನೆಗಳ ಬೆಳವಣಿಗೆ”\footnote{2. ಇದರ ಪದಶಃ ವರದಿ ಲಭ್ಯವಿಲ್ಲ.} ಎಂಬ ಮೊದಲ ಉಪನ್ಯಾಸವನ್ನು ಟಕರ್ ಹಾಲ್ನಲ್ಲಿ ನೀಡಿದರು.

ತಮ್ಮ ಧರ್ಮಗಳ ಮೂಲದ ಬಗ್ಗೆ ಸಾಂಪ್ರದಾಯಿಕ ಕ್ರೈಸ್ತರ, ಮಹಮ್ಮದೀಯರ ಮತ್ತು ಹಿಂದೂಗಳ ಮನಸ್ಸಿನಲ್ಲಿರುವ ಸಮಾನ ಅಭಿಪ್ರಾಯಗಳನ್ನು ಕುರಿತು ಭಾಷಣಕಾರರು ಸಂಗ್ರಹವಾಗಿ ಹೇಳಿದರು. ದೂರದಿಂದಲೇ ಲೋಕದ ಆಗುಹೋಗುಗಳ ಮೇಲೆ ಪ್ರಭಾವ ಬೀರುತ್ತಿರುವ ಅವುಗಳನ್ನು ನಿಯಂತ್ರಿಸುತ್ತಿರುವ ದೇವನೊಬ್ಬನಿಂದ ತಮ್ಮ ಪ್ರವಾದಿ ನಿಗೂಢ ರೀತಿಯಲ್ಲಿ ಅಂತಃ ಪ್ರೇರಣೆ ಪಡೆದಿದ್ದನೆಂದು ಇವರಲ್ಲಿ ಒಬ್ಬೊಬ್ಬರೂ ನಂಬುವರು. ಇದಕ್ಕೆ ಪ್ರತಿಯಾಗಿ, ನವಯುಗದ ವೈಜ್ಞಾನಿಕ ಮನಸ್ಸು, ವಿದ್ಯಮಾನಗಳಿಗೆ ಕಾರಣಗಳನ್ನು ಬಾಹ್ಯ ಅಥವಾ ಪ್ರಕೃತ್ಯತೀತ ಮೂಲಗಳಲ್ಲಿ ಅರಸುವ ಬದಲು, ಆಯಾ ಸಂಗತಿಗಳಲ್ಲಿ, ಅವುಗಳ ಪರಿಸ್ಥಿತಿಗಳಲ್ಲಿಯೇ ಅರಸಲು ಪ್ರಯತ್ನಿಸುತ್ತಿದೆ.

ಮೇಲ್ನೋಟಕ್ಕೆ ಈ ಸಂಶೋಧನಾ ವಿಧಾನವು ಧರ್ಮದ ಪ್ರಧಾನ ಸಂಗತಿಗಳನ್ನು ತೆಗೆದುಕೊಳ್ಳುವಂತೆ ತೋರಿದರೂ, ನಿಜವಾಗಿ ಅದು ದೇವರಿಗೆ ಆರೋಪಿ ಸಲಾದ ಆಧ್ಯಾತ್ಮಿಕ ಲಕ್ಷಣಗಳು ಮತ್ತು ಸ್ವರ್ಗನರಕಗಳನ್ನು ಹುಟ್ಟುಹಾಕುವ ಮನಸ್ಸಿನ ಅವಸ್ಥೆಗಳು ಇವೆಲ್ಲ ತನ್ನಲ್ಲೇ ಇರುವುವೆಂಬುದನ್ನು ಕಂಡುಕೊಳ್ಳುವಂತೆ ಮಾಡಿತು. ಈ ನವೀನ ವೈಚಾರಿಕ ಶೋಧನೆ ಬೈಬಲ್, ಕೊರಾನ್, ವೇದಗಳಂತಹ ಪುರಾತನ ಧಾರ್ಮಿಕ ಗ್ರಂಥಗಳಿಂದ ಪರಂಪರಾಗತವಾಗಿ ಬಂದಿರುವ ಅದೆಷ್ಟೋ ಸಂಗತಿಗಳನ್ನು ವಿರೋಧಿಸುವಂತೆ ಕಂಡರೂ, ಅಂತಹ ವಿರೋಧ ಬಹುಮಟ್ಟಿಗೆ ತೋರಿಕೆಯದೇ ಹೊರತು ನಿಜವಲ್ಲ. ಏಕೆಂದರೆ, ಪುರಾತನ ಪ್ರಬೋಧಕರ ಮತ್ತು ಪ್ರವಾದಿಗಳ ಅನುಭವದ ಗ್ರಹಿಕೆ ನಿಜ. ಆದರೆ ಅವರ ಅನುಭವಗಳು ಹಿಂದೆಂದೂ ಕಾಣದಿದ್ದ, ಅರಿವಿಗೆ ಬಾರದಿದ್ದ, ಆದರೆ ತಮ್ಮ ಆತ್ಮದೊಳಗಿನಿಂದಲೇ ಅನಾವರಣವಾದ ಉನ್ನತಿಯ ಬೆಳಕು ಎಂಬುದನ್ನರಿಯುವ ಬದಲು, ಅದನ್ನು ಬಾಹ್ಯ ಮಾಧ್ಯಮಗಳಿಗೆ ಆರೋಪಿಸುವುದರಲ್ಲಿ ತಪ್ಪಾಗಿದೆ.

ಭಾಷಣಕಾರರು ಸ್ವರ್ಗನರಕಾದಿಗಳ ಬಗ್ಗೆ, ಅಪರಕ್ರಿಯಾಪದ್ಧತಿಗಳ ಬಗ್ಗೆ ಕೆಲವು ಸಾಮಾನ್ಯ ನಂಬಿಕೆಗಳನ್ನು ಹಾಗೂ ಪೂರ್ವಿಕರ ಮನಸ್ಸಿನ ಮೇಲಾದ ಪ್ರಭಾವ ಹೇಗೆ ನಮ್ಮಸುತ್ತಣ ಪ್ರಕೃತಿಶಕ್ತಿಗಳ ಮೂರ್ತೀಕರಣಕ್ಕೆ ಕಾರಣವಾಯಿತು ಎಂಬುದನ್ನು ವಿವರಿಸಿದರು....

