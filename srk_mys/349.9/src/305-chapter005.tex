
\chapter{“ಈಶಾನುಸರಣ” ಪುಸ್ತಕದ ಬಗ್ಗೆ ಸ್ವಾಮಿ ವಿವೇಕಾನಂದರು ಬರೆದುಕೊಂಡ ಟಿಪ್ಪಣಿ.}
%\protect\footnote{1. ಸ್ವಾಮಿ ವಿವೇಕಾನಂದರು ಥಾಮಸ್ ಏ. ಕೆಂಪಿಸ್ನ \enginline{“Imitation of christ”}ಪುಸ್ತಕವನ್ನು “ಈಶಾನುಸರಣ” ಎಂಬ ಹೆಸರಿನಲ್ಲಿ ಬಂಗಾಳಿಗೆ ಅನುವಾದಿಸಿದ್ದಾರೆ.}

\begin{center}
ಭಾಗ - ೧
\end{center}

\begin{center}
ಅಧ್ಯಾಯ - ೧
\end{center}

\begin{center}
(ಈ ಶಾನುಸರಣೆಯನ್ನೂ ಪ್ರಾಪಂಚಿಕತೆಯ ತಿರಸ್ಕಾರವನ್ನೂ ಕುರಿತು.)
\end{center}

೧. “ಯಾರು ನನ್ನನ್ನು ಅನುಸರಿಸುತ್ತಾನೊ ಅವನು ಕತ್ತಲೆಯನ್ನು ಪ್ರವೇಶಿಸುವುದಿಲ್ಲ” - ಎಂದು ಪ್ರಭು ಹೇಳುತ್ತಾನೆ \enginline{(John 8.12)(The imitation of christ V.1.)}

ಸ್ವಾಮಿ ವಿವೇಕಾನಂದರ ಟಿಪ್ಪಣಿ: ಭಗವದ್ಗೀತೆ ೭.೧೪ -

ದೈವೀ ಹ್ಯೇಷಾ ಗುಣಮಯೀ ಮಮ ಮಾಯಾ ದುರತ್ಯಯಾ~।

ಮಾಮೇವ ಯೇ ಪ್ರಪದ್ಯಂತೇ ಮಾಯಾಮೇತಾಂ ತರಂತಿ ತೇ~॥

ಸ್ವಾಮಿ ವಿವೇಕಾನಂದರ ಅನುವಾದ: ನನ್ನ ದಿವ್ಯವಾದ ಈ ಮಾಯೆಯು ಗುಣಮಯವಾದುದು ಮತ್ತು ದಾಟಲು ಅತ್ಯಂತ ಕಷ್ಟಕರವಾದುದು. ಆದರೆ ನನ್ನನ್ನು ಆಶ್ರಯಿಸುತ್ತಾರೊ ಅವರು ಭವಸಾಗರವನ್ನು ದಾಟುತ್ತಾರೆ.

೨. ಯೇಸು ಕ್ರಿಸ್ತನ ಜೀವನವನ್ನು ಕುರಿತು ಚಿಂತಿಸುವುದೇ ನಮ್ಮ ಪ್ರಧಾನ ಸಾಧನೆಯಾಗಲಿ \enginline{(The Imitation of christ v.1.)}

ಸ್ವಾಮಿ ವಿವೇಕಾನಂದರ ಟಿಪ್ಪಣಿ: ‘ಅಧ್ಯಾತ್ಮರಾಮಾಯಣ’.

ಧ್ಯಾತ್ವೈವ ಮಾತ್ಮಾನಮಹರ್ನಿಶಂ ಮುನಿಃ~।

ತಿಷ್ಠೇತ್ಸದಾ ಮುಕ್ತಸಮಸ್ತಬಂಧನಃ~॥

ಪ್ರಕಾಶಕರ ಅನುವಾದ: ಹಗಲು ರಾತ್ರಿ ಆತ್ಮನನ್ನು ಕುರಿತು ಧ್ಯಾನಿಸುತ್ತ ಮುನಿಯು ಎಲ್ಲ ಬಂಧನಗಳಿಂದ ಮುಕ್ತನಾಗಿರಲಿ.

೩. ಕ್ರಿಸ್ತನ ತತ್ತ್ವವು ಸಂತರ ತತ್ತ್ವಗಳೆಲ್ಲಕ್ಕಿಂತ ಮಿಗಿಲಾದುದು - ಆತ್ಮಭಾವವುಳ್ಳ ವನು ಅದರಲ್ಲಿ ಅಡಗಿರುವ ಅಮೃತಾಹಾರವನ್ನು ಪಡೆಯುತ್ತಾನೆ. \enginline{(The Imitation of christ v.2.)}

ಸ್ವಾಮಿ ವಿವೇಕಾನಂದರ ಟಿಪ್ಪಣಿ: ಇಸ್ರೇಲಿಗಳು ಮರುಭೂಮಿಯಲ್ಲಿ ಆಹಾರವಿಲ್ಲದೆ ನರಳುತ್ತಿದ್ದಾಗ ಭಗವಂತನು ಅವರ ಮೇಲೆ ಒಂದು ಬಗೆಯ “ಆಹಾರದ” ಮಳೆಗರೆದನು.

೪. ಆದರೆ ಹೀಗೂ ಉಂಟು: ಕ್ರಿಸ್ತನ ಸಂದೇಶವನ್ನು ಕೇಳಿದ ಅನೇಕರು ಅದರಿಂದ ಪ್ರಭಾವಿತರಾಗುವುದಿಲ್ಲ. ಏಕೆಂದರೆ ಅವರು ಕ್ರಿಸ್ತನ ಭಾವವನ್ನು ಹೊಂದಿರುವುದಿಲ್ಲ. ಯಾರು ಸಂಪೂರ್ಣವಾಗಿಯೂ ಹೃತ್ಪೂರ್ವಕವಾಗಿಯೂ ಕ್ರಿಸ್ತನ ಮಾತುಗಳನ್ನು ಅರ್ಥ ಮಾಡಿಕೊಳ್ಳುತ್ತಾರೆಯೊ, ಅವರು ತಮ್ಮ ಜೀವನವನ್ನು ಕ್ರಿಸ್ತನ ಜೀವನಕ್ಕೆ ಹೊಂದಾಣಿಕೆ ಯಾಗುವಂತೆ ಮಾಡಲು ಪ್ರಯತ್ನಿಸಬೇಕು. \enginline{(The Imitation of christ v.2.)}

ಸ್ವಾಮಿ ವಿವೇಕಾನಂದರ ಟಿಪ್ಪಣಿ: ಭಗವದ್ಗೀತೆ ೨.೨೯

ಶ್ರುತ್ವಾಪ್ಯೇನಂ ವೇದ ನ ಚೈವ ಕಶ್ಚಿತ್~।

ಸ್ವಾಮಿ ವಿವೇಕಾನಂದರ ಅನುವಾದ: ಬೇರೆಯವರು ಅದರ ಬಗ್ಗೆ ಕೇಳಿಯೂ ಅರ್ಥಮಾಡಿಕೊಳ್ಳುವುದಿಲ್ಲ.

ವಿವೇಕ ಚೂಡಾಮಣಿ, ೬೨

ನ ಗಚ್ಛತಿ ವಿನಾ ಪಾನಂ ವ್ಯಾಧಿರೌಷಧಶಬ್ದತಃ

ವಿನಾ ಪರೋಕ್ಷಾನುಭವಂ ಬ್ರಹ್ಮಶಬ್ದೈರ್ನ ಮುಚ್ಯತೇ~।

ಪ್ರಕಾಶಕರ ಅನುವಾದ: ಕೇವಲ ಔಷಧ ಶಬ್ದವನ್ನು ಉಚ್ಚರಿಸಿದರೆ ರೋಗವು ಗುಣವಾಗುವುದಿಲ್ಲ; ಔಷಧವನ್ನು ಸೇವಿಸಬೇಕು. ಹಾಗೆಯೇ ಭಗವದ್ ಸಾಕ್ಷಾತ್ಕಾರವಿಲ್ಲದೆ ಕೇವಲ ಬ್ರಹ್ಮಶಬ್ದದ ಉಚ್ಚಾರಣೆಯಿಂದ ಮುಕ್ತಿ ದೊರೆಯುವುದಿಲ್ಲ.

ಮಹಾಭಾರತ, ೧೨.೩೦.೯೧

ಶ್ರುತೇನ ಕಿಂ ಯೇನ ನ ಧರ್ಮಮಾಚರೇತ್~।

ಪ್ರಕಾಶಕರ ಅನುವಾದ: ಧರ್ಮವನ್ನು ಆಚರಿಸದೆ ವೇದಾಧ್ಯಯನದಿಂದ ಏನು ಪ್ರಯೋಜನ?

೫. ನೀನು ನಮ್ರತಾ ಶೂನ್ಯನಾಗಿದ್ದು ಅದರ ಮೂಲಕ ತ್ರಿಮೂರ್ತಿಗಳನ್ನು ಅಸಂತೋಷಪಡಿಸುತ್ತಿದ್ದರೆ, ತ್ರಿಮೂರ್ತಿಗಳ ವಿಷಯವನ್ನು ಕುರಿತುವಾದಿಸುವುದರಿಂದ ಏನು ಪ್ರಯೋಜನ? \enginline{(The Imitation of christ v.3.)}

ಸ್ವಾಮಿ ವಿವೇಕಾನಂದರ ಟಿಪ್ಪಣಿ:

ಕ್ರಿಶ್ಚಿಯನರ ಪ್ರಕಾರ ದೇವರು \enginline{(Father)}, ಪವಿತ್ರಾತ್ಮ \enginline{(Holy Ghost)} ಮತ್ತು ಪುತ್ರ \enginline{(Son)} ಇವರು ಒಂದೇ ಸತ್ಯದ ಮೂರು ರೂಪಗಳು.

೬. ಕೇವಲ ಶಬ್ದಗಳು ವ್ಯಕ್ತಿಯನ್ನು ಪವಿತ್ರನನ್ನಾಗಿಯೂ ನ್ಯಾಯಪರನನ್ನಾಗಿಯೂ ಮಾಡುವುದಿಲ್ಲ; ಸದ್ಗುಣ ಸಂಪನ್ನವಾದ ಜೀವನದಿಂದ ಅವನು ಭಗವಂತನಿಗೆ ಪ್ರಿಯ ನಾಗುತ್ತಾನೆ. \enginline{(The Imitation of christ V.iii.)}

ಸ್ವಾಮಿ ವಿವೇಕಾನಂದರ ಟಿಪ್ಪಣಿ (ವಿವೇಕ ಚೂಡಾಮಣಿ - ೫೮)

ವಾಗ್ವೈಖರೀ ಶಬ್ದಝರೀ ಶಾಸ್ತ್ರವ್ಯಾಖ್ಯಾನ ಕೌಶಲಂ

ವೈದುಷ್ಯಂ ವಿದುಷಾಂ ತದ್ವತ್ ಭುಕ್ತಯೇ ನ ತು ಮುಕ್ತಯೇ~॥

ಸ್ವಾಮಿ ವಿವೇಕಾನಂದರ ಅನುವಾದ: ಅದ್ಭುತ ರೀತಿಯಲ್ಲಿ ವಾಕ್ಯಜೋಡಣೆ, ನಿರರ್ಗಳವಾಗಿ ಮಾತನಾಡುವುದು ಮತ್ತು ವಿವಿಧ ರೀತಿಯಲ್ಲಿ ಶಾಸ್ತ್ರವ್ಯಾಖ್ಯಾನ ಮಾಡುವ ಕೌಶಲ - ಇವು ಕೇವಲ ಭೋಗಕ್ಕಾಗಿಯೇ ಹೊರತು, ಮುಕ್ತಿಗಾಗಿ ಅಲ್ಲ - ಇದು ಧರ್ಮವಲ್ಲ.

೭. ನೀನು ಇಡೀ ಬೈಬಲನ್ನೂ ಎಲ್ಲ ತತ್ತ್ವಜ್ಞಾನಿಗಳ ಮಾತುಗಳನ್ನೂ ಕಂಠಪಾಠ ಮಾಡಿದ್ದರೆ, ಅದರಿಂದ ಯಾವ ಪ್ರಯೋಜನವೂ ಇಲ್ಲ. \enginline{(The Imitation of christ v.3.)}

ಸ್ವಾಮಿ ವಿವೇಕಾನಂದರ ಟಿಪ್ಪಣಿ: (ಉಲ್ಲೇಖ ಮಾತ್ರ)- \enginline{1 corinthians 13.2.}

೮. “ಎಲ್ಲವೂ ವ್ಯರ್ಥ, ಖಂಡಿತ ವ್ಯರ್ಥ” \enginline{(Eccles)} - ಭಗವಂತನನ್ನು ಪ್ರೀತಿಸುವುದು ಮತ್ತು ಅವನ ಸೇವೆಯನ್ನು ಮಾತ್ರ ಮಾಡುವುದನ್ನು ಬಿಟ್ಟು ಎಲ್ಲವೂ ವ್ಯರ್ಥ. \enginline{(The Imitation of christ v.3.)}

ಸ್ವಾಮಿ ವಿವೇಕಾನಂದರ ಟಿಪ್ಪಣಿ: ಮಣಿರತ್ನಮಾಲಾ

ಕೇ ಸಂತಿ ಸಂತೋಣಿಖಿಲವೀತರಾಗಾಃ~।

ಅಪಾಸ್ತಮೋಹಾಃ ಶಿವತತ್ತ್ವನಿಷ್ಠಾಃ~॥

ಪ್ರಕಾಶಕರ ಅನುವಾದ: ಯಾರು ಎಲ್ಲ ಆಸೆಗಳಿಂದ ಮುಕ್ತರಾಗಿದ್ದಾರೆಯೋ, ಮೋಹ ಮುಕ್ತರಾಗಿದ್ದಾರೆಯೋ ಮತ್ತು ಶಿವತತ್ತ್ವದಲ್ಲಿ ನಿಷ್ಠರಾಗಿದ್ದಾರೆಯೊ ಅವರೇ ನಿಜವಾಗಿಯೂ ಸಾಧುಗಳು.

೯. “ಕಣ್ಣು ನೋಡುವುದರಿಂದ ತೃಪ್ತನಾಗುವುದಿಲ್ಲ, ಕಿವಿ ಕೇಳಿ ಸಂತೃಪ್ತವಾಗುವುದಿಲ್ಲ” ಎಂಬ ಗಾದೆಯನ್ನು ಯಾವಾಗಲೂ ನೆನೆಸಿಕೊಳ್ಳುತ್ತಿರಬೇಕು. \enginline{(The Imitation of christ v.5.)}

ಸ್ವಾಮಿ ವಿವೇಕಾನಂದರ ಟಿಪ್ಪಣಿ: (ಉಲ್ಲೇಖ ಮಾತ್ರ) - \enginline{(Eccler 1.8)}

೧೦. ಆದ್ದರಿಂದ ದೃಷ್ಟ ವಸ್ತುಗಳಿಂದ ನಿನ್ನ ಹೃದಯವನ್ನು ಹಿಂತೆಗೆದುಕೊಂಡು ಅದೃಷ್ಟವಾದುದರ ಕಡೆಗೆ ನಿನ್ನ ಗಮನವನ್ನು ಹರಿಸು. ಏಕೆಂದರೆ ಯಾರು ಕಾಮವನ್ನು ಅರಸಿ ಹೋಗುತ್ತಾರೊ ಅವರು ತಮ್ಮ ಅಂತರಂಗವನ್ನು ಮಲಿನಗೊಳಿಸುತ್ತಾರೆ ಮತ್ತು ಭಗವಂತನ ಅನುಗ್ರಹದಿಂದ ಚ್ಯುತರಾಗುತ್ತಾರೆ. \enginline{(The Imitation of christ v.5.)}

ಸ್ವಾಮಿ ವಿವೇಕಾನಂದರ ಟಿಪ್ಪಣಿ: ಮಹಾಭಾರತ, ೨.೬೩ (ಯಯಾತಿಯ ಕಥೆ)

ನ ಜಾತು ಕಾಮಃ ಕಾಮಾನಾಮುಪಭೋಗೇನ ಶಾಮ್ಯತಿ~।

ಹವಿಷಾ ಕೃಷ್ಣವರ್ತ್ಮೇವ ಭೂಯ ಏವಾಭಿವರ್ಧತೇ~॥

ಸ್ವಾಮಿ ವಿವೇಕಾನಂದರ ಅನುವಾದ: ಆಸೆಗಳು, ಅವುಗಳನ್ನು ಪೂರ್ಣಗೊಳಿಸು ವುದರ ಮೂಲಕ ತೃಪ್ತಿಹೊಂದುವುದಿಲ್ಲ. ಅಗ್ನಿಗೆ ತುಪ್ಪವನ್ನು ಸುರಿದರೆ ಅದು ವೃದ್ಧಿ ಯಾಗುವಂತೆ ಆಸೆಗಳೂ ವೃದ್ಧಿಯಾಗುತ್ತವೆ.

\begin{center}
\textbf{ಅಧ್ಯಾಯ - ೩}
\end{center}

\begin{center}
\textbf{ಸತ್ಯತತ್ತ್ವವನ್ನು ಕುರಿತು}
\end{center}

೧೧. ಅಂಧಕಾರ ಮತ್ತು ಗುಪ್ತ ವಸ್ತುಗಳನ್ನು ಕುರಿತುವಾದವಿವಾದ ಮಾಡುವುದರಿಂದ ಏನು ಪ್ರಯೋಜನ? ಏಕೆಂದರೆ, ಇವುಗಳನ್ನು ತಿಳಿಯದಿದ್ದದ್ದರಿಂದ ಕೊನೆಯ ನಿರ್ಣಯದ ದಿನ ನಾವು ಭರ್ತ್ಸನೆಗೆ ಒಳಗಾಗುವುದಿಲ್ಲ. \enginline{(The Imitation of christ v.1.)}

ಸ್ವಾಮಿ ವಿವೇಕಾನಂದರ ಟಿಪ್ಪಣಿ: ಕ್ರೈಸ್ತಧರ್ಮದ ಪ್ರಕಾರ ಕೊನೆಯದಿನ ದಂದು (ಪ್ರಲಯದ ದಿನ) ಭಗವಂತನು ಎಲ್ಲ ಜೀವಿಗಳ ಬಗ್ಗೆ ತೀರ್ಪನ್ನು ನೀಡುತ್ತಾನೆ. ಅವರವರ ಗುಣ ಮತ್ತು ಅವಗುಣಗಳ ಪ್ರಕಾರ ಸ್ವರ್ಗವನ್ನೊ ನರಕವನ್ನೊ ಅವರಿಗೆ ನೀಡುತ್ತಾನೆ.

೧೨. ಯಾರು ಶಾಶ್ವತ ಶಬ್ದ ಕೇಳಿರುವರೊ ಅವರು ಅನೇಕ ಮತಗಳಿಂದ ಮುಕ್ತರಾಗಿರುತ್ತಾರೆ. \enginline{(The Imitation of christ v.2.)}

ವಿವೇಕಾನಂದರ ಟಿಪ್ಪಣಿ: ಮೇಲೆ ಉಲ್ಲೇಖಿಸಿರುವ ಶಬ್ದವು ವೇದಾಂತಿಗಳ ‘ಮಾಯೆ’ ಗೆ ಸಮಾನವಾಗಿದೆ ಎನ್ನಬಹುದು. ಇದೇ ಕ್ರಿಸ್ತನ ರೂಪದಲ್ಲಿ ಅಭಿವ್ಯಕ್ತವಾಯಿತು.

\begin{center}
\textbf{ಅಧ್ಯಾಯ - ೫}
\end{center}

\begin{center}
\textbf{ಶಾಸ್ತ್ರಾಧ್ಯಯನವನ್ನು ಕುರಿತು}
\end{center}

೧೩. ಶಾಸ್ತ್ರದಲ್ಲಿ ಸತ್ಯವನ್ನು ಅರಸಬೇಕೇ ಹೊರತು, ಪಾಂಡಿತ್ಯವನ್ನಲ್ಲ. ಶಾಸ್ತ್ರದ ಪ್ರತಿಯೊಂದು ಭಾವನ್ನೂ, ಅದನ್ನು ಯಾವ ಭಾವದಿಂದ ಬರೆದಿರುವರೊ, ಅದೇ ಭಾವದಿಂದ ಓದಬೇಕು. \enginline{(The Imitation of christ v.1.)}

ಸ್ವಾಮಿ ವಿವೇಕಾನಂದರ ಟಿಪ್ಪಣಿ: ಕಠೋಪನಿಷತ್, ೧.೨.೯.

ನೈಷಾ ತರ್ಕೇಣ ಮತಿರಾಪನೇಯಾ~॥

ಸ್ವಾಮಿ ವಿವೇಕಾನಂದರ ಅನುವಾದ: ಮನಸ್ಸನ್ನು ವ್ಯರ್ಥವಾದದಿಂದ ವಿಚಲಿತ ಗೊಳಿಸಬಾರದು. ಏಕೆಂದರೆ ಅದುವಾದದ ವಿಷಯವಲ್ಲ, ಅನುಭವದ ವಿಷಯ.

೧೪. ಶಾಸ್ತ್ರಕಾರನು ಬಹುಶ್ರುತನೇ ಅಥವಾ ಅಲ್ಪಶ್ರುತನೇ ಎಂಬುದರ ಬಗ್ಗೆ ಚಿಂತಿಸ ಬೇಡ. ಶುದ್ಧ ಸತ್ಯದ ಮೇಲಿನ ಪ್ರೀತಿ ನಿನ್ನನ್ನು ಶಾಸ್ತ್ರವನ್ನು ಓದುವಂತೆ ಪ್ರೇರೇ ಪಿಸಲಿ. \enginline{(The Imitation of christ v.1.)}

ಸ್ವಾಮಿ ವಿವೇಕಾನಂದರ ಟಿಪ್ಪಣಿ: ಮನುಸ್ಮೃತಿ, ೨.೨೩೮.

ಆದದೀತ ಶುಭಾಂ ವಿದ್ಯಾಂ ಪ್ರಯತ್ನಾದವರಾದಪಿ~।

ವಿವೇಕಾನಂದರ ಅನುವಾದ: ಅಂತ್ಯಜನಿಂದಾದರೂ ಸೇವೆಯ ಮೂಲಕ ಪರಜ್ಞಾನವನ್ನು ಪಡೆಯಬೇಕು.

\begin{center}
\textbf{ಅಧ್ಯಾಯ - ೬}
\end{center}

\begin{center}
\textbf{ತೀವ್ರ ವ್ಯಾಮೋಹವನ್ನು ಕುರಿತು}
\end{center}

೧೫. ಯಾವಾಗಲಾದರೂ ಒಬ್ಬ ವ್ಯಕ್ತಿಯು ಏನನ್ನಾದರೂ ತೀವ್ರವಾಗಿ ಬಯಸಿದರೆ, ಆ ಸಂದರ್ಭದಲ್ಲಿ ಅವನು ಅಶಾಂತನಾಗುತ್ತಾನೆ. \enginline{(The Imitation of christ v.1.)}

ವಿವೇಕಾನಂದರ ಟಿಪ್ಪಣಿ: ಭಗವದ್ಗೀತೆ ೨.೬೭

\begin{verse}
ಇಂದ್ರಿಯಾಣಾಂ ಹಿ ಚರತಾಂ ಯನ್ಮನೋಣಿನು ವಿಧೀಯತೇ~।\\ತದಸ್ಯ ಹರತಿ ಪ್ರಜ್ಞಾಂ ವಾಯುರ್ನಾವಮಿವಾಂಭಸಿ~॥
\end{verse}

ವಿವೇಕಾನಂದರ ಅನುವಾದ: ಅಲೆದಾಡುವ ಇಂದ್ರಿಯಗಳನ್ನು ಮನಸ್ಸು ಅನುಸರಿಸಿ ಹೋಗುತ್ತಿದ್ದರೆ ಆ ವ್ಯಕ್ತಿಯ ವಿವೇಕವು, ನದಿಯ ಮೇಲಿನ ದೋಣಿಯು ಗಾಳಿಯಿಂದ ಎಳೆಯಲ್ಪಡುವಂತೆ, ಹೊರಟುಹೋಗುತ್ತದೆ.

೧೬. ಅಹಂಕಾರಿಯೂ ಲೋಭಿಯೂ ಎಂದೂ ಶಾಂತನಾಗುವುದಿಲ್ಲ. ನಿರಹಂಕಾರಿ ಮತ್ತು ವಿನೀತಭಾವವುಳ್ಳವರು ಶಾಂತಿಯಿಂದಿರುತ್ತಾರೆ.

ತನಗೆ ತಾನೇ ಸಂಪೂರ್ಣ ಮೃತನಾಗದ ವ್ಯಕ್ತಿಯು ಅತ್ಯಲ್ಪ ವಿಷಯದಲ್ಲಿಯೂ ಪ್ರಲೋಭಿತನಾಗಿ ಪರವಶನಾಗುತ್ತಾನೆ. \enginline{(The Imitation of christ v.1.)}

\begin{verse}
ಸ್ವಾಮಿ ವಿವೇಕಾನಂದರ ಟಿಪ್ಪಣಿ: ಭಗವದ್ಗೀತೆ, ೨.೬೨-೩\\ಧ್ಯಾಯತೋ ವಿಷಯಾನ್ ಪುಂಸಃ ಸಂಗಸ್ತೇಷೂಪಜಾಯತೇ~।\\ಸಂಗಾತ್ ಸಂಜಾಯತೇ ಕಾಮಃ ಕಾಮಾದ್ ಕ್ರೋಧೋಽಭಿಜಾಯತೇ~॥
\end{verse}

\begin{verse}
ಕ್ರೋಧಾದ್ಭವತಿ ಸಂಮೋಹಃ ಸಂಮೋಹಾದ್ ಸ್ಮೃತಿವಿಭ್ರಮಃ~।\\ಸ್ಮೃತಿಭ್ರಂಶಾತ್ ಬುದ್ಧಿನಾಶೋ ಬುದ್ಧಿನಾಶಾತ್ ವಿನಶ್ಯತಿ~॥
\end{verse}

ಪ್ರಕಾಶಕರ ಅನುವಾದ: ಇಂದ್ರಿಯ ವಿಷಯಗಳನ್ನು ಚಿಂತಿಸುತ್ತ ಅವುಗಳಲ್ಲಿ ಆಸಕ್ತಿಯುಂಟಾಗುತ್ತದೆ. ಆಸಕ್ತಿಯಿಂದ ಆಸೆಯುಂಟಾಗುತ್ತದೆ. ಆಸೆ ಕ್ರೋಧಕ್ಕೆ ಕಾರಣವಾಗುತ್ತದೆ. ಕ್ರೋಧದಿಂದ ಮೋಹವುಂಟಾಗುತ್ತದೆ. ಇದರಿಂದ ವಿವೇಕ ನಾಶವಾಗುತ್ತದೆ. ವಿವೇಕ ನಾಶದಿಂದ ಸರ್ವನಾಶವಾಗುತ್ತದೆ.

೧೭. ವಿಷಯಾಸಕ್ತ ವ್ಯಕ್ತಿಯಲ್ಲಿಯೂ ಹೊರಗಿನ ವಿಷಯಗಳಲ್ಲಿ ಮತ್ತನಾದವನಲ್ಲಿಯೂ ಶಾಂತಿಯಿರುವುದೇ ಇಲ್ಲ. ಆಧ್ಯಾತ್ಮಿಕನೂ ಭಕ್ತನೂ ಆದವನಲ್ಲಿ ಮಾತ್ರ ಶಾಂತಿ ಯಿರುತ್ತದೆ. \enginline{(The Imitation of christ v.2.)}

\begin{myquote}
ಸ್ವಾಮಿ ವಿವೇಕಾನಂದರ ಟಿಪ್ಪಣಿ: ಭಗವದ್ಗೀತೆ, ೨.೬೦
\end{myquote}

\begin{verse}
ಯತತೋ ಹ್ಯಪಿ ಕೌಂತೇಯ ಪುರುಷಸ್ಯ ವಿಪಶ್ಚಿತಃ~।\\ಇಂದ್ರಿಯಾಣಿ ಪ್ರಮಾಥೀನಿ ಹರಂತಿ ಪ್ರಸಭಂ ಮನಃ~॥
\end{verse}

ಪ್ರಕಾಶಕರ ಅನುವಾದ: ಕುಂತೀಪುತ್ರನೆ, ಪರಿಪೂರ್ಣತೆಗಾಗಿ ಪ್ರಯತ್ನಿಸುತ್ತಿರುವ ಜ್ಞಾನಿಯ ಮನಸ್ಸನ್ನೂ ಕೂಡ ಬಲಯುತವಾದ ಇಂದ್ರಿಯಗಳು ಸೆಳೆದುಬಿಡುತ್ತವೆ.

