
\chapter{ಅಧ್ಯಾಯ ೭: ಶ‍್ರೀನಗರದ ಜೀವನ}

ಸ್ಥಳ: ಶ‍್ರೀನಗರ.

ಕಾಲ: ೧೮೯೮ರ ಜೂನ್ ೨೨ರಿಂದ ಜುಲೈ ೧೫ರವರೆಗೆ.

ಈಗಲೂ ಹಿಂದಿನಂತೆಯೇ ಬೆಳಗಿನ ಹೊತ್ತು ನಾವು ದೀರ್ಘಕಾಲದ ಸಂಭಾಷಣೆಯಲ್ಲಿ ನಿರತರಾಗಿರುತ್ತಿದ್ದೆವು – ಕಾಶ್ಮೀರವು ಹಾದು ಬಂದ ವಿವಿಧ ಧರ್ಮಗಳ ಕಾಲಗಳು, ಬೌದ್ಧಧರ್ಮಾನುಯಾಯಿಗಳ ನೈತಿಕತೆ, ಶಿವಾರಾಧನೆಯ ಇತಿಹಾಸ ಅಥವಾ ಬಹುಶಃ ಕನಿಷ್ಕನ ಆಳ್ವಿಕೆಯಡಿ ಶ‍್ರೀನಗರದ ಸ್ಥಾನ – ಹೀಗೆ ಒಮ್ಮೊಮ್ಮೆ ಒಂದೊಂದು ವಿಷಯ ವಿರುತ್ತಿತ್ತು. ಒಂದು ಸಲ ನಮ್ಮಲ್ಲೊಬ್ಬರೊಡನೆ ಅವರು ಬೌದ್ಧಧರ್ಮದ ವಿಚಾರವಾಗಿ ಮಾತನಾಡುತ್ತಿರುವಾಗ, ಇದ್ದಕ್ಕಿದ್ದಂತೆ, “ಪ್ರಪಂಚವು ಈ ಹೊತ್ತಿನವರೆಗೂ ಯಾವುದಕ್ಕೆ ಸಿದ್ಧವಾಗಿರಲಿಲ್ಲವೋ ಅದನ್ನು ಬೌದ್ಧಧರ್ಮವು ಅಶೋಕನ ಕಾಲದಲ್ಲಿಯೇ ಮಾಡಲು ಯತ್ನಿಸಿತ್ತು!” ಎಂದರು. ಅವರು ಹೇಳುತ್ತಿದ್ದುದು ಧರ್ಮಗಳನ್ನು ಒಗ್ಗೂಡಿಸುವ ಪ್ರಯತ್ನದ ವಿಚಾರವಾಗಿ – ಅಶೋಕನ ಧರ್ಮ ಸಾಮ್ರಾಜ್ಯ ಸಂಘಟನೆಯ ಅದು ಒಂದು ಅದ್ಭುತ ಪರಿಭಾವನೆಯಾಗಿತ್ತು. ಆ ನಂತರ ಇಡೀ ಮಾನವಜನಾಂಗದ ಸಮಷ್ಟಿಪ್ರಜ್ಞೆಯ ಮೇಲೆ ಹಕ್ಕಿರುವುದು ತಮಗೇ ಎಂಬಂತೆ ಇಸ್ಲಾಂ ಹಾಗೂ ಕ್ರೈಸ್ತಧರ್ಮಗಳು ಅಲೆಗಳೋಪಾದಿಯಲ್ಲಿ ಏರಿ ಬಂದು ಅನೇಕ ಬಾರಿ ಅದನ್ನು ಹಾಳುಗೆಡವಿದವು; ಕೊನೆಗೆ ಇಂದು ನಮ್ಮಳವಿಗೆ ಸಿಗುವಷ್ಟು ಕಾಲದಲ್ಲಿ, ಅದರ ಸಾಧ್ಯತೆಯನ್ನೂ ತೋರಿಸಿಕೊಟ್ಟವು!

ಇನ್ನೊಂದು ಸಲ ಮಾತಿನ ನಡುವೆ ಮಧ್ಯ ಏಷ್ಯಾದಿಂದ ದಂಡೆತ್ತಿ ಬಂದ ಛಂಗೇಸ್ ಖಾನ್ ವಿಚಾರ ಬಂತು. ಅವರು ಭಾವುಕರಾಗಿ, “ಜನರು ಅವನನ್ನು ನೀಚ, ಕ್ರೂರಿ ಆಕ್ರಮಣಕಾರ ಎಂದು ಹೇಳುವುದನ್ನು ನೀವು ಕೇಳಿರಬಹುದು, ಆದರೆ ಅದು ನಿಜವಲ್ಲ! ಈ ಮಹಾ ಚೇತನಗಳು ಎಂದಿಗೂ ದುರಾಸೆಯವರಲ್ಲ, ನೀಚರಲ್ಲ! ಏಕತ್ವದ ಕಲ್ಪನೆ, ಈ ಪ್ರಪಂಚವನ್ನು ಒಂದುಗೂಡಿಸಬೇಕೆಂಬ ಮಹಾಸ್ಫೂರ್ತಿ, ಆತನಲ್ಲಿ ತುಂಬಿತ್ತು. ಹೌದಲ್ಲ, ನೆಪೋಲಿಯನ್ ಸಹ ಅದೇ ಅಚ್ಚಿನಲ್ಲಿ ಮೂಡಿ ಬಂದವನು – ಹಾಗೆಯೇ ಆ ಇನ್ನೊಬ್ಬ ಅಲೆಗ್ಸಾಂಡರ್ ಸಹಾ. ಕೇವಲ ಈ ಮೂವರು ಮಾತ್ರ – ಅಥವಾ ಬಹುಶಃ ಒಂದೇ ಚೇತನ ಮೂರು ಬೇರೆಬೇರೆ ವಿಜಯಗಳಲ್ಲಿ ಪ್ರಕಟಗೊಂಡಿರ ಬಹುದೇನೋ!” ಎಂದು ನುಡಿದರು. ಹಾಗೆಯೇ ಮಾತನಾಡುತ್ತ, ಆ ನಂತರ ಧರ್ಮ ದಲ್ಲೂ ಸಹ, ದೇವರಲ್ಲಿ ಮಾನವನ ಐಕ್ಯವನ್ನು ಸ್ಥಾಪಿಸುವ ದಿವ್ಯಸ್ಫೂರ್ತಿಯಿಂದ ತುಂಬಿದ ಒಂದೇ ಚೇತನವು ಲೋಕದಲ್ಲಿ ಪುನಃ ಪುನಃ ಪ್ರಕಟವಾಗುತ್ತಿದೆ ಎಂಬ ತಮ್ಮ ನಂಬಿಕೆಯತ್ತ ಹೊರಳಿದರು.

ಈ ದಿನಗಳಲ್ಲಿ ‘ಪ್ರಬುದ್ಧ ಭಾರತ’ವನ್ನು ಮದರಾಸಿನಿಂದ ಹೊಸದಾಗಿ ಸ್ಥಾಪಿಸಿರುವ ಮಾಯಾವತಿಯ ಆಶ್ರಮಕ್ಕೆ ಸ್ಥಳಾಂತರಿಸುವ ವಿಚಾರ ನಮ್ಮೆಲ್ಲರ ಮನಸ್ಸಿನಲ್ಲಿ ತುಂಬಿಕೊಂಡಿದ್ದಿತು. ಸ್ವಾಮಿಗಳು ಅದಕ್ಕೆ ಕೊಟ್ಟಿದ್ದ ಸುಂದರ ಹೆಸರೇ ಸೂಚಿಸುವಂತೆ, ಈ ಪತ್ರಿಕೆಯ ಬಗ್ಗೆ ಅವರಿಗೆ ವಿಶೇಷ ಪ್ರೀತಿ. ಅದರ ಅಂಗೋಪಾಂಗಗಳನ್ನು ಸ್ಥಾಪಿಸುವ ವಿಚಾರದಲ್ಲಿಯೂ ಸಹ ಅವರು ಯಾವಾಗಲೂ ಕಾತರರಾಗಿದ್ದರು. ನವಭಾರತದ ಶಿಕ್ಷಣದಲ್ಲಿ ಈ ಪತ್ರಿಕೆಯ ಮೌಲ್ಯವೆಷ್ಟೆಂಬುದು ಅವರ ಮನಸ್ಸಿನಲ್ಲಿ ನಿಚ್ಚಳವಾಗಿದ್ದಿತು. ಈ ಮಾಧ್ಯಮದ ಮೂಲಕವಲ್ಲದೆ ಇನ್ನಿತರ ಕೆಲಸಕಾರ್ಯಗಳ ಹಾಗೂ ಪ್ರಚಾರದ ಮೂಲಕ ತಮ್ಮ ಗುರುಗಳ ದಿವ್ಯಸಂದೇಶ ಹಾಗೂ ಹೊಸ ಚಿಂತನೆಯ ಧಾರೆಯನ್ನು ಪ್ರಸರಿಸುವ ಅಗತ್ಯವನ್ನು ಅವರು ಮನಗಂಡಿದ್ದರು. ಆದ್ದರಿಂದಲೇ, ತಮ್ಮ ಪತ್ರಿಕಾ ಯೋಜನೆಗಳ ಭವಿಷ್ಯ ಹಾಗೂ ವಿವಿಧ ಕೇಂದ್ರಗಳ ಮೂಲಕ ನಡೆಯುವ ಕೆಲಸ – ಇವುಗಳನ್ನೇ ಅವರು ದಿನ ದಿನವೂ ಕನಸು ಕಾಣುತ್ತಿದ್ದರು. ಸ್ವಾಮಿ ಸ್ವರೂಪಾನಂದರ ಸಂಪಾದ ಕತ್ವದಲ್ಲಿ ಹೊರಬರಲಿರುವ ಪತ್ರಿಕೆಯ ಮೊಟ್ಟಮೊದಲ ಸಂಚಿಕೆಯ ವಿಚಾರವಾಗಿಯೇ ದಿನದಿನವೂ ಮಾತನಾಡುವರು. ಅಲ್ಲದೆ ಒಂದು ಮಧ್ಯಾಹ್ನ ನಾವೆಲ್ಲರೂ ಒಟ್ಟಿಗೆ ಕುಳಿತಿ ದ್ದಾಗ ತಾವು “ಒಂದು ಕಾಗದವನ್ನು ಬರೆಯಲೆತ್ನಿಸಿದೆ, ಆದರದು ಹೀಗೆ ಮೂಡಿ ಬಂದಿತು!” ಎನ್ನುತ್ತ ಒಂದು ಹಾಳೆಯನ್ನು ಹಿಡಿದುಕೊಂಡು ಬಂದರು...(ನೋಡಿ “To The Awakened India”, Complete Works, IV: 387–89).

\textbf{ಜೂನ್ ೨೬.}

ಗುರುದೇವರು ನಮ್ಮನ್ನೆಲ್ಲ ಬಿಟ್ಟು ಯಾವುದಾದರೊಂದು ಮೌನವಾಂತ ಸ್ಥಳಕ್ಕೆ ತಾವೊಬ್ಬರೇ ಹೋಗಬೇಕೆಂದು ಬಯಸುತ್ತಿದ್ದರು. ಆದರೆ, ಇದನ್ನು ಅರಿತಿದ್ದ ನಾವು, ವರ್ಣರಂಜಿತ ಊಟೆಗಳ ಸ್ಥಳವಾದ “ಕ್ಷೀರಭವಾನಿ” (ಅಥವಾ ತಾಯಿಯ ಹಾಲು)ಗೆ ನಾವೂ ಜೊತೆಗೂಡಿ ಬರುವೆವು ಎಂದು ಹಠ ಹಿಡಿದಿದ್ದೆವು. ಈವರೆಗೆ ಯಾರೊಬ್ಬ ಮುಸಲ್ಮಾನ ಅಥವಾ ಕ್ರೈಸ್ತರೂ ಅಲ್ಲಿಗೆ ಹೋಗಿಲ್ಲ ಎಂಬ ಐತಿಹ್ಯವಿತ್ತು; ಮುಂದೆ ನಮ್ಮೆಲ್ಲರಿಗೂ ಅತಿ ಪವಿತ್ರವಾದ ಹೆಸರಾಗಿ ಉಳಿಯುವಂತಾದ ಆ ಸ್ಥಳದ ದಿವ್ಯದರ್ಶನವು ನಮಗೆ ಪ್ರಾಪ್ತವಾದುದಕ್ಕಾಗಿ ನಾವೆಷ್ಟು ಋಣಿಗಳಾಗಿದ್ದರೂ ಸಾಲದು...

\textbf{ಜೂನ್ ೨೯.}

ಇನ್ನೊಂದು ದಿನ, ನಾವು ಸದ್ದಿಲ್ಲದೆ ಹೊರಟು ತಖ್ತ್–ಇ–ಸುಲೇಮಾನ್ ಸಂದರ್ಶಿಸಿದೆವು. ಸುಮಾರು ಎರಡು ಅಥವಾ ಮೂರು ಸಾವಿರ ಅಡಿ ಎತ್ತರದ ಸಣ್ಣ ಬೆಟ್ಟವೊಂದರ ಮೇಲೆ ಅತ್ಯಂತ ಸುದೃಢವಾಗಿ ಕಟ್ಟಲ್ಪಟ್ಟ ಒಂದು ದೇಗುಲ ಇದು. ಬಹು ಶಾಂತವೂ ಸುಂದರವೂ ಆಗಿದ್ದ ಇಲ್ಲಿಂದ ಪ್ರಖ್ಯಾತ ತೇಲುವ ಉದ್ಯಾನವೂ ಸೇರಿದಂತೆ ಕೆಳಭಾಗದಸುತ್ತಮುತ್ತಣ ಮೈಲಿಗಟ್ಟಲೆ ಪ್ರದೇಶವನ್ನು ನೋಡಬಹುದಾಗಿದ್ದಿತು. ಹಿಂದೂಗಳ ಪ್ರಕೃತಿ ಪ್ರೇಮದ ವಿಚಾರ ಬಂದಾಗ, ದೇವಸ್ಥಾನಗಳನ್ನೂ ಶಿಲ್ಪ ವೈಭವದ ಸ್ಮಾರಕಗಳನ್ನೂ ಕಟ್ಟಲು ಆರಿಸಿಕೊಂಡ ಸ್ಥಳಗಳನ್ನು ವಿಮರ್ಶಿಸುತ್ತ ಸ್ವಾಮಿಗಳು ಕೊಡುತ್ತಿದ್ದ ಮಹತ್ವದ ಒಂದು ಉದಾಹರಣೆ ಈ ತಖ್ತ್ – ಇ– ಸುಲೇಮಾನ್. ಸಾಧುಗಳು ಪ್ರಕೃತಿಯ ದೃಶ್ಯಗಳನ್ನು ಕಂಡು ಆನಂದಿಸುವುದಕ್ಕಾಗಿ ಪರ್ವತಾಗ್ರಗಳಲ್ಲಿ ವಾಸಿಸುತ್ತಾರೆ ಎಂದವರು ಲಂಡನ್ನಲ್ಲಿ ಹೇಳಿದ್ದ ಹಾಗೆ, ಈಗಲೂ ಸಹ ಸ್ವಾಮಿಗಳು ನಮ್ಮ ಭಾರತೀಯ ಜನರು ವಿಶಿಷ್ಟ ಪ್ರಾಮುಖ್ಯತೆಯ ಸುಂದರ ಸ್ಥಳಗಳನ್ನು ತಮ್ಮ ಆರಾಧನೆಯ ಪೀಠಗಳನ್ನಾಗಿ ಮಾಡಿಕೊಳ್ಳುವ ಮೂಲಕ ಪವಿತ್ರವಾಗಿಸುತ್ತಾರೆ ಎಂಬುದಕ್ಕೆ ಒಂದೊಂದಾಗಿ ಉದಾಹರಣೆಗಳನ್ನು ಕೊಡಲಾರಂಭಿಸಿದರು. ಇಡಿಯ ಕಣಿವೆಯಲ್ಲೇ ಉನ್ನತ ಸ್ಥಾನದಲ್ಲಿದ್ದು ಬೆಟ್ಟದ ಶಿಖರದ ಮೇಲೆ ಕಿರೀಟಪ್ರಾಯವಾಗಿರುವ ಈ ಪುಟ್ಟ ತಖ್ತ್ ಇದಕ್ಕೊಂದು ಉತ್ತಮ ಉದಾಹರಣೆ ಎಂಬುದನ್ನು ಪ್ರತ್ಯೇಕವಾಗಿ ಹೇಳಬೇಕಾಗಿರಲಿಲ್ಲ.

ಆ ದಿನಗಳಲ್ಲಿ ಮೃದುಮಧುರವಾದ ಅನೇಕ ಭಾವನೆಗಳು ಮನಸ್ಸಿಗೆ ಬರುತ್ತಿದ್ದವು ಉದಾಹರಣೆಗೆ:

\begin{myquote}
ಅದರಿಂದ, ಹೇ ತುಳಸಿ, ಎಲ್ಲರೊಟ್ಟಿಗೆ ಬದುಕು\\ಏಕೆಂದರಾ ದೇವ ಯಾವ ವೇಷದಿ ಬರುವ\\ಯಾವಾಗ ನಿನ್ನ ಬಳಿ ಬಂದು ನಿಲ್ಲುವನೆಂದು\\ಹೇಳಬಲ್ಲವರಾರು?
\end{myquote}

\begin{myquote}
ಈ ಎಲ್ಲದರೊಳಗೆ ಅಡಗಿರುವನವನೊಬ್ಬ\\ಎಲ್ಲದರ ಲಯಕರ್ತ, ಎಲ್ಲದರ ಆಧಾರ,\\ಎಲ್ಲ ಜೀವರುಗಳನು ಬಡಿದೆಬ್ಬಿಸುವನಾತ\\ಆದೊಡೆಯು ಎಲ್ಲ ಗುಣ ಮೀರಿ ನಿಂತವನಾತ!
\end{myquote}

\begin{myquote}
ಸೂರ್ಯ ಚಂದ್ರರು ತಾರೆಗಳಲ್ಲಿ ಬೀರವು ಬೆಳಕ.
\end{myquote}

ಸೀತೆಯನ್ನು ಮೋಸದಿಂದ ವಶಪಡಿಸಿಕೊಳ್ಳಲು ರಾಮನ ರೂಪವನ್ನು ತಳೆಯುವಂತೆ ರಾವಣನಿಗೆ ಯಾರೋ ಸಲಹೆ ನೀಡಿದಾಗ ಅವನೆಂದನಂತೆ: “ನಾನು ಅದನ್ನು ಯೋಚಿಸಿಲ್ಲವೆಂದುಕೊಂಡೆಯಾ? ಆದರೆ ಮನುಷ್ಯನೊಬ್ಬನ ರೂಪವನ್ನು ತಾಳುವುದಕ್ಕೆ ಅವನನ್ನು ಕುರಿತು ಧ್ಯಾನ ಮಾಡಬೇಕಾಗುತ್ತದೆ; ರಾಮನೋ ಸಾಕ್ಷಾತ್ ಭಗವಂತ; ನಾನು ಅವನನ್ನು ಮನಸ್ಸಿನಲ್ಲಿ ಪರಿಭಾವಿಸಿದರೆ ನನಗೆ ಬ್ರಹ್ಮಪದವಿಯೂ ಸಹ ಹುಲ್ಲು ಕಡ್ಡಿಗೆ ಸಮಾನ ಎನ್ನಿಸುತ್ತದೆ. ಆಗ ನಾನು ಹೇಗೆತಾನೆ ಹೆಣ್ಣೊಬ್ಬಳನ್ನು ಕುರಿತು ಚಿಂತಿಸ ಬಲ್ಲೆ?”

“ಆದ್ದರಿಂದ, ತೀರ ಸಾಮಾನ್ಯರ ಬದುಕಿನಲ್ಲಿ, ಕೊನೆಗೆ ಕೊಲೆಗಡುಕರು, ಪಾಪಿಗಳು ಎನ್ನಿಸಿಕೊಂಡವರ ಬದುಕಿನಲ್ಲಿ ಸಹ, ಇಂತಹ ಸೆಳಕುಗಳು ಕಾಣಸಿಗುತ್ತವೆ” ಎಂದರು ಸ್ವಾಮಿಗಳು. ಅದೆಂದಿಗೂ ಇರುವುದು ಹೀಗೆಯೇ. ಯಾರೊಬ್ಬರ ಅಥವಾ ಯಾವುದಾದರೊಂದು ಕಾರ್ಯದ ಕೇಡಿಗತನ ಅಥವಾ ಹೀನತನವನ್ನು ಪರಿಗಣಿಸದೆ, ಅವರು ಮಾನವ ಜೀವನವನ್ನು ಭಗಂವತನ ಅಭಿವ್ಯಕ್ತಿ ಎಂದು ಉದ್ದಕ್ಕೂ ಪ್ರತಿಪಾದಿಸುತ್ತಲೇ ಬಂದರು.

“ಉಳಿದೆಲ್ಲ ಪ್ರಪಂಚಕ್ಕೂ ಯಾವುದು ಕಾಳರಾತ್ರಿಯಾಗಿರುವುದೋ, ಅದರಲ್ಲಿ ಆತ್ಮ ಸಂಯಮವುಳ್ಳ ಯೋಗಿಯು ಎಚ್ಚೆ ತ್ತಿರುವನು. ಉಳಿದೆಲ್ಲ ಪ್ರಪಂಚಕ್ಕೂ ಯಾವುದು ಜೀವನ ಎನ್ನಿಸಿಕೊಳ್ಳುತ್ತದೆಯೋ, ಅದು ಅವನಿಗೆ ನಿದ್ರೆ”.

ಒಂದು ದಿನ ಥಾಮಸ್ ಎ ಕೆಂಪಿಸ್ ವಿಚಾರವಾಗಿ ಮಾತನಾಡುತ್ತ, ಸ್ವಾಮಿಗಳು ತಾವು ತಮ್ಮ ಪರಿವ್ರಾಜಕ ದಿನಗಳಲ್ಲಿ ಗೀತೆ ಮತ್ತು ಇಮಿಟೇಷನ್ ಆಫ್ ಕ್ರೈಸ್ಟ್ ಎಂಬೆರಡೇ ಗ್ರಂಥಗಳು ತಮ್ಮೊಡನೆ ಇದ್ದವು ಎಂಬುದನ್ನು ಹೇಳಿದರು; ಆ ಪಾಶ್ಚಾತ್ಯ ಸಂನ್ಯಾಸಿಯನ್ನು ಜ್ಞಾಪಿಸಿಕೊಂಡೊಡನೆ ಯಾವಾಗಲೂ ತಮಗೆ ನೆನಪಾಗುವುದು ಒಂದು ಮಾತ್ರ ಎಂದರು; ಅದೆಂದರೆ:

\begin{myquote}
ಮೌನವಾಗಿರಿ ನೀವು ಹೇ ಪ್ರವಾದಿಗಳೆ\\ಮೌನವಾಗಿರಿ ನೀವು ಹೇ ಜಗದ್ಗುರುವೆ\\ನೀನೊಬ್ಬ ನುಡಿ ದೇವ, ಎನ್ನಾತ್ಮದೊಳಗೆ!\\ಅಲ್ಲದೆ, ಮತ್ತೆ–\\ಝೇಂಕರಿಸುವ ಜೇನ ತಾಳುವೀ ಮೃದುಶೀರ್ಷ\\ಹಕ್ಕಿ ಕೂತರೆ ತಾನು ತಾಳಬಹುದೆ?\\ಹಾಗಾಗಿ ಉಮೆ ನೀನು ತಪಕೆಂದು ತೆರಳದಿರು\\ಬಾ ತಾಯಿ, ಎನ್ನೆದೆಯ ಪುತ್ಥಳಿಯೆ ಬಾ–\\ಬಾರಮ್ಮ, ಎನ್ನಾತ್ಮನಾನಂದ ನೀನಮ್ಮ!\\ಬಂದೆನ್ನ ಹೃದಯಸಿಂಹಾಸನದಿ ಕೂರಮ್ಮ\\ನೋಡುವೆನು ನಿನ್ನನ್ನು, ನೋಡಿಯೇ ನೋಡುವೆನು\\ಬಾಲ್ಯದಿಂದಲು ಮೊಗವ ನೋಡುತ್ತಲೆ ಬಂದಿಹೆನು\\ಅರಿವೆ ನಿನ್ನಯ ನೋವ, ವೇದನೆಯ, ತೊಂದರೆಯ –\\ಆದಕಾರಣ ತಾಯೆ, ಬಂದೆನ್ನ ಹೃತ್ಪದ್ಮ–\\ದಲಿ ಕುಳಿತು ನೆಲೆನಿಲ್ಲು, ಅನವರತ ಉಳಿದುಬಿಡು.
\end{myquote}

ಗೀತೆಯ ಬಗ್ಗೆ – “ಹೇಡಿತನದ, ದೌರ್ಬಲ್ಯದ ಒಂದೇ ಒಂದು ಛಾಯೆ ಇಲ್ಲದ ಆ ಅದ್ಭುತ ಗೀತೆ” ಯ ಬಗ್ಗೆ – ಆಗಾಗ ದೀರ್ಘವಾದ ಸಂಭಾಷಣೆ ನಡೆಯುತ್ತಿತ್ತು. ಸ್ತ್ರೀಯರಿಗೆ ಶೂದ್ರರಿಗೆ ಜ್ಞಾನವನ್ನು ಕೊಡಮಾಡಲಾಗಿಲ್ಲ ಎಂದು ದೂರುವುದು ಅಸಂಬದ್ಧ ಎಂದು ಅವರು ಒಂದು ದಿನ ಹೇಳಿದರು. ಏಕೆಂದರೆ, ಉಪನಿಷತ್ತುಗಳ ಸಾರ ಸರ್ವಸ್ವವೂ ಗೀತೆಯಲ್ಲಿದೆ; ಗೀತೆಯ ನೆರವಿಲ್ಲದೆ ಉಪನಿಷತ್ತುಗಳನ್ನು ಅರ್ಥಮಾಡಿಕೊಳ್ಳುವುದು ಅಸಾಧ್ಯ; ಅಲ್ಲದೆ ಸ್ತ್ರೀಯರೂ ಎಲ್ಲ ಜಾತಿಯವರೂ ಮಹಾಭಾರತವನ್ನು ಓದಬಹುದಲ್ಲ!

\textbf{ಜುಲೈ ೪.}

ಬಲು ಲವಲವಿಕೆಯಿಂದ, ಗುಟ್ಟಾಗಿ, ಸ್ವಾಮಿಗಳೂ ಮತ್ತು ಅವರ ಅಮೆರಿಕನ್ ಅಲ್ಲದ ಒಬ್ಬ ಶಿಷ್ಯನೂ ಜುಲೈ ೪ರ ಆಚರಣೆಗಾಗಿ ತಯಾರಿ ನಡೆಸಿದ್ದರು. ಬೆಳಗಿನ ತಿಂಡಿಯ ಹೊತ್ತಿಗೆ ಸದಸ್ಯರೆಲ್ಲರೂ ಒಟ್ಟಾಗುವಾಗ ಅಮೆರಿಕಾದ ರಾಷ್ಟ್ರೀಯ ಹಬ್ಬದ ದಿನದಂದು ಅವರನ್ನು ಸ್ವಾಗತಿಸಲು ಅಮೆರಿಕಾದ ಧ್ವಜ ಇಲ್ಲವಲ್ಲ ಎಂದು ನಾವು ಮಾತನಾಡಿಕೊಂಡದ್ದು ಸ್ವಾಮಿಗಳ ಕಿವಿಗೆ ಬಿದ್ದಿತ್ತು. ಜುಲೈ ೩ರಂದು ಸಂಜೆಯ ಹೊತ್ತಿಗೆ ಬಲು ಆನಂದೋದ್ರೇಕದಿಂದ ಅವರು ಪಂಡಿತ್ (ಬ್ರಾಹ್ಮಣ) ದರ್ಜಿಯೊಬ್ಬನನ್ನು ಕರೆ ತಂದಿದ್ದರು; ಹೇಗೆ ಮಾಡಬೇಕೆಂದು ಹೇಳಿಕೊಟ್ಟರೆ ಇವನು ತಯಾರಿಸಿ ಕೊಡಬಲ್ಲ ಎಂದು ವಿವರಿಸಿದರು. ಅಮೆರಿಕನ್ನರು ತಮ್ಮ ಸ್ವಾತಂತ್ರ್ಯೋತ್ಸವದ ದಿನದಂದು ಬೆಳಗ್ಗೆ ಮುಂಚೆ ಟೀ ಕುಡಿಯುವುದಕ್ಕೆಂದು ಅಡುಗೆ–ಊಟದ ದೋಣಿಯನ್ನು ಹತ್ತಿದಾಗ, ಅಲ್ಲಿತ್ತು ಅಮೆರಿಕನ್ ಬಾವುಟ! ದೋಣಿಯ ಕೂವೆ ಕಂಬಕ್ಕೆ ಪಟ್ಟೆಗಳೂ ನಕ್ಷತ್ರಗಳೂ ಒರಟೊರಟಾಗಿ ಮೂಡಿದ್ದ ಬಟ್ಟೆಯ ಬಾವುಟವನ್ನು ಮೊಳೆ ಬಡಿದು ಹಸಿರು ತೋರಣಗಳಿಂದ ಅಲಂಕರಿಸಿದ್ದರು. ಈ ಹಬ್ಬದಲ್ಲಿ ಹಾಜರಿರುವುದಕ್ಕಾಗಿ ಸ್ವಾಮಿಗಳು ತಮ್ಮ ಪ್ರಯಾಣವೊಂದನ್ನು ಮುಂದೂಡಿದ್ದರು; ಸ್ವಾತಂತ್ರ್ಯೋತ್ಸವದ ಆಚರಣೆಯಲ್ಲಿ ಗಟ್ಟಿಯಾಗಿ ಓದಲಿರುವ ಭಾಷಣದ ಜೊತೆಗೆ ಸೇರಿಸಲೆಂದು ತಾವೇ ಒಂದು ಕವನವನ್ನೂ ರಚಿಸಿ ತಂದಿದ್ದರು...(ನೋಡಿ, \enginline{"To The Fourth of July", Complete Works, V: 439–440})

\textbf{ಜುಲೈ ೫}

ಈ ಸಂಜೆ ಯಾರೋ ಒಬ್ಬರು ತನ್ನ ಮದುವೆ ಯಾವಾಗ ಆಗಬಹುದೆಂದು ತನ್ನ ತಟ್ಟೆಯಲ್ಲಿದ್ದ ಚೆರಿ ಬೀಜಗಳನ್ನು ಎಣಿಸಿ ಭವಿಷ್ಯ ನೋಡುವ ಮೂಲಕ ಅವರಿಗೆ ನೋವು ಉಂಟುಮಾಡಿದರು. ಅವರು ಹೇಗೋ ಮಾಡಿ ಈ ಆಟವನ್ನು ಸಹಿಸಿಕೊಂಡರು; ಆದರೆ ಮಾರನೆಯ ದಿನ ಬೆಳಗ್ಗೆ ಆದರ್ಶ ತ್ಯಾಗದ ಸ್ಫೂರ್ತಿ ತುಂಬಿದ ಮೂರ್ತಿಯಾಗಿ ಕಂಡು ಬಂದರು.

\textbf{ಜುಲೈ ೬}

ಮೃದುವಾಗಿ ತನ್ನನ್ನು ತಾನು ಸಾಮಾನ್ಯರ ಜೊತೆಗೆ ಒಂದಾಗಿಸಿಕೊಳ್ಳುವ ತನ್ನ ಬಯಕೆಯನ್ನು ಯಾವಾಗಲೂ ಸೂಕ್ಷ್ಮವಾಗಿ ಅಭಿವ್ಯಕ್ತಪಡಿಸುತ್ತಿದ್ದ ಸ್ವಾಮಿಗಳು, ಅದೇ ಭಾವದಿಂದ “ಗೃಹಸ್ಥ ಜೀವನ, ಮದುವೆ ಮುಂತಾದವುಗಳ ಛಾಯೆ ನನ್ನ ಮನಸ್ಸಿನಲ್ಲೂ ಆಗಾಗ್ಗೆ ಬಂದುಹೋಗುತ್ತಿರುತ್ತವೆ!” ಎಂದು ಘೋಷಿಸಿದರು. ಆದರೆ ಗೃಹಸ್ಥಜೀವನ ವನ್ನೇ ದೊಡ್ಡದೆನ್ನುವರನ್ನು ಹೀಗಳೆಯುವುದಕ್ಕಲ್ಲ, ಅವರು ಈ ಸಂದರ್ಭದಲ್ಲಿ ಧಾರ್ಮಿಕ ಜೀವನವನ್ನು ಎತ್ತಿಹಿಡಿಯ ಹೊರಟದ್ದು. “ಜನಕನಂತೆ ಆಗುವುದು ಅಷ್ಟೊಂದು ಸುಲಭವೆ? ನಿರ್ಲಿಪ್ತನಾಗಿ ಸಾಮ್ರಾಟನ ಸಿಂಹಾಸನದ ಮೇಲೆ ಕುಳಿತುಕೊಳ್ಳುವುದು? ಕೀರ್ತಿಗಾಗಲಿ ಸಿರಿಸಂಪತ್ತುಗಳಿಗಾಗಲಿ ಹೆಂಡತಿಮಕ್ಕಳುಗಳಿಗಾಗಲಿ ಎಳ್ಳಷ್ಟೂ ಮನಸ್ಸು ಸೋದರಿ ನಿವೇದಿತಾ ಬರೆದ ಪುಸ್ತಕದಿಂದ ಆಯ್ದ ಭಾಗಗಳು ೩೮೩೩೮೨ ಸ್ವಾಮಿ ವಿವೇಕಾನಂದರ ಕೃತಿಶ್ರೇಣಿ೩೮೦ ಸ್ವಾಮಿ ವಿವೇಕಾನಂದರ ಕೃತಿಶ್ರೇಣಿಸೋದರಿ ನಿವೇದಿತಾ ಬರೆದ ಪುಸ್ತಕದಿಂದ ಆಯ್ದ ಭಾಗಗಳು ೩೮೧ಕೊಡದಿರುವುದು? ಪಶ್ಚಿಮದೇಶಗಳಲ್ಲಿ ಒಬ್ಬರಾದ ಮೇಲೊಬ್ಬರು ನನಗೆ ಹೇಳಿದರು, ತಾವು ಈ ಸ್ಥಿತಿಯನ್ನು ತಲುಪಿರುವೆವು ಎಂದು. ಆದರೆ ನಾನು ‘ಅಂತಹ ಮಹಾಪುರುಷರು ಭಾರತದಲ್ಲಿ ಹುಟ್ಟಲಿಲ್ಲ!’ ಎಂದು ಮಾತ್ರ ಹೇಳಬಲ್ಲವನಾದೆ” ಎಂದರು.

ಆಮೇಲೆ ಇನ್ನೊಂದು ಕಡೆಗೆ ತಿರುಗಿ ಶ್ರೋತೃಗಳಲ್ಲೊಬ್ಬರಿಗೆ ಇಂತೆಂದರು: “ಗೃಹಸ್ಥ ನಿಗೂ ಸಂನ್ಯಾಸಿಗೂ ಇರುವ ವ್ಯತ್ಯಾಸವು, ಸಾಸಿವೆ ಕಾಳಿಗೂ ಮೇರುಪರ್ವತಕ್ಕೂ ಇರುವ ವ್ಯತ್ಯಾಸದಂತೆ, ಪುಟ್ಟ ಕೊಳಕ್ಕೂ ಅನಂತಸಾಗರಕ್ಕೂ ಇರುವ ವ್ಯತ್ಯಾಸದಂತೆ, ಒಂದು ಮಿಣುಕುಹುಳುವಿಗೂ ಸೂರ್ಯಪ್ರಕಾಶಕ್ಕೂ ಇರುವ ವ್ಯತ್ಯಾಸದಂತೆ, ಎನ್ನುವುದನ್ನು ನಿಮಗೆ ನೀವು ಹೇಳಿಕೊಳ್ಳುವುದನ್ನು, ನಿಮ್ಮ ಮಕ್ಕಳಿಗೆ ತಿಳಿಸಿಕೊಡುವುದನ್ನು, ಎಂದಿಗೂ ಮರೆಯದಿರಿ!”

“ಪ್ರತಿಯೊಂದೂ ಭಯದ ಭಾರದಿಂದ ಕೂಡಿದೆ: ತ್ಯಾಗ ಮಾತ್ರವೇ ಭಯರಹಿತವಾದದ್ದು”. “ಆದರ್ಶವನ್ನು ಕಂಡಿದ್ದಷ್ಟು ಮಟ್ಟಿಗೆ, ಆದರ್ಶಜೀವನಕ್ಕೆ ಸಾಕ್ಷಿಯಾಗಿದಷ್ಟು ಮಟ್ಟಿಗೆ, ಆ ದೆಸೆಯಿಂದ ಇತರರ ಯಶಸ್ಸಿಗೆ ಕಾರಣರಾಗಿದ್ದಷ್ಟು ಮಟ್ಟಿಗೆ, ವಂಚಕ ಸಾಧುಗಳು ಕೂಡ ಧನ್ಯರು, ತಮ್ಮ ಸಂಕಲ್ಪಿತ ವ್ರತದಿಂದ ತಪ್ಪಿದವರೂ ಧನ್ಯರು!”

“ಎಂದೆಂದಿಗೂ, ಯಾವ ಕಾಲಕ್ಕೂ, ಆದರ್ಶವನ್ನು ನಾವು ಮರೆಯುವಂತಾಗ ದಿರಲಿ!” ಅಂತಹ ಕ್ಷಣಗಳಲ್ಲಿ, ತಾವು ಯಾವ ಚಿಂತನೆಯನ್ನು ಪ್ರಾತ್ಯಕ್ಷೀಕರಿಸಲು ಹೊರಟಿರುತ್ತಿದ್ದರೋ ಅದರಲ್ಲಿ ಸಂಪೂರ್ಣವಾಗಿ ಒಂದಾಗಿಬಿಟ್ಟಿರುತ್ತಿದ್ದರು; ಪ್ರಕೃತಿಯ ನಿಯಮವೊಂದು ಕಠೋರ ಅಥವಾ ಕ್ರೂರ ಎಂದೆನ್ನಿಸುವಂಥದೇ ಅರ್ಥದಲ್ಲಿ ಅವರ ಪ್ರತಿಪಾದನೆಯೂ ಕೂಡ ಆ ಗುಣಗಳನ್ನು ಪಡೆದುಕೊಳ್ಳಬಹುದಾಗಿತ್ತು. ಕುಳಿತುಕೊಂಡು ಕೇಳುತ್ತಿದ್ದ ನಮಗೆ, ನಾವೇನು ನಿರಪೇಕ್ಷದೊಂದಿಗೆ, ಅಗೋಚರದೊಂದಿಗೆ ಮುಖಾಮುಖಿಯಾಗಿರುವೆವೋ ಎಂಬಂತಹ ಅನುಭವವಾಗುತ್ತಿತ್ತು.

ಜುಲೈ ೪ರ ನಿಜವಾದ ಕಾರ್ಯಕ್ರಮದ ಅಂಗವಾಗಿ ಢಾಲ್ ಸರೋವರಕ್ಕೆ ಭೇಟಿ ಕೊಟ್ಟು ಶ‍್ರೀನಗರಕ್ಕೆ ಹಿಂದಿರುಗುತ್ತಿದ್ದಾಗ ಇದೆಲ್ಲವೂ ಆಗಿದ್ದು...

ಮಾರನೆಯ ಭಾನುವಾರ–ಎಂದರೆ ಜುಲೈ ೧೦ರಂದು–ಅನಿರೀಕ್ಷಿತವಾಗಿ ಧೀರ ಮಾತಾ ಹಾಗೂ ಜಯಾ ಹಿಂದಿರುಗಿದರು. ಸ್ವಾಮಿಗಳು ಸೋನಾಮಾರ್ಗದ ದಾರಿಯಿಂದ ಅಮರನಾಥದ ಕಡೆಗೆ ಹೊರಟುಹೋದರು, ಮುಂದೊಂದು ದಿನ ಹಿಂದಿರುಗುವರು ಎಂಬುದು ಈಗ ವಿವಿಧ ಮೂಲಗಳಿಂದ ನಮಗೆ ತಿಳಿದುಬಂದಿತು. ಅವರ ಕೈಯಲ್ಲಿ ಕಾಸೊಂದೂ ಇರಲಿಲ್ಲ; ಆದರೂ ಸಹ ಹಿಂದೂ ಮಾತೃಭೂಮಿಯಲ್ಲಿ ಈ ಸಂಗತಿ ಅವರ ಅಭಿಮಾನಿಗಳಿಗೆ ಯಾವ ಆತಂಕವನ್ನೂ ಉಂಟುಮಾಡಲಾರದು...

\textbf{ಜುಲೈ ೧೫.}

ಏತಕ್ಕಾಗಿ ನಾವು ಹೊರಗೆ ಹೊರಟಿರುವೆವು? ಶುಕ್ರವಾರದ ದಿನ ಸಂಜೆ ಐದು ಗಂಟೆಯ ಹೊತ್ತಿಗೆ ಸುಮ್ಮನೆ ನದಿಯ ಉದ್ದಕ್ಕೂ ಸಾಗುವುದಕ್ಕೆಂದು ಇನ್ನೇನು ಹೊರಡ ಲಿದ್ದೆವು. ಆಗ ಸೇವಕರಿಗೆ ದೂರದಲ್ಲಿ ತಮ್ಮ ಪರಿಚಿತ ಗೆಳೆಯರು ಬರುತ್ತಿರುವುದು ಕಾಣಿಸಿತು; ಅವರಿಂದ ಸ್ವಾಮಿಗಳ ದೋಣಿಯು ನಮ್ಮ ಕಡೆಗೇ ಬರುತ್ತಿದೆ ಎಂಬ ವರ್ತ ಮಾನ ತಿಳಿಯಿತು.

ಸುಮಾರು ಒಂದು ಗಂಟೆಯ ನಂತರ ಅವರು ನಮ್ಮೊಡನೆ ಇದ್ದರು – ಹಿಂದಿರುಗಿ ಬಂದಿರುವುದಕ್ಕೆ ತಮಗೆಷ್ಟು ಆನಂದವಾಗಿದೆ ಎಂದು ವಿವರಿಸುತ್ತಿದ್ದರು. ಈ ಸಾರಿ ಬೇಸಿಗೆಯು ಎಂದಿಗಿಂತ ಹೆಚ್ಚು ಪ್ರಖರವಾಗಿದ್ದು, ಕೆಲವು ಹಿಮನದಿಗಳು ಕರಗಿ ಬಿಟ್ಟಿರುವುದರಿಂದ ಅಮರನಾಥಕ್ಕೆ ಹೋಗುವ ಸೋನಾಮಾರ್ಗದಲ್ಲಿ ಸಾಗುವುದು ಸಾಧ್ಯವಿರಲಿಲ್ಲ. ಹೀಗಾಗಿ ಸ್ವಾಮಿಗಳು ಹಿಂದಿರುಗಬೇಕಾಯಿತಂತೆ.

ಆದರೆ, ಕಾಶ್ಮೀರದಲ್ಲಿ ಕಳೆದ ಈ ಕೆಲವು ತಿಂಗಳುಗಳಲ್ಲಿ ನಾವು ಅವರಲ್ಲಿ ಕಂಡ ಮೂರು ಸಾಕ್ಷಾತ್ಕಾರದ ಹಾಗೂ ದಿವ್ಯಾನಂದದ ಮಹಾ ಉತ್ಕರ್ಷಗಳಲ್ಲಿ ಮೊದಲನೆ ಯದು ಇದೇ ಕ್ಷಣದಿಂದ ಆರಂಭವಾಯಿತು. ಅವರು ತಮ್ಮ ಗುರುಗಳ ಬಗ್ಗೆ ಹೇಳುತ್ತಿದ್ದ ಈ ಸತ್ಯವನ್ನು ನಾವು ಪ್ರತ್ಯಕ್ಷವಾಗಿ ಪ್ರಮಾಣೀಕರಿಸಿ ನೋಡುತ್ತಿರುವೆವೋ ಎಂಬಂತೆ ಇತ್ತು: “ನಿಜವಾಗಿಯೂ ಒಂದು ರೀತಿಯ ಅಜ್ಞಾನವಿದೆ. ತನ್ನ ಕೆಲಸವನ್ನು ಮಾಡಿಸುವುದಕ್ಕಾಗಿ ಅದನ್ನು ಜಗನ್ಮಾತೆಯೇ ಅಲ್ಲಿ ಇರಿಸಿದ್ದಾಳೆ. ಆದರೆ ಅದು ಒಂದು ತೆಳ್ಳ ನೆಯ, ಮಿತಪಾರದರ್ಶಕವಾದ ಕಾಗದದ ಪೊರೆಯಂತೆ. ಅದು ಯಾವ ಕ್ಷಣದಲ್ಲಿಯಾದರೂ ಹರಿದುಹೋಗಬಹುದು.”

