
\chapter{ಭಗವದ್ಗೀತೆ - ೧}

(೧೯೦೦ರ ಮೇ ೨೪ರಂದು ಸ್ಯಾನ್ಫ್ರಾಂಸಿಸ್ಕೊದಲ್ಲಿ ನೀಡಿದ ಉಪನ್ಯಾಸದ ಟಿಪ್ಪಣಿ - ಫ್ರಾಂಕ್ ರೊಡ್ಹಾಮಲ್ ತೆಗೆದುಕೊಂಡುದು.)

ಗೀತೆಯು ವೇದಗಳ ಸಾರ. ಇದು ನಮ್ಮ ಬೈಬಲ್ ಅಲ್ಲ, ಉಪನಿಷತ್ತುಗಳು ನಮ್ಮ ಬೈಬಲ್. ಅದು ಉಪನಿಷತ್ತುಗಳ ಸಾರ ಮತ್ತು ಉಪನಿಷತ್ತಿನ ಅನೇಕ ವಿರೋಧಾತ್ಮಕ ಭಾವನೆಗಳನ್ನು ಸಮನ್ವಯಗೊಳಿಸುತ್ತದೆ.

ವೇದಗಳು ಎರಡು ಭಾಗಗಳಾಗಿ ವಿಭಾಗಗೊಂಡಿವೆ - ಕರ್ಮಕಾಂಡ ಮತ್ತು ಜ್ಞಾನಕಾಂಡ. ಕರ್ಮಕಾಂಡವು ಯಾಗ ಯಜ್ಞಗಳನ್ನು ಹೊಂದಿದೆ. ಊಟ, ಬದುಕು, ದಾನ - ಇತ್ಯಾದಿಗಳನ್ನು ಕುರಿತ ನಿಯಮಗಳನ್ನೂ ಹೊಂದಿದೆ. ಜ್ಞಾನವು ಅನಂತರ ಬಂದಿತು ಮತ್ತು ಅದು ರಾಜರಿಂದ ಪ್ರತಿಪಾದಿಸಲ್ಪಟ್ಟಿತು.

ಕರ್ಮಕಾಂಡವು ಪುರೋಹಿತರ ಹಿಡಿತದಲ್ಲಿತ್ತು ಮತ್ತು ಅದು ಇಂದ್ರಿಯ ಜೀವನಕ್ಕೆ ಮಾತ್ರ ಸೀಮಿತವಾಗಿತ್ತು. ಸ್ವರ್ಗಕ್ಕೆ ಹೋಗಿ ಶಾಶ್ವತ ಸುಖವನ್ನು ಪಡೆಯುವುದಕ್ಕಾಗಿ ಸತ್ಕಾರ್ಯವನ್ನು ಮಾಡಬೇಕೆಂದು ಅದು ಬೋಧಿಸಿತು. ಈ ಯಜ್ಞವು ಮಾನವನು ಬಯಸುವುದೆಲ್ಲವನ್ನೂ ಒದಗಿಸುವ ಸಾಮರ್ಥ್ಯವುಳ್ಳದ್ದೆಂದು ಭಾವಿಸಲಾಗಿತ್ತು. ಅದು ಎಲ್ಲ ವರ್ಗದವರಿಗೂ ಒಳಿತು ಕೆಡಕುಗಳೆರಡನ್ನೂ ಒದಗಿಸಿತು. ಪುರೋಹಿತನ ಮಧ್ಯಸ್ತಿಕೆ ಯಿಲ್ಲದೆ ಯಜ್ಞದಿಂದ ಏನನ್ನೂ ಪಡೆಯಲಾಗುತ್ತಿರಲಿಲ್ಲ. ಆದ್ದರಿಂದ, ವ್ಯಕ್ತಿಗೆ ಏನಾದರೂ ಬೇಕಿದ್ದರೆ - ಅದು ಅವನ ಶತ್ರುನಾಶವಾದರೂ ಆಗಿರಬಹುದು - ಅವನು ಪುರೋಹಿತನಿಗೆ ಹಣವನ್ನು ಕೊಟ್ಟರೆ ಆಯಿತು; ಪುರೋಹಿತನು ಯಜ್ಞದ ಮೂಲಕ ಅವನ ಆಸೆಯನ್ನು ಪೂರೈಸುತ್ತಿದ್ದ. ಆದ್ದರಿಂದ ಪುರೋಹಿತರ ಹಿತದೃಷ್ಟಿಯಿಂದ ಕರ್ಮಕಾಂಡವನ್ನು ಉಳಿಸಿಕೊಳ್ಳಬೇಕಾಗಿತ್ತು. ಅದೇ ಅವರ ಜೀವನೋಪಾಯವಾಗಿತ್ತು. ಅವರು ತಮ್ಮ ಎಲ್ಲ ಶಕ್ತಿಯ ಮೂಲಕ ಕರ್ಮಕಾಂಡವನ್ನು ಅದರ ಯಥಾರ್ಥರೂಪದಲ್ಲಿ ಉಳಿಸಿಕೊಳ್ಳಲು ಸರ್ವಪ್ರಯತ್ನ ಮಾಡಿದರು. ಅನೇಕ ಯಜ್ಞಗಳು ತುಂಬ ಸಂಕೀರ್ಣವಾಗಿದ್ದವು ಮತ್ತು ಕೆಲವನ್ನು ಆಚರಿಸುವುದಕ್ಕೆ ವರ್ಷಗಳು ಹಿಡಿಯುತ್ತಿದ್ದವು.

ಜ್ಞಾನಕಾಂಡವು ಅನಂತರ ಬಂದಿತು, ಅದನ್ನು ಬೆಳಕಿಗೆ ಬಂದವರು ರಾಜರು. ಅದು ವಿಶೇಷವಾಗಿ ರಾಜರ ಸ್ವತ್ತಾಗಿತ್ತು. ಶ್ರೇಷ್ಠ ರಾಜರುಗಳಿಗೆ ಕರ್ಮಕಾಂಡವು ಅನವಶ್ಯಕವಾಗಿ ಕಂಡಿತು; ಆದ್ದರಿಂದ ಅದನ್ನು ನಾಶಗೊಳಿಸಲು ತಮ್ಮೆಲ್ಲ ಪ್ರಯತ್ನವನ್ನೂ ಮಾಡಿದರು. ಈ ಜ್ಞಾನಕಾಂಡವು ದೇವರು, ಆತ್ಮ, -ವಿಶ್ವ ಇತ್ಯಾದಿಗಳ ಜ್ಞಾನವನ್ನು ಕುರಿತು ಹೇಳುತ್ತದೆ. ಈ ರಾಜರುಗಳಿಗೆ ಕರ್ಮಕಾಂಡವಾಗಲಿ, ಪುರೋಹಿತರ ತಾಂತ್ರಿಕ ವಿದ್ಯೆಯಾಗಲಿ ಬೇಕಿರಲಿಲ್ಲ. ಅವರು ಇವೆಲ್ಲ ಡಾಂಭಿಕತೆ ಎಂದರು. ಪುರೋಹಿತರು ಸಂಭಾವನೆಗಾಗಿ ರಾಜರ ಬಳಿಗೆ ಬಂದಾಗ ದೇವರು ಮತ್ತು ಆತ್ಮದ ಕುರಿತಾಗಿ ಅವರನ್ನು ಪ್ರಶ್ನಿಸಿದರು. ಪುರೋಹಿತರಿಗೆ ಉತ್ತರಿಸಲಾಗದಿದ್ದಾಗ ಅವರನ್ನು ಹಿಂದಕ್ಕೆ ಕಳಿಸಲಾಯಿತು. ಪುರೋಹಿತರು ಹಿಂತಿರುಗಿ ತಮ್ಮ ಹಿರಿಯರನ್ನು ಈ ಪ್ರಶ್ನೆ ಕೇಳಿದಾಗ ಅವರಿಗೂ ಉತ್ತರಿಸಲಾಗಲಿಲ್ಲ. ಅನಂತರ ಅವರು ಪುನಃ ರಾಜರ ಬಳಿಗೆ ಹೋಗಿ ಅವರ ಶಿಷ್ಯರಾದರು. ಈಗ ಯಜ್ಞಗಳು ಅತ್ಯಲ್ಪ ಬಳಕೆಯಲ್ಲಿವೆ. ಹೆಚ್ಚಿನವುಗಳನ್ನು ನಾಶಗೊಳಿಸ ಲಾಯಿತು, ಅತ್ಯಂತ ಸರಳವಾದ ಭಾಗಗಳು ಮಾತ್ರ ಉಳಿದಿವೆ.

ಉಪನಿಷತ್ತುಗಳಲ್ಲಿ ಕರ್ಮಸಿದ್ಧಾಂತವಿದೆ. ಕರ್ಮವೆಂದರೆ ಕಾರ್ಯ ಕಾರಣ ನಿಯಮವನ್ನು ಮನುಷ್ಯನ ವರ್ತನೆಗೆ ಅನ್ವಯಿಸುವುದು. ಈ ಸಿದ್ಧಾಂತದ ಪ್ರಕಾರ ನಾವು ಯಾವಾಗಲೂ ಕರ್ಮಮಾಡಬೇಕು ಮತ್ತು ದುಃಖದಿಂದ ಮುಕ್ತಿಯನ್ನು ಹೊಂದುವುದಕ್ಕೆ ಒಂದೇ ಮಾರ್ಗವೆಂದರೆ ಸತ್ಕರ್ಮವನ್ನು ಮಾಡುವುದು ಮತ್ತು ಅದರ ಸತಙಲವನ್ನು ಅನುಭವಿಸುವುದು. ಸತ್ಕರ್ಮವನ್ನು ಮಾಡುತ್ತ ಜೀವಿಸಿದರೆ ಮರಣಾನಂತರ ಸ್ವರ್ಗದಲ್ಲಿ ಶಾಶ್ವತ ಸುಖವನ್ನು ಅನುಭವಿಸಬಹುದು. ಸ್ವರ್ಗದಲ್ಲಿಯೂ ನಾವು ಕರ್ಮದಿಂದ ಮುಕ್ತರಾಗುವುದಿಲ್ಲ - ಆದರೆ ಅದು ಸತ್ಕರ್ಮ, ದುಷ್ಕರ್ಮವಲ್ಲ.

ಜ್ಞಾನಕಾಂಡವು ಎಲ್ಲ ಬಗೆಯ ಕರ್ಮವನ್ನೂ, ಅದು ಎಷ್ಟೇ ಒಳ್ಳೆಯದಾಗಿರಲಿ, ನಿರಾಕರಿಸುತ್ತದೆ ಮತ್ತು ಎಲ್ಲ ಬಗೆಯ ಭೋಗವನ್ನೂ ತ್ಯಜಿಸಬೇಕೆನ್ನುತ್ತದೆ. ಈ ತತ್ತ್ವದ ಪ್ರಕಾರ ಎಲ್ಲ ಬಗೆಯ ಕರ್ಮ, ಎಲ್ಲ ಬಗೆಯ ಸುಖ ಮೂರ್ಖತನ ಮತ್ತು ಅವು ಸ್ವಾಭಾವಿಕವಾಗಿಯೇ ಅಶಾಶ್ವತ. “ಇವೆಲ್ಲವೂ ಅಂತ್ಯವುಳ್ಳವುಗಳು, ಆದ್ದರಿಂದ ಇವುಗಳನ್ನು ಅಂತ್ಯಗೊಳಿಸಿ, ಇವು ವ್ಯರ್ಥ” - ಎಂದು ಜ್ಞಾನಕಾಂಡವು ಹೇಳುತ್ತದೆ. ಎಲ್ಲ ದುಃಖಗಳಿಗೂ ಕಾರಣ ಅಜ್ಞಾನ, ಆದ್ದರಿಂದ ಜ್ಞಾನವೇ ಇದಕ್ಕೆ ಪರಿಹಾರ, ಎಂದು ಅದು ಸಾರುತ್ತದೆ.

ಒಬ್ಬನು ಪೂರ್ವಕರ್ಮದಿಂದ ಬಂಧಿತನಾಗಿದ್ದಾನೆ ಎನ್ನುವುದೆಲ್ಲ ಅರ್ಥಹೀನ. ಬೆಳಕಿನ ಒಂದು ಕಿರಣವು ಎಂಥ ಅಂಧಕಾರವನ್ನೂ ಕ್ಷಣಾರ್ಧದಲ್ಲಿ ನಾಶಮಾಡುತ್ತದೆ. ಒಂದು ಬೆಂಕಿಯ ಕಿಡಿ ಎಂಥ ಹತ್ತಿಯ ರಾಶಿಯನ್ನೂ ಸಂಪೂರ್ಣ ನಾಶಗೊಳಿಸಬಲ್ಲದು. ಅನಂತ ಕಾಲದಿಂದ ಕೊಠಡಿಯಲ್ಲಿ ಅಂಧಕಾರವಿದ್ದರೆ ದೀಪವು ಕ್ಷಣಾರ್ಧದಲ್ಲಿ ಅದನ್ನು ನಿವಾರಿಸುತ್ತದೆ. ಆದ್ದರಿಂದ ಪ್ರತಿಯೊಬ್ಬ ಜೀವನು ಎಂಥ ಅಜ್ಞಾನಾಂಧಕಾರದಲ್ಲಿ ದ್ದರೂ, ಅವನು ಸಂಪೂರ್ಣವಾಗಿ ಪೂರ್ವಕರ್ಮದಿಂದ ಬದ್ಧನಾಗಿರುವುದಿಲ್ಲ, ಅವನು ಬಂಧ ಮುಕ್ತನಾಗಲು ಯುಗಯುಗಾಂತರಗಳು ಶ್ರಮಿಸಬೇಕಾಗಿಲ್ಲ. “ದಿವ್ಯಜ್ಯೋತಿಯ ಒಂದು ಕಿರಣವು ಅವನ ದಿವ್ಯಸ್ವರೂಪವನ್ನು ಅವನಿಗೆ ಪ್ರಕಾಶಪಡಿಸಿ ಅವನನ್ನು ಮುಕ್ತಗೊಳಿಸಬಲ್ಲದು.”

ಗೀತೆಯು ಈ ಎಲ್ಲ ವಿರೋಧಾತ್ಮಕ ತತ್ತ್ವಗಳನ್ನು ಸಮನ್ವಯಗೊಳಿಸುತ್ತದೆ. ಶ‍್ರೀಕೃಷ್ಣನು ನಿಜವಾಗಿ ಇದ್ದನೆ ಎಂಬ ವಿಷಯದ ಬಗ್ಗೆ ನನಗೆ ತಿಳಿಯದು. “ಅವನ ಬಗ್ಗೆ ಅನೇಕ ಕಥೆಗಳಿವೆ, ಆದರೆ ಅವುಗಳನ್ನು ನಾನು ನಂಬುವುದಿಲ್ಲ.”

“ಅವನು ನಿಜವಾಗಿ ಇದ್ದನು ಮತ್ತು ಚಿಂತಿಸಿದನು ಎಂಬುದರ ವಿಷಯದಲ್ಲಿ ನನಗೆ ತುಂಬ ಸಂದೇಹವಿದೆ. ಅವನು ಅವತರಿಸದಿದ್ದರೇ ಒಳ್ಳೆಯದೆನಿಸುತ್ತದೆ, ಹಾಗಾಗಿದ್ದರೆ ಒಂದು ದೇವತೆ ಕಡಿಮೆಯಾಗುತ್ತಿತ್ತು.”

