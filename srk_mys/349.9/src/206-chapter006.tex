
\chapter{ಭಗವದ್ಗೀತೆ - ೨}

(೧೯೦೦ರ ಮೇ ೨೬ರಂದು ಸ್ಯಾನ್ಫ್ರಾಂಸಿಸ್ಕೊದಲ್ಲಿ ನೀಡಿದ ಉಪನ್ಯಾಸದ ಟಿಪ್ಪಣಿ - ರೊಡ್ಹಾಮಲ್ರಿಂದ ತೆಗೆದುಕೊಂಡುದು.)

ಕ್ರಿಶ್ಚಿಯನರಿಗೆ ಹೊಸ ಒಡಂಬಡಿಕೆ ಇರುವಂತೆ ಗೀತೆಯು ಹಿಂದೂಗಳಿಗಿರುವುದು. ಇದು ಐದು ಸಾವಿರ ವರ್ಷದ ಹಿಂದಿನದು ಮತ್ತು ಕುರುಕ್ಷೇತ್ರ ಯುದ್ಧ ಭೂಮಿಯಲ್ಲಿ ಇದು ಬೋಧಿಸಲ್ಪಟ್ಟಿತು. ನಾನು ಮೊದಲೇ ಹೇಳಿರುವಂತೆ ವೇದಗಳು ಕರ್ಮಕಾಂಡ ಮತ್ತು ಜ್ಞಾನಕಾಂಡ ಎಂದು ವಿಭಾಗಗೊಂಡಿವೆ.

ಜ್ಞಾನ ಕಾಂಡದ ಪ್ರತಿಪಾದಕರಾದ ರಾಜರುಗಳು ಮತ್ತು ಪುರೋಹಿತರ ನಡುವೆ ದೊಡ್ಡ ವಿರೋಧವೇರ್ಪಟ್ಟಿತು. ಪುರೋಹಿತರಿಗೆ ಜನಬಲವಿತ್ತು, ಏಕೆಂದರೆ ಜನ ಸಾಮಾನ್ಯರ ಮನಸ್ಸಿಗೆ ಮೆಚ್ಚಿಕೆಯಾಗುವ ಎಲ್ಲ ಉಪಾಯಗಳೂ ಅವರ ಕೈಯಲ್ಲಿ ದ್ದುವು. ರಾಜರುಗಳಲ್ಲಿ ಆಧ್ಯಾತ್ಮಿಕ ಶಕ್ತಿಯಿತ್ತು. ಆರ್ಥಿಕ ಬಲವಿರಲಿಲ್ಲ; ಆದರೆ ಅವರು ಅಧಿಕಾರದಲ್ಲಿದ್ದುದರಿಂದ ಹೋರಾಟವು ಬಹು ದೀರ್ಘವಾಗಿತ್ತು ಮತ್ತು ಕಠಿನವಾಗಿತ್ತು. ರಾಜರುಗಳಿಗೆ ಸ್ವಲ್ಪ ಜಯ ಸಿಕ್ಕಿದರೂ, ಅವರ ಭಾವನೆಗಳು ಜನಸಾಮಾನ್ಯರಿಗೆ ನಿಲುಕ ದಾದುದರಿಂದ, ಕರ್ಮಕಾಂಡವೇ ಜನಸಾಮಾನ್ಯರಿಗೆ ಹೆಚ್ಚು ಮೆಚ್ಚಿಕೆಯಾಗಿತ್ತು.

ಯಾವುದೇ ಧಾರ್ಮಿಕ ಪಂಥವು ಜನ ಸಾಮಾನ್ಯರಲ್ಲಿ ಮೇಲುಗೈಯನ್ನು ಪಡೆದಾಗ ಲೆಲ್ಲ ಅದಕ್ಕೊಂದು ಆರ್ಥಿಕ ಹಿನ್ನೆಲೆಯಿರುತ್ತದೆ ಎಂಬುದನ್ನು ಯಾವಾಗಲೂ ನೆನಪಿನಲ್ಲಿಡಿ. ಧರ್ಮದ ಆರ್ಥಿಕ ಅಂಶವು ಮಾತ್ರ ಜನರ ಮನಸ್ಸನ್ನು ಸೆಳೆಯುತ್ತದೆಯೇ ಹೊರತು ಅದರ ಆಧ್ಯಾತ್ಮಿಕ ಅಥವಾ ತಾತ್ತ್ವಿಕ ಅಂಶವಲ್ಲ. ನೀವು ಬೀದಿಯಲ್ಲಿ ಒಂದು ವರ್ಷಪರ್ಯಂತ ತತ್ತ್ವವನ್ನು ಬೋಧಿಸಿದರೂ ಕೆಲವೇ ಮಂದಿ ಅನುಯಾಯಿಗಳನ್ನೂ ನೀವು ಪಡೆಯಲಾರಿರಿ. ಆದರೆ ನೀವು ತುಂಬ ಅರ್ಥಹೀನವಾದುದನ್ನು ಬೋಧಿಸಿದರೂ, ಅದರಲ್ಲಿ ಆರ್ಥಿಕ ಪ್ರಯೋಜನವಿರುವುದಾದರೆ, ಇಡೀ ಜನ ಸಮೂಹವೇ ನಿಮ್ಮ ಹಿಂದೆ ಬರುತ್ತದೆ.

ವೇದಗಳು ಎಷ್ಟು ಪುರಾತನವಾದುದೆಂದರೆ ಅವುಗಳನ್ನು ಯಾರು ಬರೆದ ರೆಂಬುದೇ ತಿಳಿಯದು. ಸಾಂಪ್ರದಾಯಿಕ ಹಿಂದೂಗಳ ಪ್ರಕಾರ ವೇದಗಳು ಲಿಖಿತ ಸಾಹಿತ್ಯವಲ್ಲ; ಅವು ಸ್ವರಬದ್ಧವಾಗಿ ಉಚ್ಚರಿಸಲ್ಪಟ್ಟ ಶಬ್ದಗಳಿಂದ ಕೂಡಿದೆ. ಈ ಬೃಹತ್ ಧಾರ್ಮಿಕ ಗ್ರಂಥರಾಶಿಯು ಸಾವಿರಾರು ಸಂಪುಟಗಳನ್ನು ಹೊಂದಿವೆ. ಯಾರಿಗೆ ಯಥಾರ್ಥ ಉಚ್ಚಾರಣೆ ಮತ್ತು ಸ್ವರ ತಿಳಿದಿದೆಯೊ ಅವನಿಗೆ ಮಾತ್ರ ವೇದಗಳು ತಿಳಿದಿರುತ್ತವೆ, ಉಳಿದವರಿಗೆ ಅವನ್ನು ತಿಳಿಯುವುದು ಸಾಧ್ಯವಿಲ್ಲ. ಪುರಾತನ ಕಾಲದಲ್ಲಿ ಕೆಲವು ರಾಜಮನೆತನದವರು ವೇದಗಳ ರಕ್ಷಕರಾಗಿದ್ದರು. ವಂಶದ ಮುಖ್ಯಸ್ಥನು ತನ್ನ ಪಾಲಿಗೆ ಬಂದ ವೇದಭಾಗದ ಪ್ರತಿಯೊಂದು ಶಬ್ದವನ್ನೂ ಸ್ವರಬದ್ಧವಾಗಿ ಸ್ವಲ್ಪವೂ ತಪ್ಪದೆ ಉಚ್ಚರಿಸಬಲ್ಲವನಾಗಿದ್ದನು. ಅವನ ಬುದ್ಧಿಶಕ್ತಿ ಅಪಾರ ಮತ್ತು ಜ್ಞಾಪಕಶಕ್ತಿ ಅದ್ಭುತ.

ಕರ್ಮಕಾಂಡವನ್ನು ನಂಬುವ ಕಚ್ಚಾ ಸಂಪ್ರದಾಯವಾದಿಗಳು ದೇವರನ್ನಾಗಲಿ ಆತ್ಮವನ್ನಾಗಲಿ ನಂಬುವುದಿಲ್ಲ; ಈ ಭೌತಿಕ ಅಥವಾ ಆಧ್ಯಾತ್ಮಿಕ ಜಗತ್ತಿನಲ್ಲಿರುವ ನಾವು ಮಾತ್ರವೇ ನಿಜವಾಗಿ ಅಸ್ತಿತ್ವದಲ್ಲಿರುವವರು. ವೇದಗಳಲ್ಲಿ ದೇವರ ಬಗ್ಗೆ ಇರುವ ಪ್ರಸ್ತಾಪವು ಏನು ಎಂದು ಅವರನ್ನು ಕೇಳಿದರೆ, ಅದಕ್ಕೆ ಯಾವ ಅರ್ಥವೂ ಇಲ್ಲ ಎನ್ನುತ್ತಾರೆ. ಶಬ್ದಗಳನ್ನು ಸರಿಯಾದ ರೀತಿಯಲ್ಲಿ ಉಚ್ಚರಿಸಿದಾಗ ಅವುಗಳಲ್ಲಿ ಒಂದು ಶಕ್ತಿ ಇದೆ, ಅದರಿಂದ ನಿರ್ದಿಷ್ಟ ಪ್ರತಿಫಲವುಂಟಾಗುತ್ತದೆ - ಎಂದು ಅವರು ಹೇಳುತ್ತಾರೆ. ಇದರಿಂದ ಆಚೆಗೆ ಅವುಗಳಿಗೆ ಯಾವ ಅರ್ಥವೂ ಇಲ್ಲ.

ನೀವು ಆಲೋಚನೆಯನ್ನು ಅದುಮಿಟ್ಟಾಗ ಅದು ಕೇವಲ ಕಣ್ಮರೆಯಾಗುತ್ತದೆಯಷ್ಟೆ. ಆದರೆ ಅದು ಒತ್ತಿ ಹಿಡಿದ ಸ್ಪ್ರಿಂಗ್ನಂತೆ ಇರುತ್ತದೆ. ಯಾವುದೇ ಸಂದರ್ಭದಲ್ಲಾದರೂ ಸ್ಟ್ರಿಂಗಿನಂತೆ ಎಲ್ಲ ಶಕ್ತಿಯೊಡನೆ ಮೇಲೇಳಬಹುದು ಮತ್ತು ಕ್ಷಣಾರ್ಧದಲ್ಲಿ ದೀರ್ಘ ಕಾಲದ ಪರಿಣಾಮವನ್ನು ಉಂಟುಮಾಡಬಹುದು.

ಒಂದು ಚಟಾಕು ಸುಖವೂ ಸೇರಿನಷ್ಟು ದುಃಖವನ್ನು ತರುತ್ತದೆ. ಸುಖವಾಗಿ ಅಭಿವ್ಯಕ್ತವಾಗುವ ಶಕ್ತಿಯೇ ಇನ್ನೊಂದು ಸಂದರ್ಭದಲ್ಲಿ ದುಃಖವಾಗಿ ವ್ಯಕ್ತವಾಗುತ್ತದೆ. ಒಂದುಬಗೆಯ ಸಂವೇದನೆ ನಿಂತಕೂಡಲೆ ಇನ್ನೊಂದು ಬಗೆಯದು ಪ್ರಾರಂಭವಾಗುತ್ತದೆ. ಆದರೆ ತುಂಬ ಮುಂದುವರಿದ ಕೆಲವರಲ್ಲಿ ಎರಡು ಅಥವಾ ಅನೇಕ ಬಗೆಯ ಆಲೋಚನೆಗಳು ಒಂದೇ ಸಂದರ್ಭದಲ್ಲಿ ಕ್ರಿಯಾಶೀಲವಾಗಿರಬಹುದು. ಒಂದು ಆಲೋಚನೆಯನ್ನು ಅದುಮಿದಾಗ ಅದು ಯಾವಾಗಲಾದರೂ ತನ್ನ ಎಲ್ಲ ಶಕ್ತಿ ಯೊಡನೆ ಮೇಲೇಳಬಹುದು.

“ಮನಸ್ಸು ತನ್ನ ಸ್ವರೂಪದಂತೆಯೇ ವರ್ತಿಸುತ್ತದೆ. ಮನಸ್ಸಿನ ಕ್ರಿಯೆಯೆಂದರೆ ಸೃಷ್ಟಿ. ಶಬ್ದವು ಆಲೋಚನೆಯನ್ನು ಅನುಸರಿಸುತ್ತದೆ, ರೂಪವು ಶಬ್ದವನ್ನು ಅನುಸರಿಸುತ್ತದೆ. ಈ ಭೌತಿಕ ಮತ್ತು - ಮಾನಸಿಕ ಸೃಷ್ಟಿಯೆಲ್ಲವೂ ನಿಲ್ಲಬೇಕು. ಆಗಲೇ ಮನಸ್ಸು ಆತ್ಮನ ಬೆಳಕನ್ನು ಪ್ರತಿಬಿಂಬಿಸುತ್ತದೆ.”

ನನ್ನ ಗುರುದೇವರಿಗೆ ತಪ್ಪಿಲ್ಲದೆ ತಮ್ಮ ಹೆಸರನ್ನೂ ಬರೆಯುವುದಕ್ಕೆ ಬರುತ್ತಿರಲಿಲ್ಲ. ಅವರು ತಮ್ಮ ಹೆಸರನ್ನು ಬರೆಯುವಾಗ ಮೂರು ಅಕ್ಷರದೋಷಗಳು ಇರುತ್ತಿದ್ದುವು. ಅಂಥ ವ್ಯಕ್ತಿಯ ಪದತಲದಲ್ಲಿ ನಾನು ಕುಳಿತಿದ್ದೆ.

ನೀವು ಒಮ್ಮೆ ಮಾತ್ರ ಪ್ರಕೃತಿಯ ನಿಯಮವನ್ನು ಉಲ್ಲಂಘಿಸುತ್ತೀರಿ; ಅದು ಕಟ್ಟ ಕೊನೆಯಲ್ಲಿ, ಆಗ ನಿಮಗೆ ಪ್ರಕೃತಿ ಏನೂ ಅಲ್ಲ.

