
\chapter{ಗೀತೆಯ ತರಗತಿ}

(೧೯೦೦ರ ಜೂನ್ ೧೬ರಂದು ನ್ಯೂಯಾರ್ಕ್ನಲ್ಲಿ ನೀಡಿದ ತರಗತಿ ಉಪನ್ಯಾಸದ ಟಿಪ್ಪಣಿ - ಸೋದರಿ ನಿವೇದಿತ ಬರೆದುಕೊಂಡುದು ಮತ್ತು ಇದು ಅವಳು ಮಿಸ್ ಮ್ಯಾಕ್ಲಿಯೋಡ್ಳಿಗೆ ಬರೆದ ಪತ್ರದ ಭಾಗವಾಗಿದೆ.)

ಈ ಬೆಳಿಗ್ಗೆ ಗೀತೆಯ ಮೇಲಿನ ಪಾಠ ತುಂಬ ಚೆನ್ನಾಗಿತ್ತು. ಅತ್ಯುನ್ನತ ಭಾವನೆಗಳು ಎಲ್ಲರಿಗಾಗಿ ಅಲ್ಲ ಎಂಬ ವಿಷಯದ ಮೇಲಿನ ದೀರ್ಘ ಚರ್ಚೆಯೊಂದಿಗೆ ಅದು ಪ್ರಾರಂಭವಾಯಿತು. ಯಾವುದೇ ಬಗೆಯ ಸೇವೆಯನ್ನು ಅಸಹ್ಯವೆಂದು ಪರಿಗಣಿಸು ವವನಿಗೆ ಅಪ್ರತೀಕಾರವು ಆದರ್ಶವಾಗಲಾರದು. ಅಪ್ರತೀಕಾರವು ಕೋಪಗೊಂಡ ಮಗುವನ್ನು ತಾಯಿಯು ಸಮಾಧಾನಪಡಿಸುವಾಗ ವ್ಯಕ್ತವಾಗುತ್ತದೆ. ದುರ್ಬಲನು ಅದನ್ನು ಅನುಸರಿಸಿದರೆ ಅದು ಆ ಆದರ್ಶದ ಅಪಹಾಸ್ಯವಾಗುತ್ತದೆ.

ನಾವು ಯಥಾರ್ಥರಾಗಿರೋಣ. ನಾವು ಏನಾಗಿಲ್ಲವೊ ಅದನ್ನು ಜನರಿಗೆ ತೋರ್ಪಡಿಸುವುದರಲ್ಲಿ ನಮ್ಮ ಶಕ್ತಿಯ ಹತ್ತನೆ ಒಂಬತ್ತು ಭಾಗ ವ್ಯಯವಾಗುತ್ತದೆ. ನಾವು ಏನಾಗಿರುವೆವೊ ಅದರಂತೆ ಸ್ವಾಭಾವಿಕವಾಗಿ ವರ್ತಿಸುವುದರ ಮೂಲಕ ಆ ಶಕ್ತಿಯನ್ನು ಸರಿಯಾದ ರೀತಿಯಲ್ಲಿ ಬಳಸಿಕೊಂಡಂತಾಗುತ್ತದೆ. ಅವತಾರಪುರುಷನಿಗೆ ಪ್ರಾರ್ಥನೆಯೊಂದಿಗೆ ಇದು ಹೀಗೆ ಮುಂದುವರಿಯಿತು:

ಯಾರ ಪಾದವು ದೇವತೆಗಳಿಂದ ಪೂಜಿಸಲ್ಪಟ್ಟಿದೆಯೋ ಆ ಜಗದ್ಗುರುವಾದ ನಿನಗೆ ನಮಸ್ಕಾರ. ನೀನು ಭವರೋಗ ವೈದ್ಯ. ನೀನು ದೇವತೆಗಳಿಗೆಲ್ಲ ಗುರು. ನಿನಗೆ ನಮ್ಮ ನಮಸ್ಕಾರಗಳು. ನಿನ್ನನ್ನೇ ನಮಸ್ಕರಿಸುತ್ತೇವೆ, ನಿನ್ನನ್ನೇ ನಮಸ್ಕರಿಸುತ್ತೇವೆ, ನಿನ್ನನ್ನೇ ನಮಸ್ಕರಿಸುತ್ತೇವೆ.

ಈ ಪ್ರಾರ್ಥನೆಯನ್ನು ಸ್ವಾಮೀಜೀಯವರು ಭಾರತೀಯರಾಗದಲ್ಲಿ ಪಠಿಸಿದರು.

ಉಪನ್ಯಾಸದುದ್ದಕ್ಕೂ ಸಮಸ್ಯೆಯನ್ನು ಗ್ರಹಿಸುವ ವಿಷಯದಲ್ಲಿ ಕ್ರಿಸ್ತ ಮತ್ತು ಬುದ್ಧ ಇವರು ಶ‍್ರೀಕೃಷ್ಣನಿಗಿಂತ ಕೆಳಗಿನವರು ಎಂಬ ಸೂಚನೆಯಿತ್ತು. ಕ್ರಿಸ್ತ ಮತ್ತು ಬುದ್ಧ ಅತ್ಯುನ್ನತ ನೈತಿಕತೆಯನ್ನು ಎಲ್ಲರಿಗೂ ಅನ್ವಯವಾಗುವಂತೆ ಬೋಧಿಸಿದರು. ಆದರೆ ಕೃಷ್ಣನಾದರೂ ಸಮಗ್ರವನ್ನೂ ಅದರ ಎಲ್ಲ ಅಂಶಗಳೊಂದಿಗೆ ಅರ್ಥಮಾಡಿಕೊಂಡಿದ್ದನು, ಅವುಗಳ ಬೇರೆ ಬೇರೆ ಆದರ್ಶಗಳನ್ನೂ ಅರ್ಥಮಾಡಿಕೊಂಡಿದ್ದನು. ಅದಕ್ಕನುಗುಣವಾಗಿ ಅವನು ಬೋಧಿಸಿದನು.

