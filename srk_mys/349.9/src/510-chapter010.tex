
\chapter{ಅಧ್ಯಾಯ ೧೦: ಅಮರನಾಥದ ಗುಡಿ}

ಸ್ಥಳ: ಕಾಶ್ಮೀರ.

ಕಾಲ: ೧೮೯೮ರ ಜುಲೈ ೨೯ರಿಂದ ಆಗಸ್ಟ್ ೮ರವರೆಗೆ.

\textbf{ಜುಲೈ ೨೯.}

ಇಲ್ಲಿಂದ ಮುಂದೆ ನಾವು ಸ್ವಾಮಿಗಳನ್ನು ನೊಡಿದ್ದು ಬಹು ಸ್ವಲ್ಪ. ತೀರ್ಥಯಾತ್ರೆಯ ಉತ್ಸಾಹದಿಂದ ತುಂಬ ತುಳುಕುತ್ತಿದ್ದ ಅವರು, ದಿನಕ್ಕೆ ಒಮ್ಮೆ ಮಾತ್ರ ಆಹಾರ ಸೇವಿಸುತ್ತ, ಸಾಧುಗಳ ಹೊರತಾಗಿ ಇನ್ನಾರ ಜೊತೆಯನ್ನೂ ಬಯಸದೆ ಜೀವಿಸಲಾರಂಭಿಸಿದರು. ಕೆಲವೊಮ್ಮೆ ಕೈಯಲ್ಲಿ ಜಪಸರ ಹಿಡಿದುಕೊಂಡು ಬಿಡದಿಯ ತಾಣಕ್ಕೆ ಬರುತ್ತಿದ್ದರು. ಈ ಹೊತ್ತು ರಾತ್ರಿ ನಮ್ಮ ಗುಂಪಿನ ಇಬ್ಬರು, ಪವಿತ್ರ ಚಿಲುಮೆಗಳಲ್ಲಿ ಕೇಂದ್ರೀಕೃತವಾಗಿದ್ದ ಧಾರ್ಮಿಕ ಹಳ್ಳಿಜಾತ್ರೆಯಂತಿದ್ದ ಬಾವನ್ನಲ್ಲಿ ತಿರುಗಾಡಿ ಬರುವುದಕ್ಕೆಂದು ಹೊರ ಟರು. ಆ ಬಳಿಕ ಧೀರಮಾತಾಳ ಜೊತೆಗೆ ಹೋಗಿ, ಹಿಂದೀ ಮಾತನಾಡುವ ಸಾಧುಗಳ ದಟ್ಟಣೆ ನೆರೆದಿದ್ದ ಡೇರೆಯ ದ್ವಾರದಲ್ಲಿ ಕುಳಿತು, ಸ್ವಾಮಿಗಳನ್ನು ಪ್ರಶ್ನಿಸುತ್ತಿದ್ದ ಅವರ ಸಂಭಾಷಣೆಯನ್ನು ಕೇಳುವುದು ಸಾಧ್ಯವಾಯಿತು.

ಗುರುವಾರ ಪಹಲ್ಗಾಮ್​ ತಲುಪಿದ ನಾವು ಕಣಿವೆಯ ಕೆಳಭಾಗದ ಕೊನೆಯಲ್ಲಿ ನಮ್ಮ ಬಿಡದಿಯನ್ನು ಸ್ಥಾಪಿಸಿಕೊಂಡೆವು. ನನಗೆ ಪ್ರವೇಶ ಕೊಟ್ಟಿದ್ದೇಕೆ ಎಂಬ ಪ್ರಶ್ನೆಯ ಮೇಲೆ ಸ್ವಾಮಿಗಳು ಪ್ರಬಲವಾದ ವಿರೋಧವನ್ನು ಎದುರಿಸಬೇಕಾಯಿತು ಎಂಬುದನ್ನು ಕಂಡೆವು. ಸ್ವಾಮಿಗಳನ್ನು ಬೆಂಬಲಿಸುತ್ತಿದ್ದ ನಗ್ನ ಸಾಧುಗಳಲ್ಲೊಬ್ಬರು “ನಿಮಗೆ ಈ ಸಾಮರ್ಥ್ಯವಿರುವುದೇನೋ ನಿಜ, ಸ್ವಾಮೀಜಿ, ಆದರೆ ಅದನ್ನು ನೀವು ಪ್ರಕಾಶಕ್ಕೆ ತರತಕ್ಕದ್ದಲ್ಲ!” ಎಂದರು. ಸ್ವಾಮಿಗಳು ಈ ಮಾತಿಗೆ ತಲೆಬಾಗಿದರು. ಏನೇ ಇರಲಿ, ಆ ಹೊತ್ತು ಮಧ್ಯಾಹ್ನ ಸ್ವಾಮಿಗಳು ತಮ್ಮ ಮಗಳನ್ನು ಬಿಡದಿಯ ಉದ್ದಕ್ಕೂ ಆಶೀರ್ವಾದ ಪಡೆಯುವುದಕ್ಕೆಂದು ಕರೆದೊಯ್ದರು - ಇದರ ನಿಜವಾದ ಅರ್ಥ ಎಲ್ಲರಿಗೂ ಭಿಕ್ಷೆಯನ್ನು ನೀಡುವುದು - ಇವರನ್ನು ಶ‍್ರೀಮಂತರೆಂದುಕೊಂಡರೋ, ಅಥವಾ ಇವರನ್ನು ತಮಗಿಂತ ಹೆಚ್ಚು ಶಕ್ತಿಶಾಲಿಗಳೆಂದು ಮನ್ನಿಸಿದರೊ, ಅಂತೂ ಮಾರನೆಯ ದಿನವೇ ನಮ್ಮಗಳ ಡೇರೆಗಳನ್ನೆಲ್ಲ ಮೇಲಕ್ಕೆಕೊಂಡೊಯ್ದು ಸಾಧುಗಳ ಬೀಡಿನ ನಡುವೆಯೇ ಒಂದು ರಮ್ಯವಾದ ದಿಬ್ಬದ ಮೇಲೆ ಸ್ಥಳಾಂತರಿಸಲಾಯಿತು....

\textbf{ಜುಲೈ ೩೦.}

ಮುಂದಿನ ತಂಗುದಾಣ ಚಂದನ್ವಾಡಿಗೆ ಹೋಗುವ ಹಾದಿ ಅದೆಷ್ಟು ಸುಂದರವಾಗಿತ್ತು! ಅಲ್ಲಿ ನಾವು ಕಮರಿಯೊಂದರ ಅಂಚಿನಲ್ಲಿ ಬೀಡು ಬಿಟ್ಟೆವು. ಮಧ್ಯಾಹ್ನವೆಲ್ಲ ಮಳೆ ಸುರಿಯಿತು. ಸ್ವಾಮಿಗಳು ನನ್ನನ್ನು ಮಾತನಾಡಿಸುವುದಕ್ಕೆಂದು ಕೇವಲ ಐದು ನಿಮಿಷಗಳ ಭೇಟಿಗೆ ಬಂದರು. ಆದರೆ ಇನ್ನಿತರಯಾತ್ರಿಕರಿಂದ, ನನಗೆ ಅದೆಷ್ಟೋ ಮನ ಮುಟ್ಟುವಂತಹ ಆದರ ದೊರೆಯಿತು...

...ಚಂದನ್ ವಾಡಿಯ ಬಳಿ ಬಂದಾಗ ಸ್ವಾಮಿಗಳು ನಾನು ನನ್ನ ಮೊಟ್ಟ ಮೊದಲ ನೀರ್ಗಲ್ಲ ಪರ್ವತಾರೋಹಣವನ್ನು ಕೈಕೊಳ್ಳಬೇಕೆಂದು ಆಗ್ರಹಿಸಿದರು; ಆಸಕ್ತಿ ಕೆರಳಿಸುವ ಪ್ರತಿಯೊಂದು ವಿವರವನ್ನೂ ನನಗೆ ತಿಳಿಸಿ ಹೇಳಿದರು. ಮುಂದೆ ಅನೇಕ ಸಾವಿರ ಅಡಿಗಳ ಅದ್ಭುತ ಆರೋಹಣದ ಅನುಭವ ನನ್ನದಾಯಿತು. ನಂತರ ಪರ್ವತಗಳನ್ನು ಒಂದಾದ ಮೇಲೊಂದರಂತೆಸುತ್ತಿಸುತ್ತಿ ಹೋಗುವ ಕಿರಿದಾದ ಕಾಲುಹಾದಿಯಲ್ಲಿನ ದೀರ್ಘ ನಡಿಗೆ; ಕೊನೆಗೆ ಇನ್ನೊಂದು ಕಡಿದಾದ ಪರ್ವತಾರೋಹಣ. ಮೊದಲನೆಯ ಪರ್ವತದ ತುಟ್ಟತುದಿಯಲ್ಲಿ ಉಣ್ಣೆಯಂಥ ಎಲೆಗಳ, ಬಿಳಿಹೂಗಳ ಪುಟ್ಟಗಿಡಗಳ ನೆಲಹಾಸು. ಅನಂತರ, ನಿಷ್ಕ್ರಿಯ ನೀರಿನಿಂದ ಕೂಡಿದ ಹಾದಿಯು ಶೇಷನಾಗಕ್ಕಿಂತ ಇನ್ನೂ ಐನೂರು ಅಡಿ ಎತ್ತರಕ್ಕೆ ಹಾದುಹೋಯಿತು; ಕೊನೆಗೊಮ್ಮೆ ನಾವು ಸುಮಾರು ಹದಿನೆಂಟು ಸಾವಿರ ಅಡಿ ಎತ್ತರದ ಮಂಜು ಮುಸುಕಿದ ಶಿಖರಗಳೆಡೆಯಲ್ಲಿ ಚಳಿಯ ಜೌಗು ಪ್ರದೇಶವೊಂದರಲ್ಲಿ ಬೀಡುಬಿಟ್ಟೆವು. ಫರ್ (ಭದ್ರದಾರು) ಮರಗಳು ತೀರ ಕೆಳಗಿದ್ದವು; ಹಾಗಾಗಿ ಕೂಲಿಗಳು ಮಧ್ಯಾಹ್ನ ಹಾಗೂ ಸಂಜೆಯಿಡೀ ದಿಕ್ಕುದಿಕ್ಕುಗಳಲ್ಲಿ ಸಂಚರಿಸಿ ಜೂನಿಪರ್ (ಎಣ್ಣೆ ಇರುವ ಒಂದು ಜಾತಿಯ ಹಸಿರುಪೊದೆ)ಗಳನ್ನು ಸಂಗ್ರಹಿಸಬೇಕಾಯಿತು. ತಹಶೀಲ್ದಾರರ, ಸ್ವಾಮೀಜಿಯವರ ಹಾಗೂ ನನ್ನ ಡೇರೆಗಳು ಹತ್ತಿರವಾಗಿದ್ದವು; ಸಂಜೆಗತ್ತಲಿನಲ್ಲಿ ಇವುಗಳ ಮುಂದೆ ದೊಡ್ಡದಾಗಿ ಬೆಂಕಿಯನ್ನು ಹಚ್ಚಿದೆವು. ಆದರೆ ಅದು ಚೆನ್ನಾಗಿ ಉರಿಯಲಿಲ್ಲ; ಅನೇಕ ಅಡಿಗಳ ಕೆಳಕ್ಕೆ ಹಿಮನದಿಯೇ ಇತ್ತು. ಬೀಡುಬಿಟ್ಟ ಮೇಲೆ ನಾನು ಸ್ವಾಮಿಗಳನ್ನು ನೋಡಲಿಲ್ಲ.

ಪಂಚತರಣಿ - ಐದು ಝರಿಗಳ ಪ್ರದೇಶ - ಅಲ್ಲಿಂದ ಹೆಚ್ಚೇನೂ ದೂರವಿರಲಿಲ್ಲ. ಅಲ್ಲದೆ ಅದು ಶೇಷನಾಗಕ್ಕಿಂತ ಕೆಳಕ್ಕೆ ಇತ್ತು; ಚಳಿ ಶುಷ್ಕವಾಗಿದ್ದರೂ ಉಲ್ಲಾಸ ದಾಯಕವಾಗಿತ್ತು. ನಮ್ಮ ಬಿಡದಿಯ ಮುಂದೆ ಅಗಲವಾದ ಒಂದು ನದೀಪಾತ್ರ; ಅದರ ನಡುವೆ ಹರಿಯುತ್ತಿದ್ದವು ಆ ಐದು ಝರಿಗಳು.ಯಾತ್ರಿಕರು ಒಂದರಲ್ಲಿ ಮುಳುಗಿ ಎದ್ದ ಮೇಲೆ ಒದ್ದೆ ಬಟ್ಟೆಯಲ್ಲಿಯೇ ಇನ್ನೊಂದಕ್ಕೆ ಹೋಗುತ್ತ ಅವೆಲ್ಲವುಗಳಲ್ಲೂ ಮೀಯಬೇಕಾಗಿತ್ತು. ಈ ಆಚರಣೆಯಿಂದ ಜಾರಿಕೊಳ್ಳಲೆತ್ನಿಸದೆ ಸ್ವಾಮೀಜಿ ಈ ವಿಚಾರದಲ್ಲಿ ನಿಯಮವನ್ನು ಅಕ್ಷರಶಃ ಪಾಲಿಸಿದರು....

ಈ ಔನ್ನತ್ಯಗಳಲ್ಲಿ ಅನೇಕ ಬಾರಿ ಬೃಹತ್ ವೃತ್ತಾಕಾರದಲ್ಲಿಸುತ್ತುವರೆದ ಮಂಜು ಮುಸುಕಿದ ಶಿಖರಗಳ ಆವರಣದಲ್ಲಿ ನಾವಿರುವ ದಿವ್ಯಾನುಭವವನ್ನು ಪಡೆದೆವು - ಹಿಂದೂಗಳ ಮನಸ್ಸಿಗೆ ಬೂದಿಬಳಿದ ದೇವರ ಕಲ್ಪನೆಯನ್ನು ತಂದುಕೊಟ್ಟಿರುವ ಆ ಮೌನವಾಂತ ಭೌಮಗಳು!

\textbf{ಆಗಸ್ಟ್ ೨.}

ಮಂಗಳವಾರ, ಆಗಸ್ಟ್ ೨ರಂದು, ಅಮರನಾಥದ ಮಹಾದಿನ,ಯಾತ್ರಿಕರ ಮೊದಲ ಗುಂಪು ಬಹುಶಃ ಬಿಡದಿಯನ್ನು ಎರಡು ಗಂಟೆಗೆ ಬಿಟ್ಟಿರಬೇಕು! ನಾವು ಪೂರ್ಣಚಂದ್ರನ ಬೆಳಕಿನಲ್ಲಿ ಬಿಟ್ಟು ಹೊರಟೆವು. ಕಿರಿದಾದ ಕಣಿವೆಯೊಳಗಿನಿಂದ ನಾವು ಹೋಗುತ್ತಿದ್ದಂತೆ ಸೂರ್ಯೋದಯವಾಯಿತು. ಈ ಭಾಗದ ಪಯಣ ಕ್ಷೇಮಕರವಲ್ಲ. ಆದರೆ ನಮ್ಮ ದಂಡಿಗಳನ್ನು ಬಿಟ್ಟು ಪರ್ವತ ಹತ್ತುವುದಕ್ಕೆ ಆರಂಭಿಸುತ್ತಲೂ ನಿಜವಾದ ಅಪಾಯ ಎದುರಾಯಿತು... ನಂತರ, ಕೊನೆಗೊಮ್ಮೆ ಮುಂದಿನ ಏರಿನ ತಳಕ್ಕೆ ಹೋಗಿ ಸೇರಿದ ಮೇಲೆ, ನಾವು ಹಿಮನದಿಯೊಟ್ಟಿಗೇ ಒಂದಾದ ಮೇಲೊಂದರಂತೆ ಮೈಲಿಗಳನ್ನು ಕಷ್ಟಪಟ್ಟು ಸವೆಸಬೇಕಾಯಿತು, ಗುಹೆಯನ್ನು ತಲುಪಲು....

ಇಷ್ಟರಲ್ಲಿ ತೀರ ಆಯಾಸಗೊಂಡಿದ್ದ ಸ್ವಾಮಿಗಳು ಹಿಂದೆ ಬಿದ್ದಿದ್ದರು... ಕೊನೆಗೊಮ್ಮೆ ಬಂದು ತಲುಪಿದ ಅವರು, ತಾವು ಸ್ನಾನ ಮಾಡುವುದಾಗಿ ಹೇಳಿ ನನ್ನನ್ನು ಮುಂದಕ್ಕೆ ಕಳುಹಿಸಿದರು. ಅರ್ಧಗಂಟೆಯ ನಂತರ ಅವರು ಗುಹೆಯನ್ನು ಹೊಕ್ಕರು. ಮಂದಹಾಸವನ್ನು ಬೀರುತ್ತ, ಅರ್ಧವೃತ್ತಾಕಾರದ ಒಂದು ಕಡೆ ಮಂಡಿಯೂರಿ ನಮಸ್ಕರಿಸಿದರು; ನಂತರ ಇನ್ನೊಂದು ಕಡೆಯೂ ಹಾಗೆಯೇ ಮಾಡಿದರು. ಜಾಗವು ಅಲ್ಲೊಂದು ದೇಗುಲವನ್ನೇ ಕಟ್ಟಬಹುದಾದಷ್ಟು ದೊಡ್ಡದಾಗಿದ್ದಿತು; ಗಾಢವಾದ ನೆರಳಿನ ಆಯಕಟ್ಟಿನ ಗೂಡೊಂದರಲ್ಲಿ ನೀರ್ಗಲ್ಲ ಶಿವಲಿಂಗವು ತನ್ನದೇ ಪಾಣಿಪೀಠದ ಮೇಲೆ ತಾನೇ ಆವಿರ್ಭವಿಸಿರುವಂತೆ ಇತ್ತು. ಕೆಲವು ನಿಮಿಷಗಳಾದ ನಂತರ ಸ್ವಾಮಿಗಳು ಗುಹೆಯಿಂದ ಹೊರಗೆ ಬಂದರು.

ಅವರಿಗೆ ಸ್ವರ್ಗವೇ ಬಗೆಕಣ್ಣ ಮುಂದೆ ತೆರೆದಂತಾಗಿತ್ತು. ಶಿವನ ಪಾದಗಳನ್ನು ಅವರು ಸ್ಪರ್ಶಿಸಿದ್ದರು. “ತಲೆಸುತ್ತಿ ಬಂದು ಬೀಳದ ಹಾಗೆ” ತಮ್ಮನ್ನು ತಾವು ಕಷ್ಟಪಟ್ಟು ನಿಯಂತ್ರಿಸಿಕೊಳ್ಳಬೇಕಾಯಿತೆಂದು ಅನಂತರ ಅವರು ಹೇಳಿದರು. ಆದರೆ ಅವರ ದೇಹದ ಆಯಾಸ ಅದೆಷ್ಟಾಗಿತ್ತೆಂದರೆ ಹೃದಯವು ಆಗಲೇ ತನ್ನ ಕೆಲಸವನ್ನು ನಿಲ್ಲಿಸಿ ಬಿಡಬೇಕಾಗಿತ್ತು, ಬದಲಿಗೆ ಶಾಶ್ವತವಾಗಿ ಹಿರಿದಾಗಿಬಿಟ್ಟಿದೆ ಎಂದು ಆ ನಂತರ ವೈದ್ಯ ರೊಬ್ಬರು ಹೇಳಿದರು. ಅಚ್ಚರಿ ಅವರ ಗುರುಗಳ ಆ ಮಾತು - “ಅವನು ತಾನಾರೆಂಬುದನ್ನು ಯಾವಾಗ ಅರಿತುಕೊಳ್ಳುವನೋ ಆಗಲೇ ಈ ಶರೀರವನ್ನು ತ್ಯಜಿಸಿ ಬಿಡುವನು!” ಎಂಬುದು - ಸಫಲವಾಗುವುದಕ್ಕೆ ಅದೆಷ್ಟು ಸಮೀಪ ಬಂದಿದ್ದರು ಅವರು!

ಅರ್ಧಗಂಟೆಯ ನಂತರ, ನನ್ನ ಹಾಗೂ ಆ ದಯಾಮಯ ನಗ್ನ ಸಾಧುವಿನ ಜೊತೆಗೆ ಝರಿಯ ಪಕ್ಕದ ಬಂಡೆಯೊಂದರ ಮೇಲೆ ಕುಳಿತು ಬುತ್ತಿಯನ್ನು ಬಿಚ್ಚಿ ತಿನ್ನು ತ್ತ “ನಾನು ಅದೆಂತಹ ಆನಂದಾನುಭವವನ್ನು ಪಡೆದೆ!” ಎಂದರವರು. “ಮಂಜುಗಡ್ಡೆಯ ಲಿಂಗವು ಸಾಕ್ಷಾತ್ಶಿವನೆಂದೇ ಬಗೆದೆ. ಕಳ್ಳ ಬ್ರಾಹ್ಮಣರು ಇಲ್ಲಿರಲಿಲ್ಲ, ಯಾವ ವ್ಯಾಪಾರವಿಲ್ಲ, ಯಾವ ಅಡಚಣೆಯೂ ಇಲ್ಲ. ಎಲ್ಲವೂ ಪೂಜಾಮಯ! ನಾನೆಂದೂ ಯಾವ ತೀರ್ಥಯಾತ್ರೆಯಲ್ಲೂ ಇಷ್ಟೊಂದು ಆನಂದವನ್ನು ಅನುಭವಿಸಿರಲಿಲ್ಲ!”

ಆ ನಂತರ ಅವರು ಆಗಾಗ ತಮ್ಮನ್ನು ಆನಂದಪರಾಕಾಷ್ಠೆಗೆಕೊಂಡೊಯ್ದ ಅಂದಿನ ದರ್ಶನದ ಅದ್ಭುತ ಅನುಭವವನ್ನು ಕುರಿತು ಹೇಳುತ್ತಿದ್ದರು. ಆ ಶ್ವೇತವರ್ಣದ ನೀರ್ಗಲ್ಲಿನ ಶಿವಲಿಂಗದ ಕಾವ್ಯಮಯತೆಯನ್ನು ವರ್ಣಿಸುವರು; ಬೇಸಿಗೆಯ ಒಂದು ದಿನ ತಮ್ಮ ಕುರಿಗಳನ್ನು ಅರಸಿಕೊಂಡು ತಿರುಗುತ್ತಿದ್ದ ಕುರುಬ ಹುಡಗರ ಗುಂಪೊಂದು ಮೊಟ್ಟಮೊದಲು ಆ ಗವಿಯನ್ನು ಪ್ರವೇಶಿಸಿ ಎಂದೂ ಕರಗದ ಆ ಶಿವಲಿಂಗವನ್ನು - ಭಗವಂತನ ಸನ್ನಿಧಿಯನ್ನು -ಆವಿಷ್ಕರಿಸಿರಬೇಕು ಎಂದು ಊಹಿಸಿದವರೂ ಅವರೇ. ತಮಗೆ ಅಲ್ಲಿ ಅಮರನಾಥನ ಕೃಪೆ ಲಭಿಸಿತೆಂದೂ, ತಾನು ಅನುಮತಿ ಕೊಡುವವರೆಗೆ ದೇಹತ್ಯಾಗ ಮಾಡಕೂಡದೆಂದು ಭಗವಂತನ ಆಣತಿ ತಮಗಾಯಿತೆಂದೂ ಯಾವಾಗಲೂ ಹೇಳುತ್ತಿದ್ದರು. ನನಗೆ ಅವರು ಹೇಳಿದರು: “ನಿನಗೀಗ ಅರ್ಥವಾಗುವುದಿಲ್ಲ. ಆದರೆ ನೀನು ತೀರ್ಥಯಾತ್ರೆ ಮಾಡಿ ಬಂದಿರುವೆ, ಅದು ಕೆಲಸಮಾಡುತ್ತಾ ಹೋಗುತ್ತದೆ. ಕಾರಣಗಳು ಪರಿಣಾಮಗಳ ನ್ನುಂಟುಮಾಡಲೇಬೇಕು. ಆ ನಂತರ ನೀನು ಅರ್ಥಮಾಡಿಕೊಳ್ಳುವೆ. ಪರಿಣಾಮಗಳು ಬಂದೇ ಬರುವುವು”.

ಮಾರನೆಯ ಬೆಳಗ್ಗೆ ಅಲ್ಲಿಂದ ಪಹಲ್ ಗಾಮ್​ಗೆ ನಾವು ಹಿಂದಿರುಗಿ ಹೊರಟ ಹಾದಿ ಅದೆಷ್ಟು ಸುಂದರವಾಗಿತ್ತು! ಅಂದಿನ ರಾತ್ರಿ ನಾವು ಮುಂದಿನ ಹಂತದವರೆಗೆ ನಡೆದು ಮಂಜು ಮುಸುಕಿದ ಹಾದಿಯಲ್ಲೇ ಒಂದು ಕಡೆ ಡೇರೆಗಳನ್ನು ಸ್ಥಾಪಿಸಿಕೊಂಡು ತಂಗಿದೆವು. ಇಲ್ಲಿ ನಾವು ಕೂಲಿಯೊಬ್ಬನಿಗೆ ಕೆಲವು ಆಣೆಗಳನ್ನು ಕೊಟ್ಟು ನಮ್ಮ ಒಂದು ಪತ್ರವನ್ನು ತೆಗೆದುಕೊಂಡು ಮುಂದೆ ಹೋಗೆಂದು ಕಳುಹಿಸಿದೆವು. ಆದರೆ ನಾವು ಮಾರನೆಯ ದಿನ ಮಧ್ಯಾಹ್ನ ತಲುಪಿದಾಗ ಅದು ಅನವಶ್ಯಕವಾಗಿತ್ತೆಂದು ಗೊತ್ತಾಯಿತು; ಏಕೆಂದರೆ ಬೆಳಗಿನಿಂದಲೂ ತಂಡ ತಂಡವಾಗಿ ಡೇರೆಗಳನ್ನು ಹೊತ್ತು ಬರುತ್ತಿದ್ದಯಾತ್ರಿಕರ ಗುಂಪುಗಳು ಅತ್ಯಂತ ಸ್ನೇಹದಿಂದ ನಮ್ಮ ಕುಶಲವನ್ನು ವಿಚಾರಿಸುತ್ತ, ನಾವು ಬರುವ ಸುದ್ದಿಯನ್ನು ಮುಂದಕ್ಕೆಕೊಂಡೊಯ್ಯುತ್ತ ಸಾಗುತ್ತಿದ್ದರು. ರಾತ್ರಿ ಕಳೆದು ಬೆಳಗಾಗುತ್ತಲೂ ಸೂರ್ಯೋದಯಕ್ಕಿಂತ ಮುಂಚೆಯೇ ನಾವು ದಾರಿ ನಡೆಯತೊಡಗಿದ್ದೆವು. ನಮ್ಮ ಮುಂದೆ ಸೂರ್ಯನು ಉದಯಿಸಿ ಮೇಲೆ ಬರುತ್ತಿದ್ದಂತೆ, ಹಿಂದುಗಡೆ ಚಂದ್ರನು ಮುಳುಗುತ್ತಿದ್ದಂತೆ, ನಾವು ಮೃತ್ಯುಸರೋವರದ ಮೇಲೆ ಹಾದುಹೋದೆವು. ಇಲ್ಲಿ ಒಂದಾನೊಂದು ವರ್ಷ ಹಿಮ ನದಿಯ ಪ್ರವಾಹಕ್ಕೆ ಸಿಕ್ಕಿ ಸುಮಾರು ನಲವತ್ತು ಮಂದಿಯಾತ್ರಿಕರು ಭೂಗತರಾಗಿದ್ದರು. ಇದಾದ ನಂತರ ನಾವು ಕಡಿದಾದ ಒಂದು ಶಿಖರದಿಂದ ಕೆಳಗಿಳಿಯುವ ಕಿರಿದಾದ ಒಂದು ಮೇಕೆದಾಟಿನಂತಹ ದಾರಿಯನ್ನು ಹಿಡಿದು ಸಾಗಿ ಪ್ರಯಾಣದ ದೂರವನ್ನು ಬಹು ಮಟ್ಟಗೆ ಕಡಿಮೆ ಮಾಡಿಕೊಳ್ಳುವುದಕ್ಕೆ ಸಾಧ್ಯವಾಯಿತು. ಈ ಹಾದಿಯಲ್ಲಿ ನಾವು ಚತು ಷ್ಪಾದಿಗಳಾಗಿಯೇ ಸಾಗಬೇಕಾಯಿತು; ಪ್ರತಿಯೊಬ್ಬರೂ ಅಂಬೆಗಾಲಿಡುತ್ತ ಜೀವವನ್ನು ಕೈಯಲ್ಲಿಟ್ಟುಕೊಂಡು ಮುನ್ನಡೆಯಬೇಕಾಯಿತು. ತಳವನ್ನು ಸೇರಿದಾಗ ಅಲ್ಲಿ ಗ್ರಾಮಸ್ಥರು ಬೆಳಗಿನ ತಿಂಡಿಯಂಥದನ್ನು ಸಿದ್ಧಪಡಿಸುತ್ತಿದ್ದರು. ಒಲೆಗಳು ಉರಿಯುತ್ತಿದ್ದವು, ಚಪಾತಿಗಳು ಬೇಯುತ್ತಿದ್ದವು, ಟೀ ಬಿಸಿಯಾಗಿ ಕುಡಿಯುವುದಕ್ಕೆ ಸಿದ್ಧವಾಗಿದ್ದಿತು. ಇಲ್ಲಿಂದ ಮುಂದಕ್ಕೆಯಾತ್ರಿಕರ ಒಂದೊಂದು ಗುಂಪೂ ಮುಖ್ಯವಾಹಿನಿಯನ್ನು ಬಿಟ್ಟು ಬೇರೆಯಾಗಿ ತಮ್ಮ ತಮ್ಮ ದಾರಿಯನ್ನು ಹಿಡಿಯತೊಡಗಿದರು, ಇಷ್ಟು ದಿನಗಳೂ ಇಡಿಯ ಪ್ರವಾಸದುದ್ದಕ್ಕೂ ನಮ್ಮೆಲ್ಲರಲ್ಲಿಯೂ ಬೆಳೆದಿದ್ದ ನಾವೆಲ್ಲರೂ ಒಂದೇ ಎಂಬ ಒಗ್ಗಟ್ಟಿನ ಭಾವನೆಯು ಕ್ರಮೇಣ ಕಡಿಮೆಯಾಗುತ್ತ ಹೋಯಿತು.

ಆ ಸಂಜೆ ಪಹಲ್ ಗಾಮ್​ಗಿಂತ ಸ್ವಲ್ಪ ಎತ್ತರದ ದಿಣ್ಣೆಯೊಂದರ ಮೇಲೆ ಪೈನ್ ದಿಮ್ಮಿಗಳನ್ನು ಉರಿಸಿ ಮಾಡಿದ ದೊಡ್ಡ ಬೆಂಕಿಯಲ್ಲಿ ಕಾಯಿಸಿಕೊಳ್ಳುತ್ತ,ಸುತ್ತಲೂ ನೆಲಕ್ಕೆ ಹಾಸಿದ ಜಮಖಾನಗಳ ಮೇಲೆ ಕುಳಿತು ನಾವೆಲ್ಲರೂ ಮಾತನಾಡತೊಡಗಿದೆವು. ಮಿತ್ರರಾದ ನಗ್ನ ಸ್ವಾಮಿಗಳೂ ನಮ್ಮನ್ನು ಬಂದು ಕೂಡಿಕೊಂಡರು; ಬೇಕಾದಷ್ಟು ತಮಾಷೆಮಾಡಿಕೊಂಡೆವು; ಕಾಡುಹರಟೆ ಹೊಡೆದೆವು. ಎಲ್ಲರೂ ಹೊರಟುಹೋಗಿ ನಮ್ಮ ಪುಟ್ಟ ಗುಂಪು ಮಾತ್ರವೇ ಉಳಿದ ಮೇಲೆ, ಜೊತೆಗೆ ಉಳಿದದ್ದು ತಲೆಯ ಮೇಲಿನ ಚಂದ್ರ, ಎತ್ತರದ ಮಂಜಿನ ಶಿಖರಗಳು, ಭೋರ್ಗರೆವ ನದಿಗಳು ಹಾಗೂ ಪರ್ವತದ ಪೈನ್ ವೃಕ್ಷಗಳು. ಸ್ವಾಮಿಗಳು ಅಮರನಾಥದ ಗುಹೆಯ ಬಗ್ಗೆ, ಶಿವನ ಬಗ್ಗೆ ಹಾಗೂ ಮನಸ್ಸಿನಲ್ಲುಳಿದಿದ್ದ ಮಹಾದರ್ಶನದ ಬಗ್ಗೆ ಮಾತನಾಡಿದರು.

\textbf{ಆಗಸ್ಟ್ ೮.}

ಮಾರನೆಯ ದಿನ ನಾವು ಇಸ್ಲಾಮಾಬಾದ್ ಕಡೆಗೆ ಹೊರೆಟೆವು; ಸೋಮವಾರ ಬೆಳಗ್ಗೆ ನಾವು ಬೆಳಗಿನ ತಿಂಡಿಯನ್ನು ಮುಗಿಸುತ್ತಿದ್ದಂತೆ ನಮ್ಮನ್ನೆಲ್ಲ ಕ್ಷೇಮವಾಗಿ ಶ‍್ರೀನಗರಕ್ಕೆ ಕರೆತಂದು ಸೇರಿಸಲಾಯಿತು.

