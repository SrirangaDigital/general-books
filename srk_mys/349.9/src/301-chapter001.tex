
\addtocontents{toc}{\protect\vspace{-0.7cm}}

\part{ಬರಹಗಳು (ಮೂಲ ಮತ್ತು ಅನುವಾದ)}

\chapter{ಬರಹಗಳು ಆಕಾಶ}

(ಈ ಲೇಖನವು ಅನಾಮಿಕವಾಗಿ “ನ್ಯೂಯಾರ್ಕ್ ಮೆಡಿಕಲ್ ಟೈಮ್ಸ್”ನಲ್ಲಿ ಮೊದಲು ಪ್ರಕಟವಾಯಿತು. ಈ ಪ್ರತಿಷ್ಠಿತವಾದ ಪತ್ರಿಕೆಯನ್ನು ಪ್ರಾರಂಭಿಸಿದವನು ಎಡ್ವರ್ಡ್ಗರ್ನ್ಸೆ).

ಘಟನೆಗಳನ್ನು ಅವುಗಳ ಹೋಲಿಕೆಗಳ ಆಧಾರದ ಮೇಲೆ ವರ್ಗೀಕರಿಸುವುದು ವೈಜ್ಞಾನಿಕ ಜ್ಞಾನದ ಮೊದಲ ಹೆಜ್ಜೆ - ಬಹುಶಃ ಇಷ್ಟೇ. ಘಟನೆಗಳನ್ನು ಅವುಗಳೆಲ್ಲದರಲ್ಲೂ ಅನುಸ್ಯೂತವಾಗಿರುವ ಸಮಾನ ಗುಣಗಳ ಆಧಾರದ ಮೇಲೆ ವ್ಯವಸ್ಥಿತವಾಗಿ ವರ್ಗೀಕರಿಸುವದು ಮತ್ತು ಸಮಾನ ಸನ್ನಿವೇಶಗಳಲ್ಲಿ ಈ ವರ್ಗಗಳು ಎಂದೆಂದೂ ಒಂದೇ ರೀತಿ ವರ್ತಿಸುತ್ತವೆ ಎಂಬ ನಂಬಿಕೆ ಇದನ್ನೇ ನಿಯಮ ಎಂದು ಕರೆಯುವುದು.

ಹೀಗೆ ವೈವಿಧ್ಯತೆಯಲ್ಲಿ ಏಕತೆಯನ್ನು ಕಾಣುವುದೇ ಜ್ಞಾನ. ಸಮಾನ ಲಕ್ಷಣಗಳುಳ್ಳ ಈ ಜ್ಞಾನ ಬೇರೆ ಬೇರೆ ಕಂಡಿಗಳೊಳಗೆ ಶೇಖರವಾಗಿರುತ್ತವೆ. ನಾವು ಒಂದು ಹೊಸ ವಿಷಯವನ್ನು ಪರಿಶೀಲಿಸಿದಾಗ ಆದರ ಸಮಾನತೆಯನ್ನು ನಮ್ಮ ಮನಸ್ಸಿನ ಕಂಡಿಗಳೊಳಗೆ ಹುಡುಕುತ್ತೇವೆ. ಅದಕ್ಕೆ ಸಮಾನವಾದದು ದೊರೆತರೆ ಕೂಡಲೆ ಆ ಹೊಸದನ್ನು ಸ್ವೀಕರಿಸಿ ಬಿಡುತ್ತೇವೆ. ದೊರೆಯದಿದ್ದರೆ ನಾವು ಆ ಹೊಸ ವಿಷಯವನ್ನು ತಿರಸ್ಕರಿಸುತ್ತೇವೆ, ಅಥವಾ ಅಂಥದೇ ಬೇರೆ ವಿಷಯಗಳು ದೊರೆತು ಒಂದು ಹೊಸ ವರ್ಗವನ್ನು ಪ್ರಾರಂಭಿಸುವವರೆಗೆ ಕಾಯುತ್ತೇವೆ.

ಅಸಾಧಾರಣವಾದ ಒಂದು ವಿಷಯ ನಮಗೆ ಕ್ಲೇಶವನ್ನುಂಟುಮಾಡುತ್ತದೆ; ಆದರೆ ಅಂಥದೇ ಅನೇಕ ವಿಷಯಗಳು ನಮಗೆ ದೊರೆತರೆ ಅದು ನಮಗೆ ಕ್ಲೇಶವನ್ನುಂಟು ಮಾಡುವುದಿಲ್ಲ - ಆದರೂ ಅವುಗಳ ಕಾರಣದ ಜ್ಞಾನ ಮೊದಲಿನಂತೆಯೇ ಉಳಿದಿರಬಹುದು.

ನಮ್ಮ ಜೀವನದ ಸಾಮಾನ್ಯ ಅನುಭವಗಳು ಯಾವುದೇ ಶಾಸ್ತ್ರಗಳಲ್ಲಿ ಹೇಳಿರುವ ಪವಾಡಗಳಿಗಿಂತ ಕಡಮೆಯೇನೂ ಅದ್ಭುತವಲ್ಲ; ಪವಾಡಗಳ ವಿಷಯದಲ್ಲಿಯೂ ಕೂಡ. ಆದರೆ ಪವಾಡಗಳು “ಅಸಾಧಾರಣ,” ಮತ್ತು ದಿನನಿತ್ಯದ ಅನುಭವಗಳು “ಸಾಧಾರಣ”. “ಅಸಾಧಾರಣ” ವು ನಮ್ಮ ಮನಸ್ಸನ್ನು ದಿಗ್ಭ್ರಮೆಗೊಳಿಸುತ್ತದೆ, “ಸಾಧಾರಣ” ವು ತೃಪ್ತಗೊಳಿಸುತ್ತದೆ. ಜ್ಞಾನಕ್ಷೇತ್ರವು ಅತ್ಯಂತ ವೈವಿಧ್ಯಮಯವಾದುದು ಮತ್ತು ಕೇಂದ್ರದಿಂದ ಹೆಚ್ಚು ಬೇರೆಯಾದಂತೆಲ್ಲ, ತ್ರಿಜ್ಯಗಳೂ ಪರಸ್ಪರ ದೂರವಾಗುತ್ತ ಹೋಗುತ್ತವೆ.

ಪ್ರಾರಂಭದಲ್ಲಿ ಬೇರೆ ಬೇರೆ ವಿಜ್ಞಾನಗಳಿಗೆ ಯಾವುದೇ ಸಂಬಂಧವಿಲ್ಲವೆಂದು ಪರಿಗಣಿಸ ಬರಹಗಳುಲಾಗಿತ್ತು; ಆದರೆ ಹೆಚ್ಚು ಜ್ಞಾನವು ದೊರೆತಂತೆಲ್ಲ - ಅಂದರೆ ಕೇಂದ್ರದೆಡೆಗೆ ನಾವು ಹೆಚ್ಚು ಮುಂದುವರಿದಂತೆಲ್ಲ - ತ್ರಿಜ್ಯಗಳು ಪರಸ್ಪರ ಹತ್ತಿರ ಬರುತ್ತ ಹೋಗುತ್ತವೆ ಮತ್ತು ಕೊನೆಯಲ್ಲಿ ಅವು ಒಂದು ಕೇಂದ್ರದಲ್ಲಿ ಒಂದಾಗುವಂತೆ ತೋರುತ್ತವೆ. ಅವು ಎಂದಾದರೂ ಒಂದಾಗುತ್ತವೆಯೆ?

ಗ್ರೀಕ್ ಮತ್ತು ಭಾರತದ ಋಷಿಗಳು ವಿಶೇಷವಾಗಿ ಮನಸ್ಸಿನ ಅಧ್ಯಯನದ ಕಡೆಗೆ ತಮ್ಮ ಗಮನವನ್ನು ಹರಿಸಿದರು. ಮಾನವನ ಅಂತರಂಗದ ಅಧ್ಯಯನದಿಂದಲೇ ಎಲ್ಲ ಧರ್ಮಗಳ ಉಗಮವಾಗಿದೆ. ಇಲ್ಲಿ ಏಕತೆಯನ್ನು ಅರಸುವ ಪ್ರಯತ್ನ ನಡೆದಿದೆ. ಧರ್ಮ ವಿಜ್ಞಾನವು ಅತ್ಯಂತ ವ್ಯಾಪ್ತಿಯುಳ್ಳ ಪ್ರತಿಜ್ಞೆ (ತಥ್ಯಾಂಶ)ಯನ್ನು ಆಧರಿಸುವುದರಿಂದ, ಅದರಲ್ಲಿ ಈ ಏಕತೆಯನ್ನು ಅರಸುವ ಪ್ರಯತ್ನ ಬಹಳ ಹುರುಪಿನಿಂದ ನಡೆದಿರುವುದನ್ನು ಕಾಣುತ್ತೇವೆ. ಕೆಲವು ಧರ್ಮಳು ಈ ಮೂಲ ಏಕತೆಯನ್ನು ಅರಸುವಲ್ಲಿ ದ್ವೈತವನ್ನು ಮೀರಿಹೋಗಲಿಲ್ಲ. ಈ ಮೂಲ ಯಾವುದೆಂದರೆ ಒಂದು ದೇವತೆ ಮತ್ತೊಂದು ಸೈತಾನ. ಮತ್ತೆ ಕೆಲವರು ಒಂದು ಚೇತನಾತ್ಮಕವಾದ ಕಾರಣವನ್ನು ಕಂಡು ಹಿಡಿಯುವವರೆಗೆ ಮುಂದುವರಿದರು. ಇನ್ನು ಕೆಲವರು ಇದನ್ನೂ ಮೀರಿ ಹೋದರು, ಸಗುಣ ದೇವತೆಯನ್ನೂ ಮೀರಿಹೋದರು ಮತ್ತು ಅನಂತ ಸತ್ಯವನ್ನು ಕಂಡುಹಿಡಿದರು. ಮಾನವ ಕಲ್ಪನೆಗೆ ಸೀಮಿತವಾದ ಸಗುಣ ದೇವತೆಯನ್ನೂ ಮೀರಿಹೋಗುವ ಧೈರ್ಯವನ್ನು ಪಡೆದಿರುವ ಸಿದ್ಧಾಂತಗಳಲ್ಲಿ ಮಾತ್ರ ಇಡಿ ಭೌತಿಕ ಜಗತ್ತನ್ನು ಒಂದು ಏಕತೆಯಲ್ಲಿ ಪರ್ಯವಸಾನಗೊಳಿಸುವ ಪ್ರಯತ್ನವನ್ನು ನೋಡುತ್ತೇವೆ.

ಇದರ ಪರಿಣಾಮವಾಗಿ ಹಿಂದೂಗಳು “ಆಕಾಶ” ಎಂಬ ತತ್ತ್ವವನ್ನೂ ಗ್ರೀಕರು \enginline{"ether"} ಎಂಬ ತತ್ತ್ವವನ್ನೂ ಕಂಡುಹಿಡಿದರು.

ಈ “ಆಕಾಶವು”, ಮನಸ್ಸಿಗೂ ಮುಂಚೆ ಬರುವ ಪ್ರಪ್ರಥಮ ಭೌತ ಅಭಿವ್ಯಕ್ತಿ ಎಂದು ಹಿಂದೂಗಳು ನಂಬುತ್ತಾರೆ. ಈ ಆಕಾಶದಿಂದಲೇ ಈ ಎಲ್ಲವೂ ಅಭಿವ್ಯಕ್ತಗೊಂಡವು.

ಇತಿಹಾಸವು ಪುನರಾವರ್ತಿಸುತ್ತದೆ; ಹತ್ತೊಂಬತ್ತನೆ ಶತಮಾನದ ಕೊನೆಯ ಭಾಗದಲ್ಲಿ ಈ ಹಳೆಯ ಸಿದ್ಧಾಂತವೇ ಹೆಚ್ಚು ಶಕ್ತಿಯುತವಾಗಿ, ಹೆಚ್ಚು ಬೆಳಕಿನೊಡನೆ ತಲೆಯೆತ್ತುತ್ತಿದೆ.

ವಿವಿಧ ಭೌತಿಕ ಶಕ್ತಿಗಳ ನಡುವೆ ಪರಸ್ಪರ ಸಂಬಂಧವಿರುವುದರಿಂದ ವಿವಿಧ ಜ್ಞಾನ ಶಾಖೆಗಳ ನಡುವೆಯೂ ಪರಸ್ಪರ ಸಂಬಂಧವಿದೆ ಮತ್ತು ಈ ಎಲ್ಲ ಸಾಮಾನ್ಯ ವರ್ಗಗಳ ನಡುವೆಯೂ ಜ್ಞಾನದ ಏಕತೆ ಇದೆ ಎಂಬುದು ಹಿಂದೆಂದಿಗಿಂತಲೂ ಹೆಚ್ಚು ಸ್ಪಷ್ಟವಾಗಿ ಈಗ ಪ್ರಮಾಣೀಕರಿಸಲಾಗುತ್ತಿದೆ.

ಬೆಳಕು ಬೆಳಗುವ ವಸ್ತುಗಳಿಂದ ಹೊರಟ ದ್ರವ್ಯಕಣಗಳಿಂದ ಕೂಡಿದ್ದಿದ್ದರೆ, ವಕ್ರೀ ಕರಣದ ನಿಯಮಾನುಸಾರ, ಅದು ಗಾಳಿಯ ಮೂಲಕಕ್ಕಿಂತ ದ್ರವ ಮತ್ತು ಘನ ಪದಾರ್ಥಗಳ ಮೂಲಕ ಹೆಚ್ಚು ವೇಗವಾಗಿ ಚಲಿಸಬೇಕಿತ್ತು ಎಂಬುದನ್ನು ನ್ಯೂಟನ್ ತೋರಿಸಿದನು.

ಬೆಳಕು ಸ್ಥಿತಿಸ್ಥಾಪಕ ಮಾಧ್ಯಮದ ಮೂಲಕ ಅಲೆಯೋಪಾದಿಯಲ್ಲಿ ಚಲಿಸುತ್ತದೆ ಎಂದು ಭಾವಿಸಿದರೆ, ಮೇಲಿನ ನಿಯಮವು ಸತ್ಯವಾಗಬೇಕಾದರೆ, ಬೆಳಕು ಅನಿಲದ ಮೂಲಕಕ್ಕಿಂತ ಘನ ಮತ್ತು ದ್ರವಗಳ ಮೂಲಕ ಹೆಚ್ಚು ನಿಧಾನವಾಗಿ ಚಲಿಸಬೇಕು - ಎಂದು ಹ್ಯುಜೆನ್ಸ್ \enginline{(Huyghens)} ತೋರಿಸಿದ್ದಾನೆ. ಫಿಜಾ \enginline{(Fizeau)} ಮತ್ತು ಫೋಕಾಲ್ಟ್ \enginline{(Foucault)} ಈ ಮೇಲಿನ ಹೇಳಿಕೆಯನ್ನು ಪ್ರಯೋಗಾತ್ಮಕವಾಗಿ ಸಾಬೀತು ಪಡಿಸಿದ್ದಾರೆ.

ಆದ್ದರಿಂದ ಬೆಳಕು ಕಂಪನಾತ್ಮಕ ಮಾಧ್ಯಮವನ್ನು ಹೊಂದಿರಬೇಕು ಮತ್ತು ಆ ಮಾಧ್ಯಮವು ಸರ್ವವ್ಯಾಪಿಯಾಗಿರಬೇಕು. ಈ ಮಾಧ್ಯಮವನ್ನೇ \enginline{"ether"} (ಆಕಾಶ) ಎಂದು ಕರೆಯುವುದು.

ವಿಶ್ವ ಆಕಾಶ ಸಿದ್ಧಾಂತವು ಬೆಳಕಿನ ವಿಕಿರಣ, ವಕ್ರೀಕರಣ, ಏಕೀಕರಣ - ಈ ಕ್ರಿಯೆಗಳನ್ನು ವಿವರಿಸುತ್ತದೆ ಎಂಬ ಅಂಶವು ಮೇಲಿನ ಸಿದ್ಧಾಂತವನ್ನು ಹೆಚ್ಚು ಬಲಪಡಿಸುತ್ತದೆ.

ಹೀಗೆ ಇತ್ತೀಚೆಗೆ ಗುರುತ್ವಾಕರ್ಷಣೆ, ಅಣುವಿಕ್ರಿಯೆ, ಕಾಂತಕ್ರಿಯೆ ಮತ್ತು ವಿದ್ಯುತ್ ಕ್ರಿಯಾಕರ್ಷಣೆ ಮತ್ತು ವಿಕರ್ಷಣೆ ಇವೂ ಮೇಲಿನ ಸಿದ್ಧಾಂತಾನುಸಾರ ವಿವರಿಸಲ್ಪಟ್ಟಿವೆ.

ಸಂವೇದ್ಯ ಮತ್ತು ಸುಪ್ತವಾದ ಉಷ್ಣ, ವಿದ್ಯುಚ್ಛ ಕ್ತಿ ಮತ್ತು ಕಾಂತಶಕ್ತಿ ಇವೆಲ್ಲವೂ ಇತ್ತೀಚೆಗೆ ಸರ್ವವ್ಯಾಪಿ ಆಕಾಶ - ಸಿದ್ಧಾಂತದ ಮೂಲಕ ತೃಪ್ತಿಕರವಾಗಿ ವಿವರಿಸಲ್ಪಟ್ಟಿವೆ.

ವಿಲ್ಹೇಮ್​ ವೇಬರ್ನ ಸಂಶೋಧನೆಗಳನ್ನು ಆಧರಿಸಿದ ತನ್ನ ಗುಣಿಕೆಯ ಆಧಾರದ ಮೇಲೆ ಜೊಲ್ನರ್ \enginline{(zollner)} ನು ಹೀಗೆ ಭಾವಿಸುತ್ತಾನೆ: ಆಕಾಶಕಾಯಗಳ ನಡುವೆ ನಡೆಯುವ ಪ್ರಾಣಶಕ್ತಿಯ ವಿನಿಮಯ ಎರಡು ರೀತಿಯ ಪ್ರಭಾವಕ್ಕೆ ಒಳಪಡುತ್ತದೆ - ಅವಾವುವೆಂದರೆ ತರಂಗಾಯಿತ ಮಾಧ್ಯಮ ಮತ್ತು ಪ್ರತ್ಯಕ್ಷ ಕಣಗಳು.

ವೇಬರ್ ಅಣುಗಳು \enginline{(moleculs)} ಅವಕ್ಕಿಂತಲೂ ಸಣ್ಣ ಕಣಗಳಿಂದ ಕೂಡಿವೆ ಎಂಬುದನ್ನು ಕಂಡುಹಿಡಿದನು. ಆ ಕಣಗಳನ್ನು ವಿದ್ಯುತ್ಕಣಗಳು ಎಂದು ಅವನು ಕರೆದನು. ಅವು ಅಣುಗಳಲ್ಲಿ ನಿರಂತರ ವೃತ್ತಾಕಾರದಲ್ಲಿ ಚಲಿಸುತ್ತವೆ. ಈ ವಿದ್ಯುತ್ ಕರಣಗಳು ಆಂಶಿಕವಾಗಿ ಧನವಾಗಿಯೂ ಆಂಶಿಕವಾಗಿ ಋಣವಾಗಿಯೂ ಇವೆ.

ಸಮಾನ ವಿದ್ಯುತ್ಕಣಗಳು ಪರಸ್ಪರ ಅಪಕರ್ಷಿಸುತ್ತವೆ ಮತ್ತು ಭೇದವುಳ್ಳ ಕಣಗಳು ಪರಸ್ಪರ ಆಕರ್ಷಿಸುತ್ತವೆ. ಪ್ರತಿಯೊಂದು ಅಣುವೂ ಸಮಾನ ಪ್ರಮಾಣದ ವಿದ್ಯುತ್ಕಣಗಳನ್ನು ಹೊಂದಿರುತ್ತವೆ, ಅದಲ್ಲದೆ ಮಿಗುತಾಯವಾದ ಧನ ಮತ್ತು ಋಣ ಕಣಗಳನ್ನೂ ಹೊಂದಿದ್ದು ಇವು ಸಮತೋಲವನ್ನು ಕಾಪಾಡುತ್ತವೆ.

ಇದರ ಆಧಾರದ ಮೇಲೆ ಜೊಲ್ನರ್ ಕೆಳಗಿನ ಹೇಳಿಕೆಗಳನ್ನು ಮಾಡುತ್ತಾನೆ:

(೧) ಅಣುಗಳು ಅಧಿಕ ಸಂಖ್ಯೆ ವಿದ್ಯುತ್ ಕಣಗಳಿಂದ ಕೂಡಿವೆ ಮತ್ತು ಇವು ಅಣು ವಿನಲ್ಲಿ ನಿರಂತರಸುತ್ತುತ್ತಿರುತ್ತವೆ.

(೨) ಅಣುವಿನ ಆಂತರಿಕ ಚಲನೆಯು ಒಂದು ಮಿತಿಯನ್ನು ಮೀರಿ ವೃದ್ಧಿಯಾದರೆ, ಆಗ ವಿದ್ಯುತ್ ಕಣಗಳು ಹೊರಗೆ ಸಿಡಿದು ಬರುತ್ತವೆ. ಅವು ಅನಂತರ ಒಂದು ಆಕಾಶಕಾಯದಿಂದ ದೇಶದ \enginline{(Space)} ಮೂಲಕ ಚಲಿಸುತ್ತ ಇನ್ನೊಂದು ಆಕಾಶಕಾಯವನ್ನು ತಲುಪುತ್ತವೆ. ಅವು ಅಲ್ಲಿ ಪ್ರತಿಬಿಂಬಿಸಲ್ಪಡುತ್ತವೆ ಇಲ್ಲವೇ ಹೀರಿಕೊಳ್ಳಲ್ಪಡುತ್ತವೆ.

(೩) ವಿದ್ಯುತ್ ಕಣಗಳು ಯಾವ ಮಾಧ್ಯಮದ ಮೂಲಕ ಚಲಿಸುತ್ತವೆಯೊ ಅದನ್ನೇ ಭೌತವಿಜ್ಞಾನಿಗಳು \enginline{"ether"} (ಆಕಾಶ) ಎಂದು ಕರೆಯುವುದು.

(೪) ಈ ಆಕಾಶ ಕಣಗಳು ಎರಡು ಬಗೆಯ ಚಲನೆಗಳನ್ನು ಹೊಂದಿವೆ: (೧) ಅವುಗಳ ಸ್ವಂತ ಚಲನೆ; (೨) ತರಂಗಾತ್ಮಕ ಚಲನೆ. ಈ ಎರಡನೆಯ ಚಲನೆಗೆ ಅಣು ವಿನಲ್ಲಿರುವ ಆಕಾಶಕಣಗಳಿಂದ ಪ್ರೇರಣೆಯು ದೊರೆಯುತ್ತದೆ.

(೫) ಈ ಅತಿ ಚಿಕ್ಕ ಕಣಗಳ ಚಲನೆಯು ಆಕಾಶಕಾಯಗಳ ಚಲನೆಗೆ ಸಂವೇದಿಯಾಗಿರುತ್ತದೆ.

ಇದರ ತಾತ್ಪರ್ಯ: ಆಕಾಶಕಾಯಗಳ ನಡುವಿನ ಗುರುತ್ವಾಕರ್ಷಣ ನಿಯಮವು ಅತಿ ಚಿಕ್ಕ ಕಣಗಳಿಗೂ ಅನ್ವಯವಾಗುತ್ತದೆ.

ಈ ಮೇಲಿನ ಊಹೆಯ ಪ್ರಕಾರ ನಾವು ಯಾವುದನ್ನು ದೇಶವೆಂದು \enginline{(space)} ಕರೆಯುತ್ತೇವೆಯೊ ಅದು ನಿಜವಾಗಿಯೂ ವಿದ್ಯುತ್ ಕಣಗಳಿಂದ ಅಥವಾ ಆಕಾಶದಿಂದ ತುಂಬಿದೆ.

ಜೊಲ್ನರ್ನು ವಿದ್ಯುತ್ ಕಣಗಳ ವಿಷಯದಲ್ಲಿ ಈ ಕೆಳಗಿನ ಆಸಕ್ತಿಪೂರ್ಣ ಅಂಶಗಳನ್ನು ಕಂಡುಹಿಡಿದಿದ್ದಾನೆ:

ವೇಗ: ಒಂದು ಸೆಕೆಂಡಿಗೆ ೫೦೧೪೩ ಮೈಲಿಗಳು.

೪೨೦೦೦, ಮಿಲಿಯನ್ ಆಕಾಶಕಣಗಳು ಒಂದು ನೀರಿನ ಅಣುವಿನಲ್ಲಿರುತ್ತವೆ!

ಒಂದೊಂದು ಕಣಗಳ ನಡುವಿನ ಅಂತರ ೦.೦೦೩೨ ಮಿಲಿಮೀಟರ್.

ನಾವೀಗ ತಿಳಿದಿರುವಂತೆ, ಪ್ರಕೃತಿಯ ಬೇರೆ ಬೇರೆ ಘಟನೆಗಳನ್ನು ವಿವರಿಸಲು ಆಕಾಶ ಸಿದ್ಧಾಂತವು ಅತ್ಯುತ್ತಮವಾದುದು.

ಈ ಆಕಾಶವು ವಿದ್ಯುತ್ ಅಥವಾ ಬೇರೆ ಯಾವುದಾದರೂ ಕಣಗಳಿಂದ ಕೂಡಿದೆ ಎಂಬ ವಿಷಯವೂ ಅತ್ಯಂತ ಮೌಲ್ಯವುಳ್ಳದ್ದು. ಆದರೆ ಈ ಕಣಗಳು ಎಷ್ಟೇ ಚಿಕ್ಕವಾಗಿ ದ್ದರೂ ಅವುಗಳ ನಡುವೆ ಅಂತರವಿರಲೇಬೇಕು ಮತ್ತು ಈ ಅಂತರವನ್ನು ಯಾವುದು ತುಂಬುತ್ತದೆ ಎಂಬುದು ಪ್ರಶ್ನೆ. ಇನ್ನೂ ಸೂಕ್ಷ್ಮ ಕಣಗಳಿಂದ ತುಂಬಿವೆ ಎಂದರೆ ಅವುಗಳ ಅಂತರವನ್ನು ತುಂಬುವ ಅವಕ್ಕಿಂತಲೂ ಸೂಕ್ಷ್ಮವಾದ ಕಣಗಳನ್ನು ಊಹಿಸಬೇಕಾಗುತ್ತದೆ. ಹೀಗೆಯೇ ಇದು ಮುಂದುವರಿಯುತ್ತದೆ.

ದೇಶದಲ್ಲಾಗುವ ಘಟನೆಗಳನ್ನು ವಿವರಿಸುವುದಾದರೂ, ಈ ಆಕಾಶ ಸಿದ್ಧಾಂತವು ಅಥವಾ ಆಕಾಶವು ಕಣಗಳಿಂದ ಕೂಡಿದೆ ಎಂಬ ಸಿದ್ಧಾಂತವು ದೇಶದ ಸ್ವರೂಪವನ್ನೇ ವಿವರಿಸುವುದಿಲ್ಲ.

ಅಣುಗಳನ್ನು ಒಳಕೊಳ್ಳುವ ಆಕಾಶವು ಅಣುಗಳ ಚಲನೆಯನ್ನು ವಿವರಿಸುವು ದಾದರೂ ಅದು ತಾನೇ ದೇಶವನ್ನು ವಿವರಿಸಲಾರದು. ಏಕೆಂದರೆ ಆಕಾಶವು ದೇಶದಲ್ಲಿದೆ ಎಂದು ಮಾತ್ರ ನಾವು ಭಾವಿಸಲು ಸಾಧ್ಯ. ಆದ್ದರಿಂದ ಈ ದೇಶವನ್ನು ವಿವರಿಸುವುದು ಯಾವುದಾದರೂ ಇದ್ದರೆ ಅದು ಈ ಅನಂತ ದೇಶವನ್ನೂ ತನ್ನ ಪರಿಧಿಯೊಳಗೆ ಸೇರಿಸಿಕೊಳ್ಳುವಂತಿರಬೇಕು. ಅನಂತ ಮನಸ್ಸಲ್ಲದೆ ಇನ್ನಾವುದು ತಾನೆ ಈ ಅನಂತ ದೇಶವನ್ನು ಒಳಕೊಳ್ಳಬಲ್ಲದು?

