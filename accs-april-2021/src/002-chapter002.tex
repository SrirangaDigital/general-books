\chapter{ON THE PROBABILITY DISTRIBUTIONS\\ INDUCED BY INTEGER SEQUENCES\\ GENERATED BY RECURRENCE RELATIONS\\ WITH POSITIVE INTEGER COEFFICIENTS}

%~ \centerline{{\LARGE\sl Understanding IP Routing}}
\vskip 0.8cm

\begin{center}
{\large\uppercase{$\text{Arulalan Rajan, R. Vittal Rao, H. S. Jamadagni}$}} 

\vskip -6pt

Department of Electronic Systems Engineering,\\ Indian Institute of Science,\\ Bengaluru - 560012, India 

\end{center}


\begin{center}
{\large\uppercase{$\text{Ashok Rao}$}} 

\vskip -6pt

Consultant, 165, 11th main,\\ Saraswathipuram,\\ Mysore, India

\end{center}



\vfill




\newpage

\begin{multicols}{2}

\section*{Abstract}

The classical Fibonacci sequence is known to exhibit many fascinating properties. In this paper, we explore the Fibonacci sequence and integer sequences generated by second order linear recurrence relations with positive integer coeffcients from the point of view of probability distributions that they induce. We obtain the generalizations of some of the known limiting properties of these probability distributions and present certain optimal properties of the classical Fibonacci sequence in this context. In addition, we also look at the self linear convolution of linear recurrence relations with positive integer coefficients. Analysis of self linear convolution is focused towards locating the maximum in the resulting sequence. This analysis, also highlights the in uence that the largest positive real root, of the ``characteristic equation" of the linear recurrence relations with positive integer coefficients, has on the location of the maximum. In particular, when the largest positive real root is 2, the location of the maximum is shown to depend on the sequence length being odd or even.

\textbf{Key words.} Fibonacci sequence, recurrence relations, probability distribution, limiting properties, linear convolution

\textbf{AMS subject classifications.} 11B37, 11B39, 60C05, 60C09, 60F99

\section{Introduction}\label{section-1}

Starting with any sequence $\{f[n]\}_{n\in \mathbb N}$ of positive real numbers, we can generate a probability distribution on $\{1; 2; : : : ;N\}$ for every $N \in \mathbb N$ as follows: Let $X$ be a random variable taking values $1; 2; : : : ;N$ such that

$$
P(X=n) = \displaystyle{\frac{f[n]}{\displaystyle{\sum_{k=1}^N}f[k]}}
$$

In particular, if $f[n]$ are all positive integers, then we can generate such probability distributions using only integers. In a recent work, Neal \cite{art2-key01} has investigated such probability distributions arising out of the Fibonacci sequence and obtained some limiting properties of these distributions as $N \rightarrow \infty$.

While exploring integer sequences from the point of view of generating window functions in the context of designing digital filters for signal processing applications \cite{art2-key02}, we found some generalizations of Neal's limiting properties for a class of integer sequences and an optimal property of the Fibonacci sequence in this class. In this paper, we present these results.

\section{Probability Distributions induced by Fibonacci Sequence}\label{section-2}

The classical Fibonacci sequence \cite{art2-key03}, \cite{art2-key04}, is the positive integer sequence $f[n]$ defined by the second order recurrence relation

\numberwithin{equation}{section}

  \begin{equation}
  f[n] = f[n-1] + f[n-2]\label{eq-2.1}
    \end{equation}

with initial conditions $f[1] = 1$, $f[2] = 1$.
The ratio of successive terms of the Fibonacci sequence converges to the well known golden ratio $\varphi$, i.e.,
\begin{equation}
\lim_{k \rightarrow \infty} \frac{f[k+1]}{f[k]} = \varphi = \displaystyle{\frac{1 + \sqrt5}{2}}\label{eq-2.2}
\end{equation}

For any positive integer $N$, consider the Fibonacci sequence, $f[1], f[2], \ldots, f[N]$, of length $N$. Using the Fibonacci sequence, we can define a discrete probability distribution as follows. We define the probability mass function as\footnote{${\bar{(.)}}$: associated with increasing sequence}
$$
p_{_{\bar{N}}}[1], p_{_{\bar{N}}}[2], \ldots, p_{_{\bar{N}}}[N]
$$
\begin{equation}
 p_{_{\bar{N}}}(n) = \frac{f[n]}{\displaystyle{\sum_{k =1}^N f[k]}} \hspace{0.3in} \forall n = 1, 2, \ldots, N \label{eq-2.3}
 \end{equation}

Let $\bar{X}$ be a random variable in $\{1,2, \ldots, N\}$ such that  
 
\begin{equation}
P(\bar{X} = n) = p_{_{\bar{N}}}(n); \hspace{0.3in} 1 \leq n\leq N.\label{eq-2.4}
\end{equation}
 
In his work, Neal \cite{art2-key01} showed that, $E(\bar{X})$, the mean or expectation of $\bar{X}$, grows linearly as $N$ increases and hence 
\begin{equation}
\lim_{N\rightarrow \infty} E(\bar{X}) = \infty\label{eq-2.5}
\end{equation} 

Similarly, let $\underline{X}$ be a random variable in $\{1,2, \ldots, N\}$, where the probabilities, $p_{_{\underline{N}}}[n]$, correspond to the Fibonacci sequence, of length $N$, in the decreasing order \footnote{${(\underline{.})}$ : associated with decreasing sequence}. We define the probability mass function,
\begin{equation}
P(\underline{X} = n) = \frac{f[N+1-n]}{\displaystyle{\sum_{k =1}^N f[k]}} = p_{_{\underline{N}}}(n) \hspace{0.2in} ; \hspace{0.2in} 1 \leq n\leq N.\label{eq-2.6}
 \end{equation}

For such a random variable, $\underline{X}$, Neal \cite{neal} observed the following limiting properties:
\begin{equation}
\lim_{N\rightarrow \infty} E(\underline{X}) = \varphi + 1\label{eq-2.7}
\end{equation}
\begin{equation}
\lim_{N\rightarrow \infty} Var(\underline{X}) = 2\varphi + 1\label{eq-2.8}
\end{equation}
where $E(\underline{X})$ denotes the mean or expectation of $\underline{X}$ and $Var(\underline{X})$ denotes the variance of $\underline{X}$.

The aim of this paper, as mentioned earlier, is to show that such limiting properties are characteristic for integer sequences defined by linear recurrence relations , of any order, with positive integer coefficients. For the sake of simplicity, we shall first discuss the second order recurrence relation. 
 
\section{Second Order Recurrence Relations}\label{section-3} 

Let $a$, $b$ be any two positive integers and consider the integer sequence defined by the second order recurrence relation
\begin{equation}
f[n] = a f[n-1] + b f[n-2] \hspace{0.1in}  \textup{where} \hspace{0.1in}  f[1] = 1; f[2]= 1;\label{eq-3.1}
\end{equation}
 
The characteristic equation of this recurrence relation is given by
\begin{equation}
r^2 - ar - b = 0.\label{eq-3.2}
\end{equation}
It is well known that the solution to this Eq.~\eqref{eq-3.2} is given by
\begin{equation*}
f[n] =  C[R(a,b)^n - R_s(a,b)^n]
\end{equation*}
where, 
\begin{equation*}
R(a,b) = \displaystyle{\frac{a +\sqrt{a^2 + 4b}}{2}}
\end{equation*}
is the positive real root. The other root of the equation is 
\begin{equation*}
R_s = - \frac{b}{R(a,b)}
\end{equation*}
and 
\begin{equation*}
C = \displaystyle{\frac{1}{\sqrt{a^2 + 4b}}}
\end{equation*}
We note that,

\begin{eqnarray}
R(a,b) &\geq& \varphi \hspace{0.1in} \textup{for all positive integers} \hspace{0.1in} a, b\\\label{eq-3.3}
R(a,b) &=& \varphi   \hspace{0.1in}\textup{when}\hspace{0.1in} a = b = 1\\\label{eq-3.4}
       & &\textup{(classical Fibonacci sequence)}\nonumber
\end{eqnarray}


\vspace{-.6cm}


We further observe that the ratio of successive terms of this sequence converges to $R(a,b)$, i.e.,

\begin{equation}
\lim_{k \rightarrow \infty} \frac{f[k+1]}{f[k]} = R(a,b)\label{eq-3.5}
\end{equation}
As in section \ref{???}, for any positive integer $N$, consider the above sequence, of length $N$, i.e.,
\begin{center}
$f[1],f[2], \ldots, f[N]$
\end{center} 
and define probability mass functions,
$$
p_{_{\bar{N}}}(n)[1],p_{_{\bar{N}}}(n)[2], \ldots, p_{_{\bar{N}}}(n)[N]
$$
as
\begin{equation}
 p_{_{\bar{N}}}(n) = \frac{f[n]}{\displaystyle{\sum_{k =1}^N f[k]}} \hspace{0.3in} \forall n = 1, 2, \ldots, N\label{eq-3.6}
\end{equation}
Consider a random integer, $\bar{X}$, in $\{1,2, \ldots, N\}$, such that
\begin{equation}
P(\bar{X} = n) = p_{_{\bar{N}}}(n)\label{eq-3.7}
\end{equation} 
Similarly let $\underline{X}$ be a random variable in $\{1,2, \ldots, N\}$, (with probabilities corresponding to the decreasing sequence) such that, 
\begin{equation}
P(\underline{X} = n) = \frac{f[N+1-n]}{\displaystyle{\sum_{k =1}^N f[k]}} = p_{_{\underline{N}}}(n) \hspace{0.3in} \forall n = 1, 2, \ldots, N\label{eq-3.8}
\end{equation} 
 In the rest of the paper, for short, we shall denote $R(a,b)$ by $R$ and $R_S(a,b)$ by $R_S$.
 
\section{The Main Theorems on the Limiting Properties}\label{section-4} 

We prove the following limiting properties.

Analogous to eq.(\ref{??}), we have,

\begin{thm}
 \label{thm-1}
\begin{equation*}
\lim_{N \rightarrow \infty} E(\bar{X}) = \infty
\end{equation*}
\end{thm}
Even though $E(\bar{X})$ grows as $N$ increases, its variance still converges. In fact, we have,\\ 
 \begin{thm}
 \label{thm-2}
\begin{equation*}
\lim_{N \rightarrow \infty} Var(\bar{X}) = \frac{R}{(R-1)^2}
\end{equation*}
 \end{thm}
In the case of $\underline{X} $, both $E(\underline{X})$ and $Var(\underline{X})$ converge as $N \rightarrow \infty$ and the limit of $Var(\underline{X})$ is the same as that of $Var(\bar{X})$. We have, analogous to eq.(\ref{MeanY}) and eq.(\ref{VarianceY}), the following limits.
\begin{thm}
 \label{thm-3}
\begin{equation}\nonumber
\lim_{N \rightarrow \infty} E(\underline{X}) = \frac{R}{(R-1)}
\end{equation}
 \end{thm}
 \begin{thm}
 \label{thm-4}
\begin{equation}\nonumber
\lim_{N \rightarrow \infty} Var(\underline{X}) = \frac{R}{(R-1)^2}
\end{equation}
 \end{thm}
We, then, easily observe using these properties that when $a = b = 1$, the limits of eq.(\ref{MeanNX}), eq.(\ref{MeanNY}) and eq.(\ref{VarianceNY}) reduce to the limits, eq.(\ref{MeanX}), eq.(\ref{MeanY}), eq.(\ref{VarianceY}), obtained for the Fibonacci sequence, by Neal \cite{art2-key01}, described in section \ref{FibonacciSequence}. Using eq.(\ref{opt1}) and eq.(\ref{opt2}), we can prove the following optimality properties of the Fibonacci sequence:
\begin{thm}
 \label{thm-5}
\begin{eqnarray*}
\label{MeanMax}
\max{\left(\lim_{N\rightarrow \infty} E(\underline{X}): a, b \in \mathbb N \right)} &=& \lim_{N \rightarrow \infty}\left(E(\underline{X}):a=b=1\right)\\
 &=& \varphi + 1
\end{eqnarray*}
\end{thm}
\begin{thm}
\label{thm-6}
{\fontsize{7.4}{8.4}\selectfont\begin{eqnarray*}
\label{VarianceMax}
\max{\left (\lim_{N\rightarrow \infty}Var(\bar{X}): a,b \in \mathbb N\right)} &=&\max{\left(\lim_{N\rightarrow \infty}Var(\underline{X}): a,b \in \mathbb N\right)}\\
&=& \lim_{N \rightarrow \infty}\left(Var(\underline{X}):a=b=1\right)= 2\varphi + 1
\end{eqnarray*}}

 \end{thm}

We now proceed to prove these theorems.

\section{Some notations used in the following sections}\label{section-5}
In order to prove the aforesaid theorems, we make use of certain finite sums, presented in \textbf{Appendix.\ref{Prelims}} and \textbf{Appendix.\ref{FiniteSums}}. We list here, some of the notations used in the subsequent sections.
%~ \begin{center}
\begin{tabular}{cccc}
  Increasing sequence & ~ &~& Decreasing Sequence\\[.2cm]
 $\bar{S} = \displaystyle{\sum_{n=1}^{N}}f[n]$& ~ &~& $\underline{S} = \displaystyle{\sum_{n=1}^{N}}f[N+1-n]$ \\
 $\bar{S}_1 = \displaystyle{\sum_{n=1}^{N}}n f[n]$ & ~ &~& $\underline{S}_1 = \displaystyle{\sum_{n=1}^{N}}n f[N+1-n]$ \\
 $\bar{S}_2 = \displaystyle{\sum_{n=1}^{N}}n^2 f[n]$ &~& ~&$\underline{S}_2 = \displaystyle{\sum_{n=1}^{N}}n^2 f[N+1-n]$ 
\end{tabular}
%~ \end{center}

\section{Proofs of Theorems related to distributions induced by increasing sequences}\label{section-6}

We look at the mean and variance of the random variable $\bar{X}$, defined as in section \ref{FibonacciSequence}.

\textsc{Theorem {\ref{th:MeanLinear}}:}
\begin{eqnarray*}
\displaystyle{\lim_{N \rightarrow \infty}} E(\bar{X}) = \infty  
 \end{eqnarray*}
\begin{proof}
We have,
\begin{equation}
E(\bar{X}) = \frac{\bar{S}_1}{\bar{S}}\label{eq-6.1}
\end{equation}
Using eq.(\ref{nfn1}) and eq.(\ref{SumIncrease}), we get
\begin{eqnarray}
E(\bar{X})&\sim& \frac{(NR^{N+2}-(N+1)R^{N+1}+R)(R-1)}{(R-1)^2 R^{N+1}}\nonumber\\
&\sim& \frac{NR-N-1}{R-1}\nonumber\\
\Rightarrow E(\bar{X}) &\sim& N - \frac{1}{R-1}\\\label{eq-6.2}
\nonumber E(\bar{X}) &\sim& \mathcal{O}(N)
\end{eqnarray}
\end{proof}

Theorem \ref{th:MeanLinear}, now follows from eq.(\ref{expx}).

We also observe the following: 

 If we now define $\bar Y$ = $(N+1)- \bar{X}$, so that $\bar{Y}$ is also a random variable taking value 1, 2, 3, $\ldots$, $N$ and $\bar{X}$ has been in a sense centralized, then by eq.(\ref{expx}),
 \begin{equation*}
 E(\bar{Y}) \sim 1 + \frac{1}{R-1} = \frac{R}{R-1}
 \end{equation*}
Hence 
\begin{equation*}
\displaystyle{\lim_{N \rightarrow \infty}} E(\bar{Y}) = \frac{R}{R-1} 
\end{equation*}
This $\bar{Y}$ corresponds to $\underline{X}$, which we describe in sec. \ref{Decreasing}.

\textsc{Theorem {\ref{th:VarianceConstantX}}:}

\begin{eqnarray*}
\lim_{N \rightarrow \infty} Var(\bar{X}) = \frac{R}{(R -1)^2}
\end{eqnarray*}

\begin{proof}\renewcommand{\qedsymbol}{}
We have,
\begin{equation}
E(\bar{X}^2) = \frac{\bar{S}_2}{\bar{S}}\label{eq-6.3}
\end{equation}

Using eq.(\ref{s2}) and eq.(\ref{SumIncrease}), we get
\begin{equation}
E(\bar{X}^2) \sim \frac{N^2(R-1)^2 - 2N(R-1) + R+1}{(R-1)^2}\label{eq-6.4}
\end{equation}

Using eq.(\ref{expx}) and eq.(\ref{ex2}), we get

{\fontsize{6.8}{7.8}\selectfont\begin{eqnarray}
Var(\bar{X})&=& E(\bar{X}^2)-E(\bar{X})^2\nonumber \\
&\sim &\frac{N^2(R-1)^2 - 2N(R-1) + R+1}{(R-1)^2}-\left(N - \frac{1}{R-1}\right)^2\nonumber\\
\Rightarrow \lim_{N \rightarrow \infty}Var(\bar{X})& = & \frac{(R+1)-1}{(R-1)^2}\nonumber
\end{eqnarray}}

Thus, we get the limit of the variance as,
\begin{equation}
\lim_{N \rightarrow \infty}Var(\bar{X}) = \frac{R}{(R-1)^2}\label{eq-6.5}
\end{equation}
\end{proof}

It is interesting to note from eq.(\ref{expx}) and eq.(\ref{varx}) that, even though the mean increases linearly as the length of the sequence, the variance converges, to a function of $R$.

\section{Proofs of Theorems related to distributions induced by decreasing sequences}\label{section-7}

In this section, we look at the probability distribution generated by reversing the sequence values. We recall that the probability of the random variable $\underline{X}$ taking a value $n$ was defined as, 
\begin{equation}
P(\underline{X} = n) =\frac{f[N+1-n]}{\displaystyle{\sum_{k=1}^{N}f[N+1-k]}} = p_{_{\underline{N}}}(n)\label{eq-7.1}
\end{equation}

\textsc{Theorem {\ref{th:MeanConst}}:}
\begin{eqnarray*}
\displaystyle{\lim_{N \rightarrow \infty}}E(\underline{X}) = 1 + \displaystyle{\frac{1}{R-1}} =  \displaystyle{\frac{R}{R-1}}
\end{eqnarray*}

\begin{proof}\renewcommand{\qedsymbol}{}
We have,
\begin{equation}
E(\underline{X}) = \frac{\underline{S}_1}{\underline{S}}
\end{equation}
From eq.(\ref{nfnd}), we get
\begin{eqnarray}
E(\underline{X}) &=& \frac{(N+1)\bar{S} - \bar{S}_1}{\bar{S}}\nonumber\\
&=& (N+1) - \frac{\bar{S}_1}{\bar{S}}\nonumber\\
&\sim& (N+1) - (N - \frac{1}{R-1}) \hspace{0.1in}\textup{by eq.(\ref{expx}})\nonumber\\
\label{meand}\Rightarrow  \lim_{N\rightarrow \infty}E(\underline{X}) &=& 1 + \displaystyle{\frac{1}{R-1}} = \displaystyle{\frac{R}{R-1}}
\end{eqnarray}
\end{proof}

\textsc{Theorem {\ref{th:VarianceConstantY}}:}
\begin{eqnarray*}
\displaystyle{\lim_{N \rightarrow \infty}}E(\underline{X}^2) = 1 + \displaystyle{\frac{3R-1}{(R-1)^2}}
\end{eqnarray*}
\begin{proof}\renewcommand{\qedsymbol}{}
\begin{equation}
Var(\underline{X}) = E(\underline{X}^2) - E(\underline{X})^2\label{eq-7.4}
\end{equation}
Using eq.(\ref{n2fnd}) we first obtain $E(\underline{X}^2)$.
{\fontsize{6}{7}\selectfont
\begin{eqnarray}
E(\underline{X}^2) &=& \frac{\bar{S}_2}{\underline{S}} \nonumber\\
&\sim& \frac{(N+1)^2\bar{S}}{\bar{S}} - \frac{2(N+1)\bar{S}_1}{\bar{S}} + \frac{\bar{S}_2}{\bar{S}}\nonumber\\
&\sim& (N+1)^2 - 2(N+1)E(\bar{X}) +E(\bar{X}^2)\nonumber\\
&\sim& (N+1)^2 - 2(N+1)\left(N-\frac{1}{R-1}\right) + \left(\frac{N^2(R-1)^2 - 2N(R-1)+(R+1)}{(R-1)^2}\right)\nonumber\\
&\sim& \frac{(R-1)^2 +3R -1}{(R-1)^2}\nonumber
\end{eqnarray}}
\normalsize
\begin{equation}
\Rightarrow \lim_{N \rightarrow \infty}E(\underline{X}^2) = 1 + \displaystyle{\frac{3R-1}{(R-1)^2}}\label{eq-7.5}
\end{equation} 
From eq.(\ref{ey2}) and eq.(\ref{meand}), we get 
\begin{eqnarray}
Var(\underline{X}) &=& E(\underline{X}^2) - E(\underline{X})^2\nonumber\\
&\rightarrow& 1 + \frac{3R-1}{(R-1)^2} - \frac{R^2}{(R-1)^2}\nonumber\\ 
\Rightarrow \lim_{N \rightarrow \infty}Var(\underline{X}) &=& \frac{R}{(R-1)^2}\label{eq-7.6}
\end{eqnarray}
\end{proof}

Thus, we find that, the limit of the variance of the probability distribution with probability defined by eq.(\ref{prob}) to be a simple function $R$, the limit of the ratio of successive elements.

\section{Mean and Variance of discrete probability distributions induced by Fibonacci Sequence}\label{section-8}

Now, we look at the probability distribution induced by the standard Fibonacci sequence. For such a distribution, we obtain the value of mean and variance,using eq.(\ref{expx}) and eq.(\ref{ey2}) respectively, by using $a = 1 $, $b = 1$, and $R = \varphi$.

\subsection{Distributions induced by Increasing Sequence}\label{subsection-8.1}

By eq.(\ref{expx}) and eq.(\ref{varx}), we have 
$$
\nonumber E(\bar{X})  \sim \mathcal{O}(N)
$$
and 
\begin{eqnarray}
\lim_{N \rightarrow \infty}Var(\bar{X}) &=& \frac{R}{(R-1)^2} = \frac{\varphi}{(\varphi-1)^2}\nonumber\\
\Rightarrow \lim_{N \rightarrow \infty}Var(\bar{X}) &=& 2\varphi+1\label{eq-8.1}
\end{eqnarray}

Neal \cite{art2-key01} looks at the mean and variance of a Fibonacci distribution. It may be pointed out that the fact that the variance converges in this case was not observed by Neal. The above derivations show that, in this case, though the mean increases linearly as the length of the sequence, the variance converges to a simple function of the golden ratio $\varphi$.

\subsection{Distributions induced by Decreasing Sequence}\label{subsection-8.2}
By eq.(\ref{meand}), we have
\begin{equation}
\nonumber \displaystyle{\lim_{N \rightarrow \infty}}E(\underline{X}) =  \displaystyle{\frac{R}{R-1}} = \varphi + 1
 \end{equation}

From eq.(\ref{VarY}), we have
\begin{eqnarray}
\displaystyle{\lim_{N \rightarrow \infty}}Var(\underline{X}) &=& \frac{R}{(R-1)^2} =\frac{\varphi}{(\varphi-1)^2} \nonumber \\
\Rightarrow \displaystyle{\lim_{N \rightarrow \infty}}Var(\underline{X}) &=& 2\varphi + 1\label{eq-8.2}
\end{eqnarray}
Comparing eq.(\ref{varf}) and eq.(\ref{varyd}), we find that, in the case of the classical Fibonacci sequence, the variance of the distribution converges to the same value, irrespective of whether the probability values are increasing or decreasing.

In the following sections, we make a few observations and remarks related to the classical Fibonacci sequence.

\section{Ratio of Variance to Mean}\label{section-9}
We observe the curious fact that, from eq.(\ref{MeanNY}) and eq.(\ref{VarianceNY}),
\begin{equation}
\lim_{N\rightarrow \infty} \frac{Var(\underline{X})}{E(\underline{X})} = \frac{1}{R-1}\label{eq-9.1}
\end{equation}
In the case $a = b = 1$, that is, in the case of the distribution induced by the Fibonacci sequence, since $R = \varphi$
\begin{equation}
\lim_{N\rightarrow \infty} \frac{Var(\underline{X})}{E(\underline{X})} = \frac{1}{\varphi-1} = \varphi\label{eq-9.2}
\end{equation}
Thus the ratio of the variance of $\underline{X}$ to the expectation of $\underline{X}$ also converges to the Golden Ratio, $\varphi$.

\section{Optimality Probabilistic limit properties of the Fibonacci Sequence}\label{section-10}
In this section, we establish certain optimal probabilistic limit properties of the classical Fibonacci sequence. 

\subsection{Mean}\label{subsection-10.1}
\textsc{Theorem {\ref{th:MeanGoldenRatio}}:}
\begin{eqnarray*}
\label{MeanMax1}
\max{\left(\lim_{N\rightarrow \infty} E(\underline{X}): a, b \in \mathbb N\right)} = \lim_{N \rightarrow \infty}\left(E(\underline{X}) : a,b = 1\right) = \varphi + 1
\end{eqnarray*}
\begin{proof}\renewcommand{\qedsymbol}{} 
 We know that 
\begin{eqnarray}
 R &\geq& \varphi\nonumber\\
 \Rightarrow R-1 &\geq& \varphi -1 \nonumber\\
\Rightarrow \frac{1}{R-1} &\leq& \frac{1}{\varphi-1}\label{eq-10.1}
 \end{eqnarray}
 But $\varphi$ = $\displaystyle{\frac{1}{\varphi-1}}$. \\Hence, from eq.(\ref{compare}), we get
{\fontsize{8}{9}\selectfont\begin{eqnarray}
\frac{1}{R-1} \leq \varphi\nonumber\\
\Rightarrow 1 + \frac{1}{R-1} &\leq& 1+\varphi\\\label{eq-10.2}
\displaystyle{\frac{R}{R-1}} &\leq& \varphi + 1 \nonumber\\
\Rightarrow \lim_{N\rightarrow \infty}\left(E(\underline{X}):a,b\in \mathbb N\right) &\leq& \lim_{N\rightarrow \infty}\left(E(\underline{X}): a = b = 1\right)\label{eq-10.3}
\end{eqnarray}}
Thus, \\
$\max{\left(\displaystyle{\lim_{N\rightarrow \infty}} E(\underline{X}): a, b \in \mathbb N\right)} = \displaystyle{\lim_{N \rightarrow \infty}}\left(E(\underline{X}) : a,b = 1\right) = \varphi + 1$
\end{proof}

\subsection{Variance}\label{subsection-10.2}

Among the family of distributions, induced by second order linear recurrence relations with positive integer coefficients, $a$ and $b$, the Fibonacci sequence gives the maximum limit for $Var(\underline{X})$, i.e.\\
\textsc{Theorem {\ref{th:VarianceGoldenRatio}}:}
{\fontsize{6}{7}\selectfont\begin{eqnarray}
\max{\left(\lim_{N\rightarrow \infty}Var(\bar{X}): a,b \in \mathbb N\right)}&=& \max{\left(\lim_{N\rightarrow \infty}Var(\underline{X}): a,b \in \mathbb N\right)}\\\label{eq-10.4}
 &=& \lim_{N \rightarrow \infty}\left(Var(\underline{X}):a,b=1\right) = 2\varphi + 1\label{eq-10.5}
\end{eqnarray}}

We prove this as follows:
\begin{proof}\renewcommand{\qedsymbol}{} 
\begin{eqnarray}
 R &\ge& \varphi\nonumber\\
 \Rightarrow \frac{1}{R} &\le& \frac{1}{\varphi}\nonumber\\
\Rightarrow -\frac{1}{R} &\ge& - \frac{1}{\varphi}\nonumber\\
\Rightarrow 1-\frac{1}{R} &\ge& 1 - \frac{1}{\varphi}\nonumber\\
\Rightarrow R\left(1-\frac{1}{R}\right)^2 &\ge& \varphi\left(1 - \frac{1}{\varphi}\right)^2\nonumber\\
\Rightarrow \frac{1}{R\left(1-\frac{1}{R}\right)^2} &\le& \frac{1}{\varphi\left(1 - \frac{1}{\varphi}\right)^2}\label{eq-10.6}
\end{eqnarray}
From eq.(\ref{vareq}), we get 
{\fontsize{7}{8}\selectfont\begin{eqnarray}
\frac{R}{(R-1)^2} \le \frac{\varphi}{(\varphi-1)^2} = 2\varphi + 1\nonumber\\
\Rightarrow \frac{R}{(R-1)^2} &\le& 2\varphi + 1\\\label{eq-10.7}
\Rightarrow \label{varmaxi1}\lim_{N\rightarrow \infty} \left(Var(\underline{X}):a,b\in \mathbb N\right) &\leq& \lim_{N\rightarrow \infty}\left(Var(\underline{X}): a = b = 1\right)\label{eq-10.8}
\end{eqnarray}}
\end{proof}
From eq.(\ref{varmaxi1}), we conclude that the limit variance of discrete probability distributions, induced by decreasing sequence and generated using second order linear recurrence, is maximum for the one generated using Fibonacci sequence.

In the next section, we consider the probability distributions generated by higher order recurrence relation with positive integer coefficients.

\section{Higher order Recurrence relation with positive integer coefficients}\label{section-11}
From Section.\ref{Increasing} and Section.\ref{Decreasing}, one can observe that the mean and variance of the probability distributions, induced by general second order linear recurrence relation with positive integer coefficients, depend only on the largest positive real root, $R$, of the associated ``characteristic equation''. It turns out that, even in the case of any other higher order recurrence relation with positive integer coefficients, it is the largest positive real root that determines the mean and variance. We now proceed to prove the same.
Let 
\begin{equation}
 f[n] = a_{n-1}f[n-1] + \ldots + a_{n-k}f[n-k] \hspace{0.3in} \forall a_i \in \mathbb{N}\label{eq-11.1}
\end{equation}
be a $k^{th}$ order recurrence relation. The characteristic equation of this recurrence relation is given by,
 \begin{equation}
    f(x) = x^k - a_{n-1}x^{k-1}- \ldots - a_{n-k}\label{eq-11.2}
 \end{equation}
\begin{lem}\label{lemma-1}
Eq.(\ref{higherordgenrec}), has atleast one real positive root and if $\alpha$ is the largest positive real root, then all the other roots (real and complex) lie within the disc $|z| < \alpha$.
\end{lem}

Using Lemma \ref{LemCharEq}, it is easy to infer the following:\\
With $R = \alpha$,
\begin{itemize}
\item The same asymptotic relations as in eq.(\ref{meand})and eq.(\ref{VarY}) hold. 
\item The same limiting properties as in eq.(\ref{MeanY}) and eq.(\ref{VarianceY}) hold.
\item The optimality properties of the Fibonacci sequence as in Theorem \ref{th:MeanGoldenRatio} and in Theorem \ref{th:VarianceGoldenRatio} hold good. The reason is that, the largest positive real root, $\alpha > \varphi$, the golden ratio.
\end{itemize}
Thus, Theorem \ref{th:MeanGoldenRatio} can be generalized as 
\begin{eqnarray*}
\max{\left(\lim_{N\rightarrow \infty} E(\underline{X}): a_i, i \in \mathbb N \right)} = \lim_{N \rightarrow \infty}\left(E(\underline{X}):a_1 = a_2 = 1, a_{i}=0, \forall i> 2\right) = \varphi + 1
\end{eqnarray*}
and Theorem \ref{th:VarianceGoldenRatio} can be further generalized as 
\begin{eqnarray*}
\max{\left(\lim_{N\rightarrow \infty}Var(\bar{X}): a_i, i \in \mathbb N\right)}&=& \max{\left(\lim_{N\rightarrow \infty}Var(\underline{X}): a_i, i \in \mathbb N\right)}\\
\nonumber &=& \lim_{N \rightarrow \infty}\left(Var(\underline{X}):a_1=a_2=1, a_{i}=0,\forall i > 2\right) = 2\varphi + 1
\end{eqnarray*}

Now, we proceed to prove Lemma \ref{LemCharEq}.
\begin{proof}\renewcommand{\qedsymbol}{} 
By Descartes rule of signs, there exists a positive root. Since, \\ 
 \begin{center}
 $ f(0) < 0$, $f(\infty) > 0$ \\
 \end{center}
Let $\alpha$ be the largest positive root. \\

\subsection{\texorpdfstring{$\alpha$ is simple root}}\label{subsection-11.1}

Claim: The root is simple. \\
Let $f(x)$ be as in eq.(\ref{higherordgenrec}). Since $\alpha$ is a root, we have
 \begin{equation}
\alpha^k - a_{n-1}\alpha^{k-1} + \ldots + a_{n-k} = 0\label{eq-11.3}
 \end{equation}
Suppose, if $\alpha$ is a multiple root, then we must have
\begin{equation}
 f'(\alpha) = 0 \label{eq-11.4}
\end{equation}
\begin{equation}
\Rightarrow k\alpha^{k-1}-(k-1)a_{n-1}\alpha^{k-2} + \ldots + a_{n-k+1} = 0 \label{eq-11.5}
\end{equation}
Multiplying by $\alpha$, we get
\begin{eqnarray}
 k\alpha^{k} - (k-1)a_{n-1}\alpha^{k-1} + \ldots + a_{n-k+1}\alpha = 0\nonumber\\
 \Rightarrow  k\alpha^{k} &=& (k-1)a_{n-1}\alpha^{k-1} + \ldots + a_{n-k+1}\alpha\nonumber\\
 \Rightarrow   &<& k\left( a_{n-1}\alpha^{k-1} + \ldots +a_{n-k+1}\alpha \right)\nonumber\\
 \Rightarrow     &<& k\left( a_{n-1}\alpha^{k-1} + \ldots +a_{n-k+1}\alpha + a_{n-k}\right)\nonumber\\
 \Rightarrow \alpha^{k} &<&  a_{n-1}\alpha^{k-1} + \ldots + a_{n-k+1}\alpha + a_{n-k}\nonumber\\
 \Rightarrow\alpha^{n} &<& \alpha^{n} \nonumber
\end{eqnarray}
This contradicts eq.(\ref{alpharoot}). Hence $\alpha$ must be a simple root.

%~ \subsection{All roots lie within circle of radius $\alpha$}

Now, we prove that all other roots of eq.(\ref{higherordgenrec}) lie within the circle of radius $\alpha$. Consider eq.(\ref{higherordgenrec}), the characteristic equation of the corresponding $k^{th}$ order recurrence relation with positive integer coefficients. Since $\alpha$ is a root, 
   \begin{equation}\nonumber
     f(\alpha) = 0
   \end{equation}
Also, we have
 \begin{eqnarray}
  \displaystyle{\lim_{x\rightarrow \infty}} f(x) &=& \infty \nonumber\\
  \Rightarrow  \label{functionroots} f(x) &\ge& 0, \hspace{0.3in} \forall x \ge \alpha \\
  \textup{and}\label{functiongreaterroots}\hspace{0.1in} f(x) &>& 0  \hspace{0.3in} \forall x > \alpha           
 \end{eqnarray}
Let $\beta$ be any root (real or complex) of eq.(\ref{higherordgenrec}). Clearly, $\beta$ cannot be zero, because,
 \begin{equation}\nonumber
 f(0) = -a_{n-k} < 0
  \end{equation}
Let $b$ = $|\beta|$.\\
Claim: $b$ $\le$ $\alpha$\\
Reason: Suppose $b$ $>$ $\alpha$.\\
 Since $\beta$ is a root, we have
 \begin{eqnarray}
 f(\beta) = 0 \Rightarrow \beta^{k}  &=&  a_{n-1}\beta^{k-1} + \ldots + a_{n-k}\beta^{n-k} \nonumber\\
              \Rightarrow |\beta|^{k}  &=&  |a_{n-1}\beta^{k-1} + \ldots + a_{n-k}\beta^{n-k}|\nonumber\\
             \Rightarrow \beta^{k}  &\leq&  a_{n-1}|\beta|^{k-1} + \ldots + a_{n-k}|\beta|^{n-k}\nonumber\\
              \Rightarrow \beta^{k}-a_{n-1}|\beta|^{k-1} - \ldots - a_{n-k}|\beta|^{n-k} &\leq& 0 \nonumber\\
              \Rightarrow b^{k}-a_{n-1}b^{k-1} - \ldots - a_{n-k}b^{n-k} &\leq& 0 \nonumber\\
  \label{functionlessthanzero} \Rightarrow f(b) &\leq& 0
 \end{eqnarray}
However, since $b$ $>$ $\alpha$, by eq.(\ref{functiongreaterroots}), $f(b)$ $>$ 0. Thus, we get a contradiction and hence $b$ $\le$ $\alpha$.

Claim: $b$ = $\alpha$ \\
Since $\beta$ is a root, we have,
\begin{eqnarray}
\beta^k = a_{n-1}\beta^{k-1} + \ldots + a_{n-k}\beta^{n-k}\nonumber\\
\Rightarrow |\beta|^k &=& |a_{n-1}\beta^{k-1} + \ldots + a_{n-k}\beta^{n-k}|\nonumber
\end{eqnarray}
Since we have assumed that $|\beta|$=$b$=$a$, 
\begin{eqnarray}
\alpha^k = |a_{n-1}\beta^{k-1}+ \ldots + a_{n-k}\beta^{n-k}|\nonumber\\
a_{n-1}\alpha^{k-1}+ \ldots + a_{n-k} &=&  |a_{n-1}\beta^{k-1}+ \ldots + a_{n-k}| \hspace{0.2in}\textup{by \hspace{0.1in} eq.(\ref{alpharoot})} \\
a_{n-1}|\beta|^{k-1}+ \ldots + a_{n-k} &=&  |a_{n-1}\beta^{k-1}+ \ldots + a_{n-k}|
\end{eqnarray}
$\Rightarrow a_n-1\beta^{k-1}, \ldots, a_{n-k}$ are all complex numbers in the same direction. \\
$\Rightarrow a_{n-k}, a_{n-k+1} \beta$ are real positive\\
$\Rightarrow \beta$ is real positive\\
$\Rightarrow \beta = \alpha$.\\

Thus, based on the above claims, we find that 
\begin{itemize}
\item the only root on the circle $|z|$ = $\alpha$ is $z$ = $\alpha$ and there is no root outside this circle
\item all roots not equal to $\alpha$ are inside the circle $|z|$ = $\alpha$.
\end{itemize}

%~ \subsection{$\alpha$ is strictly greater than the golden ratio $\varphi$}
We prove that the largest positive real root of the characteristic equation, corresponding to higher order linear recurrence relation with integer coefficients, is greater than the golden ratio $\varphi$.
\begin{eqnarray}
\alpha^{k} &=&  a_{n-1}\alpha^{k-1} + \ldots + a_{n-k}\alpha^{n-k} \nonumber\\
\textup{Since all terms on the RHS are greater than 0} \nonumber\\
\Rightarrow \alpha^{k} &>&  a_{n-1}\alpha^{k-1} + a_{n-2}\alpha^{k-2}\nonumber \\
\Rightarrow \alpha^{2} &>&  a_{n-1}\alpha + a_{n-2}\nonumber
\end{eqnarray}
As $a_{n-1}$, $a_{n-2}$ are positive integers, we see that,
\begin{eqnarray}
\Rightarrow \alpha^{2} &>&  \alpha + 1\nonumber\\
\label{alphaphicompare}\Rightarrow \alpha^{2} -  \alpha - 1\ &>& 0
\end{eqnarray}
From the fact that, $\varphi$, the golden ratio, is the largest positive root of $x^2-x-1 = 0$ and that $x^2-x-1 > 0$ for $x > \varphi$, it follows that
$\alpha$ is always greater than $\varphi$.
\end{proof}














\end{multicols}
