{
\makeatletter
\def\@makechapterhead#1{%
  \vspace*{5\p@}%
\centerline{\includegraphics{images/om.jpg}}
\vspace*{10\p@}%
  {\parindent \z@ \centering \normalfont
    \ifnum \c@secnumdepth >\m@ne
      \if@mainmatter
        %\LARGE\bfseries \@chapapp\space \thechapter
	\vskip 4pt
%        \hrule height 2pt
        \par\nobreak
        \vskip 5\p@
      \fi
    \fi
    \interlinepenalty\@M
{\fontsize{30pt}{30pt}\selectfont \bfseries #1}\par\nobreak
\smallskip 

%\hrule height 2pt   
 \vskip 15\p@  
  }}
\makeatother
\chapter{ಶ್ರೀ ಮ ದ್ಭ ಗ ವ ದ್ಗೀ ತಾ}

\textbf{ಧೃತರಾಷ್ಟ್ರ ಉವಾಚ\enginline{---}}

\begin{verse}
ಧರ್ಮಕ್ಷೇತ್ರ ಕುರುಕ್ಷೇತ್ರ ಸಮವೇತಾ ಯುಯುತ್ಸವಃ\\
ಮಾಮಕಾಃ ಪಾಂಡವಾಶೈವ ಕಿಮಕುರ್ವತ ಸಂಜಯ\hfill ೧
\end{verse}
\footnotetext{
\centerline{ಶ್ರೀ ಗುರುಭ್ಯೋ ನಮಃ ಹರಿಃ ಓಂ.}

 ಶ್ರೀಕೃಷ್ಣ ಪರಮಾತ್ಮನು ದುಷ್ಟ ಶಿಕ್ಷಣಾಧಿಬಲಕಾರವನ್ನು ಮಾಡುವ ಉದ್ದೇಶ ದಿಂದ ಅವತರಿಸಿ ಅದನ್ನು ನೆರವೇರಿಸಿದ ಬಳಿಕ ಲೋಕದ ಪರಿಸ್ಥಿತಿಯನ್ನು ನೋಡಿ ತತ್ವಜ್ಞಾನವಿಲ್ಲದೆ ನರಳುತ್ತಿರುವ ಸಾಧುಜನರಲ್ಲಿ ದಯೆಮಾಡಿ ತತ್ತೋಪದೇಶದಿಂದ ಅವರನ್ನು ಆರಿಸಬೇಕೆಂದು ಸಂಕಲ್ಪಿಸಿದನು. ಆದರೆ ಪ್ರಸಂಗವಿಲ್ಲದೆ ತತ್ತೋಪದೇಶವನ್ನಾರಂಭಿಸುವುದು ಸರಿಯಲ್ಲ ; ಫಲಕಾರಿಯೂ ಆಗುವುದಿಲ್ಲವಾದ ಕಾರಣ ಒಂದು ಪ್ರಸಂಗವನ್ನು ಹೂಡಿದನು. ಕುರುಕ್ಷೇತ್ರದಲ್ಲಿ ಕೌರವಪಾಂಡವಸೈನ್ಯಗಳು ಯುದ್ಧಕ್ಕೆ ಸನ್ನದ್ಧವಾಗಿ ಇದಿರು ಬದಿರು ನಿಂತಿರುವ ಸಂದರ್ಭದಲ್ಲಿ ಸರ್ವೆಂದ್ರಿಯಪ್ರೇರಕನಾಗಿ ಹೃಷಿಕೇಶನೆನಿಸಿದ ಶ್ರೀಕೃಷ್ಣನು ಅರ್ಜುನನಿಗೆ ವಿಷಾದ ಮತ್ತು ಯುದ್ಧದಲ್ಲಿ ಜಿಹಾಸೆಗಳನ್ನು ಹುಟ್ಟಿಸಿ ದನು. ವಿಷಾದಹೊಂದಿದ ಅರ್ಜುನನು ದಿಕ್ಕು ತೋಚದೆ ಸಾರಥಿಯಾಗಿದ್ದ ಶ್ರೀಕೃಷ್ಣನನ್ನು ಮರೆಹೊಕ್ಕು ತನ್ನ ವಿಷಾದ ಜಿಹಾಸೆಗಳನ್ನು ಹೇಳಿಕೊಂಡು ಕರ್ತವ್ಯ ಧರ್ಮವನ್ನು ಉಪದೇಶಿಸಬೇಕೆಂದು ಪ್ರಾರ್ಥಿಸಿದನು. ಆಗ ಶ್ರೀಕೃಷ್ಣನು ಅರ್ಜುನನನ್ನೇ ನಿಮಿತ್ತ ಮಾಡಿ ಧರ್ಮತತ್ತೋಪದೇಶವನ್ನಾರಂಭಿಸಿದನು. ಆ ಉಪದೇಶವೇ ಭಗವದ್ಗೀತೆಯೆಂಬ ಅಧ್ಯಾತ್ಮಶಾಸ್ತ್ರವು. ಅದು ಅಶೋಟ್ಯಾನನ್ನ. ಶೋಚ೦' ೨-೧೧ ಇತ್ಯಾದಿ ಶ್ಲೋಕದಿಂದ ಆರಂಭವಾಗಿದೆ. ಅದರ ಹಿಂದಿನ ಭಾಗವಾದ ಪ್ರಥಮಾಧ್ಯಾಯ ಮತ್ತು ದ್ವಿತೀಯಾಧ್ಯಾಯದಲ್ಲಿ ಹತ್ತು ಶ್ಲೋಕಗಳು ಅರ್ಜುನನ ವಿಷಾದಜಿಹಾಸಕಥನೆಯು, ಇದೇ ಗೀತಾರಚನೆಗೆ ಮೂಲವಾದ ಪ್ರಸಂ ಗವು. ಇದನ್ನೇ ಉಪದೇಶ ಪೀಠಿಕೆಯಾಗಿ ಗೀತೆಯಲ್ಲಿ ಸೇರಿಸಿದೆ. 

ಕುರುಕ್ಷೇತ್ರದಲ್ಲಿ ಕೌರವ ಪಾಂಡವರ ಸೈನ್ಯಗಳು ಯುದ್ಧಕ್ಕೆ ಸನ್ನದ್ಧವಾದಾಗ ಹುಟ್ಟು ಕುರುಡನಾದ ಧೃತರಾಷ್ಟ್ರನು ಪುತ್ರವ್ಯಾಮೋಹದಿಂದ ವಿವೇಕಶೂನ್ಯನಾಗಿ ತನ್ನ ಮಕ್ಕಳಿಗೆ ಈ ಯುದ್ಧದಿಂದ ಕೆಡಕಾದೀತೆಂಬ ಶಂಕೆಯಿಂದ ವಿಷಾದಪಡುತ್ತಿದ್ದನು. ಆಗ ಶ್ರೀವೇದವ್ಯಾಸಸ್ವಾಮಿಯು ವಾತ್ಸಲ್ಯದಿಂದ ಧೃತರಾಷ್ಟ್ರನ ಬಳಿಗೆ ಬಂದು ರಾಜಾ! ಮುಂದಿನ ಯುದ್ಧದಲ್ಲಿ
}
