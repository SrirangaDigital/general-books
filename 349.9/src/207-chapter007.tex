
\chapter{ಭಗವದ್ಗೀತೆ - ೩}

(೧೯೦೦ರ ಮೇ ೨೬ರಂದು ಸ್ಯಾನ್ ಫ್ರಾಂಸಿಸ್ಕೊದಲ್ಲಿ ನೀಡಿದ ಉಪನ್ಯಾಸದ ಟಿಪ್ಪಣಿ)

(೧) “ನೀವು ಎಲ್ಲವನ್ನೂ ತಿಳಿದಿದ್ದರೂ ಶಿಶುಸಹಜ ಮುಗ್ಧತೆಯನ್ನು ಕೆಡಿಸಬೇಡಿ.”

(೨) “ಧರ್ಮವೆಂದರೆ ಚೇತನವನ್ನು ಚೇತನದಂತೆಯೇ ಅರಿಯುವುದು, ಚೇತನವನ್ನು ದ್ರವ್ಯದಂತೆ ಅರಿಯುವುದಲ್ಲ.”

(೩) “ನೀವು ಚೇತನ ಸ್ವರೂಪರು. ನೀವು ಚೇತನವೆಂಬುದನ್ನು ಅರಿಯಿರಿ. ಇದನ್ನು ಯಾವುದಾದರೂ ರೀತಿಯಲ್ಲಿ ಸಾಧಿಸಿ.”

(೪) “ಧರ್ಮವು ಒಂದು ಬೆಳವಣಿಗೆ.” ಪ್ರತಿಯೊಬ್ಬನೂ ತಾನೇ ಇದನ್ನು ಅನುಭವಿಸಬೇಕು.

(೫) “ಪ್ರತಿಯೊಬ್ಬನೂ, ‘ನನ್ನ ಮಾರ್ಗವೇ ಶ್ರೇಷ್ಠ’ ಎಂದು ಭಾವಿಸುತ್ತಾನೆ. ಅದು ನಿಜ, ಆದರೆ ಅದು ನಿಮಗೆ ಶ್ರೇಷ್ಠವಾದುದು, ಅಷ್ಟೇ.”

(೬) “ಚೇತನವು ಚೇತನ ರೂಪದಲ್ಲಿಯೇ ಪ್ರಕಾಶಗೊಳ್ಳಬೇಕು.”

(೭) “ಚೇತನವು ದ್ರವ್ಯದೊಂದಿಗೆ ತಾದಾತ್ಮ್ಯಹೊಂದಿದ ಸಮಯ ಇರಲೇ ಇಲ್ಲ.”

(೮) “ಪ್ರಕೃತಿಯಲ್ಲಿ ಸತ್ಯವಾದದು ಚೇತನ.”

(೯) “ಕ್ರಿಯೆ ಪ್ರಕೃತಿಯಲ್ಲಿದೆ.”

(೧೦) “ಆದಿಯಲ್ಲಿ ಆ ಸತ್ಯವೊಂದೇ ಇತ್ತು. ಅದು ನೋಡಿತು, ಆಗ ಎಲ್ಲವೂ ಸೃಷ್ಟಿ ಸಲ್ಪಟ್ಟಿತು.”

(೧೧) “ಪ್ರತಿಯೊಬ್ಬನೂ ತನ್ನ ಸ್ವಭಾವಕ್ಕನುಗುಣವಾಗಿ ವರ್ತಿಸುತ್ತಾನೆ.”

(೧೨) “ನೀವು ನಿಯಮದಿಂದ ಬದ್ಧರಾಗಿಲ್ಲ. ಅದು ನಿಮ್ಮ ಪ್ರಕೃತಿಯಲ್ಲಿದೆ. ಮನಸ್ಸು ಪ್ರಕೃತಿಯಲ್ಲಿದೆ, ಆದ್ದರಿಂದ ಅದು ನಿಯಮದಿಂದ ಬದ್ಧವಾಗಿದೆ.”

(೧೩) “ನೀವು ಧಾರ್ಮಿಕರಾಗಿರಬೇಕಿದ್ದರೆ ಧಾರ್ಮಿಕವಾದದಿಂದ ದೂರವಿರಿ.”

(೧೪) “ಸರ್ಕಾರ, ಸಮಾಜ ಇತ್ಯಾದಿಗಳು ಕೆಡುಕುಗಳು. ಎಲ್ಲ ಸಮಾಜಗಳೂ ದೋಷಯುಕ್ತ ಸಾಮಾನ್ಯೀಕರಣದ ಮೇಲೆ ನಿಂತಿವೆ. ಯಾವುದನ್ನು ಉಲ್ಲಂಘಿಸಲಾಗುವುದಿಲ್ಲವೋ ಅದೇ ನಿಯಮ.”

(೧೫) “ಪ್ರೀತಿಯು ಇತರರನ್ನು ದ್ವೇಷಿಸುವಂತೆ ಮಾಡುವುದಾದರೆ ಪ್ರೀತಿಸದಿರು ವುದೇ ಒಳ್ಳೆಯದು.”

(೧೬) “ದೌರ್ಬಲ್ಯವೇ ಮರಣದ ಚಿಹ್ನೆ; ಶಕ್ತಿ ಜೀವನದ ಚಿಹ್ನೆ.”

(ಕೆಳಗಿನ ಸಂಖ್ಯಾಬದ್ಧ ಪ್ಯಾರಾಗಳು ಮೇಲಿನ ವಾಕ್ಯಗಳೊಂದಿಗೆ ಸಂಬಂಧಿಸಿವೆ.)

೪. ಕ್ರೈಸ್ತನು ತನ್ನನ್ನು ಉದ್ಧಾರಮಾಡುವುದಕ್ಕಾಗಿ ಕ್ರಿಸ್ತನು ಪ್ರಾಣತ್ಯಾಗಮಾಡಿದ ನೆಂದು ನಂಬುತ್ತಾನೆ. ನೀವು ಕೇವಲ ಒಂದು ಮತ ತತ್ತ್ವವನ್ನು ನಂಬುತ್ತೀರಿ, ಅಷ್ಟೆ; ಮತ್ತು ಈ ನಂಬಿಕೆಯೇ ನಿಮಗೆ ಮುಕ್ತಿ. ಪ್ರತಿಯೊಬ್ಬನೂ ತನಗಿಷ್ಟವಾದ ತತ್ತ್ವವನ್ನು ನಂಬಬಹುದು, ಇಲ್ಲವೇ ನಂಬದೆ ಇರಬಹುದು. ನಮ್ಮ ಪಾಲಿಗೆ ಸಾಕ್ಷಾತ್ಕಾರವೇ ಧರ್ಮ - ಮತತತ್ತ್ವವಲ್ಲ. ಕ್ರಿಸ್ತನು ಒಂದು ಕಾಲದಲ್ಲಿ ಬದುಕಿದ್ದ ಎಂಬುದರಿಂದ ನಿಮಗೇನಾ ದಂತಾಯಿತು? ಮೋಸಸನು ದೇವರನ್ನು ಕಂಡಿದ್ದರೆ ನೀವು ಕಂಡಂತಾಗಲಿಲ್ಲ. ಹಾಗಾಗು ವುದಾದರೆ ಮೋಸಸನು ಊಟ ಮಾಡಿದ್ದೆ ಸಾಕಾಗುತ್ತದೆ, ನೀವು ಉಣ್ಣಬೇಕಾಗಿಲ್ಲ! ಒಬ್ಬ ವ್ಯಕ್ತಿಯು ಇನ್ನೊಬ್ಬನಷ್ಟೇ ಸೂಕ್ಷ್ಮನಾಗಿರುತ್ತಾನೆ. ಹಿಂದಿನ ಕಾಲದ ಮಹಾಪುರುಷರ ಜೀವನ ವೃತ್ತಾಂತವು ನಮ್ಮನ್ನು ಆಧ್ಯಾತ್ಮಿಕ ಅನುಭವವನ್ನು ಪಡೆಯಲು ಪ್ರೇರೇಪಣೆ ನೀಡುವುದಲ್ಲದೆ ಮತ್ತಾವ ಪ್ರಯೋಜನವನ್ನೂ ಉಂಟುಮಾಡುವುದಿಲ್ಲ. ಮೋಸಸ್ ಅಥವಾ ಕ್ರಿಸ್ತ ಮುಂತಾದವರು ಏನು ಮಾಡಿದರೆಂಬುದರಿಂದ ನಾವು ಮುಂದುವರಿ ಯುವುದಕ್ಕೆ ಪ್ರೇರೇಪಣೆಯಲ್ಲದೆ ಮತ್ತಾವ ಪ್ರಯೋಜನವೂ ಇಲ್ಲ.

೫. ಪ್ರತಿಯೊಬ್ಬನಲ್ಲೂ ತನ್ನದೇ ಆದ ವೈಶಿಷ್ಟ್ಯವಿರುತ್ತದೆ. ಅವನು ಅದನ್ನು ಅನು ಸರಿಸಿ, ಅದರ ಮೂಲಕ ಮುಕ್ತಿಯನ್ನು ಪಡೆಯಬೇಕು. ನಿಮ್ಮ ಗುರುವು ನಿಮ್ಮ ಸ್ವಭಾವಕ್ಕೆ ಅನುಗುಣವಾದ ಮಾರ್ಗವಾವುದು ಎಂಬುದನ್ನು ಹೇಳಬಲ್ಲವನಾಗಿರುತ್ತಾನೆ, ಮತ್ತು ಆ ಮಾರ್ಗದಲ್ಲಿ ನಡೆಯಲು ಮಾರ್ಗದರ್ಶನ ಮಾಡುತ್ತಾನೆ. ನಿಮ್ಮ ಮುಖವನ್ನು ನೋಡಿಯೇ ನಿಮ್ಮ ಸ್ವಭಾವವನ್ನು ಅರಿತು ಅವನು ನಿಮಗದನ್ನು ಸೂಚಿಸುತ್ತಾನೆ. ನೀವು ಇನ್ನೊಬ್ಬರ ಮಾರ್ಗವನ್ನು ಅನುಸರಿಸಲು ಪ್ರಯತ್ನಿಸಬಾರದು, ಏಕೆಂದರೆ ಅದು ಅವನದು, ನಿಮ್ಮದಲ್ಲ. ಮಾರ್ಗವು ನಿರ್ದಿಷ್ಟವಾದ ಮೇಲೆ, ನೀವು ಸುಮ್ಮನೆ ಕೈಕಟ್ಟಿಕೊಂಡು ನಿಂತುಬಿಡಬಹುದು, ಪ್ರವಾಹವೇ ನಿಮ್ಮನ್ನು ಕರೆದುಕೊಂಡು ಹೋಗುತ್ತದೆ. ಆದ್ದರಿಂದ ನಿಮ್ಮ ಮಾರ್ಗ ತಿಳಿದ ಮೇಲೆ ಅದರಿಂದ ಎಂದೂ ವಿಚಲಿತರಾಗಬೇಡಿ. ನಿಮ್ಮ ಮಾರ್ಗವೇ ನಿಮಗೆ ಅತ್ಯುತ್ತಮ. ಆದರೆ ಅದು ಇತರರಿಗೆ ಅತ್ಯುತ್ತಮವಾಗಿರಬೇಕೆಂದಿಲ್ಲ.

೬. ನಿಜವಾದ ಆಧ್ಯಾತ್ಮಿಕ ವ್ಯಕ್ತಿಯು ಚೇತನವನ್ನು ಚೇತನದಂತೆಯೇ ನೋಡುತ್ತಾನೆ, ದ್ರವ್ಯದಂತಲ್ಲ. ದ್ರವ್ಯವು ಅತ್ಯಂತ ಕ್ಷೀಣ ಸ್ಪಂದನದ ಸ್ಥಿತಿಯಲ್ಲಿರುವ ಚೇತನವೇ ಆಗಿದ್ದರೂ, ಚೇತನವು ಖಂಡಿತ ದ್ರವ್ಯವಾಗಲಾದು. ಚೇತನವೇ ಪ್ರಕೃತಿಯನ್ನು ಚಲಿಸುವಂತೆ ಮಾಡುತ್ತದೆ; ಅದೇ ಪ್ರಕೃತಿಯಲ್ಲಿರುವ ಸತ್ಯ. ಆದ್ದರಿಂದ ಚಟುವಟಿಕೆಯು ಪ್ರಕೃತಿಯಲ್ಲಿರೂವುದೇ ಹೊರತು ಚೇತನದಲ್ಲಿಲ್ಲ. ಚೇತನವು ಯಾವಾಗಲೂ ಬದಾಲಗದೆ ಒಂದೇ ಸಮನಾಗಿರುತ್ತದೆ ಮತ್ತು ಅದು ಶಾಶ್ವತ. ಚೇತನ ಮತ್ತು ದ್ರವ್ಯ ಇವೆರಡೂ ಯಥಾರ್ಥವಾಗಿ ಒಂದೇ ಆದರೂ, ಚೇತನವು ದ್ರವ್ಯವಾಗಲಾರದು ಮತ್ತು ದ್ರವ್ಯವು ಚೇತನವಾಗಲಾರದು, ಏಕೆಂದರೆ ಅದು ಚೇತನದ ಇನ್ನೊಂದು ರೂಪ ಅಷ್ಟೆ - ಚೇತನದ ಅತ್ಯಂತ ಕ್ಷೀಣವಾದ ಸ್ಪಂದನಾವಸ್ಥೆ. ನೀವು ಆಹಾರವನ್ನು ಸ್ವೀಕರಿಸುತ್ತೀರಿ, ಅದು ಮನಸ್ಸಾಗಿ ಪರಿಣಮಿಸುತ್ತದೆ ಮತ್ತು ಮನಸ್ಸು ತಿರುಗಿ ದೇಹವಾಗಿ ಪರಿಣಮಿಸುತ್ತದೆ. ಹೀಗೆ ಮನಸ್ಸು ಮತ್ತು ದೇಹ; ಚೇತನ ಮತ್ತು ದ್ರವ್ಯ - ಇವು ಬೇರೆ ಬೇರೆಯಾಗಿ ದ್ದರೂ ಒಂದು ಮತ್ತೊಂದಾಗಿ ಪರಿಣಮಿಸುತ್ತದೆ. ಆದರೆ ಅವು ಒಂದೇ ಅಲ್ಲ.

೮. “ಪ್ರಕೃತಿಯಲ್ಲಿ ಸತ್ಯವಾಗಿರುವುದೇ ಚೇತನ.” ಪ್ರಕೃತಿಯಲ್ಲಿ ನಡೆಯುವ ಚಟು ವಟಿಕೆಯ ಹಿಂದಿರುವ ಪ್ರಾಣವೇ ಚೇತನ. ಚೇತನವೇ ಪ್ರಕೃತಿಗೆ ಅದರ ಸತ್ಯತೆಯನ್ನೂ ಕ್ರಿಯಾಶಕ್ತಿಯನ್ನೂ ನೀಡುವುದು.

೯. “ಕ್ರಿಯೆ ಪ್ರಕೃತಿಯಲ್ಲಿದೆ.” “ಚೇತನವು ನಿಷ್ಕ್ರಿಯ. ಅದು ಏಕೆ ಕರ್ಮ ಮಾಡಬೇಕು?” ಅದು ಸುಮ್ಮನೆ ಇದೆ, ಅಷ್ಟೆ - ಅದು ಇನ್ನೇನೂ ಮಾಡಬೇಕಾಗಿಲ್ಲ. ಅದು ಶುದ್ಧ ಅಸ್ತಿತ್ವ ಮಾತ್ರ ಉಳ್ಳದ್ದು, ಅದಕ್ಕೆ ಯಾವ ಕ್ರಿಯೆಯ ಆವಶ್ಯಕತೆಯೂ ಇಲ್ಲ.

೧೨. ಇಡೀ ಪ್ರಕೃತಿಯು ನಿಯಮಬದ್ಧವಾಗಿದೆ, ತನ್ನ ಕರ್ಮನಿಯಮಕ್ಕೆ ಬದ್ಧವಾಗಿದೆ. ಈ ನಿಯಮವನ್ನು ಖಂಡಿತ ಉಲ್ಲಂಘಿಸಲಾಗುವುದಿಲ್ಲ. ನೀವು ನಿಯಮವನ್ನು ಉಲ್ಲಂಘಿಸಬಲ್ಲಿರಾದರೆ ಇಡೀ ಪ್ರಕೃತಿಯು ಕ್ಷಣಮಾತ್ರದಲ್ಲಿ ಕೊನೆಗೊಳ್ಳುತ್ತದೆ - ಪ್ರಕೃತಿ ಇನ್ನಿರುವುದಿಲ್ಲ. ಮುಕ್ತಿಯನ್ನು ಪಡೆದವನು ಪ್ರಕೃತಿಯ ನಿಯಮವನ್ನು ಮುರಿಯುತ್ತಾನೆ ಮತ್ತು ಪ್ರಕೃತಿಯು ಅವನ ಪಾಲಿಗೆ ಕರಗಿಹೋಗುತ್ತದೆ, ಅವನ ಮೇಲೆ ಯಾವ ಆಧಿಪತ್ಯವನ್ನೂ ಸಾಧಿಸಲಾರದು. ಯಾವನೇ ಆದರೂ ಒಮ್ಮೆ ಮಾತ್ರ ಪ್ರಕೃತಿಯ ನಿಯಮವನ್ನು ಮುರಿಯಬಲ್ಲನು ಮತ್ತು ಅದೇ ಅವನ ಬಂಧನದ ಕೊನೆ, ಮತ್ತೆಂದೂ ಅವನಿಗೆ ಪ್ರಕೃತಿಯ ಬಾಧೆ ಇಲ್ಲ. “ನೀವು ನಿಯಮದಿಂದ ಬದ್ಧರಾಗಿಲ್ಲ. ಅದು ನಿಮ್ಮ ಪ್ರಕೃತಿಯಲ್ಲಿದೆ. ಮನಸ್ಸು ಪ್ರಕೃತಿಯಲ್ಲಿದೆ ಮತ್ತು ಅದು ನಿಯಮಬದ್ಧವಾಗಿದೆ.”

೧೪. ನೀವು ಒಂದು ಸಂಘವಾದಿರೆಂದರೆ ಆ ಸಂಘದ ಹೊರಗಿರುವವರನ್ನೆಲ್ಲ ದ್ವೇಷಿಸಲು ಪ್ರಾರಂಭಿಸುತ್ತೀರಿ. ನೀವು ಒಂದು ಸಂಘವನ್ನು ಸೇರಿದಾಗ ಒಂದು ಬಂಧನವನ್ನು ಹಾಕಿಕೊಂಡಂತೆ - ನಿಮ್ಮ ಸ್ವಾತಂತ್ರ್ಯವನ್ನು ಮೊಟಕುಗೊಳಿಸುತ್ತೀರಿ. ನಿಯಮ ನಿಬಂಧನೆಗಳನ್ನೊಳಗೊಂಡ ಒಂದು ಸಂಘವನ್ನು ನೀವು ಏಕೆ ಸ್ಥಾಪಿಸಬೇಕು? ಇದರ ಮೂಲಕ ಎಲ್ಲರ ಸ್ವಾತಂತ್ರ್ಯವನ್ನೂ ಹಾಳುಮಾಡುತ್ತೀರಿ. ಒಬ್ಬನು ಒಂದು ಸಂಘದ ಅಥವಾ ಸಮಾಜದ ನಿಯಮವನ್ನು ಉಲ್ಲಂಘಿಸಿದನೆಂದರೆ ಉಳಿದವರು ಅವನನ್ನು ದ್ವೇಷಿಸುತ್ತಾರೆ. ಬೇರೆ ಯವರ ಮೇಲೆ ನಿಯಮ ನಿಬಂಧನೆಗಳನ್ನು ಹೇರಲು ನಿಮಗಾವ ಹಕ್ಕಿದೆ? ಅಂಥ ನಿಯಮಗಳು ನಿಜವಾದ ನಿಯಮಗಳಲ್ಲ. ಅದು ನಿಯಮವಾಗಿದ್ದರೆ ಅದನ್ನು ಮುರಿಯಲು ಸಾಧ್ಯವಾಗುತ್ತಿರಲಿಲ್ಲ. ಈ ನಿಯಮಗಳು ಉಲ್ಲಂಘಿಸಲ್ಪಡುವುದೇ ಅವು ನಿಮಯಗಳಲ್ಲ ಎಂಬುದನ್ನು ತೋರಿಸುತ್ತದೆ.

