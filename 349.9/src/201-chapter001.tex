
\addtocontents{toc}{\protect\vspace{-0.7cm}}

\part{ಭಾಷಣ ಮತ್ತು ತರಗತಿಯ ಟಿಪ್ಪಣಿಗಳು}

\chapter{ಭಾರತದ ಧರ್ಮ}

(೧೮೯೬, ಬೇಸಿಗೆಯಲ್ಲಿ ಗ್ರೀನೇಕರ್ನಲ್ಲಿ ಬೆಳಗಿನ ತರಗತಿಯ ಟಿಪ್ಪಣಿಗಳು. ಈ ಟಿಪ್ಪಣಿಗಳನ್ನು ಬರೆದುಕೊಂಡವರು ಮಿಸ್ ಎಮ್ಮಾ ಥರ್ಸ್ ಬಿ.)

ಧ್ಯಾನವು ಒಂದು ಬಗೆಯ ಪ್ರಾರ್ಥನೆ ಮತ್ತು ಪ್ರಾರ್ಥನೆಯು ಒಂದು ಬಗೆಯ ಧ್ಯಾನ. ಏನನ್ನೂ ಚಿಂತಿಸದೆ ಇರುವುದೇ ಅತ್ಯುನ್ನತ ಧ್ಯಾನ. ಆಲೋಚನೆ ಇಲ್ಲದೆ ನೀವು ಒಂದು ಕ್ಷಣ ಇರಬಲ್ಲಿರಾದರೆ ಮಹಾಶಕ್ತಿ ಲಭ್ಯವಾಗುತ್ತದೆ. ಜ್ಞಾನದ ಇಡೀ ರಹಸ್ಯ ಇರು ವುದೇ ಏಕಾಗ್ರತೆಯಲ್ಲಿ. ಹೃತ್ಪೂರ್ವಕವಾಗಿ ಭಗವಂತನನ್ನು ಪ್ರೀತಿಸುವುದರ ಮೂಲಕ ಜೀವಾತ್ಮವು ಅಭಿವೃದ್ಧಿ ಹೊಂದುತ್ತದೆ. ಮನುಷ್ಯನಲ್ಲಿರುವ ಆಲೋಚನಾ ತತ್ತ್ವವೇ ಜೀವಾತ್ಮ ಮತ್ತು ಮನಸ್ಸು ಅದರ ಕ್ರಿಯೆ. ಜೀವಾತ್ಮವು ಚೇತನ ಮತ್ತು ಮನಸ್ಸುಗಳ ನಡುವೆ ಇರುವ ಒಂದು ಸಂಪರ್ಕಸಾಧನ.

ಎಲ್ಲ ಜೀವಾತ್ಮಗಳೂ ಆಟವಾಡುತ್ತಿವೆ - ಕೆಲವು ಪ್ರಜ್ಞಾಪೂರ್ವಕವಾಗಿ, ಮತ್ತೆ ಕೆಲವು ಅಪ್ರಜ್ಞಾಪೂರ್ವಕವಾಗಿ. ಪ್ರಜ್ಞಾಪೂರ್ವಕವಾಗಿ ಆಟವಾಡುವುದನ್ನು ಕಲಿಯುವುದೇ ಧರ್ಮ.

ಗುರುವು ನಿಮ್ಮ ಶ್ರೇಷ್ಠ ಆತ್ಮವೇ ಆಗಿರುವನು.

ಅತ್ಯುನ್ನತವಾದುದನ್ನು ಅರಸಿ, ಯಾವಾಗಲೂ ಅತ್ಯುನ್ನತವಾದುದನ್ನೇ ಅರಸಿ. ಏಕೆಂದರೆ ಅತ್ಯುನ್ನತವಾದುದರಲ್ಲಿಯೇ ಶಾಶ್ವತ ಆನಂದವಿರುವುದು. ನಾನು ಬೇಟೆಯಾಡ ಬೇಕಾದರೆ, ಘೇಂಡಾಮೃಗಗಳನ್ನೇ ಬೇಟಿಯಾಡುತ್ತೇನೆ. ನಾನು ದರೋಡೆ ಮಾಡ ಬೇಕಾದರೆ ರಾಜನ ಬೊಕ್ಕಸವನ್ನೇ ದರೋಡೆ ಮಾಡುತ್ತೇನೆ. ಅತ್ಯುನ್ನತವಾದುದನ್ನು ಅರಸಿ.

(ಈ ಕೆಳಗಿನವು ಸ್ವಾಮಿ ವಿವೇಕಾನಂದರ ಭಾರತೀಯ ಶಾಸ್ತ್ರವಾಕ್ಯಗಳ ಭಾವಾನುವಾದ - ವಿಶೇಷವಾಗಿ ದತ್ತಾತ್ರೇಯನ ‘ಅವಧೂತಗೀತೆ’)

ನೀವು ಬದ್ಧರು ಎಂದು ಭಾವಿಸಿದರೆ ನೀವು ಬದ್ಧರು. ನೀವು ಮುಕ್ತರು ಎಂದು ಭಾವಿಸಿ ದರೆ ನೀವು ಮುಕ್ತರು. ನನ್ನ ಮನಸ್ಸು ಈ ಪ್ರಪಂಚದ ಆಸೆಗಳಿಂದ ಎಂದೂ ಬದ್ಧವಾಗಿ ರಲಿಲ್ಲ; ಏಕೆಂದರೆ ಶಾಶ್ವತವಾದ ನೀಲಾಕಾಶದಂತೆ, ನಾನು ಸಚ್ಚಿದಾನಂದ ಸ್ವರೂಪ. ಸಹೋದರನೇ, ನೀನೇಕೆ ಅಳುತ್ತೀಯೆ, ನಿನಗೆ ದುಃಖವಾಗಲಿ ದೌರ್ಭಾಗ್ಯವಾಗಲಿ ಇಲ್ಲ? ಸಹೋದರನೆ, ನೀನೇಕೆ ಅಳುತ್ತೀಯೆ, ನಿನಗೆ ಬದಲಾವಣೆಯಾಗಲಿ, ಮೃತ್ಯುವಾಗಲಿ ಇಲ್ಲ. ನೀನು ಶಾಶ್ವತ ಸತ್ಸ್ವರೂಪನು.

ದೇವರೆಂದರೆ ಏನು ಎಂಬುದು ನನಗೆ ಗೊತ್ತು. ಆದರೆ ನಾನು ಅವನ ಬಗ್ಗೆ ನಿನಗೆ ಹೇಳಲಾರೆ. ದೇವರು ಹೇಗಿರುವನೆಂಬುದು ನನಗೆ ಗೊತ್ತಿಲ್ಲ. ನಾನು ಅವನ ಬಗ್ಗೆ ನಿನಗೆ ಹೇಗೆ ಹೇಳಲಿ? ಸಹೋದರನೆ, ನೀನೇ ಅವನು ಎಂಬುದನ್ನು ನೋಡುತ್ತಿಲ್ಲವೇನು? ಅಲ್ಲಿ ಇಲ್ಲಿ ಅವನನ್ನು ಹುಡುಕಿಕೊಂಡು ಏಕೆ ಅಲೆಯುವೆ? ಅರಸಲೇ ಬೇಡ. ಆ ಸ್ಥಿತಿಯೇ ದೇವರು. ನಿನ್ನಲ್ಲಿಯೇ ನೀನಿರು. ವಿವರಿಸಲಾಗದ, ನಮ್ಮ ಹೃದಯಾಂತರಾಳದಲ್ಲಿ ಅನುಭವಿಸಲ್ಪಡುವ ಆತ್ಮವೇ ನೀನಾಗಿರುವೆ. ಎಲ್ಲ ಹೋಲಿಕೆಗೂ ಮೀರಿದ, ಎಲ್ಲ ಮಿತಿಯನ್ನೂ ಮೀರಿದ, ಆಕಾಶದಂತೆ ಬದಲಾಗದ ಆತ್ಮವೇ ನೀನಾಗಿರುವೆ. ಆ ಪರಮ ಪವಿತ್ರವಾದುದನ್ನೇ ತಿಳಿ, ಮತ್ತೇನನ್ನೂ ಅರಸಬೇಡ.

ಯಾವುದನ್ನು ಪ್ರಕೃತಿಯ ಬದಲಾವಣೆಯು ಸ್ಪರ್ಶಿಸಲಾಗದೊ, ಯಾವುದು ಎಲ್ಲ ಆಲೋಚನೆಗಳನ್ನೂ ಮೀರಿರುವುದೊ, ಅವ್ಯಯವೊ, ಅಚಲವೊ, ಯಾವುದನ್ನು ಎಲ್ಲ ಶಾಸ್ತ್ರಗಳೂ ಸಾರುವುವೊ, ಎಲ್ಲ ಋಷಿಗಳೂ ಪೂಜಿಸುವರೊ. ಅದನ್ನು ಮಾತ್ರ ಅರಸು, ಮತ್ತಾವುದನ್ನೂ ಅರಸಬೇಡ.

ಎಲ್ಲ ಹೋಲಿಕೆಗೂ ಮೀರಿರುವುದು, ಅನಂತ ಏಕತೆ-ಯಾವ ಹೋಲಿಕೆಯೂ ಅಸಾಧ್ಯ. ಮೇಲೂ ಜಲ, ಕೆಳಗೂ ಜಲ, ಬಲಗಡೆಯೂ ಜಲ, ಎಡಗಡೆಯೂ ಜಲ - ನಿಸ್ತರಂಗ, ನಿಶ್ಚಲ. ಸಂಪೂರ್ಣ ಮೌನ, ಶಾಶ್ವತ ಆನಂದ. ಆ ಸ್ಥಿತಿಯು ನಿನ್ನ ಹೃದಯಕ್ಕೆ ಉಂಟಾಗುತ್ತದೆ. ಇನ್ನೇನನ್ನೂ ಅರಸಬೇಡ. ನೀನೇ ನಮ್ಮ ತಂದೆ, ನಮ್ಮ ತಾಯಿ, ನಮ್ಮ ಬಂಧು, ನೀನೇ ಈ ಜಗತ್ತಿನ ಹೊರೆಯನ್ನು ಹೊತ್ತಿರುವೆ. ನಮ್ಮ ಜೀವನದ ಹೊರೆಯನ್ನು ಹೊರಲು ಸಹಾಯಕನಾಗು. ನೀನೇ ನಮ್ಮ ಸ್ನೇಹಿತ, ನಮ್ಮ ಪ್ರಿಯತಮ, ನಮ್ಮ ಪತಿ. ನೀನೇ ನಮ್ಮ ಅಂತರಾತ್ಮ.

ನಾಲ್ಕು ಬಗೆಯ ಜನರು ನನ್ನನ್ನು ಪೂಜಿಸುತ್ತಾರೆ. ಕೆಲವರು ಭೌತಿಕ ಜಗತ್ತಿನ ಸುಖವನ್ನು ಬಯಸುತ್ತಾರೆ, ಕೆಲವರು ಅರ್ಥವನ್ನು ಬಯಸುತ್ತಾರೆ, ಮತ್ತೆ ಕೆಲವರು ಧರ್ಮವನ್ನು ಬಯಸುತ್ತಾರೆ. ಇನ್ನು ಕೆಲವರು ನನ್ನನ್ನು ಪ್ರೀತಿಸುವುದರಿಂದ ಪೂಜಿಸುತ್ತಾರೆ.

ನಿಜವಾದ ಪ್ರೀತಿಯೆಂದರೆ ಪ್ರೀತಿಗಾಗಿ ಪ್ರೀತಿ. ನಾನು ಆರೋಗ್ಯವನ್ನಾಗಲಿ, ಅರ್ಥ ವನ್ನಾಗಲಿ, ಜೀವನವನ್ನಾಗಲಿ ಅಥವಾ ಮೋಕ್ಷವನ್ನಾಗಲಿ ಬಯಸುವುದಿಲ್ಲ. ನನ್ನನ್ನು ಎಷ್ಟೇ ನರಕಕ್ಕೆ ಬೇಕಾದರೂ ಕಳಿಸು. ಆದರೆ ನಾನು ನಿನ್ನನ್ನು ಪ್ರೀತಿಗಾಗಿ ಪ್ರೀತಿಸುವಂತಾಗಲಿ. ಮೀರಾಬಾಯಿಯು ಪ್ರೀತಿಗಾಗಿ ಪ್ರೀತಿ ಎಂಬ ತತ್ತ್ವವನ್ನು ಬೋಧಿಸಿದಳು.

ನಮ್ಮ ಈಗಿನ ಪ್ರಜ್ಞಾ ಸ್ತರವು ಅನಂತ ಚಿತ್ತಸಾಗರದ ಒಂದು ಸಣ್ಣ ಅಂಶ. ಈ ಪ್ರಜ್ಞೆಗೆ ನಿಮ್ಮನ್ನು ನೀವು ಸೀಮಿತಗೊಳಿಸಿಕೊಳ್ಳ ಬೇಡಿ.

ಜೀವದ ಅಭಿವೃದ್ಧಿಗಾಗಿ ಮೂರು ಅತ್ಯಂತ ಮುಖ್ಯ ಆವಶ್ಯಕತೆಗಳಿವೆ: (೧) ಮಾನವ ಜನ್ಮ (೨) ಮುಮುಕ್ಷುತ್ವ (೩) ಮಹಾಪುರುಷರ ಆಶ್ರಯ. ಆ ಮಹಾಪುರುಷನು ಮನೋವಾಕ್ಕಾಯವಾಗಿ ಸದ್ಗುಣ ಸಂಪನ್ನನಾಗಿರುತ್ತಾನೆ, ಅವನ ಒಂದೇ ಜೀವನೋದ್ದೇಶ ಜಗತ್ತಿಗೆ ಹಿತವನ್ನು ಉಂಟುಮಾಡುವುದಾಗಿರುತ್ತದೆ, ಅವನು ಇತರರ ಗುಣಗಳನ್ನೇ, ಅವು ಸಾಸಿವೆ ಕಾಳಿನಷ್ಟಿದ್ದರೂ, ಪರ್ವತದಂತೆ ನೋಡುತ್ತಾನೆ. ತನ್ನನ್ನು ವಿಶಾಲಗೊಳಿಸುತ್ತ ಇತರ ರಿಗೂ ವಿಶಾಲಗೊಳ್ಳಲು ಸಹಾಯ ಮಾಡುತ್ತಾನೆ.

‘ಯೋಗ’ ಎಂಬ ಶಬ್ದವು ಹಾಗೂ ಇಂಗ್ಲಿಷಿನ’ Yoke’ ಇವೆರಡೂ ಒಂದೇ ಧಾತುವಿನಿಂದ ಬಂದಿವೆ. ಇದರ ಅರ್ಥ ಕೂಡಿಸುವುದು ಎಂದು. ಆದ್ದರಿಂದ ಯೋಗ ಎಂದರೆ ಜೀವಾತ್ಮ ಮತ್ತು ಪರಮಾತ್ಮರ ಮಿಲನ.

ಈಗ ಯಾಂತ್ರಿಕವಾಗಿರುವ ಕ್ರಿಯೆಗಳೆಲ್ಲ ಹಿಂದೊಮ್ಮೆ ಪ್ರಜ್ಞಾಪೂರ್ವಕವಾಗಿದ್ದವು. ನಮ್ಮ ಮೊದಲನೆಯ ಸಾಧನೆಯೇ ಈ ಯಾಂತ್ರಿಕ ಕ್ರಿಯೆಗಳ ಜ್ಞಾನವನ್ನು ಪಡೆಯುವುದು. ಎಲ್ಲ ಯಾಂತ್ರಿಕ ಕ್ರಿಯೆಗಳನ್ನೂ ಸಚೇತನಗೊಳಿಸಿ ಪ್ರಜ್ಞಾಪೂರ್ವಕವಾಗಿ ಮಾಡು ವುದೇ ಇದರ ಉದ್ದೇಶವಾಗಿದೆ. ಅನೇಕ ಯೋಗಿಗಳು ತಮ್ಮ ಹೃದಯದ ಕ್ರಿಯೆಯನ್ನು ನಿಗ್ರಹಿಸ ಬಲ್ಲರು.

ಹಿಂದಿನ ಪ್ರಜ್ಞಾಸ್ಥಿತಿಗೆ ಹೋಗಿ, ನಾವು ಮರೆತಿರುವುದನ್ನು ಮತ್ತೆ ಹೊರತರುವುದು ಒಂದು ಸಾಮಾನ್ಯ ಶಕ್ತಿ. ಈ ಕ್ರಿಯೆಯನ್ನು ತ್ವರಿತಗೊಳಿಸಬಹುದು. ಪ್ರಜ್ಞಾಂತರಾಳದಲ್ಲಿರುವ ಎಲ್ಲ ಜ್ಞಾನವನ್ನೂ ಹೊರತರಲು ಸಾಧ್ಯ. ಇದೇ ಯೋಗ. ನಮ್ಮ ಹೆಚ್ಚಿನ ಕ್ರಿಯೆಗಳು ಹಾಗೂ ಆಲೋಚನೆಗಳು ಯಾಂತ್ರಿಕ ಅಥವಾ ನಮ್ಮ ಪ್ರಜ್ಞೆಯ ಹಿಂದೆ ನಡೆಯುತ್ತವೆ. ಯಾಂತ್ರಿಕ ಕ್ರಿಯೆಗಳ ಸ್ಥಾನವು ಮಿದುಳಿನ ಅತ್ಯಂತ ಹಿಂದಿನ ಚಾಚುಭಾಗ \enginline{(Medulla oblongata)} ಹಾಗೂ ಮಿದುಳು ಬಳ್ಳಿಗಳಾಗಿರುತ್ತವೆ.

ಹಿಂದಿನ ಪ್ರಜ್ಞಾ ಸ್ಥರಕ್ಕೆ ಹೋಗುವುದು ಹೇಗೆಂಬುದೇ ಪ್ರಶ್ನೆ. ನಾವು ಚೈತನ್ಯದಿಂದ ಮನಸ್ಸು ದೇಹಗಳ ಮೂಲಕ ಹೊರಬಂದಿರುವೆವು. ಈಗ ಪುನಃ ಚೈತನ್ಯಕ್ಕೆ ಹಿಂದಿರುಗಬೇಕಾಗಿದೆ. ಮೊದಲು ಉಸಿರಾಟವನ್ನು ನಿಗ್ರಹಿಸಬೇಕು, ಅನಂತರ ನರ ಮಂಡಲವನ್ನು ನಿಗ್ರಹಿಸಬೇಕು, ಅನಂತರ ಮನಸ್ಸಿನ ಮೇಲೆ ಹತೋಟಿಯನ್ನು ಸಾಧಿಸಬೇಕು. ಕೊನೆಯಲ್ಲಿ ಚೈತನ್ಯ ಅಥವಾ ಆತ್ಮನಲ್ಲಿ ನೆಲೆಗೊಳ್ಳಬೇಕು. ಈ ಪ್ರಯತ್ನದಲ್ಲಿ ಅತ್ಯುನ್ನತವಾದುದನ್ನೇ ಬಯಸುತ್ತೇನೆ ಎಂಬ ವಿಷಯದಲ್ಲಿ ನಾವು ಸಂಪೂರ್ಣ ಪ್ರಾಮಾಣಿಕರಾಗಿರಬೇಕು.

ನಿಯಮಗಳಲ್ಲೆಲ್ಲ ಶ್ರೇಷ್ಠವಾದು ಏಕಾಗ್ರತೆ. ಮೊದಲು ಎಲ್ಲ ನರಶಕ್ತಿಗಳನ್ನೂ ದೇಹದ ಜೀವಾಣುವಿನಲ್ಲಿ ಅಡಗಿರುವ ಎಲ್ಲ ಶಕ್ತಿಗಳನ್ನೂ ಕ್ರೋಢೀಕರಿಸಿ, ಇಚ್ಛಾಪೂರ್ವಕವಾಗಿ ಅದನ್ನು ಒಂದು ದಿಕ್ಕಿನಲ್ಲಿ ಹರಿಸಿ. ಅನಂತರ ಸೂಕ್ಷ್ಮ ದ್ರವ್ಯವಾದ ಮನಸ್ಸನ್ನು ಕೇಂದ್ರೀಕೃತಗೊಳಿಸಿ. ಮನಸ್ಸಿನಲ್ಲಿ ಅನೇಕ ಸ್ತರಗಳಿವೆ. ಕ್ರೋಢೀಕರಿಸಲ್ಪಟ್ಟ ನರಶಕ್ತಿಯನ್ನು ಸುಷುಮ್ನಾ ನಾಳದ ಮೂಲಕ ಹರಿಯಬಿಟ್ಟಾಗ ಮನಸ್ಸಿನ ಒಂದು ಸ್ತರವು ತೆರೆಯಲ್ಪಡುತ್ತದೆ. ಅದನ್ನು ಒಂದು ಚಕ್ರದಲ್ಲಿ ಏಕಾಗ್ರಗೊಳಿಸಿದಾಗ ಒಂದು ಭೂಮಿಕೆಯು ತೆರೆಯ ಲ್ಪಡುತ್ತದೆ. ಹೀಗೆ ಒಂದು ಭೂಮಿಯಿಂದ ಇನ್ನೊಂದು ಭೂಮಿಗೆ ಹೋಗುತ್ತ ಕೊನೆಯಲ್ಲಿ ಅದು ಸಹಸ್ರಾರವನ್ನು ತಲುಪುತ್ತದೆ. ಇದು ಸುಪ್ತ ಶಕ್ತಿಯ ಕೇಂದ್ರವಾಗಿದೆ, ಕ್ರಿಯೆ ನಿಷ್ಕ್ರಿಯೆಗಳೆರಡರ ಮೂಲಸ್ಥಾನವಾಗಿದೆ.

ಈ ಪ್ರಪಂಚದಲ್ಲಿಯೇ, ಈ ಜನ್ಮದಲ್ಲಿಯೇ ನಮ್ಮ ಎಲ್ಲ ಅನುಭವಗಳನ್ನೂ ಮುಗಿಸ ಬಲ್ಲೆವೆಂಬ ಭಾವನೆಯೊಂದಿಗೆ ಪ್ರಾರಂಭಿಸಿ. ಈ ಜನ್ಮದಲ್ಲಿಯೇ, ಈ ಕ್ಷಣದಲ್ಲಿಯೇ ಪರಿಪೂರ್ಣರಾಗಬೇಕೆಂಬ ಗುರಿಯನ್ನು ಹೊಂದಿರಬೇಕು. ಈ ಕ್ಷಣದಲ್ಲಿಯೇ ಸಾಧಿಸುತ್ತೇವೆಂಬ ಛಲವುಳ್ಳವರಿಗೆ ಯಶಸ್ಸು ಖಂಡಿತ. ‘ನಾನು ಶ್ರದ್ಧೆಯನ್ನೇ ಅವಲಂಬಿಸುತ್ತೇನೆ, ಏನೇ ಆಗಲಿ’ ಎಂದು ದೃಢವಾಗಿ ಹೇಳುವವನಿಗೆ ಪರಿಪೂರ್ಣತೆ ಸಿದ್ಧಿಸುತ್ತದೆ. ಆದ್ದರಿಂದ ಈ ಕ್ಷಣದಲ್ಲಿಯೇ ಮುಗಿಸಬೇಕೆಂದು ತಿಳಿದುಕೊಂಡು ಮುಂದುವರಿಯಿರಿ. ಹೋರಾಡಿ, ಹೋರಾಡಿ; ನೀವು ಯಶಸ್ವಿಯಾಗದಿದ್ದರೆ ಅದು ನಿಮ್ಮ ದೋಷವಲ್ಲ. ಜಗತ್ತು ನಿಮ್ಮನ್ನು ಸ್ತುತಿಸಲಿ ಅಥವಾನಿಂದಿಸಲಿ; ಭೂಮಿಯ ಐಶ್ವರ್ಯವೆಲ್ಲ ನಿಮ್ಮ ಪಾಲಿಗೆ ಬರಲಿ ಅಥವಾ ನೀವು ದರಿದ್ರಾತಿದರಿದ್ರರಾಗಲಿ; ಸಾವು ಈ ಕ್ಷಣವೇ ಬರಲಿ ಅಥವಾ ನೂರಾರು ವರ್ಷಗಳ ನಂತರ ಬರಲಿ; ನೀವು ಹಿಡಿದ ಮಾರ್ಗದಿಂದ ಚಲಿಸದಿರಿ. ಎಲ್ಲ ಸದ್ಭಾವನೆಗಳು ಅಮೃತಪ್ರಾಯವಾದವು, ಅವು ಬುದ್ಧರನ್ನೂ ಕ್ರಿಸ್ತರನ್ನೂ ನಿರ್ಮಿಸುತ್ತವೆ. ನಿಮ್ಮ ಮನಸ್ಸಿಗೆ ಗೋಚರವಾಗುವ ಘಟನೆಗಳನ್ನು ಅಭಿವ್ಯಕ್ತಗೊಳಿಸುವ ಸಾಧನಗಳೇ ನಿಯಮಗಳು. ಭೌತಿಕ ಘಟನೆಗಳನ್ನು ಗ್ರಹಿಸಿ ಅವುಗಳನ್ನು ಏಕತೆಗೆ ತರುವ ನಿಮ್ಮ ವಿಧಾನವೇ ನಿಯಮಗಳು. ಪ್ರತಿಯೊಂದು ನಿಯಮವೂ ವೈವಿಧ್ಯತೆಯಲ್ಲಿ ಏಕತೆಯನ್ನು ಕಾಣುವುದೇ ಆಗಿದೆ. ಭೌತಿಕ, ಮಾನಸಿಕ ಮತ್ತು ಆಧ್ಯಾತ್ಮಿಕ ಭೂಮಿಕೆಗಳಲ್ಲಿ ಮನಸ್ಸನ್ನು ಏಕಾಗ್ರಗೊಳಿಸುವುದರ ಮೂಲಕವೇ ಜ್ಞಾನವನ್ನು ಪಡೆಯಲಾಗುತ್ತದೆ. ಮನಸ್ಸಿನ ಶಕ್ತಿಯನ್ನು ಏಕಾಗ್ರಗೊಳಿಸಿ ಅನೇಕದರಲ್ಲಿ ಏಕವನ್ನು ಕಂಡುಹಿಡಿಯುವುದೇ ಜ್ಞಾನವಾಗಿದೆ.

ಏಕತೆಗೆ ಸಾಧನವಾಗುವುದೆಲ್ಲವೂ ನೈತಿಕವಾದುದು; ಭೇದಕ್ಕೆ ಕಾರಣವಾಗುವು ದೆಲ್ಲವೂ ಅನೈತಿಕವಾದುದು. ಅದ್ವಿತೀಯವಾದುದನ್ನು ತಿಳಿಯಿರಿ, ಅದೇ ಪರಿಪೂರ್ಣತೆ. ಎಲ್ಲದರಲ್ಲಿಯೂ ಅಭಿವ್ಯಕ್ತವಾಗುತ್ತಿರುವ ಸತ್ಯವೇ ಎಲ್ಲ ಅಸ್ತಿತ್ವದ ತಳಹದಿ. ಎಲ್ಲ ಧರ್ಮ, ಎಲ್ಲ ಜ್ಞಾನ ಈ ಕೇಂದ್ರದಲ್ಲಿ ಪರ್ಯವಸಾನವಾಗಬೇಕು.

(ಈ ಕೆಳಗಿನದು, ೧೮೯೪ರಲ್ಲಿ ಗ್ರೀನೇಕರ್ನಲ್ಲಿ ಭಾನುವಾರ ಬೆಳಗಿನ ತರಗತಿಯ ಸಂದರ್ಭದಲ್ಲಿ ಎಮ್ಮಾ ಥರ್ಸ್ಬಿ ತೆಗೆದುಕೊಂಡ ಹೊಂದಾಣಿಕೆಯಿಲ್ಲದ ವಾಕ್ಯಗಳ ಸಂಗ್ರಹ.)

\begin{myquote}
ನಾನು ಸಚ್ಚಿದಾನಂದ\\ಶಿವೋ ಣಿಹಂ, ಶಿವೋಣಿ ಹಂ
\end{myquote}

ಯಾರು ಇತರರ ಹಣವನ್ನು ತುಚ್ಛವಾಗಿಯೂ, ಎಲ್ಲ ಸ್ತ್ರೀಯರನ್ನೂ ತನ್ನ ತಾಯಿಯಾಗಿಯೂ ಕಾಣುತ್ತಾನೆಯೋ ಅವನೇ ಪಂಡಿತ. ಶಾಂತಿ.

\begin{myquote}
ಬೌದ್ಧ ಪ್ರಾರ್ಥನೆ:\\ಜಗತ್ತಿನ ಸಂತರೆಲ್ಲರಿಗೂ ನಾನು ನಮಿಸುತ್ತೇನೆ\\ಧರ್ಮಸಂಸ್ಥಾಪಕರಿಗೆ ನಾನು ನಮಿಸುತ್ತೇನೆ.\\ಎಲ್ಲ ಪವಿತ್ರಾತ್ಮರಿಗೂ ನಾನು ನಮಿಸುತ್ತೇನೆ.\\ಭೂಮಿಯ ಮೇಲಿದ್ದ ಎಲ್ಲ ಧರ್ಮದ ಪ್ರವಾದಿಗಳಿಗೂ ನಮಿಸುತ್ತೇನೆ.\\ಹಿಂದೂ ಪ್ರಾರ್ಥನೆ:\\ಈ ಜಗತ್ತಿನ ಸೃಷ್ಟಿಕರ್ತನನ್ನು ನಾನು ಧ್ಯಾನಿಸುತ್ತೇನೆ,\\ಅವನು ನಮ್ಮ ಬುದ್ಧಿಯನ್ನು ಜಾಗ್ರತಗೊಳಿಸಲಿ.
\end{myquote}

