
\chapter{ಅಧ್ಯಾಯ ೨: ನೈನಿತಾಲ್ ಮತ್ತು ಆಲ್ಮೋರಗಳಲ್ಲಿ}

ವ್ಯಕ್ತಿಗಳು: ಸ್ವಾಮಿ ವಿವೇಕಾನಂದರು, ಗುರುಭಾಯಿಗಳು, ಧೀರಮಾತಾ, ಜಯಾ ಎಂಬ ಹೆಸರಿನವಳು ಮತ್ತು ಸೋದರಿ ನಿವೇದಿತಾಳನ್ನೊಳಗೊಂಡ ಯೂರೋಪಿಯನ್ ಶಿಷ್ಯರುಗಳ ಮತ್ತು ಅತಿಥಿಗಳ ಗುಂಪು.

ಸ್ಥಳ: ಹಿಮಾಲಯ.

ಕಾಲ: ೧೮೯೮ರ ಮೇ ೧೧ರಿಂದ ಮೇ ೨೫ರವರೆಗೆ.

ಬುಧವಾರ ಸಂಜೆ ಹೌರಾ ನಿಲ್ದಾಣವನ್ನು ಬಿಟ್ಟು ಶುಕ್ರವಾರ ಬೆಳಗ್ಗೆ ಹಿಮಾಲಯದ ದರ್ಶನವಾಗುವರೆಗೆ ಪಯಣಿಸಿ ಬಂದ ನಮ್ಮ ಗುಂಪು ಸಾಕಷ್ಟು ದೊಡ್ಡದೇ ಆಗಿತ್ತು; ಎರಡು ಗುಂಪೆಂದೇ ಹೇಳಬಹುದು...

ಮೂರು ಸಂಗತಿಗಳು ನೈನಿತಾಲ್ನ್ನು ಸುಂದರವನ್ನಾಗಿಸಿದ್ದವು - ಗುರುದೇವರು ಸಂತೋಷದಿಂದ ತಮ್ಮ ಶಿಷ್ಯರಾದ ಖೇತ್ರಿ ಮಹಾರಾಜರನ್ನು ನಮಗೆ ಪರಿಚಯಿಸಿದ್ಧ; ನರ್ತಕಿಯರು ನಮ್ಮನ್ನು ಕಂಡು ಸ್ವಾಮಿಜಿ ಎಲ್ಲಿರುವರು ಎಂದು ವಿಚಾರಿಸಿದ್ದು, ಇತರರ ವಿರೋಧದ ನಡುವೆಯೂ ಸ್ವಾಮಿಗಳು ಅವರನ್ನು ಬರಮಾಡಿಕೊಂಡಿದ್ದು; ಮುಸ್ಲಿಂ ಸಭ್ಯಗೃಹಸ್ಥನೊಬ್ಬನು ಬಂದು “ಸ್ವಾಮೀಜಿ, ಮುಂದೆ ಯಾರಾದರೂ ನಿಮ್ಮನ್ನು ಭಗ ವಂತನ ವಿಶಿಷ್ಟ ಅವತಾರವೆಂದು ಉದ್ಘೋಷಿಸುವುದಾದರೆ, ಮುಸ್ಲಿಮನಾದ ನಾನೇ ಅವರಲ್ಲಿ ಮೊದಲಿಗನೆಂಬುದನ್ನು ನೆನಪಿಡಿ!” ಎಂದದ್ದು.

ರಾಜಾ ರಾಮಮೋಹನ ರಾಯ್​ ಬಗ್ಗೆ ಸ್ವಾಮಿಗಳಿಂದ ದೀರ್ಘವಾದ ಭಾಷಣವನ್ನು ನಾವು ಕೇಳಿದ್ದೂ ಇಲ್ಲಿಯೇ. ಈ ಪ್ರಬೋಧಕನ ಸಂದೇಶದಲ್ಲಿ ಎದ್ದು ಕಾಣುವ ಮೂರು ಅಂಶಗಳನ್ನು ಸ್ವಾಮಿಗಳು ಶ್ರುತಪಡಿಸಿದರು - ಆತ ವೇದಾಂತವನ್ನು ಒಪ್ಪಿಕೊಂಡಿದ್ದು, ಆತ ದೇಶಭಕ್ತಿಯನ್ನು ಬೋಧಿಸಿದ್ದು, ಮತ್ತು ಹಿಂದೂಗಳ ಸರಿಸಮಾನವಾಗಿಯೆ ಮುಸ್ಲಿಮರನ್ನೂ ಪ್ರೀತಿಸಿದ್ದು. ಸ್ವಾಮಿಗಳು ಈ ಎಲ್ಲ ಅಂಶಗಳಲ್ಲಿಯೂ ರಾಜಾ ರಾಮಮೋಹನ ರಾಯ್​ ಪ್ರತಿಪಾದಿಸಿದ್ದ ವಿಸ್ತೃತ ಭವಿಷ್ಯದೃಷ್ಟಿಯ ಕಾರ್ಯಕ್ರಮವನ್ನು ತಾವು ಕೈಗೆತ್ತಿಕೊಂಡಿರುವುದಾಗಿ ಸಾರಿದರು.

ನಾವು ಬೆಟ್ಟದ ಮೇಲೆ ಇದ್ದ ಎರಡು ದೇವಸ್ಥಾನಗಳಿಗೆ ಭೇಟಿಕೊಟ್ಟ ಪರಿಣಾಮವಾಗಿ ನಡೆದದ್ದು ನರ್ತಕಿಯರ ಪ್ರಸಂಗ... ಇಲ್ಲಿ ಪೂಜೆ ಸಲ್ಲಿಸುತ್ತಿದ್ದಾಗ ಇಬ್ಬರು ನರ್ತಕಿಯರು ನಮಗೆ ಕಾಣಿಸಿದರು. ತಮ್ಮ ನರ್ತನವನ್ನು ಮುಗಿಸಿ ನಮ್ಮ ಬಳಿಗೆ ಬಂದ ಅವರೊಡನೆ ನಾವು ಹರಕು-ಮುರುಕು ಭಾಷೆಯಲ್ಲಿ ಮಾತನಾಡತೊಡಗಿದೆವು. ಅವರನ್ನು ನಗರದ ಗೌರವಾನ್ವಿತ ಸ್ತ್ರೀಯರೆಂದುಕೊಂಡಿದ್ದ ನಾವು, ಅನಂತರ ಸ್ವಾಮಿಗಳು ಅವರನ್ನು ಅಟ್ಟಿಬಿಡಲು ನಿರಾಕರಿಸಿದಾಗ ಜನರಲ್ಲಿ ಎದ್ದ ಬಿರುಗಾಳಿಯನ್ನು ಕಂಡು ನಿಬ್ಬೆರ ಗಾದೆವು. ಅವರು ಈ ಮೊದಲೇ ಅನೇಕ ಬಾರಿ ನಮಗೆ ಹೇಳಿದ್ದ ಖೇತ್ರಿಯ ನರ್ತಕಿಯ ಪ್ರಸಂಗ ನೈನಿತಾಲ್ನ ಈ ನರ್ತಕಿಯರನ್ನು ಕುರಿತಾದದ್ದೇ ಇರಬಹುದೆಂದು ಯೋಚಿಸಿದ್ದು ಸರಿಯೆ? ನರ್ತನವನ್ನು ನೋಡಲು ಆಹ್ವಾನಿಸಲ್ಪಟ್ಟಾಗ ಸ್ವಾಮಿಗಳು ಸಿಟ್ಟಾಗಿದ್ದರು; ಆದರೆ ಒತ್ತಾಯದಿಂದಾಗಿ ಬರಬೇಕಾದಾಗ ಅವಳು ಹಾಡಿದ್ದಳು:

\begin{myquote}
ನೋಡದಿರೆನ್ನಯ ಅವಗುಣ, ಪ್ರಭುವೆ!\\ಸಮದರ್ಶಿಯು ನೀನಲ್ಲವೆ, ಹರಿಯೆ?\\ಬ್ರಹ್ಮನೊಳೊಂದಾಗಿಸು ಇಬ್ಬರನು!
\end{myquote}

\begin{myquote}
ಗುಡಿಯೊಳು ಮೂರ್ತಿಯು ಕಬ್ಬಿಣವೊಂದು\\ಕಟುಕನ ಕತ್ತಿಯು ಮತ್ತೊಂದು,\\ಸ್ಪರ್ಷಮಣಿ ತಾ ಸೋಂಕಿದರೆರಡೂ\\ಥಳಥಳಿಸುವೆ ತಾ ಹೊನ್ನಾಗಿ?
\end{myquote}

\begin{myquote}
ಯಮುನೆಯೊಳಿರುವುದು ನೀರಹನಿಯೊಂದು,\\ಕೊಚ್ಚೆಯೊಳಿರುವುದು ಮತ್ತೊಂದು;\\ಆ ಹನಿಯೆರಡೂ ಪಾವನವಾಗವೆ\\ಗಂಗೆಯ ಸೇರಲು ತೀರ್ಥದಲಿ?
\end{myquote}

\begin{myquote}
ಆದೊಡೆ, ಅವಗುಣ ನೋಡದೆ, ದೊರೆಯೆ,\\ಬ್ರಹ್ಮನೊಳೊಂದಾಗಿಸು ಇಬ್ಬರನು!\\ಸಮದರ್ಶಿಯು ನೀನಲ್ಲವೆ, ಹರಿಯೆ?
\end{myquote}

ಆಗ, ಗುರುದೇವರೇ ಹೇಳಿದಂತೆ, ಅವರ ದೃಷ್ಟಿಯಿಂದ ತರತಮಗಳು ಮಾಯವಾದುವಂತೆ; ನಿಜವಾಗಿಯೂ ಸಮಸ್ತ ಜನರೂ ಒಂದೇ ಎಂಬ ಭಾವ ತಾನೇ ತಾನಾಗಿ ಮೂಡಿತು. ಅಲ್ಲಿಂದ ಮುಂದೆ ಅವರು ಯಾರನ್ನೂ ಹೀಗಳೆಯದಾದರು...

ನಾವು ಆಲ್ಮೋರಕ್ಕೆಂದು ನೈನಿತಾಲ್ಬಿಟ್ಟು ಹೊರಟಾಗ ಸಂಜೆಯ ಬಿಸಿಲು ಇಳಿಮುಖವಾಗಿತ್ತು; ಕಾಡಿನ ನಡುವೆ ಪ್ರಯಾಣ ಮಾಡುತ್ತಿರುವಾಗಲೇ ರಾತ್ರಿಯಾಯಿತು... ಮರಗಳ ನಡುವೆ, ಪರ್ವತದ ತಪ್ಪಲಿನಲ್ಲಿ ವಿಲಕ್ಷಣವಾಗಿ ನೆಲೆಗೊಂಡಿದ್ದ ಡಾಕ್ ಬಂಗಲೆ ಯೊಂದನ್ನು ನಾವು ತಲುಪಿ ಸ್ಪಲ್ಪ ಹೊತ್ತಾದ ಮೇಲೆ ಸ್ವಾಮೀಜಿ ಅವರ ಗುಂಪಿನೊಂದಿಗೆ ಅಲ್ಲಿಗೆ ಬಂದರು; ಜೊತೆಗಿದ್ದ ತಮ್ಮ ಅತಿಥಿಗಳನ್ನು ಮುದಗೊಳಿಸುವ ಎಲ್ಲವನ್ನೂ ಗಮನಿಸುತ್ತ, ತಮಾಷೆಮಾಡುತ್ತ ಹಾಸ್ಯರಸವನ್ನು ಹೊರಸೂಸುತ್ತಿದ್ದರು...

ಆಲ್ಮೋರಕ್ಕೆ ಬಂದ ದಿನದಿಂದಲೇ ಸ್ವಾಮಿಗಳು ಬೆಳಗಿನ ತಿಂಡಿಯ ಹೊತ್ತಿಗೆ ನಮ್ಮೊಂದಿಗೆ ಬಂದು ಕುಳಿತು ಅನೇಕ ಗಂಟೆಗಳ ಕಾಲ ಮಾತಾಡುವ ತಮ್ಮ ಹಿಂದಿನ ಅಭ್ಯಾಸವನ್ನು ಪುನಃ ಪ್ರಾರಂಭಿಸಿದರು. ಯಾವಾಗಲೂ ಅವರನಿದ್ದೆ ತುಂಬ ಹಗುರವಾಗಿರುತ್ತಿದ್ದಿತು; ಸಾಮಾನ್ಯವಾಗಿ ಮುಂಜಾನೆ ಇನ್ನಿತರ ಸಂನ್ಯಾಸಿಗಳೊಡನೆ ತಮ್ಮ ಬೆಳಗಿನ ಪಥಸಂಚಲನವನ್ನು ಮುಗಿಸಿ ಹಿಂದಿರುಗುವಾಗ ನಮ್ಮೊಂದಿಗೆ ಬಂದು ಕಲೆಯುತ್ತಿದ್ದರು. ಅಪರೂಪವಾಗಿ ಕೆಲವೊಮ್ಮೆ ಸಂಜೆಯಲ್ಲೂ ಶತಪಥ ಹಾಕುತ್ತಿರುವ ಅವರನ್ನು ಕಾಣ ಬಹುದಾಗಿತ್ತು; ಅಥವಾ ಅವರು ಮತ್ತು ತಮ್ಮ ಜೊತೆಯವರು ತಂಗಿದ್ದ ಕ್ಯಾಪ್ಟನ್ ಸೆವಿಯರ್ ಇದ್ದಲ್ಲಿಗೆ ನಾವೇ ಹೋಗುತ್ತಿದ್ದೆವು. ಆಗ ಒಮ್ಮೆ ಅವರೂ ನಾವಿದ್ದಲ್ಲಿಗೆ ನಮ್ಮನ್ನು ಭೇಟಿಮಾಡಲು ಬಂದಿದ್ದರು.

ನೆನಪಿಸಿಕೊಳ್ಳಲು ನೋವಾಗುವ ಆದರೆ ಶೀಲಸಂವರ್ಧನೆಗೆ ಕಾರಣವಾಗು ವಂಥ ಒಂದು ಹೊಸ ವಿಚಿತ್ರ ಸಂಗತಿ ಆಲ್ಮೋರದ ಈ ಬೆಳಗಿನ ಸಂಭಾಷಣೆಯೊಂದಿಗೆ ಸೇರಿಕೊಂಡಿತು. ಕುತೂಹಲಕಾರಿ ಅಪನಂಬಿಕೆಯ ಕಹಿಯೊಂದು ಕಡೆ; ಕಿರಿಕಿರಿ ಹಾಗೂ ಅವಿಧೇಯತೆ ಇನ್ನೊಂದು ಕಡೆ ಇರುವಂತೆ ಕಂಡಿತು. ಈ ಕಾಲದಲ್ಲಿ ಸ್ವಾಮಿಗಳ ಇತ್ತೀಚಿನ ಶಿಷ್ಯಳಾಗಿದ್ದವಳು ಒಬ್ಬ ಆಂಗ್ಲ ಸ್ತ್ರೀ ಎಂಬುದನ್ನು ಸ್ಮರಿಸಿಕೊಳ್ಳಬೇಕು. ಬೌದ್ಧಿಕವಾಗಿ ಈ ಸಂಗತಿ ಎಷ್ಟರ ಮಟ್ಟಿಗೆ ಅರ್ಥವತ್ತಾದದ್ದು -ಬ್ರಿಟಿಷ್ ಜನಾಂಗವು ತಮ್ಮ ಆದರ್ಶ, ಇತಿಹಾಸ ಹಾಗೂ ಕೃತಿಗಳ ಹಿನ್ನೆಲೆಯಲ್ಲಿ ಭಾರತವನ್ನು ಅರ್ಥಮಾಡಿಕೊಳ್ಳುವ ವಿಚಾರದಲ್ಲಿ ಅದೆಂತಹ ಪೂರ್ವಗ್ರಹದ ಇರುವಿಕೆಯನ್ನು ಸೂಚಿಸುತ್ತಿತ್ತು, ಯಾವಾಗಲೂ ಸೂಚಿಸುತ್ತದೆ ಎನ್ನುವ ಕಲ್ಪನೆ ಸ್ವಾಮಿಗಳಿಗೆ ಮಠದಲ್ಲಿ ಅವಳ ದೀಕ್ಷೆಯಾಗುವ ದಿನದವರೆಗೆ ಇರಲಿಲ್ಲ. ಆಗ ಅವರು ಅವಳನ್ನು ಸಂಭ್ರಮದಿಂದ ಒಂದು ಪ್ರಶ್ನೆಯನ್ನು ಕೇಳಿದ್ದರು - ನೀನೀಗ ಯಾವ ದೇಶಕ್ಕೆ ಸೇರಿರುವೆ ಎಂದು. ಆಗ ಅವಳು ಕೊಟ್ಟ ಉತ್ತರ ಅವರನ್ನು ದಂಗುಬಡಿಸಿತು. ಬ್ರಿಟಿಷ್ ಧ್ವಜದ ಬಗ್ಗೆ ಅವಳು ಅದೆಂತಹ ನಿಷ್ಠೆ, ಪೂಜ್ಯ ಭಾವನೆಗಳನ್ನು -ಭಾರತೀಯ ಸ್ತ್ರೀಯೊಬ್ಬಳು ತನ್ನ ಗುರುವಿನ ಬಗ್ಗೆ ಇಟ್ಟುಕೊಳ್ಳ ಬಹುದಾದಂತಹ ಭಾವನೆಯನ್ನು -ಇಟ್ಟುಕೊಂಡಿರುವಳು ಎಂಬುದನ್ನು ಕಂಡು ಅವರಿಗೆ ಸೋಜಿಗವೆನಿಸಿತು. ಆ ಕ್ಷಣದಲ್ಲಿ ಅವರಿಗಾದ ನಿರಾಸೆ, ಅಚ್ಚರಿಗಳನ್ನು ಹೌದೋ ಅಲ್ಲವೋ ಎಂಬಂತೆ ಕಾಣ ಬಹುದಾಗಿತ್ತು. ಹೊರಗೆ ತೋರಿದ್ದು ಗಾಬರಿಯ ಒಂದು ದೃಷ್ಟಿ, ಅಷ್ಟೇ. ಈ ಶಿಷ್ಯೆ ಕೇವಲ ಮೇಲುಮೇಲಿನ ಭಾವನೆಯಿಂದ ತಮ್ಮ ಗುಂಪನ್ನು ಸೇರಿರುವುದನ್ನು ಅರಿತ ಮೇಲೂ ಅವರ ಆತ್ಮವಿಶ್ವಾಸವಾಗಲಿ ಶಿಷ್ಯಾದರಣೆಯಾಗಲಿ, ಹಿಮಾಲಯದಿಂದ ಕೆಳಗಿಳಿದು ಬಯಲುಪ್ರದೇಶಕ್ಕೆ ಬಂದ ನಂತರದ ವಾರಗಳಲ್ಲಿ ಒಂದಿನಿತೂ ಕಡಿಮೆಯಾಗಿರಲಿಲ್ಲ.

ಆದರೆ ಆಲ್ಮೋರದಲ್ಲಿದ್ದಾಗ ನಾವೆಲ್ಲ ಮತ್ತೊಮ್ಮೆ ಶಾಲೆಗೆ ಹೋಗಲಾರಂಭಿಸಿರು ವೆವೋ ಎಂಬಂತಿತ್ತು.... ಇದಕ್ಕಿಂತ ಹೆಚ್ಚೇನೂ ಇರಲಿಲ್ಲ; ಯಾವುದೇ ಅಭಿಪ್ರಾಯದ, ಮತಸಿದ್ಧಾಂತದ ಹೇರಿಕೆ ಇರಲಿಲ್ಲ; ಪಕ್ಷಪಾತನಿವಾರಣೆಗಿಂತ ಹೆಚ್ಚೇನೂ ಇರಲಿಲ್ಲ. ಈ ಭಯಾನಕ ಅನುಭವದ ಕೊನೆಯಲ್ಲಿ ಜನಾಂಗ, ರಾಷ್ಟ್ರ ಮೊದಲಾದ ಭಾವನೆಗಳನ್ನು ತ್ಯಜಿಸಿದ ಮೇಲೂ ಸ್ವಾಮಿಗಳು ಹೊಸ ಅಭಿಪ್ರಾಯದ ಅನು ಮೋದನೆಯನ್ನು, ಯಾವುದೇ ನಂಬಿಕೆಯ ಉದ್ಘೋಷಣೆಯನ್ನು ನಮ್ಮಿಂದ ಮಾಡಿಸಲಿಲ್ಲ. ಬದಲಿಗೆ ಇಡಿಯ ಪ್ರಶ್ನೆಯನ್ನೇ ಕೈಬಿಟ್ಟರು. ಅವರ ಶ್ರೋತೃ ಸ್ವತಂತ್ರರಾಗಿಯೇ ಇರುವಂತೆ ಇತ್ತು. ಆದರೆ ಭಾವನೆಯ ಮತ್ತು ಚಿಂತನೆಯ ಹೊಸದೊಂದು ನಿಲುವನ್ನು ಅನಾವರಣಗೊಳಿಸಿದ್ದರು; ಅದೆಷ್ಟು ಪರಿಪೂರ್ಣವಾಗಿತ್ತು, ಎಷ್ಟು ಶಕ್ತಿಯುತವಾಗಿತ್ತು ಎಂದರೆ, ತನ್ನದೇ ಪರಿಶ್ರಮದಿಂದ ಈ ಎರಡೂ ಪಕ್ಷಪಾತ ಧೋರಣೆಗಳನ್ನು ತನ್ನ ಮನ ವೊಪ್ಪುವ ಹಾಗೆ ವೈಚಾರಿಕವಾಗಿ ಸುಧಾರಣೆಮಾಡಿಕೊಳ್ಳುವವರೆಗೂ ಆಕೆಗೆ ವಿಶ್ರಾಂತಿಯೇ ಸಾಧ್ಯವಿಲ್ಲವಾಯಿತು.

ಕೆಲವು ವಾರಗಳ ನಂತರ, ಯಾವುದೋ ಘಟನೆಯ ಬಗೆಗಿನ ನಿಷ್ಪಕ್ಷಪಾತ ನ್ಯಾಯ ತೀರ್ಮಾನ ಸಾಧಾರಣವಾಗಿ ಇರುವುದಕ್ಕಿಂತಲೂ ಹೆಚ್ಚು ಉದ್ರೇಕಗೊಳಿಸುವಂತಿದ್ದಾಗ, “ನಿನ್ನಂತಹ ರಾಷ್ಟ್ರಭಕ್ತಿ ನಿಜವಾಗಿಯೂ ಒಂದು ಪಾಪ!” ಎಂದು ನುಡಿದುಬಿಟ್ಟರು. “ಜನರ ಕಾರ್ಯಗಳು ಬಹುಮಟ್ಟಿಗೆ ಸ್ವಾರ್ಥದ ಅಭಿವ್ಯಕ್ತಿಗಳಾಗಿರುತ್ತವೆ ಎನ್ನುವುದರ ಕಡೆಗೆ ನಿಮ್ಮ ಗಮನ ಸೆಳೆಯಲು ಇಚ್ಛಿಸುತ್ತೇನೆ. ಒಂದು ಜನಾಂಗದ ಎಲ್ಲರೂ ದೇವದೂತ ರುಗಳಂತೆ ಎಂಬಂತಹ ಕಲ್ಪನೆ ಇದಕ್ಕೆ ವಿರುದ್ಧವಾಗಿದೆ. ಇಂತಹ ಬಲವಾಗಿ ನಿಶ್ಚಯಗೊಂಡ ಅಜ್ಞಾನ ಕೆಡುಕಲ್ಲದೆ ಇನ್ನೇನು!”...

ಹೀಗೆ, ಆಲ್ಮೋರದ ಈ ಬೆಳಗಿನ ಸಂಭಾಷಣೆಗಳು ಸಾಮಾಜಿಕ, ಸಾಹಿತ್ಯಿಕ ಮತ್ತು ಕಲೆಯನ್ನು ಕುರಿತ, ಆಳವಾಗಿ ಬೇರೂರಿದ ಪೂರ್ವಗ್ರಹಗಳ ಮೇಲೆ ಆಕ್ರಮಣದ ರೂಪವನ್ನು ಪಡೆದುಕೊಂಡವು; ಭಾರತೀಯ ಹಾಗೂ ಯೂರೋಪಿಯನ್ ಭಾವನೆಗಳ, ಇತಿಹಾಸಗಳ ಮೇಲಣ ಬಹುಮೂಲ್ಯವಾದ, ದೀರ್ಘಾವಧಿಯ ಚರ್ಚೆಗಳಾಗತೊಡಗಿ ದವು. ಚರ್ಚೆಯ ನಡುವೆ ಇದ್ದಾಗ ಯಾವುದೇ ಸಮಾಜದ ಅಥವಾ ದೇಶವನ್ನು ಕುರಿತಾದ ಬೈಗುಳವನ್ನು ತೀವ್ರವಾಗಿ ವಿರೋಧಿಸಿ ಆಕ್ರಮಣ ಮಾಡುವುದು ಸ್ವಾಮಿಗಳ ಒಂದು ವಿಶಿಷ್ಟ ಅಭ್ಯಾಸವಾಗಿತ್ತು; ಆದರೆ ಅವರು ಹೊರಟು ಹೋದಮೇಲೆ ಅವರು ಕೇವಲ ಗುಣಗಳನ್ನು ಮಾತ್ರವೇ ಎತ್ತಿಹಿಡಿದಿರುವುದು ವೇದ್ಯವಾಗುತ್ತಿತ್ತು. ಅವರು ಶಿಷ್ಯ ರನ್ನು ಯಾವಾಗಲೂ ಪರೀಕ್ಷಿಸುತ್ತಿರುತ್ತಿದ್ದರು; ಬಹುಶಃ ಈ ವಿಶಿಷ್ಟ ರೀತಿಯ ಸಂಭಾಷಣೆಗಳನ್ನು, ಸ್ತ್ರೀಯರೂ ಯೂರೋಪಿಯನ್ನರೂ ಆಗಿರುವ ಶಿಷ್ಯೆಯರ ಪ್ರಾಮಾಣಿಕತೆ ಹಾಗೂ ಧೈರ್ಯವನ್ನು ಒರೆಗೆ ಹಚ್ಚುವುದಕ್ಕಾಗಿ ಉದ್ದೇಶಪೂರ್ವಕವಾಗಿ ರೂಪಿಸುತ್ತಿದ್ದರೆಂದು ತೋರುತ್ತದೆ.

