
\chapter{ಜಗತ್ತಿಗೆ ಕ್ರಿಸ್ತನ ಸಂದೇಶ}

(೧೯೦೦, ಮಾರ್ಚ್ ೧೧ರಂದು ಸ್ಯಾನ್ಫ್ರಾಂಸಿಸ್ಕೊದಲ್ಲಿ ನೀಡಿದ ಉಪನ್ಯಾಸದ ಟಿಪ್ಪಣಿ - ಫ್ರಾಂಕ್ ರೋಡ್ಹಾಮಲ್ರಿಂದ ತೆಗೆದುಕೊಂಡುದು.)

ಪ್ರತಿಯೊಂದೂ ಅಲೆಯೋಪಾದಿಯಲ್ಲಿ ಪ್ರಗತಿಹೊಂದುತ್ತದೆ. ನಾಗರಿಕತೆಯ ಮುನ್ನಡೆ, ಜಗತ್ತುಗಳ ಪ್ರಗತಿ ಅಲೆಯೋಪಾದಿಯಲ್ಲಿರುತ್ತದೆ. ಎಲ್ಲ ಮಾನವ ಕ್ರಿಯೆಗಳು - ಕಲೆ, ಸಾಹಿತ್ಯ, ವಿಜ್ಞಾನ, ಧರ್ಮ - ಅಲೆಯಂತೆ ಮುಂದುವರಿಯುತ್ತವೆ.

ಬೃಹತ್ ಅಲೆಗಳು ಒಂದಾದಮೇಲೊಂದರಂತೆ ಬರುತ್ತವೆ; ಈ ಅಲೆಗಳ ನಡುವೆ ಅಂತರವಿರುತ್ತದೆ, ಶಾಂತತೆಯಿರುತ್ತದೆ, ವಿಶ್ರಾಂತಿಯಿರುತ್ತದೆ.

ಎಲ್ಲ ಜೀವಿಗಳಿಗೂ ನಿದ್ರೆಬೇಕು, ವಿಶ್ರಾಂತಿಬೇಕು. ಈ ಅವಧಿಯಲ್ಲಿ ಅವು ಮುಂದಿನ ಕ್ರಿಯೆಗೆ ಹೊಸ ಶಕ್ತಿಯನ್ನೂ ಪಡೆಯುತ್ತವೆ. ಹೀಗೆಯೇ ಎಲ್ಲ ಪ್ರಗತಿಯೂ ಸಾಗುತ್ತದೆ, ಎಲ್ಲ ಜೀವಿಗಳೂ ಮುಂದುವರಿಯುತ್ತವೆ. ಎಲ್ಲವೂ ಕ್ರಿಯೆ ಮತ್ತು ವಿಶ್ರಾಂತಿಯ ಮೂಲಕ ಮುಂದುವರಿಯುತ್ತವೆ. ಅನಂತ ಪ್ರಗತಿಯ ಶ್ರೇಣಿಯಲ್ಲಿ ಒಂದಾದಮೇಲೊಂದರಂತೆ ಅಲೆಗಳು ಬರುತ್ತವೆ.

ಎಲ್ಲದರಂತೆಯೇ ಧರ್ಮವೂ ಕೂಡ ಅಲೆ ಅಲೆಯಾಗಿ ಮುಂದುವರಿಯುತ್ತದೆ. ಪ್ರತಿಯೊಂದು ಅಲೆಯ ತುದಿಯಲ್ಲಿಯೂ ಬೋಧಕನಾದ, ಮಹಾಶಕ್ತಿಶಾಲಿಯಾದ ಆಧ್ಯಾತ್ಮಿಕ ಮಹಾಪುರುಷನೊಬ್ಬನು ಇರುತ್ತಾನೆ. ಅಂತಹ ಒಬ್ಬ ಮಹಾಪುರುಷನೇ ನಜರತ್ ನ ಜೀಸಸ್.

