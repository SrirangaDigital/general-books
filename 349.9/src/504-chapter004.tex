
\chapter{ಅಧ್ಯಾಯ ೩: ಆಲ್ಮೋರದಲ್ಲಿ ಬೆಳಗಿನ ಸಂಭಾಷಣೆ}

ಸ್ಥಳ: ಆಲ್ಮೋರ.

ಕಾಲ: ೧೮೯೮ರ ಮೇ ಮತ್ತು ಜೂನ್.

ಮೊದಲಿನ ದಿನದ ಬೆಳಗಿನ ಸಂಭಾಷಣೆ - ಪ್ರಾಚ್ಯದಲ್ಲಿ ಶೀಲ, ಪಶ್ಚಿಮದಲ್ಲಿ ಸತ್ಯ - ಹೀಗೆ ನಾಗರಿಕತೆಯ ಕೇಂದ್ರ ಆದರ್ಶಗಳ ಬಗ್ಗೆ ನಡೆದಿತ್ತು. ಈ ಆದರ್ಶದ ಅನುಸಾರವಾಗಿಯೆ ಹಿಂದೂ ಮದುವೆಯ ಸಂಪ್ರದಾಯಗಳು ಹೊರಹೊಮ್ಮಿದುವೆಂದೂ ಜೊತೆಗೆ ಸ್ತ್ರೀರಕ್ಷಣೆಯ ಆವಶ್ಯಕತೆಯೂ ಸೇರಿದೆಯೆಂದೂ ಅವರು ಸಮರ್ಥಿಸುತ್ತಿದ್ದರು. ಅಲ್ಲದೆ ಈ ಇಡೀ ವಿಷಯಕ್ಕೂ ನಿರಪೇಕ್ಷ ತತ್ತ್ವಕ್ಕೂ ಹೇಗೆ ಸಂಬಂಧವಿದೆಯೆಂಬುದನ್ನು ಚಿತ್ರಿಸತೊಡಗಿದರು.

ಇನ್ನೊಂದು ದಿನ ಬೆಳಗ್ಗೆ ಅವರು ಬ್ರಾಹ್ಮಣ, ಕ್ಷತ್ರಿಯ, ಬನಿಯಾ (ವೈಶ್ಯ), ಶೂದ್ರ ರೆಂಬ ನಾಲ್ಕು ವರ್ಣಗಳು ಹೇಗೆ ಇವೆಯೋ ಹಾಗೆ, ನಾಲ್ಕು ಮಹತ್ತರವಾದ ರಾಷ್ಟ್ರೀಯ ಕ್ರಿಯೆಗಳೂ ಸಹ - ಹಿಂದೂಗಳಿಂದ ಸಾಂಗಗೊಳಿಸಲ್ಪಟ್ಟ ಧಾರ್ಮಿಕತೆ (ಅಥವಾ ಪೌರೋಹಿತ್ಯ), ರೋಮನ್ ಚಕ್ರಾಧಿಪತ್ಯದಿಂದ ಸೈನ್ಯ, ವರ್ತಮಾನದ ಇಂಗ್ಲೆಂಡ್ನಿಂದ ವ್ಯಾಪಾರ ಹಾಗೂ ಭವಿಷ್ಯದ ಅಮೆರಿಕಾದಿಂದ ಪ್ರಜಾಪ್ರಭುತ್ವ - ಇವೆ ಎಂದು ಪ್ರಾರಂಭ ಮಾಡಿದರು. ಇಲ್ಲಿ ಅವರು ಒಬ್ಬ ಪ್ರವಾದಿಯಂತೆ, ಹೇಗೆ ಅಮೆರಿಕಾ ದೇಶವು ಸಹಕಾರ ಸ್ವಾತಂತ್ರ್ಯಗಳೆಂಬ ಶೂದ್ರರ ಎರಡು ಸಮಸ್ಯೆಗಳನ್ನು ಪರಿಹರಿಸಲಿದೆ ಎಂಬುದನ್ನು ಕುರಿತು ತಮ್ಮ ಉಜ್ವಲ ಪ್ರತಿಭಾವಾಣಿಯನ್ನು ಮೆರೆಯುತ್ತ, ಅಮೆರಿಕಾದವನಲ್ಲದ ಶ್ರೋತೃವೊಬ್ಬರ ಕಡೆಗೆ ತಿರುಗಿ, ಆ ಜನರು ತಮ್ಮ ಮೂಲ ನಿವಾಸಿಗಳಿಗಾಗಿ ಉದಾರವಾಗಿ ಕಲ್ಪಿಸಿಕೊಡಲೆತ್ನಿಸುತ್ತಿರುವ ಸೌಲಭ್ಯಗಳನ್ನು ಉದಾಹರಿಸತೊಡಗಿದರು.

ಅದು ಮೊಘಲ್ ಚರಿತ್ರೆಯ ಅಥವಾ ಭಾರತ ಇತಿಹಾಸದ ಸಾರ ಸಂಗ್ರಹವೇ ಆಗುತ್ತಿತ್ತು. ಅದರ ಹಿರಿಮೆ ಅವರಿಗೆಂದಿಗೂ ಬೇಸರ ತರುತ್ತಿರಲಿಲ್ಲ. ಬೇಸಿಗೆ ಯುದ್ದಕ್ಕೂ ಮತ್ತೆ ಮತ್ತೆ ಅವರು ದೆಹಲಿ ಆಗ್ರಾಗಳ ವರ್ಣನೆ ಮಾಡಿದ್ದೂ ಮಾಡಿದ್ದೇ. ಒಮ್ಮೆಯಂತೂ ಅವರು ತಾಜ್ ಮಹಲನ್ನು ಕುರಿತು “ಮಬ್ಬುಮುಸುಕು, ಮತ್ತೆಯೂ ಮಬ್ಬು, ಆಹಾ ಹೇಳುವುದೇನು - ಒಂದು ಗೋರಿ!” ಎಂದರು.

ಇನ್ನೊಂದು ಸಲ ಷಹಜಹಾನ್ ಬಗ್ಗೆ ಮಾತನಾಡುತ್ತ ಇದ್ದಕ್ಕಿದ್ದಂತೆ ಸ್ಪೋಟಿಸಿದ ಉತ್ಸಾಹದಿಂದ “ಆಹಾ! ವಂಶದ ಜಯಕೀರ್ತಿ ಅವನು! ಚರಿತ್ರೆಯಲ್ಲಿ ಕಾಣಸಿಗದ ಅದ್ವಿತೀಯ ಸೌಂದರ್ಯಾರಾಧಕ, ಸೌಂದರ್ಯವಿವೇಚಕ - ಸ್ವಯಂ ಕಲಾರಸಿಕ! ಚಿರಸ್ಥಾಯಿಯಾಗಿಸಿರುವ ಅವನ ಕೈಬರಹವನ್ನು ನಾನು ನೋಡಿರುವೆ. ಭಾರತದ ಕಲೆಯ ಅಮೂಲ್ಯ ಸಿರಿ ಅದು. ಎಂತಹ ಪ್ರತಿಭಾಶಾಲಿ!” ಎಂದು ಉದ್ಗರಿಸಿದರು.

ಅವರು ಕಣ್ಣಿನಲ್ಲಿ ನೀರು ತುಂಬಿಕೊಂಡು, ಸುಲಭವಾಗಿ ಅರ್ಥಮಾಡಿಕೊಳ್ಳ ಬಹುದಾದ ಭಾವಾವೇಶದಿಂದ ಇನ್ನೂ ಪದೇ ಪದೇ ಹೇಳುತ್ತಿದ್ದುದು ಅಕ್ಬರನ ಬಗ್ಗೆ. ಅಂತೆಯೇ ಆಗ್ರಾದಲ್ಲಿ ಮೇಲೊಂದು ಗುಮ್ಮಟವಿಲ್ಲದೆ ಗಾಳಿ-ಬಿಸಿಲುಗಳಿಗೆ ಒಡಿಕೊಂಡಿರುವ ಸಿಕಂದರನ ಗೋರಿಯ ಬಗ್ಗೆಯೂ ಸಹ.

ಆದರೆ ಎಲ್ಲಕ್ಕಿಂತ ಮಿಗಿಲಾಗಿ ಮಾನವ ಭಾವನೆಗಳ ವಿಶ್ವರೂಪದ ಆದ್ಯಂತವೂ ಗುರುದೇವರಿಗೆ ಗೊತ್ತಿತ್ತು. ಒಮ್ಮೆ ಭಾವವಶರಾಗಿ ಚೈನಾ ಬಗ್ಗೆ ಅದು ಪ್ರಪಂಚದ ಸಂಪತ್ತಿನ ಭಂಡಾರವೆಂಬಂತೆ ಮಾತನಾಡಿದರು; ಚೈನಾದ ದೇಗುಲಗಳ ಹೆಬ್ಬಾಗಿಲಿನಲ್ಲಿ ತಾವು ಹಳೆಯ ಬಂಗಾಳಿ (ಕುಟಿಲ್?) ಅಕ್ಷರಗಳನ್ನು ನೋಡಿ ರೋಮಾಂಚಿತರಾದುದನ್ನು ಸ್ಮರಿಸಿಕೊಂಡರು.

ಪೌರ್ವಾತ್ಯರ ಬಗ್ಗೆ ಪಾಶ್ಚಾತ್ಯರ ಅಭಿಪ್ರಾಯಗಳಲ್ಲಿದ್ದ ಅಸ್ಪಷ್ಟತೆ ಸಲೀಸಾಗಿತ್ತು. ಅವರ ಶ್ರೋತೃವೊಬ್ಬರು ಪೌರ್ವಾತ್ಯ ಜನಾಂಗದವರಲ್ಲಿ ಅಪ್ರಾಮಾಣಿಕತೆ ಎಂಬುದೊಂದು ಕುಖ್ಯಾತ ಗುಣ ಎಂದು ಆಪಾದಿಸಿದರು... ಸ್ವಾಮಿಗಳಿಗೆ ಇದನ್ನು ಸಹಿಸಲಾಗಲಿಲ್ಲ. ಅಸತ್ಯ! ಸಾಮಾಜಿಕ ನಿಷ್ಠುರತೆ! ಇವು ತೀರ, ಸಾಪೇಕ್ಷ ಸಂಗತಿಗಳಲ್ಲದೆ ಮತ್ತೇನು? ಅಪ್ರಾಮಾಣಿಕತೆಯ ಬಗ್ಗೆ ಹೇಳುವುದಾದರೆ, ಮನುಷ್ಯರು ಮನುಷ್ಯರನ್ನು ನಂಬದಿದ್ದರೆ ಸಾಮಾಜಿಕ ಬದುಕು, ವಾಣಿಜ್ಯ ವ್ಯವಹಾರ, ಯಾವುದೇ ರೂಪದ ಸಹಕಾರ ಇತ್ಯಾದಿಗಳು ಒಂದೇ ಒಂದು ದಿನ ನಡೆಯಲು ಸಾಧ್ಯವೆ? ಅಸತ್ಯವೆಂಬುದು ಸಾಮಾಜಿಕ ಶಿಷ್ಟಾಚಾರಕ್ಕೆ ಆವಶ್ಯಕವೆ? ಇದು ಪಾಶ್ಚಾತ್ಯ ಕಲ್ಪನೆಗಿಂತ ಹೇಗೆ ಭಿನ್ನ? ಇಂಗ್ಲಿಷ್ ಮನುಷ್ಯ ಯಾವಾಗಲೂ ತಾನು ತೋರಿಸಿಕೊಳ್ಳುವಂತೆ ನಿಜವಾಗಿ ವಿಷಾದಿಸುವನೆ ಹಾಗೂ ಸಂತೋಷಪಡುವನೆ? ಆದರೂ ತರತಮದಲ್ಲಿ ವ್ಯತ್ಯಾಸವಿದೆ ಎನ್ನು ತ್ತೀರಾ? ಇರಬಹುದು - ಕೇವಲ ತರತಮದಲ್ಲಿ ಮಾತ್ರ!

ಅಥವಾ ಹೀಗೆಯೇ ಮಾತನಾಡುತ್ತ ಇಟಲಿಯವರೆಗೂ ಹೋಗುವರು; “ಯೂರೋಪಿನ ದೇಶಗಳಲ್ಲೆಲ್ಲ ಅತಿ ಮಹತ್ವದ್ದು - ಧರ್ಮದ, ಕಲೆಯ ನೆಲೆವೀಡು; ಅಂತೆಯೇ ಮ್ಯಾಜಿ ನಿಯ, ಸೌರ್ವಭೌಮ ಸಂಘಟನೆಯ ದೇಶ; ಕಲ್ಪನೆಗಳ, ಸಂಸ್ಕೃತಿಯ, ಸ್ವಾತಂತ್ರ್ಯದ ತಾಯಿ!” ಎನ್ನುವರು.

ಒಂದು ದಿನ ಮಾತನಾಡುತ್ತ ತಾವು ಸಂನ್ಯಾಸಿಯಾಗಿ ವರ್ಷವೆಲ್ಲ ಸಂಚರಿಸುತ್ತ ಕೊನೆಗೆ ರಾಯಘಡದಲ್ಲಿ ತಂಗಿದ್ದು, ಮರಾಠರು, ಶಿವಾಜಿ ಮುಂತಾದವರನ್ನು ಸ್ಮರಿಸಿಕೊಂಡರು. “ಭಾರತದಲ್ಲಿ ಸಂನ್ಯಾಸಿಯಾದವನು ಕಾವಿವಸ್ತ್ರದ ಮರೆಯಲ್ಲಿ ಅಭಿನವ ಶಿವಾಜಿಯಂತೆ ಇರದಿದ್ದರೆ ಅಧಿಕಾರಶಾಹಿಗೆ ಹೆದರಬೇಕಾಗುತ್ತದೆ!” ಎಂದರು.

ಆರ್ಯರು ಯಾರು, ಎಂತು ಎಂಬ ಪ್ರಶ್ನೆ ಅವರ ಗಮನವನ್ನು ಆಗಾಗ ಸೆಳೆ ಯುತ್ತಿತ್ತು. ಅವರ ಮೂಲ ಸಂಕೀರ್ಣವಾದುದು ಎನ್ನುತ್ತ, ಸ್ವಿಟ್ಸರ್ಲೆಂಡ್ನಲ್ಲಿದ್ದಾಗ ಜನ ರನ್ನು ನೋಡಿ ಹೇಗೆ ತಾನು ಚೈನಾದಲ್ಲಿರುವೆನೋ ಎಂದೆನಿಸಿತು, ಹೇಗೆ ಅಷ್ಟೊಂದು ಹೋಲಿಕೆಯಿತ್ತು ಎಂದು ನೆನಪಿಸಿಕೊಳ್ಳುವರು; ನಾರ್ವೆಯ ಕೆಲವು ಭಾಗಗಳಲ್ಲಿ ದ್ದಾಗಲೂ ಹಾಗೆಯೇ ಅನ್ನಿಸಿತಂತೆ. ಅನಂತರ ದೇಶಗಳು ಹಾಗೂ ಜನಗಳ ಬಾಹ್ಯ ರೂಪದ ಬಗ್ಗೆ ಇನ್ನೂ ಅನೇಕ ಸಂಗತಿಗಳನ್ನು ಹೇಳಿದರು; ಹೂಣರು ಟಿಬೆಟ್ಗೆ ಬಂದಿ ದ್ದುದನ್ನು ಗುರುತಿಸಿದ ಹಂಗೇರಿಯನ್ ವಿದ್ವಾಂಸರೊಬ್ಬರ ಗೋರಿ ಡಾರ್ಜಿಲಿಂಗ್ನಲ್ಲಿರುವುದರ ಕತೆ ಇತ್ಯಾದಿಗಳನ್ನು ನಿರ್ಭಾವುಕವಾಗಿ ತಿಳಿಸಿದರು....

 ಕೆಲವೊಮ್ಮೆ ಸ್ವಾಮಿಗಳು ಭಾರತದ ಇಡೀ ಚರಿತ್ರೆಯನ್ನು ಬ್ರಾಹ್ಮಣ ಕ್ಷತ್ರಿಯರ ನಡುವಣ ಹೋರಾಟ ಎಂಬಂತೆ ಚಿತ್ರಿಸುತ್ತ, ದೇಶದ ಬಂಧಮೋಚನೆಯ ಕುರುಹೆಂಬಂತೆ ಹೇಗೆ ಕ್ಷತ್ರಿಯರೇ ಯಾವಾಗಲೂ ಗೆಲವು ಸಾಧಿಸುತ್ತಿದ್ದರು ಎಂದು ವಿವರಿಸುವರು. ಇಂದಿನ ಬಂಗಾಳದ ಕಾಯಸ್ಥರು ಮೌರ್ಯರಿಗಿಂತಲೂ ಮೊದಲಿದ್ದ ಕ್ಷತ್ರಿಯರ ಕುಲದ ವರಾಗಿರುವರು ಎಂಬ ತಮ್ಮ ನಂಬಿಕೆಯನ್ನು ಅತ್ಯುತ್ತಮ ಕಾರಣಗಳಿಂದ ಸಾಧಿಸುವರು. ಪರಸ್ಪರ ವಿರುದ್ಧವಾದ ಎರಡು ಸಂಸ್ಕೃತಿ ರೂಪಗಳನ್ನು ಚಿತ್ರಿಸುವರು - ಒಂದು, ಪಾರಂಪರಿಕವಾದ, ಸಾಂಪ್ರದಾಯಿಕವಾದ, ಗಾಢವಾಗುತ್ತಲೇ ಮುನ್ನಡೆಯುವ ರೂಢಿಗತ ಪದ್ಧತಿಗಳಿಂದ ಕೂಡಿದ್ದು; ಇನ್ನೊಂದು ಸೆಣೆಸಿ ನಿಲ್ಲುವ, ಔದಾರ್ಯದ, ಭಾವತೀವ್ರತೆಯ ದೃಷ್ಟಿಯುಳ್ಳದ್ದು. ರಾಮ, ಕೃಷ್ಣ ಮತ್ತು ಬುದ್ಧ - ಈ ಮೂವರು ಬ್ರಾಹ್ಮಣರಾಗಿ ಹುಟ್ಟದೆ ರಾಜಮನೆತನಗಳಲ್ಲಿ ಕ್ಷತ್ರಿಯರಾಗಿ ಹುಟ್ಟಿದ್ದು ಚಾರಿತ್ರಿಕ ಮುನ್ನಡೆಯ ಹಾದಿಯಲ್ಲಿ ನಿಗೂಢವಾಗಿ ಸ್ಥಾಯಿಯಾಗಿರುವ ನಿಯಮದಿಂದಾಗಿ. ಇಂತಹ ವಿರೋಧಾಭಾಸದ ಸಂದರ್ಭದಲ್ಲಿ ‘‘ಕ್ಷತ್ರಿಯರಿಂದ ಆವಿಷ್ಕರಿಸಲ್ಪಟ್ಟ ಧರ್ಮ’’ವಾದ ಬೌದ್ಧ ಧರ್ಮವನ್ನು ಬ್ರಾಹ್ಮಣರನ್ನು ಹತ್ತಿಕ್ಕಲೆಂದೇ ಹುಟ್ಟಿದ ಜಾತಿವಿನಾಶಕ ಧರ್ಮ ಎಂದು ಹೀಗಳೆಯಲಾಯಿತು!

 ಬುದ್ಧನ ಬಗ್ಗೆ ಅವರು ಮಾತನಾಡುತ್ತಿದ್ದ ಸಂದರ್ಭಗಳು ತುಂಬ ಮಹತ್ವದ್ದವಾಗಿದ್ದವು. ಬ್ರಾಹ್ಮಣವಿರೋಧಿ ಭಾವನೆ ಎಂಬಂತೆ ತೋರುತ್ತಿದ್ದ ಯಾವುದೋ ಒಂದು ಮಾತಿನಿಂದಾಗಿ ಅವರನ್ನು ಸರಿಯಾಗಿ ಅರ್ಥಮಾಡಿಕೊಳ್ಳಲಾರದೆ ಹೋದ ಶ್ರೋತೃವೊಬ್ಬರು ‘‘ಸ್ವಾಮಿ, ನೀವು ಬೌದ್ಧರೆನ್ನುವುದು ನನಗೆ ಗೊತ್ತಿರಲಿಲ್ಲ!’’ ಎಂದರು.

ಆ ಹೆಸರಿನ ಸ್ಫೂರ್ತಿಯಿಂದಾಗಿ ಪ್ರಜ್ವಲಗೊಂಡು ತೇಜಸ್ವಿಯಾಗಿ ಪ್ರಕಾಶಿಸುತ್ತಿದ್ದ ಸ್ವಾಮಿಗಳು ಅವಳ ಕಡೆಗೆ ತಿರುಗಿ ಹೀಗೆಂದರು: ‘‘ತಾಯಿ, ನಾನು ಬುದ್ಧನ ಸೇವಕರ ಸೇವಕರ ಸೇವಕ. ಆತನ ಹಾಗೆ ಇದ್ದವರಾದರೂ ಯಾರು? ತನಗಾಗಿ ಒಂದೇ ಒಂದಾದರೂ ಕಾರ್ಯ ಮಾಡದ ಭಗವಂತ ಅವನು - ಇಡೀ ಪ್ರಪಂಚವನ್ನೇ ಆಲಿಂಗಿಸಿದ ಹೃದಯ ಅವನದು! ಎಷ್ಟು ದಯಾಪೂರ್ಣನವನು - ರಾಜಕುಮಾರ ಹಾಗೂ ಸಂನ್ಯಾಸಿ -ಒಂದು ಮೇಕೆಯ ಮರಿಯನ್ನು ಉಳಿಸುವಸಲುವಾಗಿ ತನ್ನ ಜೀವನವನ್ನೇ ಬಲಿಕೊಡಲು ಸಿದ್ಧನಾದವನು! ಅದೆಷ್ಟು ಪ್ರೇಮಮಯಿ ಎಂದರೆ ಹೆಣ್ಣು ಹುಲಿಯೊಂದರ ಹಸಿವನ್ನು ತಣಿಸಲು ತನ್ನ ಬದುಕನ್ನೇ ತ್ಯಾಗಮಾಡಲು ಅನುವಾದವನು - ಪರಯನೊಬ್ಬನ ಸತ್ಕಾರವನ್ನು ಸ್ವೀಕರಿಸಿ ಹೃತ್ಪೂರ್ವಕವಾಗಿ ಆಶೀರ್ವದಿಸಿದನು! ನಾನು ಬಾಲಕನಾಗಿದ್ದಾಗ ನಾನಿದ್ದ ಕೊಠಡಿಗೆ ಬಂದ ಅವನ ಪಾದಗಳಿಗೆ ನಾನು ನಮಸ್ಕರಿಸಿರುವೆನು! ಏಕೆಂದರೆ ನನಗೆ ಗೊತ್ತಿತ್ತು, ಆತ ಸಾಕ್ಷಾತ್ ಭಗವಂತನೇ ಎಂದು!’’

ಬುದ್ಧನ ಬಗ್ಗೆ ಈ ಲಹರಿಯಲ್ಲಿ ಅವರು ಮಾತನಾಡಿದ್ದು ಅದೆಷ್ಟೋ ಬಾರಿ - ಕೆಲವೊಮ್ಮೆ ಬೇಲೂರಿನಲ್ಲಿದ್ದಾಗ, ಇನ್ನು ಕೆಲವೊಮ್ಮೆ ಅನಂತರದಲ್ಲಿ. ಒಮ್ಮೆ ಅವರು ಆತನಿಗೆ ಉಣಬಡಿಸಿದ ಸುಂದರ ವೇಶ್ಯಾಂಗನೆ ಅಂಬಾಪಾಲಿಯ ಕಥೆಯನ್ನು ಹೇಳಿದರು...

ರಾಷ್ಟ್ರೀಯ ಭಾವನೆಯೇ ಯಾವಾಗಲೂ ತಾನೇ ತಾನಾಗಿ ವಿಜೃಂಭಿಸುತ್ತಿರಲಿಲ್ಲ. ಏಕೆಂದರೆ ಅಭಾವವು ವಿಸ್ತೃತವಾಗಿ ತೋರಿದ ಒಂದು ಬೆಳಗ್ಗೆ ಭಕ್ತಿಯನ್ನು ಕುರಿತ ಸಂಭಾಷಣೆ ದೀರ್ಘವಾಗಿ ನಡೆಯಿತು. ಚೈತನ್ಯದೇವನಿಗಿಂತಲೂ ಮುನ್ನಿನ ಕುಲೀನ ಬಂಗಾಳಿ ರಾಯ್​ ರಮಾನಂದನೆಂಬ ಭಕ್ತ ಸುಂದರವಾಗಿ ಚಿತ್ರಿಸಿರುವ ಪ್ರೇಮಿಯೊಂದಿಗಿನ ಪರಿಪೂರ್ಣ ಐಕ್ಯವನ್ನು ಕುರಿತು ಹೇಳಿದರು:

\begin{myquote}
ಬೆರೆತವಾ ಅಕ್ಷಿಗಳು ನಾಲ್ಕು. ಬದಲಾದವೆರಡು ಆತ್ಮಗಳು.\\ನೆನಪಿಲ್ಲ ನನಗೀಗ ನಾನು ಸ್ತ್ರೀ ಆತ ಪುರುಷನೋ\\ಅಥವಾ ನಾ ಪುರುಷ ಆತ ಸ್ತ್ರೀಯಾಗಿದ್ದೆವೊ ಎಂಬುದು!\\ಗೊತ್ತಿರುವುದಿಷ್ಟೆ ನಾವಿಬ್ಬರಿದ್ದೆವು ಪ್ರೇಮ ಕೊನರಿತು\\ಈಗ ಇರುವುದು ಒಬ್ಬರೆ!
\end{myquote}

ಅದೇ ದಿನ ಬೆಳ್ಳಗ್ಗೆ ಅವರು ಪರ್ಷಿಯಾದ ಬಾಬಿ ಪಂಥದ ಅನುಯಾಯಿಗಳನ್ನು, ಅವರ ಹುತಾತ್ಮ ಕಾಲವನ್ನು ಕುರಿತು ಸ್ಫೂರ್ತಿಯಿತ್ತ ಸ್ತ್ರೀ ಮತ್ತು ಆರಾಧಕ ಕೆಲಸಾರ ಪುರುಷನ ಕುರಿತು ಮಾತನಾಡಿದರು. ಎಳೆವಯಸ್ಸಿನವರ ಹಿರಿಮೆಯನ್ನು ಎತ್ತಿಹಿಡಿಯುತ್ತ, ವೈಯಕ್ತಿಕ ಅಭಿವ್ಯಕ್ತಿಯನ್ನು ಬಯಸದೆ ಪ್ರೇಮಿಸಬಲ್ಲ ಅಂಥವರ ಒಳ್ಳೆಯತನದ ಮತ್ತು ಅವರಲ್ಲಿ ಸುಪ್ತವಾಗಿರುವ ಶಕ್ತಿಯನ್ನು ಕುರಿತು ಹೇಳಿದರು. ಹೊಸ ಪರಿಚಯದವರಿಗೆ ವಿಷಯಕ್ಕಿಂತ ಖಚಿತತೆಯಿಂದಾಗಿಯೇ ವಿಚಿತ್ರವೆನಿಸಿ ಅಚ್ಚರಿಯುಂಟುಮಾಡಬಹುದಾದ ತಮ್ಮದೇ ಆದವಾದದ ಬಗ್ಗೆ ಅವರು ದೀರ್ಘವಾಗಿ ವ್ಯಾಖ್ಯಾನ ಮಾಡಿದ್ದು ಆಗಲೇ.

ಇನ್ನೊಂದು ದಿನ, ಸೂರ್ಯೋದಯದ ಹೊತ್ತಿಗೆ, ಇನ್ನೂ ಶಿಖರಗಳ ಮೇಲಣ ಮಂಜು ಉಷಾಕಿರಣಗಳಿಂದ ಕಂಗೊಳಿಸುತ್ತಿರುವಾಗ, ತೋಟದ ಕಡೆಯಿಂದ ಬಂದ ಅವರು ಶಿವ ಉಮೆಯರ ಕುರಿತಾಗಿ ಮಾತನಾಡತೊಡಗಿದರು - ಶಿಖರದ ಮೇಲಣ ಮಂಜು ಶಿವ ಮತ್ತು ಅದನ್ನು ಹೊಳೆಹೊಳೆಯುವಂತೆ ಬೆಳಗುತ್ತಿರುವ ಉಷಾಕಿರಣವೇ ಜಗನ್ಮಾತೆ ಎಂದರು! ಏಕೆಂದರೆ ಆ ಕಾಲದಲ್ಲಿ ಅವರ ಮನಸ್ಸುನ್ನು ತುಂಬಿಕೊಂಡಿದ್ದುದು ದೇವರೇ ವಿಶ್ವವಾಗಿರುವುನು ಎಂಬ ಭಾವ - ವಿಶ್ವದ ಒಳಗೂ ಅಲ್ಲ, ಹೊರಗೂ ಅಲ್ಲ ಅವನಿರುವುದು, ವಿಶ್ವವು ಆತನ ಬಿಂಬವೂ ಅಲ್ಲ, ಅವನೇ ಎಲ್ಲವೂ; ಈ ವಿಶ್ವವಾಗಿರುವದು ಅವನೇ ಎಂಬಂತಹ ಭಾವ.

ಕೆಲವೊಮ್ಮೆ ಇಡೀ ಬೇಸಿಗೆಯ ಉದ್ದಕ್ಕೂ ಗಂಟೆಗಟ್ಟಲೆ ಕುಳಿತು ಅವರು ಕಥೆಗಳನ್ನು ಹೇಳುವರು - ಹಳೆಯ ಗ್ರೀಕ್ ಪ್ರಪಂಚದ ಮಾನವನಿರ್ಮಾಣದ ಪುರಾಣ ಕಥೆಗಳನ್ನು ಹೆಚ್ಚಾಗಿ ಹೋಲುವ ಹಿಂದೂಧರ್ಮದ ಅಜ್ಜಿಕಥೆಗಳನ್ನು. ನಮ್ಮ ಕಡೆಯ ಶಿಶುವಿಹಾರದ ಕಥೆಗಳನ್ನು ಅವು ಹೋಲುತ್ತಿರಲಿಲ್ಲ. ಇವುಗಳಲ್ಲಿ ಅತ್ಯುತ್ತಮವೆಂದು ನನಗನಿಸಿದ್ದು ಶುಕನ ಕಥೆ; ಅದನ್ನು ಮೊಟ್ಟಮೊದಲು ಕೇಳಿದ ಸಂಜೆ ನಾವು ಶಿವಪರ್ವತಗಳನ್ನೂ, ಅನಾವರಣಗೊಂಡ ಆಲ್ಮೋರದ ರಮ್ಯದೃಶ್ಯಗಳನ್ನೂ ನಿಬ್ಬೆರಗಾಗಿ ನೋಡುತ್ತಿದ್ದೆವು...

ಶುಕನೇ ನಿಜವಾಗಿ ಸ್ವಾಮಿಗಳಿಗೆ ಪ್ರಿಯನಾದ ಸಂತ. ಅವರ ಪ್ರಕಾರ ಈ ಪ್ರಪಂಚ, ಈ ಬದುಕು ಕೇವಲ ಮಕ್ಕಳಾಟದ ಹಾಗೆ ಕಾಣುವಂತಹ ಅತ್ಯುನ್ನತ ಸಾಕ್ಷಾತ್ಕಾರವನ್ನು ಪಡೆದವನವನು ಶುಕ. ನಂತರ, ಅದೆಷ್ಟೋ ಕಾಲ ಸರಿದ ಮೇಲೆ, ಶ‍್ರೀರಾಮಕೃಷ್ಣರು ಎಳೆಯ ವಯಸ್ಸಿನ ಅವರನ್ನು ಕುರಿತು “ನನ್ನ ಶುಕ!” ಎಂದು ಹೇಳುತ್ತಿದ್ದರೆಂಬ ಸಂಗತಿ ನಮಗೆ ತಿಳಿಯಿತು. ಒಮ್ಮೆ ಎದ್ದು ನಿಂತು ಶಿವನ ಮಾತೆಂಬಂತೆ “ಭಗವದ್ಗೀತೆಯ ನಿಜವಾದ ಅಂತರಾರ್ಥ ನನಗೆ ಗೊತ್ತು, ಶುಕನಿಗೂ ಗೊತ್ತು, ವ್ಯಾಸನಿಗೆ ಬಹುಶಃ ಅಲ್ಪಸ್ವಲ್ಪ ಗೊತ್ತಿದ್ದರೂ ಗೊತ್ತಿರಬಹುದು!” ಎಂದು ಹೇಳಿದಾಗ ಅವರ ಆ ಮುಖಾರವಿಂದವನ್ನು, ದೂರದ ಆನಂದದ ನಿಗೂಢ ಆಳಕ್ಕೇ ಹೋಗುವಂತಹ ಅವರ ಆ ದೃಷ್ಟಿಯನ್ನು ನಾನೆಂದಿಗೂ ಮರೆಯಲಾರೆ.

ಆಲ್ಮೋರದಲ್ಲಿ ಇನ್ನೊಂದು ದಿನ ಸ್ವಾಮಿಗಳು ಪುರಾತನ ಹಿಂದೂ ಸಂಸ್ಕೃತಿಯ ಕಿನಾರೆಯ ಮೇಲೆ ನವಪ್ರಜ್ಞೆಯ ಮೊದಲನೆಯ ಅಲೆಯುರುಳಿ ಬಂದು ತೊಳೆದ ಹಾಗೆ, ಮಾನವೀಯ ಅನುಕಂಪದೊಂದಿಗೆ ಬಂಗಾಳದಲ್ಲಿ ಉದಯಿಸಿದ ಮಹನೀಯರ ಜೀವನಗಳನ್ನು ಕುರಿತು ಮಾತನಾಡಿದರು. ಈಗಾಗಲೇ ರಾಮಮೋಹನ ರಾಯ್​ ಬಗ್ಗೆ ನಾವು ನೈನಿತಾಲ್ನಲ್ಲಿದ್ದಾಗ ಕೇಳಿದ್ದೆವು. ಈಗ ಪಂಡಿತ ವಿದ್ಯಾಸಾಗರರ ಬಗ್ಗೆ “ಉತ್ತರ ಹಿಂದೂಸ್ಥಾನದಲ್ಲಿನ ನನ್ನ ವಯಸ್ಸಿನ ಯಾರೊಬ್ಬರ ಮೇಲೂ ಅವರ ನೆರಳು ಬೀಳದೆ ಇರಲಾರದು!” ಎಂದು ಉದ್ಗರಿಸಿದರು. ಈ ಎಲ್ಲ ಮಹನೀಯರುಗಳೂ ಶ‍್ರೀರಾಮ ಕೃಷ್ಟರೂ ಕೆಲವೇ ಮೈಲಿಗಳ ಅಂತರದಲ್ಲಿ ಹುಟ್ಟಿದವರು ಎಂದು ನೆನಪಿಸಿಕೊಳ್ಳುವುದೆಂದರೆ ಅವರಿಗೆ ಮಹದಾನಂದವಾಗುತ್ತಿದ್ದಿತು.

ಸ್ವಾಮಿಗಳು ಈಗ ನಮಗೆ ವಿದ್ಯಾಸಾಗರರನ್ನು “ಬಹುಪತ್ನೀಪದ್ಧತಿಯನ್ನು ನಿಲ್ಲಿಸಿದ ಮತ್ತು ವಿಧವಾವಿವಾಹವನ್ನು ಸಾಧ್ಯವಾಗಿಸಿದ ಮಹನೀಯ” ರೆಂದು ಪರಿಚಯಿಸಿದರು. ಆದರೆ ವಿದ್ಯಾಸಾಗರರನ್ನು ಕುರಿತು ಅವರಿಗೆ ಬಹು ಪ್ರಿಯವಾಗಿದ್ದ ಒಂದು ಕಥೆಯಿದೆ. ಒಂದು ದಿನ ಅವರು ಶಾಸನಸಭೆಯಿಂದ ಮನೆಗೆ ಹಿಂದಿರುಗುತ್ತ, ಅಂತಹ ಸಂದರ್ಭಗಳಲ್ಲಿ ಆಂಗ್ಲ ಪೋಷಾಕನ್ನು ಧರಿಸುವುದು ಯುಕ್ತವೇ ಎಂದು ಯೋಚಿಸುತ್ತಿದ್ದರಂತೆ. ಅವರ ಮುಂಭಾಗದಲ್ಲಿ ತನ್ನದೇ ವಿಶಿಷ್ಟ ಶೈಲಿಯಲ್ಲಿ ಉಡುಪನ್ನು ಧರಿಸಿದ್ದ ಮೊಘಲ್ ಸ್ಥೂಲ ದೇಹಿಯೊಬ್ಬ ಸಾವಕಾಶವಾಗಿ ಮನೆಗೆ ಹೋಗುತ್ತಿದ್ದನಂತೆ. ಇದ್ದಕ್ಕಿದ್ದಂತೆ ಯಾರೋ ಒಬ್ಬ ಓಡಿ “ಸ್ವಾಮಿ, ನಿಮ್ಮ ಮನೆಗೆ ಬೆಂಕಿ ಬಿದ್ದಿದೆ!” ಎಂದು ಸುದ್ದಿ ಮುಟ್ಟಿಸಿದನಂತೆ. ಮೊಘಲನು ಮೊದಲಿನಂತೆಯೇ ಸಾವಕಾಶವಾಗಿ ನಡೆಯುವುದನ್ನು ಮುಂದುವರೆಸಲು, ಆತನು ಆಶ್ಚರ್ಯಚಕಿತನಾಗಿ “ಇದೇನು ಸ್ವಾಮಿ?” ಎನ್ನಲು ಮೊಘಲನು ಸಿಟ್ಟಿನಿಂದ ಅವನ ಕಡೆಗೆ ತಿರುಗಿ “ಮೂರ್ಖ! ಕೆಲವು ಕಡ್ಡಿಗಳು ಉರಿದುಹೋಗುತ್ತಿವೆ ಎಂದು ನಾನು ನನ್ನ ತಲೆ ತಲಾಂತರದಿಂದ ಬಂದ ನಡೆಯುವ ಶೈಲಿಯನ್ನು ಬದಲಾಯಿಸಬೇಕೆ?” ಎಂದನಂತೆ. ಹಿಂದೆಯೇ ನಡೆದು ಬರುತ್ತಿದ್ದ ವಿದ್ಯಾಸಾಗರರು ಆಗಲೇ ನಿರ್ಧರಿಸಿದರಂತೆ - ಕೋಟು ಬೂಟುಗಳನ್ನು ತಾನು ಎಂದಿಗೂ ತೊಡುವುದಿಲ್ಲ, ಚಾದರ ಧೋತಿ ಚಪ್ಪಲಿಗಳಲ್ಲಿಯೇ ಇರುತ್ತೇನೆ ಎಂದು.

ವಿದ್ಯಾಸಾಗರರ ತಾಯಿ ಬಾಲವಿಧವೆಯರ ಪುನರ್ವಿವಾಹದ ಸಾಧ್ಯತೆಯ ಸಲಹೆ ಮುಂದಿಟ್ಟಾಗ, ಅವರು ಒಂದು ತಿಂಗಳ ಏಕಾಂತವಾಸಕ್ಕೆ ತೆರಳಿ ಸಂಬಂಧಪಟ್ಟ ಶಾಸ್ತ್ರಗಳ ನ್ನೆಲ್ಲ ಅಧ್ಯಯನ ಮಾಡಿದ ಸಂಗತಿಯನ್ನು ಶಕ್ತಿಯುತವಾಗಿ ಕಣ್ಣಿಗೆ ಕಟ್ಟುವಂತೆ ನಮ್ಮೆದರು ಚಿತ್ರಿಸಿದರು. “ಏಕಾಂತವಾಸದಿಂದ ಹೊರಗೆ ಬಂದ ಅವರು ಪುನರ್ವಿವಾಹವು ಶಾಸ್ತ್ರ ವಿರೋಧಿಯಲ್ಲವೆಂಬ ಅಭಿಪ್ರಾಯವನ್ನು ತಳೆದಿದ್ದರು. ಈ ಅಭಿಪ್ರಾಯಕ್ಕೆ ತಮ್ಮ ಸಮ್ಮತಿ ಯಿದೆ ಎಂಬುದಾಗಿ ಪಂಡಿತರೆಲ್ಲರ ಸಹಿಗಳನ್ನು ಸಂಗ್ರಹಿಸಿದರು. ಆ ನಂತರ ಕೆಲವು ದೇಶೀಯ ರಾಜರುಗಳ ಒತ್ತಡಕ್ಕೆ ಕಟ್ಟುಬಿದ್ದು ಪಂಡಿತರುಗಳು ತಾವು ಕೊಟ್ಟಿದ್ದ ಸಮ್ಮತಿಯನ್ನು ಹಿಂದೆಕ್ಕೆ ತೆಗೆದುಕೊಂಡರು. ಸರ್ಕಾರವು ಈ ಚಳುವಳಿಯನ್ನು ಬೆಂಬಲಿಸದೆ ಇದ್ದಿದ್ದರೆ ಅದು ಮುಂದುವರೆಯಲಾಗುತ್ತಿರಲಿಲ್ಲ - ಈಗ, ಸಮಸ್ಯೆಯು ಸಾಮಾಜಿಕವಾಗಿರುವು ದಕ್ಕಿಂತ ಹೆಚ್ಚಾಗಿ ಆರ್ಥಿಕವಾಗಿಯೇ ನೆಲೆಗೊಳ್ಳುವಂತಾಗಿದೆ”.

ವ್ಯಕ್ತಿಯೊಬ್ಬನು ಬಹುಪತ್ನೀ ಪದ್ಧತಿಯನ್ನು ಕೇವಲ ತನ್ನ ನೈತಿಕ ಬಲ ಮಾತ್ರದಿಂದ ನಿಲ್ಲಿಸಲು ಸಾಧ್ಯವಾಯಿತೆಂದಾದರೆ, ಆತನು “ತೀವ್ರ ಆಧ್ಯಾತ್ಮಿಕ ಶಕ್ತಿಯುಳ್ಳವನಾಗಿರಬೇಕು” ಎಂದು ನಮಗೂ ನಂಬಬೇಕೆನಿಸಿತು. ರೂಢಿಗತ ಸಂಪ್ರಾದಾಯವೊಂದಕ್ಕೆ ಸಂಬಂಧಿಸಿದಂತೆ ಭಾರತೀಯ ಉಪೇಕ್ಷೆ ಎಂತಿರಬಹುದೆಂಬುದನ್ನು ಅರಿತಾಗ ನಿಜಕ್ಕೂ ನಮಗೆ ಅಚ್ಚರಿ - ಈ ದೈತ್ಯ ಶಕ್ತಿಯ ಮಹನೀಯರು ೧೮೬೪ ರ ಕ್ಷಾಮದಲ್ಲಿ ೧, ೪೦, ೦೦೦ ಜನರು ಹಸಿವಿನಿಂದಲೂ ಖಾಯಿಲೆಯಿಂದಲೂ ಸತ್ತಾಗ ದೇವರನ್ನು ಸಂಪೂರ್ಣವಾಗಿ ನಿರಾಕರಿಸಿ ತಾತ್ತ್ವಿಕವಾಗಿ ಅಜ್ಞೇಯವಾದಿಯಾದನಂತೆ.

ಬಂಗಾಳವನ್ನು ವಿದ್ಯಾವಂತ ದೇಶವನ್ನಾಗಿಸಿದ ಈ ಮಹನೀಯರ ಜೊತೆಗೆ ಸ್ವಾಮಿಗಳು ಡೇವಿಡ್ ಹೇರ್ ಎಂಬ ನಾಸ್ತಿಕನಾದ ವೃದ್ದ ಸ್ಕಾಚ್ಮನ್ ಒಬ್ಬರ ಹೆಸರನ್ನೂ ಸೇರಿಸಿದರು. ಈತನಿಗೆ ಕಲ್ಕತ್ತದ ಕ್ರೈಸ್ತ ಪುರೋಹಿತಶಾಹಿಯು ಕ್ರೈಸ್ತ ಸಂಪ್ರದಾಯದ ಅಪರ ಸಂಸ್ಕಾರವನ್ನು ಮಾಡಗೊಡಲಿಲ್ಲ. ಕಾಲರಾ ತಗುಲಿದ ಶಿಷ್ಯನೊಬ್ಬನ ಸೇವೆ ಮಾಡುತ್ತಿರುವಾಗ ಸೋಂಕು ತಗುಲಿ ಈತ ಸತ್ತಾಗ, ಈತನ ಅನುಯಾಯಿಗಳು ಶವವನ್ನು ತಾವೇ ಹೊತ್ತು ಸಾಗಿಸಿ ಜೌಗೊಂದರಲ್ಲಿ ಹೂತು, ಗೋರಿಯನ್ನು ಒಂದುಯಾತ್ರಾಸ್ಥಳವನ್ನಾಗಿಸಿದ ರಂತೆ. ಈ ಸ್ಥಳವೀಗ ಕಾಲೇಜ್ ಸ್ಕ್ವೇರ್ ಎನ್ನಿಸಿಕೊಂಡಿದೆ - ವಿದ್ಯಾಕೇಂದ್ರವಾಗಿದೆ. ಆತನು ಸ್ಥಾಪಿಸಿದ ಶಾಲೆಯು ವಿಶ್ವವಿದ್ಯಾನಿಲಯದ ಒಳಗಡೆ ಇದೆ. ಈ ಹೊತ್ತು ಕಲ್ಕತ್ತದ ವಿದ್ಯಾರ್ಥಿಗಳು ನಿತ್ಯವೂ ಆತನ ಗೋರಿಗೆಯಾತ್ರೆ ಮಾಡುವಂತಾಗಿದೆ.

ಸಂಭಾಷಣೆ ಸಹಜವಾಗಿಯೆ ತಿರುವು ಪಡೆದುಕೊಂಡ ಈ ಸಂದರ್ಭವನ್ನು ಪಯೋಗಿಸಿಕೊಂಡು ಈ ದಿನ ಸ್ವಾಮಿಗಳನ್ನು ಕ್ರೈಸ್ತಧರ್ಮವು ನಿಮ್ಮ ಮೇಲೆ ಎಂತಹ ಪ್ರಭಾವವನ್ನು ಬೀರಿದೆ ಎಂದು ನಾವೇ ಪ್ರಶ್ನೆ ಕೇಳಿದೆವು. ಇಂಥದೊಂದು ಹೇಳಿಕೆ ನಮ್ಮಿಂದ ಬಂದುದನ್ನು ಕೇಳಿದ ಅವರಿಗೆ ತಮಾಷೆ ಎನ್ನಿಸಿತು. ತಮ್ಮ ಮೇಲೆ ಆಗಿರುವ ಕ್ರೈಸ್ತಪ್ರಚಾರಕರೊಬ್ಬರ ಏಕೈಕ ಪ್ರಭಾವ ಎಂದರೆ ತಮ್ಮ ಸ್ಕಾಚ್ ಗುರು ಮಿ. ಹೇಸ್ಟಿ ಅವರದ್ದು ಎಂದು ಬಹು ಹೆಮ್ಮೆಯಿಂದ ಹೇಳಿದರು. ಶೀಘ್ರಕೋಪಿಯಾಗಿದ್ದ ಈ ವೃದ್ಧ ಗರೀಬನಂತೆ ಬದುಕಿದ್ದರು-ಅವರಿದ್ದ ಕೊಠಡಿಯೇ ಅವರ ಶಿಷ್ಯರುಗಳ ವಾಸಸ್ಥಾನವೂ ಆಗಿತ್ತು. ಅವರೇ ಮೊಟ್ಟಮೊದಲು ಸ್ವಾಮಿಗಳನ್ನು ಶ‍್ರೀರಾಮಕೃಷ್ಣರ ಬಳಿಗೆ ಕಳುಹಿಸಿದ್ದು; ಭಾರತದಲ್ಲಿ ತಮ್ಮ ಕೊನೆಯ ದಿನಗಳಲ್ಲಿ ಅವರು “ಹೌದು, ಹುಡುಗಾ, ನೀನೆನ್ನುವುದು ಸರಿ! ಎಲ್ಲವೂ ದೇವರೇ ಆಗಿದೆ ಎನ್ನುವುದು ನಿಜ!” ಎನ್ನುತ್ತಿದ್ದರಂತೆ. “ಅವರೆಂದರೆ ನನಗೆ ಬಹು ಹೆಮ್ಮೆ!” ಎಂದು ಉದ್ಘೋಷಿಸುತ್ತಿದ್ದಂತೆಯೇ ಸ್ವಾಮಿಗಳು “ಆದರೆ ಅವರು ನನ್ನನ್ನು ಕ್ರೈಸ್ತನನ್ನಾಗಿ ಮಾಡಿದರು ಎಂದು ಎಷ್ಟುಮಾತ್ರಕ್ಕೂ ನೀವು ಅಂದುಕೊಳ್ಳುವಂತಿಲ್ಲ” ಎಂದೂ ಸೇರಿಸಿದರು...

ಅಷ್ಟೊಂದು ಗಂಭೀರವೆನಿಸದ ವಿಷಯಗಳಿಂದ ಕೂಡಿದ ಕಥೆಗಳೂ ಸಹ ರೋಚಕವಾಗಿರುತ್ತಿದ್ದವು. ಉದಾಹರಣೆಗೆ, ಅಮೆರಿಕಾದ ನಗರವೊಂದರಲ್ಲಿ ಒಂದು ಕಡೆ ಸ್ವಾಮಿಗಳು ಬಾಡಿಗೆ ಮನೆಯಲ್ಲಿದ್ದುಕೊಂಡು ತಮ್ಮ ಅಡುಗೆಯನ್ನು ತಾವೇ ಮಾಡಿಕೊಳ್ಳುತ್ತಿದ್ದ ರಂತೆ. ಇದಕ್ಕಾಗಿ ಅಕ್ಕಪಕ್ಕಗಳಲ್ಲಿ ಓಡಾಡುವಾಗ “ನಿತ್ಯವೂ ಹುರಿದ ಟರ್ಕಿಯನ್ನೇ ತಿನ್ನುವ ನಟಿಯೊಬ್ಬಳನ್ನು” ಹಾಗೂ ನಿತ್ಯವೂ ಪಿಶಾಚಿಗಳಂತೆ ವೇಷ ಹಾಕಿಕೊಂಡು ಜನರನ್ನು ವಂಚಿಸುತ್ತಿದ್ದ ದಂಪತಿಗಳನ್ನು ನೋಡಿದರಂತೆ. ಒಮ್ಮೆ ಸ್ವಾಮಿಗಳು ಆ ದಂಪತಿಗಳಲ್ಲಿ ಗಂಡನಿಗೆ ಜನಗಳನ್ನು ವಂಚಿಸುವುದು ತರವಲ್ಲ ಎಂದು ಅನುನಯಿಸುತ್ತ “ನೀನು ಹೀಗೆ ಮಾಡತಕ್ಕುದೇ ಅಲ್ಲ!” ಎಂದರಂತೆ. ಆಗ ಹೆಂಡತಿ ಹಿಂದಿನಿಂದ ಬಂದು ಕಾತರದಿಂದ “ಹೌದು ಸ್ವಾಮಿ! ಅದನ್ನೇ ನಾನೂ ಅವನಿಗೆ ಹೇಳುವುದು; ಏಕೆಂದರೆ ಅವನೇ ಪ್ರೇತದಂತೆ ನಟಿಸುವುದು, ಮತ್ತು ಮಿಸೆಸ್ ವಿಲಿಯಮ್ಸ್ ದುಡ್ಡನ್ನೆಲ್ಲ ಸಂಗ್ರಹಿಸುವುದು!” ಎಂದಳಂತೆ.

ಇಂತಹ ಪ್ರೇತ ಸಂಪರ್ಕಸಭೆಯೊಂದರಲ್ಲಿ, ವಿದ್ಯಾವಂತನಾಗಿದ್ದ ತರುಣ ಎಂಜಿ ನಿಯರ್ ಒಬ್ಬನು “ಆ ದಪ್ಪನಾದ ಮಿಸೆಸ್ ವಿಲಿಯಮ್ಸ್ ಪರದೆಯ ಹಿಂದುಗಡೆಯಿಂದ ತನ್ನ ತೆಳ್ಳಗಿದ್ದ ತಾಯಿಯಾಗಿ ಕಾಣಿಸಿದಾಗ, ಅಚ್ಚರಿಯಿಂದ, ತಾಯಿ, ನೀನು ಪ್ರೇತ ಪ್ರಪಂಚದಲ್ಲಿ ಹೇಗೆ ಬೆಳೆದು ಬಿಟ್ಟಿರುವೆ!” ಎಂದು ಉದ್ಗರಿಸಿದನಂತೆ!

ಸ್ವಾಮಿಗಳು “ಈ ಹಂತದಲ್ಲಿ ನಾನು ಎದೆಯೊಡೆದೆ; ಆ ಮನುಷ್ಯನಿಗೆ ಇನ್ನು ಯಾವ ಭರವಸೆಯೂ ಇಲ್ಲ ಎನಿಸಿತು” ಎಂದರು. ಅದೇ ಲಹರಿಯಲ್ಲಿ ಅವನು ರಷ್ಯನ್ ಚಿತ್ರಕಾರ ನೊಬ್ಬನ ಕಥೆಯನ್ನೂ ಹೇಳಿದನಂತೆ. ಚಿತ್ರಕಾರನು ರೈತನೊಬ್ಬನ ಗತಿಸಿದ ತಂದೆಯ ಚಿತ್ರವನ್ನು ಒಮ್ಮೆ ಚಿತ್ರಿಸಬೇಕಾಗಿ ಬಂದಿತು. ಆಣತಿಯಿತ್ತ ಮಗ ಕೊಟ್ಟ ವಿವರಣೆ “ಅಯ್ಯೋ ಶಿವನೆ! ನಮ್ಮಪ್ಪನ ಮೂಗಿನ ಮೇಲೆ ಮಚ್ಚೆಯೊಂದಿತ್ತೆಂದು ನಿನಗೆ ಹೇಳಲಿಲ್ಲವೆ!” ಎಂಬು ದಷ್ಟೇ ಆಗಿತ್ತು. ಚಿತ್ರಕಾರನು ಕೊನೆಗೆ ಯಾವನೋ ಒಬ್ಬನ ಮುಖವನ್ನು ಚಿತ್ರಿಸಿ, ಮೂಗಿನ ಮೇಲೆ ದೊಡ್ಡದೊಂದು ಮಚ್ಚೆಯನ್ನಿಟ್ಟು, ಮಗನನ್ನು ಬಂದು ನೋಡು ಎಂದು ಕರೆದ ನಂತೆ. ಅವನು ಬಂದು ಚಿತ್ರದ ಮುಂದೆ ನಿಂತವನೇ ಭಾವಾವಿಷ್ಟನಾಗಿ, “ಅಪ್ಪಾ! ಅಪ್ಪಾ! ಈ ಹಿಂದೆ ನಾನು ನಿನ್ನನ್ನು ನೋಡಿದಾಗಿನಿಂದ ಇತ್ತೀಚೆಗೆ ಎಷ್ಟೊಂದು ಬದಲಾಗಿಬಿಟ್ಟಿರುವೆ!” ಎಂದು ಗೋಳಿಟ್ಟನಂತೆ! ಇದಾದ ಮೇಲೆ ತರುಣ ಎಂಜಿನಿಯರ್ ಸ್ವಾಮಿಗಳೊಂದಿಗೆ ಇನ್ನೇನೂ ಮಾತನಾಡಲಿಲ್ಲವಂತೆ - ಕಥೆಯ ಅಂತರಾಳ ಗೊತ್ತಾಗಿರಬೇಕು ಎಂದುಕೊಂಡು. ಸ್ವಾಮಿಗಳಿಗೆ ಈಗ ನಿಜಕ್ಕೂ ಅಚ್ಚರಿಯಾಯಿತಂತೆ.

ಏನೇ ಆದರೂ, ಇಂತಹ ಸಾಮಾನ್ಯ ಕುತೂಹಲದ ಸಂಗತಿಗಳ ನಡುವೆಯೂ, ಆಂತರ್ಯದ ಹೋರಾಟ ಹೆಚ್ಚು ತ್ತಲೇ ಹೋಯಿತು. ಗುರುಗಳಿಗೆ ಸೇವೆಯ ಹಾಗೂ ಶಾಂತಿಯ ಆವಶ್ಯಕತೆಯಿದೆ ಎಂದು ನಮ್ಮ ಗುಂಪಿನಲ್ಲಿದ್ದ ಹಿರಿಯೊಬ್ಬರ ಮನಸ್ಸಿಗೆ ಚಿಂತೆ ಒತ್ತಿಕೊಂಡು ಬಂತು. ಬದುಕಿನಯಾತನೆಯ ಅಚ್ಚರಿಯ ಬಗ್ಗೆ ಅನೇಕ ಬಾರಿ ಮಾತ ನಾಡಿದರು. ಅಂತಹ ಆವಶ್ಯಕತೆಯ ಚಿಹ್ನೆಗಳು ಸ್ವಾಮಿಗಳಲ್ಲಿ ಅದೆಷ್ಟಿದ್ದವೋ ಯಾರು ಹೇಳಬಲ್ಲರು? ಒಂದೋ ಎರಡೋ ಶಬ್ದ, ಅಲ್ಪ ಮಾತ್ರ, ಆದರೆ ಸಾಕಾದಷ್ಟು ನುಡಿ ದಿರಬಹುದು. ಎಷ್ಟೋ ಹೊತ್ತಾದ ಮೇಲೆ, ಅವರು ಹಿಂದಿರುಗಿ ಬಂದು ತಾವು ಮೌನ ಬಯಸುವುದಾಗಿಯೂ, ಒಬ್ಬರೇ ಕಾಡಿಗೆ ಹೋಗಿ ಶಾಂತಿಯನ್ನು ಅರಸುವುದಾಗಿಯೂ ಹೇಳಿದರು.

ಅನಂತರ, ಮೇಲಕ್ಕೆ ನೋಡುತ್ತ, ಬಾನಿನಲ್ಲಿ ಪ್ರಕಾಶಿಸುತ್ತಿದ್ದ ಬಾಲಚಂದ್ರನನ್ನು ನೋಡಿ, “ಮಹಮ್ಮದೀಯರು ಬಾಲಚಂದ್ರನಿಗೆ ತುಂಬ ಪ್ರಾಮುಖ್ಯತೆ ಕೊಡುತ್ತಾರೆ. ನಾವೂ ಸಹ ಬಿದಿಗೆ ಚಂದ್ರನೊಂದಿಗೆ ಹೊಸ ಬದುಕನ್ನು ಪ್ರಾರಂಭಿಸೋಣ!” ಎಂದರು. ತಮ್ಮ ಮಗಳನ್ನು ಹಾರ್ದಿಕವಾಗಿ ಆಶೀರ್ವದಿಸಿದರು - ಅವಳು, ಹಳೆಯ ಬಾಂಧವ್ಯ ಬಿಟ್ಟುಹೋದುದನ್ನು ಯೋಚಿಸುತ್ತ, ಹೊಸ ಗಹನವಾದೊಂದು ಜೀವನ ಲಭ್ಯವಾಗಿರು ವುದರ ಕನಸನ್ನೂ ಕಾಣಲಾರದೆ, ಈ ಹೊತ್ತು ಅದೆಷ್ಟು ನಿಗೂಢವಾಗಿದೆ, ಹೇಗೆ ಇನಿದಾಗಿ ಕಳೆಯುತ್ತಿದೆ ಎಂದು ಮಾತ್ರ ಅರಿತುಕೊಂಡಳು...

\textbf{ಮೇ ೨೫.}

ಅವರು ಹೊರಟುಹೋದರು. ಆವೊತ್ತು ಬುಧವಾರ. ಶನಿವಾರ ಹಿಂದಿರುಗಿ ಬಂದರು. ಕಾಡಿನಲ್ಲಿ ಪ್ರತಿನಿತ್ಯ ಮೌನದಿಂದ ಹತ್ತು ಗಂಟೆಗಳನ್ನು ಕಳೆಯುತ್ತಿದ್ದರು; ಆದರೆ ಸಂಜೆ ತಮ್ಮ ಡೇರೆಗೆ ಹಿಂದಿರುಗುತ್ತಿದ್ದಂತೆಸುತ್ತಣ ಕಾತರದ ಮುಖಗಳು ಆ ಭಾವವನ್ನೊಡ್ ಯುವುವೋ ಎಂಬಂತಾಗುತ್ತಿತ್ತು; ಮತ್ತೆ ಅವರು ಹೊರಟುಹೋಗುತ್ತಿದ್ದರು. ಆದರೂ ಅವರ ಮುಖ ದಿವ್ಯ ತೇಜಸ್ಸಿನಿಂದ ಪ್ರಕಾಶಿಸುತ್ತಿತ್ತು. ಮತ್ತೊಮ್ಮೆ ಕಳೆದುಹೋದ ಕಾಲದ ಸಂನ್ಯಾಸಿಯಾಗಿಬಿಟ್ಟಿದ್ದರು - ಬರಿಗಾಲಿನಲ್ಲಿ ನಡೆಯಬಲ್ಲ, ಚಳಿ-ಸೆಕೆಗಳನ್ನು ಸಹಿಸಿಕೊಳ್ಳಬಲ್ಲ, ಲಭ್ಯ ಅಲ್ಪಾಹಾರದ ಮೇಲೆಯೇ ಜೀವಿಸಬಲ್ಲ, ಪಾಶ್ಚಾತ್ಯ ಜಗತ್ತಿನಲ್ಲಿ ಕಳೆದುಹೋಗದ ಸಂನ್ಯಾಸಿ...

\textbf{ಜೂನ್ ೨.}

ಒಂದು ಶುಕ್ರವಾರ ಬೆಳಗ್ಗೆ ನಾವೆಲ್ಲರೂ ಕಾರ್ಯಮಗ್ನರಾಗಿದ್ದಾಗ, “ಗುಡ್ವಿನ್ ನೆನ್ನೆ ರಾತ್ರಿ ಊಟಿಯಲ್ಲಿ ತೀರಿಕೊಂಡರು” ಎಂಬ ತಂತಿ ಸಮಾಚಾರ ಒಂದು ದಿನ ತಡವಾಗಿ ಬಂದು ತಲುಪಿತು. ಹರಡುತ್ತಿದ್ದ ಟೈಫಾಯ್ಡ್ ಜ್ವರದ ಸೋಂಕಿನಿಂದ ಮೃತಪಟ್ಟ ಮೊದಲನೆಯವರಲ್ಲಿ ನಮ್ಮ ಈ ಗೆಳೆಯರೂ ಒಬ್ಬರೆಂದು ಕಾಣುತ್ತದೆ. ಕೊನೆಯ ಉಸಿರಿನಲ್ಲಿ ಸಹ ಅವರು ಸ್ವಾಮಿಜಿಯ ಬಗ್ಗೆಯೇ ಮಾತನಾಡುತ್ತಿದ್ದರು, ಅಲ್ಲದೆ ಕೊನೆಯ ಕ್ಷಣದಲ್ಲಿಯೂ ಅವರನ್ನು ತಮ್ಮ ಪಕ್ಕದಲ್ಲಿ ನೋಡಲು ಬಯಸಿದರೆಂದು ತೋರುತ್ತದೆ.

\textbf{ಜೂನ್ ೫.}

ಭಾನುವಾರ ಸಂಜೆ ಸ್ವಾಮಿಗಳು ಮನೆಗೆ ಬಂದರು. ಗೇಟಿನ ಮೂಲಕ ಒಳಗೆ ಬಂದವರು ಟೆರೇಸಿನ ಮೇಲಕ್ಕೂ ಬಂದರು; ಅಲ್ಲಿ ಕುಳಿತು ನಮ್ಮೊಂದಿಗೆ ಸ್ವಲ್ಪಹೊತ್ತು ಮಾತನಾಡಿದರು. ನಮ್ಮಲ್ಲಿದ್ದ ಸುದ್ದಿಯೇನೂ ಅವರಿಗೆ ಗೊತ್ತಾಗಿರಲಿಲ್ಲ; ಆದರೂ ಅದಾಗಲೇ ಅವರ ಮನಸ್ಸಿಗೆ ಕತ್ತಲು ಕವಿದಂತಿತ್ತು. ಯಾರನ್ನು ತಾವು ಕೇವಲ ಶ‍್ರೀರಾಮಕೃಷ್ಣರಿಗೆ ಎರಡನೆಯ ಸ್ಥಾನದಲ್ಲಿಟ್ಟು ಪ್ರೀತಿಸಿದ್ದರೋ ಆ ಸಾಧುವನ್ನು, ತನಗೆ ಕಚ್ಚಿದ ಹಾವನ್ನು “ಪ್ರೇಮಿ ಕಳು ಹಿಸಿದ ದೂತ” ಎಂದಿದ್ದ ಆ ಸಾಧುವನ್ನು ನಮಗೆ ನೆನಪಿಸುವುದಕ್ಕಾಗಿ ಎಂಬಂತೆ ಮೌನ ಮುರಿದರು. “ಪವಹಾರಿ ಬಾಬಾ ತನ್ನೆಲ್ಲ ಯಜ್ಞಗಳನ್ನೂ ಮುಗಿಸಿ ಪೂರ್ಣಾಹುತಿಯಾಗಿ ತನ್ನ ದೇಹವನ್ನೇ ಅರ್ಪಿಸಿದನು; ಹೋಮಾಗ್ನಿಯಲ್ಲಿ ತನ್ನನ್ನೇ ತಾನು ದಹಿಸಿಕೊಂಡನು” ಎಂದು ತಿಳಿಸುವ ಪತ್ರ ತಮಗೆ ಬಂದಿದೆ ಎಂದರು. ಶ್ರೋತೃಗಳಲ್ಲೊಬ್ಬರು, “ಸ್ವಾಮಿ, ಅದು ತಪ್ಪಲ್ಲವೆ?” ಎಂದು ಅಚ್ಚರಿ ವ್ಯಕ್ತಪಡಿಸಿದರು.

“ನಾನು ಹೇಗೆ ಹೇಳಲಿ?” ಎಂದರು ತುಂಬ ತಳಮಳಗೊಂಡಿದ್ದ ಸ್ವಾಮಿಗಳು. “ತೀರ್ಮಾನ ಕೊಡುವುದಕ್ಕೆ ಅವರು ತುಂಬ ಮೇಲ್ಮಟ್ಟದವರು. ತಾವೇನು ಮಾಡುತ್ತಿರುವರೆಂಬುದು ಅವರಿಗೇ ಗೊತ್ತಿದ್ದಿರಬೇಕು.”

ಇದಾದ ಮೇಲೆ ಮೌನ ಹೆಪ್ಪುಗಟ್ಟಿತು. ಸಂನ್ಯಾಸಿಗಳ ಗುಂಪು ಅಲ್ಲಿಂದ ಮುನ್ನಡೆಯಿತು. ಇನ್ನೊಂದು ಸುದ್ದಿಯನ್ನು ಅವರಿಗಿನ್ನೂ ಹೇಳಲಾಗಿರಲಿಲ್ಲ.

\textbf{ಜೂನ್ ೬.}

ಮಾರನೆಯ ಬೆಳಗ್ಗೆ ಬೇಗನೆ ಬಂದಾಗ ಸ್ವಾಮಿಗಳು ಉತ್ತಮ ಭಾವದಲ್ಲಿದ್ದರು. ನಾನು ಈ ಹೊತ್ತು ನಾಲ್ಕು ಗಂಟೆಗೇ ಎದ್ದಿರುವೆನು ಎಂದರು. ನಮ್ಮಲ್ಲೊಬ್ಬರು ಮೆಲ್ಲನೆ ಎದ್ದು ಅವರ ಬಳಿಗೆ ಹೋಗಿ ಗುಡ್ವಿನ್ ಇನ್ನಿಲ್ಲವಾದ ಸುದ್ದಿಯನ್ನು ತಿಳಿಸಿದರು. ಶಾಂತ ರೀತಿಯಲ್ಲೇ ವಜ್ರಾಘಾತ ತಗುಲಿತು. ಕೆಲವು ದಿನಗಳ ನಂತರ ಈ ಸುದ್ದಿ ತಲುಪಿದ ಸ್ಥಳದಲ್ಲಿ ನಾನು ಇರಲಾರೆ, ನನ್ನ ನಿಷ್ಠಾವಂತ ಶಿಷ್ಯನ ಚಿತ್ರ ಸತತವಾಗಿ ತಮ್ಮ ಮನಸ್ಸಿನ ಮುಂದೆ ಬರುತ್ತದೆ ಎಂದು ತಮ್ಮ ದೌರ್ಬಲ್ಯವನ್ನು ಹೇಳಿಕೊಂಡರು. ಹೀಗೆ ಒಬ್ಬರ ನೆನಹನ್ನು ಹೇರಿಕೊಂಡಿರುವುದು ಒಂದು ನಾಯಿಯ ಅಥವಾ ಮೀನಿನ ಗುಣಲಕ್ಷಣಗಳನ್ನು ಉಳಿಸಿಕೊಂಡಿರುವುದಕ್ಕಿಂತ ಹೆಚ್ಚೇನೂ ಪೌರುಷಯುಕ್ತವಾದುದಲ್ಲ ಎಂದರು. ಮನುಷ್ಯನಾದವನು ಈ ಭ್ರಮೆಯಿಂದ ಪಾರಾಗಬೇಕು. ಸತ್ತವರು ಇಲ್ಲೇ, ನಮ್ಮ ಪಕ್ಕದಲ್ಲೇ ಎಂದಿನಂತೆ ಇದ್ದಾರೆ ಎಂದುಕೊಳ್ಳಬೇಕು; ಅವರಿಲ್ಲ, ನಮ್ಮನ್ನು ಬಿಟ್ಟುಹೋದರು ಎನ್ನು ವುದೇ ಭ್ರಮೆ ಎಂದೂ ತಮ್ಮನ್ನು ತಾವು ಭರ್ತ್ಸನೆ ಮಾಡಿಕೊಂಡರು. ಅನಂತರ ದೇವ ರಿಚ್ಛೆಯಂತೆಯೇ ವಿಶ್ವ ಮುಂದಕ್ಕೆ ಸಾಗುವುದು ಎಂದು ಊಹಿಸಿಕೊಳ್ಳುವುದರ ಮೂರ್ಖತೆಯ ಬಗ್ಗೆ ಏನೋ ಕಹಿಮಾತನ್ನು ಹೇಳುವರು. “ಗುಡ್ವಿನ್ನನ್ನು ಸಾಯಿಸಿ ದುದಕ್ಕಾಗಿ ಅಂತಹ ದೇವರನ್ನು ಘಾತಿಸಿಬಿಡುವುದು ನಮ್ಮ ಹಕ್ಕಲ್ಲವೆ, ಕರ್ತವ್ಯವಲ್ಲವೆ! ಗುಡ್ವಿನ್ ಬದುಕಿದ್ದಿದ್ದರೆ ಅದೆಷ್ಟೋ ಕಾರ್ಯಸಾಧನೆ ಮಾಡಿರುತ್ತಿದ್ದ!” ಎಂದು ಉದ್ಗರಿಸಿದರು. ಭಾರತದಲ್ಲಿ ಇದನ್ನು ಅತ್ಯಂತ ಧಾರ್ಮಿಕವೆಂದೇ ಪರಿಗಣಿಸುವರು, ಏಕೆಂದರೆ ಭಾವಗಳಲ್ಲೆಲ್ಲ ಇದು ಯಾವ ಅಳುಕೂ ಇಲ್ಲದ, ಸತ್ಯಸ್ಯ ಸತ್ಯವಾದ ಭಾವ!

ಈ ಉದ್ಗಾರದ ಬಗ್ಗೆ ತಿಳಿಸುವಾಗ, ಇದರ ಜೊತೆಗೆ ನಾನು ಹೆಚ್ಚು ಕಡಿಮೆ ಒಂದು ವರ್ಷದ ನಂತರ ಕೇಳಿದ, ನಾವೆಲ್ಲ ಹಗಲುಗನಸು ಕಾಣುತ್ತ ಸುಖಿಸುವುದನ್ನು ಕುರಿತ ಭರ್ತ್ಸನೆಯನ್ನೂ ಹೇಳಬೇಕು. ಆಗ ಅವರೆಂದರು: “ಪ್ರತಿಯೊಬ್ಬ ಸಣ್ಣ ಪುಟ್ಟ ಅಧಿಕಾರಿ, ನ್ಯಾಯಾಧೀಶನಿಗೆ ಸಹ ನಿವೃತ್ತಿ, ವಿಶ್ರಾಂತಿ ಎಂಬುದು ಉಂಟು. ಕೇವಲ ಭಗವಂತನಿಗೆ, ಆ ಚಿರಂತನ ನ್ಯಾಯಾಧೀಶನಿಗೆ ಮಾತ್ರ ಸ್ವಾತಂತ್ರ್ಯವೆಂಬುದಿಲ್ಲ. ಚಿರಕಾಲದವರೆಗೆ ಆತನು ನ್ಯಾಯತೀರ್ಮಾನ ಕೊಡುತ್ತಲೇ ಕುಳಿತಿರಬೇಕು, ಅದೇಕೆ?”

ಆದರೆ ಈ ದಿನ ಬೆಳಗ್ಗೆ ಸ್ವಾಮಿಗಳು ತಮಗಾದ ನಷ್ಟದ ಬಗ್ಗೆ ಶಾಂತರಾಗಿಯೇ ಇದ್ದರು; ನಮ್ಮೊಂದಿಗೆ ಕುಳಿತು ಹರಟೆಯನ್ನೂ ಹೊಡೆದರು. ಆ ದಿನ ಅವರು ಭಕ್ತಿಯಿಂದ ತುಂಬಿ ತುಳುಕುತ್ತ, ಆತ್ಮನನ್ನು ಅಲೆಗಳ ಮೇಲೇರಿಸಿ ವೈಯಕ್ತಿ ಕತೆಗೆ ಎಟುಕದಷ್ಟು ಎತ್ತರಕ್ಕೆಕೊಂಡೊಯ್ಯುವ, ಆದರೂ ತನ್ನ ವ್ಯಕ್ತಿತ್ವದ ಇನಿದಾದ ಮಾಯಾಜಾಲದಿಂದ ತಪ್ಪಿಸಿಕೊಳ್ಳಲು ಭಕ್ತನನ್ನು ಬಿಟ್ಟುಹೋಗುವ, ದಿವ್ಯವಾದ ತಪಸ್ಸಿನ ಭಾವದಲ್ಲಿದ್ದರು.

ತ್ಯಾಗದ ಬಗ್ಗೆ ಆ ದಿನ ಬೆಳಗ್ಗೆ ಅವರು ಹೇಳಿದ್ದು ಶ್ರೋತೃಗಳಲ್ಲೊಬ್ಬರಿಗೆ ಅರಗಿಸಿಕೊಳ್ಳಲು ಆಗದ ಉಪದೇಶವೆನಿಸಿತು. ಅವರು ಮತ್ತೊಮ್ಮೆ ಬಂದಾಗ ಅವಳು, ಬಂಧನವಿಲ್ಲದಂತೆ ಪ್ರೇಮಿಸುವುದರಲ್ಲಿ ನೋವೇನೂ ಇರದು, ಅಂತಹ ಪ್ರೇಮವೇ ಒಂದು ಆದರ್ಶ ಎಂಬುದು ತನ್ನ ನಂಬಿಕೆ - ಎಂದು ನಿವೇದಿಸಿಕೊಂಡಳು.

ಸ್ವಾಮಿಜಿ ಅವಳ ಕಡೆಗೆ ತಿರುಗಿ ಇದ್ದಕ್ಕಿದ್ದಂತೆ ಗಂಭೀರರಾದರು. “ತ್ಯಾಗವಿಲ್ಲದ ಈ ಭಕ್ತಿಯ ಕಲ್ಪನೆಯಾದರೂ ಏನು? ಅದು ಅತ್ಯಂತ ಮಾರಕವಾದುದು!” ಎಂದರು. ಅಲ್ಲೇ ಒಂದು ಗಂಟೆಗಿಂತಲೂ ಹೆಚ್ಚು ಕಾಲ ನಿಂತು, ಒಬ್ಬನು ನಿಜಕ್ಕೂ ಅಸಕ್ತನಾಗಿ ಉಳಿಯ ಬೇಕಾದರೆ ಅಗತ್ಯವಾಗಿ ತನ್ನ ಮೇಲೆ ತಾನು ವಿಧಿಸಿಕೊಳ್ಳಬೇಕಾದ ತೀವ್ರತರ ಆತ್ಮಸಂಯಮವನ್ನು ಕುರಿತು, ಸ್ವಾರ್ಥ ಉದ್ದೇಶಗಳು ಪಾರದರ್ಶಕವಾಗಿರಬೇಕಾದ ಅಗತ್ಯ, ಹಾಗೂ ಮೃದುವಾಗಿ ಹೂವಿನಂತಹ ಜೀವವನ್ನೂ ಸಹ ಬದುಕಿನ ದೌಷ್ಟ್ಯದ ಕಲೆಗಳು ಹೇಗೆ ಮಲಿನಗೊಳಿಸಬಹುದು ಎಂಬುದನ್ನು ಕುರಿತು ಮಾತನಾಡಿದರು. ಒಬ್ಬ ಮನುಷ್ಯ ಯಾವಾಗ ತಾನು ಈ ಪಥದಲ್ಲಿ ಕ್ಷೇಮವಾಗಿರುವೆನೆಂದುಕೊಳ್ಳಬಹುದು ಎಂದು ಕೇಳಿದಾಗ ಉತ್ತರವಾಗಿ ಸ್ವಲ್ಪ ಬೂದಿಯನ್ನು ಕೊಟ್ಟ ಭಾರತದ ಸಂನ್ಯಾಸಿನಿ ಯೊಬ್ಬಳ ಕಥೆಯನ್ನು ಹೇಳಿದರು. ಏಕೆಂದರೆ ಮನೋವಿಕಾರವನ್ನು ನಿಗ್ರಹಿಸಲು ಅಗತ್ಯವಾದ ಹೋರಾಟ ದೀರ್ಘಕಾಲದ್ದು ಹಾಗೂ ತೀಕ್ಷ್ಣತರದ್ದು; ಯಾವುದೇ ಕ್ಷಣದಲ್ಲಿ ಗೆದ್ದೆನೆಂದುಕೊಳ್ಳುವವನು ಸೋತವನಾಗಬಹುದು...

...ಕಾಶ್ಮೀರದಲ್ಲಿ ಕೆಲವು ವಾರಗಳು ಕಳೆದ ನಂತರ, ಮತ್ತೆ ಅವರು ಸಹಾನುಭೂತಿಯಿಂದ ಮಾತನಾಡಲಾರಂಭಿಸಿದರು. ನಮ್ಮಲ್ಲಿ ಒಬ್ಬರು ಅವರನ್ನು ಅಂದು ಹಾಗೆ ನೀವು ಪ್ರದೀಪ್ತಗೊಳಿಸಿದ ಭಾವವು ಯೂರೋಪಿನಲ್ಲಿ ಒಂದು ಜಾಡ್ಯವೆಂದು ಪರಿಗಣಿಸ ಲ್ಪಡುವ ನೋವಿನ ಆರಾಧನೆಯಲ್ಲವೆ - ಎಂದು ಕೇಳಿದರು.

“ಹಾಗಿದ್ದರೆ ನಲಿವಿನ ಆರಾಧನೆ ಬಹಳ ಶ್ರೇಷ್ಠವಾದುದೆ?” ಎಂಬುದು ಅವರ ತಕ್ಷಣದ ಪ್ರತಿಕ್ರಿಯೆಯಾಗಿತ್ತು. ಸ್ವಲ್ಪ ಸುಮ್ಮನಿದ್ದು ಅನಂತರ “ಆದರೆ ನಾವು ನಿಜವಾಗಿ ನೋವನ್ನಾಗಲಿ ನಲಿವನ್ನಾಗಲಿ ಆರಾಧಿಸುವವರಲ್ಲ. ಇವೆರಡರ ಮೂಲಕವೂ ನಾವು ಅರಸುವುದು ಈ ಎರಡಕ್ಕೂ ಅತೀತವಾಗಿರುವುದನ್ನು” ಎಂದರು.

\textbf{ಜೂನ್ ೯.}

ಈ ಗುರುವಾರ ಬೆಳಗ್ಗೆ ಸಂಭಾಷಣೆ ಕೃಷ್ಣನನ್ನು ಕುರಿತದ್ದಾಗಿತ್ತು. ಸ್ವಾಮಿಗಳ ಮನಸ್ಸಿನ ವೈಶಿಷ್ಟ್ಯ ಮತ್ತು ಅವರು ಹುಟ್ಟಿಬೆಳೆದ ಹಿಂದೂ ಸಂಸ್ಕೃತಿಯ ವೈಶಿಷ್ಟ್ಯವೆಂದರೆ, ಒಂದು ದಿನ ಒಂದು ಕಲ್ಪನೆಯ ಚಿತ್ರಣಕ್ಕೆ ಹಾಗೂ ಅದರ ಆನಂದಾನುಭವಕ್ಕೆ ತಮ್ಮನ್ನು ತಾವು ಒಡ್ಡಿಕೊಂಡರೆ ಮಾರನೆಯ ದಿನವೇ ನಿಷ್ಕರುಣ ಆತ್ಮ ವಿಶ್ಲೇಷಣಕ್ಕೊಳಪಟ್ಟು ಘಾಸಿ ಗೊಂಡ ಯೋಧರಂತಾಗಿಬಿಡುತ್ತಿದ್ದುದು. ಒಂದು ಕಲ್ಪನೆಯು ಆಧ್ಯಾತ್ಮಿಕವಾಗಿ ಮೌಲಿಕವಾಗಿದ್ದು ಸತ್ಯವಾಗಿದ್ದರೆ ಸಾಕು; ಅದರ ವಸ್ತುನಿಷ್ಠ ವಾಸ್ತವತೆ ಹೇಗೇ ಇರಲಿ - ಅಂತಹ ತಮ್ಮ ಜನಾಂಗದ ನಂಬಿಕೆಯನ್ನು ತಮ್ಮದೇ ನಂಬಿಕೆ ಎಂಬಂತೆ ಸ್ವೀಕರಿಸುತ್ತಿದ್ದರು. ಅವರ ಬಾಲ್ಯದಲ್ಲಿ ಮೊಟ್ಟಮೊದಲು ಇಂತಹ ಚಿಂತನೆಯನ್ನು ಅವರಿಗೆ ಸಲಹೆಮಾಡಿದವರು ಅವರ ಗುರುಗಳೇ. ಧಾರ್ಮಿಕವಾದ ಯಾವುದೋ ಒಂದು ಸಂಗತಿಯ ಐತಿಹಾಸಿಕ ಸತ್ಯತೆಯ ಬಗ್ಗೆ ಅವರು ಅನುಮಾನವನ್ನು ವ್ಯಕ್ತಪಡಿಸಿದಾಗ, ಶ‍್ರೀ ರಾಮಕೃಷ್ಣರು “ಏನು? ಹಾಗಿದ್ದರೆ ನೀನು ಆ ಕಲ್ಪನೆಯನ್ನು ನಮಗಿತ್ತವರು ತಾವೇ ಅದಾಗಿದ್ದರು ಎಂದು ಭಾವಿಸುವುದಿಲ್ಲವೇ?” ಎಂದಿದ್ದರಂತೆ.

ಎಂದಮೇಲೆ, ಅವರು ಆಗಿಂದಾಗ್ಯೆ ಹೇಳುತ್ತಿದ್ದಂತೆ, ಕೃಷ್ಣನ ಅಸ್ತಿತ್ವವನ್ನು, ಕ್ರೈಸ್ತನ ಅಸ್ತಿತ್ವದ ಹಾಗೆಯೆ, “ಸಾಧಾರಣ ರೀತಿಯಲ್ಲಿ” ಅನುಮಾನಿಸುತ್ತಿದ್ದರು. ಧರ್ಮಬೋಧಕರಲ್ಲಿ ಬುದ್ಧ ಹಾಗೂ ಮಹಮ್ಮದ್ ಇಬ್ಬರು ಮಾತ್ರ “ಶತ್ರುಮಿತ್ರರಿಬ್ಬರನ್ನೂ” ಹೊಂದಿದ್ದ ಅದೃಷ್ಟಶಾಲಿಗಳು; ಏಕೆಂದರೆ ಅದು ಅವರ ಬದುಕಿನ ಚಾರಿತ್ರಿಕತೆಯನ್ನು ವಿವಾದ ತೀತವಾಗಿಸುತ್ತದೆ.ಕೃಷ್ಣನದು ಮಾತ್ರ ಉಳಿದೆಲ್ಲರಿಗಿಂತಲೂ ವಿಶಿಷ್ಟವಾದ ಒಂದು ಛಾಯಾ ರೂಪ. ಕೈಯಲ್ಲಿ ಗೀತೆಯನ್ನು ಹಿಡಿದಿರುವ ಸುಂದರ ಪುರುಷನ ಕಲ್ಪನೆಯಲ್ಲಿ ಒಬ್ಬ ಕವಿ, ಗೋಪಾಲಕ, ರಾಜ್ಯವಾಳುವ ದೊರೆ, ಮಹಾಯೋಧ ಮತ್ತು ಋಷಿ - ಈ ಎಲ್ಲವೂ ಸಮ್ಮಿಳಿತವಾಗಿದೆ.

ಆದರೆ ಇಂದು ಕೃಷ್ಣನು “ಪರಿಪೂರ್ಣ ಅವತಾರ” ಎನ್ನಿಸಿಕೊಂಡಿದ್ದಾನೆ. ರಥದ ಕುದುರೆಗಳ ಲಗಾಮನ್ನು ನಿಯಂತ್ರಿಸುತ್ತ, ರಣರಂಗವನ್ನು ಪಕ್ಷಿವೀಕ್ಷಣೆ ಮಾಡುವುದರಲ್ಲಿಯೇ ಬಲಾಬಲಗಳು ಚದುರಿರುವುದನ್ನು ಗಮನಿಸಿಕೊಂಡು, ಅದೇ ಕ್ಷಣದಲ್ಲಿಯೇ ರಾಜಮನೆತನದ ತನ್ನ ಪ್ರಿಯಶಿಷ್ಯನಿಗೆ ಗೀತೆಯ ನಿಗೂಢ ಆಧ್ಯಾತ್ಮಿಕ ಸತ್ಯಗಳನ್ನು ಬೋಧಿಸು ತ್ತಿರುವ ಪಾರ್ಥಸಾರಥಿಯ ಅದ್ಭುತವಾದ ಒಂದು ಚಿತ್ರವನ್ನು ಜನರು ಅನುಸರಿಸುತ್ತಿರುವರು.

ಈ ಬೇಸಗೆಯಲ್ಲಿ ನಾವು ಉತ್ತರ ಭಾರತದಲ್ಲಿ ಪ್ರವಾಸವನ್ನು ಕೈಗೊಂಡು ಹಳ್ಳಿಗಾಡುಗಳ ಮೂಲಕ ಸಂಚರಿಸುತ್ತಿದ್ದಾಗ ನಿಜಕ್ಕೂ ನಮಗೆ ಕೃಷ್ಣನ ಕಲ್ಪನೆ ಜನಮನದಲ್ಲಿ ಅದೆಷ್ಟು ಆಳವಾಗಿ ನೆಲೆಗೊಂಡಿದೆ ಎಂಬುದನ್ನು ಅರಿತುಕೊಳ್ಳುವ ಅನೇಕ ಅವಕಾಶಗಳು ದೊರಕಿದವು. ರಸ್ತೆಪಕ್ಕದ ಹಳ್ಳಿಗಳಲ್ಲೆಲ್ಲ ಜನರು ಪ್ರದರ್ಶಿಸಿದ ಹಾಡು - ಕುಣಿತಗಳೆಲ್ಲ ರಾಧಾ ಮತ್ತು ಕೃಷ್ಣರನ್ನೇ ಕುರಿತಾದವುಗಳಾಗಿದ್ದವು. ಭಾರತದಲ್ಲಿ ಕೃಷ್ಣನ ಆರಾಧಕರು ಪ್ರೇಮ ಭಾವೋ ದ್ದೀಪನೆಯ ಸಾಧ್ಯತೆಗಳನ್ನೆಲ್ಲ ಸಮಗ್ರವಾಗಿ ಶೋಧಿಸಿಬಿಟ್ಟಿರುವರು ಎಂದರು ಸ್ವಾಮಿಗಳು. ಅವರಿಗಿಷ್ಟವಾದ ಈ ಅಭಿಪ್ರಾಯಕ್ಕೆ ಪ್ರತಿಕ್ರಿಯೆಯಾಗಿ ನಾವೇನೂ ಹೇಳಲಾರದಾದೆವು...

ಆದರೆ ಈ ಎಲ್ಲ ದಿನಗಳಲ್ಲಿ ಸ್ವಾಮಿಗಳು ತಾವೊಬ್ಬರೇ ದೂರ ಹೋಗಿ ಏಕಾಂಗಿಯಾಗಿರಬೇಕೆಂದು ತಹತಹಪಡುತ್ತಿದ್ದರು. ಯಾವ ಸ್ಥಳದಲ್ಲಿ ಗುಡ್ವಿನ್ ಇನ್ನಿಲ್ಲವಾದ ಸಮಾಚಾರವನ್ನು ಕೇಳಿದ್ದರೋ ಅದು ಅವರಿಗೆ ಸಹಿಸಲಸಾಧ್ಯವಾಗಿದ್ದಿತು; ಅಲ್ಲಿಗೆ ಪತ್ರಗಳು ಮೇಲಿಂದ ಮೇಲೆ ಬರುವುದು, ಅಲ್ಲಿಂದ ಪತ್ರಗಳನ್ನು ಬರೆಯುವುದು ಈ ನೋವನ್ನು ಇನ್ನೂ ಹೆಚ್ಚಿಸಿತ್ತು. ಒಂದು ದಿನ ಅವರು ಶ‍್ರೀರಾಮಕೃಷ್ಣರು ಹೊರಗೆಲ್ಲ ಭಕ್ತಿ ಎಂಬಂತೆ ಕಂಡರೂ ನಿಜವಾಗಿ ಒಳಗೆಲ್ಲ ಜ್ಞಾನವಾಗಿದ್ದರು, ಆದರೆ ತಾವು ಹೊರಗೆ ಜ್ಞಾನವಾಗಿರುವಂತೆ ಕಾಣುತ್ತಿದ್ದರೂ ಒಳಗೆ ಭಕ್ತಿ ತುಂಬಿದ್ದು, ಅದರಿಂದಾಗಿ ತಾವು ಒಬ್ಬ ಹೆಂಗಸಿನ ಹಾಗೆ ದುರ್ಬಲರೆಂದೂ ಹೇಳಿದರು.

ಒಂದು ದಿನ ಅವರು ಯಾರೋ ಬರೆದಿದ್ದ ಕೆಲವು ತಪ್ಪು ಸಾಲುಗಳನ್ನು ತೆಗೆದುಕೊಂಡು ಹೋಗಿ, ಒಂದು ಪುಟ್ಟ ಕವನದೊಂದಿಗೆ ಹಿಂದಿರುಗಿದರು; ಅದನ್ನು ಗುಡ್ ವಿನ್ರ ವಿಧವೆ ತಾಯಿಗೆ ನಿಮ್ಮ ಮಗನ ನೆನಪಿಗಾಗಿ ಎಂದು ಕಳುಹಿಸಿಕೊಟ್ಟರು...(ಕೃತಿಶ್ರೇಣಿ ಸಂಪುಟ ೮, ಪುಟ ೩೬೫) ಆ ನಂತರ, ಮೂಲದ ಏನನ್ನೂ ಉಳಿಸದೆ ಇದ್ದುದಕ್ಕಾಗಿ, ಆಕೆಗೆ ನೋವಾಗ ಬಹುದೆಂದೆನಿಸಿ (ಏಕೆಂದರೆ ಆಕೆಯ ಸಾಲುಗಳು “ಮೂರು ಛಂದಸ್ಸು”ಗಳಲ್ಲಿದ್ದವು), ಪ್ರಾಸ ಛಂದಸ್ಸುಗಳಿಗೆ ಅನುಗುಣವಾಗಿ ಪದಗಳನ್ನು ಕೊಂದುಬಿಡುವುದಕ್ಕಿಂತ ಕಾವ್ಯದ ಪರಿಭಾವನೆ ಹೆಚ್ಚು ಮುಖ್ಯವಾದದ್ದು ಎಂದು ಮುಂತಾಗಿ ದೀರ್ಘ ವಿವರಣೆ ಕೊಟ್ಟರು\footnote{1. ಪುಸ್ತಕದ ಮೂಲಪ್ರತಿಯಲ್ಲಿ ಸೋದರಿ ನಿವೇದಿತಾ ಈ ಟಿಪ್ಪಣಿಯನ್ನು ಸೇರಿಸಿರುವಳು:

“ಸೋದರಿ ಸಾರಾ ಅವರ ಅಪೇಕ್ಷೆಯಂತೆ ನಾನಿಲ್ಲಿ ಮೂಲ ಕವನದ ಸಾಲುಗಳನ್ನು ಕೊಡುತ್ತಿರುವೆನು:

\begin{myquote}
ತಾರೆ ಚೆಲ್ಲಿದ ಪಥದಿ ನಡೆದೋಡು, ಓ ಜೀವ,\\ನಿನ್ನ ಧ್ಯಾನದ ಹಾದಿಯಲಿ ನೀನು ವಿಸ್ತರಿಸು,\\ಕಾಲ, ಗ್ರಹಿಕೆಯ ಮಂಜಿನಿಂದೊದೆದು-\\ಮೇಲೇಳು!
\end{myquote}

\begin{myquote}
ಹೃದಯದಾಳದ ಪ್ರೇಮಗೇಹದಿಂದ ಮೀರಿದುದು\\ಪೂರ್ಣಗೊಂಡಿತು ಸೇವೆ, ಸಂದಿತಲ್ಲವೆ ಸೇವೆ\\ಧೂಪಗಂಧವು, ಪುಷ್ಪ, ಸಂಗೀತ ತುಂಬಿಹುದು\\ನೀನಿದ್ದ ಜಾಗವನು,\\ದಣಿದ ಜೀವವೆ ಮಲಗು!
\end{myquote}

\begin{myquote}
ಬಿಡಿಸಿಕೊಂಡೆಯ ನೀನು ಬಂಧನಗಳೆಲ್ಲವನು\\ಮುಗಿಸಿದೆಯ ಶೋಧವನು, ಮುಳುಗಿದೆಯ ಸಾಗರದಿ\\ಕರಗಿದೆಯ ಅಮೃತದೊಳೊಂದಾಗಿ ನೀನು!\\ನಂದದಾತ್ಮನೆ ನಿನಗೆ\\ಶಾಂತಿಯಿರಲೆನುವೆ.
\end{myquote}

ಇದರಲ್ಲಿ ಮೂರು ಲಯಗಳಿವೆ - ಎಂದರು ಗುರುದೇವರು. ಮತ್ತೆ - ಹೃದಯದಾಳದ ಪ್ರೇಮಗೇಹದಿಂದ ಹೊಮ್ಮಿದುದು ಎಂಬ ಸಾಲು ನಿಜವಾಗಿಯೂ ಕಾವ್ಯಧ್ವನಿಯನ್ನು ಸ್ಪಷ್ಟವಾಗಿ ಉಲಿಯುತ್ತಿದೆ - ಎಂದೂ ನುಡಿದರು.}. ತಮ್ಮ ದೃಷ್ಟಿಯಲ್ಲಿ ಮಿಥ್ಯೆ, ಭಾವಾತಿರೇಕದ್ದು ಎಂದೆನಿಸಿದ ಒಂದು ಸಹಾನುಭೂತಿ ಅಥವಾ ಅಭಿಪ್ರಾಯದ ವಿಚಾರದಲ್ಲಿ ಅವರು ತುಂಬ ಕಠೋರರಾಗಿದ್ದಿರಬಹುದು. ಆದರೆ ಸೋತ ಒಂದು ಪ್ರಯತ್ನದ ರಕ್ಷಣೆಗೆ, ಅದನ್ನು ಎತ್ತಿಹಿಡಿಯುವುದಕ್ಕೆ ಗುರು ದೇವರು ಯಾವಾಗಲೂ ಸಿದ್ಧರಾಗಿರುತ್ತಿದ್ದರು.

ಆ ತಾಯಿಯು ತಮ್ಮ ದುಃಖದ ನಡುವೆಯೇ -ದೂರದಲ್ಲಿ ತೀರಿಕೊಂಡ ನನ್ನ ಮಗನ ಮೇಲಿನ ನಿಮ್ಮ ದಿವ್ಯಪ್ರಭಾವಕ್ಕಾಗಿ ವಂದಿಸುವೆನು ಎಂದು ಬರೆದು ತಿಳಿಸಿದ ಮಾರೋಲೆ ಅದೆಷ್ಟು ಸಂತೋಷಭರಿತವಾಗಿದ್ದಿತು!

\textbf{ಜೂನ್ ೧೦.}

ಶ‍್ರೀರಾಮಕೃಷ್ಣರನ್ನು ಮರಣೋನ್ಮುಖರನ್ನಾಗಿಸಿದ ಖಾಯಿಲೆಯ ವಿಚಾರವನ್ನು ನಾವು ಕೇಳಿದ್ದು ಆಲ್ಮೋರದಲ್ಲಿದ್ದ ಕೊನೆಯ ದಿನ ಮಧ್ಯಾಹ್ನ. ಕರೆಯಿಸಲಾಗಿದ್ದ ಡಾ. ಮಹೇಂದ್ರ ಲಾಲ ಸರ್ಕಾರ್ ಅವರು, ಖಾಯಿಲೆಯು ಗಂಟಲಿನ ಕ್ಯಾನ್ಸರ್ ಎಂದು ತಿಳಿಸಿದರಲ್ಲದೆ ಅದೊಂದು ಸೋಂಕುರೋಗ ಎಂದು ಶಿಷ್ಯರುಗಳನ್ನು ಎಚ್ಚರಿಸಿಯೂ ಹೋಗಿದ್ದರು. ಅರ್ಧಘಂಟೆಯ ನಂತರ ಅಲ್ಲಿಗೆ ಬಂದ “ನರೇಂದ್ರ” ನು, - ಅವರಿಗೆ ಆಗ ಆ ಹೆಸರಿದ್ದಿತು - ಅವರೆಲ್ಲರೂ ಒಂದಾಗಿ ಸೇರಿ ಖಾಯಿಲೆಯ ಭೀಕರತೆಯನ್ನು ಕುರಿತು ಚರ್ಚಿಸುತ್ತಿ ದ್ದುದನ್ನು ನೋಡಿದನು. ಅವರಿಗೆ ವೈದ್ಯರು ಹೇಳಿದ್ದೆಲ್ಲವನ್ನೂ ಕೇಳಿಸಿಕೊಂಡನು. ಕೆಳಗಡೆ ದೃಷ್ಟಿ ಹಾಯಿಸಿದ ಅವನಿಗೆ ಶ‍್ರೀರಾಮಕೃಷ್ಣರ ಪಾದಗಳ ಬಳಿ ಇದ್ದ ಅವರು ಅರ್ಧ ಕುಡಿದಿಟ್ಟಿದ್ದ ಗಂಜಿಯ ಪಾತ್ರೆಯು ಕಾಣಿಸಿತು. ಗಾಯವಾಗಿದ್ದ ಗಂಟಲಿನಲ್ಲಿಯ ಅನ್ನ ನಾಳದ ಇಕ್ಕಟ್ಟಿನ ಮೂಲಕ ನುಂಗುವ ಕಷ್ಟದ ಪ್ರಯತ್ನದಲ್ಲಿ ಮತ್ತೆ ಮತ್ತೆ ಹೊರಗೆ ಬಂದು ಬಿಡುತ್ತಿದ್ದ ಅದರ ಜೊತೆಗೆ ಅಲ್ಲಿಯ ಕಫ ಹಾಗೂ ಖಾಯಿಲೆಗೆ ಕಾರಣವಾಗಬಹುದಾದ ಸೋಂಕುನ್ನುಳ್ಳ ಕೀವು ಸಹ ಗಂಜಿಯ ಜೊತೆಗೆ ಮಿಶ್ರವಾಗಿದ್ದಿರಬಹುದು. ಅವರೆಲ್ಲರ ಎದುರಿಗೆ ಅದನ್ನು ಅವನು ಎತ್ತಿಕೊಂಡು ಕುಡಿದುಬಿಟ್ಟನು. ಆ ನಂತರ ಕ್ಯಾನ್ಸರ್ ನ ಸೋಂಕಿನ ವಿಚಾರವಾಗಿ ಯಾರೊಬ್ಬರೂ ಎಂದೂ ಉಸಿರೆತ್ತಲಿಲ್ಲ.

