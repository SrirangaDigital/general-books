
\chapter{ಟಿಪ್ಪಣಿ}

(ಸ್ವಾಮಿ ವಿವೇಕಾನಂದರಿಂದಲೇ ಬರೆಯಲ್ಪಟ್ಟ ದಿನಾಂಕವಿಲ್ಲದ, ಶಿರೋನಾಮವಿಲ್ಲದ ಟಿಪ್ಪಣಿ)

ನನ್ನ ನರಗಳು ನನ್ನ ಮಿದುಳಿನ ಮೇಲೆ ಪ್ರಭಾವಬೀರುತ್ತವೆ - ಮಿದುಳು ಪ್ರತಿ ಕ್ರಿಯೆಯನ್ನು ಕಳಿಸುತ್ತದೆ. ಈ ಪ್ರತಿಕ್ರಿಯೆಯನ್ನೇ ಮಾನಸಿಕ ದೃಷ್ಟಿಯಿಂದ ನಾವು ಜಗತ್ತು ಎಂದು ಕರೆಯುವುದು.

ನರಗಳ ಮೂಲಕ ಯಾವುದೊ ಒಂದು ಮಿದುಳಿನ ಮೇಲೆ ಪರಿಣಾಮವ ನ್ನುಂಟು ಮಾಡುತ್ತದೆ. ಇದರ ಪ್ರತಿಕ್ರಿಯೆಯೇ ಈ ಜಗತ್ತು.

ದೃಶ್ಯವು ದೇಹದ ಹೊರಗೆ ಮಾತ್ರ ಏಕಿರಬೇಕು - ಒಳಗೂ ಏಕಿರಬಾರದು?

ಏಕೆಂದರೆ ಹಿಂದಿನ ಪ್ರತಿಕ್ರಿಯೆಯ ಮೂಲಕ ಸೃಷ್ಟಿ ಸಲ್ಪಟ್ಟ ಹೊರ ಜಗತ್ತು ನಮ್ಮ ಮೇಲೆ ಪರಿಣಾಮವನ್ನುಂಟುಮಾಡಿ ಮುಂದಿನ ಪ್ರತಿಕ್ರಿಯೆಗೆ ಕಾರಣವಾಗುತ್ತದೆ.

ಹೀಗೆ ಒಳಗು ಹೊರಗಾಗುತ್ತದೆ ಮತ್ತು ಇನ್ನೊಂದು ಕ್ರಿಯೆಯನ್ನು ಸೃಷ್ಟಿಸುತ್ತದೆ. ಈ ಆಂತರಿಕ ಕ್ರಿಯೆಯು ಇನ್ನೊಂದು ಪ್ರತಿಕ್ರಿಯೆಯನ್ನು ಸೃಷ್ಟಿಸುತ್ತದೆ. ಇದು ಮತ್ತೆ ಬಹಿರ್ಮುಖವಾಗಿ ಅಂತರಂಗದ ಮೇಲೆ ಪರಿಣಾಮವನ್ನುಂಟುಮಾಡುತ್ತದೆ.

ವಸ್ತುಸತ್ತಾವಾದ ಮತ್ತು ಭಾವಸತ್ತಾವಾದ ಇವುಗಳಲ್ಲಿ ಹೊಂದಾಣಿಕೆಯನ್ನು ಉಂಟುಮಾಡುವ ಒಂದೇ ಮಾರ್ಗ ಇದು: ಒಂದು ಮಿದುಳು ಆಂತರಿಕ ಪ್ರತಿ ಕ್ರಿಯೆಯಿಂದ ಸೃಷ್ಟಿಸಿದ ಜಗತ್ತಿನಿಂದ ಇನ್ನೊಂದು ಮಿದುಳು ಪ್ರಭಾವಿತವಾಗುತ್ತದೆ. ಅಂದರೆ ಈ ರೀತಿ ಸೃಷ್ಟಿಸಲ್ಪಟ್ಟ ಜಗತ್ತು ದೃಶ್ಯ ಮತ್ತು ಮನಸ್ಸಿನ ಮಿಶ್ರಣ. ಈ ಪ್ರತಿ ಕ್ರಿಯೆಯು ಇನ್ನೊಂದು ಮಿದುಳಿನ ಮೇಲೆ ಪರಿಣಾಮವನ್ನುಂಟುಮಾಡುತ್ತದೆ ಮತ್ತು ಅದರ ಪಾಲಿಗೆ ಇದು ಬಾಹ್ಯ ದೃಶ್ಯವಾಗಿರುತ್ತದೆ.

ಆದ್ದರಿಂದ ಹೀಗೆ ನೂರಾರು ಹಿಂದಿನ ಮಿದುಳುಗಳು ಸೃಷ್ಟಿಸಿದ ಪ್ರಭಾವದ ವಲಯದೊಳಗೆ ನಾವು ಬಂದಾಗ ನಾವು ಈಗ ನೋಡುತ್ತಿರುವಂತೆ ಈ ಜಗತ್ತನ್ನು ಅನುಭವಿಸುತ್ತೇವೆ.

ಮನಸ್ಸು ಎಂಬುದು ದ್ರವ್ಯದ ಒಂದು ರೂಪ. ನಿರಂತರವಾಗಿ ಬದಲಾಗುತ್ತಿರುವ ಪ್ರಾಕೃತಿಕ ಘಟನೆಯ ಒಂದು ರೂಪ ದ್ರವ್ಯವಾದರೆ, ಇನ್ನೊಂದು ರೂಪ ಮನಸ್ಸು. ಈ ಶಾಶ್ವತವಾದ ಘಟನಾ ಜಾಲವನ್ನು ಸಾಕ್ಷಿಯಾಗಿ ನೋಡುವವನೊಬ್ಬನಿರಬೇಕು - ಅವನೇ ಬ್ರಹ್ಮ ಎಂದು ಕರೆಯಲ್ಪಡುವ ಮೂಲ ಅಸ್ತಿತ್ವ.

