
\addtocontents{toc}{\protect\vspace{-0.7cm}}


\part{ಉಕ್ತಿಗಳು ಮತ್ತು ವಚನಗಳು}

\chapter{ಸಂಕ್ಷಿಪ್ತಗಳ ಪಟ್ಟಿ}

ಈ ಭಾಗದಲ್ಲಿ ಸ್ವಾಮಿ ವಿವೇಕಾನಂದರ ನುಡಿಗಳನ್ನು ಮಾತ್ರವೇ ಉದ್ಧರಣ ಚಿಹ್ನೆಗಳ ಒಳಗೆ ಅಳವಡಿಸಲಾಗಿದೆ. ಆಕರಗಳನ್ನು ಕೆಳಕಂಡ ಸಂಕ್ಷಿಪ್ತಗಳಿಂದ ಗುರುತಿಸಬಹುದಾಗಿದೆ:

\enginline{\textbf{ND:} Burke, Marie Louise. Swami Vivekananda in the West: New Discoveries. 6 vols. Calcutta: Advaita Ashrama, 1983-87.}

\enginline{\textbf{CWSN:} Nivedita. Sister. The Complete Works of Sister Nivedita. Vol. 1. Calcutta: Advaita Ashrama, 1982.}

\enginline{\textbf{LSN:} Nivedita. Sister. Letters of Sister Nivedita 2 vols. Compiled and edited by Sankari Prasad Basu. Calcutta: Nababharat Publishers. 1982.}

\enginline{\textbf{VIN:} Basu, Sankari Prasad and Ghosh. Sunil Bihari, eds. Vivekananda in Indian Newspapers: 1830-1920 Calcutta: Dineshchanda Basu, Basu Bhattacharya and Co., 1969}

೧. ೧೮೯೩ರ ಆಗಸ್ಟ್ ಮೆಸಾಚುಸೆಟ್ಸ್ನ ಸಲೆಮ್​ನಲ್ಲಿನ ವುಡ್ಸ್ರವರ ನಿವಾಸವನ್ನು ಬಿಟ್ಟು ಸ್ವಾಮಿ ವಿವೇಕಾನಂದರು ಹೊರಟ ಸಂದರ್ಭ. ಮಿಸೆಸ್ ಪ್ರಿನ್ಸ್ ವುಡ್ಸ್ ಅವರ ವಿವರಣೆ: ಸ್ವಾಮಿ ವಿವೇಕಾನಂದರು ತಮ್ಮ ಅತ್ಯಮೂಲ್ಯ ಆಸ್ತಿಯಾಗಿದ್ದ ಊರು ಗೋಲನ್ನು ಆಗ ತರುಣ ವೈದ್ಯ ವಿದ್ಯಾರ್ಥಿಯಾಗಿದ್ದ ಡಾ. ವುಡ್ಸ್ ಅವರಿಗೆ ಹಾಗೂ ತಮ್ಮ ಟ್ರಂಕ್ ಮತ್ತು ಉಣ್ಣೆಯ ದುಪ್ಪಟಿಯನ್ನು ಮಿಸೆಸ್ ಕಾತೆ ಟಿ. ವುಡ್ಸ್ ಅವರಿಗೂ ಕೊಡುತ್ತ ಹೇಳಿದ್ದು:

“ಈ ದೊಡ್ಡ ದೇಶದಲ್ಲಿ ನನ್ನನ್ನು ಮನೆಯವನಂತೆಯೇ ಆದರಿಸಿದ ನನ್ನ ಸ್ನೇಹಿತ ರಿಗೆ ನಾನು ನನ್ನ ಅತ್ಯಮೂಲ್ಯವಾದ ಆಸ್ತಿಯನ್ನು ಮಾತ್ರವೇ ಕೊಡತಕ್ಕದ್ದು”. (\enginline{ND 1:42)}

೨. ೧೮೯೪ ಫೆಬ್ರವರಿ ೧೧ರಂದು ಸ್ವಾಮಿ ವಿವೇಕಾನಂದರು ಲೂಯಿ ರೊಸೆಲೆಟ್ ಅವರ “ಭಾರತ ಮತ್ತು ಅಲ್ಲಿಯ ದೇಶೀಯ ರಾಜರುಗಳು-ಮಧ್ಯಭಾರತ ಹಾಗೂ ಮುಂಬಯಿ ಮತ್ತು ಬಂಗಾಳ ಪ್ರಾಂತ್ಯಗಳಲ್ಲಿ ಪ್ರವಾಸ” ಎಂಬ ಪುಸ್ತಕದ ಬೆನ್ನಿನ ಮೇಲೆ ಬರೆದದ್ದು ಪ್ರತಿಲೇಖನ:

“ಭವಿಷ್ಯದ ಬಗ್ಗೆ ತೀರ ಕಾತರರಾದವರಿಗೆ ಕೇವಲ ಒಂದೇ ಒಂದು ಪರಿಹಾರವಿದೆ ಎಂದು ನಾನು ಹೇಳುತ್ತೇನೆ - ಅದೇ ಮೊಣಕಾಲೂರಿ ಪ್ರಾರ್ಥಿಸುವುದು”.

(\enginline{ND 1:225})

೩. ಚಿಕಾಗೋ ಸರ್ವಧರ್ಮ ಸಮ್ಮೇಳನದಲ್ಲಿ ಸ್ವಾಮಿ ವಿವೇಕಾನಂದರು ಮಾಡಿದ ಪ್ರಾರ್ಥನೆಯಿಂದ ಆರಿಸಿದ್ದು:

“ಈ ವಿಶ್ವದ ಕಷ್ಟಕೋಟಲೆಗಳ ಭಾರವನ್ನು ಹೊರುವವನು ನೀನು; ಈ ಜೀವನವೆಂಬ ಸಣ್ಣ ಹೊರೆಯನ್ನು ತಾಳಿಕೊಳ್ಳಲು ನನಗೆ ನೆರವಾಗು”. (\enginline{ND 2:32})

೪. ಚಿಕಾಗೋ ಸರ್ವಧರ್ಮ ಸಮ್ಮೇಳನದಲ್ಲಿ ಸ್ವಾಮಿ ವಿವೇಕಾನಂದರು ಮಾಡಿದ ಇನ್ನೊಂದು ಪ್ರಾರ್ಥನೆಯಿಂದ ಆರಿಸಿದ್ದು:

“ಈ ನಿಯಮಗಳೆಲ್ಲದರ ಅಧಿಷ್ಠಾನವಾಗಿ, ಬಲದ ಹಾಗೂ ದ್ರವ್ಯದ ಪ್ರತಿಯೊಂದು ಕಣದಲ್ಲಿಯೂ ಹಾಸುಹೊಕ್ಕಾಗಿ ನಿಲ್ಲುವ ಏಕನು ಅವನು. ಗಾಳಿ ಬೀಸುವುದು, ಬೆಂಕಿ ಉರಿಯುವುದು, ಮೋಡಗಳು ಮಳೆಗರೆಯುವುದು ಮತ್ತು ಮೃತ್ಯು ಈ ಭೂಮಿಯಲ್ಲಿ ಹೊಂಚುಹಾಕುತ್ತಿರುವುದು ಅವನ ಆಣತಿಯಿಂದಲೇ. ಅವನ ಸ್ವರೂಪವೆಂಥದು? ಸರ್ವವ್ಯಾಪಿಯಾದ ಅವನು ಶುದ್ಧ, ರೂಪರಹಿತ, ಸರ್ವಶಕ್ತ, ದಯಾಮಯ. ನಮ್ಮೆಲ್ಲರ ಪಿತೃವೂ ನೀನೇ; ನಮ್ಮೆಲ್ಲರ ಪ್ರೀತಿಯ ಗೆಳೆಯನೂ ನೀನೇ”. (\enginline{ND 2:22})

೫. ಮೇರಿ ಟಿ. ರೈಟ್ ಅವರ ೧೮೯೪ ಮೇ ೧೨ರ ದಿನಚರಿಯಿಂದ:

ಭಾರತದ ಉತ್ತಮ ಜಾತಿಯವರಲ್ಲಿ ವಿಧವೆಯರು ವಿವಾಹವಾಗುವುದಿಲ್ಲ, ಎಂದ ರವರು; ಕೀಳು ಜಾತಿಯ ವಿಧವೆಯರು ಮಾತ್ರ ಮದುವೆಯಾಗಬಹುದು, ತಿಂದು ಕುಡಿದು ನರ್ತಿಸಬಹುದು, ಎಷ್ಟು ಮಂದಿ ಗಂಡಂದಿರನ್ನು ಬೇಕಾದರೂ ಆರಿಸಿಕೊಳ್ಳಬಹುದು, ಎಲ್ಲರಿಗೂ ವಿಚ್ಛೇದನ ಕೊಡಬಹುದು; ಸಾರಾಂಶವೆಂದರೆ ಈ ದೇಶದ ಉತ್ತಮರ ಸಮಾಜದ ಲಾಭಗಳೆಲ್ಲವನ್ನೂ ಪಡೆಯಬಹುದು...

“ನಾವು ಮತಾಂಧರಾದಾಗ, ನಮ್ಮನ್ನು ನಾವು ಹಿಂಸಿಸಿಕೊಳ್ಳುತ್ತೇವೆ, ರಥದ ಚಕ್ರದ ಕೆಳಗೆ ಬೀಳುತ್ತೇವೆ, ಕತ್ತನ್ನು ಸೀಳಿಕೊಳ್ಳುತ್ತೇವೆ, ಮುಳ್ಳುಹಾಸಿನ ಮೇಲೆ ಮಲಗುತ್ತೇವೆ; ಆದರೆ ನೀವು ಮತಾಂಧರಾದಾಗ ನೀವು ಇನ್ನೊಬ್ಬರ ಕತ್ತನ್ನು ಕೊಯ್ಯುವಿರಿ, ಬೆಂಕಿಯಲ್ಲಿ ಎಸೆದು ಹಿಂಸಿಸುವಿರಿ, ಅವರನ್ನು ಮುಳ್ಳುಹಾಸಿನ ಮೇಲೆ ಮಲಗಿಸುವಿರಿ! ನಿಮ್ಮ ಯೋಗ ಕ್ಷೇಮವನ್ನು ಮಾತ್ರ ನೀವು ಚೆನ್ನಾಗಿಯೇ ನೋಡಿಕೊಳ್ಳುವಿರಿ!” (\enginline{ND ೨:೫೮-೫೯})

೬. ಮೈನೆಯ ಗ್ರೀನೆಕರ್ನಲ್ಲಿ ಮಾಡಿದ ಸ್ವಾಮಿಗಳ ಬೋಧನೆಯೊಂದನ್ನು ಉದ್ಧರಿಸಿ ಗ್ರೀನೆಕರ್ ವಾಯ್ಸ್ ಪತ್ರಿಕೆಯ ೧೮೯೪ರಲ್ಲಿನ ವರದಿಯೊಂದರ ಸಾರಾಂಶದಿಂದ:

“ನೀವು, ನಾನು ಮತ್ತು ವಿಶ್ವದ ಪ್ರತಿಯೊಂದೂ ಸಹ ನಿರಪೇಕ್ಷವೇ ಆಗಿರುವೆವು-ಭಾಗಗಳಲ್ಲದ ಇಡಿಯ ನಿರಪೇಕ್ಷವೇ. ಆ ಇಡಿಯ ನಿರಪೇಕ್ಷವೇ ನೀವಾಗಿರುವಿರಿ”. (\enginline{ND 2:150})

೭. ಸೋದರಿ ನಿವೇದಿತಾ ೧೮೯೯ರ ಮಾರ್ಚ್ ೫ರಂದು ಮಿಸ್ ಜೋಸೆಫಿನ್ ಮ್ಯಾಕ್ಲಿಯಾಡ್ಳಿಗೆ ಬರೆದ ಕಾಗದವೊಂದರಲ್ಲಿ:

“ಅಂತರಂಗದಲ್ಲಿ ನಾನೊಬ್ಬ ಅನುಭಾವಿ, ಮಾರ್ಗಾಟ್, ಈ ವಿಚಾರವೆಲ್ಲ ಕೇವಲ ತೋರಿಕೆಯದು - ನಿಜವಾಗಿ ನಾನು ಯಾವಾಗಲೂ ಚಿಹ್ನೆಗಳನ್ನು, ವಸ್ತುವನ್ನು ಅರಸುತ್ತಿರುತ್ತೇನೆ - ಆದಕಾರಣ ನನ್ನಿಂದ ದೀಕ್ಷೆ ಪಡೆದವರ ಭವಿಷ್ಯವೇನಾಗುವುದೆಂದು ನಾನು ತಲೆ ಕೆಡಿಸಿಕೊಳ್ಳುವುದಿಲ್ಲ. ಸಂನ್ಯಾಸಿಗಳಾಗಬೇಕೆಂದು ಅವರು ಸಾಕಷ್ಟು ತೀವ್ರವಾಗಿ ಅಪೇಕ್ಷಿಸಿದರೆ, ಮುಂದಿನದು ನನ್ನ ಕೆಲಸವಲ್ಲ ಎಂದು ನನಗನಿಸುತ್ತದೆ. ಅದರಲ್ಲಿ ಕೆಡುಕಿನ ಭಾಗವೂ ಇದೆ. ಕೆಲವೊಮ್ಮೆ ನಾನು ಮಾಡಿದ ಘೋರಕ್ಕೆ ಪ್ರತಿಯಾಗಿ ತುಂಬ ಕಷ್ಟವನ್ನು ಅನುಭವಿಸಬೇಕಾಗುತ್ತದೆ -ಆದರೆ ಅದರಲ್ಲೂ ಒಂದು ಪ್ರಯೋಜನವಿದೆ. ಇವೆಲ್ಲವುಗಳ ನಡುವೆಯೂ ಅದು ನನ್ನನ್ನು ಇನ್ನೂ ಸಂನ್ಯಾಸಿಯಾಗಿಯೇ ಉಳಿಸಿದೆ - ಅದೇ ನನ್ನ ಹೆಬ್ಬಯಕೆ, ರಾಮಕೃಷ್ಣ ಪರಮಹಂಸರು ನಿಜವಾಗಿಯೂ ಇದ್ದ ಹಾಗೆ ನೈಜ ಸಂನ್ಯಾಸಿಯಾಗಿ ಸಾಯುವುದು - ಕಾಮದಿಂದ ಮುಕ್ತನಾಗಿ - ಮತ್ತು ಸಿರಿಯ ಆಸೆಯಿಂದ ಮುಕ್ತನಾಗಿ, ಕೀರ್ತಿಯ ಹಂಬಲದಿಂದ ಮುಕ್ತನಾಗಿ. ಈ ಕೀರ್ತಿಯ ಹಂಬಲವೇ ಕೇಡುಗಳಲ್ಲೆಲ್ಲಾ ಅತಿ ಕೆಟ್ಟದ್ದು”. (\enginline{ND 3:128-29})

೮. ಜಾನ್ ಹೆನ್ರಿ ರೈಟ್ ೧೮೯೬ರ ಮಾರ್ಚ್ ೨೭ರಂದು ಮೇರಿ ಟಪ್ಪನ್ ರೈಟ್ಗಳಿಗೆ ಬರೆದ ಕಾಗದದಲ್ಲಿ - ಇದರಲ್ಲಿ ಸ್ವಾಮಿ ವಿವೇಕಾನಂದರು ಇಂಗ್ಲೆಂಡ್ ಸಹ ಭಾರತದಂತೆಯೇ ಜಾತಿಗಳಿಂದ ಕೂಡಿದೆ ಎಂದಿರುವರು:

“ಎರಡು ಜಾತಿಯವರಿಗಾಗಿ ನಾನು ಎರಡು ಪ್ರತ್ಯೇಕ ತರಗತಿಗಳನ್ನು ನಡೆಸಬೇಕಾಯಿತು. ಉನ್ನತ ಜಾತಿಯವರಿಗೆ - ಈ ಲೇಡಿ ಮತ್ತು ಆ ಲೇಡಿ ಮಾನ್ಯ ಇವರಿಗಾಗಿ ಮತ್ತು ಮಾನ್ಯ ಇವರಿಗಾಗಿ - ನಾನು ಬೆಳಗಿನ ಹೊತ್ತು ತರಗತಿಗಳನ್ನು ನಡೆಸಿದೆ; ಎಗ್ಗಿಲ್ಲದೆ ಬರುತ್ತಿದ್ದ ಕೀಳು ಜಾತಿಯವರಿಗೆ ಸಂಜೆಯ ತರಗತಿಗಳನ್ನು ನಡೆಸಿದೆ”. (\enginline{ND 4:73})

೯. ೧೮೯೬ರ ಬೇಸಿಗೆಯಲ್ಲಿ, ಸ್ವಿಟ್ಸರ್ಲೆಂಡ್ನ ಪುಟ್ಟ ದೇಗುಲವೊಂದರಲ್ಲಿ ಕನ್ಯೆ ಮೇರಿಯ ಪಾದಾರವಿಂದಗಳಲ್ಲಿ ಪುಷ್ಪಾರ್ಚನೆ ಮಾಡುತ್ತ ಸ್ವಾಮಿ ವಿವೇಕಾನಂದರೆಂದರು:

“ಏಕೆಂದರೆ, ಅವಳೂ ಸಹ ತಾಯಿಯೇ.” (\enginline{ND 4:276})

೧೦. ಲಂಡನ್ನಿನ ಗ್ರೇ ಕೋಟ್ ಉದ್ಯಾನವನಗಳಲ್ಲಿ ನಡೆದ ಸ್ವಾಮಿ ವಿವೇಕಾನಂದರ ಸಂಭಾಷಣೆಗಳಿಂದ ಉದ್ಧರಿಸುತ್ತ ಮಿ. ಜೆ. ಜೆ. ಗುಡ್ವಿನ್ ೧೮೯೬ರ ಅಕ್ಟೋಬರ್ ೨೩ ರಂದು ಮಿಸೆಸ್ ಓಲ್ ಬುಲ್ ಅವರಿಗೆ ಬರೆದ ಕಾಗದದಿಂದ:

“ಉನ್ನತ ಆದರ್ಶವನ್ನಿಟ್ಟುಕೊಳ್ಳುವುದೇನೋ ಒಳ್ಳೆಯದೇ, ಆದರೆ ಅದನ್ನು ತುಂಬ ಉನ್ನತವನ್ನಾಗಿ ಮಾಡಬೇಡಿ. ಉನ್ನತಾದರ್ಶವು ಮನುಷ್ಯರನ್ನು ಮೇಲಕ್ಕೆತ್ತುತ್ತದೆಯೇನೋ ಸರಿಯೇ, ಆದರೆ ಅಸಾಧ್ಯವಾದ ಆದರ್ಶವು ಅವರನ್ನು ಅದರ ಆ ಅಸಾಧ್ಯತೆಯಿಂದಾಗಿಯೇ ಕೆಳಗಿಳಿಸಿಬಿಡುತ್ತದೆ”. (\enginline{ND 4:385})

೧೧. ಇಂಗ್ಲಿಷ್ ಜನರನ್ನು ಕುರಿತಾದ ಸ್ವಾಮಿ ವಿವೇಕಾನಂದರ ಮಾತನ್ನು ಉದ್ಧರಿಸುತ್ತ ಸ್ವಾಮಿ ಅಭೇದಾನಂದರು ದಿನಚರಿಯ ೧೮೯೬ರ ನವೆಂಬರ್೨೦ ರಂದು ಬರೆದ ಬರಹ:

“ಇಲ್ಲಿಯ ಜನರ ರೂಢಿಗಳು, ನಡವಳಿಕೆ, ರಾಜಕೀಯಗಳನ್ನು ಅರಿತುಕೊಳ್ಳದೆ ನೀವು ಅವರೊಡನೆ ಸ್ನೇಹ ಬೆಳೆಸುವುದು ಸಾಧ್ಯವಿಲ್ಲ. ಶ‍್ರೀಮಂತರ, ಸುಸಂಸ್ಕೃತರ ಮತ್ತು ಬಡವರ ಬಾಹ್ಯ ಚರ್ಯೆಗಳನ್ನು ನೀವು ಅರಿತುಕೊಳ್ಳಬೇಕಾಗುತ್ತದೆ”. (\enginline{ND 4:478})

೧೨. ಸ್ವಾಮಿ ವಿವೇಕಾನಂದರ “ಅನುಷ್ಠಾನ ವೇದಾಂತ-೪” ಉಪನ್ಯಾಸದ ಕೊನೆಯ ಲ್ಲಿನ ಅಪ್ರಕಟಿತ ಮಾತೊಂದನ್ನು ಉದ್ಧರಿಸುತ್ತ ಮಿ. ಜೆ. ಜೆ. ಗುಡ್ವಿನ್ ೧೮೯೬ರ ನವೆಂಬರ್ ೧೧ರಂದು ಮಿಸೆಸ್ ಓಲ್ ಬುಲ್ ಅವರಿಗೆ ಬರೆದ ಕಾಗದದಲ್ಲಿ:

“ಮಾಯೆ ಸಂಪೂರ್ಣವಾಗಿ ಕಳೆದುಹೋಗುವವರೆಗೆ ಒಂದು ಜೀವವು ಬ್ರಹ್ಮವನ್ನು ನಿರಪೇಕ್ಷವಾಗಿ ಎಂದಿಗೂ ಪಡೆದುಕೊಳ್ಳಲಾರದು. ಮಾಯೆಯಲ್ಲೇ ಬಿಡಲ್ಪಟ್ಟ ಜೀವ ವೊಂದು ಇನ್ನೂ ಉಳಿದಿರುವುದಾದರೆ, ಯಾವುದೊಂದು ಆತ್ಮವೂ ನಿರಪೇಕ್ಷವಾಗಿ ಮುಕ್ತವಾಗಲಾರದು... ವೇದಾಂತಿಗಳಲ್ಲಿ ಈ ವಿಚಾರವಾಗಿ ಅಭಿಪ್ರಾಯಭೇದವುಂಟು.” (\enginline{ND 4: 481})

೧೩. ಸ್ವಾಮಿ ವಿವೇಕಾನಂದರ ಕೊನೆಯ ದಿನಗಳನ್ನು ಕುರಿತು ಗುರುಭಾಯಿ ಯೊಬ್ಬರಿಗೆ ಸ್ವಾಮಿ ಶಾರದಾನಂದರು ಬರೆದ ಒಂದು ಕಾಗದದಿಂದ:

ಕೆಲವೊಮ್ಮೆ ಅವರೆನ್ನುತ್ತಾರೆ, “ಸಾವು ನನ್ನ ಹಾಸಿಗೆಯ ಬಳಿಗೇ ಬಂದಿದೆ; ನಾನೂ ಸಾಕಷ್ಟು ಕೆಲಸಮಾಡಿದ್ದೇನೆ, ಆಟವಾಡಿದ್ದೇನೆ; ನನ್ನ ಕೊಡುಗೆ ಏನೆಂಬುದನ್ನು ಲೋಕ ಅರಿತುಕೊಳ್ಳಲಿ; ಅದನ್ನು ಅರ್ಥಮಾಡಿಕೊಳ್ಳಲು ಸಾಕಷ್ಟು ದೀರ್ಘಕಾಲವೇ ಬೇಕಾಗುತ್ತದೆ”. (\enginline{ND 4:521})

೧೪. ಸೋದರಿ ನಿವೇದಿತಾ ೧೮೯೮ರ ಅಕ್ಟೋಬರ್ ೧೩ರಂದು ಮಿಸೆಸ್ ಆ್ಯಶ್ಟನ್ ಜಾನ್ಸರ್ ಅವರಿಗೆ ಕಾಶ್ಮೀರದಿಂದ ಬರೆದ ಕಾಗದವೊಂದರಲ್ಲಿ ಸ್ವಾಮಿ ವಿವೇಕಾನಂದರ ಆಧ್ಯಾತ್ಮಿಕ ಭಾವವನ್ನು ವಿವರಿಸಿದ್ದಾಳೆ:

ಈ ಕ್ಷಣದಲ್ಲಿ ಅವರಿಗೆ “ಒಳ್ಳೆಯದನ್ನು ಮಾಡುವುದು” ಎಂದರೇ ಭಯಾನಕವೆನ್ನಿಸುತ್ತದೆ. ಎಲ್ಲವನ್ನೂ ಮಾಡುವುದು ತಾಯಿಯೊಬ್ಬಳೇ. ದೇಶಭಕ್ತಿ ಎನ್ನುವುದೊಂದು ತಪ್ಪು ಕಲ್ಪನೆ. ಎಲ್ಲವೂ ತಪ್ಪುಕಲ್ಪನೆಗಳೇ. ತಾಯಿಯೊಬ್ಬಳೇ ಎಲ್ಲವೂ... ಎಲ್ಲ ಮಾನವರೂ ಒಳ್ಳೆಯವರೇ, ನಾವು ಮಾತ್ರ ಎಲ್ಲರನ್ನೂ ತಲುಪುವುದಕ್ಕಾಗುತ್ತಿಲ್ಲ... ನಾನಿನ್ನು ಯಾರಿಗೂ ಏನನ್ನೂ ಬೋಧಿಸುವುದಿಲ್ಲ. ಬೋಧಿಸುವುದಕ್ಕೆ ನಾನು ಯಾರು?... ಸ್ವಾಮೀಜಿ ಸತ್ತುಹೋಗಿದ್ದಾರೆ”.(\enginline{ND 5:3-4})

೧೫. ೧೮೯೯ ಜೂನ್ ೧೯ರಂದು ಬೇಲೂರು ಮಠದಲ್ಲಿ ನೆರೆದಿದ್ದ ಸಂನ್ಯಾಸಿಗಳನ್ನೂ ಬ್ರಹ್ಮಚಾರಿಗಳನ್ನೂ ಉದ್ದೇಶಿಸಿ ಮಾಡಿದ ಭಾಷಣದ ಉಪಸಂಹಾರದ ಮಾತುಗಳನ್ನು ಜ್ಞಾಪಿಸಿಕೊಂಡು ಮಿ. ಶಚೀಂದ್ರನಾಥ ಬಸು ಬರೆದ ಕಾಗದವೊಂದರಿಂದ:

“ಮಕ್ಕಳೇ, ನೀವೆಲ್ಲರೂ ಮನುಷ್ಯರಾಗಬೇಕು. ಇದನ್ನೇ ನಾನು ಅಪೇಕ್ಷಿಸುವುದು! ನೀವು ಸ್ವಲ್ಪ ಯಶಸ್ಸನ್ನುಗಳಿಸಿದರೂ, ನನಗೆ ನಾನು ಬದುಕಿದ್ದು ಸಾರ್ಥಕವೆನ್ನಿಸುತ್ತದೆ.” (\enginline{ND 5:17})

೧೬. ೧೮೯೯ರ ಬೇಸಿಗೆಯಲ್ಲಿ ಒಂದು ದಿನ ಸಂಜೆ ಸ್ವಾಮಿ ಶಾರದಾನಂದರ ಜೊತೆಗೆ ಮಾತನಾಡುತ್ತಿರುವಾಗ:

“ಜನರಿಗೆ ವಾಸ್ತವಿಕವಾಗಿರುವುದನ್ನು, ದೇಹಬಲ ಹೊಂದಿರುವುದನ್ನು ಕಲಿಸಬೇಕು. ಅಂತಹ ಒಂದು ಡಜನ್ ಪುರುಷಸಿಂಹರು ಲೋಕವನ್ನು ಜಯಿಸಬಲ್ಲರು, ಮಿಲಿಯ ಗಟ್ಟಲೆ ಕುರಿಗಳಲ್ಲ. ಎಷ್ಟೇ ಉನ್ನತವಾದುದಾಗಿರಲಿ, ವೈಯಕ್ತಿಕ ಆದರ್ಶದ ಅನುಕರಣೆಯನ್ನು ಜನರಿಗೆ ಕಲಿಸಬಾರದು”. (\enginline{ND 5:17})

೧೭. ಸ್ವಾಮಿ ವಿವೇಕಾನಂದ ಮತ್ತು ಸ್ವಾಮಿ ತುರೀಯಾನಂದರ ಜೊತೆಗೆ ೧೮೯೯ರಲ್ಲಿ ಅಮೆರಿಕಾಕ್ಕೆ ಮಾಡಿದ ಪ್ರವಾಸದ ಬಗ್ಗೆ ಮಿಸೆಸ್ ಮೇರಿ ಸಿ. ಫಂಕ್ ಅವರ ನೆನಹುಗಳಿಂದ:

“ಈ ಮಾಯೆಯೆಲ್ಲವೂ ಇಷ್ಟು ಸುಂದರವಾಗಿರಬೇಕಾದರೆ, ಅದರ ಹಿಂದಿರುವ ಸತ್ಯವು ಇನ್ನೆಷ್ಟು ಸುಂದರವಾಗಿರಬಹುದೆಂದು ಊಹಿಸಿಕೋ!” (\enginline{ND 5:76})

“(ಸಮುದ್ರವನ್ನೂ ಆಕಾಶವನ್ನೂ ತೋರಿಸುತ್ತ)ಸಮಸ್ತ ಕಾವ್ಯದ ಸಾರವೇ ಅಲ್ಲಿರುವಾಗ, ಕವನವನ್ನೇನು ಓದುತ್ತಿಯೆ?” (ಅದೇ)

೧೮. ೧೮೯೯ರ ಸೆಪ್ಟೆಂಬರ್ ೩ರಂದು ಮಿಸ್ ಜೋಸೆಫಿನ್ ಮ್ಯಾಕ್ಲಿಯಾಡ್ ಮಿಸೆಸ್ ಓಲ್ ಬುಲ್ ಅವರಿಗೆ ಬರೆದ ಕಾಗದದಲ್ಲಿ:

“ಅತ್ಯಂತ ಆವಶ್ಯಕತೆಯ ಕ್ಷಣದಲ್ಲಿ ಮನುಷ್ಯನು ನಿಲ್ಲುವುದು ಒಬ್ಬಂಟಿಯಾಗಿಯೇ”. (\enginline{ND 5:122})

೧೯. ರಿಡ್ಜ್ ಲಿ ಮೇನರ್ನಲ್ಲಿದ್ದಾಗ ಸೋದರಿ ನಿವೇದಿತಾ ೧೮೯೯ರ ಅಕ್ಟೋಬರ್ ೨೭ರಂದು ಬರೆದ ದಿನಚರಿಯಿಂದ - ಇದರಲ್ಲಿ ಸ್ವಾಮಿ ವಿವೇಕಾನಂದರು ಓಲಿಯಾ ಬುಲ್ ವಾಘನ್ ಅವರ ಬಗ್ಗೆ ತಮ್ಮ ಕಾಳಜಿಯನ್ನು ವ್ಯಕ್ತಪಡಿಸಿದ್ದಾರೆ:

“ದುಃಸ್ವಪ್ನಗಳು ಆರಂಭವಾಗುವುದು ಯಾವಾಗಲೂ ಸುಖವಾಗಿಯೇ - ಅತ್ಯಂತ ಕೆಟ್ಟಗಳಿಗೆಯಲ್ಲಿ ಮಾತ್ರ ಕನಸು ಒಡೆದುಹೋಗುತ್ತದೆ - ಸಾವು ಬದುಕಿನ ಕನಸನ್ನು ಒಡೆಯುವುದು ಹಾಗೆ. ಸಾವನ್ನೇ ಪ್ರೀತಿಸು”. (\enginline{ND 5:138})

೨೦. ಮಿಸೆಸ್ ಜೋಸೆಫಿನ್ ಮ್ಯಾಕ್ಲಿಯಾಡ್ ೧೮೯೯ರ ಡಿಸೆಂಬರ್ನಲ್ಲಿ ಸೋದರಿ ನಿವೇದಿತಾಳಿಗೆ ಬರೆದ ಕಾಗದದಿಂದ:

“ಕ್ಯಾಲಿಫೋರ್ನಿಯಾದವರು ನನ್ನ ಬಗ್ಗೆ ಇಟ್ಟುಕೊಂಡಿರುವ ಕಲ್ಪನೆಗಳೆಲ್ಲ ಬಂದಿರುವುದು ಚಿಕಾಗೋದಿಂದಲೇ”. (\enginline{ND 5:179})

೨೧. ಸ್ವಾಮಿ ವಿವೇಕಾನಂದರು ಮಿ. ಬಾಂಗರ್ಟ್ ಅವರಿಗೆ ಹೇಳಿದ್ದೆಂದು ಮಿಸೆಸ್ ಆ್ಯಲಿಸ್ ಹ್ಯಾನ್ಸ್ಬ್ರೋ ಅವರು ತಮ್ಮ ನೆನಪುಗಳಲ್ಲಿ ಉದ್ಧರಿಸಿರುವುದು:

“ನಾನು ಅದೇ ವಿಷಯದ ಮೇಲೆ ಮಾತನಾಡಬಲ್ಲೆ, ಆದರೆ ಅದು ಅದೇ ಉಪನ್ಯಾಸವಾಗುವುದಿಲ್ಲ”. (\enginline{ND 5:230})

೨೨. ಪ್ರೇಕ್ಷಣೀಯ ಸ್ಥಳಗಳಿಗೆ ಕರೆದೊಯ್ಯುವ ತಮ್ಮ ಪ್ರಯತ್ನಕ್ಕೆ ಸ್ವಾಮಿ ವಿವೇಕಾ ನಂದರು ತೋರಿದ ಪ್ರತಿಕ್ರಿಯೆಯನ್ನು ಕುರಿತು ಮಿಸೆಸ್ ಆ್ಯಲಿಸ್ ಹ್ಯಾನ್ಸ್ಬ್ರೋ ಅವರು ತಮ್ಮ ನೆನಪುಗಳಲ್ಲಿ ಉದ್ಧರಿಸಿರುವುದು:

“ದೃಶ್ಯಾವಳಿಗಳನ್ನು ನನಗೆ ತೋರಿಸಬೇಡಿ. ನಾನು ಹಿಮಾಲಯವನ್ನು ನೋಡಿರುವೆ! ದೃಶ್ಯಗಳನ್ನು ನೋಡಲು ಹತ್ತು ಹೆಜ್ಜೆ ಇಡಲಾರೆ; ಆದರೆ (ಮಹಾತ್ಮನಾದ) ಮನುಷ್ಯನೊಬ್ಬನನ್ನು ನೋಡಲು ನಾನು ಸಾವಿರ ಮೈಲಿಗಳನ್ನು ಕ್ರಮಿಸಲು ಸಿದ್ಧನಿರುವೆ!” (\enginline{ND 5:244})

೨೩. ಶಿಶುಶಿಕ್ಷಣದ ಸಮಸ್ಯೆಯ ಬಗ್ಗೆ ಸ್ವಾಮಿ ವಿವೇಕಾನಂದರ ಆಸಕ್ತಿಯನ್ನು ಕುರಿತು ಮಿಸೆಸ್ ಆ್ಯಲಿಸ್ ಹ್ಯಾನ್ಸ್ಬ್ರೋ ಅವರು ತಮ್ಮ ನೆನಪುಗಳಲ್ಲಿ ಉದ್ಧರಿಸಿರುವುದು:

ಶಿಕ್ಷೆ ಕೊಡುವುದರಲ್ಲಿ ಅವರಿಗೆ ನಂಬಿಕೆಯಿಲ್ಲ. ಅದರಿಂದ ತಾವು ಪ್ರಯೋಜನವನ್ನೇನೂ ಪಡೆದಿಲ್ಲ ಎಂದರು; ಅಲ್ಲದೆ “ಮಗುವೊಂದನ್ನು ಭಯಪಡಿಸುವಂಥ ಏನನ್ನೂ ನಾನು ಮಾಡಲಾರೆ” ಎಂದೂ ಸೇರಿಸಿದರು. (\enginline{ND 5:253})

೨೪. ಹದಿನೇಳು ವರ್ಷ ವಯಸ್ಸಿನ ರಾಲ್ಙ ವೈಕಾಫ್ನಿಗೆ ದೇವರ ಬಗ್ಗೆ ವಿವರಿಸುತ್ತ ಸ್ವಾಮಿ ವಿವೇಕಾನಂದರು ಹೇಳಿದ್ದೆಂದು ಮಿಸೆಸ್ ಆ್ಯಲಿಸ್ ಹ್ಯಾನ್ಸ್ಬ್ರೋ ಬರೆದಿಟ್ಟಿದ್ದಾರೆ:

“ನಿನ್ನ ಕಣ್ಣುಗಳನ್ನು ನೀನೇ ನೋಡಿಕೊಳ್ಳಬಲ್ಲೆಯಾ? ದೇವರೂ ಹಾಗೆಯೇ. ಅವನು ನಿನಗೆ ನಿನ್ನ ಕಣ್ಣುಗಳಷ್ಟೇ ಹತ್ತಿರ. ನೀನವನನ್ನು ನೋಡಲಾಗದೆ ಇದ್ದರೂ ಅವನು ನಿನ್ನವನೇ”. (\enginline{ND 5:254})

೨೫. ಭಾರತವನ್ನು ಆಕ್ರಮಿಸಿಕೊಂಡಿರುವ ಕೀಳುಜಾತಿಯ ಇಂಗ್ಲಿಷ್ ಸೈನಿಕರ ಬಗ್ಗೆ ಸ್ವಾಮಿ ವಿವೇಕಾನಂದರ ಅಭಿಪ್ರಾಯವೆಂದು ಮಿಸೆಸ್ ಆ್ಯಲಿಸ್ ಹ್ಯಾನ್ಸ್ಬ್ರೋ ತಮ್ಮ ನೆನಪುಗಳಲ್ಲಿ ಬರೆದಿರುವುದು:

“ಇಂಗ್ಲಿಷ್ ಮನುಷ್ಯನ ಮನೆಯನ್ನು ಯಾರಾದರೂ ನಾಶಪಡಿಸಿದರೆ, ಇಂಗ್ಲಿಷ್ ಮನುಷ್ಯ ಅವನನ್ನು ಕೊಲ್ಲುವನು, ಅದು ಸರಿಯಾದದ್ದೇ. ಆದರೆ ಹಿಂದೂ ಸುಮ್ಮನೆ ಕುಳಿತು ಗಹಗಹಿಸಿ ನಗುವನು!

“ನಾವು ಹೋರಾಟದ ಸ್ವಭಾವದವರಾಗಿದ್ದಿದ್ದರೆ ಕೈಬೆರಳೆಣಿಕೆಯಷ್ಟು ಇಂಗ್ಲಿಷ್ ಜನರು ಭಾರತವನ್ನಾಳಲು ಸಾಧ್ಯವಾಗುತ್ತಿತ್ತೆಂದು ಭಾವಿಸುವೆಯಾ? ಸಮರಾಸಕ್ತಿಯನ್ನು ತಂದು ಕೊಡುವ ಆಸೆಯಿಂದ ನಾನು ಭಾರತದ ಉದ್ದಗಲಕ್ಕೂ ಮಾಂಸಭಕ್ಷಣೆಯನ್ನು ಬೋಧಿಸುತ್ತಿರುವೆ!”. (\enginline{ND 5:256})

೨೬. ಮಿಸೆಸ್ ಆ್ಯಲಿಸ್ ಹ್ಯಾನ್ಸ್ಬ್ರೋ ಸ್ಮರಿಸಿಕೊಂಡಿರುವ ಕ್ಯಾಲಿಫೋರ್ನಿಯಾದ ಪಸ ಡೇನದ ಒಂದಾನೊಂದು ಪ್ರವಾಸ ಸಮಯದಲ್ಲಿ ಕ್ರಿಶ್ಚಿಯನ್ ಸೈನ್ಸ್ ಸ್ತ್ರೀಯೊಬ್ಬಳು ಜನರಿಗೆ ಒಳ್ಳೆಯವರಾಗಿರುವಂತೆ ಬೋಧಿಸಬೇಕೆಂದು ಸ್ವಾಮಿ ವಿವೇಕಾನಂದರಿಗೆ ಸಲಹೆಯಿತ್ತಾಗ:

“ಒಳ್ಳೆಯವನಾಗಿರುವುದಕ್ಕೆ ನಾನೇಕೆ ಹಂಬಲಿಸಬೇಕು? (ಕೈಯಿಂದ ಅಲ್ಲಿದ್ದ ಮರಗಳನ್ನು ಹಾಗೂ ಇತರ ದೃಶ್ಯಗಳನ್ನು ತೋರಿಸುತ್ತ) ಇವೆಲ್ಲವೂ ಆತನ ಕೆಲಸ. ಆತನ ಕಾರ್ಯಗಳಿಗೆ ನಾನು ವಿಷಾದಿಸುವುದೆ? ನೀವು ಜಾನ್ ಡೋನನ್ನು ಸುಧಾರಿಸಬೇಕೆಂದಿದ್ದರೆ, ಹೋಗಿ ಅವನೊಂದಿಗೆ ವಾಸಿಸಿ; ಅವನನ್ನು ಸುಧಾರಿಸಲೆತ್ನಿಸಬೇಡಿ. ನಿಮ್ಮಲ್ಲಿ ಏನಾದರೂ ದಿವ್ಯತೆಯ ಜ್ಯೋತಿಯಿದ್ದರೆ, ಅವನು ಅದನ್ನು ಪಡೆದುಕೊಳ್ಳುತ್ತಾನೆ”.(\enginline{ND 5:257})

೨೭. ಮಿಸೆಸ್ ಆ್ಯಲಿಸ್ ಹ್ಯಾನ್ಸ್ಬ್ರೋ ಅವರ ನೆನಪುಗಳಿಂದ:

“ಒಂದು ಸಲ ನೀವು ಮಾಡಬೇಕಾದ ಕಾರ್ಯವನ್ನು ಪರಿಗಣಿಸಿದ ಮೇಲೆ, ಬೇರೆ ಯಾವುದೂ ಅದರಿಂದ ನಿಮ್ಮನ್ನು ದೂರ ಮಾಡದಂತೆ ನೋಡಿಕೊಳ್ಳಿ. ನಿಮ್ಮ ಹೃದಯವನ್ನು ಕೇಳಿಕೊಳ್ಳಿ - ಇನ್ನೊಬ್ಬರನ್ನಲ್ಲ - ಹಾಗೂ ಅದರ ನಿರ್ದೇಶನದಂತೆ ಮಾಡಿ”. (\enginline{ND 5:311})

೨೮. ಕ್ಯಾಲಿಫೋರ್ನಿಯಾದ ಓಕ್ಲ್ಯಾಂಡ್ನಲ್ಲಿ ೧೯೦೦ರ ಮಾರ್ಚ್ನಲ್ಲಿ ಉಪ ನ್ಯಾಸವೊಂದರಲ್ಲಿ ಮಿ. ಫ್ರಾಂಕ್ ರ್ಹೊಡ್ಹ್ಯಾಮಲ್ ಬರೆದುಕೊಂಡ ಟಿಪ್ಪಣಿಯಿಂದ:

“ಗಂಡ ಹೆಂಡತಿಯನ್ನು ಹೆಂಡತಿಗಾಗಿಯಾಗಲಿ, ಹೆಂಡತಿ ಗಂಡನನ್ನು ಗಂಡನಿಗಾಗಿಯಾಗಲಿ ಎಂದೂ ಪ್ರೀತಿಸಿಲ್ಲ. ಗಂಡ ಹೆಂಡತಿಯನ್ನು ಪ್ರೀತಿಸುವುದು ಅವಳಲ್ಲಿರುವ ದೇವರಿಗಾಗಿ; ಹೆಂಡತಿ ಗಂಡನನ್ನು ಪ್ರೀತಿಸುವುದು ಅವನಲ್ಲಿರುವ ದೇವರಿಗಾಗಿ\footnote{1. ನೋಡಿ, ಬೃಹದಾರಣ್ಯಕ ಉಪನಿಷತ್, \enginline{II.4.5}}. ಯಾರ ಬಳಿಗೆ ನಾವು ಪ್ರೀತಿಯಿಂದ ಸೆಳೆಯಲ್ಪಟ್ಟರೂ ಅದು ಅವರಲ್ಲಿರುವ ದೇವರಿಂದಾಗಿ. ಎಲ್ಲದರಲ್ಲಿರುವ, ಎಲ್ಲರಲ್ಲಿರುವ ದೇವರೇ ನಮ್ಮನ್ನು ಪ್ರೀತಿಸುವಂತೆ ಮಾಡುವುದು. ಪ್ರೇಮವೆಂದರೆ ದೇವರೊಬ್ಬನೇ... ಪ್ರತಿಯೊಬ್ಬರಲ್ಲಿಯೂ ಆತ್ಮರೂಪದಲ್ಲಿ ದೇವನಿರುವನು; ಉಳಿದೆಲ್ಲವೂ ಕೇವಲ ಸ್ವಪ್ನ, ಒಂದು ಭ್ರಮೆ”. (\enginline{ND 5:362})

೨೯. ಕ್ಯಾಲಿಫೋರ್ನಿಯಾದ ಓಕ್ಲ್ಯಾಂಡ್ನಲ್ಲಿ ೧೯೦೦ರ ಮಾರ್ಚ್ನಲ್ಲಿ ನಡೆದ ಉಪನ್ಯಾಸದ ವೇಳೆಗೆ ಮಿ. ಫ್ರಾಂಕ್ ರ್ಹೊಡ್ಹ್ಯಾಮಲ್ ಬರೆದುಕೊಂಡ ಟಿಪ್ಪಣಿಯಿಂದ:

“ಆಹಾ, ನಿಮ್ಮನ್ನು ನೀವೇನಾದರೂ ಅರಿತುಕೊಂಡಿದ್ದಿದ್ದರೆ! ನೀವೇ ಆತ್ಮಗಳು, ನೀವೇ ದೇವರುಗಳು. ಪಾಷಂಡಿತನವೆಂದು ನನಗೆ ಯಾವಾಗಲಾದರೂ ಅನ್ನಿಸಿದ್ದರೆ, ಅದು ನಾನು ನಿಮ್ಮನ್ನು ಮನುಷ್ಯರೆಂದು ಕರೆಯುವಾಗ”. (\enginline{ND 5:362})

೩೦. ಸ್ವಾಮಿ ವಿವೇಕಾನಂದರು ೧೯೦೦ರ ಮಾರ್ಚ್ನಲ್ಲಿ ಕೊಟ್ಟ ಭಾರತದ ಮೇಲಣ ಸ್ಯಾನ್ ಫ್ರಾನ್ಸಿಸ್ಕೋ ಉಪನ್ಯಾಸಮಾಲಿಕೆಯ ಬಗ್ಗೆ ಮಿ. ಥಾಮಸ್ ಜೆ. ಅಲನ್ ಅವರ ನೆನಪುಗಳಿಂದ:

“ಕೈಗಳನ್ನುಪಯೋಗಿಸುವುದು ಹೇಗೆಂದು ನಮಗೆ ಕಲಿಸುವುದಕ್ಕೆ ನೀವು ಯಂತ್ರಕುಶಲಿಗಳನ್ನು ಕಳುಹಿಸಿ, ನಾವು ನಿಮಗೆ ಆಧ್ಯಾತ್ಮಿಕತೆಯೆಂದರೇನೆಂಬುದನ್ನು ತಿಳಿಸುವುದಕ್ಕಾಗಿ ಪ್ರಚಾರಕರನ್ನು ಕಳಿಸುತ್ತೇವೆ”. (\enginline{ND 5:365})

೩೧. ಟರ್ಕ್ ಸ್ಟ್ರೀಟ್ನಲ್ಲಿನ ಫ್ಲಾಟ್ ಒಂದರಲ್ಲಿ ಅಡುಗೆಮಾಡುತ್ತಿದ್ದಾಗ ಸ್ವಾಮಿ ವಿವೇಕಾನಂದರು ಹೇಳಿದ ತಾತ್ತ್ವಿಕ ಅಂಶಗಳ ಬಗ್ಗೆ ಮಿಸೆಸ್ ಎಡಿತ್ ಅಲನ್ ಅವರ ನೆನಪುಗಳಿಂದ:

“ ‘ಎಲೈ ಅರ್ಜುನ, ಪ್ರಾಣಿಗಳೆಲ್ಲವನ್ನೂ ಕುಂಬಾರನ ಚಕ್ರವೊಂದರ ಮೇಲೆ ಕೂರಿಸಿರುವ ಹಾಗೆ ಮಾಯೆಯಿಂದ ತಿರುಗಿಸುತ್ತಾ ಭಗವಂತನು ಎಲ್ಲಾ ಜೀವರುಗಳ ಹೃದಯದಲ್ಲಿ ನೆಲೆಸಿರುತ್ತಾನೆ’ (ಭಗವದ್ಗೀತಾ, ೧೮.೬೧). ದಾಳವನ್ನೆಸೆದಿರುವ ಹಾಗೆ, ಇದೆಲ್ಲವೂ ಮೊದಲೇ ತೀರ್ಮಾನವಾಗಿರುತ್ತದೆ; ಜೀವನದಲ್ಲಿಯೂ ಹಾಗೆಯೇ. ಚಕ್ರವು ತಿರುಗುತ್ತಲೇ ಇರುತ್ತದೆ, ಮತ್ತೊಮ್ಮೆ ಅದೇ ಸಂಯೋಜನೆ ಮೇಲೆದ್ದು ಬರುತ್ತದೆ; ಹೂಜಿ ಲೋಟಗಳು ಮೊದಲಿನಿಂದಲೂ ಇದ್ದ ಹಾಗೆಯೇ ಆ ಈರುಳ್ಳಿ ಆಲೂಗಡ್ಡೆಗಳೂ ಸಹ. ನಾವೇನು ತಾನೆ ಮಾಡಲು ಸಾಧ್ಯ, ಅಮ್ಮ ಅವನೇ ನಮ್ಮನ್ನು ಸಂಸಾರಚಕ್ರದ ಮೇಲೆ ಕೂರಿಸಿರುವುದು” (\enginline{ND 6:17})

೩೨. ಭೋಜನಾನಂತರದ ಸಂಭಾಷಣೆಯೊಂದರ ಬಗ್ಗೆ ಮಿಸೆಸ್ ಎಡಿತ್ ಅಲನ್ ಅವರ ನೆನಪುಗಳಿಂದ:

“ಸುಮಾರು ಇನ್ನೂರು ವರ್ಷಗಳೊಳಗೇ ನಾನು ಪುನಃ ಬರಲಿರುವೆ ಎಂದು ಗುರುಗಳು ಹೇಳುತ್ತಿದ್ದರು - ನಾನೂ ಅವರ ಜೊತೆಗೆ ಬರುವೆ. ಗುರು ಬರುವಾಗ ತನ್ನವರನ್ನೂ ಕರೆತರುವನು”. (\enginline{ND 6:17})

೩೩. ಸ್ವಾಮಿ ವಿವೇಕಾನಂದರು ೧೯೦೦ರಲ್ಲಿ ಕ್ಯಾಲಿಫೋರ್ನಿಯಾದ ಸ್ಯಾನ್ ಫ್ರಾನ್ಸಿಸ್ಕೋದಲ್ಲಿದ್ದಾಗ ಅವರ “ಅಡುಗೆಮನೆ” ಸಲಹೆಗಳ ಬಗ್ಗೆ ಮಿಸೆಸ್ ಎಡಿತ್ ಅಲನ್ ಅವರ ನೆನಪುಗಳಿಂದ:

“ನೆಲದ ಮೇಲೆ ತೆವಳುತ್ತಿರುವ ಆ ಇರುವೆಗಿಂತ ನನ್ನನ್ನು ನಾನು ದೊಡ್ಡವನೆಂದುಕೊಂಡರೆ ನಾನು ಅಜ್ಞಾನಿಯೇ ಸರಿ”. (\enginline{ND 6:19})

“ತಾಯಿ, ಉದಾರ ಮನಸ್ಸಿನವರಾಗಿರಿ; ಯಾವಾಗಲೂ ಎರಡು ಮಾರ್ಗಗಳನ್ನೂ ಪರಿಗಣಿಸಿರಿ. ನಾನು ಔನ್ನತ್ಯದಲ್ಲಿರುವಾಗ ‘ಶಿವೋಣಿಹಮ್​, ಶಿವೋಣಿಹಮ್​: ಅಹಂ ಬ್ರಹ್ಮಾಸ್ಮಿ, ಅಹಂ ಬ್ರಹ್ಮಾಸ್ಮಿ!’ ಎನ್ನುತ್ತೇನೆ; ಹೊಟ್ಟೆನೋವು ಬಂದು ನರಳುತ್ತಿರುವಾಗ, ‘ತಾಯಿ, ನನ್ನ ಮೇಲೆ ಕೃಪೆ ಮಾಡು! ಎನ್ನುತ್ತೇನೆ.” (ಅದೇ.)

“ಸಾಕ್ಷಿಯಾಗಿರುವುದನ್ನು ಕಲಿಯಿರಿ. ಬೀದಿಯಲ್ಲಿ ಎರಡು ನಾಯಿಗಳು ಜಗಳವಾಡುತ್ತಿರುವಾಗ ನಾನು ಅಲ್ಲಿಗೆ ಹೋದರೆ ಜಗಳದಲ್ಲಿ ಸಿಕ್ಕಿಹಾಕಿಕೊಳ್ಳುತ್ತೇನೆ; ಆದರೆ ಸುಮ್ಮನೆ ಕೊಠಡಿಯಲ್ಲಿ ಕುಳಿತು ಕಿಟಕಿಯಿಂದ ಜಗಳವನ್ನು ನೋಡುತ್ತಿದ್ದರೆ ಕೇವಲ ಸಾಕ್ಷಿಯಾಗಿರುತ್ತೇನೆ. ಹಾಗೆ, ಸಾಕ್ಷಿ ಆಗಿರುವುದನ್ನು ಕಲಿಯಿರಿ.” (ಅದೇ.)

೩೪. ಸ್ವಾಮಿ ವಿವೇಕಾನಂದರು ೧೯೦೦ರಲ್ಲಿ ಕ್ಯಾಲಿಫೋರ್ನಿಯಾದ ಸ್ಯಾನ್ ಫ್ರಾನ್ಸಿ ಸ್ಕೋದಲ್ಲಿದ್ದಾಗಿನ ವೈಯಕ್ತಿಕ ಸಂಭಾಷಣೆಯೊಂದರ ಬಗ್ಗೆ ಮಿ. ಥಾಮಸ್ ಜೆ. ಅಲನ್ ಅವರ ನೆನಪುಗಳಿಂದ:

“ನಾವು ಮುಂದುವರೆಯುವುದು ತಪ್ಪಿನಿಂದ ಸತ್ಯದ ಕಡೆಗಲ್ಲ, ಆದರೆ ಒಂದು ಸತ್ಯದಿಂದ ಇನ್ನೊಂದು ಸತ್ಯದೆಡೆಗೆ. ಆದ್ದರಿಂದ, ಯಾರೊಬ್ಬರನ್ನೂ ಅವರೀಗ ಮಾಡುತ್ತಿರುವುದಕ್ಕಾಗಿನಿಂದಿಸ ಕೂಡದು. ಏಕೆಂದರೆ, ಅವರು ಈ ಕ್ಷಣದಲ್ಲಿ ಅವರಿಗೆ ಸಾಧ್ಯವಿರುವ ಒಳ್ಳೆಯದನ್ನೇ ಮಾಡುತ್ತಿದ್ದಾರೆ. ಮಗು ಚೂಪಾದ ಚಾಕುವನ್ನು ಕೈಗೆ ತೆಗೆದುಕೊಂಡಿದ್ದರೆ, ಅದರ ಕೈಯಿಂದ ಚಾಕುವನ್ನು ಕಿತ್ತುಕೊಳ್ಳಲು ಹೋಗಬೇಡಿ. ಬದಲಿಗೆ ಅದಕ್ಕೊಂದು ಸೇಬಿನ ಹಣ್ಣನ್ನು ಕೊಡಿ, ಅಥವಾ ಒಂದು ಆಕರ್ಷಕ ಬಣ್ಣದ ಬೊಂಬೆಯನ್ನು ಕೊಡಿ; ತಂತಾನೆ ಅದು ಚಾಕುವನ್ನು ಕೆಳಕ್ಕೆ ಹಾಕುತ್ತದೆ. ಆದರೆ ಯಾರು ಬೆಂಕಿಯಲ್ಲಿ ಕೈಯಿಡುವರೋ ಅವರು ಸುಟ್ಟುಕೊಳ್ಳುತ್ತಾರೆ; ನಾವು ಕಲಿಯುವುದು ಅನುಭವದಿಂದ ಮಾತ್ರವೇ”.(\enginline{ND 6:42})

೩೫. ಸ್ವಾಮಿ ವಿವೇಕಾನಂದರು ೧೯೦೦ರಲ್ಲಿ ಸ್ಯಾನ್ ಫ್ರಾನ್ಸಿಸ್ಕೋದ ಉಪನ್ಯಾಸವೊಂದರ ನಂತರ ಮನೆಗೆ ನಡೆದುಹೋಗುವ ಸಂದರ್ಭದ ಬಗೆಗೆ ಮಿಸೆಸ್ ಆ್ಯಲಿಸ್ ಹ್ಯಾನ್ಸ್ಬ್ರೋ ಅವರ ನೆನಪುಗಳಿಂದ:

“ನೀವು ‘ನನ್ನ ಮಾತುಗಳೆಂದರೆ ಚೈತನ್ಯ ಹಾಗೂ ಜೀವನ’ ಎಂದು ಕ್ರಿಸ್ತನು ಹೇಳಿದ್ದನ್ನು ಕೇಳಿರುವಿರಿ. ಹಾಗೆಯೇ ನನ್ನ ಮಾತುಗಳೂ ಸಹ ಚೈತನ್ಯ ಹಾಗೂ ಜೀವನ; ಅವು ಸುಡುಸುಡುತ್ತ ನಿಮ್ಮ ಮೆದುಳಿನೊಳಕ್ಕೆ ಇಳಿಯುತ್ತವೆ; ನೀವು ಅವುಗಳಿಂದ ಎಂದಿಗೂ ದೂರವಾಗಲಾರಿರಿ!.(\enginline{ND 6:57-58})

೩೬. ಸ್ವಾಮಿ ವಿವೇಕಾನಂದರ ಉದಾತ್ತ ಹೃದಯವನ್ನು ಕುರಿತು ಮಿಸೆಸ್ ಆ್ಯಲಿನ್ ಹ್ಯಾನ್ಸ್ಬ್ರೋ ಅವರ ೧೯೦೦ರ ಸ್ಯಾನ್ ಫ್ರಾನ್ಸಿಸ್ಕೋದಲ್ಲಿನ ನೆನಪುಗಳಿಂದ:

“ಮಾನವರನ್ನು ಪ್ರೀತಿಸುವ ನಾನು ಮತ್ತೆ ಹುಟ್ಟಿಬರಬೇಕಾಗಬಹುದು”. (\enginline{ND 6:79})

೩೭. ಸ್ವಾಮಿ ವಿವೇಕಾನಂದರು ೧೯೦೦ರ ಮೇ ತಿಂಗಳಿನಲ್ಲಿ ಕ್ಯಾಲಿಫೋರ್ನಿಯಾದ ಟೇಲರ್ ಕ್ಯಾಂಪ್ನಲ್ಲಿದ್ದಾಗ ಮಿಸೆಸ್ ಜಾರ್ಜ್ ರೂರ್ಬಾಕ್ ಅವರ ನೆನಪುಗಳಿಂದ:

“ಈ ದೇಶದ ಚಿಕಾಗೋದಲ್ಲಿ, ನಾನು ಮಾಡಿದ ಮೊಟ್ಟಮೊದಲನೆಯ ಭಾಷಣದಲ್ಲಿ, ಸಭಿಕರನ್ನು ಕುರಿತು ನಾನು ‘ಅಮೆರಿಕಾದ ಸೋದರ ಸೋದರಿಯರೆ’ ಎಂದು ಸಂಬೋಧಿಸಿದೆ; ಅವರೆಲ್ಲರೂ ಮೇಲೆದ್ದು ನಿಂತದ್ದು ನಿಮಗೆ ಗೊತ್ತು. ಅವರು ಹಾಗೆ ಏಕೆ ಮಾಡಿದರು, ನನ್ನಲ್ಲಿ ಏನಾದರೂ ವಿಚಿತ್ರ ಶಕ್ತಿಯಿತ್ತೇ, ಎಂದು ನೀವು ಆಶ್ಚರ್ಯಪಟ್ಟಿರಬಹುದು. ನನೆಗ ಒಂದು ಶಕ್ತಿಯಿದ್ದದ್ದು ನಿಜ; ಅದೆಂದರೆ - ನನ್ನ ಜೀವನದಲ್ಲಿ ನಾನು ಒಂದೇ ಒಂದು ಬಾರಿಯೂ ಲೈಂಗಿಕ ಯೋಚನೆ ಮನಸ್ಸಿನಲ್ಲಿ ಮೂಡುವುದಕ್ಕೆ ಅವಕಾಶ ಕೊಡಲಿಲ್ಲ. ನಾನು ನನ್ನ ಮನಸ್ಸನ್ನು, ಚಿಂತನಾಲಹರಿಯನ್ನು ನಿಯಂತ್ರಿಸಿದೆ; ಮನುಷ್ಯನು ಸಾಮಾನ್ಯವಾಗಿ ಆ ದಿಶೆಯಲ್ಲಿ ಬಳಸುವ ಶಕ್ತಿಸಾಮರ್ಥ್ಯಗಳನ್ನು ನಾನು ಉನ್ನತ ಸ್ತರದಲ್ಲಿ ಇರಿಸಿದೆ; ಅದು ಎಷ್ಟು ಬಲವಾದ ಶಕ್ತಿಯಾಗಿ ಬೆಳೆಯಿತೆಂದರೆ, ಯಾವುದೂ ಅದನ್ನು ಪ್ರತಿರೋಧಿಸಲಾರದಂತಾಯಿತು.” (\enginline{ND 6:155})

೩೮. ಬಹುಶಃ ನ್ಯೂಯಾರ್ಕ್ನಲ್ಲಿ ಸ್ವಾಮಿ ತುರೀಯಾನಂದರೊಡನೆ ನಡೆದ ಸಂಭಾಷಣೆ ಯೊಂದರಲ್ಲಿ:

“ಮೇಲಿನಿಂದ ಕರೆ ಬಂದಿದೆ: ‘ಬಾ, ಬಂದುಬಿಡು, ಇತರರಿಗೆ ಬೋಧಿಸುವುದನ್ನು ಕುರಿತು ತಲೆ ಕೆಡಿಸಿಕೊಳ್ಳಬೇಕಾದ ಆವಶ್ಯಕತೆಯಿಲ್ಲ’ ಎಂಬುದಾಗಿ. ಆಟ ಮುಗಿಯಬೇಕು ಎನ್ನುವುದು ಅಜ್ಜಿಯ\footnote{1. ಅಜ್ಜಿ ಎಂದರೆ - ಮಕ್ಕಳ ಕಣ್ಣಾಮುಚ್ಚಾಲೆ ಆಟದಲ್ಲಿ ಮಗು ಬಂದು ಮುಟ್ಟಿದೊಡನೆ ಆಟವನ್ನು ಕೊನೆಗೊಳಿಸುವ ಹಿರಿಯ.} ಇಚ್ಛೆ”. (\enginline{ND 6:373})

೩೯. ‘ಪ್ರಬುದ್ಧ ಭಾರತ’ ದ ೧೯೦೨ರ ಜುಲೈ ಸಂಚಿಕೆಯಲ್ಲಿ ಬಂದ ಶ್ಲಾಘನೆಯೊಂದರಲ್ಲಿ “ಪಾಶ್ಚಾತ್ಯ ಶಿಷ್ಯ” ನೊಬ್ಬ ಹೀಗೆಂದು ಬರೆದಿರುವನು:

“ವಿಗ್ರಹಭಂಜಕರ ಮೇಲೆ ಸ್ವಾಮಿಗಳಿಗೆ ಸಹಾನುಭೂತಿಯೇನೂ ಇರಲಿಲ್ಲ; ಏಕೆಂದರೆ ಅವರೇ ಜಾಣ್ಮೆಯಿಂದ ಹೇಳಿದಂತೆ, “ನಿಜವಾದ ತತ್ತ್ವವೇತ್ತನಾದವನು ಯಾವುದನ್ನೂ ನಾಶ ಮಾಡಲು ಹೋಗುವುದಿಲ್ಲ, ಆದರೆ ಎಲ್ಲರಿಗೂ ಸಹಾಯಮಾಡಲು ಪ್ರಯತ್ನಿಸುತ್ತಾನೆ.” (\enginline{VIN 638})

೪೦. ೧೮೯೯ರ ಅಕ್ಟೋಬರ್ ೯ರಂದು ಸೋದರಿ ನಿವೇದಿತಾ, ಮಿಸ್ ಜೋಸೆಫಿನ್ ಮ್ಯಾಕ್ಲಿಯಾಡ್ಳಿಗೆ ಬರೆದ ಕಾಗದವೊಂದರಲ್ಲಿ ಸ್ವಾಮಿ ವಿವೇಕಾನಂದರನ್ನು ಸ್ಮರಿಸಿಕೊಂಡಿರುವುದು:

ಅವರು ಎಷ್ಟೊಂದು ವಿಮುಖರಾಗಿಬಿಟ್ಟಿದ್ದಾರೆ - “ಪ್ರಪಂಚದಲ್ಲಿ ನಿನ್ನ ಬದುಕು ಆತ್ಮ ಚಿಂತನೆಯಲ್ಲದೆ ಇನ್ನೇನೂ ಆಗುವುದು ಬೇಡ”. \enginline{(LSNI: 213)}

೪೧. ೧೮೯೯ರ ಅಕ್ಟೋಬರ್ ೧೮ರಂದು ಸೋದರಿ ನಿವೇದಿತಾ ಮಿಸ್ ಜೋಸೆಫಿನ್ ಮ್ಯಾಕ್ಲಿಯಾಡ್ಳಿಗೆ ಬರೆದ ಕಾಗದವೊಂದರಲ್ಲಿ ಸ್ವಾಮಿ ವಿವೇಕಾನಂದರು ಭೋಜನಾ ನಂತರ ಮಿಸೆಸ್ ಓಲ್ಬುಲ್ ಅವರೊಡನೆ ಸಂಭಾಷಿಸಿದ್ದನ್ನು ಸ್ಮರಿಸಿಕೊಂಡಿರುವುದು:

“ನೋಡಿ, ಪ್ರೀತಿ ಎನ್ನುವುದು ಒಂದು, ಐಕ್ಯತೆ ಎನ್ನುವುದು ಇನ್ನೊಂದು. ಐಕ್ಯವಾಗುವುದು ಪ್ರೀತಿಸುವುದಕ್ಕಿಂತ ಹೆಚ್ಚಿನದು.

“ನಾನು ಧರ್ಮವನ್ನು ‘ಪ್ರೀತಿಸು’ವುದಿಲ್ಲ. ಅದರೊಡನೆ ಐಕ್ಯವಾಗಿಬಿಡುತ್ತೇನೆ. ಅದೇ ನನ್ನ ಜೀವನ. ಯಾವನೊಬ್ಬನೂ ತನ್ನ ಜೀವನವೇ ಯಾವುದರಲ್ಲಿ ಕಳೆದು ಹೋಗಿದೆಯೋ, ಯಾವುದರಲ್ಲಿ ಅವನು ನಿಜವಾಗಿಯೂ ಸಫಲ ಸಾಧನೆ ಮಾಡಿರುವನೋ ಅದನ್ನು ಪ್ರೀತಿಸುವುದಿಲ್ಲ. ನಾವು ಯಾವುದನ್ನು ಪ್ರೀತಿಸುವೆವೋ ಅದು ಇನ್ನೂ ನಾವೇ ಆಗಿಬಿಟ್ಟಿರುವುದಿಲ್ಲ. ನಿಮ್ಮ ಪತಿ, ಸಂಗೀತಕ್ಕಾಗಿಯೇ ಯಾವಾಗಲೂ ನಿಷ್ಠರಾಗಿದ್ದರೋ ಅವರು ಅದನ್ನು ಪ್ರೀತಿಸಲಿಲ್ಲ. ಅದಕ್ಕೆ ಹೋಲಿಸಿದರೆ ಇನ್ನೂ ಕಲಿಯುವುದು ಬೇಕಾದಷ್ಟಿದ್ದ ಎಂಜಿನಿಯರಿಂಗ್ ಕ್ಷೇತ್ರವನ್ನು ಪ್ರೀತಿ ಸಿದರು. ಇದೇ ಭಕ್ತಿಗೂ ಜ್ಞಾನಕ್ಕೂ ಇರುವ ವ್ಯತ್ಯಾಸ; ಈ ಕಾರಣಕ್ಕಾಗಿಯೇ ಜ್ಞಾನವು ಭಕ್ತಿಗಿಂತ ಹೆಚ್ಚಿನದು”. \enginline{(LSN II:216)}

೪೨. ೧೯೦೪ರ ಅಕ್ಟೋಬರ್ ೧೫ರಂದು ನಿವೇದಿತಾ, ಮಿಸ್ ಜೋಸೆಫಿನ್ ಮ್ಯಾಕ್ಲಿಯಾಡ್ಳಿಗೆ ಬರೆದ ಕಾಗದವೊಂದರಲ್ಲಿ ಸ್ವಾಮೀಜಿ ತಮ್ಮ ಆಧ್ಯಾತ್ಮಿಕ ಪ್ರಬೋಧನೆಯ ಬಗ್ಗೆ ಹೇಳಿರುವುದನ್ನು ಸ್ಮರಿಸಿಕೊಂಡಿರುವುದು:

“ಅವರು ಹೊರಟುಹೋದ ಮೇಲೆ ಮಾತ್ರವೇ ಅವರಿಗೆ ತಿಳಿಯುವುದು, ಅವರಿಂದ ತಾವೆಷ್ಟು ಪಡೆದುಕೊಂಡಿರುವೆವು ಎನ್ನುವುದು”. \enginline{(LSN II:686)}

೪೩. ೧೯೦೪ರ ನವೆಂಬರ್ ೫ರಂದು ಸೋದರಿ ನಿವೇದಿತಾ ಆಲ್ಬರ್ಟಾ ಸ್ಟರ್ಗ್ಸ್ (ಲೇಡಿ ಸ್ಯಾಂಡ್ವಿಚ್)ಳಿಗೆ ಬರೆದ ಕಾಗದವೊಂದರಲ್ಲಿ ರಿಡ್ಜ್ಲಿ ಮೇನರ್ನಲ್ಲಿ ಇದ್ದಾಗ ಸ್ವಾಮಿ ವಿವೇಕಾನಂದರು ತ್ಯಾಗದ ಬಗ್ಗೆ ಮಾತನಾಡಿದ್ದನ್ನು ಸ್ಮರಿಸಿಕೊಂಡಿರುವುದು:

“ಭಾರತದಲ್ಲಿ ನಾವು ಎಂದಿಗೂ ಅಲ್ಪವಾದುದರಸಲುವಾಗಿ ಹಿರಿದಾದುದನ್ನು ತ್ಯಜಿಸಬೇಕು ಎಂದು ಹೇಳುವುದಿಲ್ಲ. ಸೌಲಭ್ಯಗಳಲ್ಲಿಯೋ ಸುಖಗಳಲ್ಲಿಯೋ ಮುಳುಗಿಹೋಗಿರುವು ದಕ್ಕಿಂತ ಸಂಗೀತ ಅಥವಾ ಸಾಹಿತ್ಯದಲ್ಲಿ ತಾದಾತ್ಮ್ಯ ಹೊಂದುವುದು ಎಷ್ಟೋ ಮೇಲು; ಇದಕ್ಕಿಂತ ಭಿನ್ನವಾದದ್ದನ್ನು ನಾವೆಂದೂ ಹೇಳುವುದಿಲ್ಲ”. \enginline{(LSNI:690)}

೪೪. ೧೯೦೯ರ ನವೆಂಬರ್ ೧೯ರಂದು ಸೋದರಿ ನಿವೇದಿತಾ ಅವರು ಮಿಸ್ ಜೋಸೆಫಿನ್ ಮ್ಯಾಕ್ಲಿಯಾಡ್ ಅವರಿಗೆ ಬರೆದ ಪತ್ರದಲ್ಲಿ:

“ಬೆಂಕಿಯಲ್ಲಿ ಕೈಯಿಟ್ಟರೆ ಕೈ ಸುಟ್ಟೇ ಸುಡುತ್ತದೆ -ನಮಗೆ ಅದರ ಸಂವೇದನೆ ತಟ್ಟಿದರೂ, ತಟ್ಟದಿದ್ದರೂ - ಅಂತೆಯೇ ಭಗವಂತನ ಹೆಸರನ್ನು ಕೀರ್ತಿಸುವವನಿಗೂ ಸಹ.” \enginline{(LSN II: 1030)}

೪೫. ೧೯೧೦ರ ಜುಲೈ ೬ರಂದು ಸೋದರಿ ನಿವೇದಿತಾ ಅವರು ಡಾ. ಟಿ. ಕೆ. ಕೇಯ್ನ್ ಅವರಿಗೆ ಬರೆದ ಪತ್ರದಲ್ಲಿ ದಾಖಲಿಸಿದ ಸ್ವಾಮಿ ವಿವೇಕಾನಂದರ ಶ‍್ರೀರಾಮಕೃಷ್ಣ ಸ್ಮರಣೆ:

“ತಾವು ಯಾರಿಗಾದರೂ ಬೋಧಿಸುತ್ತಿರುವೆನೆಂದು ಕಲ್ಪಿಸಿಕೊಳ್ಳಲು ಅವರಿಗೆಂದೂ ಸಾಧ್ಯವಾಗುತ್ತಿರಲಿಲ್ಲ. ಅವರಿದ್ದುದು ಬಣ್ಣ ಬಣ್ಣದ ಚೆಂಡುಗಳೊಂದಿಗೆ ಆಟವಾಡುವ ವರಂತೆ; ಇತರರು ತಮಗೆ ಬೇಕಾದುದನ್ನು ತಾವೇ ಆರಿಸಿಕೊಳ್ಳಲಿ ಎಂದು ಬಿಡುತ್ತಿದ್ದರು”. \enginline{(LSN II:1110)}

೪೬. ಸ್ವಾಮಿ ವಿವೇಕಾನಂದರು ತಮ್ಮೊಂದಿಗೆ ನಡೆಸಿದ ಸಂಭಾಷಣೆಯನ್ನು ೧೮೯೯ರಲ್ಲಿ ರಿಡ್ಜ್ಲಿ ಮೇನರ್ನಿಂದ ಸೋದರಿ ನಿವೇದಿತಾ ಅವರು ಮಿಸ್ ಜೋಸೆಫಿನ್ ಮ್ಯಾಕ್ಲಿ ಯಾಡ್ ಅವರಿಗೆ ಬರೆದ ಪತ್ರದಲ್ಲಿ ಸ್ಮರಿಸಿಕೊಂಡಿರುವುದು:

ಪ್ರವಾದಿಯು ಶ‍್ರೀರಾಮಕೃಷ್ಣರ ವಿಚಾರವಾಗಿ ಮಾತನಾಡುವುದನ್ನು ಅಷ್ಟಾಗಿ ನಾನೆಂದೂ ಕೇಳಿಲ್ಲ. ನಾನು ಹಿಂದೆಯೇ ಕೇಳಿದ್ದಂತೆ, ಜನರನ್ನು ಕುರಿತಾದ ಅವರ (ತಮ್ಮ ಗುರುಗಳ) ವಿವೇಚನೆ ತಪ್ಪಾದುದೇ ಇಲ್ಲ ಎಂದು ಮಾತ್ರ ನಮಗೆ ಹೇಳಿದರು....

“ಆದ್ದರಿಂದಲೇ ನೋಡಿ”, ಸ್ವಾಮಿಗಳೆಂದರು, “ನನ್ನ ಭಕ್ತಿ ಎಂದರೆ ನಾಯಿಯ ನಿಷ್ಠೆಯ ಹಾಗೆ. ಹೆಜ್ಜೆಹೆಜ್ಜೆಗೂ ನಾನು ತಪ್ಪುತ್ತಿದ್ದೆ, ಆದರೆ ಅವರು ಮಾತ್ರ ಯಾವಾಗಲೂ ಒಪ್ಪವಾಗಿಯೇ ಇರುತ್ತಿದ್ದರು; ಈಗ ನಾನು ಅವರ ವಿವೇಚನೆಯನ್ನು ಕಣ್ಣುಮುಚ್ಚಿಕೊಂಡು ನಂಬುವೆ”. ಆನಂತರ ಸ್ವಾಮಿಗಳು, ಹೇಗೆ ಅವರು ತಮ್ಮಲ್ಲಿಗೆ ಬಂದ ಯಾರನ್ನಾದರೂ ವಶಪಡಿಸಿಕೊಂಡುಬಿಡುತ್ತಿದ್ದರು, ಮತ್ತು ಹೇಗೆ ಎರಡು ನಿಮಿಷಗಳಲ್ಲಿ ಅವರ ಬಗ್ಗೆ ಎಲ್ಲವನ್ನೂ ತಿಳಿದುಕೊಂಡುಬಿಡುತ್ತಿದ್ದರು ಎಂಬುದನ್ನು ಹೇಳಿದರು. ನಮ್ಮ ಪ್ರಜ್ಞೆ ಎಂಬುದು ಅದೆಷ್ಟು ಅಲ್ಪವಾದುದು ಎಂಬುದನ್ನು ತಾನು ಅದರಿಂದ ಅರಿತುಕೊಂಡೆ ಎಂದು ನುಡಿದರು. \enginline{(LSN II:1263)}

೪೭. ೧೯೦೦ರ ಜನವರಿ ೨೭ರಂದು ಸೋದರಿ ನಿವೇದಿತಾ ಅವರು ಸೋದರಿ ಕ್ರಿಸ್ಟೈನ್ ಅವರಿಗೆ ಬರೆದ ಪತ್ರದಿಂದ:

“ಮಾನವರ ಆವಶ್ಯಕತೆಗಳನ್ನು ಬೇರೊಂದು ಬೆಳಕಿನಲ್ಲಿ ತಾನು ನೋಡಲಾರಂಭಿಸಿರುವೆ ಎಂದು ಸ್ವಾಮಿಗಳು ಈ ಹೊತ್ತು ನುಡಿದರು. ಯಾವುದು ತಾತ್ತ್ವಿಕವಾಗಿ ಸಹಾಯ ಮಾಡುವುದು ಎಂಬುದು ತನಗೆ ಈಗಾಗಲೇ ಖಚಿತವಾಗಿ ಗೊತ್ತಿದೆ; ಆದರೆ ಮಾರ್ಗೋಪಾಯಗಳಾವುವು ಎಂಬ ಸಮಸ್ಯೆಗೆ ಪರಿಹಾರನ್ನು ಕಂಡುಹಿಡಿಯಲು ನಿತ್ಯವೂ ಗಂಟೆಗಟ್ಟಲೆ ಶ್ರಮಿಸು ತ್ತಿರುವೆ ಎಂದರು. ಇಲ್ಲಿಯವರೆಗೆ ತನಗೆ ಗೊತ್ತಿದ್ದುದು ಗುಹೆಯೊಂದರಲ್ಲಿ ಏಕಾಂಗಿಯಾಗಿ, ೪೪೦ ಸ್ವಾಮಿ ವಿವೇಕಾನಂದರ ಕೃತಿಶ್ರೇಣಿ ಉಕ್ತಿಗಳು ಮತ್ತು ವಚನಗಳು ೪೪೧ ಯಾರೊಬ್ಬರ ಹಂಗಿಲ್ಲದೆ ಇರುವವರಿಗೆ ಅಗತ್ಯವಾದುದುದು ಮಾತ್ರ; ಆದರೆ ಈಗ ತಾನು “ಮಾನವ ಜನಾಂಗಕ್ಕೆ ನಿತ್ಯಜೀವನದ ಒತ್ತಡಗಳ ನಡುವೆಯೂ ಶಕ್ತಿದಾಯಕವಾದ ಯಾವುದೋ ಒಂದನ್ನು ಕೊಡಲಿದ್ದೇನೆ” ಎಂದರು. \enginline{(LSN II:1264)}

೪೮. ೧೯೦೨ರ ಜುಲೈ ೭ರಂದು ಸೋದರಿ ನಿವೇದಿತಾ ಅವರು ಸೋದರಿ ಕ್ರಿಸ್ಟೈನ್ ಅವರಿಗೆ ಬರೆದ ಪತ್ರವೊಂದರಲ್ಲಿ ದಾಖಲಿಸಿದಂತೆ ಸ್ವಾಮಿ ವಿವೇಕಾನಂದರು ೧೯೦೨ರ ಜುಲೈ ೪ರಂದು ಬೇಲೂರು ಮಠದಲ್ಲಿ ಸಂನ್ಯಾಸಿಗಳಿಗೆ ತೆಗೆದುಕೊಂಡ ತರಗತಿಯಲ್ಲಿ ಹೇಳಿದ ಮಾತು:

“ನನ್ನನ್ನು ಅನುಕರಣೆ ಮಾಡಬೇಡಿ. ಯಾರು ಅನುಕರಣೆ ಮಾಡುವರೋ ಅವರನ್ನು ಒದ್ದೋಡಿಸಿ” \enginline{(LSN II:1270)}

೪೯. ಸ್ವಾಮಿ ವಿವೇಕಾನಂದರು ಆತ್ಮಮೋಕ್ಷದ ಆದರ್ಶವನ್ನು ಕುರಿತು ಹೇಳಿದ ಒಂದು ಮಾತು ಮಾನವಸೇವೆಯೇ ವ್ಯಕ್ತಿಯ ಆದರ್ಶವೆಂಬ ಪಾಶ್ಚಾತ್ಯ ಕಲ್ಪನೆಗೆ ವಿರೋಧವಾಗಿರುವಂತೆ ಕಂಡುಬಂದಾಗ, ಅದಕ್ಕೆ ಸ್ವಾಮಿಗಳು ತೋರಿದ ಪ್ರತಿಕ್ರಿಯೆ:

“ಇದರಿಂದ ಸಮಾಜಕ್ಕೇನೂ ಲಾಭವಾಗುವುದಿಲ್ಲ ಎಂದು ನೀವೆನ್ನುವಿರಿ. ಈ ಆಕ್ಷೇಪಣೆ ಸಲ್ಲುವಂತಾಗಬೇಕಾದರೆ, ಮೊದಲು ಸಮಾಜವನ್ನು ಸುಸ್ಥಿತಿಯಲ್ಲಿರಿಸುವುದು ಎಂಬುದು ಒಂದು ಲಕ್ಷ್ಯ ವಿಷಯವೆಂದು ನೀವು ಸಾಧಿಸಬೇಕಾಗುತ್ತದೆ”. \enginline{(CWSN 1:19)}

೫೦. ಸೋದರಿ ನಿವೇದಿತಾ ಬರೆದಿರುವುದು:

“ತಾವು ಪರಿವ್ರಾಜಕ ಪ್ರಬೋಧಕರಾಗಿರುವ ವಿಷಯವನ್ನೊಮ್ಮೆ ಎತ್ತಿಕೊಂಡ ಅವರು ಧಾರ್ಮಿಕ ಸಂಘಟನೆಯ ಬಗ್ಗೆ ಭಾರತಕ್ಕೆ ಹಿಂಜರಿಕೆ; ಅಥವಾ ಯಾರೋ ಹೇಳಿರುವ ಹಾಗೆ “ಒಂದು ಚರ್ಚ್ನಲ್ಲಿ ಕೊನೆಗೊಳ್ಳುವ ಶ್ರದ್ಧೆ” ಯ ಬಗ್ಗೆ ಹಿಂಜರಿಕೆ ಎಂದರು. “ಸಂಘಟನೆ ಯಾವಾಗಲೂ ಹೊಸ ಕೇಡುಗಳನ್ನು ಹುಟ್ಟಿಹಾಕುತ್ತದೆ ಎಂದೇ ನಾವು ನಂಬುತ್ತೇವೆ”.

ಹಣದ ಮೇಲಿನ ವ್ಯಾಮೋಹದಿಂದಾಗಿ ಆಗ ಪಶ್ಚಿಮದಲ್ಲಿ ಜಾರಿಯಲ್ಲಿದ್ದ ಕೆಲವು ಧಾರ್ಮಿಕ ಉತ್ಥಾನಗಳು ಬೇಗನೆ ಸಾಯುವುವು ಎಂದು ಅವರು ಭವಿಷ್ಯ ನುಡಿದರು. ಅಲ್ಲದೆ, “ಮಾನವನು ಸತ್ಯದಿಂದ ಸತ್ಯಕ್ಕೆ ಮುಂದುವರೆಯುತ್ತಾನೆಯೇ ಹೊರತು ತಪ್ಪಿನಿಂದ ಸತ್ಯದೆಡೆಗೆ ಅಲ್ಲ” ಎಂದು ಉದ್ಘೋಷಿಸಿದರು. \enginline{(CWSN 1:19-20)}

೫೧. “ಈ ವಿಶ್ವವು ಒಂದು ಜೇಡರ ಬಲೆಯಂತೆ, ಮನಸ್ಸುಗಳು ಜೇಡರ ಹುಳುಗಳಂತೆ; ಏಕೆಂದರೆ ಮನಸ್ಸು ಏಕವೂ ಹೌದು, ಅನೇಕವೂ ಹೌದು” \enginline{(CWSN 1:21)}

೫೨. “ತಮ್ಮ ಮನಸ್ಸಿಗೆ ಒಪ್ಪಿಗೆಯಾಗುವುದು ಕಷ್ಟವಾಗಿದೆಯಲ್ಲಾ ಎಂದು ಯಾರೂ ಖೇದ ಪಡುವುದು ಬೇಡ! ನಾನು ನನ್ನ ಗುರುವಿನ ಜೊತೆಗೆ ಆರು ವರ್ಷಗಳ ಕಾಲ ಹೋರಾಡಿದೆ - ಅದರ ಫಲವಾಗಿ ಕ್ರಮಿಸಬೇಕಾಗಿರುವ ಮಾರ್ಗದ ಪ್ರತಿಯೊಂದು ಅಂಗುಲವನ್ನೂ ನಾನು ಅರಿತಿರುವೆ! ಪ್ರತಿಯೊಂದು ಅಂಗುಲವನ್ನೂ ಸಹ!. \enginline{(CWSN 1:22)}

೫೩. ಭಕ್ತಿಮಾರ್ಗವು ಸ್ವಾರ್ಥರಾಹಿತ್ಯದ ಅದೆಂತಹ ಉತ್ತುಂಗಕ್ಕೆ ಕರೆದೊಯ್ಯುತ್ತದೆ, ಹೇಗೆ ಅದು ಒಬ್ಬನಲ್ಲಿರುವ ಪರಮೋಚ್ಚ ಅಂಶವನ್ನು ಹೊರಕ್ಕೆ ತರುತ್ತದೆ ಎಂಬುದನ್ನು ವಿಶದ ಪಡಿಸುತ್ತಿದ್ದಾಗ ಸ್ವಾಮಿ ವಿವೇಕಾನಂದರು ಹೇಳಿದ್ದು:

“ವ್ಯಾಘ್ರವೊಂದು ಬರುತ್ತಿರುವ ಹಾದಿಯಲ್ಲಿ ಒಂದು ಮಗು ಇದೆ ಎಂದು ಊಹಿಸಿಕೊಳ್ಳಿ! ಆಗ ನಿಮ್ಮ ಸ್ಥಾನವೆಲ್ಲಿರುತ್ತದೆ? ವ್ಯಾಘ್ರದ ಬಾಯಲ್ಲಿ - ನಿಮ್ಮಲ್ಲಿ ಯಾರೊಬ್ಬರಾದರೂ ಅಷ್ಟೇ - ನನಗೆ ಅದು ಖಂಡಿತವಿದೆ!”. \enginline{(CWSN 1:24)}

೫೪. “ಈ ಎಲ್ಲವೂ ಯಾವುದರಿಂದ ವ್ಯಾಪ್ತವಾಗಿದೆಯೋ, ಅದನ್ನೇ ಭಗವಂತನೆಂದು ತಿಳಿಯಿರಿ!”. \enginline{(CWSN 1:27)}

೫೫. ಧರ್ಮದ ಬಗ್ಗೆ ಸ್ವಾಮಿ ವಿವೇಕಾನಂದರ ಧೋರಣೆಯನ್ನು ಕುರಿತು:

ಧರ್ಮವೆಂದರೆ ವ್ಯಕ್ತಿಯೊಬ್ಬನ ಬೆಳವಣಿಗೆಯ ಸಂಗತಿ, “ಯಾವಾಗಲೂ ಆಗಿರುವ, ಆಗಬೇಕಾಗಿರುವ ಪ್ರಶ್ನೆ!”. \enginline{(CWSN 1:28)}

೫೬. “ಸುಲಭ ಜಯಕ್ಕಾಗಿ ನೀವು ಗಂಧರ್ವಸೈನ್ಯದಳವನ್ನೇ ತರುವ ಹಾಗಿದ್ದಾಗಲೂ ಸಹ ಕ್ಷಮೆಯನ್ನು ಪ್ರದರ್ಶಿಸಿ”. ವಿಜಯವು ಇನ್ನೂ ಅನುಮಾನಾಸ್ಪದವಾಗಿರುವಾಗ ಮಾತ್ರ, ಹೇಡಿಯಾದವನು ಮಾತ್ರವೇ ತನ್ನ ಇನ್ನೊಂದು ಕೆನ್ನೆಯನ್ನು ತೋರಿಸುವನು. \enginline{(CWSN 1:28-29)}

೫೭. “ಅದರಿಂದ ಒಬ್ಬ ಮಾನವಪ್ರಾಣಿಗೆ ನಿಜವಾಗಿಯೂ ಸಹಾಯವಾಗುವ ಹಾಗಿ ದ್ದರೆ, ಖಂಡಿತಕ್ಕೂ ನಾನು ಪಾತಕವನ್ನು ಮಾಡಿಯೇನು, ಆ ಮೂಲಕ ಚಿರಕಾಲ ನರಕಕ್ಕೆ ಹೋದೇನು!”. \enginline{(CWSN 1:34)}

೫೮. ಒಂದು ಉಪನ್ಯಾಸ ಮುಗಿದ ನಂತರ, ಸೋದರಿ ನಿವೇದಿತಾಳನ್ನೂ ಒಳಗೊಂಡ ಸಣ್ಣದೊಂದು ಗುಂಪನ್ನುದ್ದೇಶಿಸಿ:

“ನನ್ನಲ್ಲೂ ಒಂದು ಮೂಢನಂಬಿಕೆಯಿದೆ - ಅಂಥದೇನೂ ಅಲ್ಲ ಬಿಡಿ, ವೈಯಕ್ತಿಕವಾದ ಒಂದು ಮೂಢನಂಬಿಕೆ! - ಒಮ್ಮೆ ಬುದ್ಧನಾಗಿ ಬಂದಿದ್ದವನೇ ಆ ನಂತರ ಕ್ರಿಸ್ತನಾಗಿ ಬಂದ ಎಂದು”. \enginline{(CWSN 1:35)}

೫೯. ಭಾರತದಲ್ಲಿ ಸೇವೆ ಸಲ್ಲಿಸುವ ಸೋದರಿ ನಿವೇದಿತಾ ಅವರ ಇಚ್ಛೆಯನ್ನು ಸ್ವಾಮಿ ವಿವೇಕಾನಂದರಿಗೆ ತಿಳಿಸಿದ ನಂತರ:

“ನನ್ನ ಮಟ್ಟಿಗೆ ಹೇಳುವುದಾದರೆ, ನಾನು ನನ್ನ ಜನರಿಗಾಗಿ ಮಾಡುತ್ತೇನೆಂದು ಕೈಗೊಂಡಿರುವ ಈ ಕಾರ್ಯಕ್ಕಾಗಿ ಇನ್ನೂರು ಸಲ ಬೇಕಾದರೂ ಹುಟ್ಟಿಬರುವೆ”. \enginline{(CWSN 1:36)}

೬೦. ಘಟನೆಯೊಂದನ್ನು ಸೋದರಿ ನಿವೇದಿತಾ ನೆನೆಸಿಕೊಂಡಿರುವುದು:

“ಅವರು ಒಮ್ಮೆ ಖೇತ್ರಿ ಮಹಾರಾಜನ ಜೊತೆಗೆ ಸವಾರಿ ಮಾಡುತ್ತಿದ್ದರು. ಅವರ ತೋಳಿನಲ್ಲಿ ಗಾಯವಾಗಿ ಧಾರಾಕಾರ ರಕ್ತಸ್ರಾವವಾಗುತ್ತಿರುವುದನ್ನು ಸ್ವಾಮೀಜಿ ನೋಡಿದರಂತೆ. ಆ ಗಾಯ ತಾವು ಮುಂದೆ ಹೋಗಲು ಅನುವಾಗುವಂತೆ ಮಹಾರಾಜರು ಮುಳ್ಳುಕೊಂಬೆಯೊಂದನ್ನು ಬದಿಗೆಳೆದುಕೊಂಡಿದ್ದರಿಂದ ಆದದ್ದು ಎಂದು ತಿಳಿಯಿತಂತೆ. ಸ್ವಾಮಿಗಳು ಸ್ನೇಹಪೂರ್ವಕವಾಗಿ ಆಕ್ಷೇಪಿಸಲು, ಆ ರಜಪೂತ ಮಹಾರಾಜರು ಅದೇನು ಮಹಾ ದೊಡ್ಡ ಸಂಗತಿ ಎಂಬಂತೆ ನಕ್ಕು, “ನಾವು ಯಾವಾಗಲೂ ಸಂಪ್ರದಾಯ - ಶ್ರದ್ಧೆಯನ್ನು ರಕ್ಷಿಸಬೇಕಾದವರಲ್ಲವೆ, ಸ್ವಾಮೀಜಿ?” ಎಂದರಂತೆ.

ಕಥೆಯನ್ನು ಮುಂದುವರೆಸುತ್ತ ಸ್ವಾಮಿಗಳೆಂದರು, “ಅನಂತರ, ಸಂನ್ಯಾಸಿಯೊಬ್ಬನಿಗೆ ಅಂತಹ ದೊಡ್ಡ ಮರ್ಯಾದೆಯನ್ನು ನೀವು ತೋರಿಸಬೇಕಾದದ್ದಿಲ್ಲ ಎಂದು ಅವರಿಗೆ ಹೇಳಬೇಕು ಎಂದುಕೊಳ್ಳುವಷ್ಟರಲ್ಲಿಯೇ, ಬಹುಶಃ ಅವರು ಹೇಳುವುದೂ ಸರಿಯಾದದ್ದೇ ಇರಬೇಕು ಎಂದು ನನಗನ್ನಿಸಿತು. ಯಾರಿಗೆ ಗೊತ್ತು? ನಾನೂ ಒಂದು ಕ್ಷಣಕಾಲ ಈ ನವನಾಗರಿಕತೆಯ ಥಳಕುಬೆಳಕಿನಲ್ಲಿ ಸಿಕ್ಕಿಹಾಕಿಕೊಂಡಿರಬಹುದು ಎಂದುಕೊಂಡೆ”.

ಅನೇಕರ ಯೋಗಕ್ಷೇಮವನ್ನೂ ಬಹಳಷ್ಟು ಕೆಲಸವನ್ನೂ ಮೈಮೇಲೆ ಎಳೆದುಕೊಂಡಿರುವ ಬೇಲೂರಿನ ಮುಖ್ಯಸ್ಥರಿಗಿಂತ ಹೋದಲ್ಲೆಲ್ಲ ಹೆಸರು ಬದಲಾಯಿಸಿಕೊಳ್ಳುತ್ತ, ಜ್ಞಾನವನ್ನು ಹರಡುತ್ತ ಸಾಗುತ್ತಿದ್ದ ಪುರಾತನ ಕಾಲದ ಪರಿವ್ರಾಜಕ ಸಂನ್ಯಾಸಿಯೇ ದೊಡ್ಡವನು ಎಂದು ತನಗೆ ಅನ್ನಿಸುತ್ತದೆ ಎಂದು ಯಾರೋ ಒಬ್ಬರು ವಿರೋಧಿಸಿದಾಗ, ಸ್ವಾಮಿಗಳು ಸುಮ್ಮನೆ “ಹೌದು, ನಾನು ಬಂಧನಕ್ಕೆ ಸಿಕ್ಕಿಕೊಂಡಿರುವೆ” ಎಂದು ಬಿಟ್ಟರು. \enginline{(CWSN 1:43)}

೬೧. ಸೋದರಿ ನಿವೇದಿತಾ ಬರೆದಿರುವಳು:

“ಪಶ್ಚಿಮದೇಶದಲ್ಲಿ ಒಂದು ದಿನ ಅವರು ಒಂದಾನೊಂದು ಕಾಲದಲ್ಲಿ ಚಿತ್ತೂರಿನ ರಾಣಿಯಾಗಿದ್ದ ಸಂತ ಮೀರಾಬಾಯಿಯ ವಿಚಾರವಾಗಿ ಮಾತನಾಡುತ್ತಿದ್ದರು. ಅರ ಮನೆಯ ಒಳಗಡೆ ಇರುವುದಕ್ಕೆ ಒಪ್ಪಿದ್ದೇ ಆದರೆ ಎಲ್ಲ ಸ್ವಾತಂತ್ರ್ಯವನ್ನೂ ನೀಡಿದ ಅವಳ ಪತಿಯ ವಿಚಾರವನ್ನೂ ಹೇಳಿದರು. ಆದರೆ ಮೀರಾ ಬಂಧನದಲ್ಲಿರಲು ಒಪ್ಪಲಿಲ್ಲ ಎಂದರು. ನೆರೆದಿದ್ದವರಲ್ಲಿ ಯಾರೋ ಒಬ್ಬರು ಅಚ್ಚರಿಯಿಂದ “ಏಕೆ ಒಪ್ಪಬಾರದಾಗಿತ್ತು?” ಎಂದು ಪ್ರಶ್ನಿಸಿದಾಗ ಸ್ವಾಮಿಗಳು “ಏಕೆ ಒಪ್ಪಬೇಕಾಗಿತ್ತು? ಅವಳೇನು ಇಲ್ಲಿ ಈ ಕೊಳಚೆಯ ಕೂಪದಲ್ಲಿ ಬದುಕಿದ್ದಳೆ?” ಎಂದರು. \enginline{(CWSN 1:44)}

೬೨. ವರ್ಷಗಳು ಕಳೆದಂತೆಲ್ಲ, ಸ್ವಾಮೀಜಿ ನಿರ್ದಿಷ್ಟ ಯೋಜನೆಗಳನ್ನು ಹಾಕಲು, ಅಜ್ಞಾತದ ಬಗ್ಗೆ ಅಧಿಕಾರಯುತವಾಗಿ ಮಾತನಾಡಲು, ಹೆಚ್ಚಾಗಿ ಧೈರ್ಯವಹಿಸುತ್ತಿರಲಿಲ್ಲ:

“ಒಟ್ಟಿನ ಮೇಲೆ, ನಮಗೆ ಗೊತ್ತಿರುವುದಾದರೂ ಏನು? ಎಲ್ಲವನ್ನೂ ಬಳಸುವವಳು ತಾಯಿ. ನಾವು ಕೇವಲ ತಡಕಾಡುತ್ತಿರುವುದಷ್ಟೇ.” \enginline{(CWSN 1:44)}

೬೩. ಸ್ವಾಮಿ ವಿವೇಕಾನಂದರನ್ನು ಉಲ್ಲೇಖಿಸುತ್ತ ಸೋದರಿ ನಿವೇದಿತಾ ನೆನಪಿಸಿಕೊಂಡಿರುವಳು:

“ಕಾರಣವಿಲ್ಲದೆ, ‘ಉದ್ದೇಶ’ವಿಲ್ಲದೆ ಇರುವುದು ಅಲ್ಲದಿದ್ದರೆ, ಪ್ರೇಮವು ಪ್ರೇಮವಾಗುವುದಿಲ್ಲ, ಒತ್ತಾಯವಾಗುತ್ತದೆ, ಆಗ್ರಹವಾಗುತ್ತದೆ...”. \enginline{(CWSN 1:52)}

೬೪. ಸ್ವಾಮಿ ವಿವೇಕಾನಂದರನ್ನು ಕುರಿತು ಸೋದರಿ ನಿವೇದಿತಾ ಬರೆದಿರುವಳು:

“ಬ್ರಿಟಿಷರನ್ನು ಅವರ ನಾಡಿನಲ್ಲೇ ನೋಡಿದ ಮೇಲೆ ನಿಮಗೆ ಅವರ ಮಹತ್ಸಾಧನೆ ಏನೆನಿಸುತ್ತದೆ ಎಂದು ತಮ್ಮ ಕೆಲವು ದೇಶೀಯರೇ ಪ್ರಶ್ನಿಸಿದಾಗ “ಅವರು ಆತ್ಮಗೌರವವನ್ನೂ ವಿಧೇಯತೆಯನ್ನೂ ಹೇಗೆ ಒಗ್ಗೂಡಿಸಿಟ್ಟುಕೊಳ್ಳಬೇಕೆಂದು ಅರಿತಿರುವರು” ಎಂದು ಉತ್ತರಿ ಸಿದರು. \enginline{(CWSN 1:54)}

೬೫. ಸ್ವಾಮಿ ಸದಾನಂದರು ವರದಿ ಮಾಡಿರುವಂತೆ, ಸ್ವಾಮಿ ವಿವೇಕಾನಂದರು ಇನ್ನೂ ಬೆಳಗಾಗುವುದಕ್ಕಿಂತ ಮುಂಚೆಯೇ ಎದ್ದು ಇತರರನ್ನೂ ಎಬ್ಬಿಸುತ್ತ ಹಾಡುತ್ತಿದ್ದರು:

\begin{myquote}
“ದಿವ್ಯಾಮೃತವನು ಕುಡಿಯಲು ಬಯಸುವ-\\ರೆಲ್ಲರು ಏಳಿರಿ! ಎದ್ದೇಳಿ!” \enginline{(CWSN 1:56)}
\end{myquote}

೬೬. ಸೋದರಿ ನಿವೇದಿತಾ ಸ್ಮರಿಸಿಕೊಂಡಿರುವಳು:

ಆ ಕಾಲದಲ್ಲಿ (ಸ್ವಾಮಿಗಳ ಪರಿವ್ರಾಜಕ ದಿನಗಳಲ್ಲಿ, ಆಲ್ಮೋರದ ಬಳಿ) ಒಂದು ಪರ್ವತ ತಪ್ಪಲಿನ ಗ್ರಾಮವನ್ನು ಮೇಲಿನಿಂದ ನೋಡಬಹುದಾಗಿದ್ದ ಗುಹೆಯೊಂದರಲ್ಲಿ ಕೆಲವು ತಿಂಗಳುಗಳು ಇದ್ದರು. ಈ ಅನುಭವವನ್ನು ಕುರಿತು ಅವರು ಎರಡು ಬಾರಿ ಪ್ರಸ್ತಾಪಿಸಿದ್ದು ಮಾತ್ರ ನನಗೆ ಗೊತ್ತು. ಒಂದು ಸಲ ಅವರು “ನನ್ನ ಇಡೀ ಜೀವನದಲ್ಲಿ ಇನ್ನಾವುದೂ ನನ್ನ ಕರ್ತವ್ಯ ಪ್ರಜ್ಞೆಯನ್ನು ಉದ್ದೀಪಿಸಲಿಲ್ಲ. ಆ ಗುಹಾಜೀವನದಿಂದ ಕೆಳಗಿಳಿದು ಬಯಲಿನಲ್ಲಿ ಅತ್ತಿತ್ತ ಅಡ್ಡಾಡು ಎಂದು ಯಾರೋ ನನ್ನನ್ನು ಅಲ್ಲಿಂದ ಕೆಳಕ್ಕೆ ಎಸೆದಂತಿತ್ತು” ಎಂದರು. ಇನ್ನೊಮ್ಮೆ ಯಾರೋ ಒಬ್ಬರಿಗೆ ಹೇಳಿದರು: “ಒಬ್ಬನ ಬದುಕು ಯಾವ ರೂಪದಲ್ಲಿ ಸಾಗುತ್ತಿದೆ ಎನ್ನುವುದು ಅವನನ್ನು ಸಾಧುವ ನ್ನಾಗಿ ಮಾಡುವುದಿಲ್ಲ. ಏಕೆಂದರೆ ಗುಹೆಯಲ್ಲಿ ಕುಳಿತಿದ್ದರೂ ಸಹ ಒಬ್ಬನ ಇಡೀ ಮನಸ್ಸಿನಲ್ಲಿ ರಾತ್ರಿಯ ಹೊತ್ತಿಗೆ ಎಷ್ಟು ರೊಟ್ಟಿಗಳು ಸಿಕ್ಕಬಹುದು ಎಂಬ ಪ್ರಶ್ನೆಯೇ ತುಂಬಿಕೊಂಡಿರಬಹುದು!” \enginline{(CWSN 1:61)}

೬೭. ತಮ್ಮದೇ “ತಾಯಿ ಕಾಳಿ” ಎಂಬ ಕವನವನ್ನು ಕುರಿತು:

“ಸಂಕಷ್ಟ ದುಃಖಗಳನತ್ತಿತ್ತ ಎಸೆಯುತ್ತ”, ಎಂದು ತಮ್ಮ ಕವನದಿಂದಲೇ ಅವರು ಉಲ್ಲೇಖಿಸಿದರು -

\begin{myquote}
ಹುಚ್ಚುವರಿದು ಕುಣಿಯುತ\\ಸಂಕಟಗಳನೀಡಾಡುತ\\ಬಾ, ತಾಯಿ, ಬಾ!\\ನಲಿನಲಿಯುತ ಬಾ!\\ಕರಾಳಿ ಕಾಳಿ ನಿನ್ನ ಹೆಸರು, ಮೃತ್ಯುವೆ ನಿನ್ನುಸಿರು!\\ನಿನ್ನಡಿಗಳ ಕಂಪನದಿಂ\\ನಿರ್ನಾಮವು ಜಗವು!
\end{myquote}

“ಅದರಲ್ಲಿನ ಒಂದೊಂದು ಪದವೂ ನಿಜವಾಯಿತು” ಎಂದರು, ಮಧ್ಯದಲ್ಲೇ ತಡೆದು.

\begin{myquote}
ಅಳಲುಗಳನು ಪ್ರೀತಿಸಿ,\\ಮೃತ್ಯುವನಾಲಿಂಗಿಸಿ,\\ಪ್ರಳಯ ನಾಟ್ಯಗೈವ ಧೀರ-\\ನೆಡೆಗೆ ತಾಯಿ ಬರುಳು!
\end{myquote}

“ಅವನೆಡೆಗೆ ಖಂಡಿತ ಆ ತಾಯಿ ಬರುವಳು. ನಾನು ಅದನ್ನು ಸಾಧಿಸಿ ತೋರಿಸಿದ್ದೇನೆ. ಏಕೆಂದರೆ, ನಾನೇ ಮೃತ್ಯುವಿನ ರೂಪವನ್ನು ಅಪ್ಪಿಕೊಂಡಿದ್ದೇನೆ!” \enginline{(CWSN 1: 98-99)}

೬೮. ಬಾಲಕಿಯರ ಶಾಲೆಯ ಬಗ್ಗೆ ತನ್ನ ಯೋಜನೆಯನ್ನು ಕುರಿತು ಸೋದರಿ ನಿವೇದಿತಾ:

“ಒಂದು ವಿಚಾರದಲ್ಲಿ ಮಾತ್ರ ಅವರು (ಸ್ವಾಮಿ ವಿವೇಕಾನಂದರು) ಅಚಲರಾಗಿದ್ದರು. ಅದೆಂದರೆ ಭಾರತೀಯ ಸ್ತ್ರೀಯರ ವಿದ್ಯಾಭ್ಯಾಸಕ್ಕಾಗಿ ದುಡಿಯುವುದು. ಅದಕ್ಕೆ ತಮ್ಮ ಹೆಸರನ್ನೂ ಕೊಡಲಡ್ಡಿಯಿಲ್ಲ; ನಾನು ಬಯಸುವ ರೀತಿಯಲ್ಲಿ ಅದನ್ನೊಂದು ಪಂಥವನ್ನಾಗಿಯೂ ಬೆಳೆಸಬಹುದು. “ಒಂದು ಪಂಥದ ಮೂಲಕವೇ ನೀನು ಎಲ್ಲ ಪಂಥಗಳಿಗೂ ಅತೀತಳಾಗಿ ಮೇಲೇರಲು ಬಯಸುವೆ”. \enginline{(CWSN 1:102)}

೬೯. ಪುಟ್ಟ ಗೂಡಿನಂತಹ ಗೋಪಾಲೇರ್ ಮಾಳ ಕೊಠಡಿಗೆ ಸೋದರಿ ನಿವೇದಿತಾ ಭೇಟಿ ಕೊಟ್ಟ ಸಂದರ್ಭದಲ್ಲಿ:

“ಆಹಾ! ಇದು ಪುರಾತನ ಭಾರತ ನೀನು ನೋಡಿ ಬಂದಿರುವ- ಕಣ್ಣೀರುಕೋಡಿಯ, ಪ್ರಾರ್ಥನೆಯ ಭಾರತವನ್ನು, ಉಪವಾಸ - ಜಾಗರಣೆಗಳ ಭಾರತವನ್ನು, ಎಂದಿಗೂ ಹಿಂದಿರುಗಿ ಬಾರದಂತೆ ಹೊರಟು ಹೋಗುತ್ತಿರುವ ಭಾರತವನ್ನು!”. \enginline{(CWSN 1:109)}

೭೦. ರಾಮಕೃಷ್ಣ ಮಹಾಸಂಸ್ಥೆಯ ಧ್ಯೇಯಗಳ ಮೇಲೆ:

ಶ‍್ರೀ ರಾಮಕೃಷ್ಣ ಮಹಾಸಂಸ್ಥೆಯ ಧ್ಯೇಯಗಳ ಅವರ ನಿರೂಪಣೆಯಲ್ಲಿಯೂ ಅದೇ ಉದ್ದೇಶವೇ ಹೊರಹೊಮ್ಮಿತು - “ಪ್ರಾಚ್ಯ ಪಾಶ್ಚಾತ್ಯಗಳ ಆದರ್ಶಗಳನ್ನು ವಿನಿಮಯ ಮಾಡಿಕೊಳ್ಳುವುದು”... \enginline{(CWSN 1:113)}

೭೧. ಸೋದರಿ ನಿವೇದಿತಾಳಿಗೆ ಶಿವಪೂಜೆಯನ್ನು ಹೇಳಿಕೊಟ್ಟು ಆದ ಮೇಲೆ, ಅದರ ಪರಾಕಾಷ್ಠೆ ಎಂಬಂತೆ ಸ್ವಾಮಿ ವಿವೇಕಾನಂದರು ಬುದ್ಧನ ಪಾದಗಳಿಗೆ ಪುಷ್ಪಾರ್ಚನೆ ಮಾಡಿಸಿದರು. ಆ ನಂತರ, ತನ್ನ ಬಳಿಗೆ ಮಾರ್ಗದರ್ಶನಕ್ಕಾಗಿ ಬರಲಿರುವ ಎಲ್ಲ ಜೀವರುಗಳನ್ನೂ ಉದ್ದೇಶಿಸಿರುವಂತೆ ಹೀಗೆಂದು ನುಡಿದರು:

“ಹೋಗಿ, ಅನುಸರಿಸಿ - ಬುದ್ಧದರ್ಶನವಾಗುವುದಕ್ಕಿಂತ ಮುನ್ನವೇ ಐನೂರು ಬಾರಿ ಜನ್ಮತಳೆದು ತನ್ನ ಜೀವನವನ್ನು ಇತರರಿಗಾಗಿ ಕೊಟ್ಟ ಆ ಮಹಾಪುರುಷನನ್ನು!” \enginline{(CWSN 1:114)}

೭೨. ಕಾಶ್ಮೀರದಲ್ಲಿನ ಒಂದು ತೀರ್ಥಯಾತ್ರೆಯಿಂದ ಹಿಂದಿರುಗಿದಾಗ:

“ಈ ದೇವರುಗಳು ಕೇವಲ ಪ್ರತೀಕಗಳಲ್ಲ! ಭಕ್ತರು ದರ್ಶನಮಾಡಿರುವ ರೂಪಗಳು ಅವರು!”. \enginline{(CWSN 1:120)}

೭೩. ತುಂಬ ಹಿಂದೆ ನಾನು ಕೇಳಿದ್ದ ಸ್ವಾಮಿ ವಿವೇಕಾನಂದರ ಮಾತುಗಳನ್ನು ಸೋದರಿ ನಿವೇದಿತಾ ಸ್ಮರಿಸಿಕೊಂಡಿರುವುದು:

“ಇಂದ್ರಿಯ ಸಂವೇದನೆಯ ಮಂಜಿನ ಮೂಲಕ ನೋಡಿದ ನಿರಾಕಾರ ಬ್ರಹ್ಮವೇ ಸಾಕಾರ ದೇವರು”. \enginline{(CWSN 1:120)}

೭೪. ಭಾರತದಲ್ಲಿ ಪಾತಕ ಅಪರೂಪವೆಂಬುದನ್ನು ನೆನಪಿಸಿದಾಗ ಸ್ವಾಮಿ ವಿವೇಕಾನಂದರು ಹೇಳಿದ್ದು:

“ನನ್ನ ದೇಶದಲ್ಲಿ ದೇವರು ಬೇರೆ ವಿಧವಾಗಿರುತ್ತಿತ್ತೆ? - ಆತ! -ಮೃತ್ಯು ದೇವತೆಯ ಒಳ್ಳೆಯ ತನವಲ್ಲದೆ ಇನ್ನೇನು?”. \enginline{(CWSN 1:123)}

೭೫. ಸ್ವಾಮಿ ವಿವೇಕಾನಂದರು ಹೇಳಿದ್ದು:

“ಇಡಿಯ ಜೀವನವೇ ಒಂದು ಹಂಸಗೀತೆ! ಈ ಸಾಲುಗಳನ್ನು ಮರೆಯದಿರಿ:

\begin{myquote}
ಹೃದಯವೇದನೆಯಾದ ಸಿಂಹವು\\ದೊಡ್ಡ ಗರ್ಜನೆಯೀವುದು.\\ತಲೆಯ ಕುಟುಕಿದ ಸರ್ಪ ತನ್ನಯ\\ಹೆಡೆಯನಗಲಿಸಿಬಿಡುವುದು.\\ಆಳ ಗಾಯವ ಪಡೆದ ಮನುಜನು\\ಆತ್ಮದಿವ್ಯತೆ ಮೆರೆವನು”. \enginline{(CWSN 1:124)}
\end{myquote}

೭೬. ನಾಗಮಹಾಶಯರು (ದುರ್ಗಾಚರಣ ನಾಗ್) ಮರಣ ಹೊಂದಿದರೆಂದು ಕೇಳಿದಾಗ:

“ಅವರು ರಾಮಕೃಷ್ಣ ಪರಮಹಂಸರ ಅತ್ಯುತ್ತಮ ಕೃತಿಗಳಲ್ಲಿ ಒಬ್ಬರಾಗಿದ್ದರು.” \enginline{(CWSN 1:129)}

೭೭. ಶ‍್ರೀರಾಮಕೃಷ್ಣರ ಪರಿವರ್ತಕ ಸಾಮರ್ಥ್ಯ ಬಗ್ಗೆ ಸ್ವಾಮಿ ವಿವೇಕಾನಂದರು ಹೇಳಿದ್ದು:

“ಒಬ್ಬರ ಜೀವನವನ್ನು ರಾಮಕೃಷ್ಣ ಪರಮಹಂಸರು ಸ್ಪರ್ಶಿಸುವುದೆಂದರೆ ಅದೇನು ತಮಾಷೆಯೆ? ಅಂತಹ ಕೇವಲ ಅಲ್ಪ ಸಂಸ್ಪರ್ಶನದ ಮೂಲಕವೆ ತಮ್ಮ ಬಳಿಗೆ ಬಂದ ಜನರಿಂದ ಹೊಸ ಸ್ತ್ರೀ ಪುರುಷರನ್ನೇ ಅವರು ಲೀಲಾಜಾಲವಾಗಿ ಸೃಷ್ಟಿಸಿಬಿಡುತ್ತಿದ್ದರು!.” \enginline{(CWSN 1:130)}

೭೮. ನಿಜವಾದ ಸಂನ್ಯಾಸಿಯ ಧೋರಣೆ ಹೇಗಿರಬೇಕು ಎನ್ನುವುದರ ಬಗ್ಗೆ ಮಾತ ನಾಡುತ್ತ ಸ್ವಾಮಿ ವಿವೇಕಾನಂದರು ಹೇಳಿದ್ದು:

“ಹೃಷೀಕೇಶದಲ್ಲಿ ನಾನು ಅನೇಕ ಮಹಾತ್ಮರನ್ನು ನೋಡಿದೆ. ಹುಚ್ಚರಂತೆ ತೋರುತ್ತಿದ್ದ ಒಬ್ಬರು ಜ್ಞಾಪಕವಾಗುತ್ತಾರೆ. ನಗ್ನರಾಗಿ ಬೀದಿಯಲ್ಲಿ ಬರುತ್ತಿದ್ದ ಅವರ ಮೇಲೆ ಹುಡುಗರು ಕಲ್ಲು ತೂರುತ್ತಿದ್ದರು. ಮುಖ, ಕುತ್ತಿಗೆಗಳಿಂದ ರಕ್ತ ಧಾರಾಕಾರವಾಗಿ ಹರಿಯುತ್ತಿದ್ದರೂ ಅವರು ಆನಂದದಿಂದ ಉನ್ಮತ್ತರಾಗಿ ನಗುತ್ತಿದ್ದರು. ನಾನು ಅವರನ್ನು ಕರೆದುಕೊಂಡು ಹೋಗಿ, ಗಾಯವನ್ನು ತೊಳೆದು, ರಕ್ತವನ್ನು ನಿಲ್ಲಿಸಲು ಅವುಗಳ ಮೇಲೆ ವಿಭೂತಿಯನ್ನು ಉದುರಿಸಿದೆ. ಉದ್ದಕ್ಕೂ ನಗುತ್ತಲೇ ಇದ್ದ ಅವರು ಕಲ್ಲು ತೂರುತ್ತಿದ್ದ ಆ ಬಾಲಕರ ಹಾಗೂ ತಮ್ಮ ಆಟದ ತಮಾಷೆ ಆನಂದಗಳನ್ನು ಕುರಿತೇ ಮಾತನಾಡಿದರು. ‘ಆ ದೇವನಾಟವಿದಲ್ಲವೆ!’ ಎಂದರವರು.

“ಇಂಥವರಲ್ಲಿ ಎಷ್ಟೋ ಜನ ಇತರರಿಂದ ತಮಗೆ ತೊಂದರೆಯಾಗಿದಿರಲೆಂದು ಅಡಗಿಕೊಂಡಿರುತ್ತಾರೆ. ಜನರೆಂದರೆ ಅವರಿಗೆ ದೊಡ್ಡ ತೊಂದರೆ. ಒಬ್ಬರಂತೂ ತಮ್ಮ ಗವಿಯ ಮುಂಭಾಗದಲ್ಲಿ ಮಾನವ ಮೂಳೆಯನ್ನು ಚೆಲ್ಲಾಡಿದ್ದರು - ಶವಗಳನ್ನು ತಿಂದು ಬದುಕುವವರು ತಾವು ಎಂಬ ಭ್ರಮೆಯನ್ನು ಮೂಡಿಸಲು. ಇನ್ನೊಬ್ಬರು ತಾವೇ ಕಲ್ಲು ಬೀರುತ್ತಿದ್ದರು. ಹೀಗೇ.....

“ಕೆಲವೊಮ್ಮೆ ಅವರಿಗೆ ಸೆಳಕು ಹೊಡೆದ ಹಾಗೆ ಜ್ಞಾನೋದಯವಾಗಿಬಿಡುತ್ತದೆ. ಉದಾಹರಣೆಗೆ, ಒಬ್ಬ ಹುಡುಗ ಉಪನಿಷತ್ತಿನ ಪಾಠಗಳಿಗಾಗಿ ಅಭೇದಾನಂದರ ಬಳಿಗೆ ಬರುತ್ತಿದ್ದ. ಒಂದು ದಿನ ಅವನು ತಿರುಗಿ ಕೇಳಿದ - ಸ್ವಾಮಿ, ಇದೆಲ್ಲ ನಿಜವೇ?

“ಹೌದಪ್ಪ ಖಂಡಿತವಾಗಿಯೂ ಸತ್ಯ. ಸಾಕ್ಷಾತ್ಕರಿಸಿಕೊಳ್ಳುವುದು ಕಷ್ಟವಾದರೂ, ಎಲ್ಲವೂ ಸತ್ಯ’ ಎಂದರು ಅಭೇದಾನಂದರು.

“ಮಾರನೆಯ ದಿನ, ಆ ಹುಡುಗ ಮೌನಿ ಸಾಧುವಾಗಿಬಿಟ್ಟಿದ್ದ; ನಗ್ನನಾಗಿ ಕೇದಾರನಾಥದ ಕಡೆಗೆ ಹೊರಟಿದ್ದ!

“ಏನಾಯಿತು ಅವನಿಗೆ ಎಂದು ನೀವು ಕೇಳಬಹುದು. ಅವನು ಮೌನಿಯಾಗಿದ್ದ!

“ಆದರೆ, ಸಂನ್ಯಾಸಿಗೆ ಪೂಜೆ, ತೀರ್ಥಯಾತ್ರೆ, ತಪಸ್ಸು ಇವು ಯಾವುದರ ಆವಶ್ಯ ಕತೆಯೂ ಇಲ್ಲ. ಆದರೆ ಈ ಒಂದಾದ ಮೇಲೊಂದರಂತೆ ತೀರ್ಥಯಾತ್ರೆ ಮಾಡುವುದು, ದೇಗುಲದಿಂದ ದೇಗುಲಕ್ಕೆ ಸಂಚರಿಸುವುದು, ಒಂದಾದ ಮೇಲೊಂದರಂತೆ ವ್ರತೋಪವಾಸಾದಿ ತಪಸ್ಸನ್ನು ಕೈಕೊಳ್ಳುವುದು ಇವುಗಳ ಉದ್ದೇಶವೇನು? ಹಾಗೆ ಅವನು ಪುಣ್ಯಸಂಪಾದನೆ ಮಾಡಿ ಅದನ್ನು ಲೋಕಕ್ಕೆ ಧಾರೆಯೆರೆಯುತ್ತಿದ್ದಾನೆ!” \enginline{(CWSN 1:133)}

೭೯. ಶಿಬಿ ರಾಣಾನ ಕಥೆಯ ಬಗ್ಗೆ:

“ಆಹಾ, ಹೌದು! ಇವು ನಮ್ಮ ರಾಷ್ಟ್ರದ ಹೃದಯದಾಳಕ್ಕೇ ಇಳಿಯುವಂಥ ಕಥೆಗಳು! ಸಂನ್ಯಾಸಿಯಾದವನು ಸತ್ಯಸಾಕ್ಷಾತ್ಕಾರ ಹಾಗೂ ಲೋಕಸಂಗ್ರಹ ಎಂಬ ಎರಡನ್ನು ವ್ರತವಾಗಿ ಸ್ವೀಕರಿಸಿರುತ್ತಾನೆಂಬುದನ್ನು ಎಂದಿಗೂ ಮರೆಯದಿರಿ! ಅಲ್ಲದೆ, ಸಂನ್ಯಾಸಿಗೆ ಇರಬೇಕಾದ ಎಲ್ಲಕ್ಕಿಂತ ಕ್ಲಿಷ್ಟವಾದ ಆವಶ್ಯಕತೆ ಎಂದರೆ ಸ್ವರ್ಗದ ಯೋಚನೆಯನ್ನು ಅವನು ತ್ಯಜಿಸಿರಬೇಕು!” \enginline{(CWSN 1:134)}

೮೦. ಸೋದರಿ ನಿವೇದಿತಾಳಿಗೆ ಹೇಳಿದ್ದು:

“ತಾಮಸಿಕ, ರಾಜಸಿಕ ಮತ್ತು ಸಾತ್ವಿಕ ಎಂಬ ಮೂರು ಬಗೆಯ ದಾನಗಳಿವೆ ಎಂದು ಗೀತೆ ಹೇಳುತ್ತದೆ. ಸ್ವಭಾವದ ಯಾವುದೋ ಆವೇಗಕ್ಕೊಳಗಾಗಿ ಮಾಡುವುದು ತಾಮಸಿಕ ದಾನ; ಅದರಿಂದ ಯಾವಾಗಲೂ ತಪ್ಪು ಆಗುತ್ತಲೇ ಇರುತ್ತದೆ. ಜಿಂಕೆಗೆ ತನ್ನ ಸಾಧು ಸ್ವಭಾವವೊಂದನ್ನು ಬಿಟ್ಟು ಇನ್ನೇನೂ ತಿಳಿಯದು. ತನ್ನ ಹಿರಿಮೆಯನ್ನು ಹೆಚ್ಚಿಸಿಕೊಳ್ಳಲು ಮಾಡುವುದು ರಾಜಸಿಕ ದಾನ. ಇನ್ನು ಸತ್ಪಾತ್ರನಿಗೆ, ಸರಿಯಾದ ಸಮಯದಲ್ಲಿ, ಯೋಗ್ಯವಾದ ರೀತಿಯಲ್ಲಿ ಕೊಡುವುದು ಸಾತ್ವಿಕ ದಾನ...

“ಸಾತ್ವಿಕ ದಾನವೆಂದೊಡನೆ ನನಗೆ ಒಬ್ಬರು ಪಾಶ್ಚಾತ್ಯ ಭದ್ರಮಹಿಳೆ ನೆನಪಾಗುತ್ತಾರೆ. ಅವರಲ್ಲಿ ನಾನು ಅಂತಹ ಸರಿಯಾದ ರೀತಿಯ, ಸರಿಯಾದ ಸಮಯದ, ಸತ್ಪಾತ್ರನಿಗೆ ಕೊಡುವ, ಎಂದೂ ಯಾವ ತಪ್ಪೂ ಆಗದ, ಸದ್ದಿಲ್ಲದ ದಾನವನ್ನು ಗಮನಿಸಿದೆ.

“ನನ್ನ ಮಟ್ಟಿಗೆ ಹೇಳುವುದಾದರೆ, ದಾನದಿಂದಲೂ ಕೂಡ ಬೇಕಾದಷ್ಟು ಮುಂದು ವರೆಯಬಹುದು ಎನ್ನುವುದನ್ನು ಕಲಿಯುತ್ತಿದ್ದೇನೆ...

“ವಯಸ್ಸಾಗುತ್ತ ಬಂದಂತೆ, ನಾನು ಸಣ್ಣ ಸಣ್ಣ ಸಂಗತಿಗಳಲ್ಲಿ ದೊಡ್ಡತನವನ್ನು ಕಾಣಲು ಪ್ರಯತ್ನಿಸುತ್ತಿದ್ದೇನೆ. ಮಹಾತ್ಮ ಎಂದೆನಿಸಿಕೊಂಡ ಮನುಷ್ಯ ಏನನ್ನು ಉಂಡುಡುತ್ತಾನೆ, ಸೇವಕರನ್ನು ಹೇಗೆ ನಡೆಸಿಕೊಳ್ಳುತ್ತಾನೆ ಎಂದು ತಿಳಿದುಕೊಳ್ಳಲು ಬಯಸುತ್ತೇನೆ. ಒಬ್ಬ ಸರ್ ಫಿಲಿಪ್ ಸಿಡ್ನಿ\footnote{1. ಸರ್ ಫಿಲಿಪ್ ಸಿಡ್ನಿ (೧೫೫೪-೧೫೮೬): ಇಂಗ್ಲಿಷ್ ಕವಿ, ಯೋಧ, ಹಾಗೂ ರಾಜಕಾರಣಿ.} ಯ ಹಿರಿಮೆಯನ್ನು ನಾನು ಕಾಣಬೇಕು! ಸಾಯುವ ಸಮಯದಲ್ಲಿಯೂ ಸಹ ಇನ್ನೊಬ್ಬರ ಆವಶ್ಯಕತೆಗಳನ್ನು ಪರಿಗಣಿಸದವರೇ ಬಹಳ ಮಂದಿ.

“ಆದರೆ ದೊಡ್ಡ ಸ್ಥಾನದಲ್ಲಿರುವ ಯಾರಾದರೂ ದೊಡ್ಡವರಾಗಿಯೇ ಆಗುತ್ತಾರೆ! ಹೇಡಿಯಾದವನೂ ಸಹ ಪ್ರಖರ ಬೆಳಕಿನಲ್ಲಿ ಧೈರ್ಯಶಾಲಿಯಾಗಿ ಕಾಣಿಸಿಕೊಳ್ಳುತ್ತಾನೆ. ಪ್ರಪಂಚ ನೋಡಿಯೇ ನೋಡುತ್ತದೆ. ಯಾರ ಹೃದಯ ಬಡಿದುಕೊಳ್ಳು ವುದನ್ನೇ ನಿಲ್ಲಿಸುತ್ತದೆ? ತನ್ನಿಂದಾಗುವ ಅತ್ಯುತ್ತಮ ಕಾರ್ಯ ಮಾಡುವವರೆಗೂ ಯಾರ ನಾಡಿ ರಭಸಗೊಳ್ಳುವುದಿಲ್ಲ?

“ಹುಳುವೊಂದು ತನ್ನ ಕರ್ತವ್ಯವನ್ನು ಮೌನವಾಗಿ, ಬಿಟ್ಟೂಬಿಡದೆ, ಕ್ಷಣದಿಂದ ಕ್ಷಣಕ್ಕೆ, ಘಳಿಗೆಯಿಂದ ಘಳಿಗೆಗೆ ಮಾಡುತ್ತ ಹೋಗುವುದಿದೆಯಲ್ಲ, ಅದು ನನಗೆ ಹೆಚ್ಚುಗಾರಿಕೆ ಎನ್ನಿಸುತ್ತದೆ.” \enginline{(CWSN 1:137)}

೮೧. ಮಹಾಪುರುಷ ಯಾರು - ಅವತಾರನೆ, ಗುರುವೆ, ಋಷಿಯೆ - ಎಂಬ ಬಗ್ಗೆ:

“ಭಾರತ ನಿಮಗೆ ಅರ್ಥವಾಗದು! ಕಂಡಂತೆಯೇ ಕಂಡಂತೆಯೇ ನಾವು ಭಾರತೀಯರು ಮಾನವ ಆರಾಧಕರು! ಮಾನವನೇ ನಮ್ಮ ದೇವರು!.” \enginline{(CWSN 1:144)}

೮೨. ಇನ್ನೊಂದು ಸಂದರ್ಭದಲ್ಲಿ ಸ್ವಾಮಿ ವಿವೇಕಾನಂದರು “ಮಾನವ ಆರಾಧಕರು” ಎನ್ನುವುದನ್ನು ಬೇರೆಯೇ ಅರ್ಥದಲ್ಲಿ ಬಳಸಿದರು:

“ಮಾನವ ಆರಾಧನೆ ಎಂಬುದು ಭಾರತದಲ್ಲಿ ಬೀಜರೂಪದಲ್ಲಿದೆ; ಅದು ಯಾವಾಗಲೂ ವಿಕಾಸವಾದುದಿಲ್ಲ. ನೀವು ಅದನ್ನು ಅಭಿವೃದ್ಧಿಪಡಿಸಬೇಕು. ಅದನ್ನೇ ಕಲೆಯನ್ನಾಗಿಸಿ, ಅದರ ಮೇಲೆ ಕಾವ್ಯವನ್ನು ಬರೆಯಿರಿ. ಮಧ್ಯಯುಗೀನ ಯೂರೋಪಿನಲ್ಲಿದ್ದ ಹಾಗೆ ತಿರುಕರ ಪಾದಗಳ ಆರಾಧನೆಯನ್ನು ಸ್ಥಾಪಿಸಿರಿ. ಮಾನವ ಆರಾಧಕರಾಗಿರಿ.” (CWSN 1:144-145)

೮೩. ಸೋದರಿ ನಿವೇದಿತಾಳಿಗೆ ಹೇಳಿದ್ದು:

“ಮಹಮ್ಮದೀಯರಲ್ಲಿ ಒಂದು ವಿಚಿತ್ರ ಪಂಥದವರಿದ್ದಾರೆ. ಅವರೆಷ್ಟು ಧರ್ಮಾಂಧರೆಂದರೆ, ಆಗತಾನೆ ಹುಟ್ಟಿದ ಶಿಶುವನ್ನು ಅಂತರಿಕ್ಷಕ್ಕೆ ತೋರಿಸಿ ‘ದೇವರು ನಿನ್ನನ್ನು ಸೃಷ್ಟಿಸಿದ್ದರೆ ಸಾಯಿ! ಆಲಿ ನಿನ್ನನ್ನು ಸೃಷ್ಟಿಸಿದ್ದರೆ ಬದುಕು!’ ಎನ್ನವರಂತೆ. ಮಗುವಿಗೆ ಅವರು ಹೇಳುವ ಇದನ್ನೇ ನಾನು ನಿನಗೆ ಈ ರಾತ್ರಿ ವಿರುದ್ಧಾರ್ಥದಲ್ಲಿ ಹೇಳುವೆ: ನೀನು ನನ್ನಿಂದ ತಯಾರಾದವಳಾಗಿದ್ದರೆ, ನಾಶವಾಗು! ತಾಯಿ ನಿನ್ನನ್ನು ತಯಾರು ಮಾಡಿದ್ದರೆ, ಬದುಕು!” \enginline{(CWSN 1:151)}

೮೪. ನೀಗ್ರೋ ಎಂಬ ತಪ್ಪು ತಿಳಿವಳಿಕೆಯಿಂದ ಹೋಟೆಲುಗಳಿಗೆ ಪ್ರವೇಶ ನಿರಾಕರಿ ಸಿದ್ದೆವು ಎಂದು ದಕ್ಷಿಣದ ಪ್ರತಿಷ್ಠಿತ ವ್ಯಕ್ತಿಗಳು ಸ್ವಾಮಿ ವಿವೇಕಾನಂದರನ್ನು ಕ್ಷಮಾಪಣೆ ಕೇಳಿಕೊಂಡ ಎಷ್ಟೋ ಕಾಲದ ನಂತರ ಸ್ವಾಮಿಗಳು ತಮ್ಮ ಬಗ್ಗೆ ತಾವೇ ಹೀಗೆಂದರು:

“ಏನು! ಇನ್ನೊಬ್ಬರ ಕ್ಷಮತೆಯ ಮೇಲೆ ಮೇಲೇರುವುದೆ! ಅದಕ್ಕಾಗಿ ನಾನು ಈ ಭೂಮಿಗೆ ಬಂದಿಲ್ಲ!.... ಗೌರವರ್ಣದ ಆರ್ಯನ್ ಪಿತೃಗಳಿಗೆ ನಾನು ಋಣಿಯಾಗಿರುವುದಾದರೆ, ಪೀತವರ್ಣದ ಮಂಗೋಲಿಯನ್ ಪಿತೃವಿಗೆ ಇನ್ನೂ ಋಣಿಯಾಗಿರುವೆ, ಎಲ್ಲಕ್ಕಿಂತ ಹೆಚ್ಚಾಗಿ ಕಪ್ಪುಬಣ್ಣದ ನೀಗ್ರೋ ಪಿತೃವಿಗೆ ಋಣಿಯಾಗಿರುವೆ!” \enginline{(CWSN 1:153)}

೮೫. ಸೈಂಟ್ ಮೈಕೆಲ್ ಪರ್ವತದ ಮೇಲಿದ್ದ ಮಧ್ಯಯುಗೀನ ಖೈದಿಗಳನ್ನಿಡುತ್ತಿದ್ದ ನೆಲ ಮಾಳಿಗೆಯ ಬಂದೀಖಾನೆಯನ್ನು ನೋಡಿದಾಗ:

“ಆಹಾ! ಧ್ಯಾನ ಮಾಡುವುದಕ್ಕೆ ಎಂಥ ಅಚ್ಚರಿಯ ತಾಣ!” \enginline{(CWSN 1:154)}

೮೬. ರೋಮನ್ ಚಕ್ರಾಧಿಪತ್ಯದ ಪಳೆಯುಳಿಕೆ ರೂಕ್ಷವೆಂದೂ ಜಪಾನಿನ ವಿವಾಹ ಕಲ್ಪನೆ ಭಯಾನಕ ಎಂದೂ ತಮ್ಮ ಅಭಿಪ್ರಾಯವಾಗಿದ್ದರೂ ಸಹ, ಸ್ವಾಮಿ ವಿವೇಕಾ ನಂದರು ಒಂದು ಜನಾಂಗದ ದೋಷಗಳನ್ನು ಪರಿಗಣಿಸದೆ ವಿಧಾಯಕ ಆದರ್ಶಗಳನ್ನೇ ಎತ್ತಿ ಹಿಡಿಯುತ್ತಿದ್ದರು:

“ರಾಷ್ಟ್ರಭಕ್ತಿಗೆ ಜಪಾನೀಯರು! ಶುದ್ಧತೆಗೆ ಹಿಂದೂಗಳು! ಪೌರುಷಕ್ಕೆ ಯೂರೋಪಿ ಯನ್ನರು! ಅರ್ಥಮಾಡಿಕೊಳ್ಳುವುದರಲ್ಲಿ ಇಂಗ್ಲಿಷರನ್ನು ಬಿಟ್ಟರೆ ಲೋಕದಲ್ಲಿ ಇನ್ನು ಯಾರೂ ಇಲ್ಲ! ಎಂದಾದ ಮೇಲೆ, ಒಬ್ಬ ಮಾನವನ ಹಿರಿಮೆಯೆಷ್ಟಿರಬೇಕು!” \enginline{(CWSN 1:160)}

೮೭. ೧೮೯೩ರಲ್ಲಿ ಅಮೆರಿಕಾಕ್ಕೆ ಹೊರಡುವುದಕ್ಕೆ ಮುಂಚೆ ಸ್ವಾಮಿ ವಿವೇಕಾನಂದರು ತಮ್ಮ ಬಗ್ಗೆ ತಾವು ಹೇಳಿಕೊಂಡಿರುವುದು:

“ನಾನು ಬೋಧಿಸಹೊರಟಿರುವ ಧರ್ಮದ ಹಿನ್ನೆಲೆಯಲ್ಲಿ ಬೌದ್ಧಧರ್ಮವು ಸೆಡ್ಡು ಹೊಡೆದುನಿಂತ ಒಂದು ಮಗು, ಹಾಗೂ ಕ್ರೈಸ್ತಧರ್ಮವು ತನ್ನೆಲ್ಲ ಆಡಂಬರದ ನಡುವೆಯೂ ದೂರದ ಒಂದು ಪ್ರತಿಧ್ವನಿ!” \enginline{(CWSN 1:161)}

೮೮. ಪ್ರಪಂಚವನ್ನು ತ್ಯಾಗಮಾಡುವುದಕ್ಕೆಂದು ಬುದ್ಧನು ಹೆಂಡತಿಯನ್ನು ಬಿಟ್ಟು ಹೊರಟ ರಾತ್ರಿಯನ್ನು ವರ್ಣಿಸುತ್ತ ಸ್ವಾಮಿ ವಿವೇಕಾನಂದರೆಂದರು:

“ಅವನ ಮನಸ್ಸನ್ನು ಪೀಡಿಸುತ್ತಿದ್ದ ಸಮಸ್ಯೆಯಾದರೂ ಏನು? ಅದೇಕೆ, ಲೋಕಕ್ಕಾಗಿ ಅವನು ತ್ಯಾಗಮಾಡುತ್ತಿದ್ದ ಅವಳು! ಅವಳಿಗಾಗಿಯೇ ಅವನ ಮನಸ್ಸಿನ ಹೋರಾಟ ವಿದ್ದದ್ದು! ತನಗಾಗಿ ಅವನು ಯಾವುದನ್ನೂ ಲೆಕ್ಕಿಸುತ್ತಿರಲಿಲ್ಲ!” \enginline{(CWSN 1:172)}

೮೯. ಬುದ್ಧನು ತನ್ನ ಪತ್ನಿಗೆ ಮನಮುಟ್ಟವ ಹಾಗೆ ವಿದಾಯ ಹೇಳಿದ್ದನ್ನು ವರ್ಣಿಸಿಯಾದ ಮೇಲೆ ಸ್ವಾಮಿಗಳೆಂದರು:

“ಲೋಕೈಕವೀರರ ಹೃದಯಗಳನ್ನು ಕುರಿತು ನೀವೆಂದೂ ಚಿಂತಿಸಿಲ್ಲವೆ? ಎಂಥ ಮಹತ್ವದ, ಔನ್ನತ್ಯದ ಹೃದಯ ಅವರದು, ಅದ್ಭುತ - ಅಲ್ಲದೆ ಬೆಣ್ಣೆಯಂತೆ ಮೃದು!” \enginline{(CWSN 1:172)}

೯೦. ಸ್ವಾಮಿ ವಿವೇಕಾನಂದರು ಮಾಡಿದ ಬುದ್ಧನ ಸಾವಿನ ವರ್ಣನೆ ಮತ್ತು ಅದಕ್ಕೂ ಶ‍್ರೀರಾಮಕೃಷ್ಣರ ಸಾವಿಗೂ ಇರುವ ಹೋಲಿಕೆ:

ವೃಕ್ಷವೊಂದರ ಕೆಳಗೆ ಹೇಗೆ ಕಂಬಳಿಯೊಂದನ್ನು ಹಾಸಿದ್ದರು, ಅದರ ಮೇಲೆ ಹೇಗೆ ಬುದ್ಧನು “ಸಿಂಹದ ಹಾಗೆ, ಬಲ ಮಗ್ಗುಲಾಗಿ” ಮಲಗಿದ್ದನು, ಎಂಬುದನ್ನು ಅವರು ವಿವರಿಸಿದರು. ಆಗ ಇದ್ದಕ್ಕಿದ್ದಂತೆ ಯಾರೋ ಒಬ್ಬನು ಉಪದೇಶವನ್ನು ಪಡೆಯುವುದಕ್ಕಾಗಿ ಬಂದನು. ಏನೇ ಆಗಲಿ, ತಮ್ಮ ಗುರುವಿನ ಮೃತ್ಯುಶಯ್ಯೆಯ ಬಳಿ ಶಾಂತಿಯನ್ನು ಕಾಯ್ದುಕೊಳ್ಳಬೇಕು ಎಂದು ಪರಿತಪಿಸುತ್ತಿದ್ದಾಗ ಹೀಗೆ ಕರೆಯದೆ ಬಂದ ಅನಾಹೂತನನ್ನು ಶಿಷ್ಯರುಗಳು ಓಡಿಸುವುದರಲ್ಲಿದ್ದರು; ಆದರೆ ಬಂದವನ ಮಾತನ್ನು ಕೇಳಿಸಿಕೊಂಡಿದ್ದ ಬುದ್ಧನು “ಇಲ್ಲ, ಇಲ್ಲ! ಯಾರು ಕಳುಹಿಸಲ್ಪಟ್ಟಿರುವನೋ\footnote{1. ಮೂಲ ಪದ “ತಥಾಗತ” ಎಂದು. ಸ್ವಾಮಿ ವಿವೇಕಾನಂದರು, “ನಿಮ್ಮ ಮೆಸೈಯಾ’ \enginline{(Messiah)} ಪದಕ್ಕೆ ಸರಿಸಾಟಿಯಾಗುವಂಥದು” ಎಂದು ವಿವರಿಸಿದರು.} ಅವನು ಸರ್ವದಾ ಸಿದ್ಧನಾಗಿರುವನು” ಎಂದೆ ನ್ನುತ್ತ ಮೊಳಕೈ ಮೇಲೆ ಎದ್ದು ಕುಳಿತನು; ಬಂದವನಿಗೆ ಉಪದೇಶವನ್ನು ನೀಡಿದನು. ಇದು ನಾಲ್ಕು ಬಾರಿ ಪುನರಾವರ್ತನೆಗೊಂಡಿತು; ಆ ಮೇಲೆಯೆ ಬುದ್ಧನು ತಾನು ಸಾಯಲು ಸ್ವತಂತ್ರನೆಂಬುದಾಗಿ ಭಾವಿಸಿದನು. “ಆದರೆ ಅಳುತ್ತಿರುವುದಕ್ಕಾಗಿ ಆನಂದನನ್ನು ಮೊದಲು ಗದರಿಸಿನು. ಬುದ್ಧನೆಂದರೆ ಒಬ್ಬ ವ್ಯಕ್ತಿಯಲ್ಲ, ಆದರೆ ಸಾಕ್ಷಾತ್ಕಾರದ ಒಂದು ಸ್ಥಿತಿ; ಯಾರೊ ಬ್ಬರು ಬೇಕಾದರೂ ಆ ಸ್ಥಿತಿಯನ್ನು ತಲುಪಬಹುದು ಎಂದನು. ಕೊನೆಯುಸಿರನ್ನು ಎಳೆಯುತ್ತಿರುವಂತೆಯೇ, ಯಾರನ್ನೂ ಪೂಜಿಸಬೇಡಿ ಎದ್ದು ಅವರನ್ನು ನಿಷೇಧಿಸಿದನು.”

ನಿತ್ಯನೂತನ ಕಥೆ ಮುಂದುವರೆದು ಕೊನೆಗೊಂಡಿತು. ಆದರೆ ಕೇಳುತ್ತಿದ್ದವರಿಗೆ ಅತ್ಯಂತ ಮಹತ್ವದ ಕ್ಷಣವೆಂದೆನಿಸಿದ್ದು, ಕಥೆ ಹೇಳುತ್ತಿದ್ದವರು “ಮೊಳಕೈ ಮೇಲೆ ಎಂದು ಕುಳಿತನು; ಬಂದವನಿಗೆ ಉಪದೇಶವನ್ನು ನೀಡಿದನು” ಎನ್ನುವಾಗ ತಡೆಹಿಡಿದು - ಅಲ್ಪ ಅಧ್ಯಾಹಾರವೆಂಬಂತೆ -

“ನೋಡಿ, ರಾಮಕೃಷ್ಣ ಪರಮಹಂಸರ ವಿಚಾರದಲ್ಲಿ ನಾನೇ ಇದಕ್ಕೆ ಪ್ರತ್ಯಕ್ಷದರ್ಶಿ” ಎಂದಾಗ, ಕೇಳುತ್ತಿದ್ದವರ ಬಗೆಗಣ್ಣ ಮುಂದೆ ಬಂದದ್ದು, ನೂರಾರು ಮೈಲಿ ಪ್ರಯಾಣ ಮಾಡಿ ಕಾಶೀ ಪುರಕ್ಕೆ ಸಾವಿನ ಸಮ್ಮುಖಕ್ಕೇ ಬಂದು ಸೇರಿದ ಆ ಮನುಷ್ಯ; ಅವರಿಂದ ಉಪದೇಶ ಪಡೆ ಯುವ ಅದೃಷ್ಟವಿದ್ದ ಆ ಮನುಷ್ಯ. ಇಲ್ಲಿಯೂ ಸಹ ಶಿಷ್ಯರುಗಳು ಅವನನ್ನು ಹತ್ತಿರಕ್ಕೆ ಸೇರಿಸು ತ್ತಿರಲಿಲ್ಲ; ಆದರೆ ಶ‍್ರೀರಾಮಕೃಷ್ಣರೇ ಮಧ್ಯೆ ಪ್ರವೇಶಿಸಿ, ಬಂದವನನ್ನು ಹತ್ತಿರಕ್ಕೆ ಕರೆದರು; ಅವನಿಗೆ ಉಪದೇಶವನ್ನು ನೀಡಿದರು. \enginline{(CWSN 1:175-176)}

೯೧. ಬೌದ್ಧ ಸಿದ್ಧಾಂತದ ತಾತ್ತ್ವಿಕ ಹಾಗೂ ಚಾರಿತ್ರಿಕ ಮಹತ್ವದ ಬಗ್ಗೆ ಟೀಕೆ ಮಾಡುತ್ತ:

“ರೂಪ, ಭಾವನೆ, ಸಂವೇದನೆ, ಚಲನೆ ಮತ್ತು ಜ್ಞಾನ - ಇವು ನಿರಂತರ ಪ್ರವಹನ ಹಾಗೂ ಸಮ್ಮಿಲನದಲ್ಲಿರುವ ಐದು ಸಂಗತಿಗಳು. ಮಾಯೆ ಇರುವುದು ಇವುಗಳಲ್ಲಿಯೇ. ಯಾವು ದಾದರೊಂದು ತರಂಗದಿಂದ ಏನನ್ನೂ ಸ್ಥಾಪಿಸಲಾಗದು, ಏಕೆಂದರೆ ಅದು ಅಸ್ತಿತ್ವದಲ್ಲಿ ಇರದು. ಇತ್ತು, ಆದರೆ ಈಗ ಹೋಯಿತು ಎಂದಾಗುತ್ತದೆ. ಹೇ ಮಾನವ, ನೀನೇ ಸಾಗರವೆಂದು ಅರಿತುಕೋ! ಆಹ್, ಇದು ಕಪಿಲನ ಸಿದ್ಧಾಂತ, ಆದರೆ ಆತನ ಶಿಷ್ಯೋತ್ತಮನಿಗೆ (ಬುದ್ಧನಿಗೆ) ಅದನ್ನು ಜೀವಂತವನ್ನಾಗಿ ಮಾಡುವಂತಹ ಹೃದಯವಿತ್ತು!” \enginline{(CWSN 1:176)}

೯೨. ಬೌದ್ಧರ ಮೊಟ್ಟಮೊದಲ ಸಮಾಲೋಚನ ಸಭೆ ಹಾಗೂ ಅದರ ಅಧ್ಯಕ್ಷರ ಆಯ್ಕೆಯಲ್ಲಿನ ವಿವಾದದ ಬಗ್ಗೆ:

“ಅವರ ಶಕ್ತಿ ಯಾವುದಾಗಿತ್ತೆಂದು ನೀವು ಕಲ್ಪಿಸಿಕೊಳ್ಳಬಲ್ಲಿರಾ? ಆನಂದನಾಗಿರಬೇಕು ಎಂದು ಒಬ್ಬರು ಹೇಳುತ್ತಾನೆ, ಏಕೆಂದರೆ ಅವನಿಗೆ ಆನಂದನನ್ನು ಕಂಡರೆ ಪರಮ ಪ್ರೀತಿ. ಇನ್ನೊಬ್ಬನು ಮುಂದೆ ಬಂದು ಕೂಡದು! ಏಕೆಂದರೆ ಆನಂದನು ಮೃತ್ಯು ಶಯ್ಯೆಯ ಬಳಿ ಅತ್ತಿರು ವನು ಎನ್ನುತ್ತಾನೆ. ಹೀಗೆ ಆನಂದನನ್ನು ಹಿಂದೆ ಹಾಕುತ್ತಾರೆ!” \enginline{(CWSN 1:177)}

೯೩. ಮರುಹುಟ್ಟು ಪಡೆಯುವುದನ್ನು ಒಂದು ನಂಬಿಕೆಯ ವಿಚಾರ ಎನ್ನುವುದರ ಬದಲು “ವೈಜ್ಞಾನಿಕ ಊಹನೆ” ಎಂದು ಪರಿಗಣಿಸಬಾರದೇಕೆ ಎನ್ನುವ ಬಗ್ಗೆ:

“ಏಕೆ, ದೇಹದಲ್ಲಿ ಕಳೆಯುವುದನ್ನು ಒಂದು ಜೀವನವೇ ಮಿಲಿಯ ವರ್ಷಗಳ ಸೆರೆವಾಸದ ಹಾಗಿರುವಾಗ, ಅವರು ಅನೇಕ ಜೀವನಗಳ ನೆನಪನ್ನು ಜಾಗ್ರತಗೊಳಿಸ ಬಯಸು ವರಲ್ಲ! ಈ ಹೊತ್ತಿನವರೆಗೂ ಅದರಲ್ಲಿನ ಕೇಡನ್ನು ಅನುಭವಿಸಿದ್ದೇ ಸಾಕು!.... ಹೌದು! ಬೌದ್ಧಧರ್ಮವೇ ಸರಿ! ಪುವರ್ಜನ್ಮವೆನ್ನುವುದು ಕೇವಲ ಒಂದು ಮರೀಚಿಕೆ! ಆದರೆ ಈ ದರ್ಶನವನ್ನು ತಲುಪಬೇಕಾದದ್ದು ಅದ್ವೈತಮಾರ್ಗದ ಮೂಲಕ ಮಾತ್ರವೇ!” \enginline{(CWSN 1:180-181)}

೯೪. ನಜರೇತ್ನ ಜೀಸಸ್ನ ದಿನಗಳಲ್ಲಿ ನಾನೇನಾದರೂ ಪ್ಯಾಲಸ್ಟೈನ್ನಲ್ಲಿ ಇದ್ದಿದ್ದರೆ, ಆತನ ಪಾದಗಳನ್ನು ನಾನು ತೊಳೆಯುತ್ತಿದ್ದೆ - ನನ್ನ ಕಣ್ಣೀರಿನಿಂದಲ್ಲ, ನನ್ನ ಹೃದಯದ ರಕ್ತದಿಂದ!” \enginline{(CWSN 1:189)}

೯೫. “ಆದ್ದರಿಂದ ಅದ್ವೈತಿಯಾದವನಿಗೆ ಏಕಮಾತ್ರ ಧ್ಯೇಯವೆಂದರೆ ಪ್ರೇಮವೇ.... ಆನಂದದಿಂದ ತನ್ನ ಹಾದಿಯನ್ನು ಕ್ರಮಿಸಬೇಕಾದವನು ಮುಕ್ತಿ ಪ್ರದಾಯಕನೇ ಹೊರತು ಮುಕ್ತನಾದವನಲ್ಲ!” (CWSN 1:197-198)

೯೬. ಶಿಷ್ಯನೊಬ್ಬನ ಜೀವನದಲ್ಲಿ ಸಂಯಮದ ಆವಶ್ಯಕತೆಯನ್ನು ಕುರಿತು:

ಭಾವುಕತೆಯ ಲವಲೇಶವೂ ಇಲ್ಲದಂತೆ ಆತ್ಮಸಾಕ್ಷಾತ್ಕಾರ ಮಾಡಿಕೊಳ್ಳಲು ಹೋರಾಟ ನಡೆಸು!... ಎಲೆಗಳುದುರುವ ದೃಶ್ಯವನ್ನು ವೀಕ್ಷಿಸು, ಆದರೆ ದೃಶ್ಯದ ಭಾವವನ್ನು ಮುಂದೆಂದಾದರೂ ಅಂತರಂಗದಿಂದ ಹೊರತಂದುಕೊ!” \enginline{(CWSN 1:20-208)}

“ಎಚ್ಚರ! ಬ್ರೆಡ್ಡಾಗಲಿ ಮೀನಾಗಲಿ ಕೂಡದು! ಪ್ರಪಂಚದ ಯಾವುದೇ ಥಳುಕಾಗಲಿ ಕೂಡದು! ಇವೆಲ್ಲವನ್ನೂ ಬಿಟ್ಟುಬಿಡಬೇಕು. ಬೇರುಸಮೇತ ಕಿತ್ತೊಗೆಯಬೇಕು. ಅದು ಭಾವುಕತೆ - ಇಂದ್ರಿಯಸಂವೇದನೆಯ ಉಕ್ಕಿಹರಿಯುವಿಕೆ. ಅದು ವರ್ಣಗಳಲ್ಲಿ, ದೃಶ್ಯವಾಗಿ, ಶ್ರಾವ್ಯವಾಗಿ, ಸಂಯೋಜನೆಯಾಗಿ ನಿನ್ನೆಡೆಗೆ ಬರುತ್ತದೆ. ಕತ್ತರಿಸಿ ಹಾಕು ಅದನ್ನು, ಅದನ್ನು ದ್ವೇಷಿಸುವುದನ್ನು ಕಲಿ. ಅದು ತೀಕ್ಷ್ಣ ವಿಷ!” (ಅದೇ, 207-208)

೯೭. ವಿವಿಧ ನಮೂನೆಗಳ ಮೌಲಿಕತೆಯ ಬಗ್ಗೆ:

“ಎರಡು ಬೇರೆ ಬೇರೆ ಬುಡಕಟ್ಟುಗಳು ಬೆರೆತು ಒಂದಾಗುತ್ತವೆ, ಅವುಗಳಿಂದ ಒಂದು ಶಕ್ತಿಯುತವಾದ ವಿಶಿಷ್ಟವಾದ ನಮೂನೆ ಹೊರಹೊಮ್ಮುತ್ತದೆ. ಶಕ್ತಿಯುತವಾದ, ವಿಶಿಷ್ಟವಾದೊಂದು ನಮೂನೆಯೇ ಕ್ಷಿತಿಜದ ಭೌತಿಕ ತಳಹದಿ. ವಿಶ್ವವೆಲ್ಲವೂ ಒಂದೆಂಬ ಭಾವೈಕ್ಯದ ಮಾತನಾಡುವುದಕ್ಕೆ ಚೆನ್ನಾಗಿರುತ್ತದೆ; ಆದರೆ ಇನ್ನೂ ಮಿಲಿಯ ವರ್ಷಗಳು ಕಳೆದರೂ ಲೋಕವು ಅದಕ್ಕೆ ಸಿದ್ಧವಾಗಿರುವುದಿಲ್ಲ!

ನೆನಪಿಡಿ! ಹಡಗೊಂದು ಹೇಗಿರುತ್ತದೆ ಎಂದು ನೀವು ತಿಳಿಯಬೇಕಾದರೆ, ಹಡಗು ಇರುವ ಹಾಗೆಯೇ ಅದರ ವೈಶಿಷ್ಟ್ಯಗಳನ್ನು ಅರಿತುಕೊಳ್ಳಬೇಕು - ಉದ್ದ, ಅಗಲ, ಆಕಾರ, ಯಾವ ವಸ್ತುವಿನಿಂದ ಕಟ್ಟಲ್ಪಟ್ಟಿದೆ ಇತ್ಯಾದಿ. ಒಂದು ರಾಷ್ಟ್ರವನ್ನು ಅರ್ಥಮಾಡಿಕೊಳ್ಳುವುದಕ್ಕೂ ಸಹ ನಾವು ಅದನ್ನೇ ಮಾಡಬೇಕು. ಭಾರತವು ವಿಗ್ರಹಾರಾಧಕ ದೇಶವಾಗಿದೆ. ಅದಕ್ಕೆ ನೀವು ಅದು ಇರುವ ಹಾಗೆಯೇ ಸಹಾಯ ಮಾಡಬೇಕು. ಅದನ್ನು ಬಿಟ್ಟು ಹೋದವರು ಅದಕ್ಕೆ ಏನನ್ನೂ ಮಾಡಲಾರರು!” \enginline{(CWSN 1:209)}

೯೮. ವಿದ್ಯಾರ್ಥಿಜೀವನದಲ್ಲಿ ಬ್ರಹ್ಮಚರ್ಯದ ಭಾರತೀಯ ಆದರ್ಶ ಹೇಗಿರಬೇಕು ಎಂಬುದನ್ನು ವಿವರಿಸುತ್ತ ಸ್ವಾಮಿ ವಿವೇಕಾನಂದರು:

“ಬ್ರಹ್ಮಚರ್ಯವೆಂಬುದು ನಿಮ್ಮ ರಕ್ತನಾಳಗಳಲ್ಲಿ ಜ್ವಲಿಸುತ್ತಿರುವ ಬೆಂಕಿಯಂತಿರಬೇಕು!” \enginline{(CWSN 1:216)}

೯೯. ಆಯ್ಕೆಯ ವಿವಾಹಕ್ಕೆ ಬದಲಾಗಿ ಏರ್ಪಡಿಸಲ್ಪಟ್ಟ ವಿವಾಹದ ಬಗ್ಗೆ ಸ್ವಾಮಿ ವಿವೇಕಾನಂದರು ಇಂತೆಂದರು:

“ಅದೆಂತಹ ವೇದನೆ ತುಂಬಿದೆ ಈ ದೇಶದಲ್ಲಿ! ಎಷ್ಟೊಂದು ನೋವು! ಸ್ಪಲ್ಪಮಟ್ಟಿಗೆ ಯಾವಾಗಲೂ ಇರುವಂಥದೇ ಎನ್ನುವುದೇನೋ ನಿಜ. ಆದರೆ ಯೂರೋಪಿಯನ್ನ ರನ್ನೂ ಅವರ ವೈವಿಧ್ಯಮಯ ಪದ್ಧತಿಗಳನ್ನೂ ನೋಡಿದ ಮೇಲೆ ಅದು ಹೆಚ್ಚಿದೆ. ಇನ್ನೊಂದು ಮಾರ್ಗವೂ ಇದೆ ಎನ್ನುವುದು ಸಮಾಜಕ್ಕೆ ಗೊತ್ತಾಗಿದೆ!

(ಯೂರೋಪಿಯನ್ನರೊಬ್ಬರಿಗೆ) “ನಾವು ತಾಯ್ತನವನ್ನು ಔನ್ನತ್ಯಕ್ಕೇರಿಸಿದ್ದೇವೆ ಮತ್ತು ನೀವು ಪತ್ನೀತ್ವವನ್ನು ಬೆಳೆಸಿರುವಿರಿ; ಪರಸ್ಪರ ಸ್ವಲ್ಪ ಕೊಟ್ಟುತೆಗೆದುಕೊಳ್ಳುವುದರಿಂದ ಇಬ್ಬರಿಗೂ ಲಾಭವಾಗಬಹುದೆಂದು ನನಗನ್ನಿಸುತ್ತದೆ.

“ಭಾರತದಲ್ಲಿ ಪತ್ನಿಯಾದವಳು ತನ್ನ ಮಗನನ್ನು ಸಹ ಪತಿಯನ್ನು ಪ್ರೀತಿಸುವಷ್ಟು ಪ್ರೀತಿಸುವುದನ್ನು ಕಲ್ಪಿಸಿಕೊಳ್ಳಲೂ ಆರಳು. ಅವಳು ಸತಿ ಎನ್ನಿಸಿಕೊಂಡಿರುವವಳು. ಆದರೆ ಗಂಡನಾದವನು ತನ್ನ ತಾಯಿಯನ್ನು ಪ್ರೀತಿಸುವಷ್ಟು ಅವಳನ್ನು ಪ್ರೀತಿಸಬೇಕಾದದ್ದಿಲ್ಲ. ಆದ್ದರಿಂದ ಪರಸ್ಪರ ಪ್ರೇಮವನ್ನು ಅಹೈತುಕ ಪ್ರೇಮದಷ್ಟು ಉನ್ನತವೆಂದು ಭಾವಿಸುವುದಿಲ್ಲ. ಅದು ‘ಅಂಗಡಿ ಇಟ್ಟಂತೆ’. ಗಂಡಹೆಂಡಿರ ಸ್ಪರ್ಶದ ಸೌಖ್ಯಕ್ಕೆ ಭಾರತದಲ್ಲಿ ಅವಕಾಶವಿಲ್ಲ. ಇದನ್ನು ನಾವು ಪಾಶ್ಚಿಮಾತ್ಯದಿಂದ ತೆಗೆದುಕೊಳ್ಳ ಬೇಕಾಗಿದೆ. ನಿಮ್ಮಿಂದ ನಾವು ನಮ್ಮ ಆದರ್ಶವನ್ನು ನವೀಕರಿಸಿಕೊಳ್ಳಬೇಕಾದ ಆವಶ್ಯಕತೆ ಯಿದೆ. ಅಂತೆಯೇ, ನಿಮಗೂ ಸಹ ನಮ್ಮಲ್ಲಿ ತಾಯ್ತನಕ್ಕಿರುವ ಭಕ್ತಿಯನ್ನು ಸ್ವಲ್ಪ ತೆಗೆದುಕೊಳ್ಳ ಬೇಕಾದ ಆವಶ್ಯಕತೆಯಿದೆ. \enginline{(CWSN 1:221-222)}

೧೦೦. ದಯಾರ್ದ್ರಹೃದಯರಾಗಿ ಶಿಷ್ಯನೊಬ್ಬನಿಗೆ ಹೇಳಿದ್ದು:

“ಈ ಮನೆ, ಮದುವೆಗಳ ಯೋಚನೆಯ ನೆರಳು ಒಮ್ಮೊಮ್ಮೆ ಮನಸ್ಸಿನಲ್ಲಿ ಹಾದು ಹೋದರೆ ನೀನು ಚಿಂತಿಸುವ ಅಗತ್ಯವಿಲ್ಲ. ನನಗೂ ಸಹ ಅವು ಆಗಿಂದಾಗ್ಯೆ ಬರುವುದುಂಟು!” \enginline{(CWSN 1:222)}

೧೦೧. ಸ್ನೇಹಿತನೊಬ್ಬ ತೀರ ಏಕಾಂಗಿಯಾಗಿರುವನು ಎಂದು ಕೇಳಿದಾಗ:

“ಪ್ರತಿಯೊಬ್ಬ ಕಾರ್ಯನಿರತನಿಗೂ ಇಂತಹ ಭಾವ ಒಮ್ಮೊಮ್ಮೆ ಬರುತ್ತದೆ!” \enginline{(CWSN 1: 222)}

೧೦೨. ಹಿಂದೂಗಳ ಮತ್ತು ಬೌದ್ಧರ ಸಂನ್ಯಾಸದ ಮತ್ತು ಸಂನ್ಯಾಸವಲ್ಲದ ಆದರ್ಶಗಳ ಬಗ್ಗೆ:

“ಅನೇಕ ಆದರ್ಶಗಳನ್ನು ಹಿಂದೂಧರ್ಮವು ಎತ್ತಿ ಹಿಡಿಯುವುದಾದರೂ, ಅದರ ಹಿರಿಮೆ ಇರುವುದೇ - ಇವುಗಳಲ್ಲಿ ಯಾವುದನ್ನೂ ಇದು ಮಾತ್ರವೇ ನಿಜವಾದ ಪಥವೆಂದು ಹೇಳುವ ಧಾರ್ಷ್ಟ್ಯವನ್ನು ತೋರಿಸದಿರುವುದರಲ್ಲಿ. ಈ ವಿಚಾರದಲ್ಲಿ ಅದು ಬೌದ್ಧ ಧರ್ಮಕ್ಕಿಂತ ಭಿನ್ನವಾಗಿದೆ. ಉಳಿದೆಲ್ಲ ಮಾರ್ಗಗಳಿಗಿಂತಲೂ ಸಂನ್ಯಾಸವನ್ನೇ ಉನ್ನತವೆಂದೂ ಪರಿಪೂರ್ಣತೆಯನ್ನು ಪಡೆಯಬೇಕೆನ್ನುವ ಎಲ್ಲ ಜೀವಿಗಳೂ ಅದನ್ನು ಅನುಸರಿಸಬೇಕೆಂದೂ ಅದು ಹೇಳುತ್ತದೆ. ಮಹಾಭಾರತದಲ್ಲಿ ಬರುವ (ಕೌಶಿಕನೆಂಬ) ತರುಣ ಬ್ರಹ್ಮಚಾರಿಯನ್ನು ಮೊದಲು ಗೃಹಿಣಿ ಯೊಬ್ಬಳಿಂದಲೂ ಆ ನಂತರ ವ್ಯಾಧನೊಬ್ಬನಿಂದಲೂ ಉಪದೇಶ ಪಡೆಯಲು ಕಳುಹಿಸುವ ಕಥೆ ಇದನ್ನು ಸ್ಪಷ್ಪಪಡಿಸಲು ಸಾಕು. ಈತ ಕೇಳಿದ ಪ್ರಶ್ನೆಗೆ ಉತ್ತರವಾಗಿ ಇವರಲ್ಲಿ ಪ್ರತಿಯೊಬ್ಬರೂ ‘ಇದ್ದಲ್ಲಿಂದಲೇ ನನ್ನ ಕರ್ತವ್ಯವನ್ನು ನಿರ್ವಹಿಸುವುದರ ಮೂಲಕ ನಾನು ಬೆಳಕನ್ನು ಕಂಡುಕೊಂಡಿದ್ದೇನೆ’ ಎಂದು ಹೇಳುತ್ತಾರೆ. ದೇವರೆಡೆಗೆ ಕರೆದೊಯ್ಯದೆ ಇರುವ ಮಾರ್ಗವೇ ಇರಲಾರದು. ಯಶಸ್ಸನ್ನು ಪಡೆದುಕೊಳ್ಳುವ ಪ್ರಶ್ನೆ ಕೊನೆಗೆ ಅವರವರ ಅಂತರಂಗದ ಹಸಿವನ್ನೇ ಅವಲಂಬಿಸಿದೆ”. \enginline{(CWSN 1:223)}

೧೦೩. ಹೊಸದಾಗಿ ತಾನು ಪ್ರೇಮದಲ್ಲಿ ಸಿಕ್ಕಿಹಾಕಿಕೊಂಡಿರುವೆನೆಂದು ಉದ್ಘೋಷಿಸಿದ ಹುಡುಗಿಯೊಬ್ಬಳಿಗೆ ಪ್ರೇಮಿಯಲ್ಲಿಯೇ ಆದರ್ಶವನ್ನು ಕಾಣಬೇಕೆಂಬ ಕಲ್ಪನೆಯ ವಿಚಾರವಾಗಿ ಪ್ರತಿಕ್ರಿಯಿಸುತ್ತ ಸ್ವಾಮಿ ವಿವೇಕಾನಂದರೆಂದರು:

“ಈ ಕಲ್ಪನೆಗೇ ಅಂಟಿಕೋ! ಎಲ್ಲಿಯವರೆಗೆ ನೀವಿಬ್ಬರೂ ಒಬ್ಬರಲ್ಲೊಬ್ಬರು ಆದರ್ಶ ವನ್ನೇ ಕಾಣುವಿರೋ, ಅಲ್ಲಿಯವರೆಗೆ ನಿಮ್ಮ ಆರಾಧನಾಭಾವ ಹಾಗೂ ಸುಖ ಹೆಚ್ಚುವುದೇ ಹೊರತು ಕಡಿಮೆಯಾಗುವುದಿಲ್ಲ”. \enginline{(CWSN 1:224)}

೧೦೪. “ಅತ್ಯುನ್ನತ ಸತ್ಯ ಯಾವಾಗಲೂ ಅತಿ ಸರಳವಾಗಿರುವುದು”. \enginline{(CWSN 1:226)}

೧೦೫. ಅಮೆರಿಕಾದಲ್ಲಿನ ಪ್ರೇತಸಂಪರ್ಕಾಧಿವೇಶನಗಳ ಬಗ್ಗೆ ಸ್ವಾಮಿ ವಿವೇಕಾನಂದರ ಟೀಕೆ:

“ಯಾವಾಗಲೂ ಅತ್ಯಂತ ಸರಳವಾದ ರೀತಿಯಲ್ಲಿ ನಿರ್ವಹಿಸಲಾಗುವ ಮಹತ್ತಮ ವಂಚನೆ”. \enginline{(CWSN 1:227-233)}

೧೦೬. ವ್ಯಕ್ತಿಯೊಬ್ಬನು ದೇಹವೋ ಆತ್ಮವೋ ಎಂಬ ವಿಚಾರದಲ್ಲಿ ಪ್ರಾಚ್ಯ ಪಾಶ್ಚಾತ್ಯ ದೃಷ್ಟಿಕೋನಗಳ ಬಗ್ಗೆ:

“ಪಾಶ್ಚಾತ್ಯ ಭಾಷೆಗಳಲ್ಲಿ ವ್ಯಕ್ತವಾಗುವಂತೆ ಮನುಷ್ಯನೆಂದರೆ ಆತ್ಮವೊಂದನ್ನು ಹೊಂದಿರುವ ದೇಹ; ಪ್ರಾಚ್ಯ ಭಾಷೆಗಳಲ್ಲಿ ವ್ಯಕ್ತವಾಗುವುದಾದರೋ ಅವನು ದೇಹಧಾರಿಯಾದ ಆತ್ಮ ಎಂದು”. \enginline{(CWSN 1:236-237)}

೧೦೭. ಗುರುಗಳ ಮೇಲೆ ಸ್ವಾಮಿ ವಿವೇಕಾನಂದರಿಗೆ ಇರುವ ಭಕ್ತಿ ಗೌರವಗಳ ಬಗ್ಗೆ:

“ಅವರ ಮೇಲಿನ ಪ್ರೇಮ ಒಂದಿನಿತೂ ಕಡಿಮೆಯಾಗದ ಹಾಗೆ ಅವತಾರವಾದ ಅವರನ್ನು ನಾನು ಟೀಕಿಸಬಲ್ಲೆ! ಆದರೆ ಬಹಳಷ್ಟು ಜನರು ಹಾಗಿರುವುದಿಲ್ಲವೆಂದು ನನಗೆ ಗೊತ್ತು; ತಮ್ಮ ಭಕ್ತಿಯನ್ನು ಕಾಯ್ದುಕೊಳ್ಳುವುದೇ ಕ್ಷೇಮವೆಂದು ಅವರು ಭಾವಿಸುವರು!”. \enginline{(CWSN 1:252)}

“ಒಂದು ನಾಯಿಯ ನಿಷ್ಠೆಯ ಹಾಗೆ ನನ್ನದು! ಅದೇಕೆ ಎಂದು ತಿಳಿಯಲು ಅಪೇಕ್ಷಿಸುವುದಿಲ್ಲ ನಾನು! ಕೇವಲ ಅನುಸರಿಸುವುದರಲ್ಲೇ ನಾನು ತೃಪ್ತ! (ಅದೇ, 252-233)

೧೦೮. ರಾಮಕೃಷ್ಣ ಪರಮಹಂಸರು ತಮ್ಮ ಪ್ರತಿಯೊಂದು ದಿನವನ್ನೂ ಒಂದೆರಡು ಗಂಟೆಗಳ ಕಾಲ ತಮ್ಮ ಕೊಠಡಿಯಲ್ಲಿಯೇ ‘ಸಚ್ಚಿದಾನಂದ!’ ಎಂದೋ ‘ಶಿವೋಣಿ ಹಮ್​!’ ಎಂದೋ ಅಥವಾ ಇನ್ನಾವುದಾದರೂ ಪವಿತ್ರ ನಾಮವನ್ನು ಉಚ್ಚರಿಸುತ್ತ ನಡೆ ದಾಡಿ ಪ್ರಾರಂಭಿಸುತ್ತಿದ್ದರು”. \enginline{(CWSN 1:255)}

೧೦೯. ತಮ್ಮ ಮಹಾಸಮಾಧಿಗೆ ಕೆಲವು ತಿಂಗಳಿದ್ದಾಗ ಸ್ವಾಮಿ ವಿವೇಕಾನಂದರು ಹೇಳಿದ್ದು:

“ಎಷ್ಟೊಂದು ಸಲ ಗುರುವಾದವನು ಶಿಷ್ಯರ ಜೊತೆಗೇ ಯಾವಾಗಲೂ ಇದ್ದು ಅವರನ್ನು ಹಾಳು ಮಾಡುತ್ತಾನೆ! ಒಮ್ಮೆ ಪರಿಣತಿ ಕೊಟ್ಟಾದ ಮೇಲೆ ಮುಂದಾಳು ಅನುಯಾಯಿಗಳನ್ನು ಸ್ವತಂತ್ರವಾಗಿ ಬಿಡುವುದು ಮುಖ್ಯ; ಇಲ್ಲದಿದ್ದರೆ ಅವರು ಬೆಳೆಯುವುದೇ ಇಲ್ಲ!”. \enginline{(CWSN 1:260)}

೧೧೦. ತಮ್ಮ ಮಹಾಸಮಾಧಿಗೆ ಕೆಲವು ದಿನಗಳಿದ್ದಾಗ ಸ್ವಾಮಿ ವಿವೇಕಾನಂದರು ಹೇಳಿದ್ದು:

“ನಾನು ಸಾವಿಗೆ ಸಿದ್ಧತೆ ಮಾಡಿಕೊಳ್ಳುತ್ತಿದ್ದೇನೆ. ಮಹತ್ತರ ತಪಸ್ಸು ಧ್ಯಾನಗಳು ನನ್ನ ಮೇಲೆ ಆವಾಹಿತವಾಗಿವೆ; ನಾನು ಸಾಯುವುದಕ್ಕೆ ಸಿದ್ಧನಾಗುತ್ತಿದ್ದೇನೆ”. \enginline{(CWSN 1:261-262)}

೧೧೧. ಕಾಶ್ಮೀರದಲ್ಲಿ ಕಾಹಿಲೆ ಬಿದ್ದು ಚೇತರಿಸಿಕೊಂಡ ನಂತರ ಎರಡು ದುಂಡು ಕಲ್ಲುಗಳನ್ನು ಎತ್ತಿ ಹಿಡಿದುಕೊಂಡು ಸ್ವಾಮಿ ವಿವೇಕಾನಂದರು ಹೇಳಿದ್ದು:

“ಯಾವಾಗ ಸಾವು ಸಮೀಪಿಸಿದಂತೆನಿಸುತ್ತದೆಯೋ ಆಗ ನನ್ನೆಲ್ಲ ದೌರ್ಬಲ್ಯವೂ ಹೊರಟು ಹೋಗುವುದು. ನನಗೆ ಭಯವಾಗಲಿ, ಸಂಶಯವಾಗಲಿ, ಬಾಹ್ಯದ ಚಿಂತನೆಯಾಗಲಿ ಇರುವುದಿಲ್ಲ. ಸುಮ್ಮನೆ ಸಾವಿಗೆ ಸಿದ್ಧನಾಗತೊಡಗುತ್ತೇನೆ. (ಕೈಯಲ್ಲಿದ್ದ ಎರಡು ದುಂಡು ಗಲ್ಲುಗಳನ್ನು ಪರಸ್ಪರ ಸಂಘಟ್ಟಿಸುತ್ತ) ನಾನು ಅದರಷ್ಟು ಕಠಿಣ - ಏಕೆಂದರೆ ನಾನು ಭಗವಂತನ ಪಾದಾರ ವಿಂದಗಳನ್ನು ಸ್ಪರ್ಶಿಸಿರುವೆ!” \enginline{(CWSN 1:262)}

