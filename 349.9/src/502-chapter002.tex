
\chapter{ಅಧ್ಯಾಯ ೧: ಗಂಗೆಯ ಬಳಿಯ ಒಂದು ಮನೆ}

ಸ್ಥಳ: ಗಂಗಾದಡದಲ್ಲಿನ ಒಂದು ಕುಟೀರ.

ಕಾಲ: ೧೮೯೮ರ ಮಾರ್ಚ್ನಿಂದ ಮೇವರೆಗೆ.

ಗಂಗೆಯ ಬಳಿಯ ಈ ಮನೆಯ ಬಗ್ಗೆ ಒಮ್ಮೆ ಗುರುದೇವರು “ಧೀರಮಾತೆಯ ಆ ಪುಟ್ಟ ಮನೆ ಸ್ವರ್ಗದಂತಿರುವುದು, ಏಕೆಂದರೆ ಅದರಲ್ಲಿ ಮೊದಲಿನಿಂದ ಕೊನೆಯವರೆಗೂ ಇರುವುದು ಪ್ರೀತಿ ಮಾತ್ರವೇ” ಎಂದಿದ್ದರು.

ಅದು ನಿಜವಾಗಿಯೂ ಹಾಗೆಯೇ ಇದ್ದಿತು. ಒಳಗೆ ಅವಿಚ್ಛಿನ್ನ ಸಾಮರಸ್ಯ, ಹೊರಗೆ ಒಂದೊಂದೂ ಸುಂದರ-ಹಸಿರು ಹುಲ್ಲಿನ ಹಾಸು, ಎತ್ತರದ ತೆಂಗಿನ ಮರಗಳು, ಕಾಡಿನ ನಡುವೆಯ ಕಂದುಬಣ್ಣದ ಪುಟ್ಟ ಹಳ್ಳಿಗಳು; ಶಿವನ ಕೃಪೆಯನ್ನು ನಮಗೆ ತಂದು ಕೊಡುವುದಕ್ಕಾಗಿಯೋ ಎನ್ನುವಂತೆ ಪಕ್ಕದಲ್ಲಿಯ ಮರದೆತ್ತರದಲ್ಲಿ ಗೂಡು ಕಟ್ಟಿಕೊಂಡಿದ್ದ ನೀಲಕಂಠಪಕ್ಷಿ. ಬೆಳಗ್ಗೆ ಮನೆಯ ಹಿಂಭಾಗಕ್ಕೆ ಬೀಳುತ್ತಿದ್ದ ನೆರಳು; ಮಧ್ಯಾಹ್ನವಾದ ಮೇಲೆ ಮುಂಭಾಗದಲ್ಲಿ ನಾವು ಗಂಗೆಯನ್ನು - ಸಿಂಹಿಣಿಯಂತಹ ಗಂಗಾಮಾತೆಯನ್ನು -ಧ್ಯಾನಿಸುತ್ತ ಕುಳ್ಳಿರಬಹುದಾಗಿತ್ತು; ಎದುರಿಗೆ ಕಾಣಿಸುವಂತಿದ್ದ ದಕ್ಷಿಣೇಶ್ವರ.

ಹಳೆಯ ಸಂಪ್ರದಾಯದವರು ಒಬ್ಬರಲ್ಲ ಒಬ್ಬರು ಬರುತ್ತಿದ್ದರು; ನಾವು ಗುರುದೇವರ ಎಂಟು ವರ್ಷ ಅವಧಿಯ ಪರ್ಯಟನೆಯ ಬಗ್ಗೆ, ಊರಿಂದೂರಿಗೆ ಅವರ ಹೆಸರು ಬದಲಾಗುತ್ತಿದ್ದ ಬಗ್ಗೆ, ನಿರ್ವಿಕಲ್ಪ ಸಮಾಧಿಯ ಬಗ್ಗೆ, ಹಾಗೂ ಸಾಧಾರಣವಾಗಿ ನೋಡುವುದಕ್ಕೆ ಸಿಕ್ಕದ, ಪದಗಳಲ್ಲಿ ಅಭಿವ್ಯಕ್ತವಾಗದಷ್ಟು ನಿಗೂಢವಾದ, ಮತ್ತು ಅತ್ಯಂತ ಪ್ರೀತಿ ಪಾತ್ರರಾದವರು ಮಾತ್ರ ನೋಡಿರಬಹುದಾದ ಅವರ ಅಂತರಂಗದ ಆ ಅತಿ ಪವಿತ್ರವಾದ ಅಳಲಿನ ಬಗ್ಗೆ ಅಲ್ಪಸ್ವಲ್ಪ ಅರಿತುಕೊಂಡೆವು. ಅಲ್ಲಿಯೂ ಸಹ ಬರುತ್ತಿದ್ದುದು ಸ್ವಯಂ ಗುರುದೇವರೇ - ತಮ್ಮ ಗಾಯನದ, ಕವನಗಳ ತುಣುಕುಗಳೊಂದಿಗೆ; ಉಮೆ ಶಿವರ, ರಾಧೆ ಕೃಷ್ಣರ ತಮ್ಮ ಕಥೆಗಳೊಂದಿಗೆ.

ಹೊಸದೊಂದು ಪ್ರಜ್ಞೆಯ ಮೊದಲ ದ್ರವ್ಯವು ಸ್ಪಷ್ಟ ಹಾಗೂ ಪ್ರತ್ಯೇಕ ಅನುಭವಗಳ ಸರಣಿಯಾಗಿರಬೇಕು ಎಂಬುದನ್ನು ಅವರು ತಿಳಿದಂತಿತ್ತು. ಆ ಅನುಭವಗಳ ಪರಸ್ಪರ ಸಂಬಂಧ, ಅನುಕ್ರಮಗಳ ಕಲ್ಪನೆಗಾಗಿ ಶ್ರಮಿಸಲು ಶಿಷ್ಯನ ಮನಸ್ಸನ್ನು ಪ್ರಚೋದಿಸು ವಂತಿರಬೇಕು... ಹೆಚ್ಚಾಗಿ ಅವರು ನಮ್ಮೆದುರು ಚಿತ್ರಿಸುತ್ತಿದ್ದುದು ಭಾರತೀಯ ಧರ್ಮಗಳ ಬಗ್ಗೆಯೇ -ಈ ಹೊತ್ತು ಒಂದನ್ನು ಕುರಿತು ಹೇಳಿದರೆ ನಾಳೆ ಇನ್ನೊಂದು - ತತ್ಕ್ಷಣದ ಸ್ಪೂರ್ತಿಯಂತೆ ಅವರಿಗೆ ತೋರಿದ್ದನ್ನು ಅವರು ಆಯ್ದುಕೊಳ್ಳುತ್ತಿದ್ದರು ಎಂದು ನಮಗೆ ಅನ್ನಿಸುತ್ತಿತ್ತು. ಆದರೆ ಅವರು ನಮ್ಮೆದುರು ತೆರೆದಿಡು ತ್ತಿದ್ದುದು ಧರ್ಮವನ್ನಷ್ಟೇ ಅಲ್ಲ; ಕೆಲವೊಮ್ಮೆ ಇತಿಹಾಸ, ಮತ್ತೊಮ್ಮೆ ಜಾನಪದ, ಇನ್ನು ಕೆಲವು ಸಂದರ್ಭಗಳಲ್ಲಿ ಜನಾಂಗೀಯ, ಜಾತೀಯ, ರೂಢಿಗತ ಅಸಾಮಂಜಸ್ಯಗಳು, ದೋಷವೈವಿಧ್ಯಗಳು - ಹೀಗೆ ಏನು ಬೇಕಾದರೂ ಆಗಿರಬಹುದಾ ಗಿತ್ತು. ವಾಸ್ತವವಾಗಿ, ನಾವು ಕೇಳುತ್ತಿದ್ದಂತೆ ಇಡಿಯ ಭಾರತವೇ ಶ್ರೇಷ್ಠತಮವೂ ಕೊಟ್ಟ ಕೊನೆಯದಾದುದೂ ಆದ ಒಂದು ಪುರಾಣದ ರೂಪದಲ್ಲಿ ಅವರ ಬಾಯಿಂದ ತಾನೇ ತಾನಾಗಿ ನುಡಿಯುತ್ತಿರುವಂತೆ ಇತ್ತು.

ಅವರು ಗ್ರಹಿಸಿದ್ದ ಇನ್ನೊಂದು ಮಹತ್ತರ ಮನೋವೈಜ್ಞಾನಿಕ ರಹಸ್ಯವೆಂದರೆ, ನಮಗೆ ಮೊದಲು ಕಷ್ಟವಾಗುವಂತೆ ತೋರುವ ಅಥವಾ ವಿಕರ್ಷಿಸುವಂತೆ ಇರಬಹು ದಾದ ಯಾವುದನ್ನೂ ಸಹ ನಮಗಾಗಿ ಸುಲಭಗೊಳಿಸುವುದಕ್ಕೆ ಪ್ರಯತ್ನಿಸದೆ ಇರುವುದು. ಯೂರೋಪಿಯನ್ ಮನಸ್ಸಿಗೆ ಆಸ್ವಾದಿಸಲು ಅಸಾಧ್ಯವೆಂದೇ ತೋರಬಹುದಾದ ಭಾರತೀಯ ಸಂಗತಿಗಳ ವಿಷಯದಲ್ಲಿ ನಮ್ಮ ಅನುಭವದ ಪ್ರಾರಂಭದಲ್ಲಿಯೇ- ಅವುಗಳ ಪರಾಕಾಷ್ಠೆಯ ಸ್ವರೂಪವನ್ನು ನಮ್ಮೆದುರು ಇಟ್ಟುಬಿಡುತ್ತಿದ್ದರು. ಉದಾಹರಣೆಗಾಗಿ, ಗೌರಿ ಮತ್ತು ಶಂಕರರ ಏಕರೂಪತೆಯ ಬಗೆಗಿನ ಕೆಲವು ಪದ್ಯಗಳನ್ನು ಉದಾಹರಿಸುತ್ತಿದ್ದರು:

\begin{myquote}
ಒಂದು ಪಾರ್ಶ್ವದಿ ಬೆಳೆವ ಕರಿಯ ಗುಂಗುರು ನೀಳ\\ಕೇಶವಿನ್ನೊಂದರಲಿ ಅದುವೆ ಜಡೆಗಟ್ಟಿದೆ.\\ಒಂದು ಪಾರ್ಶ್ವದಿ ದಿವ್ಯ ಹೂಹಾರಗಳ ಶೋಭೆ\\ಇನ್ನೊಂದರಲಿ ಅಸ್ಥಿ ಕರ್ಣಕುಂಡಲ, ಹಾವು.\\ಒಂದೆಡೆ ಬಳಿದ ಬೂದಿ, ಗಿರಿಶಿಖರ ಮಂಜಿನೊಲು\\ಇನ್ನೊಂದರಲಿ ಉಷೆಯ ಹೊನ್ನಬೆಳಕಿನ ರಾಗ.\\ಎರಡಾಗಿ ಒಡೆದಿರುವ ರೂಪವದು ದೇವನದು-\\ಒಂದರ್ಧ ಪುರುಷನದು, ಇನ್ನರ್ಧ ನಾರಿಯದು;\\ಅರ್ಧನಾರೀಶ್ವರನ ರೂಪ ತಳೆದಿಹ ದೇವ.
\end{myquote}

ಸಂಭಾಷಣೆಯ ವಿಷಯ ಯಾವುದೇ ಆಗಿದ್ದರೂ, ಅದು ಕೊನೆಗೊಳ್ಳುತ್ತಿದ್ದುದು ಅನಂತ ತೆಯ ಧ್ವನಿಯಲ್ಲಿಯೇ... ಆರಂಭದಲ್ಲಿ ಸಮಾಜ ಸಂಸ್ಕೃತಿ ಸಾಹಿತ್ಯ ವಿಜ್ಞಾನ ಇತ್ಯಾದಿ ಯಾವುದಾದರೊಂದು ವಿಷಯವನ್ನು ತೆಗೆದುಕೊಳ್ಳುವ ಹಾಗೆ ತೋರಿದರೂ, ಕೊನೆಯಲ್ಲಿ ನಮಗೆ ಅದು ಆತ್ಯಂತಿಕ ದರ್ಶನಕ್ಕೊಂದು ಉದಾಹರಣೆ ಎಂದು ಭಾಸವಾಗುವ ಹಾಗೆ ಮಾಡಿಬಿಡುತ್ತಿದ್ದರು. ಅವರ ಮಟ್ಟಿಗೆ ಲೌಕಿಕವೆನ್ನಬಹುದಾದ ಯಾವುದೊಂದೂ ಇರಲಿಲ್ಲ. ಬಂಧನವೆಂದರೆ ಅವರಿಗೆ ಆಗುತ್ತಿರಲಿಲ್ಲ. “ಸಂಕೋಲೆಗಳನ್ನು ಹೂಗಳಿಂದ ಮರೆಮಾಚು”ವವರನ್ನು ಕಂಡರೆ ಅವರು ಭಯಭೀತರಾಗುತ್ತಿದ್ದರಾದರೂ, ನಿಜವಾದ ವಿಮರ್ಶಕನಂತೆ ಉನ್ನತ ನೆಲೆಯ ಕಲೆಗೂ ಇದಕ್ಕೂ ಇರುವ ವ್ಯತ್ಯಾಸವನ್ನು ಗುರುತಿಸದೆ ಬಿಡುತ್ತಿರಲಿಲ್ಲ.

ಒಂದು ದಿನ ಕೆಲವು ಯೂರೋಪಿಯನ್ ಅತಿಥಿಗಳನ್ನು ಎದುರುಗೊಳ್ಳುವವರಿದ್ದೆವು. ಇದಕ್ಕಿದ್ದಂತೆ ಅವರು ಪರ್ಷಿಯನ್ ಕಾವ್ಯದ ಬಗ್ಗೆ ದೀರ್ಘವಾಗಿ ಮಾತನಾಡಲಾರಂಭಿ ಸಿದರು. ಒಂದು ಸಲ “ನನ್ನ ಪ್ರೇಮಿಯ ಮುಖದ ಮಚ್ಚೆಯೊಂದಕೆ ನಾನು, ಕೊಡಲಿರುವೆ ಸಮರ್ಖಂಡದೆಲ್ಲ ಸಿರಿಯನು ಕೂಡ!” ಎಂಬ ಸಾಲನ್ನು ತಾವು ಉದ್ಧರಿಸುತ್ತಿರುವುದನ್ನು ಇದ್ದಕ್ಕಿದ್ದ ಹಾಗೆ ಮನಗಂಡ ಅವರು, ನಮ್ಮ ಕಡೆಗೆ ತಿರುಗಿ, ಶಕ್ತಿಯುತವಾಗಿ “ಪ್ರೇಮಕವನವೊಂದನ್ನು ಪರಿಭಾವಿಸಲಾರದವನಿಗೆ ನಾನೊಂದು ಹುಲ್ಲುಕಡ್ಡಿಯನ್ನೂ ಕೊಡಲಾರೆ!” ಎಂದು ಒತ್ತಿ ಹೇಳಿದರು. ಅವರ ಮಾತು ಸಹ ಚಾಟೂಕ್ತಿಗಳಿಂದ ಕೂಡಿ ರಸವತ್ತಾಗಿರುತ್ತಿತ್ತು.

ಅದೇ ಮಧ್ಯಾಹ್ನ, ದೀರ್ಘವಾದೊಂದು ರಾಜಕೀಯವಾದ-ಪ್ರತಿವಾದಗಳಿಂದ ಕೂಡಿದ ಸಂಭಾಷಣೆಯ ನಡುವೆ ಅವರೆಂದರು: “ನಾವು ಒಂದು ರಾಷ್ಟ್ರವೆಂದಾಗ ಬೇಕಾದರೆ, ಎಲ್ಲರಿಗೂ ಸಾಮಾನ್ಯವಾದ ಪ್ರೇಮದ ಜೊತೆಗೆ, ಎಲ್ಲರೂ ಭಾಗಿಗಳಾಗಿ ಬೇಕಾಗಿರುವ ದ್ವೇಷವೂ ಅಗತ್ಯವೆಂದು ತೋರುತ್ತದೆ!”

ಕೆಲವು ತಿಂಗಳ ನಂತರ ಅವರೊಮ್ಮೆ, ತಾವು ಖಚಿತ ಜೀವಿತೋದ್ದೇಶವುಳ್ಳವರ ಎದುರಿಗೆ ಉಮಾ ಮತ್ತು ಶಿವ ಇಬ್ಬರ ಹೊರತಾಗಿ ಬೇರಾವ ದೇವರುಗಳ ವಿಚಾರವಾಗಿಯೂ ಮಾತನಾಡುವುದಿಲ್ಲ ಎಂದರು. ಏಕೆಂದರೆ ಶಿವ ಹಾಗೂ ತಾಯಿಯೇ ಆ ಕೆಲಸಗಾರರನ್ನು ಸೃಜಿಸಿದ್ದು. ಎಲ್ಲ ವಿಷಯಗಳ ಕೊನೆಯೂ ಹೇಗೆ ಭಕ್ತಿಯೇ ಆಗಿದೆ ಎಂಬುದರ ಅರಿವು ಆ ಕ್ಷಣಕ್ಕೆ ಅವರಿಗೆ ಇತ್ತೇ ಎಂದು ನಾನು ಚಿಂತಿಸಿದ್ದುಂಟು. ನರದೌರ್ಬಲ್ಯಕ್ಕೊಳಗಾದವರಿಗೆ ಆಧ್ಯಾತ್ಮಿಕ ಭಾವುಕತೆ ಕೇವಲ ಒಂದು ಭೋಗವಿಲಾಸದಂತೆ ಆಗಬಹುದೆಂಬ ಭಯವಿದ್ದರೂ, ಭಗವಂತನ ಅಮಲಿನಲ್ಲಿ ಒಬ್ಬನು ಹೇಗೆ ಕಳೆದುಹೋಗಿಬಿಡಬಹುದು ಎಂಬುದನ್ನು ಕುರಿತು ಅಲ್ಪ ಸೂಚನೆಯನ್ನಾದರೂ ಕೊಡದೆ ಇರುವುದಕ್ಕೆ ಅವರಿಗೆ ಸಾಧ್ಯವಾಗುತ್ತಿರಲಿಲ್ಲ. ಅದಕ್ಕೇ ಅವರು ಕೆಲವು ಪದ್ಯ ಸ್ವಾಮಿ ವಿವೇಕಾನಂದರೊಂದಿಗೆ ಪ್ರವಾಸ - ಕೆಲವು ಟಿಪ್ಪಣಿಗಳು ೩೬೩ ೩೬೨ ಸ್ವಾಮಿ ವಿವೇಕಾನಂದರ ಕೃತಿಶ್ರೇಣಿಗಳನ್ನು ನಮ್ಮೆದುರು ಹಾಡುತ್ತಿದ್ದರು. ಉದಾಹರಣೆಗೆ -

\begin{myquote}
ಬೃಂದಾವನದೊಳು ಸುಂದರ ತೋಟದಿ\\ರಾಧೆಯ ರಾಣಿಯ ಮಾಡಿಹರು;\\ಅರಮನೆ ದ್ವಾರದಿ ಕೊಳಲನು ಊದುತ\\ಕೃಷ್ಣನು ತಾನೇ ನಿಂತಿಹನು.\\“ತನ್ನಯ ಪ್ರೇಮದ ಸಂಪತ್ತೆಲ್ಲವ\\ಈಗಲೆ ರಾಧೆಯು ಹಂಚುವಳು.\\ಬಾಗಿಲ ಭಟ ನಾನಾಗಿದ್ದರು ಸಹ\\ಪ್ರವೇಶವಿರುವುದು ಎಲ್ಲರಿಗೆ.\\ಆರ್ದ್ರತೆ ಬಯಸುವರೆಲ್ಲರು ಬನ್ನಿರಿ\\ಸುಮ್ಮನೆ ರಾಧೆಗೆ ಜಯವೆನ್ನಿ\\ಪ್ರೇಮದ ರಾಜ್ಯಕೆ ಲಗ್ಗೆಯಿಡಿ.”
\end{myquote}

ಅಥವಾ ತಮ್ಮ ಗೆಳೆಯನು\footnote{1. ಬಂಗಾಳಿ ನಾಟಕಕಾರ ಶ‍್ರೀ ಗಿರೀಶ್ಚಂದ್ರ ಘೋಷ್.} ಬರೆದ ಈ ಹಾಡಿಗೆ ಮಾರುಲಿಯಂತಿರುವ ವೃಂದಗಾನವನ್ನು ಅವರು ನಮಗಿತ್ತರು:

ಪುರುಷರು: ನೀನೆ ಜೀವದ ಜೀವ\\ನೀಲಿ ಕಂಗಳ ಚೆಲುವ\\ಪೀತಾಂಬರಧಾರಿ!

ಸ್ತ್ರೀಯರು: ಬೃಂದಾವನದಾ ಗೋಪಾಲಕನೆ\\ಗೋಪಿಯರೆದುರಿಗೆ ಕಾಲೂರಿ ನಿಂತೆ\\ಶ್ಯಾಮಶರೀರನೆ ಹೇ ಕೃಷ್ಣಾ!

ಪುರುಷರು: ಜಯಜಯವೆನುವುದು ಎನ್ನಯ ಜೀವ\\ನರರೂಪದ ಶ‍್ರೀ ದೇವನಿಗೆ

ಸ್ತ್ರೀಯರು: ಚೆಲುವೆಲ್ಲವೂ ನಿನದೆ-ಗೋಪಿಯರೆಮಗೆ

ಪುರುಷರು: ಯಜ್ಞೇಶ್ವರನೆ, ದೀನ ರಕ್ಷಕನೆ, ಹೇ ಕೃಷ್ಣ!

ಸ್ತ್ರೀಯರು: ರಾಧೆಯ ಪ್ರೇಮದ ಕಣ್ಣೀರಿನಲಿ\\ತೇಲುತಲಿಹೆ ನೀ ಶ‍್ರೀಕೃಷ್ಣ!.....

\textbf{ಮಾರ್ಚ್ ೨೫}

....ಈ ಕಾಲದಲ್ಲಿ ಸ್ವಾಮಿಗಳು ಕುಟೀರಕ್ಕೆ ಬೆಳಗ್ಗೆ ಬೇಗನೆ ಆಗಮಿಸಿ ತುಂಬ ಹೊತ್ತು ಅಲ್ಲೇ ಕಳೆಯುವ, ಸಂಜೆ ಇಳಿಹೊತ್ತಾದ ಮೇಲೆ ಮತ್ತೊಮ್ಮೆ ಬರುವ ದಿನಚರಿಯನ್ನು ರೂಢಿಸಿಕೊಂಡಿದ್ದರು. ಏನೇ ಇರಲಿ, ಈ ಭೇಟಿಯ ಎರಡನೆಯ ದಿನ ಬೆಳಗ್ಗೆ - ಎಂದರೆ (ಕ್ರಿಸ್ತನ ಅವತಾರದ) ಮುನ್ಸೂಚನೆಯ ಹಬ್ಬದ ದಿನ - ನಾವು ಮೂವರನ್ನೂ ಮಠಕ್ಕೆ ಕರೆದುಕೊಂಡು ಹೋದರು; ಅಲ್ಲಿ ಪೂಜಾಕೊಠಡಿಯಲ್ಲಿ ನಮ್ಮಲ್ಲಿ ಒಬ್ಬಳಿಗೆ ದೀಕ್ಷೆಯನ್ನು ಕೊಟ್ಟು ಬ್ರಹ್ಮಚಾರಿಣಿಯನ್ನಾಗಿ ಮಾಡಿದರು. ಅಂದಿನ ಬೆಳಗ್ಗೆ ಅತ್ಯಂತ ಆನಂದ ದಾಯಕವಾಗಿತ್ತು.

ದೀಕ್ಷೆಯ ನಂತರ ಮಹಡಿಯ ಮೇಲಕ್ಕೆ ಕರೆದೊಯ್ದರು. ಅಲ್ಲಿ ಸ್ವಾಮಿಗಳು ಶಿವ ಯೋಗಿಯಂತೆ ವಿಭೂತಿ ಬಳಿದುಕೊಂಡು, ಮೂಳೆಯ ಕರ್ಣಕುಂಡಲಗಳನ್ನೂ ಜಟೆಯನ್ನೂ ಧರಿಸಿದರು; ಒಂದು ಗಂಟೆಯ ಕಾಲ ಭಾರತೀಯವಾದ್ಯವನ್ನು ತಾವೇ ನುಡಿಸಿಕೊಂಡು ಭಾರತೀಯ ಸಂಗೀತವನ್ನು ಹಾಡಿದರು.

ಸಂಜೆ, ದೋಣಿಯಲ್ಲಿ ಗಂಗಾನದಿಯನ್ನು ದಾಟುತ್ತಿರುವಾಗ, ತಮ್ಮ ಹೃದಯವನ್ನು ನಮ್ಮೆದುರು ತೆರೆದಿಟ್ಟರು - ತಮ್ಮ ಗುರುವು ನಂಬಿಕೆಯಿಂದ ತಮಗೆ ವಹಿಸಿದ್ದ ಕಾರ್ಯಗಳ ಬಗೆಗಿನ ತಮ್ಮೆದೆಯ ಆತಂಕಗಳನ್ನು, ಪ್ರಶ್ನೆಗಳನ್ನು ಹೇಳಿಕೊಂಡರು.

ಇನ್ನೊಂದು ವಾರದಲ್ಲಿಯೇ ಡಾರ್ಜಿಲಿಂಗ್ಗೆ ಹೊರಟುಹೋದರು; ಪ್ಲೇಗ್ ಉದ್ಘೋಷಣೆ ಪುನಃ ಅವರನ್ನು ಹಿಂದಕ್ಕೆ ಕರೆತರುವವರೆಗೆ ನಾವು ಅವರನ್ನು ಮತ್ತೊಮ್ಮೆ ನೋಡಲಿಲ್ಲ.

\textbf{ಮೇ ೩.}

ನಮ್ಮಲ್ಲಿ ಇಬ್ಬರು ಶ‍್ರೀಮಾತೆಯವರ ಮನೆಯಲ್ಲಿ ಅವರನ್ನು ಭೇಟಿಯಾದೆವು. ರಾಜಕಾರಣದ ಗಾಢಾಂಧಕಾರ ಕವಿದಿತ್ತು. ಬಿರುಗಾಳಿ ಏಳುವುದೇನೋ ಎಂಬಂತಿತ್ತು... ಪ್ಲೇಗ್ನ ಭಯದಿಂದ ಜನರು ದಂಗೆಯೆದ್ದಿದ್ದರು. ನಮ್ಮಿಬ್ಬರ ಕಡೆಗೆ ತಿರುಗಿ ಗುರು ದೇವರು ಹೇಳಿದರು: “ಕಾಳಿಯ ಅಸ್ತಿತ್ವವನ್ನು ಕುರಿತು ಜರೆಯುವ ಕೆಲವರಿದ್ದಾರೆ. ಆದರೂ, ಅವಳೇ ಇಂದು ಜನರ ನಡುವೆ ಹೊರಟುಬಿಟ್ಟಿದ್ದಾಳೆ. ಭಯದಿಂದ ಜನರು ದಿಕ್ಕುಗೆಟ್ಟಿದ್ದಾರೆ; ಮೃತ್ಯುವನ್ನು ನಿಭಾಯಿಸುವುದಕ್ಕಾಗಿ ಸೈನ್ಯವನ್ನು ಕರೆಸಲಾಗಿದೆ. ದೇವರು ಒಳಿತಾಗಿ ಹೇಗೋ ಹಾಗೆ ಕೆಡುಕು ಮೈತಳೆಯುವುದಿಲ್ಲವೆಂದು ಯಾರು ತಾನೆ ಹೇಳಬಲ್ಲರು? ಆದರೆ ಹಿಂದೂ ಮಾತ್ರವೇ ಅವನನ್ನು ಕೆಡುಕಿನ ರೂಪದಲ್ಲೂ ಆರಾಧಿಸುವ ಧೈರ್ಯ ಮಾಡಬಲ್ಲ.”

ಸೋಂಕು ತಾಂಡವವಾಡುತ್ತಿತ್ತು; ಜನರಲ್ಲಿ ಧೈರ್ಯ ತುಂಬುವ ಕಾರ್ಯ ನಡೆದಿತ್ತು; ಆದರೂ ಅವರು ಹಿಂದಿರುಗಿ ಬಂದ ನಂತರ ಸಾಧ್ಯವಾದಷ್ಟೂ ಹಳೆಯ ದಿನಚರಿಯನ್ನು ಅನುಸರಿಸತೊಡಗಿದೆವು. ಈ ಸಾಧ್ಯತೆಯ ಕಾರ್ಮೋಡ ಕವಿದಿದ್ದಷ್ಟು ಕಾಲವೂ ಅವರು ಕಲ್ಕತ್ತದಿಂದ ಹೊರಡಲಿಲ್ಲ. ಆದರೆ ಅದು ಕಳೆದು ಹೋಯಿತು; ಅದರೊಂದಿಗೆ ಆ ಆನಂದದ ದಿನಗಳೂ ಹೊರಟುಹೋದವು; ನಾವೂ ಹೊರಡುವ ಕಾಲ ಬಂದೇ ಬಂದಿತು.

