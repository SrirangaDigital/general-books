
\chapter{ಜ್ಞಾನದೆಡೆಗೆ ಮೊದಲ ಹೆಜ್ಜೆ}

(೧೮೯೫, ಡಿಸೆಂಬರ್ ೧೧ರಂದು ನ್ಯೂಯಾರ್ಕ್ನಲ್ಲಿ ಜ್ಞಾನಯೋಗದ ಮೇಲೆ ನೀಡಿದ ಉಪನ್ಯಾಸ. ಸ್ವಾಮಿ ಕೃಪಾನಂದರು ಬರೆದುಕೊಂಡದ್ದು)

ಜ್ಞಾನ ಎಂಬ ಶಬ್ದವು ‘ಜ್ಞ’ ಎಂಬ ಧಾತುವಿನಿಂದ ಬಂದಿದೆ. ಜ್ಞ ಎಂದರೆ ತಿಳಿ ಯುವುದು. ಇದೇ ಧಾತುವಿನಿಂದಲೇ ನಿಮ್ಮ ಇಂಗ್ಲಿಷ್ ಶಬ್ದ’ ಓ್ಞಟಡಿ’ ಎಂಬುದು ಬಂದಿದೆ. ಜ್ಞಾನಯೋಗವೆಂದರೆ ಜ್ಞಾನದ ಮೂಲಕ ಯೋಗ. ಜ್ಞಾನಯೋಗದ ಗುರಿಯೇನು? ಮುಕ್ತಿ. ಯಾವುದರಿಂದ ಮುಕ್ತಿ? ನಮ್ಮ ಅಪರಿಪೂರ್ಣತೆಗಳಿಂದ, ಜೀವನ ದುಃಖದಿಂದ ಮುಕ್ತಿ. ನಾವು ಏಕೆ ದುಃಖಿಗಳಾಗಿದ್ದೇವೆ? ನಮ್ಮ ದುಃಖಕ್ಕೆ ಬಂಧನವೇ ಕಾರಣ. ಯಾವುದು ಬಂಧನ? ಪ್ರಕೃತಿಯೇ ಬಂಧನ. ಯಾರು ನಮ್ಮನ್ನು ಬಂಧಿಸಿರುವುದು? ನಾವೇ.

ಇಡೀ ಜಗತ್ತು ಕಾರ್ಯಕಾರಣ ಬದ್ಧವಾಗಿದೆ. ಒಳ ಜಗತ್ತಿನಲ್ಲಾಗಲಿ ಅಥವಾ ಹೊರ ಜಗತ್ತಿನಲ್ಲಾಗಲಿ ಕಾರಣವಿಲ್ಲದ್ದು ಯಾವುದೂ ಇಲ್ಲ. ಪ್ರತಿಯೊಂದು ಕಾರಣಕ್ಕೂ ಒಂದು ಕಾರ್ಯ ಇದ್ದೇ ಇರಬೇಕು.

ನಾವು ಬಂಧನದಲ್ಲಿರುವುದಂತೂ ನಿಜ. ಇದನ್ನು ಪ್ರಮಾಣೀಕರಿಸಬೇಕಾಗಿಲ್ಲ. ಉದಾಹರಣೆಗೆ, ಈ ಗೋಡೆಯ ಮೂಲಕ ತೂರಿಕೊಂಡು ಕೋಣೆಯಿಂದ ಹೊರಗೆ ಹೋಗಲು ನಾನು ಇಷ್ಟಪಡುತ್ತೇನೆ. ಆದರೆ ಅದು ನನಗೆ ಸಾಧ್ಯವಿಲ್ಲ. ನನಗೆ ಖಾಯಿಲೆ ಬರದೆ ಇದ್ದಿದ್ದರೆ ತುಂಬ ಚೆನ್ನಾಗಿರುತ್ತಿತ್ತು. ಆದರೆ ಅದನ್ನು ನಾನು ನಿರೋಧಿಸಲಾರೆ. ನನಗೆ ಸಾಯದೆ ಇರಲು ಇಷ್ಟ. ಆದರೆ ನಾನು ಸಾಯಲೇಬೇಕು. ನನಗೆ ಏನೇನೊ ಮಾಡಬೇಕೆಂಬಾಸೆ. ಆದರೆ ಮಾಡಲಾರೆ. ಇಚ್ಛೆ ಇದೆ, ಆದರೆ ನಾವು ಆಸೆಯನ್ನು ಪೂರ್ತಿಗೊಳಿಸಲಾರೆವು. ನಮಗೆ ಯಾವುದಾದರೂ ಆಸೆಯಿದ್ದು ಅದನ್ನು ಪೂರ್ತಿಗೊಳಿಸಲಾಗದ ಸ್ಥಿತಿಯಲ್ಲಿದ್ದರೆ, ಅದರ ಪರಿಣಾಮವೇ ದುಃಖ. ಈ ದುಃಖಕ್ಕೆ ಕಾರಣರಾರು? ನಾವೇ. ನಾನು ಪಡುವ ದುಃಖಕ್ಕೆಲ್ಲ ನಾನೇ ಕಾರಣ.

ಮಗು ಹುಟ್ಟಿದಾಗಿನಿಂದಲೇ ದುಃಖ ಪ್ರಾರಂಭವಾಗುತ್ತದೆ. ದುರ್ಬಲ, ಅಸಹಾಯಕವಾಗಿ ಅದು ಜಗತ್ತನ್ನು ಪ್ರವೇಶಿಸುತ್ತದೆ. ಜೀವನದ ಪ್ರಥಮ ಚಿಹ್ನೆಯೇ ಅಳು. ಪ್ರಾರಂಭದಿಂದಲೇ ಈ ದುಃಖ ಇರುವುದಾದರೆ ನಾವೇ ಅದರ ಕಾರಣರಾಗಿರಲು ಹೇಗೆ ಸಾಧ್ಯ? ಇದಕ್ಕೂ ಹಿಂದೆ ಅದರ ಕಾರಣಕರ್ತರಾಗಿದ್ದೆವು. (ಇಲ್ಲಿ ಸ್ವಾಮಿ ವಿವೇಕಾನಂದರು ತುಂಬ ಸ್ವಾರಸ್ಯಕರವಾದ ಪುನರ್ಜನ್ಮ ಸಿದ್ಧಾಂತವನ್ನು ಕುರಿತು ವಿಸ್ತಾರವಾಗಿ ಹೇಳಿದರು ಮತ್ತು ಮುಂದುವರಿಸಿದರು.)

ಪುನರ್ಜನ್ಮ ಸಿದ್ಧಾಂತವನ್ನು ಅರ್ಥಮಾಡಿಕೊಳ್ಳಬೇಕಾದರೆ, ಈ ಜಗತ್ತಿನಲ್ಲಿ ಶೂನ್ಯದಿಂದ ಯಾವುದೂ ಉತ್ಪ ತ್ತಿಯಾಗುವುದಿಲ್ಲ ಎಂಬುದನ್ನು ತಿಳಿದುಕೊಳ್ಳಬೇಕು. ಜೀವಾತ್ಮ ಎನ್ನುವುದು ಇರುವುದಾದರೆ ಅದು ಶೂನ್ಯದಿಂದ ಏನಾದರೂ ಉತ್ಪ ತ್ತಿಯಾಗುವು ದಾದರೆ, ಹಾಗೆಯೇ ಅದು ಶೂನ್ಯದಲ್ಲಿ ಲಯವಾಗಬೇಕು. ನಾವು ಶೂನ್ಯದಿಂದ ಬಂದಿದ್ದರೆ ಶೂನ್ಯಕ್ಕೆ ಹಿಂತಿರುಗಬೇಕು. ಆದಿವುಳ್ಳದ್ದೆಲ್ಲ ಅಂತ್ಯವನ್ನು ಹೊಂದಿರಲೇಬೇಕು. ಆದ್ದರಿಂದ ನಮ್ಮ ಸ್ವರೂಪವಾದ ಆತ್ಮಕ್ಕೆ ಆದಿ ಇರಲು ಸಾಧ್ಯವಿಲ್ಲ. ನಾವು ಯಾವಾಗಲೂ ಇದ್ದೆವು.

ನಾವು ಹಿಂದೆ ಇದ್ದಿಲ್ಲದೆ ಇದ್ದರೆ ನಮ್ಮ ಈಗಿನ ಅಸ್ತಿತ್ವಕ್ಕೆ ಯಾವ ವಿವರಣೆಯೂ ಇಲ್ಲ. ಮಗುವು ಕಾರಣಗಳ ಮೂಟೆಯೊಂದಿಗೆ ಜನಿಸುತ್ತದೆ. ಮಗುವಿನಲ್ಲಿ ಮೃತ್ಯು ಭಯ, ಅನೇಕ ಪ್ರವೃತ್ತಿಗಳು ಇವೇ ಮುಂತಾದ ಎಷ್ಟೋ ಅಂಶಗಳನ್ನು ನೋಡುತ್ತೇವೆ. ಮಗುವಿಗೆ ಹಿಂದಿನ ಹಿಂದಿನ ಅನುಭವಗಳನ್ನು ಒಪ್ಪಿಕೊಳ್ಳದೆ ಇವುಗಳನ್ನು ಬೇರೆ ಯಾವ ರೀತಿಯಲ್ಲಿಯೂ ವಿವರಿಸಲಾಗುವುದಿಲ್ಲ. ಹಾಲು ಕುಡಿಯುವುದಕ್ಕೆ ಮತ್ತು ತನ್ನದೇ ಆದ ರೀತಿಯಲ್ಲಿ ಕುಡಿಯುವುದಕ್ಕೆ ಮಗುವಿಗೆ ಯಾರು ಕಲಿಸಿದರು? ಅದಕ್ಕೆ ಎಲ್ಲಿಂದ ಈ ಜ್ಞಾನ ಬಂತು? ಅನುಭವವಿರದೆ ಜ್ಞಾನವಿರಲಾರದೆಂಬುದು ನಮಗೆ ಗೊತ್ತು. ಮಗುವಿನಲ್ಲಿ ಜ್ಞಾನವು ತಾನೇ ತಾನಾಗಿ ಸುಙರಿಸುತ್ತದೆ ಅಥವಾ ಅದು ಅದರ ಹುಟ್ಟುಗುಣ ಎಂದು ಹೇಳುವುದು ಸಾಧಿಸಬೇಕಾದ ವಿಷಯವನ್ನೇ ಸಿದ್ಧವೆಂದಿಟ್ಟುಕೊಂಡಂತಾಗುತ್ತದೆ. \enginline{(Petitio Principle)} ಬೆಳಕು ಗಾಜಿನ ಮೂಲಕ ಹಾದು ಹೋಗುವುದಕ್ಕೆ ಕಾರಣವೇನು ಎಂದು ಒಬ್ಬನು ಕೇಳಿದರೆ ಅದು ಪಾರದರ್ಶಕವಾಗಿರುವುದರಿಂದ ಎಂದು ಹೇಳಿದಂತೆ ಇದು. ಇದೊಂದು ಉತ್ತರವೇ ಅಲ್ಲ, ಏಕೆಂದರೆ ಅವನ ಮಾತನ್ನೇ ಒಂದು ಪಾರಿಭಾಷಿಕ ಶಬ್ದವಾಗಿ ಪರಿವರ್ತಿಸುತ್ತಿದ್ದೇನೆ ಅಷ್ಟೆ. ಪಾರದರ್ಶಕ ಎಂದ ರೇನೇ ಯಾವುದರ ಮೂಲಕ ಬೆಳಕು ಹಾದು ಹೋಗುವುದೊ ಅದು. ಅದೇ ಅವನ ಪ್ರಶ್ನೆಯೂ ಕೂಡ. ಗಾಜಿನ ಮೂಲಕ ಏಕೆ ಬೆಳಕು ಹಾದು ಹೋಗುತ್ತದೆ ಎಂಬ ಪ್ರಶ್ನೆಗೆ ಅದು ಗಾಜಿನ ಮೂಲಕ ಹಾದುಹೋಗುವುದರಿಂದ ಎಂದು ನಾನು ಉತ್ತರಿಸಿದಂತಾಯಿತು.

ಹಾಗೆಯೇ, ಈ ಪ್ರವೃತ್ತಿಗಳು ಮಗುವಿನಲ್ಲಿ ಏಕಿವೆ ಎಂಬುದು ಪ್ರಶ್ನೆ. ಸಾವನ್ನೇ ನೋಡದೆ ಅದಕ್ಕೆ ಸಾವಿನ ಭಯ ಇರಲು ಹೇಗೆ ಸಾಧ್ಯ? ಅದು ಹುಟ್ಟುತ್ತಿರುವುದು ಇದೇ ಮೊದಲಬಾರಿಯಾದರೆ ಎದೆಹಾಲನ್ನು ಕುಡಿಯಬೇಕೆಂಬ ಜ್ಞಾನ ಅದಕ್ಕೆ ಹೇಗೆ ಬಂತು? ಅದು ಹುಟ್ಟುಗುಣ ಎಂದು ಹೇಳಿದರೆ ಪ್ರಶ್ನೆಯನ್ನು ಹಿಂತಿರುಗಿಸಿದಂತಾ ಗುತ್ತದೆ. “ಇದು ಹುಟ್ಟು ಗುಣ” ಎಂದು ಮುಂತಾಗಿ ಅರ್ಥಹೀನ ಮಾತನಾಡುವು ದಕ್ಕಿಂತ “ನನಗೆ ತಿಳಿಯದು” ಎಂದು ಹೇಳುವುದು ಉತ್ತಮ.

ಅಭ್ಯಾಸಕ್ಕಿಂತ \enginline{(habit)} ಬೇರೆಯಾದ ಸ್ವಭಾವವಾಗಲಿ, ಹುಟ್ಟುಗುಣವಾಗಲಿ ಇಲ್ಲ. ಅಭ್ಯಾಸವು ವ್ಯಕ್ತಿಯ ಎರಡನೆಯ ಸ್ವಭಾವ ಮತ್ತು ಅದು ಮೊದಲನೆಯ ಸ್ವಭಾವ ಕೂಡ. ನಿಮ್ಮ ಎಲ್ಲ ಸ್ವಭಾವವೂ ಅಭ್ಯಾಸದ ಪರಿಣಾಮ, ಮತ್ತು ಅಭ್ಯಾಸವು ಅನುಭವದ ಪರಿಣಾಮ. ಅನುಭವದಿಂದಲ್ಲದೆ ಯಾವುದೇ ಜ್ಞಾನವು ಸಾಧ್ಯವಿಲ್ಲ.

ಆದ್ದರಿಂದ ಈ ಮಗುವೂ ಕೂಡ ಹಿಂದೆ ಅನುಭವವನ್ನು ಪಡೆದಿದ್ದಿರಬೇಕು. ಆಧುನಿಕ ಭೌತ ವಿಜ್ಞಾನವೂ ಇದನ್ನು ಒಪ್ಪುತ್ತದೆ. ಮಗುವು ಅನುಭವ ಸಮೇತವಾಗಿ ಬರುತ್ತದೆ ಎಂಬುದನ್ನು ಅದು ನಿಃಸಂದೇಹವಾಗಿ ಪ್ರಮಾಣೀಕರಿಸುತ್ತದೆ. ಹಿಂದಿನ ಕೆಲವು ತತ್ತ್ವಜ್ಞಾನಿಗಳು ಭಾವಿಸಿದಂತೆ ಖಾಲಿ ಮನಸ್ಸಿನೊಡನೆ ಪ್ರಪಂಚಕ್ಕೆ ಬರುವುದಿಲ್ಲ. ಜ್ಞಾನದ ಬುತ್ತಿಯೊಡನೆ ಸಿದ್ಧವಾಗಿಯೇ ಬರುತ್ತದೆ. ಇಲ್ಲಿಯವರೆಗೆ ಎಲ್ಲ ಸರಿಯಾಗಿಯೇ ಇದೆ.

ಮಗುವು ತನ್ನೊಡನೆ ತರುವ ಈ ಜ್ಞಾನಬುತ್ತಿಯು ಹಿಂದಿನ ಅನುಭವದಿಂದ ಬಂದದ್ದು ಎಂಬುದನ್ನು ಆಧುನಿಕ ವಿಜ್ಞಾನ ಒಪ್ಪುತ್ತದೆ. ಆದರೆ ಆ ಅನುಭವವು ಮಗು ವಿನದಲ್ಲ ಅದರ ತಂದೆ, ತಾತ, ಮುತ್ತಾತಂದಿರದು ಎಂದು ಅದು ಪ್ರತಿಪಾದಿಸುತ್ತದೆ. ಆನುವಂಶಿಕವಾಗಿ ಜ್ಞಾನವು ಬರುತ್ತದೆ ಎಂದು ವಿಜ್ಞಾನಿಗಳು ಹೇಳುತ್ತಾರೆ.

ಇದು ಹಳೆಯ “ಹುಟ್ಟುಗುಣ” ಸಿದ್ಧಾಂತಕ್ಕಿಂತ ಒಂದು ಹಂತ ಮೇಲಿನದಾಗಿದೆ. “ಹುಟ್ಟುಗುಣ” ಸಿದ್ಧಾಂತ ಮಕ್ಕಳಿಗೆ ಮತ್ತು ಮೂರ್ಖರಿಗೆ ಯೋಗ್ಯವಾದುದು, ಅದೊಂದು ಶಬ್ದಚಮತ್ಕಾರವಷ್ಟೆ - ಅದರಲ್ಲಿ ಯಾವ ಅರ್ಥವೂ ಇಲ್ಲ. ವಿಚಾರಶಕ್ತಿ ಯುಳ್ಳ, ತರ್ಕಬದ್ಧವಾಗಿ ಮಾತನಾಡಬಲ್ಲ ಯಾವನೂ ಸಹಜ ಪ್ರವೃತ್ತಿಗಳನ್ನು “ಹುಟ್ಟು ಗುಣ” ಎಂದು ವಿವರಿಸುವ ಸಾಹಸ ಮಾಡುವುದಿಲ್ಲ. ಹಾಗೆ ವಿವರಿಸಿದರೆ ಶೂನ್ಯದಿಂದ ಏನಾದರೂ ಉತ್ಪ ತ್ತಿಯಾಗುತ್ತದೆ ಎಂದಂತಾಗುತ್ತದೆ.

ಆದರೆ ಆನುವಂಶಿಕ ಸಿದ್ಧಾಂತವು, ಹಳೆಯದಕ್ಕಿಂತ ಮೇಲಾದರೂ, ಅಪರಿ ಪೂರ್ಣವಾದುದು. ಏಕೆ? ನಾವು ಭೌತಿಕ ಗುಣಗಳ ವರ್ಗಾವಣೆಯನ್ನು ತಿಳಿದುಕೊಳ್ಳಬಹುದು, ಆದರೆ ಮಾನಸಿಕ ವರ್ಗಾವಣೆಯನ್ನು ಅರ್ಥಮಾಡಿಕೊಳ್ಳಲಾರೆವು.

ನಾನೊಂದು ಜೀವಾತ್ಮ - ಯಾವುದು ನನ್ನನ್ನು ಒಬ್ಬ ತಂದೆಯ ಮೂಲಕ ಅವನ ಕೆಲವು ಗುಣಗಳ ಮೂಲಕ ಹುಟ್ಟುವಂತೆ ಮಾಡುತ್ತದೆ? ಯಾವುದು ನನ್ನನ್ನು ಮತ್ತೆ ಬರುವಂತೆ ಮಾಡುತ್ತದೆ? ಕೆಲವು ಗುಣಗಳನ್ನು ಹೊಂದಿರುವ ತಂದೆಯು ಒಂದು ಕಾರಣವಿರಬಹುದು. ಪೂರ್ವ ಅಸ್ತಿತ್ವವುಳ್ಳ ಮತ್ತು ಪುನಃ ಜನ್ಮವೆತ್ತಬೇಕಾಗಿರುವ ಒಂದು ವಿಶಿಷ್ಟ ಆತ್ಮವಾದ ನನ್ನನ್ನು ಒಬ್ಬ ವ್ಯಕ್ತಿಯ ದೇಹವನ್ನು ಪ್ರವೇಶಿಸುವಂತೆ ಮಾಡುವುದಾವುದು? ಇದಕ್ಕೆ ಸೂಕ್ತ ವಿವರಣೆಯನ್ನು ನೀಡಬೇಕಾದರೆ ಆನುವಂಶಿಕತೆ ಮತ್ತು ನನ್ನ ಹಿಂದಿನ ಅನುಭವ ಎರಡನ್ನೂ ಒಪ್ಪಿಕೊಳ್ಳಬೇಕಾಗುತ್ತದೆ. ಇದನ್ನೇ ಕರ್ಮ ಸಿದ್ಧಾಂತ ಎಂದು ಕರೆಯುವುದು.

ಉದಾಹರಣೆಗೆ, ನನ್ನ ಪೂರ್ವಕರ್ಮಗಳೆಲ್ಲವೂ ಮದ್ಯಸೇವನೆಗೆ ಸಂಬಂಧಿ ಸಿದ್ದರೆ ನಾನು ಸ್ವಾಭಾವಿಕವಾಗಿಯೇ ಮದ್ಯಸೇವನೆಯ ಲಕ್ಷಣಗಳಿರುವ ವ್ಯಕ್ತಿಗಳ ಕಡೆಗೇ ಆಕರ್ಷಿತನಾಗುತ್ತೇನೆ. ನನಗೆ ಯೋಗ್ಯವಾದ ಲಕ್ಷಣಗಳನ್ನು ವರ್ಗಾಯಿಸುವ ತಂದೆ ತಾಯಿಗಳಿಂದ ಉತ್ಪ ತ್ತಿಯಾಗುವ ದೇಹದ ಪ್ರಯೋಜನವನ್ನೇ ನಾನು ಪಡೆಯುತ್ತೇನೆ. ಕೆಲವು ಅನುಭವಗಳು ಆನುವಂಶಿಕವಾಗಿ ವರ್ಗಾಯಿಸಲ್ಪಡುವುದು ನಿಜ. ಇಂತಹ ವಿಶಿಷ್ಟ ಆನುವಂಶಿಕ ಲಕ್ಷಣಗಳೊಂದಿಗೆ ನನ್ನನ್ನು ಕೂಡಿಸುವುದು ನನ್ನ ಹಿಂದಿನ ಅನುಭವ.

ಆನುವಂಶಿಕ ಸಿದ್ಧಾಂತವು ಭೌತಿಕ ಮನುಷ್ಯನಿಗೆ ಮಾತ್ರ ಸಂಬಂಧಿಸಿದುದು, ಆಂತರಿಕ ಜೀವಾತ್ಮನ ವಿಷಯದಲ್ಲಿ ಅದು ಅಪೂರ್ಣವಾಗುತ್ತದೆ. ಮಾನವನಲ್ಲಿ ಜೀವಾತ್ಮವೆಂಬುದು ಇಲ್ಲವೇ ಇಲ್ಲ, ಮಾನವನೆಂದರೆ ಕೆಲವು ಭೌತಿಕ ಶಕ್ತಿಗಳಿಂದ ಕೆಲಸ ಮಾಡುವ ಪರಮಾಣುಗಳ ಮೊತ್ತ, ಎಂಬ ಭೌತಿಕ ದೃಷ್ಟಿಯನ್ನು ಇಟ್ಟುಕೊಂಡರೂ ಆನುವಂಶಿಕ ಸಿದ್ಧಾಂತವು ಪೂರ್ಣ ವಿವರಣೆಯನ್ನು ನೀಡುವುದಿಲ್ಲ.

ಆನುವಂಶಿಕ ಸಿದ್ಧಾಂತದ ಮುಖ್ಯ ತೊಂದರೆ ಏನೆಂದರೆ, ಮಾನವನಲ್ಲಿ ಜೀವಾತ್ಮ ನಿಲ್ಲದೆ ಇದ್ದು ಅವನು ಕೇವಲ ಪರಮಾಣುಗಳ ಮೊತ್ತವಾಗಿದ್ದರೆ, ತಂದೆಯ ಶಕ್ತಿ ಮಗುವಿಗೆ ವರ್ಗಾಯಿಸಲ್ಪಡುವುದರಿಂದ ಅವನಿಗೆ ಹೆಚ್ಚು ಮಕ್ಕಳಾದಂತೆಲ್ಲ ಅವನು ಸತ್ತ್ವಹೀನನಾಗುತ್ತ ಬರಬೇಕು; ಏಳು ಎಂಟು ಮಕ್ಕಳನ್ನು ಪಡೆದ ತಂದೆ ಕೊನೆಯಲ್ಲಿ ಕಡು ಮೂರ್ಖನಾಗಬೇಕಾಗುತ್ತದೆ. ಇಲಿಗಳಂತೆ ಸಂತಾನೋತ್ಪ ತ್ತಿಯಾಗುವ ಭಾರತ ಮತ್ತು ಚೈನಾದಲ್ಲಿ ಬರೀ ಮೂರ್ಖರೇ ತುಂಬಿರಬೇಕು. ಆದರೆ ಇದಕ್ಕೆ ವಿರುದ್ಧವಾಗಿ, ಭಾರತ ಮತ್ತು ಚೀನಾದಲ್ಲಿ ಹುಚ್ಚರ ಸಂಖ್ಯೆ ಅತ್ಯಂತ ಕಡಿಮೆ.

‘ಆನುವಂಶಿಕತೆ’ ಎಂಬ ಶಬ್ದದ ಅರ್ಥವೇನು ಎಂಬುದು ಪ್ರಶ್ನೆ. ಅದೊಂದು ದೊಡ್ಡ ಪದ. ಬೇರೆ ಎಷ್ಟೊ ಅಸಂಭವ ಅರ್ಥಹೀನ ಪದಗಳು ಬಳಕೆಯಲ್ಲಿರುವಂತೆ ಇದನ್ನೂ ಅದರ ಅರ್ಥ ತಿಳಿಯದೆ, ನಾವು ಬಳಸುತ್ತಿದ್ದೇವೆ. ಆನುವಂಶಿಕತೆ ಎಂದರೆ ಏನು ಎಂದು ನಾನು ನಿಮ್ಮನ್ನು ಕೇಳಿದರೆ ಅದರ ಬಗ್ಗೆ ನಿಮಗೆ ಯಾವ ಸ್ಪಷ್ಟ ಕಲ್ಪನೆಯೂ ಬರುವುದಿಲ್ಲ, ಏಕೆಂದರೆ ಅದಕ್ಕೆ ಸಂಬಂಧಿಸಿದಂತೆ ಯಾವ ಭಾವನೆಯೂ ಇಲ್ಲ.

ನಾವು ಈ ವಿಷಯವನ್ನು ಇನ್ನೂ ಕೂಲಂಕಷವಾಗಿ ಅಧ್ಯಯನ ಮಾಡೋಣ. ಉದಾಹರಣೆಗೆ ಇಲ್ಲಿ ಒಬ್ಬ ತಂದೆ ಇದ್ದಾನೆ. ಅವನಿಗೆ ಒಂದು ಮಗು ಹುಟ್ಟುತ್ತದೆ. ತಂದೆಯಲ್ಲಿರುವ ಗುಣಗಳು ಮಗುವನ್ನು ಪ್ರವೇಶಿಸುತ್ತವೆ. ಒಳ್ಳೆಯದು, ತಂದೆಯ ಗುಣಗಳು ಮಗುವಿಗೆ ಹೇಗೆ ಬಂದವು? ಯಾರಿಗೂ ಗೊತ್ತಿಲ್ಲ. ಈ ಅಂತರವನ್ನು ತುಂಬಲು ಆಧುನಿಕ ಭೌತಶಾಸ್ತ್ರಜ್ಞರು ಆನುವಂಶಿಕತೆ ಎಂಬ ದೊಡ್ಡ ಪದವನ್ನು ಕಂಡು ಹಿಡಿದರು. ಆನುವಂಶಿಕತೆ ಎಂದರೇನೆಂದು ಯಾರಿಗೂ ತಿಳಿಯದು.

ಅನುಭವದ ಮಾನಸಿಕ ಗುಣಗಳೆಲ್ಲ ಹೇಗೆ ಸಂಕ್ಷೇಪಗೊಂಡು ಜೀವದ್ರವದ ಒಂದು ಜೀವಕೋಶದಲ್ಲಿ ನೆಲಸುತ್ತಿದೆ? ಒಂದು ಹಕ್ಕಿಯ ಜೀವದ್ರವಕ್ಕೂ ಮಾನವ ಮಿದುಳಿನ ಜೀವದ್ರವಕ್ಕೂ ಯಾವ ವ್ಯತ್ಯಾಸವೂ ಇಲ್ಲ. ಭೌತಿಕ ಗುಣಗಳ ವರ್ಗಾ ವಣೆಯ ವಿಷಯದಲ್ಲಿ ನಾವು ಇಷ್ಟು ಮಾತ್ರ ಹೇಳಬಹುದು: ಎರಡು ಅಥವಾ ಮೂರು ಜೀವದ್ರವ ಕೋಶಗಳು ತಂದೆಯ ದೇಹದಿಂದ ಪ್ರತ್ಯೇಕಿಸಲ್ಪಡುತ್ತವೆ. ಆದರೆ ಯುಗ ಯುಗಾಂತರಗಳ ಮಾನವ ಅನುಭವಗಳೆಲ್ಲವೂ ಆ ಕೆಲವು ಜೀವದ್ರವ ಕೋಶಗಳಲ್ಲಿ ಅಡಕವಾಗಿವೆ ಎಂದು ಊಹಿಸುವುದು ಎಂಥ ಅರ್ಥಹೀನ! ಈ ಆನು ವಂಶಿಕತೆ ಎಂಬ ಪದದ ಮೂಲಕ ಭಯಂಕರ ಗುಳಿಗೆಯೊಂದನ್ನು ನಿಮಗೆ ನುಂಗಲು ಹೇಳುತ್ತಿರುವರು.

ಹಿಂದಿನ ಕಾಲದಲ್ಲಿ ಚರ್ಚುಗಳಿಗೆ ಗೌರವವಿತ್ತು. ಈಗ ವಿಜ್ಞಾನ ಆ ಸ್ಥಾನವನ್ನು ಪಡೆದಿದೆ. ಹಿಂದಿನ ಕಾಲದಲ್ಲಿ ಜನರು ಯಾವುದನ್ನೂ ವಿಚಾರಮಾಡಿ ನೋಡುತ್ತಿ ರಲಿಲ್ಲ, ಬೈಬಲ್ ಓದುತ್ತಿರಲಿಲ್ಲ. ಆದ್ದರಿಂದ ಪುರೋಹಿತರಿಗೆ ತಮಗಿಷ್ಟವಾದದ್ದನ್ನು ಬೋಧಿಸುವುದಕ್ಕೆ ಒಳ್ಳೆಯ ಅವಕಾಶವಿತ್ತು. ಹಾಗೆಯೇ ಈಗಿನ ಕಾಲದಲ್ಲಿಯೂ ಬಹುತೇಕ ಜನರು ವಿಚಾರ ಮಾಡುವುದಿಲ್ಲ ಮತ್ತು ವೈಜ್ಞಾನಿಕ ಎನಿಸಿಕೊಳ್ಳುವು ದೆಲ್ಲವನ್ನೂ ಭಯಮಿಶ್ರಿತ ಗೌರವದಿಂದ ಕಾಣುತ್ತಾರೆ. ಹಿಂದೆ ಚರ್ಚಿನಲ್ಲಿದ್ದುದಕ್ಕಿಂತಲೂ ಕೆಡುಕಾದ ಪೌರೋಹಿತ್ಯವು ಈಗ ಅಸ್ತಿತ್ವಕ್ಕೆ ಬರುತ್ತಿದೆ ಎಂಬುದನ್ನು ನೀವು ನೆನಪಿನಲ್ಲಿಡಬೇಕು. ಅದೇ ವೈಜ್ಞಾನಿಕ ಪೌರೋಹಿತ್ಯ - ಇದು ಎಷ್ಟು ಯಶಸ್ವಿಯಾಗಿದೆ ಯೆಂದರೆ ಧಾರ್ಮಿಕ ಪೌರೋಹಿತ್ಯಕ್ಕಿಂತ ಹೆಚ್ಚು ಅಧಿಕಾರಯುತವಾಗಿ ನಮ್ಮನ್ನು ಆಳುತ್ತದೆ.

ಈ ಆಧುನಿಕ ಪುರೋಹಿತರು ಬಹಳ ಶ್ರೇಷ್ಠರು ನಿಜ. ಆದರೆ ಯಾವ ಧಾರ್ಮಿಕ ಪುರೋಹಿತನಿಗಿಂತಲೂ ಹೆಚ್ಚಾಗಿ ಇವರು ಕೆಲವೊಮ್ಮೆ ಅದ್ಭುತ ವಿಷಯಗಳನ್ನು ನಂಬ ಬೇಕೆಂದು ನಮಗೆ ಹೇಳುತ್ತಾರೆ. ಅಂಥ ಒಂದು ಅದ್ಭುತ ವಿಷಯವೇ ಈ ಆನುವಂಶಿಕ ಸಿದ್ಧಾಂತ. ನನಗಿದುವರೆಗೂ ಇದು ಅರ್ಥವಾಗಿಲ್ಲ. ಆನುವಂಶಿಕತೆ ಎಂದರೇನೆಂದು ಕೇಳಿದರೆ “ತಂದೆಯ ಗುಣಗಳು ಜೀವದ್ರವ ಕೋಶಕ್ಕೆ ಅಂಟಿಕೊಳ್ಳುವುದು” ಎಂದು ಉತ್ತರಿಸುತ್ತಾರೆ. ಈ ವಿವರಣೆಗಳನ್ನೆಲ್ಲ ಕೇಳುತ್ತ ಹೋದಂತೆ ನನ್ನ ಮನಸ್ಸು ಗೊಂದಲಮಯವಾಗಿ ಇಡೀ ವಿಷಯವೇ ಜುಗುಪ್ಸೆಯನ್ನು ಹುಟ್ಟಿಸುತ್ತದೆ.

ನಾವು ಆಲೋಚನೆಯನ್ನು ಉತ್ಪ ತ್ತಿಮಾಡುತ್ತೇವೆ ಎಂಬುದು ಸತ್ಯ.ನಾನು ಈಗ ಮಾತನಾಡುತ್ತಿದ್ದೇನೆ, ಇದು ನಿಮ್ಮ ಮಿದುಳಿನಲ್ಲಿ ಆಲೋಚನೆಗಳನ್ನು ಏಳಿಸು ತ್ತಿದೆ. ಅಂದರೆ, ನನ್ನ ಆಲೋಚನೆಗಳು ನಿಮ್ಮ ಮಿದುಳು ಮತ್ತು ಮನಸ್ಸಿಗೆ ವರ್ಗಾಯಿ ಸಲ್ಪಟ್ಟು ಬೇರೆ ಆಲೋಚನೆಗಳನ್ನೂ ಉತ್ಪ ತ್ತಿಮಾಡುತ್ತಿದೆ. ಇದು ಪ್ರತಿನಿತ್ಯ ನಡೆಯುವ ಕ್ರಿಯೆ.

ನಾವು ಯಾವುದನ್ನು ಅರ್ಥಮಾಡಿಕೊಳ್ಳುತ್ತೇವೆಯೊ ಆ ನಿಲುವನ್ನು ಆಶ್ರಯಿಸುವುದು ಯಾವಾಗಲು ವೈಚಾರಿಕವಾದುದು. ಆಲೋಚನೆಯ ವರ್ಗಾವಣೆಯು ನಮಗೆ ಸ್ಪಷ್ಟವಾಗಿ ಅರ್ಥವಾಗುವ ವಿಷಯ. ಆದ್ದರಿಂದ ಆಲೋಚನೆಯ ವರ್ಗಾ ವಣೆಯ ಭಾವನೆಯನ್ನು ನಾವು ಸ್ವೀಕರಿಸಲು ಸಾಧ್ಯ, ಜೀವದ್ರವದ ಮೂಲಕ ಆನು ವಂಶಿಕ ಗುಣಗಳ ವರ್ಗಾವಣೆಯ ಭಾವನೆಯನ್ನು ಹಾಗೆ ಸ್ವೀಕರಿಸಲಾಗುವುದಿಲ್ಲ. ನಾವು ಆ ಸಿದ್ಧಾಂತವನ್ನು ಸಂಪೂರ್ಣವಾಗಿ ತೊಡೆದುಹಾಕಬೇಕಾಗಿಲ್ಲ, ಆದರೆ ಆಲೋಚನಾ ವರ್ಗಾವಣೆಗೆ ಹೆಚ್ಚು ಒತ್ತನ್ನು ನೀಡಬೇಕು.

ತಂದೆಯು ಆಲೋಚನೆಯನ್ನು ರವಾನಿಸುವುದಿಲ್ಲ. ಆಲೋಚನೆಯು ಮಾತ್ರ ಆಲೋಚನೆಯನ್ನು ಕಳಿಸಬಲ್ಲದು. ಹುಟ್ಟಿದ ಮಗುವು ಆಗಲೇ ಆಲೋಚನಾ ರೂಪದಲ್ಲಿ ಇತ್ತು. ನಾವೆಲ್ಲರೂ ಆಲೋಚನಾ ರೂಪದಲ್ಲಿ ಯಾವಾಗಲೂ ಇದ್ದೆವು ಮತ್ತು ಮುಂದೆಯೂ ಯಾವಾಗಲೂ ಇರುತ್ತೇವೆ.

ನಾವು ಆಲೋಚಿಸಿದಂತೆ ನಮ್ಮ ದೇಹ ಪರಿಣಮಿಸುತ್ತದೆ. ಪ್ರತಿಯೊಂದೂ ಆಲೋಚನೆಯಿಂದಲೇ ನಿರ್ಮಿತವಾಗಿರುವುದು. ಹೀಗೆ ನಾವೇ ನಮ್ಮ ಜೀವನದ ನಿರ್ಮಾತೃಗಳು. ನಾವು ಮಾಡುವುದಕ್ಕೆಲ್ಲ ನಾವೇ ಜವಾಬ್ದಾರರು. “ನಾನೇಕೆ ದುಃಖಿಯಾಗಿದ್ದೇನೆ” ಎಂದು ಕೂಗಾಡುವುದು ಮೂರ್ಖತನ. ಇದು ಭಗವಂತನ ದೋಷವೂ ಅಲ್ಲ.

ಒಬ್ಬನು ಸೂರ್ಯನ ಬೆಳಕಿನ ಸಹಾಯದಿಂದ ನಿಮ್ಮ ಮನೆಗೆ ಕನ್ನಹಾಕಿ ಎಲ್ಲ ಕದ್ದುಕೊಂಡು ಹೋಗುತ್ತಾನೆ. ಅವನೇ ಪೋಲಿಸರ ಕೈಗೆ ಸಿಕ್ಕಿಕೊಂಡಾಗ “ಎಲೈ ಸೂರ್ಯನೆ, ಏಕೆ ನೀನು ನನ್ನನ್ನು ಕದಿಯುವಂತೆ ಮಾಡಿದೆ?” ಎಂದು ಕೂಗಬಹುದು. ಅದು ಸೂರ್ಯನ ತಪ್ಪು ಅಲ್ಲವೇ ಅಲ್ಲ, ಏಕೆಂದರೆ ಸಾವಿರಾರು ಜನರು ಅದೇ ಸೂರ್ಯನ ಬೆಳಕಿನಿಂದ ಇತರರಿಗೆ ಎಷ್ಟೋ ಒಳ್ಳೆಯದನ್ನು ಮಾಡಿರುವರು. ಆ ಮನುಷ್ಯನಿಗೆ ಕದಿಯಲು ಸೂರ್ಯನು ಹೇಳಲಿಲ್ಲ.

ನಾವು ಏನು ಬಿತ್ತಿರುತ್ತೇವೆಯೊ ಅದರಂತೆ ಬೆಳೆಯನ್ನು ತೆಗೆಯುತ್ತೇವೆ. ನಾವು ಪಡುವ ದುಃಖಕ್ಕೆಲ್ಲ, ನಮ್ಮ ಬಂಧನಗಳಿಗೆಲ್ಲ ನಾವೇ ಕಾರಣ. ಈ ಪ್ರಪಂಚದಲ್ಲಿ ಯಾರನ್ನೂ ನಾವು ದೂಷಿಸುವಂತಿಲ್ಲ. ದೇವರನ್ನಂತೂ ದೂರುವಂತೆಯೇ ಇಲ್ಲ.

“ದೇವರು ಏಕೆ ಈ ದುಷ್ಟ ಜಗತ್ತನ್ನು ಸೃಷ್ಟಿಸಿದನು?” ಇದು ಅವನ ಕಾರ್ಯವಲ್ಲ, ನಾವೇ ಅದನ್ನು ಕೆಡಿಸಿರುವುದು, ನಾವೇ ಅದನ್ನು ಉತ್ತಮವನ್ನಾಗಿ ಮಾಡ ಬೇಕಾಗಿದೆ. “ದೇವರು ಏಕೆ ನನ್ನನ್ನು ಇಷ್ಟೊಂದು ದುಃಖಿಯನ್ನಾಗಿ ಮಾಡಿರುವನು?” ಅವನು ಹಾಗೆ ಮಾಡಲಿಲ್ಲ. ಎಲ್ಲರಿಗೆ ಕೊಡುವಂತೆ ಒಂದೇ ಶಕ್ತಿಯನ್ನು ನನಗೂ ಕೊಟ್ಟಿರುವನು. ನಾನೇ ಈ ಸ್ಥಿತಿಗೆ ಬಂದಿರುವೆನು.

ನಾನೇ ಮಾಡಿರುವುದಕ್ಕೆ ದೇವರನ್ನು ದೂರಬೇಕೆ? ಅವನ ದಯೆ ಯಾವಾಗಲೂ ಸಮಾನವಾಗಿರುತ್ತದೆ. ಅವನ ಬೆಳಕು ದುಷ್ಟ ಶಿಷ್ಟರ ಮೇಲೆ ಸಮಾನವಾಗಿ ಬೀಳುವುದು. ಅವನ ಗಾಳಿ, ಅವನ ನೀರು, ಅವನ ಭೂಮಿ ದುಷ್ಟ ಶಿಷ್ಟರಿಬ್ಬರಿಗೂ ಸಮಾನವಾದ ಅವಕಾಶವನ್ನು ನೀಡುವುವು. ದೇವರು ಯಾವಾಗಲೂ ದಯಾಮಯ ತಂದೆಯಾಗಿರುವನು. ನಮ್ಮ ಕರ್ಮದ ಫಲಗಳನ್ನು ಭೋಗಿಸುವುದಷ್ಟೇ ನಮ್ಮ ಪಾಲಿಗಿರುವುದು.

ಮೊದಲನೆಯದಾಗಿ, ನಾವು ಯಾವಾಗಲೂ ಇದ್ದೆವು, ಎರಡನೆಯದಾಗಿ ನಮ್ಮ ಅದೃಷ್ಟವನ್ನು ನಾವೇ ರೂಪಿಸುವವರು - ಎಂಬುದನ್ನು ತಿಳಿಯುತ್ತೇವೆ. ವಿಧಿ ಎಂಬುದು ಯಾವುದೂ ಇಲ್ಲ. ನಮ್ಮ ಜೀವನವು ಪೂರ್ವ ಕರ್ಮಗಳ ಫಲ. ನಮ್ಮ ಕರ್ಮಕ್ಕೆ ನಾವೇ ಜವಾಬ್ದಾರರಾದರೆ ಅದನ್ನು ನಿವಾರಿಸುವುದೂ ನಮ್ಮ ಕೈಯಲ್ಲೇ ಇದೆ.

ಈ ಕರ್ಮಬಂಧನದಿಂದ ಕಳಚಿಕೊಳ್ಳುವ ವಿಧಾನವನ್ನು ತೋರಿಸಿಕೊಡುವುದೇ ಒಟ್ಟು ಜ್ಞಾನಯೋಗದ ಸಾರ. ಒಂದು ಕಂಬಳಿಹುಳುವು ತನ್ನ ದೇಹದ ದ್ರವ್ಯದಿಂದಲೇ ತನ್ನಸುತ್ತ ಗೂಡನ್ನು ಹಣೆದುಕೊಳ್ಳುತ್ತದೆ, ಅನಂತರ ಅದರಲ್ಲೇ ಬದ್ಧವಾಗುತ್ತದೆ. ಅದು ಅಲ್ಲಿ ಎಷ್ಟೇ ಅಳಬಹುದು, ಕಿರಿಚಾಡಬಹುದು; ಯಾರೂ ಅದರ ಸಹಾಯಕ್ಕೆ ಬರುವುದಿಲ್ಲ. ತಾನೇ ಕಷ್ಟ ಪಟ್ಟು ಆ ಗೂಡನ್ನು ಸೀಳಿಕೊಂಡು ಹೊರಬರಬೇಕು. ಆಗ ಅದು ಸುಂದರ ಚಿಟ್ಟೆಯಾಗಿರುತ್ತದೆ. ಇದೇ ನಮ್ಮ ಬಂಧನಕ್ಕೂ ಅನ್ವಯವಾಗುತ್ತದೆ. ನಾವು ಯುಗ ಯುಗಾಂತರಗಳಿಂದಸುತ್ತಾಡುತ್ತಿದ್ದೇವೆ. ಈಗ ಈ ಬಂಧನದಿಂದ ದುಃಖಪಡುತ್ತಿದ್ದೇವೆ. ಆದರೆ ಸುಮ್ಮನೆ ದುಃಖಿಸಿದರೆ, ರೋದಿಸಿದರೆ ಏನೂ ಪ್ರಯೋಜನವಿಲ್ಲ. ಈ ಬಂಧನಗಳನ್ನು ಕತ್ತರಿಸಿಹಾಕಲು ನಾವು ಪ್ರಯತ್ನಿಸಬೇಕು.

ಬಂಧನದ ಮುಖ್ಯ ಕಾರಣವೇ ಅಜ್ಞಾನ. ಮಾನವನು ಸ್ವಭಾವತಃ ದುಷ್ಟನಲ್ಲ. ಪವಿತ್ರತೆಯೇ ಅವನ ಸ್ವರೂಪ. ಪ್ರತಿಯೊಬ್ಬನೂ ದಿವ್ಯಸ್ವಭಾವದವನು. ನೀವು ನೋಡುವ ಪ್ರತಿಯೊಬ್ಬ ವ್ಯಕ್ತಿಯೂ ಸ್ವರೂಪತಃ ಭಗವಂತನೇ. ಈ ಸ್ವರೂಪವು ಅಜ್ಞಾನದಿಂದ ಆವೃತವಾಗಿದೆ, ಈ ಅಜ್ಞಾನವೇ ನಮ್ಮನ್ನು ಬಂಧಿಸಿರುವುದು. ಎಲ್ಲ ದುಷ್ಟತನಕ್ಕೂ ಕಾರಣ ಅಜ್ಞಾನ. ಜ್ಞಾನವು ಜಗತ್ತನ್ನು ಒಳ್ಳೆಯದನ್ನಾಗಿ ಮಾಡುತ್ತದೆ.

ಜ್ಞಾನವು ಎಲ್ಲ ದುಃಖವನ್ನು ನಿವಾರಿಸುತ್ತದೆ. ಜ್ಞಾನವು ನಮ್ಮನ್ನು ಮುಕ್ತರನ್ನಾಗಿ ಮಾಡುತ್ತದೆ. ಇದು ಜ್ಞಾನಯೋಗದ ಭಾವನೆ: ಜ್ಞಾನವೇ ನಮ್ಮನ್ನು ಮುಕ್ತರನ್ನಾಗಿಸುವುದು. ಯಾವ ಜ್ಞಾನ? ರಸಾಯನಶಾಸ್ತ್ರ, ಭೌತಶಾಸ್ತ್ರ, ಖಗೋಳಶಾಸ್ತ್ರ ಮುಂತಾದ ಜ್ಞಾನವೆ? ಅವುಗಳ ಸಹಾಯ ಅತ್ಯಲ್ಪ. ಆದರೆ ಮುಖ್ಯವಾದ ಜ್ಞಾನ ನಮ್ಮ ಸ್ವಭಾವವನ್ನು ಕುರಿತಾದದ್ದು. “ನಿನ್ನನ್ನು ಅರಿತುಕೊ”. ನೀವು ಯಾರು, ನಿಮ್ಮ ನಿಜವಾದ ಸ್ವರೂಪ ವೇನು ಎಂಬುದನ್ನು ನೀವು ಅರಿಯಬೇಕು. ನಿಮ್ಮೊಳಗಿರುವ ಅನಂತ ಸ್ವಭಾವದ ಅರಿವು ನಿಮಗಿರಬೇಕು. ಆಗ ನಿಮ್ಮ ಬಂಧನಗಳು ಕಳಚುತ್ತವೆ.

ಹೊರಗಿನದನ್ನೇ ಅಧ್ಯಯನ ಮಾಡುತ್ತ ಮಾನವನು ತಾನು ಬರಿ ಶೂನ್ಯ ಎಂದು ಭಾವಿಸತೊಡಗುತ್ತಾನೆ. ಈ ಅಪಾರ ಪ್ರಾಕೃತಿಕ ಶಕ್ತಿ, ಈ ಅದ್ಭುತ ಪರಿಣಾಮಗಳು, ಇಡೀ ಮಾನವ ಸಮೂಹವೇ ಕ್ಷಣಾರ್ಧದಲ್ಲಿ ಅಳಸಿಹೋಗಬಹುದು; ಒಂದು ಅಗ್ನಿಪರ್ವತ ಸಿಡಿದು ಇಡೀ ಖಂಡವನ್ನೇ ಛಿದ್ರಛಿದ್ರವಾಗಿ ಮಾಡಬಹುದು. ಇವುಗಳನ್ನು ನೋಡುತ್ತ, ಅಧ್ಯಯನ ಮಾಡುತ್ತ ಮಾನವನು ತಾನೆಷ್ಟು ದುರ್ಬಲ ಎಂದು ಭಾವಿಸತೊಡಗುತ್ತಾನೆ. ಆದ್ದರಿಂದ ಬಾಹ್ಯ ಪ್ರಕೃತಿಯ ಅಧ್ಯಯನವು ವ್ಯಕ್ತಿಯನ್ನು ಬಲಶಾಲಿಯನ್ನಾಗಿ ಮಾಡುವುದಿಲ್ಲ. ಆದರೆ ಮಾನವನ ಆಂತರಿಕ ಸ್ವಭಾವ ವೊಂದಿದೆ. ಅದು ಅಗ್ನಿಪರ್ವತ ಸ್ಫೋಟಕ್ಕಿಂತ, ಯಾವುದೇ ಪ್ರಕೃತಿ ನಿಯಮಕ್ಕಿಂತ ಮಿಲಿಯಗಟ್ಟಲೆ ಹೆಚ್ಚು ಶಕ್ತಿಯುತವಾದುದು. ಅದು ಪ್ರಕೃತಿಯನ್ನೇ ಗೆಲ್ಲುತ್ತದೆ, ಎಲ್ಲ ನಿಯಮಗಳನ್ನೂ ಜಯಿಸುತ್ತದೆ. ಅದು ಮಾತ್ರ ಮಾನವನಿಗೆ ಅವನೇನಾಗಿರುವನೊ ಅದನ್ನು ತಿಳಿಸಿಕೊಡುತ್ತದೆ.

“ಜ್ಞಾನವೇ ಶಕ್ತಿ” ಎಂಬ ಮಾತಿದೆ ಅಲ್ಲವೆ? ಜ್ಞಾನದ ಮೂಲಕವೇ ಶಕ್ತಿ ಬರುವುದು. ಮನುಷ್ಯನು ಜ್ಞಾನವನ್ನು ಪಡೆಯಬೇಕು. ಅನಂತ ಶಕ್ತಿಯನ್ನು ಪಡೆ ದಿರುವ ಒಬ್ಬ ವ್ಯಕ್ತಿಯಿದ್ದಾನೆ ಎಂದಿಟ್ಟುಕೊಳ್ಳಿ. ಅವನು ಸ್ವಭಾವತಃ ಸರ್ವಜ್ಞನೂ ಸರ್ವಶಕ್ತನೂ ಆಗಿರುತ್ತಾನೆ. ಇದನ್ನು ಅವನು ತಿಳಿಯಬೇಕು. ತನ್ನಾತ್ಮದ ಅರಿವು ಅವನಿಗೆ ಹೆಚ್ಚು ಆದಂತೆಲ್ಲ, ಅವನು ಹೆಚ್ಚು ಶಕ್ತಿಯನ್ನು ವ್ಯಕ್ತಗೊಳಿಸುತ್ತಾನೆ, ಮತ್ತು ಅವನ ಬಂಧನಗಳು ಕಳಚಿ, ಕೊನೆಯಲ್ಲಿ ಅವನು ಮುಕ್ತನಾಗುತ್ತಾನೆ.

ನಮ್ಮನ್ನು ನಾವು ಅರಿಯುವುದು ಹೇಗೆ? ಇದೇ ಈಗಿರುವ ಪ್ರಶ್ನೆ. ಆತ್ಮನನ್ನು ಅರಿ ಯುವ ಅನೇಕ ಮಾರ್ಗಗಳಿವೆ. ಆದರೆ ಜ್ಞಾನಯೋಗದ ವಿಧಾನ ಬೌದ್ಧಿಕ ವಿಚಾರವೊಂದೇ. ವಿಚಾರವೇ ಆಧ್ಯಾತ್ಮಿಕ ಒಳನೋಟವಾಗಿ ವಿಕಾಸಹೊಂದಿ ನಾವೇನಾಗಿರು ವೆವೆಂಬುದನ್ನು ತೋರಿಸುತ್ತದೆ.

ಇಲ್ಲಿ ನಂಬಿಕೆಯ ಪ್ರಶ್ನೆಯೇ ಇಲ್ಲ. ಯಾವುದನ್ನೂ ನಂಬಕೂಡದು - ಇದೇ ಜ್ಞಾನಿಯ ಭಾವನೆ. ಯಾವುದನ್ನು ನಂಬಬೇಡಿ, ಎಲ್ಲವನ್ನೂ ಸಂದೇಹಿಸಿ - ಇದೇ ಮೊದಲನೆಯ ಹೆಜ್ಜೆ. ವಿಚಾರವಾದಿಯಾಗುವ ಸಾಹಸ ಮಾಡಿ. ವಿಚಾರವು ಎಲ್ಲಿಗೆ ಕರೆದುಕೊಂಡು ಹೋದರೂ ಅದನ್ನು ಅನುಸರಿಸುವ ಸಾಹಸ ಮಾಡಿ.

“ನಾನು ವಿಚಾರವಾದಿ” ಎಂದು ಜನರು ಹೇಳುವುದನ್ನು ನಾವುಸುತ್ತಲೂ ಕೇಳುತ್ತೇವೆ. ಆದರೆ ವಿಚಾರವಾದಿಯಾಗಿರುವುದು ಅಷ್ಟು ಸುಲಭವಲ್ಲ. ವಿಚಾರ ವಂತನಾಗಿ ವಿಚಾರವನ್ನೇ ಅನುಸರಿಸುವ ವ್ಯಕ್ತಿಯ ಮುಖವನ್ನು ನೋಡಲು ಏಷ್ಟೇ ದೂರವಾದರೂ ಹೋಗುತ್ತೇನೆ. ಹೇಳುವಷ್ಟು ಸುಲಭ, ಮಾಡುವಷ್ಟು ಕಷ್ಟ ಯಾವುದೂ ಇಲ್ಲ. ನಾವು ಮೂಢನಂಬಿಕೆಗಳನ್ನು ಅನುಸರಿಸಲೇಬೇಕಾಗಿದೆ. ಹಳೆಯ ಪುರಾತನ ಮೂಢನಂಬಿಕೆಗಳು; ರಾಷ್ಟ್ರೀಯ ಅಥವಾ ಒಟ್ಟು ಮಾನವಕೋಟಿಗೆ ಸಂಬಂಧಿಸಿದ ಮೂಢನಂಬಿಕೆಗಳು; ಕುಟುಂಬ, ಸ್ನೇಹಿತರು, ಗ್ರಂಥ, ಭಾಷೆ, ಲಿಂಗಭೇದ ಮುಂತಾದುವುಗಳಿಗೆ ಸಂಬಂಧಿಸಿದ ಹತ್ತು ಹಲವು ಮೂಢನಂಬಿಕೆಗಳನ್ನು ನಾವು ಅನುಸರಿಸುತ್ತೇವೆ.

ಆದರೂ ವಿಚಾರದ ಬಗ್ಗೆ ಮಾತನಾಡುತ್ತೇವೆ. ನಿಜವಾಗಿಯೂ ಕೆಲವೇ ವ್ಯಕ್ತಿಗಳು ವಿಚಾರ ಮಾಡಬಲ್ಲರು. “ನಾನು ಯಾವುದನ್ನೂ ನಂಬಲು ಇಷ್ಟಪಡುವುದಿಲ್ಲ. ಕತ್ತಲೆಯಲ್ಲಿ ತಡಕಾಡುವುದು ನನಗಿಷ್ಟವಿಲ್ಲ. ನಾನು ವಿಚಾರ ಮಾಡಬೇಕು” ಎಂದು ವ್ಯಕ್ತಿಯು ಹೇಳುವುದನ್ನು ನೀವು ಕೇಳಿರುವಿರಿ. ಅಂತೂ ಅವನು ವಿಚಾರಮಾಡುತ್ತಾನೆ. ಆದರೆ ತಾನು ಅತ್ಯಂತ ಪ್ರೀತಿಯಿಂದ ಅಪ್ಪಿಕೊಂಡಿರುವ ವಿಷಯಗಳನ್ನು ವಿಚಾರವು ಪುಡಿ ಪುಡಿ ಮಾಡಲು ಪ್ರಾರಂಭಿಸಿದಾಗ, “ಬೇಡ, ಇಷ್ಟೇ ಸಾಕು! ವಿಚಾರವು ಎಲ್ಲಿಯವರೆಗೆ ನನ್ನ ಆದರ್ಶಗಳನ್ನು ತುಂಡರಿಸುವುದಿಲ್ಲವೊ ಅಲ್ಲಿಯವರೆಗೆ ಸರಿ. ಅಲ್ಲಿಯೇ ಅದನ್ನೂ ನಿಲ್ಲಿಸಬೇಕು” ಎನ್ನುತ್ತಾನೆ. ಆ ವ್ಯಕ್ತಿ ಖಂಡಿತ ಜ್ಞಾನಿಯಾಗಲಾರ,. ಅವನು ಜೀವನದುದ್ದಕ್ಕೂ, ಮುಂದಿನ ಜೀವನದಲ್ಲಿಯೂ ಬಂಧನವನ್ನು ಹೊತ್ತುಕೊಂಡು ಹೋಗುತ್ತಾನೆ. ಮತ್ತೆ ಮತ್ತೆ ಮೃತ್ಯುವಶನಾಗುತ್ತಾನೆ. ಇಂಥ ವ್ಯಕ್ತಿಗಳು ಜ್ಞಾನಕ್ಕೆ ಯೋಗ್ಯರಲ್ಲ. ಅವರಿಗೆ ಭಕ್ತಿಯೋಗ, ಕರ್ಮಯೋಗ, ರಾಜಯೋಗ ಮುಂತಾದ ಬೇರೆ ಮಾರ್ಗಗಳಿವೆ. ಜ್ಞಾನಯೋಗ ಅವರಿಗಲ್ಲ.

ಅತ್ಯಂತ ಧೈರ್ಯಶಾಲಿಯು ಮಾತ್ರ ಈ ಮಾರ್ಗವನ್ನು ಅನುಸರಿಸಲು ಸಾಧ್ಯ. ಯಾವ ಚರ್ಚನ್ನೂ ನಂಬದ, ಯಾವ ಮತಕ್ಕೂ ಸೇರಿರದ ಅಥವಾ ತಾನು ಯಾವುದನ್ನೂ ನಂಬುವುದಿಲ್ಲ ಎಂದು ಹೆಮ್ಮೆಕೊಚ್ಚಿಕೊಳ್ಳುವ ವ್ಯಕ್ತಿಯು ವಿಚಾರವಾದಿ ಯೆಂದು ಭಾವಿಸಬೇಡಿ. ಅದು ಖಂಡಿತ ಸುಳ್ಳು. ಆಧುನಿಕ ಕಾಲದಲ್ಲಿ ಇದೊಂದು ಫ್ಯಾಶನ್ ಆಗಿದೆ.

ಯಾವುದನ್ನೂ ನಂಬದೆ ಇದ್ದರೆ ಮಾತ್ರ ವಿಚಾರವಾದಿಯಾಗುವುದಿಲ್ಲ. ವಿಚಾರ ಮಾಡುವುದು ಮಾತ್ರವಲ್ಲದೆ, ವಿಚಾರವು ಹೇಳಿದಂತೆ ನಡೆದುಕೊಳ್ಳಬೇಕು. ಈ ದೇಹವು ಒಂದು ಭ್ರಮೆ ಎಂದು ವಿಚಾರವು ಹೇಳಿದರೆ ಅದನ್ನು ನೀವು ತ್ಯಜಿಸಲು ಸಿದ್ಧರಿರು ವಿರಾ? ಶೀತೋಷ್ಣಗಳು ಇಂದ್ರಿಯಗಳ ಭ್ರಮೆಯೆಂದು ವಿಚಾರವು ಹೇಳುತ್ತದೆ, ನಿಮಗೆ ಅವನ್ನು ಸಹಿಸಿಕೊಳ್ಳುವ ಶಕ್ತಿಯಿದೆಯೆ? ಇಂದ್ರಿಯ ಪ್ರತ್ಯಕ್ಷವೆಲ್ಲವೂ, ಅಸತ್ಯವೆಂದು ವಿಚಾರವು ಹೇಳಿದರೆ ಇಂದ್ರಿಯಾನುಭವವನ್ನೇ ಅಲ್ಲಗೆಳೆಯಲು ಸಿದ್ಧರಿರುವಿರಾ? ನಿಮಗೆ ಆ ಧೈರ್ಯವಿದ್ದರೆ ನೀವು ವಿಚಾರವಾದಿಗಳು.

ವಿಚಾರದಲ್ಲಿ ನಂಬಿಕೆಯಿರಿಸಿ ಸತ್ಯವನ್ನು ಅನುಸರಿಸುವುದು ಅತ್ಯಂತ ಕಠಿಣ. ಈ ಪ್ರಪಂಚವು ಮೂಢನಂಬಿಕೆಯ ಅಥವಾ ಅರೆ ಮನಸ್ಸಿನ ಡಾಂಭಿಕರಿಂದ ತುಂಬಿ ಹೋಗಿದೆ. ಅರೆಮನಸ್ಸಿನ ಡಾಂಭಿಕರಿಗಿಂತ ಮೂಢನಂಬಿಕೆ ಮತ್ತು ಅಜ್ಞಾನಗಳೇ ಮೇಲು. ಡಾಂಭಿಕರು ಯೋಗ್ಯರಲ್ಲ, ಅವರು ನದಿಯ ಎರಡೂ ದಡಗಳಲ್ಲಿ ಕಾಲಿಟ್ಟುಕೊಂಡಿರುತ್ತಾರೆ.

ಯಾವುದಾದರೊಂದನ್ನು ತೆಗೆದುಕೊಳ್ಳಿ, ನಿಮ್ಮ ಆದರ್ಶವನ್ನು ನಿಶ್ಚಯಿಸಿಕೊಳ್ಳಿ ಮತ್ತು ಸಾಯುವವರೆಗೂ ಧೈರ್ಯದಿಂದ ಅದನ್ನು ಅನುಸರಿಸಿ. ಅದೇ ಮುಕ್ತಿಗೆ ಹಾದಿ. ಅರೆಮನಸ್ಸಿನಿಂದ ಏನನ್ನೂ ಸಾಧಿಸಲಾಗುವುದಿಲ್ಲ. ಮೂಢನಂಬಿಕೆಯವರಾಗಿ, ನಿಮಗಿಷ್ಟವಾದರೆ ಮತಾಂಧರೂ ಆಗಿ, ಆದರೆ ಏನಾದರೂ ಆಗಿರಿ. ನಿಮ್ಮಲ್ಲಿ ಸತ್ತ್ವವಿದೆ ಎಂಬುದನ್ನು ತೋರಿಸಿ. ಸುಮ್ಮನೆ ಡೋಲಾಯಮಾನ ಮನಸ್ಸಿನಿಂದ ಕೂಡಿರ ಬೇಡಿ. ಎಲ್ಲದರಲ್ಲೂ ಮನಸ್ಸನ್ನಿರಿಸುವ ಅವರಿಗೆ ಬೇಕಾಗಿರುವುದು ಕೇವಲ ನರಗಳ ಕಚಗುಳಿ ಅಷ್ಟೆ. ಅದೊಂದು ರೀತಿಯ ಅಮಲು. ಮುಂದೆ ಈ ಭಾವೋದ್ರೇಕವೇ ಒಂದು ಅಭ್ಯಾಸವಾಗುತ್ತದೆ.

ಪ್ರಪಂಚವು ಇಂತಹ ವ್ಯಕ್ತಿಗಳಿಂದ ತುಂಬಿಹೋಗುತ್ತಿದೆ. ಕ್ರಿಸ್ತನ ಶಿಷ್ಯರು ಭೂಮಿಯ ಲವಣ (ಸಾರ)ವಾಗಿದ್ದರೆ \enginline{(Salt of the earth),} ಈ ಜನರು ಭೂಮಿಯ ಬೂದಿ ಅಥವಾ ಕಲ್ಮಶವಾಗಿದ್ದಾರೆ. ಆದರಿಂದ ಸ್ಪಷ್ಟವಾಗಿ ಆಲೋಚಿಸಿ ವಿಚಾರವನ್ನು ಅನುಸರಿಸುವುದು ಏನೆಂಬುದನ್ನು ನೋಡೋಣ. ಅನಂತರ ವಿಚಾರವನ್ನು ಅನುಸರಿಸುವುದಕ್ಕೆ ಇರುವ ಅಡಚಣೆಗಳನ್ನು ಅರ್ಥಮಾಡಿಕೊಳ್ಳೋಣ.

ಮೊದಲನೆಯ ಅಡಚಣೆಯೇ, ಸತ್ಯವನ್ನು ತಲುಪಲು ನಮಗಿರುವ ಅನಿಚ್ಛೆ. ಸತ್ಯ ನಮ್ಮೆಡೆಗೆ ಬರಬೇಕೆಂದು ನಾವು ಬಯಸುತ್ತೇವೆ. ನನ್ನ ಪ್ರಯಾಣದ ಸಮಯದಲ್ಲಿ ಅನೇಕರು ನನಗೆ ಹೇಳಿರುವರು: “ನೀವು ಹೇಳುವ ಧರ್ಮ ಅನುಕೂಲಕರವಾದು ದಲ್ಲ. ಅನುಕೂಲಕರವಾದ ಧರ್ಮವನ್ನು ನಮಗೆ ಕೊಡಿ,” ಎಂದು. “ಅನುಕೂಲಕರ ಧರ್ಮ” ಎಂದರೆ ಏನೊ ನನಗೆ ಅರ್ಥವಾಗುವುದಿಲ್ಲ. ನನಗೆ ಯಾರೂ ಅನುಕೂಲ ಕರ ಧರ್ಮವನ್ನು ಎಂದೂ ಬೋಧಿಸಲಿಲ್ಲ. ನಾನು ಬಯಸುವುದು ಸತ್ಯವನ್ನು - ಅದು ಅನುಕೂಲವೊ ಅನಾನುಕೂಲವೊ ಮುಖ್ಯವಲ್ಲ. ಸತ್ಯವು ಯಾವಾಗಲೂ ಅನು ಕೂಲವಾಗಿರಬೇಕೆ? ಸತ್ಯವು ಅನೇಕ ವೇಳೆ ಆಘಾತವನ್ನು ನೀಡುತ್ತದೆ ಎಂಬುದು ನಮ್ಮ ಅನುಭವಕ್ಕೆ ಬಂದಿದೆ. ಅಂತಹ ವ್ಯಕ್ತಿಗಳೊಡನೆ ಬಹಳ ಮಾತನಾಡಿದ ಮೇಲೆ ಅವರ ಅಭಿಪ್ರಾಯವೇನೆಂಬುದು ನನಗೆ ಗೊತ್ತಾಗಿದೆ. ಈ ಜನರು ಒಂದು ರೂಢಿಗೆ ಬದ್ಧರಾಗಿದ್ದಾರೆ, ಅದರಿಂದ ಹೊರಬರುವ ಧೈರ್ಯ ಅವರಿಗಿಲ್ಲ. ಸತ್ಯವೇ ಅವರ ಭಾವನೆ ಗನುಗುಣವಾಗಿ ಹೊಂದಿಕೊಳ್ಳ ಬೇಕಾಗಿದೆ.

ನಾನು ಒಮ್ಮೆ ಓರ್ವ ಮಹಿಳೆಯನ್ನು ಸಂಧಿಸಿದೆ. ಆಕೆಗೆ ತನ್ನ ಮಕ್ಕಳು, ಮನೆ, ಹಣ ಎಲ್ಲದರ ಮೇಲೂ ಪ್ರೀತಿ. ದೇವರನ್ನು ಪಡೆಯುವುದಕ್ಕೆ ಒಂದೇ ಒಂದು ಮಾರ್ಗವೆಂದರೆ ಎಲ್ಲವನ್ನೂ ತ್ಯಜಿಸುವುದು - ಎಂದು ನಾನು ಬೋಧಿಸಿದುದನ್ನು ಕೇಳಿ ಅವಳು ಮರುದಿನವೇ ಬರುವುದನ್ನು ನಿಲ್ಲಿಸಿದಳು. ಮುಂದೊಂದು ದಿನ ಅವಳು ಬಂದು, ನಾನು ಬೋಧಿಸುವ ಧರ್ಮ ತುಂಬ ಅನಾನುಕೂಲವಾಗಿರುವುದೇ ತಾನು ಬರದೇ ಇರುವುದಕ್ಕೆ ಕಾರಣ ಎಂದು ತಿಳಿಸಿದಳು. “ನಿನಗೆ ಅನುಕೂಲಕರವಾದ ಧರ್ಮ ಯಾವುದು?” ಎಂದು ನಾನು ಕೇಳಿದೆ. ಅದಕ್ಕವಳು, “ನಾನು ನನ್ನ ಮಕ್ಕಳಲ್ಲಿ, ನನ್ನ ಹಣದಲ್ಲಿ, ನನ್ನ ವಜ್ರದಲ್ಲಿ ದೇವರನ್ನು ಕಾಣಲು ಇಷ್ಟಪಡುತ್ತೇನೆ” ಎಂದು ಉತ್ತರಿಸಿದಳು.

ನಾನು ಹೇಳಿದೆ: “ಬಹಳ ಒಳ್ಳೆಯದು. ಈಗ ಇವೆಲ್ಲವೂ ನಿನ್ನೊಡನೆ ಇವೆ. ಸಾವಿರಾರು ವರ್ಷಗಳ ಪರ್ಯಂತವೂ ನೀನು ಇವುಗಳನ್ನು ನೋಡುತ್ತಿರಬೇಕು. ಅನಂತರ ಏನೋ ಒಂದು ಆಘಾತಕ್ಕೊಳಗಾಗಿ ನಿನ್ನಲ್ಲಿ ವಿವೇಕೋದಯವಾಗುತ್ತದೆ. ಆ ಸಮಯ ಬರುವವರೆಗೂ ನೀನು ದೇವರನ್ನು ಪಡೆಯಲಾಗುವುದಿಲ್ಲ. ಅಲ್ಲಿಯವರೆಗೂ ನಿನ್ನ ಮಕ್ಕಳಲ್ಲಿ, ನಿನ್ನ ಐಶ್ವರ್ಯದಲ್ಲಿ, ನಿನ್ನ ನೃತ್ಯಗಳಲ್ಲಿ ನೀನು ದೇವರನ್ನು ನೋಡುತ್ತಿರು.”

ಇಂತಹ ವ್ಯಕ್ತಿಗಳಿಗೆ ಇಂದ್ರಿಯ ಸುಖವನ್ನು ತ್ಯಜಿಸುವುದು ಬಹಳ ಕಷ್ಟ. ಅಸಾಧ್ಯ ವೆಂದೇ ಹೇಳಬಹುದು. ಅನೇಕ ಜನ್ಮಗಳಿಂದ ಅದು ಅವರಲ್ಲಿ ಬಲವಾಗಿ ಬೇರು ಬಿಟ್ಟಿದೆ. ಹಂದಿಗೆ ತನ್ನ ವಾಸಸ್ಥಾನವನ್ನು ಬಿಟ್ಟು ಸುಂದರ ಭವನದಲ್ಲಿ ಇರಲು ಹೇಳಿದರೆ ಅದು ಪ್ರಾಣಬಿಡುತ್ತದೆ. “ಬೇಡ, ನಾನು ಅಲ್ಲೇ ಇರುತ್ತೇನೆ” ಎಂದು ಅದು ಆರಚುತ್ತದೆ.

(ಇಲ್ಲಿ ಸ್ವಾಮಿ ವಿವೇಕಾನಂದರು ಮೀನು ಮಾರುವವಳ ಕಥೆಯನ್ನು ಹೇಳಿದರು: “ಒಮ್ಮೆ ಮೀನು ಮಾರುವವಳೊಬ್ಬಳು ಹೂಮಾರುವವಳ ಮನೆಗೆ ಅತಿಥಿಯಾಗಿ ಹೋಗುತ್ತಾಳೆ. ಮೀನು ಮಾರಿದನಂತರ ಖಾಲಿ ಬುಟ್ಟಿಯೊಡನೆ ಅವಳು ಅಲ್ಲಿಗೆ ಹೋಗಿದ್ದಳು. ಹೂವುಗಳಿರುವ ಕೋಣೆಯಲ್ಲಿ ಅವಳಿಗೆ ಮಲಗುವ ವ್ಯವಸ್ಥೆ ಮಾಡ ಲಾಯಿತು. ಆದರೆ ಹೂವಿನ ಪರಿಮಳದಿಂದಾಗಿ ಅವಳಿಗೆ ಎಷ್ಟು ಹೊತ್ತಾದರೂ ನಿದ್ರೆಯೇ ಬರಲಿಲ್ಲ. ಹೂವಿನವಳು ಅವಳ ಸ್ಥಿತಿಯನ್ನು ನೋಡಿ, ‘ಏಕೆ ಆ ಕಡೆ ಈ ಕಡೆ ಹೊರಳಾಡುತ್ತಿರುಯವೆಯಲ್ಲ?’ ಎಂದು ಕೇಳಿದಳು. ಮೀನಿನ ಹೆಂಗಸು ಹೇಳಿದಳು, ‘ಏಕೊ ಗೊತ್ತಿಲ್ಲ, ಬಹುಶಃ ಹೂವಿನ ವಾಸನೆ ತೊಂದರೆ ಕೊಡುತ್ತಿರಬಹುದು. ಆ ಮೀನಿನ ಬುಟ್ಟಿಯನ್ನು ತಂದುಕೊಡುತ್ತಿಯಾ? ಅದರಿಂದ ನಿದ್ರೆ ಬರಬಹುದು?”)

ನಮ್ಮ ಪಾಡೂ ಇದೇ. ಬಹುಪಾಲು ಜನರಿಗೆ ಮೀನಿನ ವಾಸನೆಯೇ ಇಷ್ಟ. ಈ ಪ್ರಪಂಚ, ಇಂದ್ರಿಯ ಸುಖ, ಐಶ್ವರ್ಯ, ಮಡದಿಮಕ್ಕಳು - ಇವೇ ಮೀನುವಾಸನೆ. ಇದು ನಮ್ಮನ್ನು ಆಕ್ರಮಿಸಿಕೊಂಡುಬಿಟ್ಟಿದೆ. ಇದನ್ನು ಮೀರಿದುದನ್ನು ನಾವು ಕೇಳಲಾರೆವು, ನೋಡಲಾರೆವು; ಯಾವುದೂ ಇದನ್ನು ಮೀರಿ ಹೋಗುವುದೇ ಇಲ್ಲ. ಇದೇ ಪ್ರಪಂಚ.

ದೇವರು, ಸ್ವರ್ಗ, ಆತ್ಮ ಈ ಸಂಬಂಧವಾದ ಮಾತುಗಳೆಲ್ಲ ಸಾಮಾನ್ಯ ಮನುಷ್ಯನಿಗೆ ಏನೂ ಅರ್ಥವಾಗುವುದಿಲ್ಲ. ಅವನಿಗೆ ಇಲ್ಲೇ ಸ್ವರ್ಗವಿದೆ. ಈ ಜಗತ್ತನ್ನು ಮೀರಿದು ದಾವುದೂ ಅವನ ಕಲ್ಪನೆಗೆ ನಿಲುಕದು. ಉನ್ನತವಾದ ವಿಷಯವನ್ನು ಅವನಿಗೆ ಹೇಳಿದಾಗ, “ಅದು ಅನುಕೂಲಕರ ಧರ್ಮವಲ್ಲ. ನಮಗೆ ಅನುಕೂಲಕರವಾದುದನ್ನು ಕೊಡಿ” ಎಂದು ಹೇಳುತ್ತಾನೆ. ಆ ಧರ್ಮ ಯಾವುದೆಂದರೆ ಅವನು ಈಗ ಏನು ಮಾಡುತ್ತಿರುವನೊ ಅದಕ್ಕೆ ಸಂಬಂಧಿಸಿದುದು.

ಆತನು ಒಬ್ಬ ಕಳ್ಳನಾಗಿದ್ದರೆ, ನೀವು ಅವನಿಗೆ ಕಳ್ಳತನವೇ ಶ್ರೇಷ್ಠವಾದುದೆಂದು ಹೇಳಿದರೆ, “ಇದು ಅನುಕೂಲಕರವಾದ ಧರ್ಮ” ಎಂದು ಅವನು ಹೇಳುತ್ತಾನೆ. ಅವನು ಮೋಸಗಾರನಾಗಿದ್ದರೆ ಅವನು ಮಾಡುತ್ತಿರುವುದೇ ಸರಿಯೆಂದು ನೀವು ಅವನಿಗೆ ಹೇಳಬೇಕು. ಆಗ ಅವನು ನಿಮ್ಮ ಬೋಧನೆಯನ್ನು “ಅನುಕೂಲಕರವಾದ ಧರ್ಮ”ವೆಂದು ಸ್ವೀಕರಿಸುತ್ತಾನೆ. ತಮಗೆ ರೂಢಿಯಾಗಿರುವುದನ್ನು ಬಿಟ್ಟು ಹೊರಗೆ ಬರಲು, ಮೀನಿನ ಬುಟ್ಟಿ ಮತ್ತು ಅದರ ವಾಸನೆಯನ್ನು ಬಿಟ್ಟುಬಿಡಲು ಜನರಿಗೆ ಇಷ್ಟವಿಲ್ಲ. ಇದೇ ತೊಂದರೆ. “ನಮಗೆ ಸತ್ಯಬೇಕು” ಎಂದು ಅವರು ಹೇಳಿದರೆ ಅದರ ಅರ್ಥ ಅವರಿಗೆ ಮೀನಿನ ಬುಟ್ಟಿಬೇಕು ಎಂದು.

ನೀವು ಯಾವಾಗ ಜ್ಞಾನವನ್ನು ಪಡೆಯುತ್ತೀರಿ? ನೀವು ಸಾಧನ ಚತುಷ್ಟಯ ಸಂಪನ್ನರಾದಾಗ. ಇಹಾಮುತ್ರ ಸುಖ ಭೋಗದ ಆಸೆಯನ್ನು ತ್ಯಜಿಸಬೇಕು. ಈ ಜೀವನದ ಸುಖವೆಲ್ಲವೂ ವ್ಯರ್ಥ. ಅವು ಇಷ್ಟಬಂದಂತೆ ಬಂದು ಹೋಗಲಿ ಅಷ್ಟೆ.

ನೀವು ನಿಮ್ಮ ಪೂರ್ವ ಕರ್ಮದಿಂದ ಏನನ್ನು ಪಡೆದಿರುವಿರೊ ಅದನ್ನು ಯಾರೂ ನಿಮ್ಮಿಂದ ಕಸಿದುಕೊಳ್ಳಲಾರರು. ನೀವು ಐಶ್ವರ್ಯಕ್ಕೆ ಅರ್ಹರಾಗಿದ್ದರೆ ನೀವು ಕಾಡಿನ ಮಧ್ಯದಲ್ಲಿ ಅಡಗಿಕೊಂಡಿದ್ದರೂ ನಿಮಗೆ ಅದು ಬಂದೇ ಬರುತ್ತದೆ. ಒಳ್ಳೆಯ ಆಹಾರ ಬಟ್ಟೆಗಳನ್ನು ಪಡೆಯಲು ನೀವು ಅರ್ಹರಾಗಿದ್ದರೆ ನೀವು ಉತ್ತರ ಧ್ರುವದಲ್ಲಿದ್ದರೂ ಅವು ನಿಮಗೆ ಹೇಗೊ ಬರುತ್ತವೆ. ಹಿಮಕರಡಿಯು ನಿಮಗೆ ಅವನ್ನು ತಂದುಕೊಡುವುವು. ನೀವು ಅವಕ್ಕೆ ಅರ್ಹರಲ್ಲದಿದ್ದರೆ, ಇಡೀ ಜಗತ್ತನ್ನೇ ನೀವು ಗೆದ್ದರೂ ಹಸಿವೆಯಿಂದ ಸಾಯು ತ್ತೀರಿ. ಆದ್ದರಿಂದ ಇವುಗಳ ವಿಷಯಕ್ಕೆ ಏಕೆ ತಲೆ ಕೆಡಿಸಿಕೊಳ್ಳುತ್ತೀರಿ? ಅವುಗಳ ಪ್ರಯೋಜನವಾದರೂ ಎಷ್ಟರ ಮಟ್ಟಿನದು?

ನಾವು ಮಕ್ಕಳಂತೆ, ಈ ಜಗತ್ತು ತುಂಬ ಸುಂದರವಾಗಿದೆ, ಭೋಗರಾಶಿಯು ನಮಗಾಗಿ ಕಾದು ಕುಳಿತಿದೆ ಎಂದು ಭಾವಿಸುತ್ತೇವೆ. ಇದು ಕೇವಲ ಶಾಲೆಯ ಮಕ್ಕಳ ಕನಸು. ಅವನು ಪ್ರಪಂಚವನ್ನು ಪ್ರವೇಶಿಸಿದಾಗ ಅವನ ಕನಸೆಲ್ಲ ಮಾಯವಾಗುತ್ತದೆ. ದೇಶದ ವಿಚಾರವೂ ಹೀಗೆಯೇ. ಭಗ್ನಾವಶೇಷದ ಮೇಲೆಯೇ ಪ್ರತಿಯೊಂದು ನಗರವೂ ಕಟ್ಟಲ್ಪಟ್ಟಿದೆ, ಪ್ರತಿಯೊಂದು ಕಾಡೂ ಹಿಂದೆ ನಗರವಾಗಿತ್ತು ಎಂಬುದನ್ನು ಅವರು ಅರಿತಾಗ ಈ ಜಗತ್ತಿನಲ್ಲಿ ಎಲ್ಲವೂ ವ್ಯರ್ಥ ಎಂದು ನಂಬುತ್ತಾರೆ.

ಹಿಂದೆ ವಿಜೃಂಭಿಸಿದ ಜ್ಞಾನಶಕ್ತಿ ಐಶ್ವರ್ಯಶಕ್ತಿಗಳೆಲ್ಲವೂ ಮಾಯವಾದುವು. ಪುರಾತನ ಕಾಲದ ಎಷ್ಟೊ ವಿಜ್ಞಾನಗಳು ಹೇಳ ಹೆಸರಿಲ್ಲದಂತೆ ನಶಿಸಿಹೋದವು. ಹೇಗೆಂಬುದು ಯಾರಿಗೂ ಗೊತ್ತಿಲ್ಲ. ಇದು ನಮಗೆ ಅದ್ಭುತ ಪಾಠವನ್ನು ಕಲಿಸುತ್ತದೆ. ಈ ಜಗತ್ತಿನಲ್ಲಿ ಎಲ್ಲವೂ ನಶ್ವರ, ಎಲ್ಲವೂ ವ್ಯರ್ಥ, ಇವೆಲ್ಲವೂ ಮನುಷ್ಯನ ಸತ್ವವನ್ನು ಹೀರುತ್ತವೆ ಅಷ್ಟೆ. ಇದು ನಮಗೆ ಅರ್ಥವಾದರೆ ಈ ಜಗತ್ತು, ಅದು ನೀಡುವ ಸುಖ ಎಲ್ಲವೂ ಜುಗುಪ್ಸೆಯನ್ನು ಹುಟ್ಟಿಸುತ್ತದೆ. ಇದೇ ವೈರಾಗ್ಯ, ಜ್ಞಾನದ ಮೊದಲ ಮೆಟ್ಟಿಲು.

ಮನುಷ್ಯನ ಸ್ವಾಭಾವಿಕ ಪ್ರವೃತ್ತಿಯೇ ಇಂದ್ರಿಯ ವಿಷಯಗಳೆಡೆಗೆ ಹೋಗುವುದು. ಇಂದ್ರಿಯಗಳಿಂದ ವಿಮುಖನಾದರೆ ಅವನು ಭಗವಂತನೆಡೆಗೆ ಹೋಗುತ್ತಾನೆ. ಆದ್ದರಿಂದ ಪ್ರಾಪಂಚಿಕ ವಿಷಯಗಳಿಂದ ವಿಮುಖರಾಗುವುದೇ ನಾವು ಕಲಿಯ ಬೇಕಾದ ಪ್ರಥಮ ಪಾಠ.

ಈ ಸಂಸಾರ ಸಾಗರದಲ್ಲಿ ಮುಳುಗುತ್ತ ಕ್ಷಣಕಾಲ ತೇಲುತ್ತ, ಮತ್ತೆ ಮುಳುಗುತ್ತ - ಹೀಗೆ ಮೇಲೆ ಕೆಳಗೆ ಏಳುತ್ತಬೀಳುತ್ತ ಎಷ್ಟು ಕಾಲ ಕಳೆಯುತ್ತೀರಿ? ಈ ಕರ್ಮಚಕ್ರದಲ್ಲಿ ಎಷ್ಟು ಕಾಲಸುತ್ತುತ್ತಿರುವಿರಿ? ಎಷ್ಟು ಸಾವಿರ ಬಾರಿ ನೀವು ರಾಜರಾಗಿದ್ದಿರಿ? ಎಷ್ಟು ಸಾವಿರ ಬಾರಿ ನೀವು ಬಡತನದ ಕೂಪದಲ್ಲಿ ಬಿದ್ದಿಲ್ಲ? ಎಷ್ಟು ಸಾವಿರ ಬಾರಿ ಅಧಿಕಾರದ ಉತ್ತುಂಗಕ್ಕೆ ಏರಿಲ್ಲ? ಮತ್ತೆ ಕರ್ಮಪ್ರವಾಹಕ್ಕೆ ಸಿಕ್ಕಿ ಕೆಳಗೆ ಎಳೆಯಲ್ಪಡುತ್ತೀರಿ. ಈ ದಾರುಣ ಕರ್ಮಚಕ್ರವು ವಿಧವೆಯ ಕಣ್ಣೀರನ್ನಾಗಲಿ, ಅನಾಥರ ರೋದನವನ್ನಾಗಲಿ ಲೆಕ್ಕಿಸುವುದಿಲ್ಲ.

ಎಷ್ಟು ಕಾಲ ಹೀಗೆಯೇಸುತ್ತುವಿರಿ? ಜೀವನವಿಡೀ ಸೆರೆಮನೆಯಲ್ಲಿ ಕಳೆದ ಮುದುಕನೊಬ್ಬನು ಬಿಡುಗಡೆಯಾದಾಗ ಮತ್ತೆ ಅದೇ ಕತ್ತಲೆ ಕೋಣೆಗೆ ತನ್ನನ್ನು ಕಳು ಹಿಸಲು ಕೇಳಿಕೊಂಡನಂತೆ. ಅಂಥ ವ್ಯಕ್ತಿಯಂತೆ ಇರಲು ನಿಮಗೆ ಇಷ್ಟವೆ? ನಮ್ಮೆಲ್ಲರ ಸ್ಥಿತಿಯೂ ಹೀಗೆಯೇ ಆಗಿದೆ. ಜಗತ್ತೆಂಬ ಈ ಕೊಳಕು ಕತ್ತಲೆಯ ಕೋಣೆಯಲ್ಲಿಯೇ ಯಾವಾಗಲೂ ಇರುವುದಕ್ಕೆ ನಮ್ಮ ಶಕ್ತಿ ಮೀರಿ ಪ್ರಯತ್ನಿಸುತ್ತೇವೆ. ಈ ಬೀಭತ್ಸ ಭ್ರಾಂತಿಮಯ ಜಗದಂಗಣದಲ್ಲಿ ಕಾಲ್ಚೆಂಡಿನಂತೆ ಯಾವಾಗಲೂ ಒದೆಸಿಕೊಳ್ಳುತ್ತ ಇರಲು ನಾವು ಬಯಸುತ್ತೇವೆ.

ನಾವು ಪ್ರಕೃತಿಯ ಕೈಯಲ್ಲಿ ಗುಲಾಮರಾಗಿದ್ದೇವೆ. ನಾವು ಒಂದು ರೊಟ್ಟಿಯ ಚೂರಿಗೆ ಗುಲಾಮರು, ಪ್ರಶಂಸೆಗೆ ಗುಲಾಮರು,ನಿಂದೆಗೆ ಗುಲಾಮರು, ಮಡದಿ ಮಕ್ಕಳಿಗೆ ಗುಲಾಮರು, ಪತಿಗೆ ಗುಲಾಮರು, ಪ್ರತಿಯೊಂದಕ್ಕೂ ಗುಲಾಮರು. ಗೂನು ಬೆನ್ನಿನ ಅಥವಾ ಕುರೂಪಿಯಾದ ಹುಡುಗನ ಜೀವನವನ್ನು ಸುಖಮಯವನ್ನಾಗಿ ಮಾಡಲು ನಾನು ಜಗತ್ತನ್ನೆಲ್ಲಸುತ್ತಾಡುತ್ತೇನೆ, ಭಿಕ್ಷೆ ಬೇಡುತ್ತೇನೆ, ಕದಿಯುತ್ತೇನೆ, ಏನು ಬೇಕಾದರೂ ಮಾಡುತ್ತೇನೆ. ಏಕೆಂದರೆ ಆ ಹುಡುಗ ನನ್ನ ಮಗನಾಗಿರುವುದರಿಂದ. ಆದರೆ ಅದೇ ಸಂದರ್ಭದಲ್ಲಿ ಸುಂದರ ದೇಹ ಮನಸ್ಸುಳ್ಳ ಮಿಲಿಯಗಟ್ಟಲೆ ಬಾಲಕರು ಹೊಟ್ಟೆಗಿಲ್ಲದೆ ಸಾಯುತ್ತಿರಬಹುದು. ಅದನ್ನು ನಾನು ಲೆಕ್ಕಿಸುವುದೇ ಇಲ್ಲ. ಅವರನ್ನೆಲ್ಲ ಕೊಲ್ಲುವುದಕ್ಕೂ ನಾನು ಹಿಂಜರಿಯುವುದಿಲ್ಲ. ಇದನ್ನೇ ನೀವು ಪ್ರೀತಿಯೆಂದು ಕರೆಯುವುದು. ನಾನಂತೂ ಹಾಗೆ ಕರೆಯುವುದಿಲ್ಲ. ಇದು ಕ್ರೌರ್ಯ.

ಸುಂದರ ದೇಹಮನಸ್ಸುಗಳುಳ್ಳ, ಸದ್ಗುಣಸಂಪನ್ನರಾದ ಸಾವಿರಾರು ಮಹಿಳೆ ಯರು ಉಪವಾಸದಿಂದ ಸಾಯುತ್ತಿರುವರು. ನಾನಿದನ್ನು ಲೆಕ್ಕಿಸುವುದೇ ಇಲ್ಲ. ಆದರೆ ನನ್ನವಳಾದ ಈ ಜನನಿಯು - ಅವಳು ದಿನಕ್ಕೆ ಮೂರು ಬಾರಿ ನನ್ನನ್ನು ಹೊಡೆಯಬಹುದು, ದಿನವೆಲ್ಲ ನನ್ನನ್ನು ಬಯ್ಯುತ್ತಿರಬಹುದು - ಆದರೂ ಅವಳಿಗಾಗಿ ನಾನು ಬೇಡುತ್ತೇನೆ, ಸಾಲಮಾಡುತ್ತೇನೆ, ಮೋಸಮಾಡುತ್ತೇನೆ, ಕದಿಯುತ್ತೇನೆ, ಆಕೆಗೊಂದು ಒಳ್ಳೆಯ ಸೀರೆಯನ್ನು ತಂದುಕೊಡುವುದಕ್ಕಾಗಿ ನಾನು ಏನು ಬೇಕಾದರೂ ಮಾಡುತ್ತೇನೆ.

ಇದನ್ನು ನೀವು ಪ್ರೀತಿ ಎನ್ನು ತ್ತೀರಾ? ನಾನಂತೂ ಇಲ್ಲ. ಇದು ಕೇವಲ ಆಸೆ, ಪ್ರಾಣಿ ಸಹಜ ಆಸೆ - ಮತ್ತೇನಲ್ಲ. ಇವುಗಳಿಂದ ದೂರವಾಗಿ. ಇಂತಹ ಬೀಭತ್ಸ ಕನಸುಗಳಿಗೆ ಕೊನೆಯೇ ಇಲ್ಲವೆ? ಇವುಗಳನ್ನು ಅಂತ್ಯಗೊಳಿಸಿ.

ಜೀವನದ ಸುಖಭೋಗಗಳ ಬಗ್ಗೆ ಮನಸ್ಸಿನಲ್ಲಿ ಜಿಗುಪ್ಸೆಯುಂಟಾಗುವ ಸ್ಥಿತಿ ಮುಟ್ಟಿದಾಗ ನಾವು ಪ್ರಕೃತಿಯಿಂದ ವಿಮುಖರಾಗುತ್ತೇವೆ. ಇದು ಮೊದಲನೆಯ ಹಂತ. ಎಲ್ಲ ಆಸೆಗಳನ್ನೂ, ಸ್ವರ್ಗಪ್ರಾಪ್ತಿಯ ಆಸೆಯನ್ನೂ ತ್ಯಜಿಸಬೇಕು.

ಈ ಸ್ವರ್ಗಗಳಾದರೊ ಎಂಥವು? ಯಾವಾಗಲೂ ದೇವರ ಸ್ತೋತ್ರಪಾಠ ಮಾಡು ತ್ತಿರುವುದು.ಯಾತಕ್ಕಾಗಿ? ಯಾವಾಗಲೂ ಅಲ್ಲಿಯೇ ಇರಬೇಕೆಂದು, ಒಳ್ಳೆಯ ಆರೋಗ್ಯ ಕರ ದೇಹವನ್ನು ಹೊಂದಿರಬೇಕೆಂದು!ಸುತ್ತಲೂ ಆಹ್ಲಾದಕರ ಬೆಳಕಿನಿಂದ ಆವೃತವಾಗಿರಬೇಕೆಂದು! ಇಷ್ಟೇ ತಾನೆ! ತಲೆಯಸುತ್ತ ಪ್ರಭಾವಳಿ, ಆಕಾಶದಲ್ಲಿ ಹಾರಾಡಲು ರೆಕ್ಕೆಗಳು, ಗೋಡೆಯನ್ನು ತೂರಿಕೊಂಡು ಹೋಗುವ ಶಕ್ತಿ - ಇವೆಲ್ಲವನ್ನು ಪಡೆದರೆ ತಾನೆ ಏನು ಪ್ರಯೋಜನ?

ಎಂಥ ಶಕ್ತಿಯಿದ್ದರೂ ಅದು ಒಂದಲ್ಲ ಒಂದು ದಿನ ಹೊರಟು ಹೋಗಲೇಬೇಕು. ವಿವಿಧ ಬಗೆಯ ಸ್ವರ್ಗಗಳು ಇರಬಹುದು, ಆದರೆ ದೇಹವೇನೂ ಶಾಶ್ವತವಾಗಿ ಉಳಿ ಯುವುದಿಲ್ಲ. ಅಲ್ಲಿಯೂ ಮೃತ್ಯುವು ನಮ್ಮನ್ನು ಹಿಂಬಾಲಿಸುವುದು.

ಎಲ್ಲಾಸಂಯೋಗವೂ ವಿಯೋಗದಲ್ಲಿ ಪರ್ಯವಸಾನಗೊಳ್ಳುವುದು. ಸೂಕ್ಷ್ಮವಾಗಲಿ ಸ್ಥೂಲವಾಗಲಿ, ಎಲ್ಲ ದೇಹಗಳೂ ಒಟ್ಟಿಗೆ ಸೇರಿದಾಗ ಯಾವುದೇ ಆಕರ್ಷಣೆಯ ಬಲದಿಂದ ಅವು ಒಂದಾಗಿರುತ್ತವೆ. ಆ ಕಣಗಳು ಮತ್ತೆ ಪ್ರತ್ಯೇಕವಾಗುವ ಸಮಯ ಬಂದೇ ಬರುತ್ತದೆ. ಇದು ಶಾಶ್ವತ ನಿಯಮ. ಸ್ವರ್ಗದಲ್ಲಿರುವ ದೇಹವಾಗಲಿ, ಭೂಮಿಯ ಮೇಲಿರುವ ದೇಹವಾಗಲಿ - ಅದು ಸ್ಥೂಲವಾಗಿರಲಿ, ಸೂಕ್ಷ್ಮವಾಗಿರಲಿ - ಅದು ಮೃತ್ಯುವಶವಾಗಲೇಬೇಕು.

ಆದ್ದರಿಂದ ಇಹಾಮುತ್ರ ಸುಖಭೋಗಗಳ ಆಸೆಯನ್ನು ತ್ಯಜಿಸಬೇಕು. ಜನರಿಗೆ ಸುಖದ ಬಯಕೆ ಸ್ವಾಭಾವಿಕವಾಗಿರುತ್ತದೆ. ಇಲ್ಲಿ ಅವರ ಸುಖದ ಬಯಕೆ ಪೂರ್ತಿಯಾಗದಿದ್ದರೆ ಸತ್ತಮೇಲೆ ಯಾವುದೋ ಲೋಕದಲ್ಲಿ ಸುಖವನ್ನು ಅನುಭವಿಸುತ್ತೇವೆ ಎಂದು ಭಾವಿಸುತ್ತಾರೆ. ಈ ಜನ್ಮದಲ್ಲಿಯೇ ನಮಗೆ ಜ್ಞಾನ ಪ್ರಾಪ್ತಿಯಾಗದಿದ್ದರೆ ಮುಂದೆ ಆಗುತ್ತದೆ ಎಂಬುದು ಯಾವ ನಿಶ್ಚಯ?

ಮಾನವನ ಗುರಿಯೇನು? ಸುಖವೇ ಅಥವಾ ಜ್ಞಾನವೆ? ಖಂಡಿತ ಸುಖವಲ್ಲ. ಸುಖ ಅಥವಾ ದುಃಖವನ್ನು ಅನುಭವಿಸುವುದಕ್ಕಾಗಿ ಮಾನವನು ಹುಟ್ಟಿಲ್ಲ. ಜ್ಞಾನವೇ ಗುರಿ. ಜ್ಞಾನವೇ ನಾವು ಅನುಭವಿಸಬೇಕಾದ ಏಕೈಕ ಸುಖ.

ಎಲ್ಲ ಇಂದ್ರಿಯ ಸುಖಗಳೂ ಪ್ರಾಣಿಸಹಜವಾದುದು. ಜ್ಞಾನಸುಖವನ್ನು ನಾವು ಹೆಚ್ಚು ಪಡೆದಂತೆ ಇಂದ್ರಿಯ ಸುಖಗಳು ಕಡಿಮೆಯಾಗುತ್ತವೆ. ಮನುಷ್ಯನು ಪ್ರಾಣಿ ಸಮಾನನಾಗಿದ್ದಷ್ಟೂ ಇಂದ್ರಿಯಗಳಲ್ಲಿ ಹೆಚ್ಚು ಸುಖವನ್ನು ಪಡೆಯುತ್ತಾನೆ. ಯಾವ ಮನುಷ್ಯನೂ ಒಂದು ಸೊಣಕಲು ನಾಯಿಯಷ್ಟು ಅತ್ಯಾಸಕ್ತಿಯಿಂದ ಊಟಮಾಡಲಾರ. ಒಂದು ಸಾಮಾನ್ಯ ಗೂಳಿಯು ಉಣ್ಣುವುದರಲ್ಲಿ ಸುಖವನ್ನು ಅನುಭವಿಸುವಂತೆ ಯಾವ ಮನುಷ್ಯನೂ ಅನುಭವಿಸಲಾರ. ಅವುಗಳ ಇಡೀ ಜೀವವೇ ಆ ತಿನ್ನುವುದರಲ್ಲಿ ರುತ್ತದೆ. ಶ‍್ರೀಮಂತರು ಆ ಸುಖಕ್ಕಾಗಿ ಲಕ್ಷಾಂತರ ಹಣವನ್ನು ಸುರಿಯಬಹುದು - ಆದರೆ ಅವರು ಆ ರೀತಿ ಸುಖಿಸಲಾರರು.

ಈ ಜಗತ್ತು ಪೂರ್ಣ ಸಮತೋಲವನ್ನು ಹೊಂದಿರುವ ಒಂದು ಸಮುದ್ರವಿದ್ದಂತೆ. ಅಲ್ಲಿ ಒಂದು ಅಲೆ ಏಳಬೇಕಾದರೆ ಇನ್ನೊಂದು ಕಡೆ ಇಳಿತವಿರಲೇಬೇಕು. ಈ ಜಗತ್ತಿನ ಒಟ್ಟು ಶಕ್ತಿಯ ಮೊತ್ತ ಒಂದೇ ಸಮನಾಗಿರುವುದು. ಒಂದು ಕಡೆ ನೀವದನ್ನು ಉಪಯೋಗಿ ಸಿದರೆ ಇನ್ನೊಂದು ಕಡೆ ಅದು ಕಡಮೆಯಾಗುತ್ತದೆ. ಪ್ರಾಣಿ ಆ ಶಕ್ತಿಯನ್ನು ಪಡೆದಿದೆ, ಆದರೆ ಅದು ಇಂದ್ರಿಯಗಳಲ್ಲೇ ವ್ಯಯವಾಗುತ್ತದೆ. ಅದರ ಇಂದ್ರಿಯಗಳು ಮನುಷ್ಯನ ದಕ್ಕಿಂತ ನೂರು ಪಟ್ಟು ಬಲಯುತವಾಗಿರುತ್ತವೆ.

ನಾಯಿಯ ಘ್ರಾಣ ಶಕ್ತಿಯನ್ನು ನೋಡಿ! ಅದು ಹೆಜ್ಜೆಯ ಗುರುತು ಹಿಡಿದು ಹೋಗುವುದನ್ನು ನೋಡಿ! ನಮಗದು ಸಾಧ್ಯವಿಲ್ಲ. ಹಾಗೆಯೆ ಕಾಡು ಮನುಷ್ಯನೂ ಕೂಡ. ಅವನ ಇಂದ್ರಿಯಗಳು ಪ್ರಾಣಿಗಳಿಗಿಂತ ಕಡಮೆ ಸೂಕ್ಷ್ಮ, ಆದರೆ ನಾಗರಿಕ ಮಾನವನಿಗಿಂತ ಹೆಚ್ಚು ಸೂಕ್ಷ್ಮ.

ಪ್ರತಿಯೊಂದು ದೇಶದಲ್ಲಿಯೂ ಕೂಡ ಕೆಳವರ್ಗದ ಜನರು ಭೌತಿಕವಾದುದರಲ್ಲಿ ಹೆಚ್ಚು ಸುಖವನ್ನು ಕಾಣುತ್ತಾರೆ. ಸುಸಂಸ್ಕೃತ ವ್ಯಕ್ತಿಗಿಂತ ಅವರ ಇಂದ್ರಿಯಗಳು ಹೆಚ್ಚು ಶಕ್ತಿಯುತವಾಗಿರುತ್ತವೆ. ನೀವು ವಿಕಾಸದ ಹಾದಿಯಲ್ಲಿ ಮುಂದುವರಿದಂತೆ ಆಲೋಚನಾ ಶಕ್ತಿಯು ವೃದ್ಧಿಸುವುದನ್ನೂ ಅದಕ್ಕನುಗುಣವಾಗಿ ಇಂದ್ರಿಯ ಶಕ್ತಿಗಳು ಕ್ಷೀಣಿಸುವುದನ್ನೂ ನೋಡುತ್ತೀರಿ.

ಅನಾಗರಿಕ ವ್ಯಕ್ತಿಯನ್ನು ನೀವು ತೀವ್ರವಾಗಿ ಘಾಯಗೊಳಿಸಿದರೆ ಒಂದೇ ದಿನದಲ್ಲಿ ಆ ಘಾಯ ವಾಸಿಯಾಗುತ್ತದೆ. ಆದರೆ ನಿಮಗೆ ಒಂದು ಸಣ್ಣ ಘಾಯವಾದರೂ ವಾರಗಟ್ಟಳೆ ಅಥವಾ ತಿಂಗಳುಗಟ್ಟಳೆ ಅನುಭವಿಸಬೇಕಾಗುತ್ತದೆ. ಅವನು ವ್ಯಕ್ತಪಡಿಸುವ ಶಕ್ತಿ ನಿಮ್ಮಲ್ಲಿಯೂ ಇದೆ. ಆದರೆ ಆ ಶಕ್ತಿಯನ್ನು ನೀವು ಮಿದುಳನ್ನು ಬೆಳೆಸುವುದರಲ್ಲಿ, ಆಲೋಚನೆಗಳನ್ನು ಸೃಷ್ಟಿಸುವುದರಲ್ಲಿ ಉಪಯೋಗಿಸುತ್ತೀರಿ. ಹಾಗೆಯೇ ಸುಖಭೋಗದ ವಿಷಯದಲ್ಲಿಯೂ ಕೂಡ. ಇಂದ್ರಿಯ ಸುಖವನ್ನು ಅನುಭವಿಸುತ್ತ ಪಶುಸಮಾನರಾಗಿರಿ, ಇಲ್ಲವೇ ಇವುಗಳನ್ನೆಲ್ಲ ತ್ಯಾಗ ಮಾಡಿ ಮುಕ್ತರಾಗಿ.

ಶ್ರೇಷ್ಠ ನಾಗರಿಕತೆಗಳು ನಾಶವಾದುವು ಏಕೆ? ವಿಷಯಸುಖದಿಂದಾಗಿ. ಅವರು ಎಷ್ಟು ಕೆಳಮಟ್ಟಕ್ಕೆ ಇಳಿದರೆಂದರೆ, ಭಗವಂತನ ಕೃಪೆಯಿಂದ, ಬರ್ಬರ ಜನಾಂಗಗಳು ಬಂದು ಅವರನ್ನು ನಿರ್ನಾಮ ಮಾಡಿದರು. ಇದರಿಂದಾಗಿ ಮಾನವಪಶುಗಳ ರೇಗಾಟ ಕೂಗಾಟಗಳನ್ನು ನಾವು ನೋಡದಂತಾಯಿತು. ಇಂದ್ರಿಯ ಭೋಗದಿಂದ ಪಶು ಸಮಾನರಾದ ಜನಾಂಗಗಳನ್ನು ಬರ್ಬರರು ಕೊಂದುಹಾಕಿದರು. ಇದರಿಂದ ಡಾರ್ವಿನ್ನನು ಹೇಳುವ ಸರಣಿಲೋಪವು ಆಗುವಂತಾಯಿತು.

ನಿಜವಾದ ನಾಗರಿಕತೆಯೆಂದರೆ ನಗರಗಳಲ್ಲಿ ವಾಸಿಸುವುದಲ್ಲ, ಮೂರ್ಖ ಜೀವನ ನಡೆಸುವುದಲ್ಲ. ದೇವರೆಡೆಗೆ ಹೋಗುವುದು, ಇಂದ್ರಿಯಗಳನ್ನು ನಿಗ್ರಹಿಸುವುದು ಮತ್ತು ತನ್ನ ಆತ್ಮನರಮನೆಯಲ್ಲಿ ಒಡೆಯನಂತೆ ಬಾಳುವುದು - ಇದೇ ನಿಜವಾದ ನಾಗರಿಕತೆ.

ನಾವು ಎಂತಹ ಗುಲಾಮರಾಗಿದ್ದೇವೆ ಎಂಬುದನ್ನು ಯೋಚಿಸಿ ನೋಡಿ. ಪ್ರತಿ ಯೊಂದು ಸುಂದರ ರೂಪವೂ, ಪ್ರತಿಯೊಂದು ಸುಶ್ರಾವ್ಯ ಧ್ವನಿಯೂ ಕೂಡಲೆ ನನ್ನನ್ನು ಆಕರ್ಷಿಸುತ್ತದೆ. ಪ್ರತಿಯೊಂದುನಿಂದಾಶಬ್ದವೂ ನನ್ನನ್ನು ದೂರತಳ್ಳುತ್ತದೆ. ಪ್ರತಿ ಯೊಬ್ಬ ಮೂರ್ಖನೂ ನನ್ನ ಮನಸ್ಸಿನ ಮೇಲೆ ಪ್ರಭಾವ ಬೀರುತ್ತಾನೆ. ಜಗತ್ತಿನ ಪ್ರತಿ ಯೊಂದು ಚಲನೆಯೂ ನನ್ನ ಮೇಲೆ ಮುದ್ರೆಯೊತ್ತುತ್ತದೆ. ಇದು ಬಾಳಲು ಯೋಗ್ಯವಾದ ಜೀವನವೆ?

ಆದ್ದರಿಂದ ಈ ಭೌತಿಕ ಅಸ್ತಿತ್ವದ ದುಃಖವು ನಿಮಗೆ ಅರಿವಾದಾಗ, ಇಂತಹ ಜೀವನ ಯೋಗ್ಯವಾದುದಲ್ಲ ಎಂದು ದೃಢವಾದಾಗ ನೀವು ಇಟ್ಟಂತಾಗುತ್ತದೆ.

