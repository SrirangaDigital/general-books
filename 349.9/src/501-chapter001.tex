
\addtocontents{toc}{\protect\vspace{-0.7cm}}


\part{ಸ್ವಾಮಿ ವಿವೇಕಾನಂದರೊಂದಿಗೆ ಪ್ರವಾಸ – ಕೆಲವು ಟಿಪ್ಪಣಿಗಳು}

%\\(ಸೋದರಿ ನಿವೇದಿತಾ ಬರೆದ ಪುಸ್ತಕದಿಂದ ಆಯ್ದ ಭಾಗಗಳು)\protect\footnote{\enginline{1}ಈ ಮುಂದಿನ ಕೃತಿಯಲ್ಲಿ ಸ್ವಾಮಿ ವಿವೇಕಾನಂದರ ವಿಚಾರ ಹಾಗೂ ಮಾತುಗಳನ್ನು ಒಳಗೊಂಡ ಭಾಗಗಳನ್ನು ಮಾತ್ರ ಆರಿಸಲಾಗಿದೆ. ಈ ಸಂಭಾಷಣೆಗಳಿಗೆ ಹಿನ್ನೆಲೆಯಾಗಿರುವ ಸನ್ನಿವೇಶಗಳನ್ನೂ ಸ್ಪಷ್ಟತೆಗಾಗಿ ಉಳಿಸಿಕೊಳ್ಳಲಾಗಿದೆ.}}


ಈ ಮುಂದಿನ ಕೃತಿಯಲ್ಲಿ ಸ್ವಾಮಿ ವಿವೇಕಾನಂದರ ವಿಚಾರ ಹಾಗೂ ಮಾತುಗಳನ್ನು ಒಳಗೊಂಡ ಭಾಗಗಳನ್ನು ಮಾತ್ರ ಆರಿಸಲಾಗಿದೆ. ಈ ಸಂಭಾಷಣೆಗಳಿಗೆ ಹಿನ್ನೆಲೆಯಾಗಿರುವ ಸನ್ನಿವೇಶಗಳನ್ನೂ ಸ್ಪಷ್ಟತೆಗಾಗಿ ಉಳಿಸಿಕೊಳ್ಳಲಾಗಿದೆ.

\chapter{ಮೊದಲ ಮಾತು}

ವ್ಯಕ್ತಿಗಳು: ಸ್ವಾಮಿ ವಿವೇಕಾನಂದರು, ಗುರುಭಾಯಿಗಳು\footnote{1. ಆಧ್ಯಾತ್ಮಿಕ ಸಹೋದರರು; ಒಬ್ಬನೇ ಗುರುವಿನ ಶಿಷ್ಯರುಗಳಿಗೆ ಹಾಗೆಂದು ಹೆಸರು.

2. ಧೀರಮಾತಾ ಹಾಗೂ ಜಯಾ ಅಮೆರಿಕಾದವರು; ನಿವೇದಿತಾ ಬ್ರಿಟನ್ನಿನವಳು.

\begin{flushright}
-ಪ್ರಕಾಶಕರು
\end{flushright}}, ಧೀರಮಾತಾ(ಮಿಸೆಸ್ ಓಲ್ ಬುಲ್), ಜಯಾ ಎಂಬ ಹೆಸರಿನವಳು (ಮಿಸ್ ಜೋಸೆಫಿನ್ ಮ್ಯಾಕ್ಲಿಯಾಡ್) ಮತ್ತು ಸೋದರಿ ನಿವೇದಿತಾಳನ್ನೊಳಗೊಂಡ ಯೂರೋಪಿಯ್​ ಶಿಷ್ಯರುಗಳು ಮತ್ತು ಅತಿಥಿಗಳ ಗುಂಪು.

ಸ್ಥಳ: ಭಾರತದ ವಿವಿಧ ಭಾಗಗಳು.

ಕಾಲ: ೧೮೯೮ನೆಯ ಇಸವಿ.

ಈ ವರ್ಷದ ದಿನಗಳು ಸುಂದರವಾಗಿದ್ದವು. ಈ ದಿನಗಳಲ್ಲಿ ಆದರ್ಶಗಳು ಸತ್ಯವಾಗಿ ಪರಿಣಮಿಸಿದ್ದವು. ಮೊದಲು ಗಂಗೆಯ ಬಳಿಯ ಗುಡಿಸಲಿನಲ್ಲಿ; ನಂತರ ಹಿಮಾಲಯದ ನೈನಿತಾಲ್ ಮತ್ತು ಅಲ್ಮೋರಗಳಲ್ಲಿ; ಅನಂತರ ಕಾಶ್ಮೀರದಲ್ಲಿ ಅಲ್ಲಿ - ಇಲ್ಲಿ ಅಲೆದಾಟಗಳಲ್ಲಿ - ಈ ಎಲ್ಲೆಡೆ ಮರೆಯಲಾರದ ಘಳಿಗೆಗಳು, ನಮ್ಮ ಬದುಕಿನು ದ್ದಕ್ಕೂ ಚಿರಕಾಲ ಪ್ರತಿಧ್ವನಿಸಲಿರುವ ಮಾತುಗಳು, ಅಲ್ಲದೆ ಒಮ್ಮೆಯಾದರೂ ದಿವ್ಯ ಸೌಂದರ್ಯಾನುಭೂತಿಯ ಕ್ಷಣಿಕದರ್ಶನ.

ಎಲ್ಲವೂ ಲೀಲೆಯಂತೆ ನಡೆದುಹೋಯಿತು.

ಟೀಕೆಟಿಪ್ಪಣಿಗಳೇ ಇಲ್ಲವಾದ, ಅಜ್ಞಾನಿಯ ಹಾಗೂ ವಿನೀತನ ದೃಷ್ಟಿಯಲ್ಲಿ ಲೋಕವನ್ನು ನೋಡುವ ಮೂಲಕ ಅವನೊಂದಿಗೆ ಏಕೀಭವಿಸುವಂತಹ ಪ್ರೇಮವನ್ನು ಅನು ಭವಿಸಿದೆವು; ಅಸಾಧಾರಣ ಪ್ರತಿಭೆಯ ಲೋಕವಿಶಾಲವಾದ ಚಾಪಲ್ಯದ ಲಹರಿಯನ್ನು ನೋಡಿ ನಕ್ಕೆವು; ಧೀರೋದಾತ್ತ ಅಗ್ನಿಯ ಸಮ್ಮುಖದಲ್ಲಿ ಚಳಿ ಕಾಯಿಸಿಕೊಂಡೆವು; ದಿವ್ಯ ಶಿಶುವು ಎಚ್ಚರಗೊಳ್ಳುವುದರ ಸಾಕ್ಷಿಯಾಗಿ ನಾವಲ್ಲಿದ್ದೆವು.

ಆದರೆ ಈ ಯಾವುದರಲ್ಲೂ ಶ‍್ರೀಮದ್ಗಾಂಭೀರ್ಯದ ಲವಲೇಶವೂ ಇರಲಿಲ್ಲ. ನೋವು ನಮ್ಮೆಲ್ಲರಿಗೂ ಹತ್ತಿರ ಬಂದು ಮುತ್ತಿಕ್ಕುತ್ತಿತ್ತು. ಮಹತ್ವಪೂರ್ಣ ವರ್ಷಗಳು ಉರುಳು ತ್ತಲೇ ಇದ್ದವು. ಆದರೆ ಹಿರಣ್ಯಜ್ಯೋತಿಗೆ ಏರಿಸಲ್ಪಟ್ಟ ಆಂತರ್ಯದ ಶೋಕವು ವಿನಾಶ ಕಾರಿಯಾಗುವ ಬದಲು ದೀಪ್ತವಾಯಿತು.

ಬಾರಾಮುಲ್ಲಾದಲ್ಲಿ ಅರಳಿದ್ದ ಪದ್ಮಪುಷ್ಕರಗಳು; ಇಸ್ಲಾಮಾಬಾದಿನ ಸೂಚೀಪರ್ಣ ವೃಕ್ಷಗಳಡಿಯ ಎಳೆಯ ಭತ್ತದ ಪೈರು; ಹಿಮಾಲಯದ ಅರಣ್ಯಗಳಲ್ಲಿನ ನಕ್ಷತ್ರ ಬೆಳಕಿನ ದೃಶ್ಯವೈಭವ; ಹಾಗೂ ದೆಹಲಿ, ತಾಜ್ಗಳ ರಾಜವೈಭವದ ಭವ್ಯತೆ. ಇವುಗಳಲ್ಲಿ ಕೆಲವನ್ನು ಪ್ರಯತ್ನಪಟ್ಟು ಸ್ಮರಿಸಿಕೊಳ್ಳುತ್ತಿರಬೇಕೆನ್ನಿಸುತ್ತದೆ. ಪದಗಳಲ್ಲಿ - ವ್ಯರ್ಥವೇ; ಸ್ಮರಣಿಕೆಯ ಬೆಳಕಿನಲ್ಲಿ ಅವು ಶಾಶ್ವತವಾಗಿ ನೆಲೆಗೊಂಡಿರುತ್ತವೆ. ಜೊತೆಗೆ ನಾವು ಅನವರತವೂ ಆಪ್ತವಾಗಿ ನೆಚ್ಚುವ, ನಮ್ಮ ಬರುವಿಕೆಯಿಂದ ಉಲ್ಲಸಿತರಾದ ಆ ಮೃದುಸ್ವಭಾವದ ಸಭ್ಯ ಜನರ ನೆನಪು.

ಹೊಸ ಶ್ರದ್ಧೆಗಳು ಉದಯಿಸಿದ ಭಾವಲಹರಿಗಳ ಬಗ್ಗೆ, ಅಂತಹ ಶ್ರದ್ಧೆಯ ಸೂರ್ತಿಯನ್ನು ತುಂಬುವ ವ್ಯಕ್ತಿಗಳ ಬಗ್ಗೆ, ನಾವು ಒಂದಿಷ್ಟನ್ನು ಕಲಿತೆವು. ಏಕೆಂದರೆ, ಯಾರನ್ನೂ ನಿರಾಕರಿಸದ, ಎಲ್ಲರನ್ನೂ ಆಲಿಸುವ, ಎಲ್ಲರ ಮನಸ್ಸನ್ನೂ ಅರಿಯುವ ಮೂಲಕ ತನ್ನೆಡೆಗೆ ಸೆಳೆದುಕೊಳ್ಳುವ ವ್ಯಕ್ತಿಯೊಬ್ಬರೊಡನೆ ನಾವಿದ್ದೆವು. ಅಲ್ಪತೆಯೆಲ್ಲವನ್ನೂ ಅಳಿಸಿಹಾಕು ವಂತಹ ವಿನಯ, ದಲಿತರ ಮೇಲಣ ದಯೆಗಾಗಿ, ಶೋಷಣೆಯ ನಿರ್ನಾಮಕ್ಕಾಗಿ ಸಾಯಲೂ ಸಿದ್ಧವಾಗುವಂತಹ ತ್ಯಾಗ, ಬರಲಿರುವ ಹಿಂಸೆಯ, ಮೃತ್ಯುವಿನ ಪದಾಘಾತಗಳನ್ನೂ ಮುದಗೊಳಿಸುವಂತಹ ಪ್ರೇಮ ಹೇಗಿರುತ್ತದೆ ಎಂಬುದನ್ನರಿತೆವು. ಪ್ರಭುವಿನ ಪಾದಗಳನ್ನು ಕಣ್ಣೀರಿನಿಂದ ತೊಳೆದು, ತನ್ನ ಕೇಶರಾಶಿಯಿಂದ ಅವುಗಳನ್ನೊತ್ತಿದ ಆ ಸ್ತ್ರೀಯ ಜೊತೆಗೆ ನಾವೂ ಕೈಜೋಡಿಸಿದೆವು. ಸಂಬರ್ಭ ಒದಗಲಿಲ್ಲವೆಂದಲ್ಲ, ಅವಳ ಆ ಆತ್ಮಪ್ರಜ್ಞೆಯ ಭಾವತೀವ್ರತೆ ನಮಗೆ ಒದಗಿ ಬರಲಿಲ್ಲ.

ಗತಿಸಿಹೋದ ಸಾಮ್ರಾಟರುಗಳ ಉದ್ಯಾನವನದಲ್ಲಿನ ವೃಕ್ಷದ ಅಡಿಯಲ್ಲಿ ಕುಳಿತಿದ್ದ ನಮಗೆ ದರ್ಶನವೊಂದರ ಮೂಲಕ ಈ ಭೂಮಿಯಲ್ಲಿನ ಸಮಸ್ತ ಶ‍್ರೀಮಂತ ಅದ್ಭುತಗಳೂ ಆತ್ಮನ ಮಹತ್ತಿಗೊಂದು ಅಲಂಕಾರದ ದೇಗುಲದ ರೂಪದಲ್ಲಿ ಕಂಗೊಳಿಸಿದವು. ಚರ್ಚುಗಳ ಬಹುಸ್ತರದ ಕಿಟಕಿಗಳು, ರಾಜರುಗಳ ರತ್ನಖಚಿತ ಸಿಂಹಾಸನಗಳು, ಮಹಾ ಯೋದ್ಧ ಸೇನಾಪತಿಗಳ ಧ್ವಜಗಳು, ಪುರೋಹಿತರುಗಳ, ರೈತರ ವೇಷಗಳು, ನಗರವಾಸಿಗಳ ವಿಶಿಷ್ಟ ವಾಸಸ್ಥಳಗಳು, ಮುಂತಾದವೆಲ್ಲವೂ ಬಗೆಗಣ್ಣ ಮುಂದೆ ಬಂದುಹೋದವು; ಎಲ್ಲವನ್ನೂ ನಾವು ತಿರಸ್ಕರಿಸಿದೆವು.

ಅನ್ಯದೇಶೀಯರಿಂದ ತಿರಸ್ಕರಿಸಲ್ಪಡುವ, ದೇಶೀಯರಿಂದ ಪೂಜಿಸಲ್ಪಡುವ ಭಿಕ್ಷಾಟನೆಯ ಉಡುಪಿನಲ್ಲಿ ನಾವು ಅವರನ್ನು ನೋಡಿರುವೆವು; ಬೆವರು ಹರಿಸಿಗಳಿಸಿದ ತಿನಿಸು, ಗುಡಿಸಲ ಆಶ್ರಯ ಮತ್ತು ಕಾಲುಹಾದಿಯ ನಡುಗೆ - ಇಂತಹ ಹಿನ್ನೆಲೆಯೇ ಈ ಬದುಕಿಗೆ ಸೂಕ್ತವೆನ್ನಿಸದೆ ಇರದು...ಪಾಮರರು ಅವರನ್ನು ಪ್ರೀತಿಸಿದಂತೆಯೇ ಮುತ್ಸದ್ದಿಗಳು, ಪಂಡಿತರೆನ್ನಿಸಿಕೊಂಡವರು ಸಹ ಪ್ರೀತಿಸುತ್ತಿದ್ದರು. ಅವರಿಲ್ಲದಾಗ ಅಂಬಿಗರು ಅವರು ಹಿಂದಿರುಗುವುದನ್ನೇ ಕಾಯುತ್ತ ನದಿಯನ್ನು ನೋಡುತ್ತಿರುತ್ತಿದ್ದರು, ಅವರಿಗೆ ಸೇವೆ ಸಲ್ಲಿಸುವಸಲುವಾಗಿ ಸೇವಕರು ಅತಿಥಿಗಳೊಂದಿಗೆವಾದಮಾಡುತ್ತಿದ್ದರು. ಈ ಎಲ್ಲದರ ನಡುವೆ ಕ್ರೀಡಾಮನೋಭಾವದ ಪರದೆ ಎಂದಿಗೂ ಬಿದ್ದುಹೋಗುತ್ತಿರಲಿಲ್ಲ. ಅವರು ಕ್ರೀಡಿಸುತ್ತಿದ್ದುದು “ಭಗವಂತ” ನೊಂದಿಗೆ ಮನದಾಳದಲ್ಲಿ. ಅವರಿಗೆ ಅದು ತಿಳಿದಿರುತ್ತಿದ್ದಿತು.

ಅಂತಹ ದಿವ್ಯ ಘಳಿಗೆಗಳನ್ನು ಪಡೆದಿದ್ದವರಿಗೆ ಬದುಕು ಹೆಚ್ಚು ಇನಿದಾಗಿರುತ್ತಿತ್ತು. ಹೆಚ್ಚು ಶ‍್ರೀಮಂತವಾಗಿರುತ್ತಿತ್ತು; ದೀರ್ಘ ರಾತ್ರಿಗಳಲ್ಲಿ ಮರಗಳ ಎಲೆಗಳ ನಡುವೆ ಸುಯ್ಯುವ ಸಮೀರಣನೂ ಸಹ “ಮಹಾದೇವ! ಮಹಾದೇವ!” ಎಂದು ಉದ್ಘೋಷಿಸುತ್ತಿರುವಂತಿತ್ತು!

