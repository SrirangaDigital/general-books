\chapter[{\sl Rasa:} From \textsl{Nāṭya-śāstra} to Bollywood]{{\sl Rasa:} From \textsl{Nāṭya-śāstra} to Bollywood}\label{chapter\thechapter:begin}
%~ \footnotetext[1]{pp.~\pageref{chapter\thechapter:begin}--\pageref{chapter\thechapter:end}. In: Kannan, K S (Ed.) (2018) {\sl Śāstra-s Through the Lens of Western Indology - A Response}. Chennai: Infinity Foundation India.}
\Authorline{Charu Uppal}
\lhead[\small\thepage\quad Charu Uppal]{}
\index{Natyasastra@\textsl{Nāṭya-śāstra}}

\textsl{Rasa}, meaning gist, is the essence that one feels while experiencing an art piece, be it performative or static art. In the Indian context, and as applied to both the performer and the audience, Rasa is considered an \textsl{alaukika}\index{alaukika@\textsl{alaukika}} (other worldly) experience. An integral part of aesthetics, both Indian and Greek (although European performing arts moved away from the original concept of Greek aesthetics), is improvisation on the rules that are suggested for a clear structure, which by definition is fluid and allows room for ‘newness. Using Bharat Gupt’s\index{Gupt, Bharat} study of poetics and \textsl{Nāṭya-śāstra},\index{Natyasastra@\textsl{Nāṭya-śāstra}} this paper will focus on the similarities in Indian and Greek aesthetics,\index{Greek!aesthetics} also highlighting when and why the contemporary notion of aesthetics in European theatre moved away from the Greek, which was more similar to the Indian sensibility. There will also be a focus on the concept of \textsl{hieropraxis} (art as worship, pleasing both people and Gods), which was common, to Indian and Greek art forms. Finally, the paper will illustrate, through examples of Bollywood\index{Bollywood} and interviews with \textsl{Bharatanāṭya}\index{Bharatanatya@\textsl{Bharatanāṭya}} teachers (in Sweden), how improvisation, and newness is brought into various aspects of performance arts, thereby challenging Prof. Sheldon Pollock’s\index{Pollock, Sheldon} reading of the \textsl{Nāṭya-śāstra}\index{Natyasastra@\textsl{Nāṭya-śāstra}} as being rigid and frozen in time, and as devoid of bringing novelty, thereby making the work irrelevant to our times. 

\newpage

\begin{flushright}
{\bf Performance arts and culture}\\[2pt]
Let \textsl{Nāṭya}\index{natya@\textsl{nāṭya}} (drama and dance) be the fifth Vedic scripture\\[2pt]
Combined with an epic story,\\[2pt]
tending to virtue, wealth, joy and spiritual freedom,\\[2pt]
it must contain the significance of every scripture,\\[2pt]
and forward every art.\\[2pt]
--- \textsl{Nāṭya-śāstra} 1.14--15\\[2pt]
In loving the spiritual, you cannot despise the earthly.\\[2pt]
Joseph Campbell\index{Campbell, Joseph}
\end{flushright}

\medskip

Although the story of Indra and the ants, (from the \textsl{Brahmavaivarta Purāṇa}) -\index{Purana@Purāṇa!Brahmavaivarta}\index{Brahmavaivartapurana@Brahmavaivarta Purāṇa} where Lord Viṣṇu, in the form a young boy with blue skin, visits Indra and convinces him to question his ego driven involvement in the world -- is quite well known, the story that follows is usually forgotten. Humbled by Viṣṇu's visit, Indira decides to renounce the world and become a yogi and meditate on the lotus feet of Viṣṇu. Indrāṇī, the beautiful wife of Indra, is upset by the news that Indra wishes to renounce the world, and goes to a priest for counsel. Understanding the dilemma of the queen, the priest says that he has a solution which would be pleasing both to Indra and Indrāṇī. As both approach Indra sitting on his throne, a symbol of power and authority, the priest says, 

\begin{myquote}
"Now, I wrote a book for you many years ago on the art of politics. You are in the position of the king of the gods. You are a manifestation of the mystery of Brahma in the field of time. This is a high privilege. Appreciate it, honor it, and deal with life as though you were what you really are. And besides, now I am going to write you a book on the art of love so that you and your wife will know that in the wonderful mystery of the two that are one, Brahma is radiantly present also." 

\hfill (Campbell\index{Campbell, Joseph} \textsl{et al} 1991:79).
\end{myquote}

The story is meant to illustrate the recognition and acknowledgement in Hinduism that both \textsl{pāramārthika}\index{paramarthika@\textsl{pāramārthika}} and \textsl{vyāvahārika}\index{vyavaharika@\textsl{vyāvahārika}} are important for a smooth functioning of \textsl{saṁsāra}. In the story, Indra finds that in life, 

\begin{myquote}
‘…..he can represent the eternal as a symbol...of the Brahman. So each of us is, in a way, the Indra of his own life. You can make a choice, either to throw it all off and go into the forest to meditate, or to stay in the world, both in the life of your job, which is the kingly job of politics and achievement, and in the love life with your wife and family.’

\hfill (Campbell\index{Campbell, Joseph} \textsl{et al} 1991:79).
\end{myquote}

All arts are supposed to give us a reflection of the \textsl{pāramārthika}\index{paramarthika@\textsl{pāramārthika}} in the \textsl{vyāvahārika}.\index{vyavaharika@\textsl{vyāvahārika}} In Indic tradition, the ultimate purpose of art, other than merely entertaining its audiences, has always been to bridge the gap between the worldly (\textsl{vyāvahārika}) and the transcendent (\textsl{pāramārthika})---by bringing together aspects of \textsl{laukika}\index{laukika@\textsl{laukika}} (the worldly) in the arts such as it points towards ways of experiencing the \textsl{alaukika.}\index{alaukika@\textsl{alaukika}}

\section*{Purpose of Art in Indic Tradition:}

Called the fifth Veda (like the Purāṇa-\textsl{s} and the Itihāsa), the \textsl{Nāṭya-śāstra}\index{Natyasastra@\textsl{Nāṭya-śāstra}}\endnote{\textsl{Nāṭya-śāstra---} where words were extracted out of \textsl{Ṛgveda},\index{Rgveda@\textsl{Ṛgveda}}\index{Veda-s@\textsl{Veda}-s!Rg@\textsl{Ṛg}} music from \textsl{Sāmaveda},\index{Samaveda@\textsl{Sāmaveda}}\index{Veda-s@\textsl{Veda}-s!Sama@\textsl{Sāma}} \textsl{abhinaya} out of \textsl{Yajurveda}\index{Yajurveda@\textsl{Atharvaveda}}\index{Veda-s@\textsl{Veda}-s!Yajus@\textsl{Yajus}} and \textsl{rasa} out of \textsl{Atharvaveda}\index{Atharvaveda@\textsl{Atharvaveda}}\index{Veda-s@\textsl{Veda}-s!Atharva@\textsl{Atharva}} and combined them with muthoi or Itihāsa to complete the fifth Veda called Nāṭya Veda (\textsl{Nāṭya-śāstra}:\index{Natyasastra@\textsl{Nāṭya-śāstra}} 1:11-19 as cited in Gupt\index{Gupt, Bharat} 2006:70).}---was composed to make the knowledge of the \hbox{Veda-s}\index{Veda-s@\textsl{Veda}-s} accessible to all factions of society, by evoking an experience in the audience that was commensurate with their own abilities of understanding art, and was beyond academic and formal knowledge. Though remembered more as a poet than a philosopher, Nobel Laureate Tagore,\index{Tagore, Rabindranath} in his essays defied all definitions of art by combining and comparing poetry with philosophy. Tagore believed that art relieved the audience from the clutches of reality, and moved the viewer/spectator into the other-worldly, not unlike what is proposed in the \textsl{Nāṭya-śāstra}. \index{Natyasastra@\textsl{Nāṭya-śāstra}}

\begin{myquote}
‘the artist helps us to forget the bonds with the world, and reveals to us the invisible connections by which we are bound up with eternity. True art withdraws our thoughts from the mere machinery of life, and lifts our souls above the meanness of it. It releases the self from the restless activities of the world, and takes us out ‘of the noisy sick-room of ourselves’.
\hfill (Radhakrishnan 1918:122)\endnote{These sentences follow the above quoted lines ‘It disengages the mind from its imprisonment in the web of customary associations and routine ideas. The secret of all art lies in self-forgetfulness. The poet or the artist sets free the poet or the artist in us. And this he can do only if his artistic creation is born of self-forgetful joy. The true artist lifts himself above the worldly passions and desires into the spiritual mood where he waits for the light. (Radhakrishnan\index{Radhakrishnan, S.} 1918:122).}
\end{myquote}

Attributed to Bharata Muni,\index{Bharatamuni} the \textsl{Nāṭya-śāstra}\index{Natyasastra@\textsl{Nāṭya-śāstra}} (NS) is a treatise on performing arts which, in over 6000 verses\endnote{The complete version contains 12000 verses. (as per Śāradātanaya)} and 36 chapters, provides guidelines on topics such as dramatic composition, how to structure a play and how to construct a stage, styles of acting, types of body movements, costumes and make up, role and goals of an art director. The \textsl{Nāṭya-śāstra} even makes comments on musical scales, musical instruments and how to integrate music with art performance.

\newpage

Some scholars think that the \textsl{Nāṭya-śāstra},\index{Natyasastra@\textsl{Nāṭya-śāstra}} whose dates are still debated, may not be the oldest work of its kind, but definitely the one that has survived (\index{Schwartz, S. L.}Schwartz 2004). The \textsl{Nāṭya-śāstra} that has influenced various forms of arts in India, namely, dance, music and literary traditions in India, is also known for propounding the \textsl{Rasa}\index{rasa theory@\textsl{Rasa} Theory} Theory, which stresses that although entertainment is a definite desired effect of performance arts, it is not the primary goal. This paper tries to establish how ‘\textsl{rasa}'\index{rasa@\textsl{rasa}} the core principle that defines ‘enjoyment or pleasure’ received from experiencing an artistic performance (or even eating a palatable meal), is, by nature, ever-present awaiting its manifestation through participation in a special experience, and that it must not be seen through the limited words such as aesthetic(s) or performance or pleasure (Schwartz 2004, Cush \textsl{et al} 2012). In addition, the paper establishes that sacredness is not unique to Indian texts, and that even pre-Christian European drama had a strong element of sacredness to it.

Before we move further, it is important to establish what constitutes a \textsl{Śāstra---} in \textsl{Nāṭya-śāstra}.\index{Natyasastra@\textsl{Nāṭya-śāstra}} Though misunderstood to be a dictate, \hbox{\textsl{Śāstra}-s} are an instrument of discipline (\textsl{śāsana}) and have been open to amendments, additions and subtractions, and therefore not rigid in their recommendations. Contrary to how some Western Indologists have approached them, \textsl{Śāstra}-s are guidelines for managing and creating through a particular art form or activity (Gupt 2006)\endnote{Lectures given at the Kalakshetra. 2012. \url{https://www.youtube.com/watch?v=bzVIrxjXIPo}}. Therefore, \textsl{Śāstra} has a \textsl{lakṣya} -- a purpose directed towards a discipline e.g.\@ if one wants to learn about governance one approaches \textsl{Arthaśāstra}, and if one wants to write poetry, one would consult \textsl{Nāṭya-śāstra}. However, it is important that \textsl{Śāstra-s} be approached for their usefulness with \textsl{śraddhā}, which is not blind faith but a curiosity for learning from a text. A text approached by \textsl{śraddhā}\endnote{\textsl{Śrāddha}, the ritual of acknowledging ancestors, is conducted because ancestors are considered worthy of respect and value.} will be approached for the value it has because until there is a belief in its value, the text’s essence will not reveal itself to the learner (Gupt 2006).\index{Gupt, Bharat}  Only those who have \textsl{śraddhā} and respect the texts have the \textsl{adhikāra} to read analyze and comment on the \hbox{\textsl{Śāstra}-s}. Furthermore, Bharata Muni\index{Bharatamuni} gave instructions on who qualifies to be a critic. Other than the knowledge of dance, music, customs and acting, a critic according to Bharata Muni, must have an open mind, which Pollock\index{Pollock, Sheldon} shies from, as he approaches the concept of ‘\textsl{rasa’}\index{rasa@\textsl{rasa}} with a pre-ascertained theory.

\newpage

Using works of several scholars -- but primarily Bharat Gupt,\index{Gupt, Bharat} Rajiv Malhora\index{Malhotra, Rajiv} and David Mason\index{Mason, David} -- this paper critiques Pollock’s\index{Pollock, Sheldon} misinterpretation of \textsl{rasa}, which is frozen in time and does not allow any novel creations. Pollock, who also studies the \textsl{Śāstra}-s, \textsl{sans} a sacred attitude, cannot be considered an \textsl{adhikārin}, for he discards the \textsl{alaukika}\index{alaukika@\textsl{alaukika}} and retains only the \textsl{laukika}.\index{laukika@\textsl{laukika}} As Malhotra\index{Malhotra, Rajiv} has argued, in Indic traditions \textsl{laukika} and \textsl{alaukika} are usually inseparable, as is also evident from the above story about Indra and Viṣṇu. Furthermore, the paper attempts at explaining the basic concept of \textsl{Rasa} Theory,\index{rasa theory@\textsl{Rasa} Theory} how it cannot be located or created, but must surface from the vast ocean of human consciousness due to a confluence of several factors. In addition, using David Mason’s\index{Mason, David} work, the paper explains \textsl{rasa}\index{rasa@\textsl{rasa}} as a conscious state. Finally, the paper establishes how modern mythology, namely India’s film industry continues to reflect the concept of ‘\textsl{rasa}' in various ways. 

\textbf{Pollock’s lens and interpretation of Indic art and ‘\textsl{Rasa}'}: Since the main aim of this paper is to challenge some of the assumptions and interpretations made by Pollock,\index{Pollock, Sheldon} we begin with his outlook on the concept of ‘\textsl{rasa}'.\index{rasa@\textsl{rasa}} Anyone familiar with Pollock’s writings knows that though he is a thorough scholar, with a vast knowledge of Indic traditions, he does not consider the context, and therefore does not see the whole, but only parts of Indian traditions\endnote{Pollock’s self-assurance is evident in several of his writings, e.g. within the first two paragraphs of his article of 2012, ‘\textsl{Vyakti} and the History of \textsl{Rasa}, he lays out a problem, and establishes himself correct in his position, merely by stating, ‘my account was correct’. He also self-cites himself in many of his papers proposing one idea and then taking them as if they were already proven valid merely by their publications.}. His insistence on desacralizing and ignoring the religious aspect of Indian arts, as propounded by several scholars (Schwartz\index{Schwartz, S. L.} 2004, Malhotra\index{Malhotra, Rajiv} 2011), actually disqualifies him of the \textsl{adhikāra} to comment on the \textsl{Śāstra}-s, which are prescribed to be studied in a sacred context. It is important to emphasize that theories and areas of study are lenses that are used as a guideline for viewing and interpreting the world. A very common example given is how many different interpretations can there be in a simple act of holding a door -- especially, if a man is holding the door for a woman? While an architect might notice the design and size of the door, a mathematician might look at the angle the door is held at, a gender studies scholar may study the same situation as a man-woman power vs. politeness equation. In that sense, using the Marxist theory of ‘aestheticization of power’, Pollock\index{Pollock, Sheldon} arrives at a conclusion that \textsl{kāvya}\index{kavya@\textsl{kāvya}} was essentially produced by a nexus of \textsl{brāhmaṇa}-\textsl{kṣatriya} who, making use of ‘embedded oppressive Vedic ideas to numb the masses into having a false sense of involvement with their rulers and, thereby, offering complete obedience’. (Malhotra 2016). For all the theories of oppression that Pollock propounds, he does not acknowledge the power of \textsl{Nāṭya-śāstra},\index{Natyasastra@\textsl{Nāṭya-śāstra}} in making the \textsl{Veda}-s\index{Veda-s@\textsl{Veda}-s} available to all (Malhotra 2016)\index{Malhotra, Rajiv}---including those who were not allowed to study them and those who due to their own limitations of understanding could not fathom them. 

Furthermore, Malhotra (2016) argues that Pollock not only treats \textsl{Veda}-s and \textsl{Śāstra}-s as irrelevant and focuses on \textsl{Kāvya}\index{kavya@\textsl{kāvya}} as “the primary field of cultural production” but also tries to remove any sacred connection between \textsl{Veda}-s, \textsl{Nāṭya-śāstra} and the subsequent \textsl{kāvya}. Therefore, Pollock’s prescription is simple -- a social disruption in India, by rejecting the sacred, the spiritual, the transcendental, (Malhotra\index{Malhotra, Rajiv} 2016) -- all the qualities, that make Indian texts, universal---to free it from oppression that is inherent in its texts. Several scholars including Pollock’s\index{Pollock, Sheldon} mentor Ingalls have warned against using Western lenses to study Indian texts, (Malhotra 2011, Malhotra 2016, Schwartz\index{Schwartz, S. L.} 2004), especially since in Western culture and literature, there has always been a distinction between religion and philosophy (Schwartz\index{Schwartz, S. L.} 2004:3) that renders the Western lens for reading of the Indic texts futile. Furthermore, while Pollock uses chronology and authorship to establish his points\endnote{First, while in the \textsl{Abhinavabhāratī}\index{Abhinavabharati@\textsl{Abhinavabhāratī}} it is found at the end of Abhinava’s\index{Abhinavagupta} review of the ideas of Bhaṭṭa Nāyaka,\index{Bhattanayaka@Bhaṭṭa Nāyaka} it is not self- evident that the verse is to be attributed to him. (Pollock 2012:242). It is of course entirely possible that a verse from Bhaṭṭa Nāyaka’s work could have been circulating anonymously and found its way into the \textsl{Vyaktiviveka}\index{Vyaktiviveka@\textsl{Vyaktiviveka}} (indeed, he may have taken it from the \textsl{Abhinava-bhārati} itself, though I know of no evidence that he had access to this work). (Pollock 2012:242).

More tellingly, we might wonder why Abhinavagupta should quote the verse immediately after citing two other verses from the work of Bhaṭṭa Nāyaka if it were not by the same author. (Pollock 2012:242).

Regardless of whether or not we ascribe the verse to Bhaṭṭa Nāyaka (though I think we should), or accept as genuine the reading \textsl{saṁvedanākhyayā} given in the manuscripts of the \textsl{Abhinavabhāratī}\index{Abhinavabharati@\textsl{Abhinavabhāratī}} (though I think we must), the verse would still appear to be the first instance of the migration of the idea of \textsl{vyakti} from its linguistic sense of manifestation of a latent meaning in the text, to its psychological sense, the revelation of a new consciousness in the viewer/reader. (Pollock 2012:244).}, it must be acknowledged it could be more than one author who wrote this treatise. As early as Abhinavgupta it was believed that the surviving text of \textsl{Nāṭya-śāstra} was not the work of a single Bharata, but that is was a coalition rendered by Bharata Muni\index{Bharatamuni} by combining the separate sets of three schools, \textsl{Brāhma-mata}, \textsl{Sadāśiva-mata} and \textsl{Bharata-mata} (Gupt\index{Gupt, Bharat} 2006). Furthermore, Śāradātanaya’s comment that Bharata Muni only reduced a treatise with 12000 verses to 6000, implies that the sage credited with authorship of \textsl{Nāṭya-śāstra} --may have contributed only partially to the work. In fact, this is not unique to \textsl{Nāṭya-śāstra} but rather typical of the way in which all Indian \textsl{Śāstra}-s were compiled. Great minds like Pāṇini, Vātsyāyana, Manu or Kauṭilya reviewed a particular branch of learning, gave it shape and coherence, reconciled differences of opinion but still left room for later additions (Gupt 2006). The concept of authorship and copyright itself is Western, and therefore using authorship and even chronology as criteria in the case of ancient texts is not always a reliable method of analysis. 

The following section analyses how theatre in both Indian and Greek\index{Greek!theatre} context was considered both a ritual and worship. 

\textbf{Sacredness in Indian and Greek Drama:}\index{Greek! drama} Although Gupt\index{Gupt, Bharat} (2006) does not recommend examining genres across cultures, because each should be studied in the context of its own history and culture, he considers comparisons of different modes of performance to be very instructive for a better understanding, mainly because he does not consider written texts the same as plays with performances that involved bodily gestures and languages. While sacredness is an accepted part of the Indian performance arts, not many know that even Greek theatre, before the advent of Christianity was instructive on the sacred---Heiropraxis. And yet, contrary to common belief, ancient Greek, Indian and Egyptian drama very well preserved the difference between theatre activity and religious rites. (Gupt 2006).

\begin{myquote}
``in ancient times festivals could be held on certain auspicious days only. Much has been written on the rhythms of these days only and their links with the seasonal and astronomical cycles. Whatever the logic, the purpose behind these fixations was always unambiguously to celebrate the visitation of something greater than man. Theatre was an integral part of this event. It was a substructure of the macro structure of the feast itself. It took the shape it did because it was enacted as one essential ceremony in a chain of many ceremonies.'' 
\hfill (Gupt\index{Gupt, Bharat} 2006:63)
\end{myquote}

Not solely a performance, ancient drama was actually both a prayer and a ritual, inviting and welcoming the gods, while sharing it with fellow human beings, whereas modern secular drama today is only used for its entertainment value. Gupt (2006), like many other scholars, cautions against a Darwinian mind-set to understand drama. 

\begin{myquote}
``but again we must distinguish between ritual and drama by recognizing their ends. One is prayer, the other is pleasure. One is essentially a rite of passage, from the point of not having to having, from being here to there, whereas the other may be called a rite of message, from person to person from the artist to audience. Attempts to place ritual myth and drama in a chain of evolutionary growth are not a representation of actually history but a result of Darwinian mind-set. We need not look upon them as one leading to another.''
\hfill(Gupt 2006:66)
\end{myquote}

\begin{myquote}
A ritual and drama have (\textsl{muthos}) the intersecting and coinciding ends, therefore some rituals turn into drama and others remain the same. \textsl{Garba} dance which was a ritual is presently enjoyed merely as an entertainment, while the swing festival of \textsl{teej}, which was an entertainment for the rainy season is now celebrated as a ritual. 

\hfill(Gupt\index{Gupt, Bharat} 2006:66).
\end{myquote}

In fact, while the\index{Greek!theatre} theatre of the Greek was reserved mainly for big celebrations, the same in India was even performed for family celebrations. Gupt equates \textsl{Ramlila} with \textsl{Eidolon} where a common understanding was that the Gods were themselves present at the performance as Divine spectators, making theatre a sacred viewing (Gupt 2006). Obligatory theater going ended with the advent of Christianity that considered drama an unholy, even a satanic act (Gupt 2006). Following that, drama and several other art forms gradually faded from the cultural scene. The revival of theatre’s link to the sacred happened after Europe came in contact with the traditions of Asia and Africa (Gupt 2006:64). In fact, the entire genre of performance studies was created only a few decades ago by Richard Schekhner, who was inspired by the Indian tradition in the 1950s (Gupt 2012).  However, although ancient drama was not merely ritualistic or merely religious, it came to be associated with worship (as such) because it was performed only on religious occasions and often within the premises of religious institutions. (Gupt\index{Gupt, Bharat} 2006). 

\vskip .1cm

The following section discusses the development of Indian drama, and the centrality of the sacred to Indian drama, the characteristics that make an ideal audience as listed by Bharata Muni,\index{Bharatamuni} how Indian and Greek\index{Greek!drama} drama developed independent of each other, and why unlike the \textsl{Poetics}, the principles of \textsl{Nāṭya-śāstra}\index{Natyasastra@\textsl{Nāṭya-śāstra}} actually can be applied both to drama and poetry. 

\vskip .1cm

\textbf{Indian drama:} The history of theatrical shows on festive situations is not so well documented in India, as it is in Greece. Since there is no mention, even in the \textsl{Nāṭya-śāstra}, about any precedent from where theatre could have been developed, scholars have often made several guesses on the development of the \textsl{daśa-rūpaka}-s\index{dasarupakas@\textsl{daśa-rūpaka}-s} (ten genres of acting) (Gupt 2006).  It is believed that the first Indian drama was puppetry and probably that is why the narrator is called a ‘\textsl{Sūtra-dhāra’}. However, Gupt (2006) argues that the proponents of the theory did not consider the possibility that this thread-bearer (\textsl{Sūtradhāra)} was so called because, like all architects, he carried a thread to measure the land for the theatre building---which was constructed anew each time a performance session was held (Gupt 2006:70). 

\vskip .1cm

\begin{myquote}
‘like all architects he carried a thread to measure the land for the theater building which was constructed anew each time a performance session was held. Besides the master director is called the \textsl{acarya} in the NS [\textsl{Nāṭya-śāstra}], and he was most likely not the same as \textsl{Sutradhara}. 
\hfill (Gupt\index{Gupt, Bharat} 2006:70)
\end{myquote}

Regardless, it is clear that there was more emphasis on symbolism through dance, which was an integral part of both Indian and Greek\index{Greek!theatre} theatre, in Indian drama rather than dialogue (Gupt 2006:66-67). Nevertheless, influenced by the biases of European realistic theatre, the orientalists of late 19th and 20th centuries focused more on ‘dialogue’ as opposed to symbolism and body movements, and looked for evidence in Sanskrit literature to support their preferences, finally to be found in the \textsl{Saṁhitā}-s. 

\begin{myquote}
These scholars therefore ransacked the Sanksrit literature for the earliest examples of dialogue and found them in the \textsl{Saṁhitā}-s of the \textsl{Ṛgveda}.\index{Rgveda@\textsl{Ṛgveda}}\index{Veda-s@\textsl{Veda}-s!Rg@\textsl{Ṛg}} Hence developed the theory of Vedic dialogues as precursors of Indian drama. As some examples of mime have been referred to in earlier Sanskrit texts, it was thought by one set of sanskritists that some sort of puppetry was the precedental form of Indian theatre. Similarly a group of scholars emphasized the secrets that the etymology of the word \textsl{sūtradhāra} may hold. This prologue speaker and director, it was argued was originally a string manipulator of the puppets who retained the nomenclature even when he became the play manager. 

\hfill (Gupt 2006:70)
\end{myquote}

Furthermore, the misconception mainly championed by Keith, postulates that Indian drama could not have come into its own without the highly developed structures of the \textsl{Rāmāyaṇa}\index{Ramayana@\textsl{Rāmāyaṇa}} and the \textsl{Mahābhārata},\index{Mahabharata@\textsl{Mahābhārata}} that were sung and narrated (Gupt 2006). While the dates of \textsl{Nāṭya-śāstra}\index{Natyasastra@\textsl{Nāṭya-śāstra}} are debated it has been established that it predates its Greek\index{Greek!drama} counterpart, the \textsl{Poetics} (Gupt 2006). Since \textsl{Poetics} was written after the best Greek works had been created, and \textsl{Nāṭya-śāstra} was compiled before any Indian plays were composed, the former is more ‘empirical’ and is concerned with ‘literary excellence’, where as \textsl{Nāṭya-śāstra} is more ‘emancipatory’ and pays attention to the ‘formulating principles of performance.’ (Gupt\index{Gupt, Bharat} 2006:14-15). 

While Pollock\index{Pollock, Sheldon} at several instances questions how a work written for drama could have been used to appraise a work of poetry\endnote{What remains troubling in this tentative reconstruction of mine is that the Indian tradition seems to have only rarely gestured toward, and never fully acknowledged, the transmutation of \textsl{vyakti} from a linguistic into a psychological phenomenon. The most telling case, I believe, is that of Ruyyaka\index{Ruyyaka} (c. 1150 C.E.). In his commentary on the \textsl{Vyaktiviveka}\index{Vyaktiviveka@\textsl{Vyaktiviveka}} he sets out to justify precisely what Mahima Bhaṭṭa\index{Mahimabhatta@Mahimabhaṭṭa} had sought to refute, namely, the applicability of \textsl{vyakti} to the notion of \textsl{rasa}.} -- the main reason that Bharata Muni\index{Bharatamuni} does not differentiate between \hbox{\textsl{daśa-rūpaka}-s}\index{dasarupakas@\textsl{daśa-rūpaka}-s} (the ten dramatic genres) from the Purāṇa-s or poetry, is that the \textsl{Nāṭya-śāstra} was written before any Indian plays were composed (Gupt 2006). Perhaps Pollock uses his understanding of Greek\index{Greek!poetics} \textsl{Poetics} where there is a clear distinction between drama and poetry, to question how \textsl{Nāṭya-śāstra} can comment both on poetry and drama. Gupt (2006) even disagrees with the notion that poetry may have come to India via Alexander, because he states that other than the possibility of \textsl{mimesis} (which might have developed into \textsl{anukaraṇa}) there are no signs of influence of poetics on \textsl{Nāṭya-śāstra}, or vice-versa. 

In the \textsl{Nāṭya-śāstra}\index{Natyasastra@\textsl{Nāṭya-śāstra}} (1.7-15), while commenting on the origin of drama, Bharata\index{Bharatamuni} states that due to the lack of audio-visual entertainment at the time, coupled with over -- indulgence in sensual pleasures that prevented people from contemplating higher values, he created drama as a positive distraction (as cited in Rangacharya\index{Rangacharya, A} 1966:66). Furthermore, it is important to acknowledge that, Bharata made the Indian performance arts deliberately and very conscientiously available to all. The ongoing accusations from Pollock\index{Pollock, Sheldon}
 that \hbox{\textsl{Śāstra}-s} were a conspiracy between Brahmins and the Kshatriyas can be countered by the fact that Bharata Muni, commented not only on the characters on the stage but also those in the auditorium---making the viewing egalitarian (Rangacharya\index{Rangacharya, A}
 1966). One of the dilemmas of Bharata when he advanced \textsl{Rasa} Theory\index{rasa theory@\textsl{Rasa} Theory} and defined dramatic representation was that though he believed in the equal availability of art to all, and the shows at the time had degraded to the level of ‘\textsl{grāmya}’ (vulgar), which made him want to raise the level of performances. However, if the shows were to be above the level of ‘\textsl{grāmya}' they might become too elitist, and a playwright will be compelled to restrict his audience to those with a higher level of understanding of the arts. Bharata\index{Bharatamuni} Muni resolved the conflict by suggesting that the shows should be open to all and use well-known stories, and even love stories so that ‘drama would capture the hearts of people of different tastes’ (Rangacharya\index{Rangacharya, A} 1966:74). While it seems that a combination of well-known stories and love stories is not likely to attract those with subtle tastes, Bharata Muni’s theories on \hbox{\textsl{sandhi}-s} and \textsl{rasa}-s\index{rasa@\textsl{rasa}} and enlightenment ‘could tempt a high-brow audience’ (Rangacharya 1966:74):

\begin{myquote}
A spectator is one who has no obvious faults, who is attached to drama, whose senses are not liable to distraction, who is clever in guessing (putting two and two together), who can enjoy (others’) joys and sympathize with (others’) sorrows, who suffers with those who suffer, and who has all these nine qualities in himself. 

\hfill (as cited in Rangacharya\index{Rangacharya, A} 1966:74)
\end{myquote}

Bharata\index{Bharatamuni} does not consider a person lacking imagination, is inebriated, is easily distracted or is not interested in drama and merely accompanying another spectator, an ideal audience. Therefore, a spectator according to Bharata must be able to ‘lose himself in the characters on the stage, their joys and sorrows (Rangacharya\index{Rangacharya, A} 1966:74). Such detailed and well thought out definitions and explanations form the basis of \textsl{Rasa} Theory,\index{rasa theory@\textsl{Rasa} Theory} which makes it relevant for the evaluation of art in all times. 

\textbf{\textsl{Rasa} and \textsl{Rasa} Theory:} What is ‘\textsl{Rasa}'?\index{rasa@\textsl{rasa}}
 ‘\textsl{Rasa}\endnote{In Ayurvedic terminology, the word \textsl{rasa} was used to denote the vital juice that the digestive system extracts from food and which is later converted into blood, flesh, bones, marrow, fat and sperm (\textsl{Suśruta Saṁhitā},\index{Susrutasamhita@\textsl{Suśrutasaṁhitā}} XIV, as cited in Gupt 2006:261).}’ is the term that Dewey lamented did not exist in English, a word that combines both the ‘artistic’ and the ‘aesthetic’ (Thampi 1965). Primarily derived from a reference to cuisine and concept of taste, ‘\textsl{rasa}' can mean essence, gist, or flavor. Bharata Muni\index{Bharatamuni} uses the word as an ‘extract’, since it is ‘worthy of being tasted (Gupt\index{Gupt, Bharat} 2006:261)\endnote{\textsl{Rasaḥ kaḥ padarthaḥ, ucyate āsvādyatvāt} \textsl{Nāṭya-śāstra}: 6:31 (as cited in Gupt 2006:261).} and considers it paramount, for without \textsl{rasa} no other purpose of an art is fulfilled (Rangacharya 1966).  

How do we use a word used to describe a dish to critique a dance performance? 

Just as a result of mixing of various spices and herbs to create a dish, a taste is produced in the one who consumes it, Bharata Muni\index{Bharatamuni} says that \textsl{rasa}\index{rasa@\textsl{rasa}} is produced by mixing of various \textsl{bhāva}-s (emotions) expressed in a performance in the consciousness of a spectator. The moment(s) between when a person consuming a meal, finishes his meal, sits in silence in contemplation of what he/she has experienced and before he/she expresses enjoyment -- is \textsl{rasa} (Rangacharya\index{Rangacharya, A} 1966). The experience of \textsl{Rasa} is similar to a ‘waking up’ of a feeling that has always existed, that though belongs to the consumer of the meal alone, does not reside anywhere in any of the spices, and may not be experienced the same way by the one who made the meal or any other consumer of the meal. ‘\textsl{Rasa} is both a state of being of the spectator and a climatic state’ (Baumer\index{Baumer, R. V. M}\index{Brandon, J. R.} and Brandon 1993:211).

While the later authors have tried to complicate this aspect (Rangacharya 1966), in reality the concept of \textsl{rasa}\index{rasa@\textsl{rasa}} is quite simple, something that tries to grasp the experience, resulting from subjective combination of taste buds, individual taste (preference for certain foods/arts or the ability to taste/understand) and the skill of the cook/playwright. Simplistically, \textsl{rasa} can be explained by the delight a person shows while consuming a meal that combines all tastes---sweet, pungent, hot, sour etc., in addition to other gestures such as facial or verbal expressions, of praise (Rangacharya 1966:76) e.g.\@ smacking lips, closing eyes or licking fingers. Does the person talk separately about each taste? Does one taste stand out the more? No, it is a combination of several factors, which though may be listed, cannot individually account for the final effect, which requires the consumer of the meal, and takes into account the preference for certain tastes. 

While \textsl{rasa}\index{rasa@\textsl{rasa}} is something that can be relished, enjoyed, appreciated like taste in food, or melody in music, and body’s movement in a dance, \textsl{bhāva} is conveyed by more concrete movements---e.g. bodily gestures, words, acting, expression etc. \textsl{Rasa}, which is only one of the eleven elements that a \textsl{Nāṭya}\index{nāṭya@\textsl{nāṭya}} (drama) consists of, is derived from ten other elements\endnote{\textsl{bhāva,\index{bhava@\textsl{bhāva}} abhinaya,\index{abhinaya@\textsl{abhinaya}} dharma, vṛtti, pravṛtti}-s, \textsl{sidhi}-s, \textsl{svara, ātodya, gāna} and \textsl{raṅga}.}. Similar to the experience of consuming a meal, ‘\textsl{rasa}’ emerges in watching the union of various \textsl{bhāva}-s (Rangacharya\index{Rangacharya, A} 1966:260). It must be noted, that \textsl{rasa} is the final stage that follows many others and refers to the unity of aesthetic experience combining the following eight \textsl{bhāva}-s (emotions), and not the other way around. 
\begin{center}
\begin{tabular}{|l|l|}
\hline
\textsl{Śṛṅgāra} & Love\\
\textsl{Hāsya}\index{hasya@\textsl{hāsya}}\index{rasa@\textsl{rasa}!hasya@\textsl{hāsya}} & Humor\\
\textsl{Karuṇa}\index{karuna@\textsl{karuṇa}}\index{rasa@\textsl{rasa}!karuna@\textsl{karuṇa}} & Compassion\\
\textsl{Raudra} & Horror\\
\textsl{Vīra} & Heroic\\
\textsl{Bhayānaka}\index{bhayanaka@\textsl{bhayānaka}}\index{rasa@\textsl{rasa}!bhayanaka@\textsl{bhayānaka}} & Fear\\
\textsl{Bībhatsa}\index{bibhatsa@\textsl{bībhatsa}}\index{rasa@\textsl{rasa}!bibhatsa@\textsl{bībhatsa}} & Awesome\\
\textsl{Adbhuta} & Wonder\\
\hline
\end{tabular}
\end{center}

A ninth emotion, \textsl{śānta\index{santa@\textsl{śānta}}\index{rasa@\textsl{rasa}!santa@\textsl{śānta}} bhāva} --- which is not recognized in drama --- was probably added later, since according to Bharata, all art leads to contentment. Bharata Muni uses four words in analyzing the conception of \textsl{bhāva}-s---which can be considered stages that lead to \textsl{rasa}. 
\begin{enumerate}
\itemsep=1pt
\item The external factor \textsl{vibhāva},\index{vibhava@\textsl{vibhāva}} (the cause), 
\item The immediate and involuntary reaction, \textsl{anubhāva},\index{anubhava@\textsl{anubhāva}} which is subjective
\item Voluntary control of the reaction, \textsl{vyabhicāri-bhāva}\index{vyabhicaribhava@\textsl{vyabhicāri-bhāva}}
\item The interval between involuntary show of expression and voluntary blocking of expressions called \textsl{sthāyibhāva}---
\end{enumerate}

‘It is the \textsl{sthāyibhāva} that is arrived at after the first three stages creates a ‘\textsl{rasa’} (Rangacharya\index{Rangacharya, A} 1966:79). That momentary rest, that the person eating the meal takes before expressing his delight, is the \textsl{sthāyibhāva}---the\index{sthayi-bhava@\textsl{sthāyibhāva}} moment of total immersion in the experience or enjoyment or reaction is the master, the ruler, and earlier three \hbox{\textsl{bhāva}-s} are subservient, or ‘servants’ (Rangacharya 1966:79) only secondary, to \textsl{rasa}, although all the \textsl{bhāva}-s jointly contribute to the ‘\textsl{rasa}.’\index{rasa@\textsl{rasa}} Since \textsl{sthāyi-bhāva} is the most dominant among the four, it is considered to constitute the \textsl{rasa}.

Just as mixing different tastes \textsl{rasa} is experienced, similarly mixing different \textsl{bhāva}-s---the \textsl{sthāyibhāva}\index{sthayi-bhava@\textsl{sthāyibhāva}} is transformed into \textsl{rasa}.\index{rasa@\textsl{rasa}} The success of a performance is determined by the extent of the appreciative spectators relishing a particular \textsl{rasa}. An artist’s ability to create within the boundaries of these rules indicates his ability to create \textsl{rasa} (Schwartz\index{Schwartz, S. L.} 2004). 


\textsl{Nāṭya-śāstra}\index{Natyasastra@\textsl{Nāṭya-śāstra}} states that the primary goal of an artistic performance is to transport the spectator in the audience into a parallel reality, which is beyond the physical, full of wonder, where he/she can experience the essence of his own consciousness, such that it leads him to reflect on spiritual and moral enquiry (Schwartz 2004). 

\textsl{Rasa} Theory,\index{rasa theory@\textsl{Rasa} Theory} though earlier associated only with drama, presently includes both poetry and drama (Rangacharya\index{Rangacharya, A} 1966:75), and expresses the primary goals of performing arts in India in all the major literary, philosophical, and aesthetic texts, and it provides the cornerstone of the oral traditions of transmission. It is also essential to the study production of any performance arts (Cush \textsl{et al} 2012), which in India always have a religious sensibility (Schwartz\index{Schwartz, S. L.} 2004). At this juncture, it is important to emphasize the \textsl{alaukika}\index{alaukika@\textsl{alaukika}} aspect of \textsl{rasa}, as a main defining quality of performance, going beyond text, as it combines acting, dancing performance and induces a \textsl{religious} response (Schwartz 2004). Scholars have reiterated that religion, art and philosophy in India were so intertwined that it is possible to study its religions through its performance arts (Schwartz 2004), challenging the methods used by Pollock\index{Pollock, Sheldon} that desacralizes all the \textsl{śāstra}-s. Furthermore, for Abhinavgupta, 

\begin{myquote}
“the aesthetic experience is... self luminous and self conscious, devoid of all duality and multiplicity... ‘in art, the purified state of undifferentiated experience was \textsl{rasa} or \textsl{ananda’}. Thus \textsl{rasa} becomes ‘a state of consciousness’ akin to the bliss of an enlightened soul” 

\hfill (Schwartz 2004:17)
\end{myquote}


Infact, the oft quoted ‘follow your bliss’ of Joseph Campbell,\index{Campbell, Joseph} implies that our true selves are revealed to ourselves in following what brings our soul to the level of a divine experience. Kapila Vatsyayan\index{Vatsyayan, Kapila} states that “Indian art is not religious, neither is there a theology of aesthetics, but the two fields interpenetrate because they share the basic world-view in general that of \textsl{moksha} and liberation in particular.'' (as cited in Schwartz\index{Schwartz, S. L.} 2004:17). 

While Pollock states that it is the viewer who makes the ‘judgment’\endnote{In short, what Ānandavardhana\index{Anandavardhana@Ānandavardhana} wants to understand is the basic mechanism immanent in the text by which \textsl{rasa} is made manifest in the character, and why this mechanism cannot be comprised under the normal verbal modalities of literal or figurative signification (\textsl{abhidhā},\index{abhidha@\textsl{abhidhā}} \textsl{lakṣaṇā}).\index{laksana@\textsl{lakṣaṇā}} Like all his predecessors he shows no interest whatever in \textsl{rasa} as an epistemological problem let alone in the subjective aspect of \textsl{rasa},\index{rasa@\textsl{rasa}} that is, the question of how the viewer/reader experiences it, though of course it is the viewer/reader who is always the one making the judgments about the successful or unsuccessful manifestation of \textsl{rasa} on the basis of his antecedent reactions. (Pollock 2012:235).} on \textsl{rasa}, it is important to note that ‘\textsl{rasa’} is an experience, not a judgment, nor evaluation.  Basically, Pollock\index{Pollock, Sheldon} believes that \textsl{rasa}\index{rasa@\textsl{rasa}} need not be visible, but since it cannot be located it must not exist the way it was understood. It is not clear why Pollock finds it hard to understand, because even to a school-teacher, after having taught for several years, it is apparent that the essence of understanding of a class lecture often rests on the prior reading/ effort /work, attention, interest in class, and understanding level of each student, which is reflected accordingly in the ‘aha moments’ in the class. What if \textsl{rasa} is explained as a state of consciousness? The following section discusses the debates about the universality of \textsl{rasa}, and how recent scholars have tried to explain it in terms of a mental state that cannot be located but only experienced, although it is reflected in certain physical changes.

\textbf{\textsl{Rasa}, as a conscious state:} There have been several debates among scholars about the universality of \textsl{rasa} (Baumer\index{Baumer, R. V. M}\index{Brandon, J. R.} and Brandon 1993). Can \textsl{rasa} be experienced only by Indians or only as response to an Indian drama? Some see \textsl{rasa} as ‘culture bound’, 

\vskip -.05cm

\begin{myquote}
\textsl{Rasa}\index{rasa@\textsl{rasa}} cannot be a universal concept, for the \textsl{rasa} response depends upon specific and selected cultural conditions (Deutsch). \textsl{Rasa} is not a possible response when a spectator is witnessing a Western tragedy (Gerow). \textsl{Rasa} must be culture-bound, since most of the \textsl{Nāṭya-śāstra}\index{Natyasastra@\textsl{Nāṭya-śāstra}} is taken up with describing which particular theatrical and dramatic arrangement of elements if appropriate to stimulate one or another \textsl{rasa} experience; the resulting Sanskrit play and its performance consequently are wholly different in kind from, say a Greek\index{Greek!tragedy} tragedy. 

\vskip -.13cm

\hfill (Baumer\index{Baumer, R. V. M}\index{Brandon, J. R.} and Brandon 1993:211-212)
\end{myquote}

\vfill\eject

For a non-Indian to experience \textsl{rasa}, a ‘cultural conditioning’ is a pre-requisite, as prescribed by \textsl{Nāṭya-śāstra.}\index{Natyasastra@\textsl{Nāṭya-śāstra}} However, other scholars consider \textsl{rasa} universal, equating it with ‘aesthetic joy’ (Raghvan and Shanta Gandhi, as cited in Baumer and Brandon 1993:212).  Regardless, Baumer and Brandon (1993) highlight that \textsl{rasa} being associated with emotions rather than intellect is (wrongly) denigrated in the West, because there is an enormous difference in say emotions derived/experienced from soap opera and \textsl{rasa}---for, the process is crude in Western soap opera, it is marvelously refined and artistic in India (Baumer\index{Baumer, R. V. M}\index{Brandon, J. R.} and Brandon 1993:212).

Gupt\index{Gupt, Bharat} (2006) however, compares \textsl{rasa} to catharsis, which he says is not mere relief, but should be 

\begin{myquote}
regarded as restoration to a state of pleasure not generally experienced [while] the process of \textsl{rasa} emergence requires the removal of obstructions [...].Katharsis\index{katharsis@\textsl{katharsis}} and \textsl{rasa}, with their separate points of emphasis, both begin with purification and end in delight. 

\hfill (Gupt 2006:272-73)
\end{myquote}

It is this experience of catharsis that is so accepted in appraising Western art performances that can be likened to \textsl{rasa} in the Indian context, implying that a similar concept was elucidated in both the West and the East. In fact, Richard Schechner has developed a performance theory combining the East and West concepts called, ‘\textsl{rasa} aesthetics’\index{rasa aesthetics@”\textsl{rasa } aesthetics”} which considers it from the point of view of changes, which occur in the nervous system. 

However, Mason (2015) not only considers \textsl{rasa} to be alive and universal, but also disagrees with the new theory of ‘\textsl{rasa} aesthetics’ as proposed by Richard Schekner, because he stresses that ‘\textsl{rasa’} and ‘aesthetics’ have little in common. 

\begin{myquote}
\textsl{Rasa}\index{rasa@\textsl{rasa}} is a conscious state having its own unitary and subjective quality, as well as its own content determined by memory and knowledge. Based on precepts of neural Darwinism, as articulated by Edelman, we can articulate \textsl{rasa} as a state of consciousness that arises from the contingent interactivity of brain systems, intentionality and attention. There is no \textsl{rasa} for a person not paying attention.    
\hfill (Mason\index{Mason, David} 2015:103)
\end{myquote}

Here Mason (2015) places attention at the center, without which no \textsl{rasa} can be experienced, no matter how aesthetic a performance is, for human consciousness ‘is a process, not a thing’, and while \textsl{rasa} ‘has a relationship with emotions, it is not dependent on them, nor brought out by them’ (Mason\index{Mason, David} 2015:101). Using cognitive theory and the universality of the ways human bodies interact with their environment, Mason counters Gerow’s (like Pollock’s)\index{Pollock, Sheldon} ideas that \textsl{rasa}\index{rasa@\textsl{rasa}} belongs to the past, and concludes that there is such a thing as a universal human experience, despite the recent trend of relativizing that experience:

\begin{myquote}
New historicism among other critical approaches thinks of human experience as fundamentally and inextricably embedded in particular cultural circumstances, and there are certainly very good reasons, as postcolonial theory has insisted, to resist tendencies to conflate disparate experiences, since such conflations often empty the histories of particular peoples of the meaning they uniquely derive from their experience. Even so, cognitive theory of the recent couple of decades employs compelling neurological evidence to argue that some human experience derives fundamentally from the ways in which human bodies interact with their physical environments available in the world, and given the limits on the range of ways in which human bodies can interact with those environments, the notion that we can recognize some experience and some meaning across cultures and historical periods is not absurd. While acknowledging the significant influence of unique cultures on the development and appreciation of art, renowned neuroscientist V.S. Ramachandran\index{Ramachandran, V. S.} argues that ’10 percent’ of art and art appreciation comes from ‘artistic universals’ (Ramachandran 2004:41).  

\hfill(Mason\index{Mason, David} 2015:102). 
\end{myquote}

These commonalities, say some scholars, arise from an experience that is grounded in the body, yet not located in it (Mason 2015)\endnote{Those universal elements that account for the infinitely unique and yet commonly understood phenomena of art derive from, according to philosopher Mark Johnson, a common human ‘grounding of metaphors in bodily experience’. (Johnson\index{Johnson, Mark} 2007:259), as cited in Mason\index{Mason, David} (2015:102).}. Using Edelman’s theory on consciousness and cognitive theory, Mason (2015) illustrates how \textsl{qualia},\index{qualia@\textsl{qualia}} subjectivity of an object, is used by the brain to survey and understand its environment. Just as consciousness works with \textsl{qualia} to contextualize an object in its environment, ‘\textsl{rasa} accompanies the disclosure of \textsl{bhāva-s’} (Mason\index{Mason, David} 2015:106). English words such as feeling and emotion (which even Johnson states are not the same (Mason 2015:106)) cannot be equated with \textsl{bhāva}, which is sensation itself. Despite its culinary origins. The concept of \textsl{rasa} is not to be likened to the physical mechanism of tasting, as Pollock and other modern scholars seem to do.

\textsl{Rasa}\index{rasa@\textsl{rasa}} according to Gerow is an organizing principle -- an ‘emotional consciousness’, where all ‘elements of the play’ are seen as one unit, contributing to the final experience (Mason\index{Mason, David} 2015:107). For that reason Bharata Muni,\index{Bharatamuni} stresses the characteristics of the \textsl{sumanas} (audience members) –as being ‘of a like mind with the production’ and \textsl{budha} (knowledgeable or conscious) (Mason 2015:107). Aware that the performance itself is transitory, and that without the appropriate \textsl{characteristics} of the audience as listed in \textsl{Nāṭya-śāstra},\index{Natyasastra@\textsl{Nāṭya-śāstra}} \textsl{rasa} cannot be created, since the ‘audience member’s very self is the site of \textsl{rasa}’ (Mason 2015:107) A spectator cannot be given or asked to expect \textsl{rasa} because it requires spectator’s involvement, through his understanding of the performance, and depends on:

\begin{myquote}
neural pathways that form and operate as a consequence of a particular audiences member’s life experiences, but that respond to a new experiences in creative, astonishing and unpredictable ways 

\hfill (Mason 2015:108) 
\end{myquote}

This ‘mode of reflection’ resulting from the stimulus of the theatrical event that takes the spectator deep into a self-aware consciousness is ‘\textsl{rasa’}, and like consciousness it is not a result of experiences, but one with it.

\textbf{\textsl{Rasa}\index{rasa@\textsl{rasa}} Continues into Bollywood:}\index{Bollywood} \textsl{Rasa} as a method of performance has been an integral part of Indian cinema, giving it a distinct presence apart from Western cinema. In contrast to the Western method acting, where an actor must embody the character he plays, the \textsl{rasa} method emphasized conveying an emotion, as demonstrated in his films by the Oscar winning director Satyajit Ray,\index{Ray, Satyajit} who influenced many directors in the West. \textsl{Rasa}, as a concept itself became the theme and was used as a part of the plot, in \textsl{Naya Din, Nayi Raat} where the nine main characters, all of which were played by Sanjeev Kumar, each represented a different \textsl{rasa}.\index{rasa@\textsl{rasa}} \textsl{Rasa} should not be confused with genre which ‘attempts to convey an emotion through characters, situations, or \textsl{mise-en-scene}’ (Kumar 2014:5). But according to \textsl{Rasa} Theory, the performers must become ‘the living embodiment of the \textsl{rasa} they are depicting’ (Kumar 2014:5). Ganti (2013) argues that when scholars criticize Bollywood\index{Bollywood} for not having a genre/plot/style that is because they are refusing to examine and explore Bollywood on its own terms, and are applying non-Bollywood concepts to its study. Bollywood defies any adherence to genre, mainly because it has borrowed much from \textsl{Nāṭya-śāstra}\index{Natyasastra@\textsl{Nāṭya-śāstra}} and often applied the concept of \textsl{rasa} in bringing back audiences to the theatre. Using the example of supporting cast in Bollywood\index{Bollywood} movies, Kumar (2014) explains how certain actors who had very limited screen time were not only used successfully but also helped create a stereotype, which has continued in Bollywood till this day in attracting audiences. So, using six actors who represented certain \textsl{rasa} e.g. Om Prakash (\textsl{śānta}\index{santa@\textsl{śānta}}\index{rasa@\textsl{rasa}!santa@\textsl{śānta}} -- old man), Leela Misra (\textsl{karuṇa}\index{karuna@\textsl{karuṇa}}\index{rasa@\textsl{rasa}!karuna@\textsl{karuṇa}} -- adorable aunt), Helen (\textsl{śṛṅgāra}\index{srngara@\textsl{śṛṅgāra}}\index{rasa@\textsl{rasa}!srngara@\textsl{śṛṅgāra}} -- seductress), Kanhaiya Lal (\textsl{bībhatsa}\index{bibhatsa@\textsl{bībhatsa}}\index{rasa@\textsl{rasa}!bibhatsa@\textsl{bībhatsa}} -- cunning money lender), Tun-Tun/Uma Devi (\textsl{hāsya}\index{hasya@\textsl{hāsya}}\index{rasa@\textsl{rasa}!hasya@\textsl{hāsya}} -- clumsy maid), Lalita Pawar (\textsl{bhayānaka}\index{bhayanaka@\textsl{bhayānaka}}\index{rasa@\textsl{rasa}!bhayanaka@\textsl{bhayānaka}} -- wicked mother-in-law), Kumar illustrates how each of these characters were popular because they ‘represented the social network of the spectator’ (Kumar 2014:7). They formed an aspect of everyday life of spectators and through their roles, were able to evoke a \textsl{rasa} in the audience by combining various \textsl{bhāva}-s. The audiences returned repeatedly to watch these characters, despite their limited screen time. The mere presence of these actors, evoked a \textsl{rasa}\index{rasa@\textsl{rasa}} from memory and knowledge of the spectators, which was one with their lived experience of knowing similar characters in real life, making the movie viewing experience seem like being a part of an extended family.

\begin{thebibliography}{99}
\itemsep=2pt
\bibitem[]{chap4_item1}
Baumer, R. V. M., \& Brandon, J. R. (1993). \textsl{Sanskrit Drama in Performance.} Vol.~2. New Delhi: Motilal Banarsidass Publication. 

\bibitem[]{chap4_item2}
Campbell, J., Moyers, B. D., \& Flowers, B. S. (1991). \textsl{The Power of Myth} (1st edition). New York: Anchor Books.

\bibitem[]{chap4_item3}
Cush, D., Robinson, C., \& York, M. (2012). \textsl{Encyclopedia of Hinduism.} New York: Routledge.

\bibitem[]{chap4_item4}
Chakravorty, P. (2004). “Dance, Pleasure and Indian Women as Multisensorial Subjects.” \textsl{Visual Anthropology}, 17(1). pp.~1--17. 

\bibitem[]{chap4_item5}
Daboo, J. (2009). “To learn Through the Body: Teaching Asian Forms of Training and Performance in Higher Education.” \textsl{Studies in Theatre and Performance}, 29(2). pp.~121--131.

\bibitem[]{chap4_item6}
Ganti, T. (2013). \textsl{Bollywood: a Guidebook to Popular Hindi Cinema.} New York: Routledge. 

\bibitem[]{chap4_item7}
Gupt, Bharat (2006). \textsl{Dramatic Concepts Greek \& Indian: A Study of the Poetics and the Nāṭya-śāstra.} New Delhi: DK Printworld.

\bibitem[]{chap4_item8}
---\kern3pt(2012). Lectures for Kalashetra. \url{https://www.youtube.com/watch?v=zR2nCmB3AMo.} Accessed on 20th November 2016.

\bibitem[]{chap4_item9}
---\kern3pt(2015). “Re-establishing Śāstra-s”. In \textsl{IndiaFacts}. \url{http://indiafacts.org/re-establishing-Śāstra-s-indian-education/}. Accessed on 20th November 2016.

\bibitem[]{chap4_item10}
Kumar, C. B. (2014). “The Popularity of the Supporting Cast in Hindi Cinema.” \textsl{South Asian Popular Culture}, 12(3). pp.~189--198.

\bibitem[]{chap4_item11}
Mason, D. (2015). “\textsl{Brat Tvam Asi}: \textsl{Rasa} as a conscious state.” In Nair (2015). pp.~99--112.

\bibitem[]{chap4_item12}
Nair, S. (Ed.). (2015). The \textsl{Nāṭya-śāstra} and the Body in Performance: Essays on Indian Theories of Dance and Drama. New York: McFarland.

\bibitem[]{chap4_item13}
Pollock, S. (2012). “Vyakti and the History of \textsl{Rasa}.” \textsl{Vimarsha, Journal of Rashtriya Sanskrit Sansthan} (World Sanskrit Conference Special Issue), 6. pp.~232--253.

\bibitem[]{chap4_item14}
Rangacharya, A. (1966). \textsl{Introduction to Bharata's Nāṭya-Śāstra.} Bombay: Popular Prakashan.

\bibitem[]{chap4_item15}
Malhotra, R. (2013). \textsl{Being Different: An Indian Challenge to Western Universalism}. Noida: Harpercollins India.

\bibitem[]{chap4_item16}
---\kern3pt(April 4, 2016). “Rajiv Malhotra explains the challenges of understanding Sheldon Pollock.” \textsl{Swarajya}. Online Edition. Accessed on 24th May 2016.

\bibitem[]{chap4_item17}
---\kern3pt(2016). \textsl{The Battle for Sanskrit: Is Sanskrit Political or Sacred, Oppressive or Liberating, Dead or Alive?} India: Harper Collins.

\bibitem[]{chap4_item18}
Radhakrishnan, S. (1918). \textsl{The Philosophy of Rabindranath Tagore.} Macmillan. 

\bibitem[]{chap4_item19}
Schwartz, S. L. (2004). \textsl{Rasa}: \textsl{Performing the divine in India}. New York: Columbia University Press. 

\bibitem[]{chap4_item20}
Thampi, G. M. (1965). " \textsl{Rasa}" as Aesthetic Experience. \textsl{Journal of Aesthetics and Art Criticism.} pp.~75--80.
\end{thebibliography}

\theendnotes 
