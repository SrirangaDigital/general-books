\makeatletter
\def\@makeschapterhead#1{%
  \vspace*{50\p@}%
  {\parindent \z@ \raggedleft
    \normalfont
    \interlinepenalty\@M
    \LARGE \bfseries  #1\par\nobreak
    \vskip 20\p@
  }}
\makeatother

\chapter*{Our Contributors\\ {\rm\sl\small (in alphabetical order of last names)}}\label{contributors}


\lhead[\small\thepage\quad Our Contributors]{}
\rhead[]{\small Our Contributors\quad \thepage}
\chead[]{}
\cfoot[]{}

\section*{Naresh Cuntoor}

Dr.~Naresh Prakash Cuntoor is a Senior Research Scientist, Intelligent Automation Inc.,
Rockville, MD, US. He has an M.S. and PhD from the Department of Electrical and
Computer Engineering, University of Maryland. His research interests include human
activity recognition, scene understanding, perceptual organization and computer vision for
robotics applications. He pursues Sanskrit with keen interest and is a volunteer for
Samskrita Bharati USA.

\section*{Sreejit Datta}

Sreejit Datta is currently pursuing his PhD in Comparative Literature at the Centre
for Comparative Literature, Bhasha Bhavana, Visva Bharati, Shantiniketan. He is
an accomplished musician and has been obtaining training since the past 15 years
without a break He was awarded the Young Artiste Scholarship in the category
Hindustani Light Classical Music (with a specialization in Rabindrasangeet,
Tagore's songs) by the Ministry of Culture, Government of India for a two-year
advanced training in his chosen field in the year 2010.

\section*{R. Ganesh}

Shatāvadhānī R.~Ganesh is an Engineer and Metallurgist by training, a Sanskrit and Kannada
poet, and a practitioner of the traditional art form of Avadhāna that is chiefly characterized by
the composition of extempore poetry. He has performed more than 1000 Avadhāna-s till date.
He is also a Sanskrit and Kannada scholar of great standing. He regularly gives discourses on
a variety of subjects pertaining to Indian culture and Classical Indian, mostly Sanskrit and
Kannada, poetry. He has composed several original poems in Sanskrit and Kannada and has
translated many Sanskrit works into Kannada as well as Kannada works into Sanskrit. He
takes pleasure in training the younger generation to compose Classical Sanskrit and Kannada
poetry. He is a recipient of numerous awards, prominent among which are the Rajyotsava
prashasti in 1992 and the Badarayan Vyas Samman in 2003. His D.Litt Degree is on the art of
Avadhāna in Kannada language.

\section*{K. Gopinath}

K.~Gopinath is a professor at Indian Institute of Science in the Computer Science and
Automation Department. His research interests are primarily in the computer systems area
(Operating Systems, Storage Systems, Systems Security and Systems Verification). He is
currently an associate editor of IEEE Letters of Computer Society (2018-); previously he was
also an associate editor of ACM Trans. on Storage (2010-2017). His education has been at
IIT-Madras (B.Tech'77), University of Wisconsin, Madison (MS'80) and Stanford University
(PhD'88). He has also worked at AMD (Sunnyvale) ('80-'82), and as a PostDoc ('88-'89) at
Stanford. He is also interested in understanding the Indic contributions in the area of
computer science and more broadly in S\&T.

\section*{Ashay Naik}

Ashay Naik is a software developer at Matific Ltd. and has just released his first book
\textsl{Natural Enmity: Reflections on the Niti and Rasa of the Panchatantra [Book 1]}. He has a
Masters in Information Technology from the Queensland University of Technology,
Australia and an Honours in Sanskrit from the University of Sydney, Australia.

\section*{Shankar Rajaraman}

Shankar Rajaraman is an allopathic doctor, a psychiatrist, and an award-winning Sanskrit
poet. Some of his Sanskrit works include Bhārāvatārastava, a collection of 50 verses in
praise of Shiva, \textsl{Nipuṇaprāghuṇaka}, a Sanskrit drama of the Bhāṇa type that deals with a
contemporary theme, \textsl{Vaidyopahāsakalikā}, a satire in 40 verses on physicians, and
\textsl{Devīdānavīya}, a poem in 3 chapters that illustrates the various categories of constrained
poetry for which the Karnataka Samskrit University awarded him with the Professor M.
Hiriyanna Sanskrit Works Award in 2013. His latest published work is \textsl{Citranaiṣadham}, a
first-of-its-kind Sanskrit narrative composed entirely in the \textsl{zig-zag gomūtrikābandha} pattern.
He has been awarded the Badrayan Vyas Samman for the year 2016 and Bannanje Puraskara
for the year 2017. He has also translated texts from Sanskrit to English, Sanskrit to Kannada,
and Kannada to Sanskrit. His doctoral thesis on self-conscious emotions is situated at the
crossroads of psychology and Sanskrit poetics.

\section*{Charu Uppal}

Charu Uppal, is a Senior Lecturer at Karlstad University in Sweden. Her research, which
generally follows under the broad umbrella of media studies focuses on the role of media in
bringing about social change, identity formation and mobilizing citizens towards cultural and
political activism. Her work has appeared in journals such as Journal of Creative
Communication, International Communication Gazette and Global Media and
Communication.
