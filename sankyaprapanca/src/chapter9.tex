\chapter{aparimeVya saMKeyxgaLu}

$ 4,9,16,25\cdots$ muMtAda saMKeyxgaLu pUNaRvagaR saMKeyxgaLu. ivugaLige pUNaRvagaR mUlagaLive. Adare 
$2,3,5,7$ muMtAda saMKeyxgaLige pUNaRvagaRmUlaviruvudilalx ivelAlx ivelAlx apameVriya saMKeyxgaLu. I 
riVtiya saMKeyxgaLige vagaRmUlavanunx kaMDu hiDiyalu sAdhayxvilalx. AdarU sariyAda utatxrakekx 
samiVpada beleyanunx kaMDuhiDiyabahudu.

udAharaNege: \qquad $2$ ra vagaRmUla
$$
\sqrt{2} = 1.41421
$$
$2$ ra vagaRmUlavanunx $5$ dashamAMsha sAthxnadavarege bareyalAgide. idanunx hiVgeyeV muMduvarisabahudu I riVtiya saMKeyxgaLanunx aparimeVya saMKeyxgaLu enunxtetxVve.

peYthAgorasfna parxmeVyada parxkAra {\bf aparimeVya saMKeyxgaLa vagaRmUlavanunx} \-reVKAgaNitada riVti kaMDuhiDiyabahudu.

peYthAgorasana parxmeVya namagelAlx tiLidide. 

laMbakoVna tirxBujadalilx vikaNaRda meVlina vagaRvu uLideraDu bAhugaLa vagaRgaLa motatxkekx sama.

\begin{center}
{\rm fig1}
\end{center}
{\rm OAB} {\text oMdulaMbakoVna tirxBuja}\\
{\rm OAB} $= 90\degree$\\
{\rm OA} matutx {\rm AB} oMdu seMmiV Adare peYthAgorasfna parxmeVyada parxkAra
\begin{align*}
OB^2 &= OA^2+AB^2\\
&= 1^2+1^2\\
&= 1+1\\
OB^2 &= 2\\
OB &= \sqrt{2}
\end{align*}
vikaNaR {\rm OB} ya udadxvanunx ALedAga $\sqrt{2}$ ra belegotAtxgutatxde.
\begin{center}
{\rm fig2}
\end{center}

\begin{gather*}
\text{Iga}\; {\rm OBC} \;\text{tirxBujadalilx}\; {\rm B} = 90\degree\\
{\rm OB}=\sqrt{2}, {\rm BC} = 1 \text{seMmiV}
\end{gather*}

\begin{align*}
OC^2 = OB^2+BC^2\\
= (\sqrt{2})^2+(\sqrt{1})^2\\
=2+1\\
OC^2= 3\\
OC = \sqrt{3}
\end{align*}

$3$ omdu aparimeVya saMKeyx adara vagaRmUlavanunx {\rm OC} yanunx aLeyuva mUlaka tiLiyabahudu.
\begin{center}
{\rm fig3}
\end{center}
\begin{gather*}
\text{Iga}\; {\rm OCB} \;\text{tirxBujadalilx}\; {\rm c} = 90\degree\\
{\rm OC}=\sqrt{3}, {\rm CD} = 1 \text{seMmiV}
\end{gather*}
\begin{align*}
OD^2 &= OC^2+CD^2\\
&= (\sqrt{3})^2+(\sqrt{1})^2\\
&=3+1\\
OD^2&= 4\\
OC &= \sqrt{4}\\
&= 2
\end{align*}
hiVgeyeV muMduvarisidare $5,6,7,8,9,10$ ra {\bf vagaR mUlavanunx kaMDuhiDiyabahudu.}

meVle heVLiruva saMKeyxgaLalilx $5,6,7,8,10$ ivugaLelAlx aparimeVya saMKeyxgaLu.
\begin{flalign*} 
\sqrt{1}&=1.000\\
\sqrt{2}&=1.414\\
\sqrt{3}&=1.732\\
\sqrt{4}&=2.000\\
\end{flalign*}



\begin{tikzpicture}
\draw (0,0) -- (4,0) -- (0,4)--(0,0);
\end{tikzpicture}.
