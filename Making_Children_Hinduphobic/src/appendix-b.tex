\chapter{Appendix B: May 17, 2017 Letter To Sbe}

May 17, 2017

Dear Members of the Instructional Quality Commission and the State Board of Education:

Greetings!

Reviewing the drafts by publishers of the History and Social Science sixth-grade textbooks—in particular by Houghton MifflinHarcourt Publishing Company and Discovery Education—I was struck by how these textbooks are quite explicit in equating Hinduism to casteism to hierarchy to oppression. Mathematically put, this is what the discourse on Hinduism looks like: Hinduism = caste = hierarchy = oppression. These equations are extremely problematic in terms of how Hinduism has been or is pertaining to its cosmology and worldview. In addition they, having their source in the “History-Social Science Content Standards for California Public Schools: Kindergarten Through Grade Twelve” (henceforth called Content Standards) and the “History-Social Science Framework: Adapted by the State Board of Education on July 14, 2016” (henceforth called the Framework), lead to stereotyping and bullying of Hindu children as many pieces of testimony to IQC, over the past many months, suggest. Consequently, the aforementioned representations of Hinduism, either in textbooks or in the Framework, are in gross violation of the following guidelines or codes as mandated in the “Standards for Evaluating Instructional Materials for Social Content: 2013 Edition.” Let us call it the “Social Content Document”: 
\begin{quote}
The Education Code sections referenced in this document are intended to help end stereotyping in instructional materials by showing diverse people in positive roles contributing to society. Instructional materials used by students in California public schools should never portray in an adverse or inappropriate way the groups referenced in the laws. (California Department of Education, 2016, p. 2)
\end{quote}
In the Framework, Hinduism is discussed between pages 216-219, and between lines 865-935. Out of these, lines 865-920, discuss the issue of caste, which basically leaves 15 lines dedicated to other issues (which, though not covered in this letter, incidentally also lead to negative portrayals). An overemphasis on caste essentializes the conflation of Hinduism with caste, limits the portrayal of Hinduism, and narrows it expanse. Consequently, it violates 3 and 3a of the “Social Content Document”:

\begin{quote}
3. Special purpose—limited portrayals. Several kinds of circumstances make it necessary to modify requirements regarding proportion and balance of portrayals. These circumstances do not eliminate the need to carefully review for adverse reflection or derogatory references, but they do make it difficult to achieve the usual kind of required balance.
\renewcommand{\theenumi}{\alph{enumi}}
\renewcommand{\labelenumi}{\theenumi.}
\begin{enumerate}
\item Narrow focus—limited scope and content. An evaluator must consider the number of characters presented and the relationships among them; if the material includes only three or four main characters or if all of the main characters are members of the same family, obviously it will be unrealistic to expect portrayal of a wide diversity of ethnic groups or roles and contributions. If the setting is restricted to a limited locale, such as an inner-city ghetto or a sparsely settled desert region, the possibilities for showing a wide range of socioeconomic groups in a wide range of activities are necessarily limited. \textit{Materials with a narrow focus and/or limited portrayals should be clearly identified as such so that no false impressions are conveyed}. (California Board of Education, 2013, p.\ 3, my italics) 
\end{enumerate}
\end{quote}

The continued conflation of Hinduism with caste, and by default with hierarchy and oppression causes shame, impacting the healthy identity formation of the Hindu children. There are innumerable instances where the Hindu kids have either distanced themselves with the heritage of their parents and ancestors or have lived a life of double identity, walking extra miles to dis-identify themselves from their tradition in their own social and public sphere. The Framework and the majority of the current textbooks in production are in violation of the educational codes, section 51501, 60040 (b), and 60044 (a), pertaining to matters of “Ethnic and Cultural Groups.” The “Social Content Document” outlining the purpose of standards states: 
\begin{quote}
The standards project the cultural diversity of society; instill in each child a sense of pride in his or her heritage; develop a feeling of self-worth related to equality of opportunity; eradicate the roots of prejudice; and thereby encourage the optimal individual development of each student. (California Department of Education, 2013, p.\ 5)
\end{quote}
The conjoining of Hinduism with caste, far from instilling pride, instills shame and a lack of self-worth. They become targets of prejudice among their peers, and their healthy development gets impacted. IQC has been familiar with the consequences as many children have, over the past many months during the course of hearings for the revision of the Framework, testified in front of the commission regarding the negative consequences of the negative and skewed portrayal of Hinduism. As a result, the current discourse on Hinduism in the Framework and the textbooks consequently violate the “adverse reflection” clause of the standards under the “Ethnic and Social Groups.” 
\begin{quote}
\textit{Adverse reflection.} Descriptions, depictions, labels, or rejoinders that tend to demean, stereotype, or patronize minority groups are prohibited. (California Board of Education, 2013, p.\ 6)
\end{quote}
Further, the standards in the “Social Content Document” pertaining to religious matters specifically state the following:
\begin{quote}
The standards enable all students to become aware and accepting of religious diversity while being allowed to remain secure in any religious beliefs they may already have. (California Board of Education, 2013, p.\ 10)
\end{quote}
And that
\begin{quote}
The standards will be achieved by depicting, when appropriate, the diversity of religious beliefs held in the United States and California, as well as in other societies, without displaying bias toward or prejudice against any of those beliefs or religious beliefs in general. (California Board of Education, 2013, p.\ 10)
\end{quote}
Given that the above standards emanate from the constitutions of the United States and California, the document mandates compliance insisting

\begin{quote}
1.\ \textit{Adverse reflection}. No religious belief or practice may be held up to ridicule and no religious group may be portrayed as inferior.

2.\ \textit{Indoctrination.} Any explanation or description of a religious belief or practice should be presented in a manner that does not encourage or discourage belief or indoctrinate the student in any particular religious belief. (California Board of Education, 2013, p. 10)
\end{quote}

Based on the testimonies of the Hindu children provided to the IQC, it should be more than apparent the discourse in current textbooks emanating from the Framework (which incidentally is not much different from the earlier one vis-à-vis Hinduism and caste) does not allow them to remain secure in the beliefs of their tradition or the practices of their parents. The Framework, while it denigrates Hinduism simultaneously speaks positively about other religions such as Judaism, Buddhism, and Jainism (as it should and must). Regarding Judaism, it states:
\begin{quote}
While their state did not long survive, their religion, which became known as Judaism, made an enduring contribution of morality and ethics to Western civilization…. While many of main teachings of Judaism, such as a weekly day of rest, observance of law, practice of righteousness and compassion, and belief in one God, originated in the early traditions of the Jews, other early traditions disappeared over time to be replaced by increased emphasis on morality and commitment to study. (California Board of Education, 2016, p. 204)
\end{quote}
Regarding the Buddha and Buddhism, it states: 
\begin{quote}
Through the story of his life, his Hindu background, and his search for enlightenment, students may learn about his fundamental ideas: suffering, compassion, and mindfulness. Buddhism teaches that the path to liberation from the wheel of death and rebirth is through the transformation of selfish desires. (California Board of Education, 2016, pp.\ 219--220) 
\end{quote}
With regards to Jainism, the following are the Framework’s contentions:
\begin{quote}
Jainism, a religion that embraced the dharmic idea of ahimsa, or nonviolence, paralleled the rise of Buddhism. Jainism promoted the idea of ahimsa (non-violence to all life), especially in the form of vegetarianism. It has continued to play a role in modern India, notably in Mohandas Gandhi’s ideas of nonviolent disobedience. (California Board of Education, 2016, p.\ 220)
\end{quote}
The Framework in other words singles out Hinduism, exposing its adherents—the Hindu kids—to ridicule and subtly portrays it to be inferior. In the contemporary world, no kid would want to be associated with a system of belief which is being represented as hierarchical and oppressive—this is not rocket science and it should not be considered as such. Besides, whether Hinduism is inherently hierarchical and oppressive or not is a matter of considerable academic debate. Singling it out for negative portrayals tantamount to prejudice and discrimination. It is time that IQC and SBE notice it and go on a course correction. 

\begin{thebibliography}{99}
\bibitem{apx-b-key1} California Board of Education (2013). \textit{Standards for evaluating instructional materials for social content: 2013 edition}. Retrieved from \url{http://www.cde.ca.gov/ci/cr/cf/lc.asp}

\bibitem{apx-b-key2} California Board of Education (2016). \textit{History-social science framework: Adapted by the State Board of Education on July 14, 2016}. Retrieved from \url{http://www.cde.ca.gov/ci/hs/cf/sbedrafthssfw.asp}
\end{thebibliography}
