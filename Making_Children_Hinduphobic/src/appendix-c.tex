\chapter{California Law}

\textit{\textbf{State Law: Education Code Sections}}

\noindent
\textbf{51501}: “The state board and any governing board shall not adopt any textbooks or other instructional materials for use in the public schools that contain any matter reflecting adversely upon persons on the basis of race or ethnicity, gender, religion, disability, nationality, or sexual orientation”

\textbf{60040}: “When adopting instructional materials for use in the schools, governing boards shall include only instructional materials which, in their determination, accurately portray the cultural and racial diversity of our society, including: (a) The contributions of both men and women in all types of roles, including professional, vocational, and executive roles.”

\textbf{60044 (a), (b)}: “A governing board shall not adopt any instructional materials for use in the schools that, in its determination, contain: (a) Any matter reflecting adversely upon persons on the basis of race or ethnicity, gender, religion, disability, nationality, or sexual orientation, occupation, (b) Any sectarian or denominational doctrine or propaganda contrary to law.”

\noindent
\textit{\textbf{Standards For Evaluation of Instructional Material for Social Contents 2013 Edition (page 5 and 9)}}

\noindent
Purpose of Standards “To project the cultural diversity of society; instill in each child a sense of pride in his or her heritage; develop a feeling of\break self-worth related to equality of opportunity; eradicate the roots of prejudice; and thereby encourage the optimal individual development of each student.

To enable all students to become aware and accepting of religious diversity while being allowed to remain secure in any religious beliefs they may already have.

