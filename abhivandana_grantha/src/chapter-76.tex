\chapter{A Great Teacher Sri Gangadhara Bhat}

\begin{center}
\Authorline{Alexander Medin}
\smallskip
Norway

\end{center}
The autumn of 1998 I had the great privilege of meeting Gangadhara Bhat. I wanted to study Sanskrit and he put me in contact with Alwar, who became my teacher for two years. As I spent time at the Sanskrit College, often waiting for Alwar, I noticed the energy and enthusiasm of one man looking after the needs of the students, principals, other member of the faculty and guests. I soon came to know that Gangadhara Bhat was the unquestionable inner support that held the whole school together. Later when I had the privilege to sit in on his classes I gained an inspiration in the field of Sanskrit I had never previously experienced.

Simply put, Gangadhara Bhat is the greatest teacher that I ever had. His depth of knowledge is one thing, his great sense of humour and enthusiasm another, but most importantly his actions speaks louder than words through the care, warmth and support he gives to other people. If I ever met a guru in India that shows the way through ignorance and darkness it is he, V. Gangadhara Bhat.

Gangadhara Bhat had a way of caring for all of his students and despite his many responsibilities, he managed to do more! In addition to all his classes at the Sanskrit College he would teach at home. Morning and evening, and I was impressed how he managed his time, never complained and was always positive in whatever he was doing.

In 2004 when I brought Ted Proferez, a professor from Harvard teaching in London Gangadhara Bhat was the first person I introduced him to. Ted also met many other great scholars in Mysore, but I know for sure that it was Gangadhara Bhat that inspired him the most. As a matter of fact he claims that six months of study with him was worth 3 years of study at the University. This was also my experience. I did a BA and MA from London University. But the times I had with Gangadhara Bhat was far superior to any degree or title one could ever get. I felt close to the essence of life and drinking from the source of Vidya.

During my 20 years of coming forth and back to India, there is one person I always make a point to visit every time. Not so much for studies anymore, but simply just tremendous gratitude for having the privilege to know him and to have been his student. In all my travels around the world, and in India at large, I have never met such a great teacher that unquestionably has richness and depth of the Shastras, but more importantly he is a caring, loving and supporting human being. The true marks of dharmika.

Now, as this great man is finally about to retire I feel sad for the Sanskrit Collge of Mysore, but lucky for his native place of Siddhapur, which will have such a great person back in their community. I know Gangadhara Bhat loves plants and his magic fingers and touch can make everything grow. Humans and plants the same. I feel tremendously grateful to have been touched by his presence and hehas taught me the greatest lesson of life, which is to give, never to get.

Thank you honourable teacher and friend. Spending time with you and learning from you have been some of the best times of my life. These experiences has given me new perspectives and opened my eyes to the things that really matters; to do good and be good. This highest form of dharma I have found in you.

May your retirement be filled with joy, health, moments of peace and may you live long, because the world is fortunate to have you.
