{\fontsize{14}{16}\selectfont
\chapter{ಪರೋಪಕಾರಿ ಗುರುಗಳು}

\begin{center}
\Authorline{ರಾಘು ಕೆಳದಿ}
\smallskip
ಪೂರ್ವವಿದ್ಯಾರ್ಥೀ\\
(2011.   \enginline{-}   2016.)
\addrule
\end{center}
ನಾನು ನಮ್ಮ ಗುರುಗಳಾದ ಗಂಗಾಧರ ಭಟ್ಟರ ಬಗ್ಗೆ ಬರೆಯುವಷ್ಟು ದೊಡ್ಡ ಶಿಷ್ಯನಲ್ಲ. ಅದಕ್ಕಾಗಿ ನಾನು ಅವರಿಂದ ಪಡೆದ ಜ್ಞಾನಗಳನ್ನು ಹಾಗೂ ಅವರಿಂದ ಪಡೆದ ಗುಣಗಳನ್ನು ವಿವರಿಸುವುದಕ್ಕೆ ಇಚ್ಛಿಸುತ್ತೇನೆ.

ಗುರುಗಳು ಯವಾಗಲೂ ಕೂಡ ನಿಷ್ಕಲ್ಮಷ ಮನಸ್ಸಿನವರು. ಸಂಸ್ಕೃತ ಸಂಸ್ಕೃತಿಗೆ ಹೆಚ್ಚು ಬೆಲೆ ಕೊಡುವ ವ್ಯಕ್ತಿತ್ವ. ಯಾರು ಸಂಸ್ಕೃತವನ್ನು ಓದಲು ಮೈಸೂರಿಗೆ ಬಂದರೂ ಕೂಡ ಮೊದಲು ಹುಡುಕುವುದು, ಕೇಳುವುದು ಗಂಗಾಧರ ಭಟ್ಟರು ಎಲ್ಲಿ ಸಿಗುತ್ತಾರೆ ಎಂದು. ಅನೇಕ ಪ್ರದೇಶಗಳಿಂದ ಸಂಸ್ಕೃತವನ್ನು ಅಥವಾ ಇತರೆ ಶಿಕ್ಷಣವನ್ನು ಅರಸಿ ಬಂದವರಿಗೆ ಮೊದಲ ವಸತಿ ಮತ್ತು ಊಟದ ವ್ಯವಸ್ಥೆ ಗಂಗಾಧರ ಭಟ್ಟರ ಮನೆಯಲ್ಲೇ. ಸಾಮಾನ್ಯವಾಗಿ ಮೈಸೂರಿನಲ್ಲಿ ಸಂಸ್ಕೃತ ಎಂದರೆ ಗಂಗಾಧರ ಭಟ್ಟರು ಎನ್ನುವಷ್ಟು ಅವರು ಪ್ರಸಿದ್ಧಿ ಪಡೆದಿದ್ದಾರೆ. ನಾವು ವಿದ್ಯಾರ್ಥಿಗಳಾದ ಸಮಯದಲ್ಲಿ ಭಾಷಣ ಸ್ಪರ್ಧೆ ಬಹಳಷ್ಟು ಬರುತ್ತಿತ್ತು. ಅದರಲ್ಲಿ ಮೈಸೂರು ವಿದ್ಯಾರ್ಥಿಗಳ ಜೊತೆ ಗಂಗಾಧರ ಭಟ್ಟರು ಬಂದಿದ್ದಾರೆ ಎಂದರೆ ಜಯ ಅವರಿಗೇ ಎಂಬ ಭಾವನೆ ಮತ್ತು ನಿರ್ಭಯ ಎಲ್ಲಾ ವಿದ್ವಾಂಸರಲ್ಲು ಇರುತ್ತಿತ್ತು. ಅಷ್ಟರ ಮಟ್ಟಿಗೆ ಪ್ರಸಿದ್ಧಿ ಪಡೆದಂತಹ ವ್ಯಕ್ತಿತ್ವ ಅವರದ್ದು. ಹಾಗೆಯೇ ಅವರು ಸಭೆಗಳಲ್ಲಿ ಮಾಡುವ ಭಾಷಣಗಳು ನಮ್ಮ ಮೇಲೆ ಧನಾತ್ಮಕ ಪರಿಣಾಮ ಬೀರುತ್ತಿತ್ತು. ಅವರ ವಾಗ್ವೈಖರಿ, ಅವರು ಭಾಷಣದಲ್ಲಿ ಬಳಸುವ ಶಬ್ದಝರಿ ಮತ್ತು ಅವರ ಶಾಸ್ತ್ರ ವ್ಯಾಖ್ಯಾನ ಕೌಶಲ  \enginline{-}   ಇವುಗಳು ನಮಗೆ ಮತ್ತೆಲ್ಲೂ ಕೇಳಿಸಿಕೊಳ್ಳಲು ಸಾಧ್ಯವಿರಲಿಲ್ಲ. ಅವರು ಮಾತನ್ನಾಡುತ್ತಿದ್ದರೆ ನಮಗೆ ಅರಿವಿಲ್ಲದಂತೆಯೇ ನಮ್ಮ ದೃಷ್ಟಿ ಮತ್ತು ಚಿತ್ತ ಅವರಲ್ಲಿಯೇ ಸ್ಥಿರವಾಗುತ್ತಿತ್ತು. ಅಂತಹ ಮಹಾನ್ ಭಾಷಣಕಾರರು ನಮ್ಮ ಗುರುಗಳಾದ ಗಂಗಾಧರ ಭಟ್ಟರು.
	
ಹಾಗೆಯೇ ಅವರೊಂದಿಗೆ ಕಳೆದ ನನ್ನ ಕೆಲವು ಅನುಭವಗಳನ್ನು ಹಂಚಿಕೊಳ್ಳುತ್ತೇನೆ. ನಾನು ಮೊದಲು ಮೈಸೂರಿಗೆ ಬಂದು ಸಾಹಿತ್ಯ ಎರಡನೇ ವರ್ಷಕ್ಕೆ ಪ್ರವೇಶ ಪಡೆದೆ. ಮೊದಲನೇ ವರ್ಷದ ಪರೀಕ್ಷೆಯಲ್ಲಿ ಅನುತ್ತೀರ್ಣನಾದೆ. ಯಾವ ವಿಷಯ ಎಂದು ನೋಡಿದರೆ ತರ್ಕ. ಮುಂದೇನು ಎಂದು ಯೋಚನೆ ಮಾಡಿದಾಗ ಇದ್ದಿದ್ದ ಮರುಪರೀಕ್ಷೆಯಲ್ಲಿ ಓದಿ ಉತ್ತೀರ್ಣನಾದೆ. ಆಗ ಅವರು ಅನುತ್ತೀರ್ಣನಾದದ್ದು ಒಳ್ಳೆಯದೇ ಆಯಿತು, ಇಲ್ಲದಿದ್ದರೆ ಪಾಠದ ಕಡೆ ಗಮನ ಹರಿಸುತ್ತಿರಲಿಲ್ಲ ಎಂದು ತಿಳಿ ಹೇಳಿದರು. ನನ್ನ ಸ್ನೇಹಿತ ಮತ್ತು ಗುರು ಸ್ಥಾನದಲ್ಲಿರುವ ವಿಜಯಣ್ಣನ ಪರಿಚಯದಿಂದ ಗಂಗಾಧರ ಭಟ್ಟರ ಮನೆಗೆ ಹೋಗಲು ಪ್ರಾರಂಭ ಮಾಡಿದೆ. ಅಂದು ಪ್ರಾರಂಭವಾದ ಓಡನಾಟ ಆಟ, ಪಾಠ ಮತ್ತು ಊಟಗಳಿಂದ ಇಂದಿನ ವರೆಗೂ ಮುಂದುವರೆಯುತ್ತಲೇ ಇದೆ. ಇಷ್ಟಾದರೂ ವರೊಂದಿಗೆ ಮಾತನಾಡಲು ಎನೋ ಒಂದು ಗೌರವದ ಭಯ. ಅವರ ಪತ್ನಿಯವರೂ ಸಹೃದಯರು. ನಾವು ಅವರನ್ನು ಯಾವಾಗಲೂ ಅಮ್ಮ ಎಂದೇ ಸಂಬೋಧಿಸುತ್ತೇವೆ. ನಾವು ಅವರ ಮನೆಗೆ ಹೋದಾಗಲೆಲ್ಲ ಸತ್ಕಾರ ಇದ್ದೇ ಇರುತ್ತದೆ. ಉಪಚಾರ ಇಲ್ಲದೇ ತಿರುಗಿ ಬಂದದ್ದೇ ಇಲ್ಲ. ಇವತ್ತಿಗೂ ಏನೇ ವಿಶೇಷ ಅಡುಗೆ ಮಾಡಿದರೂ ನನ್ನನ್ನು ನೆನಪಿಸಿಕೊಂಡು ಸದಾ ಸ್ವಾಗತಿಸುತ್ತಲೇ ಇರುತ್ತಾರೆ.
	
ನಾನು ವ್ಯಾಕರಣ ಶಾಸ್ತ್ರದಲ್ಲಿ ವಿದ್ವತ್ ಮಾಡಿದ ನಂತರ ಶಿಕ್ಷಕ ವೃತ್ತಿಯನ್ನು ಆರಿಸಿಕೊಂಡು ಸಂಸ್ಕೃತ ಶಿಕ್ಷಕನಾದೆ. ನನಗೆ ಮೊದಲಿನಿಂದಲೂ ಅವರ ಮೇಲಿದ್ದ ಗೌರವ ನಾನೂ ಶಿಕ್ಷಕನಾದ ಮೇಲೆ ಇಮ್ಮಡಿಯಾಯಿತು. ಅದರಲ್ಲಿ ಬರುವ ಎಲ್ಲ ರೀತಿಯ ಸಮಸ್ಯೆಗಳನ್ನು ಬಗೆಹರಿಸುತ್ತಾ ಬಂದಿದ್ದಾರೆ ನಮ್ಮ ಗುರುಗಳು. ಶಾಲೆಯಲ್ಲಿ ಸಂಸ್ಕೃತಕ್ಕೆ ಸಂಭಂಧಿಸಿದ ಯಾವುದೇ ಸ್ಪರ್ಧೆ ಅಥವಾ ಕಾರ್ಯವಿರಲಿ ಭಯವಿರಲಿಲ್ಲ. ಏಕೆಂದರೆ ನಮ್ಮ ಗುರುಗಳಿದ್ದಾರೆ ಎನ್ನುವ ಧೈರ್ಯ, ವಿಶ್ವಾಸ. ಕಾಲಭೇಧವಿಲ್ಲದೇ ನಾವು ಅವರ ಬಳಿ ಭಾಷಣವನ್ನು ಬರೆಸಿಕೊಡಿ ಎಂದು ಕೇಳುತ್ತಿದ್ದೆವು. ಅವರು ಕಿಂಚಿತ್ತೂ ವ್ಯಸ್ತಚಿತ್ತರಾಗದೇ ತತ್ಕ್ಷಣದಲ್ಲೇ ಕಾಲಮಿತಿಗನುಗುಣವಾಗಿ ಭಾಷಣವನ್ನು ಬರೆದುಕೊಳ್ಳಲು ಹೇಳುತ್ತಿದ್ದರು. ಅವರ ಸಾಮರ್ಥ್ಯವೆಂತದ್ದೆಂದರೆ ಒಂದು ವಿಷಯವನ್ನು ಒಬ್ಬರಿಗೆ ಅನೇಕ ರೀತಿಯಲ್ಲಿ ಬರೆಸುವ ಕಲೆ, ಹಾಗೆಯೇ ಅನೇಕ ವಿಷಯಗಳನ್ನು ಏಕಕಾಲದಲ್ಲಿ ಅನೇಕರಿಗೆ ಬರೆಸುವಷ್ಟು ವ್ಯವಧಾನ. ಇಂತಹ ಅಸಾಮಾನ್ಯರು ನಮ್ಮೊಂದಿಗೆ ಸಾಮಾನ್ಯರಂತಿರುವುದು ನಮಗೇ ಅಚ್ಚರಿ. ಇಂತಹ ಅಪ್ರತಿಮ ವಿದ್ವಾಂಸರನ್ನು ಪಡೆದ ನಾವೇ ಧನ್ಯರು. ಇತ್ತೀಚಿಗೆ ಅವರ ಆರೋಗ್ಯದಲ್ಲಿ ಸ್ವಲ್ಪ ಏರುಪೇರಾಗಿದ್ದರೂ ಅವರು ಯಾವತ್ತೊ ಅವರ ನೋವನ್ನು ತೋರ್ಪಡದೇ ತನ್ನ ಶಿಷ್ಯರ ಜ್ಞಾನದಾಹವನ್ನು ತೀರಿಸುತ್ತಾ ಬಂದಿರುತ್ತಾರೆ. ಆದ್ದರಿಂದ ಅಂತಹ ಗುರುಗಳಿಗೆ ಶಿಷ್ಯನಾಗಿರುವುದು ನನ್ನ ಸೌಭಾಗ್ಯ.
	
ನನ್ನಂತಹ ಅನೇಕ ವಿಧ್ಯಾರ್ಥಿಗಳನ್ನು ಮಕ್ಕಳಿಗಿಂತಲೂ ಹೆಚ್ಚಾಗಿ ನೋಡಿಕೊಂಡ ದಂಪತಿಗಳೀರ್ವರನ್ನು ಮೈಸೂರಿನಿಂದ ಬೀಳ್ಕೊಡಲು ಶಿಷ್ಯಕೋಟಿಯ ಮನಸ್ಸು ಒಪ್ಪದ ಪರಿಸ್ಥಿತಿಯಲ್ಲಿದೆ. ಆದರೂ ಅನಿವಾರ್ಯವಾಗಿದೆ. ಅವರು ಎಲ್ಲೇ ಇದ್ದರೂ ಅವರ ಜ್ಞಾನದ ದಾಹ ಎಂದಿಗೂ ಕಡಿಮೆ ಆಗದು. ಅವರೇ ಹೇಳಿದ ಒಂದು ಸುಭಾಷಿತದ ಅಭಿಪ್ರಾಯ ಇಂತಿದೆ   \enginline{-}   \textbf{“ಒಬ್ಬ ಶ್ರೇಷ್ಠ ಗುರುವಿದ್ದರೆ ಮಂದಾಧಿಕಾರಿಯೂ ಇವರ ಪಾಠವನ್ನು ಕೇಳಿ ಉತ್ತಮಾಧಿಕಾರಿಯಾಗಿ ಜ್ಞಾನವನ್ನು ಪಡೆಯುತ್ತಾನೆ ಎಂಬುದರಲ್ಲಿ ಯಾವುದೇ ಸಂಶಯವಿಲ್ಲ”.}

ಅದ್ದರಿಂದ ಅವರ ಪಾರದರ್ಶಕತೆ, ಪ್ರಾಮಾಣಿಕತೆ, ದಕ್ಷತೆ ಮತ್ತು ಸಹೃದಯತೆ ಅವರನ್ನು ಸದಾ ಕಾಪಾಡಲಿ ಎಂದು ಆ ಭಗವಂತನಲ್ಲಿ ಪ್ರಾರ್ಥಿಸುತ್ತೇನೆ.

\articleend
}
