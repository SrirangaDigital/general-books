{\fontsize{14}{16}\selectfont
\chapter{ಹಿತೈಷೀ  \enginline{-}  ಮನೀಷೀ}

\begin{center}
\Authorline{ವಿ। ಶಂಕರ. ಲ. ಭಟ್ಟ ಉಂಚಳ್ಳಿ}
\smallskip

ಉಪನ್ಯಾಸಕರು\\ 
ಶ್ರೀರಾ.ರಾ.ಸಂಸ್ಕೃತ ಕಾಲೇಜು\\
ಶ್ರೀ ಸ್ವರ್ಣವಲ್ಲಿ\\
ಶಿರಸಿ, ಉತ್ತರಕನ್ನಡ
\addrule
\end{center}

ಉತ್ತರಕನ್ನಡಜಿಲ್ಲೆಯ ಸಿದ್ದಾಪುರತಾಲೂಕಿನ ಅಪ್ಪಟ ಹವ್ಯಕ ಮಾಣವಕನೊಬ್ಬ ಚಾಮುಂಡಿಯ ತಪ್ಪಲಿನ ಮೈಸೂರಿಗೆ ಹೋಗಿ ವಿದ್ವಜ್ಜನರ ಮನಸೂರೆಗೊಳ್ಳುವ ವ್ಯಕ್ತಿತ್ವವನ್ನು ಸಂಪಾದಿಸಿಕೊಂಡಿದ್ದು ಸಾಹಸವೇ ಸರಿ. ಆ ಮಾಣವಕನೇ ವಿದ್ವನ್ಮಣಿಯಾಗಿ, ಅವನೇ ಅವರಾಗಿ, ಅನವದ್ಯವಿದ್ಯಾಮಣಿಹಾರಭೂಷಿತರಾಗಿ, ಮಾನನೀಯರಾಗಿ ನಮ್ಮೆಲರ ನೆಚ್ಚಿನ ಗಂಗಾಧರ ಭಟ್ಟರಾಗಿ ವಿರಾಜಿಸುತ್ತಿರುವುದು ನಮಗೆಲ್ಲ ಸಂತೋಷವನ್ನು ತಂದಿದೆ.

1982ರಲ್ಲಿ ನಾನು  ಎಸ್.ಎಸ್.ಎಲ್.ಸಿ ಯಲ್ಲಿ ಉತ್ತೀರ್ಣನಾಗಿ ಮೈಸೂರಿನ ಶ್ರೀಮನ್ಮಹಾರಾಜ ಸಂಸ್ಕೃತಕಾಲೇಜಿಗೆ ಸಂಸ್ಕೃತಾಧ್ಯಯನಕ್ಕಾಗಿ ಆಗಮಿಸಿದೆ. ಅದೇ ವರ್ಷ ವಿದ್ವಾನ್ ಪಿ.ಜಿ.ಭಟ್ಟ ಹೊಸ್ತೋಟ ಇವರ ಮುಖಾಂತರ ಶ್ರೀಯುತ ಭಟ್ಟರ ಪರಿಚಯವಾಯಿತು. ಪರಿಚಯದಿಂದ ಅವರ ಕುರಿತು ಪೂಜ್ಯಭಾವ ಒಡಮೂಡಿತು. ಪೂಜ್ಯಪೂಜಾವ್ಯತಿಕ್ರಮವಾಗದಂತೆ ಅಂದಿನಿಂದ ಇಂದಿನವರೆಗೂ ಅವರ ಒಡನಾಟ ಇರಿಸಿಕೊಂಡಿದ್ದೇನೆ. ಅಂದಿನ ಆ ದಿನಗಳಲ್ಲಿ ಶ್ರೀಯುತ ಭಟ್ಟರು ಶಂಕರವಿಲಾಸ ಸಂಸ್ಕೃತ ಪಾಠಶಾಲೆಯಲ್ಲಿ ಪ್ರಧಾನ ಅಧ್ಯಾಪಕರಾಗಿದ್ದರು.

ಸಿಂಹಕ್ಕೆ ಗೋವತ್ಸಗಳ ಜೊತೆ ಸರಸವಾಡುವ ನೀರಸ ಅವಕಾಶ ಸಿಕ್ಕಂತೆ ಭಟ್ಟರಿಗೆ ಆ ಅವಕಾಶ ಒದಗಿತ್ತು. ಉದ್ದಾಮ ನ್ಯಾಯಶಾಸ್ತ್ರ ಪಂಡಿತರಿಗೆ ಪ್ರಥಮಾ, ಕಾವ್ಯ ತರಗತಿಯ ಬಾಲಪಾಠ ಮಾಡುವ ಕರ್ತವ್ಯ, ಅದು ವಿಧಿಯ ಸಂಕಲ್ಪವಾಗಿತ್ತು.  ಆದರೆ ಅವರು ಒದಗಿಬಂದ ಕಾಯಕವನ್ನು ಕುತ್ಸಿತವೆಂದು ಭಾವಿಸದೇ  ಉತ್ಸುಕತೆಯಿಂದಲೇ ಮಕ್ಕಳಿಗೆ ಪಾಠಮಾಡಿದರು. ಆಡುವ ಮಕ್ಕಳಿಗೆ ಸಂಸ್ಕೃತ ಕಲಿಯುವ ಗೀಳು ಹತ್ತಿಸಿದರು. ಸದಾ ಒಂದಿಲ್ಲೊಂದು ಚಟುವಟಿಕೆಗಳನ್ನು ಮಕ್ಕಳಿಂದ ಮಾಡಿಸುತ್ತ ಪಾಠಶಾಲೆಗೆ ನವಚೈತನ್ಯವನ್ನು ತುಂಬಿದರು. “ಕ್ರಿಯಾಸಿದ್ಧಿಃ ಸತ್ವೇ ಭವತಿ ಮಹತಾಂ ನೋಪಕರಣೇ’’ ಎಂಬ ಮಾತಿಗೆ ನಿದರ್ಶನರಾದರು.

ಸಯ್ಯಾಜಿರಾವ್ ರಸ್ತೆಯಲ್ಲಿ ಸಂಸ್ಕೃತಕಾಲೇಜಿನಿಂದ ಚಿಕ್ಕಮಾರ್ಕೆಟಿಗೆ ಹೋಗುವ ರಸ್ತೆಯ ಬಲಪಾರ್ಶ್ವದ   \enginline{-}   ಎಪ್   \enginline{-}   32 ಸಂಖ್ಯೆಯ ಹಳೆಯಕಾಲದ ಮನೆಯಲ್ಲಿ ಅವರ ವಾಸ್ತವ್ಯ. ಅದು ಕೇವಲ ಅವರ ಮನೆ ಅಲ್ಲ, ಅನೇಕರಿಗೆ ಅವರ ಮನೆಯಂತೆ ಆಗಿತ್ತು. ವಿದ್ಯಾಭ್ಯಾಸಕ್ಕಾಗಿ ಹಲವರು ಅವರ ಮನೆಯಲ್ಲಿ ವಾಸವಾಗಿದ್ದರು.

ನ್ಯಾಯಶಾಸ್ತ್ರ ಪಂಡಿತರಾದ ಅವರು ವೇದಾಂತ, ಯೋಗ, ವ್ಯಾಕರಣಾದಿ ಇತರ ಶಾಸ್ತ್ರಗಳ ಸಾರವನ್ನೂ ತಮ್ಮ ಅರಿವಿನ ಪರಿಧಿಯಲ್ಲಿ ಇರಿಸಿಕೊಂಡವರು. ಬಿ.ಕಾಂ ಪದವೀಧರರಾಗಿ ಆಧುನಿಕ ವಿದ್ಯಾಕ್ಷೇತ್ರದಲ್ಲಿಯೂ ಸುದೃಢ ಪ್ರಜ್ಞೆ ಉಳ್ಳವರು. ಸ್ವರ್ಣಪುಷ್ಪಕ್ಕೆ ಸುಗಂಧವು ಸೇರಿದಂತೆ ಸಂಸ್ಕೃತಾಭಿಜ್ಞತೆಯ ಜೊತೆಗೆ ಸೇರಿದ ಆಂಗ್ಲಭಾಷಾಸಂವಹನ ಸಾಮರ್ಥ್ಯವು ಅವರ ವ್ಯಕ್ತಿತ್ವಕ್ಕೆ ಮೆರಗು ನೀಡಿತು. 

ಆದ್ದರಿಂದ ಅನೇಕ ವಿದೇಶಿಗರೂ ಶಾಸ್ತ್ರ ಜಿಜ್ಞಾಸುಗಳಾಗಿ ಅವರ ಬಳಿ ಜ್ಞಾನಭಿಕ್ಷೆಗಾಗಿ ಬರುತ್ತಿದ್ದರು. “ಕೇತಕೀಗಂಧಮಾಘ್ರಾಯ ಸ್ವಯಮಾಯಾಂತಿ ಷಟ್ಪದಾಃ” ಎಂಬ ಕವಿವಾಣಿ ಇಲ್ಲಿ ಸ್ಮರಣೀಯವೆನಿಸುತ್ತದೆ.

ಉತ್ತರಕನ್ನಡಜಿಲ್ಲೆಯ ಸಿರಸಿ, ಸಿದ್ದಾಪುರ, ಯಲ್ಲಾಪುರ ಪ್ರದೇಶಗಳಿಂದ ಅನೇಕ ವಿದ್ಯಾರ್ಥಿಗಳು ಸಂಸ್ಕೃತಾಭ್ಯಾಸಕ್ಕಾಗಿ ಮೈಸೂರಿಗೆ ಬರುವುದು ಮೊದಲಿನಿಂದಲೂ ನಡೆದುಬಂದ ಸಂಗತಿ. 

ನಾನಿದ್ದ ಆ ಕಾಲದಲ್ಲಿ ಪಾಠಶಾಲೆಗೆ ಸೇರುವ ವಿದ್ಯಾರ್ಥಿಗಳಿಗೆ ವಾಸ್ತವ್ಯ ಮತ್ತು ಊಟದ ವ್ಯವಸ್ಥೆ ಮಾಡಿಕೊಳ್ಳುವುದು ಕಷ್ಟವಾಗುತ್ತಿತ್ತು. ಯಾವ ತರಗತಿಗೆ ಸೇರಬೇಕು? ಯಾವ ಶಾಸ್ತ್ರ ಓದಬೇಕು ? ಇವೇ ಮುಂತಾದ ಪ್ರಶ್ನೆಗಳು ಎದುರಾಗುತ್ತಿದ್ದವು. ಆಗ ನೆರವಿಗೆ ಬರುತ್ತಿದ್ದವರು ಗಂಗಾಧರ ಭಟ್ಟರು.

ಸಂಸ್ಕೃತ ಕಾಲೇಜಿನ ಅನೇಕ ವಿದ್ಯಾರ್ಥಿಗಳು ಶಾಸ್ತ್ರಪಾಠಕ್ಕಾಗಿ, ಇನ್ನು ಕೆಲವರು ಭಾಷಣವಿಷಯ ಸಂಗ್ರಹಕ್ಕಾಗಿ ಮತ್ತೆ ಕೆಲವರು ಇಂಗ್ಲಿಷ್ ಭಾಷಾ ಜ್ಞಾನಕ್ಕಾಗಿ ಹೀಗೆ ಹಲವರು ಅವರ ಬಳಿ ಬಂದು ತಮಗೆ ಬೇಕಾದುದನ್ನು ಪದೆದುಕೊಳ್ಳುತ್ತಿದ್ದರು. ಆದ್ದರಿಂದ ಅವರ ಮನೆಯು ಪುಟ್ಟ ವಿದ್ಯಾಕೇಂದ್ರವೇ ಆಗಿತ್ತು. ಶ್ರೀಯುತ ಭಟ್ಟರು ಸರಸ್ವತೀ ಉಪಾಸಕರೇ ಹೊರತೂ ಲಕ್ಷ್ಮೀಸಂಗ್ರಹ  \enginline{-}  ವ್ಯಸನಾಸಕ್ತರಲ್ಲ.

ಸರಸ್ವತೀ ಉಪಾಸನೆಯ ಫಲವಾಗಿ ಮುಂದೆ ಅವರು ಸರಸ್ವತೀಪ್ರಾಸಾದದಲ್ಲಿ ನ್ಯಾಯಶಾಸ್ತ್ರ ಉಪನ್ಯಾಸಕರಾಗಿ ಸೇರುವಂತಾಯಿತು. ತಡವಾಗಿಯಾದರೂ ಈ ಹುದ್ದೆ ಅವರಿಗೆ ದೊರೆತದ್ದು, ನ್ಯಾಯಶಾಸ್ತ್ರವನ್ನು ಓದುವ ವಿದ್ಯಾರ್ಥಿಗಳಿಗೆ ವರದಾನವಾಯಿತು. “ಜೀವನ್ ಭದ್ರಾಣಿ ಪಶ್ಯತಿ” ಅವರು ಅನುಭವಿಸಿದ ಸತ್ಯವಾಯಿತು.
\begin{verse}
ಶ್ಲಿಷ್ಟಾ ಕ್ರಿಯಾ ಕಸ್ಯಚಿದಾತ್ಮಸಂಸ್ಥಾ ಸಂಕ್ರಾಂತಿರನ್ಯಸ್ಯ ವಿಶೇಷಯುಕ್ತಾ~।\\
ಯಸ್ಯೋಭಯಂ ಸಾಧು ಸ ಶಿಕ್ಷಕಾಣಾಂ ಧುರಿ ಪ್ರತಿಷ್ಠಾಪಯಿತವ್ಯ ಏವ~॥
\end{verse}
ಕೆಲವರು ವಿಷಯವನ್ನು ಬಲ್ಲವರಿರುತ್ತಾರೆ. ಆದರೆ ಅವರು ಇತರರಿಗೆ ಮನಮುಟ್ಟುವಂತೆ ಹೇಳುವಲ್ಲಿ ಸೋಲುತ್ತಾರೆ. ಇನ್ನು ಕೆಲವರು ಹೇಳುವ ಕೌಶಲ ಉಳ್ಳವರಾಗಿದ್ದರೂ, ಸ್ವತಃ ವಿಷಯವನ್ನು ಅರ್ಥೈಸಿಕೊಳ್ಳಲು ಅಸಮರ್ಥರಾಗಿರುತ್ತಾರೆ. ಸ್ವಯಂ ವಿಷಯಜ್ಞಾನ   \enginline{-}   ಪರಪ್ರಸ್ತುತಿ ಪಾಟವ ಈ ಎರಡನ್ನೂ ಹೊಂದಿದ ಶಿಕ್ಷಕನೇ ನಿಜವಾದ ಶ್ರೇಷ್ಠ ಶಿಕ್ಷಕನೆನಿಸುತ್ತಾನೆ. ಆದರ್ಶ ಶಿಕ್ಷಕನ ಲಕ್ಷಣವನ್ನು ಸಾರುವ ಕವಿಕುಲಗುರುವಿನ ಈ ಸದುಕ್ತಿಗೆ ಸೂಕ್ತ ಉದಾಹರಣೆಯಾಗಿ ನಿಲ್ಲುತ್ತಾರೆ ಶ್ರೀ ಗಂಗಾಧರ ಭಟ್ಟರು.

ಬೋಧನ, ಲೇಖನ, ಪ್ರವಚನ, ಸಾಮಾಜಿಕ ಹಿತಚಿಂತನ ಈ ಎಲ್ಲ ವಿಷಯಗಳೂ ಏಕತ್ರ ಸಂಗಮಿಸಿದ ಭಟ್ಟರಂತಹ ಸಂಸ್ಕೃತ ವಿದ್ವಾಂಸರು ವಿರಳರೆಂದೇ ಹೇಳಬೇಕು. 

ವೈಯಕ್ತಿಕವಾಗಿ ಕಷ್ಟಗಳೆಂಬ ವಿಷವನ್ನು ಪಾನಮಾಡಿ, ಜ್ಞಾನವೆಂಬ ಗಂಗೆಯನ್ನು ತಲೆಯಲ್ಲಿ ಧರಿಸಿಕೊಂಡ ಗಂಗಾಧರ ಭಟ್ಟರು ಅಂಕಿತನಾಮವನ್ನೇ ಅನ್ವರ್ಥನಾಮವನ್ನಾಗಿಸಿಕೊಂಡಿದ್ದಾರೆ.

ನಾನು ವಿದ್ಯಾರ್ಥಿಯಾಗಿದ್ದಾಗ ಮತ್ತು ಪ್ರದೋಷಸಂಘದ ಕಾರ್ಯದರ್ಶಿಯಾಗಿ ಕಾರ್ಯನಿರ್ವಹಿಸುತ್ತಿದ್ದಾಗ ಅವರಿಂದ ದೊರೆತ ಮಾರ್ಗದರ್ಶನ, ಉಪಕಾರಗಳನ್ನು ಮರೆಯಲಾರೆ.ಉತ್ತರಕನ್ನಡ ಜಿಲ್ಲೆಯ ಹೆಮ್ಮೆಯ ಪುತ್ರರಾದ ನೇರ ನಡೆ ನುಡಿಯ ವಿದ್ವಾನ್ ಗಂಗಾಧರ ಭಟ್ಟರನ್ನು ಕಂಡಾಗ ಈ ಮುಂದಿನ ಸುಭಾಷಿತವು ನೆನಪಿನಂಗಳದಲ್ಲಿ ಸಂಚರಿಸುತ್ತದೆ.
\begin{verse}
ಮನೀಷಿಣಃ ಸಂತಿ ನ ತೇ ಹಿತೈಷಿಣಃ ಹಿತೈಷಿಣಃ ಸಂತಿ ನ ತೇ ಮನೀಷಿಣಃ।\\
ಸುಹೃಚ್ಚ ವಿದ್ವಾನಪಿ ದುರ್ಲಭೋ ನೃಣಾಂ ಯಥೌಷಧಂ ಸ್ವಾದು ಹಿತಂ ಚ ದುರ್ಲಭಮ್~॥
\end{verse}
ಉದರಕ್ಕೆ ಹಿತವೂ, ಅಧರಕ್ಕೆ ಸಿಹಿಯೂ ಆದ ಔಷಧವು ಹೇಗೆ ದುರ್ಲಭವೋ ಹಾಗೆಯೇ ಹತೈಷಿಗಳೂ, ಮನೀಷಿಗಳೂ ಆದ ವಿದ್ವಾಂಸರು ಲೋಕದಲ್ಲಿ ಸಿಗುವುದು ಅತ್ಯಂತ ವಿರಳವೇ ಸರಿ. 

ಹಿತೈಷೀ ಮನೀಷಿಗಳಾದ ಶ್ರೀ ಗಂಗಾಧರಭಟ್ಟರಿಗೆ ಸಮರ್ಪಿತವೀ ನುಡಿನಮನ.

\articleend
}
