{\fontsize{14}{16}\selectfont
\chapter{ನಂದತಿ ನಂದತಿ ನಂದತ್ಯೇವ}                           

\begin{center}
\Authorline{ವಿ~।। ವೆಂಕಟಾಚಾರ್ಯ}
\smallskip

ಭೂತಪೂರ್ವ ಪ್ರಾಂಶುಪಾಲರು\\
ಶ್ರೀಮನ್ಮಹಾರಾಜಸಂಸ್ಕೃತ  ಮಹಾಪಾಠಶಾಲಾ\\
ಮೈಸೂರು
\addrule
\end{center}
ಲೋಕದಲ್ಲಿ ಪರಸ್ಪರ ಸ್ನೇಹ ಬಾಂಧವ್ಯಗಳು  ಕಾಲಕರ್ಮ ಸಂಯೋಗದಿಂದ ಉಂಟಾಗುತ್ತವೆ. ಅವುಗಳಲ್ಲಿ ಕೆಲವು ಅಪರೂಪಕ್ಕೆಂಬಂತೆ ಸುದೀರ್ಘಾನುವಿತ್ತಿಯಾಗಿ ಬಾಳುತ್ತವೆ. ಇಂತಹುದೇ ಒಂದು ಅಪರೂಪದ ಆತ್ಮೀಯತೆ ನನಗೆ ವಿ ~। ಶ್ರೀಗಂಗಾಧರ ಭಟ್ಟರಲ್ಲಿ. 

ಶ್ರೀಯುತ ಗಂಗಾಧರ ಭಟ್ಟರು ಶ್ರೀಮನ್ಮಹಾರಾಜ ಸಂಸ್ಕೃತ ಮಹಾಪಾಠ\-ಶಾಲೆಗೆ ವಿದ್ಯಾರ್ಥಿಯಾಗಿ ದಾಖಲಾದಾಗಲೇ ನನಗೆ ಮೊಟ್ಟಮೊದಲ ಅವರ ಪರಿಚಯ. ಈ ಪರಿಚಯವು ಮುಂದಿನ ನನ್ನ ಭವಿಷ್ಯದಲ್ಲಿನ ಅನೇಕ ಉತ್ಕರ್ಷಕ್ಕೆ ನಾಂದಿಯಾಯಿತು.  ಅವರು ವಿದ್ಯಾರ್ಥಿಯಾಗಿ ಶಾಲೆಗೆ ಸೇರ್ಪಡೆಯಾದ ಕಾಲ ನನ್ನ ಅಧ್ಯಾಪನದ\break ತಾರುಣ್ಯವೂ ಆಗಿತ್ತು. ಅವರು ನ್ಯಾಯಶಾಸ್ತ್ರದ ವಿದ್ಯಾರ್ಥಿಯಾಗಿದ್ದರು. ವಿದ್ವತ್ \hbox{ತರಗತಿ}ಗಳಿಗೆ ನಡೆಯುತ್ತಿದ್ದ ಸಾಮಾನ್ಯ ವಿಷಯವಾಗಿದ್ದ ನನ್ನ ಕೌಮುದೀ ಪಾಠಕ್ಕೂ ವಿದ್ಯಾರ್ಥಿಯಾಗಿದ್ದರು.

ಪಾಠದ ಮಧ್ಯೆ ಅವರು ನ್ಯಾಯಶಾಸ್ತ್ರದ ಹಿನ್ನೆಲೆಯಲ್ಲಿಯಿಂದ ಕೋರುತ್ತಿದ್ದ ಅವರ ಪ್ರಶ್ನೆಗಳಿಗೆ ಉತ್ತರಿಸಲು ನಾನು ಸಹ ನ್ಯಾಯಶಾಸ್ತ್ರದ ಹಿನ್ನೆಲೆಯಲ್ಲಿಯೇ ವ್ಯಾಕರಣವನ್ನು ಬೋಧಿಸಲು ವ್ಯಾಕರಣದ ಜೊತೆಗೆ ನ್ಯಾಯಶಾಸ್ತ್ರದ ಅಧ್ಯಯನವನ್ನೂ ಹುರಿ\-ಗೊಳಿಸಿಕೊಳ್ಳಬೇಕಾಯಿತು. 

ಇದು ಅವರ ವಿದ್ಯಾರ್ಥಿಯ ದೆಸೆಯಿಂದ ನನಗಾದ ಪ್ರಯೋಜನ. ಜೊತೆಗೆ ಅವರ ಬುದ್ಧಿ ಪ್ರಖರತೆಗೂ ಸಾಕ್ಷ್ಯ. ಈ ಮೊದಲೇ ಒಂದು ಹಂತದ ವರೆಗೆ ನಡೆದಿದ್ದ ನನ್ನ ನ್ಯಾಯಶಾಸ್ತ್ರದ ಅಧ್ಯಯನವನ್ನು ಹಿರಿಯ ವಿದ್ವನ್ಮಣಿಗಳಾದ ದಿ ~। ಮಹಾ\-ಮಹೋಪಾಧ್ಯಾಯ ಶ್ರೀ ರಾಮಭದ್ರಾಚಾರ್ಯರು, ಶ್ರೀಯುತ ವಿದ್ವಾನ್ ವೆಂಕಣ್ಣಾಚಾರ್ಯರು ಮತ್ತು ವಿದ್ವಾನ್ ಶ್ರೀನಾಥಾಚಾರ್ಯರು ಹಾಗೂ ಇಂದಿಗೂ ನನಗೆ ವೇದಾಂತವನ್ನು ಬೋಧಿಸುತ್ತಾ ಆಧ್ಯಾತ್ಮಿಕ ಕ್ಷೇತ್ರದಲ್ಲಿ ಮಾರ್ಗದರ್ಶನ ಮಾಡುತ್ತಲಿರುವ ರಾಷ್ಟ್ರಪ್ರಶಸ್ತಿ ವಿಜೇತರಾದ ಮಹಾಮಹೋಪಾಧ್ಯಾಯ ಶ್ರೀ॥ಉ॥ ವೇ ವಿದ್ವಾನ್ ಕೆ.ಎಸ್. ವರದಾಚಾರ್ಯರಲ್ಲೂ ಮುಂದುವರಿಸುವ ಕಾಲಕ್ಕೆ ನಾವಿಬ್ಬರೂ ಸಹಪಾಠಿಗಳು.

ಹೀಗೆ ಸತೀರ್ಥರಾದದ್ದು ಪರಸ್ಪರ ಆತ್ಮೀಯತೆ ಮತ್ತಷ್ಟು ನಿಕಟವಾಗಲು ಕಾರಣ\-ವಾಯಿತು. ಈ ಆತ್ಮೀಯತೆಯೇ ಮುಂದೆ ಶ್ರೀಯುತ ಭಟ್ಟರ ಒತ್ತಾಯಕ್ಕೆ ಕಟ್ಟುಬಿದ್ದು ಸಂಸ್ಕೃತ ಎಂ.ಎ. ಪದವಿಯಲ್ಲೂ ಉತ್ತೀರ್ಣನಾಗಲು ಕಾರಣವಾಯಿತು. (ಎಂ.ಎ. ಪರೀಕ್ಷಾ ಸಂದರ್ಭದಲ್ಲಿ ಅವರ ಅಣತಿಯಂತೆ ಅವರ ವಿದ್ಯಾರ್ಥಿಗಳು ಆಟೋರಿಕ್ಷಾವನ್ನು ತಂದು ಅದರಲ್ಲಿ ನನ್ನನ್ನು ಬಲವಂತವಾಗಿ ಕೂರಿಸಿ ಪರೀಕ್ಷಾ ಕೊಠಡಿಗೆ ದಬ್ಬಿ ಬರುತ್ತಿದ್ದುದನ್ನು ಮರೆಯಲುಂಟೆ ) ಈ ವೇಳೆಗೆ ಅವರೂ ನಾನು ಸಹೋದ್ಯೋಗಿಗಳಾದೆವು. 

ಸಹೋದ್ಯೋಗಿಯಾದ ನಾನು ಕಾಲೇಜಿನ ಪ್ರಾಂಶುಪಾಲ ಹುದ್ದೆಯನ್ನು ನಿರ್ವಹಿಸುವ ಸಂದರ್ಭದಲ್ಲಿ ಎದುರಾದ ಎಡರು ತೊಡರುಗಳು ಶ್ರೀಯುತ ಭಟ್ಟರ ಸಮಯೋಚಿತ ಸಲಹೆ, ಕಾರ್ಯಪರತೆ ಮತ್ತು ಸೂಕ್ತ ಸಂಯೋಜನೆ ಹಾಗೂ ಪರಮ \hbox{ಗೋಪ್ಯತೆ}ಗಳಿಂದ ಪಾರಾದ ಸಂದರ್ಭಗಳು ಅದೆಷ್ಟೋ, ಇದೇ ಸಂದರ್ಭದಲ್ಲಿ ಶ್ರೀಯುತ\break ಭಟ್ಟರಂತೆಯೇ ಆತ್ಮೀಯತೆಯ ಅಪ್ಪುವಿಕೆಯಿಂದ ನನಗೆ ಸಹಕರಿಸಿದ ಅಂದಿನ ಶಾಲಾ ಸಿಬ್ಬಂದಿ ವೃಂದಕ್ಕೂ ಇಂದಿಗೂ ಸಹ ನಾನು ಆಭಾರಿ. 

ಆದರೆ ಪಾಠಶಾಲೆಯ ಶೈಕ್ಷಣಿಕ ಚಟುವಟಿಕೆಗಳಲ್ಲಿ, ವಿವಿಧ ಕಾರ್ಯಕ್ರಮಗಳ ಸಂಯೋಜನೆಯಲ್ಲಿ, ಅವುಗಳ ಯಶಸ್ಸಿನಲ್ಲಿ ಮತ್ತು ಒಟ್ಟಾರೆ ಶಾಲೆಯ ಸರ್ವತೋಮುಖ ಬೆಳವಣಿಗೆಯ ಹಿನ್ನೆಲೆ ಮುನ್ನೆಲೆಗಳಲ್ಲಿ ಶ್ರೀಯುತ ಗಂಗಾಧರ ಭಟ್ಟರ ಪಾತ್ರ ಬಹುವಾಗಿದೆ. ಎಲ್ಲವನ್ನೂ ಇಲ್ಲಿ ಹೇಳಲಾಗದು. ಅವರ ಬಗ್ಗೆ ನನ್ನ ಹೃದಯವು ಭಾವಿಸುವ ವಿಷಯಗಳು ಬಹಳ. ಒಂದೆರಡನ್ನು ವ್ಯಕ್ತಪಡಿಸಿದ್ದೇನಷ್ಟೆ.

ವಿದ್ಯಾರ್ಥಿಯಾಗಿ, ಸತೀರ್ಥರಾಗಿ, ಸ್ನೇಹಿತರಾಗಿ, ಸಹೋದ್ಯೋಗಿಯಾಗಿ, ಹಿತೈಷಿಯಾಗಿ, ಕೊನೆಗೆ ಆತ್ಮೀಯರಾಗಿರುವವರು ಶ್ರೀಯುತ ಗಂಗಾಧರ ಭಟ್ಟರು. 
\eject

ಅಂತೆಯೇ ಅಂದಿನ ಮತ್ತು ಇಂದಿನ ವಿದ್ಯಾರ್ಥಿ ವೃಂದ ಹಾಗು ಸಹೋದ್ಯೋಗಿ ವೃಂದ ಮತ್ತು ಆಡಳಿತ ಕಾರ್ಯ ನಿರ್ವಾಹಕ ವೃಂದಗಳು ಇಂದಿಗೂ ನೀಡುತ್ತಲಿರುವ ವಾತ್ಸಲ್ಯಭರಿತ ಪ್ರೀತಿಯ ಅಪ್ಪುಗೆಯ ಮತ್ತು ಇವೆಲ್ಲಕ್ಕೂ ಕಾರಣವಾದ ಶ್ರೀವಿದ್ಯಾಗಣಪತಿ ಸಹಿತ ಪಾಠಶಾಲೆಯ ಸುಮಧುರ ಸ್ಮೃತಿಯ ಚಿಲುಮೆಯ ಹೊಮ್ಮುವಿಕೆಯಿಂದ ಹಚ್ಚ ಹಸಿರಾಗಿ ಸುಮಧುರ ಬಾಂಧವ್ಯ ತುಂಬಿದ ನನ್ನ ಹೃದಯವು ಸದಾ ಸರ್ವಕಾಲದಲ್ಲೂ  ನಂದತಿ ನಂದತಿ ನಂದತ್ಯೇವ.

\articleend
}
