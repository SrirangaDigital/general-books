{\fontsize{14}{16}\selectfont
\chapter{ನಮ್ಮ ನಲ್ಮೆಯ ಶ್ರೀಗಂಗಾಧರ ಭಟ್ಟರು}

\begin{center}
\Authorline{ಸ್ವಾಮೀ ಕರುಣಾಕರಾನಂದ}
\smallskip

ರಾಮಕೃಷ್ಣ ಮಠ ಹಾಗೂ ರಾಮಕೃಷ್ಣ ಮಿಷನ್,\\
ಬೇಲೂರು ಮಠ \\
ಹೌರಾ \\ 
ಪಶ್ಚಿಮ ಬಂಗಾಳ  \enginline{-}  711202
\addrule
\end{center}

2013 ರಲ್ಲಿ `ದಿನಕರಿ' ಪಾಠವನ್ನು ಮಾಡಿಸಿಕೊಳ್ಳಲು ಆಚಾರ್ಯರನ್ನು ಹುಡುಕುತ್ತಿದ್ದ ಸಮಯ. ಶ್ರೀ ಗಂಗಾಧರ ಭಟ್ಟರನ್ನು ಅದಾಗಲೇ ನಾನು ನೋಡಿದ್ದೆನಾದರೂ ಅವರ ಬಳಿ ಪಾಠ ಹೇಳಿಸಿಕೊಳ್ಳುವುದಕ್ಕಾಗಿ ಅರಿಕೆ ಮಾಡಿಕೊಳ್ಳಲು ಹಿಂಜರಿಯುತ್ತಿದ್ದೆ. ಕಾರಣ ಇಷ್ಟೆ, ಅವರು ಪಾಠ ಹೇಳಿಕೊಡಲು ಒಪ್ಪುವರೋ ಇಲ್ಲವೋ ಎಂಬ ಶಂಕೆ ಕಾಡುತ್ತಿತ್ತು. ಆದರೂ ಅವರಿವರು ಹೇಳಿದ್ದರು, `ಕೇಳಿ ನೋಡಿ. ಅವರು ಬಹಳ ಉದಾರಿಗಳು. ನಿಮಗೆ ಪಾಠ ಮಾಡಲು ಖಂಡಿತ ಒಪ್ಪುವರು' ಎಂದು. ನಾನೂ ಕೂಡ ಪ್ರತಿ ದಿನ ಪಾಠಶಾಲೆಗೆ ಬರುತ್ತಿದ್ದ ಸಮಯದಲ್ಲಿ ಅವರ ಧ್ವನಿಯನ್ನು ಕೇಳುತ್ತಲೇ ಇದ್ದೆ. ಗಂಗಾಧರ ಭಟ್ಟರು ಎಲ್ಲಿ ಪಾಠ ಮಾಡುತ್ತ ಇದ್ದಾರೆಂದು ಯಾರನ್ನೂ ಕೇಳಬೇಕಾಗಿರಲಿಲ್ಲ. ಅವರ ಗಟ್ಟಿಯಾದ ಸ್ಪಷ್ಟವಾದ ಉಚ್ಚಾರಣೆಯಿಂದ ಕೂಡಿದ ಧ್ವನಿ ಜಿ.ಪಿ.ಎಸ್ ನಂತೆ ತನ್ನ ಮೂಲವನ್ನು ದೂರದಿಂದಲೇ ತೋರಿಸಿಕೊಡುತ್ತಿತ್ತು. ಧ್ವನಿ ಗಟ್ಟಿಯಾದರೂ, ಯಾರೇ ಆದರೂ ಆ ಧ್ವನಿಯನ್ನು ಕೇಳಿದಾಗ ಅದರಲ್ಲಿ ವಿದ್ಯಾದಾನದ ಕಳಕಳಿ, ವಿದ್ಯಾರ್ಥಿಯ ಮೇಲಿದ್ದ ಪ್ರೀತಿ ಮಿಳಿತವಾಗಿರುವುದು ಅವರ  ಗಮನಕ್ಕೆ ಬರದೆ ಹೋಗುತ್ತಿರಲಿಲ್ಲ. ನಾನು ಹೋಗಿ ಅವರ ಬಳಿ ನನ್ನ ಕೋರಿಕೆ ಮುಂದಿಟ್ಟಾಗ ಮಹನೀಯರು ಅತ್ಯಂತ ಸಂತೋಷದಿಂದ ಒಪ್ಪಿಕೊಂಡರು. ಆಗ ನನಗಾದ ಸಂತೋಷಕ್ಕೆ ಪಾರವೇ ಇಲ್ಲ. 

ಇನ್ನು ಪಾಠ ಶುರುವಾದ ಮೇಲಂತೂ ನನಗೆ ಪ್ರತಿ ತರಗತಿಯೂ `ಅಮೃತೋತ್ಸವ'. ಪ್ರತಿಯೊಂದು ಅಂಶವನ್ನು ಬಿಡಿಸಿ, ಕಪ್ಪು ಹಲಗೆಯ ಮೇಲೆ ಚಿತ್ರಗಳ ಸಹಾಯದಿಂದ ವಿದ್ಯಾರ್ಥಿಗಳ ಮನಸ್ಸಿಗೆ ಮುಟ್ಟುವಂತೆ ಪಾಠ ಮಾಡುವುದರಲ್ಲಿ ಇರುವ ಅವರ ಉತ್ಸಾಹ ತಾಳ್ಮೆಗಳೆರಡೂ ನಿಜಕ್ಕೂ ಅನುಕರಣೀಯ.  `ವಿಷಯವನ್ನು ಅರ್ಥೈಸುವುದು ಬಹಳ ಮುಖ್ಯ, ಸಂಸ್ಕೃತದಲ್ಲೋ, ಮಾತೃಭಾಷೆಯಲ್ಲೋ. ಮೊದಲು ಮಾತೃಭಾಷೆಯಲ್ಲಿ ಪಾಠ ಮಾಡಿ ನಂತರ ಅದನ್ನೇ ಸಂಸೃತದಲ್ಲಿ ಮಾಡಿದರೆ ವಿಷಯ ಇನ್ನೂ ಚೆನ್ನಾಗಿ ಮನಸ್ಸಿನಲ್ಲಿ ಕುಳಿತುಕೊಳ್ಳುತ್ತದೆ' ಎನ್ನುವುದು ಶಾಸ್ತ್ರಾಧ್ಯಾಪನದಲ್ಲಿ ಭಟ್ಟರ ವ್ಯಾವಹಾರಿಕ ಸಲಹೆ. ಸಂಸ್ಕೃತದ ಜ್ಞಾನವಿರದೆ, ಸಂಸ್ಕೃತ ಕಲಿಯಲು ಸಮಯವಿರದೆ  ಕೇವಲ ಶಾಸ್ತ್ರವನ್ನು ಅಧ್ಯಯನ ಮಾಡ ಬಯಸುವ ಕೆಲವು ಪ್ರಾಮಾಣಿಕ ಜಿಜ್ಞಾಸುಗಳು ಬಂದಾಗ ಅವರಿಗೆ ವಿದ್ಯಾದಾನ ಮಾಡದೆ ಹಾಗೇ ಮರಳಿ ಕಳಿಸಿಬಿಟ್ಟರೆ ಶಾಸ್ತ್ರಪ್ರವರ್ತನ ಹೇಗೆ ಸಾಧ್ಯವಾದೀತು. ಹೀಗೇ ಶಾಸ್ತ್ರಗಳ ಅಧ್ಯಯನವಿಲ್ಲದೆ ಅವುಗಳು ನಿಧಾನವಾಗಿ ನಷ್ಟಪ್ರಾಯವಾಗುತ್ತಿವೆ. ಇದೇ ವಿಚಾರಸರಣಿ ಭಟ್ಟರ ಸಲಹೆಯ ಹಿಂದಿತ್ತು. ನನ್ನ ಅಧ್ಯಾಪನದಲ್ಲಿಯೂ ಈ ಸಲಹೆಯನ್ನು ಅನುಷ್ಠಾನಕ್ಕೆ ತಂದು ಅನೇಕ ಪ್ರಯೋಜನವನ್ನು ಕಂಡುಕೊಂಡಿದ್ದೇನೆ. 

ಭಟ್ಟರು ತಮ್ಮ ನಿವೃತ್ತ ಜೀವನದಲ್ಲಿಯೂ ವಿದ್ಯೆಯನ್ನು ಅರಸಿ ಬರುವ ಅನೇಕರಿಗೆ ತಮ್ಮ ಅಮೂಲ್ಯ ವಿದ್ಯೆಯನ್ನು ಧಾರೆಯೆರೆಯುವಂತಾಗಲಿ. ಅವರಿಗೆ ಭಗವಂತನು ಆಯುರಾರೋಗ್ಯವನ್ನು ದಯಪಾಲಿಸಿ ಅವರ ಮನೋವಾಂಛಿತವನ್ನು ಪೂರ್ಣಗೊಳಿಸಲಿ ಎಂಬುದೇ ನಮ್ಮ ಹಾರೈಕೆ.

\articleend
}
