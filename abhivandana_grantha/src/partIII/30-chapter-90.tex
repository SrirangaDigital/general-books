{\fontsize{14}{16}\selectfont
\chapter[ಅಪ್ರತಿಮ ಜ್ಞಾಪಕಶಕ್ತಿಯ ವಿ~॥  ಗಂಗಾಧರ ಭಟ್ಟರು]{ಅಪ್ರತಿಮ ಜ್ಞಾಪಕಶಕ್ತಿಯ \\ವಿ~॥ ಗಂಗಾಧರ ಭಟ್ಟರು}
\begin{center}
\Authorline{ಶ್ರೀಧರ ಆರ್. ಭಟ್ಟ}

ಪತ್ರಕರ್ತರು\\
ನಂಜನಗೂಡು
\addrule
\end{center}

ನಾನು ಜೀವನದ ದಾರಿ ಅರಸುತ್ತ ನಂಜನಗೂಡಿಗೆ ಬಂದು ನೆಲೆಸಿದ ಪ್ರಾರಂಭದ ದಿನಗಳು, ಬಹುಶಃ 1984ರ ಆರಂಭ ಕಾಲವಿರಬಹುದು, ಅದಾಗಲೇ ನಾನು ಆಂದೋಲನ ಪತ್ರಿಕೆಯ ಬಳಗದಲ್ಲಿ ಸೇರಿಕೊಂಡಿದ್ದ  ಒಂದು ಸಮಾರಂಭದ ನಿಮಿತ್ತ ಮೈಸೂರಿನ  ಗೋವಿಂದ\-ರಾವ್ ಮೆಮೋರಿಯಲ್ ಹಾಲ್‍ಗೆ   ಹೋಗಿದ್ದೆ. ಅಲ್ಲಿ  ಗಂಗಾಧರ ಭಟ್ಟರ ಹಾಗೂ ನನ್ನ  ಪ್ರಥಮ ಭೇಟಿಯಾದ ನೆನಪು. ಆ ಮೊದಲೇ ನಮ್ಮ ತಾಲೂಕಿನವರೇ ಆಗಿದ್ದ ಶ್ರೀಯುತ ಗಂಗಾಧರ ಭಟ್ಟರ ಕುರಿತಾಗಿ ಕೇಳಿದ್ದೆ. ಉಪ್ಪಿಟ್ಟಿನ ಸುಬ್ಬಣ್ಣನೆಂದು ಅವರಿಂದ ಕರೆಸಿಕೊಳ್ಳುತ್ತಿದ್ದ ನನ್ನ ಸೊದರ ಭಾವ ಕಸಿಗೆ ಸುಬ್ಬಣ್ಣ  ಅವರ ಪಾಂಡಿತ್ಯ, ಅಗಾಧವಾದ ಜ್ಞಾನ ಭಂಡಾರದ ಬಗೆಗೆ ನನಗೆ ಆಗಲೇ ತಿಳಿಸಿದ್ದ. ಸದಾ  ಹಸನ್ಮುಖಿಯಾಗಿರುತ್ತಿದ್ದ  ಅವರನ್ನು  ಸ್ನೇಹಿತರು ಮತ್ತು ಶಿಷ್ಯವೃಂದದವರು ಯಾವಾಗಲೂ ಮುತ್ತಿಕೊಂಡೇ ಇರುತ್ತಾರೆ. ಅಂದೂ ಹಾಗೇ,  ಇಂದೂ ಹಾಗೇ. ಅಂದು ನಾನೇ ಮುಂದಾಗಿ ಅವರ  ಪರಿಚಯ ಮಾಡಿಕೊಂಡೆ. ಸಾಧಾರಣ ಮೈಕಟ್ಟಿನ  ಗಂಗಾಧರ ಭಟ್ಟರ ವಾಮನಾವತಾರ ನನಗೆ ಪರಿಚಯ ಆದದ್ದು  ಹಾಗೆ.

ನಮ್ಮ  ಸಿರ್ಸಿ, ಸಿದ್ದಾಪುರ  ಭಾಗದ ಜನರ ಪಾಲಿಗೆ ಮೈಸೂರಿನ ವಿದ್ಯಾಭ್ಯಾಸ  ಎಂದರೆ ಅವರೆಲ್ಲರಿಗೆ ನೆನಾಪಾಗುತ್ತಿದ್ದುದು  ಈ ಗಂಗಾಧರ ಭಟ್ಟರು. ಬರಿಗೈನಲ್ಲಿ ಮೈಸೂರಿಗೆ ಬಂದವರ ಪಾಲಿನ ಅಪದ್ಬಾಂಧವರಾಗಿ  ಅವರಿಗೆ ಶಿಕ್ಷಣ ಆರೋಗ್ಯ ಹಾಗೂ ಉದ್ಯೋಗಕ್ಕಾಗಿ   ತನ್ನ ಕೈ ಮೀರಿ  ಉದಾರತೆಯ ಹಸ್ತ ಚಾಚುತ್ತಿದ್ದ  ಅವರು ಎಲ್ಲರಿಗೂ ದಾರಿ ದೀಪವೇ  ಆಗಿದ್ದಾರೆ. ಅವರೋ ಸಂಸ್ಕೃತ ವಿದ್ವಾಂಸರು. ಆದರೆ ನಾನು  ಅವರದೇ ತಾಲೂಕಿನವನು ಎಂಬುದನ್ನು ಬಿಟ್ಟರೆ  ಅವರೊಡನೆಯ  ಸ್ನೇಹ ಹಸ್ತಕ್ಕೆ  ಬೇಕಾದ ಯೋಗ್ಯತೆಗಳೇನೂ ನನ್ನಲ್ಲಿ ಇರಲಿಲ್ಲ. ಆದರೂ ಅವರ ಹಾಗೂ ನನ್ನ ಭಾಂದವ್ಯ ಆ ಕಾಲದಿಂದ ಪ್ರಾರಂಭವಾಗಿ ಗಟ್ಟಿಯಾಗುತ್ತ ಕೊನೆಗೆ ವಿದ್ವಾಂಸರಾದ ಅವರು  ನನಗೆ ಗಂಗಣ್ಣನಾದರು. ಆ ಬಾಂಧವ್ಯವು ಅಂದಿನಿಂದ ಇಂದಿನವರೆಗೂ ಮುಂದುವರಿದೇ ಇದೆ. ಆಗ ಅವರ ಮನೆ  ರಮಾವಿಲಾಸ ರಸ್ತೆಯ ಕೆಳಭಾಗದಲ್ಲಿ ಅಯ್ಯಂಗಾರ ಮೆಸ್ ನ ಹಿಂಭಾಗದಲ್ಲಿತ್ತು. ಅಂದೇ ಅವರ ಮನೆಗೆ ನನ್ನ ಪ್ರವೇಶ ಆಯಿತು. ಆಗ ಅವರ ಜೊತೆಯಲ್ಲಿ ಸಹೋದರಿ ಹೇಮಾವತಿ ಹಾಗೂ ರತ್ನಾವತಿಯವರಿದ್ದರು. ಮನೆ ಅಗಲೇ  ದಾಸೋಹದ ಮಠದಂತಿತ್ತು. ಸದಾ ಅವರಲ್ಲಿಗೆ ಜ್ಞಾನ ಪಿಪಾಸುಗಳಾಗಿ ಬರುವ ವಿದ್ಯಾರ್ಥಿಗಳು ಹಾಗೂ ಪರದೇಶೀಯರಾಗಿ ಮೈಸೂರಿಗೆ ಆಗಮಿಸುವ ಶಿಕ್ಷಣಾರ್ಥಿಗಳಿಂದ  ಸದಾ ಭರ್ತಿಯಾಗಿರುತ್ತಿತ್ತು. ಪಾಪ ಆ ಸಹೋದರಿಯರು ಶಿಕ್ಷಣಕ್ಕಾಗಿ ಮನೆಯಿಂದ ಹೊರಹೋಗುವಾಗ   ಮಾಡಿಡುತ್ತಿದ್ದ ಅಡುಗೆ ಅವರು ಹಸಿದು ಬರುವ ವೇಳಗೆ ಇವರ ದಾಸೋಹದ ಪರಿಯಿಂದಾಗಿ  ಖಾಲಿಯಾಗಿ ಅವರ ಪಾಲಿಗೆ ತೊಳೆಯಬೇಕಾಗಿದ್ದ (ನಾನೂ  ಸೇರಿದಂತೆ ಮನೆಗೆ ಬಂದವರೆಲ್ಲ ಊಟಮಾಡಿಟ್ಟು ಹೋದ) ತಟ್ಟೆ ಲೋಟ ಮಾತ್ರ ಕಂಗೊಳಿಸುತ್ತಿತ್ತು. ಗಂಗಾಧರ ಭಟ್ಟರ ನಿತ್ಯ ದಾಸೋಹದ ಪರಿ ಇದಾಗಿತ್ತು. 

ಆಕೃತಿಯಲ್ಲಿ ವಾಮನನಂತಿದ್ದ ಅವರು ಪಾಠ ಪ್ರವಚನಕ್ಕೆ ನಿಂತರೆಂದರೆ ಆಗಾಧವಾದ ಪಾಂಡಿತ್ಯದ ಪ್ರಭಾವ ಅವರ ಬಾಯಿಂದ ಸರಾಗವಾಗಿ ಮಾರ್ದನಿಸುತ್ತಿತ್ತು. ಅವರ ಈ ಪ್ರಾವೀಣ್ಯದಿಂದಾಗಿ  ಅವರಲ್ಲಿಗೆ ವಿದೇಶಿಗರೂ ಸಹ  ಅಗಮಿಸಿ ಪಾಠ ಹೇಳಿಸಿಕೊಳ್ಳುತ್ತಾರೆ, ಇಂದಿಗೂ ಅದು ಮುಂದುವರೆದಿದೆ.

ಆಡು ಮುಟ್ಟದ ಸೊಪ್ಪಿಲ್ಲ ಗಂಗಾಧರ ಭಟ್ಟರ ಅರಿವಿಗೆ ಬಾರದ ವಿಷಯಗಳಿಲ್ಲ, ಎಂಬ ಪಾಂಡಿತ್ಯ ಅವರದ್ದು.  ಅವರಲ್ಲಿಗೆ ಸಂಸ್ಕೃತದ ವಿದ್ಯಾರ್ಥಿಗಳು ಮಾತ್ರವಲ್ಲ, ಆಯುರ್ವೇದ, ವಿಜ್ಞಾನ, ಸಾಹಿತ್ಯ , ಕಲೆ ಸೇರಿದಂತೆ  ಎಲ್ಲ ಪ್ರಕಾರದವರೂ ಸಹ ಜ್ಞಾನಾರ್ಜನೆಗಾಗಿ ಆಗಮಿಸುತ್ತಿರುವದನ್ನು ನಾನು ಸ್ವತಃ ಕಣ್ಣಾರೆ  ಕಂಡಿದ್ಧೇನೆ. ಅವರ ಚರ್ಚೆಯಲ್ಲಿ ಕೆಲವೊಮ್ಮೆ ನಾನು  ಮೂಕ ಪ್ರೇಕ್ಷಕನಾಗಿರುತ್ತಿದ್ದ ದಿನಗಳೂ ಇರುತ್ತಿದ್ದವು.

ಮೈಸೂರು ನಗರದ ವಿದ್ಯಾರ್ಥಿಗಳ ಪಾಲಿಗೆ ಎಲ್ಲಿಯೇ ಭಾಷಣ ಸ್ಪರ್ಧೆಗಳಾಗಲಿ  ಪ್ರಬಂಧ ಸ್ಪರ್ಧೆಗಳಾಗಲಿ  ನಡೆಯುತ್ತವೆ ಎಂದಾದಲ್ಲಿ   ಅವರೆಲ್ಲ ಧಾಂಗುಡಿ ಇಡುವದು ಜ್ಞಾನ ಭಂಡಾರವಾದ ಈ ಭಟ್ಟರ ಮನೆಗೆ. ಗಂಗಾಧರ ಭಟ್ಟರು ಹೇಳಿದ  ಬರೆಸಿದ ವಿಷಯಗಳ ಭಟ್ಟಿ ಇಳಿಸಿಕೊಂಡು ಹೋಗಿ   ಅಲ್ಲಿ ಆ ವಿಷಯಗಳನ್ನು  ಮಂಡಿಸಿ   ಚರ್ಚಾ ಸ್ಪರ್ಧೆಯ ಪರ ವಿರೋಧದ ಎರಡೂ ಪುರಸ್ಕಾರಗಳು ಇವರ  ಶಿಷ್ಯರಿಗೇ ಲಭ್ಯವಾದ ಉದಾಹರಣೆಗಳೂ ಸಹ ಸಾಕಷ್ಟಿವೆ. ಬಂದವರಿಗೆ, "ನ ಹಿ - ಇಲ್ಲ" ಎಂದು ಹೇಳಿದವರಲ್ಲ  ನಮ್ಮ ಗಂಗಣ್ಣ.  ಸಂಸ್ಕೃತ, ವೇದ, ಉಪನಿಷತ್ ಎಲ್ಲದರ ಬಗ್ಗೆ ಅಪಾರ ಪಾಂಡಿತ್ಯಗಳಿಸಿದ್ದ ನಮ್ಮ ಗಂಗಣ್ಣನಲ್ಲಿ ಈ ವಿಷಯಗಳ ಕುರಿತು ಯಾವುದೇ  ಪ್ರಶ್ನೆ ಎದುರಾದರೂ ಕ್ಷಣದಲ್ಲಿ ಅದರ ವಿವಿರ ಕಂಪ್ಯೂಟರನಂತೆ ಕರಾರುವಾಕ್ಕಾಗಿ ಲಭ್ಯವಿರುತ್ತಿತ್ತು. ಹಾಗಾಗಿಯೇ ಅವರ ಸ್ನೇಹಿತರ ಬಳಗವೂ ಸದಾ ವೃದ್ಧಿಯಾಗುತ್ತಲೇ ಇದೆ. 
ಗಂಗಣ್ಣ ಎಂದೂ ಹೆಸರು ಹಾಗೂ ಧನದ ಹಿಂದೆ ಬಿದ್ದವರಲ್ಲ, ಹಾಗೇನಾದರೂ ಆಗಿದ್ದರೆ ಅವರು ಧಾರೆ ಎರೆವ  ವಿಷಯಗಳಿಂದಲೇ  ಲಕ್ಷಾಂತರ ಸಂಪಾದಿಸಬಹುದಿತ್ತು. ಪ್ರತಿ ನಿತ್ಯವೂ ಕರಾರುವಾಕ್ಕಾಗಿ ಮನೆಗೆ ಬಂದವರಿಗೆ ಪಾಠ ಮಾಡುತ್ತಿದ್ದ ಅವರು ಯಾರಿಂದಲೂ ಬಿಡಿಗಾಸು ಸಹ ಪಡೆಯದೇ ಅಲ್ಲಿಂದ ಇಲ್ಲಿಯವರಿಗೂ ಉಚಿತವಾಗಿಯೇ ಜ್ಞಾನ ದಾಸೋಹದ ಔತಣ  ಬಡಿಸಿ ತೃಪ್ತಿ ಕಾಣುತ್ತಿದ್ದಾರೆ.
ಇಂಥ ಜ್ಞಾನ ಭಂಡಾರದ ಗಂಗಣ್ಣ  ನಮ್ಮೊಡನೆ ಯಕ್ಷಗಾನದ ತಾಳ ಮದ್ದಲೆಯ ಅರ್ಥವನ್ನೂ ಹೇಳಿದ್ದರು. ಆ ಸಮಯದಲ್ಲಿ ಅವರ ಚಾಕಚಕ್ಯತೆ ಭಾಷಾ ಪೌಢಿಮೆ, ಸಮಯ ಸ್ಫೂರ್ತಿಯ ಜ್ಞಾನಧಾರೆಯ ಶುದ್ಧ ತರ್ಕದಿಂದ ಕೂಡಿದ  ವಾದವಿವಾದಗಳ ಆಸ್ವಾದಕನಾದವರಲ್ಲಿ ನಾನೂ ಒಬ್ಬನಾಗಿದ್ಧೇನೆ. ಮನಸ್ಸು ಮಾಡಿದ್ದರೆ ಅವರು ಯಕ್ಷಗಾನ ತಾಳ-ಮದ್ದಲೆಯಲ್ಲೂ ಹೆಸರು ಗಳಿಸಬಹುದಿತ್ತು ಅಂಥ ಪಾಡಿತ್ಯ ಅವರದು.  

ಈ ರೀತಿಯ ಜ್ಞಾನ ಭಂಡಾರದ ಗಣಿ ಗಂಗಣ್ಣ ಈಗ ನಿವೃತ್ತಿಯ ಅಂಚು ತಲುಪುತ್ತಿದ್ದಾರೆ ಎಂದು ನಂಬಲೇ ಅಸಾಧ್ಯ. ಜೊತೆಗೆ ನನ್ನ ವಯಸ್ಸು  ಸಹ ಜ್ಞಾಪಕವಾಗಲಾರಂಭಿಸಿದೆ. ಸದಾ ಚಟುವಟಿಕೆಯಿಂದ ಇರುವ ನಮ್ಮ ಪ್ರೀತಿಯ ಗಂಗಣ್ಣ  ಆರೋಗ್ಯವಾಗಿ  ನೂರು ವಂಸಂತಗಳನ್ನು  ಕಾಣುತ್ತ  ತನ್ನ ಜ್ಞಾನ ಭಂಡಾರದ ಪರಿಮಳವನ್ನು ಪಸರಿಸುತ್ತಿರಲಿ ಎಂಬುದೇ ನಮ್ಮಲ್ಲರ ಅಪೇಕ್ಷೆಯಾಗಿದೆ.


\articleend
}
