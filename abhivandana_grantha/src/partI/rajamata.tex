ರಾಜಮಾತೆ ಡಾ || ಪ್ರಮೋದಾದೇವಿ ಒಡೆಯರ್ ರವರ ಶುಭಸಂದೇಶ
	
ನಮ್ಮ ಮಹಾರಾಜರ ಪರಂಪರೆಯಲ್ಲೊಬ್ಬರಾದ ಮುಮ್ಮಡಿ ಕೃಷ್ಣರಾಜ ಒಡೆಯರು 1876 ರಲ್ಲಿ ಮೈಸೂರಿನಲ್ಲಿ ಶ್ರೀಮನ್ಮಹಾರಾಜ ಸಂಸ್ಕೃತ ಮಹಾಪಾಠಶಾಲೆಯನ್ನು ಸ್ಥಾಪಿಸಿದರು. ಸ್ಥಾಪನೆಯಾದ ಅಲ್ಪಾವಧಿಯಲ್ಲಿಯೇ ವಿಶ್ವದಲ್ಲೆಲ್ಲ ಪ್ರಸಿದ್ಧವಾದ ವಿದ್ಯಾಕೇಂದ್ರವಾಗಿ ಈ ಪಾಠಶಾಲೆ ಬೆಳೆಯಿತು, ಬೆಳಗಿತು. ಇದರಿಂದ ಪಾಠಶಾಲೆಯ ಸ್ಥಾಪನೆ ಸಾರ್ಥಕವಾಯಿತು. ಆ ಸಾರ್ಥಕ್ಯಭಾವ ನಮ್ಮ ಅನುಭವದ ವಿಷಯವಾಗಿದೆಯೆಂಬುದನ್ನು ಈ ಸಂದರ್ಭದಲ್ಲಿ ವ್ಯಕ್ತಪಡಿಸಲು ನಮಗೆ ಅತ್ಯಂತ ಹರ್ಷವಾಗುತ್ತಿದೆ.

ವಿಶ್ವಸ್ತರದಲ್ಲಿ ಗೌರವಪಡೆದ ವಿದ್ವಾಂಸರು ಅಧ್ಯಾಪಕರಾಗಿದ್ದು ವಿದ್ಯಾದೇವಿಯ ಆರಾಧನೆ ಮಾಡಿ ಈ ಪಾಠಶಾಲೆಯನ್ನು ವಿದ್ಯಾಕ್ಷೇತ್ರವನ್ನಾಗಿಸಿದ್ದಾರೆ. ಇದರಿಂದ ವಿದ್ಯಾತೀರ್ಥದ ಅವಗಾಹನಕ್ಕಾಗಿ ವಿವಿದೆಡೆಯಿಂದ ಆಗಮಿಸುವ ಅಸಂಖ್ಯಾತ ವಿದ್ಯಾರ್ಥಿಗಳು ವಿದ್ಯಾಸ್ನಾತರಾಗಿ ಜೀವನವನ್ನು ಸಾರ್ಥಕಪಡಿಸಿಕೊಂಡಿದ್ದಾರೆ. ಹೀಗೆ ಪಾಠಶಾಲೆ ಅನತಿ ಕಾಲದಿಂದ ಸಂತತವಾಗಿ ನಡೆದು ಬಂದಿದೆ.

ಇಂತಹ ವಿದ್ವತ್ ಪರಂಪರೆಯಲ್ಲಿ ಪ್ರಕೃತ ಬೆಳಕಿಗೆ ಬಂದಿರುವವರು ವಿದ್ವಾಂಸರಾದ ಶ್ರೀಮಾನ್ ಗಂಗಾಧರ ಭಟ್ಟರು.  ಅನತಿದೂರದ ಸಿದ್ದಾಪುರದಿಂದ ಮೈಸೂರಿಗೆ ಬಂದು ಇದೇ ಪಾಠಶಾಲೆಯಲ್ಲಿ ಅಧ್ಯಯನ ಮಾಡಿ ಇಲ್ಲಿಯೇ ಅಧ್ಯಾಪಕರೂ ಆಗಿ, ನಿವೃತ್ತರಾಗುತ್ತಿದ್ದಾರೆ. ದೇಶ-ವಿದೇಶದ ಸಾವಿರಾರು ವಿದ್ಯಾರ್ಥಿಗಳನ್ನು ಸ್ವಂತ ಮಕ್ಕಳಂತೆ ಭಾವಿಸಿ ಅವರಿಗೆ ವಿದ್ವತ್ತನ್ನು ಧಾರೆಯೆರೆದು ವಿದ್ಯಾರ್ಥಿಗಳ ಕಣ್ಮಣಿಯಾಗಿದ್ದಾರೆ. ತಮ್ಮ ವಿಶಿಷ್ಟವಾದ ವಿದ್ಯೆ, ಪ್ರತಿಭೆ, ಸಾಮರ್ಥ್ಯದಿಂದ ಸಮಾಜಕ್ಕೆ ಅಪಾರ ಸೇವೆಯನ್ನು ಸಲ್ಲಿಸಿ ವ್ಯಾಪಕವಾದ ಅಭಿಮಾನಿ ಸಮಾಜವನ್ನು ಹೊಂದಿದ್ದಾರೆ. ಶ್ರೀಯುತರ ಬದುಕು ಸಮಾಜಕ್ಕೆ ಆದರ್ಶಪ್ರಾಯವಾಗಿದೆ. 

ಇಂತಹ ಮಹಾನ್ ವಿದ್ವಾಂಸರಿಗೆ ಅವರ ವಿದ್ಯಾರ್ಥಿವೃಂದ ಅಭಿವಂದನ ಕಾರ್ಯಕ್ರಮವನ್ನು ಹಮ್ಮಿಕೊಂಡಿದ್ದು ಇವರ ಬಗ್ಗೆ ಒಂದು ಸಾಕ್ಷ್ಯಚಿತ್ರವನ್ನು ನಿರ್ಮಾಣ ಮಾಡಿರುವುದಲ್ಲದೇ, ಅಭಿವಂದನ ಗ್ರಂಥವೊಂದನ್ನು ಪ್ರಕಾಶಪಡಿಸುತ್ತಿದ್ದಾರೆ. ವಿದ್ಯೆಯನ್ನು ಪಡೆದ ವಿದ್ಯಾರ್ಥಿಗಳು ತಮ್ಮ ವಿದ್ಯಾಗುರುಗಳನ್ನು ಈ ಎಲ್ಲ ರೀತಿಯಲ್ಲಿ ಅಭಿವಂದಿಸುತ್ತಿರುವುದು  ಸ್ತುತ್ಯರ್ಹವಾದ ವಿಷಯವಾಗಿದೆ. 

ಈ ಕಾರ್ಯಕ್ರಮಕ್ಕೆ ನಮ್ಮನ್ನು ಆಹ್ವಾನಿಸಿ ತಮ್ಮ ವಿದ್ಯಾಮಯಜೀವನಕ್ಕೆ ಪಾಠಶಾಲೆಯಿಂದ ಆದ ಸೌಕರ್ಯಕ್ಕಾಗಿ ನಮಗೆ ಅನಂತ ಕೃತಜ್ಞತೆಯನ್ನು ವ್ಯಕ್ತಪಡಿಸುತ್ತಿದ್ದಾರೆ. ಕಾರ್ಯಕ್ರಮದಲ್ಲಿ ಪಾಲ್ಗೊಳ್ಳುತ್ತಿರುವುದು ನಮಗೆ ಸಾರ್ಥಕ್ಯವನ್ನು ಆನಂದವನ್ನು ಉಂಟುಮಾಡುತ್ತಿದೆ.

ಇಂತಹ ಸಂದರ್ಭದಲ್ಲಿ ಶ್ರೀಮಾನ್ ಗಂಗಾಧರ ಭಟ್ಟರನ್ನು ಹೃತ್ಪೂರ್ವಕವಾಗಿ ನಾವು ಅಭಿನಂದಿಸುತ್ತೇವೆ. ಅವರ ನಿವೃತ್ತಿಜೀವನ ಸ್ವಸ್ಥವಾಗಿರಲಿ. ಅವರ ಪಾಠಪ್ರವಚನಗಳು ಮುಂದುವರೆಯಲಿ. ಅವರನ್ನು ಅಭಿನಂದಿಸುವ ವಿದ್ಯಾರ್ಥಿಗಳು ಶ್ರೀಯುತರ ಪರಂಪರೆಯನ್ನು ಬೆಳೆಸಿ ಲೋಕದಲ್ಲಿ ವಿದ್ಯೆಯನ್ನು ಬೆಳಗಿಸಲಿ ಎಂದು ಭಗಂತನಲ್ಲಿ ಪ್ರಾರ್ಥಿಸುತ್ತೇವೆ.

ಹೇಮಲಂಬನಾಮಸಂವತ್ಸರ
ವೈಶಾಖ-ಶುದ್ಧ-ಏಕಾದಶೀ-ಭಾನುವಾರ
ದಿನಾಂಕ : 11.02.2018							ಡಾ || ಪ್ರಮೋದಾದೇವಿ ಒಡೆಯರ್
ಸ್ಥಳ : ಮೈಸೂರು
