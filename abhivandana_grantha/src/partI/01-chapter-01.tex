\chapter{ಅಭಿವಂದನಪತ್ರ}

\begin{verse}
ವೀರೋ ಗಂಗಾಧರೋ ಧೀಮಾನ್ ನ್ಯಾಯವಿದ್ಯಾವಿಶಾರದಃ |\\
ಪ್ರಬೋಧಕಶ್ಚಿರಂಜೀಯಾತ್ ದೀವ್ಯಾತ್ತು ಶತಶಾರದಃ ||
\end{verse}

ಪೂಜ್ಯ ಗಂಗಾಧರ ಭಟ್ಟರೇ !

ಗಂಧದಗುಡಿ ಮಹೀಸುರಪುರೀ ಎಂದು ಪ್ರಸಿಧವಾಗಿದ್ದ ಈಗಿನ ಮೈಸೂರು ತಮಗೆ ಎರಡನೆಯ ತವರು. ತಮ್ಮ ಜೀವನಸುಮ ಅರಳಿ ಪರಿಮಳಿಸುತ್ತಿರುವುದು ಮೈಸೂರಿನಲ್ಲಿ. ರೇವತೀವಿಘ್ನೇಶ್ವತೀರ್ಥರಿಂದ ಜನ್ಮಪಡೆದು ಹುಟ್ಟೂರು ಅಗ್ಗೇರೆಯಲ್ಲಿ ಮೊದಲ ವಿದ್ಯಾಭ್ಯಾಸ ಪೂರೈಸಿ, ವಿಶೇಷ ಅಧ್ಯಯನಕ್ಕಾಗಿ 1975ರಲ್ಲಿ ಮೈಸೂರಿಗೆ ಬಂದಿರಿ. ಕ್ರಮವಾಗಿ ಪಿಯುಸಿ, ಬಿಕಾಂ, ಎಂ.ಎ. ವಿದ್ವತ್ ಪರೀಕ್ಷೆಗಳಲ್ಲಿ ಉತ್ತಮಶ್ರೇಣಿಯಲ್ಲಿ ಉತ್ತೀರ್ಣರಾದಿರಿ. ಅದೇ ಸಮಯದಲ್ಲಿ ಶ್ರೀಮನ್ಮಹಾರಾಜಸಂಸ್ಕೃತಮಹಾಪಾಠಶಾಲೆಯಿಂದ ನವೀನನ್ಯಾಯವಿದ್ವದುತ್ತಮಾಪರೀಕ್ಷೆಯಲ್ಲಿಯೂ ವೈಶಿಷ್ಟ್ಯಶ್ರೇಣಿಯನ್ನು ತಮ್ಮದಾಗಿಸಿಕೊಂಡಿರಿ. ಮಹಾಮಹೋಪಾಧ್ಯಾಯ ಶ್ರೀ ಎನ್.ಎಸ್. ರಾಮಭದ್ರಾಚಾರ್ಯರ ಪಾಠ, ಉಪದೇಶ, ಮಾರ್ಗದರ್ಶನಗಳ ಭಾಗ್ಯ ತಮಗೊದಗಿತು. ಅದರಿಂದ ತಾವು ತಮ್ಮ ಮನೋಮುಕುಲವನ್ನು ಅರಳಿಸಿಕೊಂಡಿರಿ. ಜೀವನಸುಮವನ್ನು ಬೆಳೆಸಿ ಬೆಳಗಿಸಿಕೊಂಡಿರಿ. ವಾಕ್ಕನ್ನು ಪರಿಷ್ಕರಿಸಿಕೊಂಡಿರಿ. ತಮ್ಮ ಪಾಠ-ಪ್ರವಚನಗಳಿಗೆ ಪ್ರೇರಣೆಯನ್ನು ಪಡೆದಿರಿ. ಇದಲ್ಲದೇ, ಪಂಡಿತರತ್ನಂ ಶ್ರೀ ಕೆ.ಎಸ್.ವರದಾಚಾರ್ಯರು, ಶ್ರೀ ವಿಶ್ವೇಶ್ವರ ದೀಕ್ಷಿತರು, ಶ್ರೀ ಈ.ಎಸ್.ವೆಂಕಣ್ಣಾಚಾರ್ಯರೇ ಮೊದಲಾದ ವಿದ್ವಾಂಸರ ಪಾಠದಿಂದ ತಮ್ಮ ಮೇಧೆಯನ್ನು ಇನ್ನೂ ವೃದ್ಧಿಸಿಕೊಂಡಿರಿ. ಇದರಿಂದ ತಾವು ಬಹುಶ್ರುತ ವಿದ್ವಾಂಸರಾದಿರಿ. ಬಹು ವಿಶ್ರುತರೂ ಆದಿರಿ.

ಗಾಯತ್ರಿಯ ಸ್ಫೂರ್ತಿ ತಮ್ಮೊಳಗಿದೆ. ಅದರ ವಿಸ್ತಾರರೂಪದ ವೇದಾಧ್ಯಯನ ತಮಗೆ ಮೊದಲಾಯಿತು. ವೇದದ ವಿಸ್ತಾರವಾದ ಶಾಸ್ತ್ರಗಳನ್ನು ತಮ್ಮದಾಗಿಸಿಕೊಂಡಿರಿ. ಶಾಸ್ತ್ರಸಂಕುಲದಲ್ಲಿ ನ್ಯಾಯಶಾಸ್ತ್ರ ತಮಗೊಲಿಯಿತು. ಆಯುರ್ವೇದ, ಅರ್ಥಶಾಸ್ತ್ರಗಳಲ್ಲಿ ತಾವು ರಮಿಸಿದಿರಿ. ಆಧುನಿಕ, ಲೌಕಿಕ ವಿದ್ಯೆಗಳೂ ಸಹ ತಮಗೆ ತಿರೋಹಿತವಾಗದೇ ಪುರೋಹಿತವಾದವು. ಬುದ್ಧಿಪ್ರಚೋದಕವಾದ ವಾಗ್ವಿಲಾಸ ತಮ್ಮ ಶಕ್ತಿ. ಅದರಿಂದ ತಾವು ಸೂಜಿಗಲ್ಲಿನಂತೆ ನಮ್ಮನ್ನು ಸೆಳೆದಿರಿ. ಸೆಳೆದು ತಾವು ಆಸ್ವಾದಿಸಿದ ಶಾಸ್ತ್ರಮಧುವನ್ನು,  `ಏಕಃ ಸ್ವಾದು ನ ಭುಂಜೀತ’ ಎಂಬಂತೆ ನಮಗೆಲ್ಲ ವಿತರಿಸಿದಿರಿ. ವಿತರಿಸಿ ನಮ್ಮೊಡನೆಯೇ ತಾವೂ ಆನಂದಿಸಿದಿರಿ. ತಾವು ಅಧ್ಯಯನಮಾಡುವಾಗಲೇ ಅಧ್ಯಾಪನ ಕಾರ್ಯಕ್ಕೂ ತೊಡಗಿದವರು. ಶ್ರೀಶಂಕರವಿಲಾಸ ಸಂಸ್ಕೃತ ಪಾಠಶಾಲೆ ತಮ್ಮ ಕನಸಿನ ಕೂಸು. ಅದನ್ನು ವಿದ್ಯಾಲಯವನ್ನಾಗಿ ಬೆಳೆಸಿದಿರಿ. ಮುಂದೆ ತಾವೇ ಓದಿದ ಇದೇ ಸಂಸ್ಕೃತ ಮಹಾಪಾಠಶಾಲೆ ತಮ್ಮನ್ನು ಕೈಬೀಸಿ ಕರೆಯಿತು. ತಾವು ಪ್ರಾಧ್ಯಾಪಕರಾದಿರಿ, ನೂರಾರು ಮಕ್ಕಳಿಗೆ ನಿರಂತರವಾಗಿ ಪಾಠಮಾಡಿ ಶಾಸ್ತ್ರದ ಬೆಳಕನ್ನು ಅವರ ಅಂತರಂಗದಲ್ಲಿ ಬೆಳಗಿಸಿದಿರಿ. 

ಧರ್ಮವನ್ನು ಆಚರಣೆಯಿಂದ ರಕ್ಷಿಸುವ ಮನಸ್ಸು ತಮ್ಮದು. ಉಪನಯನ ಮೂಲದ ಆಚಾರದಿಂದ ವಿವಾಹಮೂಲದ ವ್ಯವಹಾರಕ್ಕೆ ಅಡಿ ಇಟ್ಟಿರಿ. ಶಿವನಾದ ಗಂಗಾಧರನನ್ನು ಶಿವೆಯಾದ ಶೈಲಜೆಯು ವರಿಸಿದಂತೆ ಗಂಗಾಧರರಾದ ತಮನ್ನು ಶೈಲಜಾ ಎಂಬ ಕನ್ಯೆಯು ವರಿಸಿದುದು ವಿಧಿವಿಲಾಸವೇ ಸರಿ. ಇಂತು ಗೃಹಸ್ಥರಾದ ತಾವು ಶ್ರೌತಸ್ಮಾರ್ಥಕರ್ಮಾನುಷ್ಠಾನ ನಿಷ್ಠರಾದಿರಿ. ಗೃಹಸ್ಥಾಶ್ರಮದ ಪರಮಧರ್ಮವಾದ ಆತಿಥ್ಯ ನಿಷ್ಠೆಯಲ್ಲಿ ತಮಗೆ ಸರಿದೊರೆ ಯಾರಿಲ್ಲ. ಎಂಥ ಕಷ್ಟ ಕಾಲದಲ್ಲಿಯೂ ಮನೆಗೆ ಬಂದ ನಮಗೆ ಉಪಚರಿಸದೇ ಕಳುಹಿಸಿದವರಲ್ಲ. ಮಕ್ಕಳಿಗೆ ಮಾತಾಪಿತೃಗಳು ತೋರುವ ಸಕಲವಿಧ ವಾತ್ಸಲ್ಯವನ್ನೂ ನಮಗೆ ತೋರಿದಿರಿ. ಸ ಪಿತಾ ಪಿತರಸ್ತಾಸಾಂ ಕೇವಲಂ ಜನ್ಮಹೇತವಃ ಎಂಬ ಅಂಶವನ್ನು ನಾವು ತಮ್ಮಲ್ಲಿ ಕಾಣುವಂತೆ ನಡೆದುಕೊಂಡಿರಿ. ನಮ್ಮಲ್ಲೆಲ್ಲರಲ್ಲೂ ಸ್ವಂತ ಸಂತಾನದ ಭಾವವನ್ನೇ ಭಾವಿಸಿ ಆನಂದವನ್ನೇ ಅನುಭವಿಸಿದಿರಿ. 

ರಸ, ನಿಷ್ಠೆ ತಮ್ಮ ಬುದ್ಧಿಗೆ ಒಲಿದಿದೆ. ತಾವು ಭಾವುಕರು. ಭಾವಕರೂ ಹೌದು. ಬುದ್ಧಿಯಿಂದ ನಿರ್ಣಯಿಸಿ ಭಾವದಿಂದ ವರ್ತಿಸುವ ತಮ್ಮ ಅಕೃತ್ರಿಮ ಶೈಲಿ ನಮಗೆ ಆದರ್ಶ. ವಾಚ್ಯಕ್ಕಿಂತ ಧ್ವನಿ ನಿಮ್ಮ ಸಹಜ ದನಿ.  ಸ್ವಾಭಿಮಾನ ತಮಗೆ ಸ್ವಭಾವ. ಸ್ವಾವಲಂಬನ ತಮ್ಮ ಸ್ವರೂಪ. ಹಾಗಾಗಿ ಬಂದ ಕಷ್ಟಕ್ಕೆಲ್ಲ ತಾವೊಬ್ಬರೇ ತಲೆಕೊಟ್ಟಿರಿ, ಸುಖವನ್ನು ಕುಟುಂಬಕ್ಕೂ ಸಮಾಜಕ್ಕೂ ವಿತರಿಸಿದಿರಿ.

ಭಟ್ಟರೇ ! ತಮ್ಮ ನಡೆನುಡಿಗಳು ನಮಗೆ ಆಪ್ಯಾಯನ. ಪ್ರೀತಿ ನೀತಿಗಳು ಚಿತ್ತಾಕರ್ಷಕ. ನ್ಯಾಯ-ನಿಷ್ಠುರತೆ ಒಂದು ಕೌತುಕ.  ನಿತ್ಯ ನ್ಯಾಯಕ್ಕಾಗಿ ಹೋರಾಡುವ, ನ್ಯಾಯಾದಿ ಶಾಸ್ತ್ರಗಳನ್ನೂ ಪಾಠಮಾಡುವ ಪ್ರಾಮಾಣಿಕವಾಗಿ ಬದುಕು ಸಾಗಿಸುತ್ತಿರುವ ತಮ್ಮನ್ನು ಅಭಿವಂದಿಸುವ ಸದವಕಾಶದಿಂದ ನಮ್ಮ ಹೃದಯ ತುಂಬಿ ಧನ್ಯವಾಗಿದೆ. ಪ್ರಕೃತ ವಿದ್ಯಾಗಣಪತಿಯ ದಿವ್ಯಾನುಗ್ರಹವಿರುವ ಈ ಸರಸ್ವತೀಪ್ರಾಸಾದದಲ್ಲಿ ಮಹೀಸುರಪುರಿಯ ಮಹಾರಾಜವಂಶಕ್ಕೆ ಸೇರಿದ ಮಹಾರಾಜ್ಞೀ ರಾಜಮಾತಾ ಪ್ರಮೋದಾ ದೇವಿಯವರ ಸಾನ್ನಿಧ್ಯದಲ್ಲಿ, ಅನೇಕ ವಿದ್ಯೋಪಾಸಕ ವಿದ್ವಾಂಸರ ಸನ್ನಿಧಿಯಲ್ಲಿ ತಮ್ಮನ್ನು ನ್ಯಾಯಗಂಗಾಧರ ಎಂದು ಸಂಬೊಧಿಸಿ ಅಭಿವಂದಿಸಲು ಹರ್ಷಿಸುತ್ತೇವೆ. ಭಗವಂತ ತಮಗೂ ತಮ್ಮ ಕುಟುಂಬಕ್ಕೂ ಆಯುರಾರೋಗ್ಯಭಾಗ್ಯವನ್ನು ದಯಪಾಲಿ ಇತೋಪ್ಯತಿಶವನ್ನುಂಟುಮಾಡಲಿ ಎಂದು ತಮ್ಮ ಪ್ರೀತಿಯ ನ್ಯಾಯಶಾಸ್ತ್ರವಿದ್ಯಾರ್ಥಿಸಮೂಹದ ಅಭಿವಂದನಸಮಿತಿ ಆಶಿಸುತ್ತದೆ.

\begin{verse}
ಶ್ರೀಕಂಠದತ್ತಹಸ್ತಾವಲಂಬಪ್ರಮೋದಯಾರ್ಯಯಾ~|\\
ನಿಧೀಯತಾಂ ಪದೇ ಪ್ರಾಂಶೌ ನ್ಯಾಯಗಂಗಾಧರಸ್ಸದಾ~||
\end{verse}

\centerline{|| ಭದ್ರಂ ಶುಭಂ ಮಂಗಲಮ್ ||}
\bigskip

ಮಾಘ-ಕೃಷ್ಣ- ಏಕಾದಶೀ, ಭಾನುವಾರ\hfill				           ನ್ಯಾಯಶಾಸ್ತ್ರವಿದ್ಯಾರ್ಥಿವೃಂದ

ದಿನಾಂಕ - 11.02.2018\hfill  						     ಶ್ರೀಮನ್ಮಹಾರಾಜಸಂಸ್ಕೃತಮಹಾಪಾಠಶಾಲಾ

ಸ್ಥಳ - ಮಹೀಸುರಪುರೀ - ಮೈಸೂರು\hfill				        \textbf{ಗಂಗಾಧರಭಟ್ಟರ ಅಭಿವಂದನಸಮಿತಿ}
