{\fontsize{14}{16}\selectfont

\chapter[ಮುಂದುವರೆಯುವ ಮುನ್ನ .......ಸಂಪಾದಕ\enginline{-}ಹೃದಯ]{\qquad ಮುಂದುವರೆಯುವ ಮುನ್ನ .......\\ \qquad  ಸಂಪಾದಕ\enginline{-}ಹೃದಯ}

\begin{wrapfigure}{l}{0.25\textwidth}
\centerline{\includegraphics[scale=0.5]{figures/guruprasad.png}}
\end{wrapfigure}

 
ಶಿಷ್ಯವತ್ಸಲರಾದ ಶ್ರೀಮಾನ್ ಗಂಗಾಧರ ಭಟ್ಟರಿಗೆ ಅವರ ವಿದ್ಯಾರ್ಥಿಗಳಾದ ನಾವು ಯಥಾಮತಿ ಅಭಿವಂದನೆಯನ್ನು ಸಾಂಗೋಪಾಂಗವಾಗಿ ಸಮರ್ಪಿಸಿದ್ದೇವೆಂದು ಭಾವಿಸಿದ್ದೇವೆ. ಕಾರ್ಯಕ್ರಮದ ಅಂಗವಾಗಿ ಅಂದೇ ಅಭಿವಂದನ ಗ್ರಂಥಮುಕುಲವೊಂದನ್ನೂ ಸಮರ್ಪಿಸಿದ್ದೆವು. ಅಂದರೆ, ಅಂದು ಅದು ಪೂರ್ಣವಾಗಿ ಅರಳಿರಲಿಲ್ಲ. ಈಗ ಅದು ಸುಗಂಧ\break ಬೀರುವ ಮಕರಂದಸ್ರವಿಸುವ ಪರಾಗಗಳಿಂದ ಪೂರ್ಣವಾದ ಗ್ರಂಥಸುಮವಾಗಿ ಅರಳಿ ನಿಂತಿದೆ. ಈ ಸುಮ, ವಾಚಕ ಭ್ರಮರಗಳಿಗೆ ಸುಮನೋಹರವಾಗಬಹುದೆಂಬ ನಂಬಿಕೆ ನಮ್ಮದು.

ಅಭಿನಂದನ ಶಬ್ದ ಮತ್ತು ಅದರ ಪ್ರಯೋಗ ನಮಗೆಲ್ಲ ಚಿರಪರಿಚಿತವಾಗಿದೆ. ಆದರೆ ವಿದ್ಯಾರ್ಥಿಗಳಾದ ನಾವು ನಮ್ಮ ಆಚಾರ್ಯರಿಗೆ ವಂದನೆಯನ್ನು ಸಂಭಾವಿಸಿ ಅದರ \break ಸಮರ್ಪಣೆಗೆ ‘ಅಭಿವಂದನ’ ಕಾರ್ಯಕ್ರಮ ಎಂದು ಹೆಸರಿಸುವುದು ಉಚಿತವೆಂದು ತಿಳಿದು ಆ ಕಾರ್ಯವನ್ನು ಸಂಪನ್ನಗೊಳಿಸಿದ್ದೇವೆ. ಗ್ರಂಥವೂ ಅಭಿವಂದನ ಗ್ರಂಥವೇ ಆಗಿದೆ. ಇದಕ್ಕೆಲ್ಲ ನಮಗೆ ಗಂಗಾಧರ ಭಟ್ಟರೇ ಸ್ಫೂರ್ತಿ. ಅವರು ಈ ಹಿಂದೆಯೇ ತಮ್ಮ ಗುರುಗಳಾದ ಮಹಾಮಹೋಪಾಧ್ಯಾಯ ಎನ್.ಎಸ್.ರಾಮಭದ್ರಾಚಾರ್ಯರನ್ನು\break ಗೌರವಿಸಿ, ಕೃತಜ್ಞತೆಯನ್ನು ಸಮರ್ಪಿಸಲು ಹಮ್ಮಿಕೊಂಡ ಕಾರ್ಯಕ್ರಮಕ್ಕೆ, ಸಮಾಜದಲ್ಲಿ ಬಹು  ಬಳಕೆಯಲ್ಲಿರುವ ಅಭಿನಂದನ ಪದ ಬಿಟ್ಟು ಅಭಿವಂದನದಲ್ಲಿ \hbox{ಔಚಿತ್ಯವನ್ನು} ಭಾವಿಸಿ ಮೊದಲ ಬಾರಿಗೆ ಅಭಿವಂದನ ಪದವನ್ನು ಬಳಸಿ ಕಾರ್ಯಕ್ರಮ ಮಾಡಿದರು. \hbox{ಅಭಿನಂದನ} ಪದ ವಿದ್ಯಾರ್ಥಿಗಳು ಆಚಾರ್ಯರಿಗೆ ಅರ್ಪಿಸುವ ವಂದನೆಗಿಂತ ಸ್ವಲ್ಪ \hbox{ಭಿನ್ನವಾದ} ಅರ್ಥದಲ್ಲಿ ಸಮಾಜದಲ್ಲಿ ಬಳಕೆಯಲ್ಲಿದೆ. ದಿನಂಪ್ರತಿ ಆ ಹೆಸರಿನಲ್ಲಿ ನಡೆಯುವ ಕಾರ್ಯಕ್ರಮಗಳನ್ನು ನಾವು ನೋಡುತ್ತಿದ್ದೇವೆ. ಎಷ್ಟೇ ಉತ್ಕೃಷ್ಟವಾದ ಅರ್ಥವುಳ್ಳ ಪದಗಳಾದರೂ ಅದು ಅತಿಯಾಗಿ ಪ್ರಯುಕ್ತವಾಗಲು ಪ್ರಾರಂಭವಾದರೆ ಆ ’ಪದ’ ತನ್ನ ಸ್ಥಾನದಿಂದ ಚ್ಯುತವಾಗಿ, ಪದ\enginline{-}ಪದಾರ್ಥಗಳ ಭಾವ ಲಘುವಾಗಿಬಿಡುವ ಸಂಭವವೇ ಹೆಚ್ಚು. ಸದ್ಯ ಅಭಿವಂದನ ಪದಕ್ಕೆ ಇನ್ನೂ ಆ ಪರಿಸ್ಥಿತಿ ಬಂದಿಲ್ಲ. ಹಾಗಾಗಿ ಇದೇ ಪದ ನಮ್ಮ ನಡೆನುಡಿಗೆ ಅನುಗುಣವಾವಾಗಿದೆ ಎನ್ನಿಸಿತು. ಅದನ್ನೇ ಬಳಸಿ ನಡೆಸಿದ ಕಾರ್ಯಕ್ರಮ ಸಾರ್ಥಕವಾಯಿತು. ಗ್ರಂಥವೂ ಅರ್ಥಪೂರ್ಣವಾಯಿತು.

ನಾವು ಕಾರ್ಯಕ್ರಮ ಮಾಡುವ ಸಂದರ್ಭ ಹೇಮಲಂಬದಿಂದ ವಿಲಂಬದೆಡೆಗೆ\break ಸಂಕ್ರಾಂತವಾಗುತ್ತಿದ್ದ ಸಂಧಿಕಾಲ. ಆ ಸಂವತ್ಸರ ಧರ್ಮವೋ ಏನೋ ! ಕಾರ್ಯಕ್ರಮದ ಅನಂತರ ನಡೆಯಬೇಕಿದ್ದ ಗ್ರಂಥಪ್ರಸವ ಗಜಗರ್ಭದಂತೆಯೇ ಸಾಕಷ್ಟು ವಿಲಂಬವೇ ಆಯಿತು. ಅದಿರಲಿ, ಆದರೆ ಗ್ರಂಥದ ಪ್ರಕಾಶನವೆಂದರೆ ಅದು ಒಬ್ಬ ಗರ್ಭಿಣಿಯ\break ಜೀವನಕ್ಕಿಂತ ಭಿನ್ನವೇನಲ್ಲ \enginline{-} ದೀರ್ಘ ಕಾಲಾಪೇಕ್ಷೀ, ಅಷ್ಟೇ ಶ್ರಮದಾಯಕ.\break ಗರ್ಭವತಿಗೆ ತನ್ನೊಳಗೇ ಇರುವ ಗರ್ಭದ ಬಗೆಗೆ ಏನೆಲ್ಲ ಕುತೂಹಲ ಇರಬಹುದೋ\break ಅದು ಸಂಪಾದಕನಿಗೂ ಉಂಟು. “ಕ್ಲೇಶಃ ಫಲೇನ ಹಿ ಪುನರ್ನವತಾಂ ವಿಧತ್ತೇ” ಎನ್ನುವಂತೆ ಪ್ರಸವವಾದೊಡನೆ ಕ್ಲೇಶವೆಲ್ಲ ಮರೆತು ಯಾವ ಆನಂದ \hbox{ತಾಯಿಗುಂಟೋ,} ಗ್ರಂಥವು ಪ್ರಕಟವಾದಾಗ ಸಂಪಾದಕನಿಗಾಗುವ ಸಂತೋಷ ಅದಕ್ಕಿಂತ ಕಡಿಮೆಯದೇನೂ ಅಲ್ಲ. ಮಗುವನ್ನು ನೋಡುವತನಕ ತಾಯಿಗೆ ನೆಮ್ಮದಿಯಿಲ್ಲ, ಪತಿಯ\break ಕೈಯ್ಯಲ್ಲಿಡುವ ತನಕ ಸಮಾಧಾನವಿಲ್ಲ.  ಅಂತೆಯೇ ಸಂಪಾದಕನಿಗೂ ಮುದ್ರಣಗೊಂಡ ಪುಸ್ತಕ ನೋಡುವ ತನಕ ನೆಮ್ಮದಿಯಿಲ್ಲ. ಲೇಖಕರ ಕೈಯ್ಯಲ್ಲಿಡುವ ತನಕ \hbox{ಸಮಾಧಾನವಿಲ್ಲ.} ಹಾಗಾಗಿ ಒಂದು ಕೃತಿ ಅದು ಸಂಪಾದಕನ ಶಿಶುವೇ ಎಂಬುದರಲ್ಲಿ ಸಂದೇಹವಿಲ್ಲ. ಲೇಖನಗಳ ಆಕಲನರೂಪವಾದ ಇಂತಹ ಗ್ರಂಥದ ಮಟ್ಟಿಗೆ ಹೇಳುವುದಾದರಂತೂ  ಸಂಪಾದಕನಷ್ಟೇ ಸಂತೋಷ ಲೇಖಕರಿಗೂ ಉಂಟು. ಈ ಸಂತೋಷಾನುಭವಕ್ಕೆ ನಮ್ಮೆಲ್ಲರ ನಿರೀಕ್ಷೆ ಮೀರಿದ ಸಮಯ ಸಂದಿದೆ. ನೈಯಾಯಿಕರು ಹೇಳುವಂತೆ ಇಲ್ಲಿಯೂ ವಿಘ್ನಪ್ರಾಚುರ್ಯವನ್ನು ಭಾವಿಸಬೇಕಷ್ಟೆ !!


ಪ್ರಕಾಶನದ ವಿಷಯದಲ್ಲಿ ಇಂಗ್ಲೆಂಡಿನ ‘ಆಕ್ಸ್ ಫರ್ಡ್’ \enginline{-} ಅತ್ಯಂತ ಶಿಸ್ತುಬದ್ಧವಾದ ಗ್ರಂಥ ಪ್ರಕಾಶನ ಸಂಸ್ಥೆ, ಅದು ಒಂದು ಪುಸ್ತಕವನ್ನು ಹದಿಮೂರು ವರ್ಷಗಳ ಅವಧಿಯಲ್ಲಿ ಪ್ರಕಟಿಸಬೇಕೆಂದು ಯೋಜನೆಯನ್ನು ಹಾಕಿಕೊಂಡಿತು. ಆದರೆ ಆ ಯೋಜನೆಯನ್ನು ಮುಗಿಸಲು ಅದು ಎಪ್ಪತ್ತು ವರ್ಷಗಳನ್ನು ತೆಗೆದುಕೊಂಡಿತು ಎಂದು ವಿದ್ವಾಂಸರಾದ ಎನ್. ಬಾಲಸುಬ್ರಹ್ಮಣ್ಯನವರು ಉಲ್ಲೇಖಿಸಿದ್ದಾರೆ. ಹಾಗಾಗಿ ಪುಸ್ತಕ ಪ್ರಕಾಶನದಲ್ಲಿ ಅದರಲ್ಲೂ ಅನೇಕ ಲೇಖಕರನ್ನು ಅವಲಂಬಿಸಿ ಸಿದ್ಧಪಡಿಸಬೇಕಾದ ಪುಸ್ತಕವನ್ನು  ಅಲ್ಪ  ಅವಧಿಯಲ್ಲಿ ಮುದ್ರಿಸುವ ಪ್ರತಿಜ್ಞೆ ಪ್ರತಿಜ್ಞೆಯೇ ಆಗಿ \hbox{ಉಳಿಯುವುದೇ} ಹೆಚ್ಚು ಎಂದರೆ ಅತಿಶಯವಲ್ಲ. 
ಆರಂಭದಲ್ಲಿ ಲೇಖನ ಸಂಗ್ರಹಕ್ಕಾಗಿ ನಾವು ತೊಡಗಿದಾಗ ಗ್ರಂಥ ವಿಷಯ ಗ್ರಂಥಿಯಾಗಿಯೇ ಉಳಿದುಬಿಡುವ ಆತಂಕವಿತ್ತು. ಯದ್ಯಪಿ ಗ್ರಂಥವನ್ನು ಅಭಿವಂದನ ಕಾರ್ಯಕ್ರಮದಲ್ಲೇ ಪೂರ್ಣಗೊಳಿಸಬೇಕೆಂದಿದ್ದರೂ ಅದನ್ನು ಮುಂದೂಡುವುದಕ್ಕೆ ಇದೂ ಒಂದು ಕಾರಣ. ಆದರೆ ನಿಧಾನವಾಗಿ ಲೇಖನಗಳು ಬರಲು\break ಆರಂಭವಾದವು. ಬಹು ಲೇಖನಗಳು ನಮ್ಮ ಕೈಸೇರಲು ವಿಲಂಬವಾದರೂ ಕೊನೆಯಲ್ಲಿ ಬಂದವುಗಳ ಸಂಖ್ಯೆಯನ್ನು ಗಮನಿಸಿದಾಗ ಅಭಿಮಾನಿ ಲೇಖಕರ ಬಗ್ಗೆ ನಮ್ಮ ಅಭಿಮಾನ ಇಮ್ಮಡಿಸಿದ್ದು ಸುಳ್ಳಲ್ಲ. ಆದರೆ ಲೇಖನಗಳನ್ನೆಲ್ಲ ಗ್ರಂಥದ ಸ್ವರೂಪಕ್ಕೆ  ಅಳವಡಿಸಿ ನೋಡಿದಾಗ, ಆದ ಗ್ರಂಥದ ಗಾತ್ರ ನೋಡಿ ಮತ್ತೆ ಗಾಬರಿಯಾಯಿತು. ಕಾರಣ ಅದು ಆರುನೂರು ಪುಟಗಳನ್ನು ದಾಟಿತ್ತು. ಆದರೆ ಶ್ರೀಮಾನ್ ನರಸಿಂಹ ಹೆಗಡೆ, ಹೊನ್ನೇಹದ್ದ,  ಅವರ ಸುಪುತ್ರರಾದ ಶ್ರೀ ಗಣಪತಿ ಹೆಗಡೆ ಮತ್ತು ಶ್ರೀ ಸೂರ್ಯ\break ನಾರಾಯಣ ಹೆಗಡೆಯವರ ಔದಾರ್ಯ, ಅಂತೆಯೇ ಕೆಲವು ವಿದ್ಯಾರ್ಥಿಗಳ ವಿಶೇಷ ಸಹಕಾರ  ಈ ಕಾರ್ಯವನ್ನು ಸಂಪನ್ನಗೊಳಿಸುವಂತೆ ಮಾಡಿತು.


ಏನೆಲ್ಲ ಪರಿಕರಗಳಿದ್ದರೂ ಇಂತಹ ಗ್ರಂಥಕ್ಕೆ ಜೀವಾಳ ಲೇಖಕರೇ ವಿನಾ ಮತ್ತಾರೂ ಅಲ್ಲ. ಪ್ರಕೃತ  ಸಮಸ್ತ ಲೇಖಕರು ಗಂಗಾಧರ ಭಟ್ಟರ ಬಗೆಗೆ ತಮ್ಮ ಆತ್ಮೀಯತೆಯನ್ನು, ಗೌರವವನ್ನು ಲೇಖನಗಳ ಮೂಲಕ ವ್ಯಕ್ತಪಡಿಸಿದ್ದಾರೆ. ಇಲ್ಲಿ ಲೇಖನ ಬರೆದವರಲ್ಲಿ ಬಹುತೇಕರು ನಮ್ಮ ಅಧ್ಯಾಪಕ ವೃಂದದವರೇ ಆಗಿದ್ದಾರೆ. ಹಾಗೆಂದು ಉಳಿದವರೂ ಸಹ ತತ್ಸಮಾನರೇ ವಿನಾ ಅನ್ಯರಲ್ಲ. ಗಂಗಾಧರ ಭಟ್ಟರ ವಿದ್ಯಾರ್ಥಿಗಳಲ್ಲೂ ಅನೇಕರು ಲೇಖನ ಸೇವೆ ಸಲ್ಲಿಸಿದ್ದಾರೆ.

ವಾಸ್ತವವಾಗಿ, ಮೊದಲು ಲೇಖನಕ್ಕಾಗಿ ಲೇಖಕರನ್ನು ವಿನಂತಿಸುವ ಪತ್ರದಲ್ಲಿ ನಾವು ಗೌರವ ಸಂಭಾವನೆಯನ್ನು ನೀಡುವ ಬಗ್ಗೆ ಪ್ರಸ್ತಾಪಿಸಿದ್ದೆವು, ಆದರೆ ಕೆಲವರು ಇದನ್ನು ವಿರೋಧಿಸಿದರು. “ಗಂಗಾಧರ ಭಟ್ಟರ ಬಗ್ಗೆ ನಮಗಿರುವ ಅಭಿಮಾನದಿಂದ ನಾವಿದನ್ನು\break ಕೊಡುತ್ತಿರುವುದೇ ವಿನಾ ಸಂಭಾವನೆಯನ್ನು ಭಾವಿಸಿ ಕೊಡುತ್ತಿರುವುದಲ್ಲ. ಇಂತಹ \hbox{ಗ್ರಂಥಗಳಿಗೆ} ಅಭಿಮಾನಿಗಳು ಬರೆಯುತ್ತಾರೆಯೇ ಹೊರತು ಸಂಭಾವನೆಯ ಅಪೇಕ್ಷೆಯುಳ್ಳವರಲ್ಲ, ಇಂತಹ ವ್ಯವಹಾರ ರೂಢಿಯಲ್ಲೂ ಇಲ್ಲ. ಹಾಗಾಗಿ ನಿಮ್ಮ\break ನಿರ್ಧಾರದಲ್ಲಿ ಔಚಿತ್ಯವಿಲ್ಲ”, ಎಂದು ಖಡಕ್ಕಾಗಿ, ಅಷ್ಟೇ ವಿಶ್ವಾಸಪೂರ್ವಕವಾಗಿ ತಮ್ಮ ಅಭಿಪ್ರಾಯವನ್ನು ವ್ಯಕ್ತಪಡಿಸಿದರು. ಅಲ್ಲಿಗೆ ನಾವು ಸಂಭಾವನೆಯ ಯೋಚನೆಯನ್ನು\break ಕೈಬಿಡುವ ತೀರ್ಮಾನ ತೆಗೆದುಕೊಂಡೆವು. ತನ್ಮೂಲಕ ಲೇಖಕರು ಉತ್ತಮರ್ಣರಾದರು. ಹಾಗಾಗಿ, ಗಂಗಾಧರ ಭಟ್ಟರ ಆತ್ಮೀಯರೂ ಅವರ ವಿದ್ಯಾರ್ಥಿಗಳೂ ಆಗಿರುವ ನಮ್ಮಲ್ಲಿ ಪ್ರಸಾದ ಭಾವಭರಿತರೂ ಆದ ಸಮಸ್ತ ಲೇಖಕರಿಗೆ ಸಂಪಾದಕನ ಮತ್ತು ಸಮಿತಿಯ\break ಗೌರವಪೂರ್ವಕವಾದ ಭೂರಿ ಭೂರಿ ನಮನಗಳು ಸಲ್ಲುತ್ತವೆ.


ಗ್ರಂಥವನ್ನು ಮೂರು ವಿಭಾಗವಾಗಿ ವಿಂಗಡಿಸಿದೆ. ಮೊದಲನೆಯದು ಅಭಿವಂದನ ಕಾರ್ಯಕ್ರಮಕ್ಕೆ ಸಂಬಂಧಿಸಿದ ಸಚಿತ್ರ ವರದಿ. ಹಾಗಾಗಿ ಇದು ಚಿತ್ರ ಸಂಪುಟ. ಎರಡನೆಯದು ಶಾಸ್ತ್ರಸಂಪುಟ \enginline{-} ಇದು ಸಂಸ್ಕೃತದಲ್ಲಿದ್ದು ವಿವಿಧ ಶಾಸ್ತ್ರಸಂಬಂಧಿ ಲೇಖನಗಳನ್ನು ಒಳಗೊಂಡಿದೆ. ಮೂರನೆಯದು  ಒಡನಾಡಿ ಸಂಪುಟ. \hbox{ಗಂಗಾಧರ ಭಟ್ಟರ} ಒಡನಾಡಿಗಳು ಅವರೊಡನೆ ಇರುವ ಒಡನಾಟದ ಅನುಭವಗಳನ್ನು ಇಲ್ಲಿ\break ಸ್ವರಸವಾಗಿ ಸ್ಮರಿಸಿಕೊಂಡಿದ್ದಾರೆ. ಈ ಲೇಖನಗಳು ಅನೇಕ ಭಾಷೆಗಳಲ್ಲಿವೆ. \hbox{ಲೇಖಕರಲ್ಲಿ} ಕೆಲವರು ಶಾಸ್ತ್ರದ ಲೇಖನವನ್ನು, ಕೆಲವರು ಗಂಗಾಧರ ಭಟ್ಟರ ಬಗೆಗಿನ ಲೇಖನವನ್ನು, ಮತ್ತೆ ಕೆಲವರು ಎರಡೂ ವಿಭಾಗಕ್ಕೆ ಸಲ್ಲುವ ಲೇಖನ\-ವನ್ನೂ ಕೊಟ್ಟು ಉಪಕರಿಸಿದ್ದಾರೆ. ಈ ಎಲ್ಲ ವಿಷಯಗಳನ್ನು ಒಳಗೊಂಡು ಗ್ರಂಥ ಆರುನೂರು ಪುಟಗಳನ್ನು \hbox{ದಾಟಿದೆ.} ಹಾಂ! ಹಾಗೆಂದು ನಮಗೆ ಗ್ರಂಥದ ಈ ಬೃಹತ್ತತೆಯೇ ಬೀಗುವ ವಿಷಯವಾಗಿಲ್ಲ. 'ಚಿತ್ರ\-ಸಂಪುಟ' ಕಾರ್ಯಕ್ರಮವನ್ನು ಕಣ್ಣಿಗೆ ಕಟ್ಟುವಂತೆ ಮಾಡಿದರೆ, \hbox{'ಶಾಸ್ತ್ರಸಂಪುಟ'} ಗಂಭೀರ ಚಿಂತನೆಗಳಿಂದ ಕೂಡಿದ ನಾನಾ ಶಾಸ್ತ್ರದ ಸಂಶೋಧನ ಬರಹಗಳನ್ನು ಒಳಗೊಂಡಿದೆ. ಮೂರನೆಯ 'ಒಡನಾಡಿ' ಸಂಪುಟವಂತೂ ಗಂಗಾಧರ ಭಟ್ಟರ ಜೀವನಕ್ಕೆ \hbox{ಆದರ್ಶವಾಗಿದೆ.} ಹಾಗಾಗಿ ಒಂದೇ ಗ್ರಂಥ  ವಿಭಿನ್ನ ರಸ \hbox{ಆಸ್ವಾದಕರಿಗೂ} ಒಂದು ಸಮಾರಾಧನೆಯಾಗಿದೆ. ಇದು ನಿಜವಾಗಿ ನಮ್ಮ ಹೆಮ್ಮೆಗೆ ಕಾರಣವಾದ ಸಂಗತಿ. ಇಷ್ಟಾಗಿಯೂ ಎಲ್ಲ ವೇದಗಳ ಬಗ್ಗೆಯೂ ಒಂದೊಂದು ಲೇಖನಗಳನ್ನು ಪಡೆದುಕೊಳ್ಳುವ ನಮ್ಮ ಅಪೇಕ್ಷೆ ಫಲಿಸಿಲ್ಲ. ಡಾ ॥ ಟಿ.ವಿ.ಸತ್ಯನಾರಾಯಣರವರು ಕೊಡಮಾಡಿದ ಅಥರ್ವವೇದ ವಿಷಯಕ ಲೇಖನವೇ ಮೂರೂ ವೇದಗಳಿಗೆ ಪ್ರತಿನಿಧಿಯಾಗಿದೆ. ಎಲ್ಲ ವೇದಗಳ ಲೇಖನಗಳೂ ಇಲ್ಲಿ ಒದಗಿ ಬಂದಿದ್ದರೆ ಗ್ರಂಥಗೌರವ ಇನ್ನೂ \hbox{ಹಿಗ್ಗುತ್ತಿತ್ತೇನೋ~!} ಈ ಒಂದು ನ್ಯೂನತಾ ಭಾವ ನಮ್ಮನ್ನು ಕಿಂಚಿತ್ ಕಾಡಿದ್ದು ಸುಳ್ಳಲ್ಲ.
ಇನ್ನು, \hbox{ಮೂರನೇ} ವಿಭಾಗದ ಲೇಖನಗಳು ಶ್ರೀಯುತರ ಜೀವನದ ಬಗೆಗೆ ಸಾಕಷ್ಟು  ಸ್ವಾರಸ್ಯಕರವಾದ ವಿಷಯಗಳನ್ನು ಅನಾವರಣ ಮಾಡುವಂಥವು. ಗಂಗಾಧರ ಭಟ್ಟರ ಅಣ್ಣತಂಗಿಯರೂ ಇಲ್ಲಿ ಲೇಖನಗಳನ್ನು ಬರೆದಿದ್ದಾರೆ. ಇವು ಅವರ ಬಾಲ್ಯಕಾಲದ ಕೆಲವು ಅಪರೂಪದ ಘಟನೆಗಳನ್ನು ತೆರೆದಿಡುತ್ತವೆ. ಇವುಗಳಿಂದ ವಿಶೇಷವಾಗಿ ಒಂದು ಅಂಶವನ್ನು ನಾವು \hbox{ಗುರುತಿಸಬೇಕು} \enginline{-} ಅದೇನೆಂದರೆ, ಗಂಗಾಧರ ಭಟ್ಟರು ಮನೆಗೆ ಮಾರಿಯಾಗಿ ಊರಿಗೆ ಉಪಕಾರಿಯಾದವರೂ ಅಲ್ಲ. ಕುಟುಂಬ \hbox{ಸ್ವಾರ್ಥಿಯಾಗಿ} ಸಮಾಜಕ್ಕೆ ದಕ್ಕದವರೂ ಅಲ್ಲ. ಭೇದವನ್ನು ಭಾವಿಸದೆ ಅವೆರಡನ್ನೂ ಸಮಾನವಾಗಿ ಕಂಡರು, ಸಮಾನವಾಗಿ \hbox{ತೂಗಿಸಿದರು.} ನಮ್ಮವರೆಂಬ ಅಭಿಮಾನ, ಸರ್ವಸಮಾನ ಸ್ನೇಹ, \hbox{ಹಾಗಿದ್ದೂ} ಎಲ್ಲೂ \hbox{ಮೋಹವಿಲ್ಲದ} ನ್ಯಾಯನಿಷ್ಠ ನಿಲುವು \enginline{-} ಇದು ಗಂಗಾಧರ ಭಟ್ಟರ ವ್ಯಕ್ತಿತ್ವದ ವೈಶಿಷ್ಟ್ಯ. ಇದನ್ನು ನಾವು ಆ ಲೇಖನಗಳಿಂದ ಗುರುತಿಸಬಹುದು.


ಗ್ರಂಥದ ಇದೇ ವಿಭಾಗಕ್ಕೆ ನಮ್ಮ ನಾಡಿನ ಪ್ರಸಿದ್ಧ ವಿದ್ವಾಂಸರೂ ಆಶುಕವಿಗಳೂ ಆಗಿರುವ ಡಾ~॥ ಹೆಚ್.ವಿ.ನಾಗರಾಜರಾವ್ ರವರು ಗಂಗಾಧರ ಭಟ್ಟರ ವ್ಯಕ್ತಿತ್ವದ ಬಗೆಗೆ ಸುಂದರವಾದ ಶ್ಲೋಕಗಗಳನ್ನು ರಚಿಸಿ ಕೊಟ್ಟಿದ್ದಾರೆ. “ವಿದ್ಯೆ ಇರುವವನಲ್ಲಿ\break ವಿನಯವಿಲ್ಲ, ವೈದುಷ್ಯವಿರುವವನು ದಯಾವಂತನಲ್ಲ” ಎಂಬ ಈ ಲೋಕೋಕ್ತಿ ಸುಳ್ಳು ಎಂಬುದು ಗಂಗಾಧರ ಭಟ್ಟರನ್ನು ನೋಡಿದರೆ ಸ್ಪಷ್ಟವಾಗುತ್ತದೆ ಎಂಬಿತ್ಯಾದಿ\break ಅಂಶಗಳನ್ನು ಬಹಳ ಸ್ವಾರಸ್ಯವಾಗಿ ಹೇಳಿದ್ದಾರೆ. ಮಹಾನ್ ವಿದ್ವಾಂಸರಾದ ಮೇಲು\-ಕೋಟೆಯ ಅರೈಯ್ಯರ್ ಶ್ರೀರಾಮಶರ್ಮರು ಕಾವ್ಯರಚನಾ ಚತುರರು. ಅವರು ವಯೋಧರ್ಮದ ಅಸಹಕಾರದ ನಡುವೆಯೂ ಅಭಿಮಾನದಿಂದ ಅಭಿವಂದನ ಕಾರ್ಯ\-ಕ್ರಮದಲ್ಲಿ ದಿನಪೂರ್ತಿ ಉಪಸ್ಥಿತರಿದ್ದರು. ಆ ಸಂದರ್ಭದ ಪ್ರಭಾವದಿಂದ ಸಭೆಯಲ್ಲಿಯೇ ಕೆಲವು ಶ್ಲೋಕಗಳನ್ನು ರಚಿಸಿದರು. ತಮ್ಮ ರಚನೆಯಲ್ಲಿ ಅವರು ಗಂಗಾಧರ ಭಟ್ಟರನ್ನು \enginline{-}“ಹವ್ಯಕವ್ಯೋಮ\enginline{-}ಭಾಸ್ಕರ” ಎಂದು ಹಾಡಿದ್ದಾರೆ. ಅಲ್ಲದೇ, ಭಟ್ಟರು ನಿತ್ಯ ವಹ್ನಿಧೂಮಗಳ ದೃಷ್ಟಾಂತದೊಡನೆ ಪಾಠವನ್ನು ಮಾಡುತ್ತಿದ್ದವರಾದ್ದರಿಂದ (ತರ್ಕ\break ಎಂದೊಡನೆ ನೆನಪಾಗುವುದೇ ವಹ್ನಿಧೂಮಗಳಷ್ಟೇ ! ತಮಾಶೆಯಾಗಿ ಹೇಳಬೇಕೆಂದರೆ ನೈಯಾಯಿಕರಿಗೂ ವಹ್ನಿಧೂಮಗಳಿಗೂ ನಿಯತ ಸಾಹಚರ್ಯ) ಅದೇ ವಹ್ನಿ ಮತ್ತು ಧೂಮಗಳನ್ನು  ಗಂಗಾಧರ ಭಟ್ಟರ ಪ್ರಶಂಸೆಗೆ ಬಳಸಿಕೊಂಡ ಬಗೆ ಮಾತ್ರ ಅದ್ಭುತ \enginline{-} “ಜ್ಞಾನವೆಂಬ ಆಜ್ಯಾಹುತಿಯಿಂದ ಹೊಗೆ(ದೋಷ)ಇಲ್ಲದ ಶುದ್ಧಾಗ್ನಿಯಂತೆ ಗಂಗಾಧರ ಭಟ್ಟರು ಬೆಳಗುತ್ತಾರೆ \enginline{-} ವಿಧೂಮಃ ಪಾವಕ ಇವ ಜ್ಞಾನಾಜ್ಯಾಹುತಿಭಿರ್ಜ್ವಲನ್” ಎಂದು ಉದ್ಗರಿಸಿದ್ದಾರೆ. ಅವರ ಈ ಪದ್ಯಪಂಕ್ತಿ ಸಹೃದಯರನ್ನು ಒಮ್ಮೆ ಗಂಭೀರಭಾವದೆಡೆಗೆ\break ಸೆಳೆಯದಿರದು. ಅವರ ಆ ಶ್ಲೋಕಸರಣಿಯೇ ಬಹಳ ಗಂಭೀರವಾಗಿದೆ. ಅಂತೆಯೇ, ಇನ್ನೊಂದು ಪದ್ಯಮಾಲಿಕೆಯನ್ನು ವಿ ॥ ಮಂಜುನಾಥ.ಜಿ. ಭಟ್ಟರು ರಚಿದ್ದಾರೆ. ಇವರು ಗಂಗಾಧರ ಭಟ್ಟರ ಊರಿನ ಪಕ್ಕದ ಊರಿನವರೇ ಆಗಿದ್ದು ಅವರನ್ನು ಬಾಲ್ಯದಿಂದಲೂ ಬಲ್ಲವರು. ಅವರು ಶ್ರೀಯುತರ ಜೀವನದ ವಿಷಯವಾಗಿ “ಗಂಗಾಧರೋ  ಭಟ್ಟವರೋ\break ವಿರಾಜತಾಮ್” ಎಂಬ ಲಲಿತವಾದ ಪದ್ಯಮಾಲಿಕೆಯನ್ನು  ರಚಿಸಿದ್ದಾರೆ. ಆ\break ಮಾಲಿಕೆ ಸಂಸ್ಕೃತದಲ್ಲಿದ್ದರೂ ಭಿನ್ನಭಾಷಾಭಿಜ್ಞರನ್ನೂ ಸೆಳೆಯುವುದರಲ್ಲಿ ಸಂದೇಹವಿಲ್ಲ. ಮಾಲಿಕಾ ಸಂರಚನೆಯೇ ದೊಡ್ಡ ವಿಷಯವಲ್ಲ. ಅದು ಮೂರ್ತಿಗೆ ತಕ್ಕ ಅಲಂಕಾರ ಆಗಿ ಮೂರ್ತಿಯ ಸೊಬಗು ಮತ್ತೂ ಚೆನ್ನಾಗಿ ಬೆಳಗುವಂತಾದರೆ ಆಗ ಮಾಲಿಕಾ ಅಲಂಕಾರ ಸಾರ್ಥಕ, ಹಾಗಲ್ಲದೇ ಮಾಲಿಕೆ ಮೂರ್ತಿಯನ್ನೇ ಮುಚ್ಚಿ, ತಾನೇ ಎದ್ದು ತೋರುವಂತಾದರೆ ಅಂತಹ ಅಲಂಕಾರವನ್ನು, ಅಲಂ \enginline{-} ಬೇಡ ಎನ್ನಬೇಕಾಗುತ್ತದೆ. ಆದರೆ ಈ \hbox{ಪದ್ಯಮಾಲಿಕೆ} ಮೂರ್ತಿಯ ಸೊಬಗನ್ನು ಇತೋಪ್ಯತಿಶಯವಾಗಿ ಸಂವರ್ಧಿಸಿದೆ ಎಂಬುದರಲ್ಲಿ ಸಂದೇಹವಿಲ್ಲ. ಈ ಎಲ್ಲ ಕಾವ್ಯಕೋವಿದರಿಗೆ ನಮ್ಮ ನಮನಗಳು \hbox{ಸಲ್ಲುತ್ತವೆ.}

 
ಹಿಂದಿನ ವಿಭಾಗದಂತೆ ಈ ವಿಭಾಗದಲ್ಲೂ ನಮಗೊಂದು ಉಲ್ಲೇಖನೀಯ\break ಕೊರತೆಯಾಗಿದೆ, ಗಂಗಾಧರ ಭಟ್ಟರ ಗುರುಕುಟುಂಬ \enginline{-} ಶ್ರೀಮಾನ್ ಎನ್.ಎಸ್.ರಾಮ\-ಭದ್ರಾಚಾರ್ಯರ ಮನೆಯಿಂದ ನಾವು ನಿರೀಕ್ಷಿಸಿದ್ದ ಬಹುಮುಖ್ಯ ಲೇಖನ, ಅವರಲ್ಲಿ ಹೆಚ್ಚು ಪರಿಚಯವಿದ್ದ ಶ್ರೀಮತಿ ಕಲ್ಯಾಣಿಯವರ ತೀವ್ರ ಅನಾರೋಗ್ಯದಿಂದ ನಮಗೆ ಪ್ರಾಪ್ತವಾಗಲಿಲ್ಲ. ಇದು ಇಲ್ಲಿಯ ಕೊರತೆಯೇ. ಏಕೆಂದರೆ ಗಂಗಾಧರ ಭಟ್ಟರು ವಿದ್ಯಾರ್ಥಿಗಳಿಗೆ ಹೇಗೆ ಬಂಧುವಾಗಿದ್ದರೋ ಅಂತೆಯೇ ಅವರ ಗುರುಗಳಿಗೂ ಅಷ್ಟೇ ಬಂಧುವಾಗಿದ್ದರು. ಹಾಗಾಗಿ ಅಲ್ಲಿಯ ಲೇಖನ ಶ್ರೀಯುತರ ವಿಶಿಷ್ಟ ವ್ಯಕ್ತಿತ್ವಕ್ಕೆ ಒಂದು ಸ್ವಚ್ಛ ಕನ್ನಡಿಯಾಗಿರುತ್ತಿತ್ತು. ಅದೊದಗದಿದ್ದುದು ಈ ಗ್ರಂಥದ ಬಹುಮುಖ್ಯ ಕೊರತೆ ಎಂದೇ ನಮ್ಮ ಭಾವನೆ.

ಈ ಎರಡು ವಿಭಾಗದ ಮಧ್ಯದಲ್ಲಿ ಮಧ್ಯಮಣಿಯಾಗಿ ಲೇಖನವಿಶೇಷವೊಂದು ಪೋಣಿಸಲ್ಪಟ್ಟಿದೆ. ಅದು ಅಭಿವಂದ್ಯರಾದ ಗಂಗಾಧರ ಭಟ್ಟರದೇ. ಅವರನ್ನು ನಾವು, “ಗ್ರಂಥಕ್ಕೆ ತಾವೂ ತಮ್ಮ ಜೀವನದ ಬಗ್ಗೆ ಒಂದು ಲೇಖನವನ್ನು ನೀಡಬೇಕೆಂದು\break ವಿನಂತಿಸಿದೆವು. ಅವರೆಂದೂ ಯಾರ ಅಪೇಕ್ಷೆಯನ್ನೂ ಉಪೇಕ್ಷಿಸಿದವರಲ್ಲವಲ್ಲ !\break ಅನಾರೋಗ್ಯದ ಮಧ್ಯೆಯೂ ಲೇಖನವನ್ನು ದಯಪಾಲಿಸಿದ್ದಾರೆ. ಅವರ ಲೇಖನವನ್ನು ಅವರ ಕುಟುಂಬಿನಿಯಾದ \enginline{-} ಶ್ರೀಮತಿ ಶೈಲಜಾರವರೇ ಟಂಕಿಸಿಕೊಟ್ಟಿದ್ದಾರೆ. ಸರಳವೂ ಶೃಂಖಲಾಲಂಕೃತವೂ ಅಷ್ಟೇ ನಿರಾಡಂಬರವೂ ಆದ ಈ ಲೇಖನ ಅವರು ನಡೆದು ಬಂದ ಹಾದಿಯನ್ನು ತಿಳಿಸುತ್ತದೆ. ಅದಕ್ಕಾಗಿ ನಾವು ಅವರಿಬ್ಬರಿಗೂ ಪ್ರತ್ಯೇಕವಾಗಿ ಭೂರಿ ಕೃತಜ್ಞತೆಯನ್ನು ಸಮರ್ಪಿಸುತ್ತೇವೆ.


ಸಂಸ್ಕೃತಜ್ಞರು ವ್ಯಾಸ, ಸಮಾಸ ಇವೆರಡಲ್ಲೂ ಸಮಾನವಾಗಿ ಪ್ರೌಢಿಮೆಯನ್ನು\break ಕಾಯ್ದುಕೊಳ್ಳಲು ಸಮರ್ಥರಾದರೂ ಹಲವರಿಗೆ ಸಮಾಸದಲ್ಲಿ ಒಲವು ಹೆಚ್ಚು. ಸಂಸ್ಕೃತ ಭಾಷೆಯ ಸಂರಚನೆಯೂ ಅದಕ್ಕೇ  ಪೋಷಕವಾಗಿರುವುದು ಸುಳ್ಳಲ್ಲ. ಬೇರೆ ಭಾಷೆಗಳಲ್ಲಿ ಅನೇಕ ಪುಟಗಳಲ್ಲಿ ಹೇಳುವುದನ್ನು ಈ ಭಾಷೆಯಲ್ಲಿ ಕೆಲವೇ ವಾಕ್ಯಗಳಲ್ಲಿ ಹೇಳಿಬಿಡಬಹುದು. ಈ ದೃಷ್ಟಿಯಿಂದ ಇಲ್ಲಿಯ ಲೇಖನಗಳು ದೀರ್ಘವೆಂದು ಯಾರಿಗಾದರೂ\break ಅನ್ನಿಸಬಹುದು. ಅಂತೆಯೇ ಕನ್ನಡದ ಲೇಖನಗಳೂ ಸಹ. ಅವಾದರೋ ಗಂಗಾಧರ ಭಟ್ಟರೊಡನಿರುವ ಅವರವರ ಒಡನಾಟದ ವಿಷಯಗಳಾದ್ದರಿಂದ ಆ ವಿಸ್ತಾರಕ್ಕೆ ತಕ್ಕಂತೆ ಲೇಖನಗಳ ವಿಸ್ತರವಿದೆ. ಇನ್ನು, ಲೇಖನಗಳೇನಿದ್ದರೂ ನಾನಾ ಕ್ಷೇತ್ರದ ಓದುಗ\-ರನ್ನು ತಾನೆ ಲಕ್ಷಿಸಿರಬೇಕು ! ಅಲ್ಲಿ ಕೆಲವರು ಸಂಕ್ಷೇಪವನ್ನು ಅಪೇಕ್ಷಿಸಿದರೆ, ಮತ್ತೆ ಕೆಲವರು\break ವಿಸ್ತರವನ್ನು ನಿರೀಕ್ಷಿಸುತ್ತಾರೆ. ಆಯಾಯಾ ಸಂದರ್ಭ, ವಿಷಯಗಳಿಗೆ ಅನುಗುಣವಾಗಿ ಎರಡರಲ್ಲೂ ಔಚಿತ್ಯವೂ ಇಲ್ಲದಿಲ್ಲ. ಎಲ್ಲಿ ಪ್ರಯೋಜ್ಯ ಅಧಿಕಾರಿಯ ನಿರ್ಣಯ ಇಲ್ಲವೋ ಅಲ್ಲಿ ವ್ಯಾಸ ಸಮಾಸಗಳನ್ನೂ ನಿರ್ಬಂಧಿಸಲಾಗದು. ಆದರೆ ವಿಸ್ತರ ಸಣ್ಣ\-ಪುಟ್ಟ ವಿಷಯಗಳನ್ನೂ ವಿಶದವಾಗಿ ವಿವರಿಸುವುದರಿಂದ ಸಾಮಾನ್ಯ ಓದುಗರ ಕುತೂಹಲ\break ತಣಿಯುತ್ತದೆ. ಮಾತ್ರವಲ್ಲ, ಸಣ್ಣ ಘಟನೆಗಳೂ ದಾಖಲೆಯಾಗಿ ಉಳಿಯುತ್ತ\-ವೆ. ಹಾಗಾಗಿ ದೈರ್ಘ್ಯಕ್ಕೆ ಅಂಜಿ, ಲೇಖನಕಾರ್ಪಣ್ಯದಿಂದ ಹೇಳಬೇಕಿರುವ ಯಾವ ವಿಷಯವೂ\break ಬಿಟ್ಟುಹೋಗಬಾರದೆಂದು ಮುಕ್ತವಾಗಿ ಬರೆಯುವಂತೆ ನಾವು  ಲೇಖಕರಲ್ಲಿ ಆಗಾಗ\break ಜ್ಞಾಪಿಸಿದ್ದುಂಟು. ತಿಳಿಯದವರಿಗೂ, ವಿವರವನ್ನು ಅಪೇಕ್ಷಿಸುವವರಿಗೂ ಕೊರತೆಯ ಭಾವ ಬರದಿರಲೆಂಬುದು ನಮ್ಮ ಭಾವನೆ.


ಇಲ್ಲಿ ಈ ಗ್ರಂಥದ ಮುಖಪುಟ ಚಿತ್ರದ ಬಗೆಗೆ ಮತ್ತು ಗ್ರಂಥಕ್ಕೆ ಬಳಸಿದ\break ಹೆಸರಿಗೆ ಹಿನ್ನೆಲೆಯಲ್ಲಿರುವ ಕೆಲವು ಚಿಂತನೆಗಳನ್ನು ಕೊಂಚ ಉಲ್ಲೇಖಿಸಬೇಕಿದೆ \enginline{-} ಏಕೆಂದರೆ, ಬಹುತೇಕರಿಗೆ ತಿಳಿದಿರುವಂತೆ ನಮ್ಮ ಆಚಾರ್ಯರಾದ ಗಂಗಾಧರ ಭಟ್ಟರು \enginline{-} ಅವರು ಒಬ್ಬ ವ್ಯಕ್ತಿಯಲ್ಲ \enginline{-} ಅನೇಕ ವ್ಯಕ್ತಿಗಳ ಸಂಗಮ, ಬಹುಮುಖ ಪ್ರತಿಭೆ. ಸಮಾಜಕ್ಕೆ ಬಹು ಆಕರ್ಷಕ ವ್ಯಕ್ತಿ. ವ್ಯವಹಾರ ಚತುರರು. ಅತಿಶಯ \hbox{ಔದಾರ್ಯವುಳ್ಳವರು.} ಕ್ಷೇತ್ರ ಸಂಸ್ಕೃತದ್ದಾದರೂ ಅದಕ್ಕೇ ಸೀಮಿತರಾದವರಲ್ಲ. ವಿದ್ಯಾರ್ಥಿದೆಸೆಯಿಂದಲೇ ಆರಂಭವಾದ ಅವರ ಪಾಠ ನಿವೃತ್ತಿಯಾದರೂ ನಿಂತಿಲ್ಲ. ಅವರ ಪಾಠದ ಪಾಟವಕ್ಕೆ\break ಆಕರ್ಷಿತರಾಗದ ವಿದ್ಯಾರ್ಥಿಗಳಿಲ್ಲ. ನ್ಯಾಯಶಾಸ್ತ್ರ ಪ್ರಧಾನ ವಿಷಯವಾದರೂ ಇತರ\break ಶಾಸ್ತ್ರಗಳಲ್ಲೂ ಅವರ ಕೃಷಿ ಕಡಿಮೆಯದೇನಲ್ಲ. ತರ್ಕ ಕರ್ಕಶವಾದರೂ, ಕಾವ್ಯ,\break ನಾಟಕಗಳ ಪಾಠದಲ್ಲೂ ಅವರು ರಸವನ್ನು ಜಿನುಗಿಸಿದವರು. ಹಾಗೆಂದು ಕೇವಲ\break ಪಾಠಶೂರರಲ್ಲ. ವಿದ್ಯೆಯನ್ನು  ಕಲೆಯಾಗಿಸಿ ಅಭಿನಯಿಸುವ ನಯ, ನೈಪುಣ್ಯವೂ\break ಅವರಿಗುಂಟು. ಕಾವ್ಯ, ಶಾಸ್ತ್ರಕೃಷಿಯಲ್ಲೇನೋ ಪರಿಣತರೆಂದು \hbox{ಭೌತಿಕಕೃಷಿಯಲ್ಲೇನೂ} ಹಿಂದೆ ಬಿದ್ದವರಲ್ಲ, ಇಷ್ಟೆಲ್ಲ ಇದ್ದರೂ ಅವರಾಗಿ ಅವರು ಯಾವುದರ ಹಿಂದೂ ಬಿದ್ದವರಲ್ಲ~!! "ಪ್ರತಿಭಾಸಂಯಮಿ"ಯಾಗಿ ಬಾಳಿದವರು. ಎಲ್ಲವನ್ನೂ ದೇಶ, ಕಾಲ, ಅಗತ್ಯ ಮತ್ತು ಔಚಿತ್ಯಕ್ಕೆ ಅನುಗುಣವಾಗಿ ನಿರ್ವಾಹ\-ಮಾಡಿದವರು. ಖ್ಯಾತಿ, ಲಾಭ, \hbox{ಪೂಜೆಗಳು} ಇವರ ಹತ್ತಿರ ಸುಳಿಯಲೂ ಇಲ್ಲ.  ಇದು ಅವರ ನಿಜವಾದ ಆದರ್ಶ. ಅವರ ಜೀವನವನ್ನು ಆಧರಿಸಿ ಒಂದು ಸಿನೆ\enginline{(cine)}ಯನ್ನೇ ತಯಾರಿಸ\- ಬಹುದು. ಹೀಗಿರುವಾಗ ಅವರ ಬಹುಮುಖ ವ್ಯಕ್ತಿತ್ವಕ್ಕೆ ಹೊಂದುವ, ಅವರ ಕುರಿತಾದ ಈ ಗ್ರಂಥಕ್ಕೆ ಒಂದು ಹೆಸರು, ಮುಖಚಿತ್ರನಿರ್ಮಾಣ ಒಂದು ಸವಾಲೇ ಆಗಿತ್ತು.


ಆದರೆ ಈ ಮೇಲಿನ ಅಂಶಗಳೆಲ್ಲ ಏನೇ ಇದ್ದರೂ ಗಂಗಾಧರ ಭಟ್ಟರೆಂದರೆ ಎಲ್ಲಕ್ಕಿಂತ ಮೊದಲು ನಮ್ಮ ಕಣ್ಣಿಗೆ ಕಟ್ಟುವುದು ಅವರ ಪಾಠ. ಅದು ಅವರ ಜೀವನದಲ್ಲಿ ಹಾಸು\enginline{-}ಹೊಕ್ಕಾಗಿದೆ. ಅವರ ಜೀವನವೇ ಪಾಠ, ಪಾಠವೇ ಅವರ \hbox{ಜೀವಾಳ.} ಪಾಠ ಎಂದಾಗ ಅದು ಜ್ಞಾನ, ವಿಜ್ಞಾನಮೂಲವಾದುದಷ್ಟೆ ! ಆ ಜ್ಞಾನ\enginline{-}ವಿಜ್ಞಾನಗಳಾದರೋ ಪರಂಪರಾ\-ಪ್ರಾಪ್ತವಾದುವು \enginline{-} ಪರಂಪರೆ ತಪೋಮೂಲವಾದುದು \enginline{-} ತಪಸ್ಸು ಭಗವನ್ಮೂಲ\-ವಾದುದು, ಇದು ನಮ್ಮ ಭಾರತೀಯ ವಿದ್ಯಾಸಂಪ್ರದಾಯದ ಸರಣಿ. ಆದ್ದರಿಂದ ಯಾರಲ್ಲೇ ಆಗಲಿ ಅವರಲ್ಲಿರುವ ವಿದ್ಯೆ, ಕಲೆ, ಪ್ರತಿಭೆಗಳಿಗೆ ಈ ಸರಣಿಯೇ ಮೂಲ ಸ್ರೋತಸ್ಸು. ಹಾಗಾಗಿ ಈ ಅಭಿವಂದನ ಗ್ರಂಥದ ಮುಖಚಿತ್ರ ಈ ಚಿಂತನೆಯನ್ನೇ ಅವಲಂಬಿಸುವುದು ಉಚಿತ, ಇದರಲ್ಲೇ ಮೇಲೆ ಹೇಳಿದ ಅಂಶಗಳೆಲ್ಲವೂ ಗತಾರ್ಥವಾಗುವುದು ಎಂದು ಭಾವಿಸಿ ಆ ಭಾವವನ್ನೇ ಚಿತ್ರಿಸುವ ಯತ್ನ ಇಲ್ಲಿ ನಡೆದಿದೆ. ಮುಖ\-ಪುಟದಲ್ಲೂ ಶ್ರೀಯುತರ ವ್ಯಕ್ತಿತ್ವದ ಸ್ಥಾನವನ್ನು ಗುರುತಿಸುವ ಕಿರು ಪ್ರಯತ್ನ ಇದಾಗಿದೆ.
ಈ ಮೇಲಿನ ಹಿನ್ನೆಲೆಯಲ್ಲಿ ಯೋಚಿಸಿದಾಗ ಉಂಟಾದ ಸ್ಫುರಣೆಯಿಂದ ಚಿತ್ರವೊಂದು ಮೂಡಿಬಂತು. ಈ ಬೌದ್ಧಿಕ ಚಿತ್ರವನ್ನು ಭೌತಿಕವಾಗಿ ಚಿತ್ರಿಸಲು ಬಹುಮಟ್ಟಿಗೆ  ಪ್ರಯತ್ನಿಸಿದ್ದೇವೆ. ಅದರ ಆಶಯ, ಸ್ವರೂಪ ಹೀಗಿದೆ \enginline{-}


ಚಿತ್ರದ ಉತ್ತುಂಗದಲ್ಲಿ ಬೆಳಕಿನರೂಪದಲ್ಲಿ ಮಾತ್ರ ಇರುವ ಎಲ್ಲಕ್ಕೂ ಮೂಲವಾದ ಪರಬ್ರಹ್ಮದ ಕಲ್ಪನೆಯಿದೆ. ಅದರ ಕೆಳಗೆ ಸೂಕ್ಷ್ಮದೃಷ್ಟಿಗೆ ಮಾತ್ರ ಗೋಚರವಾಗುವ ಯೋಗಶಾಯಿಯಾದ ವಿಷ್ಣುವಿದ್ದಾನೆ. ಆ ವಿಷ್ಣುವಿನ ಪಾದದ ಬಳಿ ಚತುರ್ಮುಖ ಬ್ರಹ್ಮ(ಬ್ರಹ್ಮಾ)ನಿದ್ದಾನೆ. ಅವನು ತನ್ನ ಕಮಂಡಲು ತೀರ್ಥದಿಂದ ವಿಷ್ಣುಪಾದವನ್ನು\break ಅಭಿಷೇಚಿಸುತ್ತಿದ್ದಾನೆ. ಆ ಅಭಿಷೇಕತೀರ್ಥ ಗೌರೀಶಂಕರ ಶಿಖರವಾಸಿಯಾದ ಶಿವನ\break ಶಿರಸ್ಸನ್ನು ಅಲಂಕರಿಸಿದೆ. ಅಲ್ಲಿಂದ ಅದು ಸಪ್ತ ಸ್ರೋತಸ್ಸಾಗಿ ಇಳೆಗೆ ಇಳಿದು ನದಿಯಾಗಿ ಹರಿದು ಭೂಮಿಯನ್ನೆಲ್ಲ ಪಾವನಗೊಳಿಸುತ್ತಾ ಸರಿತ್ಪತಿಯಾದ ಸಾಗರವನ್ನು ಸೇರುತ್ತದೆ. ಆ ನದಿಯನ್ನು ನಾವು ಗಂಗಾನದೀ ಎನ್ನುತ್ತೇವೆ. ಇದನ್ನೇ ಜ್ಞಾನಿಗಳು ಜ್ಞಾನದ\break ಪ್ರವಾಹವೆಂದು ತಿಳಿದು ಜ್ಞಾನಗಂಗಾ ಎನ್ನುತ್ತಾರೆ. ಈ ಭೌತಿಕ ಪ್ರವಾಹ ಜ್ಞಾನಾನುಭವಕ್ಕೆ\break ತೊಡಕಾಗಿರುವ ಪಾಪಗಳನ್ನು ಪರಿಹರಿಸಿಕೊಂಡು ಜ್ಞಾನಾನುಭವವನ್ನು ಹೊಂದಲು ಉತ್ತಮ ಸಾಧನವಾಗಿದೆ \enginline{-} ಎಂಬುದು ಆಶಯ.  ಜ್ಞಾನಪ್ರವಾಹ ರೂಪದ ಗಂಗೆ ಮಾನವ ಶರೀರದಲ್ಲಿ ಶಕ್ತಿಯ ರೂಪದಲ್ಲಿ ಪ್ರವಹಿಸುವುದನ್ನು ಯೋಗಿಗಳು ಗುರುತಿಸುತ್ತಾರೆ. ಯೋಗಿಗಳು ಅದರಲ್ಲಿ ಅವಗಾಹನ ಮಾಡುತ್ತಾ ಜ್ಞಾನಾನುಭವದೆಡೆಗೆ ಸಾಗುತ್ತಾರೆ.\break ಆ ಅವಗಾಹನದ ಪ್ರತೀಕವಾಗಿ ಭೌತಿಕವಾದ ಗಂಗೆಯ ಅವಗಾಹನ ಸಂಪ್ರದಾಯದಲ್ಲಿ ಬಂದಿದೆ. ಇಂತಹ ಜ್ಞಾನನದಿಯ ಬದಿಯಲ್ಲಿ ಜ್ಞಾನಾನುಭವಕ್ಕಾಗಿ, ಅದಕ್ಕೆ ಪೋಷಕವಾದ ಮತ್ತು ಮಾನವ ಶರೀರಕ್ಕೂ ಸಂಕೇತವಾದ ಅಶ್ವತ್ಥ ವೃಕ್ಷದ ಮೂಲದಲ್ಲಿ ತಪಸ್ಸು\-ಮಾಡುತ್ತಿರುವ ಒಬ್ಬ ಯೋಗಿಯಿದ್ದಾನೆ. ಇನ್ನೊಂದು ಬದಿಯಲ್ಲಿ ಅದೇ ಯೋಗಿ ತಾನು\break ತಪಸ್ಸಿನಿಂದ ಪಡೆದ ಜ್ಞಾನ, ವಿಜ್ಞಾನಗಳನ್ನು ಶಿಷ್ಯರಿಗೆ, ಅದೇ ವೃಕ್ಷದ ಪದತಲದಲ್ಲಿ ಕುಳಿತು ಬೋಧಿಸುತ್ತಿದ್ದಾನೆ. ಇಂತಹ ಪ್ರದೇಶ ಪ್ರಶಸ್ತವಾದ \hbox{ಧರ್ಮಭೂಮಿಯೆಂದು} ನಿರ್ಭಯವಾಗಿ ನಿಂತಿರುವ ಕೃಷ್ಣಮೃಗಗಳು ಸಾರುತ್ತವೆ. ಇವೆಲ್ಲ ಪರೋಕ್ಷವಾದ\break ವಿಷಯಗಳಾಗಿವೆ. ಅದರಿಂದ ಈ ವಿಷಯಗಳ ಸಾರವನ್ನು ಸಾರುವ ಚಿತ್ರಗಳನ್ನು ನೆರಳು\break ಚಿತ್ರವಾಗಿ ಚಿತ್ರಿಸಿದೆ. (ಬ್ರಹ್ಮ\enginline{-}ವಿಷ್ಣು\enginline{-}ಮಹೇಶ್ವರರ ಚಿತ್ರ ಮಾತ್ರ ಗುಪ್ತವಾಗಿದೆ. ಕಾರಣ ದೇವತೆಗಳು ಪರೋಕ್ಷಪ್ರಿಯರು, ಸ್ಥೂಲದೃಷ್ಟಿಗೆ ಗೋಚರಿಸುವವರಲ್ಲ. ಹಾಗಾಗಿ ಅದನ್ನು ಸಂಕೇತಿಸಲು ಸೂಕ್ಷ್ಮವಾಗಿಯೇ ಚಿತ್ರಿಸಲಾಗಿದೆ. ಆದರೆ ಎಲ್ಲದಕ್ಕೂ ಮೇಲಿರುವ ಪರಬ್ರಹ್ಮ ಪರಾದೃಷ್ಟಿಗೆ ಮಾತ್ರ ನಿಲುಕುವಂತಹುದು. ಅದು ದೇವತೆಗಳಿಗಿಂತ ಇನ್ನೂ ಸೂಕ್ಷ್ಮವಾದುದು, ಹಾಗಾಗಿ ಮತ್ತೂ ಸೂಕ್ಷ್ಮವಾಗಿ ಚಿತ್ರಿಸಬೇಕು. ಆದರೆ ಅದರ ಸ್ವರೂಪವೇ ಸರ್ವವ್ಯಾಪಕವಾದ ಬೆಳಕಾಗಿದ್ದುದರಿಂದ  ಆ ಬೆಳಕನ್ನು ಬೆಳಗುವಂತೆಯೇ\break ಚಿತ್ರಿಸಲಾಯಿತು.)\break
ಗುರು\enginline{-}ಶಿಷ್ಯ ಸಂಬಂಧ, ಅಧ್ಯಯನ\enginline{-}ಅಧ್ಯಾಪನಗಳು ಆಯಾ ದೇಶ\enginline{-}ಕಾಲಗಳಲ್ಲಿ\break ಯಾರಿಂದಲೇ ಸಂಪನ್ನವಾಗುತ್ತಿವೆಯೆಂದರೂ ಅವೆಲ್ಲವೂ ಭಾರತೀಯ ವಿದ್ಯಾ\break ಸಂಪ್ರದಾಯದ ನೆರಳಿನಲ್ಲಿಯೇ ಸಾಗುತ್ತವೆ ಎಂಬುದನ್ನು ಪ್ರತ್ಯೇಕವಾಗಿ ಹೇಳಬೇಕಿಲ್ಲ. (ಇದನ್ನು ಹೊರದೃಷ್ಟಿಯಿಂದ ಗುರುತಿಸುವುದು ಸಾಧ್ಯವಾಗದಿದ್ದರೂ ವಿದ್ಯೆಯ\break ಅಂತಃಸ್ರೋತಸ್ಸು ಹರಿಯದೆ ವಿದ್ಯೋಪಾಸನೆ ಸಾಧ್ಯವಿಲ್ಲ, ಕೆಲವೊಮ್ಮೆ ಆ ಸ್ರೋತಸ್ಸು\break ಪ್ರಭೂತವಾಗಿ ವ್ಯಕ್ತವಾಗಬಹುದು, ಇನ್ನೊಮ್ಮೆ ಸುಪ್ತವಾಗಿರಬಹುದು, ಇರುವ ವಿಷಯ ಮಾತ್ರ ಅದೇ !) ಪ್ರಕೃತ ಗಂಗಾಧರ ಭಟ್ಟರು ಅಧ್ಯಯನ\enginline{-}ಅಧ್ಯಾಪನದಲ್ಲೇ ತಮ್ಮ ಜೀವನದ ಸಾರವನ್ನು ಭಾವಿಸಿ ಬಾಳಿದವರಾದ್ದರಿಂದ ಒಮ್ಮೆ ಆ ವಿದ್ಯಾಸರಣೀ ಒಂದು ಮಟ್ಟದಲ್ಲಿ ಜಾಗೃತವಾಗಿದೆಯೆಂದರೆ ಅತಿಶಯವಲ್ಲ. ಇದು ನಮಗೆ ಪ್ರತ್ಯಕ್ಷ ವಿಷಯವೇ ಆಗಿದೆ. ಅಸಂಖ್ಯ ವಿದ್ಯಾರ್ಥಿಗಳಿಗೆ ಅವರು ಪಾಠ \enginline{-} ಪ್ರವಚನ ಮಾಡಿದುದು ಮಾತ್ರವಲ್ಲದೇ ಅಶನ, ವಸನ, ಶಾಲೆ, ಕಾಲೇಜ್, ಹಾಸ್ಟೇಲ್ ಇತ್ಯಾದಿ ಇತ್ಯಾದಿ ವ್ಯವಸ್ಥೆ\break ಗಳನ್ನು ಮಾಡಿಕೊಟ್ಟಿದ್ದಾರೋ ಲೆಕ್ಖವಿಲ್ಲ. ಏನಿಲ್ಲವೆಂದರೂ ಈವರೆಗಿನ ಅವರ ಬದುಕಿನಲ್ಲಿ ಅವರು ಒಂದು ಬೃಹತ್ ಗುರುಕುಲದ ಕೆಲಸವನ್ನು ನಿರ್ವಹಿಸಿದ್ದಾರೆ ಎಂದರೆ ಅನೇಕರು ನಂಬಲಾರರು. ಆದರೆ ಅದು ವಾಸ್ತವ. ಹಾಗಾಗಿ ಆ ಜ್ಞಾನಗಂಗೆಯ ಪ್ರವಾಹವೆಂಬ ಪರಂಪರೆಯಲ್ಲೇ ಶ್ರೀಯುತರ ಬದುಕಿಗೆ ಒಂದು ದೃಢವಾದ ಸ್ಥಾನವಿದೆ. ಇವುಗಳನ್ನೆಲ್ಲ ಲಕ್ಷಿಸಿ ಅದಕ್ಕೆ ಸಾಂಕೇತಿಕವಾಗಿ ಗಂಗಾನದಿಯ ಬದಿಯಲ್ಲಿ ಗಂಗಾಧರ ಭಟ್ಟರು ಅಧ್ಯಾಪನದಲ್ಲಿ\break ನಿರತರಾಗಿರುವ ಚಿತ್ರವನ್ನು ಅಳವಡಿಸಿದೆ. ಈ ಮೇಲಿನ ಹಿನ್ನೆಲೆಯಲ್ಲಿ ಚಿಂತಿಸಿದಾಗ ಮೂಲತಃ ಜ್ಞಾನದಿಂದಲೇ ಇವೆಲ್ಲ ಅರಳಿದುದು, ಅದೇ ಎಲ್ಲಕ್ಕೂ ಮೂಲ ಎಂಬುದು ಸ್ಪಷ್ಟ. ಈ ಗ್ರಂಥದಲ್ಲಿ ಸಾಕ್ಷಾತ್ ಜ್ಞಾನಕ್ಕೆ ಸಾಧನಗಳಾಗಿ ಜ್ಞಾನದ \hbox{ಪ್ರತಿನಿಧಿಗಳಂತೆಯೇ} ಇರುವ ಶಾಸ್ತ್ರಗಳು ಪ್ರಧಾನ ಸ್ಥಾನ ಪಡೆದುದರಿಂದ ಗ್ರಂಥಕ್ಕೆ ಜ್ಞಾನಗಂಗಾಧರ ಎಂದು ನಾಮಕರಣ ಮಾಡಲಾಯಿತು. ಇಷ್ಟೇ ಅಲ್ಲದೆ, ಗಂಗಾಧರ ಎಂಬುದು ಮೂಲತಃ \hbox{ವಿದ್ಯಾಧಿದೇವನಾದ} ಶಿವನ ಹೆಸರು. ಅವನು ಜ್ಞಾನಕ್ಕೆ ಇನ್ನೊಂದು ರೂಪದ \hbox{ಪ್ರತಿನಿಧಿ} ಯಾದ ಗಂಗೆಯನ್ನು ಧರಿಸಿ (ಜ್ಞಾನ)ಗಂಗಾಧರನಾಗಿದ್ದಾನೆ. \hbox{ಅಯಾಚಿತವಾಗಿ} ಶ್ರೀಯುತರಿಗೂ ಅದೇ ಹೆಸರು ಜನ್ಮಾರಭ್ಯ ಪ್ರಾಪ್ತವಾಗಿದೆ. ಹಾಗಾಗಿ ಒಂದು ದೃಷ್ಟಿಯಿಂದ ಜ್ಞಾನ ಸಂಬಂಧದ ಸೂಚನೆ ಅವರ ನಾಮಧೇಯದಲ್ಲಡಗಿದೆ. \hbox{ಅದಕ್ಕನುಗುಣವಾಗಿ} ಅವರು ಕಲೆ, ವ್ಯವಹಾರ, ರಾಜಕೀಯವಾದ ಅಸಾಧಾರಣ ಸಾಮರ್ಥ್ಯಗಳಿದ್ದೂ  ಆ \hbox{ವ್ಯಕ್ತಿಯಾಗದೇ}, ಜ್ಞಾನಗಂಗೆಯ ಧರಿಸಿ ಪಾಠ\enginline{-}ಪ್ರವಚನಗಳನ್ನು ಪ್ರವಹಿಸಿ ಸಾರ್ಥಕ \hbox{ನಾಮರಾಗಿದ್ದಾರೆ.} ಈ ಹಿನ್ನೆಲೆಯಿಂದ ಗ್ರಂಥಕ್ಕೆ ಜ್ಞಾನಗಂಗಾಧರ  ಎಂಬ ಹೆಸರೇ ಉಚಿತವಾಗಿ ಕಂಡಿತು. ಹಾಗಂತ, ನ್ಯಾಯ\- ಗಂಗಾಧರ ಎಂದಿಡಬೇಕೆಂದೂ  ಸಲಹೆ ಸಹ ಬಂದಿತ್ತು. ಅದು ಅವರ ಬದುಕಿಗೂ \hbox{ಅನ್ವಯಿಸುವಂಥದ್ದೇ} ಆಗುತ್ತಿತ್ತು. ನ್ಯಾಯಕ್ಕಾಗಿ ಅವರು ನಡೆಸಿದ ಹೋರಾಟ ಅಷ್ಟಿಷ್ಟಲ್ಲ. ಹಾಗಾಗಿ \hbox{ವಿದ್ಯಾರ್ಥಿದೆಸೆಯಿಂದ} ಅವರ ಬದುಕನ್ನು ಕಂಡವರಿಗೆ ಆ ಚಿತ್ರಣವೇ ಕಣ್ಣೆದುರಿಗೆ ನಿಲ್ಲುವುದರಲ್ಲಿ ಸಂಶಯವಿಲ್ಲ. ಆದರೆ ಅದನ್ನೇ ಗಣಿಸಿ ಗ್ರಂಥಕ್ಕೆ ಆ ಹೆಸರಿಟ್ಟಾಗ ಸಾಮಾನ್ಯ ಜನರ ಮನಸ್ಸು ಅವರನ್ನು ನ್ಯಾಯಶಾಸ್ತ್ರದ ಸೀಮೆಗಷ್ಟೇ ಕಟ್ಟಿಕೊಳ್ಳಬಹುದು ಎನ್ನಿಸಿತು. ವಾಸ್ತವವಾಗಿ ಓದುವ ಮತ್ತು ಪಾಠದ ಅವಧಿಯಲ್ಲಿ ಅವರು ನ್ಯಾಯಶಾಸ್ತ್ರವನ್ನು  ಓದಿ, ಪಾಠವನ್ನು \hbox{ಮಾಡಿದ್ದರೂ} ಸಮ\enginline{-}ಸಮವಾಗಿ ಬೇರೆಯ ವಿಷಯಗಳನ್ನೂ ಓದಿದರು, ಬಿ.ಕಾಮ್ ಪದವಿಯನ್ನೂ ಪಡೆದರು. ಅವರ ಪಾಠವೂ ಸಹ  ನ್ಯಾಯಶಾಸ್ತ್ರಕ್ಕೇ ಸೀಮಿತವಾಗಿರಲಿಲ್ಲ. ಕಾವ್ಯ, ನಾಟಕ, ಆಯುರ್ವೇದ, ಅರ್ಥಶಾಸ್ತ್ರಗಳೆಲ್ಲ ಅವರ ಪಾಠದ ವಿಷಯಗಳಾದವು. ಇಂಗ್ಲಿಷ್, ಕ್ವಚಿತ್ತಾಗಿ ಹಿಂದಿಭಾಷೆ\-ಗಳನ್ನೂ ಪಾಠಮಾಡಿರುವುದುಂಟು. ಯೋಗಾಯೋಗ\-ವೆಂಬಂತೆ ಯೋಗಶಾಸ್ತ್ರದ ಯೋಗವೂ ಅವರಿಗೊದಗಿತು. ಆ ಸಂಬಂಧವಾಗಿ ಅವರು ವಿದೇಶಕ್ಕೂ ಹೋಗಿ\-ಬಂದರು. ಹೀಗಿರುವಾಗ ನ್ಯಾಯಶಾಸ್ತ್ರಕ್ಕಷ್ಟೇ ಅವರನ್ನು ಸೀಮಿತಗೊಳಿಸಿದರೆ ಅವರಿಗೆ ಅನ್ಯಾಯ ಮಾಡಿದಂತೆಯೇ ಅನ್ನಿಸಿತು. ಹಾಗಾಗಿ ನ್ಯಾಯ ಶಬ್ದ ಬಿಟ್ಟು ಜ್ಞಾನ ಶಬ್ದವನ್ನು ಬಳಸುವುದೇ ನ್ಯಾಯವೆನಿಸಿತು. ಗ್ರಂಥ ಜ್ಞಾನಗಂಗಾಧರವಾಯಿತು.

ಇವೆಲ್ಲದರಿಂದ ಗಂಗಾಧರಭಟ್ಟರನ್ನು ಅತಿಶಯಿಸಿ ಹೊಗಳಿ ಅಟ್ಟಕ್ಕೇರಿಸುವುದು ನಮ್ಮ ಉದ್ದೇಶವಲ್ಲ. ಅವರಂತೂ ಅದನ್ನು ಕಿಂಚಿತ್ತೂ ಸಹಿಸುವವರಲ್ಲ. ನಮಗೂ ಅತ್ಯುಕ್ತಿ ಅಭ್ಯಾಸವಿಲ್ಲ. ಅವರಲ್ಲಿಯೂ ಮನುಷ್ಯ ಸಹಜವಾದ ದೊಷಗಳೇನೂ ಇಲ್ಲ ಎಂಬುದು ನಮ್ಮ ಅಭಿಪ್ರಾಯವಲ್ಲ. ಗ್ರಂಥದಲ್ಲಿ ಪ್ರಾಸಂಗಿಕವಾಗಿ ಅಂತಹ ವಿಷಯಗಳನ್ನು ಉಲ್ಲೇಖಿಸಿರುವ ಲೇಖನಗಳೂ ಇವೆ. ಮಗುವಿಗೆ ದೃಷ್ಟಿ ಆಗದಿರಲೆಂದು ದೃಷ್ಟಿಬೊಟ್ಟು ಇಡುವುದುಂಟಷ್ಟೆ! ಅಂತೆಯೇ ಈ ಲೇಖನಗಳು ಎಂದು ಭಾವಿಸಿ ಅವುಗಳನ್ನು ಯಥಾವತ್ ಪ್ರಕಟಿಸಿದ್ದೇವೆ. ಚಂದ್ರನಲ್ಲಿ ಕಲೆ ಇದ್ದರೂ ಅದು ಹೇಗೆ ಅವನ ಕಾಂತಿಯನ್ನು ಕುಂದಿಸುವ ಕಲಂಕವಲ್ಲವೋ ಹಾಗೆಯೇ ಅವು ಅವರ ವ್ಯಕ್ತಿತ್ವಕ್ಕೆ ಕಲಂಕವೆಂದು ಅನ್ನಿಸುವುದಿಲ್ಲ.\break ಅಷ್ಟಕ್ಕೂ ಸಕಲಕಲಾಪರಿಪೂರ್ಣತೆ ಭಗವಂತನಿಗೆ ಮಾತ್ರ ಎಂಬುದು ಸರ್ವವಿದಿತ !

 
ಇಷ್ಟೆಲ್ಲ ಹಿನ್ನೆಲೆಯಿಂದ ಅವರಿಂದ ಉಪಕೃತರಾದ ನಾವು \enginline{-} ವಿದ್ಯಾರ್ಥಿಗಳು, ಅವರನ್ನು ಗೌರವಿಸುವ ಕಾರ್ಯ ಕೈಗೊಂಡೆವು. ಗ್ರಂಥಪ್ರಕಟಿಸುವ ಸಂಕಲ್ಪ ಮಾಡಿದೆವು. ಈ ನಮ್ಮ ಕಾರ್ಯದಿಂದ ಸಮಾಜಕ್ಕೂ ಒಂದು  ಉತ್ತಮ ಸಂದೇಶ ಸಿಕ್ಕಿರಲೂ ಸಾಕು. ಅಂದರೆ, ಸಹ ನಾವವತು । ಸಹ ನೌ ಭುನಕ್ತು । ಸಹ ವೀರ್ಯಂ ಕರವಾವಹೈ । ತೇಜಸ್ವಿ \break ನಾವಧೀತಮಸ್ತು । ಮಾ ವಿದ್ವಿಷಾವಹೈ । ಎಂಬ ಋಷಿನುಡಿಯಂತೆ ವಿದ್ಯಾರ್ಥಿ ಮತ್ತು ಅಧ್ಯಾಪಕರ ಉತ್ತಮ ಬಂಧಕ್ಕೂ, ಅಧ್ಯಾಪಕರಲ್ಲಿ ವಿದ್ಯಾರ್ಥಿಗಳಿಗಿರಬೇಕಾದ ಗೌರವ, ಕೃತಜ್ಞತೆಗಳಿಗೂ ಇದೊಂದು ಮಾದರಿಯೆಂದು ಯಾರಿಗೇ ಅನ್ನಿಸಿದರೂ ನಮ್ಮ ಕಾರ್ಯಕ್ರಮದ ಅವಾಂತರ ಫಲವೂ ನಮಗೆ ದೊರಕಿತೆಂದು ಸಂತೋಷ\break ಪಡುತ್ತೇವೆ.


ಇಂತಹ ನಮ್ಮ ಕಾರ್ಯಕ್ರಮಕ್ಕೆ ರಾಜಮಾತೆ ಡಾ~॥ ಪ್ರಮೋದಾದೇವಿಯವರು\break ಆಗಮಿಸಿರುವುದು ಕಾರ್ಯಕ್ರಮದ ಗೌರವವನ್ನು ಹೆಚ್ಚಿಸಿದ್ದು ನಿಜ, ಅಂತೆಯೇ ಪ್ರಕೃತ ಗ್ರಂಥಕ್ಕೆ ಶುಭಸಂದೇಶವನ್ನು ದಯಪಾಲಿಸಿ ಗ್ರಂಥಗೌರವದ ವೃದ್ಧಿಗೂ ಅವರು ಕಾರಣರಾಗಿದ್ದಾರೆ. ಅವರಿಗೆ ಗಂಗಾಧರ ಭಟ್ಟರ ನಡೆಯಿಂದ ಅವರ ಬಗೆಗೂ ಪಾಠಶಾಲೆಯ ಬಗೆಗೂ  ಅಭಿಮಾನ ಮೂಡಿರುವುದರಲ್ಲಿ ಸಂಶಯವಿಲ್ಲ. ಅದರ ಫಲವಾಗಿ ಅವರು ಕಾರ್ಯ\-ಕ್ರಮಕ್ಕೂ ಆಗಮಿಸಿ ಸಂದೇಶವನ್ನೂ ನೀಡಿದ್ದಾರೆ. ಇದರಿಂದ ನಮಗೆ ಮೈಸೂರು ಸಂಸ್ಥಾನವೇ ನಮ್ಮೊಂದಿಗಿರುವ ಭಾವ ಮೂಡಿದೆ. ಅವರಿಗೆ ನಾವು ಎಷ್ಟು  ಕೃತಜ್ಞ\-ರಾಗಿದ್ದರೂ ಕಡಿಮೆಯೇ. ವಂದನಾತ್ ನೋಪಚಾರಕಮ್ ಎಂಬಂತೆ ಅವರಿಗೆ ಅನಂತ ವಂದನೆಗಳನ್ನು  ಸಲ್ಲಿಸುತ್ತೇವೆ.

  
ಅಭಿವಂದನ ಸಮಿತಿಯ ಗೌರವಾಧ್ಯಕ್ಷರಾದ ಡಾ~॥ ನಿರಂಜನ \hbox{ವಾನಳ್ಳಿಯವರು} ನಮಗೆ ಆರಂಭದಿಂದಲೂ ಅತ್ಯಂತ ಉತ್ಸಾಹವನ್ನೂ, ಬೆಂಬಲವನ್ನೂ ನೀಡುತ್ತಾ ಗೌರವಾಧ್ಯಕ್ಷರ ನುಡಿಯನ್ನು ನೀಡಿ, ಒಡನಾಡಿ ವಿಭಾಗಕ್ಕೂ ಲೇಖನವನ್ನು  ಬರೆದಿದ್ದಾರೆ. ಅವರಿಗೆ ನಮ್ಮ ಅನಂತ ವಂದನೆಗಳು. ಇದೇ ರೀತಿಯಲ್ಲಿ ಸಮಿತಿಯ ಅಧ್ಯಕ್ಷರಾದ \hbox{ವಿ~॥ ಕೆ.ಎಲ್. ರಾಘವರವರು} ಶಾಸ್ತ್ರವಿಭಾಗಕ್ಕೆ ಸಂಶೋಧನ ಲೇಖನವನ್ನೂ, ಒಳನುಡಿ \hbox{ವಿಭಾಗಕ್ಕೆ} ಇನ್ನೊಂದು ಲೇಖನವನ್ನೂ, ಅಧ್ಯಕ್ಷೀಯವನ್ನೂ ನೀಡಿದ್ದಾರೆ. ಅವರಿಗೆ ಅನಂತ ವಂದನೆಗಳು.


ಇನ್ನು, ಈ ಲೇಖನಗಳ ಡಿಟಿಪಿ ಕಾರ್ಯ ಸಾಮಾನ್ಯ ವಿಷಯವಾಗಿರಲಿಲ್ಲ, ಅವೆಲ್ಲ\-ವನ್ನೂ ವಿದ್ಯಾರ್ಥಿಗಳೇ ಟಂಕಿಸಿರುವುದಲ್ಲದೇ, ತದನಂತರದ ಸ್ಖಾಲಿತ್ಯ ಪರಿಶೀಲನೆ\-ಯಲ್ಲೂ ಸಾಕಷ್ಟು ಸಹಕಾರ ನೀಡಿದ್ದಾರೆ. ಈ ಎಲ್ಲ ವಿದ್ಯಾರ್ಥಿಗಳೂ ಪರಸ್ಥಳದಲ್ಲಿ ವಾಸಿಸುವವರು. ಹಾಗಾಗಿ ಅನೇಕ ಕಾರ್ಯಗಳನ್ನು ಅಂತರ್ಜಾಲದಲ್ಲಿಯೇ ನಡೆಸ\-ಬೇಕಾಯಿತು. ಅದನ್ನು ವಿದ್ಯಾರ್ಥಿಗಳು ನಿರ್ವಹಿಸಿಕೊಟ್ಟಿರುತ್ತಾರೆ. ಅವರಲ್ಲೂ ಕೆಲವರ \hbox{ಸಹಕಾರವನ್ನು} ಮರೆಯುವಂತೆಯೇ ಇಲ್ಲ. ಇಂತಹ ನಮ್ಮ \hbox{ವಿದ್ಯಾರ್ಥಿಬಂಧುಗಳಿಗೆ} ಅನಂತ ಶುಭಕಾಮನೆಗಳು. 
ವಿದ್ಯಾರ್ಥಿಗಳು ಅವರವರಿಗೆ ಸೌಲಭ್ಯವಿರುವ ತಾಂತ್ರಿಕ\break ವ್ಯವಸ್ಥೆಯಲ್ಲಿ ಲೇಖನಗಳನ್ನು ಡಿಟಿಪಿ ಮಾಡಿದ್ದಾರೆ. ಕೆಲವು ಲೇಖಕರು ತಮ್ಮ ಲೇಖನ\-ವನ್ನು ತಾವೇ ಡಿಟಿಪಿ ಮಾಡಿಕೊಟ್ಟಿರುವುದೂ ಇದೆ. ಆದರೆ ಡಿಟಿಪಿಯಾದ ಲೇಖನಗಳು, ವಿವಿಧ ಮಾದರಿಯ ತಂತ್ರಜ್ಞಾನದ ಬಳಕೆಯ ಕಾರಣದಿಂದಲೂ, ಭಿನ್ನ ಭಿನ್ನ ಅಕ್ಷರಗಳ ಅಳತೆ ಮತ್ತು ವಿನ್ಯಾಸಗಳ ಕಾರಣದಿಂದಲೂ ಗ್ರಂಥಕ್ಕೆ ಬೇಕಾದಂತೆ ಒಂದೇ ಸ್ವರೂಪಕ್ಕೆ ಅಳವಡಿಸುವ ಕಾರ್ಯ ಮಾತ್ರ ನಮ್ಮೆಲ್ಲ ನಿರೀಕ್ಷೆಯನ್ನು ಮೀರಿ ಶ್ರಮ ಮತ್ತು ಕಾಲವನ್ನು ಅಪೇಕ್ಷಿಸಿತು. ಲೇಖನದ ಎಷ್ಟೋ ಅಂಶಗಳನ್ನು ಪುನಃ \hbox{ಟಂಕಿಸುವಂತಾದುದೂ} ಉಂಟು. ಇದರೊಂದಿಗೆ ಇನ್ನೊಂದು ವಿಚಾರವೆಂದರೆ - ಕಂಪ್ಯೂಟರ್ ತಂತ್ರಾಂಶದಲ್ಲಿ ಲೇಖನಗಳ ಸಾಲುಗಳು ಅದರದೇ ಆದ ರೀತಿಯಲ್ಲಿ ಯೋಜಿತಗೊಳ್ಳುತ್ತವೆ. ಅಲ್ಲಿ ಶಬ್ದಗಳನ್ನು  \hbox{ವಿಂಗಡಿಸಿಕೊಳ್ಳುವಾಗ} ಒಂದೇ ಶಬ್ದವನ್ನು\enginline{-} ಪದವನ್ನು  ಅದು ವಿಭಾಗಿಸಿಬಿಡುತ್ತದೆ. ಇದು ಓದುಗರಿಗೆ ಇಷ್ಟವಾಗುವುದಿಲ್ಲ.  ಇದನ್ನು ಸರಿಪಡಿಸಲು ಹೋದರೆ ಶಬ್ದ ಶಬ್ದಗಳ ಮಧ್ಯೇ ಅಕ್ಷರ ಅಕ್ಷರಗಳ ಮಧ್ಯೇ ಸಾಕಷ್ಟು ಜಾಗವುಳಿದು,  ಪುಸ್ತಕದ ಇಡೀ ಪುಟವನ್ನು ನೋಡಿದರೆ ಹಾವು ಹರಿಯುವ ರೀತಿಯಲ್ಲಿ ಅಥವಾ ನದೀ ಹರಿಯುವಂತೆ  \hbox{ಅಕ್ಷರಗಳನ್ನು} ಮತ್ತು ಶಬ್ದಗಳನ್ನು ಜೋಡಿಸಿದಂತೆ ಆಗಿಬಿಡುತ್ತದೆ. ಅದನ್ನು  ಆ ಪರಿಭಾಷೆಯಲ್ಲಿ \hbox{ರಿವರ್} ಎಫೆಕ್ಟ್ ಎನ್ನುತ್ತಾರೆ. ಅಂದರೆ ಅಕ್ಷರ ವಿನ್ಯಾಸದ ದೋಷ, ಒಂದು ರೀತಿಯ \hbox{ಅವಲಕ್ಷಣ} ಎಂದು ಅರ್ಥ. ಅದರಲ್ಲೂ  ಕನ್ನಡ ಲಿಪಿಗೆ ಈ ಸಮಸ್ಯೆ  ಅತ್ಯಧಿಕ. ಇದಕ್ಕೆ ಕನ್ನಡದ \hbox{ಲಿಪಿಸ್ವರೂಪ} ಒಂದು ಕಾರಣವಾದರೆ, ಆಧುನಿಕ ತಂತ್ರಾಂಶಗಳೆಲ್ಲ ಇಂಗ್ಲಿಷ್ \hbox{ಭಾಷೆಯನ್ನೇ} ಆಧರಿಸಿ, ಅದರ ರಚನೆಗೆ ತಕ್ಕಂತೆ ಸಿದ್ಧವಾಗುತ್ತಿರುವುದು ಇನ್ನೊಂದು ಕಾರಣ. ತಮಿಳು, ಮಲೆಯಾಳಮ್ ಇತ್ಯಾದಿ ಭಾಷೆಗಳಲ್ಲಿ ಈ ನೇರದಲ್ಲಿ ಬಹಳ \hbox{ಕೆಲಸವಾಗಿದೆ.} ಕನ್ನಡಿಗರಿಗೆ ಭಾಷಾಭಿಮಾನದ ಕೊರತೆ ಮತ್ತು ಇಂಗ್ಲಿಷ್ ವ್ಯಾಮೋಹದ ಕಾರಣ ಇಂತಹ ಸಮಸ್ಯೆಗಳು ಹಾಗೆಯೇ ಉಳಿದಿವೆ. ಪ್ರಕೃತ  ಇಲ್ಲಿ \hbox{ಸಾಧ್ಯವಾದಷ್ಟೂ} ಈ ಅವಲಕ್ಷಣವನ್ನು ಪರಿಹರಿಸಲು  ಪ್ರಯತ್ನಿಸಿದ್ದೇವಾದರೂ ಕೆಲವೆಡೆ \hbox{ಓದುಗರು} \hbox{ಸಹಿಸಿಕೊಳ್ಳುವುದೂ} ಅನಿವಾರ್ಯವಾಗಿರುವುದನ್ನು ಅಲ್ಲಗಳೆಯುವಂತಿಲ್ಲ. ಈ ಕಾರ್ಯ ನಮಗೆ ಸಾಕಷ್ಟು  ಸವಾಲಾಗಿದ್ದು ನಿಜ. ಇಂತಹ ಸವಾಲನ್ನು ನಿರ್ವಹಿಸಲು ಸಮರ್ಥವಾಗಿರುವ ಸಂಸ್ಥೆ  ಮೈಸೂರಿನ ಶ್ರೀರಂಗ ಡಿಜಿಟಲ್ ಸಾಫ್ಟ್ ವೇರ್ ಟೆಕ್ನಾಲಾಜಿಕಲ್ \hbox{ಪ್ರೈವೇಟ್} ಲಿಮಿಟೆಡ್. ಇಲ್ಲಿಯ ತಂತ್ರಜ್ಞರಾದ ಶ್ರೀ ಡಿ.\ ಶಿವಶಂಕರ್ ಮತ್ತು ಶ್ರೀ \hbox{ಕಿಶೋರರವರು} ಶ್ರೀ ಯೋಗಾನಂದರ ನಿರ್ದೇಶನದಲ್ಲಿ ಗ್ರಂಥಕ್ಕೆ ವ್ಯವಸ್ಥಿತ\- ರೂಪ \hbox{ಕೊಡುವಲ್ಲಿ} ಸಾಕಷ್ಟು ಶ್ರಮಿಸಿದ್ದಾರೆ. ಆದರೆ ಈ ಸಂಸ್ಥೆಯ ವ್ಯವಹಾರ ಭಾರ ನಮ್ಮ \hbox{ಧಾರಣೆಗೆ} ನಿಲುಕದೇ ಗ್ರಂಥ ವಿಲಂಬವಾದುದು ವಿಷಾದನೀಯ.  ಅವರ ಶ್ರಮಕ್ಕೆ  ನಾವು ಅತ್ಯಂತ \hbox{ಆಭಾರಿಗಳಾಗಿದ್ದೇವೆ.} ಪ್ರಕೃತ ಗ್ರಂಥ  ವಿಲಂಬವಾದರೂ ಅದು ಇಂದು ಸಂಪೂರ್ಣವಾಗಿ ಡಿಜಿಟಲ್ ಮಾಧ್ಯಮದಲ್ಲಿ ಅಳವಡಿಸಿ ಗೂಗಲ್ ಪುಸ್ತಕವಾಗಿಸಲು ಅಗತ್ಯವಿರುವ ಸಕಲ ತಾಂತ್ರಿಕ ವ್ಯವಸ್ಥೆಯನ್ನೂ ಒಳಗೊಳ್ಳುವಂತೆ ಆಗಿರುವುದು  ಸಣ್ಣ ವಿಷಯವಲ್ಲ, ಅದಾದುದು ಅತ್ಯಂತ ಸಂತಸದ ಸಂಗತಿ.

ಗ್ರಂಥದ ಮುಖಪುಟಕ್ಕೆ ಚಿತ್ರವನ್ನು ಸಾಕ್ಷಾತ್ಕರಿಸಿಕೊಟ್ಟವರು ಶ್ರೀಮಾನ್ \hbox{ಲೋಕೇಶ್} ರವರು. ನಮ್ಮ ಕಲ್ಪನೆಯನ್ನು ಚಿತ್ರವಾಗಿಸಲು ಮತ್ತು ಇರುವ ಜಾಗದಲ್ಲಿ ಅದನ್ನೆಲ್ಲ ಅಳವಡಿಸಲು ಅವರು ಸಾಕಷ್ಟು ಶ್ರಮಿಸಬೇಕಾಯಿತು. ಆದರೂ ಬಹುಮಟ್ಟಿಗೆ ಅದು ಕೂಡಿಬಂದಿದೆಯೆಂದು ನಮ್ಮ ಭಾವನೆ. ಚಿತ್ರಕಾರರಾದ ಶ್ರೀ ಲೋಕೇಶರಿಗೆ ಕೃತಜ್ಞತೆಗಳನ್ನು ತಿಳಿಸಬಯಸುತ್ತೇವೆ.

ಗ್ರಂಥವನ್ನು ಅನ್ನಪೂರ್ಣಾ ಆಪ್ಸೆಟ್  ಮುದ್ರಣಾಲಯದವರು ಮುದ\-ವಾಗುವಂತೆ ಮುದ್ರಿಸಿದ್ದಾರೆ. ಅದರ  ಮಾಲೀಕರಾದ ಶ್ರೀಮತಿ ಮೀರಾರವರಿಗೆ ನಾವು ಆಭಾರಿಗಳು.
\vskip 4pt

ಗ್ರಂಥದ ಗುಣಮಟ್ಟಕ್ಕೆ ಪ್ರಮಾಣ ಸದಸದ್ವಿವೇಕಿಗಳಾದ ಓದುಗರೇ ಎಂಬುದು\break ನಿರ್ವಿವಾದ. ಆದರೆ ಈ ಆಧುನಿಕ ಕಾಲದಲ್ಲಿ ವ್ಯಾವಹಾರಿಕ ದೃಷ್ಟಿಯಿಂದ ಗ್ರಂಥದ \hbox{ಗುಣಮಟ್ಟ} ನಿರ್ಧರಿಸುವ ವಿಭಿನ್ನವಾದ ಮಾನದಂಡದ ವ್ಯವಸ್ಥೆಯೂ ಉಂಟು. ಹಾಗೆ ಗುಣಮಟ್ಟವನ್ನು ನಿರ್ಧರಿಸುವ ಸಂಸ್ಥೆ ದೆಹಲಿಯಲ್ಲಿರುವ\enginline{-}{Rajaram Mohan Roy National Agency for ISBN} ಎಂಬುದು. ಅದು ನಿರ್ದಿಷ್ಟವಾದ ಕೆಲವು ಮಾನ\-ದಂಡಗಳನ್ನು ಅನುಸರಿಸಿ ಗುಣಮಟ್ಟ ನಿರ್ಧರಿಸುತ್ತದೆ. ಅನಂತರ ಗ್ರಂಥಕ್ಕೆ ಒಂದು ಸಂಖ್ಯೆಯನ್ನು ನೀಡುತ್ತದೆ. ಈ ಸಂಖ್ಯೆಯ ಉಲ್ಲೇಖ ಗ್ರಂಥದಲ್ಲಿದ್ದಾಗ ಅದಕ್ಕೊಂದು ವಿಶೇಷ ಮಾನ್ಯತೆಯುಂಟು. ಅಂತಹ ಗ್ರಂಥದಲ್ಲಿ ಲೇಖನ ಪ್ರಕಟವಾದರೆ ಅದರಿಂದ ಸಂಶೋಧಕ ಲೇಖಕರಿಗೆ ಕೆಲವು ಅನುಕೂಲತೆ\-ಗಳಿವೆ. ಈಗಿನ ವ್ಯಾವಹಾರಿಕ ಬದುಕಿಗೆ ಆ ಅನುಕೂಲತೆಯನ್ನು ನಾವು ನಿರಾಕರಿಸುವಂತಿಲ್ಲ. ಲೇಖನದ ಹಿಂದೆ ಸಾಕಷ್ಟು ಶ್ರಮವಿರುವುದರಿಂದ ನಮ್ಮ ಗ್ರಂಥದಲ್ಲಿ ಲೇಖನ ಪ್ರಕಟವಾಗಿ ಲೇಖಕರಿಗೆ  ಕಿಂಚಿತ್ ಅನುಕೂಲ\-ವಾದರೂ ನಮಗೊಂದು ಸಾರ್ಥಕ ಭಾವ. ಈ ದೃಷ್ಟಿಯಿಂದ ನಾವು \enginline {ISBN} ಸಂಖ್ಯೆಗೆ ಎಲ್ಲ ರೀತಿಯಿಂದ ಶ್ರಮ ವಹಿಸಿದೆವು. ಅಂತೂ ಅದನ್ನು ಪಡೆದುಕೊಳ್ಳದೇ ಬಿಡಲಿಲ್ಲ. ಅದೇನಿದ್ದರೂ, ಇದರಿಂದ ಲೇಖಕ\-ರಿಗಾಗುವ ಪ್ರಯೋಜನ ನಮಗೆ ಸಮಾಧಾನವನ್ನು ಉಂಟುಮಾಡಿದೆ. ಈ ವ್ಯವಹಾರ ಮೊದಲ ಅನುಭವವಾದ್ದರಿಂದ ಅದನ್ನು ಪಡೆದು\-ಕೊಳ್ಳುವಲ್ಲಿ ಸಾಕಷ್ಟು ವಿಲಂಬವಾಗಿದೆ. ಗ್ರಂಥ ಮುದ್ರಣದಲ್ಲಾದ ವಿಲಂಬಕ್ಕೆ  ಇದೇ ಒಂದು ಪ್ರಧಾನ ಕಾರಣವೆಂದರೆ ತಪ್ಪಿಲ್ಲ. ಇದೇನೇ ಇರಲಿ,\enginline {ISBN}ಗೆ ನಮ್ಮ ಗ್ರಂಥವನ್ನು ಪರಿಗಣಿಸಿರುವುದಕ್ಕೆ ಆ ಸಂಸ್ಥೆಗೆ ನಾವು ಕೃತಜ್ಞತೆಯನ್ನು ಸಲ್ಲಿಸುತ್ತೇವೆ.
\vskip 4pt

ಇಷ್ಟು ಗುರುಗಾತ್ರದ ಗ್ರಂಥದಲ್ಲಿ ಸ್ಖಾಲಿತ್ಯಗಳಿಲ್ಲವೆಂದರೆ ಅದು ಅಂಧ\break ವಿಶ್ವಾಸವಾದೀತು. ಪುಸ್ತಕ ಪರಿಶ್ರಮಿಗಳಿಗೆಲ್ಲ ಇದು ತಿರೋಹಿತವಾದ \hbox{ವಿಷಯವೇನೂ} ಅಲ್ಲ. ಸದಸದ್ವಿವೇಕವುಳ್ಳ ಸಹೃದಯಿಗಳು ಗುಣವನ್ನು ಗ್ರಹಿಸುವರೆಂಬ ವಿಶ್ವಾಸ \hbox{ನಮಗಿದೆ}. ಗ್ರಂಥ ವಿದ್ಯಾರ್ಥಿಗಳ ಪರಿಶ್ರಮ. ಇಲ್ಲಿ ದೋಷಗಳೇನಾದರೂ ಕಂಡುಬಂದಲ್ಲಿ \hbox{ಅವನ್ನು} ಗುರುಜನರು ನಮ್ಮ ಗಮನಕ್ಕೆ ತಂದರೆ ಅದು ಪುನಃ ಪರಿಷ್ಕಾರಕ್ಕೆ ಅತ್ಯಂತ ಸಹಾಯಕ. ಈ ದೃಷ್ಟಿಯಿಂದ ಅಂತಹ ಆಪ್ತ ಸಲಹೆಗಳನ್ನೆಲ್ಲ ಪ್ರೀತಿಯಿಂದ ಸ್ವಾಗತಿಸುತ್ತೇವೆ, \hbox{ಉಪಕೃತಿಯನ್ನು}  ಭಾವಿಸುತ್ತೇವೆ.

ಗಂಗಾಧರ ಭಟ್ಟರ ಬಗ್ಗೆ ಘಟನೆಗಳನ್ನು ಕಲೆಹಾಕಿ ಪುಸ್ತಕ ಮಾಡುವುದು ನಮ್ಮ ಉದ್ದೇಶವಲ್ಲ. ಏಕೆಂದರೆ ಒಂದಷ್ಟು ಘಟನೆಗೆ ಯಾರೂ ವಿಷಯವಾಗಬಹುದು. ಆದರೆ ಅವರೆಲ್ಲ ಸಾಹಿತ್ಯ ಸೃಷ್ಟಿಗೆ ವಿಷಯವಾಗುತ್ತಾರೆಂದು ಹೇಳಲಾಗದು. ಗಂಗಾಧರ ಭಟ್ಟರು ಹಾಗಲ್ಲ, ಅವರು ಸಾಹಿತ್ಯಕ್ಕೂ ವಿಷಯವಾಗಬಲ್ಲರು, ಅವರ ಜೀವನ \hbox{ಕಾದಂಬರಿಗೂ} ವಸ್ತುವಾಗಬಲ್ಲದು, ಅದನ್ನವಲಂಬಿಸಿ, ಒಂದು ಸಿನೆಯನ್ನೂ ತಯಾರಿಸಬಹುದು \enginline{-} ಅವರ ವ್ಯಕ್ತಿತ್ವ ಅಷ್ಟು ವ್ಯಾಪಕ. ಹಾಗಾಗಿ ಪ್ರಕೃತ ನಮ್ಮ ಉದ್ದೇಶ \enginline{-} ಸಾಹಿತ್ಯ \hbox{ಸೃಷ್ಟಿಯೇ} ವಿನಾ ಘಟನೆಗಳ ಸಂಘಟನೆಯಲ್ಲ. ಹಾಗಾಗಿ ಅವರ ನೆಪದಲ್ಲಿ ಒಂದು ಸಾಹಿತ್ಯ \enginline{-} ಗ್ರಂಥ-ರಚನೆ ನಮ್ಮ ಈ ಪ್ರಯತ್ನ. ಈ ಸಾಹಿತ್ಯ ಅವರವರ ಇಷ್ಟದ \hbox{ಸೌಹಿತ್ಯಗಳಂತೆ} ಆಸ್ವಾದನೀಯವಾಗುವುದೆಂಬ ನಂಬಿಕೆ ನಮ್ಮದು.

ಗ್ರಂಥಸಂಪಾದನ ಕಾರ್ಯಕ್ಕಾಗಿ ಉಳಿದ ವಿದ್ಯಾರ್ಥಿಗಳು ನನ್ನನ್ನು ಆಯ್ಕೆ \-ಮಾಡಿದರೂ, ಕೆಲವು ದೃಷ್ಟಿಯಿಂದ ಅವರೆಲ್ಲರಿಗಿಂತ ನಾನು ಅತ್ಯಂತ ಅವರನಾದವನು.  ಈ ಸಂದರ್ಭದಲ್ಲಿ, "ಮತ್ತಃ ಪ್ರತ್ಯವರಃ ಕಶ್ಚಿತ್ ನಾಸ್ತಿ ಸುಗ್ರೀವಸನ್ನಿಧೌ \enginline{-} ಸುಗ್ರೀವನ ಸನ್ನಿಧಿಯಲ್ಲಿರುವವರಲ್ಲೆಲ್ಲ ನಾನೇ ಅತ್ಯಲ್ಪ ಸಾಮರ್ಥ್ಯದವ" ಎಂಬ ಆಂಜನೇಯನ ಮಾತು ನನಗೆ ನೆನಪಾಗುತ್ತದೆ. ಆದರೂ ಅಯಾಚಿತವಾಗಿ ಬಂದ ಜವಾಬ್ದಾರಿಯನ್ನು ಗಂಗಾಧರ ಭಟ್ಟರ ಬಗೆಗಿನ ಪ್ರೀತ್ಯಾದರಗಳಿಂದಾಗಿ  ಉಳಿದ ವಿದ್ಯಾರ್ಥಿಗಳ ಸಹಾಯದಿಂದ ಯಥಾಶಕ್ತಿ ನಿರ್ವಹಿಸಲು ಯತ್ನಿಸಿದ್ದಾಗಿದೆ. ಇನ್ನು, ಗ್ರಂಥದ ಸೌಂದರ್ಯ \hbox{ಗಾಂಭೀರ್ಯಗಳೇ} ಸಂಪಾದಕನ ಮೊದಲ ಲಕ್ಷ್ಯವಷ್ಟೇ. ಅದಕ್ಕೆ ಅಗತ್ಯ ಸ್ವಾತಂತ್ರ್ಯವನ್ನು ಸಂಪಾದಕನಾಗಿ ಬಳಸಿಕೊಂಡಿದ್ದೇನೆ. ಹಾಗಾಗಿ ಇಲ್ಲಿ ಏನಾದರೂ ಕುಂದುಕೊರತೆಗಗಳು \hbox{ಉಂಟಾದಲ್ಲಿ} ಅವುಗಳನ್ನು ಹಿರಿಯರು \hbox{ಮನ್ನಿಸುವರೆಂದು} ನಂಬಿದ್ದೇನೆ.

ಅಂತಿಮವಾಗಿ \enginline{-} 
ನಮ್ಮ ಕಾರ್ಯಕ್ಕೆ ಪ್ರತ್ಯಕ್ಷ\enginline{-}ಪರೋಕ್ಷವಾಗಿ ಸಹಕರಿಸಿದ ಎಲ್ಲರಿಗೂ ಅನಂತ ಕೃತಜ್ಞತೆಗಳು. ಈ ಎಲ್ಲ ನಮ್ಮ ಕಾರ್ಯಕಲಾಪ ಸಂಪನ್ನವಾಗಿದ್ದರೂ, ಶ್ರೀಮಾನ್ ಗಂಗಾಧರ ಭಟ್ಟರು ಮತ್ತು ಅವರ ಸಹಧರ್ಮಿಣೀ ಶ್ರೀಮತಿ ಶೈಲಜಾರವರು ನಮ್ಮಲ್ಲಿ ಹೊಂದಿರುವ ವಾತ್ಸಲ್ಯ\enginline{-}ವಿಶ್ವಾಸವೇ ಅವೆಲ್ಲಕ್ಕೂ ಪ್ರಧಾನ ಚೈತನ್ಯವಾಗಿದೆ. ನಮ್ಮೆಲ್ಲ ಕಾರ್ಯದಿಂದ ಅವರಿಬ್ಬರ ಋಣ ತೀರಿಸಿದೆವೆಂಬ ಭ್ರಮೆ ನಮಗಿಲ್ಲ. ಅವರು ಕೊಟ್ಟಿರುವು\-ದನ್ನು ವ್ಯಕ್ತಪಡಿಸುವ ಪ್ರಯತ್ನವಿದಾಗಿದೆಯೇ ವಿನಾ ನಾವು ಅವರಿಗೆ ಏನನ್ನೂ \hbox{ಕೊಟ್ಟಿಲ್ಲ.} ಕೆರೆಯ ನೀರನು ಕೆರೆಗೆ ಚೆಲ್ಲುವ ಪ್ರಯತ್ನ ಮಾಡಿದ್ದೇವೆ. ಅವರ ಪರಿಶ್ರಮ ನಮ್ಮ ಪರಿಶ್ರಮವೂ ಆದಾಗ ಅಲ್ಲೊಂದು ತೃಪ್ತಿಗೆ ವಿಷಯವಿರಬಹುದಷ್ಟೆ ! ನಮ್ಮ  ಈ ಎಲ್ಲ ಕಾರ್ಯ\-ಕಲಾಪಗಳಲ್ಲಿ ಅವರ ವ್ಯವಹಾರ ಗಾಂಭೀರ್ಯಕ್ಕಾಗಲೀ ವಿಷಯ ಗಾಂಭೀರ್ಯಕ್ಕಾಗಲೀ ಹೊಂದದ ಅಂಶಗಳಿದ್ದರೆ ಅಂಥವುಗಳನ್ನು ಅವರು ಮನ್ನಿಸಬೇಕೆಂದು ವಿನಂತಿಸುತ್ತೇವೆ. ಅವರಿಗೆ ಆಯುರಾರೋಗ್ಯಭಾಗ್ಯ ಕೂಡಿಬರಲೆಂದು ನಮ್ಮೆಲ್ಲರ ಪ್ರಾರ್ಥನೆ. 

ಒಬ್ಬ ಅಧ್ಯಾಪಕರಾಗಿ ಪಾಠವನ್ನೇ ಜೀವನವಾಗಿಸಿಕೊಂಡ, ಜೀವನವನ್ನೆಲ್ಲ ಪಾಠವಾಗಿಸಿದ ಚೇತನಕ್ಕೆ ಈ ಗ್ರಂಥಸುಮ ಸಮರ್ಪಿತ.\\
\smallskip
\noindent
\centerline{॥ ಭೂಯಿಷ್ಠಾಂ ತೇ ನಮ ಉಕ್ತಿಂ ವಿಧೇಮ ~॥}\\
\centerline{ ॥ ಭದ್ರಂ ಶುಭಂ ಮಂಗಳಮ್ ~॥}

\medskip
\hfill\begin{tabular}{c}
\textbf{ವಿ~॥ ಗುರುಪ್ರಸಾದ. M.A}\\
\textbf{ಸಂಪಾದಕ}\\
{\small ವಿ~॥ ಗಂಗಾಧರ ಭಟ್ಟರ}\\
{\small ಅಭಿವಂದನಗ್ರಂಥ ಸಂಪಾದನ ಸಮಿತಿ}\\
guruprasaada@gmail.com\\
+91 9481818394%
\end{tabular}
\articleend
}

