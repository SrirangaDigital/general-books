{\fontsize{15}{17}\selectfont
\chapter{अद्वैततत्त्वसमीक्षा}

\begin{center}
\Authorline{विद्वान्, डा. अनन्त नागेन्द्र भट्टः,}
\smallskip
अध्यापकः संस्कृतविभागः, मानसगङ्गोत्री,\\
मैसूरु विश्वविद्यालयः,\\
मैसूरु ५७०००६
\addrule
\end{center}
श्रीशङ्कराचार्यैः प्रतिपादितो वेदान्तः ‘अद्वैतसिद्धान्तः’ इति प्रसिद्धः अस्ति~। एतदनुसारं समस्तविश्वप्रपञ्चः एकस्मिन्नेव तत्त्वे अन्तर्भूतः वर्तते~। तत्सत्तयैव स्थितः तत्प्रकाशेनैव च प्रकाशितः~। एतन्मते चेतनतत्त्वाद् आत्मनः पृथक् कस्यचिदपि अन्यस्य वस्तुनः सत्ता नेह वर्तते इत्यात्मत्वैकमात्रं सत्यम्~। स एव च उपाधिविवर्तितः अनेकजडचेतनपदार्थेष्वेव अवलोक्यते~। तस्य च स्वतः अखण्डैकत्वेन न तत्र भेदांशस्य किञ्चिन्मात्रमपि समवायः~। ‘ब्रह्म सत्यं जगन्मिथ्या’ इति शब्दचतुष्टये संक्षेपेण तत्सिद्धान्तस्य सामान्यपरिचयः विद्यते~। अस्य अद्वैतसिद्धान्तस्य विशेषपरिचयः इत्थं वर्तते~। 

\section*{आत्मा अनात्मा च}

ब्रह्मसूत्रभाष्यं प्रारिप्सुना भगवता शङ्करेण सर्वतः पूर्वं युष्मदस्मदोरित्यादिना आत्मानात्मनोः विवेचनं कृतम्~। सूक्ष्मेक्षिकया अवलोकनेन अयं जागतिकप्रपञ्चः प्रधानतया द्रष्टृदृश्यभेदेन उभयथा विभाजयितुं शक्यते~। समस्तप्रतीतीनाम् अनुभवकर्ता एकः, अनुभवविषयश्च द्वितीयः~। तत्र समस्तप्रतीतीनां चरमः साक्षी आत्मा, तदितरतत्सर्वं तत्प्रतीतिविषयः अनात्मा~। आत्मतत्त्वं हि नित्यं, निश्चलं, निर्विकारं, असङ्गतं, अखण्डं, निरञ्जनं, निर्विकल्पकं, सजातीय-विजातीय-स्वजतभेदशून्यं कूटस्थं, एकं, शुद्धचैतन्यस्वरूपं, निर्विशेषं च वर्तते~। ‘साक्षी चेता केवलो निर्गुणश्च’ इत्यादिश्रुतयः तत्र प्रमाणम्~। बुद्धिमारभ्य स्थूलभूतपर्यन्तं प्रपञ्चमात्रस्य न कश्चिदात्मना सम्बन्धः वर्तते~। जीवश्च अविद्यावशादेव देहेन्द्रियादिभिः स्वसम्बन्धं संस्थाप्य (स्वीकृत्य) आत्मानम् अन्धं विद्वांसं मूर्खं सुखिनं दुःखिनं कर्तारं भोक्तारं वा मन्यते~। इत्थं बुद्ध्यादिभिः सह प्रतीयमानं आत्मनः तादात्म्यं श्रीशंकरेण ‘अध्यास’ इति शब्देन निरूपितम्~। तत्सिद्धान्तानुसारं हि समस्तप्रपञ्चस्य सत्यत्वप्रतीतिः मायापरपर्यायाध्यासनिमित्ता इति अद्वैतवाद अपि अध्यासवादः, मायावादः वा कथ्यते~। इत्थं च यत् किंचित् दृश्यं प्रपञ्चजातं तत्सर्वं मायापरपर्यायाज्ञानेन एव विभिन्नं प्रतीयते~। वस्तुतस्तु तत्सर्वं एकं, अखण्डं, शुद्धं, चिन्मात्रं विद्यते~। ततः पृथक् न कस्यापीह वस्तुनः सत्ता वर्तते~। 

\section*{ज्ञानमज्ञानञ्च}

समस्तविभिन्नप्रतीतिस्थाने एक-अखण्ड-सत्-चित्-आनन्दघन-अनुभव एव ज्ञानम्, तत्सर्वाधिष्ठाने दृष्टिम् अदत्वैव भेदे सत्यत्वबुद्धिस्थापनं चाज्ञानम्, तात्विकदृष्ट्या यथा सर्वेषां सौवर्णालङ्काराणां कटककुण्डलादीनां सुवर्णमात्रत्वम्, विभिन्नानां च घटरुचकादीनां मृण्मयमात्राणां च मृत्तिकामात्रत्वमेव वीचितरङ्गाणां वा यथा जलाभिन्नत्वमेव, तथैव अनेकविधभेदसङ्कुलितं जगत् केवलं शुद्धं परब्रह्मैवेति न ततो भिन्नम्~। यतो न तदतिरिक्तं किमपि वस्तु वर्तते~। अत एव श्रुतिः ‘सर्वं खल्विदं ब्रह्म’ इति बोधयति~। एवंभूतः अभेदबोध एव ज्ञानम्~। यावन्न इत्थं बोधः न तावज्जीवः संसारचक्रात् मुक्तो जायते~। 

\section*{ज्ञानसाधनम्}

भगवता शङ्करेण श्रवणं, मननं, निधिध्यासनं च ज्ञानस्य साधनं स्वीकृतम्~। किन्तु, तत्त्वजिज्ञासायामेव तत्साफल्यमुक्तम्~। तज्जिज्ञासोत्पत्तौ च दैवी सम्पत्तिः सहायिकेति तन्मते विवेक-वैराग्य-समादि-षट्कसम्पत्ति-मुमुक्षुता-साधनसम्पन्नस्यैव चित्तशुध्यनन्तरं ब्रह्मतत्त्वजिज्ञासा समुत्पद्यते~। एवंभूत-चित्तशुद्ध्यर्थं निष्कामकर्मानुष्ठानस्य आत्यन्तिक-आवश्यकत्वं उक्तम्~। 

\section*{भगवद्भक्तिः}

पूज्यशंकरेण भक्तिस्तु ज्ञानोत्पत्तेः प्रधानं साधनम् अमन्यत~। फलस्वरूपेण तु ज्ञानमेव तेन स्वीक्रियते~। भक्तिं लक्षणया च तेन विवेकचूडामणौ ‘स्वरूपानुसन्धानं भक्तिरित्यभिधीयते’ इत्युक्तम्~। इत्थञ्च आत्मनः शुद्धस्वरूपस्मरणस्य भक्तित्वेन आत्मजिज्ञासोर्भक्तेः प्राधान्येऽपि न तेन सगुणोपासनं उपिक्षिता~। अत एव ‘न च कार्ये प्रत्यपत्यभिसन्धिः’ (४.३.१४) इति सूत्रभाष्ये ‘किं विषयाः पुनर्गतिश्रुतयः’ इत्युपक्रम्य ‘सगुणविषया भविष्यन्तीति सत्यकामादिर्गुणैः सगुणस्यैवोपास्यत्वात् सम्भवति गतिः’ इति स्पष्टतया सगुणोपासकानां गतिरुक्ता~। 

\section*{कर्मसन्यासः}

श्रीशंकराचार्येण स्वभाष्ये यत्र तत्र कर्मणां स्वरूपेण त्यागमुक्त्वा जिज्ञासूनां उपलब्धात्मतत्त्वावबोधानां चोभयेषां कृते सर्वकर्मसन्यासस्य अवश्यकता प्रतिपादिता~। तन्मते निष्कामकर्मणा केवलं चित्तशुद्धिः भवति~। परमपदावाप्तिस्तु कर्मसन्यासपूर्वकं श्रवण-मनन-निदिध्या\-सनादिभिः आत्मतत्त्वस्य अवबोधानन्तरं एव सम्भवति~। व्यावहारिकदशायां एतत्सिद्धान्तानुसारं ब्रह्मजीवयोः ऐक्यं अवबोधस्य उपलब्धये मानसिकशुध्यर्थं च कर्म अवश्यं कर्तव्यम्~। किन्तु, ततः परं तत्कर्मपरित्याग एव वरीयान्~। ज्ञानकर्मणोः प्रकाशान्धकारयोरिव परस्परविरुद्धतया कर्मसन्यासमन्तरेण परिपूर्णमोक्षानुपपत्तेः~। अयमेव निवृत्तिमार्गः ज्ञानिनिष्ठा वा उच्यते~। 

\section*{जीवः}

‘जीवो ब्रह्मैव नापरः’ इति श्रीशङ्करकथनानुसारं जीवो ब्रह्मणः आभासमात्रं इति तत्तुल्यस्वभवः स्वप्रकाशश्चेति घटे भग्ने घटाकाशो महाकाश इव जीव अपि बुद्धिरूप-उपाधि-विनाशानन्तरं ब्रह्मणि विलीयते~। न तदानीं तदस्तित्वं तद्भोग्यं वा किमपि अवशिष्यते~। स च जीवः अस्मिन्नेव देहे ब्रह्म साक्षात्कृत्य जीवन्मुक्तः भूत्वा, शरीरपातानन्तरं अतीतसांसारिकसुखदुःखः सच्चिदानन्दब्रह्मस्वरूपो जायते~। 
अनेकानि श्रुतिवाक्यानि जीवात्मानं अणुस्वरूपमिति प्रतिपादयन्ति~। किन्तु, अत्र अद्वैतमते जीवो ब्रह्मणः अभिन्न इति स्थाप्य, तत्तुल्य जीवोऽपि विभुः इति प्रतिपाद्यते~। सः जीवात्मा इन्द्रियेण ग्राह्ययितुमशक्यः अत्यन्तसूक्ष्मो वर्तते इति प्रतिपादयितुं उपनिषत्सु जीवात्मनः अणुत्वं पदर्शितम्~। अद्वैतमते विग्रहरूपः अन्तःकरणावच्छिन्नः परमात्मैव जीवः~। 
अद्वैतवेदान्तस्यायं मूलसिद्धान्तः यत् व्यष्टिसमष्ट्योः नकिञ्चिदन्तरं विद्यते~। अत्र व्यक्तिशरीरं व्यष्टिः, समूहात्मकञ्च जगतः समष्टिः इत्युक्तम्~। एतत्सिद्धान्तानुसारं व्यष्टौ स्वीकृतस्य स्थूलं, सूक्ष्मं, कारणं नामकशरीरत्रयस्य अभिमानिनो जीवाः क्रमशः विश्वः, तैजसः, प्राज्ञः इत्युच्यन्ते~। समष्टौ च तदभिमानि चैतन्यं क्रमशः विराट् (वैश्वानरः) सूत्रात्मा ईश्वरः (हिरण्यगर्भः) इति नाम्ना अभिधीयते~। व्यष्टि-समष्टि-अभिमानिपुरुषयोः अभिन्नत्वेऽपि आत्मा एभ्यः त्रिभ्यः पृथक् स्वतन्त्रा सत्ता वर्तते इत्यपि प्रदर्शितम्~। 

\section*{माया}

श्रीशङ्करमते माया, अविद्या, अज्ञानं चेति पर्यायाः~। तत्र ‘मीयते अमेयम् अनयेति माया’ एषा ब्रह्मणः शक्तिः विद्यते~। ब्रह्माश्रिता सैव माया तत्र विविधविवर्तानुत्पादयति इति जगतो वैविध्यम् औपाधिकमस्ति~। सदसत्कर्मफलहेतुके स्वर्गनरकविषये च श्रीशङ्करस्येदं धारणा यत्कर्मानुसारं मनुष्याणां गतिरिति धृतनवशरीरकृतशुभाशुभकर्मणां फलभोगः~। इत्थं च कृतयज्ञदानाद्युक्तशुभकर्मणो जनाः पितृयानेन सगुणब्रह्मोपासकाश्च देवयानेन मुक्तिमार्गमनुसरन्ति~। अशुभकर्मणां च कृते पुनर्जन्ममरणादिकं मायोपहितं स्थानं सुरक्षितं वर्ततेऽत्र~। 

\section*{मोक्षः}

व्यावहारिकस्य जगतः सन्तापेभ्यो मुक्तिप्राप्तिरेव तत्त्वसाधनस्य चरमं (अन्तिमं) लक्ष्यम्~। मनुष्यमात्रस्य च जीवनं यानि ध्येयानि पुरस्कृत्य प्रवर्तते तानि पुरुषार्था इति नाम्ना उच्यन्ते~। धर्म-अर्थ-काम-मोक्षा इति ते चत्वारः पुरुषार्थाः~। मोक्षश्च तेषु सर्वश्रेष्ठः, परमपुरुषार्थः~। एतद्विषये साधारणजनानामियं धारणा यदयं पुरुषार्थो नानेन शरीरेणावाप्यते~। किन्तु, श्रीशंकरस्य उपनिषदाधारेण एतन्मतं ब्रह्मज्ञानस्य अथवा आत्मज्ञानस्य प्राप्तावनेनैव शरीरेण मोक्षः समवाप्यते~। ‘यदा सर्वे विमुच्यन्ते कामा ह्यस्य हृदिस्थिताः~। तदा मर्त्योऽमृतो भवत्यत्र ब्रह्म समश्नुते’ (क. २.३.१४) इत्यादिश्रुतयः अत्र प्रमाणम्~। एषैव च मुक्तिः जीवन्मुक्तिरुच्यते~। इत्थं च ‘तत्वमसि’ इत्यादिना अभ्यासेन उत्पन्नं विशुद्धं ज्ञानं जीवस्य अनाद्यज्ञानं तज्जन्यमलिनसंस्कारजातं च विनाशति इति ‘अहं ब्रह्मास्मि’ इति रूपेण तदानीन्तनः ब्रह्मजीवयोः ऐक्यानुभव एव मोक्षः~। 

\centerline{॥ इति शम्~॥}

\articleend
}
