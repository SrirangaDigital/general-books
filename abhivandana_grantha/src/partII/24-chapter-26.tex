{\fontsize{15}{17}\selectfont
\presetvalues
\chapter{शुद्धधात्वर्थकरणक-भावनाविधानम्}

\begin{center}
\Authorline{वि~॥ हेरम्ब आर् भट्टः}
\smallskip

अध्यापकः\\
केन्द्रीयविद्यालयः,\\ 
मैसूरु
\addrule
\end{center}

परमकारुणिकेन कुमारिलभट्टेन प्रणीतायां भाट्टदीपिकायां प्रथमचतुर्थे द्वितीयाधिकरणे ‘अपि वा नामधेयं स्यात् यदुत्पत्तौ.....’ इत्यादि सूत्राण्यधिकृत्य ‘उद्भिदा यजेत पशुकामः’ इत्यादि वाक्येषु समानवृत्तिं तथा गुणकर्मबोधकानां प्रकृतविधेयगुणसमर्पकत्वं वा, उत नामधेयत्वं वा इति सन्देहः प्रदर्शितः~। तत्रायं पूर्वपक्षः~। उद्भिद्शब्दस्य भिदिर् विदारणे इति स्मृतिपर्यालोचनायां विदारणसमर्थखनित्रवाचित्वप्रसिद्धेः विधेयगुणसमर्पकत्वमेव युक्तम्~। न तु नामधेयत्वम्~। वैय्यर्थ्यप्रसङ्गात्~। 

अतश्च  ‘ उद्भिदा यजेत इत्यत्र उद्भिद्गुणविशिष्टं कर्म पशुरूपफलोद्देशेन विधीयते इति प्रथमः पूर्वपक्षः~। 

अथवा अस्य वचनस्य ताण्ड्यशाखायां ज्योतिष्टोमप्रकरणपाठात् यजतिना प्रकृतसोमयागमात्रम् अनूद्य उद्भिन्नाम गुणमात्रविधानम् इति द्वितीयः पूर्वपक्षः~। 

यदि तु एतद्वाक्यविहित - यागानुवादेन सोमेन यजेत इत्यादि वाक्येन सोमविधानं कुतो न सम्भवति? इत्याशङ्का जायेत तदा सोमप्रकरणे इतिकर्तव्यता कलापस्य पाठेन ‘सोमेन यजेत’ पाठसन्निधौ इतिकर्तव्यता विशेषस्यैव पाठेन कर्मविधित्वनिश्चयात्~। सतिचैवम् - ‘उद्भिदा यजेत पशुकामः’ इति वाक्यघटकफलपदस्य आनर्थक्यं सम्भवति इति चेन्न~। 

सर्वेभ्यो ज्योतिष्टोमः इति वचनेन ज्योतिष्टोमस्य सर्वफलार्थत्वनिश्चयेन उद्भिदा यजेत\break पशुकामः इति ‘यजि’ पदस्य  पशुफलोपधायकज्योतिष्टोमयागलक्षणायां तात्पर्यग्राहकत्वं\break सम्भवति~। अत एव अत्र विशिष्टोद्देशनिबन्धनवाक्यभेदस्य नावकाशः~। नन्वत्र उत्पत्तिविशिष्ट\-सोमस्य प्राबल्येन उत्पन्नशिष्टउद्भिद्गुणस्य निवेशः न सम्भवति इति शङ्का जायते~। अथापि पशुकाम  ज्योतिष्टोम  प्रयोगविशेषपुरस्कारेण विहितस्य खनित्रादेः सामान्यज्योतिष्टोमप्रयोगे विहितं न युक्तम् इति~। ज्योतिष्टोम प्रयोगपुरस्कारेण विहितं सोमबाधकं युक्तम्~। 

वस्तुतस्तु विचार्यमाणे, ‘उद्भिदा यजेत पशुकामः’ इत्यत्र प्रकृतसोमयागाश्रितखनित्रादि गुणे पशुरूपफलं भावयेत् इति गुणफलसम्बन्धविधिरेव युक्तः~। खनित्रादिगुणस्य काम्यत्वादेव नित्यसोमबाधकत्वम् इति तृतीयः पूर्वपक्षः~। 

‘उद्भिदा यजेत पशुकामः’ इत्यस्य विधेः गुणविधित्वाभ्युपगमे गौरवं सम्भवति~। तथा हि यत्र शुद्धधात्वर्थकरणकभावना विधानं तत्र आद्यविधिप्रकारकत्वम् अभ्युपगम्यते~। यथा ‘अग्निहोत्रं जुहोति’ इत्यत्र अन्योद्देशेन गुणविधिः द्वितीयः विधिप्रकारः~। यथा ‘ अग्निहोत्रं जुहुयात् स्वर्गकामः’ अस्य च विधेः उद्देश्यवाचके पदान्तरसापेक्षत्वात् प्रथमविध्यपेक्षया दौर्बल्यम्~। धात्वर्थमुद्दिश्य अन्यकरणकभावनाविधिः तृतीयः विधिप्रकारः~। यथा ‘दध्ना जुहोति’ इत्यत्र अस्मिन् विधौ विधेयत्वस्य धात्वर्थवृत्तित्वाभावात्, प्रत्युत पदान्तरवृत्तित्वात् द्वितीयविध्यपेक्षया दौर्बल्यम्~। यत्र तु गुणविशिष्टधात्वर्थकविधानं तत्र पञ्चमविधिप्रकारः आश्रियते~। यथा ‘सोमेन यजेत इत्यत्र च विधौ विशेषण - विधिकल्पना गौरवात् चतुर्थविध्यपेक्षया दौर्बल्यम्~। अन्योद्देशेन गुणविधिकरण - गुणविशिष्टधात्वर्थकरणक - भावनाविधानं यत्र तत्र शब्दे श्ब्दविधिप्रकारः~। यथा ‘सौर्यं चरुं निर्वपेत् ब्रह्मवर्चसकामः’ इति अर्थविधिः सर्वापेक्षया दुर्बलः~। प्रकृते उद्भिद्गुणस्य यागस्य पशुरूपफलोद्देशेन विधाने अष्टमविधिप्रकाराश्रयणम् अत्यन्तं गौरवम्~। किञ्च उद्भिद्गुणस्य यागे आनर्थक्यात् मत्वर्थलक्षणा सम्भवति~। अतः अन्योद्देशेन शुद्धधात्वर्थकरणकभावना विधानम् अङ्गीक्रियते~। तेन च द्वितीयविधिप्रकारः आश्रितो भवति इति लाघवम्~। 

\articleend
}
