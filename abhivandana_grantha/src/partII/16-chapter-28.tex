{\fontsize{15}{17}\selectfont
\chapter{अनुमानं बुभुत्सन्ते  तर्करसिकाः}

\begin{center}
\Authorline{वि. नरसिंहभट्टः, बडगु}
\smallskip

बेङ्गळूरु, पूर्वविद्यार्थी (१९९९-२००१)
\addrule
\end{center}

अस्माभिः किमर्थं शास्त्रम् अध्येतव्यम् । न्यायशास्त्रं विद्यास्थानेषु अन्यतमम् । विद्याः स्थीयन्ते अस्मिन् इति विद्यास्थानम् । विद्या इत्युक्ते परमात्मा । तद्विषयकज्ञानम् अपिविद्यापदेन अभिधत्ते । एवं च परब्रह्मविषयकज्ञानं अस्मिन् न्यायशास्त्रे अस्तीति एतन्न्यायशास्त्रं विद्यास्थानमितिव्यवहर्तुं शक्यते । न्यायशास्त्रस्य अध्ययनेनपरपदप्राप्तिः सूचिता । सर्वेषां शास्त्राणां प्रयोजनं विद्यामूलान्वेषनमेव । विद्यामूलं शास्त्रेणैव अधिगन्तव्यमिति न्यायशास्त्रस्यविद्यास्थानत्वं निश्चप्रचम् ।
\begin{verse}
अङ्गानि वेदाः चत्वारः मीमांसान्यायविस्तरः । \\
पुराणं धर्माशास्त्रं च विद्या ह्येताश्चतुर्दश ॥
\end{verse}
‘न्यायविस्तरः’ इतिशब्देन अत्र न्यायशात्रम् उक्तम् । एवं च न्यायशास्त्रेण शब्दस्य विस्तरः क्रियते, नतु अर्थस्य विस्तारः । अर्थं विस्तारयितुं न शक्यते । उदाहरणार्थं यदि अन्धकारं स्वीकुर्मः । अन्धकारः न कदापि व्यत्यस्तो भवति । परन्तु स अन्धकारः ध्वान्तः, गाढः, तमसः तिमिरः इत्यादिना नापदैः परिचितोभवति । अतः अन्धकार एक एव । वस्तुतत्त्वज्ञानं नानापदविन्यासैः परिचयितुं शक्यते इति न्यायशास्त्रस्य न्यायविस्तर इति पदप्रयोगः विहितः । न्यायशास्त्रं प्रमाणशास्त्रम् इति पदेनापि अभिधीयते ।  न्यायशास्त्रस्य प्रणेता भगवान् गौतमः एवं सूत्राणि आरभत- “प्रमाण-प्रमेय-संशय-प्रयोजन-दृष्टान्त- सिद्धान्त-अवयव-तर्क-निर्णय-वाद-जल्प-वितण्ड-हेत्वाभास-छल-जाति-निग्रहस्थानां तत्त्वज्ञानान्निःश्रेयसाधिगमः” इति । प्रामाणादिषोडशपदार्थज्ञानेन विद्यामधिगन्तुं शक्यते । प्रमाणमपि तत्त्वज्ञानविषयः । प्रमाणम् इति पदं द्विधा व्युत्पादयितुं प्रभवाम । ‘प्रमीयते इति प्रामाणम्’ भावे ल्युट्प्रत्ययेन प्रमाणम् एकम् तत्त्वम् । ‘प्रमीयते अनेन इति प्रमाणम्’ इति करणे ल्युट्प्रत्ययेन प्रमाणपदं तत्त्वज्ञानसाधनम् । प्रमाणं साध्यसाधनोभयपरम् ।

‘प्रत्यक्ष-अनुमान-उपमान-शब्दाःप्रमाणानि’ इति चत्वारि प्रमाणानि । षोडशपदार्थानां सप्तपदार्थेषु अन्तर्भावः उक्तः । सप्तपदार्थेषु प्रमाणं बुद्धिगुणान्तर्भूतम् । ‘मानाधीना मेयसिद्धिः’ इति प्रतीत्याप्रमाणैः प्रमाणादि षोडशपदार्थानां सप्तपदार्थानाम् अधिगमनं प्रमाणानां परमोद्देश्यम् ।

प्रत्यक्षम् एकम् प्रमाणम् । “इन्द्रियार्थसन्निकर्शोत्पन्नम् अव्यपदेश्यम् अव्यभिचारि व्यवसायात्मकं प्रत्यक्षम्”। इन्द्रियम् अर्थयोः सन्निकर्षः-सम्बन्धः प्रत्यक्षे कारणम् । ‘इदं जलम्’ इत्यत्र सन्निहिते जले च क्षुरिन्द्रियसम्बद्धेजलगतरूपस्य, रसनासन्निकृष्टे जलगतरसस्य, त्वक्सन्निकर्षे जलगतस्पर्शस्य च ज्ञानं भवति । प्रत्यक्षे इन्द्रियस्य सम्बन्धः प्राप्यकारित्वम् अप्राप्यकारित्वम् इति द्विविधः । चक्षुः अर्थं प्रतिगच्छति । इदं पराप्यकारित्वम् । अर्थः रसनेन्द्रियं प्राप्नोति । इन्द्रियं न गच्छति । अथापि प्रत्यक्षं भवति । एवं च द्विधा प्रत्यक्षं भवति । अपि च संयोगः, संयुक्तसमवायः, संयुक्तसमवेतसमवायः, समवायः, समवेतसमवायः, विशेष्यविशेषणभावरूपः षड्विधः सन्निकर्षः ।

यद्यपि प्रत्यक्षज्ञानं न सुकरम् । अत एव वदति जगदीशभट्टाचार्यः “प्रत्यक्षपरिकलितमप्यर्थम् ८] = अनुमानेन बुभुत्सन्ते तर्करसिकाः’ इति । कन्नडभाषायामपि एषा उक्तिः प्रसिद्धा अस्ति यत् \kan{‘ಪ್ರತ್ಯಕ್ಷಕಂಡರೂ ಪರಾಂಬರಿಸಿ ನೋಡು’} इति । अत्र प्रमाणं नाम अनुमानमेव । अनुमानस्यैव नैयायिकैः प्रामाणत्वम् अभ्युपगम्यते । यतोहि प्रत्यक्षे दोषस्य सम्भवः अर्थगतः इन्द्रियगतः अथवा सन्निकर्षगतो वा । महत्त्व-अणुत्वादिदोषैः कदाचित् न प्रत्यक्षम् । इन्द्रियेकदाचित् पित्तकामालादिदोषैः न प्रत्यक्षम् । अपि च दूरत्वसमीपत्वादिदोषैरपिप्रत्यक्षं न स्यात् । परन्तु अनुमाने एतादृशदोषाणां सम्भाव्यत्वं नास्ति । अतः अनुमानं प्रमाणेषु अभ्यर्हिततमम् इति मन्वते प्रामाणिकाः । अनुमानेनैव न्यायशास्त्रस्य शास्त्रत्वं प्रमाणितम् । अणोःअणुत्वात् न प्रत्यक्षम् । आकाशस्यपरममहत्वात् न प्रत्यक्षम् । अपि तु अणोःआकाशस्य च प्रमेयत्वम् अस्ति । प्रमेयत्वं च अनेनानुमानेनैव सिद्ध्यति । एवं च इन्द्रियाजन्यज्ञाने अनुमानस्य आवश्यकताप्रतिपादिता । परन्तु अनुमानं प्रत्यक्षपूर्वम् । प्रत्यक्षम् अनुसृत्य एवमानम् अनुमानम् । प्रत्यक्षे दोषस्य सद्भावेऽपि अनुमाने व्याप्तिपक्षधर्मतापरामर्शादीनां अपेक्षता नैव परिहरणीया । एवं च अनुमानं तर्करसिकानां स्वादुरसत्वम् अभ्युपगन्तव्यम् । 

\articleend
}
