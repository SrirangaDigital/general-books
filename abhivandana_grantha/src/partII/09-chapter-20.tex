{\fontsize{15}{17}\selectfont
\chapter{अक्षरसमाम्नायः}

\begin{center}
\Authorline{डा. मञ्जुनाथ हेगडे}
\smallskip

सहायक-प्राध्यापकः व्याकरणशास्त्रम्\\
एस्,के,एस्,वि,ए, महाविद्यालयः\\
आदिचुञ्चनगिरि क्षेत्रम् 
\addrule
\end{center}

\begin{verse}
योगेन चित्तस्य पदेन वाचां मलं शरीरस्य च वैद्यकेन ।\\
योपाकरोत्तं प्रवरं मुनीनां पतञ्जलिं प्राञ्जलिरानतोऽस्मि ॥
\end{verse}
अथ परमकारुणिको लोकहितैषी सच्चिदानन्दघनः सर्वज्ञः सर्वशास्त्रयोनिः सत्यसङ्कल्पः सकलजगदाधारी भगवान् आधिदैविकाध्यात्मिकाधिभौतिकदुःखैर्दुःखितानां जनानाम् उद्धाराय धर्मादिस्थापनाय च कृतविधावतारः सकलदुःखोपशमनाय यदृच्छया ब्रह्माणं सृष्ट्वा तस्मै तावत् कर्मोपासन-ब्रह्मैक्यज्ञानयाथार्थ्यप्रतिपादकान् आदर्शस्थानीयान् सकलविद्योपबृंहितान् पूर्वपूर्वयुगस्थान् ऋग्यजुस्सामाथर्वाख्यान्षडङ्गकांश्चतुरोवेदान् उपदिदेश । “यो वै ब्रह्माणं विदधाति पूर्वं यो वै वेदांश्च प्रहिणोति तस्मै” इति श्वेताश्वतरोपनिषच्छ्रुतेः । ब्रह्मापि तदुपदेशक्रमेणैव जगत्सृष्ट्वा वेदोक्तान्येव नामानि तज्जात्यनुगुणानि निर्माय प्रजाभ्य उपदिदेश । तत्रेयं स्मृतिः -		
\begin{verse}
सर्वेषां स तु नामानि कर्माणि च पृथक् पृथक् ।\\
वेदशब्देभ्य एवासौ पृथक् संस्थाश्च निर्ममे ॥ इति
\end{verse} 		
तत्र “मुखं व्याकरणं स्मृतम्” इति मुखस्थानीयं वेदाङ्गं व्याकरणं तावदधिकृत्य विचारयामः । तत्र व्याकरणं नाम  - अर्थविशेषमाश्रित्य स्वरप्रकृतिप्रत्ययादीन् विशेषेण संस्कारविशेषेण आसमन्तात्  वैदिकान् लौकिकांश्च शुद्धान् साधुशब्दान् व्युत्पादयतीति सर्वविदितमेव ।

तत्र तावत् वैयाकरणसिद्धान्तान् निरूपयिष्यन् व्याकरणशास्त्रस्य मूलभूतानि चतुर्दश सूत्राणि प्राह - अइउण् इत्यारभ्य हल् पर्यन्तं यावत् । किन्तु इमानि सूत्राणि मुनित्रयग्रन्थबहिर्भूतानीति कृत्वा तेषाम् अप्रामाण्यमित्यत आह - “माहेश्वराणि सूत्राणीति” । महेश्वरादागतानि इत्यर्थे तत आगतः इत्यणि कृते अनुबन्धलोपादि कार्ये जाते माहेश्वराणि महेश्वरादधिगतानीति यावत् । एतस्मादेव कारणात् एतेषां प्रामाण्यं नास्तीति प्रतिपाद्यमान पक्ष एव निरास्तः ।

‘अथापि अनर्थकवर्णराश्यात्मकानाम् एतेषां सूत्राणां वैयाकरणसिद्धान्तप्रकाशने प्रयोजनाभावात् तदिह उपन्यासो व्यर्थ इत्यत आह - अणादि संज्ञार्थानीति । अण् आदिः  यासां ताः अणादयः, अणादयश्च ताः संज्ञाः अणादिसंज्ञाः, अणादिसंज्ञा एव अर्थाः  प्रयोजनं येषां तानि अणादिसंज्ञार्थानीति सिद्धम् । 

अनर्थकवर्णराशित्वेऽपि शास्त्रगतव्यवहारभूत-अणादि संज्ञासु उपयोगसत्वान्न आनर्थक्यं सूत्राणामिति सिद्धान्तः। एतेषामेव वर्णानां यन्मूलं वर्तते तस्यैव “अक्षरम्” इति अन्वर्थं सार्थकं नामधेयं व्याकरणशास्त्रे दृश्यते । इदानीं जिज्ञासा किमिदम् अक्षरमिति ? किं तस्य प्रयोजनम् ? को वा विशेषः ? किमर्थम् अक्षराणाम् उपदेशः ? कुत्र वा प्रयोजनम् ? इति चेदाह भाष्यकरः स्वीयमहाभाष्ये - 		
\begin{verse}
अक्षरं न क्षरं विद्यात्  अश्नोतेर्वा सरोऽक्षरम् ।\\
वर्णं वाहुः पूर्वसूत्रे किमर्थम् उपदिश्यते ॥ इति
\end{verse}
न क्षीयते न क्षरति इति वा अक्षरम् । तत्र अक्षरपदस्य व्युत्पत्तिस्तावत् त्रिधा भिद्यते यथा - क्षयार्थक क्षी धातोः डरच् प्रत्ययेन सिद्धमिदं ‘क्षर’ पदं नञर्थेषु षट्सु अभावार्थे नञा सह समासेन अक्षरमिति रूपमाप्नोतीत्येके । अपरे तु सञ्चलनार्थक क्षर धातोः पचाद्यचि सिद्धमिदं क्षरपदं न क्षीयते नक्षरति इत्यर्थे नञा सह समासेन अक्षरत्वमाप्नोति इत्यामनन्ति । अस्यैव अक्षरस्य  प्रकारान्तरेण व्युत्पत्तिं संसाधयन्ति सन्तो यथा - अर्थम् अश्नुते व्याप्नोति इत्यर्थे व्याप्त्यर्थक अशू धातोः औणादिक सरन् प्रत्ययेन सिद्धमिदम् अक्षरपदमिति ।
\begin{verse}
अक्षरं प्रणवे धर्मे प्रकृतौ तपसि क्रतौ ।\\
वर्णे मोक्षे च नात्वेष शिवविष्णुविरञ्चिषु ॥ इति
\end{verse}
नानार्थरत्नमालायाम् अर्थान्तरेषु विद्यमान-अक्षरस्य विवरणं विविधार्थेषु दृश्यते । एतस्यैव अक्षरस्य व्याकरणान्तरे पूर्वव्याकरणे वर्ण इति संज्ञा । तदाह भाष्ये - “वर्णं वाहुः पूर्वसूत्रे” इति । एवं तर्हि वर्णाक्षरयोरस्ति कश्चिद्भेदः उत न इत्याकांक्षायामाह - वर्ण्यते इत्यर्थे प्रेरणार्थक वर्ण् धातोः घञि वर्णपदसिद्धिः । पक्षे विस्तारार्थक/विन्यासार्थक वर्ण धातोः अचि सिद्ध वर्णपदस्य अक्षरविन्यास एव अर्थः । इत्थञ्च सर्वेषामपि वर्णानां मूलन्तावत् अक्षर एव । मूलभूतस्य अक्षरस्य विन्यसरूप वर्णनं तु वर्णेन क्रियत इति तात्पर्यम् । तदाह कोशकारः स्वीय-अमरकोशे -
\begin{verse}
वर्णो द्विजादौ शुक्लादौ यज्ञे गुण कथासु च ।\\
स्तुतौ ना नस्त्रियां भेद रूपाक्षर विलेपने ॥  इति
\end{verse}
श्लोकवार्तिकस्य पादत्रये व्याख्याते चतुर्थपादेन तदुत्तरश्लोकवार्तिकस्य अवतारणाय प्रयोजनप्रश्नं करोति - किमर्थमुपदिश्यत इति । वर्णानाम् उपदेशः किमर्थम् ? इति प्रश्नः ।

तदाह भाष्यकारः 		
\begin{verse}
वर्णज्ञानं वाग्विषयः यत्र च ब्रह्म वर्तते । \\
तदर्थम् इष्टबुध्यर्थं लघ्वर्थञ्चोपदिश्यते ॥ इति
\end{verse}

\section*{ब्रह्मज्ञानार्थं वर्णानामुपदेशः}	

सोऽयम् अक्षरसमाम्नायः वाक् समाम्नायः पुष्पितः फलितः चन्द्रतारकवत्प्रतिमण्डितः वेदितव्यो ब्रह्मराशिः । सर्ववेदपारायणपुण्यफलावाप्तिश्च अक्षरसमाम्नायस्य ज्ञानेन भवति । मातापितरौ चास्य स्वर्गे लोके महीयेते इति परमार्थः । वर्णाः येन शास्त्रेण ज्ञायन्ते तद्वाचः विषयः । ब्रह्म तावत् वेद एव तच्च पदे वर्तते । लौकिक - वैदिकानां पदानान्तावत्  शास्त्रं विषयः । तदर्थं- शास्त्रज्ञानार्थं तन्मुखेन ब्रह्मतत्व अधिगन्तुं वर्णानामुपदेशः ।	

इष्टबुध्यर्थं वर्णानामुपदेशः -

कला, ध्मात, एणीकृत अम्बूकृतादि दोषरहित वर्णज्ञानपूर्वक-वाग्व्यवहारार्थं वर्णानामुपदेशः।

लघ्वर्थञ्च वर्णानामुपदेशः -	

अनुबन्धकरणार्थरूप प्रत्याहारद्वारा लाघवेन शास्त्रज्ञानार्थं वर्णानामुपदेशः कृतः । एतावद्भिरेव वर्णैः वाग्व्यवहार इति निश्चयः । एवञ्च अक्षरसमाम्नाय इत्यस्य श्रुतिरूप वर्णसङ्घात इत्यत्र तात्पर्यम् । तस्मादेव पुष्पितः - फलितः इत्यस्य दृष्टफलेन अभ्युदयेन पुष्पितः, अदृष्टेन निःश्रेयसेन  वा फलितः वर्णसमाम्नायः वाक्समाम्नायः सर्ववाग्रूपत्वात् सर्वमूलत्वाच्च तदध्ययनेन पुष्पितः, तदर्थज्ञानरूप ब्रह्मज्ञानेन फलित इति सिद्धान्तः ।

चन्द्र तारकवत्प्रतिमण्डितः  - इत्यनेन अनादिनिदनत्वं वर्णानां सूचितम् । तादृशोऽयं ब्रह्मराशिः वर्णराशिः अक्षरसमाम्नायः सर्वैर्वेदितव्य एव । यतो हि अस्य ज्ञानेन चित्तशुद्धिर्भवति । तेन सर्ववेदपारायणपुण्यफलावाप्तिश्च भवति । एवमेतादृशज्ञानसम्पन्न- पुत्रलाभेन मातापित्रोः इहलोक-परलोकयोर्गौरवञ्च सिध्यति । तद्यथा - युधिष्ठिरकृत राजसूयेन यागेन पाण्डोर्यथा स्वर्गेलोके पूजा प्राप्ता तथैवेति लौकिक दृष्टान्तः । इत्थञ्च सम्यगधीत-सम्यग्ज्ञात-सुप्रयुक्त-अक्षरसमाम्नायेनैव सर्वार्थसिद्धिः सर्वमूलत्वादिति शम् ।

\articleend
}
