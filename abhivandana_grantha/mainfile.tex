\documentclass[12pt,twoside,openany]{book}

\input macros

\begin{document}

\frontmatter
\cfoot[{\eng{\thepage}}]{{\eng{\thepage}}}

\selectlanguage{kannada}

\input src/titlepage

\newpage

\selectlanguage{english}
\input src/copyright
\newpage

\selectlanguage{kannada}

\input src/about

\lhead[\small\eng{\thepage}]{\small\kan{\leftmark}}
\rhead[{\small\kan{ಜ್ಞಾನಗಂಗಾಧರ}}]{\small\eng{\thepage}}
\chead[]{}
\cfoot[]{}


\renewcommand{\headrulewidth}{0.6pt}
\input src/partI/rajamata
\newpage

\tableofcontents
\clearpage

\selectlanguage{kannada}
\lhead[\small\eng{\thepage}]{\small\leftmark}

\input src/partI/editorial
\input src/partI/05-chapter-02.tex
\input src/partI/06-chapter-88.tex


\selectlanguage{kannada}
\part{ಚಿತ್ರ ಸಂಪುಟ}

\input src/partI/abhivandana_samiti
\input src/partI/invitation
\input src/partI/news_paper_cuttings
\input src/partI/varadi
\newpage

\input src/partI/01-chapter-01.tex
\clearpage

\input src/photos.tex

\input dictionary

\mainmatter

\selectlanguage{sanskrit}
\lhead[\small\eng{\thepage}]{\small\leftmark}
\part{शास्त्रसम्बन्धीनि लेखनानि}

\input src/partII/01-chapter-24.tex
\input src/partII/02-chapter-27.tex
\input src/partII/03-chapter-15.tex
\input src/partII/04-chapter-16.tex
\input src/partII/05-chapter-10.tex
\input src/partII/06-chapter-34.tex
\input src/partII/07-chapter-11.tex
\input src/partII/08-chapter-32.tex
\input src/partII/09-chapter-20.tex
\input src/partII/10-chapter-13.tex
\input src/partII/12-chapter-83.tex
\input src/partII/13-chapter-21.tex
\input src/partII/14-chapter-22.tex
\input src/partII/15-chapter-25.tex
\input src/partII/16-chapter-28.tex
\input src/partII/17-chapter-85.tex
\input src/partII/18-chapter-38.tex
\input src/partII/19-chapter-29.tex
\input src/partII/20-chapter-30.tex
\input src/partII/21-chapter-86.tex
\input src/partII/22-chapter-37.tex
\input src/partII/23-chapter-18.tex
\input src/partII/24-chapter-26.tex
\input src/partII/25-chapter-12.tex
\input src/partII/26-chapter-31.tex
\input src/partII/28-chapter-19.tex
\input src/partII/29-chapter-33.tex
\input src/partII/30-chapter-80.tex
\input src/partII/31-chapter-35.tex
\input src/partII/32-chapter-09.tex
\input src/partII/33-chapter-82.tex
\input src/partII/34-chapter-78.tex
\input src/partII/35-chapter-14.tex
\input src/partII/36-chapter-08.tex
\input src/partII/37-chapter-79.tex
\input src/partII/39-chapter-23.tex
\input src/partII/40-chapter-17.tex
\input src/partII/41-chapter-39.tex
\input src/partII/42-chapter-36.tex

\selectlanguage{kannada}

\input src/partII/11-chapter-41.tex
\input src/partII/27-chapter-40.tex

\makeatletter

\renewcommand\section{\@startsection {section}{1}{\z@}%
                                   {-2.2ex \@plus -1.3ex \@minus -.3ex}%
                                   {1ex \@plus.2ex}%
                                   {\kannadafont\Large\bfseries}}
\def\parttext{ ಶ್ರೀಯುತ ಗಂಗಾಧರ ಭಟ್ಟರ ಒಡನಾಡಿಗಳ ಒಳನುಡಿ}
\makeatother

\selectlanguage{kannada}

\part{ಒಳನುಡಿ ಸಂಪುಟ}

\input src/partIII/01-chapter-42.tex %%ನನ್ನ ಸೂರು ಮೈಸೂರು

\selectlanguage{sanskrit}

\input src/partIII/02-chapter-04.tex %%गङ्गाधरं वीक्ष्य विदन्तु सर्वे
\input src/partIII/03-chapter-03.tex %%गङ्गाधरो भट्टवरो विराजताम् 
\input src/partIII/04-chapter-05.tex %%गङ्गाधरसुधीरत्र हव्यकव्योमभास्करः
\input src/partIII/05-chapter-06.tex %%गङ्गाधरवन्दनम्
\input src/partIII/06-chapter-07.tex %%अभिनन्दवाचः
\input src/partIII/39-chapter-87.tex %%प्रशस्यः प्राध्यापकाः

\clearpage

\selectlanguage{kannada}

\input src/partIII/07-chapter-43.tex %%ನನ್ನ  ಸಹೋದರ \enginline{-} ನಾನು ಕಂಡಂತೆ
\input src/partIII/08-chapter-44.tex %%ಪ್ರಿಯ ಅಣ್ಣ   \enginline{-}   ಗಂಗಣ್
\input src/partIII/09-chapter-46.tex %%ಗಂಗಾಧರ ನಮ್ಮ ಕುಟುಂಬದ ಹೆಮ್ಮೆ
\input src/partIII/10a-chapter-45.tex %%ಸನ್ಮಿತ್ರನಾದ ಗಂಗಾಧರಣ್ಣ
\input src/partIII/30-chapter-57.tex %% ನಾ ಕಂಡಂತೆ ಗಂಗಾಧರ
\input src/partIII/25-chapter-56.tex %%ಸಜ್ಜನಪ್ರಿಯ ಗಂಗಾಧರ ಭಟ್ಟರು
\input src/partIII/22-chapter-58.tex %%ನೆನಪು ಹಳತಾಗಿಲ್
\input src/partIII/20-chapter-50.tex %%ನಮ್ಮ ಪುರೋಹಿತರು   \enginline{-}   ನಮ್ಮ ಹೆಮ್ಮೆ
\input src/partIII/11a-chapter-49.tex %%ವಿದ್ವತ್ತಿನ ಗಣಿ ಆತ್ಮೀಯ ಮಿತ್ರ
\input src/partIII/11b-chapter-47.tex %%ನಂದತಿ ನಂದತಿ ನಂದತ್ಯೇವ
\input src/partIII/11c-chapter-84.tex %%ವೇ॥ಬ್ರ॥ಶ್ರೀ ವಿದ್ವಾನ್ ಗಂಗಾಧರ ಭಟ್ಟರು
\input src/partIII/18-chapter-61.tex %%ನಮ್ಮ ನಲ್ಮೆಯ ಶ್ರೀಗಂಗಾಧರ ಭಟ್ಟರು
\input src/partIII/29-chapter-91.tex %%ನಾ ಕಂಡ ನನ್ನ ಸಹವರ್ತಿ ಗಂಗಾಧರ
\input src/partIII/19-chapter-55.tex %%ನಯ ವಿನಯ ಸಂಪನ್ನ ವಿ~॥ ಶ್ರೀಗಂಗಾಧರ ಭಟ್
\input src/partIII/23-chapter-60.tex %%ವಿದ್ವಾಂಸರಲ್ಲೊಬ್ಬ ಅಪರಂಜಿ
\input src/partIII/17-chapter-53.tex %%ಅಪರೂಪದ ಪ್ರಭೇದ ವಿ~॥ ಗಂಗಾಧರ ಭಟ್ಟರು
\input src/partIII/26-chapter-62.tex %%ಗುರವೋ ವಿರಲಾಸ್ಸಂತಿ ಶಿಷ್ಯ ಚಿತ್ತಾಪಹಾರಕಾಃ
\input src/partIII/15-chapter-52.tex %%ವಿ~। ಗಂಗಾಧರಭಟ್ಟರೊಡನಾಟದ ರಸನಿಮಿಷಗಳು
\input src/partIII/16-chapter-54.tex %%ವಿ~॥ ಗಂಗಾಧರ ಭಟ್ಟರು ನಾ ಕಂಡಂತೆ
\input src/partIII/13-chapter-48.tex %%ಹಿತೈಷೀ  \enginline{-}  ಮನೀಷೀ
\input src/partIII/21-chapter-63.tex %%ಗುರುಭ್ಯೋ ನಮಃ
\input src/partIII/28-chapter-90.tex %%ಅಪ್ರತಿಮ ಜ್ಞಾಪಕ ಶಕ್ತಿಯ ವಿ~॥ ಗಂಗಾಧರ ಭಟ್ಟರು
\input src/partIII/12-chapter-64.tex %%ರತ್ನಾಕರೋಽಯಂ ಗಂಗಾಧರಃ
\input src/partIII/14-chapter-51.tex %%ಆಡದೇ ಮಾಡಿದವರು
\input src/partIII/37-chapter-59.tex %%ಗುರುವಿನೊಡನೆ ಸಂಸ್ಕೃತದೆಡೆಗೆ
\input src/partIII/27b-chapter-66.tex %%ಗುರುಪ್ರಸಾದ
\input src/partIII/35a-chapter-70.tex %%ಶಿಷ್ಯರ ಮನಗೆದ್ದ ಸರಳಸ್ವಭಾವದ ವಿ~॥ ಗಂಗಾಧರ ಭಟ್ಟರು
\input src/partIII/31-chapter-67.tex %%ಉತ್ಸಾಹದ ಚಿಲುಮೆ ಗಂಗಾಧರ ಭಟ್ಟರು
\input src/partIII/27-chapter-66.tex %%ನಮ್ಮ ಹೆಮ್ಮೆ ನಮ್ಮ ಗಂಗಾಧರಭಟ್ಟರು
\input src/partIII/24-chapter-69.tex %%ಗಂಗಾಧರ ಭಟ್ಟರು ನಾ ಕಂಡಂತೆ...
\input src/partIII/33-chapter-72.tex %%ಆಚಾರ್ಯ ಗಂಗಾಧರಭಟ್ಟರು
\input src/partIII/36-chapter-68.tex %%ಪರೋಪಕಾರಿ ಗುರುಗಳು
\input src/partIII/34-chapter-71.tex %%ಗುರುಗುಣ ಅನಾವರಣ
\input src/partIII/38-chapter-89.tex %%ಭರತ ಮಾತೆ ನಲಿಯಲಿ
\input src/partIII/32-chapter-81.tex %%ಗುರುರೇವ ಜಗತ್ಸರ್ವಮ್
\input src/partIII/35b-chapter-65.tex %%ಗುರುಜ್ಞಾನ ಗಂಗಾಧರರು

\selectlanguage{sanskrit}
\input src/partIII/40-chapter-77.tex %%तर्कातीततर्कशात्री वि~। गंङ्गाधर भट्ट


\selectlanguage{english}

\lhead[\small\thepage]{\small\eng{\leftmark}}

\input src/partIII/42-chapter-74.tex %%Pujya Sri Gangadhar Bhat – A Walking Encyclopedia
\input src/partIII/43-chapter-75.tex %%Gem Of A Teacher
\input src/partIII/44-chapter-76.tex %%A Great Teacher Sri Gangadhara Bhat
\input src/partIII/41-chapter-73.tex %%Top Ten Qualities of My Uncle and Teacher
\label{bookend}
\end{document}

