\chapter{5G Need of Hour to Support Industry 4.0 Evolution and New Applications}

\begin{center}
{\large\uppercase{Dr. Debarbata Das}} 

\vskip -6pt

Professor,  Networking and Communication Research Lab., IIIT-Bangalore, India.
\end{center}

\vfill
\eject

\begin{multicols}{2}
Past couple of decades has seen tremendous growth in wireless technologies with successful deployment of 2G to 4G systems. It is expected that by 2020 there would be around 50 billion connected devices which will include Machine Type Communication (MTC), Internet of Things (IoT), ultra-high definition video etc. along with high mobility. Various types of application ranging from remote healthcare, education and, gaming and other enterprise industrial 4.0 IoT applications which will require enhanced mobile broadband and ultra-low reliable low latency communication systems. This is what essentially defines the broad 5G requirements, such as, 10x increase in current link level capacity, latency below 1ms for robotics applications, with increased battery life, and enhanced security for users. Meeting all these requirements are extremely challenging.  Consequently, innovation of several new technologies are needed to achieve these.

From radio interface perspective, 5G is having a new radio (5G NR) access technology based on millimeter wave (mmWave). Since mmWave has shorter wavelengths they allow designing smaller antenna array elements. Many such antenna elements can now be used for beamforming, directed towards a particular user to avoid interference with others. However, mmWave suffer high path losses hence the cell sizes are likely to be small. This leads to the case of dense deployment of cells. Connecting large number of these base-stations to the back bone network becomes a significant challenge. 5G systems envisage virtualization at all levels starting from radio access network (RAN) to the core network (CN) components for efficient usage and provisioning of network resources dynamically on demand. Thus, for access network the base-band processing are supposed to happen in cloud and the radio signals to be transmitted to light weight remote radio unit/head (RRU/RRH). Due to densification of networks, large number of RRHs has to be connected to the base-band units (BBU) which is often referred to as fronthaul. This fronthauling becomes a problem due to scarcity of fiber deployment. Hence, integrated access bearer (IAB) has been proposed to allow using the radio access technology meant for users to be used for fronthauling. Another important radio access technology which is important especially from IoT perspective is Non-orthogonal multi access (NOMA). This mechanism allows different users to access the same sub-band simultaneously under certain constraints, e.g., users are spatially separated with respect to base-station. Using NOMA, large number of IoT devices (massive IoT) can transmit/receive data over scare radio resources.

5G intention is to support wide range of applications catering primarily to three categories of enhanced bandwidth, massive IoT and ultra-reliable low latency communication. Each of these has different Quality of Service (QoS) and Quality of Experience (QoE) requirements. Hence, it is essential to partition the network resources both at access network and core network to meet these different QoS/QoE requirements and configure dedicated network slices for different types of applications pertaining to the different categories. Once, these network slices are configured they need to be managed and maintained based on dynamic network conditions. Another objective of creating these slices is to allow virtualized network functions (VNF) to be created for mobile virtual network operators (MVNO) to play a bigger role and create compelling business opportunities.

Often for applications which require low latency and are highly interactive, sending data for processing to the cloud far away from the users may not be a viable option. In such cases, edge computing where processing capabilities are available close to the users (somewhere near the base stations) can help mitigate the latency issues. Edge computing servers will be able to cache certain amount of content, various functions of protocols and applications, which can be accessed by the users quickly. 

As already mentioned above, 5G systems intend to use extensive use of virtualization at various levels to enable efficiently utilization of network and computational resources and also provide flexibility to provide various types of services. This includes virtualizing network components and also virtualizing network resources on demand provisioned based on application requirements. Network function virtualization (NFV) will not only provide flexibility in provisioning but also create a platform to deploy new type of services which is sometimes referred to as everything-as-a-service (XaaS) following service oriented architecture (SOA) approach.

With the vision to provide flexible and dynamic provisioning of different services and applications in cloud based 5G, network orchestration and management becomes a major challenge. This involves setting up of QoS profiles based on application requests and map them to network slices. In this respect, software defined networking (SDN) based approach of separating control and data plane, with a well-defined interface between them, can easy creating and maintaining data flows corresponding to network slices. SDN controllers (being the brain) can configure data flows across various wired and wireless domains on a lean data plane to achieve the QoS requirements of the applications. 

Overall functioning of such complex 5G systems will be impossible to manage manually. For optimal management of networks, self-configuration, self-healing functions and self-organization and other network orchestration functions, 5G systems has to rely on artificial intelligence and machine learning approaches, so that the network can learn over time and perform optimally meeting various constraints.

With the requirement of dense network deployment and edge computing and cloud based radio access and core networks and high QoS requirements, power consumption is obviously going to increase both at network side as well as on the device. Hence, significant research effort is expected to provide green solutions.

Also, data security is likely to play a significant role in 5G networks. Existing privacy concerns with respect to mobile social networks, secure enterprise transactions, denial of service attacks, intrusion detection, insolation of the compromised network components and recovery needs to be addressed in 5G and beyond systems.

Looking at the above topics, it is obvious that, the 5G systems are going to extremely complex with lots of challenges to be addressed and can have significant impact in our lives. This provides ample opportunities for research both at academics and in industry.

\end{multicols}
  






