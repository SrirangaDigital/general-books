\chapter{Āyurveda-lakṣyam, Vyādhiḥ}

\begin{center}
Goal of Āyurveda, Illness
\end{center}
\begin{enumerate}
\item \dev{ स्वस्थस्य स्वास्थ्यरक्षणम्, आतुरस्य विकार-प्रशमनम् आयुर्वेदस्य लक्ष्यम्।}

svasthasya svāsthyarakṣaṇam, āturasya vikāra-praśamanam āyurvedasya lakṣyam |             

Protecting the health of the healthy and removing the illness of the person with disease is the aim of Āyurveda.

\item \dev{ऋतुसन्धिषु व्याधयः जायन्ते ।}

ṛtusandhiṣu vyādhayaḥ jāyante |

In the junctions of seasons diseases manifest.

\item \dev{ जितेन्द्रियं रोगाः न अनुपतन्ति।	}

jitendriyaṃ rogāḥ na anupatanti |	

A person with sense-control will not have diseases. 
\end{enumerate}

\begin{center}
\textbf{\large Exercise 1}
\end{center}

Find the āyurvedic terms for -
\begin{enumerate}
\renewcommand{\theenumi}{\alph{enumi}}
\renewcommand{\labelenumi}{\theenumi.}
\item Health 
\item A person with disease 
\item Junction of seasons 
\item Disease 
\item A person with sense-control 
\end{enumerate}

\chapter{vāta-pitta-kapha-vaiṣamyam}
\begin{center}
Vitiation of vāta, pitta and Kapha
\end{center}
\begin{enumerate}
\item \dev{ वायु-वृद्ध्या कार्श्यं कम्पः च भवति।}

vāyu-vṛddhyā kārśyaṃ kampaḥ ca bhavati |

Excess Vayu leads to leanness and trembling in the limbs.

\item \dev{ कफ-वृद्ध्या श्वासः, कासः, आलस्यं च भवति।}

kapha-vṛddhyā śvāsaḥ, kāsaḥ, ālasyaṃ ca bhavati |

Excess kapha leads to Asthama, cough and laziness

\item \dev{पित्त-वृद्ध्या दाहः, अल्पनिद्रता च भवति।}

pitta-vṛddhyā dāhaḥ, alpanidratā ca bhavati |  

Excess pitta leads to burning sensation, lack of sleep etc.
\end{enumerate}

\begin{center}
\textbf{\large Exercise 2}
\end{center}

Find the āyurvedic terms for -
\begin{enumerate}
\renewcommand{\theenumi}{\alph{enumi}}
\renewcommand{\labelenumi}{\theenumi.}
\item Leanness
\item Trembling
\item Cough 
\item Laziness 
\item Lack of sleep 
\end{enumerate}

\chapter{jvara-saṅkramaṇam - Spreading of fever}

\begin{enumerate}
\item \dev{ज्वरः गात्रसंस्पर्शनात् सङ्क्रामति।}

jvaraḥ gātrasaṃsparśanāt saṅkrāmati |

Fever spreads by physical contact.

\item \dev{ ज्वरः निश्वासात् सङ्क्रामति।}

jvaraḥ niśvāsāt saṅkrāmati |

Fever spreads by (exhalation of) breath. 

\item \dev{ ज्वरः सहभोजनात् सङ्क्रामति।}

jvaraḥ sahabhojanāt saṅkrāmati |

Fever spreads by eating together.
\end{enumerate}

\begin{center}
\textbf{Exercise 3}
\end{center}

Find the āyurvedic terms for –
\begin{enumerate}
\renewcommand{\theenumi}{\alph{enumi}}
\renewcommand{\labelenumi}{\theenumi.}
\item fever
\item Physical contact 
\item (Exhalation of) breath
\item Eating together 
\item Spreads 
\end{enumerate}

\chapter{paramauṣadham}

\begin{center}
\textbf{The Best Medicine}
\end{center}

\begin{enumerate}
\item \dev{वात-प्रकोपस्य परमौषधं तैलम् ।}

Vāta-prakopasya paramauṣadhaṃ tailam |

Best medicine for vitiation of Vata is Oil.

\item \dev{पित्त-प्रकोपस्य परमौषधं घृतम्।}

Pitta-prakopasya paramauṣadhaṃ ghṛtam |

Best Medicine for vitiation of Pitta is Ghee.

\item \dev{कफ-प्रकोपस्य परमौषधं मधु।}

Kapha-prakopasya paramauṣadhaṃ madhu |

Best Medicine for vitiation of Kapha is Honey. 
\end{enumerate}

\begin{center}
\noindent\textbf{Exercise 4}
\end{center}

Find the āyurvedic terms for –
\begin{enumerate}
\renewcommand{\theenumi}{\alph{enumi}}
\renewcommand{\labelenumi}{\theenumi.}
\item Best medicine 
\item Oil
\item Ghee 
\item Honey
\end{enumerate}

\chapter{bheṣajaṃ vāri - Water as a medicine}

\begin{enumerate}
\item \dev{ अजीर्णे भेषजं वारि।}

ajīrṇe bheṣajaṃ vāri |

Water(vāri) is medicine for indigestion.

\item \dev{ जीर्णे वारि बलप्रदम्।}

jīrṇe vāri balapradam |

Consuming water Water consumption after digestion gives strength.

\item \dev{भोजने च अमृतं वारि।}

bhojane ca amṛtaṃ vāri | 

(Limited Quantity of) Water during food intake is elixer. 
\end{enumerate}

\begin{center}
\textbf{\large Exercise 5}
\end{center}

Find the āyurvedic terms for –
\begin{enumerate}
\renewcommand{\theenumi}{\alph{enumi}}
\renewcommand{\labelenumi}{\theenumi.}
\item Water
\item Indigestion
\item Medicine
\item During food intake
\item Elixer
\end{enumerate} 

\chapter{suhṛd-darśanam auṣadham - Seeing a friend heals}

\begin{enumerate}
\item \dev{  व्याधितस्य, शोक-युक्तस्य सुहृद्-दर्शनम् औषधम् |}

vyādhitasya, śoka-yuktasya suhṛd-darśanam auṣadham |

For a person suffering from illness and also a person who is in sorrow seeing a good-hearted friend is medicine. 

\item \dev{अर्थ-हीनस्य सुहृद्-दर्शनम् औषधम् ।}

artha-hīnasya suhṛd-darśanam auṣadham |

For person in financial difficulty seeing a good-hearted friend is medicine. 

\item \dev{देशान्तर-गतस्य सुहृद्-दर्शनम् औषधम् ।}

deśāntara-gatasya suhṛd-darśanam auṣadham |

For a person who is traveling in foreign lands, seeing a good-hearted friend is medicine.
\end{enumerate}

\begin{center}
\textbf{\large Exercise 6}
\end{center}

Find the āyurvedic terms for –
\begin{enumerate}
\renewcommand{\theenumi}{\alph{enumi}}
\renewcommand{\labelenumi}{\theenumi.}
\item Seeing the friends
\item A person with disease
\item A sad person
\item A person bereft of wealth
\item A person traveling abroad 
\end{enumerate}

\chapter{Uttamaḥ  āyurveda-vaidyaḥ}
\begin{center}
Who is a best Āyurveda Physician?
\end{center}

\begin{enumerate}
\item \dev{यः वैद्यः शरीरं सर्वं वेद सः आयुर्वेदं वेद।}

yaḥ vaidyaḥ śarīraṃ sarvaṃ  veda saḥ āyurvedaṃ veda | 

The one who knows all parts of the body (anatomy/physiology) is the physician who knows āyurveda . 

\item \dev{ यः वैद्यः शरीरं  सर्वदा वेद सः आयुर्वेदं वेद ।}

yaḥ vaidyaḥ śarīraṃ  sarvadā veda saḥ āryuvedaṃ veda | 

The one who knows the body in all seasons is the physician who knows āyurveda .

\item \dev{यः वैद्यः शरीरं  सर्वथा वेद सः आयुर्वेदं वेद ।}

yaḥ vaidyaḥ śarīraṃ  sarvathā veda saḥ āyurvedaṃ veda | 

The one who knows the body in all conditions (casued by vitiation of Vāta etc) is the physician who knows āyurveda.
\end{enumerate}

\begin{center}
\textbf{\large Exercise 7}
\end{center}

Find the āyurvedic terms for –
\begin{enumerate}
\renewcommand{\theenumi}{\alph{enumi}}
\renewcommand{\labelenumi}{\theenumi.}
\item Knows
\item Entire/all parts
\item All times
\item All conditions
\item The body
\end{enumerate}

\chapter{Āyurveda-parīkṣā}

\begin{center}
How does an Āyurvedic Physician diagnose?
\end{center}

\begin{enumerate}
\item \dev{वैद्यः दर्शनेन आतुरं परीक्षते।}

vaidyaḥ darśanena āturaṃ parīkṣate |

The physician assesses the patient by visual examination. 

\item \dev{वैद्यः स्पर्शनेन आतुरं परीक्षते।}

vaidyaḥ sparśanena āturaṃ parīkṣate | 

The physician assesses  the patient by touching. 

\item \dev{वैद्यः प्रश्नेन आतुरं परीक्षते।}

vaidyaḥ praśnena āturaṃ parīkṣate | 

The physician assesses the patient by asking (relevant) questions.
\end{enumerate}

\begin{center}
\textbf{\large Exercise 8}
\end{center}

Find the āyurvedic terms for –
\begin{enumerate}
\renewcommand{\theenumi}{\alph{enumi}}
\renewcommand{\labelenumi}{\theenumi.}
\item a physician
\item a person will illness
\item visual examination
\item touching
\item questioning
\end{enumerate}
