\begin{center}
\textbf{\LARGE Part - 1}
\end{center}

\begin{multicols}{2}
\textbf{Exercise 1}
\begin{enumerate}[a.]
\item svastha
\item śaucavidhi
\item rātri-bhojana 
\item uttiṣṭhet
\end{enumerate}
\vspace{.3cm}

\textbf{Exercise 2}
\begin{enumerate}[a.]
\item dantapavana
\item dantamāṃsa
\item kaniṣṭhikā
\item tikta
\end{enumerate}
\vspace{.3cm}

\textbf{Exercise 3}
\begin{enumerate}[a.]
\item lepana 
\item hitakara
\item teja
\item netra
\end{enumerate}
\vspace{.3cm}

\textbf{Exercise 4}
\begin{enumerate}[a.]
\item jatru
\item gaṇḍūṣa-kriyā
\item dhūma-sevana
\item karaṇīya
\end{enumerate}

\textbf{Exercise 5}
\begin{enumerate}[a.]
\item abhyaṅga
\item jarā
\item su-tvak
\item dārḍhya
\item dṛṣṭi-prasāda
\end{enumerate}

\textbf{Exercise 6}
\begin{enumerate}[a.]
\item ati-hasana
\item ati-gamana
\item ati-jāgaraṇa
\item ativyāyāma
\item anusukha
\end{enumerate}

\textbf{Exercise 7}
\begin{enumerate}[a.]
\item udvartana
\item medaḥ-pravilāyana
\item tvak-prasāda
\item aṅga
\item harati
\end{enumerate}

\textbf{Exercise 8}

\vspace{.2cm}
Goodness of bathing : vṛṣya, āyuṣya, dīpana, ūrjā-prada, bala-prada Impurities removed by bathing: kaṇḍū, mala, sveda, tandrā, tṛṣṇā, dāha, papa
\newpage

\textbf{Exercise 9}
\begin{enumerate}[a.]
\item kleśaḥ, cakṣuṣ 
\item balavardhana 
\item balahāni
\item kaṇṭha
\item adhaḥ
\end{enumerate}

\textbf{Exercise 10}
\begin{enumerate}[a.]
\item anantaram
\item āhāra
\item pūrvāhāra 
\item hita
\item mita
\end{enumerate}

\textbf{Exercise 11}
\begin{enumerate}[a.]
\item sādhya-roga
\item anyakārya
\item vega 
\item prāptuṃ
\item na kuryāt
\end{enumerate}

\textbf{Exercise 12}
\begin{enumerate}[a.]
\item kalyāṇa-mitra 
\item kāya 
\item hiṃsā, steya, anyathākāma, paiśunya , paruṣa, anṛta, sambhinnālāpa vyāpāda, abhidhyā, dṛg-viparyaya
\end{enumerate}

\textbf{Exercise 13}
\vspace{-.2cm}
\begin{enumerate}[a.]
\item pipīlikā 
\item kīṭa
\item vṛttirahita
\item ātmavat
\item śoka-pīḍita
\end{enumerate}

\textbf{Exercise 14}
\vspace{-.2cm}
\begin{enumerate}[a.]
\item arthin
\item avamanyeta
\item sampad
\item vipad
\item ekamanas
\end{enumerate}

\textbf{Exercise 15}
\vspace{-.2cm}
\begin{enumerate}[a.]
\item peśala 
\item īrṣyā
\item avisamvadi
\item pūrvabhāṣī 
\item sumukha
\end{enumerate}

\textbf{Exercise 16}
\vspace{-.2cm}
\begin{enumerate}[a.]
\item viśvāsa 
\item śaṅkā
\item ayam
\item aham
\item mama
\end{enumerate}
\vskip -.2cm

\textbf{Exercise 17}
\vspace{-.3cm}
\begin{enumerate}[a.]
\item prabhu
\item nisnehatā
\item abhiprāya
\item pīḍayet
\item lālayet
\end{enumerate}

\textbf{Exercise 18}
\vspace{-.3cm}
\begin{enumerate}[a.]
\item parasparavirodha
\item madhyama mārga 
\item rahita
\item anusaret 
\item ācāra
\end{enumerate}

\textbf{Exercise 19}
\begin{enumerate}[a.]
\item ujjvala
\item susurabhi
\item dīrgha-nakha
\item ghrāṇa
\item suveṣa
\end{enumerate}

\textbf{Exercise 20}
\vspace{-.3cm}
\begin{enumerate}[a.]
\item dhārayet 
\item padatrāṇa 
\item ātapatra 
\item yuga
\item ātyayika-kārya
\end{enumerate}

\textbf{Exercise 21}
\vspace{-.3cm}
\begin{enumerate}[a.]
\item na laṅghayet 
\item na taret
\item nāva 
\item chāyā 
\item aśuddha
\end{enumerate}

\textbf{Exercise 22}
\vspace{-.3cm}
\begin{enumerate}[a.]
\item asaṃṛta-mukha
\item vijṛmbhaṇa 	
\item kṣuti 
\item ciram
\item āsīta
\end{enumerate}

\textbf{Exercise 23}
\begin{enumerate}[a.]
\item ceṣṭā 
\item ūrdhva-jānu 
\item catuṣpatha 
\item śūnya-gṛha 
\item divā
\end{enumerate}

\textbf{Exercise 24}
\begin{enumerate}[a.]
\item īkṣeta 
\item dīpta 
\item sūkṣma
\item vahet
\item vikrayaṇa
\end{enumerate}

\textbf{Exercise 25}
\begin{enumerate}[a.]
\item dhūli 
\item udgāra
\item Hima
\item atinipuṇa
\item kalaha
\end{enumerate}

\textbf{Exercise 26}
\begin{enumerate}[a.]
\item āpaṇa
\item avadhūnana
\item cintana
\item sandhyā 
\item tyajet
\end{enumerate}

\textbf{Exercise 27}
\begin{enumerate}[a.]
\item loka 
\item parīkṣaka 
\item anukuryāt 
\item paramakaruṇā
\item paropakāra
\end{enumerate}

\textbf{Exercise 28}
\begin{enumerate}[a.]
\item naktaṃdina 
\item cintanīya
\item evaṃ 
\item kadāpi na
\item pālana
\end{enumerate}
\end{multicols}

\begin{center}
\textbf{Assignment:}
\end{center}

Organise the teachings of Dinacaryā of Ashtangahrdaya under the sub-heads –

\begin{enumerate}
\item Āhāra (food),
\item vihāra (exercise ),
\item Vyavahāra (conduct and guidance for transaction with others) and
\item Kāya-indirya-samskāra (aspects for refinement/cleansing of body and senses). (You may send in the filled assignment to - jayaraman@kym.org)
\end{enumerate}
\newpage

\begin{center}
\textbf{\LARGE Part - 2}
\end{center}
\begin{multicols}{2}
\textbf{Exercise 1}
\begin{enumerate}[a.]
\item svāsthya 
\item ātura 
\item ṛtusandhi 
\item roga 
\item jitendriya
\end{enumerate}

\textbf{Exercise 2}
\begin{enumerate}[a.]
\item kārśya 
\item kampa
\item kāsa
\item ālasya
\item alpanidratā
\end{enumerate}

\textbf{Exercise 3}
\begin{enumerate}[a.]
\item Jvara
\item gātrasaṃsparśana 
\item niśvāsa
\item sahabhojana 
\item saṅkrāmati
\end{enumerate}
\vspace{.3cm}

\textbf{Exercise 4}
\begin{enumerate}[a.]
\item paramauṣadha 
\item taila
\item ghṛta 
\item madhu 
\end{enumerate}

\textbf{Exercise 5}
\begin{enumerate}[a.]
\item vāri 
\item ajīrṇa 
\item bheṣaja 
\item jīrṇa 
\item amṛta
\end{enumerate}

\textbf{Exercise 6}
\begin{enumerate}[a.]
\item suhṛd-darśana 
\item śoka-yukta 
\item vyādhita 
\item artha-hīna 
\item deśāntara-gata
\end{enumerate}

\textbf{Exercise 7}
\begin{enumerate}[a.]
\item veda 
\item sarva  
\item sarvadā 
\item sarvathā 
\item śarīra
\end{enumerate}

\textbf{Exercise 8}
\begin{enumerate}[a.]
\item vaidya 
\item ātura 
\item darśana 
\item sparśana 
\item praśna
\end{enumerate}
\end{multicols}
