\chapter{Prātarutthānam - Waking up in the Morning}

\begin{enumerate}
\itemsep=0pt
\item \dev{स्वस्थः आयुषः रक्षार्थं ब्राह्मे मुहूर्ते उत्तिष्ठेत्~। }

svasthaḥ āyuṣaḥ rakṣārthaṃ brāhme muhūrte uttiṣṭhet~|

A healthy person to protect his longevity should wake up earlier in the morning in Brahma-muhurta. (If sunrise it at 6 - then Brahmamuhrta is between 4.24 am-5.12 am) 

\item \dev{स्वस्थः आयुषः रक्षार्थं  रात्रि-भोजनस्य जीर्णता- ऽ (अ)जीर्णतादिक-चिन्तां कुर्यात्~।}

svasthaḥ āyuṣaḥ rakṣārthaṃ  rātri-bhojanasya jīrṇata-(a)jīrṇatadika-cintāṃ kuryāt~| 

(After that) A healthy person to protect his health should think about his body - as to whether the food eaten previous night has digested or not. 

\item \dev{स्वस्थः आयुषः रक्षार्थं   शौचविधिं कुर्यात्~।}

svasthaḥ āyuṣaḥ rakṣārthaṃ   śaucavidhiṃ kuryāt~| 

(After that) A Health person to protect his longevity should cleanse his body off the impurities.
\end{enumerate}

\centerline{\textbf{Exercise~1}}

Find the āyurvedic terms for - 
\begin{enumerate}
\itemsep=0pt
\renewcommand{\theenumi}{\alph{enumi}}
\renewcommand{\labelenumi}{\theenumi.}
\item A healthy person.
\item Cleansing activity
\item Night meal 
\item should get up.
\end{enumerate}

\chapter{Danta-pavanam - Cleansing the teeth}

\begin{enumerate}
\itemsep=0pt
\item \dev{प्रातः उत्थाय, भोजनानन्तरं च दन्तपवनं करणीयम्~।}

prātaḥ utthāya, bhojanānantaraṃ ca dantapavanaṃ karaṇīyam~| 

The teeth have to be cleansed in morning after waking up and after meals.

\item \dev{कनिष्ठिका-स्थूलेन, मृद्वग्रेण, कषाय-कटु-तिक्त-रस-युतेन अर्कादि-काष्ठेन दन्तपवनं करणीयम्~।}

kaniṣṭhikā-sthūlena, mṛdvagreṇa, kaṣāya-kaṭu-tikta-rasa-yutena arkādi-kāṣṭhena dantapavanaṃ karaṇīyam~|  

One should cleanse the teeth with a short twigs/sticks of Arka and other trees that has soft-tip, has a breadth of the little finger and has taste including astringent, pungent/hot, bitter taste.  

\item \dev{दन्तमांसम् अपीडयता दन्तपवनं करणीयम्~।}

dantamāṃsam apīḍayatā dantapavanaṃ karaṇīyam~|

In the process of cleansing teeth one should not hurt the gums of the teeth.
\end{enumerate}

\centerline{\textbf{Exercise~2}}

Find the āyurvedic terms for - 
\begin{enumerate}
\itemsep=0pt
\renewcommand{\theenumi}{\alph{enumi}}
\renewcommand{\labelenumi}{\theenumi.}
\item Teeth cleansing
\item Flesh connected to teeth
\item Little finger
\item Bitter taste
\end{enumerate}

\chapter{Añjanalepanam - Applying collyrium (eye application)}

\begin{enumerate}
\itemsep=0pt
\item \dev{दन्तपावनात्  परं   नित्यं नेत्रयोः अञ्जन-लेपनं करणीयम्~।}

dantapāvanāt  paraṃ nityaṃ netrayoḥ añjana-lepanaṃ karaṇīyam~| 

After cleansing the teeth, one should apply collyrium in the eyes. 

\item \dev{अञ्जन-लेपनं नेत्रयोः हितकरं  भवति~।}

añjana-lepanaṃ netrayoḥ hitakaraṃ  bhavati~| 

Applying collyrium  is beneficial to eyes. 

\item \dev{अञ्जन-लेपनेन नेत्रयोः तेजः वर्धते~।}

añjana-lepanena netrayoḥ tejaḥ Vardhate~|  

Applying collyrium to eyes enhances the brightness in eyes.
\end{enumerate}


\centerline{\textbf{Exercise~3}}


Find the āyurvedic terms for - 
\begin{enumerate}
\itemsep=0pt
\renewcommand{\theenumi}{\alph{enumi}}
\renewcommand{\labelenumi}{\theenumi.}
\item Applying 
\item Beneficial
\item Brightness 
\item Eye
\end{enumerate}

\chapter{nasyam, dhūmasevanam, gaṇḍūṣakriyā - Nasal herbal drops, herbal fumes, Herbal mouth rinsing}

\begin{enumerate}
\itemsep=0pt
\item \dev{ततः परम्,  ऊर्ध्व-जत्रु-रोग-शमनाय नस्यं  करणीयम्~।}

tataḥ param,  ūrdhva-jatru-roga-śamanāya nasyaṃ  karaṇīyam~| 

After that, to overcome disease in the limbs above the collar bone one takes in herbal drops through nose.  

\item \dev{तथैव,  ऊर्ध्व-जत्रु-रोग-शमनाय  धूम-सेवनं करणीयम्~।}

tataḥ paraṃ,  ūrdhva-jatru-roga-śamanāya  dhūma-sevanaṃ kara\-ṇīyam~|

And also, to overcome disease in the limbs above the collar bone one should inhale medicinal/herbal fumes.

\item \dev{किञ्च, गण्डूष-क्रिया, ताम्बूलसेवन-क्रिया च  करणीया~।}

tataḥ paraṃ ca, gaṇḍūṣa-kriyā, tāmbūlasevana-kriyā ca  karaṇīyā~|

Also, fillng/rinsing the mouth with herbal fluids and also chewing betel leaf should be done.   
\end{enumerate}

\centerline{\textbf{Exercise~4}}


Find the āyurvedic terms for
\begin{enumerate}
\itemsep=0pt
\renewcommand{\theenumi}{\alph{enumi}}
\renewcommand{\labelenumi}{\theenumi.}
\item Collarbone
\item Rinsing the mouth
\item Inhaling medicinal fumes
\item Should do
\end{enumerate}

\chapter{abhyaṅgaḥ --- Oil bath}

\begin{enumerate}
\itemsep=0pt
\item \dev{नित्यम् अभ्यङ्गम् आचरेत्~।}

nityam abhyaṅgam ācaret~|

One should massage limbs of the body with oil every day. 

\item \dev{अभ्यङ्गः जरा-हरः, श्रम-हरः, वात-हरः च~।}

abhyaṅgaḥ jarā-haraḥ, śrama-haraḥ, vāta-haraḥ ca~|

Oil massage removes the weariness of oldage, tiredness and reduces vāta related issues.

\item \dev{अभ्यङ्गः दृष्टि-प्रसाद-कृत्, पुष्टि-कृत्, सु-त्वक्-कृत्, आयुष्-कृत्, निद्रा-कृत्, दार्ढ्य-कृत्~।}

abhyaṅgaḥ dṛṣṭi-prasāda-kṛt, puṣṭi-kṛt, su-tvak-kṛt, āyuṣ-kṛt, nidrā-kṛt, dārḍhya-kṛt~| 

Oil massage bestows good eye-sight, nourishment, glow in the skin, longevity, good sleep and firmness.
\end{enumerate}

\centerline{\textbf{Exercise~5}}


Find the āyurvedic terms for
\begin{enumerate}
\itemsep=0pt
\renewcommand{\theenumi}{\alph{enumi}}
\renewcommand{\labelenumi}{\theenumi.}
\item Oil application
\item Ageing
\item Glow in the skin
\item Firmness
\item Good eyesight
\end{enumerate}

\chapter{Vyāyāmaḥ - Physical exercise}

\begin{enumerate}
\item \dev{अभ्यङ्गाद् अनन्तरं,  व्यायामः करणीयः~।}

abhyaṅgād anantaraṃ,  vyāyāmaḥ karaṇīyaḥ~|

After oil application in the limbs, one should do physical exercise. 

\item \dev{व्यायामं समाप्य शरीरम् अनुसुखं मर्दनीयम्~।}

vyāyāmaṃ samāpya śarīram anusukhaṃ mardanīyam~|

After doing exercise, one should gently massage the limbs. 

\item \dev{यथा अति-हसनम्, अति-जागरणं, अति-गमनम् च न करणीयम् तथा अतिव्यायाम: अपि न करणीयः~।}

yathā ati-hasanam, ati-jāgaraṇaṃ, ati-gamanam ca na karaṇīyam tathā ativyāyāmah api na karaṇīyaḥ~|

As one should avoid excessiveness in laughter, in staying awake at night, travels, and physical exercise.
\end{enumerate}

\centerline{\textbf{Exercise~6}}

Find the āyurvedic terms for
\begin{enumerate}
\itemsep=0pt
\renewcommand{\theenumi}{\alph{enumi}}
\renewcommand{\labelenumi}{\theenumi.}
\item Excess laughter
\item Excess travel
\item Excess waking up at night
\item Excess physical exercise
\item Gently
\end{enumerate}

\chapter{Udvartanam --- Applying herbal paste}

\begin{enumerate}
\itemsep=0pt
\item \dev{व्यायामाद् अनन्तरम् उद्वर्तनं करणीयम्~।}

vyāyāmād anantaram udvartanaṃ karaṇīyam~|

After Physical exercise (and related procedures) one should apply herbal paste in the body.  

\item \dev{उद्वर्तनं कफं हरति, मेदसः प्रविलायनं च करोति~। }

udvartanaṃ kaphaṃ harati, medasaḥ pravilāyanaṃ ca karoti~|

By applying herbal paste (udvartana), Kapha is reduced and the excess fat is liquidated. 

\item \dev{उद्वर्तनम् अङ्गानां स्थिरीकरणं, त्वक्-प्रसादं च करोति~।}

udvartanam aṅgānāṃ sthirīkaraṇaṃ, tvak-prasādaṃ ca karoti~| 

Application of herbal paste(udvartana), makes the limbs firm and cleanses the skin.
\end{enumerate}

\centerline{\textbf{Exercise~7}}

Find the Āyurvedic terms for 
\begin{enumerate}
\itemsep=0pt
\renewcommand{\theenumi}{\alph{enumi}}
\renewcommand{\labelenumi}{\theenumi.}
\item Application of herbal paste
\item Liquidation of excess fat
\item Cleansing of the skin
\item Limbs of the body
\item Removes
\end{enumerate}

\chapter{Snānam 1 --- Bathing 1}

\begin{enumerate}
\itemsep=0pt
\item \dev{उद्वर्तनात् परं स्नानं करणीयम्~।}

udvartanāt paraṃ, snānaṃ karaṇīyam~| 

After Udvartana one should take bath. 

\item \dev{स्नानं वृष्यम्, आयुष्यं, दीपनं, ऊर्जा-प्रदं, बल-प्रदं  च भवति~।}

snānaṃ vṛṣyam, āyuṣyaṃ, dīpanaṃ, ūrjā-pradaṃ, bala-pradaṃ  ca bhavati~|

Bathing bestows reproductive capability, long life, nourishment, enthusiasm and strength.     

\item \dev{स्नानं कण्डूं, मलं, स्वेदं, तन्द्रां, तृष्णां, दाहं, पापं च निवारयति~।}

snānaṃ kaṇḍūṃ, malaṃ, svedaṃ, tandrāṃ, tṛṣṇāṃ, dāhaṃ, pāpaṃ ca nivārayati~|  

Bathing removes itching, dirt/impurity, sweat, laziness/dullness, thirst, burning sensation and sinful/negative intentions.
\end{enumerate}

\centerline{\textbf{Exercise~8}}

Find the āyurvedic terms for the impurities that bathing removes and the goodness that it bestows? 

\chapter{Snānam 2 --- Bathing 2}

\begin{enumerate}
\itemsep=0pt
\item \dev{उष्णजलेन कण्ठाद् अधः स्नानं करणीयम्~।}

uṣṇajalena kaṇṭhād adhaḥ snānaṃ karaṇīyam~| 

Below the neck one has to bath in warm/hot water. 

\item \dev{तेन शरीरे बलवर्धनं भवति~।}

tena śarīre balavardhanaṃ bhavati~| 

By that body attains strength. 

\item \dev{उष्णजलद्वारा शिरः-स्नानेन केशस्य चक्षुषोः च बलहानिः भवति~।}

uṣṇajaladvārā śiraḥ-snānena keśasya cakṣuṣoḥ ca balahāniḥ bhavati~|  

Warm/hot water head bath reduces strength of the eyes and also the hair.
\end{enumerate}

\centerline{\textbf{Exercise~9}}

Find the Āyurvedic terms for -
\begin{enumerate}
\itemsep=0pt
\renewcommand{\theenumi}{\alph{enumi}}
\renewcommand{\labelenumi}{\theenumi.}
\item Parts of the body that lose strength by hot water bath 
\item Attainment of strength 
\item Loss of strength
\item Neck 
\item Below
\end{enumerate}

\chapter{Āhāraḥ --- Eating food}

\begin{enumerate}
\itemsep=0pt
\item \dev{ स्नानस्य अनन्तरम् आहारः सेवनीयः~।}

snānasya anantaram āhāraḥ sevanīyaḥ~|

One should eat food after bathing. 

\item \dev{  पूर्वाहारे जीर्णे सति आहारः सेवनीयः~।}

pūrvāhāre jīrṇe sati  āhāraḥ sevanīyaḥ~|

Only after the digestion of meal eaten earlier one should eat food. 

\item \dev{  हितः मितः च आहारः सेवनीयः~।}

hitaḥ mitaḥ ca āhāraḥ sevanīyaḥ~|

One should partake food that is beneficial and one should follow moderation.
\end{enumerate}

\centerline{\textbf{Exercise~10}}

Find the Āyurvedic terms for - 
\begin{enumerate}
\itemsep=0pt
\renewcommand{\theenumi}{\alph{enumi}}
\renewcommand{\labelenumi}{\theenumi.}
\item After
\item food 
\item Food consumed earlier
\item Beneficial
\item In moderation
\end{enumerate}

\chapter{Vegaḥ, rogaḥ \& dharmaḥ --- Urges, diseases and Dharma}

\begin{enumerate}
\itemsep=0pt
\item \dev{वेग-युक्तः अन्यकार्यं न कुर्यात्~।}

vega-yuktaḥ anyakāryaṃ na kuryāt~|

When there are urges (like urination etc) one should not do any other work. (one should prioritise and address that)

\item \dev{ साध्य-रोग-निवारणं विना अन्यकार्यं न करणीयम्~।}

sādhya-roga-nivāraṇaṃ vinā anyakāryaṃ na karaṇīyam~| 

When there are diseases that can be cured one should prioritise that.

\item \dev{ सुखं प्राप्तुं धर्मम् एव आचरेत्~।}

sukhaṃ prāptuṃ dharmam eva ācaret~| 

To attain happiness one should follow only  the path of Dharma.
\end{enumerate}

\centerline{\textbf{Exercise~11}}

Find the āyurvedic terms for –
\begin{enumerate}
\itemsep=0pt
\renewcommand{\theenumi}{\alph{enumi}}
\renewcommand{\labelenumi}{\theenumi.}
\item Curable disease
\item Other work
\item Urge
\item To attain
\item Should not do
\end{enumerate}

\chapter{kalyāṇa-mitram, daśa pāpakāryāṇi --- Good Friends, Ten bad deeds}

\begin{enumerate}
\itemsep=0pt
\item \dev{भक्त्या कल्याण-मित्राणि सेवेत, न दुष्ट-मित्राणि~।}

bhaktyā kalyāṇa-mitrāṇi seveta, na duṣṭa-mitrāṇi~|

One should be devoted to friends who inspire good deeds and one should not have the company of bad.

\item \dev{कायेन वाचा मनसा दश पाप-कार्याणि न करणीयानि~।}

kāyena vācā manasā daśa pāpa-kāryāṇi na karaṇīyāni~| 

By body, speech and mind one should avoid ten sinful/negative deeds. 

\item \dev{हिंसा, स्तेयम्, अन्यथाकामः, पैशुन्यम्, परुषम्, अनृतम्, सम्भिन्नालापः, व्यापादः, अभिध्या, दृग्-विपर्ययः इति दश पापकार्याणि~।}

hiṃsā, steyam, anyathākāmaḥ, paiśunyam, paruṣam, anṛtam, sambhinnālāpaḥ vyāpādaḥ, abhidhyā, dṛg-viparyayaḥ iti daśa pāpakāryāṇi~|  

Injuring, stealing, illicit-desires, speaking ill of others (in their absence), harsh words, falsehood, unconnected/useless utterences, contemplating others doom, inability to tolerate virtues of others, not trusting the shastras/pious people are the ten  sinful/negative deeds.
\end{enumerate}

\centerline{\textbf{Exercise~12}}

Find the āyurvedic terms for -
\begin{enumerate}
\itemsep=0pt
\renewcommand{\theenumi}{\alph{enumi}}
\renewcommand{\labelenumi}{\theenumi.}
\item Good friends 
\item Body 
\item State the ten bad negative deeds to be avoided !
\end{enumerate}

\chapter{sevā, karuṇā, arcanam  1 --- Serving the needy, Compassion, Hospitality  1}

\begin{enumerate}
\itemsep=0pt
\item \dev{ वृत्तिरहितान्, व्याधितान्, शोक-पीडितान् शक्तितः अनुवर्तेत~।}

vṛttirahitān, vyādhitān, śoka-pīḍitān śaktitaḥ anuvarteta~| 

One should always help - people who are without proper means for livelihood, people suffering from illness and afflictions to the extent possible.   

\item \dev{  कीटान्, पिपीलिकाः अपि सततम् आत्मवत् पश्येत्~।}

kīṭān, pipīlikāḥ api satatam ātmavat paśyet~| 

One should always consider insects and also ants equal to oneself. 

\item \dev{  देवान्, धेनूः, विप्रान्, वृद्धान्, वैद्यान्, नृपान्, अतिथीन् च अर्चयेत्~।}

devān, dhenūḥ, viprān, vṛddhān, vaidyān, nṛpān, atithīn ca arcayet~|   

Gods, cows, learned, elderly, physicians, kings and guests should be respected.
\end{enumerate}

\centerline{\textbf{Exercise~13}}

Find the āyurvedic terms for -
\begin{enumerate}
\itemsep=0pt
\renewcommand{\theenumi}{\alph{enumi}}
\renewcommand{\labelenumi}{\theenumi.}
\item Ant
\item Insect
\item People Without means for livelihood
\item Equal to oneself
\item People Afflicted by sorrow
\end{enumerate}

\chapter{Sevā, karuṇā, arcanam  2 --- Serving the needy, Compassion, Hospitality  2}

\begin{enumerate}
\itemsep=0pt
\item \dev{ अर्थिनः न विमुखान् कुर्यात्, न अवमन्येत, न आक्षिपेत्~।}

arthinaḥ na vimukhān kuryāt, na avamanyeta, na ākṣipet~|

Never turn away seekers of monetery help, nor disrespect or use harsh words.  

\item \dev{  अपकारिषु अपि जनेषु उपकार-प्रधानः भवेत्~।}

apakāriṣu api janeṣu upakāra-pradhānaḥ bhavet~| 

Try to be helpful even to those who trouble you. 

\item \dev{ सम्पदि विपदि च एकमनाः भवेत्~।}

Sampadi, vipadi ca ekamanāḥ bhavet~|

In prosperity and in calamity a balanced mind has to be maintained.
\end{enumerate}

\centerline{\textbf{Exercise~14}}

Find the āyurvedic terms for -

\begin{enumerate}
\itemsep=0pt
\renewcommand{\theenumi}{\alph{enumi}}
\renewcommand{\labelenumi}{\theenumi.}
\item Those who seek monetary help
\item To Disrespect
\item Gain
\item Loss
\item Steady mind
\end{enumerate}

\chapter{īrṣyā-nivāraṇam, vāk-saṃyamaḥ --- Avoiding jealous and refinement of speech}

\begin{enumerate}
\itemsep=0pt
\item \dev{  फले ईर्ष्या न करणीया, हेतुविषये सा करणीया~।}

phale īrṣyā na karaṇīyā, hetuviṣaye sā karaṇīyā~| 

One should not be jealous of other’s achievement, but one should be jealous about the means (how hard would the other person have worked to achieve this. Let me also work hard etc..)

\item \dev{  काले, हितं, मितम्, अविसंवादि, पेशलं च ब्रूयात्~।}

kāle, hitaṃ, mitam, avisamvadi peśalaṃ ca brūyāt~|

One should speak - at the right time, beneficial things, in moderation, the truth/in a non-contradictory and pleasant manner. 

\item \dev{  पूर्वभाषी, सुमुखः, सुशीलः करुणामृदुः च भवेत्~।}

pūrvabhāṣī, sumukhaḥ, suśīlaḥ karuṇāmṛduḥ ca bhavet~|  

(When meeting others) One should speak first, one should have a pleasant coutenance, have good practices/conduct, have compassion and should also have a soft nature.
\end{enumerate}

\centerline{\textbf{Exercise~15}}

Find the āyurvedic terms for -

\begin{enumerate}
\itemsep=0pt
\renewcommand{\theenumi}{\alph{enumi}}
\renewcommand{\labelenumi}{\theenumi.}
\item Pleasant/sweet
\item Jealous
\item truth
\item The one who speaks first
\item Pleasant face.
\end{enumerate}

\chapter{Sukham, viśvāsaḥ, śatrutā --- Pleasures, trust, enmity}

\begin{enumerate}
\itemsep=0pt
\item \dev{एकः एव सुखी न स्यात्~।}

ekaḥ eva sukhī na syāt~|

One should not enjoy pleasures all alone. 

\item \dev{सर्वत्र विश्वासः अपि न करणीयः, सर्वत्र शङ्का अपि न करणीया~।}

sarvatra viśvāsaḥ api na karaṇīyaḥ, sarvatra śaṅkā api na karaṇīyā~|

One should not trust everyone. One should not doubt everyone. 

\item \dev{अयं मम शत्रुः, अहं तस्य शत्रुः इति कदापि न वक्तव्यम्~।}

ayaṃ mama śatruḥ, ahaṃ tasya śatruḥ iti kadāpi na vaktavyam~|  

One should never declare that he is my enemy and I am his enemy.
\end{enumerate}

\centerline{\textbf{Exercise~16}}

Find the āyurvedic terms for -
\begin{enumerate}
\itemsep=0pt
\renewcommand{\theenumi}{\alph{enumi}}
\renewcommand{\labelenumi}{\theenumi.}
\item Trust
\item Doubt
\item He
\item I
\item My
\end{enumerate}

\chapter{udyoga-vyavahāraḥ, abhiprāyaḥ, indriya-pravṛttiḥ --- Workplace conduct, Views of others, Use of Sense organs}

\begin{enumerate}
\itemsep=0pt
\item \dev{प्रभोः निस्नेहतां, स्वस्य अपमानं च न प्रकाशयेत्~।}

prabhoḥ nisnehatāṃ, svasya apamānaṃ ca na prakāśayet~| 

One should not disclose to others - one's disagreements with one's Lord/employer and instances of one's own humiliation.

\item \dev{अन्यस्य अभिप्रायं ज्ञात्वा, तस्य सन्तोषार्थं तं तथा अनुवर्तेत~।}

anyasya abhiprāyaṃ jñātvā, tasya santoṣārthaṃ taṃ tathā anuvarteta~|

One should understand others mindset and conduct oneself in such way which will please others (to avoid frictions) 

\item \dev{इन्द्रियाणि न पीडयेत्, तानि न अति-लालयेत् अपि~।}

indriyāṇi na pīḍayet, tāni na ati-lālayet api~| 

One should neither torture/mortify the limbs/parts of one’s body nor should they be satiated too much.
\end{enumerate}

\centerline{\textbf{Exercise~17}}

Find the āyurvedic terms for –
\begin{enumerate}
\itemsep=0pt
\renewcommand{\theenumi}{\alph{enumi}}
\renewcommand{\labelenumi}{\theenumi.}
\item Lord/employer
\item Lack of cordiality
\item Opinion
\item Mortify
\item Satiate
\end{enumerate}

\chapter{Puruṣārthaḥ, madhyama-mārgaḥ --- Goals of life, the middle path}

\begin{enumerate}
\itemsep=0pt
\item \dev{धर्मा(अ)र्थ-काम-रहितानि कार्याणि न करणीयानि~।}

dharmā(a)rtha-kāma-rahitāni kāryāṇi na karaṇīyāni~|

Actions that are not connected to Dharma, Artha and Kāma should not be done. 

\item \dev{धर्मा(अ)र्थ-कामेषु परस्परविरोधः न भवेत्~।}

dharmā(a)rtha-kāmeṣu parasparavirodhaḥ na bhavet~| 

Even in those actions that lead to Dharma, Artha and Kāma - the one should not oppose the other. 

\item \dev{सर्वेषु आचारेषु मध्यमं मार्गम् अनुसरेत्~।}

sarveṣu ācāreṣu madhyamaṃ mārgam anusaret~| 

In all practices and activities eccentricities and extremities should be avoided. Moderation should be followed.
\end{enumerate}

\centerline{\textbf{Exercise~18}}

Find the āyurvedic terms for -
\begin{enumerate}
\itemsep=0pt
\renewcommand{\theenumi}{\alph{enumi}}
\renewcommand{\labelenumi}{\theenumi.}
\item Mutually incompatible
\item Path of moderation 
\item Bereft
\item Should follow 
\item Practices/actions
\end{enumerate}

\chapter{Śuddhatā, cārutā --- Personal hygiene and  good appearance}

\begin{enumerate}
\itemsep=0pt
\item \dev{ मनुष्यः दीर्घ-रोम-रहितः, अ-दीर्घ-नखः, अ-दीर्घ-श्मश्रुः, शुद्ध-पाद-घ्राण-कर्णः भवेत्~।}

manuṣyaḥ dīrgha-roma-rahitaḥ, a-dīrgha-nakhaḥ, a-dīrgha-śmaśruḥ, śuddha-pāda-ghrāṇa-karṇaḥ bhavet~| 

Human beings should trim long hair (men), long nails and long beard. Legs, nose and ears (and all paths of excretion) should be clean.   

\item \dev{  मनुष्यः स्नानशीलः, सुवेषः, सुसुरभिः च भवेत्~।}

manuṣyaḥ snānaśīlaḥ, suveṣaḥ, susurabhiḥ ca bhavet~|

Human being should bathe (twice), wear good clothing, use good perfumes.   

\item \dev{  मनुष्यः अनुल्बणतया उज्ज्वलः भवेत्~।}

manuṣyaḥ anulbaṇatayā ujjvalaḥ bhavet~|

Human being’s appearence should be bright, but without being excessively prominent.
\end{enumerate}

\centerline{\textbf{Exercise~19}}

Find the āyurvedic terms for -
\begin{enumerate}
\itemsep=0pt
\renewcommand{\theenumi}{\alph{enumi}}
\renewcommand{\labelenumi}{\theenumi.}
\item Bright
\item fragrant
\item Long nail
\item Nostril
\item Good clothing
\end{enumerate}

\chapter{bhūṣaṇam, bahirgamanam --- Jewels and while venturing out}

\begin{enumerate}
\itemsep=0pt
\item \dev{सततं रत्नानि, सिद्ध-मन्त्रं, महौषधिं च धारयेत्~।}

satataṃ ratnāni, siddhamantraṃ, mahauṣadhiṃ ca dhārayet~|

One should wear precious gems and amulets filled with potent and divine medicine. 

\item \dev{आतपत्रं, पद-त्राणं, धारयन् युग-मात्र-दृक् सञ्चरेत्~।}

ātapatraṃ, padatrāṇaṃ, dhārayan yuga-mātra-dṛk sañcaret~|

one should carry umbrella, wear footwear and look only four arms length ahead and walk . 

\item \dev{आत्ययिक-कार्यार्थं रात्रौ गमन-समये दण्ड-धारी, शिरो-वेष्टन-धारी, सहायवान्  च सञ्चरेत्~।}

ātyayika-kāryārthaṃ rātrau gamana-samaye daṇḍadhārī, śiro-veṣṭana-dhārī, sahāyavān  ca sañcaret~|   

For very essential work if one has go out at night, one should carry a stick, wear a turban and  have a companion.
\end{enumerate}

\centerline{\textbf{Exercise~20}}

Find the āyurvedic terms for -
\begin{enumerate}
\itemsep=0pt
\renewcommand{\theenumi}{\alph{enumi}}
\renewcommand{\labelenumi}{\theenumi.}
\item Should wear 
\item Foot wear 
\item Umbrella 
\item Four arms length
\item a work that will not wait/essential
\end{enumerate}

\chapter{laṅghanam, taraṇam, ārohaṇam --- Jumping, swimming and climbing}
\vspace{-10pt}
\begin{enumerate}
\itemsep=0pt
\item \dev{चैत्यम्, पूज्यम्, ध्वजम्, पतितम्, छायाम्, भस्म, तुषम्, अशुद्धं, लघु-पाषाणं, लोष्टं, बलिस्थानं, स्नान-स्थानं च  न लङ्घयेत्~।}

caityam, pūjyam, dhvajam, patitam, chāyām, bhasma, tuṣam, aśuddhaṃ, laghu-pāṣāṇaṃ, loṣṭaṃ, balisthānaṃ, snāna-sthānaṃ ca  na laṅghayet~| 

One should not carelessly while crossing (can it be cross?) a caitya (a spot or place or a tree or a mound that is considered to have manifestation of some divinity/natural forces or place of worship of Bauddhas etc), respected person, flag, a person with bad conduct, shadow, ashes, husk of grains, impurities like excreta, small stones, lump of clay, altar of sacrifice and place of bathing. 

\item \dev{बाहुभ्यां नदीं न तरेत्~।}

bāhubhyāṃ nadīṃ na taret~| 

One should not swim and cross the river (unnecessarily). 

\item \dev{सन्दिग्ध-नावं, वृक्षं, दुष्टयान-वत् न आरुहेत्~।}

sandigdha-nāvaṃ, vṛkṣaṃ, duṣṭayāna-vat na āruhet~|    

As a person does not get into a vehicle which is not steady (Chariot/cart etc), similary one should not climb a weak boat and tree.
\end{enumerate}
\vspace{-10pt}

\centerline{\textbf{Exercise~21}}

Find the āyurvedic terms for -
\vspace{-10pt}
\begin{enumerate}
\itemsep=0pt
\renewcommand{\theenumi}{\alph{enumi}}
\renewcommand{\labelenumi}{\theenumi.}
\item should not Jump
\item Should not swim 
\item boat 
\item Shadow 
\item Impure
\end{enumerate}

\chapter{kṣutiḥ, aṅgaceṣṭā --- Sneezing and other activities of the limbs}

\begin{enumerate}
\itemsep=0pt
\item \dev{असंवृत-मुखः क्षुतिं, हास्यं, विजृम्भणं च न कुर्यात्~।}

asaṃvṛta-mukhaḥ kṣutiṃ, hāsyaṃ,  vijṛmbhaṇaṃ ca na kuryāt~| 

Without covering the mouth one should not sneeze, laugh or yawn. 

\item \dev{नासिकां न विकुष्णीयात्,अ-कस्मात् भुवं न विलिखेत् च~।}

nāsikāṃ na vikuṣṇīyāt, a-kasmāt bhuvaṃ na vilikhet ca~| 

Unnecessarily one should not pull one's own nose, without any purpose one should not scratch the earth (with hands or feet)

\item \dev{अङ्गैः विगुणं न चेष्टेत, उत्कटकः चिरं न आसीत च~।}

aṅgaiḥ viguṇaṃ na ceṣṭeta, utkaṭakaḥ ciraṃ na āsīta ca~| 

(A healthy person) One should not employ one's limbs in a deranged manner and one should not sit in Utkaṭāsana for long duration.
\end{enumerate}

\centerline{\textbf{Exercise~22}}

Find the āyurvedic terms for -
\begin{enumerate}
\itemsep=0pt
\renewcommand{\theenumi}{\alph{enumi}}
\renewcommand{\labelenumi}{\theenumi.}
\item without covering the mouth
\item yawning 
\item sneeze 
\item long duration
\item should be seated
\end{enumerate}

\chapter{Śramādikam --- fatigue etc}

\begin{enumerate}
\itemsep=0pt
\item \dev{देहस्य, वाचः, चित्तस्य च चेष्टां श्रमात् प्राक् विनिवर्तयेत्~।}

dehasya, vācaḥ, cittasya ca ceṣṭāṃ śramāt prāk vinivartayet~| 

One should stop the activities of the body, speech and mind just before getting tired. 

\item \dev{ऊर्ध्व-जानुः चिरं न भवेत्~।}

ūrdhva-jānuḥ ciraṃ na bhavet~| 

Knees should not be kept upwards for long time. 

\item \dev{वृक्षं, चत्वरं, चैत्य-समीप-स्थानम्, चतुष्पथम्, देवालयम्/सुरापान-स्थानम् च रात्रौ न सेवेत,  वध-स्थानम्, वनं, शून्य-गृहम्, श्मशानं च दिवा अपि न सेवेत~। }

vṛkṣaṃ, catvaraṃ, caitya-samīpa-sthānam, catuṣpatham, devālayam / surāpāna-sthānam ca rātrau na seveta,  vadha-sthānam, vanaṃ, śūnya-gṛham, śmāśānaṃ ca divā api na seveta~|      

One should not go near trees, crossroads, place close to cemetry/Caitya (see Lesson 21, sentence 1 translation), junctions of roads, temple/liquor shop, at night times. One should not go to place of execution (of criminals), forest, uninhabited  buildings and crematorium even during day time.
\end{enumerate}

\centerline{\textbf{Exercise~23}}

Find the āyurvedic terms for -
\begin{enumerate}
\itemsep=0pt
\renewcommand{\theenumi}{\alph{enumi}}
\renewcommand{\labelenumi}{\theenumi.}
\item Activity
\item above the knee
\item Junction
\item empty building 
\item day time
\end{enumerate}

\chapter{īkṣaṇam vahanam madyapānam --- Seeing, carrying and on liquor consumption}

\begin{enumerate}
\itemsep=0pt
\item \dev{सर्वथा आदित्यं न ईक्षेत, प्रततं, सूक्ष्मं, दीप्तम्, अमेध्यम्, अप्रियं च न ईक्षेत~।}

sarvathā ādityaṃ na īkṣeta, pratataṃ, sūkṣmaṃ, dīptam, amedhyam, apriyaṃ ca na īkṣeta~|  

At all times the Sun should not be looked (with naked eyes), further, one should not stare for long time, one should not strive to look at minute things and very bright things, one should avoid looking at impure and undesirable objects.  

\item \dev{भारं शिरसा न वहेत्~।}

bhāraṃ śirasā na vahet~|

Load should not be carried by placing it on the head.  

\item \dev{मद्यस्य उत्पादनं, विक्रयणं, दानं, परिग्रहणं च न कुर्यात्~।}

madyasya utpādanaṃ, vikrayaṇaṃ, dānaṃ, parigrahaṇaṃ ca na kuryāt~|   

One should not indulge in producing, selling, giving (for free?), purchasing/accepting liqour.
\end{enumerate}

\centerline{\textbf{Exercise~24}}

Find the āyurvedic terms for -
\begin{enumerate}
\itemsep=0pt
\renewcommand{\theenumi}{\alph{enumi}}
\renewcommand{\labelenumi}{\theenumi.}
\item Should see
\item Bright
\item Minute/very small
\item Should carry
\item Sale/selling
\end{enumerate}

\chapter{Tyāgaḥ --- to avoid}

\begin{enumerate}
\itemsep=0pt
\item \dev{पुरोवातम्, आतपं, धूलिं, तुषारं, चण्ड-वातं च त्यजेत्~।}

purovātam, ātapaṃ, dhūliṃ, tuṣāraṃ, caṇḍa-vātaṃ ca tyajet~|

One should avoid travelling facing oncoming wind and sun, dust, snow and whirlwind/storm.  

\item \dev{वक्र-स्थित्यां क्षुतं, उद्गारं, निद्रां, भोजनं, नदीतीरं, दुष्ट-गजादिकं, सर्पं, विषाणिनं च  त्यजेत्~।}

vakra-sthityāṃ kṣutaṃ, udgāraṃ, nidrāṃ, bhojanaṃ, nadītīraṃ, duṣṭa-gajādikaṃ, sarpaṃ, viṣāṇinaṃ ca  tyajet~| 

When the body is not erect one should not sneeze, belch, sleep and eat  also When the body is not erect/alert one should avoid bank of river, elephant in rut, snake and animals with horn (like bulls etc).   

\item \dev{हीनानाम्, अनार्याणाम्, अतिनिपुणानां च सेवां, उत्तमैः सह कलहं च त्यजेत्~।}

hīnānām, anāryāṇām, atinipuṇānāṃ ca sevāṃ, uttamaiḥ saha kalahaṃ ca tyajet~| 

One should avoid the company of vice-ridden, uncultured, calculative, and also avoid fight with the people who are superior (in virtue).
\end{enumerate}

\centerline{\textbf{Exercise~25}}

Find the āyurvedic terms for -
\begin{enumerate}
\itemsep=0pt
\renewcommand{\theenumi}{\alph{enumi}}
\renewcommand{\labelenumi}{\theenumi.}
\item Dust
\item Belching
\item Snow
\item Calculative
\item Fight
\end{enumerate}

\chapter{Tyāgaḥ, sandhyākālakriyā, madhyegamanam --- to avoid, activity in the evening, going in between}

\begin{enumerate}
\itemsep=0pt
\item \dev{हस्तस्य केशस्य अवधूननं, शत्रु-भोजनम्, आपण-भोजनं, असत्-कृतं भोजनं च  त्यजेत्~।}

hastasya keśasya avadhūnanaṃ, śatru-bhoja\-nam, āpaṇa-bhoja\-naṃ, asat-kṛtaṃ bhojanaṃ ca  tyajet~| 

One should avoid unnecessary shaking/moving of hair and hands, one should avoid accepting food from enemies, food from shops and food given without proper respect. 

\item \dev{सन्ध्याकाले भोजनम्, निद्राम्, अध्ययनम्, ध्यानम्, चिन्तनं च त्यजेत्~।}

sandhyākāle bhojanam, nidrām, adhyayanam, dhyānam, cinta\-naṃ ca tyajet~| 

In junction of time (morning/midday/evening) one should avoid eating food, sleep, study, meditation, deep contemplation . (only Sandhyāvandanam has to be done?) 

\item \dev{जलाशय-द्वयस्य मध्ये, अग्नि-द्वयस्य मध्ये, पूज्य-द्वयस्य मध्ये गमनं त्यजेत्~।}

jalāśaya-dvayasya madhye, agni-dvayasya madhye, pūjya-dvaya\-sya madhye gamanaṃ tyajet~|  

One should not go in between two water sources, two (Ritualistic) agnis, two respected elders.
\end{enumerate}

\centerline{\textbf{Exercise~26}}

Find the āyurvedic terms for -
\begin{enumerate}
\itemsep=0pt
\renewcommand{\theenumi}{\alph{enumi}}
\renewcommand{\labelenumi}{\theenumi.}
\item Shop
\item shaking/making it flow
\item Thinking
\item Junction of time  
\item Should avoid 
\end{enumerate}

\chapter{ācāryaḥ, parīkṣaṇam, sadvratam --- the teacher, examine, good vows}

\begin{enumerate}
\itemsep=0pt
\item \dev{सर्व-चेष्टासु लोकः एव आचार्यः~।}

sarvaceṣṭāsu lokaḥ eva ācāryaḥ~|

In all actions, the world is our teacher.

\item \dev{अतः परीक्षकः लौकिक-विषये लोकमेव अनुकुर्यात्~।}

ataḥ parīkṣakaḥ laukika-viṣaye lokameva anukuryāt~|

Hence, in worldly matters,  one should observe, analyse and follow the actions of the world. 

\item \dev{परम-करुणा, अपरिग्रहः, दमः, परोपकारः इत्येवं सद्व्रतं परिपूर्णम्~।}

paramakaruṇā, aparigrahaḥ, damaḥ, paropakāraḥ ityevaṃ sadvrataṃ paripūrṇam~| 

Great compassion, non-hoarding/contentment, control (of actions of body, mind and speech), selfless service to others (are very important). With these the exposition on good/healthy condut gets completed.
\end{enumerate}

\centerline{\textbf{Exercise~27}}

Find the āyurvedic terms for -

\begin{enumerate}
\itemsep=0pt
\renewcommand{\theenumi}{\alph{enumi}}
\renewcommand{\labelenumi}{\theenumi.}
\item World
\item the one who analyses
\item Should follow
\item great compassion
\item Selfless service
\end{enumerate}

\chapter{Cintanam, dinacaryā-phalam --- Thinking, outcomes of Dinacaryā}

\begin{enumerate}
\itemsep=0pt
\item \dev{नक्तंदिनानि अहं कथं यापयामि इति सदा चिन्तनीयम्~।}

naktaṃdināni ahaṃ kathaṃ yāpayāmi iti sadā cintanīyam~|

"How do I spend my time in daytime and night?" - In this manner One should be mindful of one's activity always. 

\item \dev{एवं यः चिन्तयति तस्य कदापि दुःखं न भवति~।}

evaṃ yaḥ cintayati tasya kadāpi duḥkhaṃ na bhavati~| 

A person who thinks and acts in this manner will never experience suffering. 

\item \dev{एतादृश-दिनचर्या-पालनेन आयुः, आरोग्यम्, ऐश्वर्यं, कीर्तिः, स्वर्गादि-लोकाः च नित्यं प्राप्यन्ते~।}

etādṛśa-dinacaryā-pālanena āyuḥ, ārogyam, aiśvaryaṃ, kīrtiḥ, svargādi-lokāḥ ca nityaṃ prāpyante~|

A person who follows a daily schedule in this manner will attain longlife, good health, wealth, fame, Heaven and other such higher realms.
\end{enumerate}

\centerline{\textbf{Exercise~28}}

Find the āyurvedic terms for -
\begin{enumerate}
\itemsep=0pt
\renewcommand{\theenumi}{\alph{enumi}}
\renewcommand{\labelenumi}{\theenumi.}
\item day and night 
\item Should think
\item In this manner 
\item Never
\item to follow
\end{enumerate}
