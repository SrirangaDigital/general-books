\thispagestyle{empty}

\chapter*{Introduction}

Āyurveda is a complementary system for Yoga. In most yoga teachers training courses, fundamentals of Āyurveda are taught. During the Covid 19 pandemic lockdown (first wave) period in an informal manner three sentences per day were shared with the teachers of Krishnamacharya Yoga Mandiram on Āyurvedic Dinacaryā and also on other fundamental principles of āyurveda from aṣṭāṅgahṛdaya (Reference: aṣṭāṅgahṛdaya,  sūtrasthāna, adhyāya 2) of Vāgbhaṭācārya. Twenty participants participated and received inputs from traditional Āyurvedic texts. The consolidation of the sentences thus shared has been done in the form of this book.

\noindent \textbf{What does this book contain?}

This book is in three parts 

\begin{enumerate}
\renewcommand{\theenumi}{\alph{enumi}}
\renewcommand{\labelenumi}{\theenumi.}
\item Part 1 - Sentences on Dinacaryā – healthy daily āyurvedic schedule. ( 28 sets of three sentences – total 84 sentences)
\item Part 2 - Sentences on General āyurvedic facts (8 sets of three sentences – 24 sentences) 
\item Part 3 - contains answers to the exercises given at the end of each lesson.
\end{enumerate}

With regard to the sentences in each lesson, Saṃskṛta sentences are given in devanāgarī   followed by roman transliteration. The English translation of each of the sentence is also given.

\noindent \textbf{ow to use the book?}

These sentences can be memorized and incorporated in daily life. 

Learning one page/lesson of three sentences, every day will lead to learning important fundamental lessons from Āyurveda in about 36 days. At the same time pronounciation of Saṃskṛta sentences can also learnt, that might serve to give a start in Saṃskṛta language learning.

Also, the exercises at the end of each lesson will help revise the content learnt. It is advised that the exercise of the previous day’s lesson be done on the next day to serve the purpose of revision.  Āyurveda Śatamāna on systematic study daily, shall incrementally impart Ayurvedic wisdom in daily capsules and enrich our lives with ancient wisdom of the Rishis.

\heading{Note}

It is to be noted that this book is meant to kindle interest and help set up a learning habit about Āyurvedic principles. Hence it might not answer all the technical queries on Āyurveda. One is expected to explore further and learn more after using this book as a stepping stone.

Further, regarding all the physiological and herbal prescriptions that are prescribed here, it is advised that one should verify the suitability of these prescriptions to one’s own self by consulting a Āyurvedic practitioner, as the health condition will vary from person to person.
