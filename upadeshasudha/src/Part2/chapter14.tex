\chapter{ದಾನದಶ್ರೇಷ್ಠತೆ}\label{chap14}

({\eng 1977}ರಲ್ಲಿ ಶ್ರೀ ಶ್ರೀ ಜಗದ್ಗುರು ಮಹಾಸ್ವಾಮಿಗಳವರು ಮದ್ರಾಸಿನಲ್ಲಿ ಅನುಗ್ರಹಿಸಿದ ಭಾಷಣ)

\begin{shloka}
ವಿಶುದ್ಧ ಜ್ಞಾನದೇಹಾಯ ತ್ರಿವೇದೀ ದಿವ್ಯಚಕ್ಷುಷೇ |\\
ಶ್ರೇಯಃ ಪ್ರಾಪ್ತಿ ನಿಮಿತ್ತಾಯ ನಮಸ್ಸ್ಸೋಮಾರ್ಧಧಾರಣೇ ||\\
ನಮಾಮಿ ಯಾಮಿನೀನಾಥ ಲೇಖಾಲಂಕೃತಕುಂತಲಾಮ್ |\\
ಭವಾನೀಂ ಭವಸಂತಾಪನಿರ್ವಾಪಣ ಸುಧಾನದೀಮ್ ||
\end{shloka}

ತುಪಾಕಿ ಹಿಡಿದು ಗುಂಡಿಕ್ಕಲು ಗುರಿ ಇಟ್ಟಿರುವ ಮನುಷ್ಯನು, ಗುಂಡು ಹೊಡೆಯಬೇಕಾದ ವಸ್ತುವಿಗಿಂತ ಸ್ವಲ್ಪ ಮೇಲೆ ಗುರಿ ಇಟ್ಟುಕೊಳ್ಳುತ್ತಾನೆ. ಏಕೆಂದರೆ, ಗುಂಡು ಗುರಿ ಇಟ್ಟಿರುವ ವಸ್ತುವನ್ನು ಸೇರುವುದಕ್ಕೆ ಮೊದಲು ಸ್ವಲ್ಪ ಅಲುಗಾಡುವುದು. ಹಾಗೆ ಆದಮೇಲೆ ವಸ್ತುವಿನ ಮೇಲೆ ಗುಂಡು ಬೀಳುವುದು. ಅದೇರೀತಿ ದೊಡ್ಡದಾದ ಒಳ್ಳೆಯ ಕೆಲಸಗಳಿಗಾಗಿ ಯಾವುದಾದರೂ ಸಹಾಯ ಮಾಡುವ ಸಂದರ್ಭ ನಮಗೆ ಒದಗಿದಾಗ ನಾವು ಮನಸ್ಸಿನಲ್ಲಿ ಅದಕ್ಕಾಗಿ ದೊಡ್ಡ ಕೊಡುಗೆ ಕೊಡುವುದಾಗಿ ತೀರ್ಮಾನಿಸಿಕೊಳ್ಳಬೇಕು. ಹಾಗಲ್ಲದೆ, ಪ್ರಾರಂಭದಲ್ಲೇ ಸ್ವಲ್ಪವೇ ಕೊಡುತ್ತೇವೆಂದು ತೀರ್ಮಾನಿಸಿಕೊಂಡರೆ, ಅನಂತರ ಕೊಡುವ ಸಮಯದಲ್ಲಿ ಅಷ್ಟು ಕೊಡಲು ಮನಸ್ಸು ಬರುವುದಿಲ್ಲ.

ಶ್ರೀಮಂತರಾಗಿರುವ ಹಲವರಿಗೆ ದಾನ-ಧರ್ಮಗಳು ಮಾಡಲು ಮನಸ್ಸು ಬರಲಿಲ್ಲವಲ್ಲಾ ಎನ್ನುವುದೆ ಹಲವರ ಸಂದೇಹ. ಇದಕ್ಕೆ ಸಂಬಂಧಪಟ್ಟಂತೆ ಶ್ರೀ ಶಂಕರಭಗವತ್ಪಾದರು ಹೇಳಿರುವ ಕಥೆಯೊಂದನ್ನು ನೋಡೋಣ. ದಾನ ವಿಷಯವನ್ನು ಹೇಳಿ ಆ ಕಥೆಯನ್ನು ಹೇಳದೆ ಇದ್ದರೂ ಕೂಡ, ವೇದಾಂತದಲ್ಲಿ ಆಸಕ್ತಿ ಇದ್ದೂ, ಆತ್ಮನಲ್ಲೇ ಶಾಂತಿಯಾಗಿರುವ ಒಬ್ಬನು ಹೊರಗಿನ ವಸ್ತುಗಳನ್ನು ತ್ಯಜಿಸುವುದು ಎಷ್ಟು ಆವಶ್ಯಕವೆಂದು ಆ ಕಥೆಯಿಂದ ನಾವು ತಿಳಿಯಬಹುದು.

\begin{shloka}
ದ್ರವ್ಯಂ ಪಲ್ಲವತಶ್ಚ್ಯುತಂ ಯದಿ ಭವೇತ್ ಕ್ವಾಪಿ ಪ್ರಮಾದಾತ್ ತದಾ\\
ಶೋಕಾಯಥ ತದರ್ಪಿತಂ ಶ್ರುತವತೇ ತೋಷಾಯ ಚ ಶ್ರೇಯಸೇ |\\
ಸ್ವಾತಂತ್ರ್ಯಾದ್ವಿಷಯಾಃ ಪ್ರಯಾನ್ತಿ ಯದಮೀ ಶೋಕಾಯ ತೇಸ್ಯುಶ್ಚಿರಂ |\\
ಸಂತ್ಯಕ್ತಾಃ ಸ್ವಯಮೇವ ಚೇತ್ ಸುಖಮಯಂ ನಿಃಶ್ರೇಯಸಂ ತನ್ವತೇ ||
\end{shloka}

(ಮರದಿಂದ ಬಿದ್ದುಹೋಗುವ ದ್ರವ್ಯ ಅದರ ಯಜಮಾನನಿಗೆ ಸಿಕ್ಕದೆ ಹೋದರೆ ಅದರಿಂದ ಅವನಿಗೆ ದುಃಖವಾಗುತ್ತದೆ; ಆದರೆ ಅದನ್ನು ಅವನು ಒಬ್ಬ ಶ್ರೋತಿಯನಿಗೆ ತಾನೆ ಕೊಡುವವನಾದರೆ ಅದರಿಂದ ಅವನಿಗೆ ಸಂತೋಷವೂ, ಶ್ರೇಯಸ್ಸೂ ಆಗುವುದು. ಅದೇ ರೀತಿ ಈ ಇಂದ್ರಿಯವಿಷಯಗಳು ತಮ್ಮ ಇಷ್ಟದಂತೆ ಹೋಗುವುದಾದರೆ ಅವುಗಳು ದುಃಖವನ್ನೇ ಕೊಡುವಂತಹುದಾಗುವುದು; ತಾನಾಗಿಯೇ ಅವು ಬಿಡಲ್ಪಟ್ಟರೆ ಅವು ಸುಖವನ್ನೂ, ಪರಿಪೂರ್ಣ ಆನಂದವನ್ನೂ ಕೊಡುವುವು)

ಒಬ್ಬನು ತರಕಾರಿ ತರುವುದಕ್ಕೆ ಅಂಗಡಿಗೆ ಹೋಗುತ್ತಿದ್ದಾನೆ. ಷರ್ಟಿನ ಜೇಬಿನಲ್ಲಿ ಹತ್ತು ರೂಪಾಯಿಮಾತ್ರ ಇದೆ. ಬುದ್ಧಿವಂತನಾದ ಕಳ್ಳನೊಬ್ಬನು ಯಾರಿಗೂ ಗೊತ್ತಾಗದಂತೆ ಆ ಹತ್ತು ರೂಪಾಯಿಯನ್ನು ಕದ್ದುಕೊಂಡು ಹೋಗುತ್ತಾನೆ. ಮಾರ್ಕೆಟ್ಟಿಗೆ ಹೋಗಿ ಜೇಬು ನೋಡಿಕೊಂಡಾಗಲೇ ಹಣವಿಲ್ಲವೆನ್ನುವುದು ಅವನಿಗೆ ತಿಳಿಯುವುದು. ಆಗ ``ಅಯ್ಯೋ! ಅನ್ಯಾಯವಾಯಿತು! ಹತ್ತು ರೂಪಾಯಿ ಕಳುವಾಗಿಹೋಯಿತು'' ಎಂದು ಗೋಳಾಡುತ್ತಾನೆ. ಆದರೆ ಹೋದ ರೂಪಾಯಿ ಹೋದಂತೆಯೇ.

ಇನ್ನೊಂದು ಪರಿಸ್ಥಿತಿಯಲ್ಲಿ ಇದನ್ನು ನೋಡೋಣ. ಆ ಮನುಷ್ಯನು ಈಗಲೂ ಹತ್ತು ರೂಪಾಯಿ ತೆಗೆದುಕೊಂಡು ಅಂಗಡಿಗೆ ಹೋಗುತ್ತಿದ್ದಾನೆ. ದಾರಿಯಲ್ಲಿ ಬಡವನಾದ ಒಬ್ಬ ಹುಡುಗನು ಬಂದು, ``ನಾನು ಸ್ಕೂಲಿಗೆ ಫೀಜು ಕೊಡಲು ಸಾಧ್ಯವಿಲ್ಲದೆ ಕಷ್ಟ ಪಡುತ್ತಿದ್ದೇನೆ. ತಾವು ಹತ್ತು ರೂಪಾಯಿ ಕೊಟ್ಟರೆ ಬಹಳ ಉಪಕಾರವಾಗುತ್ತದೆ'' ಎಂದು ಕೇಳುತ್ತಾನೆ. ಆಗ ಆ ಮನುಷ್ಯನ ಮನಸ್ಸು ಕರಗಿತು. ಜೇಬಿನಲ್ಲಿದ್ದ ಹಣವನ್ನು ಹುಡುಗನಿಗೆ ಕೊಟ್ಟು ತರಕಾರಿ ತೆಗೆದುಕೊಳ್ಳದೆ ಇದ್ದರೂ ಕೂಡ ಬಹಳ ಸಂತೋಷವಾಗಿ ಮನೆಗೆ ಹಿಂತಿರುಗುತ್ತಾನೆ.

ಕಳ್ಳತನವಾದಾಗಲೂ ಅವನ ಹತ್ತಿರದಿಂದ ಹತ್ತು ರೂಪಾಯಿ ಹೋಯಿತು; ದಾನ ಮಾಡಿದಾಗಲೂ ಅವನ ಹತ್ತಿರದಿಂದ ಹತ್ತು ರೂಪಾಯಿ ಹೋಯಿತು. ಆದರೆ ಮೊದಲನೆ ಪರಿಸ್ಥಿತಿಯಲ್ಲಿ ಹಣ ತಾನಾಗಿಯೇ ಅವನನ್ನು ಬಿಟ್ಟುಹೋಯಿತು. ಎರಡನೆಯ ಪರಿಸ್ಥಿತಿಯಲ್ಲಿ ಹಣವನ್ನು ಅವನು ತಾನಾಗಿಯೇ ಬಿಟ್ಟುಬಿಟ್ಟನು. ನಾವಾಗಿಯೇ ಹಣವನ್ನು ಬಿಟ್ಟುಬಿಟ್ಟರೆ,

\begin{shloka}
``ತೋಷಾಯ ಚ ಶ್ರೇಯಸೇ''
\end{shloka}

-ಎಂದು ಶ್ರೀ ಶಂಕರರು ಹೇಳಿದಂತೆ ದಾನಮಾಡಿದವನಿಗೆ ಸಂತೋಷವಿರುತ್ತದೆ, ಅಷ್ಟೇ ಅಲ್ಲದೆ ಫಲವೂ ಉಂಟು. ಆದರೆ-

\begin{shloka}
``ಸ್ವಾತಂತ್ರ್ಯಾದ್ವಿಷಯಾಃ ಪ್ರಯಾನ್ತಿ ಯದಮೀಶೋಕಾಯ''
\end{shloka}

-ಎಂದು ಹೇಳಿರುವಂತೆ, ಪ್ರಪಂಚದಲ್ಲಿ ಮನುಷ್ಯನು ಪಂಚೇಂದ್ರಿಯಗಳಿಗಾಗಿ ಎಷ್ಟೋ ವಸ್ತುಗಳನ್ನು ಸೇರಿಸಿ ಇಟ್ಟುಕೊಂಡಿರುವನು. ಆ ವಸ್ತುಗಳು ತಾವಾಗಿಯೇ ಅವನನ್ನು ಬಿಟ್ಟು ಬಿಟ್ಟು ಹೊರಟು ಹೋದರೆ ಅದಕ್ಕಾಗಿ ಅವನು ಅಳುತ್ತಾನೆ. ಆದರೆ ಅವನೇ ಆ ವಸ್ತುಗಳ ಅಸ್ಥಿರತೆಯನ್ನು, ಫಲಹೀನತೆಯನ್ನೂ ತಿಳಿದುಕೊಂಡು ಅವುಗಳನ್ನು ಬಿಟ್ಟು ಬಿಡುವುದೇ ಆದರೆ ಅವನಿಗೆ-

\begin{shloka}
`ಸುಖಮಯಂ ನಿಃಶ್ರೇಯಂ ಶ್ರೇಯಸೇ'
\end{shloka}

-ಎಂದು ಹೇಳಿದಂತೆ ಮಧುರವಾದ ಮೋಕ್ಷವೇ ದೊರೆತುಬಿಡುವುದು.

`ನಮ್ಮ ಹಣದಿಂದ ಎಷ್ಟು ಸಾಧಿಸಬಲ್ಲೆವು? ಇರುವ ಹಣವನ್ನು ಎಷ್ಟು ದಿನಗಳು ನಾವೇ ಇಟ್ಟುಕೊಂಡಿರಬಲ್ಲೆವು?'-ಎನ್ನುವ ಚಿಂತೆ ಒಬ್ಬನಿಗೆ ಇರುವುದಾದರೆ ಆಗ ಅವನಿಗೆ ದಾನ ಮಾಡಬೇಕೆಂದು ತೋರುವುದು. ಆದರೆ ಮನುಷ್ಯರು ಹಾಗೆಲ್ಲಾ ಯೋಚಿಸುವುದಿಲ್ಲ. `ಅವನ ಹತ್ತಿರ ಒಂದು ಲಕ್ಷ ರೂಪಾಯಿ ಇದೆ. ಆದ್ದರಿಂದ ನನ್ನ ಹತ್ತಿರ ಎರಡು ಲಕ್ಷರೂಪಾಯಿ ಸೇರಬೇಕು! ಅವನ ಹತ್ತಿರ ಹತ್ತು ಇದ್ದರೆ ನಾನು ಕೋಟೀಶ್ವರನಾಗಬೇಕು. ಅವನ ಹತ್ತಿರ ಒಂದು ಕೋಟಿ ರೂಪಾಯಿ ಇದ್ದರೆ ನನಗೆ ಹತ್ತು ಕೋಟಿ ರೂಪಾಯಿ ಬೇಕು, ಎನ್ನುವಂತೆ ಮನುಷ್ಯನು ಯೋಚಿಸುತ್ತಾನೆ. ಹಾಗೆ ನೋಡಿದರೆ ಕೋಟೀಶ್ವರನೂ ಊಟ ಮಾಡುವುದೂ ಅದೇ! ಐದು ರೂಪಾಯಿ ಸಂಪಾದನೆ ಮಾಡುವ ಬಡವನ ಊಟ ಮಾಡುವುದೂ ಅದೇ! ಕೋಟಿ ರೂಪಾಯಿ ನನ್ನ ಹತ್ತಿರ ಇದೆ, ಎಂದುಕೊಂಡು ಕೋಟೀಶ್ವರನು ಚಿನ್ನವನ್ನು ಅಕ್ಕಿಯನ್ನಾಗಿ ಮಾಡಿ ಅದನ್ನು ಅನ್ನವನ್ನಾಗಿ ಮಾಡಿ ಬಡಿಸಿದರೆ ಊಟ ಮಾಡುತ್ತಾನೆಯೇ? ಇಲ್ಲವೇ ಇಲ್ಲ.

ಮೀಮಾಂಸಾ ಶಾಸ್ತ್ರದಲ್ಲಿ ಒಂದು ಕಡೆ `ಕೃಷ್ಣಲಂಶ್ರಪಯೇತ್' (ಭತ್ತದಷ್ಟು ಚಿನ್ನದ ಮಣಿಗಳು) ಎನ್ನುವುದು ಬರುವುದಕ್ಕೆ ಬದಲು ಕೃಷ್ಣತಿಲಂ (ಕರಿಎಳ್ಳು) ಎಂದು ತಿಳಿದು ಅದು ಬರುವುದನ್ನು ಕುರಿತು ಹೇಳಿದೆ. ಆದರೆ ಚಿನ್ನವನ್ನು ಅಕ್ಕಿಯನ್ನಾಗಿ ಮಾಡಿ ಅದನ್ನು ಅನ್ನವನ್ನಾಗಿ ಬೇಯಿಸಿ ತಿನ್ನುವುದು ಆಗದ ಕೆಲಸ. ಸ್ಥಿತಿ ಹೀಗಿದ್ದರೂ ಕೂಡ ಮನುಷ್ಯನು ಆಸೆಯನ್ನು ಮಾತ್ರ ಬೆಳೆಸಿಕೊಂಡೇ ಹೋಗುತ್ತಿದ್ದಾನೆ. ನಮ್ಮ ಹತ್ತಿರ ಇರುವ ಹಣವನ್ನು ನಾವು ಹೇಗೆ ವಿನಿಯೋಗಿಸಬೇಕೆಂದು ಯೋಚಿಸಿ ನಮ್ಮ ಆಸೆಗಳನ್ನು ಒಂದು ಕಡೆ ನಿಲ್ಲಿಸಿಬಿಡಬೇಕು. ಹಣದ ವ್ಯಯ ಹೇಗಾಗುತ್ತದೆ ಎನ್ನುವುದನ್ನು ಕುರಿತು-

\begin{shloka}
``ದಾನಂ ಭೋಗೋ ನಾಶಃ ತಿಸ್ರೋ ಗತಯೋ ಭವನ್ತಿವಿತ್ತಸ್ಯ |\\
ಯೋ ನ ದದಾತಿ ನ ಭುಂಕ್ತೇ ತಸ್ಯ ತೃತೀಯಾ ಗತಿರ್ಭವತಿ ||
\end{shloka}

(ಹಣಕ್ಕೆ ಕೊಡುವುದು, ಅನುಭವಿಸುವುದು, ನಾಶವಾಗುವುದು, ಈ ಮೂರು ದಾರಿಗಳು. ಯಾರು ಕೊಡುವುದಿಲ್ಲವೋ. ಅನುಭವಿಸುವುದಿಲ್ಲವೋ ಅವನ ಹಣ ನಾಶವನ್ನು ಪಡೆಯುತ್ತಿದೆ)

-ಎಂದು ಶ್ಲೋಕ ಹೇಳುತ್ತದೆ. ಇದರಂತೆ ದಾನ, ಭೋಗ, ನಾಶ ಈ ಮೂರು ವಿಧವಾಗಿ ಹಣ ವ್ಯಯವಾಗುತ್ತದೆ.

ಕೆಲವರು `ನನ್ನ ಮಗನಿಗಾಗಿ ಹಣವನ್ನು ಸೇರಿಸಿಡುತ್ತೇನೆ' ಎಂದು ಹೇಳುವರು ಮಗನು ಎಷ್ಟು ದಿನಗಳವರೆಗೆ ಆ ಹಣವನ್ನು ಖರ್ಚುಮಾಡದೆ ಇರಬಲ್ಲನೆನ್ನುವುದು ಒಂದು ಪ್ರಶ್ನೆ. ಅಷ್ಟೇ ಅಲ್ಲದೆ, ಮಗನಿಗೆ ಇಟ್ಟಿದ್ದ ಹಣ ಅವನಿಗೆ ದೊರೆಯುವುದೇ ಎನ್ನುವುದು ಸಂದೇಹದಿಂದ ಕೂಡಿದ ಮತ್ತೊಂದು ಪ್ರಶ್ನೆ.

ಹಲವು ದಾನಗಳನ್ನು ಮಾಡುತ್ತಿದ್ದ ಒಬ್ಬರನ್ನು ಅವರ ಸ್ನೇಹಿತರೊಬ್ಬರು `ನೀವು ಇಷ್ಟೆಲ್ಲಾ ದಾನಗಳನ್ನು ಮಾಡುತ್ತಿದ್ದೀರಲ್ಲಾ, ನಿಮಗೆ ಹಣ ಬೇಕಾಗಿಲ್ಲವೇ' ಎಂದು ಕೇಳಿದರು. ಅದಕ್ಕೆ ಮೊದಲನೆಯವರು `ನಾನು ಬಹಳ ಸ್ವಾರ್ಥಿ. ಆದ್ದರಿಂದಲೇ ನಾನು ಈಗ ದಾನ ಮಾಡುತ್ತಿದ್ದೇನೆ. ಸತ್ತಮೇಲೆ ಹಣವನ್ನು ತಲೆಯ ಮೇಲೆ ಇಟ್ಟುಕೊಂಡು ಹೋಗುತ್ತೇನೆಂದುಕೊಳ್ಳುವವನು ಈಗ ಹಣವನ್ನು ದಾನಮಾಡಬೇಕು. ಈಗ ದಾನ ಮಾಡಿದರೆ ಅದು ಪುಣ್ಯವಾಗಿ ಬದಲಾಗಿ ಒಬ್ಬನು ಸತ್ತ ಮೇಲೆ ಕೂಡ ಜೊತೆಯಲ್ಲಿ ಬರುವುದು. ಆದರೆ ಹಣದ ರೂಪದಲ್ಲಿ ಬರುವುದಿಲ್ಲ. ಆದ್ದರಿಂದ ಈಗಲೇ ನಾನು ಕೊಟ್ಟು ಬಿಡುತ್ತೇನೆ. ನಾನು ಸ್ವಾರ್ಥವನ್ನು ಯೋಚಿಸಿಯೇ ಕೊಡುತ್ತೇನೆಯೇ ಹೊರತು ಇತರರಿಗಾಗಿ ಅಲ್ಲ' ಎಂದರು. ಇದು ಬಹಳ ನ್ಯಾಯವಾದ ವಿಷಯವೆಂದು ನನಗೆ ತೋರಿತು. ಏಕೆಂದರೆ-

\begin{shloka}
`ಮೃತೋಪ್ಯರ್ಥಂ ನ ಮೋಕ್ಷ್ಯಾಮಿ ಬದ್ಧ್ಯಾನೇಷ್ಯಾಮಿ ಮೂರ್ಧನಿ |\\
ಇತಿಚೇತ್ ಸುದೃಢಾ ಬುದ್ಧಿಃ ಪಾತ್ರೇದೇಯಮಶಂಕಿತಮ್ ||'
\end{shloka}

(ಸತ್ತಮೇಲೂ ನಾನು ಹಣವನ್ನು ಬಿಡುವುದಿಲ್ಲ. ಅದನ್ನು ತಲೆಯ ಮೇಲೆ ಕಟ್ಟಿಕೊಂಡು ಹೋಗುವೆನೆನ್ನುವ ಸದೃಢವದ ಬುದ್ಧಿಯುಳ್ಳವನು ಸತ್ಪಾತ್ರನಿಗೆ ಸಂದೇಹವಿಲ್ಲದೆ ದಾನ ಕೊಡಬೇಕು)-ಎಂದೆಲ್ಲಾ ಹೇಳಲ್ಪಟ್ಟಿದೆ.

ಮನುಷ್ಯರು ಹಣದ ಮೇಲೆ ಆಸಕ್ತಿ ಇಟ್ಟುಕೊಂಡಿರುವುದರಿಂದ ಒಳ್ಳೆಯ ಕೆಲಸವನ್ನು ಮಾಡುವಾಗ `ನಾಳೆ ಮಾಡೋಣ, ನಾಳಿದ್ದು ಮಾಡೋಣ' ಎಂದು ಕೊಂಡು ಮುಂದಕ್ಕೆ ಹಾಕುತ್ತಾರೆ. ಆದರೆ-

\begin{shloka}
`ನ ಹಿ ಯಮೋಽಸ್ಯ ಕೃತಾಕೃತಮೀಕ್ಷತೇ‌ |'
\end{shloka}

(ಇವನು ಮಾಡಿದ್ದಾನೆಯೇ, ಮಾಡಿಲ್ಲವೇ ಎನ್ನುವುದನ್ನೆಲ್ಲಾ ಯಮನು ನೋಡುವುದಿಲ್ಲ)

-ಎಂದು ಹೇಳಿದಂತೆ `ಒಬ್ಬನು ಎರಡು ವರ್ಷಗಳಾದ ಮೇಲೆ ಧರ್ಮಮಾಡಬೇಕೆಂದಿದ್ದಾನೆ. ಆದ್ದರಿಂದ ಅವನು ಧರ್ಮವೆಲ್ಲಾ ಮಾಡಿದ ಮೇಲೆ ಅವನನ್ನು ಕರೆದುಕೊಂಡು ಹೋಗೋಣ' ಎಂದು ಯಮನು ಕಾದುಕೊಂಡಿರುವುದಿಲ್ಲ. ನಿರ್ದಿಷ್ಟವಾದ ಸಮಯಕ್ಕೆ ಯಮನು, ಮನುಷ್ಯರು ನೆರವೇರಿಸದ ಹಲವು ಕೆಲಸಗಳಿದ್ದರೂ ಕೂಡ ಅವರನ್ನು ಕರೆದುಕೊಂಡು ಹೋಗುವನು. ಆದ್ದರಿಂದ ನಾವು-

\begin{shloka}
`ಅಜರಾಮರವತ್ ಪ್ರಾಜ್ಞಃ ವಿದ್ಯಾಮರ್ಥಂ ಚ ಸಾಧಯೇತ್ |\\
ಗೃಹೀತ ಇವ ಕೇಶೇಷು ಮೃತ್ಯುನಾ ಧರ್ಮಾಮಾಚರೇತ್ ||'
\end{shloka}

(ಪ್ರಾಜ್ಞನಾದವನು ವಿದ್ಯೆಯನ್ನೂ ಹಣವನ್ನೂ ಗಳಿಸುವಾಗ ತನಗೆ ಮುಪ್ಪು ಸಾವು ಇಲ್ಲವೆಂದುಕೊಂಡು ಪ್ರಯತ್ನ ಪಡಬೇಕು. ಆದರೆ ತನ್ನ ತಲೆಯ ಮೇಲೆ ಯಮನು ಕುಳಿತಿದ್ದಾನೆಂದು ಧರ್ಮವನ್ನು ಮಾಡುತ್ತಿರಬೇಕು.)

-ಎಂದು ಹೇಳಿದಂತೆ ಧರ್ಮಮಾಡಬೇಕೆಂದುಕೊಂಡು ಮನಸ್ಸಿಗೆ ಬಂದೊಡನೆ ಅದನ್ನು ನೆರವೇರಿಸಬೇಕು. ಏಕೆಂದರೆ ಯಮನು ತಲೆಯ ಜುಟ್ಟನ್ನು ಬಲವಗಿ ಹಿಡಿದುಕೊಂಡಿದ್ದಾನೆ. ನಾವು ಎಷ್ಟು ದಿನಗಳು ಇರುವೆವು ಎನ್ನುವ ನಿಶ್ಚಯವೇ ಇಲ್ಲ. ಆದ್ದರಿಂದ ಧರ್ಮಚಿಂತನೆ ಬರುತ್ತಲೇ ಅದನ್ನು ನೆರವೇರಿಸಿಬಿಡಬೇಕು.

ಕಾಮ, ಕ್ರೋಧ, ಲೋಭ ಎನ್ನುವುದರಲ್ಲಿ ಕಾಮ ಕ್ರೋಧಗಳನ್ನು ಮನುಷ್ಯನು ಗೆಲ್ಲಬಹುದು, ಹೇಗೆಂದರೆ-

\begin{shloka}
`ವಯಸಿ ಗತೇ ಕಃ ಕಾಮವಿಕಾರಃ'
\end{shloka}

-ಎಂದು ಶಂಕರರು ಹೇಳುವಂತೆ ವಯಸ್ಸಾದರೆ ಕಾಮ ಹೇಗೆ ಇರಬಲ್ಲದು? ಆದ್ದರಿಂದ ಕಾಮ ವಯಸ್ಸಾದವರಲ್ಲಿ ಇರಲು ಸಾಧ್ಯವಿಲ್ಲ. ಅದೇ ರೀತಿ ಒಬ್ಬ ಮನುಷ್ಯನು ಶಿಥಿಲಗೊಂಡರೆ ಅವನ ಕ್ರೋಧವೂ ಶಿಥಿಲಗೊಳ್ಳುವುದು. ಆದರೆ ಲೋಭವೆನ್ನುವುದು ಮಾತ್ರ ಯಾವಾಗಲೂ ಇದ್ದೇ ಇರುವುದು.

ಶ್ರೀಮಂತನು `ಹಣ ಕೂಡಿಟ್ಟರೆ ಅದು ಹೆಚ್ಚಾಗುತ್ತಾ ಹೋಗುತ್ತದೆ' ಎಂದು ಯೋಚಿಸಿರುತ್ತಾನೆ. ಒಂದು ದಿನ ಅಕಸ್ಮಾತ್ತಾಗಿ ಅವನು ಮರಣವನ್ನು ಹೊಂದುತ್ತಾನೆ. ಅವನು ಇಟ್ಟಿರುವ ಹಣದ ಬಗ್ಗೆ ಯಾರಿಗೂ ತಿಳಿಯದು. ಕೊನೆಯಲ್ಲಿ ಯಾರುಯಾರೋ ಆ ಹಣವನ್ನೆಲ್ಲಾ ತಿಂದು ಹಾಕುವರು! ಹಲವರ ಬಾಳಿನಲ್ಲಿ ಹೀಗೆ ನಡೆದಿದೆಯೆಂದು ನಾನು ಕೇಳಿದ್ದೇನೆ. ಆದ್ದರಿಂದ ಭಗವಂತನು ನಮಗೆ ಶಕ್ತಿಕೊಟ್ಟಿರುವಾಗಲೇ ನಾವು ಒಳ್ಳೆಯ ಕೆಲಸಗಳನ್ನು ಮಾಡಬೇಕು. ಪುರಾಣದಲ್ಲಿರುವ ಒಂದು ಶ್ಲೋಕ ನೋಡಿ-

\begin{shloka}
``ಪಾತ್ರೇಭ್ಯೋ ದೀಯತೇ ನಿತ್ಯಂ ಅನಪೇಕ್ಷ್ಯ ಪ್ರಯೋಜನಮ್''
\end{shloka}

(ಯೋಗ್ಯತೆ ಇರುವವನಿಗೆ ಫಲವನ್ನು ಎದುರು ನೋಡದೆ ಕೊಡಲ್ಪಡುವ....) ಆದ್ದರಿಂದ ಕೊಡಲ್ಪಡುವ ದಾನ ಫಲ (ಪ್ರತ್ಯುಪಕಾರ)ವನ್ನು ಅಪೇಕ್ಷಿಸಿ ಕೊಡುವಂತಹದಲ್ಲ, ಫಲವನ್ನು ಅಪೇಕ್ಷಿಸದೆ ಮತ್ತೆ ಹೇಗೆ ದಾನ ಕೊಡಬೇಕು?

\begin{shloka}
`ಕೇವಲಂ ಧರ್ಮಬುದ್ಧ್ಯಾ ಯತ್ ಧರ್ಮದಾನಂ ಪ್ರಚಕ್ಷ್ಯತೇ |'
\end{shloka}

(ಧರ್ಮಬುದ್ಧಿಯಿಂದ ಕೊಡಲ್ಪಡುವುದು ಧರ್ಮದಾನವೆನ್ನಲ್ಪಡುವುದು.) - ಎಂದು ಹೇಳಲ್ಪಟ್ಟಂತೆ `ದಾನ ಮಾಡುವುದು ನನ್ನ ಕರ್ತವ್ಯ, ಆದ್ದರಿಂದಲೇ ಕೊಡುತ್ತೇನೆ' ಎನ್ನುವ ಭಾವನೆ ಇದ್ದರೆ ಅದು ಧರ್ಮದಾನವಾಗುವುದು.

ಕೆಲವರು, `ನನ್ನನ್ನು ಯಾರೂ ಕೇಳಲಿಲ್ಲ. ಕೇಳಿದರೆ ಕೊಡುತ್ತಿದ್ದೆ ಆದರೆ ಯಾರೂ ನನ್ನ ಹತ್ತಿರ ಕೇಳಲು ಬರಲಿಲ್ಲ, ನಾನು ಏನು ಮಾಡುವುದು?' - ಎಂದು ಕೇಳುವರು. ಆದರೆ ಒಬ್ಬನು ದಾನವನ್ನು ಕೊಟ್ಟು ಅದರಿಂದ ಪ್ರಯೋಜನವನ್ನು ಎದುರು ನೋಡುವುದಾದರೆ-

\begin{shloka}
`ಆಹೂಯ ದೀಯತೇ ದಾನಂ ತದನಂತಫಲಂ ಶ್ರುತಮ್ |'
\end{shloka}

(ಒಬ್ಬನು ಕರೆದು ದಾನ ಕೊಟ್ಟರೆ ಅದು ಅವನಿಗೆ ಅನಂತ ಫಲವನ್ನು ಕೊಡುವುದೆಂದು ವೇದದಲ್ಲಿದೆ.)

-ಎಂದು ಹೇಳಿದಂತೆ, ಒಳ್ಳೆಯ ಕೆಲಸಗಳನ್ನು ಮಾಡಿದುದರಿಂದ ತಾನಾಗಿ ಉಂಟಾಗುವ ಫಲದ ಅನಂತತೆಯನ್ನು ಕುರಿತು ಹೇಳಲಾಗುವುದಿಲ್ಲ. `ನನ್ನ ಮನೆಗೆ ದಾನ ಕೇಳುವವರು ಬರಲಿ. ನನ್ನ ಮನೆಯಲ್ಲಿರುವವರಿಗೆಲ್ಲಾ ನಾನು ಮಾಡುವ ದಾನ ತಿಳಿಯಲಿ' ಎನ್ನುವ ಉದ್ದೇಶದಿಂದ ಮಾಡಿದ ದಾನಕ್ಕೆ ಸ್ವಲ್ಪವೂ ಫಲವಿಲ್ಲ. ಆದ್ದರಿಂದ ದಾನ ಮಾಡುವ ಕಾಳಲ್ಲಿ, ಅವರು ಕೇಳಿದರೆ ಮಾತ್ರ ದಾನಕೊಡಬೇಕೆಂದು ನಾವು ಭಾವಿಸಬಾರದು. ನಾವಾಗಿಯೇ ದಾನವನ್ನು ಕೊಟ್ಟರೆ ಅದು ಶ್ರೇಯಸ್ಸನ್ನು ಕೊಡುವುದು.

ಮತ್ತೆ ಕೆಲವರು, `ನಾವು ಅನ್ಯಾಯಮಾರ್ಗದಲ್ಲಿ ಹಣವನ್ನು ಸಂಪಾದಿಸಿದ್ದೇವೆ! ಅನ್ಯಾಯವಾಗಿ ಸೇರಿಸಿದ ಹಣವನ್ನು ದಾನಮಾಡಿದರೆ ಏನು ಪ್ರಯೋಜನ?

\begin{shloka}
`ನ್ಯಾಯೋಪಾರ್ಜಿತ ವಿತ್ತೇನ ಕರ್ತವ್ಯಂ ಹ್ಯಾತ್ಮರಕ್ಷಣಮ್'
\end{shloka}

(ನ್ಯಾಯದಿಂದ ಸಂಪಾದಿಸಿದ ಹಣದ ಮೂಲಕವೇ ಆತ್ಮರಕ್ಷಣೆ ಮಾಡಿಕೊಳ್ಳಬೇಕು.)

-ಎಂದು ಹೇಳಿದಂತೆ, `ನ್ಯಾಯವಾಗಿ ಸೇರಿಸಿಟ್ಟಿರುವ ಹಣವನ್ನು ತಾನೇ ಧರ್ಮಕಾರ್ಯಗಳಿಗೆ ತೊಡಗಿಸಬೇಕು' ಎನ್ನುತ್ತಾರೆ.

ಒಬ್ಬನು ಅನ್ಯಾಯ ಮಾರ್ಗದಲ್ಲಿ ಹಣವನ್ನು ಸಂಪಾದಿಸಿ ಅದನ್ನು ಹಾಗೆಯೇ ಬಿಟ್ಟು ಸತ್ತುಹೋದರೆ `ಅವನು ಅನ್ಯಾಯವಾಗಿ ಹಣವನ್ನು ಸಂಪಾದಿಸಿದನು' ಎನ್ನುವ ಕೆಟ್ಟ ಹೆಸರೂ, ಅದಕ್ಕೆ ತಕ್ಕ ಫಲವೂ ಉಂಟಾಗುತ್ತದೆ. ಹಣವನ್ನು ಸೇರಿಸಿ ಇಟ್ಟಿದ್ದಕ್ಕಾಗಿ ಯಾವುದೇ ಒಳ್ಳೆಯ ಫಲ ದೊರೆಯುವುದಿಲ್ಲ. ಯಾವಾಗಲೂ ಹಣವನ್ನು ಕುರಿತು ಯೋಚಿಸುವವನಿಗೆ ಈ ಪ್ರಪಂಚವನ್ನು ಬಿಟ್ಟುಹೋಗಲು ಮನಸ್ಸೇ ಬರುವುದಿಲ್ಲ. ಅಂಥವರು ಸತ್ತಮೇಲೆ ಪ್ರೇತವಾಗಿ ಹಣದ ಸುತ್ತ ಸುತ್ತುತ್ತಿರುತ್ತಾರೇನೋ! ಆದ್ದರಿಂದ ಅವರು ಸುಲಭವಾಗಿ ಮತ್ತೊಂದು ಜನ್ಮವನ್ನು ಎತ್ತಲಾರರು,

\begin{shloka}
`ಯೋಽಸದ್ಭ್ಯಾಂ ಪ್ರತಿಗೃಹ್ಯಾಪಿ ಪುನಃ ಸದ್ಭ್ಯಃಪ್ರಯಚ್ಚತಿ |\\
ಆತ್ಮಾನಂ ಸಂಕ್ರಮಂ ಕೃತ್ವಾ ಪರಾನ್ ತಾರಯತೇ ಹಿ ಸಃ ||'
\end{shloka}

[ಕೆಟ್ಟವರಿಂದ (ವಸ್ತುಗಳನ್ನು) ಪಡೆದು ಮತ್ತೆ ಅದನ್ನು ಒಳ್ಳೆಯವರಿಗೆ ಕೊಡುವವನು ತನ್ನನ್ನು ಸೇತುವೆಯನ್ನಾಗಿ ಮಾಡಿ ಇತರರನ್ನೂ ದಾಟಿಸುವವನಾಗುತ್ತಾನೆ.]

ಆದರೆ ಅಸತ್ಪುರುಷರಿಂದ ಅನ್ಯಾಯವಾಗಿ ಹಣವನ್ನು ಸಂಪಾದಿಸುವ ವ್ಯಕ್ತಿ ಆ ಹಣವನ್ನು ಸತ್ಪುರುಷರಿಗೆ ದಾನವಾಗಿ ಕೊಟ್ಟರೆ ಅದರಿಂದ ಅಸತ್ಪುರುಷರಿಗೂ ಪುಣ್ಯ ಉಂಟಾಗುತ್ತದೆ. ದಾನ ಮಾಡುವವನು ಪುಣ್ಯವನ್ನೂ ಅಸತ್ಪುರುಷರನ್ನೂ ಸೇರಿಸುವ ಸೇತುವೆಯಾಗಿ ಬೆಳಗುತ್ತಾನೆ.

ಅನ್ಯಾಯ ಮಾರ್ಗದಲ್ಲಿ ಹಣವನ್ನು ಸಂಪಾದಿಸುವುದು ತಪ್ಪು ಎನ್ನುವುದು ಬೇರೆ ವಿಷಯ. ಆದರೆ ಆ ಹಣವನ್ನು ಮೊದಲು ಕೊಟ್ಟು ಕೆಟ್ಟವರಿಗೆ ಸ್ವಲ್ಪ ಮಟ್ಟಿಗೆ ಪುಣ್ಯ ಉಂಟಾಗುತ್ತದೆ ಎನ್ನುವುದು ತಾತ್ಪರ್ಯ. ಆದ್ದರಿಂದ,

\begin{shloka}
`ಆತ್ಮಾನಂ ಸಂಕ್ರಮಂ ಕೃತ್ವಾ ಪರಾನ್ ತಾರಯತೇ ಹಿ ಸಃ ||'
\end{shloka}

-ಎಂದು ಹೇಳುವುದರಿಂದ `ಯಾರು ಯಾರಿಂದಲೋ ಯಾವ ದಾರಿಯಲ್ಲೋ ಸಂಪಾದಿಸಿದ ಹಣವನ್ನು ಹೇಗೆ ದಾನ ಮಾಡುವುದು' ಎಂದು ಒಬ್ಬನು ಯೋಚಿಸಬೇಕು. ಹಾಗೆ ಸಂಪಾದಿಸಿದ ಹಣವನ್ನು ದಾನ ಮಾಡಿದರೂ ಕೂಡ ಅದು ಕೊಟ್ಟವನಿಗೂ ಮೊದಲು ಕೊಟ್ಟ ಅಸತ್ಪುರುಷನಿಗೂ ಪುಣ್ಯವನ್ನೇ ಕೊಡುತ್ತದೆ.

ಕೆಲವರು, `ನನಗೆ ಐಶ್ವರ್ಯ ಹೆಚ್ಚಾಗಿ ಇಲ್ಲವಲ್ಲಾ! ನಾನು ಏನು ಮಾಡುವುದು? ನನ್ನ ಹತ್ತಿರ ಐಶ್ವರ್ಯವಿದ್ದಿದ್ದರೆ ಕೊಡಬಹುದಾಗಿತ್ತು'-ಎಂದು ಹಲಬುವರು. ಅದು ಸರಿಯಲ್ಲ, ಏಕೆಂದರೆ ತಿಂಗಳಿಗೆ ನೂರು ರೂಪಾಯಿ ಸಂಬಳವನ್ನು ಪಡೆಯುವ ಒಬ್ಬ ವ್ಯಕ್ತಿ `ನನಗೆ ತಿಂಗಳಿಗೆ ಸಾವಿರ ರೂಪಾಯಿ ಸಂಬಳ ಬಂದರೆ ನಾನು ದಾನಮಾಡುವೆನು' ಎಂದುಕೊಳ್ಳುತ್ತಾನೆ. ಅವನು ಆಶಿಸಿದಂತೆ ಸಾವಿರ ರೂಪಾಯಿ ಸಂಬಳ ಬಂದರೆ' ಹತ್ತು ಸಾವಿರ ರೂಪಾಯಿ ಸಂಬಳ ಪಡೆಯುವವರೇ ದಾನ ಮಾಡದೆ ಇರುವಾಗ ನಾನು ಹೇಗೆ, ಮಾಡುವುದು?' ಎಂದು ಕೇಳುತ್ತಾನೆ, ಹತ್ತು ಸಾವಿರ ರೂಪಾಯಿ ಬಂದರೆ ಲಕ್ಷ ರೂಪಾಯಿ ಸಂಪಾದಿಸುವವರೇ ದಾನ ಮಾಡುವುದಿಲ್ಲವೆಂದು ನ್ಯೂನತೆಯನ್ನು ತೋರಿಸುತ್ತಾನೆ. ತಾನು ಮಾತ್ರ ಏನೂ ಕೊಡುವುದಿಲ್ಲ.

ಆದ್ದರಿಂದ ಹೀಗೆ ನಾವು ಹೆಚ್ಚು ಹೆಚ್ಚಾಗಿ ಹಣ ಬರಬೇಕೆಂದುಕೇಳುತ್ತೇವೆಯೇ ಹೊರತು ದಾನವನ್ನು ಯಾವಾಗ ಮಾಡುವೆವು? ನಾವು ಊಟ ಮಾಡುವ ಊಟದಲ್ಲಿ ಒಂದು ಭಾಗವನ್ನು ಹಸಿವಾಗಿರುವವನಿಗೆ ಕೊಡಬಹುದಲ್ಲಾ! ನಾವೇ ಎಲ್ಲವನ್ನೂ ತಿನ್ನುವುದರಿಂದ ಯಾವ ವಿಧವಾದ ಸಾಧನೆಯನ್ನೂ ಮಾಡಿದಂತೆ ಆಗುವುದಿಲ್ಲ. ಆದರೆ ಬಡವನು ಊಟವೇ ಇಲ್ಲದೆ ಹೋದರೆ ಸತ್ತು ಹೋಗುವನು. ಆದ್ದರಿಂದ ನಾವು ಅವನಿಗೆ ಅರ್ಧವನ್ನು ಕೊಟ್ಟರೆ ಅವನೂ ಊಟ ಮಾಡಬಹುದು. ಅನಂತರ ಅವನೂ ವಸ್ತುಸಂಗ್ರಹ ಮಾಡಬಹುದು.

ಹೀಗೆಲ್ಲಾ ಪುರಾಣಗಳಲ್ಲಿ ಹೇಳಲ್ಪಟ್ಟಿದೆ. ಆದ್ದರಿಂದ ಜನರಿಗೆ ಹೆಚ್ಚು ಹೆಚ್ಚಾಗಿ ಹಣ ಬಂದಾಗ, ದಾನ ಮಾಡಿದರೇನೇ `ಶ್ರೇಯಸ್ಸು' -ಎನ್ನುವ ಭಾವನೆ ಉಂಟಾಗಬೇಕು, (ಇತರರು ಸಂತೋಷಪಡಲಿ ಎನ್ನುವುದಕ್ಕಾಗಿ ಮೊದಲು ದಾನ ಮಾಡುತ್ತೇನೆಂದು ಘೋಷಿಸಿ ಅದನ್ನು ಕೊಡನೆ ಇರುವವರೂ ಇದ್ದಾರೆ).

ಒಬ್ಬ ಅರಸನ ಆಸ್ಥಾನದಲ್ಲಿ ಸಂಗೀತಗಾರನೊಬ್ಬನು ಬಹಳ ಚೆನ್ನಾಗಿ ಹಾಡಿದನು. ಹಾಡನ್ನು ಕೇಳಿದ ಅರಸನು `ಐವತ್ತು ಸಾವಿರ ರೂಪಾಯಿ ಇವರಿಗೆ ಬಹುಮಾನ ಕೊಡಿ' ಎಂದು ಆಜ್ಞಾಪಿಸಿದನು. ಸಂಗೀತಗಾರನು ಸಂತೋಷಪಟ್ಟು ಅರಮನೆಯ ಖಜಾಂಚಿಯ ಹತ್ತಿರಕ್ಕೆ ಹೋಗಿ ಹಣವನ್ನು ಕೇಳಿದನು. ಅವನು `ಅರಸನು ಆಜ್ಞಾಪಿಸಿದರೂ ನಾನು ಕೊಡುವುದಿಲ್ಲ' ಎಂದನು. ಇದನ್ನು ಕೇಳಿದ ಸಂಗೀತಗಾರನು ಅರಸನ ಬಳಿಗೆ ಬಂದು ನಡೆದ ಘಟನೆಯನ್ನು ತಿಳಿಸಿದನು. ಅರಸನು ಸಂಗೀತಗಾರನನ್ನು ನೋಡಿ `ಅವರು ಕೊಡುವುದಿಲ್ಲವೆಂದು ಹೇಳಿದರೆ ತಪ್ಪೇನು? ನೀನು ನಿನ್ನ ಹಾಡಿನಿಂದ (ಶಬ್ದದಿಂದ) ನನ್ನನ್ನು ಸಂತೋಷಪಡಿಸಿದೆ. ನಾನು ನಿನಗೆ ಐವತ್ತು ಸಾವಿರ ರೂಪಾಯಿ ಕೊಡುತ್ತೇನೆನ್ನುವ ಶಬ್ದಗಳಿಂದ ನಿನ್ನನ್ನು ಸಂತೋಷಪಡಿಸಿದೆ. ನೀನು ಕೂಗಿಕೊಂಡಾಗ ನನಗೆ ಸಂತೋಷವಾಯಿತು, ನಾನು ಕೂಗಿಕೊಂಡಾಗ ನಿನಗೆ ಸಂತೋಷವಾಯಿತು. ಈಗ ನಿನಗೆ ಸಂತೋಷವಾಗದೆ ಇದ್ದರೆ ಅದಕ್ಕೆ ನಾನು ಹೊಣೆಯಲ್ಲ' ಎಂದನಂತೆ.

ಆದ್ದರಿಂದ `ದಾನವನ್ನು ಕೊಡುತ್ತೇನೆ' ಎನ್ನುವ ಮಾತುಗಳನ್ನು ಕೇಳಿದರೆ ಇತರರಿಗೆ ಸಂತೋಷವಾಗುತ್ತದೆ.‌ (ನಿಜವಾಗಿಯೂ ಅದೇ ರೀತಿ ನಡೆದುಕೊಳ್ಳಬೇಕು.) ಜನರಿಗೆ ಇಂಥ ಧರ್ಮಚಿಂತನೆ ಇದ್ದರೆ ಒಳ್ಳೆಯ ಕೆಲಸಗಳು ನಡೆಯುವುವು. ಆದ್ದರಿಂದ ಭಗವಂತನು ಈ ಭಾವನೆಯನ್ನು ಎಲ್ಲರಿಗೂ ಕರುಣಿಸಲೆಂದು ಪ್ರಾರ್ಥಿಸಿ ಈ ಭಾಷಣವನ್ನು ಮುಗಿಸುತ್ತಿದ್ದೇನೆ.
















































































