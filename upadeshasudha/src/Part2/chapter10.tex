\chapter{ಇಂದ್ರಿಯಾಧೀನತೆ ನಾಶಕ್ಕೆ ಕಾರಣ}\label{chap10}

\begin{shloka}
ಗಂಗಾಪೂರ ಪ್ರಚಲಿತ ಜಟಾಸ್ರಸ್ತ ಭೋಗೀಂದ್ರಭೀತಾಂ\\
ಅಲಿಂಗಂತೀಮಚಲತನಯಾಂ ಸಸ್ಮಿತಂ ವೀಕ್ಷಮಾಣಃ |\\
ಲೀಲಾಪಾಂಗೈಃ ಪ್ರಣತಜನತಾಂ ನಂದಯಂಶ್ಚಂದ್ರಮೌಲಿಃ\\
ಮೋಹಧ್ವಾಂತಂ ಹರತು ಪರಮಾನಂದ ಮೂರ್ತಿಶ್ಶಿವೋ ನಃ ||
\end{shloka}

ಎಲ್ಲರಿಗೂ ಭಗವಂತನ ಪಾದಾರವಿಂದಗಳಲ್ಲಿ ಮನಸ್ಸನ್ನು ಇಡುವುದು ಬಹಳ ಇಷ್ಟ. ಆದರೆ `ಎಷ್ಟು ದಿವಸ ಆಗಮಿಕ ಸಾಧನೆಗಳನ್ನೂ ಅನುಷ್ಠಾನಗಳನ್ನೂ ಮಾಡಿದರೂ ಕೂಡ ಇನ್ನೂ ಭಗವಂತನ ದಯೆ ದೊರೆಯಲಿಲ್ಲವಲ್ಲಾ, ನಾನು ಏನು ಪರಿಹಾರ ಮಾಡಬೇಕು' ಎಂದು ಒಬ್ಬನು ಕೇಳಬಹುದು. ಇದರ ಕಾರನವನ್ನು ಕುರಿತು ಈಗ ವಿಚಾರ ಮಾಡೋಣ. ಅನೇಕರಿಗೆ ಭಗವಂತನ ಸಾಕ್ಷಾತ್ಕಾರವನ್ನು ಪಡೆಯಬೇಕೆಂಬ ಆಸೆ ಇದೆ. ಅದಕ್ಕಾಗಿ ಯಾವುದಾದರೂ ಸಾಧನೆಯನ್ನು ಕೈಗೊಳ್ಳಬೇಕೆಂದು ಆಸೆಯೂ ಇದೆ. ಆದರೂ ಅವರು ತಮ್ಮ ಲಕ್ಷ್ಯವನ್ನು ಪಡೆಯಲು ಸಾಧ್ಯವಾಗಲಿಲ್ಲ. ಇದನ್ನು ಕುರಿತು ನೀಲಕಂಠದೀಕ್ಷಿತರು ಒಂದು ಕಡೆ,

\begin{shloka}
ಕಾಮಂ ಜನಾಃ ಸ್ಮಯನ್ತೇ ಕೈಲಾಸವಿಲಾಸವರ್ಣನಾವಸರೇ |\\
ಸಾಧನಕಥನಾವಸರೇ ಸಾಚೀಕುರ್ವನ್ತಿ ವಕ್ತ್ರಾಣಿ ||
\end{shloka}

(ಕೈಲಾಸವನ್ನು ವರ್ಣಿಸುವಾಗ ಜನರು ಸಂತೋಷದಿಂದಿದ್ದಾರೆ. ಆದರೆ `ಸಾಧನೆಗಳು' ವರ್ಣಿಸಲ್ಪಡುವಾಗ ಅವರು ತಮ್ಮ ಮುಖಗಳನ್ನು ತಿರುಗಿಸಿಕೊಂಡು ಬಿಡುತ್ತಾರೆ.)

-ಎಂದಿದ್ದಾರೆ. ಪುರಾಣಗಳಲ್ಲಿ ನಾವು ಕೈಲಾಸನಾಥನನ್ನು ಕುರಿತು ಓದುವಗ ನಮ್ಮ ಮನಸ್ಸಿನಲ್ಲಿ ಆನಂದ ಉಂಟಾಗುತ್ತದೆ. ಹೀಗಿರುವ ಕೈಲಾಸನಾಥನನ್ನು ಒಮ್ಮೆಯಾದರೂ ನಾವು ನೋಡಲಾಗುವುದಿಲ್ಲವೇ ಎಂಬ ಆಸೆಯೂ ಉಂಟಾಗುತ್ತದೆ. ಆದ್ದರಿಂದ ಕೈಲಾಸವನ್ನು ಕುರಿತು ಕೇಳುವಾಗ ಜನರು ಬಹಳವಾಗಿ ತೃಪ್ತರಾಗುತ್ತಾರೆ. ಆದರೆ ಅಂತ ಕೈಲಾಸನಾಥನ ಸಾಕ್ಷಾತ್ಕಾರ ದೊರೆಯಬೇಕಾದರೆ ನಾವು ಮಾಡಬೇಕಾದುದು ಏನು ಎನ್ನುವುದನ್ನು ಹೇಳುವಾಗ-

\begin{shloka}
`ಸಾಧನಕಥನಾವಸರೇ ಸಾಚೀಕುರ್ವನ್ತಿ ವಕ್ತ್ರಾಣಿ'
\end{shloka}

-ಎಂದು ಹೇಳಿದಂತೆ `ಇವೆಲ್ಲಾ ನಡೆಯುವ ಸಾಧನೆಗಳಲ್ಲ, ಇವುಗಳು ನಮ್ಮಿಂದ ಆಗುವವಲ್ಲ' ಎನ್ನುವ ತೀರ್ಮಾನಕ್ಕೆ ಜನರು ಬರುತ್ತಾರೆ. ಏಕೆ ಅವರು ಅಂಥಾ ತೀರ್ಮಾನಕ್ಕೆ ಬರುತ್ತಾರೆಂದರೆ,

\begin{shloka}
ಇಂದ್ರಿಯಸ್ಯೇಂದ್ರಿಯಾರ್ಥೇ ರಾಗದ್ವೇಷೌ ವ್ಯವಸ್ಥಿತೌ |\\
ತರ್ಯೋನವಶಮಾಗಚ್ಛೇತ್ ತೌ ಹ್ಯಸ್ಯ ಪರಿಪಂಥಿನೌ ||
\end{shloka}

(ಇಂದ್ರಿಯಗಳಿಗೆ ವಿಷಯಗಳಲ್ಲಿ ಆಸಕ್ತಿಯೂ, ಅನಾಸಕ್ತಿಯೂ ಇದೆ. ಅವುಗಳಿಗೆ ವಶವಾಗಬಾರದು. ಅವುಗಳೇ ಅವನಿಗೆ ಶತ್ರುಗಳು.)

-ಎಂದು ಗೀತೆಯಲ್ಲಿ ಭಗವಂತನು ಹೇಳಿದಂತೆ ಇಂದ್ರಿಯಗಳಿಗೆ ಆಸಕ್ತಿ ಅನಾಸಕ್ತಿಗಳು (ರಾಗದ್ವೇಷಗಳು) -ಎರಡೂ ಇರುವುದನ್ನು ನೋಡಿದವರಾGಇ ಮನುಷ್ಯರಿದ್ದಾರೆ.

ಶಾಸ್ತ್ರಗಳಲ್ಲಿ ಅವರವರಿಗೆ ವಿಧಿಸಲ್ಪಟ್ಟ ಕರ್ಮಗಳನ್ನು ಉಪಯುಕ್ತವಾದ ಸಮಯದಲ್ಲಿ ಮಾಡಬೇಕೆಂದಿದ್ದರೂ ಅವುಗಳನ್ನು ಮಾಡುವುದರಲ್ಲಿ ಜನರಿಗೆ ಅಷ್ಟಾಗಿ ಮನಸ್ಸಿಲ್ಲ. ಉದಾಹರಣೆಗೆ- ಸೂರ್ಯಾಸ್ತಮಾನ ಕಾಲದ ಸಂಧ್ಯಾವಂದನೆ ಮಾಡಬೇಕೆನ್ನುವುದು ಶಾಸ್ತ್ರದಲ್ಲಿ ಹೇಳಿದ್ದರೆ, ಅವರು ಅದೇ ಕಾಲದಲ್ಲೇ ಬೇರೆ ಕೆಲಸಗಳಲ್ಲಿ ತೊಡಗಿರುವರು. ಏಕೆಂದರೆ ಸಂಧ್ಯಾವಂದನೆ ಮಾಡುವುದಕ್ಕೆ ಮನಸ್ಸಿಲ್ಲ, `ನಿಮಗೆ ಸಮಯ ಸಿಕ್ಕಿದಾಗ ಮಾಡಬಹುದು' -ಎಂದು ಹೇಳಿದರೆ ಅವರು ಒಂದು ವೇಳೆ ಮಾಡಬಹುದು. ಆದರೆ ನಮ್ಮ ಹಿಂದಿನವರು.-

\begin{shloka}
`ಸಂಧೌ ಸಂಧ್ಯಾಮುಪಾಸೀತ'
\end{shloka}

(ಸಂಧ್ಯೆಯನ್ನು ಸಂಧಿಕಾಲದಲ್ಲಿ ಮಾಡಬೇಕು)

-ಎಂದು ಹೇಳಿದ್ದಾರೆ. ಆದರೆ ಆಗ ಅವರೂ ಯಾವುದಾದರೂ ಸಿನಿಮಾಗೆ ಹೋಗಬೇಕಾದರೆ, ನಾಟಕಕ್ಕೆ ಹೋಗಬೇಕಾಗಿದ್ದರೆ ಅಥವಾ ಯಾವುದಾದರೂ ಸಂಗೀತ ಕಛೇರಿಗೆ ಹೋಗಬೇಕಾಗಿದ್ದರೆ ಅವುಗಳಲ್ಲಿಯೇ ಅವರಿಗೆ ಸಮಯವಿರುವುದೇ ಹೊರತು ಸಂಧ್ಯಾವಂದನೆ ಮಾಡುವುದರಲ್ಲಿ ಅಥವಾ ಅಂಥಹುದರಲ್ಲಿ ಮನಸ್ಸನ್ನು ಇಡಲಾರರು.

\begin{shloka}
ಇಂದ್ರಿಯಸ್ಯೇಂದ್ರಿಯಾರ್ಥೇ
\end{shloka}

-ಎಂದು ಭಗವಂತನು ಹೇಳಿದಂತೆ, ಒಬ್ಬನು ತನ್ನ ಇಂದ್ರಿಯಗಳನ್ನೆಲ್ಲಾ ಸಂಯಮದಲ್ಲಿಟ್ಟುಕೊಂಡು ತನ್ನ ಶ್ರೇಯಸ್ಸಿಗಾಗಿ ಪ್ರಯತ್ನಪಡಬೇಕೆಂದು ತಿಳುವಳಿಕೆ ಕೊಟ್ಟರೂ, ಕೇಳುವುದಕ್ಕೆ ಕೆಲವರು ತಯಾರಾಗಿಲ್ಲ. ನಾವು ಕೇಳುವುದಕ್ಕೆ ತಯಾರಾಗಿದ್ದೇವೆ. ಆದರೆ ಯಾರು ಯಾವ ಉಪದೇಶವನ್ನು ಕೊಟ್ಟರೂ ಕೂಡ `ಕೇಳುವಾಗ ಬಹಳ ಚೆನ್ನಾಗಿರುತ್ತದೆ' ಎಂದು ಹೇಳುತ್ತೇವೆ. ಅದನ್ನು ಹಾಗೆ ಅನುಷ್ಠಾನ ಮಾಡುವಾಗ `ಅವುಗಳು ಬಹಳ ಕಷ್ಟವಾದವು; ಈ ಕಾಲದಲ್ಲಿ ಇದನ್ನು ಯಾರು ಮಾಡಬಲ್ಲರು? ಸ್ವಲ್ಪ ಸುಲಭವಾಗಿರುವುದನ್ನು ಮಾತ್ರ ಮಾಡಿದರೆ ಸಾಲದೇ? ಆದ್ದರಿಂದ ಸುಲಭವಾಗಿ ಕರ್ಮಗಳನ್ನು ಮಾತ್ರ ಮಾಡುತ್ತೇನೆ' ಎಂದು ಹೇಳುತ್ತೇವೆ. ಇದರ ಬಗ್ಗೆ ಅಪ್ಪಯ್ಯ ದೀಕ್ಷಿತರು ಒಂದು ಶ್ಲೋಕದಲ್ಲಿ,

\begin{shloka}
``ಉಪಾಯಂ ತೇ ಚೇತಃ ಪರಮಮುಪದಶಾಮಿ ಸ್ಮರರಿಪೋ\\
ಪದಾಂಭೋಜದ್ವಂದ್ವೇ ಪರವಶತಯಾ ರಜ್ಯಸಿ ಯತಃ |\\
ಅವಶ್ಯಂ ಭೋಕ್ತವ್ಯಾಂ ಅನುದಿನಂ ಅತರ್ಕ್ಯೋಽಪಗಮನಾಂ\\
ದಶಾಮಂತ್ಯಾಮಂತಃ ಸಪದಿ ಗಣಯ ಉಪಸ್ಥಿತತರಾಮ್ ||'
\end{shloka}

(ಮನಸ್ಸೇ! ನಿನಗೆ ಉತ್ತಮವಾದ ದಾರಿಯನ್ನು ಹೇಳುತ್ತೇನೆ. ನೀನು ಕಾಮದೇವನ ಶತ್ರುವಿನ ಪಾದಕಮಲಗಳಲ್ಲಿ ಬೇರೆ ವಿಷಯಗಳಲ್ಲಿ ಚಿಂತನೆಯುಳ್ಳವನಾಗಿದ್ದು, ಆಸಕ್ತಿಯುಳ್ಳವನಾಗಿದ್ದೀಯೇ. ಆದ್ದರಿಂದಲೆ, ಪ್ರತಿದಿನವೂ ಅವಶ್ಯವಾಗಿ, ಚಿಂತಿಸಲಾಗದ ಗತಿಯನ್ನು ಅನುಭವಿಸಿಯೇ ಆಗಬೇಕು. ಬೇರೆ ಸ್ಥಿತಿಯನ್ನು ಎಣಿಸುತ್ತಿರುವೆ.)-ಎಂದಿದ್ದಾರೆ.

ಇದೆ ವಿಷಯವನ್ನು ಕುರಿತು ನೀಲಕಂಠ ದೀಕ್ಷಿತರು ಬಹಳ ಸ್ವಾರಸ್ಯಕರವಾಗಿ ಶ್ಲೋಕವೊಂದನ್ನು ಕೊಟ್ಟಿದ್ದಾರೆ.

\begin{shloka}
`ವೇದಾ ವಾ ಸ್ಯುಃ ವಿತತರಚನಾಃ ವಿಸ್ಮರೇದೀಶ್ವರೋ ವಾ\\
ಧರ್ಮಾಧರ್ಮಸ್ಥಿತಿ ವಿರಚನಾಂ ಅಂತಕೋ ವಾ ಮೃಷಾಸ್ಯಾತ್ |\\
ನಿತ್ಯೋ ವಾ ಸ್ಯಾಂ ಅಹಮಿತಿ ಬಹೂನುಲ್ಲಿಖನ್ತಃ ಸಮಾಧೀನ್\\
ವೇದೋವೃತ್ತ್ಯಾ ಮುದಿತಮನಸಃ ಸರ್ವತೋ ನಿವೃತಾಃ ಸ್ಮಃ ||'
\end{shloka}

(ವೇದಗಳು ವ್ಯರ್ಥವಾದ ರಚನೆಗಳಾಗಬಹುದು; ಧರ್ಮವನ್ನೂ ಅಧರ್ಮವನ್ನೂ ಹೇಗೆ ವಿಂಗಡಿಸುವುದು ಎನ್ನುವುದನ್ನು ಈಶ್ವರನೂ ಮರೆಯಬಹುದು. ಅಥವಾ ಯಮನು ಸುಳ್ಳಾಗಿಯೂ, ನಾನು ನಿತ್ಯವಾಗಿಯೂ ಇರಬಹುದು; ಹೀಗೆ ಹಲವಾರು ವಿವರಣೆ ಕೊಡುತ್ತಾರೆ; ಸಂತೋಷದಿಂದ ಕೂಡಿದ ಮನಸ್ಸುಳ್ಳವರಾದ ನಾವು(ಇಂಥ) ಎಲ್ಲಾ ಎಡೆಗಳಿಂದಲೂ ತಿರಿಗಿಬಿಡುತ್ತೇವೆ.)

ಒಬ್ಬನು `ನಾನು ಪರಮಶಿವನನ್ನು ಕುರಿತು ಚಿಂತನೆಯೇ ಮಾಡಲಿಲ್ಲ. ಆದರೆ ನಾನು ತೃಪ್ತಿಯಾಗಿದ್ದೇನೆ. ತೃಪ್ತಿಯಾಗಿರುವನು ಏಕೆ ಪರಮಶಿವನನ್ನು ಚಿಂತಿಸಬೇಕು? ಪ್ರಪಂಚದಲ್ಲಿ ತೃಪ್ತಿ ಇಲ್ಲದವರು ಮಾತ್ರ ಪರಮಶಿವನನ್ನು ಚಿಂತಿಸಬೇಕು. ನಾನಾದರೋ ತೃಪ್ತಿಯಿಂದಿದ್ದೇನೆ. ಆದ್ದರಿಂದ ನಾನು ಏತಕ್ಕೆ ಚಿಂತನೆ ಮಾಡಬೇಕು'-ಎಂದು ಕೇಳುತ್ತಾನೆ. ಆದರೆ ಅವನ ಈ ಪ್ರಶ್ನೆ ಬಹಳ ತಪ್ಪಾದುದೆಂದು ನಾವು ಈಗ ನೋಡುವೆವು.

ಪ್ರಪಂಚದಲ್ಲಿ ಎಷ್ಟೋ ಕಷ್ಟಗಳನ್ನು ಮನುಷ್ಯರು ಪ್ರತಿದಿನವೂ ಅನುಭವಿಸುತ್ತಿದ್ದಾರೆ. ವೇದದಲ್ಲಿ,

\begin{shloka}
`ಅಹರಹಃ ಸಂಧ್ಯಾಮುಪಾಸೀತ'
\end{shloka}

(ಪ್ರತಿದಿನವೂ ಸಂಧ್ಯೋಪಾಸನೆಯನ್ನು ಮಾಡಬೇಕು.)

\begin{shloka}
`ಸತ್ಯಾನ್ನ ಪ್ರಮದಿತವ್ಯಮ್ |ಧರ್ಮಾನ್ನ ಪ್ರಮದಿತವ್ಯಮ್|'
\end{shloka}

(ಸತ್ಯದ ವಿಷಯದಲ್ಲಿ ಪ್ರಮಾದವಾಗಬಾರದು, ಧರ್ಮಕಾರ್ಯಗಳಿಂದ ಹಿಂಜರಿಯಬಾರದು.)

\begin{shloka}
``ಸ್ವಾಧ್ಯಾಯ ಪ್ರವಚನಾಭ್ಯಾಂ ನ ಪ್ರಮದಿತವ್ಯಮ್''
\end{shloka}

`ಕಲಿಯುವುದರಲ್ಲೂ ಕಲಿಸುವುದರಲ್ಲೂ ಗಮನ ಕಡಮೆಯಾಗಬಾರದು)

ಎಂದು ಹಲವು ವಿಧ ಕರ್ಮಗಳು ಮನುಷ್ಯರಿಗೆ ವಿಧಿಸಲ್ಪಟ್ಟಿವೆ. ಸರಕಾರ ಹಲವು ಕಾನೂನುಗಳನ್ನು ವಿಧಿಸಿದೆ. ಅದೇ ರೀತಿ ಭಗವಂತನು ಎಲ್ಲಾ ಸರಕಾರಗಳಿಗೂ ಮೇಲೆ ಸರಕಾರವಾದ್ದರಿಂದ, ಅವನು ಹೇಳಿರುವ ವೇದವನ್ನು ಒಬ್ಬನು ಅನುಸರಿಸಲಿಲ್ಲವೆಂದರೆ ಭಗವಂತನು ಸರಕಾರದಂತೆಯೇ ದಂಡನೆ ಕೊಟ್ಟೇ ಕೊಡುತ್ತಾನೆ. ಇದನ್ನು ಒಬ್ಬನು ಯೋಚನೆ ಮಾಡುವುದೇ ಇಲ್ಲ. ನಾವು ಈಗ ಬಾಳುವ ದೇಶದಲ್ಲಿರುವ ಸರಕಾರದ ಮಾತಿನಂತೆ ನಡೆಯದೆ ಇದ್ದರೆ, ಅನಂತರ ಈ ರಾಜ್ಯದಿಂದಲೇ ನಮ್ಮನ್ನು ಹೊರಹಾಕುತ್ತಾರೆನ್ನುವ ಭಯದಿಂದ `ಕಾನೂನುಗಳನ್ನು ಒಪ್ಪಲು ಸಾಧ್ಯವಿಲ್ಲ'ವೆಂದು ನಾವು ಹೇಳಲಾರೆವು. ಆದರೆ ಭಗವಂತನ ವಿಷಯದಲ್ಲಿ ಮಾತ್ರ ಜನರು `ಏಕೆ ಅವನ ವಿಧಿ-ನಿಯಮಗಳನ್ನು ಪಾಲಿಸಬೇಕು'-ಎನ್ನುವ ಭಾವನೆಯುಳ್ಳವರಾಗಿದ್ದಾರೆ. ಏಕೆಂದರೆ ಭಗವಂತನ ವಿಧಿ-ನಿಯಮಗಳೆಲ್ಲವೂ ವ್ಯರ್ಥವೆಂದು ಅಂಥಾ ಮನುಷ್ಯರ ಭಾವನೆಯಾಗಿದೆ.

ಸುಳ್ಳಾದ ಶಾಸ್ತ್ರಗಳನ್ನು ಏಕೆ ಒಪ್ಪಿಕೊಳ್ಳಬೇಕು. ಅವುಗಳೆಲ್ಲವೂ ಅನಾವಶ್ಯಕವೆಂದು ಅವರು ಭಾವಿಸುತ್ತಾರೆ.

ಭಗವಂತನು ಈಗತಾನೇ ಹುಟ್ಟಿದನೆಂದಿಲ್ಲ. ಅವನ ಹುಟ್ಟು ಅನಾದಿಯಾದುದು. ಆದರೂ ಕೆಲವರಿಗೆ ಒಂದು ಪ್ರಶ್ನೆ. `ಸಾಧಾರಣವಾಗಿ ಪ್ರಪಂಚದಲ್ಲಿರುವ ಮನುಷ್ಯರಿಗೆಲ್ಲಾ ಐವತ್ತು ಅಥವಾ ಅರವತ್ತು ವರ್ಷ ವಯಸ್ಸಾಗುತ್ತಲೇ ಮರೆಯುವಿಕೆ ಎನ್ನುವುದು ಉಂಟಾಗುತ್ತದೆ. ಆದರೆ ಐವತ್ತು ವರ್ಷ ವಯಸ್ಸಾಗುವವರಿಗೆ ಮರೆಯುವಿಕೆ ಎನ್ನುವುದು ಸಹಜವೆಂದಾದರೆ ಬಹಳ ಪುರಾತನನಾದ ಅವನಿಗೆ ಏಕೆ ಮರೆಯುವಿಕೆ ಉಂಟಾಗಬಾರದು? ಏನೋ ಸೃಷ್ಟಿಸಲು ಪ್ರಾರಂಭಿಸಿ ಭಗವಂತನು ವೇದಗಳನ್ನೂ ಉಪದೇಶಿಸಿಬಿಟ್ಟನು. ಈ ಸ್ಥಿತಿಯಲ್ಲಿ ಅವನು ಎಷ್ಟು ವಿಷಯಗಳನ್ನು ತನ್ನ ನೆನಪಿನಲ್ಲಿಟ್ಟುಕೊಂಡಿದ್ದಾನೆಂದು ಹೇಗೆ ಹೇಳುವುದು? ಭಗವಂತನಿಗೆ ತಾನು ಮಾಡಿದ ವಿಧಿ-ನಿಯಮಗಳೆಲ್ಲವೂ ನೆನಪಿನಲ್ಲಿವೆ ಎನ್ನುವ ಪರಿಸ್ಥಿತಿ ಇದ್ದರೆ ಅವನು ತಪ್ಪಿತಸ್ಥರಿಗೆ ದಂಡನೆ ಕೊಟ್ಟರೆ ಪರವಾಗಿಲ್ಲ, ಅವನಿಗೆ ತಾನು ಮಾಡಿದ ವಿಧಿ-ನಿಯಮಗಳು ಮರೆತು ಹೋಗುವುದಿಲ್ಲವೇ?' ಎನ್ನುವುದೇ ಪ್ರಶ್ನೆ.

ಭಗವಂತನಿಗೆ ಎಲ್ಲವೂ ಜ್ಞಾಪಕವಿರುತ್ತದೆ, ಅವನು ಎಲ್ಲವನ್ನೂ ಮಾಡಲು ಶಕ್ತಿಯುಳ್ಳವನು. ಅವನು `ಸರ್ವಸಮರ್ಥ'ನೆಂದು ಹೇಳಲ್ಪಟ್ಟಿದೆ. ಸಮರ್ಥನಾಗಿರುವ ಈಶ್ವರನಿಗೆ ಜ್ಞಾಪಕ ಮರೆಯುವಿಕೆ ಹೇಗೆ ಉಂಟಾಗುವುದು? ಕೆಲವರು ಐವತ್ತು ವರ್ಷಗಳಾಗುತ್ತಲೇ ತಮಗೆ ಮರೆಯುವಿಕೆ ಉಂಟಾಗಿದೆಯೆಂದು ಹೇಳುತ್ತಾರೆ ಆದರೆ ನಾವು ಕೆಲವರನ್ನು ನೋಡುತ್ತೇವೆ. ಅವರಿಗೆ ಮರೆಯುವಿಕೆ ಎನ್ನುವುದು ಎಂಭತ್ತು ವಯಸಾದರೂ ಸ್ವಲ್ಪವೂ ಇರುವುದಿಲ್ಲ. ಸಾಮಾನ್ಯರಾದ ಇಂಥಾ ಮನುಷ್ಯರಿಗೆ ವಯಸ್ಸಾದರೂ ಹೀಗೆ ಇರಲು ಸಾಧ್ಯವಾಗುವುದಾದರೆ ಭಗವಂತನಿಗೆ ಮರೆಯುವಿಕೆ ಏಕೆ ಉಂಟಾಗಬೇಕು? ನಾವೆಲ್ಲರೂ ಸಮರ್ಥರು. ಅಂದರೆ ಸಾಮರ್ಥ್ಯವುಳ್ಳವರೆಂದು ಹೆಸರು ಇಟ್ಟುಕೊಂಡರೆ ಭಗವಂತನು ಸರ್ವಸಮರ್ಥನು, ಅಂದರೆ ಎಲ್ಲವನ್ನೂ ಮಾಡಲು ಶಕ್ತಿಯುಳ್ಳವನು. ಆದ್ದರಿಂದ ಮರೆಯುವಿಕೆ ಎನ್ನುವುದು ಇಲ್ಲ.

ಆದ್ದರಿಂದ ಭಗವಂತನು, `ನನ್ನ ವಿಧಿ-ನಿಯಮಗಳನ್ನು ಅನುಸರಿಸಿ ನೀನು ನಡೆಯುವುದಿಲ್ಲ. ಆದ್ದರಿಂದ ನೀನು ದಂಡನೆಯನ್ನು ಅನುಭವಿಸಬೇಕು' -ಎಂದು ಧಾರಾಳವಾಗಿ ಹೇಳಬಹುದು. ಇನ್ನೊಂದು ಪ್ರತಿಕೂಲಪ್ರಶ್ನೆ. ನ್ಯಾಯಾಲಯದಲ್ಲಿ ಯಾವ ತೀರ್ಪನ್ನಾದರೂ ಕೊಂಡಬಹುದು. ಆದರೆ ನ್ಯಾಯಾಲಯದಲ್ಲಿ ತೀರ್ಪನ್ನು ಕೊಟ್ಟಮೇಲೆ ಆ ನ್ಯಾಯಾಲಯದಲ್ಲಿರುವ ಅಧಿಕಾರಿಗೆ ಇನ್ನೇನು ಮಾಡಲು ಸಾಧ್ಯವಿಲ್ಲ, ಪೋಲಿಸಿನವರು ಆ ತಪ್ಪಿತಸ್ಥನನ್ನು ಕರೆದುಕೊಂಡು ಜೈಲಿಗೆ ಹೋದ ಮೇಲೆ ಇವನಿಗೆ ದಂಡನೆ ಪ್ರಾರಂಭವಾಗುತ್ತದೆ. ಪೋಲಿಸಿನವರು ಕರೆದುಕೊಂಡುಹೋಗಲಿಲ್ಲವೆಂದರೆ ಇವನಿಗೆ ಸಂಬಂಧಪಟ್ಟ ತೀರ್ಪು ಪುಸ್ತಕದಲ್ಲಿ ಮಾತ್ರ ಇರುವುದು. ಅದೇ ರೀತಿ ಭಗವಂತನು ದಂಡನೆ ವಿಧಿಸಿದರೂ ಅದನ್ನು ಕೊಡುವವರು ಯಾರಿದ್ದಾರೆ ಎನ್ನುವುದೇ ಇನ್ನೊಂದು ಪ್ರತಿಕೂಲ ಪ್ರಶ್ನೆ.

ಭಗವಂತನು ವಿಧಿಸಿದ ದಂಡನೆಯನ್ನು ನಮಗೆಲ್ಲಾ ಕೊಡುವುದಕ್ಕಾಗಿಯೇ ಯಮಧರ್ಮರಾಜನು ಇರುವುದು. ಆ ಯಮಧರ್ಮರಾಜನು,

\begin{shloka}
`ಅಯಂ ಲೋಕೋ ನಾಸ್ತಿ ಪರ ಇತಿ ಮಾನೀ\\
ಪುನಃ ಪುನರ್ವಶಮಾಪದ್ಯತೇ ಮೇ |'
\end{shloka}

(ಯಾರು `ಈ ಪ್ರಪಂಚ ಮಾತ್ರವಿದೆ, ಬೇರೆ ಪ್ರಪಂಚವಿಲ್ಲ' ಎಂದು ಭಾವಿಸುತ್ತಾನೋ ಅವನು ಮತ್ತೆ ಮತ್ತೆ ನನ್ನ ವಶಕ್ಕೆ ಬರುತ್ತಾನೆ.)

\begin{shloka}
ಅಯಂ ಲೋಕೋ ನಾಸ್ತಿ ಪರ ಇತಿ ಮಾನೀ'
\end{shloka}

-ಎಂದಂತೆ ಕಳ್ಳನೂ ಭಾವಿಸುತ್ತಾನೆ. ನನ್ನ ಬುದ್ಧಿವಂತಿಕೆ ಮತ್ತು ಸಾಮರ್ಥ್ಯದ ಮುಂದೆ ಪೋಲೀಸಿನವನ ಬುದ್ಧಿವಂತಿಕೆ ಮತ್ತು ಸಾಮರ್ಥ್ಯ ಏನಿದೆ? ಎಷ್ಟೋಸಲ ಪೋಲೀಸಿನವನ ಕಣ್ಣಿಗೆ ಮಣ್ಣು ಎರಚಿದ್ದೇನೆಂದು ಕಳ್ಳನು ಹೇಳುತ್ತಲೇ ಇರುತ್ತಾನೆ. ಆದರೆ ಒಂದು ವಿಷಯವನ್ನು ತಿಳಿದುಕೊಳ್ಳಬೇಕು. ಒಂದು ಗಾದೆ ಇದೆ `ಕಳ್ಳನ ಹೆಂಡತಿ ಎಂದಿದ್ದರೂ ಮುಂಡೆ' ಎಂದು. ಏಕೆಂದರೆ, ಕಳ್ಳನು ಹತ್ತುಸಲ ಕಳ್ಳತನ ಮಾಡುತ್ತಾನೆ ಆಮೇಲೆ ಸಿಕ್ಕಿಕೊಳ್ಳುತ್ತಾನೆ. ಅದರಂತೆಯೇ ಈ ಪ್ರಪಂಚದಲ್ಲಿರುವರೂ ಕೂಡ ಇಹಲೋಕದಲ್ಲಿ ಅನುಭವಿಸುವ ಕರ್ಮ ಪ್ರಬಲವಾಗಿರುವವರಿಗೆ ಅವರು ಯಮನ ಹಿಡಿತಕ್ಕೆ ಸಿಕ್ಕದೆ ಇರಬಹುದು. ಒಬ್ಬನು ತಾನು ನಿತ್ಯವಾಗಿಯೇ ಇದ್ದು ಬಿಟ್ಟರೆ, ಆಗ ಯಮನ ಹಿಡಿತಕ್ಕೆ ಹೇಗೆ ಉಂಟಾಗುತ್ತದೆಂದು ಕೇಳಬಹುದು. ಆದರೆ ನಾವು ನಿತ್ಯರಾಗಲು ಸಾಧ್ಯವೇ ಇಲ್ಲ. ರೋಗಿಗಳು ಡಾಕ್ಟರ್‌ಗೆ ಹಣ ಕೊಡುತ್ತಲೆ ಇದ್ದಾರೆ. ಎಲ್ಲಿಯವರೆಗೆ? ಒಬ್ಬ ಡಾಕ್ಟರ್ ಸಾಲದೆಂದು ಭಾವಿಸಿ ಮದ್ರಾಸಿಗೆ ಬರುವನು. ಮದ್ರಾಸಿನಲ್ಲಿರುವ ಡಾಕ್ಟರ್‌ಗೆ ಹಣ ಕೊಟ್ಟು ನಂತರ ಲಂಡನ್‌ಗೆ ಹೋಗಬೇಕೆನ್ನುತ್ತಾನೆ. ಆದರೆ ಎಷ್ಟರವರೆಗೆ ಹಣ ಕೊಡುತ್ತಿರುವುದು? ಹಣ ಮುಗಿದು ಹೋದರೆ ಯಾವ ವೈದ್ಯನೂ ಇವನಿಗೆ ಉಪಯೋಗವಿಲ್ಲವೆನ್ನುವ ಸ್ಥಿತಿ ಉಂಟಾಗುತ್ತದೆ. ಇವನು ತನ್ನ ಶರೀರವನ್ನು ಬಿಡಬೇಕಾದುದೇ. ಆದರೆ ಮನುಷ್ಯರು `ನಿತ್ಯೋವಾಸ್ಯಾಸಿ' -ಎಂದು ಹೇಳಿದಂತೆ ತಾವು ಯಾವಾಗಲೂ ಸ್ಥಿರವಾಗಿರುವರೆಂದುಕೊಂಡಿದ್ದರೂ ಕೂಡ ಯಮನು ಒಂದು ದಿನ ಅವರಿಗೆ ದುಃಖವನ್ನು ಕೊಟ್ಟೇಕೊಡುವನು. ನಾವು ಎಷ್ಟು ಮಂದಿ ವೈದ್ಯರನ್ನು ನಮ್ಮ ಹತ್ತಿರ ಇಟ್ಟುಕೊಂಡು ನಮಗೆ ಚಿಕಿತ್ಸೆ ಮಾಡುವಂತೆ ಹೇಳಿದರೂ ಯಮನು ನಮ್ಮನ್ನು ತೆಗೆದುಕೊಂಡು ಹೋಗದೆ ಬಿಡುವುದಿಲ್ಲ. ಏಕೆಂದರೆ ಆ ಯಮನೇ ಆ ವೈದ್ಯರುಗಳನ್ನೂ ಸಹ ಒಂದು ದಿನ ತೆಗೆದುಕೊಂಡು ಹೋಗಲು ತಯಾರಾಗಿದ್ದಾನೆ. ಅಂಥ ಯಮನು ನಮ್ಮನ್ನು ಮಾತ್ರ ಏಕೆ ಬಿಟ್ಟು ಬಿಡುತ್ತಾನೆ?

ಮನೆಯಲ್ಲಿ ಇಟ್ಟಿದ್ದ ವಡೆಯೊಂದನ್ನು ಒಂದು ಚಿಕ್ಕ ಪ್ರಾಣಿ ತಿನ್ನುತ್ತಿತ್ತು; ಅದನ್ನು ನೋಡಿದ ದೊಡ್ಡ  ಇಲಿಯೊಂದು ಆ ವಡೆಯನ್ನು ತಿಂದು ಬಿಟ್ಟಿತು. ಅದನ್ನು ತಿನ್ನುತ್ತಿದ್ದ ಪ್ರಾಣಿಯನ್ನೂ ತಿಂದುಬಿಟ್ಟಿತು. ಅದೇ ರೀತಿಯಲ್ಲಿ ಯಮ ನಾವು ಯಾರಿಂದ ಕಾಪಾಡಲ್ಪಡುತ್ತೇವೆಂದು ಭಾವಿಸುತ್ತಿರುತ್ತೇವೋ, ಅಂಥ ವೈದ್ಯರನ್ನು ಮಾತ್ರ ಬಿಟ್ಟು ಬಿಡುವುದಿಲ್ಲ. ಅವರನ್ನೂ, ಅವರು ಚಿಕಿತ್ಸೆಗೆ ಒಳಪಡಿಸುವ ರೋಗಿಗಳನ್ನೂ ಅವನು ಒಂದು ದಿನ ಕರೆದುಕೊಂಡು ಹೋಗೇ ಹೋಗುತ್ತಾನೆ.; ಆದರೆ ಅಂಥ ಯಮನು ಒಬ್ಬನನ್ನು ಮಾತ್ರ ಏನೂ ಮಾಡಲಾರನು. ಆ ಒಬ್ಬನು ಯಾರು?

\begin{shloka}
`ಕಾಮಮಸ್ತು ಜಗತ್ಸರ್ವಂ ಕಾಲಸ್ಯಾಸ್ಯ ವಶಂವದಮ್ |\\
ಕಾಲಕಾಲಂ ಪ್ರಪನ್ನಾನಾಂ ಕಾಲಃ ಕಿಂ ನ ಕರಿಷ್ಯತಿ ||'
\end{shloka}

(ಈ ಕಲಿಕಾಲದ ವಶಕ್ಕೆ ಬಂದಿರುವ ಪ್ರಪಂಚವೆಲ್ಲಾ ಹೇಗೆ ಬೇಕಾದರೂ ಇರಲಿ. ಕಾಲನಿಗೆ ಕಾಲನಾಗಿರುವ ಕಾಲನು ಏನು ಮಾಡಬಹುದು?) ಯಾರು ಕಾಲನನ್ನೂ ಕೂಡ ಸುಟ್ಟು ಹಾಕಿದನೋ ಅಂಥ ಕಾಲಕಾಲನಿಗೇ ನಾವು ಶರಣಾದರೆ ಯಮ ಧರ್ಮರಾಜನು ಅಂಥವನನ್ನು ಏನೂ ಮಾಡಲಾರನು. ಆದರೆ ಅದಕ್ಕಾಗಿ ಯಾರೂ ಪ್ರಯತ್ನ ಪಡುವಂತೆ ತೋರುವುದಿಲ್ಲ.

ಪ್ರಪಂಚದಲ್ಲಿ ಮನುಷ್ಯರಿಗೆ ಹಲವು ಆಯುಧಗಳಿವೆ. ಅವುಗಳು ಯಾವುವು ಎಂದರೆ,

\begin{shloka}
ಶಬ್ದಾದಿಭಿಃ ಪಂಚಭಿರೇವ ಪಂಚ ಪಂಚತ್ವಮಾಪುಃ ಸ್ವಗುಣೇನ ಬದ್ಧಾಃ |\\
ಕುರುಂಗ-ಮಾತಂಗ-ಪತಂಗ-ಮೀನ-ಭೃಂಗಾ ನರಃ ಪಂಚಭಿರಂಚಿತಃ\\
\hspace{5.5cm} ಕಿಮ್ ||'
\end{shloka}

(ಜಿಂಕೆ, ಆನೆ, ದೀಪದ ಹುಳು, ಮೀನು, ಭೃಂಗ ಇವುಗಳು ಶಬ್ದಾದಿ ಪಂಚದಲ್ಲಿ ಸ್ವಗುಣದಿಂದ ಬದ್ಧರಾಗಿರುವುದರಿಂದ ನಾಶವನ್ನು ಪಡೆಯುತ್ತವೆ. ಇದರಲ್ಲಿಯೂ ಸಂಬಂಧವನ್ನು ಇಟ್ಟುಕೊಂಡಿರುವ ಮನುಷ್ಯನ ಗತಿ ಏನು?)

-ಎಂದು ಶಂಕರಭಗವತ್ಪಾದರು ಹೇಳಿದ್ದಾರೆ. ಸಾಮಾನ್ಯವಾಗಿ ಒಂದೊಂದು ವಿಷಯದಲಿಯೂ ಒಬ್ಬ ಜೀವನು ಆಸಕ್ತಿಯುಳ್ಳವನಾಗಿದ್ದು ಅದರಿಂದಲೇ ತನ್ನ ಪ್ರಾಣವನ್ನು ಕಳೆದುಕೊಳ್ಳುತ್ತಾನೆಂದು ಶ್ಲೋಕದಲ್ಲಿ ಅವರು ಹೇಳಿರುವುದನ್ನು ನಾವು ನೋಡುತ್ತೇವೆ.

ಜಿಂಕೆಯನ್ನು ಹಿಡಿಯಲು ಹೋಗುವ ಬೇಡನು, ಮೊದಲು ಕಾಡಿನಲ್ಲಿ ಮಧುರವಾದ ಸಂಗೀತಕ್ಕೆ ಏರ್ಪಾಡು ಮಾಡುವನು. ಜಿಂಕೆ ತನ್ನ ಕಿವಿಗಳಿಗೆ ಅಂಥ ಸಂಗೀತ ಬಹಳ ಅದ್ಭುತವಾಗಿದೆಯೆಂದುಕೊಂಡು ಶಬ್ದ ಬರುವ ಜಾಗಕ್ಕೆ ಓಡಿ ಬರುವುದು. ಆದರೆ ಈ ಜಾಗದಲ್ಲಿ ಬೇಡನು ಬಲೆಯೊಂದನ್ನು ಹಾಸಿರುವನು. ಶಬ್ದವನ್ನು ಕೇಳುವ ಆಸೆಯಿಂದ ಬರುವ ಜಿಂಕೆಗಳಿಗೆ ಕೆಳಗೆ ಹಾಸಿರುವ ಬಲೆ ಕಣ್ಣಿಗೆ ಬೀಳುವುದಿಲ್ಲ. ಯಾವುದೋ ಮಧುರವಾದ ಶಬ್ದ ಕೇಳಿ ಬರುತ್ತದಲ್ಲಾ ಎಂದು ಅದರಲ್ಲೇ ತೃಪ್ತಿ ಪಡೆಯುವ ಜಿಂಕೆಗಳು ಕೆಳಗೆ ಹಾಸಿರುವ ಬಲೆಯನ್ನು ನೋಡಲು ಬುದ್ಧಿ ಇಲ್ಲದವಾಗಿ ಬಿಡುತ್ತವೆ. ಅನಂತರ ನೇರವಾಗಿ ಬಲೆಯಲ್ಲಿ ಬಿದ್ದು ಸಿಕ್ಕಿಕೊಳ್ಳುವುವು. ಶಬ್ದವೆನ್ನುವ ಒಂದು ಇಂದ್ರಿಯಾಕರ್ಷಣೆಯಿಂದಾಗಿ ತಮ್ಮ ಪ್ರಾಣವನ್ನೇ ಕೊಡಬೇಕಾಗುತ್ತದೆ.

ಆನೆಗಳನ್ನು ಪಳಗಿಸುವುದರಲ್ಲಿ, ಹೊಸದಾಗಿ ಯಾವುದಾದರೂ ಆನೆ ದೊರೆತರೆ ಆ ಆನೆಯನ್ನು ಹಳ್ಳದಲ್ಲಿ ಹಾಕಿ ಬಿಡುತ್ತಾರೆ. ಆನೆಯನ್ನು ಹಳ್ಳದಲ್ಲಿ ಹಾಕಿದರೂ ಕೂಡ ಮನುಷ್ಯನು ಆ ಆನೆಯೊದನೆ ಹೋಗಿ ತಕ್ಷಣ ಹತ್ತಿರ ಪಳಗಿಸಿಕೊಂಡು ಬಿಡಲು ಸಾಧ್ಯವಿಲ್ಲ. ಅದು ಮನುಷ್ಯನ ಹತ್ತಿರಕ್ಕೆ ಬರಬೇಕಾದರೆ ಮೊದಲೇ ಪಳಗಿದ ಒಂದು ಹೆಣ್ನಾನೆಯನ್ನು ಇದಕ್ಕಾಗಿ ಆರಿಸಿ ಇಟ್ಟುಕೊಳ್ಳುತ್ತಾರೆ. ಹೇಗೆಂದರೆ ಈ ಹೆಣ್ಣಾನೆಯ ಹೊಟ್ಟೆಯ ಹತ್ತಿರ ಮಾವಟಿಗನು ತನ್ನನ್ನು ಒಂದು ಬಟ್ಟೆಯಿಂದ ಕಟ್ಟಿಕೊಂಡು ಈ ಹೆಣ್ಣಾನೆಯನ್ನು ಗಂಡಾನೆಯ ಹತ್ತಿರ ಹೋಗುವಂತೆ ಮಾಡುತ್ತಾನೆ. ಆ ಹೆಣ್ಣಾನೆ ಗಂಡಾನೆಯ ಹತ್ತಿರಕ್ಕೆ ಹೋದರೂ ಸಹ ಗಂಡಾನೆ ಏನೂ ಮಾಡುವುದಿಲ್ಲ. ಏಕೆಂದರೆ ತನ್ನ ಹತ್ತಿರ ಬರುತ್ತದೆಂದು ಸಂತೋಷವಾಗಿರುವುದು. ಪಳಗಿಸುವ ಮನುಷ್ಯನು ಹೆಣ್ಣಾನೆಗೆ `ಒಂದು ಸರಪಣಿಯನ್ನು ತೆಗೆದು ಗಂಡಾನೆಯ ಮೇಲೆ ಹಾಕು' ಎನ್ನುತ್ತಾನೆ. ಹೆಣ್ನಾನೆ ಹಾಗೆಯೇ ಮಾಡುವುದು. ಆಗ ಆ ಗಂಡಾನೆಗೆ ಅದು ಯಾವ ವಿಧವಾದ ಕೋಪವನ್ನೂ ಉಂಟುಮಾಡುವುದಿಲ್ಲ. ಏಕೆಂದರೆ, ಅದರ ಮಟ್ಟಿಗೆ ಹೆಣ್ಣಾನೆ ತಾನೆ ತನ್ನ ಕಾಲಿಗೆ ಸರಪಣಿ ಹಾಕುತ್ತದೆ ಎನ್ನುವುದರಿಂದ ಅದು ಸುಮ್ಮನೆ ಇರುವುದು, ಅನಂತರ ಹೆಣ್ಣಾನೆ, ಮಾವಟಿಗನ ಆಜ್ಞೆ ಪ್ರಕಾರ ಈ ಗಂಡಾನೆಯನ್ನು ಹಿಡಿದು ಎಳೆದರೂ ಕೂಡ ಹೆಣ್ಣಾನೆ ಎಲ್ಲೆಲ್ಲಿ ಓಡುವುದೋ, ಅಲ್ಲೆಲ್ಲಾ ಅದೂ ಓಡುತ್ತಿರುತ್ತದೆ. ಹೀಗೆ ಕೆಲವು ದಿನಗಳಲ್ಲಿ ಗಂಡಾನೆಯೂ ಮಾವಟಿಗನ ವಶಕ್ಕೆ ಬಂದು ಬಿಡುವುದು. ಇದೇ ರೀತಿ ಒಂದು ಹೆಣ್ಣಾನೆಯನ್ನು ಪಳಗಿಸಬೇಕಾದರೂ ಕೂಡ ಗಂಡಾನೆಯ ಸಹಾಯವನ್ನು ಪಡೆದು ಹೀಗೆ ಮಾಡುವರು. ಆದ್ದರಿಂದ, ಒಂದು ಹೆಣ್ಣಾನೆಯನ್ನು ಕಂಡರೆ `ಅದು ನನ್ನ ಸಂಗಾತಿ' ಎಂದುಕೊಂಡು ತನ್ನ ಶಕ್ತಿ ಎಷ್ಟಿದ್ದರೂ ಅದೆಲ್ಲವನ್ನು ಮರೆತು, ಇದಾದ ಮೇಲೆ ತನಗೆ ಮನುಷ್ಯನಿಂದ ಉಂಟಾಗುವ ಎಲ್ಲಾ ವಿಪತ್ತುಗಳನ್ನು ಮರೆತು, ತನ್ನ ಜೀವನವನ್ನು ಪೂರ್ತಿ ಆ ಆನೆ ಮಾವಟಿಗನಿಗೆ ಒಪ್ಪಿಸಿ ಬಿಡುತ್ತದೆ. ಇದಕ್ಕೆಲ್ಲಾ ಕಾರನ ಆನೆ; ಸ್ಪರ್ಶ ಸುಖದಲ್ಲಿ ಬಿದ್ದು ತನ್ನ ಬಾಳನ್ನೇ ಒಬ್ಬ ಮನುಷ್ಯನಿಗೆ ಅರ್ಪಣೆಮಾಡಿ ಬಿಡುವುದು.

ನಾವು ಮನೆಯಲ್ಲಿ ದೀಪದ ಹುಳುವನ್ನು ನೋಡಿದ್ದೇವೆ. ಎಲ್ಲಾದರೂ ದೀಪದ ಪ್ರಕಾಶವನ್ನು ಕಂಡರೆ ಆ ಹುಳು `ಈ ದೀಪ ನೋಡುವುದಕ್ಕೆ ಬಹಳ ಸುಂದರವಾಗಿದೆ' ಎಂದು ಭಾವಿಸಿ ಆ ದೀಪವನ್ನು ಹೊಡದು ಹೊಡದು ಕೊನೆಗೆ ಆ ದೀಪದಲ್ಲಿಯೇ ಬಿದ್ದು ತನ್ನ ಪ್ರಾಣವನ್ನು ಕಳೆದುಕೊಂಡುಬಿಡುತ್ತದೆ. ಇದಕ್ಕೆ ಕಾರಣವನ್ನು ನೋಡುವಾಗ ಬಹಳ ಸುಂದರವಾಗಿದೆಯೆಂದು ಭಾವಿಸಿ ಆ ಮನೆಯಲ್ಲಿ ಹುಳು ಆ ವಸ್ತುವಿನಿಂದಲೇ ತನ್ನ ಪ್ರಾಣವನ್ನು ಕಳೆದುಕೊಂಡು ಬಿಡುವುದು.

ಬೆಸ್ತನು ಹೇಗೆ ಮೀನು ಹಿಡಿಯುತ್ತಾನೆ? ನೆಲದ ಮೇಲಿರುವ ಮಣ್ಣುಹುಳುವನ್ನು ಅವನು ತೆಗೆದುಕೊಂಡು ಹೋಗುತ್ತಾನೆ. ಕೋಲಿನಲ್ಲಿ ಒಂದು ಮುಳ್ಳು ಇರುವುದು. ಹುಳುವಿನೊಡನೆ ಕೂಡಿದ ಕೋಲನ್ನು ಅವನು ನೀರಿಗೆ ಹಾಕುತ್ತಾನೆ, ಅದರಲ್ಲಿ ಸಿಕ್ಕಿಕೊಂಡಿರುವ ಹುಳುವನ್ನು ಕಂಡು ಮೀನುಗಳು ಹುಳುವನ್ನು ತಿನ್ನುವುದಕ್ಕಾಗಿ ತಮ್ಮ ಬಾಯಿಯನ್ನು ತೆರೆದುಕೊಂಡು ಹತ್ತಿರ ಬರುವಾಗ ಆ ಹುಳುಗಳನ್ನು ಅವುಗಳು ನುಂಗಲು ಸಾಧ್ಯವಾಗುವುದಿಲ್ಲ. ಏಕೆಂದರೆ ಹುಳುವನ್ನು ನುಂಗಲು ಪ್ರಯತ್ನಪಡುವಾಗ ಕೋಲಿನಲ್ಲಿರುವ ಮುಳ್ಳಿಗೆ ಅದು ಸಿಕ್ಕಿಕೊಂಡುಬಿಡುವುದು. ಕೊನೆಗೆ ಮೀನು ಸ್ವಾದಕ್ಕೆ ಆಸೆಪಟ್ಟು ಬೆಸ್ತನ ಕೋಲಿಗೆ ಸಿಕ್ಕಿಕೊಂಡು ತನ್ನ ಪ್ರಾಣವನ್ನೇ ಕಳೆದುಕೊಂಡುಬಿಡುತ್ತದೆ.

ದುಂಬಿ ಹೂಗಳಲ್ಲಿರುವ ರಸವನ್ನು ಆಸ್ವಾದಿಸುವುದಕ್ಕಾಗಿ ಹೂಗಳಲ್ಲಿಯೇ ಇರಬಯಸುತ್ತದೆ. ಒಮ್ಮೆ ಒಂದು ದುಂಬಿ ಒಂದು ತಾವರೆ ಹೂವಿನಲ್ಲಿ ಕುಳಿತು ಅದರ ರಸವನ್ನು ಆಸ್ವಾದಿಸುತ್ತಿತ್ತು. ಆದರೆ ರಾತ್ರಿಯಾಯಿತು. ಇನ್ನೂ ಅದರಲ್ಲೇ ಕುಳಿತಿರಬೇಕೆನ್ನುವ ಆಸೆಯಿಂದ ದುಂಬಿ ಇದ್ದಾಗ ತಾವರೆ ಹೂವು ಮುಚ್ಚಿಕೊಂಡಿತು.

\begin{shloka}
``ರಾತ್ರಿರ್ಗಮಿಷ್ಯತಿ ಭವಿಷ್ಯತಿ ಸುಪ್ರಭಾತಂ\\
ಭಾಸ್ವಾನುದೇಷ್ಯತಿ ಹಸಿಷ್ಯತಿ ಪಂಕಜಶ್ರೀಃ‌ |\\
ಇತ್ಥಂ ವಿಚಿಂತಯತಿ ಕೋಶಗತೇ ದ್ವಿರೇಫೇ\\
ಹಾ ಹನ್ತ ಹನ್ತ ನಲಿನೀಂ ಗಜ ಉಜ್ಜಹಾರ ||''
\end{shloka}

(``ರಾತ್ರಿ ಕಳೆದು ಹೋಗುವುದು, ಸೂರ್ಯೋದಯವಾಗುವುದು, ತಾವರೆ ಅರಳುವುದು' ಹೀಗೆ ಹೂವಿನೊಳಗಿದ್ದ ದುಂಬಿ ಯೋಚಿಸಿತು. ಅಯ್ಯೋ! ಅಯ್ಯೋ! ಆನೆ ತಾವರೆಯನ್ನು ಕಿತ್ತೆಸೆಯಿತು.)

ಹೀಗೆಲ್ಲಾ ಅದು ```ರಾತ್ರಿ ಮುಗಿಯುವುದು, ಸೂರ್ಯನೂ ಉದಯಿಸುವನು, ನಾನು ಉನ್ನತವಾದ ತಾವರೆ ಹೂವಿನಲ್ಲಿ ಕುಳಿತಿದ್ದೇನೆ. ಇದರಲ್ಲಿರುವ ಮಕರಂದವನ್ನೆಲ್ಲಾ ತೆಗೆದುಕೊಂಡು ಹೋಗುವುದಕ್ಕೆ ನಾನು ಬಂದಿದ್ದೇನೆ. ಬೆಳಗಾದರೂ ನನಗೇನು ನಷ್ಟ; ಬೆಳಿಗ್ಗೆ ಸೂರ್ಯನು ಪ್ರಕಾಶವಾಗಿ ಉದಯಿಸುವನು'' ಎನ್ನುವ ನಂಬಿಕೆಯಿಂದ ಅದು ಅಲ್ಲಿ ಕಾಲವನ್ನು ಕಳೆಯುತ್ತಿತ್ತು. ಬೆಳಿಗ್ಗೆ ಆ ತಟಾಕಕ್ಕೆ ನೀರು ಕುಡಿಯುವುದಕ್ಕಾಗಿ ನಾಲ್ಕು ಆನೆಗಳು ಬಂದವು. ಬೇಸಿಗೆಯಾದ್ದರಿಂದ ತಣ್ಣಗಿರುವ ನೀರನ್ನು ನೋಡಿ ಆನೆಗಳಿಗೆ ಬಹಳ ಸಂತೋಷವಾಯಿತು.

ಮೊದಲು ನೀರಿಗಾಗಿ ಬಂದ ಆನೆಗಳು ನಾಲ್ಕು ಒಂದರ ಹಿಂದೆ ಒಂದೊಂದಾಗಿ ಕೆರೆಯಲ್ಲಿ ಇಳಿದು ಸ್ನಾನಮಾಡಲು ಆರಂಭಿಸಿದುವು. ಆನೆಗಳು ಸ್ನಾನ ಮಾಡಲು ಆರಂಭಿಸಿದರೆ ಆ ಪ್ರದೇಶವೇ ಕೋಲಾಹಲಪೂರ್ಣವಾಗಿಬಿಡುವುದು ನಮಗೆ ತಿಳಿದ ವಿಷಯವೇ. ಹೀಗೆ ಒಂದೆರಡುಸಲ ಅಟ್ಟಹಾಸ ಮಾಡಿ ಆ ಕೆರೆಯನ್ನು ಬಿಟ್ಟು ಹೋದರೆ ಆನೆಗಳು ಸ್ನಾನ ಮಾಡಿದುವೆಂದುಕೊಳ್ಳಬೇಕು. ಆನೆಗಳು ಆ ಕೆರೆಯಲ್ಲಿ ಹಾಗೆ ಸ್ನಾನ ಮಾಡಿ, ಸಂತೋಷಪಡುತ್ತಿದ್ದಾಗ ಅಲ್ಲಿದ್ದ ತಾವರೆ ಹೂಗಳೆಲ್ಲಾ ಹಾಳಾದವು. ಪಾಪ, ಒಂದು ತಾವರೆ ಒಳಗಿದ್ದ ಆ ದುಂಬಿಯೂ ಸತ್ತುಹೋಯಿತು.

ಹೀಗೆ ಒಂದೊಂದರಲ್ಲಿ ಇಂದ್ರಿಯಾಸಕ್ತಿ ಇಟ್ಟುಕೊಂಡಿದ್ದರಿಂದಲೇ ಆ ಐದು ಪ್ರಾಣಿಗಳು ತಮ್ಮ ಪ್ರಾಣ ಕಳೆದುಕೊಂಡುದನ್ನು ನಾವು ನೋಡಿದೆವು, ಮನುಷ್ಯನಿಗಾದರೋ-

\begin{shloka}
``ನರಃ ಪಂಚಭಿರಂಚಿತಃ ಕಿಮ್''
\end{shloka}

-ಎಂದು ಹೇಳಿದಂತೆ, ಒಂದರಲ್ಲಿ ಮಾತ್ರವಲ್ಲದೆ ಎಲ್ಲದರಲ್ಲೂ ಇಂದ್ರಿಯಾಸಕ್ತಿ ಇದೆ. ಐದು ಇಂದ್ರಿಯಗಳೂ ಹೇಳುವಂತೆ ತನ್ನ ಬಾಳನ್ನು ನಡೆಸುತ್ತಿದ್ದಾನೆ. ಆದರೆ ಈ ರೀತಿ ಯಾವಾಗಲೂ ಇಂದ್ರಿಯಗಳು ಹೇಳಿದಂತೆ ನಡೆಯುತ್ತಿರಲು ಸಾಧ್ಯವಾಗುವುದಿಲ್ಲ. ಆದ್ದರಿಂದ ಮನುಷ್ಯನು ಸಾಧಿಸಬೇಕಾದುದು ಭಗವಂತನ ಪಾದಾರವಿಂದಗಳನ್ನೇ. ಸ್ಥಿತಿ ಹೀಗಿದ್ದರೂ ಕೂಡ ಮನುಷ್ಯನು ``ಇಂದ್ರಿಯಗಳಿಂದ ತನ್ನ ಚಿಂತನೆಯನ್ನು ಎಂದೂ ಬಿಡಲು ಮನಸ್ಸೇ ಬರುವುದಿಲ್ಲ'' ಎನ್ನುತ್ತಾನೆ. ಹಣ ಸಂಪಾದನೆ ಮಾಡುವ ಒಬ್ಬನಿಗೆ, ``ಇನ್ನೂ ಸಂಪಾದನೆ ಮಾಡಬೇಕು; ಒಂದು ಲಕ್ಷ ಬಂದರೂ ಅದು ಸಾಲದು, ಇನ್ನೂ ಹತ್ತು ಲಕ್ಷ ಬೇಕು, ಇನ್ನೂ ಹತ್ತು  ಕೋಟಿ ಬೇಕು'' ಎಂದು ತೋರುವುದು. ಹೀಗೆ ಹತ್ತು ಕೋಟಿ ಸಂಪಾದಿಸಿದ ಮೇಲೆ ನಾವು ಅವನನ್ನು ``ಈಗ ಹತ್ತು ಕೋಟಿ ಸಂಪಾದಿಸಿರುವೆ, ನೂರು ರೂಪಾಯಿ ದಾನ ಕೊಡು'' ಎಂದು ಕೇಳಿದರೆ ``ಇಲ್ಲ ನನಗೆ ಹೆಚ್ಚಾಗಿ ಹಣ ಸಿಕ್ಕಲಿ, ಆಗ ಹೆಚ್ಚಾಗಿ ದಾನ ಕೊಡುವೆನು'' ಎನ್ನುತ್ತಾನೆ. ಆದರೆ ಹೆಚ್ಚು ಹಣ ಬಂದಾಗಲೂ ಸಹ ಅಂಥವರು ದಾನಮಾಡಲು ಮುಂದೆ ಬರುವುದಿಲ್ಲ! ಆದ್ದರಿಂದ ಮನುಷ್ಯನ ಆಸೆಗೆ ಮಿತಿ ಇಲ್ಲವೆಂದು ಹೇಳಿದರೆ ಆಶ್ಚರ್ಯವೇನಿಲ್ಲ.

``ಮನುಷ್ಯನಿಗೆ ಚಿಕ್ಕ ವಯಸ್ಸು(ತಾರುಣ್ಯ) ಇರುವಾಗಲೇ ಅವನು ತಪಸ್ಸು ಮಾಡಬೇಕೆಂದು ಶಾಸ್ತ್ರಗಳಲ್ಲಿ ಹೇಳಿದ. ಏಕೆಂದರೆ ಅವನಿಗೆ ವಯಸ್ಸಾದ ಮೇಲೆ ಅವನು ಯಾವುದಾದರೂ ಒಂದು ಯಜ್ಞವನ್ನೋ, ಹೋಮವನ್ನೋ ಮಾಡಬೇಕೆಂದು ಕೊಂಡರೆ ಅವನು ಹಾಗೆ ಮಾಡಲು ಸಾಧ್ಯವಾಗುವುದಿಲ್ಲ. ಒಬ್ಬನು ತನ್ನ ತಾರುಣ್ಯದಲ್ಲಿ ಪ್ರಪಂಚದಲ್ಲಿರುವ ಇಂದ್ರಿಯಾಸ್ವಾದಗಳಲ್ಲಿ ಮನಸ್ಸಿಟ್ಟುಕೊಂಡು ವೈರಾಗ್ಯವನ್ನು ಅಭ್ಯಾಸ ಮಾಡಬೇಕೆಂದುಕೊಂಡರೆ ಅದೂ ಆಗುವುದಿಲ್ಲ; ಏಕೆಂದರೆ,

\begin{shloka}
``ನಜಾತುಕಾಮಃ ಕಾಮಾನಾಮುಪಭೋಗೇನ ಶಾಮ್ಯತಿ |\\
ಹವಿಷಾ ಕೃಷ್ಣಾವರ್ತ್ಮೇವ ಭೂಯ ಏವಾಭಿವರ್ಧತೇ ||''
\end{shloka}

(ಆಸೆ ಎನ್ನುವುದು ಅದನ್ನು ಅನುಭವಿಸುವುದರಿಂದ ಎಂದೂ ಶಮಿಸುವುದಿಲ್ಲ; ಭೋಗದಿಂದ ಅದು ತುಪ್ಪ ಹಾಕಿದ ಅಗ್ನಿಯಂತೆ ಇನ್ನೂ ಹೆಚ್ಚಾಗುತ್ತದೆ) - ಎಂದು ಹೇಳಿದಂತೆ ಯಾವುದೇ ಒಂದು ವಸ್ತುವನ್ನು ಅನುಭವಿಸುವುದರಿಂದ ಅದು ಶಮಿಸುವುದಿಲ್ಲ. ಒಬ್ಬನು ತನಗೆ ಸಾವಿರರೂಪಾಯಿ ಸಂಬಳ ಬಂದರೆ ಯಾವ ವಿಧವಾದ ಚಿಂತೆಯೂ ಇರುವುದಿಲ್ಲವೆಂದು ಹೇಳುತ್ತಾನೆ. ಕೆಲವು ದಿನಗಳು ಸಾವಿರ ರೂಪಾಯಿ ಸಂಬಳ ಬಂದಮೇಲೆ ``ಇನ್ನೊಬ್ಬನಿಗೆ ಹತ್ತು ಸಾವಿರ ರೂಪಾಯಿ ಸಂಬಳ ಬರುವುದಲ್ಲಾ, ನನಗೆ ಮಾತ್ರ ಏಕೆ ಸಾವಿರ ರೂಪಾಯಿ'' ಎಂದು ಕೇಳುತ್ತಾನೆ. ಅವನಿಗೆ ಹತ್ತು ಸಾವಿರ ಬಂದರೆ ``ಆ ಮನುಷ್ಯನು ಕುಳಿತಲ್ಲಿಯೇ ತಿಂಗಳು-ತಿಂಗಳು ಒಂದು ಲಕ್ಷ ರೂಪಾಯಿ ಸಂಪಾದಿಸುವನು. ನನಗೆ ಮಾತ್ರ ಹತ್ತು ಸಾವಿರ ಮಾತ್ರವೇನು?'' ಎನ್ನುವನು. ಹೀಗೆ ಅವನು ಯೋಚನೆ ಮೇಲೆ ಯೋಚನೆ ಮಾಡುತ್ತಾ ತನ್ನ ಆಯುಸ್ಸನ್ನು ಪೂರ್ತಿಯಾಗಿ ಇಂಥ ಯೋಚನೆಯಲ್ಲಿಯೇ ಕಳೆದು ಬಿಡುವನು. ಅವನ ಯೋಚನೆಯಂತೆ ಅವನಿಗೆ ಸಂಬಳ ದೊರೆಯುವುದಿಲ್ಲ. ಆದ್ದರಿಂದ ಅವನ ಯೋಚನೆಗಳಿಗೆ ಬೆಲೆಯೇ ಇಲ್ಲ. ಮನುಷ್ಯನು ಭಗವಂತನನ್ನು ಕುರಿತು ಯೋಚಿಸುವುದೋ, ಧರ್ಮಾನುಷ್ಠಾನ ಮಾಡಬೇಕೆಂದುಕೊಳ್ಳುವುದೋ ಆದರೆ,

\begin{shloka}
``ಅಜರಾಮರವತ್‌ಪ್ರಾಜ್ಞೋ ವಿದ್ಯಾಮರ್ಥಂ ಚ ಸಾಧಯೇತ್ |\\
ಗೃಹೀತ ಇವ ಕೇಶೇಷು ಮೃತ್ಯುನಾ ಧರ್ಮಮಾಚರೇತ್ ||''
\end{shloka}

(ವಿದ್ಯೆ ಕಲಿಯುವಾಗ, ಸಂಪಾದನೆ ಮಾಡುವಾಗ ಬುದ್ಧಿವಂತನಾದವನು ತನಗೆ ಮುಪ್ಪು, ಸಾವು ಇಲ್ಲವೆಂದು ಭಾವಿಸಿ ಅವುಗಳನ್ನು ಸಾಧಿಸಬೇಕು. ಆದರೆ ಧರ್ಮ ಮಾಡುವಾಗ ತಲೆ ಕೂದಲನ್ನು ಯಮನು ಹಿಡಿದು ಕೊಂಡಿದ್ದಾನೆಂದು ಭಾವಿಸಿ ಧರ್ಮಾಚರಣೆ ಮಾಡಬೇಕು.)-ಎಂದು ಹೇಳಿದಂತೆ ``ಮೃತ್ಯುದೇವತೆ ತನ್ನ ತಲೆ ಕೂದಲನ್ನು ಗಟ್ಟಿಯಾಗಿ ಹಿಡಿದುಕೊಂಡಿದೆ'' ಎನ್ನುವ ತೀರ್ಮಾನಕ್ಕೆ ಬಂದರೇನೇ ಹಾಗೆ ಮಾಡಲು ಸಾಧ್ಯವಾಗುವುದು. ಆದ್ದರಿಂದ ಭಗವಂತನ ಸಾಕ್ಷಾತ್ಕಾರ ಪಡೆಯಬೇಕೆನ್ನುವ ಆಸೆ ಎಲ್ಲರಿಗೂ ಇದ್ದರೂ ಸಹ, ತಮ್ಮ ಬಾಳಿನ ದಿನಗಳನ್ನು ಸಫಲವಾಗಿಸಿಕೊಳ್ಳಬೇಕೆನ್ನುವ ತೀವ್ರ ಆಸೆ ಇದ್ದರೂ ಕೂಡ, ಮಧ್ಯೆ ಬಂದಿರುವ ಪ್ರಾಪಂಚಿಕ ವಿಷಯಗಳು ಮನುಷ್ಯನ ಮನಸ್ಸನ್ನು ತಪ್ಪುದಾರಿಗೇ ಎಳೆದುಕೊಂಡು ಹೋಗುತ್ತವೆ. ಮೊದಲೇ ಹೇಳಿದಂತೆ ನಾವು ಈ ಪ್ರಪಂಚದಲ್ಲಿ ಸ್ಥಿರವಾದ ಬಾಳನ್ನು ಪಡೆಯಲು ಎಂದೂ ಸಾಧ್ಯವಿಲ್ಲ.

\begin{shloka}
``ಯೇನಾಶಾನ್ತಃ ಸಲಿಲನಿಧಯಃ ಯೇನ ಸೃಷ್ಟಾ ಪ್ರತತ್ಯೌ ||''
\end{shloka}

(ಯಾರಿಂದ ಸಮುದ್ರವೇ ತಡೆಯಲ್ಪಟ್ಟಿತೋ, ಯಾರಿಂದ ಬೇರೆ ಸ್ವರ್ಗವೇ ಸೃಷ್ಟಿಸಲ್ಪಟ್ಟಿತೋ.)

ಅಗಸ್ತ್ಯರು ತಮ್ಮ ಕೈ ತುಂಬ ಸಮುದ್ರವನ್ನು ಆಪೋಷನ ತೆಗೆದುಕೊಂಡು ಬಿಟ್ಟಿದ್ದರೆಂದು ನಾವು ಕೇಳಿದ್ದೇವೆ. ಹಾಗೆಯೇ ವಸಿಷ್ಠರು ``ಏಕೇನ ಬ್ರಹ್ಮದಂಡೇನ'' ಎಂದಂತೆ ಎಂಥ ಅಸ್ತ್ರವನ್ನಾದರೂ ತಮ್ಮ ಬ್ರಹ್ಮದಂಡದಿಂದ ತಡೆದು ನಿಲ್ಲಿಸುವ ಶಕ್ತಿಯುಳ್ಳವರಾಗಿದ್ದರೆಂದೂ ನಾವು ಕೇಳಿದ್ದೇವೆ. ಇಂಥ ಮಹಿಮೆಯುಳ್ಳವರು ಎಷ್ಟೋ ಮಂದಿ ಇದ್ದಾರೆ. ಹಾಗೆಯೇ ಪ್ರಪಂಚವನ್ನೇ ಭಯಪಡಿಸುವವರೂ ಇದ್ದಾರೆ. ಅಂಥವರು ಕೂಡ ಬಹಳ ಕಾಲ ಜೀವಿಸಲು ಸಾಧ್ಯವಾಗುವುದಿಲ್ಲ. ಹೀಗಿರುವಾಗ ನಾವು ಸ್ಥಿರವಾಗಿರಲು ಹೇಗೆ ಸಾಧ್ಯ? ಒಬ್ಬ ದೊಡ್ಡ ಡಾಕ್ಟರು ಜಪಾನಿನಲ್ಲಿದ್ದರೂ ಅವನಿಗು ಕೂಡ ಒಂದು ದಿನ ಅನಾರೋಗ್ಯವಾಗುವುದು. ಏಕೆಂದರೆ,

\begin{shloka}
``ಜಾತಸ್ಯ ಹಿ ಧ್ರುವೋ ಮೃತ್ಯುಃ ಧೃವಂ ಜನ್ಮ ಮತ್ಯಸ್ಯ ಚ |''\\
(ಹುಟ್ಟಿದವನಿಗೆ ಸಾವು ಖಂಡಿತ, ಸತ್ತವನಿಗೆ ಜನ್ಮ ಖಂಡಿತ.)
\end{shloka}

-ಎಂದು ಹೇಳಿರುವುದರಿಂದ ವಸ್ತು ಎಂದೂ ಸ್ಥಿರವಾಗಿರಲು ಸಾಧ್ಯವಿಲ್ಲ. ಹುಟ್ಟಿದ ಶರೀರವಾದ್ದರಿಂದ ಒಂದು ದಿನ ಇದು ನಶಿಸಿಯೇ ಹೋಗುವುದು. ಆಶೀರ್ವಾದ ಮಾಡುವಾಗ ``ಶತಾಯುಷ್ಮಾನ್ ಭವ'' ಎಂದು ದೊಡ್ಡವರು ಹೇಳುವರು. ಆದರೆ ಎಷ್ಟೋ ಮಂದಿ ೮೦ ವಯಸ್ಸು ಕೂಡ ದಾಟುವುದಿಲ್ಲ. ನೂರು ವಯಸ್ಸು ಒಬ್ಬನು ಬಾಳಿದರೂ ``೮೦ ವರ್ಷಗಳು ಪ್ರಪಂಚದಲ್ಲಿರುವುದನ್ನು ಅನುಭವಿಸುವುದರಲ್ಲಿ ಕಳೆದು ಬಿಡಬಹುದು. ಅದಾದಮೇಲೆ ೨೦ ವರ್ಷಕಾಲ ವೈರಾತ್ಯವನ್ನು ಅಭ್ಯಾಸ ಮಾಡಬಹುದು, ಭಗವಂತನನ್ನು ಧ್ಯಾನ ಮಾಡಬಹುದು, ಅದಕ್ಕಾಗಿ ಆ ಕಾಲವನ್ನು ಮೀಸಲಾಗಿಡಬಹುದು'' -ಎಂದು ಯಾರಾದರೂ ಯೋಚಿಸಿದರೆ, ಹಾಗೆ ಆಗುವುದಿಲ್ಲ. ಏಕೆಂದರೆ,

\begin{shloka}
``ಯಮಸ್ಯ ಕರುಣಾನಾಸ್ತಿ''\\
(ಯಮನಿಗೆ ಕರುಣೆ ಇಲ್ಲ.)
\end{shloka}

\begin{shloka}
``ನ ಹಿ ಯಮೋಽಸ್ಯ ಕೃತಾಕೃತಮೀಕ್ಷತೇ |''
\end{shloka}

(ಅವನು ತನ್ನ ಕೆಲಸವನ್ನು ಮುಗಿಸಿದನೇ, ಬಿಟ್ಟನೇ ಎನ್ನುವುದನ್ನು ಯಮನು ನೋಡುವುದಿಲ್ಲ.)

-ಎಂದು ಒಂದು ಜಾಗದಲ್ಲಿ ಹೇಳಲ್ಪಟ್ಟಿದೆ. ಯಮನು ``ಈ ಮನುಷ್ಯನು ಯಾವುದಾದರೂ ಕೆಲಸವನ್ನು ಮಾಡಿದ್ದಾನಾ, ಇಲ್ಲವಾ'' -ಎನ್ನುವುದನ್ನೆಲ್ಲಾ ನೋಡುವುದಿಲ್ಲ. ಕರ್ಮ ಮುಗಿದರೆ ಯಮಲೋಕಕ್ಕೆ ಕರೆದೊಯ್ಯುವನು. ನಾವು ಪ್ರಪಂಚದಲ್ಲಿ ನೋಡುತ್ತೇವೆ. ದೊಡ್ಡ ದೊಡ್ಡ ಹೋಟಲುಗಳಲ್ಲಿ ಒಂದು ರೂಮಿಗೆ ಒಂದು ದಿನಕ್ಕೆ {\eng 250} ರೂಪಾಯಿ ಬಾಡಿಗೆ ವಸೂಲು ಮಾಡುವರು. ಆ ಒಂದು ದಿನ ಆದಮೇಲೆ ``ಮಾರನೆಯ ದಿನವೂ ನಾನು ಈ ರೂಮಿನಲ್ಲಿರುವೆನು'' ಎಂದು ಯಾರಾದರೂ ಹೇಳಿದರೆ ಅವನಿಗೆ ಅನುಮತಿ ಕೊಡುವುದಿಲ್ಲ. ಇನ್ನೂ {\eng 250} ರೂಪಾಯಿ ಕೊಟ್ಟರೆ ಅವನು ಅಲ್ಲಿರಬಹುದು. ಬಾಯಾರಿಗೆ ಆದಾಗ ನೀರು ಕುಡಿಯಬಹುದು. ಆದರೆ ಬಾಯಾರಿಕೆ ಆದಾಗೆಲ್ಲಾ ಒಂದೊಂದು ಬಾವಿಯನ್ನು ತೋಡಿಸಿ ನೀರು ಕುಡಿಯಬೇಕೆಂದುಕೊಂಡರೆ ಅದು ಎಷ್ಟು ಕಷ್ಟವಾದ ಕೆಲಸ! ಅವನು ಬಾಯಾರಿಕೆಯಾದಾಗ ಒಂದು ಬಾವಿಯನ್ನು ತೋಡುವನು. ಮತ್ತೊಮ್ಮೆ ಬಾಯಾರಿಕೆಯಾದರೆ ಇನ್ನೊಂದು ಬಾವಿಯನ್ನು ತೋಡಬೇಕು! ಇದಕ್ಕೆಲ್ಲಾ ಎಷ್ಟು ಖರ್ಚಾಗುತ್ತದೆ! ಅದೇ ರೀತಿ ನಾವು ಎಷ್ಟು ಜನ್ಮಗಳನ್ನು ಪಡೆದಿದ್ದರೂ ಒಂದಲ್ಲ ಒಂದು ದಿನ ನಾವು ಪ್ರಾನವನ್ನು ಬಿಡಬೇಕು. `{\eng Intimation}' ಎನ್ನುವುದನ್ನು ಕೊಟ್ಟು ಯಮನು ತನ್ನ ಲೋಕಕ್ಕೆ ಕರೆದೊಯ್ಯುವುದಿಲ್ಲ. ಅವನು ನೇರವಾಗಿ ಒಬ್ಬನನ್ನು ಹಿಡಿದು ಬಿಡುತ್ತಾನೆ.

ಹೊರಗೆ ``ಭಗವಂತನಿದ್ದಾನೆ'' ಎಂದು ಹೇಳುತ್ತಾ ಒಬ್ಬನು ತಿರುಗಾಡುತ್ತಿರಬೇಕಾಗಿಲ್ಲ. ನಮ್ಮ ಮನಸ್ಸಿನಲ್ಲಿ ಯಾರು ಇದ್ದಾರೆ, ಯಾರು ಇಲ್ಲ ಎನ್ನುವುದನ್ನು ಪ್ರತಿದಿನವೂ ಯೋಚಿಸುತ್ತಿದ್ದರೆ ಅದೇ ಸಾಕು. ನಿಜಕ್ಕೂ `ತಾನು ಯಾರು' ಎನ್ನುವುದನ್ನು ಮನುಷ್ಯನು ತಿಳಿದುಕೊಂಡಿಲ್ಲ. ಏಕೆಂದರೆ ಅವನಿಗೆ ತಿನ್ನಲು ಆಹಾರಬೇಕು, ಉಡಲು ಬಟ್ಟೆ ಬೇಕು, ಇರಲು ಮನೆ ಬೇಕು. ಸ್ವಲ್ಪ ಐಶ್ವರ್ಯವಿದ್ದರೆ ಹೊರಗೆ ಹೋಗಲು ಕಾರ್ ಬೇಕು. ಎಷ್ಟು ಕೋಟಿ ರೂಪಾಯಿ ಸಂಪಾದನೆ ಮಾಡಿದರೂ ಅವನು ಇನ್ನೂ ಅಳುತ್ತಿರುತಾ ಇರುವನೇ ಹೊರತು ತನಗಿರುವ ಐಶ್ವರ್ಯದಿಂದ ತೃಪ್ತಿಪಡುತ್ತಾ ನಗುತ್ತಾ ಇರುವುದಿಲ್ಲ. ಎಂಥ ದೊಡ್ಡ ಪ್ರಭುವಾಗಿದ್ದರೂ ``ಬೇರೆ ನಗರಗಳಲ್ಲಿರುವ ಎಲ್ಲಾ ವ್ಯಾಪಾರಗಳೂ ಸರಿಯಾಗಿ ನಡೆಯುತ್ತವೆ. ಆದರೆ ಆ ಊರಿನಲ್ಲಿರುವ ನನ್ನ ವ್ಯಾಪಾರ ಮಾತ್ರ ಎಷ್ಟು ಕೋಟಿ ರೂಪಾಯಿ ಹಾಕಿದರೂ ಕೂಡ ಸರಿಯಾಗಿ ನಡೆಯಲಿಲ್ಲ. ನಾನು ಹತ್ತು ಕೋಟಿ ರೂಪಾಯಿ ಮೂಲಧನವಾಗಿ ಹಾಕಿದನು. ಆ ಹತ್ತು ಕೋಟಿ ಮಾತ್ರ ಬಂದಿತೆ ಹೊರತು ಮೇಲೆ ಯಾವ ಲಾಭವೂ ದೊರೆಯಲಿಲ್ಲ'' ಎಂದು ಹಲವರು ಹೇಳುವುದನ್ನು ನಾವು ಕೇಳಿದ್ದೇವೆ. ವ್ಯಾಪಾರದಲ್ಲಿ ಈ ರೀತಿ ನಷ್ಟವೆಲ್ಲಾ ಸಹಜವೆಂದು ತಿಳಿದು ಯಾರೂ ಅದರ ಬಗ್ಗೆ ಚಿಂತಿಸದೆ ಇರುವುದಿಲ್ಲ. ಮೇಲೆ ಮೇಲೆ ಸಂಪಾದಿಸಬೇಕೆನ್ನುವ ಆಸೆಯೇ ದಿನ ದಿನವೂ ಹೆಚ್ಚಾಗುತ್ತಾ ಹೋಗುತ್ತದೆ. ಈ ಆಸೆ ಯಾವುದರಿಂದ ಉಂಟಾಗುತ್ತದೆ? ನಾವು ಸೇರಿಸಿ ಇಟ್ಟಿರುವ ಐಶ್ವರ್ಯವನ್ನೆಲ್ಲಾ ಯಾರಿಗೆ ಕೊಡುತ್ತೇವೆಂದು ತಿಳಿಯದೆ ಇರುವುದರಿಂದ ಈ ಆಸೆ ಉಂಟಾಗುತ್ತದೆ. ಅಲ್ಲದೆ, ``ಒಂದು ವೇಳೆ ಯಮನು ಸತ್ತುಹೋದರೂ ಹೋಗಬಹುದು. ನಾನು ಯಾವಾಗಲೂ ಸ್ಥಿರವಾಗಿರುವೆನು'' ಎನ್ನುವ ಭಾವನೆಯೇ ಇದಕ್ಕೆ ಕಾರಣ. ``ಡಾಕ್ಟರುಗಳು ಸತ್ತು ಹೋದರೂ ನಾನು ಸಾಯುವುದಿಲ್ಲ. ನಾನು ಯಾವಾಗಲೂ ಶಾಶ್ವತವಾಗಿರುವೆನು'' ಎನ್ನುವ ಭಾವನೆ `ನಿತ್ಯೋವಾಸ್ಯಮ್' ಎಂದು ಹೇಳಿದಂತೆ ಸಾಯುವಾಗ ಕೂಡ ಒಬ್ಬನು ಇರುವುದನ್ನು ನಾವು ಕಾಣಬಹುದು. ನನಗೆ ಕೆಲವು ವೇಳೆ ತಮಾಷೆಯಾಗಿರುತ್ತದೆ. ಒಬ್ಬನಿಗೆ {\eng 82} ವಯಸ್ಸು ಆಗಿದೆ. ಆತನು ``ನಾನು ಕೆಮ್ಮುವಾಗ ಬಹಳ ಕಷ್ಟವಾಗುತ್ತದೆ, ನನಗೆ ಕೆಮ್ಮು ನಿಲ್ಲುವುದಕ್ಕೆ ಸನ್ನಿಧಾನ ಯಾವುದಾದರೂ ಒಂದು ದಾರಿಯನ್ನು ಹೇಳಬೇಕು'' ಎಂದು ಕೇಳುತ್ತಾನೆ. {\eng 82} ವಯಸಾದಮೇಲೆ ಕೆಮ್ಮಲು ಬಂದರೆ ಅದನ್ನು ದೂರ ಮಾಡಲು ನಾವು ಯಾವ ದಾರಿಯನ್ನು ಹೇಳುವುದು? ಬಂದ ಕೆಮ್ಮಲು ಆತನು ಇರುವವರೆಗೆ ಇದ್ದೇ ಇರುವುದು.

\begin{shloka}
``ಚಕ್ಷುಷ್ವನ್ತೇ ಚಲತಿ ದಶನೇ ಶ್ಮಶ್ರುಣಿಶ್ವೇತಮಾನೇ\\
ಸೀದತ್ಯಂಗೇ ಮನಸಿ ಕಲುಷೇ ಕಂಪಮಾನೇ ಕರಾಗ್ರೇ |\\
ದೂತೈರೇತೈರ್ದಿನಕರಭುವಃ ಶಶ್ವದುದ್ಬೋಧ್ಯಮಾನಾಃ\\
ತ್ರಾತುಂ ದೇಹಂ ತದಪಿ ಭಿಷಜಾಮೇವ ಸ್ವಾಂತಂ ವದಾಮಃ ||''
\end{shloka}

-ಎಂದು ನೀಲಕಂಠ ದೀಕ್ಷಿತರು ತಮಾಷೆಯಾಗಿ ಹೇಳಿದ್ದಾರೆ. {\eng 80}ವರ್ಷ ವಯಸ್ಸಾದ ಒಬ್ಬನಿಗೆ ಕಣ್ಣು ಸರಿಯಾಗಿ ಕಾಣುವುದಿಲ್ಲ. ಕುರುಡನೆನ್ನುವ ಹೆಸರನ್ನು ಪಡೆದುಕೊಂಡನು. ಆಗ `ಚಕ್ಷುಷ್ವನ್ತೇ' ಎನ್ನುವಂತೆ ಒಬ್ಬ ಡಾಕ್ಟರ್ ಹತ್ತಿರದಿಂದ ಇನ್ನೊಬ್ಬ ಡಾಕ್ಟರ್ ಹತ್ತಿರಕ್ಕೆ ಹೋಗುತ್ತಾನೆ. ಡಾಕ್ಟರ್, ``ಸ್ವಾಮಿ, ನಿಮ್ಮ ಕಣ್ಣುಗಳ ಶಕ್ತಿ ಹೋಗಿಬಿಟ್ಟಿದೆ'' ಎನ್ನುತ್ತಾರೆ. ಇನ್ನೊಬ್ಬ ಡಾಕ್ಟರ್ ಹತ್ತಿರ ಆತನು ಒಳ್ಳೆಯ ಔಷಧ ಕೊಡುವನೆನ್ನುವ ಆಸೆ ಇಟ್ಟುಕೊಂಡು ಆ ಡಾಕ್ಟರ್ ಹತ್ತಿರ ಹೋಗುವನು. ಆದರೆ ಆ ಡಾಕ್ಟರ್ ಕೂಡ ಏನೂ ಮಾಡುವುದಕ್ಕಾಗುವುದಿಲ್ಲ. ವಯಸ್ಸಾದುದರಿಂದ ಕಣ್ಣು ಕಾಣದೆ ಹೋಯಿತೆನ್ನುವುದು ತೀರ್ಮಾನ.

`ಚಲತಿದಶನೇ'

ಎಂದು ಹೇಳಿದಂತೆ ಒಬ್ಬನಿಗೆ ಹಲ್ಲುಗಳು ಬಿದ್ದುಹೋಗಿವೆ.

\begin{shloka}
`ಶ್ಮಶ್ರುಣಿಶ್ವೇತಮಾನೇ'
\end{shloka}

-ಎಂದಂತೆ ತಲೆ ಕೂದಲೆಲ್ಲಾ ನೆರೆತಿದೆ. ಶರೀರವೆಲ್ಲಾ ನಡುಗುತ್ತದೆ. ಒಮ್ಮೆ ಮಹಡಿಯ ಮೇಲೆ ಒಬ್ಬರು ವಯಸ್ಸಾದವರು ಕುಳಿತಿದ್ದರು. ಆಗ ಮಹಡಿ ಮೇಲಕ್ಕೆ ಬಂದ ಒಬ್ಬರು ಹೂವಿನ ಹಾರ ಒಂದನ್ನು ಅವರಿಗೆ ಕೊಟ್ಟಾಗ, ಇನ್ನೊಬ್ಬರು, ``ಆ ಹಾರವನ್ನು ಅವರ ಕೈಗೆ ಕೊಡಬೇಡಿ'' ಎಂದರು. ಏಕೆಂದರೆ ಅವರು ಆ ಮಾಲೆಯನ್ನು ಹಿಡಿದುಕೊಂಡರೆ ಹಾರದಲ್ಲಿರುವ ಹೂಗಳೆಲ್ಲಾ ಉದುರಿಹೋಗುವುವು. ಆ ವಯಸ್ಸಾದವರ ಕೈ ಹಾಗೆ ಆಡುವುದು. ಅಷ್ಟೇ ಅಲ್ಲ-`ಮನಸಿ ಕಲುಷೇ' ಎಂದು ಹೇಳಿದಂತೆ ಅವರ ಜ್ಞಾಪಕ ಶಕ್ತಿ ಬಹಳ ಕಡಮೆಯಾಗಿದೆ. ಯಾರಾದರೂ ಬಂದರೆ ``ನೀನು ರಾಮಾಶಾಸ್ತ್ರಿಯೋ? ಗೋಪಾಲಶಾಸ್ತ್ರಿಯೋ?' ಎಂದು ಕೇಳುತ್ತಾರೆ. ಇದೆಲ್ಲವನ್ನೂ ನೋಡಿ ನಾವು ಏನು ತಿಳಿದುಕೊಳ್ಳಬೇಕೆಂದರೆ,

\begin{shloka}
`ದೂತೈರೇತೈರ್ದಿನಕರಭುವಃ ಶಶ್ವದುದ್ಬೋಧಮಾನಾಃ\\
ತ್ರಾತುಂ ದೇಹಂ ತದಪಿ ಭಿಷಜಾಮೇವ ಸಾಂತ್ವಂ ವದಾಮಃ ||'
\end{shloka}

-ಎಂದು ಹೇಳಿದಂತೆ ಒಂದು ವ್ಯಾಧಿರೂಪದಲ್ಲಿ ಯಮದೂತರಲ್ಲಿ ಒಬ್ಬನು ಬಂದನು. ಆಗ ಮನುಷ್ಯನು, ``ನಾನು ಬರುವುದಿಲ್ಲ'' ಎಂದು ತಿರಸ್ಕರಿಸಿದರೆ, ಇನ್ನೊಂದು ವ್ಯಾಧಿರೂಪದಲ್ಲಿ ಯಮನ ಇನ್ನೊಬ್ಬ ದೂತನು ಬರುತ್ತಾನೆ. ಈಗ ಒಬ್ಬನನ್ನು ತಿರಸ್ಕರಿಸಿದರೆ, ಹಲವು ವ್ಯಾಧಿಗಳು ಒಂದೊಂದಾಗಿ ಅವನ ಹತ್ತಿರ ನಾಲ್ಕು ಐದು ದೂತರುಗಳಾಗಿ ಬರುವರು. ಒಬ್ಬ ದೂತನು ಬಂದರೂ ಪರವಾಗಿಲ್ಲ. ಹಲವರು ಬಂದರೆ, ``ನಾನು ಬರುವುದಿಲ್ಲ'' ಎಂದರೂ ಯಮದೂತರು ಎತ್ತಿಕೊಂಡು ಹೋಗುವರು. ಆದ್ದರಿಂದ ಮನುಷ್ಯರು ಹಲವರು ಡಾಕ್ಟರುಗಳ ಹತ್ತಿರ ಹೋಗಿ ಹಣವನ್ನು ಕೊಡುತ್ತಿದ್ದಾರೆ, ಅಷ್ಟೇ; ತಮ್ಮ ನಿಜವಾದ ಸ್ಥಿತಿಯನ್ನು ಯೋಚನೆ ಮಾಡುವುದೇ ಇಲ್ಲ. ಅಂಥ ಮನುಷ್ಯನಿಗಿಂತಲೂ ಮೂಢನು ಇನ್ನೊಬ್ಬನು ಇರಲಾರನು.

ಈ ಕಾಲದಲ್ಲಿ ನಾನು ಕೆಲವು ಮನುಷ್ಯರನ್ನು ನೋಡಿದ್ದೇನೆ. ಅವರು ದೊಡ್ಡ ಪದವಿಯಲ್ಲಿ ಇದ್ದವರು. ಒಂದು ದಿನ ಅಂಥ ಒಬ್ಬರು ಡಾಕ್ಟರ್ ಹತ್ತಿರ ಹೋದರೆ, ``ಹೃದಯ ಸರಿಯಾಗಿಲ್ಲ, ಬ್ಲಡ್‌ಪ್ರಶರ್ ಜಾಸ್ತಿಯಾಗಿದೆ, ಸಕ್ಕರೆ ವ್ಯಾಧಿಯೂ ಇದೆ'' ಎನ್ನುವುದನ್ನು ಕೇಳಿದರೆ, ಅವರು ತಕ್ಷಣ ತನ್ನೊಡನೆ ಕೆಲಸಮಾಡಿದ ಇತರರಿಗೆ ಒಂದು ಕಾಗದ ಬರೆದು ಬಿಡುವರು. ``ನಾನು ನನ್ನ ಪದವಿಯಲ್ಲಿದಾಗ ನಿಮ್ಮೊಡನೆ ಸರಿಯಾದ ವ್ಯವಹಾರಮಾಡದೆ ಇದ್ದಿರಬಹುದು. ಯಾವುದಾದರೂ ಮಾತನ್ನು ನಿಮ್ಮ ಮನಸ್ಸಿಗೆ ನೋವಾಗುವಂತೆ ಆಡಿರಬಹುದು. ನಾನು ಪದವಿಯಲ್ಲಿದ್ದಾಗ ಹೀಗೆ ನಾನು ಮಾಡಿದ್ದನ್ನೆಲ್ಲಾ ನೀವು ಕ್ಷಮಿಸಬೇಕು. ನಿಮ್ಮೆಲ್ಲರ ಮೇಲೂ ನನಗೆ ಸಮವಾಗಿ ಪ್ರೀತಿ ಇದೆ'' ಎಂದು ಅಂಥವರು ಕಾಗದ ಬರೆಯುವರು. ಆದರೆ ಯಾವುದಾದರೂ ವ್ಯಾಧಿ ಬಂದರೆ ಅವರು ಹಾಗೆ ಕ್ಷಮೆ ಕೇಳಲು ಮುಂದೆ ಬರುವರು. ಅದು ಬರುವುದಕ್ಕೆ ಮುಂಚೆ ಪದವಿಯಲ್ಲಿದ್ದಾಗ ಇತರರೊಡನೆ ಸರಿಯಾಗಿ ವ್ಯವಹರಿಸಿ, ಪ್ರೀತಿಯಿಂದ ಇದ್ದಿದ್ದೇ ಆದರೆ ಈಗ ಏಕೆ ಕ್ಷಮೆ ಕೇಳಬೇಕು? ಮೊದಲು ಕ್ಷಮೆ ಕೇಳಿದರೆ ಅದು ಅವರ ಮರ್ಯಾದೆಗೆ ಕಡಿಮೆ ಎಂದುಕೊಳ್ಳುವರು.

ಒಬ್ಬನು ತಲೆಗೆ ಬೆಂಕಿ ಹತ್ತಿಕೊಂಡರೆ ಅವನು ನದಿಗೆ ಇಳಿಯಬೇಕೋ ಅಥವಾ ಕೊಳಕ್ಕೋ? ಈ ಕೊಳದಲ್ಲಿ ಸ್ನಾನ ಮಾಡಬಹುದೇ? ಇದು ಒಳ್ಳೆಯ ನೀರೇ? ಇದೆಲ್ಲವನ್ನೂ ಯೋಚಿಸಿಕೊಂಡು ಇರುವುದಿಲ್ಲ. ಯಾವುದಾದರೂ ನೀರನ್ನು ಕಂಡರೆ ತಕ್ಷಣ ಅದರಲ್ಲಿ ಒಂದು ಮುಳುಗು ಮುಳುಗುವನು. ಅದರಂತೆಯೇ ನಾವು ಈ ರೀತಿ ಚಿಂತನೆಯನ್ನು ಇಟ್ಟುಕೊಂಡಿದ್ದರೆ, ನಮ್ಮ ಮನಸ್ಸಿನಲ್ಲಿ ಧರ್ಮವನ್ನು ಆಚರಿಸಬೇಕೆನ್ನುವ ಭಾವನೆ ತಾನಾಗಿಯೇ ಉಂಟಾಗುತ್ತದೆ. ನಾನು ಈ ಪ್ರಪಂಚದಲ್ಲಿ ಸ್ಥಿರವಾದವನೆಂದುಕೊಂಡು ಭಗವಂತನ ಬಗ್ಗೆ ಚಿಂತನೆಯೇ ಇಲ್ಲದೆ ಇದ್ದೇನೆ. ಆದರೂ ನಮ್ಮೆಲ್ಲರಿಗೂ ಪರಮಾತ್ಮನನ್ನು ಕುರಿತು ಯೋಚಿಸಬೇಕೆಂಬ ಆಸೆ ಇದೆ. ಆದರೆ ನಿಜಕ್ಕೂ, ಪ್ರಪಂಚದಲ್ಲಿರುವ ಜನರ ಸ್ಥಿತಿ ಹೀಗಿರುವುದರಿಂದ ಆಗಮಿಕ ದಾರಿಯಲ್ಲಿ ಯಾರೂ ಪಡೆಯುವುದಿಲ್ಲ. ಈ ಶರೀರವೆನ್ನುವುದು ಎಂದೂ ಸ್ಥಿರವಲ್ಲ. ಈ ಮನುಷ್ಯ ಶರೀರ ನಮಗೆ ದೊರೆತಿರುವಾಗ ಇದನ್ನು ಸದುಪಯೋಗಪಡಿಸಿಕೊಂಡು ಸದ್ಗತಿಯನ್ನು ಪಡೆಯ ಬೇಕೆನ್ನುವ ಭಾವನೆ ಉಂಟಾದರೆ ಆಗ ಮನಸ್ಸು ಬೇರೆ ಯಾವ ವಿಷಯಕ್ಕೂ ಹೋಗುವುದಿಲ್ಲ. ಆದ್ದರಿಂದ ಇಂಥ ಭಾವನೆಯನ್ನು ನಾವು ಬೆಳಸಿಕೊಳ್ಳಬೇಕೆಂದು ದೀಕ್ಷಿತರು ಬಹಳ ಸ್ವಾರಸ್ಯಕರವಾಗಿ ಆ ಶ್ಲೋಕದಲ್ಲಿ ಹೇಳಿದ್ದಾರೆ. ನಾವು ಅವರು ಹೇಳಿದಂತೆ ನಡೆಯುತ್ತಾ ಶ್ರೇಯಸ್ಸನ್ನು ಪಡೆಯಬೇಕು.













































 

