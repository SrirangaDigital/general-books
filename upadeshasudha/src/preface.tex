{\dn श्री श्री जगद्गुरु शङ्कराचार्य महासंस्थानम् दक्षिणाम्नाय श्री शारदापीठम् शृङ्गेरी}\\


{\eng V. R. Gowri Shankar, B.E., D.I.I.Sc., AMIMA.\\
Administrator \hfill{Ref. No. 2195}\\
SRI SRINGERI MATH \&\\
ITSPROPERTIES \hfill{Camp}\\
SRINGERI - 577 139 (Karnataka)\hfill{Date: 10-12-1987}\\

Phone: 
\begin{tabular}{lll}
Office  & : & 23\\
Res  & : & 92
\end{tabular}
}

\bigskip

\begin{center}
{\large{ಗುರುಭಕ್ತಾಲಂಕಾರ ಶ್ರೀ ಪಿ. ಆರ್. ಹರಿಹರನ್ ರವರಿಗೆ\\  ನಮಸ್ಕಾರಗಳು}}

\begin{minipage}{8cm}
ತಾವು ಕಳುಹಿಸಿದ ಪುಸ್ತಕಗಳನ್ನು ಶ್ರೀ ಶ್ರೀಗಳವರು ವೀಕ್ಷಿಸಿ\-ದರು. ಜಗದ್ಗುರು
ಮಹಾಸ್ವಾಮಿಗಳವರು ಮಾಡಿರುವ ಉಪನ್ಯಾಸಗಳನ್ನು ಪುಸ್ತಕರೂಪವಾಗಿ
ಹೊರತರಬೇಕೆಂಬ ತಮ್ಮ ಸಂಕಲ್ಪವು ತುಂಬಾ ಶ್ಲಾಘನೀಯವಾಗಿದೆ. ಈ
ಗ್ರಂಥ\-ಗಳು ಭಕ್ತಜನರಿಗೆ ವಿಶೇಷವಾಗಿ ಉಪಕಾರಗಳಾಗಲೆಂದು ಶ್ರೀ
ಶ್ರೀಗಳವರು ಆಶೀರ್ವದಿಸಿರುತ್ತಾರೆ. 
\end{minipage}
\end{center}

\hfill{ಇಂತು ನಮಸ್ಕಾರಗಳು}

\hfill{(ಸಹಿ) {\eng V. R. Gowri Shankar}}

\eject

\begin{center}
{\huge\bfseries{ಅರಿಕೆ}}
\end{center}

\medskip

ದಕ್ಷಿಣಾಮ್ನಾಯ ಶ್ರೀ ಶೃಂಗೇರಿ ಶ್ರೀ ಶಾರದಾಪೀಠದ ಜಗದ್ಗುರುಗಳಾದ ಶ್ರೀ ಶ್ರೀ ಅಭಿನವ ವಿದ್ಯಾತೀರ್ಥ ಮಹಾಸನ್ನಿಧಾನದವರು 
ತಮಿಳುನಾಡಿನಲ್ಲಿ ಪ್ರವಾಸ ಮಾಡುತ್ತಲಿದ್ದಾಗ, ಭಕ್ತರ ನಿವೇದನೆಯಂತೆ, ತಾವು ತಂಗಿದ್ದ ಸ್ಥಳದಲ್ಲಿ ಸಾಯಂಕಾಲ ಅನುಗ್ರಹ 
ರೂಪವಾದ ಉಪದೇಶಾತ್ಮಕ ಉಪನ್ಯಾಸಗಳನ್ನು ಮಾಡುತ್ತಲಿದ್ದರು. ಉಪನ್ಯಾಸಗಳ ವಿಷಯಗಳಾದರೊ ನಮ್ಮ ಸನಾತನ 
ಧರ್ಮ ಹಾಗೂ ವೇದಾಂತಕ್ಕೆ ಸಂಬಂಧಿಸಿದವು. ಸರಳವಾದ ಭಾಷೆಯಲ್ಲಿ ಅಖ್ಯಾಯಿಕೆಗಳ ಮೂಲಕ ಗಹನವಾದ ತತ್ತ್ವಗಳನ್ನು 
ಜನರಿಗೆ ಮನದಟ್ಟಾಗುವಂತೆ ಪ್ರತಿಪಾದನೆ ಮಾಡುವುದು ಶ್ರೀ ಶ್ರೀ ಜಗದ್ಗುರುಗಳವರ ವೈಶಿಷ್ಟ್ಯ. ಈ ಉಪನ್ಯಾಸಗಳೆಲ್ಲವೂ ತಮಿಳು 
ಭಾಷೆಯಲ್ಲಿ ಗ್ರಂಥರೂಪದಲ್ಲಿ ಪ್ರಕಟವಾಗಿರುತ್ತವೆ. ಈ ಉಪನ್ಯಾಸಗಳೆಲ್ಲವನ್ನೂ ಕನ್ನಡ ಭಾಷೆಯಲ್ಲಿಯೂ ತರಬೇಕೆಂಬ ನಮ್ಮ 
ಅಭಿಲಾಷೆಯನ್ನು ನೆರೆವೇರಿಸಿಕೊಟ್ಟವರು ಸನ್ಮಾನ್ಯ ಡಾ|| ಎನ್. ಎಸ್. ದಕ್ಷಿಣಾಮೂರ್ತಿಯವರು. ಉಪನ್ಯಾಸಗಳನ್ನು ಕ್ರಮವಾಗಿ 
ಶ್ರೀ ಶಂಕರಕೃಪಾ ಮಾಸಪತ್ರಿಕೆಯಲ್ಲಿ ಪ್ರಕಟ ಮಾಡಿದೆವು. ಅವುಗಳನ್ನು ಒಂದು ಗ್ರಂಥರೂಪದಲ್ಲಿ ತರಬೇಕೆಂಬ ನಮ್ಮ ಅಭಿಲಾಷೆ ಕಾರ್ಯಗತವಾಗಿ, 
ಉಪನ್ಯಾಸ ಮಾಲೆಯ ಎರಡು ಭಾಗಗಳನ್ನೂ ಒಟ್ಟಿಗೆ ಸೇರಿಸಿ ಓದುಗರಿಗೆ ಅರ್ಪಿಸುತ್ತಿದ್ದೇವೆ. 

ಭಾಷಾಂತರ ಮಾಡಿಕೊಟ್ಟ ಡಾ|| ಎನ್. ಎಸ್. ದಕ್ಷಿಣಾಮೂರ್ತಿಯವರಿಗೆ ನಮ್ಮ ಕೃತಜ್ಞತೆಗಳು.

ಆಸ್ತಿಕರು ಈ ಗ್ರಂಥದ ಸಂಪೂರ್ಣ ಪ್ರಯೋಜನವನ್ನು ಪಡೆಯುವರೆಂದು ಆಶಿಸಲಾಗಿದೆ.


\vskip 1cm

\hfill{ಪ್ರಕಾಶಕರು}

