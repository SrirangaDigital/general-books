\chapter{ಶ್ರೀ ಶಂಕರ ಭಗವತ್ಪಾದಾಚಾರ್ಯರು}\label{chap3}

ಅನಾದಿಯಿಂದ ಮಾನವನ ಆಧ್ಯಾತ್ಮಿಕ, ಭೌತಿಕ ಪ್ರಯೋಜನಕ್ಕಾಗಿ ಈಶ್ವರ ನಿಃಶ್ವಾಸರೂಪವಾದ ವೇದಗಳಿವೆ. ವೇದಗಳ ಆಧಾರದ ಮೇಲೆ ಸ್ಮೃತಿಗಳು ಬೆಳಕಿಗೆ ಬಂದವು. ಅನಾದಿಯಾಗಿ ಎಂದರೆ ಮಾನವನ ಅವತರಣ ಕಾಲದಿಂದಲೂ ಎಂದು ಭಾವ. ಶಾಸ್ತ್ರಗಳು, ಸಂಪ್ರದಾಯ ನಮಗೆ ತಿಳಿಸುವುದರ ಪ್ರಕಾರ ನಮ್ಮ ಅಭ್ಯುನ್ನತಿಗಾಗಿ ಈಶ್ವರನು ನಮಗೆ ವೇದಗಳನ್ನು ನೀಡಿದನು. ವೇದಗಳು ನಾವು ಅನುಷ್ಠಾನ ಮಾಡಬೇಕಾದ ಕರ್ಮಗಳನ್ನು ತಿಳಿಸುತ್ತವೆ. ಜಗತ್ತಿನ ಸ್ಥಿತಿಯನ್ನು ನಿಯಮಿಸುವ ಶಾಸನಗಳನ್ನು ಉಪಾಸಿಸಲು ಅವು ಹೇಳುತ್ತವೆ. ಪ್ರಪಂಚದ ಸೃಷ್ಟಿಯನ್ನು ಮಾಡಿದವನ ಬಗ್ಗೆ ಯಥಾರ್ಥ ಜ್ಞಾನವನ್ನು ಅವು ತಿಳಿಸುತ್ತವೆ. ಭಿನ್ನ ಭಿನ್ನ ಸ್ಥಳಗಳಲ್ಲಿ ಈ ಕರ್ಮ, ಉಪಾಸನೆ ಮತ್ತು ಜ್ಞಾನದ ಪ್ರಾಮುಖ್ಯ ಹೇಳಲಾಗಿದೆ. ಒಂದರ ಪ್ರಾಮುಖ್ಯವನ್ನು ಕುರಿತು ಹೇಳುವಾಗ ಉಳಿದ ಎರಡರ ಪ್ರಾಮುಖ್ಯವನ್ನು ಹೇಳುವ ಹಾಗಿಲ್ಲ.

ಭೌತಿಕ ಸುಖಾದಿಗಳು, ಐಶ್ವರ್ಯಾದಿಗಳು ಬೇಕೆನ್ನುವವನು ಕರ್ಮಗಳನ್ನೂ ಉಪಾಸನೆಯನ್ನೂ ಅವಲಂಬಿಸಬೇಕು. ಆದರೆ ಜ್ಞಾನ ಸರ್ವೋತ್ತಮವಾದುದು. ಕರ್ಮ-ಉಪಾಸನೆಗಳನ್ನು ಸೋಪಾನವಾಗಿ ಉಪಯೋಗಿಸುವುದರಿಂದಲೇ ಜ್ಞಾನವನ್ನು ಪಡೆಯಲು ಸಾಧ್ಯವಾಗುವುದು. ಕರ್ಮೋಪಾಸನೆಯಲ್ಲಿಯೇ ನಿಲ್ಲುವವನಿಗೆ ಜ್ಞಾನ ಲಭಿಸುವುದಿಲ್ಲ. ಅನೇಕ ಶತಾಬ್ದಿಗಳವರೆಗೆ ಕರ್ಮೋಪಾಸನೆಗಳಿಗೆ ಪ್ರಾಮುಖ್ಯ ಕೊಡಲಾಗಿತ್ತು. ಜ್ಞಾನಹೇತುವಾದ ಶ್ರವಣ, ಮನನ, ನಿದಿಧ್ಯಾಸನಗಳನ್ನು ಮರೆತಂತಾಯಿತು. ಜನರು ಯಾಂತ್ರಿಕವಾದ ಬಾಳಿಗೆ ಅಭ್ಯಸ್ತರಾದರು. ತಮ್ಮ ತಾರ್ಕಿಕ ಲಕ್ಷ್ಯದಿಂದ ಶ್ರೇಯಸ್ಸನ್ನು ಸಾಧಿಸಬಹುದೆಂದು ಭಾವಿಸಿದರು. ಅದರಿಂದಾಗಿ ಮಾನವ ನಿರ್ಮಿತವಾದ ವೇದದೂರವಾದ ಮತಗಳು ಪ್ರಾದುರ್ಭವಿಸಿದುವು. ಇದಕ್ಕೆ ಅನೇಕ ಕಾರಣಗಳಿವೆ.

`ಭಕ್ಷ್ಯಾದ್ಯನಿಯಮ ಇತಿ ರಾಗಿಣಿಃ ಸ್ವೇಚ್ಛಯಾ ದಾರಪರಿಗ್ರಹ ಇತಿ ಕುತರ್ಕಾಭ್ಯಾಸಿನಃ ಕರ್ಮಲಾಘವಮಿತ್ಯಲಸಾಃ, ಇತಃ ಪತಿತಾನಾಮಪ್ಯಸ್ತ್ಯನುಪ್ರವೇಶ ಇತಿ ಅನನ್ಯ ಗತಿಕಾಃ|'

ಇಚ್ಛೆಗಳಿರುವವರು ಮತಾಂತರನ್ನು ಮಾಡಿಕೊಳ್ಳಲು ಕಾರಣ ಆಹಾರ ವಿಷಯವಾದ ಸ್ವಾತಂತ್ರ್ಯ. ಕುತರ್ಕ ಮಾಡುವವರ ಮತಾಂತರಕ್ಕೆ ಕಾರಣ ದಾರಪರಿಗ್ರಹಣದಲ್ಲಿ ಸ್ವೇಚ್ಚೆ. ಸೋಮಾರಿಗಳ ಮತಾಂತರಕ್ಕೆ ಕಾರಣ ಕರ್ಮಭಾರವಿಲ್ಲದೆ ಹೋದುದು. ಅನನ್ಯ ಗತಿಕರ ಮತಾಂತರಕ್ಕೆ ಕಾರಣ ಪತಿತರಿಗೆ ಕೂಡ ಪ್ರವೇಶವಿದೆ ಎನ್ನುವುದು.

ಈ ಕಾರಣಗಳಿಗಾಗಿ ಜನರು ತಪ್ಪು ದಾರಿಯನ್ನು ಹಿಡಿದರು. ಪ್ರತಿಯೊಬ್ಬರೂ ತಮಗೆ ತೋಚಿದಂತೆ ತೀರ್ಮಾನ ಮಾಡಿದರೆ ಪ್ರಮಾಣವಾದ ನಿರ್ಣಯವನ್ನು ಮಾಡುವುದಾದರೂ ಹೇಗೆ?

\begin{shloka}
`ಕೈಶ್ಚಿದಭಿಯುಕ್ತ್ಯರ್ಯತ್ನೇನೋತ್ಪ್ರೇಕ್ಷಿತಾಸ್ತರ್ಕಾ\\
ಅಭಿಯುಕ್ತತರೈರನ್ಯೈರಾಭಾಸ್ಯಮಾನಾ ದೃಶ್ಯಂತೇ||'
\end{shloka}

ಕೆಲವರು ಬುದ್ಧಿವಂತರು ತಮ್ಮ ಬುದ್ಧಿಯ ಮೂಲಕ ಯಾವುದನ್ನು ಸರಿಯೆಂದು ತೋರಿಸಿದರೋ ಅದನ್ನೇ ಹೆಚ್ಚು ಬುದ್ಧಿವಂತರು ತಪ್ಪೆಂದು ತೋರಿಸುತ್ತಿದ್ದರು.

ಮುಖ್ಯವಾಗಿ ನಾವು ಗಮನಿಸಬೇಕಾದುದು - ಭಗವಂತನಿದ್ದಾನೆ. ಅವನು ಸೃಷ್ಟಿ ಮಾಡಿದಾಗಲೇ ತಾನು ಸೃಷ್ಟಿಸಿದ ಮಾನವರಿಗಾಗಿ ನಿಯಮಗಳನ್ನು ಮಾಡಿರಬೇಕು. ಪರಿಪಾಲಿಸುವ ರಾಜನು ಕೂಡ ತನ್ನ ನಿಯಮಗಳನ್ನು ಪ್ರಕಾಶಿಸಬೇಕು. ನಮಗೆ ನಮ್ಮ ಶಾಸ್ತ್ರಗಳಿರಬೇಕು. ಶಾಸ್ತ್ರಗಳ ಪ್ರಕಾರ ಕರ್ಮಫಲವಿರಬೇಕು. ಈಶ್ವರನು ಮಾನವರನ್ನು ಪಾಲಿಸಲು ಶಾಸ್ತ್ರಗಳಲ್ಲಿ ನಿಯಮಗಳನ್ನು ಪ್ರಕಾಶಿಸಿದನು. ಅವುಗಳಿಲ್ಲದಿದ್ದರೆ ಅಯುಕ್ತವಾಗುತ್ತಿತ್ತು. ಕುದುರೆಯ ಮೇಲೆ ಲಗಾಮಿಲ್ಲದೆ ಸವಾರಿ ಮಾಡಿದಂತೆ.

ವೇದಗಳ ಮೇಲೆ ನಂಬಿಕೆ ಇಲ್ಲದಿರುವ ಮತಗಳ ಪ್ರಾಬಲ್ಯದಿಂದಾಗಿ ವೇದಮತ ಕ್ಷೀಣವಾಯಿತು. ಆ ಸಮಯದಲ್ಲಿ ಪೂರ್ಣಾನದಿ ತೀರದಲ್ಲಿ, ಬಾಲಕೃಷ್ಣ ದೇವಾಲಯದ ಹತ್ತಿರ ಪವಿತ್ರ ಕುಟುಂಬದಲ್ಲಿ ಶ್ರೀಶಂಕರ ಭಗವತ್ಪಾದರು ಪ್ರಾದುರ್ಭವಿಸಿದರು. ಅವರು ಬಹಳ ಚಿಕ್ಕ ವಯಸ್ಸಿನಲ್ಲೆ ಶಾಸ್ತ್ರಾಧ್ಯಯನವನ್ನು ಪೂರ್ತಿಗೊಳಿಸಿದರು. ಅವರು ಪಡೆದ ಜ್ಞಾನದ ಕಾರಣದಿಂದಾಗಿ ಐಹಿಕ ಸುಖಭೋಗಗಳಲ್ಲಿ ಅವರಿಗೆ ಇಚ್ಛೆ ಇಲ್ಲದೆ ಹೋಯಿತು. ವೇದಪ್ರಮಾಣವಲ್ಲದ ಮತಗಳು ಸಾರಹೀನವಾದುವೆಂದೂ, ಅಂಥಹವುಗಳನ್ನು ನಿರೋಧಿಸಬೇಕೆಂದೂ ಅವರು ಭಾವಿಸಿದರು. ಅವರು ತಮ್ಮ ತಂದೆ-ತಾಯಿಗೆ ಒಬ್ಬರೇ ಸಂತಾನ. ತಂದೆ ಸ್ವರ್ಗಸ್ಥರಾದರು. ಶ್ರೀಶಂಕರರು ಲೋಕಸಂಗ್ರಹವನ್ನು ವಾಂಛಿಸಿದರು. ಅವರು ಮಹರ್ಷಿ.

\begin{shloka}
ಅಯಂ ನಿಜಃ ಪರೋವೇತಿ ಗಣನಾ ಲಘುಚೇತಸಾಮ್|\\
ಉದಾರಚರಿತಾನಾಂ ತು ವಸುಧೈವ ಕುಟುಂಬಕಮ್||
\end{shloka}

ಅವನು ನನ್ನವನು. ಇವನು ಬೇರೆಯವನು. ಹೀಗೆ ಸಂಕುಚಿತ ಭಾವನೆಯುಳ್ಳವರು ಭಾವಿಸುತ್ತಾರೆ. ಉದಾರ ಚರಿತರಿಗೆ ಲೋಕವೆಲ್ಲಾ ಕುಟುಂಬವೇ

ಇದಕ್ಕೆ ಶ್ರೀಭಗವತ್ಪಾದರು ಉದಾಹರಣೆ, ಈ ಅಭಿಯುಕ್ತೋಕ್ತಿ ಶ್ರೀಭಗವತ್ಪಾದರ ರೂಪದಲ್ಲಿ ಅವತರಿಸಿತೆಂದು ಹೇಳಬೇಕು.

ಅವರು ತಮ್ಮ ತಾಯಿಯ ಮನಸ್ಸಿಗೆ ಕಷ್ಟಕೊಡಲು ಇಷ್ಟಪಡಲಿಲ್ಲ. ಪೂರ್ಣಾನದಿಯಲ್ಲಿ ಸ್ನಾನ ಮಾಡುತ್ತಿದ್ದಾಗ ಒಂದು ಮೊಸಳೆ ಅವರನ್ನು ಹಿಡಿದುಕೊಂಡಿತು. ಅವರು ಸಹಾಯಕ್ಕಾಗಿ ಕೂಗಿದರು. ತಾಯಿ ಬಂದು ನೋಡಿ ಗಾಬರಿಗೊಂಡಳು. ಆಕೆಯೊಡನೆ ಅವರು `ಸಂನ್ಯಾಸವೆಂದರೆ ಎಲ್ಲರಿಗೂ ಅಭಯವನ್ನು ಕೊಡುವುದು, ನನ್ನಿಂದ ಭಯಗೊಂಡು ಮೊಸಳೆ ನನ್ನನ್ನು ಹಿಡಿದುಕೊಂಡಿದೆ. ನನ್ನಿಂದ ಎಲ್ಲರಿಗೂ ಅಭಯವಾದರೆ ಇದು ನನ್ನನ್ನು ಬಿಟ್ಟು ಬಿಡುವುದು ಆದ್ದರಿಂದ ನಾನು ಸಂನ್ಯಾಸ ಸ್ವೀಕರಿಸಬೇಕು' ಎಂದರು. ನಿಸ್ಸಹಾಯಳಾದ ಆಕೆ ಒಪ್ಪಿಕೊಂಡರು. ಅವರು ಗುರುವನ್ನು ಹುಡುಕಿಕೊಂಡು ನರ್ಮದಾ ತೀರಕ್ಕೆ ಬಂದರು. ಅಲ್ಲಿ ಶ್ರೀ ಗೋವಿಂದ ಭಗವತ್ಪಾದರಿಂದ ಕ್ರಮ ಸಂನ್ಯಾಸವನ್ನು ಸ್ವೀಕರಿಸಿದರು. ಆ ಗುರುವಿನಲ್ಲಿ ಪಡೆದ ವಿದ್ಯೆಯೇ ಫಲಿಸಿತು.

ಶ್ರೀಶಂಕರರ ಶಕ್ತಿಸಾಮರ್ಥ್ಯಗಳನ್ನು ಕಂಡು ಅವರ ಗುರುಗಳು ವಿಸ್ಮಯಾನಂದಚಕಿತರಾದರು. ಲೋಕಸಂಗ್ರಹಕ್ಕಾಗಿ ಬ್ರಹ್ಮಸೂತ್ರಗಳಿಗೆ ಭಾಷ್ಯ ಬರೆಯುವಂತೆ ಗುರುಗಳು ಅವರಿಗೆ ಅಪ್ಪಣೆ ಕೊಟ್ಟರು. ಗುರುಗಳ ಅಪ್ಪಣೆ ಪ್ರಕಾರ ಶ್ರೀಶಂಕರರು ಬ್ರಹ್ಮಸೂತ್ರ-ಭಗವದ್ಗೀತಾ ಉಪನಿಷದ್ಭಾಷ್ಯಗಳನ್ನು ರಚಿಸಿದರು. ಶ್ವೇತಾಶ್ವತರೋಪನಿಷತ್ತಿಗೂ ಭಾಷ್ಯವನ್ನು ಬರೆದರು. ಮಹಾಭಾರತದಲ್ಲಿರುವ ಸನತ್ಸುಜಾತೀಯಕ್ಕೆ ಕೂಡ ವ್ಯಾಖ್ಯೆ ಬರೆದರು. ಮತ್ತೂ ಎಷ್ಟೋ ಗ್ರಂಥಗಳನ್ನು ರಚಿಸಿದರು. ಸರ್ವವ್ಯಾಪಿಯಾದ ಬ್ರಹ್ಮವೊಂದೇ ಸತ್ಯವೆಂದು ದೃಢಪಡಿಸಿದರು, ಅದೇ ತುರೀಯವೆಂದರು.

ಶ್ರೀಶಂಕರರ ಭಾಷ್ಯಗಳಲ್ಲಿ ಸುಂದರವಾದ ಉದಾಹರಣೆಗಳಿವೆ. ಶಾಸ್ತ್ರಜ್ಞಾನವಿರುವವರಿಗೆ ಶ್ರೀಶಂಕರರ ಗ್ರಂಥಗಳು ಚೆನ್ನಾಗಿ ಅರ್ಥವಾಗುತ್ತವೆ. ತಾರ್ಕಿಕರ ವಾದಗಳಿಗೆಲ್ಲ ಅವರು ಸಮಾಧಾನ ಹೇಳಿ ಅವರ ವಾದಗಳನ್ನು ಪೂರ್ವಪಕ್ಷ ಮಾಡಿದರು. ಅವರ ಗ್ರಂಥಗಳು ಪಂಡಿತರಿಗೆ ಎಷ್ಟೋ ಮೋದವನ್ನುಂಟು ಮಾಡುತ್ತವೆ. ಆದರೆ ಶ್ರೀ ಶಂಕರರು ಪಂಡಿತರಿಗಾಗಿ ಮಾತ್ರವಲ್ಲ ಭೂಮಿಯ ಮೇಲೆ ಅವತರಿಸಿದುದು ಪ್ರತಿಯೊಬ್ಬರಿಗೂ ಪ್ರಯೋಜನವಾಗಬೇಕು. ಪ್ರತಿಯೊಬ್ಬರೂ ಸಂಸಾರ ಸಾಗರವನ್ನು ದಾಟಬೇಕು ಎನ್ನುವುದು ಅವರ ಭಾವನೆ. ಅದಕ್ಕಾಗಿ ಅವರು ವಿವೇಕಚೂಡಾಮಣಿ, ಉಪದೇಶಸಾಹಸ್ರೀ, ಪ್ರಬೋಧಸುಧಾಕರ ಮುಂತಾದ ಗ್ರಂಥಗಳನ್ನು ರಚಿಸಿದರು.

ಈ ಗ್ರಂಥಗಳನ್ನು ಓದಲು ಬಹಳ ದಿನಗಳು ಬೇಕಾಗಿಲ್ಲ. ಈ ಗ್ರಂಥಗಳನ್ನು ಅರ್ಥಮಾಡಿಕೊಳ್ಳಲು ಶಾಸ್ತ್ರಜ್ಞಾನ ಕೂಡ ಅವಶ್ಯಕವಿಲ್ಲ. ಸ್ವಲ್ಪ ಶ್ರದ್ಧೆ ಮತ್ತು ಸ್ವಲ್ಪ ಸಂಸ್ಕೃತ ಭಾಷಾಜ್ಞಾನವಿದ್ದರೆ ಸಾಕು.

ಶ್ರೀಶಂಕರರ ಜನ್ಮಸ್ಥಾನ ಕೇರಳ. ಆ ಪ್ರಾಂತೀಯ ಭಾಷೆ ಅವರ ಮಾತೃಭಾಷೆಯಾಗಿದ್ದಿರಬಹುದು. ಆದರೆ ಅವರು ತಮ್ಮ ಮಾತೃಭಾಷೆಯಲ್ಲಿ ಗ್ರಂಥಗಳನ್ನು ಬರೆಯಲಿಲ್ಲ. ವಿಶಾಲ-ದೃಷ್ಟಿಯಿಂದ ಸಂಸ್ಕೃತ ಆ ಕಾಲದಲ್ಲಿ ಅಖಿಲ ಭಾರತ ವಿದ್ವದ್ಭಾಷೆಯಾದುದರಿಂದ ಆ ಭಾಷೆಯಲ್ಲಿ ಎಲ್ಲರಿಗೂ ಉಪದೇಶ ಯೋಗ್ಯವಾದ ತಮ್ಮ ಗ್ರಂಥಗಳನ್ನು ಬರೆದರು. ಎಲ್ಲ ಧರ್ಮಗಳಿಗೂ ವೇದಗಳು ಮೂಲ, ವೇದಗಳು ಸಂಸ್ಕೃತದಲ್ಲಿವೆ. ಆದ್ದರಿಂದ ಸಂಸ್ಕೃತದಲ್ಲಿ ಅವರು ಗ್ರಂಥರಚನೆ ಮಾಡಿದುದು ಸಮುಚಿತವೇ.

ಶ್ರೀಶಂಕರರು ಒಂದು ಹೊಸ ಮತವನ್ನು ಪ್ರಚಾರಮಾಡಲು ಪ್ರಯತ್ನಿಸಲಿಲ್ಲ. ಹಾಗೆ ಮಾಡುವುದು ನಿಷ್ಪ್ರಯೋಜನಕರ. `ವೇದ ಸಾರವೇನು? ವೇದವಿಹಿತ ಧರ್ಮಗಳನ್ನು ಬಿಡಬಹುದೇ? ಅವುಗಳನ್ನು ವಿಸ್ಮರಿಸಿದರೆ ನಮಗೆ ಶ್ರೇಯಸ್ಸಿಲ್ಲ. ಸತ್ಯದ ಬಗ್ಗೆ ನಮಗೆ ತಿಳಿಯದು'-ಇವು ಅವರ ಉಪದೇಶ. ಇದು ಹೊಸ ಮತವಲ್ಲ. ಹೊಸ ಆಂದೋಳನವಲ್ಲ. ವೇದಗಳು ಸಂಸ್ಕೃತದಲ್ಲಿರುವಾಗ ವೇದಸಾರವನ್ನು ಬೇರೆ ಭಾಷೆಯಲ್ಲಿ ಹೇಳುವುದು ಶಂಕರರಿಗೆ ಹೇಗೆ ಸಮಂಜಸ? ಪ್ರತಿಯೊಬ್ಬರಲ್ಲೂ ನಿತ್ಯಾನಿತ್ಯ ವಸ್ತುವಿವೇಕ, ವೈರಾಗ್ಯ, ಶಮಾದಿಷಟ್ಕ ಮತ್ತು ಮೋಕ್ಷೇಚ್ಛೆ ಇರುವುದಿಲ್ಲ. ಅವು ಇಲ್ಲದವರಿಗೆ ವೇದಾಂತ ಶ್ರವಣಯೋಗ್ಯತೆ ಇಲ್ಲ. ಇನ್ನು ಮನನ, ನಿದಿಧ್ಯಾಸಗಳು ಹೇಗೆ? ಅವರಿಗೆ ವೇದಜ್ಞಾನ ಲಾಭಯೋಗ್ಯತೆ ಇಲ್ಲ. ಅವರಿಗೂ ಒಳ್ಳೆಯದು ಆಗಬೇಕು. ಭಗವಂತನ ಅನುಗ್ರಹವಾದರೆ ಎಲ್ಲವನ್ನೂ ಸಾಧಿಸಬಹುದು. ಅದಕ್ಕಾಗಿ ಶ್ರೀಶಂಕರರು `ಪ್ರಪಂಚಸಾರ'ವೆನ್ನುವ ಮಂತ್ರ ಶಾಸ್ತ್ರಗ್ರಂಥ ಒಂದನ್ನು ರಚಿಸಿದರು. ದೇವತೆಗಳನ್ನು ಎಲ್ಲರೂ ಪ್ರಾರ್ಥಿಸಬೇಕೆನ್ನುವುದು ಅವರ ಸದಭಿಪ್ರಾಯ. ಹಾಗೆ ಪ್ರಾರ್ಥಿಸಿ ಭಗವದನುಗ್ರಹವನ್ನು ಸಂಪಾದಿಸಿ ಶ್ರೇಯ-ಪ್ರೇಯಸ್ಸುಗಳನ್ನು ಸಾಧಿಸಿಕೊಳ್ಳಲೆಂದೂ ದೇವತಾನುಗ್ರಹದಿಂದ ಪ್ರತಿಯೊಬ್ಬರೂ ವೇದಾಂತ ತತ್ತ್ವಗಳನ್ನು ತಿಳಿಯಬಲ್ಲರೆಂದೂ ಅವರು ಭಾವಿಸಿದರು. ಇಂಥ ದೊಡ್ಡ ಔದಾರ್ಯ ಭಾವದಿಂದ ಅವರು ರಚಿಸಿರುವ ಸ್ತೋತ್ರಗಳನ್ನು ಪಠಿಸಿ ಯಾರು ಬೇಕಾದರೂ ಅವುಗಳ ಪ್ರಯೋಜನವನ್ನು ಪಡೆಯಬಹುದು.

ಶ್ರೀಭಗವತ್ಪಾದರು ಆಸೇತು ಹಿಮಾಚಲ ಪರ್ಯಂತ ಎರಡು ಸಾರಿ ಪರ್ಯಟನೆ ಮಾಡಿದರು. ಅವರು ತಮ್ಮ ವಿದ್ವತ್ತಿನಿಂದ ಪಂಡಿತರೆಲ್ಲರೂ ಅವರ ವಾದವನ್ನು ಅಂಗೀಕರಿಸುವಂತೆ ಮಾಡಿದರು. ಆ ದಿನಗಳಲ್ಲಿ ಎಷ್ಟೋ ಚಿಕ್ಕ ಚಿಕ್ಕ ಮತಗಳು ಹುಟ್ಟಿಕೊಂಡಿದ್ದವು. ಅಂಥವರನ್ನು ವೇದಮತಾನುಯಾಯಿಗಳನ್ನಾಗಿ, ಮಾಡಿ - `ನೀನು ಶಿವನನ್ನಾಗಲಿ, ದೇವಿಯನ್ನಾಗಲಿ, ಗಣಪತಿಯನ್ನಾಗಲಿ, ಸೂರ್ಯನನ್ನಾಗಲಿ, ಸುಬ್ರಹ್ಮಣ್ಯನನ್ನಾಗಲಿ, ಬೇರೆ ಯಾವ ದೇವರನ್ನಾಗಲಿ ಪೂಜಿಸಿದರೂ ನಿಜಕ್ಕೂ ಸಚ್ಚಿದಾನಂದ ರೂಪನಾದ ಪರ ಬ್ರಹ್ಮನನ್ನೇ ಆರಾಧಿಸುತ್ತಿರುವೆ' ಎಂದು ಅವರು ಬೋಧಿಸಿದರು. ಯಾರಾದರೂ ತಮ್ಮ ಅಭಿರುಚಿಗೆ ಅನುಗುಣವಾಗಿ ಆರಾಧಿಸಬಹುದು. `ಏಕಂ ಸದ್ವಿಪ್ರಾ ಬಹುಧಾ ವದನ್ತಿ' - ಬ್ರಹ್ಮ ಒಂದೇ, ಜ್ಞಾನಿಗಳು ಹಲವು ವಿಧವಾಗಿ ಹೇಳುತ್ತಾರೆ.

ಇದನ್ನು ನಾವು ಮರೆಯಬಾರದು. `ಧರ್ಮಭೂಮಿ' ಎಂದರೆ ನಾವು ವಾಸಿಸುವ ಪಟ್ಟ‌ಣ, ಸುತ್ತ ಮುತ್ತಿನ ಪ್ರದೇಶ ಮಾತ್ರವೇ ಅಲ್ಲ. ಭಾರತ ದೇಶವೆಲ್ಲ `ಧರ್ಮಭೂಮಿ' ಎಂದು ಶ್ರೀಶಂಕರರು ಹೇಳಿದರು. ಅವರು ಭಾರತ ದೇಶದ ನಾಲ್ಕು ದಿಕ್ಕುಗಳಲ್ಲಿಯೂ ಪ್ರಾಕ್ ದಕ್ಷಿಣ ಪೂರ್ವೋತ್ತರಗಳಲ್ಲಿ ನಾಲ್ಕು ಮಠಗಳನ್ನು ಸ್ಥಾಪಿಸಿ ಅವುಗಳಿಗೆ ನಾಲ್ಕು ಮಹಾವಾಕ್ಯಗಳನ್ನು ಕೊಟ್ಟರು. ಈಗಲೂ ಆ ಮಠಗಳನ್ನು ನೋಡಬಹುದು.

ಅವರ ಕಾರಣದಿಂದಾಗಿ ವೇದಗಳು ಈಗಲೂ ಸ್ಥಿರವಾಗಿವೆ. ಶಿವ-ವಿಷ್ಣು ದೇವತೆಗಳ ದೇವಾಲಯಗಳು ನಿಂತಿವೆ. ಅವರು ಇಲ್ಲದೆ ಹೋಗಿದ್ದರೆ ಧರ್ಮದ ಸ್ವರೂಪ ನಮಗೆ ತಿಳಿಯುತ್ತಿರಲಿಲ್ಲ. ಸಂಸಾರ ಬದ್ಧನಾದ ಜೀವನು ವಿಮುಕ್ತಿಯನ್ನು ಪಡೆದು ಕೈವಲ್ಯವನ್ನು ಸಾಧಿಸುವ ಮಾರ್ಗವನ್ನು ಅವರು ನಮಗೆ ಬೋಧಿಸಿದರು, ಅನುಗ್ರಹಿಸಿದರು. ಅವರು ರಚನೆಯಾದ `ಮೋಹ ಮುದ್ಗರ' ದಲ್ಲಿನ ಈ ಶ್ಲೋಕ ಮನನೀಯವಾಗಿದೆ.

\begin{shloka}
ತ್ವಯಿ ಮಯಿ ಚಾನ್ಯತ್ರೈಕೋ ವಿಷ್ಣುಃ ವ್ಯರ್ಥಂ ಕುಪ್ಯಸಿ ಮಯ್ಯಸಹಿಷ್ಣುಃ|\\
ಸರ್ವಸ್ಮಿನ್ನಪಿ ಪಶ್ಯಾತ್ಮಾನಂ ಸರ್ವತ್ರೋಸ್ತೃಜ ಭೇದ ಜ್ಞಾನಮ್||
\end{shloka}

ನಿನ್ನಲ್ಲಿ, ನನ್ನಲ್ಲಿ ಎಲ್ಲರಲ್ಲಿಯೂ ಒಬ್ಬನೇ ಪ್ರಭುವಿದ್ದಾನೆ, ವ್ಯರ್ಥವಾಗಿ ನನ್ನ ಮೇಲೆ ಕೋಪಿಸಿಕೊಳ್ಳಬೇಡ. ಎಲ್ಲರಲ್ಲಿಯೂ ಆತ್ಮನನ್ನು ನೋಡು, ಭೇದಜ್ಞಾನವನ್ನು ಬಿಡು.

