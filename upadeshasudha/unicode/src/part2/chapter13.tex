\chapter{ಪರವಸ್ತುವಿನ ವಿಚಾರ}\label{chap13}

(1977ರಲ್ಲಿ ಮದ್ರಾಸಿನಲ್ಲಿ ತಮಿಳು ಭಾಷೆಯಲ್ಲಿ  ಅನುಗ್ರಹಿಸಿದ ಉಪನ್ಯಾಸ  ಕನ್ನಡ ರೂಪಾಂತರ)

ವ್ಯಾಸರು ತಮ್ಮ ಬ್ರಹ್ಮಸೂತ್ರದಲ್ಲಿ ಯಾರು ಯಾರಿಗೆ ಬ್ರಹ್ಮ ವಿಚಾರ ಮಾಡಲು ಯೋಗ್ಯತೆ ಇದೆ ಎನ್ನುವುದನ್ನು ವಿಶದವಾಗಿ ತಿಳಿಸಿದ್ದಾರೆ. ಬ್ರಹ್ಮವಿಚಾರ ಮಾಡಬೇಕಾದರೆ ಯಾವ ರೀತಿ ಮಾಡಬೇಕು? ಬ್ರಹ್ಮದ ಲಕ್ಷಣ ಏನು? ಸ್ವರೂಪ  ಯಾವುದು? ಎನ್ನುವುದೆಲ್ಲವನ್ನೂ ನಾವು ತಿಳಿದಿರಬೇಕು. ಶಂಕರ ಭಗವತ್ಪಾದರು ಬ್ರಹ್ಮದ ಸ್ವರೂಪವೇನೆಂದು ಮೊದಲು ಕೇಳಿ ಸೂತ್ರಕ್ಕೆ ವಿವರವನ್ನು ಕೊಡುವಾಗ ಅದಕ್ಕೆ ಉತ್ತರವನ್ನೂ ಅವರೇ ಕೊಟ್ಟಿದ್ದಾರೆ. 

ಜನ್ಮಾದ್ಯಸ್ಯ ಯತಃ

(ಯಾವುದರಿಂದ ಇದರ ಜನ್ಮ ಮೊದಲಾದವು)-ಎಂದು ಒಂದು ಸೂತ್ರವಿದೆ. 

ಬ್ರಹ್ಮವನ್ನು ಕುರಿತು ಪೂರ್ತಿಯಾಗಿ ತಿಳಿದು ಕೊಂಡ ಮೇಲೆ ಬ್ರಹ್ಮವಿಚಾರ ಮಾಡಬೇಕಾದುದೇ ಇಲ್ಲ. ಬ್ರಹ್ಮವನ್ನು ಕುರಿತು ನಮಗೆ ಏನೂ ಗೊತ್ತಿಲ್ಲವೆಂದರೂ ಸಹ ನಾವು ವಿಚಾರ ಮಾಡಲಾಗುವುದಿಲ್ಲ. ಆದ್ದರಿಂದಲೇ ನಾವು- 

ಯಸ್ಮಾತ್ ಜನ್ಮಸ್ಥಿತಿಭಂಗಃ

[ಯಾವುದರಿಂದ (ಪ್ರಪಂಚದ) ಸೃಷ್ಟಿ, ರಕ್ಷಣೆ, ನಾಶ...]

-ಎಂದು ಹೇಳಲ್ಪಟ್ಟಿರುನಂತೆ ಆ ಬ್ರಹ್ಮವನ್ನು ಕುರಿತು ಪ್ರಪಂಚದ ಸೃಷ್ಟಿ, ರಕ್ಷಣೆ ಮತ್ತು ನಾಶ ಇವುಗಳು  ಚೈತನ್ಯದಿಂದಲೇ ಉಂಟಾಗುತ್ತದೆಂದು ಹೇಳಿ ವಿಚಾರ ಮಾಡಲು ಸಾಧ್ಯವಾಗುತ್ತದೆ. ಬ್ರಹ್ಮವೆನ್ನುವುದು ಗುಣ, ರೂಪ ಇಲ್ಲದುದು. ಅದಕ್ಕೆ ಲಕ್ಷಣವನ್ನು ಹೇಳುತ್ತೀರಲ್ಲಾ ಎಂದು ಕೇಳಿದರೆ, ಲಕ್ಷಣಗಳು ಎರಡು ವಿಧವೆಂದು ಶಾಸ್ತ್ರಗಳಲ್ಲಿ  ಹೇಳಲ್ಪಟ್ಟಿವೆ. ಇದಕ್ಕೆ ಉದಾಹರಣೆಯನ್ನು ನೋಡೋಣ. ದೇವದತ್ತನೆಂಬ ಒಬ್ಬ  ಮನುಷ್ಯನಿದ್ದನು. ಅವನ ಮನೆ ಎಲ್ಲಿದೆ ಎಂದು ಕೇಳಿದರೆ `ಅಲ್ಲಿ, ಆ ಕಾಗೆ ಕುಳಿತಿದೆಯಲ್ಲಾ  ಅದೇ ದೇವದತ್ತನ ಮೆನೆ' ಎಂದು ಒಬ್ಬನು ಹೇಳಿದನು. ಆ ಮನುಷ್ಯನು ಆ ಮನೆಯ ಹತ್ತಿರ ಹೋಗಿ  ನೋಡಿದರೆ ಕಾಗೆ ಇಲ್ಲ. ಆದರೆ ಅದು ದೇವದತ್ತನ ಮನೆ ಹೌದೆ ಅಲ್ಲವೇ ಎಂದು ಕೇಳಿದರೆ ಅದು ದೇವದತ್ತನ ಮನೆಯೇ. ಈ ರೀತಿ ಲಕ್ಷಣವನ್ನು ಹೇಳುವುದಕ್ಕೆ `ತಟಸ್ಥ ಲಕ್ಷಣ'ವೆಂದು ಶಾಸ್ತ್ರದಲ್ಲಿ ಹೆಸರು ಕೊಡಲಾಗಿದೆ. ನಾವು ತಿಳಿದುಕೊಳ್ಳುವವರೆಗೆ ಆ  ಲಕ್ಷಣವಿರುವುದು, ಅದಕ್ಕಾಗಿಯೇ `ತಟಸ್ಥ ಲಕ್ಷಣ'ವೆಂದು ಹೆಸರು. ಆದರೆ ನಾವು ಆ ವಿಷಯವನ್ನು ತಿಳಿದುಕೊಂಡ ಮೇಲೆ ಆ ಲಕ್ಷಣ ಅಲ್ಲಿ ಇರಬೇಕೆಂಬುದಿಲ್ಲ. ಹೇಗೆ ಮೇಲೆ ಹೇಳಿದ ಉದಾಹರಣೆಯಲ್ಲಿ ಮನೆ ಕಂಡುಹಿಡಿದನಂತರವೂ ಕಾಗೆ ಅಲ್ಲಿರಬೇಕೆಂದಿಲ್ಲ. ಅದೇ ರೀತಿಯಾಗಿ ಇತರ ವಿಷಯಗಳಲ್ಲಿಯೂ ತಟಸ್ಥ ಲಕ್ಷಣವಿರುತ್ತದೆ. 

ಒಬ್ಬ ಪ್ರೇಷ್ಯ (ಕೆಲಸಗಾರ)ನನ್ನು ನಾವು ವರ್ಣನೆ ಮಾಡಬಹುದು. ಮತ್ತೊಬ್ಬನು ನೋಡುವಾಗಲೂ ಅವನು ಹಾಗೆಯೇ ಇರುವನು. ``ಅವನು ನಿನ್ನೆ  ನೋಡುವುದಕ್ಕೆ  ಕಪ್ಪಾಗಿದ್ದನು; ಇಂದು ಬೆಳ್ಳಗೆ ಆಗಿಬಿಟ್ಟಿದ್ದಾನೆ" ಎನ್ನುವುದೆಲ್ಲಾ ಇಲ್ಲ. ಹಾಗೆಯೇ ``ನಿನ್ನೆ ಅವನಿಗೆ ಎರಡು ಕಣ್ಣುಗಳಿದ್ದವು; ಇಂದು ನೋಡುವಾಗ ಒಂದೇ ಕಣ್ಣು ಆಗಿಬಿಟ್ಟದೆಯಲ್ಲಾ" ಎನ್ನುವುದೂ ಇಲ್ಲ. ಇಂಥ ಲಕ್ಷಣಗಳು ಯಾವಾಗಲೂ ಇರುವುವು. ಇವುಗಳಿಗೆ `ಸ್ವರೂಪ ಲಕ್ಷಣ'ವೆಂದು ಹೆಸರು.

ಈ ಎರಡು ಲಕ್ಷಣಗಳೂ ಒಂದು ವಸ್ತುವಿನ ಸ್ವರೂಪವನ್ನು ತಿಳಿಸುವುದಕ್ಕಾಗಿಯೇ ಇರುವುವು. ಆದ್ದರಿಂದಲೇ ಶಾಸ್ತ್ರದಲ್ಲಿ,

\begin{shloka}
(ಯಸ್ಮಾತ್ ವಿಶ್ವಮುದೇತಿ ಯತ್ರ ನಿವಸತಿ ಅನ್ತೇ ತದಪ್ಯೇತಿ")
\end{shloka}

(ಯಾವುದರಿಂದ ಪ್ರಪಂಚವೆಲ್ಲಾ ಹುಟ್ಟಿದೆಯೋ. ಎಲ್ಲಿ ಇದೆಯೋ, ಎಲ್ಲಿ ಲಯವಾಗುತ್ತದೆಯೋ) ಎಂದು ಹೇಳಿದೆ.

ಈ ಪ್ರಪಂಚವೆಲ್ಲಾ ಉಂಟಾಗಿ, ಇದ್ದು, ಅನಂತರ ನಶಿಸಿಹೋಗುವುದಕ್ಕೆ  ಕಾರಣವಾಗಿರುವುದು ಆ ಚೈತನ್ಯವೆಂದೇ ತಿಳಿದುಕೊಳ್ಳಬೇಕೆಂದು ಅಲ್ಲಿ ಹೇಳಿರುವುದು. ಅದನ್ನು ಮನಸ್ಸಿನಲ್ಲಿ ಇಟ್ಟುಕೊಂಡು ಯಾವ ಈಶ್ವರನಿಂದ ಈ ಪ್ರಪಂಚವೆಲ್ಲಾ ನಿರ್ಮಿತವಾಗಿದೆಯೋ, ಯಾವ ಈಶ್ವರನಿಂದ ಈ ಪ್ರಪಂಚವೆಲ್ಲಾ ಕಾಪಾಡಲ್ಪಡುವುದೋ, ಕೊನೆಗೆ ಯಾರಲ್ಲಿ ಇದೆಲ್ಲವೂ ಲಯವಾಗಿಬಿಡುವುದೋ ಅದನ್ನು  ಬ್ರಹ್ಮವೆಂದು ತಿಳಿದುಕೊಳ್ಳಬೇಕು. ಈ ಮೂರು ಒಂದೇ ಜಾಗದಲ್ಲಿ ಇರಬೇಕು. ಹೀಗಿಲ್ಲದೆ ಬೇರೆ ಬೇರೆ ವಸ್ತುಗಳಲ್ಲಿದ್ದರೆ ಪ್ರಪಂಚಸೃಷ್ಟಿಗೆ ಒಂದು ಕಾರಣ, ಅದನ್ನು ಕಾಪಾಡುವುದಕ್ಕೆ  ಬೇರೊಂದು ಕಾರಣ. ಹೀಗೆ ಎರಡು ಮೂರು ಕಾರಣಗಳಾಗಿ ಮತ್ತೆ ಅದ್ವೈತ ಮತವೇ-ಅದ್ವೈತ ತತ್ತ್ವವೇ ಇಲ್ಲದಂತಾಗುತ್ತದೆ. 

ಆದರೆ ನಾವು ಬ್ರಹ್ಮ, ವಿಷ್ಣು, ಮಹೇಶ್ವರನೆಂದು ಬೇರೆ ಬೇರೆಯಾಗಿ ಹೇಳುತ್ತೇವಲ್ಲಾ ಎಂದರೆ, ಒಂದು ಚೈತನ್ಯ ಹೀಗೆ ಹಲವು ವಿಧವಾದ ರೂಪಗಳನ್ನು ಪಡೆದುಕೊಳ್ಳವುದು.  ಆದ್ದರಿಂದ ಹೀಗೆ ಹೇಳುವುದರಿಂದ  ನಮ್ಮ ಅದ್ವೈತಕ್ಕೆ ಯಾವ ವಿರೋಧವೂ ಇಲ್ಲ. 

ಬ್ರಹ್ಮದಿಂದಲೇ ಪ್ರಪಂಚ ಉಂಟಾಗುವುದೆಂದು ಲಕ್ಷಣವನ್ನು ಹೇಳುವುದರಿಂದ  ಏನಾಗುತ್ತದೆಂದರೆ, ಹೇಗೆ ನಾವು ಮನೆಯನ್ನು ಗುರುತಿಸುವಾಗ  `ಕಾಗೆ ಇರುವ ಮನೆ' ಎಂದು ಹೇಳುತ್ತೇವೋ ಹಾಗೆಯೇ ಆಗುವುದು. ಆದ್ದರಿಂದಲೇ ನಾವು ಈ ಪ್ರಪಂಚವೆಲ್ಲಾ ಬ್ರಹ್ಮದಿಂದ ಕಲ್ಪಿತವಾದುದು ಎಂದು ಹೇಳಿದೆವು. ಯೋಚಿಸಿನೋಡಿದರೆ ಪ್ರಪಂಚವನ್ನು ಕುರಿತು ಮಾತನಾಡುವಾಗ ಮಾತ್ರ ಪ್ರಪಂಚವಿರುವುದೆಂದು ನಾವು ಹೇಳುವೆವು. ಆದರೆ ಬ್ರಹ್ಮಜ್ಞಾನವನ್ನು ಪಡೆದು ತನ್ನ ನಿಜ ಸ್ವರೂಪವೇನೆಂದು ತಿಳಿದುಕೊಂಡವರ ಹತ್ತಿರ  ಪ್ರಪಂಚವೆನ್ನುವುದು ಬೇಕೇ ಎಂದು ಕೇಳಿದರೆ ಅವರ ದೃಷ್ಟಿಯಲ್ಲಿ ಪ್ರಪಂಚವೆನ್ನುವುದೇ ಇಲ್ಲ.

\begin{shloka}
ಪರಬ್ರಹ್ಮವನ್ನು, ಕುರಿತು, \\
``ಸತ್ಯಜ್ಞಾನಸುಖಸ್ವರೂಪಮ್''\\
(ಸತ್ಯ-ಜ್ಞಾನ-ಸುಖ ಸ್ವರೂಪ)
\end{shloka}

-ಎಂದು ಹೇಳಲ್ಪಟ್ಟಿದೆ. ಇದೇ ಬ್ರಹ್ಮದ ಸ್ವರೂಪಲಕ್ಷಣವೆನ್ನುವುದು. ಸ್ವರೂಪ ಲಕ್ಷಣವೆಂದರೇನು? ಹೇಳುವ ವಸ್ತುವನ್ನು ಆ ಸ್ವರೂಪ ಲಕ್ಷಣವೆನ್ನುವುದು ಎಂದು ಬಿಡುವುದಿಲ್ಲ. ಸ್ವರೂಪ ಲಕ್ಷಣ, ವಸ್ತು ಇರುವವರೆಗೂ ಇದ್ದೇ ಇರುವುದು. ಬ್ರಹ್ಮವನ್ನು ಕುರಿತು ಹೇಳುವಾಗ ಸತ್ಯ, ಜ್ಞಾನ ಸುಖ ಎನ್ನುವುದು ಅದರ ಸ್ವರೂಪ ಲಕ್ಷಣವೆಂದು ಹೇಳಲಾಗುವುದು.

ಪ್ರಪಂಚದಲ್ಲಿ ಎಷ್ಟೋ ವಿಧವಾದ ಸತ್ಯಗಳು ಇವೆ. ಆದರೆ ಎಲ್ಲಾ ವಿಧವಾದ ಸತ್ಯಗಳಿಗೂ ಮೇಲಾಗಿರುವ ಸತ್ಯ ಒಂದಿದೆ. ಪರವಸ್ತು ಒಂದೇ ಮೇಲಾದ ಸತ್ಯವೆಂದು ಶಾಸ್ತ್ರಗಳು ಹೇಳುತ್ತವೆ.

ಒಬ್ಬನಿಗೆ ಹತ್ತು ವರ್ಷವೆಂದು ಯಾರಾದರೂ ಹೇಳಿದರೆ ಅವನಿಗಿಂತಲೂ ದೊಡ್ಡವನಾಗಿರುವ ಇಪ್ಪತ್ತು ವಯಸ್ಸಿನವನನ್ನು ನಾವು ತೋರಿಸಬಹುದು. ಅದಕ್ಕೆ ಮೇಲ್ಪಟ್ಟ ಮೂವತ್ತು ವರ್ಷದವನೂ ಇರುವನು. ಅದರೆ ಬ್ರಹ್ಮವೆನ್ನುವುದಕ್ಕೆ ಇಷ್ಟು  ವಯಸ್ಸೆಂದು ಬೇಗನೆ ಹೇಳಲು ಸಾಧ್ಯವಿಲ್ಲ. ನಾವು ವಯಸ್ಸನ್ನು ವಿಭಜಿಸಿ ಹೇಳುವಂತೆ ಸತ್ಯದಲ್ಲಿ ಕೊಡ ವಿಭಜನೆ ಇದೆ. ಈ ಬ್ರಹ್ಮ ಎನ್ನುವುದು ಉತ್ತಮವಾದ ಸತ್ಯ. 

 \begin{shloka}
 ``ತ್ರಿಕಾಲಾಬಾಧ್ಯತ್ವಂ ಸತ್ಯಂ"
 \end{shloka}

(ಮೂರು ಕಾಲಗಳಲ್ಲಿಯೂ ನಶಿಸಿದೆ ಇರುವುದೇ ಸತ್ಯ.) -ಎಂದು ಶಾಸ್ತ್ರ ಹೇಳುತ್ತದೆ. 


ಒಂದು ವಸ್ತು ಮೊದಲು ಇರಲಿಲ್ಲವೆನ್ನುವ ವಾಕ್ಯಕ್ಕೂ ಅವಕಾಶವಿರಬಾರದು. ಈಗ  ಇಲ್ಲ ಎನ್ನುವ ವಾಕ್ಯಾಕ್ಕೂ ಅವಕಾಶವಾಗಬಾರದು. ಮುಂದೆ ಇರುವುದಿಲ್ಲ ಎನ್ನುವ ವಾಕ್ಯಕ್ಕೂ ಅವಕಾಶವಾಗಬಾರದು. ಪ್ರಪಂಚದಲ್ಲಿ ಇಂಥ ಯೋಗ್ಯತೆಯುಳ್ಳ ವಸ್ತುಗಳನ್ನು ನಾವು ಕಾಣಲಾರೆವು. ಪ್ರಪಂಚದಲ್ಲಿ ನಾವು ಕಂಡ ಹಲವು ವಸ್ತುಗಳು ಈಗ ನಮ್ಮ ಹತ್ತಿರವಿಲ್ಲ. ಆದ್ದರಿಂದ ಮೂರು ಕಾಲಗಳಲ್ಲಿಯೂ ಇಲ್ಲ ಎನ್ನುವ ಮಾತಿಗೆ ಯಾವ ವಸ್ತು ಅವಕಾಶ ಕೊಡುವುದಿಲ್ಲವೋ ಆ ವಸ್ತುವೇ ಸತ್ಯವೆಂದು ಶಾಸ್ತ್ರದಲ್ಲಿ  ಹೇಳಿದೆ. ಚೈತನ್ಯ  ಮೊದಲು ಇರಲಿಲ್ಲ, ಎನ್ನುವುದಿಲ್ಲ. ಈಗ ಇಲ್ಲ ಎನ್ನುವುದೂ ಇಲ್ಲ. ನಾಳೆ ಇರುವುದಿಲ್ಲ ಎಂದೂ ಇಲ್ಲ. ಆದ್ದರಿಂದಲೇ-

\begin{shloka}
``ತ್ರಿಕಾಲಾಬಾಧ್ಯತ್ವಂ ಸತ್ಯಂ"
\end{shloka}

ಎಂದು ಹೇಳಿರುವುದು, ಇನ್ನೊಂದು ವಸ್ತುವನ್ನು ನೋಡಿ ಈ ಲಕ್ಷಣ ಹೇಳಿಲ್ಲ.

\begin{shloka}
``ಜಗತ್ಕಾರಣಂ ಬ್ರಹ್ಮ"
\end{shloka}

(ಪ್ರಪಂಚಕ್ಕೆ ಕಾರಣ ಪರವಸ್ತು)

ಎಂದು ಹೇಳುವ ಜಾಗದಲ್ಲಿ ಪ್ರಪಂಚವೆನ್ನುವುದನ್ನು ಮೂಲತಃ ಇಟ್ಟುಕೊಂಡು ಆ  ಲಕ್ಷಣವನ್ನು ಹೇಳಿರುವುದು.  ಪ್ರಪಂಚವನ್ನು ಇಟ್ಟುಕೊಂಡು ನಾವು ಒಂದು ಲಕ್ಷಣವನ್ನು ಹೇಳಿದರೆ ಪ್ರಪಂಚವಿರುವವರೆಗೆ ಆ ಲಕ್ಷಣವನ್ನು ಹೇಳಬಹುದು. ಪ್ರಪಂಚವಿಲ್ಲದಿದ್ದರೆ ಬ್ರಹ್ಮ ಪ್ರಪಂಚಕ್ಕೆ ಕಾರಣವೆಂದು ನಾವು ಹೇಳಲಾರೆವು. ಆದ್ದರಿಂದಲೇ-


\begin{shloka}
``ತ್ರೀಕಾಲಾಬಾಧ್ಯತ್ವಂ"
\end{shloka}

ಎನ್ನುವುದನ್ನು `ತಟಸ್ಥ ಲಕ್ಷಣ'ಕ್ಕೆ ನಾವು ಹೇಳಲಾರವು. ಆದರೆ ಮೂರು ಕಾಲಗಳಲ್ಲಿಯೂ ಇರಬಹುದಾದ ಲಕ್ಷಣ ಸ್ವರೂಪಲಕ್ಷಣ ಆಗುತ್ತದೆ. ಆದ್ದರಿಂದ ``ಆದ್ದರಿಂದ ಯಾವಾಗಲೂ'' ಬ್ರಹ್ಮ ಶಾಶ್ವತವಾಗಿರುವುದೆಂದೇ ತಾತ್ಪರ್ಯ. ``ಯಾವಾಗಲೂ'' ಎನ್ನುವ ಮಾತನ್ನು ಕಾಲಬೋಧಕವಾದ ನಮ್ಮ ಹುಟ್ಟನ್ನು ಮನಸ್ಸಿನ್ನಲ್ಲಿ ಇಟ್ಟುಕೊಂಡು ನಾವು ಸಾಧಾರಣವಾಗಿ ಉಪಯೋಗಿಸುವೆವು. ಆದರೆ ಬ್ರಹ್ಮಯಾವಾಗಲೂ ಇರುವುದು  ಎನ್ನುವಾಗ ಬ್ರಹ್ಮದ ಇರುವಿಕೆಗೆ ಕಾಲಮಿತಿ ಇಲ್ಲವೆಂದು ಅರ್ಥ. ಇದು ಭೂತಕಾಲ, ವರ್ತಮಾನ ಕಾಲ, ಭವಿಷ್ಯತ್ಕಾಲ ಈ ಮೂರು ಕಾಲಗಳಲ್ಲಿಯೂ ಇರುವಂತಹದು. ಒಂದು ವಸ್ತು ಇದ್ದರೂ, ಅದು ಇರುವುದು ಒಬ್ಬರಿಗೂ ತಿಳಿಯದೆ ಇದ್ದರೆ, ಆಗ ವಸ್ತು ಇದೆಯೆಂದು ಹೇಗೆ ಹೇಳುವುದು? ಈ ಪ್ರಪಂಚವೆಲ್ಲಾ ಯಾವಾಗ ಬ್ರಹ್ಮದಿಂದ ಕಲ್ಪಿತವಾಗಿರುವುದೋ, ಆಗ ಆ ಬ್ರಹ್ಮದ ಅರಿವು ಸೃಷ್ಟಿಮಾಡಿದವನಿಗಾದರೂ ಇದೆಯೋ ಇಲ್ಲವೋ? ಸೂರ್ಯನು ಸ್ವಯಂಪ್ರಕಾಶನಾಗಿದ್ದಾನೆ. ಸೂರ್ಯನನ್ನು ನೋಡಲು ನಮಗೆ ಬೇರೆ ಯಾವ ದೀಪವೂ ಬೇಕಾಗಿಲ್ಲ. ಬೇರೆ ಯಾವುದಾದರೂ ವಸ್ತುವನ್ನು ನೋಡಬೇಕಾದರೆ ನಮಗೆ ಸೂರ್ಯನ ಸಹಾಯ ಬೇಕು. ಒಬ್ಬನಿಗೆ ಕಣ್ಣುಗಳಿದ್ದರೆ ಸೂರ್ಯನನ್ನು ನೋಡಬಹುದು. ಅವುಗಳೇ ಬೇಕು. ಅದೇ ರೀತಿ ಈ ಪರಬ್ರಹ್ಮವೆನ್ನುವುದು ಯಾವಾಗಲೂ ಪ್ರಕಾಶಿಸುವಂತಹದು. ಇದು ಪ್ರಕಾಶವನ್ನು ಕೊಡತಕ್ಕದ್ದಾಗಿರುವುದರಿಂದ ಇದು ಜ್ಞಾನ ಸ್ವರೂಪ. ``ನಾವು ಇದ್ದರೆ ಅದೂ ಇದೆ.  ಅದನ್ನು ಕಾಣಲು ಬೇರೆ ಯಾವ ವಸ್ತುವೂ  ಬೇಕಾಗಿಲ್ಲ. ಆದ್ದರಿಂದ ಜ್ಞಾನ ಸ್ವರೂಪವಾದುದು ಎನ್ನುವ ಲಕ್ಷಣ ಸ್ವರೂಪಲಕ್ಷಣವಾಗುತ್ತದೆ. ತಟಸ್ಥ ಲಕ್ಷಣವಲ್ಲ.

ಅನಂತರ `ಸುಖಸ್ವರೂಪ' ವೆನ್ನುವುದು  ಹೇಳಲಾಗಿದೆ. ನಾವು ಪ್ರಪಂಚದಲ್ಲಿ ಯಾವಾಗಲೂ  ಸುಖವಾಗಿರಬೇಕೆಂದೇ ಆಶಿಸುತ್ತೇವೆ.

\begin{shloka}
``ಏತಸ್ಯೈವ ಆನಂದಸ್ಯ ಅನ್ಯಾನಿ ಭೂತಾನಿ ಮಾತ್ರಾಮುಪಜೀವನ್ತಿ''
\end{shloka}

(ಈ ಆನಂದದ ಒಂದು ತುಣುಕನ್ನು ಇಟ್ಟುಕೊಂಡು ಇತರ ಎಲ್ಲಾ ಭೂತಗಳೂ ಜೀವಿಸುತ್ತವೆ)

\begin{shloka}
``ಯಸ್ಮಿನ್  ನಿತ್ಯಸುಖಾಂಬುಧೌ ಗಲಿತಧೀಃ ಬ್ರಹ್ಮೈವ ಬ್ರಹ್ಮವಿತ್"
\end{shloka}

(ನಿತ್ಯವಾದ ಸುಖ ಸಮುದ್ರದಲ್ಲಿ ಮನೋಮಗ್ನನಾಗಿರುವವನು ಬ್ರಹ್ಮವನ್ನು ಅರಿತವನೇ ಅಲ್ಲ, ಅವನು ಬ್ರಹ್ಮವೇ ಆಗುವನು)


-ಎಂದೂ ಹೇಳಿರುವುದರಿಂದ ಬ್ರಹ್ಮವೆನ್ನುವುದು ಅಂಥ ಒಂದು ಆನಂದ ಸ್ವರೂಪವಾಗುತ್ತದೆ. ಆನಂದಸ್ವರೂಪಕ್ಕೆ ಎಂದೂ ಅಳಿವಿಲ್ಲ. ಆದ್ದರಿಂದ ಅದೂ ಸಹ ಪರಬ್ರಹ್ಮದ ಲಕ್ಷಣವೆಂದು ಹೇಳಲ್ಪಟ್ಟಿದೆ. ಮೇಲೆ ಹೇಳಿದಂತೆ ಮೂರು ಲಕ್ಷಣಗಳನ್ನೂ ಹೇಳಲಾಗಿದೆ. ಆದರೆ ಹೇಳುವಾಗ ಈ ಮೂರು ಒಂದೇ ಒಂದೆಂದು ಹೇಳುವರು ಬೇರೆ ಬೇರೆಯಾಗಿ ಹೇಳುವುದಿಲ್ಲ. ಸ್ವರೂಪವನ್ನು ತಿಳಿಯಲು ನಾವು ಈ ಮೂರನ್ನೂ ಇಷ್ಟು ವಿಶದವಾಗಿ ನೋಡಬೇಕಾಗಿದೆ. ಆದ್ದರಿಂದಲೇ ``ಸತ್ಯಜ್ಞಾನ ಸುಖ ಸ್ವರೂಪ'' ವೆಂದು ಹೇಳಿರುವುದು. 

\begin{shloka}
``ಅವಧಿ ದ್ವೈತಪ್ರಣಾಶೋಜ್ಝಿತಮ್''
\end{shloka}

(ಎಲ್ಲೆ, ದ್ವೈತ ಮತ್ತು ಅಳಿವು ಇಲ್ಲದುದು)

ಪ್ರಪಂಚದಲ್ಲಿ ನಾವು ಕಾಣವ ಪ್ರತಿಯೊಂದು ವಸ್ತುವಿಗೂ ಅವಧಿ (ಎಲ್ಲೆ)ಎನ್ನುವುದು ಉಂಟು. ಒಂದು ವಸ್ತುವನ್ನು ಕುರಿತು ಹೇಳುವಾಗ" ಎರಡು ಅಡಿ ಇದೆ. ನಾಲ್ಕು ಅಡಿ ಇದೆ" ಎಂದೆಲ್ಲಾ ಹೇಳುತ್ತೇವೆ. ಕಾಲದಿಂದ ಎಂದು ಎಲ್ಲೆಯನ್ನು ಪರವಸ್ತುವಿಗೆ ವಿಧಿಸಲು ಸಾಧ್ಯವಿಲ್ಲವೆನ್ನುವುದನ್ನು  ನಾವು ನೋಡಿದೆವು.  ದೇಶದಿಂದ ಅಂದರೆ ಜಾಗದಿಂದ ಕೂಡ ಪರವಸ್ತುವಿಗೆ ಯಾವ ಎಲ್ಲೆಯೂ ಇಲ್ಲ. ಏಕೆಂದರೆ ಪರವಸ್ತು ಎಲ್ಲಾ ಜಾಗದಲ್ಲೂ ವ್ಯಾಪಿಸಿರುತ್ತದೆ. ``ಒಂದು ವಸ್ತು ಈ ಜಾಗದಲ್ಲಿದೆ, ಇನ್ನೊಂದು ಜಾಗದಲ್ಲಿ ಇಲ್ಲ" ಎಂದರೆ ಆ ವಸ್ತುವಿಗೆ ಜಾಗದಿಂದ ಅಳಿವು ಇದೆ ಎಂದು ಹೇಳಿದಂತಾಯಿತು. ಆದರೆ ಪರವಸ್ತುವಿಗೆ ಯಾವ  ವಿಧವಾದ ಅಳಿವೂ ಇಲ್ಲ 

\begin{shloka}
``ಪಾದೋಽಸ್ಯ ವಿಶ್ವಾ  ಭೂತಾನಿ |\\
ತ್ರಿಪಾದಸ್ಯಾಮೃತಂ ದಿವಿ |"
\end{shloka}
 
 (ಪ್ರಪಂಚವೆಲ್ಲಾ ಪರವಸ್ತುವಿನ ಒಂದು ಭಾಗವೇ ಆಗುತ್ತದೆ. ಪರವಸ್ತುವಿನ ಮೂರು ಭಾಗವಾದರೋ ನಾಶವಾಗದೇ ಇರುವುದು....)
 
 
 -ಎಂದು ವೇದದಲ್ಲಿ ಹೇಳಲ್ಪಟ್ಟಿದೆ. ಅದೇನೆಂದರೆ ಈ ಪ್ರಪಂಚದಲ್ಲಿರುವುದೆಲ್ಲಾ ಬ್ರಹ್ಮದ ಒಂದು ತುಣುಕೇ ಆಗಿದೆಯಂದು ತಿಳಿದುಬೇಕು. ಆದ್ದರಿಂದ `ಜಾಗ' ಎನ್ನುವ ಅರ್ಥವೂ ಕೂಡ ನಾವು ಪ್ರಪಂಚವನ್ನು ನೋಡುವಾಗ ಮಾತ್ರ ಉಂಟಾಗುತ್ತದೆ.  ಪ್ರಪಂಚವಿಲ್ಲದ ಸಮಯದಲ್ಲಿ  ಅಂಥ ಅರ್ಥವೂ ಇಲ್ಲ.
 
 
 \begin{shloka}
 ``ನ ತಸ್ಯ ಪ್ರತಿಮಾಽಸ್ತಿ"
\end{shloka} 

-ಎನ್ನುವಂತೆ ಈ ಬ್ರಹ್ಮಕ್ಕೆ  ಪ್ರಪಂಚವೆನ್ನುವದನ್ನೇ ಹೇಳಲಾಗುವುದಿಲ್ಲ. ಅದು ದೊಡ್ಡ ವಸ್ತುವೆಂದು ನಾವು ಹೇಳಿದರೆ ``ಅದು ಎಷ್ಟು ದೊಡ್ಡದು" ಎಂದು ಕೇಳಬಾರದು. ಏಕೆಂದರೆ ಅದಕ್ಕಿಂತಲೂ ದೊಡ್ಡವಸ್ತು ಇಲ್ಲವೇ ಇಲ್ಲ, 

\begin{shloka}
``ಅವಧಿ ದ್ವೈತ ಪ್ರಣಾಶೋಜ್ಝತಮ್''
\end{shloka}

-ಎಂದು ಹೇಳಿರುವಲ್ಲಿ ದ್ವೈತವೆಂದರೆ ಏನು ಎನ್ನುವುದನ್ನು ಈಗ ನೋಡೋಣ. ಶಾಸ್ತ್ರದಲ್ಲ್ಲಿ `ಬ್ರಹ್ಮ' ಬೇರೊಂದು ವಸ್ತುವಿನಿಂದ ಉಂಟಾದುದಲ್ಲವೆಂದು ಹೇಳಿದೆ, ``ಇದು ಬ್ರಹ್ಮ, ಇದು ಬ್ರಹ್ಮ ಅಲ್ಲ" ಎಂದು ನಾವು ಯಾವ ವಸ್ತುವನ್ನೂ ನೋಡಿ ಹೇಳಲಾರೆವು. 


ಬ್ರಹ್ಮದ ಸತ್ಯವೇ ಪ್ರಪಂಚದಲ್ಲಿ ದೊಡ್ಡ ಸತ್ಯವೆಂದು ನಾವು ಮೊದಲೇ ಹೇಳಿದೆವು. ಆದ್ದರಿಂದ ಬ್ರಹ್ಮ ಸತ್ಯವನ್ನು ಬಿಟ್ಟು ಬೇರೆ ಯಾವ ಸತ್ಯವೂ ಇಲ್ಲದಿರುವಾಗ ಪ್ರಪಂಚದಲ್ಲಿ ಪರವಸ್ತುವನ್ನು ಬಿಟ್ಟು ಬೇರೆ ಯಾವ ವಸ್ತುವನ್ನೂ ತೋರಿಸಲು ಸಾಧ್ಯವಿಲ್ಲ. ಎರಡನೆಯ ವಸ್ತು ಎನ್ನುವುದು ಬೇರೆಯಾಗಿ ಇದ್ದರೆ ಆಗ ದ್ವೈತ ಉಂಟು. ಆದರೆ ನಾವು ಪ್ರಪಂಚವನ್ನು ನೋಡಿದ್ದೇವಲ್ಲಾ (ಅದು ಎರಡನೆಯ ವಸ್ತುವಲ್ಲವೇ?) ಎಂದು ಯಾರಾದರೂ ಕೇಳಿದರೆ ಅವರಿಗೆ ಒಂದು ಉದಾಹರಣೆಯನ್ನು ಕೊಡಬಹುದು. 

ಚಿನ್ನ ಎನ್ನುವ ಲೋಹವನ್ನು ನಾವು ಉಂಗುರ ರೂಪದಲ್ಲಿ ಮಾರ್ಪಡಿಸಿದರೂ ಅದು ಚಿನ್ನವೇ ತಾನೇ! ಚಿನ್ನವೇ ಉಂಗುರ ರೂಪದಲ್ಲಿ ನಮ್ಮ ಮುಂದೆ ಕಾಣುತ್ತದೆ. ಉಂಗುರದ ವಿಷಯವಾಗಿ ಹೇಳಿದ ಉದಾಹರಣೆಯಲ್ಲಿ ಪರಿಣಾಮವೆನ್ನುವುದು ಇದೆ. ಅಂದರೆ ಬ್ರಹ್ಮದ ವಿಷಯ ಹಾಗಲ್ಲ, ಪರಿಣಾಮವೆನ್ನುವುದು ಇಲ್ಲ. ಇದನ್ನು ವಿವರ್ತ(ತೋರಿವಿಕೆ)ಎನ್ನುತ್ತಾರೆ. ಕೆಲವು ವಸ್ತುಗಳು ನಮ್ಮ ಕಣ್ಣಿಗೆ ಬೀಳುತ್ತವೆ. ಆದರೆ ಅವು ನಿಜವಾಗಿಯೂ ಇರುವುದಿಲ್ಲ. ಯಾವುದಾದರೂ ಮೇಘವನ್ನು ನಾವು ದೂರದಿಂದ  ನೋಡಿದರೆ ಅದನ್ನು ನೋಡುವಾಗ ನಮ್ಮ ಕಣ್ಣುಗಳಿಗೆ ಮಾತ್ರ ಒಂದು ಇರುವುದಾಗಿ ತೋರುವುದು.

ಮೋಹವೆನ್ನುವ ಕತ್ತಲೆಯಲ್ಲಿ ನಮ್ಮ ಕಣ್ಣುಗಳಿಗೆ ಒಂದು ವಸ್ತು ಬೇರೆ ವಿಧವಾಗಿ ತೋರುವುದು. ಆಗ ಕೆಲವು ವಸ್ತುಗಳನ್ನು ಕಂಡು ನಾವು ಭಯಪಡುವುದೂ ಉಂಟು, ಆದರೆ ಅಲ್ಲಿ ವಸ್ತುವೆನ್ನುವುದೇ ನಿಜಕ್ಕೂ ಇಲ್ಲ. ಆದ್ದರಿಂದ ಅದನ್ನು ನೋಡಿ ಓಡಿಹೋಗಬೇಕಾಗಿಲ್ಲ. ಆದರೆ ಅಂಥ ಜ್ಞಾನ ನಮಗೆ ಇಲ್ಲ. ಬೌದ್ಧರು ``ಜ್ಞಾನವಿದ್ದರೆ  ಸಾಕಲ್ಲವೇ, ವಸ್ತು ಏತಕ್ಕೆ" ಎಂದು ಕೇಳುತ್ತಾರೆ, ವಸ್ತುವಿಲ್ಲದೆ ಜ್ಞಾನವಿಲ್ಲವೆನ್ನುವ ಸಿದ್ಧಾಂತ ನಮ್ಮ ಶಾಸ್ತ್ರಗಳಲ್ಲಿದೆ. ಆದರೆ ಒಂದು ಹಗ್ಗವನ್ನು ಕಂಡು ನಾವು ಹಾವು ಎಂದು ಭ್ರಮೆಗೆ ಒಳಗಾಗುತ್ತೇವಲ್ಲಾ ಎಂದರೆ ನಮಗೆ ಸಂಬಂಧಪಟ್ಟಂತೆ ಆಜ್ಞಾನವೆನ್ನುವ ಹಾವು ಅಲ್ಲಿದೆ. ಅಜ್ಞಾನದಿಂದ ತೋರಿದ ಆ ಹಾವು ಆಜ್ಞಾನ ದೂರವಾಗುತ್ತಲೇ ಹೊರಟುಹೋಗುತ್ತದೆ. ಹೇಗೆ ನಾವು ದೂರದಿಂದ ಮೇಘವನ್ನು ನೋಡುವಾಗ ಹಸು ಇರುವಂತೆ ತೋರಿ ಹತ್ತಿರಕ್ಕ ಹೋದಾಗ ಅದು ಇಲ್ಲದೆ ಹೋಗುವುದೋ ಹಾಗೆಯೇ ಅಜ್ಞಾನದಿಂದ ಉಂಟಾಗುವ ಹಾವಿನ ವಿಷಯ ಕೂಡ. 


ಒಬ್ಬ ಮನುಷ್ಯನನ್ನು ಅವನ ಮಗ `ಅಪ್ಪಾ' ಎಂದು ಕರೆಯುತ್ತಾನೆ. ಅವನ ಅಳಿಯ `ಮಾವ'  ಎಂದು ಕರೆಯುತ್ತಾನೆ. ಅದರೆ ಇರುವವನು ಒಬ್ಬನೇ. ಅದೇ ರೀತಿ ಒಂದೇ ವಸ್ತು ಹಲವು ವಿಧವಾಗಿ ತೋರುತ್ತಿದೆಯೆಂದು ನಾವು ಹೇಳಬಹುದು. ಬ್ರಹ್ಮವನ್ನು ಬಿಟ್ಟು ಯಾವ ವಸ್ತು ಇದ್ದರೂ. ಬ್ರಹ್ಮ ಬೇರೆ ವಸ್ತುವಿನ ಆಧೀನದಲ್ಲಿದೆ ಎಂದಾಗುವುದು. ಆದರೆ ಸ್ಥಿತಿ ಹಾಗಿಲ್ಲ. ಬ್ರಹ್ಮವನ್ನು ಬಿಟ್ಟು  ಯಾವ ವಸ್ತುವೂ ಇಲ್ಲ. ಆದ್ದರಿಂದ ದ್ವೈತವೆನ್ನುವುದೇ ಇಲ್ಲ. 

\begin{shloka}
``ಪ್ರಣಾಶೋಜ್ಝಿತಮ್"
\end{shloka}

ಎಂದು ಹೇಳಿರುವುದನ್ನು ಈಗ ನೋಡೋಣ.

ಬ್ರಹ್ಮಕ್ಕೆ ನಾಶವೆನ್ನುವುದೂ ಇಲ್ಲ. ಆದ್ದರಿಂದ ಬ್ರಹ್ಮಕ್ಕೆ ಹೇಳಿದ ಸ್ವರೂಪ ಲಕ್ಷಣಗಳು, ಯಾವಾಗಲೂ ಬ್ರಹ್ಮಾ ಒಂದೇ ವಿಧವಾಗಿರುತ್ತದೆಂದು ತೋರಿಸುವುದು. 

``ಈಶ್ವರನೂ ಇರಬಹುದು; ಬ್ರಹ್ಮನೂ ಇರಬಹುದು; ಆದರೆ  ಪರಬ್ರಹ್ಮದ ಸ್ವರೂಪವನ್ನು ತಿಳಿದುಕೊಂಡು ನಮಗೆ ಆಗಬೇಕಾದುದಾದರೂ ಏನು? ನಮ್ಮ ಸ್ವರೂಪ ತಾನೇ ನಮಗೆ ತಿಳಿಯಬೇಕು. ಆದ್ದರಿಂದ ನಮ್ಮ ಸ್ವರೂಪವೇನು" ಎನ್ನುವ ಪ್ರಶ್ನೆ ಉಂಟಾಗುತ್ತದೆ. 

ಮನುಷ್ಯನು ಹಲವು ವಿಧವಾದ ವೇಷಗಳನ್ನು ಧರಿಸುತ್ತಿದ್ದಾನೆ. ಒಂದು ಕಾಲದಲ್ಲಿ  ಅವನು(ಜಾಗತಸ್ಥಿತಿಯಲ್ಲಿ ) ಎಲ್ಲವನ್ನೂ ನೋಡುತ್ತಿದ್ದಾನೆ. ಬೇರೊಂದು ಕಾಲದಲ್ಲಿ (ಸ್ವಪ್ನಸ್ಥಿತಿಯಲ್ಲಿ) ಹಲವು ವಿಧವಾದ ಸ್ವಪ್ನಗಳನ್ನು ನೋಡುತ್ತಿದ್ದಾನೆ. ಇನ್ನೊಂದು ಕಾಲದಲ್ಲಿ (ಸುಷುಪ್ತಿಯಲ್ಲಿ) ಅವನು ನೋಡುತ್ತಿರುವುದೂ ಇಲ್ಲ; ಕಲ್ಪನೆ (ಸ್ವಪ್ನ)ಗಳನ್ನು  ಮಾಡುತ್ತಲೂ ಇಲ್ಲ. ಆಗ ನಾನು `ಇದ್ದೇನೆ' ಎನ್ನುವುದು ಮಾತ್ರ ಇರುವುದು. ಸುಷುಪ್ತಿಸ್ಥಿತಿ ಮುಗಿಯುತ್ತಲೇ; `ಇದ್ದೇನೆ' ಎನ್ನುವುದು ಮಾತ್ರ ಇರುವುದು. ಸುಷಪ್ತಸ್ಥಿತಿ ಮುಗಿಯುತ್ತಲೇ; `ನಾನು ಸುಖವಾಗಿ ನಿದ್ದೆ ಮಾಡಿದೆನು' ಎಂದೇ ಅವನು  ಯೋಚಿಸುತ್ತಾನೆ, ಹಿಂದೆ ಅವನು ಹಾಗೆ ಇದ್ದರೆ ಮಾತ್ರ ಈಗ ಆ ರೀತಿ ಹೇಳಲು ಸಾಧ್ಯವಾಗುತ್ತದೆ. ಹಿಂದೆ ಏನೂ ಇಲ್ಲದುದನ್ನು ಈಗ ಅವನು ಹೊಸದಾಗಿ ಹೇಳಲು ಸಾಧ್ಯವಿಲ್ಲ. ಆದ್ದರಿಂದ ಅವನು ಜಾಗೃತ್ ಸ್ಥಿತಿಯಲ್ಲಿ ಎಷ್ಟೋ ವಸ್ತುಗಳನ್ನೆಲ್ಲಾ ನೋಡಿ ಅವುಗಳನ್ನು ಅನುಭವಿಸುತ್ತಿದ್ದಾನೆ. ಕನಸಿನಲ್ಲೂ ಅದೇ ರೀತಿ ವಸ್ತುಗಳನ್ನು ಕಲ್ಪಿಸಿಕೊಳ್ಳುತಿದ್ದಾನೆ. ಸುಷುಪ್ತಿಸ್ಥಿತಿಯನ್ನು ಚೆನ್ನಾಗಿ ಅನುಭವಿಸುತ್ತಾನೆ. ಅವನಿಗೆ ತನ್ನ ಬಗ್ಗೆ ಅರಿವು ಆಗ ಇದೆಯೇ ಎನ್ನುವುದನ್ನು ಈಗ ನೋಡೋಣ. ಜಾಗೃತ್ ಸ್ಥಿತಿಯಲ್ಲಿರುವಾಗ `ನೀನು ಇದ್ದೀಯಾ?' ಎಂದು ಯಾರಾದರೂ ಕೇಳಿದರೆ, `ನಾನು ಇದ್ದೇನೆ' ಎಂದು ಹೇಳುವನು. `ನಾನು ಎಲ್ಲವನ್ನೂ ನೋಡುತ್ತಿದ್ದೇನೆ. ಆದ್ದರಿಂದ ನಾನು ಇದ್ದೇನೆ' ಎಂದು ಅವನು ಹೇಳುವನು. ಕನಸಿನಲ್ಲೂ ಹಾಗೆಯೇ `ನಾನು ಹಲವು ವಿಧವಾದ ವಸ್ತುಗಳನ್ನು ನೋಡುತ್ತಿದ್ದೇನೆ. ಎನ್ನುವುದರಿಂದ ನಾನು ಆಗಲೂ ಇದ್ದೇನೆ' ಎಂದೇ ಅವನು ಹೇಳುವನು. ನಿದ್ದೆ ಮಾಡುತ್ತಿದ್ದೇನೆ ಎನ್ನುವುದು ಆಗ ಅವನಿಗೆ ತಿಳಿಯದೇ ಇದ್ದರೂ, ನಿದ್ದೆ ಆದನಂತರ `ನಾನು ಸುಖವಾಗಿ ನಿದ್ದೆ ಮಾಡಿದೆ' ಎನ್ನುವ ಅರಿವು ಅವನಿಗೆ ಇರುವುದು. ಹಾಗೆ ನಿದ್ದೆ ಮಾಡುವ ಸಮಯದಲ್ಲಿ ಅನುಭವಿಸುವವನು ಒಬ್ಬನು ಇಲ್ಲದೇ ಇದ್ದರೆ, ಅನುಭವಿಸಿದುದನ್ನೂ ಅವನು ಅನುಭವಿಸದೇ ಇದ್ದರೆ, ಸುಖವಾಗಿ ನಿದ್ದೆ ಮಾಡಿದುದನ್ನು ಕುರಿತು ಜಾಗ್ರತ್ ಸ್ಥಿತಿಯಲ್ಲಿ ಅವನು ಹೇಳಲು ಸಾಧ್ಯವಿಲ್ಲ. ಆದ್ದರಿಂದ ಸುಷುಪ್ತಿಸ್ಥಿತಿಯಲ್ಲಿಯೂ ಕೂಡ ಅನುಭವವೆನ್ನುವುದು ಇದೆ. ಆದರೆ ಅಜ್ಞಾನ ಬಹಳ ತೀವ್ರವಾಗಿರುವುದರಿಂದ `ನಾನು ಈಗ ಅನುಭವಿಸಿಕೊಂಡಿದ್ದೇನೆ' ಎನ್ನುವುದು ಅವನಿಗೆ ನಿದ್ದೆಮಾಡುವಾಗ ತಿಳಿಯಲಿಲ್ಲ.

ಆದ್ದರಿಂದ ಜಾಗ್ರತ್, ಸ್ವಪ್ನ, ಸುಷುಪ್ತಿಗಳನ್ನು ಅನುಭವಿಸಿಕೊಂಡಿರುವುದು ಜೀವನ ಸ್ವರೂಪ. ಹಾಗಾದರೆ ಜೀವನೆನ್ನುವವನು ಯಾರು? ಇವನು ಒಬ್ಬನೇ ಇರುವಾಗ ಯಾವುದಾದರೂ ದೇವರನ್ನು ಉಪಾಸನೆ ಮಾಡಿದರೆ ಅವನು ಸ್ವರ್ಗವೋ, ನರಕವೋ, ಸುಖವೋ, ದುಃಖವೋ ಪಡೆಯುವಂತಾಗುವುದು. ಆದರೆ ಇವನು (ಜೀವನು)-

\begin{shloka}
`ಯ ಜ್ಜಾಗ್ರತ್ ಸ್ವಪ್ನಪ್ರಸುಪ್ತಿಷು ವಿಭಾತ್ಯೇಕಂ ವಿಶೋಕಂ ಪರಮ್'
\end{shloka}

(ಯಾವುದು ಜಾಗ್ರತ್, ಸ್ವಪ್ನ, ಸುಷುಪ್ತಿ ಸ್ಥಿತಿಗಳಲ್ಲಿ ಒಂದಾಗಿಯೂ, ಶೋಕರಹಿತವಾಗಿಯೂ, ಪರವಾಗಿಯೂ, ಪ್ರಕಾಶಿಸಿಕೊಂಡಿರುವುದೋ)-ಎಂದು ಹೇಳಿದಂತೆ ದುಃಖವಿಲ್ಲದವನಾಗಿದ್ದಾನೆ. ಏಕೆ ಹಾಗೆ ಎಂದು ಕೇಳಿದರೆ, ಜೀವನ ನಿಜವಾದ ಸ್ವರೂಪ ಜಾಗ್ರತ್ ಸ್ಥಿತಿಯೇ, ಸ್ವಪ್ನ ಸ್ಥಿತಿಯೇ ಅಥವಾ ಸುಷುಪ್ತಿ ಸ್ಥಿತಿಯೇ-ಯಾವುದು ಎನ್ನುವುದನ್ನು ನೋಡಬೇಕು.

ಬ್ರಹ್ಮವನ್ನು ನಾವು `ಸತ್ಯಜ್ಞಾನ ಸುಖಸ್ವರೂಪ' ಎಂದು ಕರೆದೆವು. ಅದು ಯಾವಾಗಲೂ ಸತ್ಯ; ಯಾವಾಗಲೂ ಸುಖವಾಗಿರುವುದು; ಯಾವಾಗಲೂ ಜ್ಞಾನಯುಕ್ತವಾಗಿಯೇ ಇರುವುದು. ಆದ್ದರಿಂದ ಬ್ರಹ್ಮಕ್ಕೆ ಜಾಗ್ರತ್ ಸ್ಥಿತಿ ಇದೆಯೆಂದರೆ ಅದರ ಲಕ್ಷಣವೇ ಜಾಗ್ರತ್ ಸ್ಥಿತಿಯೆಂದು ನಾವು ಹೇಳಿದಂತಾಗುವುದು, ಆಗ ಅದಕ್ಕೆ ಸ್ವಪ್ನ, ಸುಷುಪ್ತಿಗಳು ಇಲ್ಲವೆಂದು ಹೇಳಬೇಕು.

ಆದರೆ ಜಾಗ್ರತ್, ಸ್ವಪ್ನ, ಸುಷುಪ್ತಿ ಎನ್ನುವ ಸ್ಥಿತಿಗಳು ಜೀವನಿಗೆ ಬಂದು ಹೋಗುತ್ತಿವೆ. ಆದ್ದರಿಂದ ಯಾವುವು ಬಂದು ಹೋಗುತ್ತಿವೆಯೋ ಅವುಗಳಲ್ಲಿ ಒಂದನ್ನೂ ಸಹ ನಾವು ಬ್ರಹ್ಮದ ಸ್ವರೂಪವೆಂದು ಹೇಳಲಾರೆವು, ಈ ಸ್ಥಿತಿಗಳಿಗೆ ಜೀವನು ಸಾಕ್ಷಿಯಾಗಿ ಬೆಳಗುತ್ತಿದ್ದಾನೆ. ಬಂದು ಬಂದು ಹೋಗುವುದೆಲ್ಲವೂ ಸುಳ್ಳೇ. ಸಾಕ್ಷಿಯಾಗಿ ಬೆಳಗುತ್ತಿರುವ ಜೀವನ ಸ್ವರೂಪ ದುಃಖ ಸ್ವರೂಪವೋ, ಸುಖ ಸ್ವರೂಪವೋ ಎಂದು ಕೇಳಿದರೆ `ದುಃಖಸ್ವರೂಪ' ವೆಂದು ನಾವು ಹೇಳುವಂತಿಲ್ಲ. ಇವನಿಗೆ ದುಃಖವೆನ್ನುವುದು ಯಾವಾಗ ಬರುವುದು? ಹೊರಗಿನ ವಸ್ತುಗಳೊಡನೆ ಜೀವನು ಸಂಬಂಧವನ್ನು ಇಟ್ಟುಕೊಂಡಾಗ ಮಾತ್ರ ದುಃಖ ಉಂಟಾಗುವುದು. ಹೊರಗಿನ ವಸ್ತುಗಳೊಡನೆ ಸಂಬಂಧವನ್ನು ಇಟ್ಟುಕೊಳ್ಳದೆ ಇರುವಾಗ ಇವನಿಗೆ ದುಃಖವೆನ್ನುವುದು ಉಂಟಾಗುವುದೇ ಇಲ್ಲ. ಆದ್ದರಿಂದ ದುಃಖವೆನ್ನುವುದು ಯಾವಾಗಲೂ ಇರುವುದೆಂದು ಹೇಳುವಂತಿಲ್ಲ.

ಜೀವನು ಸುಖಸ್ವರೂಪನೇ ಎನ್ನುವುದನ್ನು ಈಗ ನೋಡೋಣ. ಪಾಯಸವನ್ನು ಕುಡಿದರೆ ಒಬ್ಬನಿಗೆ ಸುಖ ಉಂಟಾಗುವುದು ಅಥವಾ ಬೇರೆ ಯಾವುದಾದರು ವಸ್ತುವನ್ನು ತಿಂದರೆ ಸುಖವಾಗಬಹುದು. ಆದ್ದರಿಂದ ಸುಖವೆನ್ನುವುದು ಕೂಡ ವಿಷಯಗಳಿಂದ ಉಂಟಾಗುವುದು ಎಂದು ತಿಳಿದಂತಾಯಿತು. ಇನ್ನು ಸುಖವೆನ್ನುವುದು ಒಬ್ಬನಿಗೆ ವಿಷಯಗಳಿದ್ದರೇನೇ ಉಂಟಾಗುತ್ತದೆಯೇ, ಅವುಗಳು ಇಲ್ಲದೇ ಇದ್ದರೂ ಉಂಟಾಗುತ್ತದೆಯೇ ಎನ್ನುವುದನ್ನು ಕುರಿತು ನಾವು ವಿಚಾರ ಮಾಡಬೇಕು. ಪಾಯಸವನ್ನು ಕುಡಿದರೆ ಒಬ್ಬನಿಗೆ ಸುಖ ಉಂಟಾಗುತ್ತದೆ, ಎಂದು ನಾವು ನೋಡಿದೆವು. ಆದರೆ ಒಬ್ಬನು ನಿದ್ದೆ ಮಾಡುತ್ತಾನಲ್ಲಾ, ಆಗ ಅವನಿಗೆ ಸುಖವಿದ್ದಿತೇ ಎನ್ನುವುದನ್ನು ನೋಡಬೇಕು. ಆಗ ಸುಖ ಉಂಟೇ ಇಲ್ಲವೇ ಎಂದರೆ,

\begin{shloka}
`ನಿರ್ವಿಷಯಃ ಯಃ ಅನುಭವಃ ಶ್ರೂಯತೇ ಸುಖಸ್ಯ'
\end{shloka}

(ವಿಷಯವಿಲ್ಲದ ಸುಖದ ಅನುಭವ ಯಾವುದು ಇದೆಯೋ)-ಎಂದು ಹೇಳಿರುವಂತೆ ಸುಷುಪ್ತಿಯ ಸಮಯದಲ್ಲಿ ಯಾವುದೊಂದು ವಿಷಯದ ಸಂಬಂಧದಿಂದಲೂ ನಮಗೆ ಸುಖ ಉಂಟಾಗಲಿಲ್ಲವೆನ್ನುವುದು ಸ್ಪಷ್ಟವಾಗಿದೆ. ಹೊರ ವಿಷಯಗಳೇ ಇಲ್ಲದ ಸಮಯದಲ್ಲಿ ಒಬ್ಬನಿಗೆ ಸುಖ ಉಂಟಾಯಿತೆಂದರೆ, ಆಗ ಅವನ ಸ್ವರೂಪ ಮಾತ್ರವೇ ಇದ್ದಿತೆಂದು ತಾತ್ಪರ್ಯ. ಚಿನ್ನದಲ್ಲಿ ಹಲವು ಧಾತುಗಳು ಕಲೆತಿರುತ್ತವೆ. ಚಿನ್ನವನ್ನು ಶುದ್ಧಿಮಾಡಿ ಅವುಗಳೆಲ್ಲವನ್ನೂ ಬೇರ್ಪಡಿಸಿಬಿಟ್ಟರೆ, ಆಗ ಶುದ್ಧವಾದ ಚಿನ್ನ ಮಾತ್ರ ಇರುವುದಲ್ಲವೇ? ಶುದ್ಧವಾದ ಚಿನ್ನದ ಸ್ವರೂಪ ನಮಗೆ ಇತರ ಧಾತುಗಳನ್ನು ತೆಗೆದ ಮೇಲೆ ತಾನೇ ತಿಳಿಯುವುದು? ಅದೇ ರೀತಿ ಸುಷುಪ್ತಿ ಸಮಯದಲ್ಲಿ ಜೀವನ ಸ್ವರೂಪ ಆನಂದವೆಂದು ತಿಳಿಯಬೇಕು. ಅದನ್ನು ನಾವೇ ತಿಳಿದುಕೊಂಡರೆ ಅದಕ್ಕೆ ಶಾಸ್ತ್ರವೇಕೆ ಬೇಕು? ಅಜ್ಞಾನವನ್ನು ನಿವಾರಿಸಲು ಶಾಸ್ತ್ರವೆನ್ನುವುದು ಬೇಕು. ಯಾವ ಅಜ್ಞಾನದಿಂದಾಗಿ ಜಾಗ್ರತ್, ಸ್ವಪ್ನ, ಸುಷುಪ್ತಿ ಎನ್ನುವ ಮೂರು ಅವಸ್ಥೆಗಳನ್ನೂ ನಾವು ಕಾಣುತ್ತಿದ್ದೇವೋ ಆ ಅಜ್ಞಾನವನ್ನು ನಿವಾರಿಸುವುದಕ್ಕಾಗಿಯೇ ನಾವು ಶಾಸ್ತ್ರವನ್ನು ಓದಬೇಕು. ನಮ್ಮ ಸ್ವರೂಪವನ್ನು ಕುರಿತು ತಿಳುವಳಿಕೆ ನಮಗೆ ಇಲ್ಲವೇ ಇಲ್ಲ ಎಂದೇನಲ್ಲ. ಆದರೆ ನಮ್ಮ ಬಗ್ಗೆ ನಮಗೆ ತಿಳುವಳಿಕೆ ಯಾವಾಗ ಇದೆ ಎಂದು ಕೇಳಿದರೆ, ಆ ಅಜ್ಞಾನ ಸೇರಿಕೊಂಡಿರುವ ಸಮಯದಲ್ಲೂ ಇದ್ದೇ ಇದೆ. ಈ ಜಾಗ್ರತ್, ಸ್ವಪ್ನ, ಸುಷುಪ್ತಿ ಇವೆಲ್ಲವೂ ಒಬ್ಬನಿಗೆ ಯಾವುದರಿಂದ ಬಂದಿವೆ?

\begin{shloka}
`ದುಃಖಜನ್ಮ ಪ್ರವೃತ್ತಿದೋಷಮಿಥ್ಯಾಜ್ಞಾನಾನಾಂ\\
ತದುತ್ತರೋತ್ತರಾಪಾಯೇ ತದನನ್ತರಾಪಾಯಾದಪವರ್ಗಃ'
\end{shloka}

-ಎಂದು ನ್ಯಾಯಶಾಸ್ತ್ರದಲ್ಲಿ ಬರುವುದು.

ದುಃಖ ಎನ್ನುವುದು ಯಾವುದರಿಂದ ಬಂದಿತು? ಶರೀರ ಬಂದಿರುವುದರಿಂದ ದುಃಖ ಬಂದಿತು. ಶರೀರವೇ ಇಲ್ಲದಿದ್ದರೆ ದುಃಖವೇ ಇಲ್ಲ. ಏಕೆಂದರೆ ಆತ್ಮನೆನ್ನುವವನು ನಿತ್ಯನು. ಅವನಿಗೆ ಹೊರಗಿನ ವಸ್ತುಗಳಿಂದ ಉಂಟಾಗುವ ಸಂಬಂಧವೇ ಇಲ್ಲ. ವಿಷಯಗಳಿಂದ ಉಂಟಾಗುವ ಸಂಬಂಧ ಈ ಶರೀರವಿದ್ದರೇನೇ ಬರುವುದು.

ಜನ್ಮ ಯಾವುದರಿಂದ ಬಂದಿತು ಎಂದು ಕೇಳಿದರೆ, `ಪಂಚದಶಿ'ಯಲ್ಲಿ-

\begin{shloka}
`ಕುರ್ವತೇ ಕರ್ಮ ಭೋಗಾಯ ಕರ್ಮ ಕರ್ತುಂ ಚ ಭುಂಜತೇ'
\end{shloka}

(ಭೋಗಕ್ಕಾಗಿ ಕರ್ಮವನ್ನು ಮಾಡುತ್ತಾರೆ. ಕರ್ಮಕ್ಕಾಗಿ ಭೋಗವನ್ನು ಅನುಭವಿಸುವರು.)

-ಎಂದು ಹೇಳಿದಂತೆ ನಾವು ಮಾಡುವ ಕರ್ಮಗಳಿಗೆ ಅನುಗುಣವಾಗಿ ನಮಗೆ ಜನ್ಮ ದೊರೆಯುತ್ತದೆ.

ಆದ್ದರಿಂದ ಜನ್ಮವೆನ್ನುವುದು ಕರ್ಮದಿಂದ ಬಂದಿದೆ. ಈ ಕರ್ಮ ಯಾವುದರಿಂದ ಬಂದಿತು ಎನ್ನುವುದಕ್ಕೆ `ದೋಷದಿಂದ' ಎಂದು ಸೂತ್ರದಲ್ಲಿ ಹೇಳಲ್ಪಟ್ಟಿದೆ. ದೋಷವೆಂದರೆ ಏನು?

\begin{shloka}
`ರಾಗ ದ್ವೇಷಮೋಹಾ ದೋಷಾಃ'
\end{shloka}

-ಎಂದು ಹೇಳಲಾಗಿದೆ. ರಾಗ, ದ್ವೇಷ, ಮೋಹ-ಈ ಮೂರೂ ದೋಷಗಳು. ಕೆಲವು ವಸ್ತುಗಳನ್ನು ನೋಡಿದರೆ ನಮಗೆ ಅನಿಚ್ಛೆ ಉಂಟಾಗುತ್ತದೆ. ನಾವು ಮನೆಯಲ್ಲಿ ಸಾಕಿಕೊಂಡಿರುವ ಒಂದು ಚಿಕ್ಕ ನಾಯಿಮರಿಯನ್ನು ನೋಡಿದರೆ ನಮಗೆ ಮನಸ್ಸಿನಲ್ಲಿ ಸಂತೋಷ ಉಂಟಾಗುತ್ತದೆ. ಏಕೆಂದರೆ ಆ ನಾಯಿ ಕಡಿಯುವುದಿಲ್ಲ; ಹಾವು ಕಡಿಯುವುದೆನ್ನುವ ಭಯ. ಆದ್ದರಿಂದ ಒಂದು ವಸ್ತು ನಮಗೆ ಬೇಕಾಗಿದೆ, ಮತ್ತೊಂದು ವಸ್ತು ಬೇಡವೆಂದಾಗಿದೆ. ಹೀಗಿರುವ ರಾಗವೂ, ದ್ವೇಷವೂ ಮನುಷ್ಯನಿಗೆ ಇದೆ. ಈ ರಾಗ ದ್ವೇಷ ಯಾವುದರಿಂದ ಉಂಟಾಗುತ್ತದೆ ಎಂದು ಕೇಳಿದರೆ ಮೋಹದಿಂದ ಉಂಟಾಗುತ್ತದೆ ಎನ್ನುವುದೇ ಅದಕ್ಕೆ ಉತ್ತರ.

ಮೋಹವೆಂದರೆ ಏನು? ಸ್ಥಿರವಿಲ್ಲದ ಒಂದು ವಸ್ತು, ಸ್ಥಿರವಾಗಿದೆಯೆಂದು ತಿಳಿಯುವುದು ಮೋಹವೆನ್ನಲಾಗುವುದು. ಅದೇ ರೀತಿ ಅಶುಚಿಯಾಗಿರುವ ಒಂದು ವಸ್ತುವನ್ನು ಶುಚಿಯಾದುದೆಂದು ತಿಳಿಯುವುದೂ ಮೋಹವೇ. ಹಾಗೆಯೇ ಅಸುಖವಾಗಿರುವ ವಸ್ತುವನ್ನು ಮಧುರವಾಗಿರುವುದೆಂದು ತಿಳಿಯುವುದೂ ಮೋಹಕ್ಕೆ ಸೇರುತ್ತದೆ. ಉದಾಹರಣೆಗೆ-ನಾವು ಈ ಶರೀರವನ್ನು ಸ್ಥಿರವೆಂದು ತಿಳಿದಿದ್ದೇವೆಯೋ ಇಲ್ಲವೋ? ನಿತ್ಯವಾಗಿ ಮಾಡಿಕೊಳ್ಳಬೇಕೆಂದು ಭಾವಿಸುತ್ತೇವೆ.

\begin{shloka}
``ಭುಕ್ತಾ ಬಹುವೋ ದಾರಾ\\
ಲಬ್ಧಾಃ ಪುತ್ರಾಶ್ಚ ಪೌತ್ರಾಶ್ಚ"
\end{shloka}

-ಎಂದು ಒಂದು ಶ್ಲೋಕದಲ್ಲಿ ನೀಲಕಂಠ ದೀಕ್ಷಿತರು, `ಒಬ್ಬನು ಹಲವು ಮದುವೆಗಳನ್ನು ಮಾಡಿಕೊಂಡನು. ಹಲವು ಮಂದಿ ಮಕ್ಕಳನ್ನೂ ಮೊಮ್ಮಕ್ಕಳನ್ನೂ ಪಡೆದಾಯಿತು. ನೂರು ವರ್ಷಗಳೂ ಜೀವಿಸಿದನು. ಆಗ ಯಮನು, `ಬಾ' ಎಂದು ಕರೆದರೆ ಅವನು ತಯಾರಾಗಿ ಇರುವನೇ?' ಎಂದು ಹೇಳಿದ್ದಾರೆ.

ಆದ್ದರಿಂದ ಮನಸ್ಸಿನಲ್ಲಿ ತನ್ನ ಶರೀರವನ್ನೇ `ನಾನು' ಎಂದು ತಿಳಿಯುತ್ತಾನೆ. ನಾವು ಸ್ಥಿರವಾಗಿರಬೇಕೆನ್ನುವ ಭಾವನೆ ಅವನಿಗೆ ಇದೆ. ಈ ಭಾವನೆ ನ್ಯಾಯವಾದುದೇ. ಆದರೆ `ನಾನು' ಎನ್ನುವುದನ್ನು ಅವನು ಎಲ್ಲಿ ಭಾವಿಸುತ್ತಾನೋ ಅದರಲ್ಲಿ ವಿಶೇಷವಿದೆ, ಸ್ಥಿರವಲ್ಲದ ಒಂದು ವಸ್ತುವನ್ನು `ನಾನು' ಎಂದು ಭಾವಿಸಿ ಆ `ನಾನು' ಸ್ಥಿರವಾಗಬೇಕೆಂದು ಭಾವಿಸಿದರೆ ಅದಕ್ಕಿಂತಲೂ ಹುಚ್ಚುತನ ಬೇರೆ ಯಾವುದು? ಆದ್ದರಿಂದ ಮನುಷ್ಯನು ಅನಿತ್ಯವಾಗಿರುವ ಒಂದು ವಸ್ತುವನ್ನು ನಿತ್ಯವಾಗಿ ಭಾವಿಸುತ್ತಾನೆ. ಅಶುಚಿಯಾಗಿರುವ ವಸ್ತುವನ್ನು ಶುಚಿಯಾಗಿರಬೇಕೆಂದು ಭಾವಿಸುತ್ತಾನೆ.

ನಾನು ಬಹಳ ಬೆಳ್ಳಗಿದ್ದೇನೆ; ಮನ್ಮಥನಂತೆ ಇದ್ದೇನೆ ಎಂದು ಒಬ್ಬನು ಭಾವಿಸಿಕೊಂಡಿದ್ದಾನೆ. ಒಂದು ಚಿಕ್ಕ ಸೂಜಿಯಿಂದ ಅವನ ಶರೀರವನ್ನು ಚುಚ್ಚಿದರೆ ಅದರಿಂದ ತಕ್ಷಣ ನೋಡಲು ಅಸಾಧ್ಯವಾದ ರಕ್ತ ಬರುತ್ತದೆ. ನಾಲ್ಕು ದಿನಗಳು ಹಾಗೆಯೇ ಬಿಟ್ಟು ಬಿಟ್ಟರೆ ಅದರಿಂದ ಕೀವು ಬರಲು ಆರಂಭವಾಗುತ್ತದೆ. ಆದ್ದರಿಂದ ಈ ಶರೀರವೆನ್ನುವುದು ಬಹಳ ದುಃಖಕೊಡುವುದಾಗುತ್ತದೆ. ಹೀಗಿರುವಾಗ ಏತಕ್ಕೆ ಈ ದುಃಖಕರವಾದ ಶರೀರ ನಮಗೆ ಬೇಕು?

ನಾವು ಹೊಳೆಯನ್ನು ದಾಟಲು ಒಂದು ದೋಣಿಯನ್ನು ಉಪಯೋಗಿಸುತ್ತೇವೆ. ಹೊಳೆಯನ್ನು ದಾಟುವುದಕ್ಕೆ ಆ ದೋಣಿ ಯೋಗ್ಯವಾಗಿದೆಯೇ ಎಂದು ನಾವು ನೋಡುತ್ತೇವೆಯೇ ಹೊರತು, ಆ ದೋಣಿಯ ಉದ್ದ, ಅದರ ಬಣ್ಣ, ಅದರ ಅಗಲ ಇವುಗಳೆಲ್ಲವನ್ನೂ ನಾವು ನೋಡುತ್ತಾ ಇರುವುದಿಲ್ಲ. ಭಗವಂತನು ಸಂಸಾರ ಸಮುದ್ರವನ್ನು ದಾಟುವುದಕ್ಕಾಗಿ ಶಕ್ತಿಯನ್ನು ಈ ಶರೀರಕ್ಕೆ ಕೊಟ್ಟಿದ್ದಾನೆ. ಆದ್ದರಿಂದ ಈಗಲೇ ಇದನ್ನು ಇಟ್ಟುಕೊಂಡು ಸಂಸಾರಸಾಗರವನ್ನು ದಾಟಲು ಪ್ರಯತ್ನಪಡಬೇಕು. ಇದನ್ನು ಬಿಟ್ಟು, `ನಾನು ಇನ್ನೂ ಬೆಳ್ಳಗಾಗಬೇಕು. ಶರೀರ ಬೆಳ್ಳಗಾಗುವದಕ್ಕೆ ಯಾವು ಯಾವುದೋ ಪುಡಿಗಳನ್ನು ಹಚ್ಚಿಕೊಂಡು ಇದನ್ನು ಬೆಳ್ಳಗೆ ಮಾಡಿಕೊಳ್ಳಬೇಕು, ಎಂದೆಲ್ಲಾ ಯೋಚಿಸಬಾರದು. ಶರೀರ ಬೆಳ್ಳಗಿದ್ದರೇನು. ಕಪ್ಪಾಗಿದ್ದರೇನು? ಅದರಿಂದ ಯಾವ ಪ್ರಯೋಜನವೂ ಇಲ್ಲ. ಆದ್ದರಿಂದ ದುಃಖಕರವಾದ ಈ ಶರೀರವನ್ನು ದುಃಖರಹಿತವಾಗಿ ಮಾಡಿಕೊಳ್ಳಬೇಕೆಂದು ಒಬ್ಬನು ಯೋಚಿಸುತ್ತಿದ್ದಾನೆ. ಅದನ್ನು ಹಾಗೆ ಮಾಡಲು ಸಾಧ್ಯವಿಲ್ಲ. `ಅಶುಚಿಯಾಗಿದ್ದರೆ ಏನು, ಶುಚಿಯಾಗಿದ್ದರೆ ಏನು' ಎಂದು ನಾವು ಹೇಳುವುದಾದರೆ `ನಾನು ಏಕೆ ಸ್ನಾನ ಮಾಡುವುದು' ಎಂದು ಕೇಳಬಾರದು. ಈ ಶರೀರವೇ `ನಾನು' ಎನ್ನುವ ಭಾವನೆ ಇರುವವರೆಗೆ ಅವನಿಗೆ ಈ `ಶೌಚ' `ಅಶೌಚ' ಎರಡೂ ಉಂಟು. ಇಂಥ ಭಾವನೆ ಇಲ್ಲವೆಂದರೆ ನಾವು ಒಂದು ವಿಧವಾಗಿ ಶುಚಿಯಾಗಿಲ್ಲದಿದ್ದರೂ ಕೂಡ (ಆತ್ಮಜ್ಞಾನವನ್ನು ಪಡೆದಿರುವುದರಿಂದ) ಬೇರೆ ವಿಧವಾಗಿ ಶುಚಿಯೆಂದು ನಿರೂಪಿಸಬಹುದು. ಆದ್ದರಿಂದ ಈ ಶರೀರವನ್ನು ಇಟ್ಟುಕೊಂಡು ನಾವು ಪಡೆಯಬೇಕಾದುದನ್ನು ಪಡೆದರೆ ಅದೇ ಉತ್ತಮವಾದುದು. ಅದನ್ನು ಬಿಟ್ಟು ಅಶುಚಿಯಾಗಿರುವ ಈ ಶರೀರವನ್ನು `ನಾನು ಶುಚಿಪಡಿಸುತ್ತೇನೆ' ಎಂದರೆ ಅದು ಆಗದ ಕೆಲಸ.

ಅನಂತರ `ಅಸುಖ'ವೆಂದು ಹೇಳಲ್ಪಟ್ಟಿದೆ. (ಸುಖವಲ್ಲದ್ದನ್ನು ಸುಖ ಕೊಡುವುದಾಗಿ ಭಾವಿಸುವುದು ಮೋಹವೇ ಆಗುತ್ತದೆ.)

ಪ್ರಪಂಚದಲ್ಲಿ ಯಾವಾಗಲೂ ಪ್ರಿಯವಾಗಿರುವ ವಸ್ತು ಇದೆಯೇ ಎಂದು ಕೇಳಿದರೆ, ಅದಕ್ಕೆ ಶಂಕರ ಭಗವತ್ಪಾದರು ಒಂದು ಶ್ಲೋಕದಲ್ಲಿ-

\begin{shloka}
``ಯಸ್ಮಾತ್ ಯಾವತ್ ಪ್ರಿಯಂ ಸ್ಯಾದಿಹ ಹಿ ವಿಷಯತ ಸ್ತಾವದಸ್ಮಿನ್ ಪ್ರಿಯತ್ವಂ\\
ಯಾವದ್ದುಃಖಂ ಚ ಯಸ್ಮಾದ್ಭವತಿ ಖಲು ತತಸ್ತಾವದೇವಾ ಪ್ರಿಯತ್ವಮ್|\\
ನೈಕಸ್ಮಿನ್ ಸರ್ವಕಾಲೇಽಸ್ತ್ಯುಭಯಮಪಿ ಕದಾಪ್ಯಪ್ರಿಯೋಽಪಿ ಪ್ರಿಯಃಸ್ಯಾತ್\\
ಪ್ರೇಯಾನಪ್ಯಪ್ರಿಯೋ ವಾ ಸತತಮಪಿ ತತಃ ಪ್ರೇಯ ಆತ್ಮಾಖ್ಯ ವಸ್ತುಃ||"
\end{shloka}

(ಎಲ್ಲಿಯವರೆಗೆ ಒಬ್ಬನು ಒಂದು ವಸ್ತುವಿನಿಂದ ಪ್ರಿಯತ್ವವನ್ನು ಪಡೆಯುತ್ತಾನೋ ಅದುವರೆಗೆ ಆ ವಸ್ತು ಪ್ರಿಯವಾಗಿರುತ್ತದೆ; ಅದು ದುಃಖವನ್ನು ಉಂಟುಮಾಡುವವರೆಗೆ ಅದು ಪ್ರಿಯವಾಗಿರುತ್ತದೆ; ಯಾವುದೇ ಒಂದು ವಸ್ತುವಾಗಲಿ ಯಾವಾಗಲೂ ಪ್ರಿಯವಾಗಲೀ ಅಪ್ರಿಯವಾಗಲೀ ಆಗಿರುವುದಿಲ್ಲ. ಕೆಲವೊಮ್ಮೆ ಅಪ್ರಿಯವೂ ಪ್ರಿಯವಾಗಿ ಬದಲಾಗಬಹುದು; ಪ್ರಿಯವಾಗಿರುವುದು ಅಪ್ರಿಯವೂ ಆಗಬಹುದು; ಆದ್ದರಿಂದ ಆತ್ಮ ಎನ್ನುವ ವಸ್ತು ಯಾವಾಗಲೂ ಬಹಳ ಪ್ರಿಯವಾಗಿರತಕ್ಕಂತಹದು)-ಎಂದು ವರ್ಣಿಸಿದ್ದಾರೆ. ಒಬ್ಬ ತಂದೆಗೆ ತನ್ನ ಮಗನ ಮೇಲೆ ಪ್ರೀತಿ ಇರುತ್ತದೆ. ಆದರೆ ಮಗನು ತಂದೆಯ ಮಾತು ಕೇಳುವವರೆಗೆ ಮಾತ್ರ ಅವನು ತಂದೆಗೆ ಪ್ರಿಯ. ಅವನು ಹಾಗೆ ನಡೆದುಕೊಳ್ಳದಿದ್ದರೆ ಅವನನ್ನು ತಂದೆ ಬಯ್ಯುವನು. ತಂದೆ ತನಗಾಗಿ ಮಗನ ಮೇಲೆ ಪ್ರೀತಿ ತೋರಿಸುವನೇ ಹೊರತು ಆ ಪ್ರೀತಿ ಮಗನಿಗಾಗಿ ಅಲ್ಲ. ಮಗನು ತನ್ನ ಮಾತಿನಂತೆ ನಡೆದುಕೊಳ್ಳುವವರೆಗೆ, ಹಣ ಸಂಪಾದಿಸಿ ಅದನ್ನು ತನಗೂ ಕೊಡುವವರೆಗೆ ತಂದೆಗೆ ಅವನ ಮೇಲೆ ಪ್ರೀತಿ ಇರುವುದೆಂಬುದನ್ನು ನಾವು ಪ್ರಪಂಚದಲ್ಲಿ ಕಾಣುತ್ತೇವೆ.

ಹಾಗೆಯೇ ದುಃಖ ಎಲ್ಲಿಯವರೆಗೆ ಇರುವುದು ಎನ್ನುವುದೂ ನಾವು ನೋಡಬೇಕು. ಒಬ್ಬನಿಗೆ ಸಕ್ಕರೆ ಬಹಳ ಇಷ್ಟವಾಗಿರಬಹುದು, ಇನ್ನೊಬ್ಬನಿಗೆ ಅದು ಇಷ್ಟವಾದರೂ ಸಹ, ಸಕ್ಕರೆ ಖಾಯಿಲೆ ಬಂದಿರುವುದರಿಂದ ಅದನ್ನು ಅವನು ಮುಟ್ಟುವುದೇ ಇಲ್ಲ. ``ಇಷ್ಟು ದಿನಗಳು ಸಕ್ಕರೆ ತಿನ್ನುತ್ತಿದ್ದೀರಲ್ಲಾ, ಏಕೆ ಈಗ ಮಾತ್ರ ಸಕ್ಕರೆ ತಿನ್ನುವುದಿಲ್ಲ?" ಎಂದು ಕೇಳಿದರೆ, ``ಈಗ ನನಗೆ ಸಕ್ಕರೆ ಖಾಯಿಲೆ ಆಗಿದೆ. ಆದ್ದರಿಂದ ನಾನು ಸಕ್ಕರೆ ತಿನ್ನುವುದಿಲ್ಲ" ಎಂದು ಅವನು ಹೇಳುವನು. ಔಷಧವೆನ್ನುವ ಕಹಿಯಾದ ಒಂದು ವಸ್ತುವನ್ನು ನಾವೆಲ್ಲಾರೂ ``ಬೇಡ, ಬೇಡ" ಎಂದು ಸಾಧಾರಣ ಸಮಯಗಳಲ್ಲಿ ದೂರವಿಡುತ್ತಿರುತ್ತೇವೆ. ಆದರೆ ಖಾಯಿಲೆ ಬಂದರೆ ``ಮೊದಲು ಔಷಧ ತೆಗೆದುಕೊಂಡು ಬನ್ನಿ" ಎಂದು ನಾವೇ ಕೇಳಿ ತೆಗೆದುಕೊಳ್ಳುತ್ತೇವೆ. ಆಗ ನಾವು ಮೊದಲು ಪ್ರೀತಿಯನ್ನು ಇಡದ ವಸ್ತುವಿನ ಮೇಲೂ ಪ್ರೀತಿಯನ್ನು ತೋರಿಸುತ್ತೇವೆ. ಆದ್ದರಿಂದ ಒಂದೇ ವಸ್ತುವೇ ಎಲ್ಲಾ ಕಾಲದಲ್ಲಿಯೂ ಪ್ರಿಯವಾಗಿಯೋ ಅಪ್ರಿಯವಾಗಿಯೋ ಇರುವುದಿಲ್ಲ.

ಆತ್ಮ ಎನ್ನುವುದು ಯಾವಾಗಲೂ ಪ್ರಿಯವಾಗಿರುವುದು. ಆದರೆ ಮನುಷ್ಯನು ತಾತ್ಕಾಲಿಕವಾಗಿ ಸುಖವನ್ನು ಕೊಡುವ ವಿಷಯಗಳನ್ನು, ಅವುಗಳೇ ತನಗೆ ಪ್ರಿಯವಾದವು ಎಂದು ಭಾವಿಸಿ ಅವುಗಳ ಮೇಲೆಯೇ ಆಸೆಯನ್ನು ಇಟ್ಟುಕೊಂಡಿದ್ದಾನೆ. ನಿಜಕ್ಕೂ ಅವನು ಪ್ರಿಯತೆಯನ್ನು ಹೊಂದಬೇಕಾದ ಮುಖ್ಯವಾದ ವಿಷಯವೆಂದರೆ ಅದು ಆತ್ಮ.

ಹೀಗೆ ಒಬ್ಬನು ಆಸೆ ಇಟ್ಟುಕೊಳ್ಳಲು ಯೋಗ್ಯವಲ್ಲದ ವಿಷಯಗಳ ಮೇಲೆ ಆಸೆ ಇಟ್ಟುಕೊಂಡರೆ ಅದೇ ಅವನಿಗೆ ಮೃತ್ಯುವಾಗಿ ಬಿಡುವುದು. ಅದಕ್ಕೆ ಮೋಹವೆಂದು ಸೂತ್ರದಲ್ಲಿ ವರ್ಣಿತವಾಗಿರುವುದು. ಆದ್ದರಿಂದ ಈ ರಾಗ-ದ್ವೇಷ ಎರಡೂ ಬಹಳ ದೊಡ್ಡ ಶತ್ರುಗಳು. ಮೋಹವೆನ್ನುವುದು ಇದ್ದರೆ ಈ ರಾಗ-ದ್ವೇಷಗಳೂ ಇರುವುವು. ಮೋಹವೆನ್ನುವುದು ಇಲ್ಲದಿದ್ದರೆ ಈ ಪ್ರಿಯತೆ, ಅಪ್ರಿಯತೆ ಇಲ್ಲವೇ ಇಲ್ಲ.

\begin{shloka}
``ತೇಷಾಂ ಮೋಹಃ ಪಾಪೀಯಾನ್"
\end{shloka}

(ಅವುಗಳಲ್ಲಿ ಮೋಹವೇ ಹೆಚ್ಚು ಅಪಾಯಕಾರಿ)-ಎಂದು ಹೇಳಿರುವುದರಿಂದ ಮೋಹವನ್ನು ನಾವು ತೊರೆಯಬೇಕು.

ಶಾಸ್ತ್ರವನ್ನು ಓದುವಾಗ, ``ಬೇರೆ ಎಲ್ಲಾ ವಸ್ತುಗಳೂ ಮೋಹ" ವೆಂದು ಒಬ್ಬನಿಗೆ ತೋರಬಹುದು. ಆದರೆ ಓದಿದ ಪುಸ್ತಕವನ್ನು ಕೆಳಗೆ ಇಟ್ಟು ಒಡನೆ ಇನ್ನೂ ಸ್ವಲ್ಪ ಟಾನಿಕ್ ಕುಡಿದರೆ ಒಳ್ಳೆಯದಲ್ಲವೇ? ನಾವು ಪುಷ್ಟಿಯಾಗಿ ಇರಬಹುದಲ್ಲಾ" ಎಂದು ತೋರುತ್ತದೆ. ``ಏಕಂ ವಿಶೋಕಂ ಪರಂ" ಎಂದು ಹೇಳಲ್ಪಟ್ಟಿರುವುದರಿಂದ ಆತ್ಮನಿಗೆ ಶೋಕವಿಲ್ಲವೆಂದು ತಿಳಿಯುವುದು. ಈ ಆತ್ಮನು ಸುಖಸ್ವರೂಪಿಯೇ ಎನ್ನುವುದನ್ನು ಸುಷುಪ್ತಿಯಲ್ಲಿ ನಾವು ಚೆನ್ನಾಗಿ ಅನುಭವದ ಮೂಲಕ ನೋಡಬಹುದು. ಆದ್ದರಿಂದ ನಾವು ಆತ್ಮನ ಸ್ವರೂಪವನ್ನು ಕುರಿತೂ ಶೋಧಿಸಬೇಕು, ಬ್ರಹ್ಮದ ಸ್ವರೂಪವನ್ನು ಕುರಿತೂ ಶೋಧಿಸಬೇಕು. ನಾವು ಹೀಗೆ ನೋಡಿದರೆ ಬ್ರಹ್ಮವೂ ಸುಖಸ್ವರೂಪಿಯೇ. ಆತ್ಮನೂ ಸುಖಸ್ವರೂಪಿಯೇ. ಆದರೂ `ಏಕಂ ವಿಶೋಕಂ ಪರಂ' ಎಂದು ಹೇಳಲ್ಪಟ್ಟಿದೆ. ಇದಕ್ಕೆ ಮೇಲೆ ಯಾವ ವಸ್ತುವೂ ಇಲ್ಲ.

\begin{shloka}
``ಪ್ರತ್ಯಕ್ ಬ್ರಹ್ಮ ತದಸ್ಮಿನ್‌ಯಸ್ಯ ಕೃಪಯಾ ತಂ ದೇಶಿಕೇಂದ್ರಂ ಭಜೇ"
\end{shloka}

(ಯಾರ ಕೃಪೆಯಿಂದ ಒಳಗಿರುವ ಆತ್ಮನಾದ ನಾನೇ ಬ್ರಹ್ಮನಾಗಿ ಇದ್ದೇನೋ ಆ ಶ್ರೇಷ್ಠ ಗುರುವನ್ನು ಸ್ತುತಿಸುತ್ತೇನೆ.)

`ಪ್ರತ್ಯಕ್ ಬ್ರಹ್ಮ' ಎಂದು ಹೇಳಿರುವುದರಿಂದ ಪ್ರತ್ಯಗಭಿನ್ನಂ ಬ್ರಹ್ಮ (ಜೀವನಿಂದ ಬ್ರಹ್ಮ ಬೇರೆಯಲ್ಲ)-ಈ ಜೀವನೂ ಬ್ರಹ್ಮವೂ ಒಂದಾಗಿಯೇ ಇರಬೇಕು.

\begin{shloka}
``ಯದಾಹ್ಯೇವೈಷ ಏತಸ್ಮಿನುದರಮನ್ತರಂ ಕುರುತೇ| ಆತ ತಸ್ಮಿನ್ ಭಯಂ ಭವತಿ|"
\end{shloka}

(ಅದರಲ್ಲಿ ಸ್ವಲ್ಪ ಭೇದವನ್ನು ಮಾಡಿದರೂ ಸಹ ಅದರಿಂದ ಭಯ ಉಂಟಾಗುತ್ತದೆ ಎಂದು ವೇದ ಘೋಷಿಸುತ್ತದೆ.)

ಕೆಲವರು ``ನಾವು ಸಗುಣ ಈಶ್ವರನನ್ನು ಆರಾಧಿಸಿ, ಅವನ ಲೋಕಕ್ಕೆ ಹೋದರೆ ಭಯವಿಲ್ಲದೆಯೇ ಇರಬಹುದಲ್ಲಾ" ಎಂದು ಹೇಳುವರು. ಆ ಲೋಕಗಳಲ್ಲಿ ಇದ್ದರೂ ಕೂಡ,

``ದ್ವಿತೀಯಾತ್ ವೈ ಭಯಂ ಭವತಿ" (ಎರಡನೆಯ ವಸ್ತುವಿದ್ದರೆ ಭಯ ಉಂಟಾಗುತ್ತದೆ.)-ಎಂದು ಹೇಳಲ್ಪಟ್ಟಿದೆ. ಎರಡನೆಯ ವಸ್ತುವೆಂದು ಒಂದು ಇದ್ದರೆ ಆಗ ಅವನಿಗೆ ಭಯವೇ ಆಗುತ್ತದೆ. ಹೀಗಿಲ್ಲದೆ ಒಬ್ಬನು ಬ್ರಹ್ಮ ವಿಚಾರವನ್ನು ಮಾಡಿದರೆ, ಬ್ರಹ್ಮ ವಿಚಾರಮಾಡಿ ತನ್ನ ಸ್ವರೂಪವನ್ನು ತಿಳಿದುಕೊಂಡರೆ, ಅವನು ಬ್ರಹ್ಮವೇ ಆಗಿಬಿಡುವನು. ಆದ್ದರಿಂದಲೇ ``ಬ್ರಹ್ಮವಿತ್ ಬ್ರಹ್ಮೈವ ಭವತಿ'' ಎಂದು ಹೇಳಿರುವುದು.

ಆದ್ದರಿಂದ ಒಬ್ಬನು ತತ್ತ್ವವನ್ನು ಅರಿಯುವುದರಿಂದ ಅವನು ಬೇರೆ ವಿಧವಾಗಿ ತಿಳಿದುಕೊಂಡುದೆಲ್ಲಾ ದೂರವಾಗಿ ಅವನು ಬ್ರಹ್ಮವಾಗಿಯೇ ಆಗಿಬಿಡುತ್ತಾನೆಂದು ಹೇಳಲ್ಪಟ್ಟಿದೆ. ನಾವು ಸ್ವಲ್ಪಮಟ್ಟಿಗಾದರೂ ವೇದಾಂತ ವಿಚಾರವನ್ನು ಮಾಡಿ ಆತ್ಮಸಾಕ್ಷಾತ್ಕಾರವನ್ನು ಪಡೆಯಲು ಪ್ರಯತ್ನಪಟ್ಟರೆ ಅದರಲ್ಲೇ ನಾವು ಶ್ರದ್ಧೆಯನ್ನು ಇಟ್ಟುಕೊಂಡಿದ್ದರೆ-

``ಇತಃ ಕೋನ್ವಸ್ತಿ ಮೂಢಾತ್ಮಾ ಯಸ್ತುಸ್ವಾರ್ಥೇ ಪ್ರಮಾದ್ಯತೆ" (ತನ್ನ ಆತ್ಮನ ಒಳ್ಳೆಯದರ ವಿಷಯವಾಗಿ ತಪ್ಪಾಗಿ ತಿಳಿದುಕೊಂಡಿರುವವನಿಗಿಂತಲೂ ಮೂರ್ಖನು ಯಾರಾದರೂ ಇರಬಹುದೇ?)

-ಎಂದು ಭಗವತ್ಪಾದರು ಹೇಳಿದಂತೆ, ಒಂದುವೇಳೆ ಈ ಜನ್ಮದಲ್ಲಿ ನಮಗೆ ಮೋಕ್ಷ ದೊರೆಯದಿದ್ದರೂ ಆ ಸಂಸ್ಕಾರಗಳಿಂದ (ವಾಸನೆಗಳಿಂದ) ಮುಂದಿನ ಜನ್ಮದಲ್ಲಿ ನಮಗೆ ``ಆತ್ಮ ಸಾಕ್ಷಾತ್ಕಾರ" ದೊರೆಯಬಹುದು. ಆದ್ದರಿಂದ ನಾವು ಅದಕ್ಕಾಗಿ ಪ್ರಯತ್ನ ಪಡಬೇಕು. ನಾವು ಯಾವ ಉನ್ನತ ಲೋಕಕ್ಕೆ ಹೋದರೂ ``ಕ್ಷೀಣೇ ಪುಣ್ಯೇ ಮರ್ತ್ಯಲೋಕಂ ವಿಶನ್ತಿ" ಎಂದು ಹೇಳಿದಂತೆ ಪುಣ್ಯಗಳು ಮುಗಿಯುತ್ತಲೇ ಮನುಷ್ಯಲೋಕಕ್ಕೇ ಬರಬೇಕಾಗುತ್ತದೆ. ಆದ್ದರಿಂದ ನಾವು ಸ್ಥಿರವಲ್ಲದ ವಸ್ತುಗಳಿಂದ ಬೇರ್ಪಟ್ಟು ಸ್ಥಿರವಾದ ವಸ್ತುವನ್ನು ಪಡೆಯಲು ಭಗವಂತನು ನಮಗೆ ಶಕ್ತಿಕೊಟ್ಟಿದ್ದಾನೆ. ಇದಕ್ಕೆ ಮನುಷ್ಯ ಜನ್ಮವೆನ್ನುವುದು ಬಹಳ ಶ್ರೇಷ್ಠವಾದುದು. ನಾವು ಮನುಷ್ಯ ಜನ್ಮವನ್ನು ವ್ಯರ್ಥಮಾಡಿಕೊಳ್ಳಬಾರದು. ಇದು ಶಾಸ್ತ್ರಗಳ ನಿರ್ಧಾರ. ಈ ಜ್ಞಾನವನ್ನು ನಿಮಗೆ ಭಗವಂತನು ಕೊಡಲಿ ಎಂದು ಪ್ರಾರ್ಥಿಸಿ ಈ ಉಪನ್ಯಾಸವನ್ನು ಮುಗಿಸುತ್ತಿದ್ದೇನೆ.

