\chapter{ಭಕ್ತನ ಒಂದು ಪ್ರಾರ್ಥನೆ}\label{chap11}

ಶ್ರೀಶಂಕರ ಭಗವತ್ಪ್ಪದರು ಸೂತ್ರಭಾಷ್ಯ, ಉಪನಿಷದ್ಭಾಷ್ಯ, ಗೀತಾಭಾಷ್ಯ ಮುಂತಾದ ರಚನೆಗಳನ್ನು ಮಾಡಿದ್ದಾರೆ. ಅವುಗಳಲ್ಲದೆ ಬಹಳ ಸೂಕ್ಷ್ಮವಾದ ವೇದಾಂತ ಸಿದ್ಧಾಂತವನ್ನು ಸಾಮಾನ್ಯ ಜನರಿಗೆ ತಿಳಿಯುವಂತೆ ಪ್ರಕರಣಗ್ರಂಥಗಳೆಂದು ಹೇಳಲ್ಪಡುವ ರಚನೆಗಳನ್ನು ಮಾಡಿದ್ದಾರೆ. ಇಂಥ ಗ್ರಂಥಗಳು ಸಂಸ್ಕೃತ ಶಾಸ್ತ್ರದಲ್ಲಿ ಸ್ವಲ್ಪ ಮಟ್ಟಿಗೂ ಪಾಂಡಿತ್ಯವಿಲ್ಲದವರಿಗೆ ತಿಳಿಯುವುದು. ಇಷ್ಟೆಲ್ಲಾ ರಚನೆ ಮಾಡಿದ ಮೇಲೂ ಭಗವಂತನನ್ನು ಸ್ತುತಿಸುವ ಗ್ರಂಥಗಳನ್ನೂ ಬರೆದರು. ಹಾಗಿರುವ ಗ್ರಂಥಗಳಲ್ಲಿ `ಶಿವಾನಂದ ಲಹರೀ' ಎನ್ನುವುದು ಶಿವನನ್ನು ಸ್ತುತಿಸುವಂಥ ಶ್ಲೋಕಗಳನ್ನುಳ್ಳದ್ದು. ಆ ಗ್ರಂಥ ಬಹಳ ಪ್ರಸಿದ್ಧಿಯಾದುದು. ಅದರಲ್ಲಿ ಶ್ರೀಶಂಕರ ಭಗವತ್ಪಾದರು ಒಂದು ಶ್ಲೋಕದಲ್ಲಿ,

\begin{shloka}
`ಅಮಿತಮುದಮೃತಂ ಮುಹುರ್ದುಹನ್ತೀಂ\\
ವಿಮಲಭವತ್ಪದಗೋಷ್ಠಮಾವಸನ್ತೀಮ್ |\\
ಸದಯ ಪಶುಪತೇ ಸುಪುಣ್ಯ ಪಾಕಂ\\
ಮಮ ಪರಿಪಾಲಯ ಭಕ್ತಿಧೇನುಮೇಕಾಮ್ ||'
\end{shloka}

(ಕರುಣೆಯುಳ್ಳೆವನೇ ! ಪಶುಪತಿಯೇ ! ಮಿತವಲ್ಲದ ಬ್ರಹ್ಮಾನಂದವೆನ್ನುವ ಹಾಲನ್ನು ಮತ್ತೆ ಮತ್ತೆ ಕರೆಯುವಂತಹದೂ, ವಿಮಲವಾದ ನಿಮ್ಮ ಪಾದಗೋಷ್ಠದಲ್ಲಿ ಇರುವುದೂ, ಪುಣ್ಯದ ಫಲವೂ ಆಗಿರುವ ನನ್ನ ಭಕ್ತಿ ಎನ್ನುವ ಹಸು ಒಂದನ್ನು ಕಾಪಾಡು)

ಎನ್ನುತ್ತಾರೆ. ಭಗವತ್ಪಾದರು ಜ್ಞಾನವನ್ನು ಪಡೆದು ಆತ್ಮಸ್ವರೂಪದಲ್ಲೇ ಇದ್ದರು. ಅವರು ಏತಕ್ಕಾಗಿ ಭಕ್ತಿ-ರಚನೆಗಳನ್ನು ಮಾಡಬೇಕು? ಭಾಗವತದಲ್ಲಿ-

\begin{shloka}
`ಆತ್ಮಾರಾಮಾಶ್ಚ ಮುನಯೋ ನಿರ್ಗ್ರಂಥಾ ಅಪ್ಯುರುಕ್ರಮೇ |\\
ಕುರ್ವನ್ತ್ಯಹೇತುಕೀಂ ಭಕ್ತಿಮಿತ್ಥಂಭೂತಗುಣೋ ಹರಿಃ ||'
\end{shloka}

(ಆತ್ಮನಲ್ಲಿಯೇ ಆನಂದವನ್ನು ಪಡೆಯುವ, ಬಂಧಗಳಿಲ್ಲದ ಮುನಿಗಳೂ ಕೂಡ ಭಗವಂತನಿಗೆ ಭಕ್ತಿಯನ್ನು ಸಮರ್ಪಿಸುವರು. ಇದೇ ಭಗವಂತನ ಗುಣಗಳ ಮಹಿಮೆ.)

-ಎಂದು ಹೇಳಲಾಗಿದೆ. ಜ್ಞಾನಿಯಾದವರೂ ಯಾವಾಗಲೂ ತಮ್ಮ ಆನಂದವಾದ ಸ್ವರೂಪದಲ್ಲೇ ಇರುವರು. ಅವರಿಗೆ ಅದನ್ನು ಬಿಟ್ಟು ಬೇರೆ ಯಾವುದೂ ಬೇಕಾಗಿಲ್ಲ. ಏಕೆಂದರೆ ಅವರು ಅರಿತು ಕೊಳ್ಳಬೇಕಾದುದು ಯಾವುದೂ ಇಲ್ಲ, ಅವರು,

\begin{shloka}
`ಪಲಾಲಮಿವ ಧ್ಯಾನಾರ್ಥೀ ತ್ಯಜೇತ್ ಗ್ರಂಥ ಮಶೇಷತಃ'
\end{shloka}

(ಧಾನ್ಯವನ್ನು ಅಪೇಕ್ಷಿಸುವವನು ಹೊಟ್ಟನ್ನು ಬಿಡುವಂತೆ, ಒಂದು ಸ್ಥಿತಿಗೆ ಬಂದ ಮೇಲೆ ಪುಸ್ತಕಗಳನ್ನು ಬಿಟ್ಟುಬಿಡಬಹುದು.)

ಆದ್ದರಿಂದ ಪರವಸ್ತುವಿನ ಅರಿವಾದ ಮೇಲೆ ಯಾವ ಪುಸ್ತಕವನ್ನೂ ಓದಬೇಕಾಗಿಲ್ಲ. ಅಕ್ಕಿಯನ್ನು ತೆಗೆದುಕೊಂಡಮೇಲೆ ಹೊಟ್ಟನ್ನು ಹೇಗೆ ಬಿಟ್ಟು ಬಿಡುತ್ತೇವೆಯೋ ಹಾಗೆಯೇ ಜ್ಞಾನಿಯಾದವನು ಪುಸ್ತಕಗಳನ್ನು ಬಿಟ್ಟುಬಿಡಬಹುದು. ಅಂಥ ಜ್ಞಾನಿಗಳು ಕೂಡ

\begin{shloka}
``ಕುರ್ವನ್ತ್ಯಹೇತುಕೀಂ ಭಕ್ತಿಮಿತ್ಥಂಭೂತ ಗುಣೋ ಹರಿಃ''
\end{shloka}

ಎಂದು ಹೇಳಿರುವಂತೆ ಭಗವಂತನ ಮೇಲೆ ಭಕ್ತಿ ಇಟ್ಟುಕೊಂಡಿರುವವರು. ಅಂಥ ಜ್ಞಾನಿಗಳು ಯಾವುದನ್ನಾದರೂ ಸಂಪಾದಿಸಬೇಕೇ ಎಂದು ಕೇಳಿದರೆ, ಅವರು ಸಂಪಾದಿಸಬೇಕಾದುದು ಯಾವುದೂ ಇಲ್ಲ. ಹಾಗಿರುವ ಮಹಾತ್ಮರೂ ಕೂಡ ಭಗವದ್ಭಕ್ತಿಗೆ ಮುಖ್ಯತ್ವವನ್ನು ಕೊಡುತ್ತಾರೆ ಎಂದು ಭಾಗವತದಲ್ಲಿ ಹೇಳಿದೆ. ಹಾಗೆಯೇ ಶ್ರೀ ಶಂಕರ ಭಗವತ್ಪಾದರು ಆತ್ಮಾರಾಮರಾಗಿದ್ದರೂ ಕೂಡ ಭಗವಂತನನ್ನು ಕುರಿತು ಹಲವು ಸ್ತ್ರೋತ್ರಗಳನ್ನು ರಚಿಸಿದ್ದಾರೆ.

\begin{shloka}
``ಅಮಿತಮುದಮೃತಂ ಭಕ್ತಿಧೇನುಮೇಕಾಮ್''
\end{shloka}

ಶ್ರೀಶಂಕರ ಭಗವತ್ಪಾದರು ಶಿವನನ್ನು ನೋಡಿ, ``ನನ್ನ ಹತ್ತಿರ ಭಕ್ತಿ ಎನ್ನುವ ಒಂದು ಹಸು ಇದೆ. ಅದನ್ನು ನೀನು ಕಾಪಾಡಬೇಕು'' ಎಂದರು. ಏಕೆ ಹಸುವನ್ನು ಕಾಪಾಡಲು ಭಗವಂತನಿಗೆ ಹೇಳುತ್ತಾರೆಂದರೆ, ಭಗವಂತನೇ ಕಾಪಾಡಬೇಕು, ಭಗವಂತನಿಗೆ `ಪಶುಪತಿ' ಎನ್ನುವ ಒಂದು ಹೆಸರಿದೆ. ಆದ್ದರಿಂದ ಭಕ್ತನ ಹಸುವಿಗೆ ಪತಿಯಾಗಿ ಅವನು ಬೆಳಗುತ್ತಾನೆ. ಅವನೇ ಹಸುವನ್ನು ಕಾಪಾಡಬೇಕೆಂದು ಶಂಕರರು ಕೇಳಿಕೊಂಡರು. ಭಕ್ತಿ ಎನ್ನುವುದು ಒಂದು ಭಾವನೆಯನ್ನು ಸೂಚಿಸುತ್ತದೆ. ಶಂಕರರು ಏತಕ್ಕೆ ಇದನ್ನು ಹಸು ಎನ್ನುತ್ತಾರೆ? ಕಾವ್ಯದಲ್ಲಿ ಹೀಗೆ ಹೇಳುವುದು ಸಹಜ. ಆದ್ದರಿಂದ ಭಕ್ತಿಗೆ ಹೀಗೆ ಉಪಮಾನಕೊಟ್ಟರೆ ಅದರಿಂದ ತಪ್ಪೇನಿಲ್ಲ.

\begin{shloka}
``ಅಮಿತಮುದಮೃತಂ ಮುಹುರ್ದುಹನ್ತೀಂ''
\end{shloka}

ಭಕ್ತಿಭಾವನೆಯಿಂದ ಕೂಡಿದ ಅಂತಃಕರಣ ಯಾವಾಗಲೂ ಸಂತೋಷವಾಗಿರುವುದು. ಭಕ್ತನಿಗೆ ಸಂತೋಷವಿಲ್ಲದ ಕಾಲವೇ ಇಲ್ಲ. ಅವನಿಗೆ ಎಂಥ ದುಃಖವು ಬಂದರೂ, ಅವನ ಮನಸ್ಸು ಮಾತ್ರ ಆನಂದದಲ್ಲೇ ಸಂಚರಿಸುತ್ತಿರುವುದು. ಭಕ್ತಿಯ ಪ್ರಭಾವ ಅಂತಹುದೇ. ಎಲ್ಲೆ ಇಲ್ಲದ ಸಂತೋಷವೆನ್ನುವ ಹಾಲನ್ನು ಕರೆಯುವ ಹಸು ಅದು! ಯಾವಾಗ ಸಂತೋಷ ದೊರೆಯುವುದೆಂದು ಕೇಳಿದರೆ ಸಂತೋಷ ಯಾವಾಗಲೂ ಇದ್ದೇ ಇರುವುದು.

ಹಸುವನ್ನು ನಾವು ಕೊಟ್ಟಿಗೆಯಲ್ಲಿ ಕಟ್ಟಿರುತ್ತೇವೆ. ಅಂಥ ಕೊಟ್ಟಿಗೆ ಯಾವಾಗಲೂ ಅಶುದ್ಧವಾಗಿರುವುದು. ನಾವು ಶುದ್ಧ ಮಾಡುತ್ತಾ ಮಾಡುತ್ತಾ ಅದು ಮತ್ತೆ ಮತ್ತೆ ಅಶುದ್ಧವಾಗಿಯೇ ಇರುವುದು. ಏಕೆಂದರೆ ಪದೆ ಪದೆ ಸಗಣಿ, ಗಂಜಳ ಇವುಗಳಿಂದ ಜಾಗವನ್ನು ಶುದ್ಧವಾಗಿಡಲು ಸಾಧ್ಯವಾಗುವುದಿಲ್ಲ. `ಶುದ್ಧಿ' ಎನ್ನುವುದು ಬೇರೆ, `ಶುಚಿ' ಎನ್ನುವುದು ಬೇರೆ. ಶಾಸ್ತ್ರದಲ್ಲಿ ಸಗಣಿ, ಗಂಜಳ ಇಂಥವು ಬಹಳ ಪವಿತ್ರವಾದ ವಸ್ತುಗಳೆಂದು ಹೇಳಲ್ಪಟ್ಟಿದ್ದರೂ ಕೂಡ ಅವುಗಳು ಇರುವುದರಿಂದ ಕೊಟ್ಟಿಗೆ ಶುದ್ಧವಾಗಿರುವುದಿಲ್ಲ.

\begin{shloka}
``ವಿಮಲ ಭವತ್ಪದಗೋಷ್ಠಮಾವಸನ್ತೀಮ್''
\end{shloka}

ಆದರೆ ಭಕ್ತಿ ಎನ್ನುವ ಹಸುವನ್ನು ಕಟ್ಟಬೇಕಾದ ಕೊಟ್ಟಿಗೆಯಾದ ಭಗವಂತನ ಪಾದಾರವಿಂದಗಳು ವಿಮಲವಾದವು. ಆದ್ದರಿಂದ ಭಗವಂತನ ಪಾದಾರವಿಂದಗಳೇ ಭಕ್ತಿ ಎನ್ನುವ ಹಸು ಇರುವ ಜಾಗ.

ನಮ್ಮ ಹಸುವನ್ನು ಭಗವಂತನು ಏತಕ್ಕೆ ಕಾಪಾಡಬೇಕು? ಏತಕ್ಕೆಂದರೆ ಭಗವಂತನನ್ನು ನಾವು ಕರುಣಾಸಮುದ್ರನೆನ್ನುತ್ತೇವೆ! ಕರುಣಾಸಮುದ್ರನಾಗಿರುವವನಿಗೆ ಎಲ್ಲೋ ಅಲೆಯುತ್ತಿರುವ ಹಸುವನ್ನು ಕಾಪಾಡಲು ಕಷ್ಟವೇ? ಭಕ್ತರನ್ನು ಕಾಪಾಡುವುದು ಅವನಿಗೆ ಸಹಜವಾದ ಗುಣವಲ್ಲವೇ? ಆದ್ದರಿಂದ ಭಗವಂತನೇ ಹಸುವನ್ನು ಕಾಪಾಡಬೇಕೆಂದು ನಾವು ಕೇಳಿಕೊಂಡರೆ ಅದರಲ್ಲಿ ತಪ್ಪೇನಿಲ್ಲ.

ಶಾಸ್ತ್ರದಲ್ಲಿ ಕರು ಇರುವ ಹಸುವೇ ಬಹಳ ಪವಿತ್ರವಾದುದೆಂದು ಹೇಳಲ್ಪಟ್ಟಿದೆ. ಕರು ಇಲ್ಲದ ಹಸುವಿನ ಹಾಲನ್ನು ನಾವು ಕುಡಿಯಲು ಉಪಯೋಗಿಸಬಹುದೇ ಹೊರತು ಭಗವಂತನ ಅಭಿಷೇಕಕ್ಕೆ ಅದು ಉಪಯೋಗಪಡುವುದಿಲ್ಲ. ಇದರಿಂದ ಇಂಥ ಭಕ್ತಿಭಾವನೆ ಉಂಟಾಯಿತೆಂದರೆ, ಅದು ಪ್ರತಿದಿನವೂ ಹೆಚ್ಚಾಗುತ್ತಲೇ ಇರುವುದು. ಇಂಥ ಭಕ್ತಿ ಎನ್ನುವ ಹಸುವನ್ನು ಭಗವಂತನು ಕಾಪಾಡಬೇಕೆಂದು ಶಂಕರರು ಕೇಳಿಕೊಂಡರು.

ಭಗವತ್ಪಾದರು ಭಕ್ತಿಯನ್ನು ಕುರಿತು ಇಷ್ಟು ವಿವರವಾಗಿ, ಭಕ್ತಿ ಎಲ್ಲರಿಗೂ ಯಾವಾಗಲೂ ಇರಬೇಕೆಂದು ಪ್ರಾರ್ಥನೆ ಮಾಡಬೇಕಾದರೆ, ಅದಕ್ಕೆ ಅವರು ಎಷ್ಟು ಮುಖ್ಯತ್ವವನ್ನು ಕೊಟ್ಟಿರಬೇಕೆನ್ನುವುದನ್ನು ನಾವು ತಿಳಿದುಕೊಳ್ಳಬಹುದು. ಇದರಲ್ಲಿ ಮುಖ್ಯತ್ವವಿಲ್ಲವೆಂದರೆ ಭಕ್ತಿಯನ್ನು ಭಗವತ್ಪಾದರು ಇಷ್ಟು ಪ್ರಶಂಶಿಸಬೇಕಾಗಿಲ್ಲ.

ನಾವೆಲ್ಲರೂ ಮನುಷ್ಯರಾಗಿ ಹುಟ್ಟಿದ್ದೇವೆ. ಮನುಷ್ಯರಾಗಿ ಹುಟ್ಟಿದುದರಿಂದ, ಸುಖ ಉಂಟಾಗಬೇಕೆಂದು ನಾವು ಯಾವಾಗಲೂ ಯೋಚಿಸುತ್ತಿರುತ್ತೇವೆ. ಎಲ್ಲರೂ ಯಾವುದೊಂದು ಕೆಲಸವನ್ನು ಮಾಡುವಾಗಲೂ ಸುಖವಾಗಬೇಕೆಂದು ಯೋಚಿಸುವರೇ ಹೊರತು ದುಃಖವಾಗಲೆಂದು ಯಾರೂ ಯೋಚಿಸುವುದಿಲ್ಲ. ನಾವು ಇಂದು ಬಂದು ನಾಳೆಯೇ ಹೋಗುವ ಸುಖವನ್ನು ಅಪೇಕ್ಷಿಸುವುದಿಲ್ಲ. ಯಾವಾಗಲೂ ಸುಖವಿದ್ದರೇನೇ ಒಳ್ಳೆಯದೆಂದು ಭಾವಿಸುತ್ತೇವೆ. ನಾವು ಮಾಡುವ ಕರ್ಮಗಳಿಂದ ಶಾಶ್ವತವಾದ ಸುಖ ದೊರೆಯುವುದೇ ಎಂದರೆ ನಾವೇ ಶಾಶ್ವತವಾಗಿಲ್ಲದಿರುವಾಗ ಸುಖ ಹೇಗೆ ಶಾಶ್ವತವಾಗಿರಲು ಸಾಧ್ಯ? ಆದ್ದರಿಂದ ನಮಗೆ ಶಾಶ್ವತವಾದ ಸುಖ ದೊರೆಯಿತೆಂದು ಹೇಳಲಾಗುವುದಿಲ್ಲ.

ಹಳೆಯ ಗ್ರಂಥಗಳನ್ನು ನೋಡಿದರೂ, ಮಹಾತ್ಮರ ಚರಿತ್ರೆಗಳನ್ನು ಕೇಳಿದರೂ, ``ಶಾಶ್ವತವಾದ ಸುಖ ಉಂಟು; ನಾವು ಅನುಭವಿಸುವುದಕ್ಕಿಂತಲೂ, ದುಃಖವೇ ಇಲ್ಲದ, (ಹೆಚ್ಚಾದ) ಶಾಶ್ವತವಾದ ಸುಖ ಉಂಟು'' ಎನ್ನುವುದು ತಿಳಿಯುತ್ತದೆ. ಹಾಗೆ ನಮಗೆ ತಿಳಿಯುವಾಗ ಅಂಥ ಸುಖವನ್ನು ನಾವೂ ಅಪೇಕ್ಷಿಸುತ್ತೇವೆ.

ನಾವು ಮಾಡುವ ಸಾಧನೆಗಳಿಂದ ಅಂಥ ಸುಖ ಬರುವುದಿಲ್ಲವೆನ್ನುವುದನ್ನೂ ನಾವು ನೋಡುತ್ತಿದ್ದೇವೆ. ``ನಮ್ಮ ಪ್ರಯತ್ನದಿಂದ ಆ ಸುಖ ನಮಗೆ ದೊರೆಯುವುದಿಲ್ಲವೆನ್ನುವಂತಿದೆ. ಆದ್ದರಿಂದ ಅದಕ್ಕಾಗಿ ಚಿರಂತನ ಮಾರ್ಗವಿದ್ದರೆ ಹೇಳಿ, ಪ್ರಯತ್ನ ಪಡುತ್ತೇವೆ'' ಎಂದು ನಾವೇ ಕೇಳುವೆವು. ನಾವು ಎದುರು ನೋಡಿದುದು ಸಫಲವಾಗದೆ ಇರುವುದಕ್ಕೆ ಕಾರಣ ನಮ್ಮ ಪ್ರಯತ್ನ ಅದಕ್ಕೆ ಸರಿಯಾಗಿಲ್ಲದೆ ಇರುವುದೆಂದೇ ಹೇಳಬೇಕು.

\begin{shloka}
``ನ್ಯಗ್ರೋಧಬೀಜಮುಪ್ತ್ವಾಶೋಚನ್ನಿವ ನಾಮ್ರಮಸ್ಯೇತಿ''
\end{shloka}

(ಆಲದ ಬೀಜವನ್ನು ಹಾಕಿ ಮಾವಿನ ಹಣ್ಣು ಆಗಲಿಲ್ಲವೆಂದು ಚಿಂತಿಸಿದಂತೆ....)

ಆಲದ ಬೀಜವನ್ನು ಹಾಕಿ ನಾವು ಮಾವಿನಕಾಯಿ ಬರುವುದೆಂದು ಭಾವಿಸಿದರೆ ಅದು ಹೇಗೆ ಬರುವುದು? ಮಾವಿನಕಾಯಿ ಬರಬೇಕಾದರೆ ಮಾವಿನ ವಾಟೆಯನ್ನು ನಾವು ಹಾಕಿರಬೇಕು, ಆದರೆ ಮಾವಿನ ವಾಟೆಯನ್ನು ನಾವು ಹಾಕೇ ಇಲ್ಲ. ಅದೇ ರೀತಿ ನಾವು ಮಾಡಿದ ಒಂದೊಂದು ಕರ್ಮವೂ ಹೇಗಿದೆ?

\begin{shloka}
``ಯತ್ತದಗ್ರೇ ವಿಷಮಿನ ಪರಿಣಾಮೇಽಮೃತೋಪಮಮ್''
\end{shloka}

(ಯಾವುದು ಆರಂಭದಲ್ಲಿ ವಿಷದಂತೆಯೂ, ಕೊನೆಯಲ್ಲಿ ಅಮೃತದಂತೆಯೂ ಇರುವುದೋ)-ಎಂದು ಗೀತೆಯಲ್ಲಿ ಹೇಳಿದೆ.

ಕೊನೆಯಲ್ಲಿ ಸುಖ ಬರುವುದೆಂದುಕೊಂಡು ಕರ್ಮಗಳನ್ನು ಮಾಡುವೆವು. ``ನಾನು ಮಾಡಿದುದೆಲ್ಲವನ್ನೂ ಚೆನ್ನಾಗಿಯೇ ಮಾಡಿದೆನು. ಈಗ ಕಷ್ಟವೇ ಬರುತ್ತಿರುವುದು. ನಾನು ಏನು ಮಾಡಿದರೂ ಎಲ್ಲವೂ ಕಷ್ಟದಲ್ಲೇ ಮುಗಿಯುತ್ತಿದೆ'' ಎಂದು ಮನುಷ್ಯರು ಹಂಬಲಿಸುವುದನ್ನು ನಾವು ನೋಡುತ್ತೇವೆ. ಆದ್ದರಿಂದ ಮಾಡುವ ಪ್ರಯತ್ನಗಳಲ್ಲಿ ಯಾವುದೋ ತಪ್ಪು ಇದೆ ಎನ್ನುವುದನ್ನು ನಾವು ತಿಳಿಯಬೇಕು. ನಮ್ಮ ಪ್ರಯತ್ನಗಳು ಸರಿಯಾದುವಲ್ಲವೆಂದು ತೀರ್ಮಾನವಾದ ಒಡನೆ ಅದನ್ನು ನಿವರ್ತಿಸಲು ಯಾವುದಾದರೂ ಸಾಧನವಿದೆಯೋ ಎಂದು ನಾವು ಹುಡುಕಿ ನೋಡಬೇಕು. ನೋಡಿದ ಮೇಲೆ ನಮಗೆ ಸರಿಯಾದ ಪ್ರಯತ್ನದ ಅರಿವು ಉಂಟಾಗುವುದು. ಒಬ್ಬನು ತನ್ನ ಇಂದ್ರಿಯಗಳ ಮೂಲಕ ಪ್ರತ್ಯಕ್ಷ ಪ್ರಮಾಣದಿಂದಲೂ, ದೊಡ್ಡವರು ಹೇಳುವ ಶಬ್ದವೆನ್ನುವ ಶಬ್ದ ಪ್ರಮಾಣದಿಂದಲೂ ಅಂಥ ಅರಿವನ್ನು ಪಡೆಯಬಹುದು. ಯಾವ ದಾರಿಯಲ್ಲಿ ಹೋದರೆ ಶಾಶ್ವತವಾದ ಸುಖವನ್ನು ಸಾಧಿಸಬಹುದೆನ್ನುವುದನ್ನು ನಮಗೆ ದಾರಿ ತೋರಿಸಲು ಒಂದು ಗ್ರಂಥವಿದೆ, ಅಂಥ ಗ್ರಂಥಕ್ಕೆ `ವೇದ'ವೆಂದು ಹೆಸರು.

\begin{shloka}
``ಪ್ರತ್ಯಕ್ಷೇಣಾನುಮಿತ್ಯಾ ವಾ ಯಸ್ತೂಪಾಯೋ ನ ಬುಧ್ಯತೇ |\\
ಏನಂ ವಿದನ್ತಿ ವೇದೇನ ತಸ್ಮಾತ್ ವೇದಸ್ಯ ವೇದತಾ ||''
\end{shloka}

(ಯಾವ ಸಾಧನವನ್ನು ಪ್ರತ್ಯಕ್ಷದಿಂದಾಗಲಿ, ಅನುಮಾನದಿಂದಾಗಲಿ ತಿಳಿಯಲು ಆಗುವುದಿಲ್ಲವೋ ಅದನ್ನು ವೇದದಿಂದ ತಿಳಿಯಲಾಗುವುದು; ಆದ್ದರಿಂದಲೇ ವೇದಕ್ಕೆ `ವೇದ'ವೆನ್ನುವ ಸ್ಥಾನ)

ಯಾವ ವಸ್ತುವನ್ನು ನಾವು ಇಂದ್ರಿಯಗಳಿಂದಲೋ, ಮನಸ್ಸಿನಿಂದಲೋ ತಿಳಿಯಲು ಆಗುವುದಿಲ್ಲವೋ, ಅಂಥ ವಸ್ತುವನ್ನು ತಿಳಿಸುವುದರಿಂದ ಅದಕ್ಕೆ ವೇದವೆಂದು ಹೆಸರು. ಆ ವೇದ ಯಾರಿಗೆ ನಿತ್ಯವಾದ ಸುಖವನ್ನು ಅನುಭವಿಸುವ ಯೋಗ್ಯತೆ ಇದೆ ಎನ್ನುವುದನ್ನು ಹೇಳುತ್ತದೆ.

\begin{shloka}
``ತಮೇವ ವಿದಿತ್ವಾಽತಿಮೃತ್ಯುಮೇತಿ ನಾನ್ಯಃ ಪಂಥಾ ವಿದ್ಯತೇಽಯನಾಯ''
\end{shloka}

(ಅವನನ್ನು ತಿಳಿದೇ ಮೃತ್ಯುವನ್ನು ದಾಟುವನು, ಬೇರೆ ದಾರಿ ಇಲ್ಲ.)

-ಎಂದು ಹೇಳಿದಂತೆ ಪರಮಾತ್ಮನ ಸಾಕ್ಷಾತ್ಕಾರವನ್ನು ಪಡೆದರೇನೇ ನಿತ್ಯಸುಖದೊರೆಯುವುದು. ಅಲ್ಲದೆ, ಈ ಹುಟ್ಟು ಸಾವು ಅವನನ್ನು ಬಿಟ್ಟು ಹೋಗುವುದು. ಹಾಗೆ ಭಗವಂತನ ಸಾಕ್ಷಾತ್ಕಾರ ದೊರೆಯಲಿಲ್ಲವೆಂದರೆ ಅವನಿಗೆ ನಿತ್ಯ ಸುಖ ದೊರೆಯುವುದಿಲ್ಲವೆಂದು ವೇದ ಹೇಳಿದೆ. ಅಂಥ ಪರಮಾತ್ಮನನ್ನು ಏನು ಮಾಡಿದರೆ ನಾವು ಅರಿಯಬಹುದು?

\begin{shloka}
``ಆತ್ಮಾ ವಾ ಅರೇ ದ್ರಷ್ಟವ್ಯಃ ಶ್ರೋತವ್ಯೋ\\
ಮಂತವ್ಯೋನಿದಿಧ್ಯಾಸಿತವ್ಯಃ |''
\end{shloka}

([ಪ್ರಿಯಳೇ!) ಆತ್ಮನನ್ನು ನೋಡಬೇಕು, ಅದನ್ನು ಕುರಿತು ಕೇಳಬೇಕು, ಚಿಂತನೆ ಮಾಡಬೇಕು, ಒಂದು ಸ್ಥಿರ ಮನಸ್ಸಿನಿಂದ ಧ್ಯಾನ ಮಾಡಬೇಕು]

ಮೊದಲು ಚಿಂತನೆಗಳಿರುವ ದಾರಿಯಲ್ಲಿ ಒಂದು ದಾರಿಯನ್ನು ಆರಿಸಿಕೊಳ್ಳಬೇಕು. ಆತ್ಮನನ್ನು ಕುರಿತು ಹೇಳುವ ಗ್ರಂಥವನ್ನು ಓದಲು ಪ್ರಯತ್ನಪಡಬೇಕು. ಏಕೆ ಓದಬೇಕೆಂದರೆ ಓದಿದರೇನೇ ಅಲ್ಲಿ ಹೇಳಿರುವ ರೀತಿಯಲ್ಲಿ ನಡೆಯಲು ಯೋಗ್ಯತೆ ಬರುವುದು. ಏನನ್ನೂ ತಿಳಿಯದೆ ನಾವಾಗಿಯೇ, ``ಹೀಗೆಯೂ ಇರಬಹುದು, ಹಾಗೆಯೂ ಇರಬಹುದು'' ಎಂದು ಕೊಂಡರೆ ಕೊನೆಗೆ-

\begin{shloka}
``ನಾಯಂ ಲೋಕೋಽಸ್ತಿ ನ ಪರೋ ನ ಸುಖಂ ಸಂಶಯಾತ್ಮನಃ''
\end{shloka}

`ಸಂದೇಹ ಪಡುವವನಿಗೆ ಈ ಲೋಕವೂ ಇಲ್ಲ, ಪರಲೋಕವೂ ಇಲ್ಲ, ಸುಖವೂ ಇಲ್ಲ' ಎಂದು ಹೇಳಿದೆ, ಆದ್ದರಿಂದ ನಾಸ್ತಿಕನಾಗಿದ್ದರೂ ಪರವಾಗಿಲ್ಲವೆಂದು ಹೇಳಬಹುದೆನಿಸುತ್ತದೆ! ಸರಿಯಾದ ನಾಸ್ತಿಕನಾಗಿಯೂ ಇರದೆ, ಪರಲೋಕವಿದೆಯೆಂದು ಒಂದು ದಿನ ಹೇಳಿಕೊಂಡು, ಇನ್ನೊಂದು ದಿನ ಸಿನಿಮಾಗೂ ಹೋಗುತ್ತಾ ಕಾಲಕಳೆಯುತ್ತಿದ್ದರೆ ಅವನ ನಡತೆ ಯಾವುದಕ್ಕೂ ಉಪಯೋಗವಾಗುವುದಿಲ್ಲ. ಆದ್ದರಿಂದ ಒಂದು ರೀತಿಯನ್ನು ತಿಳಿದುಕೊಂಡರೇನೇ ಸರಿಯಾಗಿ ನಡೆಯಲು ಸಾಧ್ಯ. ಮೊದಲು ಆತ್ಮ ಸ್ವರೂಪ ಯಾವ ಗ್ರಂಥದಲ್ಲಿ ಹೇಳಲ್ಪಟ್ಟಿದೆಯೋ, ಆ ಗ್ರಂಥವನ್ನು ಓದಬೇಕು. ಓದುವ ರೀತಿಯೂ ಇದೆ, ಏನೋ ಸುಮ್ಮನೆ ಪುಸ್ತಕವನ್ನು ತಿರುವಿ ಹಾಕಿದರೆ ಅದು ಯಾವುದಕ್ಕೂ ಪ್ರಯೋಜನವಾಗುವುದಿಲ್ಲ.

\begin{shloka}
``ಪುಸ್ತಕಸ್ಥಾ ಚ ಯಾ ವಿದ್ಯಾ ಪರಹಸ್ತೇ ಚ ಯುದ್ಧನಮ್ |\\
ಕಾರ್ಯಕಾಲೇ ಸಮಾಯತೇ ನ ಸಾ ವಿದ್ಯಾ ನ ತದ್ಧನಮ್ ||
\end{shloka}

(ಪುಸ್ತಕದಲ್ಲಿ ಮಾತ್ರ ಇರುವ ವಿದ್ಯೆ, ಬೇರೆಯವರ ಹತ್ತಿರವಿರುವ ಹಣ ಸಮಯ ಬಂದಾಗ ಆ ವಿದ್ಯೆಯಾಗಲಿ, ಹಣವಾಗಲಿ ಉಪಯೋಗಕ್ಕೆ ಬರುವುದಿಲ್ಲ.)

``ಒಬ್ಬನಿಗೆ ಹತ್ತು ಸಾವಿರ ರೂಪಾಯಿ ಸಾಲ ಕೊಟ್ಟಿದ್ದೇನೆಂದು ಹೇಳಿದರೆ, ಆ ರೂಪಾಯಿ ನಮ್ಮ ರೂಪಾಯಿಯೆಂದು ನಾವು ಭಾವಿಸಕೂಡದು. ಏಕೆಂದರೆ ಅವನು ಮತ್ತೆ ಕೊಟ್ಟರೆ ಮಾತ್ರ ಅದು ನಮ್ಮ ಹಣ, ಅವನು ಕೊಡದೆ ಇದ್ದರೆ ಅದು ಹೇಗೆ ನಮ್ಮ ಹಣವಾಗುವುದು? ನಾವು ಕೋರ್ಟಿನಲ್ಲಿ ಕೇಸು ಹಾಕಿದರೂ ಅವನು ತಿರುಗಿ ಕೊಡಲಿಲ್ಲವೆಂದರೆ ಆಗಲೂ ಅದು ನಮ್ಮ ಹಣವಲ್ಲ. ಅದೇ ರೀತಿ ಪುಸ್ತಕವನ್ನು ಓದುವಾಗ, ``ಪುಸ್ತಕದಲ್ಲಿ ಎಲ್ಲಾ ವಿಷಯಗಳೂ ಇವೆ. ಆ ಪುಸ್ತಕವನ್ನು ನೋಡಿದರೆ ನಮಗೆ ಎಲ್ಲಾ ತಿಳಿದು ಬಿಡುತ್ತದೆ'' ಎಂದು ಹೇಳುವವರಿಗೆ ಅಂಥ ಓದು ಪ್ರಯೋಜನ ಪಡುವುದಿಲ್ಲ. ಅನುಭವವಿಲ್ಲದ ಓದೇ ಅದು. ಅನುಭವಕ್ಕೆ ಬರುವ ವಿದ್ಯೆಯನ್ನು ಅನುಭವಸ್ಥರೇ ಹೇಳಿಕೊಡಬೇಕು. ಅನುಭವವಿಲ್ಲದವರು ಹೇಳಿಕೊಟ್ಟರೆ ``ಅವರಿಗೇ ಅನುಭವವಿಲ್ಲ. ಅವರು ಪುಸ್ತಕದಲ್ಲಿರುವುದನ್ನು ಮಾತ್ರ ಹೇಳಿಕೊಡುತ್ತಾರೆ. ಅದನ್ನು ನಾನೇ ಓದಿಕೊಳ್ಳುವೆನು'' ಎಂದು ತೋರುತ್ತದೆ. ಒಂದು ಯಂತ್ರವನ್ನು ಕುರಿತು ಹೇಳಿಕೊಡುವಗ ಯಾವುದಾದರೂ ಮೊಳೆಯನ್ನು ಹಾಕಬೇಕಾದರೂ ಅದಕ್ಕೆ ಅನುಭವ ಬೇಕು. ಆದ್ದರಿಂದ ವಿಷಯವನ್ನು ಕೇಳಬೇಕೆಂದರೆ ಒಬ್ಬ ಗುರುವಿನ ಮುಖಾಂತರ ಕೇಳಬೇಕು.

ಸಾಧಾರಣವಾಗಿ ಈ ಕಾಲದಲ್ಲಿ ಒಬ್ಬರು ಯಾವುದನ್ನಾದರೂ ಹೊಸದಾಗಿ ಕಂಡುಹಿಡಿದರೆಂದರೆ, ಅವರನ್ನು ಗೌರವಿಸುವವರು ದೊಡ್ಡ ಹಣವಂತರಾಗಿಯೋ, ದೊಡ್ಡ ಅಧಿಕಾರದಲ್ಲಿರುವವರೋ ಆದರೆ ಅವರು ಆ ವ್ಯಕ್ತಿಗೆ `ಚೆಕ್' ಪುಸ್ತಕವನ್ನು ಕೊಟ್ಟು ``ನೀವು ಎಷ್ಟು ಬೇಕಾದರೂ ಖರ್ಚು ಮಾಡಿ, ಧಾರಾಳವಾಗಿ ತೆಗೆದುಕೊಳ್ಳಿ'' ಎಂದು ಕೇಳಿದರೆ ಆ ವ್ಯಕ್ತಿಗೆ ಯಾವ ವಿಧವಾದ ಚಿಂತೆಯೂ ಇರುವುದಿಲ್ಲ. ಆ ವ್ಯಕ್ತಿ ಸಂಶೋಧನೆಯಲ್ಲಿಯೇ ತನ್ನ ಮನಸ್ಸನ್ನು ಇಡಬಹುದು. ಅದೇ ರೀತಿ ವೇದಾಂತ ವಿಚಾರವನ್ನು ಮಾಡುವಾಗ ನಾವು ನಿಶ್ಚಿಂತರಾಗಿರಬೇಕು.

ಅಂಥ ನಿಶ್ಚಿಂತತೆಯ ಸ್ಥಿತಿ ಬರಬೇಕಾದರೆ ಮನಸ್ಸಿನ ಸ್ಥಿರತೆ ಅವಶ್ಯಕ; ಏಕೆಂದರೆ, ಯಾವು ಯಾವುದೋ ವಿಷಯಗಳು ನಮ್ಮ ಬಳಿಗೆ ಬಂದು ಸೇರಿದರೆ, ``ಅವುಗಳು ನಮಗೆ ಸಂಬಂಧಪಟ್ಟವುಗಳಲ್ಲ'' ಎನ್ನುವಂತೆ ನಾವು ಒಂದು ಯೋಗ್ಯತೆಯನ್ನು ಸಂಪಾದಿಸಿಕೊಳ್ಳಬೇಕು. ಅದಕ್ಕೆ ದಾರಿಯಾಗಿ ಶಮ, ದಮ, ಉಪರತಿ, ತಿತಿಕ್ಷಾ, ಶ್ರದ್ಧಾ, ಸಮಾಧಾನ ಎನ್ನುವ ಎಷ್ಟೋ ಸಾಧನಗಳು ಹೇಳಲ್ಪಟ್ಟಿವೆ. ಮುಖ್ಯವಾಗಿ ತೀವ್ರವಾದ ಮುಮುಕ್ಷತ್ವವೆನ್ನುವುದು ಬೇಕು. ``ನನ್ನ ಈ ಲಕ್ಷ್ಯಕ್ಕೆ ವಿರೋಧವಾಗಿ ಯಾವುದು ಬಂದರೂ ಅದಕ್ಕೆ ನಾನು ಒಳಗಾಗುವುದಿಲ್ಲ'' ಎನ್ನುವ ಭಾವನೆ ಇರಬೇಕು. ಇದೇ ತೀವ್ರವಾದ ಮುಮುಕ್ಷುತ್ವವಾಗುವುದು. ಅಂಥ ತೀವ್ರವಾದ ಮುಮುಕ್ಷುತ್ವ ಹೇಗೆ ಬರುವುದು? ``ಆತ್ಮ ಒಂದೇ ಸ್ಥಿರನಾಗಿ, ಸುಖಸ್ವರೂಪನಾಗಿರುವುದು. ಬೇರೆ ಯಾವುದೂ ಸುಖವಲ್ಲ'' ಎನ್ನುವ ಭಾವನೆ ಬಂದರೆ ತೀವ್ರವಾದ ಮುಮುಕ್ಷುತ್ವ ಬರುವುದು. ಹಾಗಿಲ್ಲದಿದ್ದರೆ, ಆತ್ಮವಿಚಾರಮಾಡುತ್ತಿರುವಾಗಲೇ, ಇತರ ಪ್ರಾಪಂಚಿಕ ವಿಷಯಗಳಿಂದ ಸುಖ ದೊರೆಯುತ್ತದೆನ್ನುವ ತಪ್ಪಾದ ಭಾವನೆಯಿಂದಾಗಿ ಒಂದು ಹತ್ತು ನಿಮಿಷಕಾಲ ಪ್ರಪಂಚವನ್ನು ಕುರಿತು ಅಂಥವನು ಚಿಂತಿಸುವನು. ಮತ್ತೆ ಆತ್ಮವಿಚಾರವನ್ನು ಮಾಡುವನು. ಹೀಗೆ ಚಂಚಲವಾಗಿರುವವನಿಗೆ ಆತ್ಮಸಾಕ್ಷಾತ್ಕಾರ ದೊರೆಯುವುದಿಲ್ಲ. ಆದ್ದರಿಂದ ವೈರಾಗ್ಯ ಬೇಕೆಂದು ಹೇಳಲ್ಪಟ್ಟಿದೆ. ಆದಕ್ಕಾಗಿ ಸುಮ್ಮನೆ ಪುಸ್ತಕವನ್ನು ಮಾತ್ರ ಓದಿದರೆ ಆ ಜ್ಞಾನ ಉಪಯೋಗವಾಗುವುದಿಲ್ಲ.

ತೀವ್ರವಾದ ಮುಮುಕ್ಷತ್ವ ತೀವ್ರವಾದ ವೈರಾಗ್ಯ ಇದ್ದರೇನೇ ವೇದಾಂತ ವಿಚಾರಕ್ಕೆ ಯೋಗ್ಯತೆ ಉಂಟಾಗುತ್ತದೆ. ಹೀಗೆ ವಿಚಾರ ಮಾಡುವುದೇ ನಿಜವಾದ ವಿಚಾರವಾಗುವುದು. ವಿಚಾರ ಮಾಡಿದಮೇಲೆ, ಮಾಡಿದ ವಿಚಾರದ ವಸ್ತುವನ್ನು ಚೆನ್ನಾಗಿ ಮನಸ್ಸಿನಲ್ಲಿಟ್ಟುಕೊಳ್ಳಬೇಕು. ``ಪುಸ್ತಕದಲ್ಲಿ ಹೇಳಿರುವುದನ್ನು ಒಪ್ಪಿಕೊಳ್ಳುವಂತಿಲ್ಲವಲ್ಲಾ'' ಎಂದು ಹೊರನೋಟದಿಂದ ನೋಡುವವನು ಹೇಳಿದರೆ ಅವನು ಅದನ್ನು ಚೆನ್ನಾಗಿ ಪರಿಶೀಲಿಸಬೇಕು. ಇದಕ್ಕಾಗಿ ಪುಸ್ತಕದಲ್ಲಿರುವುದನ್ನು ಹಾಗೆಯೇ ತೆಗೆದುಕೊಳ್ಳಬೇಕೆಂದಿಲ್ಲ. ತತ್ತ್ವಾನ್ವೇಷಣೆ ಮಾಡುವಾಗ ವೇದವನ್ನು ಮೂಲಭೂತವಾಗಿಟ್ಟುಕೊಂಡು ಅನ್ವೇಷಣೆ ಮಾಡಬೇಕು.

ಆದ್ದರಿಂದಲ್ಲೆ ಶ್ರವಣ, ಮನ ಎನ್ನುವುದು ದಾರಿಯಾಗಿ ಹೇಳಲ್ಪಟ್ಟಿವೆ. ಮನನ ಮಾಡಿದ ಮೇಲೆ ``ಹೀಗೆಯೇ ವಸ್ತು ಇರುವುದು'' ಎಂದು ತೀರ್ಮಾನವಾದ ನಂತರ ನಾವು ಆ ಧ್ಯಾನದಲ್ಲೇ ಇರಬೇಕು. ಧ್ಯಾನದಲ್ಲಿ ಬಹ್ಳ ಹೊತ್ತು ಇದ್ದಮೇಲೆ ಸಾಕ್ಷಾತ್ಕಾರ ಉಂಟಾಗುತ್ತದೆ. ಇಂಥ ತೀವ್ರವಾದ ಮುಮುಕ್ಷುತ್ವವೂ, ತೀವ್ರವಾದ ವೈರಾಗ್ಯವೂ ಇದ್ದರೆ ಈ ಶಮ, ದಮ, ಉಪರತಿ, ತಿತಿಕ್ಷಾ, ಶ್ರದ್ಧಾ, ಸಮಾಧಾನ ಎನ್ನುವ ಯೋಗ್ಯತೆಗಳು ಉಂಟಾಗುತ್ತವೆ. ಈ ಸಾಧನೆಗಳು ಇದ್ದರೇನೇ ವಿಚಾರ ಮಾಡುವುದಕ್ಕೆ ಯೋಗ್ಯತೆ ಉಂಟಾಗುತ್ತದೆ. ಆದ್ದರಿಂದ ಶ್ರವಣ, ಮನನ, ನಿದಿಧ್ಯಾಸನ, ಅನಂತರ ಆತ್ಮ ಸಾಕ್ಷಾತ್ಕಾರ ಎನ್ನುವುದು ಹಂತಹಂತವಾಗಿ ಹೇಳಲ್ಪಟ್ಟಿವೆ.

``ಸಾಧನಕಥನಾವಸರೇ ಸಾಚೀಕುರ್ವನ್ತಿ ವಕ್ತ್ರಾಣಿ'' ಎಂದು ಹೇಳಿದಂತೆ ಕೈಲಾಸವನ್ನು ವರ್ಣಿಸುವಾಗ ಎಲ್ಲರಿಗೂ ಸ್ವಾರಸ್ಯವಾಗಿರುವುದು. ಆದರೆ ಅದಕ್ಕೆ ಬೇಕಾದ ಸಾಧನಗಳನ್ನು ಹೇಳಿದರೆ ಅವರು ತಮ್ಮ ಮುಖಗಳನ್ನು ತಿರಿಗಿಸಿಕೊಂಡು ಹೋಗುವರು.

ಆದ್ದರಿಂದ ಸಾಧನೆ ಮಾಡುವ ಕಾಲದಲ್ಲಿ ನಾವು ವೈರಾಗ್ಯದಂಥ ಯೋಗ್ಯತೆಗಳನ್ನು ಪಡೆದು ಶ್ರೇಯಸ್ಸನ್ನು ಪಡೆಯಬೇಕು.














































































