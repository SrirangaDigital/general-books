
\chapter{Brāhmanism, Buddhism and Mīmāṁsā}\index{Buddhism}\label{chapter6}

\Authorline{-- Sharda Narayanan$^{\ast}$}\footnotetext{*pp.~\pageref{chapter6}\enginline{--}\pageref{chapter6-end}. In: Kannan, K. S and Meera, H. R. (Ed.s) (2021). \textit{Chronology and Causation: Negating Neo-Orientalism.} Chennai: Infinity Foundation India.}

\lhead[{\small\thepage}\quad\small Sharda Narayanan]{}

\begin{flushright}
\textit{(sharda.narayanan@gmail.com)}
\end{flushright}


\vspace{.2cm}

\section*{Abstract}

Western research seems to needlessly target “Brahmans” or priests as power-mongers. Modern life is very different to ancient times and India does not have a problem adapting to contemporary ways, but Western studies seem to find justification in raking up issues in distorted perspective and portray Indian society in a warped manner. Nowhere in Sanskrit treatises do we see exploitation of any community on the divisions of caste as a basis. Western study typically presents a set of data derived not from original research, but by partially re-stating the Mīmāṁsā’s\index{Mimamsa@Mīmāṁsā} own \textit{pūrva-pakṣa}\index{purvapaksa@\textit{pūrva-pakṣa}} (objections) as their modern conclusions, showing the philosophy in poor light. Modern researchers do not grasp the context or present all sides of the arguments. Among several statements of fact, the personal view of the researcher, which actually has no basis and is unwarranted, is slipped in and made to look convincing. The writings of Johannes Bronkhorst\index{Bronkhorst, Johannes} and Sheldon Pollock\index{Pollock, Sheldon} are seen to form damaging conclusions on false, concocted and misrepresented bases. Unless the reader is well-versed in the subject, it is difficult to spot the aberration. This paper attempts to academically evaluate a few issues, notably, \textit{apauruṣeyatva} of Mīmāṁsā.\index{Mimamsa@Mīmāṁsā}

\vspace{-.3cm}

\section*{1. Classical Studies in Modern Times}

Mīmāṁsā is relatively far less studied even by Indian scholars and is still taught in the traditional way, following the rigorous \textit{paramparā} of many centuries, where teacher and student converse only in Sanskrit in order to retain the high level of precision in terminology. Unfortunately for Sanskrit studies, awareness levels are so low today among the general public, that we think that anybody can pick up any portion of study as part of university research and then claim expertise in it. It is apparent that modern methodology is considered adequate to delve into any subject and form conclusions based on the scholar’s own analysis.

Native to India and also English-speaking, we are yet not native English speakers and have our limitations of time and interest to devote to studying Western Indologists’ views on Mīmāṁsā. Most traditional scholars well-versed in Mīmāmsā are not well-versed in English and do not read western writing in high-flown language; they do not contradict western writing and this perhaps leads to the impression that western study is on the right track. My own interest has been in studying the Sanskrit texts themselves, in good measure owing to the beautiful ring of the language; study of Physics, not Western humanities, formed my major in University along with Sanskrit studies which I continued to pursue into research level. This paper shall concern itself with observations on methodology and perspective on a few broad issues in Mīmāṁsā.

\vspace{-.3cm}

\section*{2. Basic Principles of Mīmāṁsā}

The Veda--s contain hymns and prose passages discussing past events, secrets of the cosmos and liturgical procedures. Mīmāṁsā as a science has very ancient origins, formulated to methodically interpret the Vedic utterances in the proper, approved manner (\textit{āmnātaḥ}, as Bhartṛhari\index{Bhartrhari@Bhartṛhari} says in \textit{Vākyapadīya}\index{Vakyapadiya@\textit{Vākyapadīya}} I. 2) as one can easily fall into a pattern of misinterpretation and wrong application. Across a large geographical expanse over many generations, it is not possible to maintain uniformity of interpretation of scripture without rules to follow. In the context of \textit{yajña},\index{yajna@\textit{yajña}} the rules of procedure are often stated clearly in Veda for a particular instance and have to be extrapolated for other rituals. Mīmāṁsā rules become crucial in such cases, where rules stated for the \textit{prakṛti},\index{prakrti@\textit{prakṛti}} i.e. the ritual in context of discussion, have to be adapted for \textit{vikṛti},\index{vikrti@\textit{vikṛti}} or extended application.

As one of the greatest writers of modern times on Mīmāṁsā,\index{Mimamsa@Mīmāṁsā} Prof. K. T. Pandurangi\index{Pandurangi, K. T.} writes -

\begin{myquote}
“The area of semantics is deeply probed by philosophy, psychology, anthropology and other human sciences that deal with the mind. It is the behavior of the mind that is reflected in the behavior of languages. Mīmāṁsā philosophy that gives utmost importance to sabda\index{sabda@\textit{sabda}}-pramana is deeply concerned with language. It studies all aspects concerning the import of language. Its studies belong to a period of our intellectual history when psychology, sociology, human sciences, etc, were not bifurcated from philosophy. Therefore its handling of the problems of language involves the approach of these disciplines also. Mīmāṁsā reveals remarkable insight on these aspects.
\end{myquote}

\begin{myquote}
“Purva-Mīmāṁsā considers language to be autonomous at three levels: (i) the relation between word and meaning, (ii) sentence-meaning, (iii) the purport of a passage or discourse. The relation between word and meaning is natural. It is not fixed by any agency or God. When a meaningful expression is expressed its meaning is also expressed. It is comprehended through elders’ conversation from generation to generation.” 

~\hfill (Pandurangi 2006: xx, 132)
\end{myquote}

So on one hand Mīmāṁsā has a marvelous science of sentence interpretation and on the other a system of philosophy. It advocates true self-understanding, maturity of knowledge gained from the study of \textit{Upaniṣad}-s in order to live a contented life without superstition and to develop renunciation from the inevitably ephemeral pleasures of the world. Knowledge, a disciplined life, fulfilling obligations and contentment are advocated for their intrinsic value in giving peace of mind leading to salvation. Mīmāṁsā promises no pearly gates, golden thrones or rows of virgins welcoming the hero with open arms in the afterlife.

\begin{myquote}
“Parthasarathi describes the state of Mokṣa\index{moksa@\textit{mokṣa}} as ‘\textit{svastha}’, i.e. the state of remaining unto himself. ……... At this stage, not only the pleasure and pain are eliminated but all other qualities such as desire, initiative, religious merit, etc, are also eliminated.” 

~\hfill (Pandurangi 2006: 567)
\end{myquote}

\begin{myquote}
“The role of \textit{ātmajñāna} is stated in the \textit{Mīmāṁsā} terminology as \textit{kratvartha}. The knowledge of \textit{ātman} as distinct from the body enables a person to undertake sacrifices which yield results in the other world and after this birth.” 

~\hfill (Pandurangi\index{Pandurangi, K. T.} 2006: 569)
\end{myquote}

The \textit{karma}\index{karma@\textit{karma}}\index{Law of Karma} theory is very important to explain the vagaries of life. (The concept of divine retribution in one way or other is held almost everywhere in the world.) The \textit{ātman} is defined ultimately by volition, which is its chief attribute. So, as M. Hiriyanna\index{Hiriyanna, M.} points out, Indian philosophy lays greater emphasis on free will than fate and is not deterministic in its outlook (Hiriyanna 1994: 109). The individual soul takes responsibility for its actions and the just deserts that follow. Emancipation, \textit{mokṣa}\index{moksa@\textit{mokṣa}} is when the soul revels in its innate glory, transcending the sorrows of the world. The saying, \textit{muktir naija-sukhānubhūtir amalā} from a verse by Śrī Vyāsatīrtha\index{Vyasatirtha@Vyāsatīrtha} (Archak 2004: 22)is well-known in Dvaita\index{Dvaita Vedanta@Dvaita Vedānta} Vedānta tradition, meaning that \textit{mukti} is the state of bliss natural to \textit{ātman}\index{atman@\textit{ātman}} without external joy or pain. This is also the view accepted by Mīmāṁsā\index{Mimamsa@Mīmāṁsā} (Dravid and Narayanan 2016:322).

\vspace{-.3cm}

\section*{3. The Western Viewpoint}

Western research seems to needlessly target “\textit{Brāhmans}” or priests as power-mongers. Priests and Vedic scholars formed a minority of the population. In a society formed of producers such as farmers, artisans and craftsmen, administrative servants, military soldiers, feudal lords, traders, rulers, teachers and priests, how much of the economy could have been appropriated by the “Brāhmaṇ-s”? “\textit{Brāhmaṇa-}s”\index{brahmana@\textit{brāhmaṇa}} as the word indicates, are the priests concerned with Vedic – rituals and the study and teaching of Veda and ancillary subjects. It is derived from the word “Brāhmaṇa-s” – the prose passages in the Veda--s that deal with knowledge of “Brahman”. Modern studies which are directed towards portraying them as imposing Vedic religion on society seem to be only divisive in their intent as there is no basis for such a view. The priests formed a part of society with its full acceptance and support. That the \textit{Brāhmaṇa} priest, despite the disciplined life imposed on him to pursue studies of the Veda--s was held in great esteem and placed at the apex of society goes to show that classical Indian society was knowledge-driven and did not hold military or economic prowess as supreme. In the frequent military skirmishes that occurred over collection of tax and tributes, the \textit{Brāhmaṇa} living quietly in the ashrams or towns, teaching youngsters the traditional learning, conducting temple worship or Vedic rituals was spared the atrocities of war and not murdered for material gain. This was again a reflection of the respect the Veda--s were held in, which found place in the \textit{Brāhmaṇa}\index{brāhmaṇa@\textit{brāhmaṇa}} as a transferred epithet.

This view may appear too idealistic but in fact points out the basic principles behind \textit{varṇāśrama vidhāna}\index{varna@\textit{varṇa}} in its origins. Dr. Pollock\index{Pollock, Sheldon} has traced the ills of modern caste problems in India to the tenets of Mīmāṁsā\index{Mimamsa@Mīmāṁsā} but there is no basis for such allegations. Firstly, these are problems of economics and lack of education, not caste problems as the Westerner may like to believe. Keeping economic assets within the community and cultural affinity rather than religious sentiment was behind caste feelings in bygone times, as perhaps even today. Secondly, the tenets of Mīmāṁsā have nothing to do with caste equations - that was the purview of Dharma śāstra-s which were descriptive codes rather than prescriptive formulations as is widely and wrongly believed. Prabhakara Mishra, one of the greatest Mīmāṁsaka\textit{-}s has famously said that \textit{Brāhmaṇatva} is not a \textit{jāti}\index{jati@\textit{jāti}} but an \textit{upādhi}, in the 7th century CE. That is, one is not born a \textit{brāhmaṇa} but becomes one due to circumstances and training from infancy. (Pandurangi\index{Pandurangi, K. T.} 2004: 79). Western scholars tend to ignore such viewpoints,

\vspace{-.3cm}

\subsection*{3.1 Some Examples from Writings of Bronkhorst}\index{Bronkhorst, Johannes}

From “The origin of Mīmāṁsā as a school of thought: a hypothesis” written by Dr.Johannes Bronkhorst - l

\begin{myquote}
“However, Mīmāṁsā...is more than merely the outcome of a continuous development of the ideas and concerns which we find in the ritual Sutras. At some period in its history Mīmāṁsā underwent one or more dramatic breaks with its predecessors, which allowed it to become an independent school of thought.” 

~\hfill (Bronkhorst 2001: 1)
\end{myquote}

There is much speculation on the researcher’s part that is presented here as scholarly study. It is not possible to truly find evidence of any dramatic reason that others did not know, unless we meticulously present data to support a pre-conceived notion. Such research cannot be considered serious study of Mīmāṁsā.

\begin{myquote}
“Mīmāṁsā never fully replaced the ritual traditions of the Vedic schools. We know, for example, that Bhartṛhari a philosopher from the fifth century C.E., though acquainted with Mīmāṁsā, refers for ritual details to the handbook of his own Vedic school, that of Manava-Maitrayaniyas (Bronkhorst 1985; 1989; 105 (375-376)). Other authors explicitly prescribe that sacrifices should adhere to the manuals of their own schools (Deshpande 1999). The Mīmāṁsāsutra itself (2.4.8-9), finally, first records the position according to which there are differences between the rituals in different Vedic schools, then rejects it. All the passages reveal a certain amount of resistance against Mīmāṁsā\index{Mimamsa@Mīmāṁsā} that was apparently felt by a number of orthodox Brahmins, presumably from the very beginning.” 

~\hfill (Bronkhorst\index{Bronkhorst, Johannes} 2001: 3)
\end{myquote}


It appears that the Westerner thinks he understands all aspects of Vedic ritual across the Indian subcontinent, across the centuries. We have already seen that there were different lines of thinking and it would naturally reflect in the practices of different parts of the country. Even today, the great variety underlying our dress, food, festivals, traditions of celebrating events such as weddings, birthdays, childbirth, etc are so diverse. Why should Vedic ritual practices be exactly the same? As in classical dance and music that all follow the same principles of \textit{Nāṭyaśāstra},\index{Natyasastra@\textit{Nāṭyaśāstra}} we see great variety even while adhering to the same basic principles. In the case of Vedic rituals too, there could be differences based on timings, seasons, availability of fruits, vegetables and animal species for sacrifice. The Vedic rituals were also of different kinds, which the Western scholar fails to see - those that were mandatory or obligatory and those that were optional. The routine rituals did not have elaborate arrangements or large expenses. Again, \textit{Kalpa Sūtra}-s\index{Kalpasutras@\textit{Kalpa Sūtra-s}} and \textit{Gṛhya Sūtra}-s\index{Grhyasutras@\textit{Gṛhya Sūtra-s}} defined rituals in addition to those that Mīmāṁsā discusses. So there is bound to be a great dismaying variety which academic scholars cannot understand.

To examine the above quoted passage, are we to understand “Mīmāṁsā” as different to “the Vedic schools” or as part of them? How does the modern scholar define “Mīmāṁsā” as different to other “Vedic schools”? It is impossible to surmise that Bhartṛhari\index{Bhartrhari@Bhartṛhari} followed a particular “handbook,” based on a quotation. How do we know that the sentence he quotes did not find common occurrence in several if not all texts, as was most likely the case? It is well-known that many maxims current in the tradition were quoted by many important writers of \textit{śāstra}; modern researchers cannot use these to date the writers or their texts.

Indian philosophical writing is full of debate; this style of writing using \textit{pūrva--pakṣa} and \textit{siddhānta} in no way endorses the last sentence quoted above from Bronkhorst, as would be apparent to anybody who is well-versed in the method of Sanskrit \textit{śāstra}.

Further, Bronkhorst enquires –

\begin{myquote}
“The schools of philosophy that arose beside Mīmāṁsā believed in the beginninglessness of the universe, to be sure, but they all accepted, unlike Mīmāṁsā,\index{Mimamsa@Mīmāṁsā} the periodic destruction and recreation of the world? Why then did Mīmāṁsā invent and accept this strange set of doctrines? What could the Mīmāṁsakas possibly gain by doing so? Predictably, none of our sources propose any answers, for these doctrines are not presented as new inventions but as eternal truths. But we are entitled to ask what benefit these strange doctrines brought with them.
\end{myquote}

\begin{myquote}
What could be the advantage for the Brahmins concerned in accepting them?” 

~\hfill (Bronkhorst\index{Bronkhorst, Johannes} 2001: 3)
\end{myquote}

It would have been considered equally strange if every single school of thought adhered to the same set of ideas. In a great university such as Nalanda, it is doubtful that all the teachers taught the same philosophy. India was known to have many diverse views in every age as it does even today. A person’s theoretical views may even evolve over his or her own lifetime and influence the practical decisions to varying degrees. Coming to Mīmāṁsā, it was first developed as a science of interpretation of sentence-meaning, not as a complete world view invented, packaged and offered to consumers as might seem to a modern scholar. In accordance with its philosophy of language, the Mīmāṁsā takes this stand on the origin of the Veda--s and the stable state of the world. Most other schools have their own explanation of what happens to the Veda--s during \textit{pralaya.}

In the Mīmāṁsā\index{Mimamsa@Mīmāṁsā} view, since \textit{śabda}\index{sabda@\textit{śabda}} is a real, eternal entity with independent ontological status, it does not undergo dissolution in time of \textit{pralaya} even. When the other philosophies question how anything can escape dissolution at \textit{pralaya}, Mīmāṁsā solution is not to accept the end of the world at all. It takes a bold stance rooted in practicality. It is all very well to single out Mīmāṁsā as unconventional on this issue, but let us not forget that none of us humans have been witness to \textit{pralaya} or creation.

\begin{myquote}
“\textit{sargādipakṣopanyāsaḥ tannirāśca \dev{।} - yadi paramevam? – sargādikāle bhagavataa prajāpatinā sarvameva sthāvarajańgama, dharmādharmau ca sŗṣṭvā ………..vedāśca pratipāditāḥ, …….śabdārthapratipattirvyavahāraśceti\index{sabda@\textit{śabda}} \dev{।}}
\end{myquote}

\begin{myquote}
“\textit{tadapyayuktam – itthambhāve pramāṇābhāvāt \dev{।}” (Śāstradīpikā: 219)}
\end{myquote}

\begin{myquote}
“\textit{na kadācidanīdṛśaṃ jagat \dev{।}”}

~\hfill (Dravid and Narayanan 2016: 292)
\end{myquote}

The tenets of Mīmāṁsā are valid in a constant world. The correct way to interpret this point is to appreciate the Mīmāṁsā’s practical perspective in not taking into account the end of the world. It lays greater emphasis on our existence in this world which surely has its end! It was essential that a school of thought provide a cogent system that did not contradict itself. Rather than a theory of Mīmāṁsā being a sudden upstart that rose to pocket the fees, why does the Westerner not see it as a logical solution offered to defend the Veda--s against irreverent allegations, much to the solace and gratitude of the faithful populace? It was not uncommon for a school of thought to take a particular stand that suited its framework better on logical issues, e.g. accepting different number of \textit{pramāṇa-}s, \textit{samavāya sambandha}, etc.

Bronkhorst’s enquiry into what the Brahmins concerned stood to gain from it is not significant and goes outside the purview of academic study. But it is worth noting how the Western researcher selectively uses the word “Brahman” and the newly fabricated but unexplained “Brahmanism”.

\vspace{-.3cm}

\subsubsection*{Discussion On \textit{Vākyapadīya}}\index{Vakyapadiya@\textit{Vākyapadīya}}

In his paper, “Studies on Bhartṛhari,\index{Bhartrhari@Bhartṛhari} 9: Vākyapadīya 2.119 and the Early History of Mīmāṁsā” at the outset in the Abstract, Bronkhorst\index{Bronkhorst, Johannes} writes

\begin{myquote}
“..Interestingly, Śabara’s\index{Sabarasvamin@Śabarasvāmin} classical work on Mīmāṁsā has abandoned this position, apparently for an entirely non-philosophical reason: the distaste felt for the newly arising group of Brahmanical temple priests.” 

~\hfill (Bronkhorst 2012: 411)
\end{myquote}

This illustrates how the Western scholar oversteps his bounds and inserts his own prejudices or compulsions of novelty into what is made to look like academic study. He is writing about the possible distaste of a section of people towards others nearly two millenia ago, phrasing it with “apparently” to be on the cautious side, while clearly attempting to present a biased picture.

\vspace{-0.25cm}

\subsubsection*{\textit{Vākyapadīya} of Bhartṛhari:}\index{Vakyapadiya@\textit{Vākyapadīya}}\index{Bhartrhari@Bhartṛhari}

\begin{verse}
\textit{astyarthaḥ sarvaśabdānāmiti\index{sabda@\textit{śabda}} pratyāyyalakṣaṇam \dev{।}}\\
\textit{apūrvadevatāsvargaiḥ samamāhurgavādiṣu \dev{॥} II. 119}\\
\textit{prayogadarśanābhyāsādākārāvagrahastu yaḥ \dev{।}}\index{prayoga@\textit{prayoga}}\\
\textit{na sa śabdasya viṣayaḥ sa hi yatnāntarāśritaḥ \dev{॥} II. 120}
\end{verse}

\vspace{-0.55cm}

\begin{flushright}
 (Iyer 1983: 58)
\end{flushright}


The article discusses Bhartṛhari’s \textit{kārikā}-s 119-120 from \textit{Vākyapadīya} Kāṇḍa II. shown above and which he translates as -

\begin{myquote}
“They say that the characteristic of what is to be conveyed is that all words have things [corresponding to them]; this applies to [words] such as ‘cow’ as much as to [the words] \textit{apūrva, devatā}\index{apūrva@\textit{apūrva}}\index{devata@\textit{devatā}} and \textit{svarga}.\index{svarga@\textit{svarga}} The grasping of the form (\textit{ ākāra}) as a result of repeatedly observing the use of a word, on the other hand, is not the realm of words, for it is based on a different effort.” 
\end{myquote}

\vspace{-.3cm}

\begin{myquote}

~\hfill (Bronkhorst\index{Bronkhorst, Johannes} 2012: 412)
\end{myquote}

He says later -

\begin{myquote}
“The second half of the above observation - “there are things corresponding to all (Sanskrit) words - will occupy us in this article. It is the most startling half, at least from a modern point of view. It is demonstrably untrue for the languages we now use, which have many words - such as “Martian”, “angel” - that do not refer to any existing entities, not at least according to an important part of their users. It presupposes a language stable in time, and which has no place for new words coined by its users. This may very well be the image that many Sanskrit users had of their language.”
\end{myquote}

\begin{myquote}
“It is clear from this response that the author of the criticized passage believed that various religious and mythological expressions correspond to reality simply because those expressions are part of the Sanskrit language.”
\end{myquote}

\begin{myquote}
“We may conclude that Uddyotakara, as Vātsyāyana before him, did\break indeed believe that all Sanskrit words (or at least nouns) designate something that exists in the “outside world”. 
\end{myquote}

\vspace{-.2cm}

\begin{myquote}

~\hfill (Bronkhorst 2012: 413)
\end{myquote}

We would like to point out here firstly, that Bhartṛhari is illustrating how language functions by recalling the idea associated with the word upon hearing it; he uses words that have a conceptual meaning rather than those referring to tangible, material entities such as “pot” or “cow” whose understanding is based on our having perceived them with sense of sight, touch etc. Bhartṛhari’s\index{Bhartrhari@Bhartṛhari} focus in the \textit{Vākyapadīya}\index{Vakyapadiya@\textit{Vākyapadīya}} is to show how language functions for every different school of thought and he shows no favoritism or hatred to the ontology of any school. He opposes the Mīmāṁsā\index{Mimamsa@Mīmāṁsā} theory of language on several counts, chiefly the divisibility of the sentence-meaning. Mīmāṁsā on the other hand severely criticises the \textit{sphoṭa} theory of Bhartṛhari. But they are in concurrence on almost every other linguistic issue. In our present context Bhartṛhari brilliantly (as always) points out that words such as \textit{apūrva},\index{apurva@\textit{apūrva}} etc serve their purpose through mental idea, as nobody has seen it as an object.

Bronkhorst\index{Bronkhorst, Johannes} uses this argument to show that the Mīmāṁsaka\textit{-}s held these things (\textit{apūrva, svarga, devatā}\index{svarga@\textit{svarga}}\index{devata@\textit{devatā}}) as “objects” and then goes on to argue that people were to believe these things existed simply because it was said in Sanskrit. We cannot agree with this view as being sound or derived from the Sanskrit arguments. Firstly, the word \textit{artha} means many things -- object, meaning, purpose, goal, wish, etc. In Bhartṛhari’s context here, it more precisely corresponds to “meaning”. Bhartṛhari gives these examples (which are used by common people, not only scholars) expressly to show that people understand word-meanings even without seeing transactions concerning these objects in the process of learning language as explained by Mīmāṁsā. Secondly, Bronkhorst extrapolates to show that the Mīmāṁsaka believed that these entities were objects, as their names (the words) were in Sanskrit. This is not warranted, as there is a simpler explanation to the situation. (Logical reasoning demands that a complicated solution is not to be favoured over a simpler solution.)

There are words such as \textit{manas, sukha, duḥkha, cintana} etc which are abstract but nobody doubts their existence. Note that the word “\textit{artha}” is used, not “\textit{bāhyārtha}” which would refer to “external object”. Puṇyarāja, in his \textit{Vṛtti} on the {Vākyakāṇḍa} of \textit{Vākyapadīya} says, \textit{“buddhyārūḍhastu śabdasyārtho,\index{sabda@\textit{śabda}} na bāhyaḥ}\dev{।}” (Iyer 1983: 58). That is, the meaning of a word is conceptual, in the mind, not an external object. Bronkhorst’s translation violates this principle. It is not as if the Mīmāṁsaka believed in a material object (a gross body, possessing mass) such as \textit{apūrva}. When a word such as “heaven” is used in a sentence, both speaker and hearer are quite sure what is meant. In another place, Bhartṛhari gives the example of “\textit{apsaras}” to illustrate a similar point. (Tripathi 1977: 573)

These words are to be understood exactly as we do “Martian” or “angel”, in Bhartṛhari’s\index{Bhartrhari@Bhartṛhari} example. On what basis does Bronkhorst differentiate? According to him, the principles of Sanskrit words cannot be applied to other languages!! Here we have a misrepresentation of \textit{śāstra} - the masters of yore never said that. Although they wrote in Sanskrit and commented on Sanskrit grammar, expecting the language to conform to exacting standards, they have never barred any relevant linguistic theory from being applied to other languages. Mīmāṁsā\index{Mimamsa@Mīmāṁsā} is explicit that the world will not obey \textit{śāstra} and that \textit{śāstra} should follow real norms that exist in the world, but the modern researcher of the subject is not aware of such maxims. Here we find a fundamentally skewed perspective to Sanskrit studies – applying principles in a misguided manner and instead of showing how universal concepts in classical studies can make sense even today, the Western scholar goes out of his way to portray issues as warped, on artificially constructed evidence.

Common words such as “cow”, “pot” etc. would have to follow the same linguistic rules in any language, but those such as \textit{`apūrva', `devatā'}\index{apurva@\textit{apūrva}}\index{devata@\textit{devatā}} and \textit{`svarga'}\index{svarga@\textit{svarga}} which are not essential in the Westerner’s world-view are selected to show seeming weakness in the theory. There is no weakness in our linguistic theory, neither has the {Mīmāṁsaka} any weird notions.

Despite his detailed discussions in this paper, Bronkhorst\index{Bronkhorst, Johannes} does not appear to have grasped the meaning of the two \textit{kārikās} on which his paper is based. We now look at what they convey. \textit{Kārikā} 119 says that it is known that all words convey meaning, by \textit{pratyāyya-pratyāyaka} relation and this is exactly the same in “\textit{apūrva-devatā-svarga}”, as it is in such words as “cow”.

Bhartṛhari has in several places pointed out that the word conveys meaning only through generalities, never with specific attributes. The power of a word, its \textit{pratyāyakatva}, is in the general meaning. The fact that the hearer gathers more upon hearing the word, is due to other factors, such as familiarity with the objects mentioned, etc. This is pointed out in \textit{kārikā} 120 which says that the fact that more information is recalled upon hearing a word, such as the form of the object (or thing) that it may correspond to, is due to other reasons such as use (\textit{prayoga})\index{prayoga@\textit{prayoga}}, having seen it (\textit{darśana})\index{darsana@\textit{darśana}} and familiarity (\textit{abhyāsa})\index{abhyasa@\textit{abhyāsa}}; it is not the capacity of the word itself, but is based on other efforts.

So in \textit{kārikā} 119, Bhartṛhari says that the core function of the word ‘cow’ from which we understand the familiar, tangible animal, cow, is exactly the same as in words such as ‘\textit{apūrva}’, ‘\textit{devatā’}\index{devata@\textit{devatā}} and ‘\textit{svarga}’,\index{svarga@\textit{svarga}} where the tangible form is not alluded to or conveyed. As an illustration, when an office colleague overhears a friend being told, “Burglars broke open your front door,” he would only understand the import of the sentence in a general sort of way, but cannot envisage the size, colour and heaviness of the door, nor the type, number and strength of the locks on it, unless he has seen them before (\textit{yatnāntara}) – through “other efforts”.

\textbf{Whereas Bhartṛhari\index{Bhartrhari@Bhartṛhari} is explaining that the word “cow” conveys as abstract a sense as the words, “\textit{apūrva}”,\index{apurva@\textit{apūrva}} “\textit{devatā}” or “\textit{svarga}”, Bronkhorst\index{Bronkhorst, Johannes} interprets it as the words, “\textit{apūrva}”, “\textit{devatā}” and “\textit{svarga}” having as concrete a referent as the word “cow”, which is diametrically opposite to the actual meaning. In this case, he is 180 degrees off the mark!}

There is no \textit{pūrva--pakṣa} in this case under discussion but Bronkhorst builds an elaborate thesis on Mīmāṁsaka’s hatred of temple priests, etc. without even discussing the main concept in the linguistic theory. While the Indian tradition has always welcomed innovative thinking and the development of theoretical concepts, modern researchers’ imagination proceeds not on academic lines, but on theorizing on motives of long ago!

\vspace{-.3cm}

\subsection*{3.2 Some Examples from Writings of Pollock}\index{Pollock, Sheldon}

\vspace{-.3cm}

\subsubsection*{\textit{Apauruṣeyatva} of Veda--s}

\vspace{-0.35cm}

Pollock writes that the Buddhists held that language was a human invention but Mīmāṁsā\index{Mimamsa@Mīmāṁsā} held that it was eternal and had no human author (Pollock 2006: 53, 55). This concerns the important issue of \textit{apauruṣeyatva} of Veda. It was not only the Buddhists who suggested that language was a human invention --- it was common perception of all schools that brought up the discussion of human speech being the starting point of verbal testimony. Starting from the \textit{Śābara Bhāṣya},\index{Sabarasvamin@Śabarasvāmin} all important texts of Mīmāṁsā discuss this issue in detail. We now look at some key arguments on the subject and try to appreciate the brilliance of Mīmāṁsā’s\index{Mimamsa@Mīmāṁsā} logic.

In testing the veracity of Vedic statements, the validity of language as a means to knowledge and the validity of verbal testimony is first analyzed. How do we comprehend the meaning of an utterance as valid? If one were to be directed by a person that the trees on the riverbank down the road were laden with fruit and went there seeking the fruits and actually found them, one ascertains the truth of the statement. But if one were to go there and find no trees with fruits, then the statement is false as it does not correspond to reality. A statement could be false if 1) it is uttered based on mistaken notions or false knowledge of the speaker, or 2) due to his intention to deceive. So normally, we rely on what reliable people tell us, who are knowledgeable and trustworthy. The truth of the statement hence depends on the virtue of the speaker in terms of being endowed with correct knowledge and also honesty. In case of error in the statement, the reason is invariably traced to the defect of the speaker, which has to be known in order to validate or invalidate his utterance. While a statement can convey meaning by rules of the language, the validity or credibility depends on the speaker. We do not believe everything we hear, but only the words of \textit{āpta}, a trustworthy person who speaks the truth. (Dravid and Narayanan 2016: 98, 392)

In the case of the Veda--s, Nyāya\index{Nyaya@Nyāya} holds God who is omniscient and without blemish, as the author of the Veda--s; the Veda--s are therefore completely reliable and their validity is beyond question.~The\break Veda--s are also imperishable and eternal in each cycle of creation; God composes them in the same form in every new cycle of creation. But Mīmāṁsā does not admit of a personal god endowed with the necessary characteristics to compose the Veda--s, for lack of evidence. Rather than resort to doubtful explanations, they consider Veda--s as valid by themselves, there being no reason to doubt their teaching.

Here it is important to understand what “permanent” or “eternal” (\textit{nitya}) means. Both {Vyākaraṇa} and Mīmāṁsā uphold the \textit{nityatā} of \textit{śabda},\index{sabda@\textit{śabda}} i.e, word, meaning and the relation between them. Patañjali\index{Patanjali@Patañjali} in \textit{Mahābhaṣya}\index{Mahabhasya@\textit{Mahābhaṣya}} and Bhartṛhari\index{Bhartrhari@Bhartṛhari} in \textit{Vākyapadīya}\index{Vakyapadiya@\textit{Vākyapadīya}} enunciate the concept of \textit{pravāhanityatā}, i.e. the principle by which a river current is considered a permanent body of water despite the water flowing past and changing every moment at any point. This is the permanence of a word. Although it may be uttered a hundred times and sound being evanescent, disappears everytime, the permanent form that is associated with it is recognized each time it is spoken or heard. [Incidentally, modern acoustics and wave patterns do tell us that a word actually has a form associated with it, which is used in modern electronics today that enable a person to speak commands into a computer].

When {Vyākaraṇa} and Mīmāṁsā\index{Mimamsa@Mīmāṁsā} say that words are \textit{nitya, śabda}\index{sabda@\textit{śabda}} is \textit{nitya} and so are the Veda--s, it means that they have been so as far back as one understands and therefore there is no point in enquiring after their origins and that the principles of that school ({Vyākaraṇa} or Mīmāṁsā) apply in a situation where the origin or end of the entity is not relevant. It defines the framework so to speak of the tenets of that particular school in explaining the avowed standpoint. (For instance, in the Classical Mechanics branch of Physics enunciated by Newton, matter is deemed different to energy and their inter-convertibility is not taken into account.)

Coming to the discussion on this matter, the Buddhists uphold only the teachings of Lord Buddha\index{Buddha, the} and reject the Veda--s. They\break are questioned on what basis can one replace the well-established, prevalent and ancient Vedic rituals by rites that are introduced by the Buddhists. What is the reason for holding their tenets as authoritative? They reply that as Lord Buddha was most noble in character and enlightened, he was \textit{sarvajña}, i.e, omniscient. Hence\break\hfill his teachings are valid beyond question. But the Mīmāṁsaka raises objection that the omniscience of Lord Buddha cannot be verified by us, living in a far-removed time and place. We only have the word of other disciples, also removed in time and place. It is\break reasonable to draw the inference that just as there is nobody omniscient in the present time, so it must be in all times past. Moreover, the disciple himself not being omniscient, how can he comprehend the Buddha’s omniscience? Kumārila\index{Kumarila Bhatta@Kumārila Bhaṭṭa} Bhaṭṭa points out that while the Buddhists speculate/posit \textit{sarvajñatva}, we\break (Mīmāṁsaka--s) speculate/posit \textit{apauruṣeyatva}, but between the two, our method is simpler!

\vspace{-.3cm}

\begin{verse}
\textit{sarvajñakalpanānyaistu vede cāpauruṣeyatā \dev{।}}\\\textit{tulyatā kalpitā yena tenedaṁ saṁpradhāryatām \dev{॥}} 
\end{verse}

\vspace{-.4cm}

\begin{flushright}
- \textit{Śloka-vārttika} II. 116 (Rai 1993: 61)
\end{flushright}

Dharmakīrti,\index{Dharmakirti@Dharmakīrti} a great Buddhist philosopher and contemporary of Kumārila, says that although omniscience cannot be proved, it is enough that the Buddha understood what \textit{dharma} was and taught people what he knew. This is accepted by Mīmāṁsaka\textit{-s}, but even then they argue, “where is the evidence that \textit{Bauddha Āgama--s} should replace the Vedic tenets?” A person’s teachings enable us to infer what he himself understands, but that in itself is not a validation. Moreover, the Buddha\index{Buddha, the} wrote no books or even uttered the canons directly. When in deep trance, \textit{samādhi},\index{samadhi@\textit{samādhi}} he could not have spoken sentences. Sermons written by disciples at a later date would only reflect that disciple’s knowledge and its validity would be limited by the attributes of that person. The Buddhists say that by the prowess of the Buddha, even the plastering in the wall and the boulders in the vicinity reverberated with the master’s teachings. This, the Mīmāṁsaka--s are unable to accept and Kumārila\index{Kumarila Bhatta@Kumārila Bhaṭṭa} calls it a matter of faith outside the range of logic, to depend on rocks and walls to deliver sermons.

\begin{verse}
\textit{sānnidhyamātratastasya puṁsaścintāmaṇeriva \dev{।}}\\
\textit{niḥsaranti yathākāmaṁ kuḍyādibhyo.api deśanāḥ \dev{॥}}
\end{verse}

\vspace{-0.5cm} 
\begin{flushright}
 \textit{Śloka-vārttika} II. 138
 \end{flushright}
 
  \begin{verse}
 \textit{evamādyucyamānan tu śraddadhānasya śobhate \dev{।}}\\
 \textit{kuḍyādiniḥsṛtatvācca nāśvāso vedanāsu naḥ \dev{॥}}
 \end{verse}
 
 \vspace{-0.5cm} 
 \begin{flushright}
  \textit{Śloka-vārttika} II. 139 (Rai 1993: 65)
  \end{flushright}


The logical standpoint of the Mīmāṁsā\index{Mimamsa@Mīmāṁsā} on there being no author of the Veda--s may have easily been misunderstood as a notion of the Veda--s floating permanently and eternally among the distant stars but that is not the true explanation! We cannot assume that the great intellectuals of ancient India were so naive as to propagate irrational ideas or that the public were so gullible as to be led by them. There was always logic to support the views and although it took a lot of training to be able to debate, there was not one profound philosophical concept that the common man was not aware of. If we were to pause to understand how God could have handed the tablets with the Ten Commandments inscribed on them to Moses, progress in reading the Bible would be slow indeed!


\section*{4. A Few Other Issues}

Prof. Pollock’s writing usually appears to set out with some agenda rather than any genuine curiosity about any issue in Sanskrit studies. In the beginning of his chapter titled “Axialism\index{Axial Theory} and Empire” in the book, “Axial Civilizations and World History”, he declares that he is seeking to establish a cause-effect relation separated by two millenia!

\begin{myquote}
“Both because the divergent modes of realizing the imperial political principle in South Asia and Europe have had reverberations across history and because they demonstrate the existence as such of alternate possibilities in transregional polity, studying them is meant as a form of “actionable” history, an attempt to produce statements about past events that can inform the conduct of present practices.” 

~\hfill (Pollock\index{Pollock, Sheldon} 2005: 400)
\end{myquote}

Clearly Prof. Pollock’s studies are not intended as academic exercises but aimed to represent history as he would like to influence future action. Perhaps his interest is in using academic activities to take him closer to policy-makers of the world, as otherwise his views on the subject are quite baffling to any scholar long familiar with Sanskrit studies.

\begin{myquote}
“The Veda’s injunction to act is meaningful precisely because it enunciates something that transcends the phenomenal, something inaccessible to observation, inference, or other form of empirical reasoning -- something in fact, irrational.” 

~\hfill (Pollock 2006: 405)
\end{myquote}

This view is for the most part a repetition of the Mīmāṁsā’s\index{Mimamsa@Mīmāṁsā} own explanation of why we cannot understand Vedic injunction by any other means, since we find it only in the Veda. The last part of the sentence, “something, in fact, irrational” is Pollock’s addition. Mīmāṁsā’s explanations are neither irrational nor beyond one’s rationale. Admittedly, people cannot grasp the Veda--s spontaneously but have to be taught by a guru in the tradition. India has long followed an oral tradition in Sanskrit studies; a book is an aid to study, but not a replacement for a good teacher. 

From the article, “Mīmāṁsā and the problem of History in Ancient India” - 

\begin{myquote}
“Then I will go on to examine in a little more detail what I think could be viewed as a confrontation with history on the part of Mīmāṁsā, and the resulting limiting conditions on histiriography imposed on the valuation of knowledge in general that Mīmāṁsā,\index{Mimamsa@Mīmāṁsā} the dominant orthodox discourse of traditional India, articulated”. 

~\hfill (Pollock 1989: 604)
\end{myquote}


\begin{myquote}
“The denial of history, for its part, raises an entirely new set of questions. To answer these we would want to explore the complex ideological formulation of traditional Indian society that privileges system over the creative role of man in history - and that, by denying the historical transformations of the past, deny them for the future and thus serve to naturalize the present and its asymmetrical relations of power.” 

~\hfill (Pollock\index{Pollock, Sheldon} 1989: 610)
\end{myquote}

That ideological formulations in Indian society privilege systems over creative role of man cannot be said to be the norm – in many instances of creative activity, such as architecture, literature and music, they actually help the individual attain far greater heights. Seeing the depth of systematization in an ancient and continuous tradition such as ours, he is mistaken that it does not allow for novelty or creativity. The situation is similar to wondering, upon seeing the intricate rules of Indian classical music, whether any new song can be sung! 

Appayya\index{Appayya Diksita@Appayya Dīkṣita} Dīkṣita was an orthodox brahmin, a traditional Vedic scholar, a Mīmāṁsaka, an {Advaitin} and a rhetorician par excellence. We cannot confine his identity to a definition that might be convenient for a modern research paper or monograph. We may mention here that Dr. Pollock translates the title of Appayya’s work \textit{Pūrvottara--Mīmāṁsāvāda--nakṣatramālā} as “The Milky Way of Discourses on Mīmāṁsā and Vedānta” and then writes further -

\begin{myquote}
“We might better capture the spirit if not the letter of the title by translating it “Collected Essays in the Prior and Posterior Analytics”, or perhaps instead, with a nod to Gademer rather than Aristotle, “... in Philosophical and Theological Hermeneutics”. 

~\hfill (Pollock 2004: 2)
\end{myquote}

But “\textit{Nakṣatramālā}” does not mean “Milky Way” - it means “group of stars” or “string of constellations”. There are traditionally twenty-seven constellations along the equator of the celestial sphere and since there are twenty-seven essays or arguments in the book the author has named it so! A necklace with twenty-seven beautiful pearls is also known as a \textit{nakṣatramālā} (Apte\index{Apte, V. S.} 1965: 333). The Sanskrit term for “Milky Way” is \textit{Ākāśa Gańgā}. Although not very significant in this case, it shows the latitude Western writers display in their translations. Here, Pollock\index{Pollock, Sheldon} covertly undermines the astronomical advances in the Indian system by using a wrong translation that is of no worth. Further, Pūrva--Mīmāṁsā always refers to Mīmāṁsā,\index{Mimamsa@Mīmāṁsā} the interpretation of the earlier portion of Veda dealing with ritual and Uttara--Mīmāṁsā\index{Uttaramimamsasutra@\textit{Uttaramīmāṁsā--sūtra--s}} always refers to Vedānta, which deals with the analysis of Upaniṣad. They cannot be translated as “prior and posterior analytics”. Where the translations go against traditionally held meanings, they have to be discarded. Language, by definition, is something that is used according to the intentions of the speaker and unless there is familiarity with the cultural context, there is bound to be error.


\section*{5. Conclusion}

In Indian treatises on \textit{śāstra} the subject under discussion was always considered more important than the historical antecedents of the writer or the place where the book had been written. Historical importance was accorded to the achievements of kings in edicts, but not in the pursuit of knowledge. It is a characteristic feature of Indian writing, but does not merit value judgements or speculation on motives. There is clear evidence that the Indian tradition understood the passage of time very well --- royal edicts of military and economic relevance record the year very clearly. That very few edicts remain in a crowded tropical country with a long history is a different matter. Many families were known to have carefully preserved records of family trees with details of native place and property, but they have now gone up in dust.

In the \textit{saṅkalpa mantra} that is chanted every time a person undertakes a formal worship ritual, his name and family are mentioned to establish his unique identity, the place in which he (or the activity at that point in time) is located is mentioned for unique configuration in space and the date is also mentioned for uniqueness in time. The Navya-Nyāya\index{Nyaya@Nyāya} philosophy had also developed technical terminology in terms of \textit{avacchedakatva} of \textit{deśa} and \textit{kāla} to distinguish between any two individual entities. Way back in the \textit{Vākyapadīya},\index{Vakyapadiya@\textit{Vākyapadīya}} Bhartṛhari\index{Bhartrhari@Bhartṛhari} speaks of \textit{mūrti--vivarta}\index{murtivivarta@\textit{mūrti--vivarta}} and \textit{kriyā--vivarta}\index{kriyavivarta@\textit{kriyā--vivarta}} which define an entity uniquely in space and time, as reflected in the four-dimensional Cartesian co-ordinates that define reality (Narayanan 2012: 79). The language is different, but the concept is the same. The American astrophysicist Carl Sagan\index{Sagan, Carl} has shown in his book and television show, “Cosmos” that for some reason the Indian civilization is unique in its astronomical scales of time that correspond to the figures in modern astrophysics.

There is not one system of knowledge that does not have some point of “singularity” as it is called in modern physics, where there is no answer and hence the question is not addressed. We can ask questions such as - 1) What was the expanding universe like, at its first moment of creation, at time t = 0? 2) What was the cause of the Big Bang? 3) What is the value of infinity divided by zero? (The answer is: “indeterminate”, in mathematics.) 4) Why does the Bible say that God created the world in six days? 5) What is defined by “day” in this context? 6) Is it a “day” on the Earth or is it on Jupiter? 7) But then what would constitute the meaning of “day” on Uranus, whose axis of rotation lies parallel to the orbital plane of the other planets? 8) Should we not understand the neighborhood of the earth, the solar system, as being encompassed within the definition of “world” as created by God? And so on and so forth. But we cannot hope for conclusive answers. Can we criticize mathematics as a science because it cannot solve simple problem such as zero divided by zero?

The \textit{Taittirīya Upaniṣad}\index{Taittiriya Upanisad@\textit{Taittirīya Upaniṣad}} speaks of the teacher and student sitting down together to study: 

“Let us be protected together; let us sup together; let us join in our endeavours; let our study lead to illumination; may we never succumb to hate”. If the Western endeavours in Sanskrit studies are not guided by this lofty ideal, they are not worth five paise.


\section*{Bibliography}

\begin{thebibliography}{99}
\itemsep=1pt
\bibitem{chap6-key01} Apte, Vaman Shivram. (1965). \textit{The Practical Sanskrit–English Dictionary}. Delhi: Motilal Banarsidass.

 \bibitem{chap6-key02} Archak, K.B. (2004). \textit{Pūrṇaprajñadarśanam}. M.A. Text Book, Course-IV. Mysore: Karnataka State Open University.

 \bibitem{chap6-key03} Arnason, Johann Pall., Eisenstadt, S. N., and Wittrock, Bjorn. (Ed.s) (2005). \textit{Axial Civilizations and World History.} Boston: Brill.

 \bibitem{chap6-key04} Bronkhorst, Johannes. (2001). “The Origin of Mīmāṁsā as a school of thought: a hypothesis”. In Kartheinnen and Koskikallio (2001). pp. 83-103.

 \bibitem{chap6-key05} —. (2012). “Studies on Bhartṛhari,9: Vākyapadīya 2.119 and the Early History of Mīmāṁsā”. \textit{Journal of Indian Philosophy}, 40, August 2012. pp 411-425.

 \bibitem{chap6-key06} Dravid, R. Mani., and Narayanan, Sharda. (2016). \textit{Śāstradīpikā - Tarka Pāda}. Chennai: Giri Trading Agency Pvt. Ltd. 

 \bibitem{chap6-key07} Hiriyanna, M. (1994). \textit{Outlines of Indian Philosophy}. Delhi: Motilal Banarsidass. 

 \bibitem{chap6-key08} Iyer, K.A. Subramania. (Ed.) (1983). \textit{The Vākyapadīya of Bhartṛhari Kāṇḍa II.} Delhi: Motilal Banarsidass.

 \bibitem{chap6-key09} Kartheinnen, Klaus., and Koskikallio, Petteri.(Ed.s) (2001).\textit{Vidyārṇavavandanam. Essays in Honour of Asko Parpola}. Helsinki: Studia Orientalia.

 \bibitem{chap6-key10} Narayanan, Sharda. (2012). \textit{Vākyapadīya - Sphoṭa, Jāti \& Vyakti}. New Delhi: DK Printworld. 

 \bibitem{chap6-key11} Pandurangi, K.T. (2004). \textit{Prakaraņapañcikā of Śālikanātha}. New Delhi: Indian Council for Philosophical Research.

 \bibitem{chap6-key12} —. (2006). \textit{PūrvaMīmāṁsā from an Interdisciplinary Point of View.} New Delhi: PHISPS. 

 \bibitem{chap6-key13} Pollock, Sheldon. (1989). “Mīmāṁsā and the Problem of History in Traditional India”. \textit{Journal ofAmerican Oriental Society.} Vol 109, No. 4. pp 603-610.

 \bibitem{chap6-key14} —. (2004). “The Meaning of Dharma and the Relationship of the Two Mīmāṁsās: Appayya Dīkṣita’s Discourse on the Refutation of a Unified Knowledge System of Purvamīmāṁsā and Uttaramīmāṁsā”. \textit{Journal of Indian Philosophy} Vol. 32. No. 5/6. pp 769-811. 

 \bibitem{chap6-key15} —. (2005). “Axialism and Empire”. In Arnason \textit{et al} (2005). pp 397-450.

 \bibitem{chap6-key16} —. (2006). \textit{The Language of the Gods in the World of Men: Sanskrit, Culture and Power in Premodern India}. Berkeley: University of California Press.

 \bibitem{chap6-key17} Rai, Ganga Sagar (Ed.) (1993). \textit{Ślokavārttikam of Kumārila Bhaṭṭa: with Nyāya Ratnākara of Pārthasārathimiśra.} Varanasi: Ratna Publications.

 \bibitem{chap6-key18} \textbf{\textit{Śāstradīpikā}.} See Swami (1996).

 \bibitem{chap6-key19} \textbf{\textit{Ślokavārttikam.}} See Rai (1993).

 \bibitem{chap6-key20} Swami, Kishoredas. (Ed.) (1996). \textit{Śāstradīpikā (Tarka Pāda)(Hindi)}. Dehra Dun: Swami Ramtirth Mission.

 \bibitem{chap6-key21} Tripathi, Ramdev. (1977). \textit{Bhāṣā Vijñān kī Bhāratīya Paramparā aur Pāṇiṇi. (Hindi).} Patna: Bihar Rashtrabhasha Parishad.

 \bibitem{chap6-key22} \textbf{\textit{Vākyapadīya.}} See Iyer (1983).
 
 \end{thebibliography}

\label{chapter6-end}
