
\chapter{\hspace{7.2cm} Othering\index{Othering} \hfill\break and Indian Population Genetics}\label{chapter10}

\Authorline{-- Murali K. Vadivelu}\footnote{pp. 301-317. In: Kannan, K. S and Meera H. R. (Ed.s) (2020). \textit{Chronology and Causation: Negating Neo-Orientalism.} Chennai: Infinity Foundation India.}

\begin{flushright}
\textit{(mkv22@cantab.net)}
\end{flushright}

\setcounter{endnote}{0}

\section*{Abstract}

Sanskrit, an ancient language of the Hindu civilisation and other Indic cultures across the globe, is hypothesised as “a source for legitimising power by a tyrannical kingship (Oriental despotism)” that was yet paradoxically subservient to the priesthood, in pre-colonial India. In simple words, the language was supposedly used by upper-caste groups for the explicit “othering” of minorities and lower-caste groups – so claims Professor Sheldon Pollock.\index{Pollock, Sheldon}

Recent population genetics\index{population genetics} analysis shows that an “abrupt” start (in genetic timescales) of castes-by-birth (used synonymously with endogamous groups) is found to coincide with the foreign invasions of India (Islamic\index{Islamic} rule under the Delhi Sultanate and the Madurai Sultanate) and appears to have been modelled on the Arab-Muslim clan-tribal endogamy.\index{endogamy} Until then the entire population of India appears to have been genetically admixing freely, that is entirely exogamous. Prior to the availability of definitive genetic data, the presence of same patrilineal (matrilineal in some sects) \textit{gotra}-s\index{gotra@\textit{gotra}} amongst the various castes classified under the different \textit{varṇa}-s\index{varna@\textit{varṇa}} was well known, yet ignored by the social scientists.

Genetic evidence also indicates that the most proximal group to the Brahmins are the lower castes at the so-called bottom of the \textit{varṇa}\index{varna@\textit{varṇa}} “hierarchy” (labelled as Scheduled Castes, and more recently as Dalits) and somewhat surprisingly the Muslims of North India.

Archival records (of the colonial British Indian government and the East India\index{East India Company} Company) from the early colonial time period show that casteism (caste-based discrimination) and untouchability was largely alien to the Hindus till the arrival of colonial education that completely replaced indigenous educational system and the implementation of land “reforms” by the British, which regressively curtailed tenancy rights and reallocated the ownership of lands out of tune with longstanding local customs.

Given the evidence, the process of “othering”,\index{othering} based on extensive empirical historical data and population genetics\index{population genetics} evidence, can only be seen as a direct imposition by invading foreign rulers since the time of the Islamic\index{Islamic invasion} incursions into India, which only worsened during the European colonial rule. Thus, Pollock’s\index{Pollock, Sheldon} hypothesis remains entirely discredited in the scientific world.

Clearly, Sanskrit is a victim of that “othering” process and claiming it as an instrument of oppression is an “attributional bias” and thus “amoral”. Such anti-empirical conclusion is arrived at by Pollock, a mere \textit{classicist}, using a putatively novel discipline of overtly textualised and materialistic dialectical philology, called political philology, which dehumanises by de-emphasising the emotional, empathetic, and contextual aspects of language, and the entire historic time periods thus studied. A de-stressing of empathetic (humanising) components can be viewed in a clinical psychological framework as the zero-negative empathy end of the empathy spectrum, seen in the pathology of psychopathy. It can be concluded, thus, that mapping the dimensions of political philology onto a psychological framework of the empathy spectrum might provide us a novel perspective to understand the mechanics of hypothesis-forming employed by its practitioners.

\vspace{-.3cm}

\section*{Background}

Sanskrit “is the golden treasure of epics, the cradle of grammar, politics and philosophy and the home of logic, dramas and criticism” in the words of Ambedkar,\index{Ambedkar, B. R.} one of India’s most “disruptive” thinkers. (Keer\index{Keer, Dhananjay V.} 1954:19) “Disruptive” cannot be any more traditional than this if, the antiquity of Sanskrit notwithstanding, the language deserves such critical acclaim. However, particularly recently Sanskrit has been subjected to a political-academic campaign border lining on hysteria. Thus it has become imperative – for diagnostic studies to be carried out on the manifestations of this phenomenon, analyse empirical data and contemporary scientific evidence, and to formulate a psychopathological framework to understand such phenomenon based on recent advances and understandings in the fields of clinical psychology (Baron-Cohen 2012).

It is claimed that “Sanskrit knowledge presents itself to us as a major vehicle of the ideological form of social power in traditional India,” and “The ideology of divine hierarchy... is an important part of the ancient knowledge of India, beginning with the post-Vedic Brāhmaṇa texts, with their neat order of social differences within a moral unity, and continuing through medieval dharmaśāstra texts, with their more messy, contingent and regionally varied codes.” (Malhotra\index{Malhotra, Rajiv} 2016: 145)

Following on the above, it is concluded that

\begin{myquote}
“Domination did not enter India with European colonialism. Quite the contrary, gross asymmetries of power – the systematic exclusion from access to material and nonmaterial resources of large sectors of the population – appear to have characterised India in particular times and places over the last three millennia and have formed the background against which ideological power, intellectual and spiritual resistance, and many forms of physical and psychological violence crystallised.” 

~\hfill (Malhotra 2016:174)
\end{myquote}

To highlight one of the most egregious of such examples from history is an infamous claim of Macaulay’s, “It is, I believe, no exaggeration to say that all the historical information which has been collected from all the books written in the Sanskrit language is less valuable than what may be found in the most paltry abridgement used at preparatory schools in England.” In the same document Macaulay\index{Macaulay, Thomas Babington} had acknowledged his abject ignorance of having “\textbf{no knowledge of either Sanscrit or Arabic}.”\endnote{ Unless stated otherwise \textbf{emphasis} (bold) found in the various quoted quotations found at various parts of this text was added by the author.}(Pritchett 2017). This can only be described as an act of monumental hypocrisy, given Macaulay's association with the drafting of the Indian Penal Code, the evidential standards of which he clearly failed to pass.


\section*{Castes in India}

Before we go further, it is imperative that we understand at least a few of the hypotheses widely accepted as “facts” with regards to Hindu civilisation and castes. Ambedkar\index{Ambedkar, B. R.} bravely attempted to reconstruct the putative origins of caste, though he was not the first or the last to do so. He made some astute observations in the the process, such as this:

\begin{myquote}
“One thing I want to impress upon you is that Manu\index{Manu} did not give the law of Caste and that he could not do so. Caste existed long before Manu. He was an upholder of it and therefore philosophised about it, but certainly he did not and could not ordain the present order of Hindu Society. His work ended with the codification of existing caste rules and the preaching of Caste Dharma. \textbf{The spread and growth of the Caste system is too gigantic a task to be achieved by the power or cunning of an individual or of a class. Similar in argument is the theory that the Brahmins created the Caste}. After what I have said regarding Manu, I need hardly say anything more, except to point out that it is \textbf{incorrect in thought and}malicious \textbf{in intent}.” 

~\hfill (Ambedkar 2014:16)
\end{myquote}

In spite of starting on the right track, coloured by the then prevailing orthodoxy in the society and his own personal ill experiences, which were indeed depressingly very many, (Keer\index{Keer, Dhananjay V.} 1954) combined with a lack of empirical data and the absence of the luxury of scientific data (population genetics,\index{population genetics} etc.) Ambedkar effectively ended up condemning the Hindus, particularly the Brahmins. He hypothesised that the class system that the Hindu society has (like other societies), divided into the four classes of priestly, military, merchant and artisan/menial used to have the flexibility of people being allowed to change their classes according to qualification. However, he says, that the open-door character was lost and they became self-enclosed units we know as castes. “Some closed the door: Others found it closed against them.” He gives the psychological interpretation as “the infection of imitation” where the non-Brahmin subdivisions whole-heartedly imitated the Brahmin caste which had turned endogamous. (Ambedkar 2014:18) Not so surprisingly Pollock's\index{Pollock, Sheldon} “othering”\index{othering} seems to be modelled almost entirely on Ambedkar's hypothesis but rephrased in a newly invented language: \textit{old wine in a new bottle}.

%\newpage

\section*{Castes in India: Their Mechanisms, Genesis and Development}

William Adam\index{Adam, William} in his report on the “State of education in Bengal, 1835–38” writes on the state of the Brahmins that the students who continued their studies till they were nearly forty used to find support either through their gurus, the gifts they received on festive occasions, through their relations and “fourthly, by begging” in case all other three ways failed. (cited in Dharampal\index{Dharampal} 1983:305)

It beggars belief that the Brahmins dedicated to studies alone, and nothing else, till “even” forty years of age and supporting themselves on begging were the ones who closed the doors on others, or were guilty of “othering”\index{othering} others. On the contrary, would not logic of human nature dictate that the Brahmins were the ones who perhaps would have found the door closed while begging for alms or when seeking matrimony, metaphorically or even literally?

W. Adam in his report mentioned above also observed that the Hindu medical men (Vaidya-s) and even Kāyastha-s were taught Sanskrit (and could study Sanskrit literature), the same could not be said of their Mahomedan counterparts, who were not taught Arabic or the sciences. He mentions the general complaint heard from the kazis that few Mahomedan priests understood Arabic though they had learnt by rote enough to allow them to perform ceremonies. (Dharampal 1983:306)

Beyond any reasonable doubt, the above establishes that the Hindu population was capable of using Sanskrit as a medium of instruction and of study for even complex subjects such as medicine, a field predominantly undertaken by a large proportion of non-Brahmins. (Dharampal 1983) While at the same time, the Muslim population had little or no knowledge of Arabic, though the Quran and its ceremonial verses were commonly used. This should be not a surprise given the fact that the caste system in Islam\index{Islam} is as regressive as it can get:

\begin{myquote}
‘As an illustration one may take the conditions prevalent among the Bengal Muslims. The Superintendent of the Census for 1901 for the Province of Bengal records the following interesting facts regarding the Muslims of Bengal:— “The conventional division of the Mahomedans into four tribes— Sheikh, Saiad, Moghul and Pathan—has very little application to this Province (Bengal). The Mahomedans themselves recognize two main social divisions, (1) Ashraf or Sharaf and (2) Ajlaf.\index{Ajlaf} \textbf{Ashraf means `noble'} and includes all undoubted descendants of foreigners and converts from high caste Hindus. All other Mahomedans including the occupational groups and all converts of lower ranks, are known by the contemptuous terms, \textbf{`Ajlaf~', `wretches' or `mean people'}: they are also called \textbf{Kamina or Itar, `base'} or \textbf{Rasil, a corruption of Rizal, `worthless'}. In some places a third class, called \textbf{Arzal\index{Arzal} or `lowest of all'}, is added. With them no other Mahomedan would associate, and they are \textbf{forbidden to enter the mosque to use the public burial ground}. “Within these groups there are castes with social precedence of exactly the same nature as one finds among the Hindus.'
\end{myquote}

I. Ashraf\index{Ashraf} or better class Mahomedans.

(1) Saiads. (2) Sheikhs. (3) Pathans. (4) Moghul. (5) Mallik. (6) Mirza.

II. Ajlaf\index{Ajlaf} or lower class Mahomedans.

\begin{enumerate}
\itemsep=0pt
\item Cultivating Sheikhs, and others who were originally Hindus but who do not belong to any functional group, and have not gained admittance to the Ashraf Community, e.g. Pirali and Thakrai.

 \item Darzi, Jolaha, Fakir, and Rangrez.

 \item Barhi, Bhathiara, Chik, Churihar, Dai, Dhawa, Dhunia, Gaddi, Kalal, Kasai, Kula Kunjara, Laheri, Mahifarosh, Mallah, Naliya, Nikari.

 \item Abdal, Bako, Bediya, Bhat, Chamba, Dafali, Dhobi, Hajjam, Mucho, Nagarchi, Nat,Panwaria, Madaria, Tuntia.

\end{enumerate}

III. Arzal or degraded class. Bhanar, Halalkhor, Hijra, Kasbi, Lalbegi, Maugta, Mehtar.”’

\vspace{-.3cm}

\begin{flushright}
(Ambedkar\index{Ambedkar, B. R.} 1941: 225–226)
\end{flushright}

The current regressive Indian caste system, projected to be a unique Hindu phenomenon either erroneously or deliberately, appears to be almost entirely modelled on the above Islamic\index{Islam} society's hierarchy than the Hindu \textit{varṇa}\index{varna@\textit{varṇa}} system, which hardly ever classified people as \textbf{wretches, mean, base, worthless and lowest of all}. Arab clan-tribal system is arguably world renowned for its endogamy\index{endogamy} and dehumanising discrimination, but far less publicised for inexplicable reasons. (Weiner\index{Weiner, Mark} 2013).

%\newpage

It is also important to understand that the Europeans had a very limited understanding of the apparently complex Indian society, and this confused them to no end. Sadly there appears to be a significant degree of disgust arising from various misunderstandings, furthering various prejudices. Dharampal (2000:~17–18) notes that the early British Governer Generals observed that Hindu rulers in fact spent very little on themselves\endnote{ This completely contradicts the hypothesis of oppressive kingship.} but gave away a lot to the brahmins and to temples. It is to be noted here that both the terms were probably used in a much wider sense at the time to include all sorts of scholarship and to the institutions that catered to needs, not just religious, but to scholarship, culture, entertainment and comfort. “Obviously, anyone who exercised some intellectual, medical or other professional skill seems to have been taken to be a Brahmin, even by fairly knowledgeable Europeans, in this period.”

Empirical data from various British and European archives and colonial archives in India that “disproves” the existence of significant caste discrimination or untouchability (“othering”).\index{othering} Dharampal\index{Dharampal} (2000: V:26–27) points out, for instance, that the village community had greater supremacy (in most cases) over land – ownership, disposal and working included. Also, regarding the question of the śūdra-ownership, he points out: “Again in Thanjavur in 1805, the number of \textit{mirasdars} (i.e. those having permanent rights in land) was put at 62,042, of which over 42,000 belonged to the sudras and castes below them.” (Dharampal 2000:~27). As regards education and the composition of students in schools, a survey the British did in the Madras Presidency indicated “the Sudras\index{sudra@\textit{śūdra}} and castes below them formed 70\%–80\% of the total students in the Tamil speaking areas, 62\% in the Oriya areas, 54\% in the Malayalam speaking areas, and 35\%–50\% in the Telugu speaking areas.” (Dharampal 2000: 29)\endnote{ Madras Presidency included the present-day Tamilnadu, major parts of Andhra Pradesh and Telangana as well as some districts of Karnataka, Kerala and Orissa. See Dharampal\index{Dharampal} (2000: 29--32) for details.}

Accounting, a subject of great practical importance and a skill considered essential in native Hindu education, was neglected in the Christian schools, for example:

\eject

\begin{myquote}
“Regarding the content of elementary teaching, Adam\index{Adam, William} mentioned various books which were used in teaching. These varied considerably from district to district, but all schools in the surveyed districts, except perhaps the 14 Christian schools, taught accounts. Also, most of them taught both commercial and agricultural accounts.” 

~\hfill (Dharampal 1983: 48)
\end{myquote}

Thus “othering”\index{othering} by de-skilling through European education has also been a major factor. Records from Punjab as well show an extensive and widespread Sanskrit schooling\index{Sanskrit schools} system (330,000 pupils at the time of annexation vs 190,000 in 1882) providing further evidence that “othering” has been imposed by foreign invaders (British colonialists) by elimination of Sanskrit. Data from “Leitner on indigenous education in the Panjab”:

“LIST OF SANSCRIT BOOKS USED (Balbodh and Akshar dipika)

\vspace{-.2cm}

\begin{enumerate}
\itemsep=0pt
\item GRAMMAR – Saraswat, Manorama, Chandrika, Bhashya, Laghu Kaumudi, Paniniya\index{Panini@Pāṇini} Vyakaran, Kaumudi, Siddhant Kaumudi, Shekar, Prakrita Prakasa

 \item LEXICOLOGY – Amar Kosh, Malini Kosh, Halayudh

 \item POETRY, THE DRAMA AND RELIGIOUS HISTORY – Raghu Vans, Mahabharat, Megh Duta, Venisanhara, Magh, Sakuntala, Kirat Arjun, Naishadha Charita, Ramayan, Mrichhakatika, Sri Mad Bhagwat, Kumara Sambhava and other Puranas\index{purana@\textit{purāṇa}}

 \item RHETORIC – Kavya Dipik, Kavya Prakash, Sahitya Darpana,\index{Sahityadarpana@\textit{Sāhitya Darpaṇa}} Dasu Rupa, Kuvlayanund

 \item MATHEMATICS, ASTRONOMY,\index{Astronomy} AND ASTROLOGY – Siddbant Shiromani, Nil Kanthi, Mahurta Chintamani, Brihat Jatak, Shighra Bodh, Parasariya, Garbh Lagana

 \item MEDICAL SCIENCE – Sham Raj, Nighant, Susruta,\index{Susruta Samhita@\textit{Suśruta Saṁhitā}} Sharang Dhar, Charaka, Bhashya Parichehed, Madhava Nidan, Vagbhat

 \item LOGIC – Nyaya\index{Nyaya@Nyāya} Sutra Vritti, Gada dhari, Vyutpattivad, Tarka-\break -lankar, Tark Sangrah, Kari kavali

 \item VEDANT – Atma Bodh Sarirak, Panch Dashi

 \item LAW – Manu\index{Manu}\index{Manusmrti@Manusmṛti} Smriti, Parasara Smriti, Yagya Valk Gautama,\index{Gautama} Mitakshara

 \item PHILOSOPHY – Sankhya\index{Sankhya@Sāṅkhya} Tatwa Kaumudi, Patanjali,\index{Patanjali@Patañjali} Sutra Britti Sutra with Bhashya, Sankhya Pravachan Bhashya, Vedanta, Vedantsar (see also above), Yoga\index{Yoga} Sutra, Vaiseshika, Siddhant Mimansa, Sutra with Muktavali Sutra with a commentary, Bhashya Artha Sangraha

 \item PROSODY – Srut Bodh, Vritta Ratnakar

 \item PROSE LITERATURE – Hitopadesa, Vasavadatta, Dasa Kumara Charita

 \item RELIGION – Rigveda\index{Rigveda@\textit{Ṛigveda}} Sanhita (rare), Samaveda,\index{Samaveda@\textit{Sāmaveda}} Mantra Bhaga, Yajurveda,\index{Yajurveda@\textit{Yajurveda}} Shukla Yajur Chhandasya Archika (very rare), Vajasneyi Sanhita”

\end{enumerate}

\vspace{-.7cm}

\begin{flushright}
(Dharampal 1983: 351–352)
\end{flushright}

Thomas Munro,\index{Munro, Thomas} in his evidence to a House of Commons committee, had observed that “if civilisation is to become an article of trade between England and India, the former will gain by the import cargo.” referring to schools established in every village for teaching, reading, writing and arithmetic. (House of Commons Papers: 1812–13, Vol. 7, p.131). (Dharampal 1983: 42 footnote 67)

To conclude the elucidation of the empirical data from the early colonial period, and before moving on to the population genetic evidence, it shall be borne in mind that the average period of schooling in the contemporary England (say, 1835 CE – 1851 CE) was around one to two years, where even writing was excluded and also that the British, hungry for revenue, targeted to exterminate the Indian cultural and religious content and structure through starving its resources. All this was done to maintain the British rule, just as the large-scale school education was deliberately neglected till Anglicised education was viably established (Dharampal\index{Dharampal} 1983: 74–75) As one can see from the above, the “absurd” superstition, for which writing was excluded from many schools in England, notwithstanding:

\begin{myquote}
“Karl Marx\index{Marx, Karl} seems to have had similar impressions of India—this, despite his great study of British state papers and other extensive material relating to India. Writing in the New York Daily Tribune on 25 June 1853, he shared the view of the perennial nature of Indian misery, and approvingly quoted an ancient Indian text which according to him placed ‘the commencement of Indian misery in an epoch even more remote than the Christian creation of the world.’ According to him, Indian life had always been undignified, stagnatory, vegetative, and passive, given to a brutalising worship of nature instead of man being the ‘sovereign of nature’—as contemplated in contemporary European thought. And, thus Karl Marx concluded: ‘Whatever may have been the crimes of England’ in India, ‘she was the unconscious tool of history’ in bringing about—what Marx so anxiously looked forward to—India’s westernisation.” 

~\hfill (Dharampal 1983: 75)
\end{myquote}

%\newpage

\section*{Population Genetics}

The final blow to the \textit{mythology of “othering”},\index{othering} castes and its origins being erroneously placed within the Hindu civilisation and its religious philosophies comes from a series of recent \textit{population genetics\index{population genetics} evidences}. From one of the northernmost state of Uttarakhand (Negi \textit{et al}. 2016) to the southernmost state of Tamil Nadu (Basu, Sarkar-Roy, and Majumder 2016), data shows that the now so-called Dalits or scheduled castes are genetically closest to, in other words belong to the same larger group as, the Brahmins and Kshatriyas castes than to the other castes. Of course, this is on the background of a completely admixed pan-Indian populace. (Moorjani \textit{et al.} 2013) This mixture of population groups across now known castes-by-birth (endogamous groups) and even geographies have lasted until very recently when it was put to an “abrupt” end during the pan-Indian Muslim rule; the “abrupt” end to exogamy (admixture) and start of endogamy\index{endogamy} (formation of castes-by-birth) also follows the wave (time lapse) of Islamic\index{Islamic invasioni} invasion, from the west of India to the east. Any social change to have occurred so “abruptly” within a few generations across such a huge geography requires unimaginable force and / or disruption. There are sufficient historical sources and evidences that could provide what these forces and disruptions could have been and their mechanisms of action. (Sanyal\index{Sanyal. Sanjeev} 2016) For one, the genesis of castes-by-birth (endogamy) as hypothesised by Ambedkar\index{Ambedkar, B. R.} (imitation and fashion) (Ambedkar 2014) or by “othering” using language structures and literature would be inordinately protracted: genetic evidence is emphatically against such a possibility at all. (Vadivelu 2016)

\vspace{.1cm}

Prior to the availability of definitive genetic data, the presence of same patrilineal (matrilineal in some sects) \textit{gotra-s}\index{gotra@\textit{gotra}} amongst the various castes classified under the different \textit{varṇa}-s\index{varna@\textit{varṇa}} was well known, yet ignored by the social scientists. Perhaps arguments and evidence were selectively chosen and further massaged to lead to a pre-determined conclusion. (Malhotra\index{Malhotra, Rajiv} and Neelakandan\index{Neelakandan, Aravindan} 2011; Ambedkar 1946)

\vspace{.1cm}

Above all, why would the Brahmins and Kshatriyas “other” their own, now genetically proven, kith and kin? Pre-1857 British Indian army (the mutineers) was made up of a significant strength of Brahmins, to the extent that they were a little over one-third of the strength. (Ambedkar 1941) Does this hold a key to the anti-Brahminism and “othering”\index{othering} promoted by vested interests who must have been, and perhaps are even now, arguably anti-Indic?

Empirical historical data from British / European archives, hitherto conveniently ignored, confirms the findings of genetic studies variously, Dharampal (2000: 58-59) cites Carpue’s observation, “The profession of astrology, and the task of making almanacs,’ says a later writer on India, ‘belong to degraded Brahmins; and the occupation of teaching military exercises, and physic, as well as the trade of potters, weavers, brasiers, fishermen, and workers in shells, belong also to the descendants (meaning the outcastes) of Brahmins.”

Moving on from the Brahmin connections of the now so-called scheduled castes to their Kshatriya connections, we find the following empirical historical data, again correlating well with the genetic evidences being unearthed:

\begin{myquote}
“The \textit{Karnam} or \textit{Conicoply}\endnote{ Ironically, \textit{Conicoply} (Kanakku Pillai) has morphed into a caste-by-birth in spite of the offices being manned by various different communities until two or three generations ago. (Dirks 2011).} (which really implied the office of the registrar of the village, a sort of secretariat, rather than a single individual) generally had an allocation of 3–4\% while the \textit{Taliar} (i.e. \textbf{the village police}, which may have included several persons) generally had an allocation of around 3\%. Incidentally, it may be useful to know that \textbf{the offices of the \textit{Taliar}, the Corn-Measurer, the settler of boundary disputes, and a few other village offices}, were generally \textbf{filled by persons from the \textit{Pariah} and allied castes}. As many will know in Maharashtra, \textbf{it was the \textit{Mahars} who constituted the village police}. … \textbf{it does imply that every person in this society enjoyed a certain dignity and that his social and economic needs were well provided for}…” 

~\hfill (Dharampal\index{Dharampal} 2000: 24)
\end{myquote}

Interestingly, Ambedkar\index{Ambedkar, B. R.} made such a prescient proposition based on his analysis of the \textit{Ṛg Veda} primarily, that \textit{śūdra}-s\index{sudra@\textit{śūdra}} are \textit{kṣatriya}-s.\index{ksatriya@\textit{kṣatriya}}\break (Ambedkar 1946) Ironically, from the above analysis Ambedkar made a better Sanskrit scholar than the Western classicist Pollock,\index{Pollock, Sheldon} perhaps because of the embedded (conscious or subconscious) cultural \textit{saṁskāra}\index{samskara@\textit{saṁskāra}} of Ambedkar, being born an “insider”? Indeed, Ambedkar appears to have had a pride in his cultural roots (\textit{saṁskṛti})\index{samskrti@\textit{saṁskṛti}} and the ensuing \textit{saṁskāra}. He noted that if one goes out of the Hindu religion, they invariably go out of the Hindu culture as well. “... Conversion to \textbf{Islam\index{Islam} or Christianity\index{Christianity} will denationalise the depressed classes}” He rightly observed that conversion into either of these Abrahamic faiths would create the danger of their domination.(Keer\index{Keer, Dhananjay V.} 1954: 280–281).

One would not need a huge stretch of imagination to conclude that Ambedkar, in spite of his criticisms of the Hindu society and religion, was a staunch Hindu culturalist (\textit{saṁskṛti})\index{samskrti@\textit{saṁskṛti}} who has clearly considered “othering”\index{othering} (denationalisation) as a defining feature of conversion to Christianity\index{Christianity} or Islam,\index{Islam} if not the religions themselves. He also clearly acknowledges alien (invading) nature of these “othering” religious philosophies in his choice of words, \textit{\textbf{national vs denationalised}}. (Keer\index{Keer, Dhananjay V.} 1954: 280–281) In fact, Ambedkar\index{Ambedkar, B. R.} was a sponsor for a constitutional amendment to make Sanskrit the national language of India. (Malhotra\index{Malhotra, Rajiv} 2016: 166)

While Hindus celebrate and celebrated the divine black (God Kṛṣṇa, Goddess Pārvatī, Draupadī, God Viṣṇ, all stone idols were and should be black, etc.) racism (“othering”) is and was celebrated in Persian and Sufi\index{Sufi poetry} poetry, and the much-maligned Hindus have been at the receiving end. On Rumi\index{Rumi} and his “love-filled” Sufi poetry:

\begin{myquote}
“One of the geographico-historical topics is the contrast of Turk and Hindu. It was used from the earliest days of Persian poetry; but it is interesting to see Rumi's application of this traditional pair of correlatives in his works, since the Turks are absolutely convinced that Mowlana* himself was a Turk, quoting one of his lines in favour of this claim. However,we shall probably never be in a position to reach any definite conclusion in this respect. Rumi's mother tongue was Persian, but he had learned, during his stay in Konya, enough Turkish and Greek to use it, now and then, in his verses. We may ask, therefore, how he represents the Turks, or the usual pair Turk-Hindu. To be sure, these words and combinations occur so frequently in his verses that one has to restrict oneself to some of the most characteristic passages from both the \textit{Divan}* and the \textit{Mathnavi}:
\end{myquote}

\begin{myquote}
There is one revealing poem, beginning with the lines:\\\textit{A \textbf{Hindu} came into }thekhanqah\textit{*.}\\\textit{'Are you not a Turk? Then }throw\textit{ him from the roof!'\\ Do you consider him and the whole of Hindustan as little, pour his special (part) on his whole (i.e. let him feel that he is part of \textbf{infidel Hindu India}). The ascendent of \textbf{India is Saturn himself},\\ and though he is high, \textbf{his name is Misfortune}.\\ He went high, but did not rescue (man) from misfortunes\\ What use has the bad wine from the cup?\\ I showed the \textbf{bad Hindu} the mirror:}\\\textit{Envy and wrath is not his sign...}\\...
\end{myquote}

\begin{myquote}
The \textit{nafs} is theHindu\endnote{ According to Sufi\index{Sufi philosophy} philosophy, ““ego” (nafs) is the lowest dimension of man’s inward existence, his animal and satanic nature” (Chittick 1983:12). However, it is an important concept as well in the gnosis (irfan) discipline in Shia Islam.\index{Islam}}, and the \textit{khanqah}* my heart...\\ The last \textit{hemistich} gives the clue: \textbf{the Hindu, always regarded as ugly, black, of evil omen (like the 'black' Saturn, the Hindu of the Sky, in astrology}), and as a mean servant of the Turkish emperors, is the \textit{nafs}, the base soul which on other occasions is \textbf{compared to an unclean black dog}. Yet, even the \textit{nafs} if successfully educated can become useful, comparable to the little Hindu-slave whose perfect loyalty will be recognized by the Shah.” 

~\hfill (Schimmel 1993: 193)
\end{myquote}

Thus one can clearly envision how, when and what would have led to the origin of castes-by-birth and caste groups, in the classicist Pollock's\index{Pollock, Sheldon} words “othering”.\index{othering} It must have and only been a foreign instrument of oppression, perhaps the earliest known use of divide and rule in India. Castes-by-birth (endogamy)\index{endogamy} that “abruptly” started during the Islamic\index{Islamic rule} rule of India had been nurtured, re-engineered and corrupted further into casteism and untouchability by the British and European colonial powers and their lackeys (missionaries, cultural sepoys, etc.) (Malhotra\index{Malhotra, Rajiv} and Neelakandan\index{Neelakandan, Aravindan} 2011)

The treatment of the “black” Hindus at the hands of the Islamic\index{Islamic invasion} invaders, particularly the Brahmins and especially with an aim to convert them to Islam,\index{Islam} is well documented by their own court historians. Ambedkar\index{Ambedkar, B. R.} highlights the treatment meted out the Hindus. Some highlights:

\begin{myquote}
“…they [muslims] were all united by one common objective and that was to destroy the Hindu faith…Mahommad bin Qasim's first act of religious zeal was \textbf{forcibly to circumcise the Brahmins} of the captured city of Debul; but on discovering that they objected to this sort of conversion, he proceeded to \textbf{put all above the age of 17 to death}, and to order all others, \textbf{with women and children, to be led into slavery}…The slaughtering of ‘infidels’ seemed to be one thing that gave Muhammad [of Ghazni] particular pleasure… \textbf{the Muslims paid no regard to booty until they had satiated themselves with the slaughter of the infidels} … \textbf{Most of the inhabitants were Brahmins with shaven heads. They were put to death}. Large number of books were found......... but \textbf{none could explain their contents as all the men had been killed, the whole fort and city being a place of study}.”
\end{myquote}

Further details in (Ambedkar 1941: 50–57) discuss how the temple at Multan was desecrated with “a piece of cow’s flesh”, the destruction of the “temple of Bishnath at Benares”, how “winning of converts became a matter of supreme urgency”, how brahmins of Old Delhi faced the choice of conversion or being burnt to death, and how thousands of Hindus were taken as slaves.

So was the door closed on non-Brahmins by Brahmins, or many people formed groups and closed their doors to others (Brahmins and government clerks) to save their daughters? Ambedkar\index{Ambedkar, B. R.} wrote further:

\begin{myquote}
“These edicts, says the historian of the period, “were so strictly carried out that the chaukidars and khuts and muqaddims \textbf{were not able to ride on horseback, to find weapon}, to wear fine clothes, or to indulge in betel...... No Hindu could hold up his head...... Blows, confinement in the stocks, imprisonment and chains were all employed to enforce payment.”
\end{myquote}

\begin{myquote}
… All this was \textbf{not the result of}mere \textbf{caprice or moral perversion}. On the other hand, what was done was \textbf{in accordance with the ruling ideas of the leaders of Islam\index{Islam} in the broadest aspects}.” 

~\hfill (Ambedkar 1941: 56–57)
\end{myquote}

Perhaps the selective and special degrees of brutality was saved for the Brahmins and their women, the Muslims of north India (excepting the old immigrant communities of Iranian Shias and Bohras living in India) are predominantly descendants of Brahmin women with minor contributions from the Middle East / Persia in their genes: population genetics\index{population genetics} evidence of sexual violence depicted by historians above. (Eaaswarkhanth \textit{et al.} 2009)

To rephrase and repeat a statement from above: Was the door closed on non-Brahmins by Brahmins, or many people formed groups and closed their doors to others (Brahmins and other groups specifically targeted for forced religious conversions) to save their daughters? It appears that the Brahmins were the ones who were “othered.”

\vspace{-.3cm}

\section*{Conclusions}

Clearly, Sanskrit is a victim of that “othering”\index{othering} process and claiming it as an instrument of oppression is an “attributional bias” and thus “amoral.” Such anti-empirical conclusion is arrived at by Pollock,\index{Pollock, Sheldon} a mere \textit{classicist}, using a putatively novel discipline of overtly textualised and materialistic dialectical philology, called political philology, which dehumanises by de-emphasising the emotional, empathetic and contextual aspects of language and the entire historic time periods thus studied. A de-stressing of empathetic (humanising) components can be viewed in a clinical psychological framework as the zero-negative empathy end of the empathy spectrum, seen in the pathology of psychopathy. Simon Baron-Cohen reports about an incident involving a crime and its perpetrator as follows:

\begin{myquote}
“Paul (not his real name, to protect his identity) is twenty-eight years old and is currently detained in a secure prison after being found guilty of murder. I was asked to conduct a diagnostic interview with him by his lawyer, and, because his violence meant it could have been unsafe for him to come to our clinic, I went to see him in the prison. He told me how he had wound up in jail. He insisted he wasn’t guilty because the man he stabbed had provoked him by looking at him from across the bar. Paul had gone over to the man and said, ‘Why were you staring at me?’ The man had replied, I assume truthfully: ‘I wasn’t staring at you. I was simply looking around the bar.’ Paul had felt incensed by the man’s answer, believing it to be disrespectful, and felt he needed to be taught a lesson. He picked up a beer bottle, smashed it on the table, and plunged the jagged end deep into the man’s face.
\end{myquote}

\begin{myquote}
Like me, the barrister at Paul’s trial was shocked by the apparent lack of remorse and the self-righteousness of his plea of not guilty. In my questioning I probed further for some evidence of moral conscience. Paul was adamant that he had simply defended himself.” 

~\hfill (Baron-Cohen 2012 Chapter 3, Paul: Type P)
\end{myquote}

Pollock's\index{Pollock, Sheldon} entire work is committed to demonstrate, against the above mountain of empirical and scientific evidence, that somehow Sanskrit and Hindus themselves are to be blamed for the infection that they are suffering from. Thence, one needs a completely novel approach to study the phenomenon called political philology and liberation philology. (Malhotra\index{Malhotra, Rajiv} 2016)

\newpage

It can be concluded, thus, mapping the dimensions of political philology onto a psychological framework of the empathy spectrum might provide us a novel perspective to understand the mechanics of hypothesis forming employed by its practitioners.


\section*{Bibliography}

\begin{thebibliography}{99}
\itemsep=0pt
\bibitem{chap10-key01} Ambedkar, B. R. (1941). \textit{Thoughts on Pakistan}. Bombay: Thacker and Co. Limited.

 \bibitem{chap10-key02} --- . (1946). \textit{Who Were the Shudras? How They Came to Be the Fourth Varna in the Indo-Aryan Society}. Bombay: Thacker and Co. Limited.

 \bibitem{chap10-key03} --- . (2014). “CASTES IN INDIA.” In \textit{Dr. Babasaheb Ambedkar: Writings and Speeches}. Vol. 1. New Delhi: Dr. Ambedkar Foundation.

 \bibitem{chap10-key04} Baron-Cohen, Simon. (2012). \textit{Zero Degrees of Empathy: A New Theory of Human Cruelty and Kindness}. London: Penguin.

 \bibitem{chap10-key05} Basu, Analabha., Sarkar-Roy, Neeta., and Majumder, Partha P. (2016). “Genomic Reconstruction of the History of Extant Populations of India Reveals Five Distinct Ancestral Components and a Complex Structure.” \textit{Proceedings of the National Academy of Sciences of the United States of America}113 (6). pp 1594–99.

 \bibitem{chap10-key06} Chittick, William C. (1983). \textit{The Sufi Path of Love.} New York: SUNY Press.

 \bibitem{chap10-key07} Dharampal. (1983). \textit{The Beautiful Tree: Indigenous Indian Education in the Eighteenth Century}. Delhi: Biblia Impex.

 \bibitem{chap10-key08} --- . (2000). \textit{Essays on Tradition, Recovery and Freedom}. Vol. V. DHARAMPAL – COLLECTED WRITINGS. Mapusa: Other India Press.

 \bibitem{chap10-key09} Dirks, Nicholas B. (2011). \textit{Castes of Mind: Colonialism and the Making of Modern India}. New Jersey: Princeton University Press.

 \bibitem{chap10-key10} Eaaswarkhanth, Muthukrishnan., Dubey, Bhawna., Meganathan, Poorlin Ramakodi., Ravesh, Zeinab., Khan, Faizan Ahmed., Singh, Lalji., Thangaraj, Kumarasamy., and Haque, Ikramul. (2009). “Diverse Genetic Origin of Indian Muslims: Evidence from Autosomal STR Loci.” \textit{Journal of Human Genetics} 54 (6). pp 340–48.

 \bibitem{chap10-key11} Keer, Dhananjay V. (1954). \textit{Dr. Ambedkar: Life and Mission}. Mumbai: Popular Prakashan.

 \bibitem{chap10-key12} Malhotra, Rajiv. (2016). \textit{The Battle for Sanskrit: Is Sanskrit Political or Sacred, Oppressive or Liberating, Dead or Alive?} Delhi: Harper Collins.

 \bibitem{chap10-key13} --- ., and Neelakandan, Aravindan. (2011). \textit{Breaking India: Western Interventions in Dravidian and Dalit Faultlines}. Delhi: Amaryllis.

 \bibitem{chap10-key14} Moorjani, Priya., Thangaraj, Kumarasamy., Patterson, Nick., Lipson, Mark., Loh, Po-Ru., Govindaraj, Periyasamy., Berger, Bonnie., Reich, David., and Singh, Lalji. (2013). “Genetic Evidence for Recent Population Mixture in India.” \textit{American Journal of Human Genetics} 93 (3). pp 422–38.

 \bibitem{chap10-key15} Negi, Neetu., Tamang, Rakesh., Pande, Veena., Sharma, Amrita., Shah, Anish., Reddy, Alla G., Vishnupriya, Satti., Singh, Lalji., Chaubey, Gyaneshwer., and Thangaraj, Kumarasamy. (2016). “The Paternal Ancestry of Uttarakhand Does Not Imitate the Classical Caste System of India.” \textit{Journal of Human Genetics} 61 (2). pp 167–72.

 \bibitem{chap10-key16} Pritchett, Frances. (2017). “Minute on Education (1835) by Thomas Babington Macaulay.” \textless \url{http://www.columbia.edu/itc/mealac/pritchett/00generallinks/macaulay/txt_minute_education_1835.html/}\textgreater. Accessed on 21 March 2018.

 \bibitem{chap10-key17} Sanyal, Sanjeev. (2016). \textit{The Ocean of Churn: How the Indian Ocean Shaped Human History}. Penguin UK.

 \bibitem{chap10-key18} Schimmel, Annemarie. (1993). \textit{The Triumphal Sun: A Study of the Works of Jalaloddin Rumi}. New York: SUNY Press.

 \bibitem{chap10-key19} Vadivelu, Murali K. (2016). “Emergence of Sociocultural Norms Restricting Intermarriage in Large Social Strata (endogamy)\index{endogamy} Coincides with Foreign Invasions of India.”\textit{Proceedings of the National Academy of Sciences of the United States of America} 113 (16): E2215–17.

 \bibitem{chap10-key20} Weiner, Mark S. (2013). \textit{The Rule of the Clan: What an Ancient Form of Social Organization Reveals about the Future of Individual Freedom}. Macmillan.

 \end{thebibliography}

\theendnotes

