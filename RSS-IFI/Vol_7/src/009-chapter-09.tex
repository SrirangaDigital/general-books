
\chapter{\hspace{1.8cm} A Rejoinder to \textit{A Rasa Reader}: \hfill\break An Insider View}\label{chapter9}

\Authorline{-- Sharda Narayanan}\footnote{pp. 275-300. In: Kannan, K. S and Meera H. R. (Ed.s) (2020). \textit{Chronology and Causation: Negating Neo-Orientalism.} Chennai: Infinity Foundation India.}

\begin{flushright}
\textit{(sharda.narayanan@gmail.com)}
\end{flushright}


\section*{1. Introduction}

India has a long and deep tradition of aesthetics that discusses the performing arts and literature. While it is not possible to explain all aspects in full detail in a single book, it is important to explain the most important concepts in a manner that even a reader who is new to the subject can appreciate and to place issues in perspective. But we find Pollock’s\index{Pollock, Sheldon} treatment of the subject rather biased without explaining the main concepts; while it is perfectly legitimate for any scholar to have his own opinions and preferences, it is not acceptable for an academic tome to misrepresent, distort and selectively explain issues according to the author’s prejudices. This paper attempts to analyze Pollock’s interpretations and translations in several specific issues and to showcase his methodology in distorting the tradition to suit his thesis.


\section*{2. Language and Style of Writing}

\textit{A Rasa Reader}\index{rasa@\textit{rasa}} (Pollock 2016) is part of a well-organized, well-funded enterprise to present Indian classical schools of thought to the Western world, presumably American scholars, students and general public. While there is not one new topic in Pollock\index{Pollock, Sheldon} (2016) that is not addressed in \textit{A History of Sanskrit Poetics} by P.V. Kane\index{Kane, P. V.} or the work of the same name by S.K. De,\index{De, S. K.} Pollock’s method is very different. He has also arranged the various writers and the translations of selected passages in the order that is conducive to the line of argument that he presents, while the texts themselves are not presented.

The annotations and comments on the developments in the classical aesthetic tradition provided by the author are often of a disparaging and prejudiced nature. Where the issue may have two sides to it, Dr. Pollock goes out of his way to portray it in negative light. It is not clear whether it is the only side he comprehends or whether it is deliberate.

For instance, from Section 1 of the Introduction -

\begin{myquote}
“For one thing, there was no unified sphere with a particular designation we could translate by the English term “art”. There were separate cultural domains of poetry (\textit{kāvya}),\index{kavya@\textit{kāvya}} drama (\textit{nāṭya}), music (\textit{saṁgīta}, consisting of vocal and instrumental music and dance), and less carefully thematized practices, with terminology also less settled, including painting (\textit{citra}), sculpture (often \textit{pusta}), architecture (for which there was no common term at all), and the crafts (\textit{kalā}), which could include many of the preceding when that was deemed necessary. In these disparate domains there was never any dispute, at least overtly, about what was and was not to be included, though sometimes works passed into and out of a given category according historically changing reading or viewing practices. Furthermore, almost everything outside the literary realm, let alone the cultural realm, remained outside classical Indian aesthetic analysis (including nature: though Shiva was a dancer, God in India was generally not an artist.)” 

~\hfill (Pollock 2016: Introduction, section 1)
\end{myquote}

The above passage reveals not only a warped outlook but also displays several factual errors. Texts of \textit{Śaiva Tantra} dating to the early centuries enumerate sixty-four \textit{kalā}-s or arts. (Jayaa 2006: 39). How convenient and novel to insert \textit{kalā} under “crafts”! It is a wrong translation and distorts the picture. \textit{Pusta} is also wrongly translated as sculpture; it is art put together with glue, paper (any cellulose material available at the time), cloth, sticks, paste and paint. Art, artist and artisan are closely related in any society and India was not any different to other cultures of the world in this. Pollock raises a non-issue as if it were of peculiar significance in India. Moreover, it is not anybody’s responsibility to convince a foreigner of the elegance of our views on Lord Śiva, but we can say that we have no reason to hold Pollock’s\index{Pollock, Sheldon} view as worthwhile. That he considers himself authority to pronounce these views is worth noting. As David Frawley writes, most Westerners do not go beyond the surface in what they see of Indian culture.

\begin{myquote}
“My intention in the present book is to help correct the distortions about Lord Shiva present in academic and popular accounts, which focus on the sensational side of Shiva and downplay his yogic implications. So far, few thinkers in the West have understood the profound wisdom and meditative insights behind such apparent figures of polytheism and nature worship as epitomized in Lord Shiva.” 

~\hfill (Frawley 2016: 13)
\end{myquote}

\begin{myquote}
“The western mind tends to reduce Shiva to iconographic and anthropomorphic appearances, which derive from a very different cultural milieu of ancient India.” 

~\hfill (Frawley\index{Frawley, David} 2016: 27)
\end{myquote}

Was there any form of art in Europe or elsewhere in the world (America did not count at that time) in the 3rd or 6th or 10th centuries that was not in India? Were there any abstract artists, just \textit{kalākāras}, anywhere in the Western world, without specific reference to the particular form of art? If you called a person an artist, would you not also be able to say what form of art it is that you mean? The term \textit{kalākāra} means artist. One who makes art is \textit{kalāṁ karoti}. In Sanskrit grammar the word would be derived as \textit{kalākara} but becomes \textit{kalākāra} to show that art is the person’s vocation and he is not just working at artistic activity for the moment.

The Veda-s and all subsequent literature celebrate God as the ultimate Creator, not in the sense of manufacturing something on an assembly line but artistically envisaging and conceptualizing before giving it form. He is referred to as the foremost “\textit{Kavi}”, artist with original creativity. Many are the poems that praise Brahmā’s artistic ability in having created the breathtakingly beautiful Sarasvatī, his first creation. A poet is called \textit{kavi} because he is truly endowed with the creative genius, creating art out of practically nothing. So not only is Pollock overstepping his authority in stating his view on what God in India generally was, but is also mistaken on the facts.

Simpler English is also seen in his writing, when he chooses, but it is not to explain a concept to advantage or portray the tradition in flattering light. Again, the merits of his observation are highly debatable and facts point out to their being ridden with error. Sample the following, from Section 1 of the Introduction.

\begin{myquote}
“As for questions of creativity and genius (\textit{pratibhā}),\index{pratibha@\textit{pratibhā}} Indian thinkers certainly were interested in them, but they never thought it necessary to develop a robust theory to account for their nature or impact on the work. Interpretation was never thematized as a discrete problem of knowledge in literary texts … Critical judgements were certainly rendered, … but literary evaluation itself was not framed as a philosophical problem. Last, while careful attention was directed to beauty (\textit{saundarya}), especially in literature (which does have a role to play in aesthetic reflection), beauty was typically disaggregated … never became an object of abstract consideration in and of itself.” 

~\hfill (Pollock\index{Pollock, Sheldon} 2016: Introduction, section 1)
\end{myquote}

This whole paragraph serves well to set the tone of Pollock’s study, but means next to nothing in every sentence. How much of abstract considerations can one discuss or share? In the very process of discussion, they cease to be abstract. Pollock’s observations are belied by the fact that Bhartṛhari\index{Bhartrhari@Bhartṛhari} has addressed these issues. \textit{Pratibhā} means flash of immediate (without mediation), intuitive illumination. \textit{Prati + bhā}, to shine, illumine. It is what enables the mind to take a decision, the creative function of the mind that can envisage and create poetry to convey that idea and emotion to others. The logical structure that it may have is beyond description.

\begin{myquote}
“It is clear from all this that, as conceived by Bhartṛhari, \textit{pratibhā }is something very comprehensive. It is a flash of understanding that takes in a situation and prompts one to do something to meet the situation.” 

~\hfill (Iyer 1969: 87)
\end{myquote}

\begin{myquote}
“Thus he (Bhartṛhari) compares \textit{pratibhā} to the latent \textit{śabda}\index{sabda@\textit{śabda}} in an infant, whence the ability to learn a particular language springs.” 

~\hfill (Narayanan 2012: 89)
\end{myquote}

Vāmana\index{Vamana@Vāmana} says that the very germ of poetic skill is creative genius, \textit{kavitvabījaṁ pratibhānam }1.16 (Shastri\index{Shastri, Haragovind} 1989: 37). This is an inborn characteristic that one has within oneself. Yes, Vāmana does say that it is something that the individual has brought along from a previous birth. This is to mean that it is an inherited characteristic, in modern parlance, and that it cannot be imparted to the person. Yes, Vāmana was not aware of DNA and RNA and how different random gene combinations can give rise to new characteristics. But even we cannot fully answer why two brothers may not be equally creative or talented and can only guess that their DNA must be different. Modern genetics is constantly developing and we never have all the answers we seek. In the same way Vāmana\index{Vamana@Vāmana} guessed that the poet may have brought his spark of genius with him at birth, the point being that it is inborn. Bhāmaha\index{Bhamaha@Bhāmaha} says that even the dull-witted can be taught the sciences (\textit{śāstra-s}) with the help of a good teacher, but the creative spark is rare indeed (Sastry 1970: 2). This does not warrant Pollock\index{Pollock, Sheldon} connecting poetic genius with transmigration, as he does towards the end of his Introduction.

As to the concept of beauty simply being a description of a thing of beauty disaggregated into its parts, it is not a correct observation. In fact, discussions go to great lengths to say that in the case of beauty, the whole is more than the sum of its parts and beauty is beyond the aggregate of its beautiful parts. A woman, for instance, may be perfect in every limb and yet not be attractive, but another who may bear some defect in a specific limb may be considered very beautiful despite the flaw! These discussions form a part of large treatises and it takes a discerning eye to spot these theoretical discussions amidst many topics.

Important issues in Indian thought in a serious academic tome such as Pollock (2016) are cursorily and perfunctorily mentioned in a rather condescending manner, just for the record whereas it is expected of a teacher to present a topic in proper perspective to students before reading translations of selected text sections, evaluating advanced discussions or embarking on value judgement. The book claims to be of interest to the general reader but does little to explain anything. The book derives its value and gravity solely from the presentation of passages from the classical tradition of India, considering that so few people, even among Indians themselves know anything on the subject.

\vspace{-.3cm}

\section*{3. Key Issues in Interpretation}

The next point in our discussion is whether Pollock has even understood important issues in proper perspective. Sanskrit language is such that it takes a great deal of sagacity to understand the import of words such as \textit{rasa, nyāya, dharma, bhāva, vyakti, kārya} etc in different contexts. What looks like the same word need not give the same meaning in two different sentences. We first examine here the \textit{Rasa} theory\index{rasa@\textit{rasa}}\index{Rasa Theory} formulated in our earliest extant source, the \textit{Nāṭyaśāstra}.\index{Natyasastra@\textit{Nāṭyaśāstra}}

\vspace{-.3cm}

\subsection*{3.1 \textit{Rasa} Theory in Drama as per Indian Tradition}

\vspace{-.3cm}

The Rasa theory is introduced in the context of a stage performance and the \textit{Nāṭyaśāstra }refers to even more ancient verses on \textit{bhāva} and \textit{rasa}. The performance is naturally conceived and rehearsed with the purpose of giving enjoyment to the audience only upon which it can be called successful. The spectator is taken to a high level of aesthetic appreciation by dramatic devices consisting of story, characters, plot, music, dance and dialogue. Why does the spectator become moved to an emotional climax? The stimulation is after all not real, staged and in no way affects the real-life interests of the spectator. The answer is the art experience called \textit{rasa} or \textit{nāṭya rasa} wherein the emotional reaction of the spectator reaches a climax. All of us have certain latent emotions within us. When a story or drama strikes these chords, we are able to empathize with the characters in it. When our feelings are fanned and taken to an acme by the turn of events, there is \textit{rasa} experience. Way back in the \textit{Nāṭyaśāstra,} Bharata has shown that aesthetic experience can be a great stress-buster, providing relief to those worried by daily cares and sorrows of the world (Rangacharya\index{Rangacharya, Adya} 1986: 4).

The latent or dormant emotions in every human being are many; those that can be raised to a climax through the process of \textit{nāṭya} are known as \textit{sthāyi-bhāva-}s.\index{sthayibhava@\textit{sthāyi-bhāva}} These are love, anger, grief, vigour, fear, revulsion, laughter and amazement. Many other emotions would boil down to one of these upon analysis, e.g. envy or jealousy would ultimately be depicted by anger at the rival’s success. Not all emotions can prevail for the duration of the performance and be taken to a zenith.

Pollock\index{Pollock, Sheldon} asks why there should be only eight \textit{sthāyi-bhāva}-s, why sentiments such as maternal love or hatred should not be included in the list and this leads him to question whether, after all, the tradition even understands what constitutes a \textit{sthāyi-bhāva }(Pollock 2012: 197). To this we may remind the reader that Bharata speaks of \textit{bhāva} in \textit{nāṭya} – dramatic emotions and is not listing all possible human emotions. \textit{Sthāyi-bhāva-s} are only those emotions which can be fanned to a crescendo, leading to \textit{rasa} experience. Maternal love is a real, powerful human (and animal) emotion that exists in the real world. Upon dramatization, the audience may feel empathy and should the story lead to the maternal sentiment being hurt, it evokes \textit{karuṇa rasa}\index{rasa@\textit{rasa}} which does not correspond to maternal love as \textit{sthāyi-bhāva}.\index{sthayibhava@\textit{sthāyi-bhāva}} Jealousy, or hatred, for example are other human emotions not listed under \textit{sthāyi-bhāva} as their depiction would depend on anger at the rival’s success, etc, which is already listed. While the Sanskrit writers of yore were admittedly experts in classification, they tempered it with common sense, divorced from which the modern scholar’s conclusions border on the absurd.

\textit{Rasa} is said to be the cumulative result of \textit{vibhāva}\index{vibhava@\textit{vibhāva}} (stimulus), \textit{anubhāva}\index{anubhava@\textit{anubhāva}} (reaction) and \textit{vyabhicāri-bhāva}\index{vyabhicaribhava@\textit{vyabhicāri-bhāva}} (transitory states), the mood that the performance evokes. The famous \textit{rasasūtra} states - “\textit{vibhāva-anubhāva-vyabhicāri-saṁyogāt rasa-niṣpattiḥ}\dev{।}" (Dwivedi 1996: 34).

When the audience beholds the empty stage (or blank movie screen in modern times) at the moment the curtain goes up, there is great anticipation and curiosity, but the spectator does not have any direction for his emotions to proceed in yet. The \textit{vibhāva-s} are the determinants that stimulate a particular mood and give direction to the series of emotions. For example, the appearance of a clown with ungainly attire and wobbly gait, leering and rubbing his belly would excite hilarity. The appearance of a young, comely lady, smiling and laughing, with downturned glances in the company of a smiling, attentive young man would elicit interest in a love theme. But if a tender lady was unsuspecting and the handsome young man came in stealthily, brandishing a sword with evil intent, it would arouse suspense and trepidation – not thoughts of love – in the audience. If the scene was dimly lit and some hideous monsters appeared in the shadows, it would incite fear or \textit{bhayānaka rasa}. Modern cinema has decided advantages in technological simulations, but even way back in ancient times, drama managed its resources well enough to depict \textit{adbhuta}, the wonderful. \textit{Vibhāva-s }are of two types, \textit{ālambana}\index{alambanavibhava@\textit{ālambana-vibhāva}} (primary or foundational) and \textit{uddīpana}\index{uddīpanabhava@\textit{uddīpana-vibhāva}} (inflaming or fanning) (Narayanan and Mohan 2017: 44).

The \textit{anubhāva-}s are the ensuant states of mind in different conditions. Unless the characters react emotionally to the situation, with the actor’s mind concentrated on the sentiment, the portrayal will not be convincing.

The \textit{vyabhicāri-bhāva-}s\index{vyabhicaribhava@\textit{vyabhicāri-bhāva}} are the changing moods enacted by the actors in different situations to convey different feelings to the spectator so that the plot can be developed, culminating in \textit{rasa}.\index{rasa@\textit{rasa}} These are enumerated as thirty-three, for the convenience of the actors to train and practice honing their skills and their number is not vital to the aesthetic theory.

As academic scholars of the present century evaluating the merits of the discussions in ancient treatises, we are like armchair critics and require greater imagination to understand what the words mean to those in the field, such as an actor, playwright or stage director. Literature, \textit{kāvya}\index{kavya@\textit{kāvya}} was divided into two categories, \textit{prekṣya}\index{preksya@\textit{prekṣya}} (to be viewed) and \textit{śravya}\index{sravya@\textit{śravya}} (to be heard), but \textit{rasa} was not divided into something that could be seen and something that could be heard. It is aesthetic, emotional relish whereby beauty is savored. Pollock’s\index{Pollock, Sheldon} caption of \textit{rasa} seen and heard is to be taken with a pinch of salt and the treatises do not substantiate that view. Nowhere is it said that \textit{rasa} is visually perceptible or visible, nor is it audible. He insists on translating \textit{rasa} as “taste” everywhere, although there are many words that suit it better. \textit{Rasānubhava}\index{rasānubhava@\textit{rasānubhava}} and \textit{rasāsvāda}\index{rasasvada@\textit{rasāsvāda}} correspond to “\textit{rasa} experience,” not “taste”. Pollock wrongly translates ‘\textit{pratyakṣa-kalpa-saṁvedanā}’ quoting from \textit{Abhinavabhāratī}\index{Abhinavabharati@\textit{Abhinavabhāratī}}\index{Abhinavabharati@Abhinava-bhāratī} as “visual perception” (Pollock 2012: 191). The word “visual” is not warranted: perception includes all the senses and here Abhinavagupta\index{Abhinavagupta} is clearly speaking of literature leading to \textit{rasa} experience when the reader is able to imagine the situation as if real, not “seeing” it as a picture.

The account of “\textit{rasa} seen” in drama discounts the effect of rhythm, music, songs and pregnant pauses that build up mood, anticipation and aesthetic delight. Can one watch a performance with ear plugs on? Bharata speaks of the tempo of \textit{nṛtta} (pure dance) building up the mood and beauty of the performance, hence aiding attainment of \textit{rasa} in addition to \textit{nṛtya} which includes \textit{āṅgika abhinaya} or acting.

Pollock (2016) gives new translation of \textit{vibhāva}\index{vibhava@\textit{vibhāva}} as “factor” but does not explain it at all. Anything that has any bearing on the equation at hand is a “factor” and the word conveys no particular information. Adya Rangacharya,\index{Rangacharya, Adya} for instance, translates it as “stimulant,” as it determines the emotional response of the spectator (\textit{vi+bhāva = viśeṣeṇa bhāvayati}). He writes that \textit{vibhāva}\index{vibhava@\textit{vibhāva}} is that which leads to a perception; so \textit{vibhāva} is a cause (Rangacharya\index{Rangacharya, Adya} 1986: 55, 64). In going into some detail about \textit{śṛṅgāra},\index{srngara@\textit{śṛṅgāra}} “sexual instinct” is mentioned in Pollock (2016) as if it were of exceptional importance in the Indian system. “Sexual instinct” is common to almost all forms of life and does not merit special attention in a book on art and aesthetics, but Pollock goes into only this basic emotion as \textit{sthāyi-bhāva},\index{sthayibhava@\textit{sthāyi-bhāva}} apparently unable to explain any other. He does not connect any of the components to show that they combine to form a comprehensive whole. He writes -

\begin{myquote}
“From such an analytical perspective the play looks like a jumble of disconnected components…... They are ultimately combined into a whole, where each component is at once preserved and subsumed,...” 

~\hfill (Pollock\index{Pollock, Sheldon} 2016: Introduction, section 3)
\end{myquote}

Referring to the play as a “jumble” of its components is as good as referring to blue cheese as a jumble of casein, mould, fungus and maggots.

The allegation that the \textit{Nāṭyaśāstra}\index{Natyasastra@\textit{Nāṭyaśāstra}} only speaks of \textit{rasa}\index{rasa@\textit{rasa}} in the character, not in the spectator, that the latter notion appeared only later in the tradition and hence classical aesthetics did not have any theory of aesthetics in the abstract sense, wherein aesthetic enjoyment was discussed, is untenable. Please note the words, “discerning \textbf{viewers} relish the stable emotions when they manifest by the acting...” -- \textit{nānābhāvābhinayavyañjitān ….sthāyibhāvān āsvādayanti sumanasaḥ \textbf{prekṣakāḥ} harṣādīṁśca adhigacchanti~{\dev ।}} - \textit{Nāṭyaśastra} (Dwivedi 1996: 90) in the sentences immediately following the first discussion on \textit{rasa }in Chapter VI.

The \textit{Nāṭyaśāstra} has maintained from the beginning that \textit{rasa} is developed in the performance, and that the connoisseur partakes of it by \textit{sādhāraṇīkaraṇa}.\index{sadharanikarana@\textit{sādhāraṇīkaraṇa}} As a modern-day parallel, can we not see the late Michael Jackson imbued with aesthetic ecstasy when he performs “Thriller” or “Heal the World” and do we not feel it too? Can it be said that it is in one and not the other?

Pollock explains his reasons for choosing his own translations in favor of those more widely used. In the Preface he says -

\eject

\begin{myquote}
“For the same reason and in the hope of recovering a sense more faithful to the tradition, I have sometimes rejected a widely used translation –“love in separation” for example, in favour of “the erotic thwarted” which reflects the aesthetic system’s own understanding of \textit{vipralambha śṛṅgāra}.”\index{srngara@\textit{śṛṅgāra}} 

~\hfill (Pollock 2016: Preface)
\end{myquote}

It is interesting that Pollock should consider himself closer to the sense more faithful to the tradition and able to reflect the aesthetic system’s own understanding (by his own assessment). Love in separation or yearning, gives one sense, but “thwarted” presents a distorted picture. While love thwarted may be unfulfilled, love that is yet to be fulfilled is not necessarily thwarted. \textit{Vipralambha} only means “not in communion” and does not warrant translation as “thwarted”. In the least, it fails to evoke the sense of aesthetic portrayal of the yearning of love. There is nothing “thwarted” about the states of \textit{vāsakasajjā nāyikā} or \textit{abhisārikā nāyikā}, important aspects of love in separation, highly celebrated conditions of \textit{śṛṅgāra}, which Pollock’s\index{Pollock, Sheldon} definition clearly does not cover.


\subsection*{3.2 \textit{Rasa} in Poetry}\index{rasa@\textit{rasa}}

Coming to poetry in Chapter One it is said -

\begin{myquote}
“And rasa as first theorized for literature in performance was emotion the spectator could \textit{see}.” 

~\hfill (Pollock 2016: Chapter One, section 1.2)(\textit{italics as in the original})
\end{myquote}

This is a fundamentally wrong notion and has no basis; it is Pollock’s innovation. Presenting the issue of literature that is heard as identical with literature that is read in private is only part of the picture. Pollock, by his own declaration, ignores the embellishments of sound, as unnecessary fuss and focuses only on the meaning of poetry to study its accomplishment of \textit{rasa} for the reader. He may not find embellishments of sound of any worth, but the fact is, the sound of an utterance is as vital as its meaning in its aesthetic appeal. Poetry heard also includes variations in tone and other forms of expression which are not available in the private reading. This may be part of Pollock’s strategy to belittle the sound of chanting of the Veda-s; for him, the Veda-s are text on paper and would be represented by their meaning, perhaps even in translation. That may be his view and what he can grasp, but our tradition has always maintained that the audible aspect is what comprises language and the written form is but an aid to recall it, as recognized by early writers on Sanskrit literature such as A. B. Keith.\index{Keith, A. B.}

\begin{myquote}
“We are apt to regard with styles largely by the sounds preferred by different writers, but there is no doubt that the effects of different sounds were more keenly appreciated in India than they are by us, and in the case of the Gitagovinda the art of wedding sound and meaning is carried out with such success that it cannot fail to be appreciated by ears far less sensitive than those of Indian writers on poetics.” 

~\hfill (Keith 1928:195)
\end{myquote}

In Section 4.3 of the Reader -

\begin{myquote}
“…when Bana… exclaims how hard it is to produce a beautiful poem and make “its rasa\index{rasa@\textit{rasa}} clear,” he is referring to emotions in the text, not its impact on the reader.” 

~\hfill (Pollock\index{Pollock, Sheldon} 2016: Introduction, section 4.3) (\textit{spelling as in the original})
\end{myquote}

This is highly disputable as it can be better translated as “with its rasa clear” and refers to the \textit{rasa }in the poem equally to that in the \textit{sahṛdaya};\index{sahrdaya@\textit{sahṛdaya}} of what worth is the \textit{rasa} in the composition unless it is relished by the reader? Emotions cannot be said to lie in the text as smoothness or roundness in a stone sculpture. It is hard to differentiate between the \textit{rasa} in the text imbued by the poet, and the \textit{rasa} in the connoisseur. The two are united in heart, in the experience that has arisen in the creator’s mind and is experienced by the spectator or reader, the \textit{sahṛdaya.} Treatises discussing aesthetic experience do not differentiate between the two. P. V. Kane\index{Kane, P. V.} writes on this verse that \textit{rasa} is that sentiment which rules a composition and which is the object of the poem to present to the mind of the reader (Kane 1918: 148).

In Section 9 Pollock says -

\begin{myquote}
“...since the time of Bhāmaha,\index{Bhamaha@Bhāmaha} the view that the cultivation of literature produces pleasure, but also “instruction” - - in this context always instruction in the four “ends of man”, love, wealth, morality and spiritual liberation with the two outcomes equally balanced. This old view came to be embodied in the very definition of rasa at a relatively early stage.” 

~\hfill (Pollock 2016: Introduction, section 9)
\end{myquote}

\newpage

Pollock seeks to show that the conceptualization and ratification of \textit{rasa} was enmeshed with instruction in the \textit{puruṣārtha-}s,\index{purusartha@\textit{puruṣārtha}} but that was not the case. Fine literature was described as not only giving pleasure, but also offering some beneficial instruction in addition, viz., learning about the ways of the world so as to advance one’s own \textit{puruṣārtha-s}.\index{purusartha@\textit{puruṣārtha}} Bhāmaha\index{Bhamaha@Bhāmaha} and other early writers take pains to state that \textit{kāvya}\index{kavya@\textit{kāvya}} should NOT seek to instruct, but only to charm and that any instruction should be a purely incidental gain. Fine literature was required to have something more to offer than pure pleasure so as not to be vacuous. Indian writers never went to the extent of modern writers such as Lev Tolstoy who said that beauty should satisfy our moral nature in purity and embody the virtue of truth and justice (Srinivasachari 1958:12).

To look into \textit{śṛṅgāra}\index{srngara@\textit{śṛṅgāra}} closely, consider the situation when a particular spectator may feel aroused at viewing Rāvaṇa proposition Sītā (or Kīcaka, Draupadī) in drama; this is more a reflection of the person’s individual proclivity than the aesthetic theory; normally, a spectator would feel revulsion and fear. \textit{Rasa} theory\index{Rasa Theory}\index{rasa@\textit{rasa}} would only accept it as \textit{śṛṅgāra} if Rāma were to express love towards Sīta or Kṛṣṇa towards Rādhā. Pleasure admittedly has many degrees, low, middling, high and supreme. The \textit{rasa} theory says that the highest aesthetic experience is one of pure bliss – it does not dictate that no person can or will feel aesthetic pleasure as a response to vulgar art.

V. M. Kulkarni discusses \textit{Rasa} theory and \textit{puruṣārtha}-s in a whole chapter of the same name and quoting Abhinavagupta\index{Abhinavagupta} writes that even of instruction in the four goals of human life, \textit{ānanda} (delight) is the final and major result (Kulkarni 1998: 89). Many writers in poetics have named \textit{vyutpatti} as one of the requirements of a poet and also one of the gains to a reader of poetry. \textit{Vyutpatti} is knowledge of the ways of the world and the cultural contexts; in other words, the sense of what is deemed right and wrong, which would also be equivalent to what could advance one’s success in the four yardsticks or goals of life, which are also called \textit{puruṣārtha}-s. So, this is an issue of common occurrence in the real world, not one of orthodoxy or dogma peculiar to Indian society. Pollock\index{Pollock, Sheldon} is unable to explain the practical value of \textit{vyutpatti }and uses it to show that \textit{rasa} theory is obsolete, belonging only to the ‘pre-modern’ era.

\eject

Consider -
\begin{myquote}
“...if Scripture commands us like a master, and history counsels us like a friend, literature seduces us like a beloved.” 

~\hfill (Pollock 2016: Introduction, section 9)
\end{myquote}

This is an incorrect translation, if it refers to Mammaṭa’s famous Verse 2 of \textit{Kāvyaprakāśa},\index{Kavyaprakasa@\textit{Kāvyaprakāśa}} which says that literature “counsels like a beloved” -\textit{ kāntā-sammitatayopadeśayuje }(Jha 1966: 2). Mammaṭa’s\index{Mammata@Mammaṭa} \textit{vṛtti} further says that while the \textit{śāstra-s }(technical treatises) command like a master, \textit{itihāsa}\index{itihasa@\textit{itihāsa}} (\textit{Purāṇa-}s\index{purana@\textit{purāṇa}} and epics) teach like a friend, literature counsels sweetly and gently like a beloved. “Seduces” is not warranted here. It is shocking that a scholar of Pollock’s\index{Pollock, Sheldon} credentials should err on a simple translation in order to present a perverted picture.

The relevance of “propriety” is also exaggerated and misrepresented in the Reader. Propriety is not so much a moral value as cogency in attainment of \textit{rasa}.\index{rasa@\textit{rasa}} Aberrations that jar, defects that mar and detract from the lucid flow of thought should be avoided in order to achieve \textit{rasa}, that is all. To put this in perspective, it would be quite distracting and considered outside of propriety to portray in English drama a peasant addressing a young Queen of England as “My bonny lass” in her court, not so much because it is improper on the yokel’s part but because it may offend the audience’s sensibilities.

The \textit{Nāṭyaśāstra}\index{Natyasastra@\textit{Nāṭyaśāstra}} speaks of the hero adhering to propriety so that the audience is able to give him free and frank admiration, lending themselves fully to enjoy the play. As an example in modern literature, if young Harry Potter had not exhibited a fine sense of ethics, there may not have been any sequel to the first story! Propriety was more a pragmatic issue than any characteristic of \textit{rasa}. To be sure, there would have been instances of debauched men and women indulging themselves in pleasure of possibly immoral and indecent situations but these do not form part of discussions on poetics. The \textit{rasa} of aesthetics was by definition impersonal, not pertaining to personal sensory pleasures. To understand what was within propriety and what was not, familiarity with the cultural ethos of the play would be required, as anybody who tried reading Walter Scott without knowing something of Scottish history would attest!

\vspace{-.3cm}

\subsection*{3.3 \textit{Rasa} in \textit{Bhakti}}\index{bhakti@\textit{bhakti}}

With the advent of the Bhakti Movement all over India, philosophical discourse shifted from highly sophisticated, technically intricate debates of Nyāya,\index{Nyaya@Nyāya} Mīmāṁsā,\index{Mimamsa@Mīmāṁsā} Advaita\index{Advaita} Vedānta, Buddhism,\index{Buddhism} etc, into a more emotional path, where the devotee attained bliss by immersion in devotion to the Supreme. With devotional \textit{stotra-s, bhajan-s, satsaṅg-s}\break and dance becoming very popular, \textit{bhakti}\index{bhakti@\textit{bhakti}} was the \textit{rasa} that people evoked in every nook and corner of the country, through Sanskrit and vernacular poetry. Naturally, \textit{bhakti} was delineated as the prominent \textit{rasa}, with its own \textit{vibhāva}-s\index{vibhava@\textit{vibhāva}} and \textit{anubhāva}-s,\index{anubhava@\textit{anubhāva}} in addition to the traditional nine, by writers such as Rūpa Gosvāmi. It is not certain that this development warrants a heading such as “No Rules for the Number of Rasas”. Metaphysics and allegories are often combined in Indian art and literature. When saints such as Caitanya Mahāprabhu\index{Caitanya Mahaprabhu@Caitanya Mahāprabhu} are regarded as the incarnation of Rādhā and Kṛṣṇa, when devotees are said to have been \textit{gopī-}s in their previous births, so blessed as to partake of the Lord’s company and love, rather than checking on the veracity of such statements, the modern scholar should pause to reflect on the gravity of the notion for those who value them and move on if he does not accept their explanations. It is relevant and coherent for those who do understand. We cannot ask, as Pollock\index{Pollock, Sheldon} does (Pollock 2016: Introduction, section 7), why the language of aesthetics is used to describe the devotees’ relationship to God.


\subsection*{3.4 Who is a \textit{Rasika}?}\index{rasika@\textit{rasika}}

After much study, there are some fundamental questions he raises in Section 9 that ought to have been answered earlier -

\begin{myquote}
“Nothing said so far, however, explains how viewers and readers are able to taste rasa\index{rasa@\textit{rasa}} in the first place and to grasp its social-moral logic.... How, in short, does a \textit{rasika}, a person able to taste rasa, come to be a \textit{rasika}?... And after all, what special training is required for getting lost in a book or film? Perhaps more than we know, since although it may seem to be a natural human capacity, Indian thinkers saw “nature” quite otherwise. A \textit{rasika} may largely be born, not made, but who is born a \textit{rasika}?” 

~\hfill (Pollock 2016: Introduction, section 9)
\end{myquote}

It has already been discussed that \textit{rasa }had no social-moral logic. That is a misinterpretation by Pollock, who often sees things “otherwise”. Two issues are discussed closely together in the treatises and he has not been able to sift them.

It is good to relate an issue with a modern example as discussions then become clearer. To be sure, anybody can intuitively enjoy a book or movie and be lost to the world. But it is well-known that if one is taught how to review a book, one would understand it so much more deeply. The same goes for movie-making. A person who knows something about direction or story technique or some other detail of movie-making might admire it even more. An average person can appreciate popular or ‘light’ music, but it takes a connoisseur to understand classical music. And the same applies to a \textit{rasika} of literature. The more one knows about its different aspects, the more one is charmed by fine poetry. The training and guidance that go into a poet-aspirant benefit the \textit{rasika}\index{rasika@\textit{rasika}} equally well.

Now, the question would arise why two people may never acquire the same skill, no matter how much training is directed at them. Talents differ hugely. One person seems gifted and quickly becomes adept; while another may simply have to choose another line, in the arts, in sports, in vocations even. (It is well-known that most of the great artists of the Western world underwent much training and apprenticeship before their creative genius could dazzle. Would Pollock\index{Pollock, Sheldon} call Michelangelo a “craftsman”?) The same goes for a sensitive spectator, a \textit{rasika}. There are some people so gifted as to deeply appreciate the creative genius’ endeavor and others who may not be so moved by it. So, in short, in addition to acquired characteristics brought by experience, a person is a function of his inborn talents and sensibilities.

Pollock translates the name of the text “\textit{Sahṛdaya Darpaṇa}”\index{sahrdaya@\textit{sahṛdaya}}\index{Sahṛdayadarpana@\textit{Sahṛdaya Darpaṇa}} as “Mirror of the Heart” which would be better suited as translation of “\textit{Hṛdaya Darpaṇa}”;\index{Hrdayadarpana@\textit{Hṛdaya-darpaṇa}} the “\textit{Sa}” in “\textit{Sahṛdaya}” appears to be left out in his translation! “\textit{Sahṛdaya},” meaning, “those who sympathetically respond to poetry in their own hearts, or sensitive spectator” (Kulkarni 1998: 6), is a very important concept in Indian aesthetics, which Pollock ignores.


\subsection*{\textit{Saṁskāra} – Memory Impressions}\index{samskara@\textit{Saṁskāra}}

Pollock makes much of the mention of memory of past lives and pretends not to see the main point, calling \textit{rasa}\index{rasa@\textit{rasa}} “contingent” on it. He says -

\begin{myquote}
“There is no doubt a good answer to the obvious question why the endless cycle of transmigration would not eventually endow all people with all predispositions, but our thinkers do not provide it.” 

~\hfill (Pollock 2016: Introduction, section 9)
\end{myquote}

In reply, we can say that they did not provide what was common knowledge that the endless cycle would constantly endow those predispositions that the person’s actions merited and no other! There is more common sense than mysticism in classical discussions on \textit{rasa} which can be explained completely in down-to-earth terms, requiring no other-worldly notions. If Pollock translates \textit{saṁskāra} as memory of past lives, he is on the wrong track. He does not appear to know the word \textit{saṁskāra} which may be behind the following statements -

\begin{myquote}
“To understand rasa\index{rasa@\textit{rasa}} as a historical form of thought, however as I try to enable the reader of this \textit{Reader} to do, is to confront a theory clearly contingent on a nonmodern worldview and understanding of literary art. Its full conceptualization is intimately tied to a number of primary, uncontested, and largely non-transferable Indian pre-suppositions - about the threefold psychophysiology of Samkhya, for example, or the storage of memories of past lives, or even transmigration.” 

~\hfill (Pollock\index{Pollock, Sheldon} 2016: Introduction, section 12)
\end{myquote}

\vspace{-.2cm}

\begin{myquote}

~\hfill (\textit{spelling and italics as in the original})
\end{myquote}

In the above statements, Pollock himself admits that he does not understand the concept of \textit{rasa }and gives us leave to discount everything he says in the Reader. He wrongly renders \textit{sattva, rajas}\index{sattva@\textit{sattva}}\index{rajas@\textit{rajas}} and \textit{tamas}\index{tamas@\textit{tamas}} as “sensitivity, volatility and stolidity”. These are only a few features of the three \textit{guṇa-}s and cannot be said to represent them, just as orange colour cannot represent sunlight. There is nothing in modern science that negates in essence the Sāṁkhya philosophy that the world is made up of matter and spirit, that the myriad variety we see in the insentient world is due to different permutations and combinations of the three characteristic features although the primordial material is one.

\eject

\textit{Saṁskāra}\index{samskara@\textit{saṁskāra}} is a primary, well-accepted and characteristically Indian concept that is very important to the understanding of \textit{rasa}. It is memory, the mental impression formed by any experience and aids in cognitive and motor functions. We are not after all, like sensory automatons, but constantly use stored experience in all our activities – speaking a language, cooking, reading, driving to the supermarket, making purchases...anything. When the experience stored by \textit{saṁskāra} is recalled, it is called remembrance. It even helps in the grasp of a sentence, as we hear only one syllable or sound at a time and the whole utterance is a collection of all the evanescent phonemes uttered in temporal sequence. When the last syllable reaches the ear, the first one is lost but we still grasp the whole sentence using \textit{saṁskāra} of all the phonemes. Repeated practice of singing, or playing a musical instrument, or a sport, or revising our lessons creates the \textit{saṁskāra} that enables us to become expert in our chosen field. A person who is cultured and refined, as the outcome of training and education, and having refined taste is also said to have well-cultivated \textit{saṁskāra}.\index{samskara@\textit{saṁskāra}}

In the case of aesthetic response, \textit{saṁskāra} is the emotional baggage that each person carries with himself or herself into the auditorium. It goes without saying that a young person in love, a fond mother, a young man going through heartbreak, a war widow, or a recently bereaved son may all respond differently to the scenes portrayed in the play according to their own emotional make-up. The ideal spectator, Bharatamuni says should have a clean slate and be totally receptive, but the influence of \textit{saṁskāra} in fact is very difficult to avoid. The same dramatic representation can evoke varying emotional response due to \textit{saṁskāra}. \textit{Saṁskāra} is not the same as \textit{vāsanā}.\index{vasana@\textit{vāsanā}}

India was not unique in operating on “uncontested notions”; it is more common than we may admit as would be apparent if, for instance, the construction of the magnificent statues of Mary, Queen of Scots and her cousin Queen Elizabeth I, representing those monarchs lying entombed beneath in Westminster Abbey, their hands folded in eternal prayer, were to be “etiolated”.


\section*{4. Misleading Subheadings}

The subheadings, which are meant to influence the reader’s perspective, give a negative connotation to the development of the tradition, suggesting that the evolution was confused, lacking a logical progression, as the relevant details are not provided. In fact, the contents of each section are not even consistent with the subheadings, which seem more like the author’s annotations rather than true representation of the discussion. While the passages presented in the section represent the developing tradition, the headings and subheadings form implicit value judgement on the development. The \textit{Rasa} Theory\index{rasa@\textit{Rasa}}\index{Rasa Theory} of Bharata as applicable to dramatic performance was universally accep\-ted across the centuries, as it is relevant even today, in the best of Broadway productions. \textit{Rasa} as the essence of ecstatic aesthetic joy was later discussed in the context of literary appreciation. It is not as if one rasa changed into the other or that the later definition replaced the earlier. If a person delights in the experience of viewing a theatrical performance, can we say that he is not capable of being moved by literature - a poem that he can read? In subsequent centuries when devotion took a prominent position in the lives of the people (and they viewed fewer dramatic performances) through \textit{satsaṅgs} and \textit{bhajan}s, \textit{bhakti}\index{bhakti@\textit{bhakti}} was delineated with much importance as a \textit{rasa},\index{rasa@\textit{rasa}} in addition to the traditional nine. Pollock’s\index{Pollock, Sheldon} subheading, “No Rules for the Number of Rasas” is almost gleeful at finding a seeming flaw in the old arguments of the tradition that did not appear to understand its own rules and forgot to seek permission from the West! The Sanskrit tradition is explicit in holding that \textit{śāstra} should be in accordance with the realities of the world and that the world will not obey the dictates of \textit{śāstra}.

\vspace{-0.3cm}

\subsection*{The Means to Cognition – \textit{Pramāṇa}-s}

\vspace{-0.3cm}

Indian philosophy is clinical in its precision of its discussions and one of the most important topics of discussion is \textit{pramāṇa} - the means to cognition. But even these rigorous \textit{pramāṇa}-s which are valid in the real world we live, are said not to explain aesthetic experience! It is beyond the logic of phenomenal or transactional world. This experience is admittedly different to any other in the real world which is ruled by empirical experience and practical considerations. Therefore, \textit{rasa} experience was termed \textit{alaukika}\index{alaukika@\textit{alaukika}} – beyond the real world and its rhyme and reason! In the context of aesthetics, the real world is considered \textit{laukika}\index{laukika@\textit{laukika}} – mundane. But art is given a superior ontological status, not explained by the rigorous \textit{pramāṇa}-s. Nyāya\index{Nyaya@Nyāya} philosophy clearly states that poetry is outside the ambit of its discussions on language when it comes to \textit{vyañjanā vṛtti} or suggestion. After promising “fresh translations” on key words, Pollock insipidly repeats the word “mundane” for \textit{laukika}, perhaps being uncertain of its full import.


\subsection*{Mundane and Supermundane}

In Section 6 of the Introduction, Pollock writes

\begin{myquote}
…”for this “savoring” of rasa, or “rapture”, as he calls it, Abhinava reserves the qualification “supermundane”. But even this assessment, and much of the understanding of literature that accompanied it, was to be overturned in the coming centuries.” 

~\hfill (Pollock 2016: Introduction, section 6)
\end{myquote}

Pollock\index{Pollock, Sheldon} is unable to annotate or explain what “supermundane” might mean. It is a very important concept that is essential to the \textit{rasa} concept and it has never been overturned. While it is generally agreed that \textit{rasa} experience is \textit{alaukika}\index{alaukika@\textit{alaukika}} there were other thinkers who tried to liken the aesthetic pleasure to other experiences in the world, pointing out its \textit{laukika}\index{laukika@\textit{laukika}} aspect. We are dealing with art experience here and no two people’s experience need be identical. Two shades of meaning do not make an issue self-contradictory or illogical. The word \textit{laukika} is used in different connotation. The fact that there are a number of views on the subject only point to the depth of discussion and analysis and it is not as if a writer is contradicting himself. If one theory is accepted as the only rigid view, it may not conform to every single person’s experience as different people may be in different stages of aesthetic sensibilities. Bharatanatyam Guru Natyavidushi Jayaa translates “\textit{alaukika}” as “spiritual”.

\begin{myquote}
“… and there are four levels of experiencing beauty.….At the level of spiritual search for meaning, the artiste is on a quest to know the deeper meaning of life and to represent this search and its results, if any, in works of art. It recognizes a reality beyond the senses, beyond imagination, and beyond the visible truths of joy and sorrow, and certainly beyond the considerations of saleability and markets. The work is an effort to realize the divine.” 

~\hfill (Jayaa 2016: 271)
\end{myquote}

There were writers who delineated rasa\index{rasa@\textit{rasa}} experience in its similarity to that in real life, as art mirrors reality in many ways; there were others whose aesthetic experience was at a different level and who therefore pointed out the transcendental level of art experience as \textit{alaukika}, Abhinavagupta\index{Abhinavagupta} being the foremost among them. If a modern scientist were asked to choose between a model of the world as made up entirely of atoms and molecules, or one with gravity waves or one with matter waves or one solely with electromagnetic waves or one made up entirely of quarks, which one should he select as the only true one? Later, in Section 7 we find -

%\newpage

\begin{myquote}
..``But here too, disagreement among later commentators, including one in the sixteenth century who boldly rejects Mammata’s\index{Mammata@Mammaṭa} position, shows the growing inadequacy of such an appraisal”. 

~\hfill (Pollock 2016: Introduction, section 7)
\end{myquote}

Variations in view on a subject such as art appreciation and aesthetics point to richness of thought rather than “growing inadequacy”. In today’s world, we have discordant theories on topics such as model of the universe, model of the atom, cosmology, economics, trade embargoes, politics, business practices, etc. to name a few. How can a large country with a long history produce concurrent writing across the centuries so that each concept is neatly formulated and packaged for the modern man’s consumption or rejection, rather like manufacture on a factory assembly line? On one hand Pollock\index{Pollock, Sheldon} says that no theory was satisfactory, but were a treatise to endorse a prior thinker’s views, he complains that the writer had nothing to add.


\section*{5. Why the Penchant for Gloom?}

Tracing the intellectual history of rasa\index{rasa@\textit{rasa}} by arranging passages concerning rasa in an order convenient to present Pollock’s line of argument is not enough to truly comprehend how the performing arts and literary styles evolved over the centuries. Change is the sign of life, and the discussions on aesthetic appreciation evolved along with changes in the trend in the arts. Changes in the theory did not stem from inherent flaws. To fully trace the development of \textit{rasa,} its significance in society and validity in human psychology is also necessary.

\eject

A good Rasa Reader ought to have a more detailed study involving historical and social conditions that make rasa relevant in every epoch. Instead, Pollock presents Indian society as confused, holding weird notions. To be sure, ancient and even medieval (and perhaps even today) India has had its share of superstitious beliefs, as anywhere in the world. Modern physics and chemistry have influenced the outlook of all people on this globe. But these do not form a part of logical discussions on poetics and are not relevant. We only need to understand human emotions and their response to stimuli. Although we live in a modern world with amenities very different to that of ancient times, our emotional responses have not changed as much as our technological advancements and hence classical Indian aesthetics still have much to offer, being as relevant today as two millennia ago. The use of electric lights, microphones, loudspeakers and mechanized curtain do not change the principles of stagecraft many of which were recorded in the \textit{Nāṭyaśāstra}.\index{Natyasastra@\textit{Nāṭyaśāstra}}

Pollock’s language is unnecessarily gloomy. In Section 3 of the Introduction he refers to “the demise of dramaturgical theory after about the thirteenth century”. He makes no reference to the many Sanskrit texts on performing arts which focused on music and dance, as theatre had moved in that direction by the thirteenth century. Drama had moved more into the vernacular languages, but was in no danger of demise. There are other places where Pollock\index{Pollock, Sheldon} makes some rather strange, meaningless, baseless and unwarranted statements, such as, in Section 8 of the Introduction “It was the Buddhists who invented compassion - and that is not the \textit{karuṇa} of aesthetic discourse.” Are any such sudden, inexplicable statements of any relevance in this discussion? Pollock appears to build on his own pre-conceived notions. There are many places where he expresses his views with no reference to the basis of the notion. He provides no quotations for us to verify the translation. This issue is discussed in the paper, “From Rasa\index{rasa@\textit{rasa}} Seen to Rasa Heard” also, which discussion we find utterly confounding. The relevance or the import of this innovative enquiry ascribing new meaning and motive to the notion of “compassion” is baffling. Bharata and Abhinavagupta\index{Abhinavagupta} are quoted on their views on what would constitute pathos in a performance, but Pollock mentions only the words “pity” and “compassion” and analyzes the Indian psyche based on his own translation of these passages.

Even stranger is the following paragraph from Section 10 -

\begin{myquote}
“Theory is related, however obscurely, to practice, and the history of rasa theory\index{Rasa Theory} roughly maps against the history of practice of Sanskrit literature - understanding “literature” in the sense accorded to the category in Sanskrit culture itself. In that sense, Sanskrit literature was an invention of the beginning of the Common Era,…” 

~\hfill (Pollock 2016: Introduction, section 10)
\end{myquote}

This paragraph truly indicates what thin ice Pollock is skating on and makes one question his ability to comment upon any classical study of India. Nowhere is theory derived independently of practice. A rule-book for administration of a pedagogic institution, for instance, would be formulated based on what a good administrator found effective, by trial and error. The rules may be recorded for the purpose of clarity, uniformity in case of change in administration or even to aid in managing franchised institutions. A gifted composer or writer may produce several successful works before theorists begin to study and describe his style. In modern science, research findings always corroborate theory. The cuisine of a region develops with the ingredients available in that geographical location and the cultural practices peculiar to the place after which gourmets speak of the distinct flavours of the region. Similarly, a language develops gradually based on which its grammar is systematized. In music and dance too, as in literature, once the artistic urges have created a distinct style, theory is formulated to describe it.

Theory is always preceded by practice. Only after the practice attains a state of some maturity can its theory be formulated. For example, in Sanskrit literary styles of expression, \textit{rīti}, the treatises tell us that initially there were three styles \textit{Vaidarbhī, Gaudīyā and Pāñcālī}, named after the geographical regions where they originated, but were practiced everywhere over time, owing to their popularity (Shastri\index{Shastri, Haragovind} 1989: 16).

Pāṇini\index{Panini@Pāṇini} has meticulously recorded the state of the language as it was spoken in his time. It is certain that classical Sanskrit must have flourished for a few hundred years prior to his treatise. When Bharata wrote or compiled the \textit{Nāṭyaśāstra},\index{Natyasastra@\textit{Nāṭyaśāstra}} it is certain that the codified practices of the Sanskrit theatre must have flourished for at least a hundred years before he began to compile them. Pāṇini explains words that mean professional actor, dancer, etc. There is no possibility of Sanskrit literature being an invention of the beginning of the Common Era, whatever be the sense in which Pollock\index{Pollock, Sheldon} uses the phrase. The literature or performing art tradition of a culture and a civilization is not to be referred to as an “invention”.

\vspace{-.3cm}

\section*{6. Universality of Indian Concepts}

It is a fact that classical thought in India on the internal working of the mind was very advanced and intricate. From linguistics to aesthetics to yoga\index{Yoga} and processes of cognition, the analysis of the human mind is unparalleled. This makes many issues discussed in the Sanskrit traditions take on universal relevance, not just pertaining to Indian society. And as the human mind has changed little in two or three millennia, many issues are as relevant today as when they were first formulated. The profundity of many concepts such as \textit{ātman,\index{atman@\textit{ātman}} manas, buddhi, dhyāna, śabda, rasa}\index{sabda@\textit{śabda}} etc. cannot be ignored. They have spread to Europe and other parts of the world through translations in the middle centuries and have influenced the development of thought in those countries, far more than is acknowledged.

It is not surprising then, that some scholar should attempt to show how rasa theory\index{Rasa Theory}\index{rasa@\textit{rasa}} can be applied in today’s scenario, in the context of the plays of Shakespeare, for example. Just as a study of \textit{rasa} from being seen to being heard has been for Pollock,\index{Pollock, Sheldon} this evaluation of \textit{rasa} theory may have attracted the scholar for its novelty and scope of academic study. But Pollock does not appear to be pleased with such tendencies – in Section 12 of the Introduction he writes:

\begin{myquote}
“…There is a proclivity in a certain strain of postcolonial thought to assert claims to conceptual priority: the precolony is always supposed to have preempted colonisation in its theoretical understanding of the world. This is demonstrated for classical Indian aesthetics by awarding it a kind of superior insight and universal applicability (“Rasa in Shakespeare” is the genre of study that I have in mind).” 

~\hfill (Pollock 2016: Introduction, section 12)
\end{myquote}

Perhaps much of the misconceptions modern people have is based on the notion that Sanskrit grammar, \textit{dharmaśāstra}-s and other works were \textit{prescriptive} texts. They were in fact \textit{descriptive} texts and point out to the prevalent practices and their justification at that time. It is not the fault of the \textit{śāstra-s} if a modern scholar does not grasp them in proper perspective. As Yāska\index{Yaska@Yāska} said way back in the 8th century B.C.E., it is not the pillar’s fault if the blind man does not see it! \textit{naiṣa sthāṇoraparādho yadenamandho na paśyati \dev{।} puruṣāparādhaḥ sa bhavati}! (Sarup 1967: 115)


\section*{7. Conclusion}

Classical Indian thought has always attracted intellectual quest among foreign scholars. But where the intentions may not be honorable, it seriously changes the color of things. Even more than in literature, in academic study, unbiased presenting of information is expected of a teacher. If Pollock (2016) is a harbinger of things to come, as a first in a series that will address philosophy, religion, linguistics, etc, it does not bode well at all.

Sanskrit treatises on virtually every topic are the tip of the iceberg, with a vast body of literature in vernacular languages that is hidden forever from us. Even if a manuscript were to be found intact, there is scarce the scholar who can decipher that ancient script or interpret what it means. But the Sanskrit text shines through the centuries yielding a wealth of information on those times. There was not one idea or concept in philosophy or science that was not known to the common man; there were no secret concepts for secret cults. It is very wrong to hold Sanskrit tradition as anything but representing society as a whole.

India’s hospitality should extend to sharing our knowledge systems with those who are interested but not to the extent of accepting anything said by anybody as the outcome of partial understanding or hidden agenda. We owe it to posterity to preserve our traditional knowledge systems, so that they have a fair chance at evaluating them and perhaps derive benefit. In pursuing novelty in Indological research, let not the Western scholar’s vision of Goddess Lakṣmī be portrayed as Goddess Sarasvatī to us! If at the end of several brilliant Western careers, we have nothing left of culture to show our children, what do we gain with our riches?

“From Rasa\index{rasa@\textit{rasa}} Seen to Rasa Heard” (Pollock\index{Pollock, Sheldon} 2012) is a novel, innovative way to phrase the study, but it is nothing more than that, an attractive caption. Pollock (2016) is simply a thesis of the author despite its limitations and appears convincingly arranged but lacks depth and balance required to be considered as serious study material. Its greatest contribution may be waking up more people to the issue of Indian aesthetics. The book itself could have been titled, “From Rasa Seen to Rasa Heard” for, a text book on \textit{Rasa} it is not.


\section*{Bibliography}

\begin{thebibliography}{99}
\itemsep=0pt
\bibitem{chap9-key01} Dwivedi, Parasnath. (Ed.) (1996). \textit{Nāṭyaśāstra of Bharatamuni with Abhinavabhārati, Part II - Chapters6-11}. Varanasi: Sampoornanand Sanskrit University.

 \bibitem{chap9-key02} Frawley, David. (2016). \textit{Shiva - The Lord of Yoga}. New Delhi: New Age Books.

 \bibitem{chap9-key03} Guenzi, Caterina., and d’Intino, Sylvia. (Ed.s) (2012)\textit{. Aux abords de la clairière}. Paris: Brepols.

 \bibitem{chap9-key04} Jayaa, Guru Natyavidhushi. (2016). \textit{Nruthya Lakshanam}. Bengaluru: Jayaa Foundation for the Promotion of Performing Arts.

 \bibitem{chap9-key05} Jha, Ganganatha. (Ed.) (2005). \textit{Kāvyaprakāśa of Mammaṭa}. Delhi: Bharatiya Vidya Prakashan.

 \bibitem{chap9-key06} Kane, P.V. (Ed.) (1918). \textit{Harṣacarita of Bāna Bhaṭṭa}. Delhi. Motilal Banarsidass.

 \bibitem{chap9-key07} — . (1971). \textit{History of Sanskrit Poetics}. Delhi: Motilal Banarsidass.

 \bibitem{chap9-key08} \textit{\textbf{Kāvyālaṅkāra}}. See Sastry (1970).

 \bibitem{chap9-key09} \textit{\textbf{Kāvyālaṅkārasūtra}. }See Shastri (1989).

 \bibitem{chap9-key10} \textit{\textbf{Kāvyaprakāśa}}. See Jha (2005).

 \bibitem{chap9-key11} Keith, A. Berriedale. (1928). \textit{A History of Sanskrit Literature}. London: Oxford University Press.

 \bibitem{chap9-key12} Kulkarni, V.M. (1998). \textit{An Outline of Abhinavagupta’s Aesthetics}. Ahmedabad: Saraswati Pustak Bhandar.

 \bibitem{chap9-key13} Narayanan, Sharda. (2012). \textit{Vākyapadīya: Sphoṭa, Jāti and Dravya}. New Delhi: DK Printworld.

 \bibitem{chap9-key14} Narayanan, Sharda., and Mohan, Sujatha. (2017).\textit{ Gītagovinda of Jayadeva}. Chennai: Ambika Aksharavali.

 \bibitem{chap9-key15} \textit{\textbf{Nāṭyaśāstra}}. See Dwivedi (1996).

 \bibitem{chap9-key16} \textit{\textbf{Nirukta}}. See Sarup (1967).

 \bibitem{chap9-key17} Pollock, Sheldon. (2012). “From Rasa Seen to Rasa Heard.” In Guenzi \textit{et al} (2012). pp. 189-207.

 \bibitem{chap9-key18} — . (2016). \textit{A Rasa Reader}. New York: Columbia University Press.

 \bibitem{chap9-key19} Rangacharya, Adya. (1986). \textit{Nāṭyaśāstra of Bharata}. Delhi: Munshiram Manoharlal.

 \bibitem{chap9-key20} Sarup, Lakshman. (Ed.) (1967). \textit{Nighaṇṭu tathā Nirukta} (Hindi). Delhi: Motilal Banarsidass.

 \bibitem{chap9-key21} Sastry, P.V. Naganatha. (Ed.) (1970).\textit{ Kāvyālaṅkāra of Bhāmaha}. Delhi: Motilal Banarsidass.

 \bibitem{chap9-key22} Shastri, Haragovind. (Ed.) (1989). \textit{Kāvyālaṅkārasūtra of Vāmana.} Varanasi: Chaukhamba Surabharati Prakashan.

 \bibitem{chap9-key23} Srinivasaschari, P. N. (1958). \textit{The Philosophy of the Beautiful}. Madras: The Mylapore Library.

 \bibitem{chap9-key24} Subramania Iyer, K.A. (1969). \textit{Bhartṛhari}. Pune: Deccan College.

 \end{thebibliography}

