
\chapter*{Our Contributors \namesinorder{(in alphabetical order of last names)}}\label{contributors}

\section*{Vishal Agarwal}

Vishal Agarwal is an Indian-American scholar and author who writes on Indian philosophy, history and contemporary Indian society. He has published his critiques on the works of scholars like Romila Thapar, Michael Witzel, Martha Nussbaum, Wendy Doniger and Paul Courtright. He has been deeply involved in the California textbook issue where he has championed the cause of Hinduism, campaigning to remove the anti-Hindu bias in the textbook contents.


\section*{Ravindra Joshi}

Ravindra Joshi was an Engineering research and simulation professional with over 20 years of experience in global MNCs. He had an M.S. in Mechanical Engineering and an M.S. in Engineering Management from Drexel University, Philadelphia. He was Board Member of WAVES (World Association of Vedic Studies) and was also the co-founder of the online Medha Journal \url{https://www.medhajournal.com/}. Long time resident in the US, Ravi was also actively involved in the local Hindu temple and was a teacher, mentor and founder of the decade old Sanatana Dharma school. He had also presented talks and papers at many conferences, including WAVES, HMEC, and the first Swadeshi Indology conference held in 2016 at IIT Chennai. He unfortunately succumbed to cardiac arrest in June 2019.

\newpage

\section*{Megh Kalyanasundaram}

Megh Kalyanasundaram is an Indian citizen with close to nine years of lived experience in China and is an alumni of ISB. His professional experience includes a stint as Market Leader at a Fortune 40 technology firm and he has served a term on the Board of a Shanghai-based not-for-profit. His research interests currently include eurocentrism, decolonization, ancient India in global and transnational history narratives with a focus on some aspects of ancient Indian chronology. Other professional pursuits have included building differentiated digital platforms of Indic texts targeted at specific learning and research needs, singing and composing (including history-specific songs). He was invited by the Indian Permanent Mission at the United Nations in recognition of his contribution to some aspects of the launch of the First International Day of Yoga (IDY) and has subsequently contributed to the IDY campaign of multiple Indian missions abroad. \url{https://independent.academia.edu/MeghKalyanasundaram}


\section*{Sharda Narayanan}

Dr. Sharda Narayanan received Ph.D in Sanskrit from JNU, New Delhi and holds Master's Degrees in Physics and Sanskrit from Bangalore University. She has several paper publications and presentations to her credit at national and international seminars. Having co-authored \textit{Sastradipika - Tarka Pada} and \textit{Gita Govinda of Jayadeva}, she has assisted in editing two volumes of papers presented at SI-3 Conference in Chennai where she currently teaches Indian Philosophy and Aesthetics.


\section*{Nilesh Nilkanth Oak}

Nilesh Nilkanth Oak is an author and writes extensively on ancient Indian history at \url{https://nileshoak.wordpress.com/}. He is currently working on 4 books and has written two books in the past which are critically acclaimed. \textit{When did the Mahabharata War Happen? The Mystery of Arundhati} was written in 2011 and was nominated for the Lakatos award given annually by the LSE. His second book \textit{The Historic Rama }was written in 2014. Nilesh holds an MBA and an M.S. in Chemical Engineering and is also Adjunct Assistant Professor at School of Indic Studies, Institute of Advanced Sciences, Dartmouth, MA, USA. He resides in Atlanta, GA, USA.



\section*{Arvind Prasad}

Dr. Arvind Prasad has a PhD in Materials Science from the University of Alberta, Canada. He is an Honorary Research Fellow at the University of Queensland, Australia. His interest lies in studying India’s history – in particular the 9 darshans of India, their role in shaping the Indian society and India’s contribution to the world – which he does in his spare time. Apart from the current paper on Swadeshi Indology, he has also performed a comparative study of fundamental arithmetical operations contained in early Indian mathematics and in the Vedic Maths of Shankaracharya Sri Bharati Krishna Tirthaji. He is currently pursuing a degree in Teaching from the University of Queensland.


\section*{Manogna H. Sastry}

Manogna Sastry is a Swadeshi Indology research scholar and published author; she is a Master of Science from the Indian Institute of Astrophysics, with a strong background in theoretical physics and mathematics. Her research interests encompass consciousness studies and civilizational studies centred around India, including focussed aspects of chronology and desacralisation. Manogna is a passionate environmentalist, involved with the solid waste management issues of Bengaluru, as well as a keen gardener. Her published papers can be accessed at:  \url{https://independent.academia.edu/ManognaSastry}.


\section*{M. V. Sunil}

Sunil Upasana is a writer and blogger with a keen interest in philosophy. He primarily writes in his native tongue, Malayalam and has authored three books to date, one a collection of short stories, the second an autobiography and the third a philosophy book which introduces Indian philosophy for a beginner. His philosophy articles have been published in various weeklies in Kerala. Holder of a technical degree, with a Diploma in computer hardware maintenance, he is now pursuing his B.A. in philosophy from IGNOU. After a stint in various IT companies like HCL Infosystems, he is now a freelancer living in Bengaluru.


\section*{Murali K. Vadivelu}

Murali is a medic and a graduate of the University of Cambridge, England. Currently involved in medical biotech entrepreneurship and inter-disciplinary research (population genetics and historical sociology of Bharat) relevant to the inculcation of a scientific rigour in the outdated fields of humanities, especially in the Indian academic environment and thus inspired skewed mainstream media discourse: putting science into social sciences.


\section*{Rajath Vasudevamurthy}

Dr. Rajath Vasudevamurthy is an Assistant Professor at B.M.S. College of Engineering, Bengaluru in the ECE department from January 2019. He has a PhD from the Department of Electrical Communication Engineering at the Indian Institute of Science, Bengaluru and a Post Doc from the Pennsylvania State University, State College, USA. He is a keen student of Saṁskṛta, Vedānta and History of Science; and actively involves himself in related learning and propagation activities.

