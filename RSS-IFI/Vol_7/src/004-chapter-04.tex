
\chapter{The Upaniṣad-s: The Source of the Buddha’s Teachings}\index{Buddha, the}\label{chapter4}

\Authorline{-- M. V. Sunil$^{\ast}$}\footnotetext{*pp. 135-162. In: Kannan, K. S and Meera, H. R. (Ed.s) (2020). \textit{Chronology and Causation: Negating Neo-Orientalism.} Chennai: Infinity Foundation India.}

\lhead[{\small\thepage}\quad\small M. V. Sunil]{}

\begin{flushright}
\textit{(sunilmv@gmail.com)}
\end{flushright}

\setcounter{endnote}{0}

\section*{Abstract}

Hinduism and its core concepts are facing many challenges today through the misinterpretation and distortion at the hands of Western academicians who are neither practitioners nor insiders. Age-old traditions of Indic civilization are subjected to scrutiny using defective methods and recast with new interpretations and dimensions. New theories which are alien to the civilization are coming forth from various quarters. One such recent ‘discovery’, proposed by American Indologist Sheldon Pollock,\index{Pollock, Sheldon} is that Buddhism,\index{Buddhism} a prominent religion of Dharma tradition, is opposed to Hinduism. Prof. Pollock and his school of neo-Orientalist\index{Neo-Orientalist} scholars, proponents of this theory, take minor differences existing between the two religions in the \textit{vyāvahārika} world, and give it an absolute meaning. They forget or willfully disregard that there are two degrees of expression, about Reality, in Indic tradition – \textit{vyāvahārika} and \textit{pāramārthika}.\index{paramarthika@\textit{pāramārthika}} While certain differences are common in \textit{vyāvahārika} level, in the \textit{pāramārthika}, everyone’s aim is same – be one with the Reality, be it \textit{Brahman}\index{Brahman@\textit{Brahman}} or \textit{Nirvāṇa}.\index{nirvana@\textit{nirvāṇa}} And this Reality is not external to the body, but internal. In this paper, I intend to do a thorough philosophical evaluation of the two, to conclude that the foundation of Hinduism and Buddhism is the same. Both are well rooted in the Vedic tradition, especially the Upaniṣad-s. This is evident by the outlook of the Upaniṣad-s and the Buddha’s\index{Buddha, the} teaching. I will strive to show that a philosophical research, rather than an evaluation of external ritualistic methods and arguments, will lead us to the conclusion that the Buddha’s teaching is almost the same as Upaniṣadic teaching, but in a new terminology. The Highest Truth represented by the Upaniṣad-s and the Buddha shares similar aspects, only the names are different. I will demonstrate that the two planes of Reality that we get from the Buddha’s teaching are also well represented in the Upaniṣad-s. In contrast to the common belief, I will attempt to show how the Buddha’s theories of Dependent Origination (\textit{pratītya-samutpāda})\index{pratityasamutpada@\textit{pratītya-samutpāda}}, No-Soul (\textit{anātman}\index{anatman@\textit{anātman}}) are also not against the Vedic tradition. In fact, it is clear that they are in harmony with the Upaniṣadic teaching.


\section*{Introduction}

It is often remarked by some scholars that, there are only two religions in the world – Hinduism and Judaism\index{Judaism}. The rest are offshoots of these two religions. Hence, Hinduism is considered to be the mother of Buddhism,\index{Buddhism} Jainism\index{Jainism}, and Sikhism\index{Sikhism} while Judaism of the other Abrahamic religions. Only these two religions have robust and independent foundations. Other religions depend on these religions for their existence, myths, ritual, lore and theology. This is a commonly accepted norm in comparative studies in religions.

Buddhism is the second religion that emerged from the Hindu/Dhārmic tradition; the first being Jainism. Buddhism starts with the teachings of Siddhārtha,\index{Siddhartha@Siddhārtha} the Buddha. Siddhārtha after attaining \textit{Nirvāṇa},\index{nirvana@\textit{nirvāṇa}}\break preached the Truth to the popular masses. This teaching is often alleged to be different from, or diametrically opposed to the then existing beliefs and customs of the masses, namely, the Vedic tradition. Though we may accept that there may have been some minor differences, because of the rise of a new sect, or say religion, great care must be taken before concluding that the new sect was totally opposed to the old traditions, because Buddhism in its outlook and tradition does not differ radically from the Indic tradition. This is very evident when we compare the culture ofthe predominantly Buddhist nations with India. The sharp contradictions, in certain matters, that Buddhism has with Vedic tradition are also, in fact, not raised by the Buddha himself. Take the example of the Buddha’s objection towards ritual sacrifices. It is the Upaniṣad-s, which first showed opposition to sacrifices. The Buddha continued to take that opposition forward vigorously. The \textit{pratītya-samutpāda}\index{pratityasamutpada@\textit{pratītya-samutpāda}} and \textit{anātman}\index{anatman@\textit{anātman}} theories of the Buddha are also related to the Upaniṣad-s. The core teaching of the Upaniṣad-s (from the Advaita\index{Advaita} viewpoint) as indicated in the expressions like \textit{tattvamasi, ahaṁ brahmāsmi}, are ultimately against the concept of an individual \textit{ātman}.\index{atman@\textit{ātman}} We are \textit{Brahman}\index{Brahman@\textit{Brahman}} at the ultimate level. Individuality is a product of \textit{avidyā}.\index{avidya@\textit{avidyā}} The Buddha’s\index{Buddha, the} \textit{Nirvāṇa}\index{nirvana@\textit{nirvāṇa}} and \textit{anātman} concepts also mirror the same ideas.

Deliberately disregarding this relationship between the Vedic and the Buddha’s teachings, Western Indologists like Pollock\index{Pollock, Sheldon} have attempted to erect a wall between the two. They assert that the Buddha’s teachings were opposed to the Veda-s. In this paper I attempt to counter their fallacious theories with proper arguments and evidence.

This paper contains sections that can nullify certain theories of Mr. Pollock. In the next two sections I give a short description of ‘Vedism Vs Buddhism’,\index{Buddhism} as an entry into the subject, and the two main philosophical tenets taught directly by the Buddha. This will be helpful to set the tone for the positions that will be discussed in this paper. Pollock’s critique of Hinduism and the Veda-s is also added for clarity. His comments on the topic are quoted. The next four sections are the core of this paper. In these I show that \textit{pratītya-samutpāda} and \textit{anātman} doctrines of the Buddha are not against Upaniṣadic teaching, but emerged from it. I quote the opinion of eminent scholars who are insiders to validate my points. The immense parallels between \textit{Nirvāṇa} and Upaniṣadic \textit{Brahman}, that make for a compelling case to state that both are same, are touched upon next. The claim that the Buddha rejected the authority of the Veda-s and the existence of two planes of Reality in the Upaniṣad-s and the Buddha’s teaching are discussed in the remaining sections. The paper then discusses the implications of the fallacious theories propounded by the neo-Orientalists and finally ends with a conclusion.

\eject

Thought development is a continuous process. So when we start from the \textit{Ṛgveda},\index{Rgveda@\textit{Ṛgveda}} the oldest of our \textit{śruti}\index{sruti@\textit{śruti}} texts and proceed to the Upaniṣad-s,\break which is the Vedānta, there is a refinement in the interpretation of the Veda-s, from the ritualistic to the philosophical. This shows the dynamism of Vedic society and its evolving capacity. We cannot choose a particular part of the Vedic compendium, compare it with the Buddha’s teaching and then conclude that the Buddha was against the Veda-s. We should evaluate the core of Vedic thought with the teaching of the Buddha.\index{Buddha, the} Only then the research and study will be impartial, and output will be balanced. Such an attempt is made here.


\section*{Vedism Vs Buddhism}\index{Buddhism}

\subsection*{Vedism}

The means to attain/realize the Ultimate Reality mentioned in Vedic literature are mainly two – \textit{karma}\index{karma@\textit{karma}} and \textit{jñāna mārga}-s\index{jnanamarga@\textit{jñāna mārga}}. It is very important to note that one way to \textit{mokṣa}\index{moksa@\textit{mokṣa}} never rejects the other in this worldview. Instead, one \textit{mokṣa-mārga} legitimates the other by giving it an evolutionary role. It is only the degree of importance given to each that differs.

Among the Vedic texts, the Veda-s and \textit{Brāhmaṇa}-s predominantly reflect realism. In them, the existence of \textit{prakṛti}\index{prakrti@\textit{prakṛti}} on its own terms is recognized and upheld. Gods are invoked and their help is requested for prosperity and fight against opponents. A dualism between man and nature, man and God is visible there. However, it would be wrong to assume that, the idea of monism is not present in the Veda-s. Even while worshipping multiple gods, Vedic people were sure that these gods are just manifestations of the One Reality\endnote{ They call him Indra, Mitra, Varuṇa, Agni, and he is heavenly nobly-winged Garutman.
~~To what is One, sages give many a title they call it Agni, Yama, Mātariśvan.

\vspace{-.3cm}

\begin{flushright}
\textit{- Ṛgveda.}\index{Rgveda@\textit{Ṛgveda}} 1.164.46
\end{flushright}
\vspace{-0.3cm}}. Thus we can see a glimpse of monism, which later gets thoroughly expanded upon in the Upaniṣad-s. While Vedic injunctions primarily give importance to actions (\textit{karma}), similar importance is given to knowledge (\textit{jnāna}) in the Upaniṣad-s.

It is also worth noting that the Upaniṣad-s did not approve of sacrifices. Also, Upaniṣadic statements like ‘\textit{tattvamasi}’, ‘\textit{ayamātmā brahma}’, if we took them in the ultimate sense, do not allude to the caste-class divide. When taking a stand that \textit{All this is} \textit{Brahman / sarvaṁ khalv  idaṁ brahma},\index{Brahman@\textit{Brahman}} the meaning to be inferred is that everyone in the world irrespective of caste and creed is divine in the ultimate sense. Division exists only at the \textit{vyāvahārika} level, where \textit{avidyā}\index{avidya@\textit{avidyā}} exists.


\subsection*{Buddhism}\index{Buddhism}

Buddhism,\index{Buddhism} in its early period, was more or less a sect than a religion, established by its celebrated founder the Buddha. His teachings resembled those in the Upaniṣad-s, but in a different terminology. He was very liberal in matters of caste, though in some \textit{sūtra}-s (\textit{Ambattha sutta})\index{Ambattha sutta@\textit{Ambattha sutta}} he seems to show a preference for \textit{kṣatriya}\index{ksatriya@\textit{kṣatriya}} over others.

The \textit{anātman}\index{anatman@\textit{anātman}} concept of the Buddha\index{Buddha, the} does not accept the existence of any unchanging constant principle in the \textit{vyāvahārika} world. But in the highest plane he also upholds a state akin to the Ultimate Reality, which he terms as \textit{Nirvāṇa}.\index{nirvana@\textit{nirvāṇa}} It is pointed out by many scholars that there are many similarities between the Upaniṣadic and the Buddha’s teachings\endnote{ “Buddha\index{Buddha, the} carries on the tradition of absolutism so clearly set forth in the Upaniṣads. For both, the Real is the Absolute which is at once transcendent to thought and, immanent in phenomena. Both take \textit{avidyā},\index{avidya@\textit{avidyā}} the beginningless and cosmic Ignorance as the root-cause of phenomenal existence and suffering. Both believe that thought is inherently fraught with contradictions and thought-categories, \textit{instead of} revealing the Real distort it, and therefore, one should rise above all views, all theories, all determinations, all thought-constructions in order to realize the Real. For both, the Real is realized in immediate spiritual experience. Both prescribe moral conduct and spiritual discipline as means to realize the Real, the fearless goal, the abode of Bliss”.

\vspace{-.3cm}

\begin{flushright}
(Sharma\index{Sharma, Chandradhar} 2007: 31-32)(\textit{spelling and italics as in the original})
\end{flushright}
\vspace{-0.3cm}}.

\vspace{-.3cm}

\section*{Two Main Philosophical Concepts of the Buddha}

There are some fundamental doctrines of the Buddha upon which the Buddhist belief and philosophy is built. Most important among them are ‘\textit{pratītya-samutpāda}’\index{pratityasamutpada@\textit{pratītya-samutpāda}} (theory of depended origination) and ‘\textit{anātman}’ (no-soul). These two are considered to be the kernel of the Buddha’s teachings.

\textit{Pratītya-samutpāda} theory states that \textbf{when this is, that is. From the arising of this, comes the arising of that. When this is not, that is not. From the cessation of this, comes the cessation of that}. The simple meaning of \textit{pratītya-samutpāda} is that, things in the mundane world arise depending upon other things. When this thing ceases to arise, the other thing also ceases. The law of causation\index{Law of Causation} is inherent in this doctrine.The real import of \textit{pratītya-samutpāda} is believed to be, as adopted by Mahāyānists, the theory of relativity.

Another key teaching of the Buddha is the \textit{anātman} theory, a natural outcome of \textit{pratītya-samutpāda}. According to this, there is no permanent agent called \textit{ātman}\index{atman@\textit{ātman}} because everything is relative. A relative entity cannot produce an unchanging, absolute entity like \textit{ātman.} Everything in the mundane world is therefore without \textit{ātman}.

Many of the Buddhist concepts are centered on these theories, particularly on \textit{pratītya-samutpāda}. These are considered as the direct teachings of the Buddha and this claim has never been disputed at any time in history.

\vspace{-.3cm}

\section*{Critique of Sheldon Pollock's Theses}

Western Indologist and professor at Colombia University, Sheldon Pollock is the most influential member of the current day Neo-Orientalist\index{Neo-Orientalist} school of Western Indological studies. He and his followers through their various theses distort Indian tradition, culture, \textit{dharma} and reformulate them into a new narrative that reflect their own worldview. This Western universalistic worldview discounts India’s oral tradition, makes \textit{kāvyā}-s\index{kavya@\textit{kāvyā}} devoid of religiosity, invents chronology\index{Chronology} for Indian literature so that it serves a pre-decided narrative, and pits the Buddha’s teachings against Hinduism and so on. Since Pollock\index{Pollock, Sheldon} is considered an authority by many on Indian tradition, his arguments about Indian tradition must be critically evaluated, and countered if found to be in contrast with what the insider tradition believes.

Pollock claims in his book, \textit{The Language of the Gods in the World of Men,} that by the rejection of the \textit{ātman},\index{atman@\textit{ātman}} Buddhism\index{Buddhism} altogether negated Upaniṣadic thought.

\begin{myquote}
“…positive transvaluations\index{transvaluation} in early Buddhism of core \textit{vaidika} values were complemented by a range of pure negations, beginning with \textit{an-atta (an-ātma)}, the denial of a personal essence, whereby the core conception of Upaniṣadic thought was cancelled.” 

~\hfill (Pollock 2006:52)(\textit{spellings as in the original})
\end{myquote}

This is a sweeping claim. By the above statement, Pollock wishes to establish that the Buddha\index{Buddha, the} wanted to cancel the Upaniṣadic teaching of \textit{ātman} concept, be it the notion of a \textit{paramātman}\index{paramatman@\textit{paramātman}} or individual \textit{ātman}. But the opinion of eminent Buddhist scholars differ quite radically from the narrative of Sheldon Pollock. W. T. Rhys Davids\index{Rhys Davids, W. T.} says –

\begin{myquote}
“Gautama was born and brought up and lived and died a Hindu.... There was not much in the metaphysics and principles of Gautama which cannot be found in one or other of the orthodox systems, and a great deal of his morality could be matched from earlier or later Hindu books. Such originality as Gautama possessed lay in the way in which he adopted, enlarged, ennobled and systematized that which had already been well said by others; in the way in which he carried out to their logical conclusion principles of equity and justice already acknowledged by some of the most prominent Hindu thinkers. The difference between him and other teachers lay chiefly in his deep earnestness and in his broad public spirit of philanthropy.” 

~\hfill (Davids 2000: 83-84)
\end{myquote}

As the opinion of major Buddhist scholars runs contrary to the claims of Pollock,\index{Pollock, Sheldon} we must critically evaluate the narrative built by Pollock through his research and examine the different \textit{ātman}\index{atman@\textit{ātman}} concepts that existed in the dharmic tradition in ancient times. Such an attempt is made here in this paper.

Continuing his false methods, Pollock\index{Pollock, Sheldon} argues that \textit{Vaidika} systems start to write down their ideas due to Buddhist influence and that the \textit{Rāmāyaṇa}\index{Ramayana@\textit{Rāmāyaṇa}} was written after the Buddha,\index{Buddha, the} with heavy borrowings from Jātaka tales. Pollock is also emphatic in his opinion that the Buddha totally despised the Veda-s.

\begin{myquote}
“Against the Mīmāṃsā\index{Mimamsa@Mīmāṃsā} tenet that the relationship between word and meaning is \textit{autpattika}, “originary” or natural—a primal, necessary, and non-arbitrary relationship (some-times absurdly reduced by its opponents to a mechanical, even magical view of reference)—Buddhists typically argued for a relationship based on pure convention (\textit{saṅketa}, also \textit{avadhi}). What was at stake for Mīmāṃsā in asserting the uncreated, eternal nature of language is the possibility that \textit{vāṅmaya}, or a thing-made-of-language—that is, a text, like the Veda—could be eternal too, something the Buddhists sought fundamentally to reject.” 

~\hfill (Pollock 2006: 52-53)
\end{myquote}

This is only partially true. the Buddha was opposed to the Veda-s to a certain extent. But this was due to the elements of ritual sacrifice present in the Veda-s rather than any disagreement with language convention. There are also opinions from some scholars and dialogues of the Buddha that suggest that the Buddha did support the ‘original, unaltered’ form of Veda-s and later, due to the way sacrificial hymns were interpreted by certain Brahmins. The Buddha was compelled to reject their authority and sanctity. This issue is also addressed in this paper.


\section*{The Upaniṣad-s: \hfill\break The Roots of Buddhist Philosophy}

In the following portions I venture to show that the three major teachings of the Buddha – \textit{pratītya-samutpāda},\index{pratityasamutpada@\textit{pratītya-samutpāda}} \textit{anātman}\index{anatman@\textit{anātman}} and \textit{nirvāṇa}\index{nirvana@\textit{nirvāṇa}} – have their roots in Upaniṣadic philosophy. This will greatly nullify Pollock’s\index{Pollock, Sheldon} theses which have an inherent tone of separation between the Buddha’s\index{Buddha, the} teaching and Vedic literature.

\vspace{-.3cm}

\section*{\textit{Pratītya-Samutpāda} of the Buddha \hfil\break and \textit{Madhu-Vidyā}\index{Madhu Vidya@Madhu-Vidyā} of Sage Dadhyañc\index{Dadhyanc@Dadhyañc, sage}}

\subsection*{\textit{Pratītya-samutpāda}}

The Buddha always tried to avoid giving answers either in the affirmative or the negative to certain questions\endnote{ These fourteen questions are – Is the world eternal? Or not? Or both? Or Neither?; Is the world finite? Or not? Or both? Or neither?; Does the Tathagata\index{Tathagata@Tathāgata} exist after death? Or not? Or both? Or neither?; Is the soul identical with the body? Or not?

\vspace{-.3cm}

\begin{flushright}
\textit{- Poṭṭhapāda Sutta}\index{Potthapada Sutta@\textit{Poṭṭhapāda Sutta}} 25-27
\end{flushright}
\vspace{-0.3cm}} in order to avoid the extremes of eternalism and annihilationism, and keep strictly to the Middle Way\index{Middle Way, the}\endnote{ “If I, Ananda, when the wandering monk Vachhagotta asked me: ‘Is there the ego?’ had answered: ‘The ego is’ then that, Ananda, would have confirmed the doctrine of the Sramanas\index{Sramana@śramaṇa} and Brahmins who believe in permanence (of the ego). If I, Ananda, had answered: ‘The ego is not’, then that, Ananda, would have confirmed the doctrine of the Sramanas and Brahmins, who believe in annihilation (of the ego).”

\vspace{-.3cm}

\begin{flushright}
\textit{Saṁyutta Nikāya}\index{Samyutta Nikaya@\textit{Saṁyutta Nikāya}} - 44, 10 (Oldenberg 1882: 272-273)
\end{flushright}
\vspace{-0.3cm}
}. As an example, for the question \textit{does the Tathāgata\index{Tathagata@Tathāgata} exist after death or not?}, the Buddha gave a thick silence as reply because he knew that if he gave ‘Yes’, it would be interpreted as promoting ‘eternalism’. On the other hand, if he gave ‘No’, he would be promoting the annihilation theory. So he remained silent\endnote{ Certain scholar opines that Buddha’s silence was due to his absolutist stand about Ultimate Reality. Thus, Chandradhar Sharma (2007: 17) says: “The ‘silence’ of Buddha on the fourteen metaphysical questions does not indicate his ignorance of metaphysics or his agnosticism or his nihilism. It indicates his absolutism by revealing that contradictions are inherent in thought and can be solved only by rising to immediate spiritual experience”.}.

The Buddha knew that the things that exist in the mundane world, neither exist nor non-exist ultimately. That being the case, what then was happening to them? The Buddha’s answer was that, they are always ‘becoming’. Things always arise depending on other things. This doctrine is known as \textit{pratītya-samutpāda} or Theory of Dependent Origination (as per Mahāyānists, the Theory of Relativity). The doctrine of \textit{pratītya-samutpāda} has profound meaning. In Buddhism,\index{Buddhism} \textit{pratītya-samutpāda} is also known as ‘Twelve Chain of Causation’\index{Law of Causation}\index{Twelve Chains of Causation}(Sogen\index{Sogen, Yamakami} 2009:85). It contains three periods – past, present and future.

In this theory of relativity, every entity in the world depends on other entities for its existence. Not only the objects in the mundane world, but also the mental states, are inter-dependent. Such entities that depend upon each other are said to be essence-less (Dasgupta\index{Dasgupta, Surendranath} 1933: 77). Continuously changing entities are therefore devoid of essence and thus ultimate existence. Ultimate existence is only for that, which exists by itself, without help from anything external.

\subsection*{\textit{Madhu-Vidyā}\index{Madhu Vidya@Madhu-Vidyā}}

\vspace{-.3cm}

Among Upaniṣad-s, the famous \textit{Madhu-vidyā} doctrine is in the oldest one, the \textit{Bṛhadāraṇyaka Upaniṣad}.\index{Brhadaranyaka Upanisad@\textit{Bṛhadāraṇyaka Upaniṣad}} It is taught by the sage Dadhyañc. \textit{Madhu-vidyā} doctrine’s teaching is multifarious. But the main theme is that, everything in this universe is interconnected and thus has no independent existence. Hence, they have no essence. Let us quote from the commentary of \textit{Bṛhadāraṇyaka Upaniṣad} by Śaṅkarācārya.\index{Sankara@Śaṅkarā}

\begin{myquote}
“Because there is mutual helpfulness among the parts of the universe including the earth, and because it is common experience that those things which are mutually helpful spring from the same cause, are of the same genus and dissolve into the same thing, therefore this universe consisting of the earth etc., on account of mutual helpfulness among its parts, must be like that. This is the meaning which is expressed in this section…” 

~\hfill (Swami Madhavananda 2011: 262)
\end{myquote}

Śaṅkarācārya here explicitly states that universal entities are in mutual helpfulness. Whatever exists by mutual helpfulness has no independent existence and so they are relative.

It is very much evident that, for a genius like the Buddha,\index{Buddha, the} the \textit{pratītya-samutpāda}\index{pratityasamutpada@\textit{pratītya-samutpāda}} doctrine can be easily developed from the \textit{Madhu-vidyā} doctrine of sage Dadhyañc\index{Dadhyanc@Dadhyañc, sage}. Also, the Law of Causation\index{Law of Causation} (\textit{kārya-kāraṇa-siddhānta}), mentioned elsewhere in the Upaniṣad-s is the foundation of \textit{pratītya-samutpāda}. In addition, depended origination is strongly based on the Law of Karma,\index{karma@\textit{karma}}\index{Law of Karma} which is very well a Vedic concept. The \textit{pugdala-dharma} \textit{śūnyatā} of Buddhists has its roots in the Upaniṣad-s.\break That being the case, (it is a fact that a prominent teaching of the Buddha that gave birth to the \textit{anātman}\index{anatman@\textit{anātman}} theory, has its roots in the Upaniṣad-s), how can Pollock\index{Pollock, Sheldon} claim that the Buddha nullified Upaniṣadic thought? In fact, the Buddha’s teaching was just a re-statement of Upaniṣadic thought from a new standpoint\endnote{ “To develop his theory, Buddha had only to rid the Upaniṣads of their inconsistent compromises with Vedic polytheism and religion, set aside the transcendental aspect as being indemonstrable to thought and unnecessary to morals, and emphasis the ethical universalism of the Upaniṣads. Early Buddhism,\index{Buddhism} we venture to hazard a conjecture, is only a restatement of the thought of the Upaniṣads from a new standpoint.”

\vspace{-.3cm}

\begin{flushright}
Radhakrishnan (2013: Vol.1 303)
\end{flushright}
\vspace{-.3cm}}.

\vspace{-.3cm}

\section*{\textit{Anātman} Theory of the Buddha \hfill\break and Unreal \textit{Jīvātman} of Upaniṣad-s}\index{Jivatman@\textit{jīvātman}}\index{atman@\textit{ātman}}

\subsection*{\textit{Anātman} theory of the Buddha:-}

\vspace{-.3cm}

This is one of the natural outcomes of the \textit{pratītya-samutpāda} theory. Since a permanent and unchanging \textit{ātman} cannot fulfill the ‘relative’ characteristic of \textit{pratītya-samutpāda}, this theory gave birth to the \textit{anātman} (no-soul) theory. This is an important doctrine of the Buddha. However, there is a lingering doubt that remains regarding this doctrine. Does the Buddha propound the \textit{anātman} theory only with reference to the relative, mundane (\textit{vyāvahārika}) world or was he applying this theory for both the mundane and trans-mundane (\textit{pāramārthika})\index{paramarthika@\textit{pāramārthika}} world?

\vspace{-.3cm}

\subsection*{\textit{Anātman} Concept in the Mundane Plane:-}\index{anatman@\textit{anātman}}

Let us take the first position. If the Buddha’s position was that there is no unchanging principle like \textit{ātman}\index{atman@\textit{ātman}} in the \textit{vyāvahārika} (relative plane of reality), then there is no doubt that it is in alignment with the Upaniṣad-s, since the central teaching of the Upaniṣad-s is non-duality between the \textit{jīvātman}\index{Jivatman@\textit{jīvātman}} and the \textit{paramātman}\index{paramatman@\textit{paramātman}} (\textit{Brahman})\index{Brahman@\textit{Brahman}}. In fact, the Upaniṣad-s also propound that there is no ultimate individual \textit{ātman} (\textit{jivātman}) in humans. Our supposition of an unchanging entity like \textit{ātman}, which is related to a single individual only, and our presumption that this is the true ultimate reality/supreme \textit{ātman}, is due to the \textit{avidyā}\index{avidya@\textit{avidyā}} or ignorance that resides in us. When we obtain \textit{brahma-vidyā}, \textit{avidyā} will be extinguished and we will realize that we are the ultimate reality, \textit{Brahman}. There is no \textit{jīvātman} in the ultimate sense.

The Buddha\index{Buddha, the} also suggested that there is no \textit{ātman} inside us permanently. To elaborate further, \textit{ātman} is a term that we give, for the combined operation of five \textit{skandha}-s and it can be annulled by knowing the four noble truths and practicing the eight fold path. Here, the Buddha clearly admits that people may feel something, like an \textit{ātman} inside them and they may experience this entity as unchanging. The Buddha did not reject the feeling inside one, which is akin to \textit{ātman}. Instead he asserted that people may feel something like an individual \textit{ātman} in them, but that thought is utterly wrong. This was Buddha’s position\endnote{ “The Tathāgata\index{Tathagata@Tathāgata} sometimes taught that the \textit{ātman}\index{atman@\textit{ātman}} exists and at other times he taught that the \textit{ātman} does not exist. When he preached that the \textit{ātman} exists and is to be the receiver of misery or happiness in the successive life as the reward of its own Karma,\index{karma@\textit{karma}} his object was to save men from falling into the heresy of Nihilism\index{Nihilism} (\textit{Uccheda-vāda}). When he taught that there is no \textit{ātman} in the sense of a creator or perceiver or an absolutely free agent, apart from the conventional name given to the aggregate of the five \textit{skandhas}, his object was to save men from falling into the opposite, heresy of Eternalism (\textit{Śāśvata-vāda}). Now which of these two views represents the truth? It is doubtless the doctrine of the denial of \textit{ātman}. This doctrine, which is so difficult to understand, was not intended by Buddha\index{Buddha, the} for the ears of those whose intellect is dull and in whom the root of goodness has not thriven. And why? Because such men by hearing the doctrine of \textit{anātman}\index{anatman@\textit{anātman}} would have been sure to fall into the heresy of Nihilism. The two doctrines were preached by Buddha for two very different objects. He taught the existence of \textit{ātman} when he wanted to impart to his hearers the conventional doctrine; he taught the doctrine of \textit{anātman} when he wanted to impart to them the transcendental doctrine.” (\textit{Prajñāpāramita Sūtra, Nāgārjuna\index{Nagarjuna@Nāgārjuna}).}

“The existence of the \textit{ātman} and of the \textit{Dharmas} (i.e., of the \textit{Ego} and of the phenomenal world) is affirmed in the Sacred Canon only provisionally and hypothetically, and never in the sense of their possessing a real and permanent nature.” (\textit{Dharmapāla\index{Dharmapala@Dharmapāla} in his commentary on the Vijñānamātra-śāstra)}.

\vspace{-.35cm}

\begin{flushright}
Sogen\index{Sogen, Yamakami} (2009: 19).
\end{flushright}
\vspace{-.3cm}
} and this is similar to the Upaniṣadic teaching.

The Upaniṣad-s say that the idea of an individual soul (\textit{jīvātman/ātman}) in human, is a product of \textit{avidyā}/ignorance\endnote{ Upaniṣad-s says \textit{sarvaṁ khalu idaṁ brahma.} Everything in this universe is Brahman\index{Brahman@\textit{Brahman}} including human beings. That means we are already \textit{Brahman}, the Ultimate Reality, by essence. Yet, ironically, Upaniṣad-s instructs us to ‘attain \textit{Brahma-Vidyā} and realize \textit{Brahman’}. People may feel this is contradictory. But in fact, this is not so because Upaniṣad-s only indicates that there is an unknown entity in us that prevents us from knowing that we are \textit{Brahman}. This unknown entity is called \textit{avidyā}.\index{avidya@\textit{avidyā}}}. By acquiring knowledge and practicing meditation, people can get rid of the ignorance and then subsequently find release from the clutch of the individual \textit{ātman} concept. He then realizes the \textit{paramātman}, or the \textit{Brahman}. Likewise the Buddha advocated to his followers that there is no real individual \textit{ātman} inside the body and if they feel so, they should know the Four Noble Truths and practice the noble eight fold path to get rid of that feeling. The path that leads to the Ultimate Truth is almost the same in both traditions. The Upaniṣad-s accord importance to austerity, knowledge, discrimination, reflection (reasoning) and meditation (\textit{nidhidhyāsana}). The Buddhist way to \textit{Nirvāṇa}\index{nirvana@\textit{nirvāṇa}}/Ultimate Truth includes these in a different package like understanding the four noble truths, practicing the noble eight fold path, meditation, self control, and so on. The similarity between the paths to the Ultimate Truth, in the Upaniṣad-s and the Buddha’s teaching is indeed clear.

\vspace{-.3cm}

\subsection*{\textit{Anātman} Concept in the Trans-mundane Plane}\index{anatman@\textit{Anātman}}

We shall now study the second stand. Was the Buddha\index{Buddha, the} advocating that there is no Ultimate Truth, like \textit{paramātman}\index{paramatman@\textit{paramātman}} or any such equivalent concept, beyond the realm of mundane world, by his \textit{anātman} concept? In fact, the rejection of the individual \textit{ātman}\index{atman@\textit{ātman}} does not warrant the rejection of the \textit{paramātman}, especially since the Buddha asserted many times that he had attained a highest level of existence, which is difficult to comprehend, which is beyond the realm of logic and which only the wise can attain\endnote{ “These, O brethren, are those other things, profound, difficult to realize, hard to understand, tranquillizing, sweet, not to be grasped by mere logic, subtle, comprehensible only by the wise, which the Tathāgata, having himself realized and seen face to face, hath set forth; and it is concerning these that they who would rightly praise the Tathāgata in accordance with the truth, should speak.”

\vspace{-.3cm}

\begin{flushright}
- \textit{Brahma Jāla Sutta}. Davids (1923: 30).
\end{flushright}}. There are practical difficulties to reject an Ultimate Reality because the relative, by default, indicates the existence of an Absolute. Without an Absolute, the relative cannot exist and sustain. While the Buddha admitted to the changing, relative character of the external world, he must have posited an Absolute realm too, without which the relative cannot sustain. Furthermore, if there is no Ultimate Reality, then a \textit{mokṣa}\index{moksa@\textit{mokṣa}} aspirant would always be in the loop of \textit{saṁsāra},\index{samsara@\textit{saṁsāra}} irrespective of how faithfully and earnestly he followed the four noble truths and the noble eight fold path. A \textit{mokṣa} aspirant can then never attain \textit{nirvāṇa}. Since an aspirant finds asylum from the relative mundane world in \textit{nirvāṇa}, \textit{nirvāṇa} itself has to be the Ultimate Reality.

\vspace{-.3cm}

\section*{Three \textit{Ātman} Concepts \hfill\break and the Buddha’s \textit{Anātman} Theory:-}

We must also consider the different concepts of \textit{ātman} that existed in India, while evaluating the Buddha’s objection towards the \textit{ātman}. This is a must because there are three \textit{ātman} concepts in India, and from the teachings of the Buddha, we can see that he rejected only two of them, leaving the third intact. The three \textit{ātman} concepts are given below.

\textbf{Individual \textit{Ātman}} – According to this concept, there is an \textit{ātman} inside each one of us and it is by itself eternal. There are many \textit{ātman-}s in the universe. Each is independent of the other. The \textit{ātman} controls the actions, and enjoys happiness and sorrow. After attaining \textit{mokṣa},\index{moksa@\textit{mokṣa}} the \textit{ātman}\index{atman@\textit{ātman}} will continue to remain independent, but in a supreme blissful state. This \textit{ātman} concept is followed by Nyāya\index{Nyaya@Nyāya} and Vaiśeṣika philosophies.

\textit{\textbf{Jīvātman}}\index{Jivatman@\textit{jīvātman}} – This is the reflection of \textit{paramātman}\index{paramatman@\textit{paramātman}} on the \textit{avidyā}\index{avidya@\textit{avidyā}} in an individual; i.e., \textit{paramātman} in the conditioned form. This \textit{ātman} vanishes when \textit{avidyā} is overcome and the person realizes \textit{Brahman}.\index{Brahman@\textit{Brahman}}

\textbf{\textit{Paramātman/Brahman}} – This is the highest level of Truth according to the Upaniṣad-s. This is the One without a second. All that exists in the universe is simply \textit{Brahman}. This is beyond the realm of logic and the senses. This is the non-dual Truth and can be directly realized through \textit{śravaṇa, manana}\index{sravana@\textit{śravaṇa}}\index{manana@\textit{manana}} and \textit{nididhyāsana}.\index{nididhyasana@\textit{nididhyāsana}}

Of these three \textit{ātman} concepts, the Buddha\index{Buddha, the} rejected only the first two\endnote{ “The Upaniṣadic seers and the Advaita\index{Advaita} Vedāntins use the word \textit{ātmā} in the sense of the pure transcendent subject, which is at once pure consciousness and bliss. Buddha and the Buddhists, on the other hand, use the word \textit{ātmā} in the sense of an empirical ego or in the sense of an eternal individual substance and reject its ultimate reality, while accepting its empirical validity…… In Buddha\index{Buddha, the} and Mahāyāna,\index{Mahayana@Mahāyāna} the denial of the self is its denial as an eternal substance; it is not the denial of the absolute Self. Anātmavada or nairātmyavāda is really the nirahaṅkāra-nirmama-vāda of Vedānta. It denies neither the empirical validity of the ego nor the ultimate reality of the Absolute Self. It is the denial only of the false notion which mistakes the empirical ego as an eternal spiritual substance and attempts to objectify the subject and realize it through thought-categories. To take the self as an eternal substance is to cling to it eternally and this is \textit{avidyā}\index{avidya@\textit{avidyā}} which is the root-cause of all attachment, desire, misery and bondage.”

\vspace{-.3cm}

\begin{flushright}
Sharma\index{Sharma, Chandradhar}(2007: 26, 27) (\textit{spellings/diacritics and italics as in the original})
\end{flushright}
\vspace{-0.28cm}}. He could not have rejected the unconditioned \textit{Brahman}, or such an equivalent Ultimate Reality because any such decision will then mean that people will be permanently entangled in \textit{saṁsāra}\index{samsara@\textit{saṁsāra}}\endnote{ “There is, O Bhikkhus, an unborn, unoriginated, uncreated, unformed. Were there not, O Bhikkhus, this unborn, unoriginated, uncreated, unformed, there would be no escape from the world of the born, originated, created, formed. Since, O Bhikkhus, there is an unborn, unoriginated, uncreated, unformed, therefore is there an escape from the born, originated, created, formed.”

\vspace{-.5cm}

\begin{flushright}
Strong (1902: 112)
\end{flushright}
\vspace{-.3cm}
}. There should be a way to overcome the hardships of \textit{saṁsāra}. Logically, it then follows that there must be a Highest Reality. The Buddha called it \textit{nirvāṇa},\index{nirvana@\textit{nirvāṇa}} and this state akin to the Upaniṣadic \textit{Brahman}.

\vspace{-.3cm}

\section*{\textit{Nirvāṇa} and Upaniṣadic \textit{Brahman}}

There were efforts to compare the ultimate realities propounded by the Upaniṣad-s and the Buddha’s teaching, at all times. In fact, if you put \textit{Brahman} and \textit{Nirvāṇa} side by side, there is no great discernible difference. They are very similar concepts. As stated in the Upaniṣad-s,\break \textit{Brahman} is all that is, and it is one without a second. These are descriptions about \textit{Brahman}, not definition. Nobody can say what \textit{Brahman} is since it is attribute-less or unqualified (from the Advaita\index{Advaita} point of view). The Buddha too did not define \textit{Nirvāṇa}. To all questions related to its definition, he remains silent.

The Buddha does not talk about anything comparable to an eternal truth apart from \textit{nirvāṇa}. This gives rise to an important confusion. Where is \textit{nirvāṇa}? Is it inside our body? Or is it outside? Or is it inside and outside? Or is it neither inside nor outside?

The fourth option can be rejected altogether because it will lead us to conclude that there is no \textit{nirvāṇa} at all. The problem with third option is that, if we attain \textit{nirvāṇa} from outside, then it will not be our essence and therefore there is a chance of losing it. The second option can also be rejected for the same reason. Thus, only the first option is a possibility to consider and this position is very important because it means that \textit{nirvāṇa} attainment is permanent. The concept of \textit{nirvāṇa}\index{nirvana@\textit{nirvāṇa}} is so depicted in the teachings of the Buddha. If it is posited that there is a possibility of losing \textit{nirvāṇa} after attaining it once, then the aspirant needs to strive for it again. However, such an idea is surely not present in the teachings of the Buddha. \textit{Nirvāṇa} attainment is permanent. And in the ultimate sense, what we can attain permanently is the one which is already inside us. That is, we must be in an enlightened state, by default. Enlightenment must not come from outside.

The Buddha\index{Buddha, the} in \textit{Mahāparinibbāṇa Sutta}\index{Mahaparinibbana Sutta@\textit{Mahāparinibbāṇa Sutta}} states the same. Chandradhar Sharma\index{Sharma, Chandradhar} in his important work \textit{‘The Advaita\index{Advaita} Tradition in Indian Philosophy’} opines as follows.

\begin{myquote}
“In a celebrated passage in the Mahāparinibbāna Sutta, the ailing Buddha says to Ananda: ‘O Ananda! I have taught the Dhamma (Dharma) without any reservation and have not kept anything secret like a tight-fisted teacher (\textit{āchārya-muṣṭi}). Now I am eighty years old and am somehow pulling on (the body) like an old tattered cart bound with ropes. I am going to leave the world soon. But there is no cause for grief as the light of Dharma is there. Ananda! my message for all of you is this: Let the Self be your light (\textit{attadīpa}, Skt. \textit{ātma-dīpa}), let the Self be your shelter (\textit{attasaraṇa}, Skt. \textit{ātma-sharaṇa}); let the Dharma (the real) be your light (\textit{dhamma-dīpa}, Skt. \textit{dharma-dīpa}), let the Dharma be your shelter (\textit{dhamma-saraṇa}, Skt. \textit{dharma-sharaṇa}); do not seek light and shelter outside.’” 

~\hfill (Sharma 2007: 30)(\textit{spellings/diacritics as in the original})
\end{myquote}

Even if, as some scholars do, the word \textit{atta (ātma)} in \textit{atta-dīpa} is interpreted as meaning just oneself without any reference to an ontological reality called ‘Self’ and the phrase ‘\textit{atta-dīpa}’ is taken to mean ‘you yourself are your light’, it has to be admitted that the Buddha is asking his disciples to seek light within and not outside. Now, if there is no true ‘Self’, then who is to seek the light and where? And if all objects as the Buddha says are perishable and miserable and the light is to be sought only in the subject, then the reality of the transcendent subject is clearly implied in this passage”

This is very similar to the \textit{Brahman} concept of the Upaniṣad-s\endnote{ It is not just \textit{nirvāṇa}\index{nirvana@\textit{nirvāṇa}} that is similar to Upaniṣadic Brahman.\index{Brahman@\textit{Brahman}} \textit{Vijñaptimātra}, the Highest Reality posited by Yogācāra\index{Yogacara@\textit{Yogācāra}} school of Mahāyāna, is also very similar to Brahman. Surendranath Dasgupta writes:
\begin{myquote}
{\fontsize{8}{10}\selectfont{
“As a ground of this ālayavijñāna we have the pure consciousness called the \textit{vijñaptimātra}, which is beyond all experiences, transcendent and pure consciousness, pure bliss, eternal, unchangeable and unthinkable. It is this one pure being as pure consciousness and pure bliss, eternal and unchangeable like the Brahman of the Vedānta, that forms the ultimate ground and ultimate essence of all appearance; even the ālayavijñāna\index{vijnana@\textit{vijñāna}} is an imposition of it, as are all the different states of it which make the world-order possible…… Thus we see that the ultimate reality is one, being self-identical, pure consciousness and pure bliss, which is thus different from the Tathatā of Aśvaghosha and very similar to the Brahman of the Upanishads.” }}

~\hfill Dasgupta\index{Dasgupta, Surendranath} (1933:119-120) (\textit{diacritics/spellings and italics as in the original})
\end{myquote}}. The Upaniṣad-s says, all are \textit{Brahman}. So we are divine by default. But we are not aware of our divine status. There must be something which prevents us from knowing our original divinity and which can be overcome through some specific methods. What prevents is \textit{avidyā}. We must overcome \textit{avidyā}\index{avidya@\textit{avidyā}} and realize \textit{Brahman}\index{Brahman@\textit{Brahman}} inside us. This Upaniṣadic concept of \textit{avidyā} and \textit{Brahman} is appearing in Buddhism\index{Buddhism} as \textit{avidyā} and \textit{nirvāṇa}.\index{nirvana@\textit{nirvāṇa}} In the Upaniṣad-s the locus of \textit{avidyā} is in the \textit{Brahman} (according to Advaita\index{Advaita} Vedānta)\endnote{ Post--Śaṅkara\index{Sankara@\textit{Śaṅkara}} Advaitin-s are divided on this topic. This topic is outside the purview of this paper.}. In that case one needs to understand where \textit{avidyā} resides in the Buddha’s teachings.

The Buddha\index{Buddha, the} states that the Twelve Chains of Causation\index{Twelve Chains of Causation}\endnote{ Twelve Chain of causation\index{Twelve Chains of Causation}\index{Law of Causation} – 1) Beginning less and cosmic Ignorance, \textit{Avidyā}.\break 2) Impressions of kārmic forces, \textit{Saṁskāra}.\index{samskara@\textit{saṁskāra}} 3) Individual consciousness, \textit{Vijñāna}. 4) Psycho-physical organism, \textit{Nāmarūpa}.\index{namarupa@\textit{Nāmarūpa}} 5) Six sense-organs including \textit{manas}, \textit{Ṣaḍāyatana}.\index{Sadayatana@\textit{Ṣaḍāyatana}} 6) Sense-object-contact, \textit{Sparśa}.\index{Sparsa@\textit{Sparśa}} 7) Sensation, \textit{Vedanā}.\index{Vedana@\textit{Vedanā}} 8) Desire for sense-enjoyment. \textit{Tṛṣṇā}. 9) Clinging to sense enjoyment, \textit{Upādana}.\index{Upadana@\textit{Upādana}} 10) Will to be born for experiencing sense-enjoyment, \textit{Bhava}.\index{bhava@\textit{bhava}} 11) Birth including rebirth, \textit{Jāti}.\index{jati@\textit{jāti}} 12) Disintegration and death, \textit{Janana-maraṇa}

\vspace{-.5cm}

\begin{flushright}
- \textit{Mahānidāna Sutta}.
\end{flushright}
\vspace{-.3cm}} starts from \textit{avidyā}\endnote{ “Although there must have been existed a complicated process in formulating the Twelve Link formula, it is undeniable that it is analogous in its way of formulation to the formulas set forth by other philosophical systems of India, such as Sāṁkhya-Yoga\index{Yoga}.”

\vspace{-.3cm}

\begin{flushright}
Nakamura (1987: 69)
\end{flushright}}. But \textit{avidyā}, being unreal (because we can destroy/avoid it by following noble eight-fold path), cannot exist by itself. \textit{Avidyā} must be rooted in a Reality which can exist by itself. The Buddha says that all things in the mundane world are relative, and hence unreal, and are the cause of pain. So \textit{avidyā} cannot be rooted in the mundane world. This then means, there is no Ultimate Reality, other than \textit{Nirvāṇa}, about which the Buddha has preached. From this, the natural conclusion is that \textit{Nirvāṇa} must stand together with \textit{avidyā} at the beginning of the twelve chains of causation\index{Law of Causation} which is akin to the \textit{Brahman-avidyā} concept of the Upaniṣad-s.

The characteristics of \textit{nirvāṇa} are also similar to that of \textit{Brahman}/\textit{para\-mātman}.\index{paramatman@\textit{paramātman}}\index{atman@\textit{ātman}} Chandradhar Sharma\index{Sharma, Chandradhar} continues.

\begin{myquote}
“\textit{Nirvāṇa}, like the Upaniṣadic \textit{ātmā}, is repeatedly described by the Buddha as calm (\textit{shānta}), immortal (\textit{amṛta}), unproduced (\textit{akṛta}), uncaused (\textit{asamskṛta}), unborn (\textit{ajāta}), undecaying (\textit{ajara}), undying (\textit{amara}), eternal (\textit{nitya}), abiding (\textit{dhruva}), unchanging (\textit{shāshvata}), highest joy (\textit{parama sukha}), blissful (\textit{Shiva}), desireless (\textit{tṛṣṇā-kṣaya}), cessation of plurality (\textit{bhava nirodha; prapanchopashama}) and the fearless goal (\textit{abhaya pada})\endnote{ \textit{Udana}, 73; \textit{Suttanipata, RatanSutta; Itivuttaka}, 112; \textit{Dhammapada},\index{Dhammapada} 18, etc.}. All the epithets (or their synonyms) which the Upaniṣadic seers use for the \textit{Ātmā}, Buddha uses for \textit{Nirvāṇa}. \textit{Ātmā} and \textit{Nirvāṇa} stand for the Inexpressible and the Ineffable Absolute which is transcendent to thought and is realized through immediate spiritual experience (\textit{bodhi} or \textit{prajñā})”. 

~\hfill (Sharma 2007: 29) (\textit{spelling/diacritics as in the original})
\end{myquote}

S. Radhakrishnan, in his \textit{magnum opus}, \textit{Indian Philosophy} quotes a verse of the Buddha, from the \textit{Udāna} which is similar to the characteristics and idea of \textit{Brahman}.

\begin{myquote}
“There is an unborn, an unoriginated, an unmade, an uncompounded; were there not, Oh mendicants, there would be no escape from the world of the born, the originated, the made and the compounded.” 

~\hfill (Radhakrishnan 2013: 319)
\end{myquote}

We can thus conclude that the concept of \textit{Nirvāṇa}\index{nirvana@\textit{nirvāṇa}} is very much similar to the Upaniṣadic \textit{Brahman}.

\vspace{-.3cm}

\section*{\textit{Vyāvahārika} and \textit{Pāramārthika}\index{paramarthika@\textit{pāramārthika}} \hfill\break in \textit{Vaidika} and Buddhist Systems}

In a number of hymns, the Buddha has asserted that he had attained a state that is difficult to attain by others. Quoting from \textit{Brahmajāla Sutta},\index{Brahmajala Sutta@\textit{Brahmajāla Sutta}} Davids (1923: 30)

\begin{myquote}
“These, O brethren, are those other things, profound, difficult to realize, hard to understand, tranquillizing, sweet, not to be grasped by mere logic, subtle, comprehensible only by the wise, which the Tathāgata,\index{Tathagata@Tathāgata} having himself realized and seen face to face, hath set forth; and it is concerning these that they who would rightly praise the Tathāgata in accordance with the truth, should speak.”
\end{myquote}

In this statement, the Buddha\index{Buddha, the} gives many epithets to the highest state that he realized – hard to understand, not to be grasped by mere logic, comprehensible only to the wise, and so on. All of these can also be applied to Upaniṣadic \textit{Brahman}\index{Brahman@\textit{Brahman}} without a single exception. When the Buddha says that this state is \textit{not to be grasped by mere logic,} it clearly indicates that the highest state is transcendental, not empirical.

The quote from \textit{Brahmajāla Sutta} also shows that, like \textit{vyāvahārika} and \textit{pāramārthika} of Vedic literature, there are two levels/planes of existence in the Buddha’s teachings too. Since the Buddha has realized a state which is difficult for others to grasp, he must be on a plane higher than the common people. Others can attain this highest level only by destroying \textit{avidyā}.\index{avidya@\textit{avidyā}} Thus two levels of consciousness are present in this conception, just like the \textit{vyāvahārika} and \textit{pāramārthika} of Vedic literature.

Another principle which points to the two levels of existence is ‘\textit{nāma-rūpa}’.\index{namarupa@\textit{nāmarūpa}} The Tathāgata has given discourses about \textit{nāma-rūpa} (name and form) many times. In his discourses, he states that the objects that we see in the mundane world exist just as \textit{nāma-rūpa}. Everything in the mundane world is in constant flux. They are in \textit{coming into} and \textit{passing by} state always, and hence have no ultimate existence. They are devoid of essence and consequently the mundane world is expressible only in \textit{nāma-rūpa}. This constant state of flux – being the state of the mundane world – means that it has to be rooted somewhere. This somewhere is then most certainly, the unchanging, transcendental plane. Here again, we encounter the concept of two planes of existence which mirror the Upaniṣadic worldview.


\section*{The Buddha's Opinion on \hfill\break the \textit{Apauruṣeyatva} of Veda-s}

It is often believed that the Buddha\index{Buddha, the} rejected the authority of Veda-s and that led him to set up a new religion. In fact, this is a baseless argument.

Basing his opinion on early Buddhist texts, mainly \textit{Aṅguttara Nikāya}\index{Anguttara Nikaya@\textit{Aṅguttara Nikāya}} and \textit{Brāhmaṇa-dhammika-sutta},\index{Brahmanadhammikasutta@\textit{Brāhmaṇa-dhammika-sutta}} Robert Spence Hardy\index{Hardy, Robert Spence} (Hardy 1866: 43-44) comments that, the Buddha himself admitted that \textbf{the Veda-s in their ‘original’ form are \textit{apauruṣeya},\index{apauruseya@\textit{apauruṣeya}} but that later certain Brahmins corrupted it by adding sacrificial hymns, due to which the Buddha ceased revering the Veda-s}. Quoting from the book –

\begin{myquote}
“the Buddha denied that the Brahmans were then in the possession of the real Veda. He said that it was given in the time of Kāsyapa (a former supreme the Buddha) to certain rishis, who, by the practice of severe austerities, had acquired the power of seeing Divine Bliss. They were Attako, Wāmako, Wāmadewo, Wessāmitto, Yamataggi, Angiraso, Bhāraddwājo, Wāsetto, Kassapo and Bhagu. The Vedas that were revealed to these rishis were subsequently altered by Brahmans, so that they are now made to defend the sacrifice of animals, and to oppose the doctrine of Buddha. It is on account of this departure from the truth, that the Buddha refused to pay them any respect.” 

~\hfill (\textit{spellings/diacritics as in the original})
\end{myquote}

If the Veda-s were corrupted by \textit{Brahman}s, then there must be a version of the Veda-s which are not corrupted. According to the Buddha this version is revealed to the previous Buddhas, and hence they must be \textit{apauruṣeya}.

\textit{Brāhmaṇa-dhammika-sutta} is very important in this respect. This \textit{sutta} discusses the state and status of Brahmins in ancient times, especially before the Veda-s get corrupted. Quoting from \textit{Brāhmaṇa-dhammika-sutta, Sutta Nipāta}\index{Brahmanadhammikasutta@\textit{Brāhmaṇa-dhammika-sutta}} (Mills 2015: 71-75).

\begin{myquote}
“Thus have I heard: At one time the Radiant One dwelt at Sāvatthī, in the Jeta Grove, Anāthapiṇḍika’s park. Then many decrepit old Kosalan brahmins, aged, elderly, advanced in years, attained to old age, those indeed of palatial abodes, went to the Radiant One and exchanged greeting with him. When this courteous and amiable talk was finished they sat down to one side. Sitting there these Brahmins of palatial abodes said,

“Master Gautama, are there now to be seen any Brahmins who practice the Brahmin Dharma of the Brahmins of old?”

“No, Brahmins, there are no Brahmins now to be seen who practice the Brahmin Dharma of the Brahmins of old.”

“It would be excellent if the good Gautama\index{Buddha, the} would speak to us upon the Dharma of the Brahmins of old if it would not be too much trouble.”

“Then Brahmins listen well and bear in mind what I shall say”.

“Indeed, venerable” said those Brahmins of palatial abodes to the Radiant One. He spoke as follows:

In ancient times the sages then\\ austerely lived, were self-restrained,\\ let go five bases of desire\\ to fare for their own benefit.

\vspace{0.05cm}

Brahmins then no cattle had,\\ no gold, no grain they hoarded up,\\ their grain, their wealth was Vedic lore—\\ this the treasure they guarded well.\\.........................


\vspace{0.05cm}


Unbeaten were Brahmins and inviolate—\\ guarded by Dharma-goodness then,\\ none hindered or obstructed them\\ when they arrived at household doors.

\vspace{0.05cm}

Until the age of eight-and-forty\\ they practiced celibate student life—\\ the brahmins of those ancient times\\ fared seeking knowledge and conduct good.

\newpage

Brahmins then did not indulge\\ in sexual intercourse out of time,\\ during menstruation,\\ but only when wives were free from this.\\ …………………

\vspace{0.05cm}

Having begged rice, butter and oil,\\ with cloths and bedding too,\\ they sought and stored these righteously,\\ and from them made a sacrifice:\\ during that sacrificial rite\\ cattle they never killed.\\ ………………

\vspace{0.05cm}

Givers of good and strength, of good\\ complexion and the happiness of health,\\ having seen the truth of this\\ cattle they never killed.

\vspace{0.05cm}

Those Brahmins then by Dharma did\\ what should be done, not what should not,\\ and so aware they graceful were,\\ well-built, fair-skinned, of high renown.\\ While in the world this lore was found\\ these people happily prospered.

But then in them corruption came\\ for little by little they observed\\ how rajahs had to splendors won\\ with women adorned and elegant,\\ …………………

\vspace{0.05cm}

 filled with crowds of women fair\\ and ringed by herds of increasing cows—\\ all this the eminent wealth of men\\ the Brahmins coveted in their hearts.

Then they composed some Vedic hymns\\ and went chanting to Okkāka king:\\ “Great your wealth and great your grain,\\ make sacrifice to us with grain and wealth”.

That rajah, Lord of chariots,\\ by Brahmins was persuaded so\\ he offered all these sacrifices:\\ of horses, men, the peg well-thrown,\\ the sacrifice of soma drink\\ the one of rich results—\\ while to the Brahmins wealth he gave:\\ ……………………

 When they had all this wealth received\\ to hoard it up was their desire\\ for they were overwhelmed by greed—\\ their craving thus increased—\\ so they composed more Vedic hymns\\ and chanting went to Okkāka king.

\vspace{0.1cm}

“As water is, and earth, as well\\ as gold, as grain as well as wealth,\\ in the same way for human beings,\\ and cattle are necessities;\\ Great your wealth and great your grain,\\ make sacrifice to us with grain and wealth”.

\vspace{0.1cm}

That rajah, lord of chariots,\\ by Brahmins was persuaded—so\\ in sacrifice, he caused to kill\\ cattle in hundreds, thousands too.

\vspace{0.1cm}

But neither with hooves nor horns\\ do cows cause harm to anyone,\\ gentle they are as sheep\\ yielding us pails of milk;\\ in spite of this the rajah seized\\ their horns, slew them by the sword.\\ ……………………

\vspace{0.1cm}

This adharmic wielding of weapons,\\ descended from times of old:\\ in this are the innocents slain,\\ while ritual priests from Dharma fell.”
\end{myquote}

The opinion of the Buddha\index{Buddha, the} about the ancient life of brahmins is very clear here. The Buddha’s opinion on the infallibility of the Veda-s can also be derived from this \textit{Sutta}. The \textit{Sutta} says that brahmins were \textbf{austerely lived, self-restrained, unbeaten, inviolable, guarded by Dharma and so on}. Because of these qualities, they were respected by people. To summarize the author: What is really implied here is that, in ancient times Brahmins possessed the non-corrupted version of the Veda and animal sacrifices were not prevalent then. Later they came under the influence of the lavish life style of the kings and began to long for wealth. They altered Vedic literature and approached the kings and persuaded them to perform sacrifices so that they acquire wealth themselves. They inserted hymns that support and validate animal sacrifice. Hundreds of animals were killed thus in the sacrifices. Gods and demons objected to this. But Brahmins did not accede and thus began to be disrespected by people and the Buddha.

These words emphasise that, the Buddha\index{Buddha, the} did revere the original\break Veda-s, where devoid of animal sacrifice. He also valued the ancient Vedic sages as indicated in the \textit{Brāhmaṇa-dhammika-sutta}.\index{Brahmanadhammikasutta@\textit{Brāhmaṇa-dhammika-sutta}} It is only after “the insertion of sacrificial hymns” that the Buddha objected to the Veda-s, hesitating to pay them the respect they formerly commanded. The Buddha was very much aligned to the knowledge based portions of the Vedic compendium, the Upaniṣad-s\endnote{ “The only metaphysics that can justify Buddha’s ethical discipline is the metaphysics underlying the Upaniṣads. Buddhism\index{Buddhism} is only a later phase of the general movement of thought of which the Upaniṣads were the earlier.”

\vspace{-.3cm}

\begin{flushright}
- Radhakrishnan (2013: 470)
\end{flushright}

\vspace{-.3cm}

“The Śākyan mission was out ‘not to destroy, but to fulfill’, to enlarge and enhance the accepted faith-in-God of their day, not by asseverating, but by making it more vital.”

\vspace{-.3cm}

\begin{flushright}
Rhys-Davids\index{Rhys Davids, W. T.}(1932: 194)
\end{flushright}
\vspace{-.3cm}}. Almost all the teachings of the Buddha can be traced back to the Upaniṣad-s\endnote{ “The Upaniṣadic seers and Buddha\index{Buddha, the} both are opposed to the view of realistic pluralism that the self is an ultimate individual substance and that there is a plurality of such eternal selves. Buddha carries on the tradition of absolutism so clearly set forth in the Upaniṣads. For both, the Real is the Absolute which is at once transcendent to thought and, immanent in phenomena. Both take \textit{avidyā},\index{avidya@\textit{avidyā}} the beginningless and cosmic Ignorance as the root-cause of phenomenal existence and suffering. Both believe that thought is inherently fraught with contradictions and thought-categories, \textit{instead of} revealing the Real distort it, and therefore, one should rise above all views, all theories, all determinations, all thought-constructions in order to realize the Real. For both, the Real is realized in immediate spiritual experience. Both prescribe moral conduct and spiritual discipline as means to realize the Real, the fearless goal, the abode of Bliss…. Both believe in the established canon of logic that it is the unreal alone which can be negated. For both, that which is negated in \textit{avidyā}, the imposed empirical character of the ‘I’, and that which is retained is the Absolute. Both use the negative dialectic, the ‘\textit{neti neti}’ (not this, not this) for indirectly pointing to the nature of the Inexpressible. All the epithets which the Upanishadic seers use for \textit{Ātmā} or Brahma (or their synonyms) Buddha uses for \textit{Nirvāṇa}.\index{nirvana@\textit{nirvāṇa}} \textit{Ātmā} and \textit{Nirvāṇa} both stand for the ineffable non-dual Absolute. It must, however, be admitted that while the Upaniṣadic seers openly identify the Absolute with the Pure Self which is at once pure consciousness and bliss, Buddha, true to his negative logic, does not expressly identify the Absolute with the Pure Self, though the implication is clearly there. He identifies the Absolute with \textit{Nirvāṇa}. Buddha’s omission to identify the Absolute with the Transcendent Self has led to the misunderstanding of his \textit{anātmavāda}. But though Buddha does not expressly identify the Absolute with the Pure Self, nowhere has he expressly denied it. His descriptions of \textit{Nirvāṇa} are similar to the descriptions of the Upaniṣadic \textit{Ātmā} and leave no doubt that he is carrying on the tradition of the Upaniṣadic absolutism.”

\vspace{-.3cm}

\begin{flushright}
- Sharma\index{Sharma, Chandradhar} (2007: 31-32) (\textit{diacritics/spellings and italics as in the original}).
\end{flushright}}.\break He never rejected the Upaniṣadic \textit{Brahman}\index{Brahman@\textit{Brahman}} in any of his \textit{suttas}\endnote{ “At first sight nothing can appear more definite than the opposition of the Buddhist an-attā, 'no-Ātman,'\index{atman@\textit{ātman}} and the Brāhman\index{Brahman@\textit{Brahman}} ātman, the sole reality. But in using the same term, Attā or Ātman, Buddhist and Brāhman are talking of different things, and when this is realized, it will be seen that the Buddhist disputations on this point lose nearly all their value. It is frankly admitted by Professor Rhys Davids\index{Rhys Davids, W. T.} that,

‘The neuter Brahman\index{Brahman@\textit{Brahman}} is, so far as I am aware, entirely unknown in the Nikāyas, and of course the Buddha's\index{Buddha, the} idea of Brahmā, in the masculine, really differs widely from that of the Upanishads.’

There is nothing, then, to show that the Buddhists ever really understood the pure doctrine of the Ātman,\index{atman@\textit{ātman}} which is ‘not so, not so’. The attack which they led upon the idea of soul or self is directed against the conception of the eternity in time of an unchanging individuality; of the timeless spirit they do not speak, and yet they claim to have disposed of the theory of the Ātman! In reality both sides were in agreement that the soul or ego (manas, ahamkāra, vijñāna,\index{vijnana@\textit{vijñāna}} etc.) is complex and phenomenal, while of that which is ‘not so’ we know nothing.”

\vspace{-.3cm}

\begin{flushright}
- Coomaraswamy\index{Coomaraswamy, Ananda K.} (1916: 199) (\textit{diacritics/spellings and italics as in the original})
\end{flushright}
\vspace{-.3cm}}. \textit{Karma siddhānta}\index{Law of Karma}\index{karma@\textit{karma}}, \textit{saṁnyāsa},\index{samnyasa@\textit{saṁnyāsa}} morality\endnote{ “‘Hence let a man take care to himself. A man who steals gold, who drinks spirits, who dishonors his Guru’s bed, who kills a Brāhman, these four falls, and as a fifth he who associates with them. But he who thus knows the five fires [Pañcāgni] is not defiled by sin even though he associates with them. He who knows this is pure, clean, and obtains the world of the blessed.’ Herein one can trace the origin of Pārsvanātha’s\index{Parsvanatha@Pārśvanātha} doctrine of four-fold restraint (cāujjāma saṁvara), Mahāvira’s\index{Mahavira@Mahāvira} five great vows (pañea mahāvvayas) and of Buddha’s five moral percepts (pañca-śīlas).”

\vspace{-.3cm}

\begin{flushright}
- Barua (1921: 96) (\textit{diacritics/spellings and italics as in the original})
\end{flushright}} and so on are all pre-Buddhistic in origin.

\vspace{-.3cm}

\section*{Implications}

Pollock’s\index{Pollock, Sheldon} effort to pit Buddhism\index{Buddhism} against Hinduism is just the beginning of a grand narrative. In the coming years more such ‘inventions’ will come forth from the neo-Orientalist\index{Neo-Orientalist} school. Their Indian counterparts, with no access to religious studies as a discipline in the largely anglicized mainstream education, built on Western theories will support such claims whole heartedly. If not countered with facts, these Western narratives will exacerbate the divide between two systems of thought that sprung from the same dharmic source.

Rajiv Malhotra\index{Malhotra, Rajiv} in his book \textit{The Battle for Sanskrit} has highlighted several red flags with regard to Pollock’s theses regarding the differences between Hinduism and Buddhism. According to Malhotra (2016: 382), “He [Pollock] obsessively looks for things he can interpret as ‘norms’ in both systems and then tries to put them in mutual contradiction as much as possible”. This is also borne out by my own extracts from Pollock’s work which I have quoted above in reference to the philosophical meeting points that I want to highlight. Malhotra then goes on to explain the mischief propagated by Pollock and his group of neo-Orientalists with respect to chronology,\index{Chronology} the adoption of Pali as the language of propagation, and various other distortions which would never be accepted by traditionalists. It is clear that by working furiously to deconstruct and exaggerate differences between dharmic streams of thought, the neo-Orientalists are trying to fragment the inherent unity of thought among the various dharmic offshoots at the foundational level. Once this is accomplished, their aim could be to pit the various offshoots irrevocably as hostile to each other and thus create the basis for more fragmentation and exclusivity claims. This could then serve geo-political interests or help predator faiths to take advantage of the falsely exaggerated faultlines. At the least, it serves to impose universalistic modes of interpreting events that are not native to the traditions. This universalism has not helped to unify the world in any way as we can see that conflicts are only increasing in every part of the world. This kind of deconstruction denies Indians the opportunity to develop alternate ways of interpreting their own traditions using their own categories which could potentially benefit the world.

The Buddha\index{Buddha, the} valued many Vedic doctrines. As seen before, most of the doctrines of the Buddha were derived from Vedic literature, the difference being that the Buddha used a different terminology for his doctrine.

\vspace{-.4cm}

\section*{Conclusion}

Pollock\index{Pollock, Sheldon} tries very hard to establish that there are foundational differences between Buddhism\index{Buddhism} and Vedic tradition. His fundamentally flawed conclusions receive endorsement and encouragement from a wide range of scholars including deracinated Indians. Through this paper, I have shown how heavily Buddhism depends on the Vedic literature, especially the Upaniṣad-s. \textit{Pratītya-samutpāda}\index{pratityasamutpada@\textit{pratītya-samutpāda}} and \textit{anātman}\index{anatman@\textit{anātman}} theories of the Buddha are not in conflict with the Vedic tradition. On the contrary, these theories show an uncanny similarity with the Vedic tradition. This being the case, I believe it is time to re-evaluate this notion of irreconcilable difference between Hinduism and Buddhism using our own sources for reference, so that we can once more unite the dharmic systems under one umbrella which will make them uniquely positioned to take on the challenges that the future holds for all of humanity.


\section*{Bibliography}

\begin{thebibliography}{99}
\itemsep=1pt
\bibitem{chap4-key01} Barua, Benimadhab. (1921).\textit{ A History of Pre-Buddhistic Indian Philosophy}. Calcutta: University of Calcutta.

 \bibitem{chap4-key02} Belvalkar, S.K and Ranade, R.D. (1927). \textit{History of Indian Philosophy: The Creative Period.} Poona: Bilvakunja Publishing House.

 \bibitem{chap4-key03} Bhattacharya, Haridas. (Ed.) (2006). \textit{The Cultural Heritage of India Volume 3: The Philosophies.} Kolkata: The Ramakrishna Mission Institute of Culture.
 
 \bibitem{chap4-key03a} \textbf{\textit{Brahmasūtra Bhāṣya.}} See Gambhirananda (2011).
 
 \bibitem{chap4-key03b} \textbf{\textit{Bṛhadāraṇyaka Upaniṣad.}} See Madhavananda (2011).
 
 \bibitem{chap4-key03c} \textbf{\textit{Chāndogya Upaniṣad.}} See Gambhirananda (2019).

 \bibitem{chap4-key04} Chatterjee, Satischandra. (2015). \textit{The Nyaya Theory of Knowledge: A Critical Study of Some Problems of Logic and Metaphysics.} New Delhi: Rupa Publications (P) Ltd.

 \bibitem{chap4-key05} Coomaraswamy, Ananda. (1916). \textit{Buddha and the Gospel of Buddhism.} New York: G.P Putnam’s Sons.

 \bibitem{chap4-key06} Dasgupta, Surendranath. (1933).\textit{ Indian Idealism.} London: Cambridge University Press.

 \bibitem{chap4-key07} Davids, W. T. (1923). \textit{Dialogues of the Buddha Vol 2.} London: Oxford University Press.

 \bibitem{chap4-key08} —. (2000). \textit{Buddhism: Being a Sketch of the Life and teachings of Gautama, the Buddha.} New Delhi: Asian Educational Services.

 \bibitem{chap4-key09} Gambhirananda, Swami. (2009). Translation. \textit{Chandogya Upanishad with the commentary of Sankaracharya.} Kolkata: Advaita Ashrama.

 \bibitem{chap4-key10} —. (2011). Translation. \textit{Brahmasutra Bhashya of Sankaracharya}. Kolkata: Advaita Ashrama.

 \bibitem{chap4-key11} —. (2012). Translation. \textit{Eight Upanishads with the commentary of Sankaracharya, 2 Volumes}. Kolkata: Advaita Ashrama.

 \bibitem{chap4-key12} Griffith, Ralph. (1896). \textit{The Hymns of the Rigveda.} Nilgiri 2nd Edition. \url{<https://www.sanskritweb.net/rigveda/}\textgreater .

 \bibitem{chap4-key13} Hardy, Robert Spence. (1866). \textit{The Legends and Theories of the Buddhists compared with History and Science}. London: Williams and Norgate.

 \bibitem{chap4-key14} Joshi, Lal Mani. (2002). \textit{Studies in the Buddhistic Culture of India}. Delhi: Motilal Banarsidass Publishers Pvt. Ltd.

 \bibitem{chap4-key15} Kalupahana, David J. (2012). \textit{Mūlamadhyamakakārikā of Nāgārjuna: The Philosophy of the Middle Way.} Delhi: Motilal Banarsidass Publishers Pvt. Ltd.

 \bibitem{chap4-key16} Madhavananda, Swami. (Tr.) (2011).  \textit{The Bṛhadāraṇyaka Upaniṣad with the commentary of Śaṅkarācārya.} Kolkata: Advaita Ashrama.

 \bibitem{chap4-key17} Malhotra, Rajiv. (2016). \textit{The Battle for Sanskrit.} New Delhi: Harper Collins.

 \bibitem{chap4-key18} Mills, Laurence Khantipalo. (2015). \textit{Sutta Nipāta}. SuttaCentral. ISBN: 978-1-329-36020-4.

 \bibitem{chap4-key19} Nakamura, Hajime. (1987).\textit{ Indian Buddhism: A Survey with Bibliographical Notes.} Delhi: Motilal Banarsidass Publishers Pvt. Ltd.

 \bibitem{chap4-key20} Oldenberg, Hermann. (1882). \textit{Buddha: His Life, His Doctrine, His Order.} (Translated by William Hoey). London: Williams and Norgate.

 \bibitem{chap4-key21} Pollock, Sheldon. (2006).\textit{ The Language of the Gods in the World of Men: Sanskrit, Culture and Power in Premodern India.} University of California Press.

 \bibitem{chap4-key22} Radhakrishnan, Sarveppalli. (2013). \textit{Indian Philosophy (2 volumes}). London: Oxford University Press.

 \bibitem{chap4-key23} Ramanan. K Venkata. (2011). \textit{Nagarjuna’s Philosophy}. Delhi: Motilal Banarsidass Publishers Pvt. Ltd.

 \bibitem{chap4-key24} Ranade, R. D. (1926). \textit{A Constructive Survey of Upanishadic Philosophy: Being a Systematic Introduction to Indian Metaphysics.} Poona: Oriental Book Agency.

 \bibitem{chap4-key25} Rhys-Davis, Mrs. (1932). \textit{A Manual of Buddhism.} London: The Sheldon Press.

 \bibitem{chap4-key26} Satprakashananda, Swami. (2009). \textit{Methods of Knowledge: According to Advaita Vedanta.} Kolkata: Advaita Ashrama.

 \bibitem{chap4-key27} Sharma, Chandradhar. (2007). \textit{The Advaita Tradition in Indian Philosophy: A Study of Advaita in Buddhism, Vedanta and Kashmira Saivism}. Delhi: Motilal Banarsidass Publishers Pvt. Ltd.

 \bibitem{chap4-key28} —. (2013). \textit{A Critical Survey of Indian Philosophy}. Delhi: Motilal Banarsidass Publishers Pvt. Ltd.

 \bibitem{chap4-key29} Sinha, Jadunath. (1999). \textit{Indian Realism}. Delhi: Motilal Banarsidass Publishers Pvt. Ltd.

 \bibitem{chap4-key30} —. (2013). \textit{Outlines of Indian Philosophy}. London: New Central Book Agency (P) ltd.

 \bibitem{chap4-key31} Sogen, Yamakami. (2009). \textit{Systems of Buddhistic Thought}. Delhi: Eastern Book Linkers.

 \bibitem{chap4-key32} Stcherbatsky, F. T. H. (2008). \textit{Buddhist Logic (2 Volumes)}. Delhi: Low Price Publications.

 \bibitem{chap4-key33} Strong, D. M. (1902). \textit{The Udana or The Solemn Utterances of The Buddha.} London: Luzac \& Co.

 \bibitem{chap4-key34} Takakusu, Janjiro. (2001). \textit{The Essentials of Buddhist Philosophy.} New Delhi: Munshiram Manoharlal Publishers Pvt. Ltd.
 
 \end{thebibliography}

\theendnotes

