
\chapter*{Volume Editorial}\label{volumeeditorial}

\begin{myquote}
\textit{The military superiority of Europe to Asia is not an eternal law of nature, as we are tempted to think, and our superiority in civilisation is a mere delusion.}
\end{myquote}
~
\vspace{-0.9cm}
\begin{flushright}
- \textbf{Bertrand Russell}\index{Russell, Bertrand}
\end{flushright}

\begin{myquote}
\textit{Cynics are only happy in making the world as barren for others as they have made for themselves.}
\end{myquote}
~
\vspace{-0.9cm}
\begin{flushright}
- \textbf{George Meredith}\index{Meredith, George}
\end{flushright}

\begin{myquote}
\textit{The learned tradition is not concerned with truth, but with the learned adjustment of learned statements of antecedent learned people.}
\end{myquote}
~
\vspace{-0.9cm}
\begin{flushright}
- \textbf{A. N. Whitehead}\index{Whitehead, A. N.}
\end{flushright}

It is a pleasure to write a few words by way of the Volume Editorial for this 7th Volume of the SI Series. The volume has ten papers contributed mostly by youngsters. The first two papers deal with issues of chronology\index{Chronology} in Prof. Sheldon Pollock’s\index{Pollock, Sheldon} writings, bringing out the lapses and deficiencies in his approach and analysis. The next four deal with Hinduism and Buddhism\index{Buddhism} in their various aspects, handling his comments on their relation. The final four papers deal with the issues - of Sanskrit Cosmopolis,\index{Cosmopolis, Sanskrit} of Nazism\index{Nazism@Nazism}, of Rasa\index{rasa@\textit{rasa}} theory\index{Rasa Theory}, and of casteism and population genetics.\index{population genetics}

The opening paper \textbf{(Ch.1)} by \textbf{Manogna Sastry and Megh Kalyanasundaram} is entitled \textbf{“A} \textit{\textbf{Pūrvapakṣa}}\textbf{ of Sheldon Pollock's Use of\break Chronology”}. Chronology is “central to comparative intellectual-\break historical practice” asserts Pollock. But the “facts”and “evidence” that have been made available so as to assist in any reconstruction of Indian history - are all the ones that were methodically constructed by Westerners in their overbearing concern for the perpetuation of colonial rule. The time has come when the very “facts” need to be examined. All along, the colonialists scuttled several indigenous voices; and the European voices that evinced some positive taste for the Indian heritage are but objects of disdain for Pollock\index{Pollock, Sheldon}. Brimming with\textit{ parti pris}, his writings occupy themselves with driving a wedge between Hinduism and Buddhism\index{Buddhism} here, or Sanskrit and the regional languages there; or depicting the Mohammedan marauders as veritable savers of Sanskrit and temples, pitting them against indigenous rules who are portrayed as working to the contrary!

The focus of the paper is an examination of the historical veracity of the the first of “the two great moments in the career of Sanskrit” when, from its primary status as no more than a religious language, Sanskrit is supposed to have “reinvented itself as a code for literary and political expression” around the beginning of the Common Era (while in the second one, situated at the beginning of the second millennium, local languages supposedly came to replace it). Our authors have scoured two key writings of Pollock, to tabulate the various dates assigned by him to various authors/rulers/dynasties/ events in India and elsewhere (nearly 70 items). They have pointed out how there are many among those items simply lacking any sources/references whatsoever, and how several inconsistencies even within and among themselves abound. In his 2003 publication, Pollock says the Śātavāhana-s\index{Satavahana@Śātavāhana} did not support Sanskrit, and in the 2006 one, he refers to the same dynasty as both willing and able to use Sanskrit for its public records: Śātavāhana-s have changed so much within three years! In the former publication their date is cited as 100 BCE to 250 CE, and in the latter as 250 BCE to 200 CE! In just three years, their start recedes by 150 years! - and of course, no grounds need to be provided - either for the earlier date or the latter: my will be done, and it is done! That’s Pollock for you.

Pollock is inconsistent in his dates of Kātyāyana/Vararuci\index{Katyayana@Kātyāyana}\index{Vararuci} and Patañjali,\index{Patanjali@Patañjali} again between his 2003 and 2006 publications. “The date of the Buddha\index{Buddha, the} is the one key point for fixing Indian chronology\index{Chronology}”. A gap of eleven centuries separates the Indus\index{Indus Valley Civilization} Valley Civilization and the date of the Buddha, with no historical points in between, even though there is a fairly large body of literature for this period. The brazen title of the 1820 publication of Cambridge viz. \textit{A Key to the Chronology of the Hindus... \underline{to Facilitate the Progress of Christianity\index{Christianity} in Hindustan}}… (\textit{underline ours}) betrays what the Indologists then were up to, and gives away equally well, a clue to what the Neo-Orientalists of today are up to. Contrary to his own claims, Pollock\index{Pollock, Sheldon} has hardly critiqued, or even cited, a host of important insider sources on issues of the identity or date of Candragupta\index{Candragupta} or the Buddha\index{Buddha, the}.

Reckoning the irresponsible strides of Pollock who says his 2006 item could have an alternate subtitle viz. “A study of Big Structures, Large Processes, and Huge Comparisons,” the authors aptly remark that a more apt subtitle would be “\underline{A Study of Unsound Structures, Illogical} \underline{Processes and Inaccurate Comparisons}”. In his overriding anxiety to adjust facts to his fancied theories, he asserts the simultaneity of the production of \textit{kāvya}\index{kavya@\textit{kāvya}} and invention of writing, but the edicts of Aśoka\index{Asoka@Aśoka}\index{Asokan edicts@Aśokan edicts} go back to 320-150 BCE as per his own admission - so he is either overlooking or ignoring the incongruity therein. It is the overbearing prejudice and anxiety that makes Pollock posit a post-Buddhist date for Jaimini\index{Jaimini}. His facile equation of Sanskrit - as little more, or little else, than the sacerdotal in the pre-Christian era, and the preeminently profane in the post-Christian era - also stands exposed. He attempts to give political colour to Buddhism’s\index{Buddhism} early rejection of Sanskrit, and subsequent capituation with Sanskrit - all remaining inexplicable for him, or indeed needing nothing but convoluted explanations, or better sophisticated concoctions. His imagination is fertile: “freedom” of Sanskrit from its sacerdotal shackles, and its ensuant politicisation, were all subsequent to - and by implication, consequent to - the influence of Western Asian and Central Asian peoples. \textit{Rāmāyaṇa}\index{Ramayana@\textit{Rāmāyaṇa}} must - for him - be post-literate. His claim of Sanskrit grammar as a tool of monopolisation and political manipulation is absurd, to say the least. It may be stated in conclusion that what we find in Pollock’s writings is fantastically conceived exuberance of theories accompanied by a frantic search for the rare select factoids that can somehow be made to fit his pet theories.

The next paper \textbf{(Ch.2)} by \textbf{Nilesh Oak} is entitled \textbf{“Astronomy\index{Astronomy} and Epic Chronology\index{Chronology}”}. Basing on the dates of the earliest manuscripts/inscriptions, Pollock assigns 200BCE to 400CE as the possible date for the \textit{Rāmāyaṇa}\index{Ramayana@\textit{Rāmāyaṇa}} and the \textit{Mahābhārata}\index{Mahabharata@\textit{Mahābhārata}}. Pollock assumes AIT (Aryan Invasion Theory\index{Aryan Invasion Theory}) ignoring many pieces of evidence to the contrary. Seeing that the arguments of Pollock betray his biases and lack scientific rigour, Oak dwells first on issues and concepts of a proper and sound scientific methodology. He opts for an examination of possible objective evidences for the lower/upper limits with respect to the chronology\index{Chronology} of events in, or time of the composition of, the texts of the \textit{Rāmāyaṇa/Mahābhārata}, and suggests testability as a key criterion of scientific evidences. All relevant evidence must be evaluated and tested. Astronomical/geological/hydrological evidences must all be looked into in the the case of the dating of the \textit{Mahābhārata} or the \textit{Rāmāyaṇa}. One can thus look for corroboration from multiple disciplines.

Restrictive evidences - the ones which rule out certain possibilities - are the ones that are to be valued the more, as they can help make our estimates narrower. The \textit{Rāmāyaṇa} and the \textit{Mahābhārata} provide many astronomical/chronological markers. The natural cycles - of 26000 years, of 72 years, and of 1000 years, to speak of but a few - figure in the reckoning of what constitutes the North Pole\index{North Pole} of the particular era/times (owing, for example, to the precession\index{Precession of Equinoxes} of the equinoxes); so too the \textit{nakṣatra-}\index{naksatra@\textit{nakṣatra}} frame of reference with respect to the timing of the Winter Solstice\index{Winter Solstice}, and the cardinal points - which follow certain cycles. Instances such as the reference to the movement of Arundhatī\index{Arundhati@Arundhatī}\index{Arundhati-Vasistha@Arundhatī-Vasiṣṭha}\index{Vasistha@Vasiṣṭha} and Vasiṣṭha asterisms, are reckoned and handled by but a handful among over a century of research scholars in this regard. Similar is the Yuddha\index{Yuddhakanda@Yuddha Kāṇḍa} Kāṇḍa astronomical reference in the \textit{Rāmāyaṇa}.

We find in Pollock\index{Pollock, Sheldon} neither the sensitivity to issues such as these, nor the integral understanding of a massive work such as the \textit{Rāmāyaṇa} which is what makes him suspect whether there is unity in even the two adjacent sections of the \textit{Rāmāyaṇa} viz. the Ayodhyā\index{Ayodhya Kanda@Ayodhyā Kāṇḍa} Kāṇḍa and the Araṇya\index{Aranya Kanda@Araṇya Kāṇḍa} Kāṇḍa which appear for him “but a congeries of utterly distinct and unrelated materials”. All his concern is towards sabotaging the poem of Vālmīki\index{Valmiki@Vālmīki} as a unitary work - precisely because it is considered by the Hindu tradition as, to press into service his very words, “the first and greatest poem venerated as such for two thousand years”.

\textbf{Ravi Joshi}’s article \textbf{(Ch.3) “Hindu-Buddhist Framework: Detonator of Western Indology”} shows how the West deploys utterly incongruent frameworks while assessing Eastern cultures: religion as a category may well suit Abrahamic faiths, but it ill serves the Eastern value-systems. Pollock reads tropes that fit Western history such as “Catholic vs. Protestant\index{Protestant Christianity}” into the Eastern as “Theistic Hierarchical Hinduism'' vs. ``Protestant Egalitarian Secular Buddhism”, or “Spiritualistic Evangelistic Buddhism” vs. “Ritualistic Escapist Hinduism''! Westerners are never tired of harping on some kind of Hinduism-\break Buddhism divide or the other. McKim Marriot\index{Marriot, McKim}, A K Ramanujan\index{Ramanujan, A. K.}, and\break Rajiv Malhotra\index{Malhotra, Rajiv} have shown the applicability of the Dharmic framework rather than at all of “religion” for the Indic systems. The equation of \textit{dharma} and religion has wrought havoc on no small a scale, and for more than a century.

Even when it comes to fixing dates, the Biblical exigencies are inviolable for a typical Westerner. Indologists - Western brand as well as their Eastern moulds - have typically, or rather systematically and relentlessly, white-washed the endless and ruthless devastation wrought by the Islamic\index{Islamic invasion} invaders of India right from Bakhtiar Khilji\index{Bakhtiar Khilji}. All current academic frameworks are Western defaults. Given that the West's first exposure to Buddhism\index{Buddhism} is from the Far East (rather than from India), and that it appears coherent and stable, as against Hinduism which must loom as but constructed and chimerical. India has been, on the whole, dethroned from the status of an exotic mother civilization to a colony of defeated kingdoms.

There is hardly anything indeed in Buddhism to mark it as any radical social revolution - including that of its evaluation of the \textit{varṇa-jāti}\index{jati@\textit{jāti}}\index{varna@\textit{varṇa}} system, the role of which is in no wise subdued or sabotaged, or questioned overtly or covertly by the Buddha\index{Buddha, the}. The Buddha maintains silence with respect to many issues - but the same cannot be claimed to amount to any categorical denial of theism. The Buddhist categories and terminologies are little else than close kins of and easy derivatives of their Hindu counterparts. The clearly discernible motive for German Indology was to demonise and displace the sound and strong traditional Brahminical scholarship, and the intellectual intrigue was ultimately aimed at usurping the same. The Axial\index{Axial Theory} hypothesis is yet another attempt to construct a new grand narrative of world history - but in no way disbanding the Western Indologists’ hackneyed premises and prejudices. For Pollock\index{Pollock, Sheldon}, all religion is essentially a cover for politics! Though Buddhist texts unabashedly declare that the Buddha learnt meditation from \textit{yogin}-s, Pollock fantasises that Buddhist meditation is a precursor to \textit{all} meditation systems - a rare “intellectual” temerity indeed! The writings of Staal\index{Staal, Frits} can indeed show how Pollock’s vile attempts at misrepresenting the concept, or practice, of \textit{yajña},\index{yajna@\textit{yajña}} are all baseless. Pollock’s endeavour - of somehow showing Buddhism as civilizationally disruptive of the Indian heritage - simply falls flat in the light of the abundance of facts to the contrary.

\textbf{M. V. Sunil}'s article \textbf{(Ch.4) “The Upaniṣad-s: The Source of the Buddha’s Teachings”} exposes the distortions of Hinduism wrought by Western academicians - who, it must be cautioned, are neither its practitioners nor insiders. Mixing up the \textit{vyāvahārika} and the \textit{pāramārthika}\index{paramarthika@\textit{pāramārthika}} is one of their handy tools. One cannot afford to ignore the commonality of philosophical approaches in lieu of the divergences in some of the external rituals in each case. The Buddha introduced but a new terminology, while handling more or less the self-same categories of thought. Hinduism and Judaism\index{Judaism} could perhaps be labelled as the original religions, and the rest are all their offsprings: Buddhism\index{Buddhism}, Jainism\index{Jainism} and Sikhism\index{Sikhism} are thus mere offshoots of Hinduism. Advaita\index{Advaita} and Buddhism alike negate the idea of the individual self. While Pollock\index{Pollock, Sheldon} seeks to make out that the core conception of the Upaniṣad-s were cancelled in early Buddhism, such suppositions/presuppositions are anticipated and annulled by authentic and more authoritative writers such as Rhys Davids\index{Rhys Davids, W. T.}. The concept of \textit{Nirvāṇa}\index{nirvana@\textit{nirvāṇa}} is after all akin to the Upaniṣadic \textit{Brahman}\index{Brahman@\textit{Brahman}}. Common are the epithets and descriptors used for \textit{Nirvāṇa} on the one hand, and for the \textit{Ātman/Paramātman}\index{atman@\textit{ātman}}\index{paramatman@\textit{paramātman}}/Self on the other. The last words of the Buddha\index{Buddha, the} were “Let the Self be your light and shelter” - which is but the entity within oneself, not outside.The Buddha was of course concerned with the dilution and decline in the ethical standards of some of the custodians of the Veda-s. A survey of the above and related ideas shows that there are no foundational differences between the Vedic and the Buddhist traditions.

\textbf{Rajath Vasudevamurthy}’s \textbf{“Vedic Roots of Buddhism” (Ch.5)} starts with an assessment of the motives of Western Indologists. Differences of opinion or approach are only common in philosophical discussions in India as elsewhere, but the Western Indologists are seen exploiting the same for petty political gains - and to serve which nonexistent differences are projected, and even magnified at the outset. Modern/Western lenses are systematically used in analysing ancient/Eastern societies and thought systems. While Max Muller\index{Müller, Max} asserts that all the faculties of ancient Indians were devoted to the inward life of the soul, Pollock fancies the opposite extreme viz. that the very language (Sanskrit), even its grammar, and too, its poetry - are all politically oriented! Pollock involves the Axial\index{Axial Theory} Theory, which as per Jaspers\index{Jaspers, Karl}, its propounder, involves a new sociopolitical formation - a like of which, however, never occurred in India, at least prior to the 20th century. Pollock concedes on the one hand that Sanskrit literary culture spread from Afghanistan to Java, and the trans-local empire; yet at the same time asserts that Buddhist thinkers produced one such moment in early South Asia; but again, that nothing like an Axial Age occurred in India prior to the 20th century. Pollock\index{Pollock, Sheldon} is a past master not only in cherry-picking but also blowing hot in one breath and blowing cold in another - or rather in the same, breath. If, as Aiyaswami Sastri\index{Sastri, Aiyaswami} says, the pre-Buddhist Jains and Ājīvikas\index{Ajivika@Ājīvika} already showed the characteristics Buddhism\index{Buddhism} displays, where then is any question of speaking of Buddhism as a breakthrough? If Pollock speaks of “a lay community of co-religionists (\textit{upāsaka-s})” as a development brought about by Buddhism, the \textit{Mahābhārata}\index{Mahabharata@\textit{Mahābhārata}} exemplifies Vidura and Dharmavyādha as prominent teachers of Vedānta - such as is fit to be, and was, venerated by even those knowledgeable in the Veda-s - despite the fact that the “lower” classes were denied access to the Veda-s. (Ignoring time-scales, the) Veda-s and the Buddha\index{Buddha, the} present a parallel: oral-teaching for a long time followed latterly - by committing the teachings to writing. If the Buddha criticised the sacrificial act of slaying an animal, we already have a superior archetype in the Veda itself (in the \textit{Mahānārāyaṇa Upaniṣad}\index{Mahanarayana Upanisad@\textit{Mahānārāyaṇa Upaniṣad}}). The Buddhist doctrine of \textit{anattā} (“non-self”) is but a continuation of the “\textit{neti neti}” of the Upaniṣad-s. While Pollock attempts to pit Pali against Sanskrit, A. N. Upadhye had shown how the two coexisted for long, and betrayed little mutual animosity. Pollock tries all tricks to explain anew the return to Sanskrit by the Buddhists but none of them explains all facts. Much is made of the supposed role of the Buddha as a social reformer, but the Buddha is not known to have suggested the removal/alteration of the \textit{varṇa/jāti}\index{jati@\textit{jāti}}\index{varna@\textit{varṇa}} system. The continuity of the caste system in India from times immemorial to our own times is a grand riddle - for economic historians and sociologists alike. Not only has Pollock not been able to explain any well-established facts with his fantastic theories, he has not ably propounded any new theory to see or show things in a new light.

%\newpage

\textbf{Sharda Narayanan}’s paper \textbf{(Ch.6) “Brāhmanism, Buddhism\index{Buddhism}, and Mīmāṁsā\index{Mimamsa@Mīmāṁsā}”} gives an overview of the Mīmāṁsā tradition, and lays bare the implausibility of claiming Mīmāṁsaka-s as the aggressors, with Buddhists as the victims. Targeting brahmins has always been high on the agenda of Western Indologists, and Pollock is no exception - despite the fact that a life of discipline, one sans pelf and power, was ordained for the brahmins, which they followed all through, generally speaking, till even recent times. Blaming the brahmin or the Mīmāṁsaka for the age-old caste system or Nazism\index{Nazism} evinces non-adherence to historical facts on the part of Pollock, and the political motivations of the master misinterpreter of history. “Aryan” interpreted on racial lines has no basis whatsoever.

The fantastic claim of Bronkhorst\index{Bronkhorst, Johannes} that the beginninglessness of the universe was propounded by the Mīmāṁsaka-s - “apparently for an entirely non-philosophical reason: the distaste felt for the newly arising group of of brahminical temple priests” - is well-refuted by her. When you prefix your statement with “apparently” (or even “virtually”), you can present any nonsensical idea without a feeling of guilt and /or, more importantly, without getting caught. The Mīmāṁsaka-s evinced little distaste towards the temple-priests; but then, are the Western scholars envious in no small measure towards the Indian scholars of yore, or even the contemporary pundits, because they are after all themselves driven - apparently, let us say - by the leftist frameworks or leftist agenda? On the whole, Pollock’s\index{Pollock, Sheldon} attack on Mīmāṁsā for its “confrontation with history” is completely off the mark.

The paper \textbf{(Ch.7)} by \textbf{Arvind Prasad} viz. \textbf{“Chronology\index{Chronology} Beyond ‘Sanskrit Cosmopolis\index{Cosmopolis, Sanskrit}’”} looks at Pollock’s ideas on Sanskrit; his very coinage ‘Sanskrit Cosmopolis’ is directed towards baselessly, hence brazenly, cooking up an idea of Sanskrit as a political tool - of all! It is only typical of Pollock to pose as a careful writer, and state preemptively that his dates are tentative and can be disputed, but then by and by press forward a little later and present it all as if he has actually already proved what he at the outset had presented with trepidation as but a hypothesis for consideration. The paper anchors on a key publication of a well-established author viz. Baldev Upadhyaya\index{Upadhyaya, Baldev} to substantiate its solid claims as against the puerile Pollockian posits. Sanskrit was developed in terms of grammar and literature, essentially as a tool for colonising, asserts Pollock - the Pollock who imagines a balkanised past of India, and so fervently dreams of its balkanised future - which can after all never come true. (Many “leading” intellectuals have cherished stupid dreams - much like Karl Marx’s “The State will wither away!”, whereas it is only such silly dreams that have withered away). Prasad is able to see through the politician in Pollock in the very coinage of the term ‘Sanskrit Cosmopolis\index{Sanskrit Cosmopolis}’, and the nonsense of supralocal ecumene. Pollock tries to provide statistics of the percentage of inscriptions in Sanskrit versus local languages in a given period, but has no documentation to be appended so as to give a semblance at least of the same. His wild theories of \textit{praśasti}\index{prasasti@\textit{praśasti}} are subverted by the \textit{praśasti}-s\break that we find composed even for the Dutch lords! Pollock\index{Pollock, Sheldon} toes the usual line of Christian evangelists in spewing venom against “power-hungry” Brahmins, but the abundance of evidence produced by Baldev Upadhyaya\index{Upadhyaya, Baldev} bears out the utter hollowness of the imaginary claims of Pollock. Pollock is befuddled in numerous ways: Hanneder\index{Hanneder, J.} and Sastry have, too, shown several fallacies in Pollock’s arguments. Pollock is indeed good - but only for obfuscation.

The paper \textbf{(Ch.8)} by \textbf{Vishal Agarwal} viz. \textbf{“Hinduism: a Precursor to Nazism?\index{Nazism}”} takes on the view of Pollock that Hinduism anticipates Nazism. Oriental scholarship did act by and large as but a handmaiden, after all, of European colonialism and imperialism. Early 18th c., Schlegel\index{Schlegel, Friedrich von} mooted the idea of a master Aryan race\index{Aryan race}; and the IE languages were linked to the Aryans. But the British could hardly bear with any racial affinity to Brahmins, the elite among the Hindus. A hierarchy of races - with Europeans, who else, at the top was framed. A premodern racism, Pollock posits, has deep roots in the śāstric tradition: the \textit{śūdra-s}\index{sudra@\textit{śūdra}},\break he imagines, were castigated like the Jews in Nazi Germany! He tries to drive a wedge between the upper classes versus not only the \textit{śūdra-s},\break but even women, Buddhists, and Jains. The sinister motivations of Pollock in his “Deep Orientalism” are but patent. Much has been made of the Aryan “stock”, and the colours - of the skin, hair and eyes; whereas in vivid contrast, sages are themselves described as dark, or even praying for black hair. It is only Nazis who equate language to race, and race to one’s looks. Unlike Nazis who sought the expulsion of Jews, never did Dharmaśāstra-s seek the expulsion of \textit{śūdra}-s.

In order to counter Pollock arguing on the basis of Mīmāṁsā\index{Mimamsa@Mīmāṁsā}, Agarwal provides the very Mīmāṁsan grounds for exactly the opposite conclusions. In any case, whoever speaks today for Hindus is branded a Hindu Nationalist; and a scholarly hatemonger that Pollock after all is, he loves to portray Brahminism as premodern racism. Preaching anti-Hinduism, Pollock is a bird of the same feather - as the Nazi scholars preaching anti-Semitism\index{anti-Semitism}.

\textbf{Sharda Narayanan}’s paper \textbf{(Ch.9)} entitled \textbf{“A Rejoinder to \textit{A Rasa\index{rasa@\textit{rasa}} Reader”}} attempts to analyse Pollock’s interpretations and translations, and to showcase his methodology in distorting the tradition. Pollock’s annotations and comments are often of a disparaging and prejudiced nature. Pollock complains how even though Śiva was a dancer, God in India was generally not an artist! She concurs with David Frawley\index{Frawley, David} who says most Westerners do not go beyond the surface in what they see of Indian culture. She feels that Pollock’s work derives its value and gravity solely from the presentation of passages from the classical tradition of India. Drawing attention to Pollock’s\index{Pollock, Sheldon} question regarding the very number of \textit{sthāyi-bhāva}\index{sthayibhava@\textit{sthāyi-bhāva}}-s, she says the modern scholar’s conclusion borders on the absurd. She objects to the segmentation of \textit{rasa} as seen and as heard. The translation of \textit{rasa} as “taste” is also not a commendable translation; \textit{pratyakṣa}\index{pratyaksa@\textit{pratyakṣa}} is not just “visual”; \textit{vibhāva}\index{vibhava@\textit{vibhāva}} rendered as “factor” is also not satisfactory; \textit{śṛṅgāra}\index{srngara@\textit{śṛṅgāra}} as “sexual instinct” is hopeless; and again, that \textit{Nāṭyaśāstra}\index{Natyasastra@\textit{Nāṭyaśāstra}} does not speak of the \textit{rasa} in the spectator is also far from truth. Pollock’s translation of \textit{vipralambha śṛṅgāra} also does not do justice to the term. Embellishments of sound are belittled by Pollock as unnecessary fuss, but even Keith\index{Keith, A. B.} had the sense to note it as a matter of keen appreciation. Speaking of “seduction” by literature, which Pollock does, is shocking, she points out. Pollock has missed the role of “propriety”, and he even questions the language of aesthetics in \textit{bhakti}\index{bhakti@\textit{bhakti}}. Pollock’s translation of text titles are, too, problematic as are his translations of \textit{saṁskāra,\index{samskara@\textit{saṁskāra}} sattva,\index{sattva@\textit{sattva}} rajas\index{rajas@\textit{rajas}}} and \textit{tamas},\index{tamas@\textit{tamas}} etc. We keep coming across contentious claims by him such as the “demise of dramaturgical theory after about the 13th c.” and that “it was the Buddhists who invented compassion” etc.. Indeed Pollock out-Keiths Keith in his mordancy and sardonicism.

\textbf{Murali K. Vadivelu}’s article \textbf{(Ch.10)} entitled \textbf{“Othering\index{othering} and Indian Population Genetics\index{population genetics}”} suggests that studies of endogamy\index{endogamy} in the Indian context have been modelled after the Arab-Muslim clan-tribal endogamy. Social scientists do not seem to have grasped the concept of \textit{gotra-s}\index{gotra@\textit{gotra}} across \textit{varṇa-s}\index{varna@\textit{varṇa}}. Archival records indicate that the colonial education system completely replaced the indigenous education systems, and also curtailed tenancy rights - all leading to much ruination, all man-made - which is to say the inhuman-British-made. Westerners have misinterpreted the role of Sanskrit as an instrument of oppression. Political Philology, mooted by Pollock, is a dehumanising ideology.

%\vspace {.15cm}

Citing the hearty acclaim by Ambedkar\index{Ambedkar, B. R.} regarding the uniqueness and richness of Sanskrit literature, Vadivelu notes how the modern political-academic campaign against it borders, rather, on hysteria. For Pollock, Sanskrit knowledge presents itself as a major vehicle of the ideological form of social power in traditional India; gross asymmetries of power characterise India over the last three millennia. As against Pollock\index{Pollock, Sheldon}, even Ambedkar admits that the caste system existed long before Manu\index{Manu}. The caste system cannot be attributed to Brahmins, asserts Ambedkar. For him, the caste system was essentially a class system. Initially brahmins were endogamous, and subsequently others followed suit. Dharampal\index{Dharampal} cites William Adam\index{Adam, William} who notes that Brahmins studied till they were nearly forty, and would even beg for their survival during their avid pursuit of studies.

%\vspace {.15cm}

On the other hand, the poor education of Muslims during those times, and the continued practice of the caste system among the very converts to Islam\index{Islam} - even during those days, are all in fact well-recorded. As to the education of \textit{śūdra-s}\index{sudra@\textit{śūdra}}, it may be mentioned that nearly 80\% of the total students in Tamil speaking regions, for example, were \textit{śūdra-s}. Thomas Munro\index{Munro, Thomas} stated in unmistakable terms that India was more civilised than England. Apparently, Muslim-rule enforced greater endogamy\index{endogamy} than might have been current. The division of society is linked to their professions by and large. Ambedkar clearly warned that conversion to Islam or Chirstianity will denationlise the depressed classes, which is so well borne out today. Ambedkar was in fact all for making Sanskrit the national language of India.

%\vspace {.15cm}

Of othering\index{othering} on the basis of colour, the typical Westerner’s practice, and charge against Hindus, the answer is found in the dark divinities viz. Lord Kṛṣṇa, Kālī, or even Draupadī. Ambedkar took note of the endless atrocities of the Muslim rulers - circumcising brahmins, making slaves of Hindu women and children, slaughter of Hindus on a large scale, plunder of temples, and so on. Especially, during Muslim rule, it is brahmins who were “othered”, and it continues to this day - bearing out and standing as a witness to - the very antithesis of Pollock’s\index{Pollock, Sheldon} claims.

It must be said in conclusion that Pollock’s Political Philology and Liberation Philology thus go against all empirical and scientific evidence.

The dictum of \textit{Yoga-Vāsiṣṭha}\index{Yogavasistha@\textit{Yoga-Vāsiṣṭha}} is after all not too off the mark:
\begin{center}
\dev{“पापा म्लेच्छा धनाढ्याश्च} \\\dev{नाना-देश्यास्सुसंहताः~।}\\\dev{बहवो लब्ध-रन्ध्राश्च} \\\dev{सामादेर्नास्पदं द्विषः~॥”}
\end{center}

“The vicious \textit{mleccha}-s from various countries, affluent and well-organised, large in number and given to accessing others’ loopholes - are veritable foes, fit indeed for no conciliatory dialogues.” Deal with the devious after their own fashion - so dictates the \textit{Mahābharata}:\index{Mahabharata@\textit{Mahābharata}}
\begin{center}
\dev{मायाचारो मायया वर्तितव्यः }\\\dev{साध्वाचारः साधुना प्रत्युपेयः~॥}
\end{center}

It remains to be added that the various contributors hold themselves responsible for their statements in their various papers. And lastly, we crave the indulgence of the readers - for, for certain practical reasons, full conformity to the standards set in the previous volumes in the Series could not be thoroughly ensured in this volume.


\vspace{0.35cm}

 \phantom{a}~ \hfill \textbf{Dr. K. S. Kannan}\\
Cāndramāna Yugādi \hfill Academic Director\\
Śārvari Samvatsara \hfill and\\
(25-March-2020)\hfill General Editor of the Series


