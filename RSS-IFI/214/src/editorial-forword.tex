\chapter*{Editorial Foreword}\label{preface5}

\vspace{-.8cm}

Besides the problem of the Two Cultures, history, even of science \& technology, has its own biases and prejudices. The West has developed its history of science \& technology tracing most of the innovations back to Greece. This Euro-centric distortion has ignored a variety of great innovations that were made in countries like India and China. When Joseph Needham brought out a series of 50 volumes on Chinese Science and Civilisation, it resulted in the recognition of the Chinese scientific innovations by the West, but the Indian contributions remained unnoticed. In India itself it had its own non-believers who dismissed the scientific innovations that India made. On the other hand, there is a jingoistic fringe which finds even hydrogen bombs, rockets, jet planes in prehistoric India. A holistic study of history of science and technology was further complicated by the rigid  periodisation of Indian history into ancient, medieval and modern periods. There are great experts for each period but hardly any who has a comprehensive and holistic view of Indian history.  Vibha Tripathi has for the first time taken a holistic view of India as far as science \& technology are concerned, right from prehistoric beginnings to modern times.  She has taken great pains in tracing the global history of iron technology, especially in West Asia and China. She has made it clear that in West Asia extensive use of iron and iron smelting starts only the I millennium BCE; even in China iron technology makes its presence felt only around 7-8th century BCE. Of course, the Chinese iron technology is based on cast iron but in India it was mostly wrought iron which later on was carburised to make steel. She has explained how in West Asia iron was treated as more expensive than gold in the pre-1100 BCE  period. The earliest iron was more ornamental than utilitarian.

\newpage

We just mentioned the yawning gap between the Two Cultures in most of the studies made on early iron technology and the analyses of references to Vedic and Buddhist literature. Vibha herself is of the opinion that iron technology was not known to early Aryans. Most of the workers on early iron technology depended upon literary and archaeological sources. There were hardly any professional metallurgists who could look at the early Indian iron technology. Luckily, things have changed now and professional metallurgists like Balasubramaniam, Bhanu Prakash, Biswas, and others, have gone into the technology aspects of early iron. 

Today we have reached a piquant situation where the antiquity of iron is being pushed back continuously but there is very little metallurgical evidence of smelting, chemical and metallographic analysis of the finds. There is a clear technological evolution from the PGW stage to the late NBPW. Chronologically it is in concordance with the West Asian and Chinese evidence. But when we trace back the origins of iron technology to early second millennium BCE, earlier than anywhere else in the world, queries are raised. Why doesn’t early iron technology make an impact on the socio-economic development and process of urbanization ? Was the early iron meteoritic? nobody has bothered to see so far. Was there an ornamental stage of precious iron in India too? Did we also begin with meteoritic iron as in West Asia? 

There is further complication as in India we have two traditions – the Greater or Margi and the Lesser, the desi tradition. If iron technology was confined to the folk tradition, one does not expect to see a socio-economic impact on the contemporary society. The tribal tradition had kept the traditional iron technology alive. Like they do even today, they are essentially a self sufficient community only occasionally in contact with the townspeople.  Thus they could not have made much impact on the socio-economic development.

The other tradition was the Greater one, which seems to have learned the iron technology only in the beginning of the first millennium BCE. Once iron technology came into the main stream, it did make an impact on the socio-economic front and in the process of urbanisaton as reflected in the growth of the city states, the mahajanapadas, high grade ceramic technology, a variety of iron artifacts for masonry, glass technology, agriculture etc. Tripathi’s grand marshalling of the total evidence would indicate that it was the folk science, the Lesser Tradition that began iron technology, though regular production could start only with pyro-technological innovations and social organisation.

The Infinity Foundation (IF) has sponsored a series of books on history of Indian science and technology. Under this project, we are covering a variety of topics like the Harappan technology and architecture, the hydraulic technology in the south, among the Harappans and also during the historical period. In iron alone we are bringing out three  volumes covering the history of iron technology through the ages, Wootz steel and monumental iron structures. The production of zinc, using distillation process in the 12th-13th century on an industrial scale, was an Indian innovation and will  be covered by another volume.. Further volumes are being planned on astronomy, textile technology, Chalcolithic technology etc; our aim is to produce about 25 volumes. We are taking up each technology from its origins to its modern condition.  At times therefore we have to produce more than one volume for a given technology.

Our main aim is to highlight the Indian contributions to the global science and  technology so that we have a distortion free  and balanced world history of science and technology.



\label{endpreface5}
