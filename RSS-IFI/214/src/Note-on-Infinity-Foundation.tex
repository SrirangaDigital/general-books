\chapter*{Note on Infinity Foundation}\label{preface3}

\vspace{-.8cm}


\textbf{Facilitating Research, Publishing and Education in: History of Indian Contributions to Science and Technology (HIST)}

\begin{myquote}
\it{“In 1750 China accounted for almost one-third, \textbf{India for almost one-quarter} and the West for less than a fifth of the world’s manufacturing output… In the following decades…the industrialization of the West led to the de-industrialization of the rest of the world".}
\end{myquote}

\vspace{-.3cm}

\begin{flushright}
 —Samuel Huntington, Clash of Civilizations, pp.86-87
\end{flushright}

It has become largely forgotten that India held its prominent economic position referenced above because of its accomplishments in science and technology. The commonly taught history of India emphasises mostly kings, invasions and conflicts. The history of Indian ideas, if taught at all, is limited to religion, philosophy and popular culture. What is being often ignored is the well-documented evidence of India’s significant contributions in metallurgy, civil engineering and architecture, water harvesting, shipping and ship building, textiles, medicinal plants, medicine, agriculture, forest management, management, astronomy and linguistics, among other disciplines.

By way of comparison, Joseph Needham’s 30+ volumes on Chinese history of science and technology have made a monumental impact in the academic repositioning of China as having a rational and progressive civilisation. This work is a premier reference in China studies worldwide. But nothing equivalent exists for India, as explained below:

\begin{myquote}
“Fear of elitism did not, happily, deter Joseph Needham from writing his authoritative account of the history of science and technology in China, and to dismiss that work as elitist history would be a serious neglect of China’s past…"
\end{myquote}

\newpage

\begin{myquote}
\textbf{“A similar history of India’s science and technology has not yet been attempted, though many of the elements have been well discussed in particular studies}. The absence of a general study like Needham’s is influenced by an attitudinal dichotomy. On the one hand, those who take a rather spiritual — even perhaps a religious — view of India’s history do not have a great interest in the analytical and scientific parts of India’s past, except to use it as a piece of propaganda about India’s greatness (as in the bloated account of what is imaginatively called ‘Vedic mathematics’, missing the really creative period in Indian mathematics by many centuries). On the other hand, many who oppose religious and communal politics are particularly suspicious of what may even look like a ‘glorification’ of India’s past. \textbf{The need for a work like Needham’s has remained unmet.}"
\end{myquote}

\vspace{-.5cm}

\begin{flushright}
— Amartya Sen, 1997
\end{flushright}

\vspace{-.2cm}


HIST is an important part of India’s story because it has been the substratum of its civilisation’s rationality and secular progress, the basis for pre-colonial Indian Ocean global trade, the foundation for building India’s future knowledge society, and a key element in projecting Brand India.

Infinity Foundation, a private non-profit foundation based in Princeton, USA, launched its HIST project with the following objectives:

\begin{itemize}
\itemsep=0pt
\item Document and discuss the history of scientific and technological achievements in the Indian subcontinent until the end of the nineteenth century, because the period after that has already been well documented.

\item Inspire creativity and self-confidence among our youth, refuting the popular notion that Indians were irrational and mystical. Replace colonial notions of intellectual dependency and revive India as a premier knowledge exporter. 

\item Bridge the socio-economic divide: It is often the rural and the underprivileged in India who have preserved key aspects of traditional knowledge, such as in medicinal plants, or traditional civil engineering and home building. By restoring legitimacy to traditional knowledge, which is decentralised and less capital intensive, we empower local/rural cultures, lifestyles and economies. If India hopes to create genuine alternatives to rural migration and the burden of high-density metro lifestyles, we need to re-engage with, and build upon, the strengths of traditional knowledge systems. 

\item Build India’s brand, cultural capital and soft power: Ancient India’s higher education and technological genius attracted the cream of Asian students to Nalanda, Taxila and other Indian centers of learning. The alumni and visiting scholars spread India’s cultural capital across Asia. Today, USA has a strategy to educate the future leaders of the world to gain influence globally. China has also embarked upon a strategy to export its higher education for geopolitical influence. If India hopes to preserve and expand its role in the information economy or in engineering design, she needs a civilizational or “cultural brand” in the same sense as the following culture specific brands:
\end{itemize}

\begin{tabular}{ll}
 Japanese-ness  &  Quality\\
 French-ness  &  Beauty/aesthetics: cosmetics, fashion,\\
              &   wine,Cannes, tourism\\ 
 Italian-ness  & Design, art, cuisine\\
 British-ness & Justice, law, rationality\\
 German-ness & Precision, manufacturing\\
 Chinese-ness  & Efficiency
\end{tabular}

In each of these cases, the soft power wielded by the country’s brand has fueled economic expansion using its civilisational advantages. It is important to restore a realistic counterbalance to India’s negative branding brought about by some of the thrusts of the social sciences until now. The following table is not intended as a generalisation but merely as indicative of some differences between two ways of seeking Indian culture – the one by social scientists and the other by technocrats such as those pursuing HIST research.

\newpage

\begin{tabular}{ll}
\textbf{Social Sciences on Indian-ness}    &  \textbf{HIST on Indian-ness}\\
 Mystical and otherworldly   &  Rational, innovative and\\
  & creative\\
 Frozen in time and backward &   Pragmatic and\\ 
 & progressive\\ 
 “Caste, dowry, sati, conflicts”  & Multicultural and\\
 & interdependent\\
 Knowledge came from invaders & Indian knowledge\\
 & systems\\
 British made us a nation & Many prior nations\\ 
 & existed\\
 Lacking self-esteem and afraid of future  & Confident and\\
        & competitive\\
 Defensive and closed & Open and globalised
\end{tabular}

Once again, Indians are increasingly being acknowledged worldwide for their brainpower, but the civilisational brand is dominated by “caste, cows, curry” images from anthropology. Furthermore, many Indians have distanced themselves from what they see as “baggage”, bringing about disconnects with their own past. HIST is a secular and all-encompassing shared past for nation-building.

It is important to bring scientists and technologists into this historical research so as to stay focused on the evidence and not the politics. The goal is to not glorify but analyse objectively, in order use the knowledge for present and future growth and to learn from past mistakes. Infinity Foundation’s HIST Project.

\textbf{Infinity Foundation’s HIST Project}

Although Infinity’s vision is to implement this unprecedented, composite and inclusive series, we would like to acknowledge and applaud many pioneering research and contributions made in this field, such as the following, and work with them to expand the discipline:
\begin{itemize}
\itemsep=0pt
\item 5 volume series on traditional agricultural practices by Indian Council for Agricultural Research;

\item 2 volume series on History of Science by Dr D.~P.~Chattopadhyay;

\item History of Indian Medicine by Dr. S. Valliathan;

\item Works of Indian National Science Academy; and 

\item many others.

\end{itemize}

The HIST project was started in 2002, after being conceived at the Foundation’s Colloquium in Woodstock that was jointly convened by Infinity Foundation and Columbia University. So far, only the series on Materials and Technologies has been started. Two other series, one on Life Sciences and another on Theoretical Sciences, are envisioned for the future.

Annual project meetings have been held all over India to review progress, share research and foster collaboration. These meetings — in places as varied as Indian Institute of Technology, Kanpur, India International Center (Delhi), Kumaon University (Nainital) and Banaras Hindu University (Varanasi) — have also served as outreach to local colleges and to the general public by giving them a chance to interact with our academic scholars. Each volume takes approximately three years to develop the final manuscript, and involves a team of scholars in addition to the author(s).

\textbf{Topic Selection Criteria}

\begin{itemize}
\itemsep=0pt
\item The following criteria were applied in selected topics for the series:

\item Time period of focus should be sciences and technologies developed until the end of the nineteenth century. 

\item Topics must have empirically verifiable and significant Indian contributions.

\item Subject matter experts should be available, especially within India. This helps the growth of research communities invested in such investigations and secures the roots of this work in India over time. 

\item Religion and “stories” should be avoided as evidence, i.e. no “Pushpak Viman” to be included. The goal is an honest exploration, not glorification. 

\item The sources and scholar must be internationally credible.

\end{itemize}

The steps include rigorous peer review and editorial review at each step, and project administration by Infinity Foundation. Infinity Foundation owns the copyright.


\textbf{About Infinity Foundation (IF)}

Infinity Foundation is a 501 (c)(3) non-profit tax-exempt private foundation that was started in 1995 as a vehicle for its founder to give back to the Indian and American societies. For further information, please contact:

Project: http://www.indiascience.org/\\
Foundation: www.infinityfoundation.com\\
Address: 53 White Oak Drive, Princeton\\
Email: infinityfoundation@gmail.com

\begin{flushright}
 \textbf{Rajiv Malhotra}\\
 \textbf{President, Infinity Foundation}
\end{flushright}

\label{endpreface3}
