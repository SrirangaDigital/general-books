\chapter*{Foreword}\label{preface}


The present book by Professor Vibha Tripathi is an excellent and up-to-date synthesis of a wide range of data on Indian iron, and I thank Professor Tripathi and the editor of the series in which the book is published, Professor D.~P.~Agrawal, for kindly giving me the privilege of academically introducing it.

\vspace{-.3cm}

\begin{center}{\textbf{(I)}}\end{center}

\vspace{-.3cm}

To argue that Indian iron and steel had a global reputation in the pre-modern period may not be enough. The extent of this reputation has to be driven home, and this one can do only by citing the relevant sources from the countries to which Indian iron and steel used to be exported. The fact that the famous Damascened swords of the mediaeval period used to be made of Indian steel, specifically south Indian steel which was known as \textit{Wootz} (derived from the local term \textit{Ukku}) among the Europeans, means that there must be an extensive body of trade documents in west Asia, which would throw light on the extent and mechanism of this trade. Early in the 18$^{\rm th}$ century the best steel in the Cairo market came from India, and as late as the 1830s one reads about the presence of Persian merchants at Nirmal in the Adilabad district of Andhra to purchase steel. There was a time, presumably in the 18$^{\rm th}$ century, when the people of Westphalia in Germany used to roll puddle steel into half-inch or three-fourth inch squares and sell them in the Hamburg market under the name of Indian steel. In the 11$^{\rm th}$ century Jewish trade documents discovered in the mediaeval quarters of Cairo there is a reference to a Jewish merchant based in Mangalore, who used to ship, among other goods, iron to the Middle Eastern and Egyptian market. Some of these data have been mentioned in my \textit{The Early Iron Use of Iron in India} (1992), and Professor Tripathi herself draws attention to some Chinese and Arab sources, but there is still scope for undertaking detailed archival research on this point. 

\vspace{-.3cm}

\begin{center}{\textbf{(II)}}\end{center}

\vspace{-.3cm}

A strong point of Professor Tripathi’s book is the emphasis she lays on the accounts of British engineers on the extent and quality of the manufacture of indigenous Indian iron and the more basic ethnographic documentation of the same. She deserves full credit for drawing attention to the survival of indigenous iron industry in some remote pockets of Jharkhand and for giving importance to the recent attempts to revive this industry. My familiarity with this topic does not extend much beyond Valentine Ball’s \textit{A Manual of the Geology of India : Part III – The Economic Geology} (1881), but those who have ploughed through the occasional records on the pre-industrial iron manufacture published in \textit{Journal of the Asiatic Society of Bengal} or the famous records of iron and steel manufacture in the early 19$^{\rm th}$ century survey reports of Francis Buchanan and Benjamin Heyne would know that there is a voluminous archival material to be discovered on this  throughout the length and breadth of the subcontinent. This is primarily for our historians to investigate. One doubts very much if the economic history of the 18$^{\rm th}$ century and early 19$^{\rm th}$ century India can be reconstructed without properly assessing the role of the contemporary iron industry. Occasionally one comes across references to the construction of bridges made of indigenous iron, and places like Tendukhera on the north bank of the Narmada are mentioned in this regard. But how many major centres of production of pre-industrial iron were there in the subcontinent at that point of time? We do not know the answer, but that does not mean it is not worth knowing, if we have to have an idea of the technological base of the country before the British moved in.

\vspace{-.3cm}

\begin{center}{\textbf{(III)}}\end{center}

\vspace{-.3cm}

Equally important in a related context is the issue of \textit{Wootz} or south Indian pre-industrial steel. Two crucial problems still remain to be worked out. First, what is the antiquity of this industry? In my 1992 monograph (pp.174-175) I argued that the vitrified crucibles and furnaces excavated by K Rajan at Kodumanal in the Coimbatore area in its early historic context (3$^{\rm rd}$ century BC, but probably earlier) denoted the early steel-making tradition of south India as recorded by Buchanan and Heyne. In view of the extensive and pre-1000 BC iron-manufacturing tradition of south India I think it is probable that the \textit{Wootz} tradition is rooted in the megalithic stage itself. However, this has to be tested in the ground, and as far as one can judge, there is no systematic interest in investigating the history and distribution of \textit{Wootz}-manufacturing sites in south India, although Sharada Srinivasan and her associates seem to have made a beginning in this in recent years. In this context one may also mention.

In my 1992 monograph again (pp. 146-147) I cited Cecil Von Schwartz’s record of steel-making by what he called the Deccanese process, whereby steel was obtained by directly reducing iron ores in crucibles. Schwartz called this “crucible cast steel”, and although in recent years there has been some interest in  the pre-industrial tradition of south Indian steel, one is not sure if Schwartz’s ‘Deccanese process’ has been archaeologically identified. In 1989  in an unpublished report Thelma Lowe reported in the Karimnagar district of northern Andhra 74 iron-producing sites, and on 13 of them \textit{Wootz} was apparently manufactured by the Deccanese process. If we are proud of our \textit{Wootz}-manufacturing tradition, we should undertake extensive field-investigations and recover as much of this history as possible. It is doubtful if the relevant manufacturing debris in the Andhra, Karnataka and Tamil Nadu countryside will still remain undisturbed within a few years.

\vspace{-.3cm}

\begin{center}{\textbf{(IV)}}\end{center}

\vspace{-.3cm}

Professor Tripathi has undertaken, among other things, an admirable review of archaeological and literary data in this volume. The literary data are no doubt very limited and do not throw much light on the technical details of iron production. This is something we cannot do anything about, but what can be done is to prepare a compendium of various terms used by the blacksmiths and indigenous iron-producers in different parts of the country in the context of their actual manufacturing process and trade. The literature of the higher strata of Indian society had no place for such mundane details, but this does not mean that these details did not exist. They existed among the ordinary craftsmen, and it is time we gave some attention to the understanding of these terms in various regions.

Archaeologically I find the present scenario very exciting. The fact that  there were iron-manufacturing centres as early as 1800-1600 BC at  places like Malhar in the fringe of the Ganga valley means that such centres came up only to supply iron implements and lumps of bloomery iron to the  settlements of the valley. Apparently there was an extensive trade network  in iron in the central Ganga plain as early as the first half of the second millennium BC. One hopes that the details of this network will soon emerge with more archaeological discoveries.

\vspace{-.3cm}

\begin{center}{\textbf{(V)}}\end{center}

\vspace{-.3cm}

Mining and metallurgy are two of the areas of ancient India, of which we can justifiably be proud. This is something which was explicitly stated by Panchanan Neogi, a teacher of Chemistry who wrote the earliest syntheses of the history of both copper and iron in ancient India in the first decade of the 20$^{\rm th}$ century. Except  some stray  C-14 dates from old mine shafts and some detailed investigations of such shafts in the Zawar mining area of Rajasthan, the history of pre-industrial mining in India still remains to be worked out. Initiative in this regard should come not merely from the historians and archaeologists but also from the people employed in the Public Sector organizations which control modern mining in these areas. There are thousands of old mine shafts in their jurisdictions, and if they do not want to know anything about their history, one can blame only the educational system of which they are products – an educational system which inculcates no professional curiosity outside the immediate needs and concerns. There is no national level laboratory to analyse ancient metallurgical objects either.

Meanwhile, we are deeply indebted to Professor Tripathi for what she has given us – an excellent analysis of what can be said about the history of iron in ancient India in the beginning of the 21$^{\rm st}$ century. Nobody has been more consistently interested in this history than Professor Tripathi.

Dilip K Chakrabarti\\ Professor of South Asian Archaeology,\\ Cambridge University
