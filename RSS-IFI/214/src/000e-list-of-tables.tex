\chapter*{List of Tables}\label{table}

Table II.1:	Iron objects in antiquity at diverse stages of development in different countries 

Table III.1A:	Some recent $^{14}$C dates of Early Iron Age

Table III.1:	Radiocarbon dates of Jhusi

Table III.2:	Radiocarbon dates of Raja Nal-Ka-Tila

Table III.3:	Radiocarbon dates of Malhar

Table III.4:	Dates of Black-and-Red Ware sites 

Table III.5:	Dates of PGW sites 

Table III.6:	Radiocarbon and TL dates of NBPW and Early Iron Age sites\\ (India and Pakistan)

Table III.7:	Dates of south Indian Megaliths 

Table III.8:	$^{14}$C dates for early iron bearing sites from the Ganga plains and the eastern Vindhyas

Table III.9:  $^{14}$C Dates from Raipura, District Sonbhadra, U.P.

Table III.10:  $^{14}$C Dates from Raipura, District Sonbhadra, U.P.

Table IV.1:	Distribution of iron objects from archaeological sites, Stage-I (Early Iron Age)

Table IV.2:	Distribution of iron objects from archaeological sites, Stage-II (Middle Iron Age) (NBPW level)

Table IV.3:	Typology of iron objects discovered during Stage-III

Table IV.4:	Iron weapons of war and hunting from Megaliths

Table V.1: Classification of iron as in RRS

Table V.2: Iron cannon

Table VI.1: Number of furnaces functional in parts of central India

Table VII. 1: Results of analysis of Taxila iron (After Marshall 1951)

Table VII. 2: Analysis of ancient and Pre-industrial iron produced in traditional furnaces.
