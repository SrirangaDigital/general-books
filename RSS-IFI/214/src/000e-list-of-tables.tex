\chapter*{\centering List of Tables}\label{table}


\vspace{-1cm}

{\fontsize{7}{10}\selectfont\begin{longtable}{ll}
Table II.1:	&Iron objects in antiquity at diverse stages of development\\[2pt]
            
            &  in different countries \\[2pt]
            
Table III.1A:&	Some recent $^{14}$C dates of Early Iron Age\\[2pt]

Table III.1:&	Radiocarbon dates of Jhusi\\[2pt]

Table III.2:&	Radiocarbon dates of Raja Nal-Ka-Tila\\[2pt]

Table III.3:&	Radiocarbon dates of Malhar\\[2pt]

Table III.4:&	Dates of Black-and-Red Ware sites \\[2pt]

Table III.5:&	Dates of PGW sites \\[2pt]

Table III.6:&	Radiocarbon and TL dates of NBPW and Early Iron Age sites\\
            
            &  (India and Pakistan)\\[2pt]
Table III.7:&	Dates of south Indian Megaliths\\ [2pt]

Table III.8:&	$^{14}$C dates for early iron bearing sites from the Ganga plains\\[2pt]
            
            & and the eastern Vindhyas\\[2pt]

Table III.9:&  $^{14}$C Dates from Raipura, District Sonbhadra, U.P.\\[2pt]

Table III.10: & $^{14}$C Dates from Raipura, District Sonbhadra, U.P.\\[2pt]

Table IV.1:&	Distribution of iron objects from archaeological sites,\\[2pt]
           & Stage-I (Early Iron Age)\\[2pt]

Table IV.2:	& Distribution of iron objects from archaeological sites,\\[2pt]
            & Stage-II (Middle Iron Age) (NBPW level)\\[2pt]

Table IV.3:&	Typology of iron objects discovered during Stage-III\\[2pt]

Table IV.4:	& Iron weapons of war and hunting from Megaliths\\[2pt]

Table V.1: & Classification of iron as in RRS\\[2pt]

Table V.2:& Iron cannon\\[2pt]

Table VI.1:& Number of furnaces functional in parts of central India\\[2pt]

Table VII. 1:& Results of analysis of Taxila iron (After Marshall 1951)\\[2pt]

Table VII. 2: & Analysis of ancient and Pre-industrial iron produced\\[2pt]
              & in traditional furnaces.\\
\end{longtable}}
