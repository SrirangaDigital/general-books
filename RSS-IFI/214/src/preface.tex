\chapter*{Preface}\label{preface1}

\vspace{-1cm}

India's contributions to development of science and technology have caught the attention of scholars in recent decades. The developments that took place in India in various fields of science, mathematics, astrology, metallurgy etc. are yet to be fully studied and documented. The Infinity Foundation, USA, has made a serious effort in this direction. Rising above the ideological divide, it has come forward to present to the world a critical and comprehensive history of Indian science and technology. The Foundation aims to bring out volumes focusing on different dimensions of science and technology. The subjects are being taken up afresh with the help of scholars under the umbrella of the Foundation. We agreed that the subject has three distinct dimensions: technology, archaeology and certain specific variety of iron produced in ancient India. It is felt that there should be separate volumes in each of these subjects. Prof. R. Balasubramanyan will deal primarily with metallurgy of iron. The present volume restricts itself to archaeological and historical issues related to iron technology.

I was invited to write a history of iron technology in India. Since I had just produced a book on a similar subject, I was somewhat diffident to start writing on almost the same theme once again immediately after. However, we agreed to enlarge the scope of the book and work on it. The present work extends its scope several times – from the origin of iron in India in the antiquity to decline of indigenous iron working in the pre-modern times. Despite the risk of some repetition that is inevitable, the present volume makes an effort to cover a large area of Indian history - from ancient to modern period – a timeframe spanning over several millennia. To compress the evidence spread so extensively both in time and space in the form of a book within such a short time space is indeed a difficult task. 

\newpage

In the course of the study, I felt that although India has a very long and efficient tradition of iron working, its history is little known to the world today. It is to such an extent that the famous steel, which India produced right from the closing centuries before the Christian era and subsequently exported to the world, came to be known after those ports through which it found way to world markets. For instance, the wootz steel, famous as Damascus steel was actually a product of ingenuity of the Indian metal smiths. However, the Arabs with whom India developed a close relationship – commercial as well as socio-cultural did a great service to the advancement of knowledge. They acquired and spread several branches of Indian science such as astronomy, mathematics, including the Indian numerals, navigation techniques to other parts of world wherever they went. This included information about the much-appreciated Indian steel.

As early as the 4$^{\rm th}$ century BCE the steel industry was so developed in India that steel was presented to Alexander as a tribute. Even earlier, in the 5$^{\rm th}$ century BCE, Ktesias received swords of Indian origin in the Royal court of Persia. Thus Indian steel finds mention in early records, as it must have been a real prized commodity in that age. To arrive at such a level of expertise, it must have got hundreds of years of experimentation to back it up. But these facts are hardly known to those interested in history of science and technology of the ancient world. It is our duty as students of archaeology and ancient technology to let it be known to the world of scholars.

There are misconceptions right from the issue of origin of iron in India. Looking at the emergence of iron, the earlier contention was that iron reached India through diffusion from the west as late as 6$^{\rm th}$ – 5$^{\rm th}$ century BCE. The Bactrians, the Greeks and the immigrating Aryans were held responsible to have brought it to the Indian subcontinent. The very foundations of those theories have now been challenged. Firstly, it is no longer tenable that metallurgically iron being too different from earlier metal techniques could not be developed independently. Secondly, the archaeological evidence coming in recent years by way of very early $^{14}$C dates is indeed more clinching. Importantly enough, those dates come from the heartland of India ruling out all possibilities of diffusion of iron through outside sources. Such early dates are not to be found in the archaeological contexts on any of the bordering lands. Thus one may safely argue for an independent origin of iron in India.

It is noteworthy that despite such an early beginning of iron, there is very little change in the social or economic condition. It has given rise to questions on the effectiveness of iron in bringing about culture change. One of the debated issues of Indian archaeology even today is the impact of iron in the process of urbanization. In dealing with this issue, we have not taken cognizance of the fact that development of metallurgy of iron must have been a very slow process. Its adaptation was still slower in an impoverished chalcolithic milieu in which stone or bone largely sufficed. Judging by the number of metal tools, it appears to be more of a luxury than anything more. It is against such a backdrop that iron technology and its impact ought to be studied. I have tried to incorporate socio-political background at different nodal points of cultural development in this study of iron technology. 

The early society shaped, dictated and canalized the direction of technology. It demanded tools and implements for war, hunting, carpentry, masonry, household jobs and building material. With improvement in metallurgical skills, iron objects could be produced in larger quantity and definitely of better quality by 600-500 BCE. It did influence the economic growth. The cause - effect relationship between technological advancement, economic development, and the consequent urban development should better be avoided here as much has been said on the subject from both angles. One thing is certain that the Indian artisan class never betrayed the society. They were capable of meeting social demands. Even seven-ton colossal structures could be manufactured with great efficiency. The Delhi Iron Pillar of 4$^{\rm th}$ – 5$^{\rm th}$ AD is an example of a well-organized iron industry. This situation continued till much later - right up to the Moghul period.

Babur on his arrival in India expressed surprise at the large number of skilled artisans available here – perhaps something he had not been familiar with. No wonder records confirm that there was no unemployment in India till 18$^{\rm th}$ –19$^{rm th}$ century. Besides, the average per capita income of India was several times higher than Europe.

It is these skilled craftsmen and their expertise that was commissioned by medieval rulers to produce military hardware to wage wars on the indigenous political system. However, the relationship between the Sultans, later Moghul emperors and the artisans was largely exploitative in nature. According to eyewitness records, it was very oppressive in nature. This indeed must have taken its toll on the creative faculties of the artisans. These points have never been considered in the context of examination of technology in the earlier works.

The Europeans, on their arrival in India were impressed by the fine quality steel being produced in India. The Dutch imported shiploads of steel objects. They even established their own manufactories at several places in Deccan. The British and the Swedish showed great enthusiasm towards iron production on a larger scale. They established several factories in different parts of the country. They were given preference over local manufacturers. The latter were, on several occasions, refused permission to run their establishments in favour of the Europeans who were enamoured by the high quality Indian steel and easy accessibility of raw material. The Indian steel that they tested was rated to be much superior to British or Swedish steel.

Perhaps lack of understanding of local temper and work conditions played a negative role as most such ventures had to be shutdown in course of time. The colonial temper of the British promoted British iron industries by importing cheap iron after the advent of blast furnaces there. The indigenous iron industry succumbed to such conditions. Centuries of exploitation and suppression by the oppressive political system, along with import of cheaper iron, which the indigenous ironworkers could not compete with put the last nail in the coffin of indigenous iron production.

Despite the odds, though the indigenous iron industry lost its vigour, it did manage to survive in certain remote parts of ore rich forested areas. Our own research and explorations in such areas have brought forth traces of survival of traditional iron workings. It has been briefly documented here. I personally feel that there is a need to reactivate this, not only for the sake of preservation of cultural heritage of India but also for allowing a whole class of society to lead a decent life with dignity. They are bearers of a legacy of the past. In the modern times of 'small is beautiful' and concern for preservation and conservation of eco-system, effort should be made to revive traditional industries. The traditional craft groups should be given right to forests and forest products for raw material. They have nurtured the forests for generations with a spirit of veneration. It is worth giving it a sympathetic consideration in the modern planning. 


In the end, I hope that the present volume on iron technology in India will be able to provide the reader an idea about the emergence, developmental stages of growth of iron metallurgy, achievements of ancient metal smiths, as well as the story of their struggle – both at personal and professional levels. This volume presents a history of iron technology; the attention and appreciation it got from traders of the ancient times from various parts of the world who flocked to the manufacturing centres to buy it. The colossal structures, the massive cannons and guns adorning important buildings and museums are a witness to the glorious past of Indian iron and steel. 

{\large{\textbf{Preface to the Second Edition}}}


India made valuable contributions to the field of science and technology especially in the fields of chemistry, alchemy, important branches of mathematics, astronomy, metallurgy etc. With increasing researches in these areas of knowledge, India’s contributions in diverse fields of science and technology are coming to light. Several efforts are underway to study and document the Indian knowledge system. The Infinity Foundation has made serious effort in this direction by taking up series of projects on important aspects of science and technology of India. The present volume on ‘History of Iron Technology in India (from Beginning to Pre-Modern Times)’ is one such area of research published by the Infinity Foundation. It attempts to narrate the origin and development of Iron in India through the ages. India has a long history of iron working lasting over nearly four thousand years. Iron technology in India had a very humble beginning in form of small bits and pieced in a Chalcolithic milieu. Iron in India seems to have developed gradually, from wrought iron to steely iron producing the Damascus steel or Wootz steel that became famous throughout the world due to its strength, malleability, ductility and with its unique watering pattern that has drawn attention of the world. For the metallurgists across the world this unique steel is still an enigma. Researches are going on in different parts of the world to understand and reproduce product akin to the crucible steel of India. Indian iron finds mention in several ancient accounts. For instance, in the 5th century BCE, Ktesia, the Greek ambassador visited the Persian court. The king and the queen mother in the court gifted him two swords which were produced and brought from India. Likewise, subsequently, in the 4$^{\rm th}$ century BCE the rulers of North-west India presented the unique Indian steel in form of ingots as attribute to Alexander. There could be several such narratives from the past which demonstrate the saliency of Indian iron and steel from very ancient times. However, our knowledge about the history of iron technology is still far from adequate. We still feel that an authentic work on indigenous iron and steel is very much required. An attempt was made to fill this lacuna by publishing a volume on ‘History of Iron Technology in India (from Beginning to Pre-Modern Times)’ in the year 2008. However, in the intervening period of more than a decade, newer information on this subject has been brought to light; hence it is felt that a new edition on this subject is necessary. The present edition attempts to include the information brought to light in the intervening period since the work was published in 2008. A large number of radiocarbon dates have come forth due to extensive excavations being carried out in different parts of India. With new excavations and more efficient ways of dating available to us, scores of reliable $^{14}$C dates have been brought forth. As a consequence, the antiquity of iron in India has been pushed back by several centuries going back to the opening centuries of the second millennium BCE. These characterizations place the origin of iron in India much earlier than most Iron Age cultures of the ancient world. With better analytical tools of investigations more detailed information is available on metallurgical know- how prevalent at various stages of cultural developments in India. Additionally, recent excavations have exposed several newer dimensions of Iron Age cultures in India. An attempt has been made here to incorporate all the latest information available to us on iron in India.

In course my study, I tried to locate the remains of traditional Indian iron working which could survive till late twentieth century. Traces of traditional Indian iron working could be located by us in certain remote parts of the country, even though at its last stages of survival. Interestingly, the ethnic communities could produce fairly good quality implements in their smithy which is preferred by the village folk over the factory-made tools available in market. They are the last strands of the grand legacy of the world famous iron technology which could produce the colossal structures like the Delhi Iron Pillar in 4$^{\rm th}$-5$^{\rm th}$ CE. Much more serious work needs to be undertaken to support and revive the ancient Indian iron technology which is a legacy of metallurgical supremacy possesses by India once. The present revised work which incorporates the latest researches in the field on iron technology in India is being presented here with the hope that it will reach the larger readership. It may also make the world aware about India’s glorious legacy in the field of metallurgy. 

\noindent
\textbf{Vibha Tripathi\\ March 2022}

\label{endpreface1}
