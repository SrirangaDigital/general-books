\chapter*{Acknowledgments to the First Edition}\label{preface6}

\vspace{-.8cm}

I would like to acknowledge the help that I received from different quarters in preparation of the manuscript. First of all I would like to thank Dr Rajiv Malhotra who not only contacted me to work on this project but later also suggested that I extend the study up to pre-modern times, rather than confining it only to ancient India – a period of study with which I was more comfortable. I took up the challenge rather reluctantly, but learnt so much in the process and discovered several new dimensions of history of indigenous industries of pre-modern India. I am equally thankful to Prof. D.P. Agrawal who entrusted this book-writing project to me and in his typical style kept persuading, even pushing me to meet the deadlines. I appreciate his perseverance. But for him I would not have finished this work as early as I did.	I am also thankful to all the friends and colleagues who are working on different projects on science and technology for their valuable suggestions during our brainstorming sessions. I am especially thankful to Sri J.P. Joshi, Sri M.C. Joshi, Dr. R.S. Bisht, Dr. Rima Hooja, Prof. C.K. Raju and others. Those working on different aspects of history of metallurgy like Dr. R. Balasubramaniam, Dr. Sharada Srinivasan, and Dr. Jeewan S. Kharakwal have also helped me with their suggestions and in defining the scope of problems that were to be taken up in our respective areas of studies.

I also thank all the friends and colleagues whose material I have freely used. Special thanks are due to Dr. Rakesh Tewari, Prof. V.N. Mishra, Prof. M.D.N. Sahi and several others. I am also thankful to authorities of the State Archaeology and Museum, Department of Andhra Pradesh (Hyderabad), for giving me access to their rich collection. I am thankful to those who have worked with traditional ironworkers of India such as Prof. N.B. Ballal of IIT, Bombay, Dr. Mahesh Sharma (who has devoted himself to work with Bishun Bharati), Sri Sunil Sahastrabuddhe and others. I am grateful to Prof. Bhanu Prakash, (Retd. from Department of Metallurgy, IT, Banaras Hindu University) for patiently discussing problems that I faced every now and then and also going through the manuscript. In the end I would like to thank my students Dr. Prabhakar Upadhyay and Ashok Kishan Mujmule for their help while writing this book. Without their help it could not have been completed within this short time. Many of the pictures that are illustrated here are from our own field investigations that are still being carried out.

I am indeed very grateful to the Infinity Foundation for initiating a project on the History of Science and Technology in India and giving us this opportunity to be a part of it. When complete, I hope it will fill in the lacunae in our knowledge about the glorious past that India had in the field of Science and Technology.

Last but not the least, I would like to admit that the work could not have been completed without the loving support of Ajit who inspires me to contribute to the field of knowledge to my utmost capability. Equally supporting have been Amit, Avijit, Nidhi and Savita. I affectionately acknowledge their support in this venture. 

\begin{flushright}
 \textbf{Vibha Tripathi}
\end{flushright}

\label{endpreface6}
