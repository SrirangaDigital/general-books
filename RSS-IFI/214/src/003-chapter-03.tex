\chapter{Origin and Dispersal of Iron in India}\label{chapter3}


The period around the closing centuries of the $2^{\rm nd}$ millennium BCE witnessed a changing configurations due to the emergence and adaptation of newer technologies. The technologies provided incentive and the infrastructure for material prosperity and over all growth. The innovative changes set the pace for a relatively vigorous socio-economic and techno-cultural change. One of the important technological innovations that took place during the period spanning over the $2^{\rm nd}$-$1^{\rm st}$ millennium BCE was the advent and development of iron metallurgy in the Indian subcontinent. In the large and diverse eco-zones of Indian subcontinent, iron technology made appearance in divergent contexts and in different ways. The exact time and circumstances of the introduction and adaptation of iron in has been investigated at different levels. In regions closer to the raw material experiments could have started easily if need for metal was felt. In other areas, the circumstances differed. Therefore, a different approach to understanding of advent and adaptation of emerging technologies has to be adopted. It is desirable to address the issue of beginning of iron in India in different eco-zones which were not easily accessible due to geographical barriers. We will attempt to do so in course of discussions that ensue. Each zone may present us with divergent chronological framework and adaptation pattern. New evidences are being brought forth by excavations being carried out every year. Radiometric characterizations from excavated sites are pushing back the antiquity of iron. In view of fresh light being thrown by issue of iron technology, the related issues have to be addressed afresh while taking cognizance of the earlier viewpoints expressed by historians/ archaeo-metallurgists from time to time. Broadly speaking, there are two main viewpoints on origin of iron in the Indian subcontinent, that is, 1- diffusion of iron from outside; 2- indigenous origin of iron. We propose to examine both the viewpoints in some detail here.  First we take a look at the diffusionistic theory of advent of iron in India. Three basic points were raised in this regard:

 (1) Metallurgists strongly believed that iron technology is too complex to be learnt independently\footnote{} (Forbes 1950). It had to be learnt and perfected under the guidance of artisans who were well adept in the process of iron working.
 
 (2) It was strongly believed that Aryans had entered India with horses and superior weapons (iron?) overpowered the indigenous inhabitants and occupied the land of seven rivers \textit{(saptasaindhav desh)}. There is reference to \textit{ayas} in the Rigveda, the earliest Aryan treatise. 
 
 (3) Iron was introduced by the Greeks or Bactrians as they were present in north-west India. This view was proposed on discovery of iron from sites like Taxila in the early parts of the $20^{\rm th}$ century the British archaeologists who were at the helm of affairs at the time. Since the Bactrians and Greeks had ruled the North-western part of the subcontinent during $5^{\rm th}$-4th centuries BCE and iron was found in those strata, it reinforced the assumption that iron was brought in by them (Gordon 1958: 50-78). Similarly, Sir Mortimer Wheeler (1958) recovered iron artifacts from excavations of megalithic sites like Brahmagiri, Maski etc. Interestingly, as per the norm and the colonial temper, both the archaeologists had ruled out the possibility of use of iron in the subcontinent prior to 600-500 BCE. These views have to be examined closely in light of recent researches. 
 
The other set of scholars worked on literary evidences on iron. Scholars like Neogi (1914) and M.N. Banerji (1929) who had gone into the rich literary accounts of India chose to closely examine the occurrence of ayas which meant iron in Sanskrit. Reference to \textit{ayas} in Early texts, including the Rigveda led to assumption that antiquity of iron coincides with early Vedic period. It has been argued that the Rigvedic Aryans were well conversant with iron \textit{(ayas)} technology wherein one comes across tools and implements made with \textit{ayas}. The question is what did Vedic \textit{ayas} stand for? The term ayas , it seems had different connotations. More in-depth researches show that the term \textit{ayas} is a generic term used for metal. As we will see below, intense debate followed on the etymology of word ayas and whether it stood for iron in the Early Vedic texts.


Lallanji Gopal (1961) synthesized the existing literary data and came to the conclusion that the word \textit{ayas} in the \textit{Rigveda}, stood for metal in general, not specifically for iron. According to him iron was introduced in India during the Later Vedic times. This interpretation of the word \textit{ayas} at the earliest stage is significant indeed to the issue of introduction of iron in India. A detailed analysis of the prevalent views on the advent of iron in India is called for here. In recent years (as discussed in chapter II), scholars have taken a second look at the diffusionistic viewpoint of introduction of iron due to the fact that iron was more probably a by-product of copper or lead metallurgy. However, we would first take up the theories of diffusion of iron technology followed by the alternative viewpoints. 

The theory of diffusion took roots because it was believed by a set of scholars, especially Indologists that the Aryans migrated from a common homeland. It was argued that the Aryans dispersed and settled in different parts of the world in course of their move; this included the Indian subcontinent. This has given rise to theory of diffusion of people and / or ideas. This includes advent of iron technology in India. It is another matter, however that the idea of Rigvedic Aryans being outsiders has now been contested by scholars on the basis of researches being carried out in the field of genetics, especially DNA of population from the concerned regions. However, following the earlier theories propagated regarding the Aryan immigration, we would first take a look at the diffusionistic viewpoint of origin of iron in India.	

\section{The Theory of Diffusion of Technology}\label{section-1}

The philological evidence suggesting similarities between Indo-European languages and Sanskrit, the common features noticeable in the Iranian sacred text the \textit{Avesta} and the \textit{Rigveda}, the oldest Ayran text and the inscriptional evidence like the Boghaz Keui (dated 1365 BCE, a treaty between Hittite king Shubbiluliuma and the Mitanni King Mattiuaza, referring to four Vedic gods) have given rise to a theory of common homeland of authors of these cultures. The linguistic affinity had gained cultural attributes in due course of time. In the specific context of introduction of iron technology in India, the theory of Aryan immigration from the west, though debated has come in handy. It is believed that iron, which was known to the people of Asia Minor in the 2nd millennium BCE, entered the Indian sub-continent with the immigrating Aryans. As examined in detail above (also see Tripathi 2001: 57-78), there are indisputable literary references to the knowledge and use of iron in the pre-1200 BCE period in Mitanni, Hittite and Egyptian records. The king of Mitanni, Tushratta sent the Egyptian Pharaoh, a dagger with iron blade with lapis lazuli studded gold handle (using 14 ‘shekels’ of gold) (quoted by Maddin 1982: 16).

Another reference belonging to the early $13^{\rm th}$ century BCE narrates the contents of a letter of a Hittite king Hattusilius III to the king of Assyria, mentioned above, “As for the good iron about which you wrote to me, there is no good iron in my store house in Kizzuwatna. The iron (ore?) is (of) too low (a grade) for smelting. I have given orders and they are (now) smelting good iron (ores?). But up till now they have not finished, I shall send (it) to you. Meanwhile I am sending to you a blade of iron for a dagger.”(Maddin op cit, 16-17). These passages are sufficient to prove that iron was a precious, prized and naturally also a scarce commodity at this stage. The metal however was being smelted in small amounts sporadically in some parts of the ancient world. The Hittites kept the knowledge of iron working a closely guarded secret, confining the art to the Anatolian Plateau till about 1200 BCE. Their monopoly was broken by disruption of the Hittite Empire by the Thraco-Phrygian invaders who forced the former to migrate to the peripheries of the Assyrian Empire. Around this very time the use of iron objects multiplies and the world witnessed large-scale migrations by warrior tribes with superior weapons, horses and horse-drawn chariots.

In close succession to this incident (c. 1000 BCE), Ghirshman (1954 : 73) observed two perceptible phenomena in the Iranian plateau: The invasion of the Indo-Europeans and the increased use of iron in Iran. In this region, iron was first noticed in the necropolis of Sialk, Cemetery-A along with a new grey ceramic. Iron, however was restricted to just a few pieces as part of royal costume being exclusively ornamental in nature. It assumed a utilitarian role only in the succeeding phase in Cemetery-B. The number of objects increased considerably from a couple of objects in previous period to a noticeably large number as well as a diversified typology reflected in a variety of objects like swords, daggers, shields, javelins, arrowheads, horse bits, head and chest ornaments of horses along with utensils and ornaments like anklets. This has been interpreted as an incursion of new cultural elements in Iran introduced from farther west or north.

In western Iran, Young (1976) defined a three-fold stratification of emergence of iron, calling them Iron I, Iron II and Iron III. Iron makes its earliest appearance by way of stray occurrences primarily as bi-metallic objects in the pre-1000 BCE period, classified as Iron I. Its presence is restricted to its use by a selected few of the society. A grey pottery is associated with this cultural phase impinging on the local Bronze Age cultures. Young held that there were widespread migrations in the western part of the Iranian Plateau in this period. There were new pottery traditions and also change in the pattern of use of metals. Iron was introduced here at this stage by itinerant metal smiths or through some kind of trade. Young cites the parallel of a known situation of influx of gypsies in certain parts of the world.

Interestingly, there is a gap between Iron I and Iron II according to Young. Iron I has been dated between \textit{c}. 1300/1250-1100 BCE. Iron II, according to Young is datable roughly to about 1000-800 BCE. Geographically, India is close to the Iranian borders. It is likely, that there were intermittent folk movements into India through this region. The inherent similarity between Avesta and Rigveda corroborates familiarity  between the two neighbouring people. This has given rise to the assumption that iron was brought in India by these incoming tribes (the Aryans?) through Iran. (It is noteworthy that there is a near absence of iron in Iron I level dated between 1300-1100; it means that they were not the people who could introduce iron in India). Recently, D.P. Agrawal (1998, 2002) has underlined the – linguistic as well as geographical and migratory pattern of movement of people into the Central Himalayas. Agrawal has related this to the beginning of iron in this part. The presence of cist burials with use of iron in Kumaon region is significant in this respect. The presence of the Caucasoid racial elements in the population of Central Himalayan region is said to be a noteworthy feature in this regard. 

Similar evidence of burials of skeletal remains with Caucasoid features has also been brought to light in China recently. The richness of iron ore and presence of iron working tribes from time immemorial give credence to the hypothesis of diffusion of people in this region. Iron has been shown to be present in about 1000-800 BCE in Kumaon – Garhwal region. There exists a possibility of migrations and technology transfer through the immigrating folk at least in these Himalayan zones (Agrawal and Tripathi 1995, Agrawal and Kharakwal 1998). Future research may throw more light on it. With this, we may go back to the issue of iron in \textit{Rigveda}. To evaluate the hypothesis of introduction of iron in India with incoming Aryans through the Iranian plateau, we need to examine in depth- a) the evidence of iron in \textit{Rigveda}; b) the cultural material including iron objects in Indo-Iranian borders for similarities that followed to substantiate the said intrusions by the Aryans in the region.

\subsection*{1. I. Iron in \textit{Rigveda}}\label{subsection-1}

%%%%%%%%%%%%8
%~ \theendnotes 

