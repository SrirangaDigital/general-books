\chapter*{List of Figures}\label{figures}

Fig.1. Map showing distribution of P.G.W., B.R.W. and N.B.P.W.\\ cultures

Fig.2 Different Zones of Iron Age Cultures of India 

Fig.3 Distribution of pre-industrial iron working sites

Fig.4 Map showing Iron Age sites in India

Fig.5 Iron smelting furnaces (reconstructed ancient furnaces)

Fig.6 cluster of furnaces,  Latifshah, Period I.

Fig.7 Ancient Furnaces at Khairadih  

Fig.8 Ancient Furnaces at Lohsan, Malhar Pd. II.

Fig.9 Ancient Furnaces at Lohsan, Malhar Pd. II. ( Courtesy R. Tewari)

Fig.10 Stone Implements used by Ancient Metalworkers

Fig.11 Iron smelting furnace, Raipura, Pre-NBPW, Period II

Fig.12 Iron forge, Raipura, stone slabs for weighing down bellows, NBPW, Period III 

Fig.13 Iron ingot from Raipura, Period III, NBPW culture 

Fig.14 A. Surgical knife, Latif Shah, NBPW Period

Fig.15 Iron  objects, period I, Latif Shah, District Chandauli 

Fig.16 sickle, Latif Shah, Period I, BSW, Pre-NBPW 

Fig.17 A. and 17 B, two views of Shaft-Hole Iron Axe, Latif Shah, (Chandauli) Period II, NBPW

Fig.18 A.  Asur Birija Furnaces at Bishunpur, Dist. Gumla, (Jharkhand). 

Fig.18 B. Bloom being removed from .the furnaces by Asur Birija  smelters at Bishunpur, Dist. Gumla (Jharkhand)
