\chapter*{\centering List of Figures}\label{figures}


\vspace{-1cm}

{\fontsize{9}{11}\selectfont\begin{longtable}{@{}p{1.3cm}@{}p{8cm}@{}}
Fig.1. & Map showing distribution of P.G.W., B.R.W. and N.B.P.W. cultures\\[2pt]

Fig.2 & Different Zones of Iron Age Cultures of India \\[2pt]

Fig.3  & Distribution of pre-industrial iron working sites\\[2pt]

Fig.4  & Map showing Iron Age sites in India\\[2pt]

Fig.5  & Iron smelting furnaces (reconstructed ancient furnaces)\\[2pt]

Fig.6  & cluster of furnaces,  Latifshah, Period I.\\[2pt]

Fig.7  & Ancient Furnaces at Khairadih \\[2pt] 

Fig.8  & Ancient Furnaces at Lohsan, Malhar Pd.~II.\\[2pt]

Fig.9  & Ancient Furnaces at Lohsan, Malhar Pd.~II.\\[2pt]
       &( Courtesy R. Tewari)\\[2pt]

Fig.10  & Stone Implements used by Ancient Metalworkers\\[2pt]

Fig.11  & Iron smelting furnace, Raipura, Pre-NBPW, Period II\\[2pt]

Fig.12  & Iron forge, Raipura, stone slabs for weighing down bellows,\\[2pt]
        &  NBPW, Period III \\[2pt]
Fig.13  & Iron ingot from Raipura, Period III, NBPW culture \\[2pt]

Fig.14 A.  & Surgical knife, Latif Shah, NBPW Period\\[2pt]

Fig.15  & Iron  objects, period I, Latif Shah, District Chandauli \\[2pt]

Fig.16  & sickle, Latif Shah, Period I, BSW, Pre-NBPW \\[2pt]

Fig.17 A. &  and 17 B, two views of Shaft-Hole Iron Axe, Latif Shah,\\[2pt]
          &  (Chandauli) Period II, NBPW\\[2pt]
Fig.18 A.  &  Asur Birija Furnaces at Bishunpur, Dist. Gumla, (Jharkhand). \\[2pt]

Fig.18 B. &  Bloom being removed from .the furnaces by Asur Birija \\ [2pt]
          &  smelters at Bishunpur, Dist. Gumla (Jharkhand)
\end{longtable}}
