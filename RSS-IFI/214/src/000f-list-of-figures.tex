\chapter*{\centering List of Figures}\label{figures}


\vspace{-1cm}

%~ {\fontsize{7}{9}\selectfont\begin{longtable}{@{}p{1cm}@{}p{8cm}@{}}
%~ Fig. 1.& Map showing distribution of P.G.W., B.R.W. and N.B.P.W. cultures\\[.3cm]
%~ Fig. 2.& Different Zones of Iron Age Cultures of India \\[.3cm]
%~ Fig. 3.& Distribution of pre-industrial iron working sites\\[.3cm]
%~ Fig. 4.& Map showing Iron Age sites in India\\[.3cm]
%~ Fig. 5 A, & Iron smelting furnaces (reconstructed ancient furnaces)\\
%~ B and C. & \\[.3cm]
%~ Fig. 6A. & Cluster of furnaces, Latif Shah, Period I. \\[.3cm]
%~ Fig. 6B.&  Furnace, querns, peslte and a chunk of ore lying nearby in a iron workshop, Latif Shah Chandauli Period I, BSW\\[.3cm]
%~ Fig. 7. A.& Surgical knife, Latif Shah, NBPW Period\\[.3cm]
%~ Fig. 7.B. &Analysis of the surgical knife, Latif Shah, NBPW Period, ISIS, RAL, Oxford. tip and handle, respectively\\[.3cm]
%~ Fig. 8. & Iron smelting furnace, Raipura, Pre-NBPW, Period II\\[.3cm]
%~ Fig. 9. & Iron forge, Raipura, stone slabs for weighing down bellows, NBPW, Period III \\[.3cm]
%~ Fig. 10 .& Iron ingot from Raipura, Period III, NBPW culture\\[.3cm]
%~ Fig. 11.& Sickle, Latif Shah, Period I, BSW, Pre-NBPW\\[.3cm]
%~ Fig. 12.& Iron  objects, period I, Latif Shah, District Chandauli \\[.3cm]
%~ Fig. 13 &  Shaft-Hole Iron Axe, Latif Shah, (Chandauli) Period II, NBPW\\
%~ A and B. &\\[.2cm]
%~ Fig. 14.& Ancient Furnaces at Lohsan, Malhar Pd. II.\\[.2cm]
%~ Fig. 15.& Ancient Furnaces at Lohsan, Malhar Pd. II.\\[.2cm]
%~ Fig. 16.  & Furnaces and stone Implements used by metalworkers,Khairadih, Pd. III\\
%~ A, B and C.   &  \\
%~ Fig. 17A.& Internal lamination structure of hoe, Dhatwa \\[.3cm]
%~ Fig. 17B.& Iron  hoe, Dhatwa \\[.3cm]
%~ Fig. 17C. Iron object and D.& Microstructure, Tadkanhalli \\[.3cm]
%~ Fig. 18 A, B and C.& Photomicrograph of Iron implements, Hatigra \\[.3cm]
%~ Fig. 19 A., B. C.& Iron slag, Mangalkot \\[.3cm]
%~ Fig. 20.& SEM of sickle, Anai Period I (Pre-NBPW), Varanasi 500X.\\[.3cm]
%~ Fig 21.& Electron micrograph of a sickle, 1000X, 15KV revealing tempered martensitic structure, Pandurajardhibi \\[.3cm]
%~ Fig. 22 A, B.& Photomicrograph of the same sickle, Pandurajar Dhibi \\[.3cm]
%~ Fig. 23 A, B and C.&Metallograph from Prakash, NBPW  \\[.3cm]
%~ Fig. 24 A, B and C.& Metallographic sections, Rajghat, NBPW\\[.3cm]
%~ Fig. 25 A, B, C and D.& Microstructures of iron samples of NBPW period \\[.3cm]
%~ Fig. 26 A, B, C and D.&  Microstructures of samples of NBPW period \\[.3cm]
%~ Fig. 27 A, B and C.& Microstructure from Sringverpur \\[.3cm]
%~ Fig. 28. A and B.& Delhi Iron Pillar \\[.3cm]
%~ Fig.29 A and B.& Iron beams of  Sun temple  (Konarak)\\[.3cm]
%~ Fig. 30. &Cannon lying at Thanjavur fort  \\[.3cm]
%~ Fig. 31.& Cannon Jahankosha: Murshidabad \\[.3cm]
%~ Fig. 32.& Cannon- Landa Kasab: Bijapur \\[.3cm]
%~ Fig. 33.& Farflier, Bijapur \\[.3cm]
%~ Fig. 34.& Gemda Top, Cannon of Aurangzeb, Bijapur \\[.3cm]
%~ Fig. 35.& Cannon with alligator shaped barrel, Bijapur \\[.3cm]
%~ Fig. 36. &Panj-mani-Adil-Shahi: Ahmed Burj, Bijapur \\[.3cm]
%~ Fig. 37.& Cannon from Malik-Sandal Burj, Bijapur \\[.3cm]
%~ Fig. 38. &Golgumbaz, Bijapur \\[.3cm]
%~ Fig 39.& Cannon at Gulbarga, Karanataka \\[.3cm]
%~ Fig. 40 A.& Furnace at Kaladhungi (Kumaon)\\[.3cm]
%~ Fig. 40 B.& Kumaon iron works. Design of blast-furnace at Dehchuari \\[.3cm]
%~ Fig. 41.& Malabar Furnace (as described by Buchanan)\\[.3cm]
%~ Fig. 42. & The Agaria Belt \\[.3cm]
%~ Fig. 43 A, B and C.& Iron Working by Asur Birija at Bishunpur, District. Gumla (Jharkhand) \\[.3cm]
%~ Fig. 44 A, B, C and D.& Iron working by Asur Birjia tribe, Wadruffnagar \\[.3cm]
%~ Fig. 45 A, B, C and D.& Iron working by Agarias at Balaghat, M.P.\\[.3cm]
%~ Fig. 46 A, B and C.& Remain of iron working at Wadruffnagar \\[.3cm]
%~ Fig. 47 A, B and C.&  Furnace and Forge at Kathua, Pratappur Sarguja, Chattisgarh \\[.3cm]
%~ Fig. 48.& Agaria iron Workers with their furnace, Ranchi\\[.3cm]
%~ Fig. 49.&  Pre-industrial Furnace, Nagpur \\[.3cm]
%~ Fig. 50.& Furance ready for smelting, Pipra, Sidhi\\[.3cm]
%~ Fig. 51.& Furnace in operation after inserting bellows; molten slag being tapped, Pipra, Sidhi \\[.3cm]
%~ Fig. 52, A, B and C.& Remains of traditional Iron Working. \\[.3cm]
%~ Fig. 53. A, B, C, D and E.& Abandoned sites of traditional iron working, Sonbhadra District\\[.3cm]
%~ Fig. 54 A and B.& Survival of iron working- Forging by Agarias, Sidhi \\[.3cm]
%~ Fig. 55 A and B.& Survival of iron working - Forging by Agarias, Sonbhadra \\[.3cm]
%~ Fig. 56.& Coal making kilns, Singhbhum, Jharkhand\\[.3cm]
%~ Fig. 57.&  Coal making kilns, Singhbhum, Jharkhand.\\[.3cm]
%~ Fig. 58.& Charcoal making kiln, Orissa\\[.3cm]
%~ Fig. 59.& Traces of charcoal making kiln, Sonbhadra (U.P.)\\[.3cm]
%~ \end{longtable}}


\begin{myquote}
Fig. 1.  Map showing distribution of P.G.W., B.R.W. and N.B.P.W. cultures\\[.2cm]
Fig. 2.  Different Zones of Iron Age Cultures of India \\[.2cm]
Fig. 3.  Distribution of pre-industrial iron working sites\\[.2cm]
Fig. 4.  Map showing Iron Age sites in India\\[.2cm]
Fig. 5 A, B and C. Iron smelting furnaces (reconstructed ancient furnace2)\\[.2cm]
Fig. 6A.   Cluster of furnaces, Latif Shah, Period I. \\[.2cm]
Fig. 6B.   Furnace, querns, peslte and a chunk of ore lying nearby in a iron workshop, Latif Shah Chandauli Period I, BSW\\[.2cm]
Fig. 7. A.  Surgical knife, Latif Shah, NBPW Period\\[.2cm]
Fig. 7.B.  Analysis of the surgical knife, Latif Shah, NBPW Period, ISIS, RAL, Oxford. tip and handle, respectively\\[.2cm]
Fig. 8. Iron smelting furnace, Raipura, Pre-NBPW, Period II\\[.2cm]
Fig. 9.  Iron forge, Raipura, stone slabs for weighing down bellows, NBPW, Period III \\[.2cm]
Fig. 10. Iron ingot from Raipura, Period III, NBPW culture\\[.2cm]
Fig. 11. Sickle, Latif Shah, Period I, BSW, Pre-NBPW\\[.2cm]
Fig. 12. Iron  objects, period I, Latif Shah, District Chandauli \\[.2cm]
Fig. 13 A and B. Shaft-Hole Iron Axe, Latif Shah, (Chandauli) Period II, NB2W\\[.2cm]
Fig. 14.  Ancient Furnaces at Lohsan, Malhar Pd. II.\\[.2cm]
Fig. 15.  Ancient Furnaces at Lohsan, Malhar Pd. II.\\[.2cm]
Fig. 16. A, B and C. Furnaces and stone Implements used by metalworkers,Khairadih, Pd. I2I\\[.2cm]
Fig. 17A.  Internal lamination structure of hoe, Dhatwa \\[.2cm]
Fig. 17B.  Iron  hoe, Dhatwa \\[.2cm]
Fig. 17C. Iron object and D.  Microstructure, Tadkanhalli \\[.2cm]
Fig. 18 A, B and C.  Photomicrograph of Iron implements, Hatigra \\[.2cm]
Fig. 19 A., B. C.  Iron slag, Mangalkot \\[.2cm]
Fig. 20.  SEM of sickle, Anai Period I (Pre-NBPW), Varanasi 500X.\\[.2cm]
Fig 21.  Electron micrograph of a sickle, 1000X, 15KV revealing tempered martensitic structure, Pandurajardhibi \\[.2cm]
Fig. 22 A, B.  Photomicrograph of the same sickle, Pandurajar Dhibi \\[.2cm]
Fig. 23 A, B and C. Metallograph from Prakash, NBPW  \\[.2cm]
Fig. 24 A, B and C.  Metallographic sections, Rajghat, NBPW\\[.2cm]
Fig. 25 A, B, C and D.  Microstructures of iron samples of NBPW period \\[.2cm]
Fig. 26 A, B, C and D.   Microstructures of samples of NBPW period \\[.2cm]
Fig. 27 A, B and C.  Microstructure from Sringverpur \\[.2cm]
Fig. 28. A and B.  Delhi Iron Pillar \\[.2cm]
Fig.29 A and B.  Iron beams of  Sun temple  (Konarak)\\[.2cm]
Fig. 30.  Cannon lying at Thanjavur fort  \\[.2cm]
Fig. 31.  Cannon Jahankosha: Murshidabad \\[.2cm]
Fig. 32.  Cannon- Landa Kasab: Bijapur \\[.2cm]
Fig. 33.  Farflier, Bijapur \\[.2cm]
Fig. 34.  Gemda Top, Cannon of Aurangzeb, Bijapur \\[.2cm]
Fig. 35.  Cannon with alligator shaped barrel, Bijapur \\[.2cm]
Fig. 36.  Panj-mani-Adil-Shahi: Ahmed Burj, Bijapur \\[.2cm]
Fig. 37.  Cannon from Malik-Sandal Burj, Bijapur \\[.2cm]
Fig. 38.  Golgumbaz, Bijapur \\[.2cm]
Fig 39.  Cannon at Gulbarga, Karanataka \\[.2cm]
Fig. 40 A.  Furnace at Kaladhungi (Kumaon)\\[.2cm]
Fig. 40 B.  Kumaon iron works. Design of blast-furnace at Dehchuari \\[.2cm]
Fig. 41.  Malabar Furnace (as described by Buchanan)\\[.2cm]
Fig. 42.   The Agaria Belt \\[.2cm]
Fig. 43 A, B and C.  Iron Working by Asur Birija at Bishunpur, District. Gumla (Jharkhand) \\[.2cm]
Fig. 44 A, B, C and D.  Iron working by Asur Birjia tribe, Wadruffnagar \\[.2cm]
Fig. 45 A, B, C and D.  Iron working by Agarias at Balaghat, M.P.\\[.2cm]
Fig. 46 A, B and C.  Remain of iron working at Wadruffnagar \\[.2cm]
Fig. 47 A, B and C.   Furnace and Forge at Kathua, Pratappur Sarguja, Chattisgarh \\[.2cm]
Fig. 48.  Agaria iron Workers with their furnace, Ranchi\\[.2cm]
Fig. 49.   Pre-industrial Furnace, Nagpur \\[.2cm]
Fig. 50.  Furance ready for smelting, Pipra, Sidhi\\[.2cm]
Fig. 51.  Furnace in operation after inserting bellows; molten slag being tapped, Pipra, Sidhi \\[.2cm]
Fig. 52, A, B and C.  Remains of traditional Iron Working. \\[.2cm]
Fig. 53. A, B, C, D and E.  Abandoned sites of traditional iron working, Sonbhadra District\\[.2cm]
Fig. 54 A and B.  Survival of iron working- Forging by Agarias, Sidhi \\[.2cm]
Fig. 55 A and B.  Survival of iron working - Forging by Agarias, Sonbhadra \\[.2cm]
Fig. 56.  Coal making kilns, Singhbhum, Jharkhand\\[.2cm]
Fig. 57.   Coal making kilns, Singhbhum, Jharkhand.\\[.2cm]
Fig. 58.  Charcoal making kiln, Orissa\\[.2cm]
Fig. 59.  Traces of charcoal making kiln, Sonbhadra (U.P.)\\[.2cm]
\end{myquote}

