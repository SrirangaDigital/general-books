\chapter*{Introduction to the Series}\label{preface2}

\vspace{-.5cm}


INDIA HAS HAD A LONG HISTORY OF CIVILISATION GOING BACK SEVERAL THOUSAND years. In the global mind this civilisation is largely associated with spiritualism, philosophy and religiosity. There is also an appreciation of great traditions that have continued till today: of performing arts in dance and music; of magnificent works in architecture, sculpture, painting; and of artisanal output of quality relating to crafts. However, over the last couple of centuries the image has been significantly one of poverty: surprising when John Milton in Paradise Lost had referred to ‘the wealth of Ormuz and of Ind’. For many, it is a land of magicians and snake-charmers; of wild animals like tigers and elephants; of mendicants, and mystics. This image is changing today with India moving powerfully into the knowledge economy and becoming a power to reckon with. Whats not known and appreciated is the extent to which a whole knowledge system, and particularly scientific thought, has been inextricably linked to the development of this ancient and continuing civilisation.

There are many reasons for this. An important one is that there has been no major work on the history of science and technology in India. This is not to say that no work has been done in this field; there is a great deal available relating to specific areas, and involving specialised discussion, more so concerning technology. But there is nothing covering science and technology as a whole, dealing with the many diverse areas that it covered, and particularly, its very different conceptual foundations, and also the extent to which it has underpinned the development of Indian society and civilisation. Because of this, in authoritative encyclopaedias (such as Encyclopaedia Britannica), Indian science and technology is dealt within a few hundred words. This must be contrasted with the situation in respect of China, following the great work of Dr Joseph Needham, who produced thirty odd volumes on the history of science  and technology of that country. This made a monumental impact in the academic repositioning of China as a scientific, rational and progressive civilisation. There is need for a similar effort with regard to Indian science and technology.

A further reason for the rather poor appreciation of the history of Indian science is that in the public eye, the development of science and technology is manifested through its innumerable technological artefacts that affect daily lives in society, and great scientific discoveries, both of which rapidly followed the developments that took place around a few hundred years ago in Europe with the birth of the modern, ongoing Scientific and Industrial Revolutions. As a result, the image of western science is so overwhelming that it swamps all earlier history.

In the West, there is major reference to the Greek origins of western philosophy and scientific thought. There was a strong coupling of Arab and Greek science, particularly as manifested in the historical library of Alexandria. It must be remembered that India was an open country, and had significant interactions with the Arab world. Also, western scholars of the pre-Christian era had information on India through reports of Greek origin. It is thus, that the ‘Zero’ and the ‘Decimal place value system’ became international and got referred to as the Arabic numerals; but these had their origins in India. Indeed ‘Zero’ has a deep philosophical meaning, that was characteristically Indian, denoting ‘nothing or emptiness’.

There are many who, without a deep appreciation of the character of indigenously developed knowledge structures in India, often propagate the idea that science (which was so intimately interwoven with these knowledge structures) came to India from the outside through Islamic and Western conquests. The fact is that India absorbed all that came in from the outside (whether by invasion or otherwise) and gave freely of itself to the East and to the West.

The following reasoning may provide an indication why it has been difficult to compile a meaningful and reasonably comprehensive history of science and technology in India.

\newpage

A great deal of knowledge transmittal in India has been through the oral tradition, having been conveyed through chants, hymns, poems and the like; this has been true even in mathematics. These do not remain in the written record, and much of it has probably been lost with the cataclysmic changes in Indian society.
What has been written has invariably been on palm leaf and such other materials, followed by paper recordings; this is particularly the case for the earlier period in history prior to the advent of printing. A great deal of the source material on which writing has taken place, has been subject to ravages of weather and insects, since India is located in the tropics characterised by high temperatures and humidity. After printed books came into existence, which ensured the availability of many copies of a text, India has not had a great history in science and technology, except in the last century.
Even when written, this was in the language then prevailing, namely Sanskrit, particularly of the archaic variety, and other related languages; there was also the problem of scripts. In recent years, one finds that those who know these scripts and languages, particularly their older forms, have little knowledge or interest in matters that are scientific. Their interests largely lie in areas of literature, philosophy, religion and the like. In contrast, those who know science, and could contribute to a meaningful analysis of recorded history, have scant knowledge of the language and the scripts of the past and often little interest.

Finally, what was transmitted, even on matters that were scientific, was mixed up with a great deal of philosophy, religion and mysticism; often practitioners of science of those days were philosophers or those high up in the socio-religious hierarchy. Knowledge in ancient India was much less compartmentalised; it was characterised by a holistic approach. To extract that which would be regarded as science from this totality is not an easy task.
For all these reasons, the overall effort relating to writing of the history of science and technology in India has been a limited and scattered effort.
From what we know already, it is clear that there was a significant development of science and technology, covering a wide range of areas, with high creativity and originality, and over a long time-span. It is important to record and understand this. The purpose would not be to go back to the past from the viewpoint of claiming how great India was. Whilst there were many great discoveries made in India, which predate the same ones later made in the West, and now named after western scientists, the purpose would also not be to claim priorities or demand changes in attributions. What is important is to carry out proper historical work to understand the developments that had taken place and to record them appropriately.

There are two important reasons why this should be done. The first is from the viewpoint of understanding the nature of society in which science can flourish. Ultimately it is a thinking-questioning society that can give rise to science. It is from this angle that Jawaharlal Nehru constantly spoke about the need for society to be imbued with a scientific temper, namely, society functioning on a rational objective basis, which can give rise to the development of science – with innovation, originality, and creativity. These qualities cannot arise in a society which is hierarchical and authoritarian. On this aspect Gautama, the Buddha had remarked:

Believe nothing merely because you have been told it or because it is traditional or because you yourself have imagined it Do not believe what your teacher tells you, Merely out of respect for the teacher But whatever after due examination and analysis You find to be conducive to the good, the benefit, The welfare of all beings,That doctrine believe and cling to, And take it as your guide.

It is important to understand why, with so much creativity and originality that had characterised Indian science and technology of the past, those developments did not last or take off on a self-sustaining basis.

The second reason why one needs to look at the history of science and technology in India is that, to a great extent, the conceptual approach in it has been different from which has characterised western development. It has tried to examine problems on a holistic basis. It has dealt with complexity. Thus the fundamental basis of the Indian system of medicine, referred to as Ayurveda, is very different from that which characterises the allopathic system. Ayurveda is a holistic system, which attempts to ensure harmony among the different components of the human body, and aspects of its functioning, including relationships with the outside world and inputs received. The approach is to ensure that the mind-body system remains in health, rather than in the treatment of disease. This involves a complex many-body synergy (including the treatment with medication, body discipline of yoga, meditation and the like) rather than the deductive, reductionist ‘active principles’ approach of allopathy; the latter has certainly yielded a plethora of miraculous results.

One is also amazed at the dimensions with which the ancients grappled with, and their many speculations on issues such as origins of the universe and cosmology that are still with us. There is a great deal that one can learn from these excursions into their thoughts.

It is, therefore, important that any such history deals not only with technological issues that are related to societal needs but to science as a whole.
There have been a number of efforts to cover the history of science and technology in India. The Indian National Science Academy publishes a journal entitled Indian Journal of History and Science under the guidance of the Indian National Commission for History of Science. One of the other, very ambitious, ongoing projects in this area is that on the ‘History of Science, Philosophy and Culture in Indian Civilisation’ of which the general editor is Prof. D.P. Chattopadhyaya. The series is projected to cover approximately seventy-five volumes, of which twelve have already been published. There are also many individual efforts such as the Sandeepani Science Gallery in Porbunder and innumerable publications that cover specific topics in science and technology such as copper and its alloys in ancient India; works on Indian medicine and medicinal plants; works on logic, where India was particularly strong, and so on.

I am glad that the Infinity Foundation has considered it appropriate to produce a series on the History of Indian Science and Technology (HIST). It is important that this effort covers not only areas of technology but also of science. It will have to bring together for this, scientists and historians of science, along with those concerned with philosophy, anthropology, religion, ancient Indian languages and many such other disciplines, who will have to interact strongly among themselves on the basis of available textual and other material. It has to be a holistic, interdisciplinary approach. This is a major task but well worth doing.

\begin{flushright}
\textbf{Prof M.~G.~K.~Menon},\\
\textbf{FRS Advisor, ISRO (Department of Space), New Delhi}
\end{flushright}



\label{endpreface2}
