\chapter{Introduction}\label{chapter1}

%~ \Authorline{Gayathri Girish\footnote{pp. 21--48. In: Meera, H. R. (Ed.) (2020). \textit{Karnāṭaka Śāstrīya Saṅgīta - Its Past, Present, and Future.} Chennai: Infinity Foundation India.}}

%~ \vspace{-.3cm}

%~ \lhead[\small\thepage\quad Gayathri Girish]{}

%~ \begin{flushright}
%~ \textit{(gayathrigirish@gmail.com)}
%~ \end{flushright}

Among the early technologies, metallurgy holds a special place. It is a success story of human endeavour to control, manipulate and transform materials into relatively more durable and efficacious forms. History of metal technology is replete with examples of new ventures in the field of metallurgy. In India, the mastery in metal casting techniques including the complex \textit{cire´ perdue´} process in the $3^{\rm rd}$ millennium BCE is perceptible in the Harappan dancing girl and the Daimabad bronzes. The heavy tools and implements of the Copper Hoards in the Gangetic Plains in \textit{circa} 2000-1500 BCE or may be somewhat earlier also illustrate metallurgical skill of the in copper utilization. In the subsequent era, a hitherto unparalleled metallurgical expertise was mastered by ancient metal workers in extraction and forging of iron and distillation of metallic zinc. 

Pure zinc could not be extracted in Europe even up to the 19th century, while use of metallic zinc in India can be traced back to the $4^{\rm th}$ -$3^{\rm rd}$ centuries BCE. Zinc production had reached industrial level in India by the $12^{\rm th}$ - $13^{\rm th}$ centuries CE. It was produced through distillation at an industrial scale as indicated by the heaps of zinc retorts found in the Zawar region of Rajasthan. Sizably large scale production of zinc continued for several centuries. Thus India may be given the credit of being the first country to master the complex technique to extract pure metallic zinc almost on an industrial scale by an indigenously developed distillation technique.

The ingenuity and the innovative spirit of metal workers are also evidenced in the techniques of manufacturing of iron and steel at early date. This is fully borne out by the records of foreign travellers and historians who visited in the past from time to time. For instance, Herodotus mentions iron arrowheads being used by the fighting army in the battle of Thermopylae in the 5th century BCE. Around the same time, Ktesias gratefully acknowledged the gift of swords of Indian steel made to him by the king and his mother at the Persian court. Quintus Curtis records that in North-western India, Alexander was presented with 100 talents of highly prized Indian steel in the form of ingots along with gold dust and other precious items in 326 BCE as a tribute. Arrian has also mentioned about import of Indian iron and steel into the Abyssinian ports. These Greek and Roman records thus bear a testimony to the importance of iron and steel produced in India and also to the fact that it was in great demand and prized in the ancient world civilizations because of its special quality. It was being exported to various parts of the world. Indian iron indeed appears to be a valuable commodity in the ancient times. Thus Indian iron was a commodity worth presenting to a monarch or other dignitaries way back in the $5^{\rm th}$ - $4^{\rm th}$ century BCE! 

In the following centuries iron technology grew manifolds, the mastery of the craft exhibits itself in the form of colossal structures like the seven-ton iron pillar at Delhi ($4^{\rm th}$ - $5^{\rm th}$ centuries AD). It has withstood the ravages of time for centuries. The technique developed by early iron workers was such that either it slowed down the rusting significantly or it almost stopped it once the oxide layer was formed. This property of rust resistance made Late Prof. T.R. Ananthraman, an eminent metallurgist of modern India, refer to Delhi Iron Pillar as the ‘Rustless Wonder of the World’. Equally significant in this regard are the large beams at the Sun temple at Konark that date back to the $9^{\rm th}$ -$10^{\rm th}$ century. The iron pillar at Dhar in central India is another iron master piece belonging to the $10^{\rm th}$ -$11^{\rm th}$ century. Besides the quality of rust resistance, these examples testify to the mastery in forging of colossal structures involving production and use of large quantity of iron having uniform properties. This in turn, also testifies to the presence of a well-organised production system capable of turning out tonnes of uniformly similar metallic iron during the early centuries of the Common Era. Forge-welding tons of iron with seamless precision is by no means a small achievement.  Little wonder that the skill and the expertise could easily be exploited subsequently by the state machinery during the medieval times for manufacturing weapons, especially cannons that adorn several important buildings today.

It may be interesting to underline the fact here that the British were surprised to see the high quality of iron and steel being produced in the traditional Indian iron workshops. They tested and rated highly the traditional Indian iron. The British engineers found it to be more suitable and strong for constructing bridges etc. than iron produced in their own production units. An important example in the case is the famous `tubular bridge' built in the early parts of the $19^{\rm th}$ century across the Menai Straits in the United Kingdom. It is categorically stated, ``...its (iron's) superiority is so marked, that at the time when the Britannia tubular bridge across the Menai Straits was under construction preference was given to the use of iron produced in India" (T.H.D. La Touché 1918). It has also been recorded that 50 tonnes of Indian steel have been used in construction of the famous London Bridge. This proves beyond doubt that traditional Indian iron and steel was superior to the pre-modern European steel. Indigenous iron was being imported from India by the British in $19^{\rm th}$ century for using it at strategically important points in iron structures like the London Bridge. The famous wootz steel is modified form of \textit{ukku} of Telugu language. Wootz, a very special kind of high carbon crucible steel, generally known as Damascus steel till recently was originally produced in India at around the beginning of the Common Era, or maybe even earlier as shown by recent excavations of megalithic sites like Kodumanal yielding evidence of steel making with crucibles, furnaces, slag, refractory material as early as 4th- 3rd century BCE (Rajan 2015: 65-79). The metallurgy of this enigmatic crucible steel making is still a subject of research among modern metallurgists. Several efforts have been made to reproduce high carbon steel with typical `watering pattern' without much success. \textit{Wootz} was being exported to the outside world through various important ports of the ancient times. Presence of the `damascened' swords and daggers in so many important collections in different parts of the world is sufficient to prove both its significance as well as the scale of its production and its extensive distribution at a fairly early date. These facts have hardly been researched and brought to the notice in the publications on history of metallurgy (of iron). Despite the glorious past of Indian iron technology, volumes dealing with history of metallurgy do not have much to say about history iron and steel in Indian.

The history of iron technology of Mesopotamia, the Greco-Roman world, Africa, China etc. are fairly well documented and hence better known to students of history of technology. One is amazed at the inadequacy of researches on Indian iron and steel. It is, therefore high time that an authentic history of iron technology in India covering its diverse aspects is produced. Indian contribution to this technology needs to be explored and given its due place in a global perspective of history of science and technology. The present volume is an attempt to present a comprehensive history of Indian iron and steel up to pre-modern times.

Reconstructing a comprehensive history of iron technology in India is indeed a difficult task. In India there is a lack of systematic documentation of different practices. The oral tradition of knowledge transmission and the frequent destructions of manuscripts at educational centres by invaders in the early centuries of the medieval period have led to this vacuum. Centres of knowledge such as those at Taxila, Nalanda, Vikramshila and private collections housed thousands of books. But the libraries of ancient times were frequently vandalized leading to an irreversible and permanent loss of scientific knowledge that existed in India.

There is very little knowledge available on the scientific basis of Indian metallurgy. There could be two reasons for it. Firstly, metallurgy was a practical skill mostly in the hands of a group of craftsman living in remote areas rich in ores and forests for charcoal as a source of energy. These craftsmen had little contact with the elite class of scholars. Secondly, possibly there existed a theoretical basis for the craft but the formulations have got lost in the course of time. It is improbable that high standards of metal technology and mastery - as illustrated by the examples mentioned above - could have been achieved without a proper scientific basis. Stray references to texts on iron metallurgy prove this point beyond doubt. This also dispels the notion that these skills were confined to a class or a caste based reservation and other restrictions associated with it. At least this must have been so at the initial stages. It is a different matter that metallurgy as a much specialised skill came to be associated with an indigenous  group of people and who got identified with it subsequently. It became the vocation of that community which were forest dwellers, may be for practical reasons like proximity to raw material. Later on such social groups were segregated and got identified with their vocations like iron working. They developed a social and cultural system of their own. It is this distant ethnic group that possesses some knowledge about iron working till date. They somehow carry the legacy of iron working. Whatever little remains with them today, need to be preserved and documented lest it gets lost forever.

The present volume on History of Science and Technology in India proposes to examine various dimensions of iron technology in India-right from its beginning to the stage of its development and the zenith it achieved, to its eventual decline after industrialization. It also proposes to look into the causes of its decline. Through a critical evaluation of sources, many of them not tapped so far, it may be possible to bring out several hitherto unknown or half known facets of history of iron technology in India. The book proposes to cover a long period of history spanning over several millennia - from emergence of iron in the second millennium BCE to the present day survival of the tradition. Though efforts have been made to collect as much information as possible up to the British period, at times the treatment of the subject may not be as comprehensive due to unavailability of necessary archival material.

Recent archaeological discoveries have attributed the first emergence of iron to copper-using societies in different parts of the subcontinent. The earlier contention of diffusion of iron has been questioned in light of new discoveries. We are faced with several questions that demand attention. How and under what circumstances metallurgy originated and evolved? Whether technology of iron was obtained through outside contacts or whether it evolved out of the existing knowledge of metal craft? In which part of the subcontinent did iron first appear? How did the metallurgy of iron develop? What are the stages of its development? Why is it that despite several attempts it not been possible for modern metallurgists to unravel the mystery of wootz steel forging techniques? When was the impact of iron felt and why was it so slow to reflect itself in socio-economic milieu? What was the pattern of adaptation of iron technology in the early society? The interface of technology and society is yet to be examined and evaluated in all its dimensions. The causes responsible for the decline of a flourishing iron industry in India have to be looked into. The present study proposes to address such unresolved issues related to early Indian iron technology (See, figures~\ref{chapter1-fig001}, \ref{chapter1-fig002} and \ref{chapter1-fig004} for early Iron Age sites and Fig.~\ref{chapter1-fig003} for pre-industrial sites). 

With new researches in archaeo-metallurgy and radiocarbon dating of recently excavated sites in India, one needs to take a fresh look at the issue of origin of iron in India. The other important issue that needs attention is the role of iron in cultural changes. We need to review and interface productivity, technology advancement, and its pattern of adaptation by society at nodal points of history. The status of metallurgy at various stages of development has to be defined and its adaptation in proper cultural context has to be studied. This has to be attempted at several stages of cultural growth. Iron metallurgy had a prolonged incubation period. Its impact on society was indeed slow. All this should be examined in detail to be able to address the unanswered questions that keep coming up every now and again. 
%~ \theendnotes 

