\chapter{Conclusion}\label{chapter8}


\lhead[\small\thepage\quad chapter VIII]{}


The history of iron in the Indian subcontinent unfolds a saga of human endeavour, a spirit of innovation in the field of metal technology for which India was well known in the ancient world. It is a story of centuries-long perseverance and sustained hard work put in by communities of metal workers spanning nearly over a period of 3000 years. Iron seems to have been discovered indigenously incidentally or accidentally in the process of metal working by the Chalcolithic metal workers somewhere around circa 1800/1700 to 1500 BCE 14C. There are dozens of 14C dates going back to opening centuries of the second or the mid- second millennium BCE showing a high antiquity of iron technology in the Indian subcontinent. Iron at this stage could be by-product of copper or lead smelting as shown by several metallurgists like Wertime (1980: 13, op. cit.), Moorey (1994:279, op. cit.) and several others. Once metallic iron was recognized, it could be extracted, albeit in small quantities to begin with. In the subsequent times with trial and error, regular production of metallic became possible. This also explains a halting production and sparse utilization of iron artifacts at the initial stages of advent of iron. Such a situation seems to have lasted for centuries till the metallurgical processes were perfected and mastered. With perseverance and experience, the desired technological acumen could be attained. The society got familiarized with the new metal and its properties and was thus in a state of readiness to adopt the new technology. The artisans at the subsequent stages of advent of iron were more capable to meet the challenges posed before them. The Age of Iron was heralded at this juncture.  

The survey of iron technology undertaken in the present study is an attempt to bring to light the status of metallurgical developments that took place in India over the centuries in different and divergent eco-zones demarcated by natural barriers which could not be overcome by the primitive cultures. Our review of over three thousand years of Indian history reveals that there have been innumerable ups and downs in the socio-economic and politico-cultural life of people in the subcontinent. Indian history is generally studied under three broad categories of ancient medieval and modern periods. There is little interaction among the experts of these branches. Therefore, there is a dearth of a holistic approach to problems across these specialized divisions within the discipline. Rising above this tendency of compartmentalised examination of events and processes, the present study has attempted to adopt a holistic view of technology – its innovations and adaptation pattern through the ages. 

It has been our approach to situate our enquiry on iron technology against the socio-cultural backdrop and to observe its repercussions on the dynamics of culture change. The study focuses on all possible dimensions related to iron technology – from the earliest recognition of iron as an independent metal in the ancient world civilizations to its utilization pattern at various stages of development at different pockets of cultures in the world; from the issue of origin and dispersal of iron technology to its development as a mature metallurgical science that caught the attention of the ancient world. As a consequence of the prolonged period of experimentation the ancient Indian ironworkers developed metallurgical techniques to produce different types of metallic iron suitable for different requirements. They successfully developed metallurgy to produce high quality steel that the world came to recognise for its special qualities. Iron as early as $5^{\rm th}$-$4^{\rm th}$ century BCE became a prized commodity (to be presented to the monarchs or the emissaries and ambassadors of the states as evident in cases of Alexander the Great or Ktesias in the Persian court). Iron became an item of export through various trading agencies, in the subsequent times. The peak lasted for centuries, till as late as the medieval ages or the on-set of the British rule in India. However, there are phases of decline of production to the virtual end of the indigenous iron industry. The causes of such phenomena have been subject of inquiry of the present study. Our sources have been archaeological, literary, and archival as well as metallurgical researches going on all over the world. Some of the emergent points may be highlighted here to conclude our study.

\vspace{-.3cm}

\subsection*{Genesis and development of iron}\label{chapter8-subsection-1}

\vspace{-.2cm}

Let us begin from the very beginning. The issue of genesis of iron technology has caught the attention of scholars throughout the world. It was believed that metallurgy of iron was considerably different from that of copper – bronze; therefore it had to be learned from those sources which were well adept in the specific technique of iron extraction, forging and smithy. Primarily because of this reason the theory of diffusion gained ground. But now as stated above, researches have shown beyond doubt that iron is a by-product of copper or lead smelting in which the oxides were present due to a variety of factors leading to accidental production or recognition of iron as a separate metal than copper or bronze used as utilitarian metal previously. It has also been argued that the possibility of the earliest production of iron was stronger amongst those societies, which were less adept in metallurgy rather than in the ones which were more advanced and better conversant with metallurgical processes. The chalcolithic communities in many parts of the world appear to be more eligible candidates for such inadvertent discoveries. In India, we come across the earliest iron in such a milieu. Its use starts with stray occurrence of a few bits and pieces to elementary hunting tools of wrought iron of inferior nature by chalcolithic communities. The metallurgy gradually improves over the centuries, leading to common occurrence of good quality iron objects. 

India is a large sub-continent with divergent eco-zones. In several of these secluded pockets, chalcolithic cultures having little inter- regional interactions co-existed in the Indian subcontinent. Copper-bronze using cultures of Middle Ganga Plains, for example, hardly had anything common with the megalithic builders of the Deccan and South India. We have seen (in Chapter II) that iron in the Iranian borders does not make an appearance before 1200 BCE. In the Swat-Gandhara area (of modern Pakistan) in the northwestern region also, iron was first found in graves around 1100-1000BCE. On the other hand, the dates from Gufkral are as early as 1500-1400 BCE (the ${}^{14}C$ date is as early as 1900 BCE), that too not far away from Gandhara region, demonstrate a strong possibility of folk movement or technology dispersal from Kashmir valley towards west or north-west rather than in the other direction. The joint evidence of literature and archaeology has proved it beyond doubt that iron had of an independent origin in India. The chronological evidence at hand also leads to the assumption that iron could have been recognised possibly at more than one centres, which were located in distant, disconnected and remote eco-zones of this large subcontinent. 

The archaeological investigations carried out in different parts of India have made one thing certain that iron emerged in the country much before it was seen in the neighbouring countries situated around it in the immediate geographical proximity. Our examination of the early contexts of occurrence of iron has amply demonstrated that the chalcolithic metal workers of eastern India in Bengal, Bihar or of the Vindhya-Kaimur plateau region in Uttar Pradesh were among the earliest users of iron in India. Same is the case of the megalith builders whether of Kashmir or of the peninsular India. They yielded evidence of iron right from the Neolithic-Megalithic overlap phase.

The local resources easily accessible in the neighbourhood were generally being tapped for raw material by early human societies – be it stone, gems or metals. In the Saryupar plains in the Middle Ganga Plain, for example, even stone was not available within hundred kilometres of these so called chalcolithic settlements. These communities were content with bone tools only. Archaeologists label them as a distinct cultural sub-group because of this characteristic feature– almost negligible use of stone tools, a few copper objects and profusion of bone tools. Beautifully carved bone points, styli, beads, pendants etc. were being fashioned by them. It is in this self sufficient and fairly self-contained socio-cultural context that iron makes its earliest appearance in India in the Vindhya-Ganga region. It is difficult to visualize diffusion of iron from distant lands in such a remote set up. The early radiocarbon dates coming from some of the sites of the region gives credence to this assumption. Additionally, their geographical proximity with ore deposits in the hilly terrains occupied from the Stone Age onwards, made it easy and possible for them to experiment with alternatives, that is, other varieties of minerals accessible to them in their vicinity. This region also has a very long survival of iron working tradition making it crucial to the history of iron in India. Till about a couple of decades back iron was being smelted in these parts of Vindhya-Kaimur ranges as well as in parts of Chhotanagpur plateau. There is still a sizeable population of Asurs and Agrarias – the traditional ironworkers who reside in these ore rich areas. One comes across remains of iron working littered in the forest clearance (see illustrations Figs.~\ref{chapter-7-fig52A}, \ref{chapter-7-fig52B}, \ref{chapter-7-fig52C} - \ref{chapter-7-fig53A}, \ref{chapter-7-fig53B}, \ref{chapter-7-fig53C}, \ref{chapter-7-fig53D}, \ref{chapter-7-fig53E}). Radiocarbon dates ranging between 17/1600 – 1500 BCE from iron-bearing horizon of the sites of Nal-Ka-Tila,  Malhar and more recently Raipura in the Vindhyan range are among the earliest dated in the country. The sites are located near Varanasi on the bank of River Ganga in Uttar Pradesh. Significantly enough, the hilly terrain is rich in variety of minerals including iron ore. In view of very early radiocarbon and the suitable cultural background, it may safely be assumed that iron has had an independent development in this region. Equally interesting is the evidence of the megalithic cultures of south India. There are sites like Hallur, Tadkanhalli and Kumaranhalli in Karnatka that have quite early evidence of iron – 1100/1200 BCE to 1400/1500 BCE and an OSL date around 2000 BCE from Hyderabad University campus. As stated above, an early date of 1800-1700 B.C has also been reported from Kashmir valley from the site of Gufkral. The testimony of dozens of Radiocarbon and TL dates necessitate a revision of the earlier theories of beginning of iron in 1100-1000 BCE.  The chalcolithic cultures in which iron makes its first appearance flourished in divergent and distant eco-zones. Most of them exhibit little evidence of cultural interaction between them as discussed in Chapter III. It stands to reason to assume that each region was evolving independently and locally with limited resources within the limits of their eco-zones. The hypothesis of a multi-centred beginning of iron in India is thus plausible and likely.  

At this point, it may also be interesting to raise the question, what made these cultures experiment with iron? One of the possible answers to it may be that these cultures had some earlier background of use of copper –a mineral that is available only in very limited quantity that too in small pockets in most areas like the Vindhya-Kaimur and most parts of the southern India. No wonder that the artisans in those regions chose to experiment with other metal bearing ores available to them in the vicinity. The recognition of iron must have been an inadvertent incident initially, leading to a long period of trials as indicated by extremely slow production and adoption of iron at the early level of its advent in the copper using cultures. The metallurgy of iron once learnt in the process could be perfected in due course of time over the centuries from simple wrought iron to steely iron; as the evidence of evolution in tool typology from elementary hunting and fishing tools to a complex tool repertoire shows. The engineering skills, if one may be allowed to call it so, attained such perfection that they could manufacture remarkable masterpieces in the succeeding ages. 

Excavations conducted in different parts of the country manifest evolutionary cultural phases and ensuing changes in growth of metallurgy. At its earliest level like the ones with predominant chalcolithic features, as discussed earlier, we come across just a few elementary iron tools of wrought iron. Even the megalithic culture of south India with relatively richer tool repertoire hardly displays a prosperous material base of this community. It is because iron tools at this early age were used mostly for war or hunting. The subject has been dealt with in detail elsewhere (Tripathi 2001).

We have already mentioned about beautiful steel swords gifted to Ktesias, the Greek ambassador. The ancient world started taking notice of the excellent steel that came to be produced in India. By $6^{\rm th}$ century BCE, as is borne out by analysis of iron objects unearthed in excavations there is further improvement in metallurgy. We start getting specimens that may be classified as mild steel. The techniques of carburisation, quenching and tempering come to be used, albeit selectively. There is an accompanying proliferation in tool typology. In one word, the technology adaptation was complete. It is around this time that we also come across references in literary texts about excellent quality steel produced in the subcontinent. In $6^{\rm th}$–$5^{\rm th}$ century BCE, Sushruta begins surgery using surgical tools. This must have required precision and high quality. These appliances had to be manufactured with great care. It stands to reason that the artisans of that age could supply such surgical instruments. Incidentally, Sushruta was a resident of ancient Varanasi (the site of Rajghat) that is situated close to the sites of Raipura, Latif Shah, Malhar and Nal-Ka-Tila mentioned above. These sites have yielded rich evidence of iron working, a resource zone for the settlements flourishing in the adjacent plains.   	

In due course of time Indian steel found an expanding market for itself with matchless products of its kind right from $4^{\rm th}$ –$3^{\rm rd}$ century BCE. The recent evidence of crucible steel from sites like Kodumnal shows that the wootz steel production had much higher antiquity than thought so far (K. Rajan, personal communication; Rajan et.al. 2021). We may say that India was at the forefront in the field of art, literature, mathematics, medicine, astronomy, astrology and other sciences including metallurgy by the opening centuries of the Christian era. The intellectuals were engaged in creating a knowledge bank in diverse fields imparting information, and training the society at large. The artisans and the craftsmen acquired the know-how in the field of art, architecture, statuary, metal casting and so on. The Śilpśāstras were composed later to train experts under the able guidance of great masters. There were well-developed theories deciding on the properties, the measurements and the carving techniques in sculpture and architecture as clearly borne out by stone sculptures one comes across during excavations. Such a tradition was fully developed towards the second half of the first millennium BCE. The skill and the system must have already taken deep roots in the oral tradition of the country before the composition of texts. The oral tradition of learning may be traced back to the Vedic times. The knowledge of metallurgy must have been acquired, dispersed and also preserved through this very process. 

In the specific context of iron technology, we have already seen that it was learnt independently through an evolutionary process. It attained unprecedented height by $4^{\rm th}$ century CE during the Gupta period. The Delhi iron pillar is an expression of the culmination of mastery achieved by the metal smiths of the $4^{\rm th}$ – $5^{\rm th}$ century CE. The tools and implements of this period clearly indicate that all the techniques of steel making were developed and they were very much in practice. As mentioned earlier, carburisation quenching and tempering techniques, that had already been learnt, came to be regularly practised at appropriate places. The artisans were astute enough to decide as to which specific type of object needed to be made in which particular way. They applied suitable technique at appropriate places. That there existed a theoretical basis for iron metallurgy is substantiated by references made to a number of texts that were written on iron metallurgy during the early medieval period.

A common belief is that iron technology in India was always prerogative of the tribal community or the lower echelons of society. True, that the ethnic communities like Asurs, Agarias and the Mundas have been recognized as traditional iron workers. But the question is: has the case been always so? Or was the metallurgy of iron been confined to traditional practitioner alone? Didn’t India ever have a science or theoretical basis of iron metallurgy?  This question needs special attention in view of the excellent quality of iron being produced in India. Answer to this lies in the literary sources, despite the large scale vandalism to which ancient texts were subjected. 

There is testimony of literary sources that suggest that iron and steel making in ancient India was not left to the folk tradition alone; it was a science in which the scholar as well as the royalty of the ancient times took keen interest. In the $11^{\rm th}$ century, we come across mention of a text Yuktikalpataru in Sanskrit which was composed by King Bhoja, a ruler of ancient ‘Dhara Nagari’ (Dhar of present times which houses the $11^{\rm th}$ century famous Dhar pillar), In his treatise on iron, Bhoja refers to two earlier texts, viz. {\it Lauha Pradip} and {\it Lauhadasp}. Thus at least we have knowledge of three texts on iron working composed before $11^{\rm th}$ CE. While it establishes the point beyond doubt that knowledge of iron had a theoretical basis and was in hands of the learned persons, including the royalties of the time. It appeared to have assumed the status of an important industry to warrant attention of a king like Bhoja. Not much later, ($11^{\rm th}$-$12^{\rm th}$ CE) the alchemy text {\it Ras Tatna Samucchaya} was composed. It may be classified as a text giving one of the finest and most detailed account on iron metallurgy. The fine classification on types and variety of iron, as well as method of production of each type, its properties etc. have been dealt with at length and with exemplary precision (see Ch.V). These accounts emphasize the fact that the presumed social barriers did not exist between theoreticians and social workers, at least at this juncture. There is another text on metallurgy {\it Sarangadhar paddhati}  dated to the $14^{\rm th}$ century CE. It deals at length with aspects of iron and steel production, especial attention has been paid to technique of sword making as well as the number of centres famous for manufacturing of swords. 

Unfortunately, the ancient texts that must have documented and demonstrated the theoretical background of iron metallurgy are no longer available to us. They must have been destroyed in the frequent mindless frenzy of invaders who targeted the monasteries and universities which housed rich libraries. Thus the invaders destroyed the very foundation of the culture and knowledge banks which were preserved in the form of hand written manuscripts. No wonder, what survives today is only in the oral tradition and that too with the practitioners who have taken in to the forests. The once flourishing iron industry somehow survived with the forest dwellers. It is now visible to us more as a craft than a science.

We have referred earlier to the annoyance expressed by the eminent Muslim scientists and philosophers like Bu Ali Sina on destruction of knowledge base by the invading forces of Mahmud in the name of Jehad. Sina, a respected Persian physician and biologist refused to accompany Mahmud to India because the latter indiscriminately destroyed Indian science and philosophy. Despite such set backs, the strong tradition of transmission of knowledge through Guru-Shishya institution which had always been a key in the knowledge transmission in India, played a role in preservation of the practice of a threatened knowledge. The excellent quality iron produced under the strict supervision of master craftsmen had attracted attention of travellers, visitors and traders who came to India from an early date.  Traders who valued iron and steel made in India continued to visit the centres of iron production till the Pre-modern times. We will briefly summarise the trade and commerce in iron and steel.

\vspace{-.3cm}

\subsection*{Trade in Iron}\label{chapter8-subsection-2}

\vspace{-.2cm}


Having surveyed the genesis and development of iron metallurgy, technological innovations, pattern of adaptation, and the heights which iron technology attained carving out a special place for itself in the ancient world, it may be interesting now to explore the trade mechanism of iron. Once the contemporary civilizations learnt about the saliency of Indian iron, it caught the fancy of the higher echelons of the society. The nobles and the kings of $5^{\rm th}$ –$4^{\rm th}$ century BCE treated Indian swords as their prized possession. It is around this very time that India developed diplomatic relations with the west starting with Achaemenians in $6^{\rm th}$ century BCE who had occupied parts of northwest. The Mauryan court received Greek ambassador of the likes of Megasthenes. After Alexander, regular passage was opened between India and the west especially Greece for diplomacy as well and commercial transactions. The Buddhist literature of about $4^{\rm th}$ –$3^{\rm rd}$ century BC recounts innumerable anecdotes of overland expeditions through caravans. There are clear references to caravan routes leading westwards through the mountain passes. There are similar narrations of traders sailing by sea with shiploads of luxury goods, as far as the Mediterranean, through the gulf. There were ports like Petra where merchants converged from different parts, such as Arabia, Mesopotamia, Syria, Palestine and Levantine ports. Indian goods reached different parts of the world through such a trade network right from $4^{\rm th}$ –$3^{\rm rd}$ century BC (Macrindle; 1979; Tarn, 1930).

From the $1^{\rm st}$ century CE onwards Indo-Roman trade had started taking roots. The Periplus of the Erythrian Sea amply demonstrates presence of flourishing trade relations between Indian and the Western world. Large vessels were sent from Barygaza (Broach) to the Persian market town of `Ommana', according to descriptions in Periplus. We have already mentioned earlier that the list of objects that were exported also included Indian iron and steel (see R. C. Majumdar, 1990). Pliny lamented the outflow of gold to India in exchange of luxury goods.

The foreign accounts of the early centuries of Common Era such as that of Periplus of Erythrian sea, give a vivid account of the flourishing trade and commerce that the affluent traders of India had established both through the overland routes as well as maritime trade relations with contemporary cultures. The passes in the Himalayan ranges leading further northwest and west were frequented by well-organized caravans. The naval routes operated from different ports from Sind (Bhambor) in the north to deep south and from coastal Gujarat to the Bay of Bengal.

The trade organizations – {\it Śreni} or guilds had come to occupy important status in society from the times of Panini, the great grammarian by $5^{\rm th}$ –$4^{\rm th}$ century BC. These Śrenis wielded great power in society. Besides enormous wealth that they possessed, they also kept their own army or fleets basically to safeguard themselves against robbers or pirates on their voyages. There is an anecdote in the Buddhist literature that Anathpindak an eminent merchant of $6^{\rm th}$ century BC had purchased the mangrove for lord Buddha by covering the entire ground of the orchard with gold coins. Such are the stories of the riches of the {\it `Setthins'} of ancient days.

During the successive centuries, especially in the Gupta period the {\it Śrenis} emerged as powerful organizations in India. The last decades of the golden period of Indian history, however, heralded a decline in economic condition including the trade and commerce. The international trade relations that the imperial Gupta had established with their contemporary world powers declined. But the story of prosperity of India had attracted invaders like the Hunas right in the $5^{\rm th}$ century CE towards the end of Gupta rule, something that never seemed to have stopped. After the Gupta imperial rule, internecine wars, feuds and rivalry between the small feudatories and states into which India was divided made it vulnerable to frequent attacks from diverse sources. 

``Global historical factors which appear to have contributed to the decline in prosperity …include invasion by fresh waves of barbarian central Asian tribes; the closure of the silk route through the Tarim basin and north-west India to the Arabian sea; and the rise of Islam. The coastal areas of Gujarat and Coromandel remained within the network of maritime trade…" (Digby, 1982).

Indian swords, as we have already discussed above were recognized for their special quality in the ancient world. We have also mentioned above about the painstaking methods with which these swords were manufactured as detailed by various texts from Varahmihir in the $5^{\rm th}$ century to {\it Sarangdhar paddhati} in the $14^{\rm th}$ CE. A large number of centres produced quality swords in India from north to south and from east to west. We frequently hear of such centres in ancient Indian literature.

These swords found their way in the world market through the well-organised trade mechanism. Even when state power became weak, trade and commerce continued to flourish. With closure of more secure routes of earlier days alternative routes were explored. The coastal routes provided alternative passage for maritime trade.

The accounts of Arab geographers relate to the trade activity through Indian sea ports. On the coast of Sindh (Indus) was Debal which was a large and busy trading centre of this region in the early medieval- medieval period. On the Gujarat coast, Khambat, Thana, Soppora and further south Sindan or Sanjan (near Bombay) right upto Malabar and Deccan there were such important coastal trading centres. The Arab writers like Ibn Khordadbah (end of $9^{\rm th}$ CE). Ibn Rosteh ($9^{\rm th}$ CE) talk of spice trade and import of along list of precious items including certain varieties of wood like teak and aloe and sandal wood, cotton cloth. These commodities were exported through sea routes. Abu Zaid mentioned about aloe wood of Kamrup (Assam) known as `Kamarubi' that sold for 200 Dinars per `mound'. We know that many of these regions of Bengal, Gujarat, Assam produced fine steel and steel swords that must have been exported through these networks.

Even the Chinese ships arrived here to purchase iron `rods' for swords. It was much sought after trade commodity by sword makers, especially of the Middle East. The traders used to make a beeline for it from different port towns of the ancient times. It was a valuable commodity with an extraordinary reputation of its efficacy and efficiency. This is fully testified by records and documents of as late as $17^{\rm th}$ –$18^{\rm th}$ century CE. Thus travellers and scholars have mentioned about the excellent Indian steel in the antiquity. The swords of Indian steel (generally referred to as Damascus steel) reached up to Europe. Marco Polo mentioned the name Ondanique which is said to be a corruption of the Persian word Hindwany or Indian. In Spanish it has been mentioned as {\it Al hinde} or {\it Al finde}, which initially meant steel and later because of the shine it came to connote mirror or foil. Arrian mentioned about a steel {\it śideros indico's} imported into Abyssinian ports. This was again Indian steel sailing out of the country to west. 

Writing about the period of $8^{\rm th}$ –$11^{\rm th}$ century CE UN Ghoshal (1964) concluded, that during this period Indian trade with the Arab world grew incessantly.

{\footnotesize{``The variety and excellence of Indian textiles, and metal work, and above all of Indian jewellery, are attested to by literary as well as epigraphic evidence. The sea and land routes of Indian teachers visiting China, Central Asia and Tibet as well as South – East Asia, were no doubt followed by the Indian merchants as well, reminiscences of whose unrecorded adventures have been partially preserved in the form of stories in contemporary prose romances. The daring and enterprise as well as the profit motive of the merchants of which we got such vivid accounts in the Jain stories"}} (is significant). (Ghosal 1964, 408-409).

\vspace{-.3cm}

\section*{The Invasions and End of Supremacy:}

\vspace{-.2cm}

This may be called the last phase of Indian enterprise and adventure. It is at this juncture that Mohammadan invasions started to loot the riches housed in temples –the tale of which had spread far and wide through the traders, visitors and scholars. The invaders targeted first the temples and treasuries returning home with enormous amounts of wealth. Later having assessed the weakness of the political system targeted and subjugated the kings and became rulers of the country suppressing the free spirit and creative urges for good.

The traders of this age somehow maintained commercial interaction with Persian and Arabic states and they went even beyond. Small states that came to occupy major parts of north India were assailable in the absence of powerful central authority. However, we come across temples (like Konark using massive iron beams which still hold part of the temple structure), architectural masterpieces and other monuments which came up during this period perhaps to assert their identities and as demonstration of status and power of the local rulers or chieftains. But inherently weak, and disunited, the petty rulers or feudal lords of early medieval times could hardly face the onslaught of the motivated Mohammedan armies frequently invading the borders. Taking advantage of the political weakness and vicissitudes of the local rulers, the Muslim expansion took over large parts of India. 

The `fragmentation of power' (as Digby puts it) took a big crunch of the volume of internal as well as external trade. The land routes deteriorated and became perilous. The powerful trade guilds declined drastically, but somehow did not disappear. Therefore, the ``recession of trade was not absolute". Though north India was politically in shambles, parts of Gujarat and south India were not that badly hit. They continued their foreign trade right upto $13^{\rm th}$ century and even later. The area of Kannauj along with the neighbouring areas under Harsh continued to flourish. The Palas held sway in Bengal upto $10^{\rm th}$ –$12^{\rm th}$ CE. The Chedies in eastern Madhya Pradesh and Gahadwalas in Kannauj survived till much later, as flourishing states.

\vspace{-.3cm}

\subsection*{Iron Production during the Medieval Age}\label{chapter8-subsection-3}

\vspace{-.2cm}

With Mohammedan invasions the attitude of ruling class towards common man and the artisans, craftsmen, skilled persons underwent changes. Heavy taxes were levied on craft production and even on farm produce. The workers were asked to render services for little or no payment at all. As the atrocities of the officials increased with time, to avoid exploitation and harassment often the workers deserted the villages. Ingenuity of such persons naturally suffered. 

In the field of iron production, the expertise of ironworkers was channelized towards military equipment and other gadgets suited for consolidation of power. During the medieval age the production capability intensified to meet the ever-increasing demands of iron for weapons. The Moghul rulers employed the skill of experts for cannon casting and gun making.

\newpage

The mines were identified and exploited; the resources were channelized in this direction and in this they were helped by the state machinery. Iron thus produced not only fulfilled the local requirements but was also exported to various countries. There existed important production centres from Gujarat to Bengal, from Kumaon and Shimla in the Himalayan region to deep South in Hyderabad and Salem. The users of the medieval ages were concerned more about the production, i.e., good quality steel to meet the growing military requirement of the state. However, the attitude towards the workers was not only of indifference but of disdain. The hardships hardly drew sympathetic attention of a class that lived in affluence and was capable of introducing innovations – the elite of the medieval ages. The iron working was an unorganised industry at that age. It was left in the hands of the craftsmen alone who acted as enslaved and forced labourers under the tyrannical supervision of petty officers of the state. There was very little option available to this class. They had to obey the officials of the state, suffer their atrocities without any recourse to law that hardly ever favoured poor workers. However, certain centres did manage to maintain their production capability, especially in Bengal, Assam, Gujarat and mostly in the southern part of the country. They continued to export iron to Middle East. With the exception of Tipu Sultan's workshop where waterpower driven plants were experimented with, no effort was made to introduce ox-driven or any such mechanical systems. When need for larger production arose, cheap human labour was pushed into the system to work day and night. 

The nature of state intervention during the medieval times, if records are to be believed, proved to be a disincentive to both the production and creative urges. There are frequent reports of instances of forced labour, exorbitant taxation, exhortations and exploitation by official machinery. The medieval period rulers hardly cared to provide any incentives to promote innovations. Even during the Moghul period the skilled class remained poor, working with the age-old tools and implements. As stated above, no efforts seem to have been made to improve their lot nor there were incentives provided for innovations. This could have been one of the fundamental causes for subsequent decline of iron industry.

During these centuries of foreign rule, the country obviously underwent through an unnerving and agonising experience. The spirit of innovation and creativity was replaced by one of preservation and an instinct for survival. To expect that the artisan class would have the courage to innovate and the energy to strengthen the infrastructure of the mechanism of production is indeed an unreasonable expectation. That they could continue with their traditional means and modes against such great odds of an oppressive rule is by itself a commendable achievement. They somehow persisted showing a remarkable resilience and they preserved the ground for future.  

\vspace{-.3cm}

\subsection*{Miserable Condition of the Artisans}\label{chapter8-subsection-4}

\vspace{-.2cm}

The survey of the social and economic status of the artisan class over a period of centuries makes it clear that after the ancient period ({\it ca} 11-1200 CE), the condition of unorganised sector of `industrial' workers deteriorated. They had little status in society. The caste system was gradually becoming more rigid. It did not even accommodate the smelters who were living in distant hills. The village smiths hardly maintained any social relationship with them. The artisan class, as pointed above was perpetually a victim of greed of petty officials of state. When they were harassed beyond their tolerance limits, they escaped to the neighbouring states to save themselves from the tyranny. They deserted their villages {\it en masse}. It has also been observed that they were kept in poverty deliberately by the petty officers even up to the Moghul period.

\vspace{-.2cm}

\subsection*{The Pre-Modern India: End of an Era}\label{chapter8-subsection-5}

\vspace{-.1cm}

In Gujarat, Bengal, the Deccan and some other parts of south India where trade in Damascus steel was in a flourishing state till much later, the iron working industry was better organized. We have talked about three types of production system: 1. The individual ventures; 2. The group venture and 3; The organized industries. Such ventures continued to exist till around $18^{\rm th}$ –$19^{\rm th}$ century. The third category was the most profit making one. They paid their labourers, supervisors and experts according their skill or a proportionate share of the product. The workers in such a system could perhaps sell the ingots at their own will and at whatever price they could get in return for it. 

\newpage

The individuals who lent their expertise to such organizations must have been very much in demand. But the ethnic societies which resided in remote areas and operated in family units were not so well off. With little understanding and/or access to market, they worked through intermediaries. They also volunteered to produce on larger scale if and when required. The instance of Tendukhera in M.P. during the British times has been elaborated upon earlier. Their product had been analysed and assessed by the British engineers. Such examination was conducted at several places – from Kumaon-Garhwal and at places in Himanchal Pradesh in the north to the Chhota Nagpur plateau that was largely occupied by the smelting communities like the Agraria and Asurs. With the experience of generations behind them, they could produce excellent quality iron and at a cheaper rate. 

In-depth examination of Indian iron and steel was made right from the $17^{\rm th}$ century by European Engineers. The British along with the other European agencies operating in the country were struck by the superb quality of steel being produced at such a low cost. We have already seen that even at a later stage, with Persian or European supervisors around, the condition hardly changed. Many Europeans like Franklin (1829) have commented on the advantage of the cheap labour and fuel available in India. They have also commented on how this could be harnessed to their best advantage as source of energy to make huge profits. The tools and techniques being used by the craftsmen and others engaged in industries of different kinds hardly changed. Lack of innovation in the fundamental production mechanism has been adversely commented upon time and again by scholars. What is noteworthy, however, is that a progressively increasing number of artisans were pushed into production of iron to meet the growing demands. This was in the nature of forced labour. In due course of time they seem to have been degraded to the position of unpaid labour force surviving at the mercy of masters or employers than independent and proud owners that they once were.

Franklin's comments substantiate this impression and are worth quoting here,

%~ \vspace{.1cm}

{\footnotesize{``The employment of so simple a forge in England would be absurd – but considering it an instrument adapted to the existing condition of the country where it is used – it assumes a different character – for such is the cheapness of labour and fuel that I question whether any other furnace would compete with it – and if by improvement it can be made capable of working on a larger scale; arsenal materials, materials for bridges and other heavy work – it certainly is an object worthy of attention as a great saving of expense might be effected by its use."}} 

Thus despite the so called `absurdity' noted by Franklin, a large number of small furnace operated by a noticeably large work force successfully met the growing demand of iron and steel. With their sheer grit they not only met the local requirements but catered to the demands of foreign traders also.

The European geologists explored and analysed the ores and made calculations to explore future prospects of production on a mass scale. Some efforts were made by them to increase the size of the smelting furnaces for higher production as suggested by Franklin, above. Some even succeeded. But no revolutionary changes could be introduced by them in iron production. Perhaps the indigenous industry was more suited for small-scale production. Low investment was involved with clay furnaces, using family or village groups to man them. Ore collection or shallow mining hardly exerted extra demands. Once the resources like ore and timber were exhausted the set up could be shifted to a new site without much cost or inconvenience. Though, on the surface, it looks more primitive but it had evolved over a long period to suit the local conditions and temper of the artisans. While many larger ‘factories’ set up in parts of Himalayas, Bengal-Bihar or South India had to be closed down after the venture became uneconomical and non-productive, the indigenous industry survived till much later. However, the weakened social fabric and the politico-economic pressures that were exerted on the $18^{\rm th}$ –$19^{\rm th}$ century craft groups were excessive.  The indigenous iron industry found it too tough to withstand it. 

\vspace{-.3cm}

\subsection*{Indigenous Iron Industry}\label{chapter8-subsection-6}

\vspace{-.2cm}

Though non-use of mineral fuels and non-introduction of flux are described to be the parameters of conservatism displayed by the indigenous iron industry, one wonders about the debacle that the European manufactures faced in different parts of the country, even after experimenting with newer and supposedly higher techniques. Most of the big plants had to be shut down within decades of their establishment. On the contrary, the indigenous system survived with their appropriateness. The traditional system had several built in advantages. It was evolved as what is termed today ``appropriate technology". Iron produced traditionally was of better quality, durability, and ductility than the factory made iron. Even till recently the villagers preferred locally produces iron in comparison to the latter. So was the case in Sweden as noted by Marie Nissen (2002).

The indigenous iron industry had its in-built weakness like the small production capability, lack of innovative inputs, and lack of organizational support, low investment capacity and a poor economic condition. To this may be added one more dimension, the social segregation of the artisan groups like smelters and smiths from the main stream. “ There were caste or caste-like barriers inhibited by social exchange and communication between the upper caste and most of the artisans groups, particularly the iron smelters of tribal origin….”commented Bhattacharya, (2002). We have little data to investigate as to when such a social segregation took place to alienate a whole class of `experts' who were the backbone of economic activity of the society. 

It has been often asked and experts have wondered whether there was any scientific basis of technology. Talking especially of iron metallurgy, it has traditionally been attributed to Asurs or the Agarias, the non-Aryan social groups residing in remote forests closer to ore bodies. It is another matter, and one not very hard to comprehend that living in such remote areas with little interaction with the mainstream cultures, these communities might have evolved their own socio-cultural and religious belief systems. But they must have been a powerful groups having their own political status and strength. It is evident from innumerable stories of Asur and Deva conflicts. But they were indeed a force to reckon with. There are countless anecdotes of their glory: whether these smelting communities have any relationship with the mythical Asurs, is difficult to say but that they were possessors of a valuable crafts – is beyond doubt. Though no written text related to their skill in metallurgy is available, but oral tradition has a long survival in India. It goes without saying that the knowledge was transmitted through the word of mouth from generation to generation and it survived for a long period look at the magnificent structures like the iron pillar at Mehrauli or the massive girders at Konark and other commemorative victory pillars makes it unlikely that these are works of unskilled folks.

\newpage

In India, tradition appears to have played a vital role not only in conservation and preservation of skills but also in promotion and expansion of arts, crafts and a variety of other knowledge systems. It is fully manifest in the masterly skill expressed through creative urges of the people. The early medieval age witnessed several monumental and iron structures like Konark beams and Dhar Pillar which were indeed products of ingenuity of the metal workers of that age. It is this infrastructure that was subsequently utilized by the rulers of the medieval period. Such human courage perseverance and innovative spirit is recognised and appreciated by scholars like Rosenberg in his comment,


``A major innovation is the one that provides a framework for a large number of subsequent innovations each of which is dependent upon, or complementary to, the original one...."He emphasised further that each innovation ``constitutes the initiation of a long sequence of path – dependent activities, typically extending over several decades..." (Rosenberg, 1994). It is true of the technological development in the field of ancient iron and steel making industry that evolved from simple wrought iron to steel production bodies of matchless quality.

Whether the indigenous iron working may be revived is an issue that is being debated at a few places. Though an uphill task, efforts have already started in this direction. It is like resurgence or may be even a resurrection of the dead from the grave. The success of such ventures will depend on the sincerity with which they are pursued; only the future will decide. But it is worth giving it a chance because it carries a legacy of a glorious past.

\vspace{-.3cm}

\subsection*{The Ethnic Artisans of Pre-Industrial Times}\label{chapter8-subsection-7}

\vspace{-.2cm}

The unorganised sector of Indian society, specially the craftsmen rarely got back what they deserved. Their mastery was appreciated but they were mostly exploited. It was true of metal crafts too, including the ironworkers. Till the industrial revolution all such skills could be defined as craft rather than an industry. Iron working was at best a flourishing cottage industry from the present day viewpoint. It was mostly run by family units. The smelters had conveniently chosen to stay closer to the ore deposits within the forested zones as Elvin remarked, `wherever there is {\it Sarai} tree there is Agaria'. Such communities like the Agaria or the Asur were generally cut off from agricultural villages and lived their lives off the forest or bartered their produce with the adjacent Agaria communities. At the most they got some return for their product from itinerant merchants who were a liaison between the remote areas and the mainstream cultures. Little wonder that they could rarely received real return for their produce and they continued to remain poor. 

The Asur and the Agaria tribes carried out this tradition of iron production. Their descendants still reside in the Vindhyan and Chota Nagpur plateaux passing through several states from Uttar Pradesh to Chhattishgarh and Orissa. They have carried this legacy till the fifties in the $20^{\rm th}$ century. On investigation the British engineers found, as mentioned earlier, that the pieces produced by the Agaria, the traditional ironworkers were far superior to the British or Swedish iron.

The family units continued. The entire community of smelters put in their labour, time and whatever little resources they had to increase production. This must have naturally benefited them materially but not sufficiently enough to bring about any perceptible change in their life style. This fact has been noticed by the foreign visitors to medieval or pre-modern India as noted earlier.

Indian swords were very much in demand. Indo-Roman trade is too well known a fact to warrant any additional details here. Over a period of time, interactions with the Middle Eastern countries grew. The port towns of that region attained the status of world market centres. The Indian made swords reached the world through them gaining the title of Damascus steel (swords). Persian and Arab traders started flocking the Indian ports for a variety of commodities – one of their primary interests was in swords with ripple mark or so called watering pattern that fetched high return. We have already mentioned about the Isphanani merchants who arrived in the Golkonda region regularly, stayed there to get the precious steel till as late as the $17^{\rm th}$ century CE or may be even later. We have also pointed at the fact that Indian steel was imported for making bridges in Britain such as in Menai Tubular Bridge and even in London Bridge. Thus the indigenous iron and steel industry carved out a special place for itself in the pre-modern world. With industrialisation and imperial designs of a foreign rule a decline set in.  

\vspace{-.35cm}

\subsection*{Decline}\label{chapter8-subsection-8}

\vspace{-.2cm}

Nevertheless, with the strengthening of the power of Muslim rulers, the socio-cultural and the economic situation in general continued to deteriorate.~The atmosphere became hostile.~It was not only detrimental to growth, prosperity and ingenious creativity of productive forces, it started eating up profits in a big way. The riches were drained towards the Middle East through Persian Gulf. It is clearly testified to by the Persian revenue records, such as {\footnotesize{``The gold at royal treasury (at Tabriz) which came from India and was coined in Tanka (therefore from Delhi Sultanate), exceeds that from any other source. The gold from India was recorded to be 1,22,000 {\it mithqals}, as against 64,000 {\it mithqals} of Egyptian gold..."}} (Simon Digby, 1984). Thus India paid a heavy price for its weaknesses - culturally, socially and economically from the medieval times onwards. It remained under the subjugation of foreign rulers right up to the British times, losing its ancient glory.

\vspace{-.35cm}

\subsection*{Concluding Remarks}\label{chapter8-subsection-9}

\vspace{-.2cm}

The iron industry could not withstand the colonial design working against its interests in a planned way. Once the modern blast furnaces came into existence in Britain, production started at a much cheaper rate. This was the last nail in the coffin of an industry which was staggering under its own weight. It could hardly compete with the cheap British pig iron being imported. On the other hand, laws enforcing non-felling of trees in the forests deprived the charcoal based indigenous iron industry of its very basic raw material. It made production of iron impossible. The powerful lobby in Britain succeeded. Wrote Geijerstam (2002),

{\footnotesize{“....free trade groups in England having imperial interest in limiting the competition and securing markets for the British industry. The cotton spinners of Lancashire and the iron and steel producers of Sheffield had fundamental common interest in this respect. They viewed India as a vital future market and considered it to be the duty of the government to facilitate trade contexts and market penetration…. This colonial character of policies of Indian iron making had been evident for a long time. Thus the demolition of indigenous iron industry appears to be part of the British policy. All possible contestants were systematically eliminated".}} 

He further mentioned about an order in 1887 by the mineralogical surveyor of Kumaon who recommended, “as the working of these metals (iron and copper) might injuriously affect important articles of British import, attention should be paid to finish off the local production capability". These statements are self-explanatory making any further effort in this direction unnecessary. This is further affirmed by modern British thinkers. Recently George Monbiot wrote an article ‘East India inc.’ in {\it The Gordian} in Britain saying, 

“Britain’s industrialization was secured by destroying the manufacturing capacity of India in 1699, the British government banned the import of woollen cloth from Ireland, and in 1700 the import of cotton cloth (or calico) from India. Both products were forbidden because they were superior to our own….” (reproduced in Hindustan Times, New Delhi, p.12, Oct. 11, 03). 

No wonder with such oppressive designs of the British government, the indigenous industries withered away in due course of time. Once the well-entrenched system was disturbed by arrival of larger plants, its resurgence must have been well nigh impossible. The reasons for it are not difficult to understand.

\begin{enumerate}
\item The factories that were established must have employed those very ironworkers as labourers paying them sufficiently but weaning them away from their house-hold units which were operative earlier.
\item It naturally disrupted the social, economic and technological structure of a delicately balanced system that was already pushed to the edge.
\end{enumerate}

The final blow came with the failure of the larger European-British and Swedish ventures - followed by import of cheap factory made pig iron from British factories. The fragile indigenous production structure succumbed to such pressures.

\section*{Bibliography}

\begin{thebibliography}{99}
\itemsep=0pt

\bibitem{chap8-key01} Agrawal, D. P. 1999. The role of Central Himalayas in Indian archaeometallurgy. In \textit{Metals in Antiquity} (Eds.) M. M. Suzanne A.Young, P. Budd, A.M. Pollard and Rob Ixer. Oxford: Archaeopress. Pp. 193-199.

\bibitem{chap8-key02} Agrawal, D. P. 2000. \textit{Ancient Metal Technology and Archaeology of South Asia}. Delhi: Aryan Books International.

\bibitem{chap8-key03} Agrawal, D. P. and J. S. Kharkwal 1998. \textit{Central Himalayas: An Archaeological, Linguistic and Cultural Synthesis}. New Delhi: Aryan Books International.

\bibitem{chap8-key04} Agrawal, D. P. and J. S. Kharkwal 2002. Outstanding problems of early Iron Age in India: need for a new approach. In \textit {Tradition and Innovation in the History of Iron Making: An Indo European Perspective} (Eds.) Girija Pande and Jan af Geijerstam. Nainital: PAHAR Parikrama. Pp. 3-20.

\bibitem{chap8-key05} Agrawal, D. P. and V. Tripathi 1999. Early Indian iron technology, Himalayan contacts and Gangetic urbanisation. In \textit{Proc. Fourth International Conference on the Beginning of the Use on Metals and Alloys (BUMA-IV)}. Matsue. Japan: The Japan Institute of Metals. Pp. 53-58.

\bibitem{chap8-key06} Agrawal, O. P. 1983. Scientific and technological examination of some objects from Ataranjikhera. In \textit{Excavations at Ataranjikhera}. R.~C.~Gaur 1983. New Delhi: Motilal Banarasidass. Pp.487-489.

\bibitem{chap8-key07} Agrawala, R. C. and V. Kumar 1976. The problems of PGW and iron in North-eastern Rajasthan. In \textit{Mahabharat, Myth and Reality} (Eds.) S.P. Gupta and K.S. Ramachandran. Delhi: Aram Prakashan. Pp. 241-44. 

\bibitem{chap8-key08} Alam, I. 2002. Iron manufacturers in Golkonda in 17$^{\rm th}$ century. In \textit{Tradition and Innovation in the History of Iron Making: An Indo European Perspective} (Eds.) Girija Pande and Jan af Geijerstam.  Nainital: PAHAR Parikrama. Pp. 98-111.

\bibitem{chap8-key09} Allchin, F. R. 1962. Upon the antiquity and methods of gold mining in ancient India. \textit{Jour. Eco. Soc. Hist. Orient} 5: 195-211.

\bibitem{chap8-key10} Anantharaman, T.R. 1999. The iron pillar at Kodachadri in Karnataka. \textit{Current Science} 76: 1428-1430.

\bibitem{chap8-key11} Antieny, A. 1987. Result of metallographic examination of pattern welded swords and lance heads in the Baltic area. Paper presented at \textit{Archaeometallurgy of Iron (1967-1987)}, Symposium Liblice, Praha.

\bibitem{chap8-key12} Atkinson, E. T. 1973. (First Pub. 1884). \textit{The Himalayan Gazetteer}, 1(1): 262-275. New Delhi: Cosmo Publication. 

\bibitem{chap8-key13} Balasubramaniam, R. 2001. \textit{Delhi Iron Pillar: New Insights}. New Delhi: IIAS Shimla and Aryan Books International.

\bibitem{chap8-key14} Balasubramaniam, R. 2008. Marvels of Indian Iron Through the Ages, New Delhi, Infinity Foundation Series and Rupa \& Co.

\bibitem{chap8-key15} Ball, V. 1881a. Geology of the districts of Manbhum and Singhbhoom. \textit{Memoirs of Geological Survey of India} 18(2): 106-146. 

\bibitem{chap8-key16} Ball, V. 1881b. \textit{A Manual of Geology of India, Pt III. Economic Geology}. Calcutta: Geological Survey of India. 

\bibitem{chap8-key17} Ballal, N. B. et al. 1990. Interim report on \textit{Indigenous Iron and Steel Technology of India}. Department of Metallurgy, IIT, Bombay.

\bibitem{chap8-key18} Banerjee, M. N. 1927. On metals and metallurgy in ancient India – metallurgy in Rigveda. \textit{Indian Historical Quarterly }3(1-2): 121-33.

\bibitem{chap8-key19} Banerjee, M. N. 1929. Iron and steel in Rgvedic age. \textit{Indian Historical Quarterly} 5: 432-36.

\bibitem{chap8-key20} Banerjee, M. N. 1932. A note on iron in Rgvedic age. \textit{Indian Historical Quarterly} 8: 364-66. 

\bibitem{chap8-key21} Banerjee, N. R. 1965. \textit{The Iron Age in India}. Delhi: Munshiram Manoharlal.

\bibitem{chap8-key22} Bardget, W. E. and J. F. Stanner. 1963. The Delhi iron pillar: a study of corrosion aspects. \textit{Journal of the Iron and Steel Institute:} 210: 3-10.

\bibitem{chap8-key23} Bhardwaj, H. C. 1979. \textit{Aspects of Ancient Indian Technology}. Varanasi: Motilal Banarasidass.

\bibitem{chap8-key24} Bhattacharya, S. 1972. Iron smelters and the indigenous iron and steel industry of India:  from stagnation to atrophy. In \textit{Aspects of Indian Culture and Society} (Ed.) Surajit Sinha. Calcutta: Indian Anthropological Society. Pp. 133-152.

\bibitem{chap8-key25} Bhattacharya, S. 1983. Eastern India. In \textit{Cambridge Economic History of India. Vol. II.} (Ed.) Dharma Kumar. Cambridge: Cambridge University Press. Pp. 270-295.

\bibitem{chap8-key26} Bhattacharya, Savyasachi. 2002. Foreword to \textit{Tradition and Innovation in the History of Iron Making: An Indo European Perspective.} (Eds.) Girija Pande and Jan af Geijerstam. Nainital: PAHAR Parikrama. pp. iv-x

\bibitem{chap8-key27} Biswas, A. K. 1996. \textit{Minerals and Metals in Ancient India}. Vol.1 \& 2. New Delhi: D. K. Printworld. 

\bibitem{chap8-key28} Biswas A. K. 2001. \textit{Minerals and Metals in Pre-Modern India}. New Delhi: D. K. Printworld.

\bibitem{chap8-key29} Bjorkman, J. K. 1973. Quoted by Robert Maddin (1974) Early Iron Metallurgy in the Near East. \textit{Journal of Transactions of the Iron and Steel Institute of Japan:} 60-61. Tokyo: Centre for Meteorite study. 

\bibitem{chap8-key30} Bose, H. K. 1968. Mining in ancient India. \textit{Transactions of the Mining. Geological and Metallurgical Institute of India} 65 (1): 83-89.

\bibitem{chap8-key31} Bronson, B. 1986. The making and selling of Wootz – a crucible steel of India. \textit{Archaeomaterials} 1(1): 13-51.

\bibitem{chap8-key32} Buchanan-Hamilton, F. 1807. \textit{A Journey from Madras through the Countries of Mysore, Canara and Malabar, Vol. II. London}.

\bibitem{chap8-key33} Chakrabarti, D. K. 1976. The beginning of iron in India. \textit{Antiquity} 50: 114-124.

\bibitem{chap8-key34} Chakrabarti, D. K. 1986. The pre-industrial mines of India. \textit{Puratattva} 16: 65-72. 

\bibitem{chap8-key35} Chakrabarti, D. K. 1992. \textit{The Early Use of Iron in India}. Delhi: Oxford University Press. 

\bibitem{chap8-key36} Chakrabarti D. K. and R. K. Chattopadhyay 1985. A preliminary report on the archaeology of Singhbhum with notes on the old mines of the district. \textit{Man and Environment} 9: 128-150.

\bibitem{chap8-key37} Chakrabarti, D. K. and N. Lahiri 1994. The Iron Age in India: the beginning and consequence. \textit{Puratattva} 24: 12-32.

\bibitem{chap8-key38} Charles, J. A. 1980. The coming of copper-base alloys and iron: a metallurgical sequence. \textit{In The Coming of the Age of Iron} (Eds.) T. A. Wertime and J. D. Muhlly. New Haven, London: Yale University Press. Pp. 151­-182. 

\bibitem{chap8-key39} Chattopadhyay, P. K. 2004. \textit{Archaeometallurgy in India}. Patna: Kashi Prasad Jayaswal Research Institute.

\bibitem{chap8-key40} Cousens, H. 1902. The iron pillar at Dhar. \textit{Archaeological Survey of India. Annual Report 1} (1902-03):  205-212.

\bibitem{chap8-key41} Craddock, P. T., I. C. Freestone, L. Willies, H.V. Paliwal, L.K. Gurjar and K.T.M. Hegde 1998. Ancient industry of lead, silver and zinc in Rajasthan. In \textit{Archaeometallurgy in India} (Ed.) Vibha Tripathi. Delhi: Sharada Publishing House. Pp. 107-133. 

\bibitem{chap8-key42} Curtis, L. C., B. P. Radhakrishna and G.K. Naidu 1990. Hutti – \textit{Gold Mine With a Future}. Bangalore: Geological Society of India.

\bibitem{chap8-key43} Dani, A. H. \textit{et al}. 1967. Timargarh and Gandhara grave culture. \textit{Ancient Pakistan Vol 3: 1-407}. Peshawar: Department of Archaeology, Ministry of Education, Government of Pakistan.

\bibitem{chap8-key44} De, S. and P.~K. Chattopadhyay 1989. Iron objects from Pandurajar Dhibi: Archaeometallurgical studies. \textit{Steel India}, A Biannual Technical Journal for the Iron and Steel Industry, 2(1): 33-41.

\bibitem{chap8-key45} Deshpande, P.~P., R.~K.~Mohanty and V.~S.~Shinde 2010. Metallographical Studies of a Steel Chisel found at Mahurajhari, Vidarbha, Maharashtra. Current Science 99(5): 636-639.

\bibitem{chap8-key46} Deo, S.~B. 1998. Inaugural address to national seminar on Indian Archaeometallurgy. In \textit{Archaeometallurgy in India} (Ed.) Vibha Tripathi. Delhi: Sharada Publishing House. Pp. XXV-XXIX. 

\bibitem{chap8-key47} Deo, S.~B. and A.~P. Jamkhedkar. 1982. \textit{Excavations at Naikund}. Bombay: Department of Archaeology and Museums. Government of Maharashtra. 

\bibitem{chap8-key48} Dharampal. 1971. \textit{Indian Science and Technology in the 18th Century}. Hyderabad: Academy of Gandhian Studies.

\bibitem{chap8-key49} Digby, Simon 1982. The maritime trade of India. In \textit{Cambridge Economic History of India, Vol. I.} (Eds.) Tapan Raychaudhuri and Irfan Habib. New Delhi: Orient Longman and Cambridge University Press. Pp. 125-162. 

\bibitem{chap8-key50} Dunn,~J.~A. 1937. The mineral deposits of Eastern Singhbhum and surrounding areas. \textit{Memoirs of Geological Survey of India}, 69 (1): 1-183.
\bibitem{chap8-key51} Dutta, A. 1992. Early stages of iron technology and the development of regional pattern in India. In \textit{Man and his Culture: A Resurgence} (Ed.) Peter S. Bellwood, New Delhi: Books \& Books. Pp. 293-308.

\bibitem{chap8-key51} Dutta, A. 1998. Iron technology at the chalcolithic phase of West Bengal. In \textit{Archaeometallurgy in India} (Ed.) Vibha Tripathi. Delhi: Sharada Publishing House. Pp. 36-43.

\bibitem{chap8-key52} Eliot, J. 1880.  Indian Industries. London:  W.~H.~Allen and Company.

\bibitem{chap8-key53} Elliot, H.~M. and J.~Dawson. 1953. \textit{History of India as told by its own historians, Vol. I}. Calcutta. Reprinted in 1972, Allahabad: Kitab Mahal.

\bibitem{chap8-key54} Elvin, V. 1942. \textit{The Agaria}. London: Oxford University Press. 

\bibitem{chap8-key55} Elvin, V. 1963. In foreword to \textit{The Asur}. K.~K.~Leuwa. New Delhi: Bhartiya Adimjati Sevak Sangh. p. viii.

\bibitem{chap8-key56} Fernandes, W.~G. Menon and P. Viegas 1988. \textit{Forests, Environment and Tribal Economy}.  Tribes of India Series. New Delhi: Indian Social Institute.

\bibitem{chap8-key57} Forbes, R.~J. 1950. \textit{Metallurgy in Antiquity}. Leiden: E.J. Brill

\bibitem{chap8-key58} Foster, W. 1921. Early Travelers in India (1616-1619). W. Foster (Ed.) London. p. 314. (Quoted in O.P. Jaggi 1989).

\bibitem{chap8-key59} Franklin, J. 1829. The Mode of Manufacturing Iron in Central India. Quoted in Dharmapal (1971) \textit{Indian Science and Technology in the 18th Century}. Hyderabad: Academy of Gandhian Studies, pp. 269-294. 

\bibitem{chap8-key60} Gaur, R.~C. 1983. \textit{Excavations at Atranjikhera}. Delhi: Munshiram Manoharlal. 

\bibitem{chap8-key61} Geijerstam, J. af. 2002. The Kumaon iron works – a colonial project. In \textit{Tradition and Innovation in the History of Iron Making: An Indo European Perspective.} (Eds.) Girija Pande and Jan af Geijerstam.  Nainital: PAHAR Parikrama. Pp. 157-201

\bibitem{chap8-key62} Ghirshman, R. 1954. Iran. Harmondsworth, Middlesex: Penguin Book.

\bibitem{chap8-key63} Ghosh, N.~C. and P.~K. Chattopadhyay 1987-88. The archaeological background and an iron sample from Hatigra. \textit{Puratattva} 18: 21-27.

\bibitem{chap8-key64} Ghose, M.~K. 1963. The Delhi iron pillar and its iron. \textit{National Metallurgical Laboratory Technical Journal} 5 (1): 31-35. 

\bibitem{chap8-key65} Ghoshal, U.~N. 1964. Economic life. In \textit{The Age of Imperial Kanauj} (Eds.) R.C. Majumdar \textit{et.al}. Bombay: Bhartiya Vidya Bhawan. Pp. 400-411. 
 
\bibitem{chap8-key66} Gogte, V.~G. 1982. Report on iron-working and furnaces. In \textit{Naikund Excavations-1978-79}. S.~B.~Deo and A.~P.~Jamkhedkar, Pune: Deccan College Pune, Pp. 52-59.

\bibitem{chap8-key67} Goitein, S.~D. 1966. \textit{Studies in Islamic History and Institutions}. Leiden: E. J. Brill.

\bibitem{chap8-key68} Gopal, L. 1961. Antiquity of iron in India. \textit{Uttar Bharati} 9: 71-86.

\bibitem{chap8-key69} Gordon, D.~H. 1950. The early use of metals in India and Pakistan. \textit{Journal of the Royal Anthropological Institute} 80: 50-78.

\bibitem{chap8-key70} Gordon, D.~H. 1958. \textit{Prehistoric Background of Indian Culture}. Bombay: Bhulabhai Memorial Institute.  

\bibitem{chap8-key71} Habib, Irfan. 1984. \textit{Non-Agriculture Production and Urban Economy}. The Cambridge Economic History of India. (Eds.) T Raychoudhary and I. Habib Delhi: Orient Longman and Cambridge University Press. Pp. 76-92.

\bibitem{chap8-key72} Hadfield, R. 1913-14. A.S.R. Archaeological Survey Report 1913-14 quoted in \textit{Taxila}, Sir John Marshall (Reprint 1975), New Delhi: Motilal Banarasidass. Pp. 534-35.

\bibitem{chap8-key73} Hadfield, R. 1925. Discussion on Friend and Thorneycraft’s paper on Ancient Iron. \textit{Journal of Iron and Steel Institute} 112: 233-235.

\bibitem{chap8-key74} Hamilton B. Captain. 1708. \textit{A new count of the East Indies Vol. 1,} p. 392. Unpublished report.

\bibitem{chap8-key75} Hannay, S.F. 1857. Notes on the iron ore statistics and economic geology of upper Assam. \textit{Journal of Asiatic Society of Bengal }25: 330. 

\bibitem{chap8-key76} Hansen, P.~M. 1958. \textit{Constitution of Binary Alloys}. New York: Mcgraw-Hill Book Company.

\bibitem{chap8-key77} Harding, R. 2004. Report on scanning electron microscope analysis of NBP sherds. \textit{Man and Environment} 29 (2): 30-36.

\bibitem{chap8-key78} Hari Narain, Jai Prakash, O.~P.~Agrawal and M.~V.~Nair. 1998. Metallographic studies of iron artifacts of domestic use tools and weapons from Sringverpur (250 BC-600 AD). Uttar Pradesh. In \textit{Archaeometallurgy in India} (Ed.) Vibha Tripathi. Delhi: Sharada Publishing House. Pp. 44-53.

\bibitem{chap8-key79} Hari Narain, A.~K. Srivastava, Jai Prakash, I.~K.~Bhatnagar and O.~P. Agrawal. 1990-91. Metallographic studies of iron implements from Soron (600 BCE-100 AD), Uttar Pradesh, \textit{Pragdhara} 1: 163-171. 

\bibitem{chap8-key80} Havart, D. 1693. en zijn op-en ondergang van coromandet (in Dutch,  Rise and decline of Coromandel). Quoted in I. Alam, 2002. Iron manufacturers in Golkonda in 17$^{\rm th}$ century. In \textit{Tradition and Innovation in the History of Iron Making: An Indo European Perspective} (Eds.) Girija Pande and Jan af Geijerstam. Nainital: PAHAR Parikrama. Pp. 98-111.

\bibitem{chap8-key81} Heath, J.~M. 1839. On Indian iron steel. \textit{Journals of Royal Asiatic Society} 5: 390-397.

\bibitem{chap8-key82} Heatly, S.~G.~T. 1843. Contribution towards a history of the development of mineral resources of India. \textit{Journal of Asiatic Society of Bengal} 12(2): 542-546.

\bibitem{chap8-key83} Hegde, K.~T.~M. 1973. A model for understanding ancient Indian iron metallurgy. \textit{Man} 8(3): 416- 421.

\bibitem{chap8-key84} Hegde, K.~T.~M. 1991. \textit{An Introduction to Ancient Indian Metallurgy}. Bangalore Geological Society of India.

\bibitem{chap8-key85} Heriwood, W.~J. 1855. Report on the metalliferous deposits of Kumaon and Garhwal in Northwestern India. Selection from the \textit{Records of the Government of India, Home Department} 8: 1-46. Calcutta. Unpublished report.

\bibitem{chap8-key86} Holland, T.~H. 1892. Preliminary report on the iron ores and iron industries of the Salem district. \textit{Records of the Geological Survey of India} 25: 137. 

\bibitem{chap8-key87} Hughes, T.~W.~H., 1873. Notes of some of the iron deposits of Chanda, Central Provinces, \textit{Records of Geologycal Survey of India} 6(4): 77-81.

\bibitem{chap8-key88} Jackson, W. 1845. Memorandum on the iron worker of Birbhum. \textit{Journal of Asiatic Society of Bengal} 14(2): 755.

\bibitem{chap8-key89} Jacob, L. 1843. Report on the iron of Kathyawar, its comparative value with British metal, the mines and the mode of smelting the ore. \textit{Journals of Royal Asiatic Society} 7: 1012.

\bibitem{chap8-key90} Jaggi, O.~P. 1989. \textit{Science and Technology in Medieval India}. Delhi: Atma Ram and Sons. 

\bibitem{chap8-key91} Jarrige, J.~F. and J.~F. Enault 1973. Recent Excavations (French) in Pakistan. In \textit{Radiocarbon and Indian Archaeology} (Eds.) D.~P.~Agrawal and A. Ghosh. Bombay: TIFR. Pp. 163-172.

\bibitem{chap8-key92} Joshi, S.~D. 1970. \textit{History of Metal Founding on the Indian Subcontinent since Ancient Times}. Ranchi: Mrs. Sushila S. Joshi.

\bibitem{chap8-key93} Jueming, Hua 1981. Ancient bronze smelting casting and early iron casting in China. Paper presented at BUMA, Beijing. 

\bibitem{chap8-key94} Jueming, Hua 1986. Metallurgical technologies in Ancient China. \textit{History of the Development in World Metallurgy} (Eds.) Hua Jueming \textit{et.al}. Bejing: Science and Technology Literature Press, cyclostyled copy.

\bibitem{chap8-key95} Kangle, R.~P. 1963. \textit{The Kautilya Arthashastra} Pt. II. Bombay: University of Bombay Studies.

\bibitem{chap8-key96} Kashyapa, J. 1959. \textit{Samyukta Nikaya}. Varanasi: Mahabodhi Society, Sarnath.

\bibitem{chap8-key97} Kedzierski, Z. and J. Stephneski 1987. \textit{Archaeometallurgy of Iron (1967-87). Symposium}. Liblice. p. 387.

\bibitem{chap8-key98} Kennedy, J.~P. 1855. Selected Record. Bombay Government, New Series No. 9: 42-48.

\bibitem{chap8-key99} Khan, I.~G. 1986. Metallurgy in medieval India – 16$^{\rm th}$ to 18$^{\rm th}$ centuries. \textit{Technology in Ancient and Medieval India} (Eds.) Aniruddha Roy and S.~K. Bagchi. Delhi: Sandeep Prakashan. Pp. 71-91 

\bibitem{chap8-key100} Krishnan, M.~S. 1952. The iron ores of India. Indian Mineral 6(3): 113-130.

\bibitem{chap8-key101} Krishnan, M.~S. 1954. Iron Ore, Iron and Steel. Bulletins of the Geological Survey of India No. 9. Series A - Economic Geology, Delhi: Geological Survey of India.

\bibitem{chap8-key102} Kulkarni, D.~A. (Ed). 1969. \textit{Ras Ratna Samuccaya}. Delhi: Meher Chand Laxman Das.

\bibitem{chap8-key103} La Touche, T.~H.~D. 1918. \textit{A Bibliography of Indian Geology and Physical Geography with an Annoted Index of Minerals of Economic Value. Part, I-B}. Calcutta: Geological Survey of India.

\bibitem{chap8-key104} Lahiri, A.~K. 1963. Some observations on corrosion resistance of ancient Delhi iron pillar and present time Adivasi iron made by primitive method. \textit{NML Technical Journal} 3(1): 46-54.

\bibitem{chap8-key105} Lal, B.~B. 1954-55. Excavations at Hastinapur and other explorations in Ganga-Satlej basin (1950-52.) \textit{Ancient India} 10-11: 5-151. 

\bibitem{chap8-key106} Lamberg-Karlovsky, C.~C. and J. Humphries 1968. The cairn burials of South-eastern Iran. \textit{East and West, New Series} 18(3-4): 269-76.

\bibitem{chap8-key107} Leuva K.~K. 1963. \textit{The Asur}. New Delhi: Bhartiya Adimjati Sevak Sangh.

\bibitem{chap8-key108} Lowe, T.~L. 1990. Refractories in high-carbon iron processing: a preliminary study of the Deccani wootz-making crucibles. In. Cross-craft and \textit{Cross-cultural Interaction in Ceramics: Ceramic and Civilization Vol. 4} (Eds.) P.E. McGovern, M.~D. Notis and W.~D. Kingery. Westvilte, OH: The American Ceramic Society. Pp. 237-251. 

\bibitem{chap8-key109} Maddin, R. 1982. The beginning of the Iron Age in the Eastern Mediterranean. \textit{Transactions of the Indian Institute of Metals}. 35(1): 14-34. 

\bibitem{chap8-key110} Maddin, R., J.~D. Muhly and T.~S. Wheeler. 1977. How the Iron Age began. \textit{Scientific American} 237(4): 127-137.

\bibitem{chap8-key111} Mahmud, S.~J. 1988. \textit{Metal Technology in Medieval India}. Delhi: Daya Publishing House. 

\bibitem{chap8-key112} Majumdar, R.~C. 1990. India and the western world: economic condition. In \textit{The Age of Imperial Unity}. (Eds.) R.~C. Majumdar \textit{et.al}. Bombay: Bhartiya Vidya Bhawan. Pp. 611-632.

\bibitem{chap8-key113} Manucci, N. 1907. \textit{Storia do Mogor} (Translated by W. Irvine). London.

\bibitem{chap8-key114} Mathur, S.~M. 1949. Iron smelting by Agaria. \textit{Indian Minerals} 3(3): 138-141.

\bibitem{chap8-key115} Maxwell-Hyslop, K.~R. 1974. Assyrian source of iron: preliminary survey of historical and geographical evidence. Iraq 36: 139-154

\bibitem{chap8-key116} McCrindle J.~W. 1979. \textit{Ancient India as Described in Classical Literature}. New Delhi: Oriental B.K.S Reprint Corporation.

\bibitem{chap8-key117} Mockler, E. 1876. On ruins in Makran. \textit{Journal of Royal Asiatic Society} 9: 121-34.

\bibitem{chap8-key118} Moorey, P.~R.~S. 1994. \textit{Ancient Mesopotamian Materials and Industries: The Archaeological Evidence}. Oxford: Claredon Press. 

\bibitem{chap8-key119} Moreland, W.~H. 1923 (Reprint 1988). \textit{From Akbar to Aurangzeb}. Delhi: Vinod Publications.

\bibitem{chap8-key120} Moreland, W.~H. 1962. (Reprint) \textit{India at the Death of Akbar}. Delhi: Atma Ram \& Sons. 

\bibitem{chap8-key121} Murray, E.~F.~O. 1940. The ancient workers of western Dhalbhum. \textit{Journal of the Asiatic Society of Bengal. Letters} 7: 79-104.

\bibitem{chap8-key122} Niogi, P. 1914. \textit{Iron in Ancient India. Calcutta}: Indian Association for Cultivation of Science.

\bibitem{chap8-key123} Nisser, M. 2002. Swedish iron and steel: the historical perspective. In \textit{Tradition and Innovation in the History of Iron Making: An Indo European Perspective} (Eds.) Girija Pande and Jan af Geijerstam.  Nainital: PAHAR Parikrama. Pp. 112-145.

\bibitem{chap8-key124} Oldham, T. 1852. Report of the examination of the districts in Damooh valley and Beerbhooma, producing iron ore. \textit{Selected Records of Bengal Government} 13: 1-34.

\bibitem{chap8-key125} Pande, G. 2002. Iron making in Kumaon: a study of Kumaon iron works. In \textit{Tradition and Innovation in the History of Iron Making: An Indo European Perspective} (Eds.) Girija Pande and Jan af Geijerstam.  Nainital: PAHAR Parikrama. Pp. 146-156. 

\bibitem{chap8-key126} Percy, J. 1861. \textit{Metallurgy, fuels refractories and copper}. London. 

\bibitem{chap8-key127} Percy, J. 1864. \textit{Metallurgy of Iron and Steel}. London: Murray.

\bibitem{chap8-key128} Paiskowski, J. 1974. Metallographic examination in ancient Iron objects. Perspectives in Palaeoanthropology (Ed.) A.~K. Ghosh. Calcutta: Firma K.~L. Mukhopodhyay. Pp. 321-29.

\bibitem{chap8-key129} Paiskowski, J. 1988. The earliest iron in the world. PACT 21: 37-46. 

\bibitem{chap8-key130} Prakash, B. 1989. Metals of iron making in early India. \textit{Archaeometallurgy of Iron (1967-87)}. Symposium. (Ed.) R. Pleiner. Liblice: UISPP Prague. Pp. 307-332. 

\bibitem{chap8-key131} Prakash, B. 1991. Metallurgy of iron and steel making and black smithy in ancient India. \textit{Indian Journal of History of Science} 26(4): 351.

\bibitem{chap8-key132} Prakash, B. 1997. Metals and metallurgy. \textit{History of Science and Technology}. Vol. I. (Ed.) A.K. Bag. New Delhi: INSA. Pp. 80-174.

\bibitem{chap8-key133} Prakash, B. 2002. The inception of iron and steel making in India: archaeometallurgical overview. In \textit{Tradition and Innovation in the History of Iron Making: An Indo European Perspective} (Eds.) Girija Pande and Jan af Geijerstam.  Nainital: PAHAR Parikrama. Pp. 21-30.

\bibitem{chap8-key134} Prakash, B. and Igaki, K. 1984. Ancient iron making in Bastar district. \textit{Indian Journal of History of Science} 19(2): 172-185. 

\bibitem{chap8-key135} Prakash, and V. Tripathi 1986. Iron technology in ancient India. \textit{Metals and Materials}. (Sept. Issue): 568-579. 

\bibitem{chap8-key136} Puri, K.~N. (N.D.) Excavations at Rairh 1938-39 (Jaipur). Department of Archaeological and Historical Research. 

\bibitem{chap8-key137} Rao, K.~N.~P. 1989. Wootz – Indian crucible steel. \textit{Metal News} 11(3): 1-8.

\bibitem{chap8-key138} Nagaraja Rao, M.~S. 1971. Protohistoric Cultures of Tungabhadra Valley, Dharwad, Department of Archaeology 

\bibitem{chap8-key139} Thomas, P.~J.~P. Nagabhusanam, D. V. Reddy. 2008 Optically Stimulated luminescence dating of heated materials using single-aliquot regenerative-dose procure: a feasibility study using archaeological artifacts from India. \textit{Journal of Archaeological Science} 35: 781-790.

\bibitem{chap8-key140} Rajan, K. 2015. Kodumanal: An Early Historic Site of South India. \textit{Man and Environment}. XL, No.2: 65-79

Rajan, K.V.P. Yathees Kumar, M. Rajesh, R. Sivanantham, J. Ranjith and J. Baskar 2021. Recent Radiometric Dates and Their Implications in Understanding the Early Writing System and Early Hisoric Archaeology of Tamil Nadu. \textit{Man and Environment} XLVI(1):57-86.

\bibitem{chap8-key141} Raychaudhury, T. 1982. Non-agricultural production (Mughal India). In \textit{The Cambridge Economic History.} Vol. 1 c. 1200 - c. 1750. (Ed.) Tapan Raychaudhury and Irfan Habib. Cambridge University press, Cambridge and Orient. Pp. 261-307.

\bibitem{chap8-key142} Renfrew, C. 1986. Varna and the emergence of wealth in prehistoric Europe. In \textit{Social Life of Things; Commodities in Cultural Perspective} (Ed.) A. Appadurai. Cambridge University Press. Pp. 141-168. 

\bibitem{chap8-key143} Richardson, F.~D. and J.~H.~E. Jeffes 1948. Thermodynamics of metallurgical reactions. \textit{Journal of Iron and Steel Institute} 160:  261.

\bibitem{chap8-key144} Roessler, K. 1995. The non-rusting iron pillar at Dhar. \textit{NML Technical Journal} 37: 143-154. 

\bibitem{chap8-key145} Roessler, Ing Habil Klaus 1997. The Big Cannon Pipe at Thanjavur. \textit{Metal News} 19(1): 1-4. 

\bibitem{chap8-key146} Rosenberg, N. 1982. Inside the black box. \textit{Technology and Economics}. London: Cambridge University Press.

\bibitem{chap8-key147} Roychowdhury, M.~K. 1953. A note on the lateritic iron ore of the Maikal range, Madhya Pradesh. \textit{Indian Minerals 7}(3): 125-137.

\bibitem{chap8-key148} Roy, S.~N. 1984. Early literary data on the use of iron in India. Paper presented in \textit{World Archaeological Congress}. Southampton. U.K (typed copy).

\bibitem{chap8-key149} Sachu, E.~C. 1983 (Reprinted). \textit{Alberuni's India (1030 AD)} Vol. I \& II. Delhi: Munshiram Manoharlal.

\bibitem{chap8-key150} Sahi, M.~D.~N. 1978. New light on painted grey ware people as revealed from the excavations at Jakhera. Man and Environment 2: 104-105. 

\bibitem{chap8-key151} Sahi, M.~D.~N. 1979. Iron at Ahar. In Essay in Indian Protohistory (Eds.) D.~P. Agrawal and D.~K Chakrabarti. New Delhi: B.R. Publishing Company. Pp. 365-68.

\bibitem{chap8-key152} Sahi, M.~D.~N. 1994. Aspects of Indian Archaeology. Jaipur: Publication Scheme. 

\bibitem{chap8-key153} Sankalia, H.~D., S.~B. Deo, Z.~D. Ansari. 1969. Excavations at Ahar. Poona: Deccan College Post Graduate and Research Institute.

\bibitem{chap8-key154} Sanyal, H. 1968. The indigenous iron industry of Birbhum. Indian Journal of Economic and Social History Review. 4: 101-108. 

\bibitem{chap8-key155} Sarkar, J. 1919. Studies in Mughal India. Calcutta: M.~C. Sarkar. 

\bibitem{chap8-key156} Sarton, G. 1931. Introduction to the History of Science Vol. 2 Pt. II. Carnegie Institute of Washington.

\bibitem{chap8-key157} Sasisekaran, B. 2004. Iron Industry and Metallurgy - A Study of Ancient Technology. Chennai: New Era Publication.

\bibitem{chap8-key158} Schoff, W.~H. 1912. The Periplus of the Erythrean Sea (translation). New York, London: Longmans Green \& Co.

\bibitem{chap8-key159} Schwartz, C.~R. 1882. Report on the principle prospects of iron working in the Chanda district, Shimla. Quoted in M.~S. Krishnan, 1952, Bulletin of the Geological Survey of India 9: 102.

\bibitem{chap8-key160} Shaffer, J.~G. 1984. Bronze age iron from Afghanistan: its implication for South Asian protohistory. In \textit{Studies in the Archaeology and Palaeoanthropology of South Asia} (Eds.) K.~A.~R. Kennedy and G.~L. Possehl. New Delhi: Oxford and IBH Publishing Company. Pp. 41-62. 

\bibitem{chap8-key161} Shamasastry, R. 1922. \textit{Arthashastra of Kautilya} (3rd Ed.). Mysore: Mysore Sanskrit Series.

\bibitem{chap8-key162} Sharma, A.~K. 1992. Excavations at Gufkral. \textit{Puratattva} 11: 19-25.

\bibitem{chap8-key163} Sharma, A.~K. 1992. Early iron users of Gufkral. In \textit{New Trends in Indian Art and Archaeology} I, (Eds.) B.~U. Nayak and N.~C. Ghose. New Delhi: Aditya Prakashan. Pp. 63-68.

\bibitem{chap8-key164} Sherby, O.~D. and J. Wadsworth. 1985. Damascus steel. Scientific American (Feb. issue): 112-120.

\bibitem{chap8-key165} Sherwani, H.~K. 1965. Golkonda. In Encyclopedia of Islam Vol. II. (Eds.) B. Lewis, C. Pallet and J. Schacht. London: LUZA \& Co. Pp. 1118-1119. 

\bibitem{chap8-key166} Shrivastava, P. 1997. \textit{Prachin Baudh Granthon Mein Varnit Dhatu Evam Dhatukarm} (in Hindi). Varanasi: Ratna Publications.

\bibitem{chap8-key167} Singh, P. 1994. \textit{Excavations at Narhan} (1984-89). Varanasi: Banaras Hindu University.

\bibitem{chap8-key168} Singh, R.~D. 1997. Mining. In \textit{History of Technology in India} Vol. I. (Ed.). A.~K. Bag. New Delhi: Indian National Science Academy. Pp. 48-79.

\bibitem{chap8-key169} Singh, R.~N., I.~C. Glover and J.~F. Merkel. 1998-99. Scientific studies of some iron objects from Senuwar. \textit{Pragdhara} 9: 123-128. Journal of UP State Archaeology Department, Lucknow. 

\bibitem{chap8-key170} Singh, R.~N. and J.~F. Merkel. 2001-01. A study of iron objects from Chirand and Taradih. \textit{Pragdhara} 11: 135-142. Journal of UP State Archaeology Department, Lucknow.

\bibitem{chap8-key171} Smith, C.~S. 1973. Bronze technology in the East. In \textit{Changing Perspectives in the Theory of Science}. (Eds.) M. Young and R. Young. London: Heimendorf.

\bibitem{chap8-key172} Smith, V.~A. 1898. The Dhar iron pillar. \textit{Journal of Royal Asiatic Society of Great Britain and Ireland}: 143-146. 

\bibitem{chap8-key173} Snodgrass, A. 1980. Iron and early metallurgy in the Mediterranean. Coming of the Age of Iron. (Eds.) T.~A.~Wertime and J.~D.~Muhly. New Haven, London: Yale University Press. Pp. 333-374.

\bibitem{chap8-key174} Srinivasan, S. and S. Ranganathan 2004. \textit{India’s Legendary Wootz Steel: An Advanced Material of the Ancient World}. Bangalore: National Institute of Advanced Studies and Indian Institute of Science.

\bibitem{chap8-key175} Stacul, G. 1969. Excavations near Ghaligai (1968) and chronological sequence of protohistorical cultures in the Swat Valley. \textit{East and West} 19: 44-87 and Plates 1 \& 2. 

\bibitem{chap8-key176} Stein, A. 1929. \textit{Archaeological Tour of Waziristan and Northern Baluchistan}. Delhi: Memoirs of Archaeological Survey of India No. 37. 

\bibitem{chap8-key177} Stein, A. 1937. \textit{Archaeological Reconnaissance in Northwestern India and Southeastern Iran}. London: Macmillan and Company.

\bibitem{chap8-key178} Subbarao, B. 1958. \textit{The Personality of India}. Baroda: M.S. University Archaeological Series No.3. 

\bibitem{chap8-key179} Sundara, A. 1975. \textit{The Early Chamber tombs of South India – A Study of the Iron Age Megalithic Monuments of North Karnataka}. Delhi: University Publishers. 

\bibitem{chap8-key180} Sundaram, C.~V., Baldev Raj and C. Rajgopalan. 1999. History of metallurgy in India. \textit{Chemistry and Chemical Techniques in India} (Ed.) B.~V. Subbarayappa. IV (I). In \textit{History of Science, Philosophy and Culture In Indian Civilization} Gen. Ed. D.~P. Chattopadhyay. Delhi: Centre for Studies in Civilizations. Pp. 23-53.

\bibitem{chap8-key181} Swarup, D. and R.~C. Misra. 1944. Indigenous smelting of iron in Mirzapur district, Uttar Pradesh. \textit{Quaternary Journal of Geological, Mining and Metallurgical Society of India} 16(4): 127-131. 

\bibitem{chap8-key182} Tarn, W.~W. 1930. \textit{Hellenistic Civilization}. Bradford: Edward Arnold \& Co.

\bibitem{chap8-key183} Tavernier: 1677. \textit{Travels in India} (French). Translated by Ball in 2 vols: New Delhi: Central Books Corporation (Reprint). 

\bibitem{chap8-key184} Taylor, S.~J. \& C.~A. Shell 1988.  Social and historical implications of Early Chinese iron technology. \textit{Proceedings of BUMA II- 1986}, R. Maddin (Ed.) MIT, USA: 205-221

\bibitem{chap8-key185} Tewari, R. 1998. In quest of early Iron Age sites in Karmanasa valley. \textit{Pragdhara} 8: 57-61. Journal of UP State Archaeology Department, Lucknow. 

\bibitem{chap8-key186} Tewari, R. 2003. The origin of iron working in India: new evidence from the central Ganga plain and the Eastern Vindhyas. \textit{Antiquity} 77: 536-544.

\bibitem{chap8-key187} Tewari, R., R.~K. Srivastava and K.~K. Sinha. 2002. Excavation at Lahuradeva, district Sant (Kabir) Nagar, Uttar Pradesh. \textit{Puratattva} 32: 54-62.

\bibitem{chap8-key188} Thapar, B.~K. 1965. Excavations at Prakash (1955): a chalcolithic site. \textit{Ancient India}. No. 20-21. New Delhi: Archaeological Survey of India. 

\bibitem{chap8-key189} Thomsen, R. 1989. Pattern welded swords from Illerup and Nydam. \textit{Archaeometallurgy of Iron (1967-1987) Symposium}, Liblice (Ed.) R. Pliener. Pp. 371-378.

\bibitem{chap8-key190} Torrens M. 1850. Note on a specimen of iron from the Dhunakar hills, Birbhum. \textit{JRAB} 19: 77-8.

\bibitem{chap8-key191} Toussaint, F. 2002. Iron and steel in ancient India. In \textit{Tradition and Innovation in the History of Iron Making: An Indo European Perspective} (Eds.) Girija Pande and Jan af Geijerstam. Nainital: PAHAR Parikrama. Pp. 39-46. 

\bibitem{chap8-key192} Tripathi, K.~K. 1995. Eran excavations: chalcolithic culture and the iron age. In \textit{Studies in Indian Art and Archaeology}. ('Krshnasmriti', Prof. K.~D. Bajpai Commemorative Volume) (Eds.) R.~K. Sharma and R.C. Agrawal. New Delhi: Aryan Books International. Pp. 58-63. 

\bibitem{chap8-key193} Tripathi, V. 1976. \textit{The Painted Grey Ware – An Iron Age Culture of Northern India}. New Delhi. Concept Publishing Company. 

\bibitem{chap8-key194} Tripathi, V. 1986. From copper to iron - a transition. \textit{Puratattva} 15: 75-79. 

\bibitem{chap8-key195} Tripathi, V. 1994. Prachin Bharat Mein Lauh Yug (Hindi) \textit{Visva Dristi} (Ed.) V. Venkatachalam. Varanasi: Sampurnanand Sanskrit Vishvidyalaya. Pp. 777-88.

\bibitem{chap8-key196} Tripathi, V. 1997. Stages of painted grey ware culture. In \textit{Indian Prehistory: 1980}. (Eds.) V.~D. Mishra and J.~N. Pal. Allahabad: Department of Ancient History, Culture and Archaeology. Pp. 318-322.

\bibitem{chap8-key197} Tripathi, V. 2001. \textit{The Age of Iron in South Asia: Legacy and Tradition}. New Delhi: Aryan Books International. 

\bibitem{chap8-key198} Tripathi, V. and A. Tripathi. 1994. Iron working in ancient India: an ethnoarchaeological study. In \textit{From Sumer to Meluhha}. Wisconsin Archaeological Report, Vol. 3, (Ed.) J.~M. Kenoyer. Wisconsin. Madison. Pp. 241-251.

\bibitem{chap8-key199} Tripathi, V. and P. Upadhyay. 2013 Raipura; an Early Iron Age Site in Vindhya-Kaimur Region, \textit{Puratattva, Jl. of Indian Archaeological Society} Vol. 43: 149-157.

\bibitem{chap8-key200} Tripathi, V. and P. Upadhyay. 2013 \textit{Anai, a Rural Site of Ancient Varanasi}. Delhi, Sharada Publishing House
 
\bibitem{chap8-key201} Tripathi, V. 2012. The Rise of civilization in the Gangetic Plain: The context of Painted Grey ware, New Delhi, Aryan Books International.

\bibitem{chap8-key202} Tripathi, V. 2014. ‘Recently Discovered Iron Working Site in Vindhya-Kaimur Region, India’, \textit{Jl. of Iron and Steel Institute of Japan, International}, (ISIJ International) Vol.54, No.5: 1010-1016

\bibitem{chap8-key203} Tripathi, V. 2018. \textit{Archaeomaterials in Early Cultures of Middle Ganga Plain (Excavations at Khairadih (1996-97)}, District Ballia, U.P. Delhi, Sharada Publishing House, 
 
\bibitem{chap8-key204} Turner, T. 1894. Production of wrought iron in small plant furnaces in India. \textit{Journal of Iron and Steel Institute} 44: 170. 

\bibitem{chap8-key205} Tylecote, R.~F. 1962. \textit{Metallurgy in Archaeology}. London: Edward Arnold. 

\bibitem{chap8-key206} Tylecote, R.~F. 1980. Furnaces, crucibles and slag. In \textit{The Coming of the Age of Iron}. (Eds.) T.~A Wertime and James D. Muhly. New Haven, London Yale University Press. Pp. 183-228.

\bibitem{chap8-key207} Tylecote, R.~F. 1992. A History of Metallurgy. Institute of Materials, UK.

\bibitem{chap8-key208} Tylecote, R.~F., J.~N. Austin and A.~E. Wraith. 1971. The Mechanism of the bloomery process in the shaft furnace. Journal of Iron and Steel Institute 209: 342.

\bibitem{chap8-key209} UP State Archives, Lucknow - Pre mutiny records - KD/MLR: S-1/18, Vol I: 26-27. 

\bibitem{chap8-key210} Upadhyay, P. 2003. \textit{An Ethono-archaeological Study of Mining and Mineral Resources of Cultures of Ganga Plains (700/600 BC-100 BC-AD)}. Thesis submitted for the award of degree of Ph.D. Banaras Hindu University Varanasi, India.

\bibitem{chap8-key211} Upadhyay, P. 2003. \textit{Minerals and Mining in Ancient India (From the Earliest Times to the Beginning of Christian Era)}. Varanasi: Kala Prakashan.

\bibitem{chap8-key212} Upadhyay, P. 2018. Excavations at Latif Shah 2016-17, \textit{Bharati} (Bulletin of Department of AIHC and Archaeology, BHU) 41:150-158.

\bibitem{chap8-key213} Verhoevens, J.~D. and L.~L. Jones. 1987. Damascus Steel. Part I: Indian Wootz Steel. \textit{Metallography} 20(2): 145-151.

\bibitem{chap8-key214} Verhoevens, J.~D. and A.~H. Pendray 1992. A reconstructed method for making Damascus steel blades. \textit{Metals, Materials and Processes}. 4(2): 93-106.

\bibitem{chap8-key215} Voysey, H.~W. 1832. Description of native manufacturer of steel in South India. \textit{Journal of Asiatic Society of Bengal} 1: 245-247.

\bibitem{chap8-key216} Waldbaum, J.~C. 1978. \textit{From bronze to iron: the transition from the bronze Age to iron age in the Eastern Mediterranean}. Goteborg: Paul Astroms Forlag.

\bibitem{chap8-key217} Waldbaum, J. 1980. The first archaeological appearance of iron and the transition to the iron age. In \textit{The Coming of the Age of Iron}. (Eds.) T.~A Wertime and James D. Muhly. New Haven, London: Yale University Press. Pp. 69­-98.

\bibitem{chap8-key218} Wertime, T.~A. 1980. The pyrotechnologic background. In \textit{The Coming of the Age of Iron}. (Eds.) T.~A Wertime and James D. Muhly. New Haven, London: Yale University Press. Pp. 1-24.

\bibitem{chap8-key219} Wheeler, R.~E.~M. 1958. Foreword to B. Subbarao. \textit{Personality of India}. Baroda: M.~S. University of Baroda.

\bibitem{chap8-key220} Willies, L. 1987: Ancient Zinc-lead-silver Mining in Rajasthan, Interim Report, \textit{Bulletin of the Peak District Mines Historical Society} 10 (2): 81-124.

\bibitem{chap8-key221} Wood, C. 1894. 'Discussion'  \textit{Journal of the Iron and Steel Institute}.  X LIV: 179.

\bibitem{chap8-key222} Yater, W.~M. 1983. The legendary steel of Damascus: forging, pattern development and heat treatment. Pt. III. \textit{The Anvils Ring} 11(4): 1-17.

\bibitem{chap8-key223} Young, T.~C. 1976. The Iranian migration into the Zagros. \textit{Iran} 5: 11-34. 

\end{thebibliography}


\label{endchapter8}
