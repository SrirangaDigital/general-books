\chapter{The Non-translatables\index{Non-translatables@Non-translatables} of\break (South) Indian Music}\label{chapter2}

\Authorline{K. Vrinda Acharya\footnote{pp. 49--75. In: Meera, H. R. (Ed.) (2020). \textit{Karnāṭaka Śāstrīya Saṅgīta - Past, Present, and Future.} Chennai: Infinity Foundation India.}}

\vspace{-.3cm}

\lhead[\small\thepage\quad K. Vrinda Acharya]{}

\begin{flushright}
\textit{(vrindacharya@gmail.com)}
\end{flushright}


\section*{Abstract}

The core concepts related to Indian knowledge systems and art forms, most of which are terms in Saṁskṛta, have very profound meanings and often deeply embedded in their cultural frameworks. They not only encode specific and unique cultural experiences and traits, like all languages, but the very form, sound and manifestation of the language carry effects that cannot be separated from their conceptual meanings.

However, there has been a tendency of translating and mapping such concepts and perspectives onto Westerns frameworks. Dharmic traditions and wisdom embodied in Saṁskṛta language get compromised, shrunk or even demolished once they are substituted with the alleged Western equivalents which are in no way adequate to precisely represent the original ideas. Concepts like \textit{dharma,\index{dharma@\textit{dharma}} yoga, ātman,\index{atman@\textit{ātman}}\index{yoga@\textit{yoga}} guru}\index{guru@\textit{guru}} and so on often get loosely/badly/wrongly translated, which leads to serious loss of meaning and ultimately aids the undesirable digestion\index{cultural digestion@cultural digestion} and appropriation.\index{appropriation@appropriation} Thus, the non-translatability\index{Non-translatables@Non-translatables} of Saṁskṛta concepts and terms is most imperative for the non-digestibility of Hindu traditions, cultural practices, knowledge systems and art forms into the Abrahamic framework.

\textit{Karnāṭaka Śāstrīya Saṅgīta} or South Indian Music is not just an art form, but a complex music culture that is fundamentally integrated with its Vedic/\textit{Sanātana Dhārmic} roots and with Hindu culture and ethos. In fact, no classical art form with a long history anywhere in the world can be viewed as separate from the cultural, historical context in which it is born, grown and sustained; and Karnatic\index{Karnatic@Karnatic} music is no exception to this. It has always been an inseparable aspect of Hindu civilization, right from its origin in \textit{Sāmaveda},\index{Samaveda@\textit{Sāmaveda}} through Śārṅgadeva,\index{Sarngadeva@Śārṅgadeva} to its evolution during the Trimūrti era and all the way to the current times. The various \textit{rāga}\index{raga@\textit{rāga}} and \textit{tāla}\index{tala@\textit{tāla}} names, technical terms like \textit{nāda,\index{nada@\textit{nāda}} śruti,\index{sruti@\textit{śruti}} rāga, mela,\index{mela@\textit{mela}} laya}, etc., which have philosophical/spiritual/metaphysical significance, are all essentially non-translatables\index{Non-translatables@Non-translatables} since they are invariably tied to \textit{Sanātana Dharma}\index{Sanatana Dharma@\textit{Sanātana Dharma}}\textit{Dharma}\index{Sanatana Dharma@\textit{Dharma}} at various levels and cannot be translated to any other language.

This paper aims to identify and elaborate upon twelve such major technical terms associated with (South) Indian Music\endnote{The paper could as well be titled ‘Non-translatables\index{Non-translatables@Non-translatables} of Karnatic\index{Karnatic@Karnatic} Music’, but most of the terms I take up here are applicable to Indian music at large (both Hindustani\index{Hindustani@Hindustani} and Karnatic) but only analysed from the point of view of South Indian/Karnatic Music.}, to examine their roots, to understand what they mainly convey and signify, and to elucidate their strong connect with the quintessence of our musical culture over the centuries. How some of the terminologies even transcend strict linear definitions, but can only be comprehended by deep experience is highlighted. Also, the commonly used English translations of these terms are taken up to analyse if at all they are any close to the originals, let alone denote them precisely.

\vspace{-.2cm}

\section*{Key words}

\vspace{-.2cm}

Karnatic Music, South Indian Music, Hindu Art forms, Non-translata\-bles,\index{Non-translatables@Non-translatables} Sanskrit, Indian Art, Samaveda, Nada, Shruti, Raga, Tala, Gamaka,\index{gamaka@\textit{gamaka}} Mela,\index{mela@\textit{mela}} Sangati,\index{sangati@\textit{sangati}} Vaggeyakara,\index{vaggeyakara@\textit{vāggeyakāra}} Sanatana Dharma, Swadeshi Indology, Vrinda Acharya

\vspace{-.2cm}

\section*{Introduction}

\vspace{-.2cm}

The core concepts related to our knowledge systems and art forms, most of which are Saṁskṛta terms, have very profound meanings and often deeply embedded in their cultural frameworks. These terms have deep spiritual, metaphysical, scientific, historical and/or cultural significance which makes it almost impossible to find corresponding terms in any other language, particularly a foreign one like English. In fact, the so-called equivalent terms are not only highly incapable of representing the originals, but also lead to dilution, distortion and even demolition of the intensity and profundity of the concepts. Malhotra\index{Malhotra, Rajiv@Malhotra, Rajiv} (2013: 9) observes that the dharmic traditions and wisdom embodied in Saṁskṛta language get compromised or even obliterated once they are substituted with Western equivalents which are not capable of accurately representing the original ideas.

Unfortunately, there has been a tendency of translating and mapping such concepts and perspectives onto Westerns frameworks. Western scholars, sometimes due to their genuine lack of deeper understanding of all dimensions of our culture; sometimes with a deliberate intent to overlook, ignore, deny or not acknowledge its greatness, richness, complexity and depth; but mostly with an ulterior motive and organised scheme to dilute our recondite concepts (and to ultimately appropriate\index{appropriation@appropriation} and digest\index{cultural digestion@cultural digestion} the same), often come up with the ‘well-known’ argument of ‘Sameness’! Furthermore, many Indians who are either naive or take great pride in being ‘Westernised’ in their thought and conduct, easily subscribe to this ‘Sameness’ principle and go on to claim that “all religions are same”, “all Gods are same”, “all scriptures say the same things, “everything is the same.. what difference does it make?” and so on. As a matter of fact, it has been drilled into our heads that attributing universality to everything is noble! In this way, the dominant/colonial culture gradually and subtly imposes its own superficial translations of weighty concepts, which the natives of the colonised culture also conveniently adopt.

For instance, let us take the term ‘\textit{dharma}’. \textit{Dharma},\index{Dharma@\textit{Dharma}} the vital concept of Indian philosophy, is a very broad idea which has no single definition. It is derived from the root ‘\textit{dhṛ dhāraṇe}’ which means ‘that which holds, supports, sustains, maintains’ –“\textit{dhāraṇād dharma ity āhuḥ}”(\textit{Mahābhārata}\index{Mahabharata@\textit{Mahābhārata}} 8.69.58). So, essentially it means the eternal unvarying cosmic law, inherent in the very nature of things. In short, it means the all-pervading cosmic order that sustains the entire creation. However, \textit{dharma} is often mistranslated as ‘Religion’ which is a purely Abrahamic concept. Religion is worship of divine that is separate from human, and is governed by a religious institution/authority. It consists of formal members and follows a standard set of rituals. 

Similarly, as per the dharmic tradition, \textit{ātman}\index{atman@\textit{ātman}} is one’s true Self, the nature of which is \textit{sat-cit-ānanda} and which can be realised through \textit{adhyātma sādhanā}. It is a reflection of the Supreme Self or \textit{Brahman}\index{brahman@\textit{Brahman}} that is present in humans, plants, animals and all creatures – “\textit{ayam ātmā brahma}” (\textit{Bṛhadāraṇyaka Upaniṣad}\index{Upanisad-s@Upaniṣad-s!\textit{Bṛhadāraṇyaka Upaniṣad}}\index{Brhadaranyaka Upanisad@\textit{Bṛhadāraṇyaka Upaniṣad}} 4.4.5) Hinduism believes in the theory of \textit{karman}\index{karma See karman@\textit{karma} See \textit{karman}}\index{karman@\textit{karman}} and reincarnation of \textit{ātman}. This is totally opposed to the Judeo-Christian notion of ‘Soul’ or ‘Spirit’ according to which all humans are born as sinners due to the initial sin by Adam and Eve. Hence the very nature of soul is sinful and only God can save it. Soul is present only in humans and there is neither \textit{karman} nor rebirth. Thus, it is clear that soul is not a synonym for \textit{ātman}.

Likewise, \textit{guru}\index{guru@\textit{guru}} is not teacher, \textit{kāla} is not time, \textit{pūjā}\index{puja@\textit{pūjā}} is not prayer/worship and so on. Many such pure Indic terms like \textit{prāṇa,\index{prana@\textit{prāṇa}} saṁskāra,\index{samskara@\textit{saṁskāra}} varṇa},\index{varna@\textit{varṇa}} etc., do not have any equivalents whatsoever in the Western vocabulary.

In this way, dharmic aspects often get loosely/badly/wrongly translated, which leads to serious loss of meaning and eventually aids the undesirable cultural digestion\index{cultural digestion@cultural digestion} and appropriation.\index{appropriation@appropriation} Thus, it is extremely important that the Saṁskṛta terms are retained in their original forms to ensure that they are associated with their origins and to help in preserving their authenticity. The non-translatability\index{Non-translatables@Non-translatables} of Saṁskṛta terms\endnote{It is important to note that the word ‘Saṁskṛta’ (meaning \textit{samyak kṛtam} or well made) itself is a non-translatable, as the word ‘Sanskrit’ is inadequate to represent what ‘Saṁskṛta’ actually stands for.} is important for the non-digestibility of Hindu traditions, cultural practices, knowledge systems and art forms into the Abrahamic framework.

\begin{myquote}
“Many Sanskrit words are simply not translatable. This non-translatability of key Sanskrit words attests to the non-digestibility of many Indian traditions. Holding on to the Sanskrit terms and thereby preserving the complete range of their meanings becomes a way of resisting colonization and safeguarding dharmic knowledge.” 

~\hfill (Malhotra\index{Malhotra, Rajiv@Malhotra, Rajiv} 2013: 220)
\end{myquote}

\begin{myquote}
“A translation of foundational Indian works from Indic to European languages is not as simple and transparent as Indic-to-Indic (e.g. Sanskrit to Tamizh). This is due to the presence of a large number of important non-translatables within Sanskrit and Indian languages that are not part of the western vocabulary, tradition, or psyche. These non-translatables end up getting mangled in the attempted translation, resulting in distorted output text filled with misleading interpretations that are biased in favour of the dominant cultural (western) perspective.…Equivalent Tamizh words for crucial Sanskrit terms existed since ancient times; if not Sanskrit terms were retained ‘as is’ along with their full range of meanings….This non-translatable\index{Non-translatables@Non-translatables} ecosystem is also a beautiful shield that helped protect Sanskriti for thousands of years. It is also protecting Tamizh Kalacharam and preserving the distinctiveness of India’s diverse regional cultures...” 

~\hfill (Tamizh Cultural Portal 2016)
\end{myquote}


\section*{Saṁskṛta Language}

No doubt all cultures have their distinct and exclusive ideas, attributes and experiences which are only best expressed in their native languages; and any translations into other languages and direct importations into other religious or cultural frameworks will not sufficiently convey the original idea. However, this aspect is all the more predominant with respect to Indian civilization owing to the inimitable nature of Saṁskṛta language. Saṁskṛta is indeed the bed rock of our \textit{saṁskṛti}\index{samskrti@\textit{saṁskṛti}} as proclaimed by the popular saying “\textit{saṁskṛtiḥ saṁskṛtāśritā}”. All branches of knowledge, philosophy, cultural practices, art forms, literature, rituals, temple traditions, festivals and so on are manifestations of this \textit{saṁskṛti} and the related Saṁskṛta terms not only encode specific and unique cultural experiences and traits (like that of any other language), but the very form, sound and manifestation of the language carry effects that cannot be separated from their conceptual meanings. Moreover, Saṁskṛta is a language based on sacred sounds and vibrations that were realised by \textit{ṛṣi}-s from deeper and higher states of realisation.

\begin{myquote}
“The sacred sounds that comprise the Sanskrit language were discovered by India’s rishis of the distant past through their inner sciences. These sounds are not arbitrary conventions but were realized through spiritual practice that brought direct experiences of the realities to which they correspond. Numerous meditation systems were developed by experimenting\index{experimentation@experimentation} with these sounds, and thus evolved the inner sciences that enable a practitioner to return to a primordial state of unity consciousness. Sanskrit provides an experiential path back to its source. It is not just a communications tool but also the vehicle for embodied learning. Employed by the spiritual leaders of India, South-east Asia and East Asia for many centuries as language, Sanskrit became the medium of expressing a distinct set of cultural systems and experiences.” 

~\hfill (Malhotra\index{Malhotra, Rajiv@Malhotra, Rajiv} 2013: 9)
\end{myquote}

\newpage

Saṁskṛta is a language that is quite unique, special and different when compared to other languages in many respects. Saṁskṛta vocabulary is amenable to etymological explanasions, and most words used are well-defined. Words and their meanings are derived from roots called ‘\textit{dhātu}-s’\index{dhatu@\textit{dhātu}} and there is a very logical and methodical grammatical process that follows before the words are finally formed.~Thus, the meanings of words are very ‘meaningful’ and not just random, customary sounds. This makes it all the more a reason for original Saṁskṛta words to be retained, as one will be bound to think what the words mean actually. For, instance, since \textit{mantra} is not the same as hymn, it compels one to know and understand what exactly a \textit{mantra} is\endnote{\textit{mananāt trāyate iti mantraḥ} - Being \textit{apauruṣeya} in origin, a \textit{Mantra} is a sound, phrase or a verse capable of producing enormous spiritual effects upon repeated chanting and contemplation. It cannot be altered or mispronounced, as it has an element of great sanctity attached to it. There is also something secretive about it, since it cannot be taught to everybody so that misuse and abuse can be avoided.}, where and why it differs, and why the difference matters a lot.

\vspace{-.3cm}

\section*{Non-translatables\index{Non-translatables@Non-translatables} of (South) Indian Music}

\textit{Karnāṭaka Śāstrīya Saṅgīta} or South Indian Music (Karnatic\index{Karnatic@Karnatic} or Carnatic Music as it is popularly called) is the present evolved form of the ancient music system of India. It is known for its sublimity and uniqueness, and stands apart when compared to other forms of music across the globe. It is not some kind of an art born in a void, but a complex music culture that is fundamentally integrated with its Vedic/ \textit{Sanātana} Dhārmic roots and with Hindu culture and ethos. In fact, music in India has always been positioned as a spiritual practice having a sacred dimension, that provides an ‘experience’ and ‘elevates’ both the practitioner (\textit{sādhaka})\index{sadhaka@\textit{sādhaka}} as well as the listener (\textit{rasika}),\index{rasika@\textit{rasika}} rather than being perceived as a mere means of ‘entertainment’. It is rather impossible to think of it divorced from this essential background. Also,

\begin{myquote}
“Today, the peninsular South of India is the only region in the whole world where music is inseparable from life and literature. The evolution of Raga\index{raga@\textit{rāga}} and Kriti, of rhythm\index{rhythm@rhythm} and dance, is a fascinating story that has to be pieced together from the profuse literature of the ages…” 

~\hfill (Ayyangar\index{Ayyangar, Rangaramanuja} 1972: vii)
\end{myquote}

The various \textit{rāga} and \textit{tāla}\index{tala@\textit{tāla}} names, nomenclatures and technical terms like \textit{nāda,\index{nada@\textit{nāda}} śruti,\index{sruti@\textit{śruti}} rāga, mela,\index{mela@\textit{mela}} laya,\index{laya@\textit{laya}} gamaka,\index{gamaka@\textit{gamaka}} saṅgati,\index{sangati@\textit{saṅgati}} kāla,\index{kala@\textit{kāla}} kālapramāṇa,\index{kalapramana@\textit{kālapramāṇa}}\break manodharma,\index{manodharma@\textit{manodharma}} anuloma,\index{anuloma@\textit{anuloma}} pratiloma,\index{pratiloma@\textit{pratiloma}} avadhāna},\index{avadhana@\textit{avadhāna}} etc., which have signifi\-cance philosophically/spiritually/metaphysically, are all essentially non-\break translatables since they are invariably tied to \textit{Sanātana Dharma}\index{Sanatana Dharma@\textit{Sanātana Dharma}}\index{Dharma@\textit{Dharma}} at various levels and cannot be represented by one-word translations of any other language.

Sambamoorthy\index{Sambamoorthy, P.@Sambamoorthy, P.} (2006: 1,2) points out that Indian music is noted for its extensive and rich nomenclature, which is a proof of the sound and comprehensive development of art in all its diverse branches. The names and terms of classification of \textit{rāga}-s,\index{raga@\textit{rāga}} \textit{tāla}-s,\index{tala@\textit{tāla}} musical forms, musical instruments, and the technical terms used to denote the \textit{svara}-s,\index{svara@\textit{svara}}\break \textit{śruti}-s,\index{sruti@\textit{śruti}} \textit{gamaka}-s\index{gamaka@\textit{gamaka}} and techniques of playing instruments testify to the scientific and analytical genius of Indian musicologists.

\begin{myquote}
“Sanskrit is the mother of all languages in the world, barring one or two. In the same manner, the notes of the octave are the contribution of Indian genius to world music. In fact, the place of Indian music in comparative musicology is the same as that of Sanskrit in comparative philology. Both Sanskrit and Indian music offer material of world-wide importance for their scientific as well as aesthetic values.” 

~\hfill (Ayyangar\index{Ayyangar, Rangaramanuja} 1972: 69)
\end{myquote}

It is however not a good thing that the entire gamut of Indian musical terminology is being denoted by simplistically, shallowly, casually, freely and loosely translated (supposedly equivalent) English terms even in books written by Indian authors/musicologists as well as commonly used by Indian musicians. This may be due to want of better terms, nevertheless these translated words are totally incapable of representing the complete essence, dimension and depth of the original terms. 

The purpose of this paper is to identify such concepts and technical terms associated with Karnatic\index{Karnatic@Karnatic} Music, to examine their roots, to understand what they essentially convey and signify, and to show how they are intertwined with the quintessence of our musical culture over the centuries. The focus will be to understand how some of the terminologies even transcend strict linear definitions, and can only be comprehended by deep experience. The paper also intends to throw light on the commonly used English translations of these terms and to analyse if at all they are any close to the originals, let alone denote them precisely.

Here I elaborate upon twelve major technical terms related to Indian music in general and Karnatic Music in particular. Definitions are taken from Bharata’s\index{Bharata@Bharata} \textit{Nāṭyaśāstra},\index{Natyasastra@\textit{Nāṭya-śāstra}} Mataṅga’s\index{Matanga@\textit{Mataṅga}} \textit{Bṛhaddeśī},\index{Brhaddesi@\textit{Bṛhaddeśī}} Cālukya Someśvara’s\index{Calukya Somesvara@\textit{Cālukya Someśvara}} \textit{Mānasollāsa},\index{Manasollasa@\textit{Mānasollāsa}} Śārṅgadeva’s\index{Sarngadeva@\textit{Śārṅgadeva}} \textit{Saṅgītaratnākara}\index{Sangitaratnakara@\textit{Saṅgītaratnākara}} (hereafter SR) and Pārśvadeva’s\index{Parsvadeva@\textit{Pārśvadeva}} \textit{Saṅgītasamayasāra}\index{Sangitasamayasara@\textit{Saṅgītasamayasāra}} (hereafter SSS) which are considered as the most authoritative treatises on Indian music. \textit{The Harvard Dictionary of Music} (hereafter HDM) and \textit{Music: An Appreciation} (hereafter Kamien (1980)) have been referred to for the definitions of the Western counterparts of these chosen terms.

\subsection*{1. \textit{Nāda}\index{nada@\textit{nāda}} ≠ Sound}

The very first \textit{pāribhāṣika pada} that any student of Indian music learns is ‘\textit{Nāda}’.

\begin{verse}
\textit{na-kāraṁ prāṇa-nāmānāṁ da-kāram analaṁ viduḥ \dev{।}}\\ \textit{jātaḥ prāṇāgni-saṁyogāt tena nādo’bhidhīyate \dev{।।}} 

~\hfill (\textit{SR}\index{Sangitaratnakara@\textit{Saṅgītaratnākara}} 1.3.6)
\end{verse}

\begin{myquote}
“The letter ‘\textit{na}’ represents \textit{Prāṇa}\index{prana@\textit{prāṇa}} or life force and the letter ‘\textit{da}’ denotes \textit{Agni} or fire. Having born out of the union of \textit{Prāṇa} and \textit{Agni}, it is thus named as ‘\textit{nāda}’.” 

~\hfill (\textit{Translation my own})
\end{myquote}

Indeed, this \textit{nāda}, originating from the primordial ‘\textit{om}’ or ‘\textit{praṇava}’, is the all-pervading eternal musical sound of the universe, from which originates all beings and all speech. It is the minutest, atom-like sound\endnote{Though I am set to prove that \textit{Nāda}\index{nada@\textit{nāda}} is not equal to sound, I have no choice but to use the same English word for explaining \textit{Nāda}, since English language is the medium of this paper. This unavoidability of using the very same English non-equivalents applies to other terms as well.}. It is identified with the \textit{bindu}\endnote{ References to this \textit{Nādabindu} can be found abundantly in many compositions. For instance, a Tiruppugaḻ by Aruṇagirināthar\index{Arunagirinathar@Aruṇagirināthar} ‘\textit{nādavindukalādi namo namo}’ and a \textit{kṛti}\index{krti@\textit{kṛti}} by Muttusvāmi Dīkṣitar\index{Muttusvami Diksita@Muttusvāmi Dīkṣita} ‘\textit{Jambūpate mām pāhi}’ (where he says \textit{anirvacanīya nādabindo})} (\textit{yato bindus tato nādaḥ}) which is the root or the centre point of the entire creation. It is undying (\textit{nitya})\endnote{\textit{yo nādas sarva-bhūtānāṁ sarva-varṇasya cāṅkuraḥ \dev{।}}\\ \textit{yo bījaṁ mantra\index{bija mantra@\textit{bīja-mantra}}-koṭīnāṁ taṁ nityaṁ praṇamāmy ahaṁ \dev{।।}}} and inexplicable (\textit{anirvacanīya}). Empirically, it can be described as a continuous, sustained, melodious, musical sound (\textit{avicchinna, dīrgha} and \textit{madhura}). It is \textit{nāda} that gives rise to \textit{śruti}-s,\index{sruti@\textit{śruti}} and these give rise to \textit{svara}-s\index{svara@\textit{svara}} and these again to \textit{rāga}-s.

\textit{Nāda} is of two types, namely \textit{āhata} and \textit{anāhata}. (\textit{āhato’nāhataś ceti dvidhā nādo nigadyate – SR} 1.2.3) Sampatkumaracharya and Ramaratnam (2000: 29) explain that, a \textit{nāda} that is produced by the conscious effort of man and is heard externally is \textit{āhata}. \textit{Anāhata}, on the other hand is internal, unstruck, non-vibratory and mystical; and is audible only to great \textit{sādhaka}-s\index{sadhaka@\textit{sādhaka}} with transcendental and experiential yogic insights.

In India, the Supreme Being has throughout been regarded as the personification, creator and lover of beauty and therefore, the aim of all forms of art (and more so music) is to experience ‘Him’. Lord Kṛṣṇa’s proclamation in the \textit{Bhagavadgīta}\index{Bhagavadgita@\textit{Bhagavadgīta}} (10.22)- “\textit{vedānāṁ sāmavedo’smi}” and the \textit{Viṣṇusahasranāman}\index{Visnusahasranaman@\textit{Viṣṇusahasranāman}} (106) calling him “\textit{sāmagāyanaḥ}” substantiate this beyond question. This fundamental notion gives rise to one of the conceptions of God i.e. as ‘\textit{Nāda-brahman}’,\index{brahman@\textit{Brahman}} which is India’s contribution to world thought. Sambamoorthy\index{Sambamoorthy, P.@Sambamoorthy, P.} (2005: 14) aptly observes that God is conceived of as \textit{Nādabrahman}\index{Nadabrahman@\textit{Nādabrahman}} – the embodiment of \textit{nāda}; and through \textit{Nādopāsanā}\index{nadopasana@\textit{nādopāsanā}} or musical meditation (\textit{saṅgītopāsanā}), one can attain celestial bliss.

\begin{verse}
\textit{caitanyaṁ sarva-bhūtānāṁ vivṛttaṁ jagadātmanā} \dev{।}\\ \textit{nādabrahma tadānandam advitīyam upāsmahe \dev{।।}}
\end{verse}

\vspace{-.5cm}

\begin{verse}

~\hfill (\textit{SR}\index{Sangitaratnakara@\textit{Saṅgītaratnākara}} 1.3.1)
\end{verse}

\begin{verse}
\textit{nādopasanayā devā brahma-viṣṇu-maheśvarāḥ \dev{।}}\\ \textit{bhavanty upāsitā nūnaṁ yasmād ete tadātmakāḥ \dev{।।}}
\end{verse}

\vspace{-.5cm}

\begin{verse}

~\hfill (\textit{SR} 1.3.2)
\end{verse}

\begin{myquote}
“I worship that blissful and matchless \textit{Nādabrahman}, which is the very consciousness in all beings and which manifests itself as the cosmos. The Gods Brahma, Viṣṇu and Śiva are indeed embodiments of \textit{Nāda} as they are worshipped through the worship of \textit{nāda}.” 

~\hfill (\textit{Translation my own})
\end{myquote}

Sadguru Tyāgarāja,\index{Tyagaraja@Tyāgarāja} in an unprecedented manner draws our attention to the transcendental effects of \textit{nādopasanā}. In scores of his compositions in words such as – ‘\textit{Nādaloluḍai brahmānanda môndave}’, ‘\textit{Praṇava-nādasudhā\index{pranavanada@\textit{praṇavanāda}} rasambilanu narākṛtiyāye}’, ‘\textit{Divyagānamūrte}’, ‘\textit{Nādopāsanace\break śaṅkara-nārāyaṇa-vidhulu vêlasiri}’, ‘\textit{Nādātmaka tyāgarāja}’, ‘\textit{Nādatanum\break aniśaṁ śaṅkaraṁ namāmi}’, etc., he highlights the greatness of worshipping the absolute music or \textit{Praṇavanāda}.

\textit{Nāda} is often plainly translated as ‘Sound’. Sound represents only \textit{śabda} (not in the sense of a word) and not \textit{nāda}. No doubt \textit{nāda} is also sound basically, but any sound cannot be considered \textit{nāda}. Only a pure musical sound which can be perceived in the aforesaid philosophical background is \textit{nāda}.\index{nada@\textit{nāda}} It is also noteworthy that no such (or similar) concept is found in any of the books related to Western musicology.


\subsection*{2. \textit{Śruti}\index{sruti@\textit{śruti}} ≠ Semitone\index{semitones@semitones}}

\textit{Śruti} literally means that which is heard. “\textit{śrūyante iti śrutayaḥ}”. While Dattila\index{Dattila@Dattila} says ‘\textit{śravaṇāt śruti-saṁjñitā}’, Mataṅga\index{Matanga@Mataṅga} elucidates thus

\begin{verse}
\textit{śru śravaṇe cāsya dhātoḥ ktin-pratyaya-samudbhavaḥ \dev{।}}\\ \textit{śravaṇendriya-grāhyatvāt dhvanir eva śrutir bhavet \dev{।।}} 
\begin{flushright}
(\textit{Bṛhaddeśī}\index{Brhaddesi@\textit{Bṛhaddeśī}} 1.3.1,2)
\end{flushright}
\end{verse}

\begin{myquote}
“Being derived from the dhātu\index{dhatu@\textit{dhātu}} ‘śru śravaṇe’ and with ktin-pratyaya, the word śruti\index{sruti@\textit{śruti}} is thus originated. Any tone or sound that can be perceived through the organ or sense of hearing becomes ‘śruti’.” 

~\hfill (\textit{Translation my own})
\end{myquote}

The nature of this \textit{śruti} is thus explained in a \textit{kārikā} to \textit{Nāradīyaśikṣa}\index{Naradiyasiksa@\textit{Nāradīyaśikṣa}}\index{Siksa@\textit{Śikṣa}} -

\begin{verse}
\textit{prathamaśravaṇācchabdaḥ śrūyate hrasvamātrakaḥ \dev{।}}\\ \textit{sā śrutiḥ saṁparijñeyā svarāvayavalakṣaṇā \dev{।।} } 
\begin{flushright}
(Ref p31 of \textit{Nāradīyaśikṣā})
\end{flushright}
\end{verse}

\begin{myquote}
“When the sound is heard at first, only short vowel is heard. That should be known as \textit{śruti}, the one which has \textit{svara}\index{svara@\textit{svara}} as its part.” 

~\hfill (\textit{Translation my own})
\end{myquote}

Rajarao\index{Rajarao, L. Mysore@Rajarao, L. Mysore} (1963: 5) explains that the most subtle and minute \textit{nāda}\index{nada@\textit{nāda}} that is audible to the ear is termed as \textit{śruti}; and broadly, it can be defined as the least difference between two consecutive \textit{svara}-s. He further says that, though in principle it can be said that \textit{śruti}-s are infinite, most ancient musicologists have accepted \textit{śruti}-s to be twenty-two in number and that they are distributed over a \textit{sthāyi}\index{sthayi@\textit{sthāyi}} or \textit{svara-saptaka} (called an octave). \textit{SR}\index{Sangitaratnakara@\textit{Saṅgītaratnākara}} says “\textit{tasya dvāvimśatir bhedāḥ śravaṇācchrutayo matāḥ}” (1.3.8) These twenty-two \textit{śruti}-s are discretely named and their exact frequencies are also ascertainable; only the names given in different texts written over different periods of time in the history of Indian music differ.

\begin{longtable}{|p{1.3cm}|p{3cm}|p{4cm}|}
\hline
\textbf{\textit{Śruti}\index{sruti@\textit{śruti}} number} & \textbf{\textit{Nāṭyaśāstra/SR}\index{Natyasastra@\textit{Nāṭya-śāstra}} nomenclature} & \textbf{Names used in recent times} \\
\hline
1 & \textit{Tīvrā} & \textit{Ṣaḍja} \\
\hline
2 & \textit{Kumudvatī} & \textit{Ekaśruti Ṛṣabha} \\
\hline
3 & \textit{Mandā} & \textit{Dviśruti Ṛṣabha} \\
\hline
4 & \textit{Chandovatī} & \textit{Triśruti Ṛṣabha} \\
\hline
5 & \textit{Dayāvatī} & \textit{Catuśruti Ṛṣabha} \\
\hline
6 & \textit{Rañjanī} & \textit{Komala Gāndhāra} \\
\hline
7 & \textit{Raktikā/Ratikā} & \textit{Sādhāraṇa Gāndhāra} \\
\hline
8 & \textit{Raudrī} & \textit{Antara Gāndhāra} \\
\hline
9 & \textit{Krodhā} & \textit{Tīvra Antara Gāndhāra} \\
\hline
10 & \textit{Vajrikā} & \textit{Śuddha Madhyama} \\
\hline
11 & \textit{Prasāriṇī} & \textit{Tīvra Śuddha Madhyama} \\
\hline
12 & \textit{Prīti} & \textit{Prati Madhyama} \\
\hline
13 & \textit{Mārjanī} & \textit{Tīvra Prati Madhyama/ Cyuta Pañcama} \\
\hline
14 & \textit{Kṣiti} & \textit{Pañcama} \\
\hline
15 & \textit{Raktā} & \textit{Ekaśruti Dhaivata} \\
\hline
16 & \textit{Sandīpanī} & \textit{Dviśruti Dhaivata} \\
\hline
17 & \textit{Ālāpinī} & \textit{Triśruti Dhaivata} \\
\hline
18 & \textit{Madantī} & \textit{Catuśruti Dhaivata} \\
\hline
19 & \textit{Rohinī} & \textit{Komala Kaiśikī Niṣāda} \\
\hline
20 & \textit{Ramyā} & \textit{Kaiśikī Niṣāda} \\
\hline
21 & \textit{Ugrā} & \textit{Kākali Niṣāda} \\
\hline
22 & \textit{Kṣobhiṇī} & \textit{Tīvra Kākali Niṣāda/ Cyuta Ṣaḍja} \\
\hline
\end{longtable}

\begin{flushright}
(Sampatkumaracharya and Ramaratnam 2000: 34)
\end{flushright}

Though this is the actual technical meaning of the term \textit{śruti}, it is also used in another sense in common parlance. It is used to mean the pitch or base key that is chosen by a musician to render his/her music. Vasanthamadhavi\index{Vasanthamadhavi@Vasanthamadhavi} (2005: 2) points out that the range in which a person’s voice is easily negotiable in three octaves is also called the \textit{śruti}\index{sruti@\textit{śruti}} of that voice. 

Parallelly, Kamien (1980:\;62) says “No matter how often a piece changes key, there usually is one main key, called the tonic or home key. The tonic key is the central key around which the whole piece is organized”. Thus, the word ‘pitch’ or ‘tonic’ used as an equivalent for \textit{śruti} makes sense only in the context of this common usage and not in the context of the deeper original concept.

There is another term called ‘Semitone’\index{semitones@semitones} used in Western music, which is defined by the HDM (2003: 768) thus “The smallest interval in use in the Western musical tradition. There are twelve such intervals to the octave i.e., between two pitches with the same pitch name. The semitone is represented on the piano keyboard by the distance between any two immediately adjacent keys, whether white or black.” So, it is the interval between two consecutive notes in a twelve-tone scale.\index{scale@scale} For example, the interval between C and C\#. But, this idea of twelve Semitones (which can only be equated to the twelve \textit{svara-sthāna}-s\index{svarasthana@\textit{svara-sthāna}} or the \textit{prakṛti\index{prakrtisvara@\textit{prakṛti svara}}-vikṛti\index{vikrtisvara@\textit{vikṛti svara}} svaraprabheda}-s in an octave) is far from the concept of twenty-two \textit{śruti}-s\index{sruti@\textit{śruti}} which is unique to Indian music. Western music theoretically recognizes microtones\index{microtones@microtones} or quartertones,\index{quartertones@quartertones} which are defined by the same dictionary (2003: 509, 697) as “an interval smaller than a semitone” and “an interval equal to half of a semitone” respectively. Yet, as cited earlier, the smallest interval in practical use in Western Music is only a Semitone; and microtones or quartertones are not so precisely defined and demonstrable and thus not as clear and comprehensible as our concept of twenty-two \textit{śruti}-s. On the other hand, it is remarkable that twenty-two \textit{śruti}-s is not just a theoretical recognition, but Indian music indeed uses these twenty-two \textit{śruti}-s.


\subsection*{3. \textit{Svara} ≠ Note}

A \textit{Nāda}\index{nada@\textit{nāda}} having a definite single frequency is termed as \textit{svara}.\index{svara@\textit{svara}} “\textit{svato rañjayati śrotṛ-cittaṁ sa svara ucyate}” (SR 1.3.25) meaning “that which pleases the mind of the listener on its own, is called a \textit{svara}.”

\begin{myquote}
\textit{śrutibhyaḥ syuḥ svarāḥ ṣaḍjarṣabha-gāndhāra-madhyamāḥ \dev{।}}\\ \textit{pañcamo dhaivataś cātha niṣāda iti sapta te \dev{।।}}\\ \textit{teṣāṁ saṁjñāḥ sa-ri-ga-ma-pa-dha-nīty aparā matāḥ \dev{।}} 

~\hfill (\textit{SR}\index{Sangitaratnakara@\textit{Saṅgītaratnākara}} 1.3.23,24)
\end{myquote}

\begin{myquote}
“\textit{Śruti}-s give rise to the seven \textit{svara}-s namely \textit{ṣaḍja, ṛṣabha, gāndhāra, madhyama, pañcama, dhaivata} and \textit{niṣāda}. And they are respectively denoted by \textit{sa, ri, ga, ma, pa, dha} and \textit{ni}.” 

~\hfill (\textit{Translation my own})
\end{myquote}

A \textit{svara} is by default called a ‘note’. However, it is interesting to observe that both \textit{svara} and note do not mean exactly the same. The HDM (2003: 571) defines a note as “a symbol used in musical notation to represent the duration of a sound and, when placed upon a staff, to indicate its pitch; more generally (especially in British usage), the pitch itself. Types of notes are classed and named according to the relationship of their durations to one another and are sometimes termed note values”. As per Kamien (1980: 42) “Pitches are notated by the placement of notes on a staff. A note is a black or white oval to which a stem and flags can be added…the higher a note is placed on the staff, the higher its pitch”. From these definitions it is clear that, a note generally means the pitches in an octave and specifically a symbolic representation of the pitch and duration in a musical notation. 

A \textit{svara}\index{svara@\textit{svara}} is much more than a note in many ways. Firstly, though technically each of the \textit{svara}-s have discrete \textit{śruti}-s\index{sruti@\textit{śruti}} (pitches), conceptually they are those that are derived from the \textit{śruti}-s and subsequently lead to creation of the \textit{rāga}-s.\index{raga@\textit{rāga}} \textit{Śruti}-s are twenty-two but \textit{svara}-s are seven. Secondly, as the definition by Śārṅgadeva\index{Sarngadeva@Śārṅgadeva} makes it clear, \textit{svara}-s have the ability to delight the human mind on their own and this is what makes them different from \textit{śruti}-s or \textit{rāga}-s.


\subsection*{4. \textit{Rāga} ≠ Melody\index{melody@melody}}

\textit{Rāga} is a Saṁskṛta word meaning affection, colour, feelings, etc. In the context of music, it is anything that rejoices the mind – \textit{rañjayatīti rāgaḥ} or \textit{rajyate’neneti rāgaḥ}. (\textit{rañj dhātu}+\textit{bhāve karaṇe vā ghañ pratyaya})

\begin{verse}
\textit{yo’sau dhvani-viśeṣas tu svara-varṇa-vibhūṣitaḥ \dev{।}}\\ \textit{rañjako jana-cittānāṁ sa ca rāga udāhṛtaḥ \dev{।।}} 

~\hfill (\textit{Bṛhaddeśī}\index{Brhaddesi@\textit{Bṛhaddeśī}} 3.1.5)
\end{verse}

\begin{myquote}
“A kind of \textit{dhvani} (sound) that is decorated with specific \textit{svara}-s (intervals) and \textit{varṇa}-s (intervallic transitions) and that which delights the minds of the people, is called a \textit{rāga}.” 

~\hfill (\textit{Translation my own})
\end{myquote}

It is rather difficult to define a \textit{rāga} in precise terms. Grossly, we can say that it is the combination and arrangement of a series of \textit{svara}-s in a particular sequence with a definite relationship to its fundamental svara or \textit{ādhāra ṣaḍja} that forms the basis for \textit{rāga}-s. The \textit{ārohaṇa}\index{arohana@\textit{ārohaṇa}} and \textit{avarohaṇa}\index{avarohana@\textit{avarohaṇa}} (translated as a scale)\index{scale@scale} is a mere entry point to understand the outline or contour of a \textit{rāga}. Though theoretically it is said that \textit{śruti}-s give rise to \textit{svara}-s, which in turn lead to formation of \textit{rāga}-s, empirically \textit{rāga}-s go beyond the \textit{svara}-s\index{svara@\textit{svara}} they are made up of and expand through fine, delicate \textit{Śruti}-s. This has been pertinently advocated by Someśvara\index{Calukya Somesvara@Cālukya Someśvara} when he says

\begin{verse}
\textit{rāgaḥ pravardhate śrutyā rajyate mānasaṁ sadā} 

~\hfill (\textit{Mānasollāsa}\index{Manasollasa@\textit{Mānasollāsa}} 4.16.123)
\end{verse}

\begin{myquote}
“\textit{Rāga} thrives or grows perpetually through \textit{Śruti}-s\index{sruti@\textit{śruti}} and thus attracts/\break pleases/touches the mind.” 

~\hfill (\textit{Translation my own})
\end{myquote}

Well does the HDM (2003: 758) say -

\begin{myquote}
“The pitches of any music in which pitch is definable can be reduced to a scale.\index{scale@scale} The concept and its pedagogical use have been especially prominent in the history of western art music. The importance of the concept in non-western systems varies considerably and is often associated with concepts of melody\index{melody@melody} construction and internal pitch relationships that go well beyond any simple ordering of pitches from lowest to highest.” 
\end{myquote}

Hence, the crux of \textit{rāga}\index{raga@\textit{rāga}} is its melodic personality and individuality (beyond its \textit{ārohaṇa}\index{arohana@\textit{ārohaṇa}} and \textit{avarohaṇa}\index{avarohana@\textit{avarohaṇa}}) that is derived from what we call as characteristic phrases or signature \textit{saṅgati}-s,\index{sangati@\textit{saṅgati}} which have been refined and defined over centuries of evolution. This well-established character of each \textit{rāga} as well as a wide variety of phrases within the \textit{rāga} bring out distinct moods and emotive aspects, thereby leading to \textit{rasānubhava},\index{rasanubhava@\textit{rasānubhava}} which indeed has to be deeply felt and experienced, rather than explained. This the reason why \textit{rāga}-s in Indian music have long been personified and considered as almost living entities. 

Also, \textit{rāga}-s are aesthetic entities and can be perceived by trained ears. A \textit{rāga}’s communicative efficacy is thus attributed to the receptivity of the listener, as much as the competence of the performer. The ability on the part of a person to recognise, distinguish and sing or play \textit{rāga}-s indicates a high degree of musical culture.

India is the home of the \textit{Rāga} system. \textit{Rāga} is the pivotal concept of Indian music. The usage of \textit{gamaka}-s\index{gamaka@\textit{gamaka}} to beautify, decorate and embellish the \textit{svara}-s and the exquisite workmanship in the creation of music has been possible in Indian music only because of this \textit{rāga} system. It is certainly the \textit{rāga} system that paved the way for the development of \textit{manodharma\index{manodharma@\textit{manodharma}} saṅgīta} or creative music along systematic lines.

\begin{myquote}
“Indian music may be styled the \textit{Raga sangita} and European music the \textit{Samvada sangita}. They may also be syled as the \textit{Ekadhvani} and \textit{Bahudhvani} systems of music respectively. Each system has its own beauties”. 

~\hfill (Sambamoorthy\index{Sambamoorthy, P.@Sambamoorthy, P.} 2005: 18)
\end{myquote}

\begin{myquote}
“The builders of the Indian system of music knew the principles of harmony…. The extensive raga system and tala\index{tala@\textit{tāla}} system which are unique features of Indian music could not have been evolved, if they had also chosen to develop their music along the lines of harmony. The ideal of absolute music has been reached here in the very concept of the raga. The concept of raga is India’s contribution to world’s musicology.” 

~\hfill (Sambamoorthy 2005: 16)
\end{myquote}

In the ancient times, \textit{rāga}-s\index{raga@\textit{rāga}} were initially referred to as \textit{mūrcchana}-s,\break \textit{jāti}-s,\index{jati@\textit{jāti}} and it is believed that it was Mataṅga\index{Matanga@Mataṅga} who was the first to introduce the term ‘\textit{rāga}’. Different \textit{rāga}-s have been in vogue at different points of time in history, some went out of usage and some newly created. Nonetheless, on the same lines as \textit{nāda},\index{nada@\textit{nada}} \textit{rāga} has always been perceived with a spiritual outlook. This can be understood from the \textit{kṛti}-s\index{krti@\textit{kṛti}} of many \textit{santa vāggeyakāra}-s\index{vaggeyakara@\textit{vāggeyakāra}} like Aruṇagirināthar,\index{Arunagirinathar@Aruṇagirināthar} who in his \textit{Tiruppugaḻ\index{Tiruppugal@\textit{Tiruppugaḻ}} ‘Kādimōdi’} has described God as dwelling in \textit{rāga–‘Rāgatturaivone’} and Tyāgarāja,\index{Tyagaraja@Tyāgarāja} who in his \textit{kṛti ‘Elarā Kṛṣṇa’} refers to the Lord as ‘\textit{Rāgarasika}’\index{rasika@\textit{rasika}} i.e., one who delights in \textit{rāga}.

Moreover, \textit{rāga} has ever been perceived as a means of transcendence and the connect or harmony with nature was always aimed at. Different \textit{rāga}-s relating to different times of the day and different seasons of the year have been identified to stimulate specific \textit{rasa}-s and cause desirable positive effects on the human body and mind.\endnote{Though this is more or less strictly followed in Hindustani music even today, the present practice of Karnatic music believes in creating an imagery of a certain season or time of the day by performing a certain \textit{rāga}\index{raga@\textit{rāga}}.}

‘Melody’\index{melody@melody} is defined by the HDM (2003: 499,500) simply as “a coherent succession of pitches”. It further says “Melody is opposed to harmony in referring to successive rather than simultaneous sounds; it is opposed to rhythm\index{rhythm@rhythm} in referring to pitch rather than duration or stress”. Kamien (1980: 48) describes melody as a series of single tones that add up to a recognizable whole. This sounds a bit too primitive and definitely does not come anywhere close to the ‘\textit{rāga}’ of Indian music. Calling it a tune, scale\index{scale@scale} or mode is also highly superficial and does no justice to the highly abstract and rich concept of \textit{rāga}. \textit{Rāga} is unique only to Indian music; if one has to comprehend and feel a \textit{rāga}, it has to be as a ‘\textit{rāga}’ only and thus it is a non-translatable.\index{Non-translatables@Non-translatables}

\begin{myquote}
“All musical cultures have melody\index{melody@melody} and melodies, and within bounds of larger stylistic consistencies, a culture’s melodies resemble and differ from one another in ways easily perceivable, if not always so easily describable; yet the consideration of melodies in relatable groups is the quickest path to grasping their individual modes of coherence…. In some musics, melody types are named and are describable entities manipulated by musicians; the Indian \textit{rāga} is the outstanding modern instance.” 

~\hfill (HDM 2003: 500)
\end{myquote}

Though, it is notable that the West at least recognizes this unparalleled entity of the Indian \textit{rāga}, it is somehow not clear as to why it calls it ‘modern’ given that the \textit{rāga} system in India has a history of almost two millennia.

Likewise, all the technical idioms and guidelines relating to \textit{rāga} like \textit{bhāṣāṅga rāga,\index{bhasanga raga@\textit{bhāṣāṅga rāga}} rakti rāga,\index{rakti raga@\textit{rakti rāga}}\index{raga@\textit{rāga}} ghana rāga,\index{ghana raga@\textit{ghana rāga}} graha-svara,\index{graha-svara@\textit{graha-svara}}\index{svara@\textit{svara}} aṁśa-svara,\index{amsasvara@\textit{aṁśa-svara}} nyāsa-svara,\index{nyasasvara@\textit{nyāsa-svara}} alpatva,\index{alpatva@\textit{alpatva}} bahutva,\index{bahutva@\textit{bahutva}} gamaka,\index{gamaka@\textit{gamaka}} saṅgati,\index{sangati@\textit{saṅgati}} viśeṣa-prayoga}, etc. are also obviously untranslatable.


\subsection*{5. \textit{Meḷa}\index{meḷa@\textit{meḷa}} has no =}

`\textit{Mila saṅgame}’ (plus \textit{ghañ pratyaya}) is the root of the word \textit{Mela} or \textit{Meḷa}, which means coming together or combination. \textit{Meḷa} is the system of scales\index{scale@scale} formulated in Karnatic\index{Karnatic@Karnatic} Music to organise \textit{rāga}-s. It facilitates systematic grouping and categorization of \textit{rāga}-s based on permutations and combinations of \textit{svara}-s. The \textit{meḷa paddhati}\index{melapaddhati@\textit{meḷa-paddhati}} is the proud innovation of South Indian musicologists like Vidyāraṇya,\index{Vidyaranya@Vidyāraṇya} Rāmāmātya,\index{Ramamatya@Rāmāmātya} Veṅkaṭamakhin\index{Venkatamakhin@Veṅkaṭamakhin} and Govindācārya\index{Govindacarya@Govindācārya} who (chronologically) were responsible for its origin and development until it evolved into a full-fledged scheme of seventy-two \textit{meḷa}-s. The \textit{meḷa}-s are scales with all seven \textit{svara}-s\index{svara@\textit{svara}} in both \textit{arohaṇa}\index{arohana@\textit{arohaṇa}} and \textit{avarohaṇa},\index{avarohana@\textit{avarohaṇa}} which translate into the \textit{Meḷakarta rāga}-s or \textit{Janaka rāga}-s\index{janaka-raga@\textit{janaka rāga}} (Parent \textit{rāga}-s) under each of which can be identified many \textit{Janya rāga}-s\index{janya raga@\textit{janya rāga}} (Derived \textit{rāga}-s).

Being inspired by the \textit{meḷa} of Karnatic Music, Hindustani\index{Hindustani@Hindustani} music devised the \textit{Thāṭ} system.\index{That system@\textit{Thāṭ system }} This is only a miniature scheme with ten \textit{thāṭ}-s or parent scales. However, this is practically alien to the European musical tradition which apparently does not have any recognized system for classification of musical scales.\index{scale@scale} Thus, the term ‘\textit{meḷa}’\index{mela@\textit{meḷa}} does not have a corresponding single term in any other vernacular or musical jargon, and hence has to be appreciated as \textit{meḷa} only.


\subsection*{6. \textit{Laya}\index{laya@\textit{laya}} ≠ Rhythm\index{rhythm@rhythm}}

\textit{Laya}, like \textit{rāga},\index{raga@\textit{rāga}} is a formless form. It is often considered to be synonymous with \textit{tāla},\index{tala@\textit{tāla}} which is not right. \textit{Laya} is not rhythm either. From the Western standpoint,

\begin{myquote}
“…the word rhythm appears on two semantic levels. In the widest sense, it is set beside the terms melody\index{melody@melody} and harmony and in a very general sense, rhythm covers all aspects of musical movement as ordered in time, as opposed to aspects of musical sound conceived as pitch (whether singly or in simultaneous combination) and timbre (tone color). In the narrower and more specific sense, rhythm shares a lexical field with meter and tempo.\index{tempo@tempo} Rhythm in that specific sense – where it can be preceded by an indefinite article – (“a rhythm”) denotes a patterned configuration of attacks that may or may not be constrained overall by a meter or associated with a particular tempo…” 

~\hfill (HDM 2003: 723)
\end{myquote}

The above description confirms that rhythm is the most gross and is simply keeping to a beat for which there may or may not be any logical explanation. Kamien (1980: 36) also mentions that rhythm can be defined as a particular arrangement of note lengths in a piece of music, sometimes matching the beat, sometimes not.

The \textit{tāla} of Indian music is also basically rhythm, but it is well-defined and represented in the form of cycles called \textit{āvarta}-s\index{avarta@\textit{āvarta}} comprising specific number of beats or \textit{akṣara}-s.\index{aksara@\textit{akṣara}} Like any other musical aspect, the \textit{tāla} is also traced to a divine origin.

\begin{verse}
\textit{ta-kāraḥ śaṅkaraḥ prokto la-kāraś śaktir ucyate \dev{।}}\\ \textit{śiva-śakti-samāyogāt tāla ity abhidhīyate \dev{।।}}
\end{verse}

\begin{myquote}
“The letter ‘\textit{ta}’ signifies Lord \textit{Śaṅkara} or \textit{Śiva} and the letter ‘\textit{la}’, Goddess \textit{Śakti} or \textit{Pārvatī}. Having born out of the coming together of \textit{Śiva} and \textit{Śakti}, it is thus named as \textit{Tāla}.\index{tala@\textit{tāla}}” 

~\hfill (\textit{Translation my own})
\end{myquote}

Unlike Western music, there exists in Indian music a great number and variety of \textit{tāla}-s. According to Sambamoorthy\index{Sambamoorthy, P.@Sambamoorthy, P.} (1998: 18): 

“The tāla\index{tala@\textit{tāla}} system is perhaps the most difficult and complicated branch of South Indian music. There is no comparison to it in the other musical systems of the world. The time measures used by all the nations put together will form but a small fraction of the innumerable varieties of rhythm\index{rhythm@rhythm} used in South Indian music”.

Coming to ‘\textit{laya}’,\index{laya@\textit{laya}} the simple musical definition of the term ‘\textit{laya}’ is \textit{kāla-pramāṇa}.\index{kala-pramana@\textit{kāla-pramāṇa}} Apparently, it can be considered as ‘tempo’.\index{tempo@tempo} Etymologically, the word \textit{laya} is derived from the Saṁskṛta root ‘\textit{līṅ śleṣaṇe}’ (plus \textit{ac pratyaya}), which means ‘to merge with’ or ‘to be absorbed in’. Thus, \textit{laya} is a highly subtle concept, which is beyond \textit{tāla} or rhythm, and encompasses both. \textit{Laya} can exist without \textit{tāla} or rhythm, but not vice versa. For instance, a form of musical expression like the \textit{rāgālāpanā}\index{alapana@\textit{ālāpanā}}\index{raga@\textit{rāga}} which is principally \textit{anibaddha} or unbound, though devoid of any \textit{tāla}, essentially has a \textit{laya}, which can be better experienced than intellectually explained, since it is highly intangible. It can only be perceived with a high level of aesthetic sense. In fact, any practitioner or serious connoisseur of Karnatic\index{Karnatic@Karnatic} music will agree to the fact that each \textit{rāga} has its own inherent \textit{laya}, without the proper understanding of which, rendering Karnatic music does not result in creating the required impact on the listeners.


\subsection*{7. \textit{Kāla} ≠ Speed}

Like \textit{laya},\index{laya@\textit{laya}} \textit{kāla} is also one of the \textit{Tāla\index{tala@\textit{tāla}}-daśa-prāṇa}-s.\index{dasaprana@\textit{daśa-prāṇa}} It is formed from the root ‘\textit{kala saṅkhyāne}’ (plus \textit{ac} and \textit{aṇ pratyaya}-s) and means counting, calculation or measurement. It is defined as the time taken to utter one \textit{akṣara}\index{aksara@\textit{akṣara}} or syllable. It is the degree of movement of music. It is the rate at which a musical unit called \textit{akṣara} is placed or fit in each beat of the \textit{tāla}. Depending upon the number of \textit{akṣara}-s for each beat, \textit{kāla}-s are determined. For example, if it is at the rate of one \textit{akṣara} per beat it is said to be first \textit{kāla}, rate of two \textit{akṣara}-s per beat is said to be second \textit{kāla}, rate of four \textit{akṣara}-s per beat is said to be third \textit{kāla} and so on, with the units or \textit{akṣara}-s doubling with every succeeding higher \textit{kāla}, however maintaining the tempo or \textit{kāla-pramāṇa}\index{kalapramana@\textit{kāla-pramāṇa}} of the \textit{tāla} constant throughout.

\newpage

I am not sure if the word ‘speed’ does any justice to this notion of \textit{kāla}. Speed is a very generic term which only indicates fastness and slowness and not necessarily the rate of geometric progression (Eg. 1,2,4,8,16 or 1.5,3,6,12,24 or 2.5,5,10,20,40) that \textit{kāla} stands for. Speed may only be accepted as the best possible approximately equivalent term for \textit{kāla} for lack of a better term.

Likewise, most of the other \textit{daśa-prāṇa}-s\index{dasaprana@\textit{daśa-prāṇa}} of \textit{tāla}\index{tala@\textit{tāla}} like \textit{kriyā,\index{kriya@\textit{kriyā}} jāti,\index{jati@\textit{jāti}} aṅga, yati,\index{yati@\textit{yati}} prastāra,\index{prastara@\textit{prastāra}} etc}., cannot be captured in words of other languages.


\subsection*{8. \textit{Gamaka}\index{gamaka@\textit{gamaka}} has no =}

\textit{Gamaka} is undoubtedly the heart and soul of Indian music, particularly Karnatic\index{Karnatic@Karnatic} Music. Being the most important feature of our music and the core of the \textit{rāga}\index{raga@\textit{rāga}} system, it lends to our music its distinctive character and personality. In other words, Karnatic music is what it is, due to its \textit{gamaka}-s. Grammatically, it is derived from the \textit{dhātu ‘gamḷ gatau’} (with \textit{ṇic} and \textit{ṇvul pratyaya}-s) and can be used in the sense of \textit{gamayati, prāpayati, bodhayati} generally and \textit{ya ātmānamiti}\index{atman@\textit{ātmān}} with reference to music.

\begin{myquote}
\textit{sva-śruti-sthāna-saṁbhūtāṁ chāyāṁ śrutyantarāśrayām \dev{।}}\\ \textit{svaro yad gamayed gīte gamako’sau nirūpitaḥ \dev{।।}} 

~\hfill (\textit{SSS}\index{Sangitasamayasara@\textit{Saṅgītasamayasāra}} 1.47)
\end{myquote}

\begin{myquote}
“When a \textit{svara}\index{svara@\textit{svara}} assumes and moves into the shade of another \textit{śruti} which is emanating from its own \textit{śruti-sthāna},\index{sruti@\textit{śruti }} it is termed as \textit{gamaka}.” 

~\hfill (\textit{Translation my own})
\end{myquote}

Rajarao\index{Rajarao, L. Mysore@Rajarao, L. Mysore} (1963:12) observes that any usage or experiment\index{experimentation@experimentation} or handling of \textit{svara}-s which is aimed at beautifying the \textit{rāga} and rendering aural pleasure to the listener, thereby bringing out the artistic skill of the musician is \textit{gamaka}. In other words, it is the graceful turn or curve or stress or shake or corner touch given to a \textit{svara} or to a group of \textit{svara}-s that emphasizes the melodic individuality of the \textit{rāga}.

In ancient musical texts like \textit{Nāradīya Śikṣā,\index{Siksa@\textit{Śikṣā}} Dattilam}\index{Dattila@Dattila} and \textit{Nātyaśāstra},\index{Natyasastra@\textit{Nātya-śāstra}} the term \textit{alaṅkāra}\index{alaṅkāra (trope)@\textit{alaṅkāra} (trope)} (and its many types) is mentioned which in later centuries came to be known as \textit{gamaka}\index{gamaka@\textit{gamaka}} (believed to be introduced by Mataṅga\index{Matanga@Mataṅga} in his \textit{Bṛhaddeśī}\index{Brhaddesi@\textit{Bṛhaddeśī}} for the first time). Since then, different musicologists have come up with different varieties of \textit{gamaka}-s with clear-cut definitions ranging from seven to fifteen in number, but at some point, it got crystallized into a scheme of ten \textit{gamaka}-s,\index{gamaka@\textit{gamaka}} which is more or less accepted even today as \textit{daśa-vidha-gamaka}.

No doubt Western music also uses musical effects like \textit{legato,\index{legato@\textit{legato}} staccato,\index{staccato@\textit{staccato}} vibrato},\index{vibrato@\textit{vibrato}} etc. But, the usage is limited as the system of music is basically harmonic; and thus they do not very well correspond to the intricate \textit{gamaka}-s of our musical system.


\subsection*{9. \textit{Saṅgati}\index{sangati@\textit{saṅgati}} has no =}

\textit{Saṅgati} is a very commonly used term in Karnatic\index{Karnatic@Karnatic} music. The word \textit{Saṅgati} is also derived from ‘\textit{gamḷ gatau}’ \textit{dhātu} (with \textit{sam upasarga} and \textit{ktin pratyaya}). In the context of a \textit{rāgālāpanā,\index{raga@\textit{rāga}}\index{alapana@\textit{ālāpanā}} saṅgati}-s refer to musical phrases that portray the multiple shades and nuances of the \textit{rāga}. A bunch of \textit{saṅgati}-s in a meaningful sequence woven aesthetically, some of which are customary and some products of the artist’s imagination and creative genius, form a complete and wholesome \textit{rāgālāpanā}. In the context of a \textit{kṛti,\index{krti@\textit{kṛti}} saṅgati}-s can be understood as successive step-by-step musical variations of a certain chosen lyrical line, with the degree of embellishment and complication gradually increasing. So, these will be considered as first \textit{saṅgati}, second \textit{saṅgati} and so on. These are mostly the creation of the \textit{vāggeyakāra},\index{vaggeyakara@\textit{vāggeyakāra}} which is \textit{kalpita} or pre-composed from the point of view of the performer, but can also be attempted by the performer based on his own creative ability. Though \textit{saṅgati}-s can be found in Hindustani\index{Hindustani@Hindustani} music to a small extent (but may not be called as such), such an idea is unimaginable in any non-Indian musical culture. Therefore, \textit{saṅgati} adds to the list of non-translatables.\index{Non-translatables@Non-translatables}


\subsection*{10. \textit{Manodharma}\index{manodharma@\textit{manodharma}} ≠ Improvisation\index{improvisation@improvisation}}

‘\textit{Manodharma}’ is the hallmark of Indian Classical Music. In fact, it is \textit{manodharma} which makes our music absolute, dynamic and special. In India, every musician is a creative artist. His creation is based upon his \textit{manodharma} (\textit{manasaḥ dharmaḥ}). He delights his listeners not only with his scholarly and polished renderings of musical compositions already composed, but also with his own improvised music. Thus, in an Indian concert, one listens not only to the music but also to the musician.

\newpage

\textit{Manodharma}\index{manodharma@\textit{manodharma}} may be roughly translated as improvisation.\index{improvisation@improvisation} The HDM (2003: 406) gives the definition of ‘Improvisation’ as “the creation of music in the course of performance” and then goes on to explain the role, significance and degree of improvisation in music cultures across the globe. It argues that it seems most appropriate to reserve the term improvisation for cultures and repertories in which a distinction from non-improvised or precomposed forms can be recognized. It further observes that the world’s cultures differ in the value placed on improvisation; in Western culture which is heavily dependent on notation, it has usually been regarded as a kind of craft, sub-ordinate to the more ‘prestigious’ art of composition.

The dictionary (2003: 407) finally recognizes that the most developed improvisatory systems are those of India and the Middle East and also that the knowledge of non-Western music helped in shaping improvisation in Western art music around 20th century.

Also, ‘Improvisation’ (which is fairly a very recent concept in Non-Indian musical genres and not a much explored one at that) only means any added improvised extension of the main composition which can also be rehearsed, whereas \textit{manodharma} is totally spontaneous. Moreover, \textit{manodharma} also represents the unique traits of a musician which evolves and matures over a period of time. It is significant to note that in an Indian music concert (both Hindustani\index{Hindustani@Hindustani} and Karnatic)\index{Karnatic@Karnatic} about 60-70\% of the concert duration is reserved for \textit{Manodharma} whereas improvisation in other genres is very minimal. Thus, though ‘Improvisation’ is not a really bad translation, it is still unable to capture the complete essence of the idea of \textit{manodharma} of Indian music.

In Karnatic Music, the various aspects of \textit{manodharma}\index{manodharma@\textit{manodharma}} like \textit{rāgālāpanā,\index{raga@\textit{rāga}}\index{alapana@\textit{ālāpanā}} tāna,\index{raga-tana-pallavi@\textit{rāga-tāna-pallavi}}\index{see raga-tana-pallavi@see \textit{rāga-tāna-pallavi}}\index{tana@\textit{tāna}} nêraval}\index{neraval@\textit{nêraval}} (\textit{deśī}\index{desi@\textit{deśī}} word), \textit{svarakalpanā,\index{svara@\textit{svara}}\index{svarakalpana@\textit{svarakalpanā}} viruttam}\index{viruttam@\textit{viruttam}} (\textit{tadbhava} of \textit{vṛttam}), \textit{śloka,\index{sloka@\textit{śloka}} ugābhoga}\index{ugabhoga@\textit{ugābhoga}} (Kannada word), \textit{laya-vinyāsa,\index{layavinyasa@\textit{laya-vinyāsa}} tani-āvartana}\index{tani avartana@\textit{tani-āvartana}} are all non-translatables\index{Non-translatables@Non-translatables} as they just cannot be expressed in any other language.


\subsection*{11. \textit{Rasa} ≠ Emotion or Sentiment}

The crux of Indian art forms is \textit{Rasa}. It is the ultimate goal of any art; any artistic attempt that has all the necessary ingredients, but is devoid of \textit{rasa}, is not considered any worthwhile (\textit{na hi rasādṛte kaścid arthaḥ pravartate - Nāṭyaśāstra}\index{Natyasastra@\textit{Nātya-śāstra}} 6.32+). The intensity, depth and comprehensiveness that have gone into the study of the theory of \textit{rasa} in the Indian tradition fails to find a parallel anywhere in the world.

\textit{Rasa} generally means ‘\textit{rasyate āsvādyate iti rasaḥ}’- that which is relished or enjoyed, the technical definition of which is given by Bharata\index{Bharata@Bharata} (\textit{Nāṭyaśāstra}\index{Natyasastra@\textit{Nātya-śāstra}} 6.32+) as “\textit{vibhāva\index{vibhava@\textit{vibhāva}}-anubhāva-vyabhicāri\index{vyabhicari bhava@\textit{vyabhicāri-bhāva}}-saṁyogād rasa-niṣpattiḥ}” which means that the basic feelings in a person termed as \textit{sthāyibhāva}-s\index{bhava@\textit{bhāva}} get evoked and themselves result as \textit{rasa}­-s due to the combination and interplay between the attendant emotional conditions called \textit{vibhāva}-s, \textit{anubhāva}-s and \textit{vyabhicāribhāva}-s and also due to their contact with the \textit{sthāyibhāva}-s.

\textit{Rasa} is translated as ‘emotion’ or ‘sentiment’, both of which may only be taken as corresponding words for \textit{bhāva} which is the basic requirement for \textit{rasānubhava},\index{rasanubhava@\textit{rasānubhava}} and not as words for \textit{rasa}. \textit{Bhāva}-s always exist in the minds of cultured connoisseurs called ‘\textit{sahṛdaya}-s’\break (also untranslatable) and transform into \textit{rasa}-s upon involvement in beautiful music, dance, drama or literature. Thus, it might be almost impossible to express the spirit of \textit{rasa} in a single non-Saṁskṛta word.


\subsection*{12. \textit{Vāggeyakāra}\index{vaggeyakara@\textit{vāggeyakāra}} ≠ Composer}

The concept of a \textit{Vāggeyakāra} (\textit{dhātu\index{dhatu@\textit{dhātu}}-mātu-kāra}) is probably special to Indian music. A \textit{Vāggeyakāra} can be defined as “\textit{vāk ca geyaṁ ca – ete karoti iti}”. He is one who aesthetically blends both \textit{vāk} and \textit{geyam} (also called as \textit{Mātu} and \textit{Dhātu}) and thereby creates a beautiful \textit{kṛti}\index{krti@\textit{kṛti}} or \textit{racanā}. His \textit{kṛti}, in which both the \textit{saṅgīta} and the \textit{sāhitya}\index{sahitya@\textit{sāhitya}} are simultaneously conceived and are thus of equal importance, is the result of an \textit{intuitional} flash, which happens spontaneously and not of a pre-planned and systematic intellectual exercise.

The word ‘composer’ is normally used to refer to a \textit{vāggeyakāra}. However, all composers need not be \textit{vāggeyakāra}-s. In the Western context where \textit{sahitya} is almost of no significance, a composer is one who composes only the music or just a tune. The same term being used in the context of Indian music, can mean someone who either conceives of the \textit{dhātu}\index{dhatu@\textit{dhātu}} first and then thinks of an appropriate \textit{sāhitya} to it or only composes music or sets tune to a \textit{sāhitya} already authored by somebody else. Also, in the West, a composer often engages in the task of composing pieces for specific music shows and may thus derive substantial sums for his compositions. His composition is often well-thought-out and a product of conscious effort. He takes a paper and pencil in his hand and writes the music, whereas for a \textit{vāggeyakāra},\index{vaggeyakara@\textit{vāggeyakāra}} his \textit{kṛti}\index{krti@\textit{kṛti}} is a mostly a product of \textit{tapas} and (divine) inspiration, the very first emergence of which is consummate. Not just that, he sings as he composes and composes as he sings. This becomes possible for him due to the extraordinary gift of genius or \textit{pratibhā} that he possesses and also develops over a period of time (a combination of \textit{śakti-vyutpatti-abhyāsa}). Thus, a \textit{vāggeyakāra} is much more than a ‘composer’ and perhaps there is no single term in the English lexicon that can mean a \textit{vāggeyakāra}. All ancient texts like SR\index{Sangitaratnakara@\textit{Saṅgītaratnākara}} clearly lay down the qualifications for \textit{vāggeyakāra}-s and classify them as \textit{uttama, madhyama} and \textit{adhama}.

The \textit{Mānasollāsa}\index{Manasollasa@\textit{Mānasollāsa}} effectively brings out the grandeur and sublimity of the idea of \textit{Vāggeyakartṛtva} in our tradition when it says “\textit{dhātu\index{dhatu@\textit{dhātu}}-mātu-kriyā\index{kriya@\textit{kriyā}}-kārī pravaraḥ parikīrtitaḥ}” (4.16.15), that is one who does not only \textit{dhātu} and \textit{mātu}, but also cogently illustrates or demonstrates his composition, is considered the best (\textit{Translation my own}). And all our great saint seer composers not only demonstrated what they composed, but they lived what they preached through their \textit{racanā}-s! That is the kind of contribution of our immortal \textit{vāggeyakāra}-s to the immortality of \textit{Karnāṭaka Śāstrīya Saṅgīta}.

The HDM (2003:194) defines a composition only as “the activity of creating a musical work; the work thus created”. It also acknowledges that “Non-western cultures vary considerably in the extent to which the concepts implied by the term composition are applicable”. It is well-known that South Indian music in the course of its long history has evolved many musical forms belonging to both Pure music and Applied music, both Technical and Melodic. Examples are \textit{alaṅkāra,\index{alankara@\textit{alaṅkāra}} gīta,\index{gita@\textit{gīta}} jatisvara,\index{jatisvara@\textit{jatisvara}} svarajati,\index{svarajati@\textit{svarajati}} varṇa,\index{varna (composition)@\textit{varṇa} (composition)} kṛti, kīrtana,\index{kirtana@\textit{kīrtana}} jāvaḷi,\index{javali@\textit{jāvaḷi}} padaṁ,\index{pada@\textit{pada}} tillāna,\index{tillana@\textit{tillāna}} devaranāma,\index{devaranama@\textit{devaranāma}} tiruppugaḻ},\index{Tiruppugal@\textit{Tiruppugaḻ}} and so on; and for all of these, there is possibly only term – a ‘musical composition’\endnote{Even the parts of a composition like \textit{Pallavi,\index{pallavi@\textit{pallavi}} Anupallavi,\index{anupallavi@\textit{anupallavi}} Caraṇa,\index{carana@\textit{caraṇa}} Ciṭṭesvara}, etc (even though some of which are arbitrary or customary terms) do not seem to find equivalents in other languages.}. This can be compared to usage of just one word ‘uncle’ to address varied relations like father’s elder brother, father’s younger brother, mother’s elder brother, mother’s younger brother, aunt’s husband, father-in-law, neighbour, male teacher as well as an unrelated elderly male friend; whereas there are specific and distinct words for each of these in all the Indian languages. This shows the dearth of vocabulary in English when it comes to representing the ideas of various facets of Indian culture.


\section*{Other Non-translatables\index{Non-translatables@Non-translatables}}

Apart from these, there are several other aspects which are exclusive to Karnatic\index{Karnatic@Karnatic} music that cannot be simply translated to any other language. An attempt has been made to furnish a comprehensive list of these in the appendix.


\section*{Conclusion}

Thus, it is evident that a poor translation is not just of the word but of a whole lot of wisdom that it embodies. Just like in other Indic domains, there is an acute need to retain and preserve these non-translatables of Indian music in their most authentic and indigenous forms, to ensure that the undesirable loss of meaning does not sooner or later lead to loss of wisdom and the obvious appropriation\index{appropriation@appropriation} and digestion\index{cultural digestion@cultural digestion} that follows. 

Malhotra\index{Malhotra, Rajiv@Malhotra, Rajiv} and Neelakandan\index{Neelakandan, Aravindan@Neelakandan, Aravindan} (2011: 88-124) have vividly demonstrated the series of processes that are being carried out to assimilate Hinduism and various elements of its culture like Karnatic Music and \textit{Bharatanāṭyam}\index{Bharatanatyam@\textit{Bharatanāṭyam}} into Christian history and dogma. The non-translatable campaign is an important step towards counter-attacking this conspiracy and protecting our dhārmic wealth in the long run.


\section*{Bibliography}

\begin{thebibliography}{99}
\itemsep=0pt

 \bibitem{chap2-key01} Ayyangar, R. Rangaramanuja. (1972). \textit{History of South Indian (Carnatic) Music: from Vedic times to present}. (n.p): Author.

 \bibitem{chap2-key02} \textbf{\textit{Bṛhadāraṇyaka Upaniṣad}}. See Madhavananda (1950).

 \bibitem{chap2-key03} \textbf{\textit{Bṛhaddeśī}}. See Sathyanarayana (1998).

 \bibitem{chap2-key04} Kamien, Roger. (1980). \textit{Music: An Appreciation}. USA: McGraw-Hill, Inc.

 \bibitem{chap2-key05} Kedarnath, Pandit. (Ed.) (1943). \textit{The Nāṭyaśāstra by Śrī Bharatamuni}. Bombay: Satyabhamabai Pandurang.

 \bibitem{chap2-key06} Madhavananda, Swami. (Trans.) (1950). \textit{The Bṛhadaraṇyaka Upaniṣad}. Almora: Advaita Ashrama.

 \bibitem{chap2-key07} \textbf{\textit{Mahābhārata}}. (1958). No editor. Gorakhpur: Gita Press.

 \bibitem{chap2-key08} Malhotra, Rajiv and Neelakandan, Aravindan. (2011). \textit{Breaking India}. New Delhi: Amaryllis.

 \bibitem{chap2-key09} Malhotra, Rajiv.\index{Malhotra, Rajiv@Malhotra, Rajiv} (2013). \textit{Being Different}. Noida: HarperCollins Publishers India.

 \bibitem{chap2-key10} \textbf{\textit{Mānasollāsa}}. See Shrigondekar (1961).

 \bibitem{chap2-key11} \textbf{\textit{Nāradīyaśikṣa}}. (2013). No editor. Bangalore: Drahyayana Pratishthana.

 \bibitem{chap2-key12} \textbf{\textit{Nāṭyaśāstra}}. See Kedarnath (1943).

 \bibitem{chap2-key13} Prabhanjanacharya, Vyasanakere. (Ed.) (2008). \textit{Stotramālikā}. Bangalore: Aitareya Prakashana Vyasanakere.

 \bibitem{chap2-key14} Rajarao, L. Mysore.\index{Rajarao, L. Mysore@Rajarao, L. Mysore} (1963). \textit{Saṅgītaśāstrasāra} (Kannada). Bengaluru: Sri Srinivasa Sangita Kalashale.

 \bibitem{chap2-key15} Randel, Don Michael. (Ed.) (2003). \textit{The Harvard Dictionary of Music}. Cambridge, Massachusetts and London: The Belknap Press of Harvard University Press.

 \bibitem{chap2-key16} Sambamoorthy, P.\index{Sambamoorthy, P.@Sambamoorthy, P.} (1998). \textit{South Indian Music}. (Book 2) Chennai: The Indian Music Publishing House.

 \bibitem{chap2-key17} — .(2005). \textit{South Indian Music}. (Book 1). Chennai: The Indian Music Publishing House.

 \bibitem{chap2-key18} — .(2006). \textit{South Indian Music}. (Book 6). Chennai: The Indian Music Publishing House.

 \bibitem{chap2-key19} Sampatkumaracharya, V. S. and Ramaratnam, V. (2000). \textit{Karnāṭaka Saṅgīta Dīpike} (Kannada). Mysore: D.V.K. Murthy.

 \bibitem{chap2-key20} \textbf{\textit{Saṅgītaratnākara}}. See Sastri (1943).

 \bibitem{chap2-key21} \textbf{\textit{Saṅgītasamayasāra}}. See Sastri (1925).

 \bibitem{chap2-key22} Sastri, T. Ganapati. (Ed.) (1925). \textit{The Saṅgītasamayasāra of Saṅgitākara Śrī Pārśvadeva}. Trivandrum: The Government of Her Highness the Maharani Regent of Travancore. 

 \bibitem{chap2-key23} Sastri, Pandit S. Subrahmanya. (Ed.) (1943). \textit{Saṅgītaratnākara of Śārṅgadeva} – Vol I. Madras: Adyar Library.

 \bibitem{chap2-key24} Sathyanarayana, R. (Ed. \& Trans.) (1998). \textit{Śri Mataṅgamuniviracita Bṛhaddeśī}. Hampi: Prasaranga, Kannada University.

 \bibitem{chap2-key25} Shrigondekar, G.K. (Ed.) (1961). \textit{Mānasollāsa of King Someśvara}. Baroda: Oriental Institute.

 \bibitem{chap2-key26} \textbf{\textit{Śrīmad Bhagavad-gītā}}. See Vireswarananda (1989).

 \bibitem{chap2-key27} \textit{The Harvard Dictionary of Music}. See Randel (2003).

 \bibitem{chap2-key28} “Tamizh Cultural Portal–the cultural network, interface, \& database for the serious Tamizh person” (Last modified on 21 Oct 2016). \url{<http://tamizhportal.org/tag/non-translatables/>.} Accessed on 28 Feb 2019.

 \bibitem{chap2-key29} Vasanthamadhavi.\index{Vasanthamadhavi@Vasanthamadhavi} (2005). \textit{Theory of Music}. Bangalore: Prism Books Pvt. Ltd.

 \bibitem{chap2-key30} Vireswarananda, Swami. (Trans.) (1989). \textit{Srimad Bhagavad-gītā}. Madras: Sri Ramakrishna Math.

 \bibitem{chap2-key31} \textbf{\textit{Viṣṇusahasranāman}}. See Prabhanjanacharya (2008).

 \end{thebibliography}

\theendnotes

\section*{Appendix}

Many more non-translatables\index{Non-translatables@Non-translatables} (some of which are Saṁskṛta, while some others are the regional vernaculars) like \textit{varase, alaṅkāra,\index{alankara@\textit{alaṅkāra}} gīta,\index{gita@\textit{gīta}} jatisvara,\index{jatisvara@\textit{jatisvara}} svarajati,\index{svarajati@\textit{svarajati}} varṇa,\index{varna (composition)@\textit{varṇa} (composition)} kṛti,\index{krti@\textit{kṛti}} kīrtana,\index{kirtana@\textit{kīrtana}} jāvaḷi,\index{javali@\textit{jāvaḷi}} padaṁ,\index{pada@\textit{pada}} tillāna,\index{tillana@\textit{tillāna}} devaranāma,\index{devaranama@\textit{devaranāma}} tiruppugaḻ,\index{Tiruppugal@\textit{Tiruppugaḻ}} graha,\index{graha-svara@\textit{graha-svara}} sama, viṣama, atīta, anāgata, anuloma,\index{anuloma@\textit{anuloma}} pratiloma,\index{pratiloma@\textit{pratiloma}} avadhāna,\index{avadhana@\textit{avadhāna}} naḍêbheda, kriya,\index{kriya@\textit{kriyā}} aṅga, jāti,\index{jati@\textit{jāti}} naḍê, gati, prastāra,\index{prastara@\textit{prastāra}} laghu, druta, anudruta, guru,\index{guru@\textit{guru}} pluta, kākapāda, yati,\index{yati@\textit{yati}} ālāpanā,\index{alapana@\textit{ālāpanā}} tāna,\index{raga-tana-pallavi@\textit{rāga-tāna-pallavi}}\index{see raga-tana-pallavi@see \textit{rāga-tāna-pallavi}}\index{tana@\textit{tāna}} nêraval,\index{neraval@\textit{nêraval}} svarakalpanā,\index{svarakalpana@\textit{svarakalpanā}}\index{svara@\textit{svara}} viruttam,\index{viruttam@\textit{viruttam}} ugābhoga,\index{ugabhoga@\textit{ugābhoga}} suḷādi, janaka rāga,\index{janaka-raga@\textit{janaka rāga}}\index{raga@\textit{rāga}} janya rāga,\index{janya-raga@\textit{janya rāga}} bhāṣāṅga rāga,\index{bhasanga raga@\textit{bhāṣāṅga rāga}} upāṅga rāga,\index{upanga raga@\textit{upāṅga rāga}} ghana rāga,\index{ghana raga@\textit{ghana rāga}} rakti rāga,\index{rakti raga@\textit{rakti rāga}} niṣādyanta rāga, vakra rāga, grahasvara, aṁśasvara,\index{amsasvara@\textit{aṁśa-svara}} nyāsasvara,\index{nyasasvara@\textit{nyāsa-svara}} alpatva,\index{alpatva@\textit{alpatva}} bahutva,\index{bahutva@\textit{bahutva}} viśeṣa-prayoga, anyasvara, mudrā, pallavi,\index{pallavi@\textit{pallavi}} anupallavi,\index{anupallavi@\textit{anupallavi}} carana,\index{carana@\textit{carana}} ciṭṭesvara, rāgamālika,\index{ragamalika@\textit{rāgamālika}} jati, śôllukaṭṭu, kônnakkol, madhyama śruti,\index{sruti@\textit{śruti}} samudāya kṛti}-s, \textit{svarākṣara,\index{svaraksara@\textit{svarākṣara}}\index{aksara@\textit{akṣara}} padagarbha, vīṇā, mṛdaṅga},\index{Mṛdaṅga@\textit{Mṛdaṅga}} etc.

