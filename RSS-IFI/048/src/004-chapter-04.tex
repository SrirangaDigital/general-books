
\chapter{Experimentation\index{experimentation@experimentation} in Karnatic\index{Karnatic@Karnatic} Music – How Far is Too Far?}\label{chapter4}

\Authorline{Radha Bhaskar\footnote{pp.~\pageref{chapter4}\enginline{-}\pageref{endchapter4}. In: Meera, H. R. (Ed.) (2021). \textit{Karnāṭaka Śāstrīya Saṅgīta - Past, Present, and Future.} Chennai: Infinity Foundation India.}}

\vspace{-.3cm}

\lhead[\small\thepage\quad Radha Bhaskar]{}

\begin{flushright}
\textit{(radha@mudhra.org)}
\end{flushright}

\vspace{-.5cm}

\section*{Abstract}

\vspace{-.2cm}

The Karnatic concert format as existent presently, stands on the edifice laid by our great \textit{vāggeyakāra}-s,\index{vaggeyakara@\textit{vāggeyakāra}} especially Tyāgarāja,\index{Tyagaraja@Tyāgarāja} Muttusvāmi Dīkṣitar\index{Muttusvami Diksita@Muttusvāmi Dīkṣita} and Śyāma Śāstri.\index{Syama Sastri@Śyāma Śāstri} Also, the structure of the \textit{Kacheri}\index{kacheri@\textit{kacheri}} as devised by Sri Ariyakuḍi Rāmānuja Ayyaṅgār\index{Ariyakudi Ramanuja Ayyangar@Ariyakuḍi Rāmānuja Ayyaṅgār} about 100 years back, has stood the test of time because it caters to both the connoisseur and the layman and is enriching and uplifting devotionally, spiritually and intellectually. The \textit{sāhitya}\index{sahitya@\textit{sāhitya}} of great \textit{vāggeyakāra}-s are \textit{bhakti}\index{bhakti@\textit{bhakti}} laden and convey lofty ideologies. Even the \textit{rāga}-s\index{raga@\textit{rāga}} used for each composition have a sense of divinity and appropriateness.

But of late, with the view of trying to reach Karnatic music to the masses, we see how the compositions of \textit{vāggeyakāra}-s have been diluted and are used as tools of experimentation and as some kind of a peg to fasten \textit{manodharma}\index{manodharma@\textit{manodharma}} aspects. Thus, the devotional content has been overlooked in an attempt to make it look novel and trendy.

This paper will analyse the content of the compositions of great \textit{vāggeyakāra}-s from various perspectives to validate how these \textit{kṛti}-s\index{krti@\textit{kṛti}}\break are not mere songs but a devotional and spiritual outpour, much beyond human comprehension. The \textit{sāhitya}\index{sahitya@\textit{sāhitya}} and \textit{saṅgīta} of \textit{kṛti}-s\index{krti@\textit{kṛti}} will be examined to show how both complement each other in a seamless manner to create a divine experience both for the artiste and the \textit{rasika}-s.\index{rasika@\textit{rasika}}

The content of traditional concerts will also be analysed to see how it is effective in conveying the spiritual and devotional elements of our tradition in the most comprehensive manner.

The paper will also take up case studies of how our concert tradition is being tampered with over the past few years and the serious impact it has had in the course taken by Karnatic music. It will also elucidate some effective ways in which novelty can be introduced in Karnatic\index{Karnatic@Karnatic} music without losing out on its core values.

\vspace{-.3cm}

\section*{Introduction}

\vspace{-.2cm}

Karnatic music as we see today is a system of music that has evolved over the past several years to its present style, form and content. While it is regarded as a traditional system, it has also offered space for several experimentations\index{experimentation@experimentation} and changes over the years and thus been a dynamic form all through.

In any sphere, experimentation is important for progression and it is all the more vital in the art field to keep it alive and moving. If the intention is only to carry forward what has been handed down to us, development will cease to happen and the art will stagnate in the process. But this experimentation has two sides to it –

\vspace{-.3cm}

\begin{enumerate}
\itemsep=0pt

 \item That which leads to the progression of the art

 \item That which disturbs its fundamental ethos and is thus detrimental to the traditional values of the art.

\end{enumerate}

\vspace{-.3cm}

Karnatic music as we see today is a highly sophisticated system with a unique blend of science and aesthetics. But the experimentations and changes in form and content have been happening constantly over several centuries to evolve to its present contour as a cumulative product. New compositions have emerged with respect to their form and content, new \textit{rāga}-s\index{raga@\textit{rāga}} have been created, new \textit{tāla}-s\index{tala@\textit{tāla}} have been explored, new instruments have been devised, new method of presenting music have emerged. Now, with the music itself becoming a formal presentation through \textit{sabhā}-s\index{sabha@\textit{sabhā}} and cultural organizations, concerts have become a variety package where artists try to experiment and discover new avenues to present the form and content of the music with a view to sustain audience interest and to make it interesting.

Earlier, \textit{lakṣaṇa}\index{laksana@\textit{lakṣaṇa}} or grammar of music was strictly adhered to and the quantum of treatises available on this aspect is testimony to how music was codified with strict discipline. While it is an accepted fact that \textit{lakṣya}\index{laksya@\textit{lakṣya}} derived from listening and learning is of great importance, it is only \textit{lakṣaṇa} that codifies it and gives it a concrete shape.

Karnatic\index{Karnatic@Karnatic} music refers to the classical music system as originating and prevalent in South India – mainly in the present-day states of Tamilnadu, Karnataka, Kerala, Andhra, and Telangana. Though there are different ways of arranging the musical material in a concert, there are also some conventional methods of presentation, which have come down to us through the \textit{guru-śiṣya paramparā}.\index{guru@\textit{guru}}\index{parampara@\textit{paramparā}} The format of concert presentation that we see today does not specifically find a place in any theoretical work but it is followed as a matter of tradition. Again, though there is a model in which Karnatic music is prevalently presented today, there has also been a lot of experimentation\index{experimentation@experimentation} in presenting its form and content with a view to give it a new and different dimension.

\vspace{-.4cm}

\section*{The Concert Format}

\vspace{-.2cm}

This is only a model representation of a typical Karnatic concert as seen today – (2-3 hours duration). -

\textit{Varṇam}

\textit{Kṛti}\index{krti@\textit{kṛti}}\index{varna (composition)@\textit{varṇa } (composition)} on Lord \textit{Vināyaka}

A brief \textit{ālāpanā}\index{alapana@\textit{ālāpanā}} followed by a composition and \textit{kalpanā svara}\index{svara@\textit{svara}}\index{svarakalpana@\textit{svarakalpanā}}

One or two more compositions and \textit{kalpanā svara} (optional)

The main piece for which detailed \textit{ālāpanā, neraval}\index{neraval@\textit{nêraval}} and \textit{svara} are performed followed by \textit{Tani āvartanam}\index{tani avartana@\textit{tani-āvartanam}}

\textit{Tukkaḍa}\index{Tukkada piece@\textit{Tukkaḍa} piece} items

The \textit{Rāgam tānam pallavi}\index{raga-tana-pallavi@\textit{rāga-tāna-pallavi}}\index{tana@\textit{tāna}}\index{RTP - see raga-tana-pallavi@RTP - see \textit{rāga-tāna-pallavi}|oldindex} which was earlier an integral part of all concerts is not so now, due to the duration of the concerts becoming shorter.

\newpage

In an article in \textit{The Hindu}, March 21, 2013, music critic K. Ganapathi outlines the salient features of a traditional concert –

\begin{myquote}
“As per the conventional pattern, the concert begins with a varnam,\index{varna (composition)@\textit{varṇa } (composition)} followed by one rendered in quick succession, without an alapana\index{alapana@\textit{ālāpana}} – for example a song on Ganesha, Saraswathi or Subrahmanya and so on. These kritis are in familiar ragas\index{raga@\textit{rāga}} such as Natta, Gowla, Hamsadhwani, Mayamaalavagowla, Sriranjani and so on. Kalpana\index{svarakalpana@\textit{svarakalpanā}} swaras, if rendered, is brief. This healthy practice also served as a warm-up up exercise for the artiste. Some musicians used to render a pancharatnakriti of Tyagaraja\index{Tyagaraja@Tyāgarāja} at this stage. The artiste then presents an alapana, preferably in a “prathimadhyama” raga, followed by the kriti in the same raga (the alapana in a particular raga is automatically followed by a kriti in the same raga), appended with neraval\index{neraval@\textit{nêraval}} and kalpanaswaras; sometimes only swaras without neraval. Thereafter, the singer presents the other items selected for the concert – one or two vilamba\index{vilambakala@\textit{vilamba kāla}} kaala kritis, sometimes preceded by a brief alapana and a couple of duritakaala (\textit{sic}) kritis, with or without swaraprastharas. Then he takes up the main raga for detailed delineation, followed by the kriti, with comprehensive neraval and swaraprastharas. At this point, the percussionists present the tani avarthanam.\index{tani avartana@\textit{tani-āvarthanam}} After that, one or two fast-paced kritis are rendered, before embarking on Ragam, thaanam and pallavi (RTP).\index{raga-tana-pallavi@\textit{rāga-tāna-pallavi}}\index{tana@\textit{tāna}}\index{RTP - see raga-tana-pallavi@RTP - see \textit{rāga-tāna-pallavi}|oldindex} RTP was an important part of the concert and it would take 45 minutes to an hour to do full justice to it. In those days, musicians enthralled the audience with an elaborate RTP, normally in a major raga. The duration of the concert used to be three to three-and-a-half hours.
\end{myquote}

\begin{myquote}
In the post-pallavi session, (called “thukkada”phase) light pieces such as thillanas, jaavalis,\index{javali@\textit{jāvali}} padams,\index{pada@\textit{pada}} bhajans, slokas\index{sloka@\textit{śloka}} or verses rich in aesthetics in ragamalika\index{ragamalika@\textit{rāgamālika}} format are rendered, before concluding the concert.” 

~\hfill (Ganapathi 2013) (\textit{spellings as in the original})
\end{myquote}

A concert of Semmanguḍi Srīnivāsa Ayyar\index{Semmangudi Srinivasa Ayyar@Semmanguḍi Srīnivāsa Ayyar} performed at The Music Academy (1986) would serve as a practical model example for this –

\textit{Intaparāka} – Māyāmāḷavagouḷa – Rūpaka\\ \textit{Śrī Mātṛbhūtam} – Kannaḍa – Cāpu\\ \textit{Paṭṭividuvarādu} – Mañjari – Ādi\\ \textit{Paṅkajalochana} – Kalyāṇī – Cāpu\\ \textit{Karuṇāsamudra} – Devagāndhārī – Ādi\\ \textit{Rāgam tānam pallavi} – Bhairavī – Sankīrṇa Jhampa\\ \textit{Śaravaṇabhava} – Ṣaṇmukhapriyā – Ādi\\ \textit{Viruttam \& Tillāna}\index{tillana@\textit{tillāna}}\index{viruttam@\textit{viruttam}} – Paras – Ādi \\ \textit{Maṅgalam}

Thus, the following points may be made with regard to a general Karnatic concert –

\begin{itemize}
\itemsep=0pt

 \item It is usually a presentation of several items arranged in a specific order, based on the artiste’s own experience, taste and conventional methods. The method of musical progression is such that a number of compositions are sung before settling on the main item, which is generally a \textit{rāgam tānam pallavi}\index{raga-tana-pallavi@\textit{rāga-tāna-pallavi}}\index{tana@\textit{tāna}}\index{RTP - see raga-tana-pallavi@RTP - see \textit{rāga-tāna-pallavi}|oldindex} or a \textit{viḷamba} (two \textit{kaḷai}) \textit{kṛti}\index{krti@\textit{kṛti}}\index{vilambakala@\textit{vilamba kāla}}.

 \item Another striking feature is the presentation of a number of compositional forms in a concert – \textit{varṇam,\index{varna (composition)@\textit{varṇa } (composition)} kṛti, rāgam tānam pallavi, viruttam,\index{viruttam@\textit{viruttam}} jāvaḷi,\index{javali@\textit{jāvaḷi}} padam,\index{pada@\textit{pada}} tillāna,\index{tillana@\textit{tillāna}} aṣṭapadi,\index{Astapadi@\textit{Aṣṭapadi}} bhajan, tiruppugaḻ}\index{Tiruppugal@\textit{Tiruppugaḻ}} etc. Many composers are also featured to lend variety in terms of the content of \textit{sāhitya}\index{sahitya@\textit{sāhitya}} and the melody.\index{melody@melody}

 \item Telugu, Tamil, Sanskrit and Kannada are the main languages in which compositions are sung

 \item Theoretically, there are many \textit{tāla}-s\index{tala@\textit{tāla}} in Karnatic\index{Karnatic@Karnatic} music but only four main \textit{tāla}-s are used – \textit{ādi, rūpaka, miśra cāpu} and \textit{khaṇḍa cāpu tāla}-s.

 \item The concert encompasses songs on various deities as each composer has given vent to his/her devotion through an outpour of songs on various deities. Some composers like Śyāma Śāstri\index{Syama Sastri@Śyāma Śāstri} and Gopālakṛṣṇa Bhārati\index{Gopalakrsna Bharati@Gopālakṛṣṇa Bhārati} have concentrated on only one deity.

\end{itemize}

This traditional format which has survived over the years, as devised by Sri Ariyakuḍi Rāmānuja Ayyaṅgār\index{Ariyakudi Ramanuja Ayyangar@Ariyakuḍi Rāmānuja Ayyaṅgār} in the early 20th century, was a great experimentation\index{experimentation@experimentation} against the background of earlier concerts which did not offer such a variety fare.

\vspace{-.3cm}

\section*{Compositions – Their Role in Concerts}

While in any system of music, be it Karnatic, Hindustani\index{Hindustani@Hindustani} or Western, the composition has a specific role to play as part of the performing genre, it has a very special place in Karnatic music. Prior to \textit{pallavi}-s\index{raga-tana-pallavi@\textit{rāga-tāna-pallavi}}\index{tana@\textit{tāna}}\index{RTP - see raga-tana-pallavi@RTP - see \textit{rāga-tāna-pallavi}|oldindex}\break and \textit{kṛti}-s\index{krti@\textit{kṛti}} forming the core of Karnatic concerts, it was the \textit{Praban\break dha}-s which formed a part of the performance genre. Since the \textit{Prabandha}-s\index{prabandha@\textit{prabandha}} were technical and lengthy, they gave way to \textit{kṛti}-s which became the core material of Karnatic \textit{kacheri}-s.\index{kacheri@\textit{kacheri}}

According to musician Sandhyāvandanam Srinivasa Rao, earlier, the musical setting consisted of an elaboration of a \textit{prasiddha rāga}\index{raga@\textit{rāga}} in \textit{mandra, madhya, tāra sthāyi}\index{mandra sthayi@\textit{mandra sthāyi}}\index{madhya sthayi@\textit{madhya sthāyi}}\index{tara sthayi@\textit{tāra sthāyi}}\index{sthayi@\textit{sthāyi}} in \textit{vilamba,\index{vilambakala@\textit{vilamba kāla}}\index{madhyamakala@\textit{madhyama kāla}}\index{drutakala@\textit{druta kāla}} madhya} and \textit{druta kāla}, then a line on God or a \textit{rāja} would be rendered or a starting phrase of a \textit{Prabandha}\index{prabandha@\textit{prabandha}} or \textit{kṛti}\index{krti@\textit{kṛti}} and after an elaborate \textit{nêraval},\index{neraval@\textit{nêraval}} a few \textit{svara}-s\index{svara@\textit{svara}} would be rendered with full \textit{rāga bhāva}.\index{bhava@\textit{bhāva}} Then, two or three devotional songs or \textit{padam}-s\index{pada@\textit{pada}} of Kṣetrajña\index{Ksetrajna@Kṣetrajña/Kṣetrayya} or Purandara Dāsa\index{Purandaradasa@Purandaradāsa} would be sung. It was in toto comparable to concerts of Hindustani\index{Hindustani@Hindustani} music.

Noted music critic N. M. Narayan has also mentioned in an article in \textit{The Hindu} (3 May, 1998)-

\begin{myquote}
“The crux of the Carnatic\index{Karnatic@Karnatic} music concert is the great composition. Before the great kirtana\index{kirtana@\textit{kīrtana}} appeared in the scene, the pallavi\index{raga-tana-pallavi@\textit{rāga-tāna-pallavi}}\index{RTP - see raga-tana-pallavi@RTP - see \textit{rāga-tāna-pallavi}|index} was the king–pin of Carnatic music and spread itself all over the concert. As a concession to the novel form of music that it represented when kirtana first appeared, it is said that the maha vidvans of the past who were all pallavi exponents condescended to \textbf{render what we recognize as the major compositions of the trinity\index{Trinity, The@Trinity, The} as “tukkadas”\index{Tukkada piece@\textit{Tukkaḍa} piece} to enliven the end part of a cutcheri}…….actually, the great compositions then filled the bill as the devotional like the bhajan in the concert of Hindusthani classical…… the musicians of those times really did not reckon with the destiny of the great kritis and kirtanas…..The exclusive quality of the pallavi became a thing of the past when the tradition of technical decoration of a theme line chosen from the song itself in extension of its rendition was discovered and later got firmly settled. The elaborate neraval and svaram mounted on the chosen theme line and their excellence made even the formal pallavi look redundant when it actually appeared in the concert. In the present day, the formal pallavi has become a rarity with numerous “pallavis” figuring in the course of a performance in the form of theme lines from the song movements on which neraval and svaras are liberally mounted.”
\end{myquote}

\vspace{-.3cm}

\begin{myquote}

~\hfill \textit{(emphasis ours)}
\end{myquote}

In a personal interview with the writer, musician K. S. Krishnamurthy observed that while it was the \textit{pallavi} which was earlier elaborated expansively, it is the composition which is predominantly used in present day concerts.

\vspace{-.3cm}


\section*{Ariyakuḍi’s\index{Ariyakudi Ramanuja Ayyangar@Ariyakuḍi Rāmānuja Ayyaṅgār} Experimentation with Concert Format}

\vspace{-.2cm}

This shift in the focus of Karnatic\index{Karnatic@Karnatic} concerts in terms of content is attributed to Ariyakuḍi Rāmānuja Ayyaṅgār, whose experimentation with regard to the set-up of a concert has been path-breaking. The rationale behind the format devised by him seemed logical and practical in terms of catering the music to a larger audience. We must understand that this change was necessitated by the fact that Karnatic music had to cater to a wider spectrum of audience and thus had to offer a variety fare. This format was found to be ideal, and came to be followed by most musicians after his time.

Ariyakuḍi created a concert style for himself that involved beginning with a \textit{varṇam},\index{varna (composition)@\textit{varṇa} (composition)} singing many songs, short \textit{rāga ālāpanā}-s\index{alapana@\textit{ālāpanā}}\index{raga@\textit{rāga}} preceding some of them, \textit{neraval}\index{neraval@\textit{nêraval}} and \textit{svara}-s\index{svara@\textit{svara}} again within limits for many pieces, followed by the \textit{rāgam tānam pallavi} which again did not last more than half an hour at most. He followed the RTP\index{raga-tana-pallavi@\textit{rāga-tāna-pallavi}}\index{tana@\textit{tāna}}\index{RTP - see raga-tana-pallavi@RTP - see \textit{rāga-tāna-pallavi}|index} with many small pieces from the \textit{Tiruppugaḻ} and other Tamil works. He also invariably sang\break \textit{tillāna}-s\index{tillana@\textit{tillāna}} composed by his \textit{guru}\index{guru@\textit{guru}} at the end of the concert. This pattern soon became the rage and it was soon demanded from all other musicians as well and is followed religiously till date.

This format was of course necessitated due to the shift in nature of audience and coming in of more \textit{sabhā}-s\index{sabha@\textit{sabhā}} which catered to a larger and varied sections of people. This concert format never failed to impress the audience as there was never a dull moment or feeling of boredom throughout the concert. It caters to both the connoisseur and the layman and is enriching and uplifting spiritually and intellectually. 

In this sense, the experimentation\index{experimentation@experimentation}\index{Ariyakudi Ramanuja Ayyangar@Ariyakuḍi Rāmānuja Ayyaṅgār} Ariyakuḍi did with regard to restructuring the content of concerts was a success and it has stood the test of time. This is evident from the fact that this format or concert formula is followed by most musicians till date. So, this also brings to the fore the point that any experimentation in order to be successful has to stand the test of time.

Later, Semmanguḍi Srīnivāsa Ayyar\index{Semmangudi Srinivasa Ayyar@Semmanguḍi Srīnivāsa Ayyar} is regarded to have taken this art of concert presentation to greater heights through the way he experimented with a variety of compositions in different \textit{rāga}-s,\break \textit{tāla}-s, tempo\index{tempo@tempo} and by various composers.

In an article in \textit{The Hindu} by V. Subrahmaniam (17 December, 2012), we see an account of this -

\begin{myquote}
“…. The art of singing an effective \textit{kutcheri} with excellent audience rapport was perfected by Semmangudi Srinivasa Iyer.\index{Semmangudi Srinivasa Ayyar@Semmanguḍi Srīnivāsa Ayyar} Blessed with longevity, he lived 93 years (1908-2003) and had a long musical career spread over 75 years. He could mesmerise audiences even close to the end of his life. His last \textit{kutcheri} — an unforgettable one — was given when he was 90-plus. Although he struggled with a recalcitrant voice, he worked untiringly during his younger days to tame it to obey his musical commands. He was, however, endowed with a very deep musical insight.
\end{myquote}

\begin{myquote}
His ascendency to the acme on the concert scene was swift and he remained there till the end. His \textit{kutcheris} were full of energy and had large audiences glued to their seats. The verve with which he performed permeated the audience and kept them alert and attentive. There was not a dull moment in his concerts. Sharp enough to perceive the slightest dip in the energy level during his concert, he would immediately give it a boost by singing a fast-paced piece or sparkling \textit{kalpana swaras}.\index{svarakalpana@\textit{svarakalpana}} Adept at the \textit{kutcheri} craft, Semmangudi was a meticulous planner and ensured that all pieces were accommodated without compromising on the liveliness. As mridangam maestro Palghat Mani Iyer once remarked to this scribe, “Srinivasa Iyer’s \textit{kutcheris} have always been successful; never do they fail.” His renditions were packed with tonal continuity, each variation smoothly dove-tailing into the next, and were, importantly, served in the right measure.
\end{myquote}

\begin{myquote}
\textbf{A class apart}
\end{myquote}

\begin{myquote}
His \textit{kutcheris} boasted an excellent sense of proportion with the raga\index{raga@\textit{rāga}} prefixes, \textit{neraval}\index{neraval@\textit{nêraval}} and \textit{kalpana swaras} in consonance with the position of the piece in the concert. The first raga essay would be a short one and progressively lengthen as the performance built up to the main piece. The complexity of improvisation\index{improvisation@improvisation} in his essays, however short, were a class apart. His slow and fast tempo\index{tempo@tempo} \textit{kalpana swara}-s were soaked in \textit{raga bhava}……….” 

\vspace{-.2cm}

~\hfill \textit{(spelling as in the original)}
\end{myquote}

\vspace{-.3cm}

\section*{Compositions – The Core Material of Concerts}

\vspace{-.2cm}

While cosmetic experimentations\index{experimentation@experimentation} and changes have been made constantly, the core material over the past 100 years has all along been the same – the compositions of our great composers. The Karnatic concert content as existent presently, stands on the edifice laid by our great \textit{vāggeyakāra}-s,\index{vaggeyakara@\textit{vāggeyakāra}} especially Tyāgarāja,\index{Tyagaraja@Tyāgarāja} Muttusvāmi Dīkśitar\index{Muttusvami Diksita@Muttusvāmi Dīkśita} and Śyāma Śāstri.\index{Syama Sastri@Śyāma Śāstri} The \textit{sāhitya}\index{sahitya@\textit{sāhitya}} of great \textit{vāggeyakāra}-s\index{vaggeyakara@\textit{vāggeyakāra}} are \textit{bhakti}\index{bhakti@\textit{bhakti}}-laden and convey lofty ideologies. Even the \textit{rāga}-s\index{raga@\textit{rāga}} used for each composition has a sense of divinity and appropriateness.

If we look at the history of music, we see that many composers have adorned the devotional scenario and poured out their \textit{bhakti}\index{bhakti@\textit{bhakti}} through the medium of songs.~Āṇḍāḷ,\index{Andal@Āṇḍāḷ} Appar,\index{Appar@Appar} Sundarar,\index{Sundarar@Sundarar} Māṇikkavācagar\index{Manikkavasakar@Māṇikkavāsakar}, Jayadeva,\index{Jayadeva@Jayadeva} Annamācārya,\index{Annamacarya@Annamācārya} Purandaradāsa,\index{Purandaradasa@Purandaradāsa} Ūttukkāḍu Venkaṭa Kavi,\index{Uttukkadu Venkata Kavi@Ūttukkāḍu Venkaṭa Kavi} Tyāgarāja,\index{Tyagaraja@Tyāgarāja} Muttusvāmi Dīkśitar,\index{Muttusvami Diksita@Muttusvāmi Dīkśita} Śyāma Śāstri, Gopālakṛṣṇa Bhārati,\index{Gopalakrsna Bharati@Gopālakṛṣṇa Bhārati} Pāpanāsam Sivan\index{Papanasam Sivan@Pāpanāsam Sivan} are some of the great \textit{vāggeyakāra}-s who have given us a treasure of compositions and all these form the core material of concerts.

\vspace{-.3cm}

\section*{Inclusion of Several Forms in Karnatic\index{Karnatic@Karnatic}\\ Concerts}

\vspace{-.2cm}

It should be noted here that unlike Hindustani\index{Hindustani@Hindustani} concerts which encompass only a limited genre as part of their performance, Karnatic music has been more open in approach, in the sense that many semi-classical forms have also been absorbed into the system and made a part of it.

Hindustani music has the \textit{Khyāl}\index{Khyal@\textit{Khyāl}} and \textit{Dhrupad}\index{Dhrupad@\textit{Dhrupad}} as the two genres presented mainly in concerts. Thus, a \textit{Khyāl} concert will include several \textit{Khyāl}-s and it may be winded up with a \textit{Bhajan}. A \textit{Dhrupad} concert will again include only this genre. But, Karnatic concert has been acquiring a different colour over the years with addition of various musical forms to it. For example, the \textit{Tukkaḍa}\index{Tukkada piece@\textit{Tukkaḍa} piece} section of a Karnatic concert has come to include so many different musical forms as a part of it – \textit{Taraṅgam,\index{Tarangam@\textit{Taraṅgam}} Aṣṭapadī,\index{Astapadi@\textit{Aṣṭapadī}} Dāsara pada}, Annamācārya\index{Annamacarya@\textit{Annamācārya}} \textit{saṅkīrtana}, Patriotic songs, \textit{Tillāna,\index{tillana@\textit{tillāna}} Utsava Sampradāya}\index{utsavasampradaya@\textit{Utsava Sampradāya}} and \textit{Divyanāma Kīrtana}-s\index{Divyanama Kirtana@\textit{Divyanāma Kīrtana}} of Tyāgarāja,\index{Tyagaraja@Tyāgarāja} \textit{Bhajan}-s, \textit{Abhang}-s, \textit{Tevāram,\index{Tevaram@\textit{Tevāram}} Tiruppugaḻ},\index{Tiruppugal@\textit{Tiruppugaḻ}} folk melodies and \textit{Kāvaḍicindu}, light songs like \textit{Rabīndra Saṅgīt, Śloka}\index{sloka@\textit{śloka}} and so on. In fact, these pieces have become so attractive that sometimes, the common audience looks forward to these pieces more than the hardcore classical material presented in concerts.

Interestingly, many \textit{Tukkaḍa} pieces have been modified to give them the status of a classical piece and this experimentation\index{experimentation@experimentation} with the forms has resulted in their distortion by way of truncation of \textit{sāhitya}\index{sahitya@\textit{sāhitya}} etc, some aspects of which are discussed later.

A closer analysis would reveal that the composition of Karnatic music is most often not presented in its original form as intended by the composer. It is ideally crafted and used as a tool to exhibit the technical skill of the artiste and this is a significant change we see over the years now. In fact, this trend of decoupling the musical aspect from the devotional aspect and tagging it as “art music” is getting more intense, as seen through the case study of many concerts by various artists.

Though \textit{manodharma\index{manodharma@\textit{manodharma}} saṅgīta} is a separate domain of creativity for the artiste, it is interesting to note that many artists exercise their creativity even within the composition by adding their own \textit{saṅgati}-s\index{sangati@\textit{saṅgati}}\break and other workmanship. Thus, this process of experimentation\index{experimentation@experimentation} with music is not just limited to \textit{manodharma} but also to the \textit{kalpita} elements in Karnatic\index{Karnatic@Karnatic} music. This can be further explained through how there are several ways of singing the same composition by different artistes though its core content may be the same. In this context, experimentation with the composition is not about changing its structure, but making cosmetic changes according to one’s individual virtuosity and creativity.

\vspace{-.3cm}

\section*{Experimental Concerts in Karnatic Music}

\vspace{-.2cm}

Many experiments have been attempted over the past thirty years to put the material of Karnatic music in a new garb. Some have been meaningful experiments while some have diminished the values of the music.

Examples of some thematic concerts conducted by various \textit{sabhā}-s\index{sabha@\textit{sabhā}}-

\vspace{-.3cm}

\begin{itemize}
\itemsep=0pt

 \item Concert based on compositions of a particular composer like Tyāgarāja,\index{Tyagaraja@Tyāgarāja} Muttusvāmi Dīkṣitar,\index{Muttusvami Diksita@Muttusvāmi Dīkṣita} Śyāma Śāstri,\index{Syama Sastri@Śyāma Śāstri} Nārāyaṇa Tīrtha,\index{Narayana Tirtha@Nārāyaṇa Tīrtha} Annamācārya,\index{Annamacarya@Annamācārya} etc. This is the most significant experimentation that has happened over the past 30 years where a composer’s day is celebrated through a series of thematic concerts of only his/her compositions.

 \item Concert on a particular deity encompassing songs by various composers. This has become a very popular concept with a number of religious festivals having concerts as part of the celebrations. Thus, a festival like Navarātri will have compositions on Devī by various composers; Śrī Kṛṣṇa Jayantī will feature various compositions on Kṛṣṇa and so on.

 \item Concert based on a particular \textit{rāga}\index{raga@\textit{rāga}} like Toḍi, Kalyāṇī, Bhairavī, etc

 \item \textit{Rāgam tānam pallavi}\index{raga-tana-pallavi@\textit{rāga-tāna-pallavi}}\index{tana@\textit{tāna}}\index{RTP - see raga-tana-pallavi@RTP - see \textit{rāga-tāna-pallavi}|index} concert

 \item Concert based on highlighting \textit{tāla\index{tala@\textit{tāla}} aspects}

\end{itemize}

\newpage

But what is important in such concerts is that though they may have a different dimension, the approach and build-up of the concert is very much as per the model of a general concert. Hence, they have stood the test of time and continue to be in practice. In fact, thematic concerts have grown to a large extent in recent years.

On the other hand, there have also been artists who have deliberately tried to move against tradition and do something different on an experimental basis.

One example of a senior artiste who did such an experimentation\index{experimentation@experimentation} is \textit{Mṛdaṅgam}\index{Mrdanga@Mṛdaṅga} maestro Guru Kāraikkuḍi Maṇi who, in order to give primary status to the \textit{Mṛdaṅgam} artiste as against the main artiste, altered the sitting position of the \textit{Mṛdaṅgam} artiste. Thus, in a few such concerts, he sat centre stage. Also, to give prime focus to the rhythmic aspects of music, he started the concert with a \textit{Tani āvartanam}.\index{tani avartana@\textit{tani-āvartanam}}   This concert was performed at Sri Krishna Gana Sabha,\index{sabha@\textit{sabhā}} Chennai in 1991.

Prof. T. R. Subramaniam had performed a concert in Chennai, where he started in an unconventional manner with the \textit{tillāna}\index{tillana@\textit{tillāna}} and ended the concert with a \textit{varṇam}.\index{varna (composition)@\textit{varṇa} (composition)} %\index{varnam@\textit{varṇam}}

Other examples are the concerts by artiste T. M. Krishna.\index{Krishna, T. M.@Krishna, T. M.} Examples of some such concerts performed by him are – singing the \textit{varṇam} as a main piece in a concert, singing the \textit{ālāpanā}\index{alapana@\textit{ālāpanā}} in a \textit{rāga} unconnected to the song that is to follow, singing \textit{kalpanā svarā}-s\index{svara@\textit{svara}}\index{svarakalpana@\textit{svarakalpanā}} in a \textit{rāga}\index{raga@\textit{rāga}} different than that of the composition to which it is sung, singing \textit{tānam} to \textit{kṛti}\index{krti@\textit{kṛti}} and also in \textit{rāga}-s independent of the \textit{kṛti}, bringing in non-devotional themes and non-religious themes into the main content of the concert and so on.~While such experiments may display novelty, they cannot be regarded as successful experiments as they have not gained acceptance in the art community.~Also, when such experiments are not taken up as a model to be followed by any other artiste, these experiments are not regarded to be worthy at all. Thus, any experiment, in order to be successful, must have followers and takers and also be able to survive the test of time.

\vspace{-.3cm}

\section*{Truncation of Forms}

We also see how other devotional forms like \textit{Aṣṭapadī,\index{Astapadi@\textit{Aṣṭapadī}} Taraṅgam}\index{Tarangam@\textit{Taraṅgam}} have been trimmed or cut short to fit into a concert structure.~This experimentation\index{experimentation@experimentation} has caused serious damage to the compositional form as such and distorted the original form of many works so as to make them ideal for concert presentation.

Example of an \textit{Aṣṭapadī} -

\begin{myquote}
\textit{candana-carcita-nīla-kalevara-pīta-vasana-vana-mālī \dev{।}}\\ \textit{keli-calan-maṇi-kuṇḍala-maṇḍita-gaṇḍa-yuga-smita-śālī \dev{।।}1 \dev{।।}}
\end{myquote}

\begin{myquote}
\textit{haririha mugdha-vadhū-nikare vilāsini vilāsati kelī-pare \dev{।।}} \textit{dhruvapadam \dev{।।}}
\end{myquote}

\begin{myquote}
\textit{pīna-payodhara-bhāra-bhareṇa hariṁ parirabhya sarāgam \dev{।}}\\ \textit{gopa-vadhūr anugāyati kācid udañcita-pañcama-rāgam \dev{।।} 2 \dev{।।}}
\end{myquote}

\begin{myquote}
\textit{kāpi vilāsa-vilola-vilocana-khelana-janita-manojam \dev{।}}\\ \textit{dhyāyati mugdha-vadhūr adhikaṁ madhusūdana-vadana-sarojam \dev{।।} 3 \dev{।।}}
\end{myquote}

\begin{myquote}
\textit{kāpi kapola-tale militā lapituṁ kim api śruti-mūle \dev{।}} \\ \textit{cāru cucumba nitambavatī dayitaṁ pulakair anukūle \dev{।।} 4 \dev{।।}}
\end{myquote}

\begin{myquote}
\textit{keli-kalā-kutukena ca kācid amuṁ yamunā-jala-kūle \dev{।}} \\ \textit{mañjula-vañjula-kuñja-gataṁ vicakarṣa kareṇa dukūle \dev{।।} 5 \dev{।।}}
\end{myquote}

\begin{myquote}
\textit{kara-tala-tāla-tarala-valayāvali-kalita-kalasvana-vaṁśe \dev{।}} \\ \textit{rāsa-rase saha nṛtya-parā hariṇā yuvatiḥ praśaśaṁse \dev{।।} 6 \dev{।।}}
\end{myquote}

\begin{myquote}
\textit{śliṣyati kām api cumbati kām api kām api ramayati rāmām \dev{।}} \\ \textit{paśyati sa-smita-cāru-tarām aparām anugacchati vāmām \dev{।।} 7 \dev{।।}}
\end{myquote}

\begin{myquote}
\textit{śrī-jayadeva-bhaṇitam idam adbhuta-keśava-keli-rahasyam \dev{।}} \\ \textit{vṛdāvana-vipine lalitaṁ vitanotu śubhāni yaśasyam \dev{।।} 8 \dev{।।}}
\end{myquote}

Meaning

\begin{myquote}
“He, whose sapphirine-bluish body is bedaubed with sandal paste, clad in ochry silks, and garlanded with a garland of basil leaves and flowers, and whose both cheeks are embellished by the sways of his gem-studded knobby ear-hangings, while he is romping, he that gleeful Krishna is now amidst a coterie of ravishing and coyly damsels, in a rapturous ronde…
\end{myquote}

%~ \newpage

\begin{myquote}
Someone, a milkmaid, eager to ease the weightiness of her bosomy bust is cleaving to Krishna in a overarching manner, and then in a heightened octave she is singing melodiously, in tune with his fluting, hence, he that gleeful Krishna is now amidst a coterie of ravishing and coyly damsels, in a rapturous ronde…
\end{myquote}

\begin{myquote}
Even someone, a meekish damsel, for being an inexpert in romancing she is helpless, but caused is the passion in her mind, by the romantic gesticulations of Krishna and even by his slanting and sliding glances of his verily flustered wide eyes, thus she at once started gazing at the beautiful lotus like face of the eliminator of demon Madhu, namely Krishna, and fixatedly contemplating on that face, thus, that gleeful Krishna is now amidst a coterie of ravishing and coyly damsels, in a rapturous ronde…
\end{myquote}

\begin{myquote}
Even someone neared him as though to say something in his ear, and on reaching her lover Krishna, his tickly face conveniently turned sideways in all ears for her, thus she kissed his cheek conveniently and amusingly, thus that gleeful Krishna is now amidst a coterie of ravishing and coyly damsels, in a rapturous ronde…
\end{myquote}

\begin{myquote}
Someone, another milkmaid, enthusiastic in the artistry of plays of passion, and passionate to play in the waters of river Yamuna with Krishna, found him frolicking in a beautiful bower at that Ashoka tree, and she is gleefully lugging him along by his dress with her hand, thus that gleeful Krishna is now amidst a coterie of ravishing and coyly damsels, in a rapturous ronde, thus, that gleeful Krishna is now amidst a coterie of ravishing and coyly damsels, in a rapturous ronde…
\end{myquote}

\begin{myquote}
While a maiden is engrossed in ronde dancing, along with zealously dancing Krishna, she is clapping her palms in rhythm\index{rhythm@rhythm} to his fluting, while doing so the sets of her heavy wrist metal roundlets are clanking in undefinable dulcet clanks, and those rhythmic clanks are intermingling with that fluting, and that flautist of mohana vamshi, the Divine Flute, namely Krishna, is singing the praises unto her, thus, that gleeful Krishna is now amidst a coterie of ravishing and coyly damsels, in a rapturous ronde..
\end{myquote}

\begin{myquote}
Such frolicsome Krishna is even cleaving to someone, and even kissing someone, and even delighting someone, a delightful damsel, and paying his attention to another beaming and most beautiful damsel, and he is going in tow after one with her slanting glances, thus, that gleeful Krishna is now amidst a coterie of ravishing and coyly damsels, in a rapturous ronde…
\end{myquote}

%~ \newpage

\begin{myquote}
This song, which endows gloriousness to the devotees of Krishna, if sung or danced to its tunes, and which contains the arcaneness about the exquisite plays of passion of Krishna in equally arcanus Vrindavan, is articulated by Jayadeva,\index{Jayadeva@Jayadeva} thus let it radiate prosperities to one and all…” 

~\hfill (Rao 2008)
\end{myquote}

Though this is the text of the original work, it has been truncated with the first and last \textit{caraṇa}\index{carana@\textit{caraṇa}} as part of the text of the song thus –

\begin{myquote}
\textit{candana-carcita-nīla-kalevara-pīta-vasana-vana-mālī \dev{।}}\\ \textit{keli-calan-maṇi-kuṇḍala-maṇḍita-gaṇḍa-yuga-smita-śālī \dev{।।} 1 \dev{।।}}\\ \textit{haririha mugdha-vadhū-nikare vilāsini vilāsati kelī-pare}\\ \textit{śrī-jayadeva-bhaṇitam idam adbhuta-keśava-keli-rahasyam \dev{।}} \\ \textit{vṛndāvana-vipine lalitaṁ vitanotu śubhāni yaśasyam \dev{।।} 8 \dev{।।}}
\end{myquote}

Thus, this experimentation\index{experimentation@experimentation} of tailoring all the devotional songs of our Hindu pantheon to make them appear in a \textit{kṛti}\index{krti@\textit{kṛti}} format has been a very significant one. This concept also applies to some compositions of Tyāgarāja\index{Tyagaraja@Tyāgarāja} and Śyāma Śāstri\index{Syama Sastri@Śyāma Śāstri} which have multiple \textit{caraṇa}-s but they are done away with when presented as part of the concert genre.

\vspace{-.3cm}

\section*{The Edifice of Karnatic\index{Karnatic@Karnatic} Music}

Coming to the edifice of Karnatic music, it stands on three basic components – \textit{rāga,\index{raga@\textit{rāga}} tāla}\index{tala@\textit{tāla}} and \textit{sāhitya}.\index{sahitya@\textit{sāhitya}} A lot of experimentations have been done in these three areas and this has resulted in each of these branches developing into highly sophisticated areas.

Prof.~P.~Sambamoorthy\index{Sambamoorthy, P.@Sambamoorthy, P.} defines \textit{rāga} as – “A raga might be defined as a melody\index{melody@melody} mould or melody type. It consists of a series of notes, which bear a definite relationship to \textit{adhara shadja} and which occur in a particular sequence.” (Vol 2: 3)

\textit{Rāga}-s have evolved in different ways over the years and the concept of \textit{janaka-janya rāga}-s\index{janaka-raga@\textit{janaka rāga }}\index{janya-raga@\textit{janya rāga}} that we use as the norm for classification of \textit{rāga}-s today is a much later concept. The Kanakāṅgi scheme that we follow today is the treasure expounded by Govindācārya\index{Govindacarya@Govindācārya} in the \textit{Saṅgraha Cūḍāmaṇi}.\index{Sangraha Cudamani@\textit{Saṅgraha Cūḍāmaṇi}} Also, the fact that a particular \textit{rāga} is reckoned as a \textit{janaka rāga} for certain reasons and another \textit{rāga} as a \textit{janya rāga} for certain other reasons has no bearing on the practical merit of the \textit{rāga} concerned. Thus, a \textit{janaka rāga} need not necessarily have more merit than a \textit{janya rāga} in terms of its melodic content and individuality. Rather, it could be said that many scales\index{scale@scale} which have come to exist as a result of the \textit{janaka rāga}\index{janaka-raga@\textit{janaka rāga}} classification survive only through permutation and combination of notes and thus, do not have an entity of their own.

\vspace{-.3cm}

\section*{Experimentation with Notes to Create New \textit{Rāga}-s}

This concept of experimentation\index{experimentation@experimentation} with permutation and combination of notes was given full form by Tyāgarāja\index{Tyagaraja@Tyāgarāja} who created so many \textit{rāga}-s in the process. These are called \textit{vinta rāga}-s.\index{vintaraga@\textit{vinta rāga}}

Some examples of \textit{kṛti}-s\index{krti@\textit{kṛti}} and their \textit{rāga}-s are given below -

\vspace{-.3cm}

\begin{enumerate}
\itemsep=0pt

 \item \textit{Nāda-tanum aniśam} – Cittarañjanī – Ādi

 \item \textit{Vara-nārada nārāyaṇa} – Vijayaśrī – Ādi

 \item \textit{Jānakī-ramaṇa} – Śuddhasīmantinī – Ādi

 \item \textit{Ênta muddo} – Bindumālinī – Ādi

 \item \textit{Mākelara vicāramu} – Ravicandrika – Ādi

 \item \textit{Durmārgacarādhamulanu} – Rañjanī – Rūpakam

 \item \textit{Śobhillu Saptasvara} – Jaganmohinī – Rūpakam

 \item \textit{Ānanda-sāgara} – Garuḍadhvani – Ādi

 \item \textit{Banṭurīti} – Haṁsanādam – Ādi

\end{enumerate}

\vspace{-.3cm}

While this experimentation with permutation and combination of notes has opened up endless new vistas for the formation of \textit{rāga}-s, not everyone may accept these scales\index{scale@scale} as \textit{rāga}-s because they do not possess a distinct identity of their own, unlike \textit{rāga}-s like Sahana, Rītigouḷa, Ānandabhairavi, Dhanyāsi, etc which have characteristic features.

In an article in \textit{The Hindu}, musician T. M. Krishna\index{Krishna, T. M.@Krishna, T. M.} has raised this point –

\begin{myquote}
“This is not about whether we find these artificial ragas appealing, this is an aesthetic question about its ideation, structuring and interpretation and subsequent impact on musical thought. On that score they fail and have over the years caused a downward spiral in the raga-ness of Carnatic\index{Karnatic@Karnatic} music. And it is important for us to be unemotional and recognise that Muthuswami Dikshitar had an early role in triggering such a thought flow. But I will say with great trepidation and absolutely no disrespect that Tyagaraja’s\index{Tyagaraja@Tyāgarāja} impact has been deleterious. This is much to do with the larger than life presence Tyagaraja has had over the Carnatic psyche.” 

~\hfill (Krishna\index{Krishna, T. M.@Krishna, T. M.} 2017)
\end{myquote}

In response to this, P. K. Doraiswamy\index{Doraiswamy, P. K.@Doraiswamy, P. K.} (2017) writes subsequently in the same paper –

\begin{myquote}
“If I have understood TMK correctly, the burden of his article is that, while most of the traditional ragas have evolved organically and holistically over long periods, most of the post-Venkatamakhi\index{Venkatamakhin@Venkatamakhin} rāgas\index{raga@\textit{rāga}} are the result of scale\index{scale@scale} manipulation and have downgraded the very concept and spirit of a rāga, and by composing in such ragas, Tyagaraja\index{Tyagaraja@Tyāgarāja} has unwittingly become a party to it. Musicians have started treating ragas as mere swara skeletons.
\end{myquote}

\begin{myquote}
There is no agreed list of organic ragas.~To make matters simple, should we say that all pre-Venkatamakhi ragas are organic? Do we go further and say that only these are to be recognised as ragas and sung? (Semmangudi\index{Semmangudi Srinivasa Ayyar@Semmanguḍi Srīnivāsa Ayyar} said famously, “Why sing Chandrajyoti when Sahana is available?”)
\end{myquote}

\begin{myquote}
Why should experimentation\index{experimentation@experimentation} with scales be considered mutilation or inferior to existing ragas? I personally believe that it is no longer possible to create new ragas of the stature and impact of a Kalyani or a Todi and that experimentation can only produce ragas with a marginal aesthetic impact. However, on principle, experimentation should continue. It is continuing even now and its products, if sufficiently attractive, are being sung along with major ragas.”
\end{myquote}

In this context, my personal observation is that though these \textit{vinta rāga}-s\index{vintaraga@\textit{vinta rāga}} conceived by Tyāgarāja may seem as permutation- combination of just notes, his experimentation opened up new vistas in music. Also, while people were aware that the 72 \textit{melakarta}-s\index{mela@\textit{mela}} could generate many scales, it was Tyāgarāja\index{Tyagaraja@Tyāgarāja} who froze these scales as \textit{rāga}-s for posterity through his immortal compositions.

While Sahana, Rītigouḷa, Ānandabhairavī, Begaḍa do hold an exalted status, the \textit{rāga}-s created by Tyāgarāja as a product of experimentation is an important milestone in the evolution of the \textit{Rāga} system of Karnatic\index{Karnatic@Karnatic} music and has shown a new path to later composers. But, it is important to acknowledge that this experimentation has stood the test of time and also many later composers have taken cue from it. Also, every scale conceived by him has been concretized and also given the requisite \textit{rāga}-ness through his own compositions.

Thus, in this path-breaking experimentation, a study of the method of creating new \textit{rāga}-s by Tyāgarāja\index{Tyagaraja@Tyāgarāja} reveals how only certain \textit{janaka\break rāga}-s\index{janaka-raga@\textit{janaka rāga }} have been taken to conceive \textit{janya}-s.\index{janya-raga@\textit{janya rāga}} This is very important from the aesthetic point of view. For example, while he has experimented in using \textit{mela}-s\index{raga@\textit{rāga}}\index{mela@\textit{mela}} like Kharaharapriya, Śaṅkarābharaṇa, Harikāmbodhi and Māyāmāḷavagauḷa for generating a number of \textit{janya rāga}-s, he has used some \textit{mela}-s for creating just one \textit{rāga} out of it, like Kāntāmaṇi (\textit{Janya} – Śruti-rañjanī).

Another example of a beautiful experimentation is the way Tyāgarāja has handled the \textit{rāga} Kīravāṇī and two of its \textit{janya}-s - Kiraṇāvaḷī and Kalyāṇavasanta. Kīravāṇī through the composition \textit{Kaligiyuṇṭe} and its \textit{janya}-s Kiraṇāvaḷī through ‘\textit{Eṭiyocanalu}’ and Kalyāṇavasanta through ‘\textit{Nādaloluḍai}’ are classic examples of a set of \textit{rāga}-s which have resulted by an experimentation\index{experimentation@experimentation} of permutation-combination. This experimentation has actually paved a new path in Karnatic\index{Karnatic@Karnatic} music.

Similarly, three \textit{rāga}-s Varāḷi, Śubhapantuvarāḷi, Pantuvarāḷi which are all \textit{mela rāga}-s and differ only in the \textit{gāndhāra} are brilliantly focused to give the distinct colour of each. The \textit{kṛti}-s\index{krti@\textit{kṛti}} ‘\textit{Eṭi janmamidi}’, ‘\textit{Ennāḷḷu ūrake}’ and ‘\textit{Śiva Śiva Śiva enarāda}’ in the respective \textit{rāga}-s show how Tyāgarāja focuses on the \textit{gāndhāra} of each to show the distinct colour.

Also, with regard to the \textit{melakarta rāga}-s, the choice of certain \textit{melakarta}-s to compose multiple compositions and some with only a single composition is an eye opener. This has a deep musical purpose as seen in his composing many \textit{kṛti}-s in \textit{mela}-s like Kharaharapriya, Kalyāṇī, Śankarābharaṇa, Toḍi etc. but just one \textit{kṛti} in \textit{mela}-s\index{mela@\textit{mela}} like Kāntāmaṇi, Vanaspati, Vāgadhīśvarī, etc.

We see that time tested \textit{rāga}-s like Toḍi, Bhairavī, Kāmbhoji, Kalyāṇī, Śaṅkarābharaṇa find new dimensions in Tyāgarāja’s\index{Tyagaraja@Tyāgarāja} compositions. The concept of “art music” came to be conceived after the time of Trinity\index{Trinity, The@Trinity, The}\index{Trinity-era, post- @Trinity-era, post-} where the \textit{manodharma}\index{manodharma@\textit{manodharma}} elements have gained a predominant position in concerts. Thus, a great divine composer as Tyāgarāja who actually enriched the \textit{Rāga} system through his contribution does not in any way deserve the allegation that he diluted the same.

\vspace{-.3cm}

\section*{Experimentation of Different Approaches to Same Rāga\index{raga@\textit{rāga}}}

\vspace{-.2cm}

Kalyāṇī by Tyāgarāja\index{Tyagaraja@Tyāgarāja} through various \textit{kṛti}-s\index{krti@\textit{kṛti}} and also comparison of the same \textit{rāga Kalyāṇī} as handled by Tyāgarāja and Dīkṣitar offer interesting study.

Śubhapantuvarāḷi by Tyāgarāja and Dīkṣitar can also be taken for a comparative study of how both composers have experimented with different \textit{gamaka}-s,\index{gamaka@\textit{gamaka}} namely \textit{kampita}\index{kampita@\textit{kampita}} and \textit{jāru}\index{jaru@\textit{jāru}} respectively to give two colours to the same scale.\index{scale@scale}

Again, the experimentation\index{experimentation@experimentation} that Dīkṣitar has done with Śankarābhara\-ṇa is remarkable. He has used the \textit{rāga} to compose classic pieces like \textit{‘Akṣayaliṅga vibho’, ‘Dakṣināmūrte’, ‘Śrī kamalāmbikayā’} as well as \textit{noṭṭu svara}-s\index{Nottusvara@Noṭṭusvara} like \textit{‘Śyāmale mīnākṣi’, ‘Śakti -sahita-Gaṇapatim’, ‘Rāmajanārdana’} and so on. That a \textit{rāga} can have different musical expressions is an idea conveyed by Dīkṣitar in a novel way through these compositions.

Similarly, Tyāgarāja was a master in experimenting with the colour of \textit{rāga}-s. While most of his compositions in Panthuvarāḷi start on the \textit{tāra ṣaḍja}, he has also experimented with the same \textit{rāga} without touching the \textit{tāra ṣaḍja} at all throughout, in the composition ‘\textit{Śobhāne}’ whose range is from \textit{mandra dhaivata}\index{mandra sthayi@\textit{mandra sthāyi}} to \textit{madhya sthāyi\index{madhya sthayi@\textit{madhya sthāyi}} dhaivata}.

In the domain of \textit{tāla}-s\index{tala@\textit{tāla}} also, Tyāgarāja was a new trend-setter. Unlike his contemporaries, he experimented with the \textit{deśādi tāla} as a rhythmic expression for many of his compositions. While some may say that the set of songs in this \textit{tāla} as devised by Tyāgarāja\index{Tyagaraja@Tyāgarāja} has been instrumental in diluting the quality of music (as against the \textit{cauka kāla\index{cauka kala@\textit{cauka kāla}} kṛti}-s which unfold the \textit{rāga}-s in a grand way), a positive point in this is the simplicity in structure which has resulted in a larger reach of the music itself. His \textit{deśādi tāla kṛti}-s have become the basic material for many learners of Karnatic\index{Karnatic@Karnatic} music as it is very easy to comprehend.

Examples of such \textit{kṛti}-s\index{krti@\textit{kṛti}} – \textit{‘Marugelara’, ‘Brova bhārama’, ‘Mākelarā’,\break ‘Banṭurīti’, ‘Êntanercina’, ‘Gānamūrte’, ‘Calamelarā’, ‘Śrī Raghukula’,\break ‘Têliyaleru Rāma’}.

\vspace{-.4cm}

\section*{Experimentations with Respect to\hfill \break Compositions}

\vspace{-.2cm}

The next aspect of Karnatic music, which has been subject to constant experimentation\index{experimentation@experimentation} is the composition. Composition is the pivotal material around which Karnatic music has evolved. The compositions of great \textit{vāggeyakāra}-s\index{vaggeyakara@\textit{vāggeyakāra}} have been the main material of presentation and all \textit{manodharma}\index{manodharma@\textit{manodharma}} aspects have been extensions of it.

\vspace{-.1cm}

While this is true of both Hindustani\index{Hindustani@Hindustani} and Karnatic music, the concept, import and purpose of a composition is quite different in the two systems. According to noted music scholar Dr. Prem Latha Sharma (1980),

\begin{myquote}
“Composition is a repository of the nuances and contours of a rāga and sets the model of the blending of text, melody\index{melody@melody} and rhythm.\index{rhythm@rhythm} Improvisation\index{improvisation@improvisation} without composition could be accepted as the highest form of art – music, but it should not be forgotten that composition not only serves as the basis of gaining entry into the intricate portals of a rāga\index{raga@\textit{rāga}} but also an interesting and enlivening corollary of independent alapti. It is a very strong component of the oral tradition that sustains rāga, tāla\index{tala@\textit{tāla}} and text. Hence the loss of its due importance is fraught with dangers like over abstraction, complete annihilation of poetic content, lack of a frame etc.”
\end{myquote}

\vspace{-.1cm}

But in Karnatic\index{Karnatic@Karnatic} music, compositions mean much more than that as the \textit{sāhitya}\index{sahitya@\textit{sāhitya}} is as profound as the music itself. Compositions of Dīkṣitar\index{Muttusvami Diksita@Muttusvāmi Dīkṣita} or Tyāgarāja\index{Tyagaraja@Tyāgarāja} for example, offer a treasure trove of information on spiritual and devotional aspects.

%~ \vspace{-.1cm}

The \textit{sāhitya} in Karnatic music predominantly addresses or describes God in various forms. The theme is, of course, varied – some are in praise of God, some describe deities and temples, some highlight teachings and doctrines of Hinduism. The Trinity\index{Trinity, The@Trinity, The} and many other composers have composed innumerable compositions to highlight various such aspects. Especially, the compositions of Trinity are a mark of their musical genius, deep sense of spirituality and intense devotion to God. \textit{Bhakti}\index{bhakti@\textit{bhakti}} was the nucleus of their lives and this was expressed through an outpour of musical compositions.

%~ \vspace{-.1cm}

On the importance and sacredness of the \textit{sāhitya} of compositions in Karnatic music, some points made in a panel discussion chaired by the author at the conference of The Music Academy on 21st December 2018, is pertinent to quote here -

\begin{itemize}
\itemsep=0pt

 \item If we look at the history of Karnatic music, it can be classified into: pre-Trinity, Trinity\index{Trinity, The@Trinity, The} and post-Trinity period. This is not just a matter of dividing the composers according to the era that they belong to. Rather, it is because of the turns and twists that music has taken over these three periods. In the pre-Trinity period, the lyrics\index{sahitya@\textit{sāhitya}} have been the most important element of compositions. This is evidently seen in forms like \textit{Tevāram,\index{Tevaram@Tevāram} Tiruppugaḻ,\index{Tiruppugal@\textit{Tiruppugaḻ}} Divyaprabandham,\index{Divya-prabandham@\textit{Divya-prabandham}}\index{prabandha@\textit{prabandha}} Taraṅgam,\index{Tarangam@\textit{Taraṅgam}} Aṣṭapadī}\index{Astapadi@\textit{Aṣṭapadī}} etc. The \textit{sāhitya}\index{sahitya@\textit{sāhitya}} is the predominant element and hence melody\index{melody@melody} has been kept very simple in all these forms. We also see that the melody is repetitive in its nature so as to give more focus on the \textit{sāhitya}. It is only in the time of the Trinity that music took a different course where we see that in these compositions, apart from the \textit{sāhitya} being rich and lofty in its content, we also see that music has evolved at a very high level, which continued through to the post-Trinity period.

 \item \textit{Sāhitya} in Karnatic\index{Karnatic@Karnatic} music is not just a cluster of words. They convey high ideologies and musicians who understand its deeper import will agree that \textit{sāhitya} and \textit{saṅgīta} are interwoven so beautifully and it is important to understand the deeper layers of both to make the music holistic.

 \item A prolific \textit{sāhitya} with profound \textit{bhāva}\index{bhava@\textit{bhāva}} warrants that the \textit{saṅgīta} should match the \textit{sāhitya}. The rich \textit{sāhitya} instills a sense of responsibility to give good music and to understand that no compromises are to be made both in the \textit{sāhitya} and \textit{saṅgīta} quotient of the composition.

\end{itemize}

In this context, let us take this common \textit{śloka}\index{sloka@\textit{śloka}} for example –

\begin{longquote}
\textit{śuklāmbara-dharam viṣṇuṁ śaśi-varṇaṁ caturbhujam \dev{।}}\\ 
\textit{prasanna-vadanaṁ dhyāyet sarva-vighnopaśāntaye \dev{।।}}
\end{longquote}

This can be just recited and still have its meaning conveyed but the impact will be multifold when sung with \textit{bhāva}.\index{bhava@\textit{bhāva}} Likewise, a passage like the \textit{Arutpa} requires a deeper musical interpretation to unfold its meaning –

\begin{longquote}
\tamil{பெற்ற தாய் தனை மக மறந்தாலும்\\ பிள்ளையைப் பெரும் தாய் மறந்தாலும்\\ உற்ற தேகத்தை உயிர் மறந்தாலும்\\ உயிரை மேவிய உடல் மறந்தாலும்\\ கற்ற நெஞ்சகம் கலை மறந்தாலும்\\ கண்கள் நின்றிமைப்பது மறந்தாலும்\\ நற்றவத்தவர் உள்ளிருந்தோங்கும்\\ நமச்சிவாயத்தை நான் மறவேனே}
\end{longquote}

\begin{longquote}
\textit{peṟṟa tāy taṉai maga maṟantālum\\ piḷḷaiyaip pêrum tāy maṟantālum\\ uṟṟa degattai uyir maṟandālum\\ uyirai meviya uḍal maṟandālum\\ kaṟṟa nêñjagam kalai maṟandālum\\ kaṇgaḷ niṉṟimaippatu maṟandālum\\ naṟṟavattavar uḷḷiruntoṅgum\\ namaccivāyttai nāṉ maṟavene}
\end{longquote}

This verse also can be just recited but when it is sung with the \textit{bhāva}\index{bhava@\textit{bhāva}} of the lyrics,\index{sahitya@\textit{sāhitya}} one gets transformed as Vaḷḷalār himself.

Some points which can be observed are -

\vspace{-.3cm}

\begin{itemize}
\itemsep=0pt

 \item \textit{Sāhitya}\index{sahitya@\textit{sāhitya}} gives \textit{saṅgīta} the exalted status and the combination of words and syllables is what makes the music sublime. The varied \textit{rasa}-s can be felt and the highest level of aesthetic experience can be attained with this combination.

 \item The Indian culture is one of sound tradition. Music without words has never been seen in our culture. Sadāśiva Brahmendra, Purandaradāsa,\index{Purandaradasa@Purandaradāsa} Annamācārya,\index{Annamacarya@Annamācārya} Tyāgarāja,\index{Tyagaraja@Tyāgarāja} Dīkṣitar\index{Muttusvami Diksita@Muttusvāmi Dīkṣita} – all sang and gave the \textit{sāhitya} a unique status. 

 \item The idea of poetry as divorced from the idea of singing has never been there. It is actually a problem to imagine music without words. The most beautiful \textit{sāhitya} that Tyāgarāja has created is the word Rāma itself. The way \textit{rā} and \textit{ma} in the context of songs are used, embellishes the beauty of the song. To pronounce the word Rāma in the context of the music is an ecstatic thing.

 \item The element of sound or \textit{dhvani} is important in music. That is why we relish songs even in languages unknown to us. Telugu or Kannada as a language may be unknown but still enjoyed by a Tamil listener and vice versa. Language evolves not just by meaning but by the beauty of sound also. 

 \item In Karnatic\index{Karnatic@Karnatic} music, sound is married to music. For example, in the \textit{Śrī rāga pañcaratna} of Tyāgarāja,\index{Tyagaraja@Tyāgarāja} we see a perfect beauty in the way the \textit{sāhityā} and \textit{saṅgīta} complement each other in each of the \textit{caraṇa}-s.\index{carana@\textit{caraṇa}}

 \item The \textit{sāhitya} of our great \textit{vāggeyakāra}-s\index{vaggeyakara@\textit{vāggeyakāra}} is a treasure of information about our spirituality, culture and deities of the Hindu pantheon. It instills \textit{bhakti}\index{bhakti@\textit{bhakti}} in the minds of the singer and listener and takes them to the highest realms of divinity.

\end{itemize}

\vspace{-.2cm}

\textit{Sāhitya}, is thus the carrier of a rich tradition also. There are some anti-\textit{sāhitya} brigades who say that music can exist by itself as an abstract form but the reality is that you cannot separate \textit{sāhitya}\index{sahitya@\textit{sāhitya}} and music. Dismissing \textit{sāhitya} as redundant or as an impediment to music is nothing short of doing harm to the great tradition of Karnatic\index{Karnatic@Karnatic} music.

Here, it is pertinent to mention how Dīkṣitar has experimented\index{experimentation@experimentation} with the western scale\index{scale@scale} of Śaṅkarābharaṇa and created tiny compositions on various Hindu deities. This innovative experimentation serves two purposes –

\vspace{-.3cm}

\begin{enumerate}
\itemsep=0pt

 \item To introduce music in a simple and innovative way.

 \item But beyond this, it is interesting how he has composed these songs on all the important Hindu Gods and Goddesses, thus giving a simple means for children to be aware of our own culture in a novel way.

\end{enumerate}

\vspace{-.3cm}

Some examples of songs and the respective God/ Goddess -

\begin{longquote}
\textit{Śakti-sahita-gaṇapatim – Gaṇeśa} \\ \textit{Śyāmale mīnākṣi – Devī}\\ \textit{Vara-śiva-bālam vallī-lolam  - Muruga}\\ \textit{Rāma janārdana rāvaṇa-mardana – Rāma} \\ \textit{Pārvatīpate sadā pālayāśu – Śiva}\\ \textit{Pāhi durge bhaktaṁ te padmakare – Durgā} \\ \textit{Varada-rāja pāhi vibho – Viṣṇu}\\ \textit{Āñjaneyaṁ sadā bhāvayāmi - Hanumān}
\end{longquote}

The composition which was held in great sanctity is, of late, subject to a lot of experimentation. It is a matter of concern that compositions are being used more as a peg to hang the \textit{manodharma}\index{manodharma@\textit{manodharma}} of the artiste rather than being presented for what it conveys about the thought process of the composer.

Experimentations with regard to taking liberty in presenting the composition according to one’s own wish, incorporating cosmetic changes may appear trivial but in the long run, it will totally distort the original composition and its true form will become obscure. In order to preserve the sanctity of the compositions, it is important for artists to exercise restraint and be faithful to the composer. The artiste does have the domain of \textit{manodharma}\index{manodharma@\textit{manodharma}} where he can show his virtuosity.

In this context, some experimentations are proving to be alarming and detrimental to the core values of Karnatic\index{Karnatic@Karnatic} music.

For example, some instrumentalists without cognizance of the \textit{sāhitya}\index{sahitya@\textit{sāhitya}} of compositions, just try to use it as a tool to display instrumental techniques and virtuosity. Such experimentations\index{experimentation@experimentation} spoil the ethos of the music system. In such cases, it would do well for artists to create their own compositions to prove their skill in whatever instrument they are playing.

\textbf{Other Trends}

\textit{Jugalbandi}\index{jugalbandi@\textit{jugalbandi}} is another latest trend where with the zeal to project two systems of music on the same dais, both get mingled and lose their identity. Here, apart from the entertainment quotient, no great musical purpose is served. Most often, we also see that in \textit{Jugalbandi}, it is always the Karnatic music which struggles to maintain its identity. Whether it is the choice of songs, \textit{rāga}, \textit{gamaka}-s\index{gamaka@\textit{gamaka}} or accents, it is seen that Karnatic musicians lean towards the Hindustani\index{Hindustani@Hindustani} system and finally, the whole concert gets a flavor of the Hindustani style.

Some very alarming trends have crept up recently where Christians have even distorted the original words of compositions like \textit{gīta}-s\index{gita@\textit{gīta}} of Purandaradāsa\index{Purandaradasa@Purandaradasa} and various \textit{svarajati}-s, and replaced them with words pertaining to their own religion. In the name of teaching Karnatic music, a religion–brainwash is being done much to the agony of Hindus. One such example is the book \textit{Christuva Tamilisai Bodhini} authored by E. K. Lakshmi Bai and Renuka Suresh. They justify the alteration by calling the original \textit{Gīta}-s as \textit{Tamizh Isai}, (which actually does not make any sense) and say that in their pursuit to reach the music to Christians, this change has been incorporated. But such tampering of the compositions in the name of experimentation and using it for cultural appropriation\index{appropriation@appropriation} and conversion is to be totally condemned. It is detrimental to the very foundation of Karnatic music. Similary, ventures like the ‘Carnatic Rock band’ are highly undesirable as they amount to vandalizing the very identity of Karnatic music.

In the instrumental front, also, we see a lot of experimentations taking place. Instruments belonging to western music like violin, mandolin, saxophone, clarinet, keyboard, guitar, drums have been used as mediums to perform Karnatic music.

Here we see that instrumental musicians hold two theories –

\begin{enumerate}
\itemsep=0pt

 \item That instrumental music should reflect the Vocal Bāni and hence portrayal of \textit{sāhitya}\index{sahitya@\textit{sāhitya}} through the instrument is also important. The Lalgudi school is one example of the expounder of this theory.

 \item Artists like Ganesh Kumaresh and Rajesh Vaidhya are of the contention that it is more important to bring out the versatility of the instrument in terms of its range, technique etc. and hence, these musicians also resort to playing their own compositions which focus on exploring the full range of the instrument.

\end{enumerate}

Artistes have been constantly researching and experimenting\index{experimentation@experimentation} on how to make these instruments adaptable to Karnatic\index{Karnatic@Karnatic} music. These experiments with various instruments have met with varied levels of success so far as adapting them best to the core values of the musical form goes.

\vspace{-.3cm}

\section*{Conclusion}

\vspace{-.2cm}

Experimentations are indeed very essential for the growth and flourishment of an art like Karnatic music. An artist’s creativity takes concrete shape only when he tries to experiment with whatever material he has acquired through training, listening, intense introspection and analysis. The outcome of the experimentation is what makes the art take different courses over different periods of time. But, the most important thing in making an experiment meaningful is in being faithful to the musical tradition in terms of preserving its core values.

While the intelligence quotient in artistes has had a great impact in recent years in making the music more thrilling and entertaining, it is important to remember that the primary purpose of Karnatic music is to enrich and elevate the listener. In this context, experimentations should be done with a sense of responsibility and integrity towards the rich tradition that has been nurtured by great \textit{vāggeyakāra}-s\index{vaggeyakara@\textit{vāggeyakāra}} and \textit{vidvān}-s over so many years.

\textit{\textbf{Note:} Since this is a practical subject, referential material in terms of books is minimal. My Ph.D work has been on a related topic – “Karnataka music concerts – an analytical study.” I have been in the field as a musician, musicologist, journalist and cultural organizer for 30 years and many things in this research paper are quoted from first-hand knowledge of the subject through practical observations. Also, I have made many path breaking experimentations\index{experimentation@experimentation} in the music field through my own organization Mudhra for the past 25 years and also been witness to several such experimental concepts by other organizations and all these provide the core material for this research paper.}


\section*{Bibliography}

\begin{thebibliography}{99}
\itemsep=0pt

 \bibitem{chap4-key01} Bhaskar, Radha. (2000). “Karnataka Music Concerts – An Analytical Study”. PhD Dissertation. University of Madras.

 \bibitem{chap4-key02} —. (2018). “Understanding Rāgas Through Compositions”. Lecture-Demonstration. 24th December 2018. Chennai: Sri Thyaga Brahma Sabha.

 \bibitem{chap4-key03} —. (2018). “Importance of Sahithya in Karnatic\index{Karnatic@Karnatic} Music”. As Moderator in Panel Discussion with Neyveli Santhanagopalan, Sriram Parasuram, Dr.R S Jaylakshmi as panelists. 21st December 2018. Chennai: Music Academy. Recorded by the Music Academy.

 \bibitem{chap4-key04} —. (2018). “Art of Listening to Music”. As Moderator in Panel Discussion with Neyveli Santhanagopalan, Sriram Parasuram, Ananya Ashok as panelists. 27th December 2018. Chennai: Sri Parthasarathy Swami Sabha.

 \bibitem{chap4-key05} Deva, B. C. (1980). \textit{Indian Music}. New Delhi: Indian Council for Cultural Relations and New Age International (P) Limited.

 \bibitem{chap4-key06} Doraiswamy,\index{Doraiswamy, P. K.@Doraiswamy, P. K.} P. K. (2017). “Organic vs inorganic”. The Hindu. 12th May 2017. $<$\url{https://www.thehindu.com/todays-paper/tp-features/tp-fridayreview/organic-vs-inorganic/article18433379.ece}$>$. Accessed on 30 Jan 2020.

 \bibitem{chap4-key07} \textbf{\textit{Gīta Govinda}}. See Rao (2008).

 \bibitem{chap4-key08} Krishna, T. M.\index{Krishna, T. M.@Krishna, T. M.} (2017). “A case of aesthetic extravagance”. \textit{The Hindu}. 04th May 2017. $<$\url{https://www.thehindu.com/entertainment/music/tyagarajas-musical-span-and-insight-reiterates-his-genius/article18384186.ece}$>$. Accessed on 30 Jan 2020.

 \bibitem{chap4-key09} Parvathi, S. (1956). “Evolution of Concerts”. PhD Dissertation. University of Madras.

 \bibitem{chap4-key10} Ranjani, S. “Facets of Krti.” M.A. Dissertation. University of Madras.

 \bibitem{chap4-key11} Rao, Desiraju H. (2008). “Jayadeva Gita Govindam”. Giirvaani.\break $<$\url{https://www.sanskritdocuments.org/sites/giirvaani/giirvaani/gg_utf/gg_utf_intro.htm}$>$. Accessed on 30 Jan 2020.

 \bibitem{chap4-key12} Rao, T. K. Govinda. (Ed.) (1999). \textit{Compositions of Tyāgarāja}. Chennai: Gnanamandir Publications.

 \bibitem{chap4-key13} Ramanujachari, C and Raghavan, V.\index{Raghavan, V.@Raghavan, V.} (1966). \textit{The Spiritual Heritage of Tyagaraja}. Madras: Sri Ramakrishna Math.

 \bibitem{chap4-key14} Ruth, Caitlin Ann. (1980). “Variability and change in three Karnataka \textit{Kriti}-s - A Study of South Indian Classical Music”. PhD Dissertation. Brown University.

 \bibitem{chap4-key15} Sambamoorthy, P.\index{Sambamoorthy, P.@Sambamoorthy, P.} (1963-69). \textit{South Indian Music}. (Six Vol.s). Madras: The Indian Music Publishing House.

 \bibitem{chap4-key16} —. (1960). \textit{History of Indian Music}. Madras: The Indian Music Publishing House.

 \bibitem{chap4-key17} —. (1962). \textit{Great Composers}. Madras: The Indian Music Publishing House.

 \bibitem{chap4-key18} Sharma, Prem Lata. (1990). ``The Treatment of Musical Composition in the Indian Textual Tradition." \textit{Journal of the Indian Musicological Society}. 21/1-2: pp. 1-6.

 \bibitem{chap4-key19} Sriram, V. (2010). “Ariyakkudi T Ramanuja Iyengar”. \textit{Madras Heritage and Carnatic\index{Karnatic@Karnatic} Music}. $<$\url{https://sriramv.wordpress.com/2010/11/24/ariyakkudi-t-ramanuja-iyengar/}$>$. Accessed on 30 Jan 2020.

 \bibitem{chap4-key20} Viswanathan, T. (1974). “Rāga Ālāpanā in South Indian Music”. Ph.D Dissertation. Wesleyan University.

 \end{thebibliography}

\label{endchapter4}
