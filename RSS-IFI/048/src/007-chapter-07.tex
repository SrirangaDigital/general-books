
\chapter{Is Karnatic\index{Karnatic@ Karnatic} Music a Bastion of Brahminical\index{brahminical@ brahminical} Patriarchy?\index{patriarchy}}\label{chapter7}

\Authorline{V. B. Arathi\footnote{pp. 185--209. In: Meera, H. R. (Ed.) (2020). \textit{Karnāṭaka Śāstrīya Saṅgīta - Past, Present, and Future.} Chennai: Infinity Foundation India.}}

\vspace{-.3cm}

\lhead[\small\thepage\quad V. B. Arathi]{}

\begin{flushright}
\textit{(arathi.vbr@gmail.com)}
\end{flushright}


\section*{Introduction}

For more than a century, misleading narratives on Indian culture and life, which violently intruded into our academia and intelligentsia during British rule, have continued to silently do the same in post-independence India. The Breaking-India forces have been generously pushing in their colonial agenda, misusing the open structure of the Hindu society, wherein healthy criticism and debate have always been welcome. In the last few decades, they and their Indian delegates, posing as benefactors to modern Indian thought, have injected many a duplicitous theory and have been cheekily grabbing the reins of the Indian cultural narrative at an alarming pace. These unverified and misleading thought lines brazenly attempt to trivialize and demonize the native culture to weaken the pride and cultural identity, thenceforth opening doors for mass conversions.

The agendum has silently but rapidly penetrated India’s polity, textbooks and mainstream media while its tentacles are extending palpably into Indian theatre, cinema, fashion and the religious dialogue in the nation. The unsuspecting and broadminded Hindu, has allowed the agendum to work its way, only to be, of late, awakened to the threats of losing his voice in his own country. The one domain, where these divisive thought seeds have somehow not penetrated much, is the domain of Karnatic\index{Karnatic@Karnatic} classical music.

However, desperate attempts to do so are evident from the recent developments in the art narrative of South India.~Misleading statements and writings by leftist voices like T. M. Krishna\index{Krishna, T. M.@Krishna, T. M.} are some symptoms.~Musicians and predisposed critics like him use their popularity and erudition to vent out their Hindu-phobia, in the pretext of introducing ‘objective’ criticism into Indian classical art. But being heavily biased, subjective and intolerant in tone, their criticisms carry neither new insights nor a holistic comprehension of the classical art tradition of India.

The focus of this paper will be around the latest injections of the caste narratives into Karnatic music domain, by musicians like T. M. Krishna (henceforth, Krishna). As supplementary factors in contesting the camouflaged allegations on Karnatic music heritage, a few points, ostensibly out of the purview of this study, are also taken up in the commencement. A broad observation of the how the ‘Breaking India’ agendum tries to penetrate the Indian cultural narrative is also done.

\vspace{-.4cm}

\section*{Distorting Key Terminologies like \textit{Brāhmaṇa}\index{brahmana @\textit{Brāhmaṇa}} and \textit{Brāhmaṇya}}

Many terminologies like \textit{dharma},\index{dharma@\textit{dharma}} \textit{śraddhā},\index{sraddha@\textit{śraddhā}} \textit{śāstra}, \textit{guru},\index{guru@\textit{guru}} \textit{brāhmaṇa}, \textit{ācāra,} etc., are key to the debates on Indian culture, religion and social aspects.~It is most important that we understand them in their true sense and context, deriving directly from the sources/texts.~Any attempt to hastily translate or substitute these terms, would dilute or shrink their meanings, damaging the life-force they carry.

For millennia, till Saṁskṛta remained the pan-Indian language for debate and documentation, these key words were understood in their true sense and context in the rational dialogue.~About a century and half ago, the outsiders’ agendum began to intervene right here. Dishonest translations of the key terminologies were done and redefined and re-interpreted to serve the divisive purpose. These tainted words and meanings were then floated across and made very common in the public discourse.~The leftist’s strong hold on post-independence India’s academia, media, cinema and theatre enabled this agendum to work its way rapidly and effectively into the commoners’ minds.

Eventually, the word \textit{Veda} (suggestive of knowledge, awareness, text that documents wisdom, dialogue and insights of many enlightened seers, etc.,) came to be understood as just an ‘old Hindu text of chants and rituals’.~\textit{Dharma (dhāraṇād dharma ityāhuḥ)}\index{dharma@\textit{dharma}}\endnote{There are many scholarly definitions on \textit{Dharma}\index{dharma@\textit{dharma}}. This is just the simplest and most common of them.} which originally suggested a concept, deed, norm or system that promotes the welfare of \textit{vyakti} (the individual) and \textit{samaṣṭi} (the society/world), came to mean just ‘a religious path or religious norm’. The word \textit{śraddhā}\index{sraddha@\textit{śraddhā}} which suggested ‘a strong conviction subject to correction and verification on experience’ now got to mean ‘unconditional faith to a theology’. Similar was done with the word \textit{brāhmaṇa}\index{brahmana@\textit{brāhmaṇa}} and \textit{brāhmaṇya}.

For millennia, the terms \textit{brāhmaṇa} and \textit{brāhmaṇya} were being used in India in rational debate in public, seldom causing any caste feel. That is because they were understood in their true sense. The word \textit{brāhmaṇa,} except in a subjective reference, never was used as a caste nomenclature, particularly in rational debates. The commonly used terms \textit{brāhmaṇottama} (noble Brahmin)and \textit{brāhmaṇādhama} (fallen Brahmin) clearly suggests that a \textit{brāhmaṇa} was regarded high not merely for his birth, but for the \textit{brāhmaṇya} he implemented in his lifestyle.

But the interventions of the outsiders’ narratives infused the terrible caste feel into these words.~The words \textit{brāhmaṇa} and \textit{brāhmaṇya} were methodically distanced from their original meaning, context, intent and content.~Many subtler imports of the words were chiseled out, making them sound absolutely subjective.~Thus, the word ‘\textit{brāhmaṇa’} was made to mean only ‘caste’. This misleading meaning and definition were even successively standardized in the global dictionaries and absolutely insulated from any verification by the original viewpoint.

Further\textit{,} the word\textit{ ‘brāhmaṇya’} which suggests strict pious adherence to \textit{ācāra} (pious life style) and \textit{vicāra} (profound thought) was cleverly substituted by the word ‘brahminism’ (Harshananda 2008: 337). The word brahminism carries an extreme caste-tone and is non-suggestive of the \textit{ācāra}-\textit{vicāra} angle.~The word \textit{brāhmaṇya} in the modern Saṁskṛta dictionaries (even after dilutions!)~means - pious/ devoted to sacred knowledge/ religious/ piety/ priestly profession, etc., But, brahminism, even according to the widely accessed Wikipedia is “the domination of Indian society by the priestly class of Brahmins and their Hindu-ideology”. Obviously, the words ‘brahmin’ and ‘brahminism’ were successfully manipulated to give the images of an ‘egotistical personality’ and ‘authoritarian system of life’. Stray instances of prejudiced brahmin individuals’ behavior were compiled and even magnified to serve as ‘proofs’ for a generalized allegation. Thus, many distorted interpretations were floated against the \textit{brāhmaṇa}\index{brahmana@\textit{brāhmaṇa}} and his every pursuit successfully.

Unfortunately, the native narratives remained either ignorant or a helpless onlooker to such semantic inflictions on its own profound terminology in last few decades. In the public space, these words have been so deeply dyed in the caste colour, that uttering the word ‘\textit{brāhmaṇa’} causes discomfort and even penitence(!) in the brahmin-borns while the non-brahmins\index{non-brahmins@non-brahmins} get sensitive, unconsciously expecting to be offended in the discourse that follows. This has its powerful implications even on rational dialogue, and tends to generate more subjective impressions than an objective understanding.

A glimpse into the original meanings of the word \textit{brāhmaṇa.} The word \textit{Brahma} has a few contextual meanings-the almighty, vital energy, \textit{Veda, yajña, guru.}\index{guru@\textit{guru}}\index{yajna@\textit{yajña}}

Accordingly, the word \textit{brāhmaṇa} gives these meanings-

\textit{brahmajñānād brāhmaṇaḥ} / \textit{brahmacaryād brāhmaṇaḥ}\index{brahmacarya@\textit{brahmacarya}}

= one who realizes the \textit{Brahman}\index{brahman@\textit{Brahman}} (the Almighty), one who has achieved self-mastery, one who studies the \textit{Veda}, one who serves / preserves the knowledge systems, one who performs the \textit{yajña-}s.

(\textit{yajña} means 1. Vedic rituals or vows 2. A duty or task done without desire for personal gain, or wherein the fruits are offered to God or to the society).\endnote{Lord Kṛṣṇa explains in many verses, even repeatedly, about how duty/work must be done as a \textit{yajña}\index{yajna@\textit{yajña}}, i.e without desperation for fruits thereof. He also explains how Gods are pleased by such work and how the doer must share the fruits of his actions with his fellow beings – \textit{yajñārthāt karmaṇaḥ}\index{karman@\textit{karman}} (\textit{Bhagavadgītā}\index{Bhagavadgita@\textit{Bhagavadgīta}} 3.9) \textit{tadarthaṁ karma kaunteya mukta-saṅgas samācara} (\textit{Bhagavadgītā} 3.9) \textit{devān bhāvayatānena te devā bhāvayantu vaḥ} (\textit{Bhagavadgītā} 3.11) \textit{karmaṇaiva hi saṁsiddhim āsthitā janakādayaḥ | loka-saṅgraham evāpi sampaśyan kartum arhasi} (\textit{Bhagavadgītā} 3.20) \textit{yogasthaḥ kuru karmāṇi saṅgaṁ tyaktvā dhanañjaya | siddhyasiddhoḥ samo bhūtvā} (\textit{Bhagavadgītā} 2.48).}

In this background, the word \textit{brāhmaṇa}\index{brahmana@\textit{brāhmaṇa}} refers to an individual or community that is -

\vspace{-.3cm}

\begin{enumerate}
\itemsep=0pt

 \item Pious, austere, meditative 

 \item Knowledgeable, resourceful, advisors and benefactors of the society 

 \item A teacher, trainer, researcher or one who preserves the knowledge traditions.

\end{enumerate}

The \textit{brāhmaṇa}\index{brahmana@\textit{brāhmaṇa}} offered his services in teaching, research, advice, policy making, law and spiritual guidance to the society, and all that for free. The society voluntarily supported his material needs out of gratitude. This was the beautiful relationship between a \textit{brāhmaṇa} and the others in the Indian society since ages.

But, the word brahmin was shrunk to mean just a ‘caste’ and coloured to such an extent that it would give the sense- ‘imprudent’ priest-hood\endnote{For instance, in Christianity, ``Priesthood is the power and authority of God given to man, including the authority to perform ordinances and to act as a leader in the church.''(\textbf{Editor's Note}: The above is a quote from Wikipedia. To cite from a standard source, ref. ``priesthood is the power and authority of God given to man by an (\textit{sic}) holy ordinance'' Matthews, Clinton. (2016). \textit{Priesthood Through Scriptures}. (Softcopy). Xlibris.com.)}. Our textbooks, newspapers, cinema, theatre and every public space possible was used to float only this distorted sense. In fact, T. M. Krishna\index{Krishna, T. M.@Krishna, T. M.} in his writings, profusely uses the words brahmin and Brahminism is their most demonic sense.

On the other hand, native scholars and thinkers hardly noticed what was being conspired. Even amongst those who did notice, few dared to fiercely counter the powerful conspiracy. Down the decades, this has created a big disadvantage to the native narrative.

Now, whenever we use these terms, we need to start by first cleansing and clarifying the intended sense quite very elaborately. Such is the brainwashing! Moreover, when someone tries to use the word \textit{brāhmaṇa} in the original sense, the ‘liberals’ vehemently object that ‘standard definitions’ are being overruled. The stigma attached to this word is so high that, people preferring to remain ‘non-controversial’ even substitute the word with words like ‘\textit{sajjana’,} ‘religious’ or ‘righteous’, etc., But these people are unknowingly causing further disadvantage to the native narrative by doing so.

We need to bring back into use vociferously, these key terminologies in their original sense and context most assertively, without substituting them.~We need to gather supportive interpretations from vedic, epic, classic and folk sources and stridently carry them into world view. Saṁskṛta scholarship and a dedicated executive team must work together to re-establish the true meanings of these terms and reach them out to the last man.

\newpage

\vspace{-.3cm}

\section*{Isolation of Saṁskṛta Rhetoric in Cultural Dialogue and Public Space}

\vspace{-.2cm}

Saṁskṛta, being the language of documentation, debate and the source of most of the terminologies used in India’s socio-religious and cultural dialogue, deserves a big representation in the Indian narratives. But the very opposite has been happening. Saṁskṛta rhetoric has been marginalized and trivialized to allow outsiders’\endnote{The word ‘Outsider’ here does not refer to the ethnicity, but to the ‘stand point’ wherein, the native socio-cultural context and perspectives of India are totally ignored and dominated by an alien narrative, which does no justice to facts.} narrative to float translations and interpretations detrimental to the native ones. Since most of our art treatises are composed in Saṁskṛta (some in \textit{desi}\index{desi@\textit{deśī}} languages too), we must duly connect back to Saṁskṛta to derive the exact spirit of our art narratives.

Step-motherly treatment to Saṁskṛta studies in the academia and lack of any representation in professional and public life in modern India, has cramped the general ability of people to comprehend the Saṁskṛta words, phrases and texts without translations.~As English proceeds to dominate every field, the ability of younger generations to comprehend the translations in regional languages is also decreasing. Not surprising that most of the urban young Indians are becoming completely dependent on ‘only English versions’ for any and every information. They are the ones who are largely plagued by misconceptions about our native features.

When critics like T. M. Krishna\index{Krishna, T. M.@Krishna, T. M.} make allegations on our art space as ‘brahminical’,\index{brahminical@brahminical} deep in our hearts we know that it is not so. But the problem is in countering them then and there.~We are not equipped enough for defense.~The left rhetoric is usually very eloquent, well phrased and is marketed in the garbs of ‘humanism and social equality’. The dubious narratives almost convince us that they are indeed talking about ‘a larger interest and progress’, while they actually work to divide and weaken the native’s ethnic identity.

Though command in English and regional languages are primary weapons to contest these voices, Saṁskṛta scholarship is most essential to defeat them in the long run.

Let us introspect. The problem is also the lack of will in our artists. Many musicians have decided that it enough to ‘perform well and get appreciated’. They mete out a second-class treatment to \textit{śāstra} and lyrical purity. Although the musical lyrics\index{sahitya@\textit{sāhitya}} in regional languages have managed to retain some purity, Saṁskṛta lyrics have been left to decay. Even the finest of musicians sometimes sing corrupt versions of Saṁskṛta lyrics, unmindful of the damage they are doing to the \textit{paramparā}\index{parampara@\textit{paramparā}}, by passing on the same to their students. Although a good number of Saṁskṛta scholars are available, very few music composers and artists in the classical, folk and cinema fields, care to approach them for verifying the purity of their Saṁskṛta lyrics.

Musicians and Musicologists must begin to introspect the depth of their understanding of the art heritage and musical terminology that they use.~After decades of left interventions in every sphere of Indian socio-cultural life, no one can rule out the influence of the dubious narratives on our art perceptions today. They therefore must associate with Saṁskṛta scholarship to cross-verify the purity of their articulation, phraseology and comprehension of the Saṁskṛta terms/ lyrics\index{sahitya@\textit{sāhitya}}. Artists need to verify their own interpretation of the art too, whether it is in line with the \textit{paramparā}\index{parampara@\textit{paramparā}}. It is alarming to see how in the present day, many artists carelessly pick up unverified ‘English definitions’ to explain important terminology to their audience and students. Musicians must become more careful and responsible in this regard. ‘Creativity’ must not mean ‘overlooking’ the very essential form and spirit of the art.

Music syllabi must be enriched with important terminologies taken directly from art treatises like the \textit{Nāṭyaśāstra}\index{Natyasastra@\textit{Nāṭya-śāstra}} and understood in their true sense and scope. This would lessen the gap between the artists’ present perception of their art and roots concepts in their art. This would also cleanse the distortions enveloping words like \textit{brāhmaṇa}\index{brahmana@\textit{brāhmaṇa}} in the art space. Particularly because Krishna\index{Krishna, T. M.@Krishna, T. M.} and his likes are using these words in the distorted sense, we can no more afford to overlook the importance of Saṁskṛta scholarship.

\vspace{-.3cm}

\section*{Indian Classical Music and\hfill \break the \textit{Rasa} Experience at its Core}

In Indian art, the external expressions have been vehicles to the \textit{rasa} factor. Picking themes, inspiration and material support from around, the artist can make many a creative attempt to explore the \textit{rasa} feel. This freedom has encouraged every Indian art tradition to develop its own indigenous form and style, bearing both similarities and differences with other art systems. It is irrelevant to interpret this freedom of the artist and the art, to be ‘original, innovative and dynamic’ as just a ‘social division’.

Artists who intended to explore more of \textit{navarasa-}s and subtle human relationships in their music, branched out to develop the light classical music (\textit{sugama saṅgīta})\index{sugamasangita@\textit{sugama-saṅgīta}} model. Those who preferred to focus more on original art technicalities in music continued within the frame of classical music. Karnatic\index{Karnatic@Karnatic} Music too comprises a lot of themes including all the \textit{navarasa-}s. But \textit{bhakti}\index{bhakti@\textit{bhakti}} however has been the more common theme here. There is no need for a hierarchical comparison of the two. Both have their charm and appeal to music lovers. Any artist is free to pick up ideas from another art form. But that should not be done at the cost of ‘compromising the elementary structure’ of their art form. This is what \textit{sampradāya-śuddhi} actually means. This \textit{sampradāya-śuddhi}\index{sampradaya@\textit{sampradāya}} is relevant not only to the art technicalities, but also to the conventions like dress, stage and occasion etc., which are indigenous to the art presentation. When an artist religiously adheres to \textit{sampradāya}, he is actually intending to protect the purity of his art internally and externally. It is ridiculous to see this as a social/brahminical\index{brahminical@brahminical} prejudice.

While technical structures and boundaries like \textit{rāga, tāla, bhāva, sāhitya}\index{raga@\textit{rāga}}\index{tala@\textit{tāla}}\index{bhava@\textit{bhāva}}\index{sahitya@\textit{sāhitya}} norms and the standardized compositions intend to protect, document and nourish the systematic evolution of the classical art from within, some religious, regional, cultural and social conventions add embellishment, dignity and an indigenous charm to the external presentation.

But of course, these external features must never suffocate the creative potentials of the artist and must be open to justifiable modifications.~But if the modifications threaten to ‘displace’ an existing form and enforce an alien one, that would naturally trigger unrest. No ardent lover of the art can watch comfortably when the very form of his beloved art is under threat.

The artists’ love for art is indeed nothing less than the ardent devotee’s \textit{bhakti}\index{bhakti@\textit{bhakti}} for God. For a true musician, music has not been about mere scholarship and entertainment. It has indeed been an \textit{upāsanā}/ ‘\textit{nadopāsanā’}\index{nadopasana@\textit{nadopāsanā}} for self-elevation. From the time of \textit{devadāsī-}s\index{devadasi@\textit{devadāsī}}\break till now, the art with all its internal and external features has always been regarded as divine. An artist is indeed a \textit{kalā tapasvin}, who constantly remains submerged in aesthetic contemplation. The concert presentations are just occasional glimpses into the intense art contemplation and practice sessions that the artist goes through. For him art is divine. He verily visualizes Goddess Sarasvatī in everything connected to his art – the musical sound (\textit{nāda}),\index{nada@\textit{nāda}} his instruments, his \textit{guru},\index{guru@\textit{guru}} the \textit{paramparā},\index{parampara@\textit{paramparā}} the stage, the audience, the compositions, the \textit{rasa} experience, the dress and social conventions. This is true with almost every form of folk and classical art in India.

It is surprising that, Krishna being a musician himself, displays contempt and insensitivity towards this beautiful sentiment and looks at it with a caste-coloured spectacle.

\vspace{-.55cm}

\section*{Generalising Karnatic Music Conventions as ‘Attempts to Brahminise’}

While targeting the brahmin, Krishna makes a sweeping generalization against the whole of Karnatic music heritage thus-

\vspace{-.1cm}

\begin{myquote}
“The presentation of a Carnatic concert is a representation of Brahminical culture”. “Carnatic\index{Karnatic@Karnatic} concert is not mere exposure to the music; it is a complete Brahmin brainwashing package.
\end{myquote}


\begin{myquote}
“Every time I rendered kirtanas,\index{kirtana@\textit{kīrtana}} expounded ragas\index{raga@\textit{rāga}} or cracked arithmetic patterns, I was holding up my community flag and waving it with gusto.” 

\vspace{-.1cm}

~\hfill (Krishna 2018)
\end{myquote}

\vspace{-.1cm}

What exactly he means by ‘brahminising’ music? – Krishna\index{Krishna, T. M.@Krishna, T. M.} is unable to logically explain. He generalizes that all conventions and costumes followed in the \textit{sabhā}\index{sabha@\textit{sabhā}} are ‘brahmin’ in appeal. If, according to him, adherence to the conventional costume (saree/ \textit{dāvaṇi}/ \textit{veṣṭi, tilakam,} etc.,), bowing down, \textit{kacheri}\index{kacheri@\textit{kacheri}}-\textit{dharma}\index{dharma@\textit{dharma}}, \textit{sabhā-dharma}, reverence for the \textit{Guru}\index{guru@\textit{guru}} and the lineage, the invocation song/\textit{śloka}\index{sloka@\textit{śloka}}-s, mutual discussions on the great stalwarts and heritage, etc., are exclusively ‘brahminical’\index{brahminical@\textit{brahminical}} (Krishna 2018), why then do devout non-brahmins\index{non-brahmins@non-brahmins} practice the same in their exclusive community events too? Saree, \textit{tilakam, dāvaṇi, veṣṭi} etc., have never been exclusively brahmin costumes, but have rather been regional/ geographical features. If all this is brahminical, what then is the non-brahmin South Indian dress code according to Krishna? He even ridicules women musicians as ‘showing lesser skin’ to appear ‘brahminical’. Does he mean to say that all non-brahmin artists wear revealing dresses?

By generalizing that all conventions are brahminical, Krishna even implicates that all non-brahmin communities are uncouth and lack any sense of social culture of their own and depend wholly on some brahmins’ dictates. This is not only \textit{not} true, but also humiliating to the great lineage of non-brahmin artists and accomplishers in history. In a hasty attempt to demonize the brahmin community, Krishna is actually insulting the non-brahmins who comprise the largest part of the Hindu population!

It is a well-known fact that the dress, cultural symbols, food and lifestyles in India are inspired mostly by the regional, geographical, occupational and seasonal factors. A South-Indian non-brahmin\index{non-brahmins@non-brahmins} can relate better to a South-Indian brahmin in terms of dress, language, festivals, food and social conventions, than he can with a North-Indian non-brahmin and vice versa. So, it is mostly the ‘regional’ and not ‘caste’ factors that influence the cultural conventions of people and communities.\endnote{In fact, it is the climate, whether and geography which define the dress style in the Indian ethos. Before Christian culture intruded the schoolings, men and women of all communities including brahmins, in Kerala and some other humid / hot regions of south and coastal belts of India, wore light clothes, covering less of the upper parts of their bodies. People living in colder regions like Himalayas naturally cover their bodies more, with thick long garments. It is only after English dress was imposed upon schools as ‘formal and modern’, nativity in dress gradually decreased in Indian masses.}

Brahmins or non-brahmins, all have had their own \textit{śāstra-}s and \textit{sampradaya-}s\index{sampradaya@\textit{sampradāya}} at all times. Each community has had its own family priests, \textit{mukhiya-}s (community heads), democratic elections, people’s representatives and experienced mentors. Each of them have their own occupational and cultural \textit{paramparā-}s\index{parampara@\textit{paramparā}} and related codes of conduct, treatises and social conventions. No one depended solely on the brahmins’ dictates ever. The countless treatises and living traditions\endnote{Treatises of native medicine, metallurgy, jewelry making, castings, physics, cattle rearing, \textit{kāma-śāstra}, perfume making, weaving, bangle making, basket weaving, pottery, warfare, cooking, polity and many more can be seen in treatises mostly in Saṁskṛta and in \textit{deśa-bhāṣā-}s or in unbroken oral traditions.} are standing proofs for this.

Every country has its own regional conventions in presenting its classical and folk arts. These are regional customs with cultural, social and aesthetic value. They do not need to pass the ‘check posts’ of dry logic to prove their worth.

Krishna’s\index{Krishna, T. M.@Krishna, T. M.} desperate attempt to generalize thus, reveals his contempt not just towards the brahmin, but towards Hindu \textit{dharma}\index{dharma@\textit{dharma}} and India’s ethnic identity at large.

Krishna’s thought-line is rooted in the leftist standpoint, that always show non-brahmins as only amateurish, uncivilized, uneducated, unskilled, ignorant and capable of being only domestics and so on in their art and literary pieces.\endnote{In a haste to demonize the brahmin image and everything that he represents, leftist writers and artists have, as a supplementary act, depicted their non-brahmin\index{non-brahmins@non-brahmins} characters mostly as illiterate, ill-dressed, helpless, incapable, dull, deprived and so on. The accounts of non-brahmins who held high positions, who were valorous, enterprising, religious, saintly and those who excelled in trade and other endeavor have seldom gained main focus in leftist writings. Similarly, many caste-based Indian movies magnify the caste divide demonize the whole brahmin community and depict all non-brahmin characters are weak/ incompetent. Mocking brahmins, priests and\break \textit{sanyāsin}-s and showing them as hypocrites, autocrats or cheats is most common in Indian cinema and the small screen to this day. Readers may refer to the countless websites and Indian movies and leftist literature that magnify beyond facts, the ‘atrocities by brahmins’. Popular examples where brahmin-bashing rules the screen are Satyajit Ray’s\textit{ Sadgati} (1981), \textit{Ankur} (1974) (both in Hindi) and \textit{Comana duḍi} (1975), \textit{Hemāvati} (1977), \textit{Ghaṭaśrāddha} (1977) (all three in Kannada) and many others. We also come across countless episodes, conversations, songs and comedy scenes that subtly or loudly ridicule and insult the brahmin class. Leftist playwrights like Girish Karnad, U. R. Ananthamurthy and others strongly influenced the narratives of the brahmin image.} The left view intentionally overlooks the fact that non-brahmins who form a considerable majority of the Indian population till date have produced countless rulers, statesmen, warriors, poets, yogis, artists and entrepreneurs at all times. Considering this, it is indeed the non-brahmin artists who must lead the defense against Krishna’s dubious allegations!

If caste was the only consideration, why then would any brahmin worship the non-brahmin Rāma who killed the brahmin Rāvaṇa? How would Sītā, whose parentage remains unknown become the goddess of many brahmin hearts? Why would Hanumān, a hero of the \textit{vānara} tribe, become the chosen deity for numerous brahmins? When a brahmin sings the verses / \textit{mantra-}s from the Veda, \textit{purāṇa}-s or epics, does he consider the parentage of the seers like Vyāsa, Vālmīki or Viśvāmitra who composed them?

The lion’s share of the credit of promoting various \textit{classical} traditions of music and dance in India goes to the talented \textit{devadāsī}\index{devadasi@\textit{devadāsī}} community. Although the \textit{saṅgīta-śāstra-}s composed by men (their communities not all ascertained) suggests that people of all communities took deep interest in professional music and dance, countless references in vedic, \textit{purāṇic} and classical lore and living traditions establish that \textit{devadāsī} community dominated the scenario.\endnote{A \textit{devadāsī}\index{devadasi@\textit{devadāsī}} had sacred responsibilities in temple traditions and social and political events (unlike a \textit{dāsī} who merely served her master). The word \textit{puṁscalī} refers to a whore, who held a role in ceremonies like \textit{vratya}. \textit{Puṁscalī} is the forerunner of the later \textit{devadāsī} (Chandra 1975: 2). The Buddhist \textit{Jātaka}\index{Jataka tales@\textit{Jātaka tales}} literature uses many words like \textit{vesī, nāriyo, gamaṇīyo, gaṇikā, nagara-dāsī, vaṇṇa-dāsī, kumbha-dāsī} etc,. The \textit{gaṇikā} was affluent and had an important position in the king’s court. She was also called \textit{vaṇṇa-dāsī} and even employed a large number of slaves (Chandra 1975: 23-24). \textit{Puṁscalī-}s and \textit{gaṇikā-}s who carried ritualistic significance were usually learned and talented and some even became queens. Born in the clan of courtesans, Śukavāṇī who was a Śatāvadhāninī and a poetess in six languages, was honoured with \textit{kanakābhiṣeka (}showeredwithgold\textit{)} and conferred the title of Madhuravāṇī by king Raghunāthanāyaka\index{Raghunatha Nayaka@Raghunātha-Nāyaka}. She composed the \textit{Rāmāyaṇasāratilakam}. (Adkoli 2007: 161-162). It is a well-known historical account that the celebrated King Kṛṣṇadevarāya’s queen Cinnammā Devī was originally a \textit{devadāsī.} (Naidu 2012: 101).} It is only in the recent centuries that brahmins are getting into full time pursuit of classical arts.

If brahmins were so contemptuous about other communities, how then could they enter the fine arts domain, knowing well that they were the forts of the \textit{devadāsī-}s for millennia? It is obviously only \textit{vidyā}-\textit{prīti}, the urge to preserve the dying traditions, which encouraged brahmins to open their doors to professional music during the colonial era. Men and women of all communities participated in Music and dance at all times. But for the \textit{devadāsī} families it was a more serious and full-time family profession. The decline of patronage\index{patronage@patronage} for \textit{devadāsī}-s\break and their social status simultaneously challenged the very existence of their fine arts traditions too. This probably led other communities to join them in preserving the arts. The big entry of men and women of other communities into music dance domain is suggestive of this.

Centuries of brutal Islamic invasions and the consequent British colonial rule, de-patronized and trampled upon the temple traditions, marginalizing the role of \textit{devadāsī-}s to mere flesh trade. Thus arose the need for the public to volunteer to save these fine arts from extinction. This was when perhaps, people from all communities were encouraged to take up fine arts too as full-time professions. This indeed shows the social responsibility that people of all communities have shown, towards preserving our classical art traditions, during adverse political conditions.

Why did not others enter this domain as much brahmins did during the same time? More objective research could throw light upon this.

But one possible reason was the commercial factor. Most brahmins were already used to a life style of study and piousness, wherein material benefits were meagre. It was therefore not too difficult for them to involve in fine arts, where too material gain was always uncertain. But, people of more profitable occupations naturally did not risk taking up fine arts as full-time pursuits.\endnote{Intending to completely gain hold on Indian economy, the British policies gradually weakened even the other occupations in India. Dharampal\index{Dharampal@Dharampal} (1971) observes in the Foreword:

“After the first few years of euphoria since Independence, a period of self-denigration set in during which educated Indians, particularly those educated in the West, took the lead. Whether in the name of modernisation, science or ideology, they ran down most, if not all, things Indian.”

We can thus infer how the imperial British rule methodically displaced native arts, occupations and knowledge systems with their impositions. In the pretext of modernization, they trivialized our native arts and sciences as ‘outdated and non-progressive’. Through heavy taxation and other cruel sanctions, they discouraged self-employment and weakened the markets forcing masses to shift to the jobs that they created. They imposed English as the official medium in India, displacing Saṁskṛta and \textit{desi}\index{desi@\textit{deśi}} languages and thus marginalizing their role to a very great extent. Saṁskṛta and Vedic studies and classical arts, were depicted as solely brahmin ventures and as non-progressive and castiest and gradually marginalized in mainstream academics and job market. Refer to English Education Act 1835.

(\textbf{Editor’s Note}: The seeds of the English Education Act can be traced to the now-\break infamous \textit{Macaulay’s Minute}, the full text of which can be seen documented in \textit{Selections From Educational Records Part 1 1789-1839.} https://archive.org/details/SelectionsFrom\break EducationalRecordsPartI1781-1839)}

In the beginning years of Post-independence India, countless native arts and occupations suffered extinction or extreme degeneration because of government apathy.\endnote{A living example from contemporary times, is how the extraordinary great literary feat like \textit{avadhāna}\index{avadhana@\textit{avadhāna}}, a fascinating classical poetic tradition, has been never given a single representation on the prestigious platforms of Kannada Sahitya Parishat, till the present day. The prejudice that works behind this intentional negligence of a great poetic tradition is obviously the left-generated myth that ‘\textit{avadhāna} is a brahmin art’, while in reality, countless poets and artists and even \textit{devadāsī-}s\index{devadasi@\textit{devadāsī}} have excelled in this art down the centuries.} Some folk arts however got some revival through government patronization to an extent. But Karnatic\index{Karnatic@Karnatic} classical music was bluntly labelled as brahminical\index{brahminical@brahminical} and given much lesser patronage\index{patronage@patronage}. This is when local patrons and art connoisseurs volunteered to support it.~This was when perhaps, the artists and connoisseurs formed their own committees and raised funds to promote their own events. Even now there are few classical musicians who go to government bodies to seek financial help, where not only do the proceedings get delayed, but they even would have to face a step-motherly treatment for pursuing the ‘Brahmin art’.

Krishna\index{Krishna, T. M.@Krishna, T. M.} actually does not mention that there are ‘no brahmins’ in this domain, but asserts that brahmins and brahminism dominate the scene. But he intentionally avoids mentioning that many brahmin \textit{guru}\index{guru@\textit{guru}}-s have indeed nurtured their non-brahmin\index{non-brahmins@non-brahmins} students with love and passion. Similarly, many brahmin musicians have high regard for their non-brahmin \textit{guru}-s.

Muttusvāmi\index{Muttusvami Diksita@Muttusvāmi Dīkṣita} Dīkṣitar, one of the Trimūrti-s, himself had students from all communities – Pillai-s, Mudaliar-s, etc., He even had a \textit{devadāsī}\index{devadasi@\textit{devadāsī}} student by the name Kamalam. Isai Veḷḷalār, Piḷḷai and Devadāsī communities have done yeomen service to Music and dance professions. ‘Tanjai Nālvar’ (Tanjavur Quartet\index{Tanjavur Quartet@Tanjavur Quartet}) was hailed as the greatest of Karnatic musicians. Muttu Tāṇḍavar (composed special \textit{padam-}s\index{pada@pada} for dance), Mārimutta Piḷḷai and Aruṇācala Kavi are well known composers who lived around Chidambaram that has a rich history of music and dance. Another prominent composer, Aruṇagirināthar, composed the famous \textit{Tiruppugaḻ}\index{Tiruppugal@\textit{Tiruppugaḻ}}.~There were and are some women \textit{Nāgasvaram} artists, including Muslim women. Till date, most of the \textit{Nāgasvaram} musicians are from the barbers’ community.

T. Chowdiah,\index{Chowdiah, T.@Chowdiah, T.} the celebrated violinist of Karnataka, hails from the Gowda community. The violinists Mahadevappa and his two sons, M. Nagaraj and M. Manjunath, the Mridangists M. L. Veerabhadrayya and his son, V. Praveen, and the famed saxophonist Kadari Gopalanath are also non-brahmins. They have all won the hearts of music lovers the world over!

The great patron of Karnatic\index{Karnatic@Karnatic} Music, K. K. Murthy (a brahmin) initiated the building of the famous Chowdiah\index{Chowdiah, T.@Chowdiah, T.} Memorial Hall in the heart of Bengaluru city, to honour the great violinist. The architectural design of the building is in the shape of a huge violin and the best of artists across the world perform here. (Even Krishna has rendered his music in this renowned hall) Where is the caste angle here? If caste was the sole criteria in music domain, why would the brahmin K. K. Murthy choose to honour the non-brahmin\index{non-brahmins@non-brahmins} Chowdiah thus? This proves the art and the artist have always been above caste for true music connoisseurs.

If the likes of Krishna can quote a few stray examples where brahmin musicians denied teaching non-brahmins, we can quote countless other examples wherein brahmin musicians have tutored non-brahmins. Top-ranking brahmin performers like Semmangudi\index{Semmangudi Srinivasa Ayyar@Semmanguḍi Srīnivāsa Ayyar} had non-brahmin students like T. M. Thyagarajan and Neyyatinkara Vasudevan. The celebrated Chembai\index{Chembai Vaidyanatha Bhagavathar@Chembai Vaidyanatha Bhagavathar} Vaidyanatha Bhagavathar tutored the non-Hindu K. J. Yesudas\index{Yesudas, K. J. @Yesudas, K. J.}. It is said that D. K. Pattammal was the first brahmin woman to perform in public. This clearly suggests that non-brahmin women were more active in this field till then.\break M. L. Vasanthakumari’s mother and musician Madras Lalithangi was not a brahmin by birth.

In the present-day music scenario, even the few caste considerations that existed here and there, are fast disappearing in the classrooms.\endnote{I personally have had more than 3 non-brahmin\index{non-brahmins@non-brahmins} teachers in classical music and dance. Never once did the caste factor even flash by in the beautiful relationship I have had with my \textit{guru}\index{guru@\textit{guru}}-s.}

\textit{Harikathā}\index{Harikatha@\textit{Harikathā}} is an art based on Karnatic classical music. \textit{Harikathā}-\textit{dāsa-}s hail from all communities and include harmonium, \textit{tabala} and \textit{naṭṭuvāṅgam} artists. The late Gururajulu Naidu and his daughter Shobha Naidu are extremely popular in this field. Others like Keshava Dasa, Achyuta Dasa and their illustrious disciples command immense admiration across the state, national and international platforms.

The celebrated artist, Bengaluru Nagaratnamma, belonged to the \textit{devadāsī}\index{devadasi@\textit{devadāsī}} community. Her offering of her lifetime earnings to build the temple on Sri Tyāgarāja’s\index{Tyagaraja@Tyāgarāja} Samādhi at Tiruvayyaru, has been duly acknowledged by connoisseurs, who have erected her life size statue opposite the Tyāgarāja shrine. Where is the caste consideration here?

The respect and jubilant reception that even Muslim artists like Zakir Hussain get in South Indian temples and \textit{sabhā-}s\index{sabha@\textit{sabhā}} proves that India’s love for artists that transcends all social boundaries.

We can indeed prepare a very long list of non-brahmin musicians from historical and contemporary data. But it pains me to pick up the caste tags of the great artists whom we admire solely for their talent and charisma. With due apologies to them all, I am presenting a random list of well-known non-brahmin\index{non-brahmins@non-brahmins} artists in the footnote.\endnote{(Names mostly taken from Tamil Nadu because that is T. M. Krishna’s cradle) Bharata Ratna M. S. Subbulakshmi,\index{Subbulakshmi, M. S.@Subbulakshmi, M. S.} M. L. Vasanthakumari’s mother Lalithangi, Dwaram Venkataswamy Naidu, Palani Subrahmanya Pillai, Muttaiaya Pillai, Panchapakesha Pillai, Kanchipuran Nayana Pillai, Chittoor Subrahmanya Pillai, Mayavaram V. R. Govindaraja Pillai, Pudukottai Dakshinamurthy Pillai, Tangavelu Pillai (Malaya),\break T. M. Thyagarajan, Neyyatinkara Vasudevan, Veena Dhanammal, T. Brinda, T. Mukta, T. Balasaraswati, T. Shankaran, T. Vishwanathan, Sempanarkoil Brothers, A. K. C. Natarajan, Sheikh.~Chinna Moulana, Kasim-Babu Brothers, Thiruvadudurai Rajarathinam Pillai, Thiruveezhimalai Brothers, Namagiripettai Krishnan, Valayappatti Subramanyam, Haridwara Mangalam, Seerkazhi Govindarajan, T. R. Mahalingam, Chidambaram Jeyaraman, P. Unnikrishnan, Madurai Somasundaram, M. M. Dandapani Desikar, M. K. Thiagaraja Bhagavatar, M. K. Govindaraja Bhagavatar, Karukuricchi Arunachalam, Meenakshi Sundaram Pillai, Kuzhikkarai Pitchaiappa, Tiruppamburam Swaminatha Pillai, Kumbakonam Rajamanickam Pillai, T. Chowdiah\index{Chowdiah, T.@Chowdiah, T.}, Kadari Gopalnath, M. Nagaraj and M. Manjunath of Karnataka. Not just non-Brahmins, but even non-Hindus like John Higgins Bhagavatar and the most popular Dr. K. J. Yesudas\index{Yesudas, K. J.@Yesudas, K. J.} command great admiration in music circles.} The names are just samples from a list that can be potentially very long.

Krishna\index{Krishna, T. M.@Krishna, T. M.} makes a cruel personal allegation on the most respected and celebrated musician Smt. M. S. Subbulakshmi- “To embrace Brahminism she distanced herself from her brother and family in Madurai…” (Krishna 2017). Krishna makes an impudent judgment around the personal life of a great artist. While there could be great many reasons for someone to live away from their kith, no gentleman would unnecessarily drag that into public gossip. Particularly, keeping in mind the great respect that Subbulakshmi Amma commands, not just in the music domain, but the whole nation, Krishna’s statements appear as nothing more than stale loose talk.

How exactly has MSS\index{Subbulakshmi, M. S.@Subbulakshmi, M. S.} ‘become brahminical’\index{brahminical@brahminical}- Krishna is unable to specify. So, he perhaps only intends to instigate some cheap speculations about the celebrity.

Pointing out to Chennai’s famous ‘Annual Music and Dance festival’ Krishna alleges- “It is the most restrictive big festival in the world” (Ramanan 2014).

Krishna implicates that the artists use caste credentials to grab opportunities there. But in such big festivals where commercials and popularity matter most, how could the organizers risk huge investments by hosting artists on caste basis? The crowd-pulling factor is undoubtedly the talent of the artist and not his caste. Given a choice, does any sensible singer choose his \textit{pakka}-\textit{vādya} artists on their caste basis compromising on the musical abilities?

National level music festivals are not easy to penetrate, particularly with mere caste credentials. Because, even if a brahmin artist manages to sneak in thus, he cannot survive the competition and remain there, if he cannot ‘deliver’ on stage. Without winning the hearts of connoisseurs each time, no artist, not even the big celebrities, can continue to reign on such platforms for long. And, even if some brahmin individuals did dominate at times, is it not cruel to make a sweeping generalization about all brahmin artists in the domain? Bringing in the subjective caste issue into the general picture of art circle, is not only unnecessary but also offensive on the part of Krishna.

In numerous private religious, cultural and educational trusts and institutions which offer services to general public, the members belonging to a certain community could be more. That should not matter as long as there are no social walls dividing people and work is going on well. If Krishna is indeed so concerned about the democracy of such event executions, he must initiate a dialogue within the committees to make the proceedings more transparent. He always has had the celebrity status to do so!

If more brahmins took to full time music more than others, other factors like the aptitude and commercials have also played a part. Moreover, in most brahmin households, children are encouraged, even insisted upon learning vocal or instrumental Karnatic\index{Karnatic@Karnatic} music. But who is stopping non-brahmins\index{non-brahmins@non-brahmins} from taking up music full time? Parents too play a big role in influencing the artistic options of their kids.

One more important factor has been the monetary returns. Music has never been a prosperous profession for most musicians. Teaching, research and art pursuits are some fields where monetary benefits are never immediate and even uncertain many a time. That is why only the most passionate ones, continue here as full-time artists.

Art is indeed a ‘calling’. No one can be forcefully initiated into art, like they can be into other trades or jobs. A truly passionate artist breaks out of all social and economic barriers to pursue his/her art.

Karnatic music draws lyrics\index{sahitya@\textit{sāhitya}} from not only the well-known\break \textit{vāggeyakāra-}s\index{vaggeyakara@\textit{vāggeyakāra}}, but also from the lesser known \textit{vāggeyakāra-}s, local poets and saints. In Karnataka, the compositions of many non-brahmins saints like Kanakadāsa,\index{Kanakadasa@Kanakadāsa} Nijaguṇa Śivayogi,\index{Nijaguna Sivayogi@Nijaguṇa Śivayogi} Śiva \textit{śaraṇa}-s and even the Muslim mystic Śiśunāḷa\index{Sisunala Sarifa@Śiśunāḷa Śarīfa} Śarīfa and others, set to music, have been welcomed with great love and admiration by music connoisseurs at all times. In fact, lines like

\begin{centerquote}
“\textit{mêndaina brāhmaṇuḍu mêṭṭu bhūmiyokkaṭe \dev{।}\\ caṇḍāluḍuṇḍeti saribhūmiyôkkaṭe} … \dev{।।}
\end{centerquote}

(by saint Annamayya\index{Annamacarya@Annamācārya} in his composition\textit{- Brahmamôkkaṭe), kula kula kulavêndu hôḍedāḍadiri} (by Saint Kanakadāsa)\index{Kanakadasa@Kanakadāsa} and others have never been opposed by any brahmin, but in fact, appreciated by all.

The music industry is not only about the singers and instrumentalists. It involves the services of instrument manufacturers, carpenters, blacksmiths, potters, engineers and other artists and artisans. Interestingly, from centuries, most of the instrument-manufacturers hail from non-brahmin\index{non-brahmins@non-brahmins} communities.

Manufacturing instruments is not only about ‘carpentry or metallurgy’, but demands a sound knowhow of the \textit{śruti}\index{sruti@\textit{śruti}} factor, octave ranges, sound amplification, resonance, acoustics and many more musical features. Indeed, many persons involved in manufacturing and repairing musical instruments are usually musicians or have a keen understanding of the art.

Each \textit{saṅgīta-kacheri}\index{kacheri@\textit{kacheri}} is indeed a beautiful team work of artists, technical, financial and marketing support workers. The manpower involved in stage arrangements, auditorium management, sound system, networking, event management, sponsoring and others like transport, packing and shipping of musical instruments etc., (the list can go on) are vitally important parts of the music industry. Professional music is therefore a big employment generation system in itself involving lots of team work and commercials.

It is ridiculous to assume that only a handful of brahmin musicians, without the involvement of any of these, have solely been managing the whole show till date!

\textit{Varṇa-paddhati}\index{varna@\textit{varṇa}} never meant caste (as understood today) in ancient and medieval India. It was only suggestive of the occupation and cultural identity. Beyond that, caste remained irrelevant. People of all communities came together to rejoice in festivals, public events, feasts and cultural programmes.

Vātsyāyana’s\index{Vatsyayana@Vātsyāyana} \textit{Kāmasūtra,}\index{Kamasutra@\textit{Kāmasūtra}} describeshowonce in a fortnight or month, people from all communities assembled in the \textit{Sarasvatī-bhavana} for the \textit{Samājotsava}\endnote{\textit{Kāmaśāstra}\index{Kamasutra@\textit{Kāmasūtra}} refers to \textit{Sarasvatī-Bhavana-}s where people from all sections of the society gathered periodically to celebrate \textit{vasantotsava} through music, dance feastings and various amusements – \textit{pakṣasya māsasya vā prajñāte’hani sarasvatyā bhavane niyuktānāṁ nityaṁ samājaḥ.} (Shastri 1929: 44).}. Local as well as invited artists, dancers and actors of other regions participated in good numbers. \textit{Vasanta-maṇṭapa-}s,\break spacious royal and temple precincts and the \textit{devadāsī-}s’\index{devadasi@\textit{devadāsī}} affluent mansions hosted music and dance events. The text even describes how the crowds and patrons gave generous gifts and money to the performers spontaneously, with due admiration. Never anywhere are the castes of the donors, performers or connoisseurs highlighted in these descriptions. The mention to ‘huge crowds’ clearly suggests that people of all communities came together to rejoice and celebrate.

The left critics are cleverly silent about such historical accounts! They instead cherry pick stray instances from a tale or even fiction, wherein a brahmin is seen displaying prejudice to ‘prove’ their point!

The worst hit by invasions then and intellectual colonization in post-independence India, is the brahmin community. The invaders brutally massacred Brahmins, intending to root out the knowledge systems (\textit{guru paramparā-}s)\index{guru@\textit{guru}}\index{parampara@\textit{paramparā}} that theypreserved. After independence, the academia, media, theatre and cinema in India, influenced directly by left rhetoric, continued to demonize the brahmin, for his very birth, denying him the credit for anything he deserved.

The apathy of the political power that took over the newly independent India, never allowed the expected Indian cultural renaissance to happen in good spirit. On the contrary, more damage was done through relentless brahmin-bashing, Hindu-bashing, caste-divides, minority appeasement and marginalization of Saṁskṛta and \textit{desi}\index{desi@\textit{deśi}} rhetoric.

\vspace{-.4cm}

\section*{\textit{Aucitya:}\index{aucitya@\textit{aucitya}} Its Absence in the Leftist Narratives on Art}

\textit{Aucitya} which played an important role in Indian art narrative for centuries, has totally been discarded in the left narratives. The concept of‘\textit{aucitya’} was greatly emphasized by aestheticians as the ‘regulatory factor’ in any art or literature. Ānandavardhana\index{Anandavardhana@Ānandavardhana} explains how elaborately in the 3rd Uddyota of his work \textit{Dhvanyāloka},\index{Dhvanyaloka@\textit{Dhvanyāloka}} about how an aesthetic creation must adhere to the norms of\textit{aucitya.}\endnote{\textit{virodhi-rasa-sambandhi-vibhāvādi\index{vibhava@\textit{vibhāva}}-parigrahaḥ...rasasya syādvirodhāya vṛttyanaucityam\index{aucitya@\textit{aucitya}} eva ca \dev{।}} (\textit{Dhvanyāloka} 3.18-19)

\textit{raso yadā prādhānyena pratipādyas tadā tatpratītau vyavadhāyakā virodhinaś ca sarvātmanaiva parihāryaḥ} (\textit{Dhvanyāloka}\index{Dhvanyaloka@\textit{Dhvanyāloka}} 3.6+) (Pathak 1997: 349).} Many \textit{kārikā-}s and \textit{vṛtti-}s discuss the significant role of \textit{aucitya} in generating \textit{rasa} and the role of \textit{anaucitya} (inappropriateness) in causing \textit{rasabhaṅga} (spoiling the \textit{rasa} experience). The aesthetician warns how the poet/artist must never take too much liberty and introduce changes, characters or an inappropriate narrative or even word usages such that they cause \textit{rasabhaṅga}. Trivializing a divine character or noble theme or well accepted and respected social or cultural aspect is one of \textit{anaucitya-}s that an artist must avoid\endnote{\textit{santi siddharasa-prakhyā ye ca rāmāyaṇādayaḥ \dev{।}} \\ \textit{kathāśrayā na tair yojyā svecchā rasavirodhinī \dev{।।}} (\textit{Dhvanyāloka} 3.14+) (Pathak 1997: 367).}. The poet or artist is advised to never trespass the limits of decency and offend a social or cultural or religious value in the name of creativity. Ānandavardhana\index{Anandavardhana@Ānandavardhana} does not spare even the great poet Kālidāsa,\index{Kalidasa@Kālidāsa} when he points out how poets, if not careful enough, may offend the dignity of divine characters.\endnote{\textit{mahākavīnām apy uttamadevatāviṣaya-prasiddha-sambhoga-śṛṅgāra-nibandhanādyanau\break cityaṁ śakti-tiraskṛtatvāt grāmyatvena na pratibhāsate | yathā kumārasambhave devīsambhoga-varṇanam |} (\textit{Dhvanyāloka}\index{Dhvanyaloka@\textit{Dhvanyāloka}} 3.6a+) (Pathak 1997: 346). \textit{Anaucitya-doṣa}\index{aucitya@\textit{aucitya}} occurs in the descriptions done by great poets too, wherein they sometimes offend the dignity of divine characters by showing them in a cheap way, but the power of their talent conceals the flaw nevertheless. For instance - \textit{Kumārasambhava} (of Kālidāsa)\index{Kalidasa@Kālidāsa} wherein the union of Goddess is described.} Along with the \textit{aucitya}\index{aucitya@aucitya} of the character and wheat he speaks, even the topic/theme directs the content of the composition.\endnote{\textit{vaktṛvācyagataucitye satyapi viṣayāśrayam anyad aucityaṁ saṅghaṭanāṁ niyacchati.}\break (\textit{Dhvanyāloka} 3.7+) (Pathak 1997: 353).}

For centuries, artists, poets and connoisseurs have contemplated on this \textit{aucitya} factor throughout their deliberations and presentations. At times, amendments have been made to the \textit{aucitya} norm, but only with due sensitivity and care.

My revered \textit{guru, Saṅgīta Kalāratna}\index{guru@\textit{guru}} Dr. T. S. Satyavathi encapsulates the \textit{aucitya} concept thus - \textit{“}In Indian classical art there is always the unbroken frame. But the frame is flexible”\endnote{ Private conversation.}. When art is used by perverts to offend, it ceases to remain beautiful and dignified in their hands. That is why \textit{aucitya} has always acted as an important regulatory norm in Indian culture and aesthetics.

When Krishna makes scornful remarks against the social conventions involved in art, he is not only insulting this \textit{aucitya-dharma}\index{dharma@\textit{dharma}} that is being duly followed in Indian art traditions, but he himself is transgressing all limits of \textit{aucitya} in his speech.

\vspace{-.3cm}

\section*{\fontsize{13pt}{15pt}\selectfont Compositions of \textit{Devadāsī}-s:\index{devadasi@\textit{devadāsī}} Less Represented?\relax}

Krishna (2018b) writes: “The erotic compositions sung by the Devadasis and other Carnatic\index{Karnatic@Karnatic} compositions that were not rooted in bhakti\index{bhakti@\textit{bhakti}} have to be brought back to centre stage…”. He sounds as if he wants a revival of the forgotten good compositions of the past. If that was the true intent, that is certainly good and deserves to be endorsed. But his words actually implythat there is contempt towards the \textit{devadāsī-}s\textit{’} compositions either simply because they comprise erotic narrations and also simply because the \textit{devadāsī} composers are non-brahmins\index{non-brahmins@non-brahmins} and that they are therefore ignored in \textit{kacheris}\index{kacheri@\textit{kacheri}}. It is true that the \textit{Saṅgīta} Trimūrti’s \textit{kṛti-}s now dominate the stage and hearts of music lovers in Karnatic music. But what is blasphemous is that Krishna absurdly associates even this fact to the caste indirectly. The Trimūrti’s \textit{kṛti-}s are most popular mainly because they opened up newer aesthetic dimensions to classical music through their \textit{kṛti-}s. Introduction of the \textit{saṅgati}\index{sangati@\textit{saṅgati}} system, tremendous scope for \textit{manodharma}\index{manodharma@\textit{manodharma}} and beautiful jugglery with \textit{rasa-}s and the intense \textit{bhakti}\index{bhakti@\textit{bhakti}} feel in their lyrics\index{sahitya@\textit{sāhitya}} all the main factors that made them win tremendous popularity. The \textit{jāvaḷi-}s,\index{javali@\textit{jāvaḷi}} \textit{padam-}s\index{pada@\textit{pada}},\textit{tillāna-}s\index{tillana@\textit{tillāna}} and \textit{varṇa-}s,\index{varna(composition)@\textit{varṇa}(composition)} no doubt carry their own unique charm, but they do not carry as much scope for \textit{manodharma}\index{manodharma@\textit{manodharma}}. However, \textit{jāvaḷi-}s and \textit{tillāna-}s are still parts of most \textit{kacheri-}s\index{kacheri@\textit{kacheri}}, though not as the prime items. These are taught, rendered, and promoted with the genuine concern to preserve them for posterity.~Also, most Karnatic\index{Karnatic@Karnatic} music \textit{kacheri}-s commence with a \textit{varṇa}.~Many \textit{varṇa-}s, have been compositions of \textit{devadāsī-}s\index{devadasi@\textit{devadāsī}} (till recent decades). If artists considered only the caste factor, such beautiful variety of compositional forms would not be available to us at all.

\vspace{-.3cm}

\section*{Tyāgarāja\index{Tyagaraja@Tyāgarāja} Svāmi: Imprisoned in Brahminical\index{brahminical@brahminical} Prejudice?}

Krishna suggests that even Tyāgarāja Svāmi was imprisoned in brahminical prejudice.\endnote{Krishna very arrogantly declares- “Tyagaraja\index{Tyagaraja@Tyāgarāja} was an extraordinary composer, yet amidst the musical genius is his Brahminical import. He was a product of his social boundaries and we need to understand that...This is one of the problems even with the spiritual—it can be casteist and doctrinaire!” (Krishna 2018: 50).} Whenever Tyāgarāja Svāmi has mentioned the words \textit{brāhmaṇa},\index{brahmana@\textit{brāhmaṇa}} \textit{vaidika}, etc., he has indeed implied the \textit{ācāra}-\textit{vicāra} aspects. In fact, he even censures the \textit{brāhmaṇa-adhama-}s (brahmin only by birth but not by practice):

In his composition \textit{bhakti-bhicchamīyave}, he openly points to brahmins who do not live up the \textit{brāhmaṇya} ideals- \textit{prāṇamuleni vāniki baṅgāru bāga cuṭṭi…tyāgarāja-nuta rāma—} (meaning: Those who study \textit{purāṇa} and \textit{āgama}\index{agama@\textit{āgama}} but are deceitful, are indeed like corpses decorated with fine clothes and jewelry). If so prejudiced about caste, why at all would Tyāgarāja care to censure his own people thus? In his composition \textit{êndaro mahānubhāvulu,} Tyāgarāja acknowledges all the ‘known and unknown’ blessed persons who silently contribute to the \textit{paramparā}\index{parampara@\textit{paramparā}} and remain modest. This clearly hints that Tyāgarāja\index{Tyagaraja@Tyāgarāja} Svāmi stood above all social boundaries in appreciating and acknowledging true talent.

\vspace{-.3cm}

\section*{Conclusion}

That there are brahmin individuals who display prejudice is a fact that cannot be denied, the sweeping generalization made by the likes of Krishna cannot be accepted. Such prejudice is not exclusive to the brahmin community. It is rather a human weakness seen in individuals across the globe. But using this as a ‘weapon’ to demonize the whole brahmin community is not only deeply flawed but points towards an ulterior motive of aiding the Breaking-India forces.

A thorough introspection needs to be done by every Indian today – particularly, by all artists. We need to stop ‘pretending not to see’ the dangerous interventions by outsiders’ narratives into our thought and social and art structures.

Some practical steps would be –

\begin{itemize}
\itemsep=0pt

 \item Musicians of all communities must unite to uphold and retain the integrity of the music domain and fight out the divisive attempts like that of Krishna.

 \item Connecting back to Saṁskṛta scholarship related to music, to verify and cleanse our understanding of our art terminologies, interpretations and their underlying spirit.

 \item Industriously weeding out the outside narratives that have intervened into out thought and art perspectives, to re-establish the native narrative assertively.

 \item Encouraging ourselves, our students and co-artists to not only perform, but also think and speak more for the cause of the \textit{paramparā.}\index{parampara@\textit{paramparā}}

 \item \textit{Swadeshi} conferences like the present one, well organized research, profound discussions and profuse debates rooted in the native narrative, must gain more mass and popularity in the coming years. The prevalent narratives on Indology by the West and the left must be exposed to intense verification and correction in the world view.

\end{itemize}

The biggest advantage India has is that, despite lack of support and powerful defense rhetoric and despite the discrimination suffered over the years, sincere scholars in both Saṁskṛta and Karnatic\index{Karnatic@Karnatic} classical music domain have continued to work selflessly to maintain the \textit{śāstra}-\textit{śuddhi} and \textit{sampradāya}\index{sampradaya@\textit{sampradāya}}-\textit{śuddhi} all through. We only have to plug in to those sources to restore effectively the Indian spirit in Indian art in a big way.

\newpage

\section*{Bibliography}

\begin{thebibliography}{99}
\itemsep=0pt

 \bibitem{chap7-key01} Adkoli, Mahesh. (2007). \textit{Saṁskṛta Kāvya.} (English Translation by V. B. Arathi). Bangalore: Bhavan’s Gandhi Centre of Science and Human Values.

 \bibitem{chap7-key02} \textit{\textbf{Bhagavadgītā}}. See Goyandaka (2016).

 \bibitem{chap7-key03} Chandra, Moti. (1975). \textit{The World of Courtesans.} Delhi: Vikas Publishing House Pvt. Ltd.

 \bibitem{chap7-key04} Dharampal. (1971). \textit{Civil Disobedience and Indian Tradition. Collected Works, Vol. II.} Mapusa, Goa: Other India Press.

 \bibitem{chap7-key05} \textit{\textbf{Dhvanyāloka}}. See Pathak (1997).

 \bibitem{chap7-key06} Goyandaka, Jayadayal (Tr.) (2016). \textit{Śrīmadbhagavadgītā} (with Hindi Ṭīkā). Gorakhpur: Gita Press.

 \bibitem{chap7-key07} Harshananda, Swami. (2008). \textit{A Concise Encyclopaedia of Hinduism.} Bangalore: Ramakrishna Math.

 \bibitem{chap7-key08} Kale, Pandurang Vaman. (1994). \textit{History of Dharmaśāstra.} Vol 5, Part 1. Poona: Bhandarkar Oriental Research Institute.

 \bibitem{chap7-key09} \textit{\textbf{Kāmasūtra.}} See Shastri (1929).

 \bibitem{chap7-key10} Krishna, T. M. (2017). “Art, culture should help us de-baggage our identities, says TM Krishna”. \textit{Deccan Chronicle.} November 25, 2017. $<$\url{https://www.deccanchronicle.com/entertainment/music/251117/art-culture-should-help-us-de-baggage-our-identities-says-tm-krishna.html}$>$. Accessed on June 18, 2019.

 \bibitem{chap7-key11} — .(2018a). \textit{Reshaping Art.} New Delhi: Aleph Book Company.

 \bibitem{chap7-key12} — .(2018b). “Can Carnatic\index{Karnatic@Karnatic} Music be de-Brahminised?”. \textit{Open.} April 11, 2018. $<$\url{https://openthemagazine.com/art-culture/can-carnatic-music-be-de-brahminised/}$>$. Accessed on June 18, 2019.

 \bibitem{chap7-key13} \textit{\textbf{Locana}}. See Pathak (1997).

 \bibitem{chap7-key14} Naidu, Amith Manohar. (2012). \textit{Dana in the History of Karnataka with special reference to Vijayanagara period 1336-1646 A.D} (PhD Thesis). Mysore: University of Mysore. $<$\url{https://shodhganga.inflibnet.ac.in/handle/10603/73279}$>$

 \bibitem{chap7-key15} Pathak, Acharya Jagannatha. (Ed.) (1997). \textit{Dhvanyāloka of Sri Ānandavardhanācārya with Locana of Abhinavagupta and Prakāśa, the Hindi Translation.} Varanasi: Chaukhambha Vidyabhavan.

 \bibitem{chap7-key16} Ramanan, Sumana. (2014). “How Caste Plays a Role in India’s Biggest Classical Music Festival”. \textit{The Wall Street Journal}. December 18, 2014. $<$\url{https://blogs.wsj.com/indiarealtime/2014/12/18/caste-and-indias-biggest-classical-music-festival/}$>$

 \bibitem{chap7-key17} Sastri, Devadutta. (Ed.) (2012). \textit{Kāmasūtram with Hindi Commentary.} Varanasi: Chowkhambha Sanskrit Sansthan.

 \bibitem{chap7-key18} Shastri, Damodar. (1929). \textit{Vātsyāyana’s Kāmasūtra with Jayamaṅgalā Ṭīkā of Yaśodhara.} Benares: Chowkhamba Sanskrit Series Office.

 \bibitem{chap7-key19} Tapasyananda, Swami. (2005). \textit{Śrīmad Bhagavad Gītā: The Scripture of Mankind.} Bangalore: Ramakrishna Math.

 \end{thebibliography}

\theendnotes

