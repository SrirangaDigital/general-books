
\chapter{Intellectual History of Pollock}\label{Anubandha1}

The development of Pollock’s\index{Pollock, Sheldon} thought can be broadly divided into five phases. The \textbf{first} phase represents the pre–\textit{Orientalism} period and includes most of the works on the \textit{Rāmāyaṇa}\index{Ramayana@\textit{Rāmāyaṇa}}. The \textbf{second }phase, in responding to Edward Said’s\index{Said, Edward} criticism, begins with Pollock trying to “exhume” power and domination within Sanskrit texts. Several papers on \textit{śāstra}–s\index{sastra@\textit{śāstra}} (and in particular Mīmāṁsā\index{Mimamsa@Mīmāṁsā})are written during this period. The \textbf{third} phase begins in 1993 with aestheticization of power or “poetics of power” – interpreting \textit{kāvya} as exercising political power. The \textbf{fourth} phase begins with the paper on \textit{Rasa}\index{rasa@\textit{rasa}} in the late nineties. The \textbf{fifth} phase is directed towards theorizing philology\index{philology} in an effort to postpone its eventual death. Pollock is sensitive to current events taking place both in the West and in India, and his academic career is strongly influenced by such events.

Pollockian philology also has three levels of interpretations: textual, political and liberation. These three aspects can be summarized as interpreting Sanskrit texts (textual) in a political framework (political), and then to use this knowledge to be “disruptive” and liberate Sanskrit tradition (liberation) from elite control. In this \textit{prabandha}, a \textit{pramāṇa-parīkṣā}\index{pramana@\textit{pramāṇa}}\index{pariksa@\textit{parīkṣā}} of the three–dimensional philology has been used to show that in all his works, while claiming to represent Indian tradition, only Western theories are imposed on Sanskrit tradition. The aim of this section of the \textit{prabandha} is to chronologically survey the works of Pollock and see if the three–dimensional philology is evident. Is this how human understanding works as he claims?

\newpage

From the perspective of \textit{pramāṇa–}s\index{pramana@\textit{pramāṇa}}, the five phases of Pollockian thought can be classified in the following manner. In the first or pre–\textit{Orientalism} phase, \textit{śloka}–s from the \textit{Rāmāyaṇa} and other Sanskrit texts (\textit{śabda}\index{sabda@\textit{śabda}}–\textit{pramāṇa}) are interpreted to establish the thesis statements. In the subsequent phases, the effort is to go beyond the texts (to textualized thought) to understand the Indian conscious, and thus nothing from the texts (\textit{śabda}–\textit{pramāṇa}) would be shown to substantiate the thesis statements. Thus, inscriptions play an important role from 1993 which are then used to incorrectly infer various theses. Edward Said\index{Said, Edward} remains a constant companion throughout Pollock’s\index{Pollock, Sheldon} career.

\vspace{-.3cm}

\section*{A1.1 Ph.D. Thesis (1977)}

\vspace{-.2cm}

(1977a)\textit{ Aspects of Versification in Sanskrit Lyric Poetry}\endnote{ This work was not available to me at the time of writing this \textit{prabandha.}}

Pollock’s doctoral thesis, as the title indicates, is about some aspects of \textit{chandas}\index{chandas@\textit{chandas}} in Sanskrit poetry. His academic career will mostly focus on the theme of Sanskrit \textit{kāvya} with the exception of a few papers on \textit{śāstra}\index{sastra@\textit{śāstra}}.

\vspace{-.3cm}

\section*{A1.2 Pre–\textit{Orientalism} Works (1979–1984)}

\vspace{-.2cm}

(1979a) “Text Critical Observations on Vālmīki\index{Valmiki@Vālmīki} Rāmāyaṇa”\index{Ramayana@\textit{Rāmāyaṇa}}

(1983a) “Some Lexical Problems in Vālmīki Rāmāyaṇa”

(1984a) “The Rāmāyaṇa Text and the Critical Edition”

The early works represent the pre–\textit{Orientalism} period (Edward Said’s criticism is yet to take hold of Pollock’s mind) where the \textit{śloka}–s from Sanskrit texts are quoted to support thesis statements. Pollock’s introduction to and translation of Ayodhyākāṇḍa\index{Ayodhyakanda@Ayodhyākāṇḍa} and Araṇyakāṇḍa\index{Aranyakanda@Araṇyakāṇḍa}, and the early papers on the \textit{Rāmāyaṇa} are based on the critical edition published by the Oriental Institute, Baroda. This edition followed the principles laid down by V. S. Sukthankar\index{Sukthankar, V. S.}, the chief editor of the critical edition of the \textit{Mahābhārata}\index{Mahabharata@\textit{Mahābhārata}} prepared by Bhandarkar Oriental Research Institute at Pune. The methodology of these editions was partly based on the method developed by German Indologists during the post–1800 period when Sanskrit was introduced to Europe on a large scale. The issue here is whether the critical editions are considered authentic by the (Indian) tradition. Pollock considers the critical edition as “provisional only” and thus still having some mistakes (\textit{śloka}–s that are included or excluded incorrectly) which are corrected in the notes to his own translation.

\begin{myquote}
In fact, when the \textbf{interpolations} of the Ramayana are \textbf{excised}, a perfectly smooth text usually does result. The editors may have erred either way in their application of it, but the \textbf{principle} itself repeatedly demonstrates its validity. And the result is remarkable: a full 25 percent of the \textbf{vulgate} (the southern recension) has been eliminated as not deriving from Valmiki’s\index{Valmiki@Vālmīki} monumental poem. (1984a: 92)
\end{myquote}

The commentators of the \textit{Rāmāyaṇa}\index{Ramayana@\textit{Rāmāyaṇa}} have discussed additional \textit{śloka}–s and sections but agree with most of the text. Thus, a text where 25 percent of the \textit{śloka}–s are “excised” cannot be considered as authentic by the tradition. Moreover, the critical edition which resulted was not attested to by even a single manuscript from out of a collection of more than hundred manuscripts.

Two aspects are important in these papers. The first is that philology\index{philology} does not have a political dimension. Consider the following passage where philology is equated to language analysis of the text.

\begin{myquote}
We have no complete \textbf{grammar} of the epic dialect, no adequate \textbf{dictionary}…and worst of all no \textbf{concordances}. Until all the evidence is fully and sensitively assembled, the \textbf{philological study} of the epic will not progress…
\end{myquote}

But in the papers from 1985 (post–Saidian\index{Said, Edward} period), philology becomes primarily political. The second aspect is that the understanding of the commentators is easily dismissed.

\begin{myquote}
As well as a few [lexical problems of the \textit{Rāmāyaṇa}] which, I believe, have been \textbf{imperfectly understood both in the Indian tradition} and the West. (1983a: 275)
\end{myquote}

In certain places, Pollock\index{Pollock, Sheldon} rejects the traditional interpretation and establishes his own viewpoint. But the three–dimensional philology repeatedly states the importance of preserving the traditional meaning along plane 2\index{Second Dimension/Traditional/Plane 2}. In these early papers, only a single dimension of meaning is seen.

(1984b) “Divine King in the Indian Epic”

(1984c) “\textit{Ātmānaṁ mānuṣaṁ manye}: \textit{Dharmākūtam}\index{Dharmakuta@\textit{Dharmākūta}} on the Divinity of Rama”\index{Rama@Rāma}

(1985a) “Rākṣasas and Others”

(1986a)\textit{ The Rāmāyaṇa of Vālmīki\index{Valmiki@Vālmīki}: An Epic of India, Volume II: Ayodhyākāṇḍa\index{Ayodhyakanda@Ayodhyākāṇḍa}}

(1991a)\textit{ The Rāmāyaṇa of Vālmīki: An Epic of India, Volume III: Araṇyakāṇḍa}\index{Aranyakanda@Araṇyakāṇḍa}

The translations of Ayodhyākāṇḍa and Araṇyakāṇḍa are probably completed before the 1985 paper on Indian intellectual history even though the publication dates show otherwise. This can be known both by internal evidence\endnote{ \textit{The Divine King in the Indian Epic} refers to the translation of Araṇyakāṇḍa (1984b: 509).} and also by the style of writing. There is generally no effort to show how the Indian tradition viewed itself which will be the predominant theme in papers written after 1985. \textit{Rakshasas and Other} and \textit{Rama’s Madness} are both included in the introduction to Araṇyakāṇḍa. \textit{Rakshasas and Other} is used to show the othering of Muslims in the later writings on the \textit{Rāmāyaṇa}\index{Ramayana@\textit{Rāmāyaṇa}}. As the titles indicate, all of them are related to the \textit{Rāmāyaṇa} and aestheticization of power is not yet theorised in these works. Epics are repeatedly referred to as “myths” and Western lens is readily adorned when discussing the \textit{ādikāvya}.

\begin{myquote}
A \textbf{work of fiction} it [\textit{Rāmāyaṇa}] no doubt is, but fiction that in both its genesis and its reference is historically constituted…(1984b:14). The status of junior members of the Indian household was, historically, not very dissimilar to that of \textbf{slaves} (as was also the case in ancient Rome) …(1984b: 14)
\end{myquote}

This is the kind of approach that Edward Said\index{Said, Edward} had critiqued. But \textit{Orientalism }had not yet entered Pollock’s academic life and thus the \textit{Rāmāyaṇa }is interpreted from a Western perspective. It should be noted that the traditional views of the medieval commentators (which should constitute the plane 2\index{Second Dimension/Traditional/Plane 2} reading) are easily dismissed.

\begin{myquote}
If comparison between literary cultures offers little help in understanding what Rāma’s madness means in Vālmīki’s epic, we do best to remain as close as possible to Indian presuppositions. Yet the \textbf{narrow presuppositions of the medieval interpretations }do not take us very far, either. (1991a: web edition)
\end{myquote}

In trying to understand Rāma’s madness, Pollock\index{Pollock, Sheldon} begins by comparing with Shakespearean\index{Shakespeare} heroes, but concludes that such comparisons do not help. Then, the traditional commentators' understanding of this episode is easily dismissed in a single line. Not only are the meanings along the three planes absent, the traditional understanding of the text is discarded as being “narrow”. But it was claimed that the three–dimensional philology\index{philology} is how the human mind works and one of the “great promises of philology” was to preserve the truth along the traditional plane. Nothing of that sort is seen here and thus in the earlier papers, only a single dimension of Pollockian meaning is present. These earlier works represent the “normal” writings of Pollock\index{Pollock, Sheldon} as there is no effort to “exhume” meanings deeply embedded in Sanskrit texts.

\vspace{-.3cm}

\section*{A1.3 On Ingalls\index{Ingalls, Daniel H. H.} (1985)}

\vspace{-.2cm}

(1985b) “Daniel Henry Holmes Ingalls”

A Pollock reader would begin with this short piece along with \textit{Rice and Ragi: Remembering URA} (2015c) to show the significant change in the style of writing resulting in the theory–ladenness of the later papers. Change is a natural process, but Said’s\index{Said, Edward} devastating criticism forces Pollock to take an artificial approach in trying to “exhume” deeply embedded theories and thereby discarding his own teacher’s advice.

\begin{myquote}
…literary study ought by preference to be suggested by the texts themselves, and \textbf{not imposed by any external theory}…a respect, too, for the Sanskrit tradition of exegesis and traditional categories of analysis…(1985b: 388)
\end{myquote}

Commenting on this passage (which is the view of Ingalls), Pollock states that “though this is not blind respect, but rather is tempered by the awareness that no tradition is altogether competent to explain itself.” The incompetence of Indian tradition to explain itself will be the basis of Pollockian philology as this concept is referred to in the philological papers also: “since no culture is competent to understand itself in its totality” (2009b: 954). Thus, ultimately one has to go beyond traditional ways of understanding and this implies that Pollockian theories will not have a basis in Sanskrit texts or oral\index{oral tradition} traditions which has been documented by this \textit{prabandha}.

\vspace{-.3cm}

\section*{A1.4 On \textit{Śāstra}\index{sastra@\textit{śāstra}} (1985–1993)}

(1985c) “The Theory of Practice and Practice of Theory in Indian Intellectual History”

(1989a) “Mīmāṁsā\index{Mimamsa@Mīmāṁsā} and the Problem of History in Traditional India”

(1989b) “The Idea of \textit{Śāstra} in Indian Tradition”

(1989c) “Playing by the Rules: \textit{Śāstra} and Sanskrit Literature”

\newpage

(1990a) “From Discourse of Ritual to Discourse of Power in Sanskrit Culture”

(1993a) \textit{Kṛtyakalpataru} of Lakṣmīdhara\index{Laksmidhara@Lakṣmīdhara}

(2011g) The Revelation of Tradition: \textit{Śruti}\index{sruti@\textit{śruti}}, \textit{Smṛti}\index{smrti@\textit{smṛti}} and the Sanskrit Discourse of Power\endnote{ This is a corrected version of a paper submitted in 1988 and published in 1997: Lienhard, S., Piovano, I., (Eds.) (1997), Lex et Litterae. Studies in Honour of our Professor Oscar Botto, Edizioni dell’Orso, Torino.}

Edward Said’s\index{Said, Edward} \textit{Orientalism} had left the field of humanities devastated and as a response to this criticism, Pollock begins the process of trying to “exhume” hidden meanings in Indian texts to show power and domination embedded in them. Beginning with the first paper on \textit{śāstra}\index{sastra@\textit{śāstra}} and in all subsequent papers, the effort is to show how the Indian tradition viewed itself, or how “actors” saw themselves. The main theme of these papers is that \textit{śāstra}–s went from being descriptive to prescriptive (normative) which was then used by the elite to control and dominate the Indian society. No \textit{pramāṇa}–s\index{pramana@\textit{pramāṇa}} from śāstric texts are shown to substantiate this supposed transformation which is formulated on the basis of some Western theory. Several of these works were analysed in the section on the five \textit{avayava}–s\index{avayava@\textit{avayava}}\break [4.6]. The \textit{Kṛtyakalpataru} of Lakṣmīdhara is included here since it is a continuation of the \textit{śāstra} theme even though it is part of \textit{Deep Orientalism}. A theory speculated in one paper is assumed in the subsequent papers to be established and this transformation (from descriptive to prescriptive) is important in later works also.

\begin{myquote}
What in the early period may have been encountered as dialectal or regional variation and described as difference was transformed by the beginning of the first millennium (in Patañjali\index{Patanjali@Patañjali}) into \textbf{prescriptive} option. (2006a: 268)
\end{myquote}

Neither Patañjali nor the commentators make any mention about this “shift” and thus the early papers are essential and form the basis of Pollock’s\index{Pollock, Sheldon} later theories. No “aestheticization of power” is seen in these works which are only about power and domination in śāstric texts. The nonexistence of the three dimensions should also be noticed in these papers.

\vspace{-.5cm}

\section*{A1.5 Deeper Orientalism? (1993)}

(1993a) “Deep Orientalism? Notes on Sanskrit and Power beyond the Raj”

\newpage

This paper is from a collection of papers published under the title “\textit{Orientalism and the Postcolonial Predicament: Perspectives on South Asia New Cultural Studies}” exploring the theme of orientalist ideas (knowledge) and the colonial project to rule India (power). Edward Said’s\index{Said, Edward} criticism of Oriental studies had a profound effect in both Europe and America and began what is now called the “post–\textit{Orientalism}” period. However, as Said himself would sadly realize and acknowledge later, not only nothing changed but the situation actually got worse as in the case of Pollock\index{Pollock, Sheldon}. Before the Saidian criticism, Pollock directly imposed Western theories and models on Sanskrit texts as is seen in the early works on the \textit{Rāmāyaṇa}\index{Ramayana@\textit{Rāmāyaṇa}}. After \textit{Orientalism}, Sanskrit texts are “exhumed” to show power structures which are very much akin to Western theories and models.

\textit{Deep Orientalism}? has two parts to it: the first concerns German Indology and the second illustrates power and domination in \textit{Kṛtyakalpataru} of Lakṣmīdhara\index{Laksmidhara@Lakṣmīdhara}, a \textit{Dharma–nibandha}. The very first footnote of this paper clearly states the source of the methodology (that of Giddens\index{Giddens, Anthony}) used in interpretation and was discussed in a previous section [4.6.5]. Both Giddens and Weber are referred to repeatedly in the \textit{Language of the Gods in the World of Men }(Giddens about 7 times and Weber about 32 times including footnotes) sometimes as \textit{pūrvapakṣin}–s, i.e., to critique their views. Weber’s theory of life chances and Giddens interpretation of it as the capacity of a “literate population” as opposed to oral one is vital to Pollock’s own theory. Mobilizing a population across time and space is a theory that will be applied to \textit{kāvya}–s from the 1993 paper on the\textit{ Rāmāyaṇa}. The mobilization of a literate population as opposed to an oral one leads Pollock to connect \textit{kāvya} with a written culture and the contradictions arising from such theorizing were discussed earlier [2.6].

\textit{Deep Orientalism}? revealed to Pollock power and domination in Indian śāstric traditions which was then connected to German Indology which supposedly assimilated these categories. A deeper\textit{ Orientalism} reveals to us that Western theoretical models are applied to Indian śāstric traditions (to infer power and domination) which is then connected to German Indology! The impossibility of a three–dimensional philology\index{philology} should be clear.

\vspace {-.4cm}

\section*{A1.6 Poetics of Power (1993)}

\vspace {-.2cm}

(1993b) “Rāmāyaṇa and Political Imagination in India”

(1995f) “Rāmāyaṇa\index{Ramayana@\textit{Rāmāyaṇa}} and Public Discourse in Medieval India”

Reflecting on the demolition of Babri\index{Babri Masjid} Masjid in 1992, Pollock\index{Pollock, Sheldon} writes this piece on the \textit{Rāmāyaṇa}. This work begins the third phase of Pollockian philology called “poetics of power” or “aestheticization of power” (\textit{Deep Orientalism} briefly discusses the “poetics of power”\endnote{ “What are we to do with such a statement in light of the above–mentioned claims about the transformative impact of colonialism and in our attempt to reconstruct a “poetics of power" for precolonial India?” (1993a: 102). This term is also repeatedly used in \textit{The Language of the Gods}.} which intends to connect the \textit{ādikāvya }with the political.) The issue here is how a political aspect within a text like the \textit{Rāmāyaṇa} is carried over for a period of two thousand years. Since the \textit{Rāmāyaṇa} does not contain references to political aspects directly, implying a lack of \textit{śabda–pramāṇa}\index{pramana@\textit{pramāṇa}}\index{sabda@\textit{śabda}}, inscriptions, temple constructions, compositions of vernacular \textit{Rāmāyaṇa}–s are used to infer the connection between politics and \textit{ādikāvya}. One has to realize that providing epigraphical evidences (as is done in \textit{The Language of the Gods}) to substantiate a theory implies that it will have no basis in the Sanskrit texts (such as the \textit{Rāmāyaṇa}, grammar, etc.).

\begin{myquote}
\textit{Rāmāyaṇa}…is, to be sure, more than a single text. For some scholars it rather approximates a literary genre, library, or language, added to, reworked, rewritten in every region and every community, and in \textbf{every century for the last twenty}. (1993b: 288)
\end{myquote}

But in spite of this vast literature in Sanskrit and Vernacular languages, not even a single text (or even a single \textit{śloka} from the innumerable texts) discusses the political nature. Thus, Pollock tires to infer from inscriptions. But, inscriptions themselves do not connect the \textit{ādikāvya} with the political. In an earlier section [2.6], what was fact (oral\index{oral tradition} composition of the\textit{ ādikāvya}) was considered fiction. Here, what is fiction (political aspect) is considered a fact and even the use of the word “imagination” in title of this paper is inspired by a French theorist:

\begin{myquote}
...in order to understand the ordering of human societies and to discern their evolutionary forces, we have to direct our attention equally \textbf{to mental phenomena}…whose intervention is unquestionably just as determinative as that of economic and demographic phenomena. (1993a: 103)
\end{myquote}

But even to understand the “mental phenomena” of a culture, one must base it on a textual source. This theory of mental phenomena (imagination) is readily applied to the \textit{ādikāvya} in an attempt to penetrate the Indian mind and there is no attempt to even represent tradition (plane 2)\index{Second Dimension/Traditional/Plane 2}. That the three–dimensional philology\index{philology} is not possible here should be obvious. An alternate title to this paper would enable the readers to grasp the nature of philology: \textit{Pollock and Political Imagination in Philology}.

\vspace{-.3cm}

\section*{A1.7 Sanskrit Cosmopolis and \textit{Kavipraśaṁsā} (1995–1997)}

(1995a) “In Praise of Poets: On the History and Function of Kavipraśaṁsā”

(1995b) “Literary History, Religion and Nation in South Asia: Introductory Note”

(1995c) “Literary History, Indian History, World History”

(1995d) “Public Poetry in Sanskrit”

(1996a) “Philology, Literature, Translation”

(1996b) “The Sanskrit Cosmopolis, A.D. 300–1300: Transculturation, Vernacularization\index{vernacularization} and the Question of Ideology”

\textit{Literary History, Indian History, World History }was examined earlier [4.6.6] and requires a further enquiry because of its importance. It was mentioned several times that Pollock’s\index{Pollock, Sheldon} theories would have no basis in Sanskrit texts and the following passage indicates the inherent lack of \textit{pramāṇa}–s\index{pramana@\textit{pramāṇa}}.

\begin{myquote}
The problems chosen are those of the \textbf{outside of the text}, of communicative contexts and practices, without which the inside of the text must remain unintelligible for any historicist understanding of literary discourse, which always has primacy in critical scholarship. (1995c: 113)
\end{myquote}

If the questions and problems are outside the text, then the answers to such questions would also be outside the text and thus outside the tradition (or plane 2\index{Second Dimension/Traditional/Plane 2} reading). Even along the historical or personalist plane, what is inside of the text (\textit{śabda–pramāṇa}\index{sabda@\textit{śabda}}) should be used to theorize about Indian society. Although Indian tradition never explicitly mentions the beginning or the transition to writing, this silence is interpreted as suppression.

\begin{myquote}
…we may need to recognize that in India, as everywhere else, writing has been a social resource that has been \textbf{kept deliberately scarce}… (1995c: 121)
\end{myquote}

Lack of \textit{pramāṇa}–s\index{pramana@\textit{pramāṇa}} and the nonexistence of an object is \textit{interpreted} as something that has been purposely kept scarce. The second important theory that this paper “establishes” is the relation between languages. Pollock’s\index{Pollock, Sheldon} unambiguous words are as follows:

\begin{myquote}
My point in framing this hypothesis [about dominant forms of culture and prestige languages] is in part to invite us to \textbf{develop such theories}, but also to make a historical point. The instances that I mention prompt us to take seriously the principle that\textbf{ Gramsci }most powerfully formulates: there is no “parthenogenesis” in cultural history. Language does not merely “produce another language,” does not change by reacting solely upon itself; on the contrary, “innovations occur through the \textbf{interference of different cultures}…” (1995c: 126)
\end{myquote}

The “interference” between Sanskrit and regional languages is an important topic to substantiate the second great moment (of “vernacularization”)\index{vernacularization} that is theorized in \textit{The Language of the Gods}. This 1995 paper will be the basis of that theory and it should be noted that no such discussion (conflict of languages) is found either in Sanskrit or vernacular texts. If there is no \textit{śabda–pramāṇa}\index{sabda@\textit{śabda}} and living traditions are denied, then theories would have their basis in \textit{upamāna–pramāṇa}\index{upamana@\textit{upamāna}} (comparing with Western models). Along the second plane (traditional meaning), the basic rule is the non–application of theory, but in this paper, we are invited to develop social theories to understand the interaction of languages. All the other papers in this period discuss similar topics and the impossibility of the three–dimensional philology\index{philology} should be clear.

\vspace{-.3cm}

\section*{A1.8 \textit{Rasa Śāstra}\index{sastra@\textit{śāstra}}\index{rasa@\textit{rasa}} (1998)}

(1998a) “Bhoja's\index{Bhoja} {\it Śṛṅgāraprakāśa}\index{Srngara-prakasa@\textit{Śṛṅgāra-prakāśa}} and the Problem of {\it Rasa}: A Historical Introduction and Annotated Translation”

This paper begins the study of \textit{Rasaśāstra} by Pollock and represents the fourth and perhaps the most important phase. The essence of this paper is the use of \textit{Rasa} by kings such as Bhoja to control their subjects or “create politically correct subjects and subjectiveness” which was discussed previously [4.6.7]. By now, three important theories are considered to be established.

\vspace{-.3cm}

\begin{enumerate}
\itemsep=0pt
\item Power and domination in \textit{śāstra}–s\index{sastra@\textit{śāstra}} – This is inferred due to the transition of \textit{śāstra–s }from being descriptive to being normative.

 \item Poetics of Power – Composition of the \textit{Rāmāyaṇa}\index{Ramayana@\textit{Rāmāyaṇa}} and other\textit{ kāvya–s }are interpreted as exercising political power.

 \item \textit{Rasa}\index{rasa@\textit{rasa}} is used by kings to control and regulate the subjects.

\end{enumerate}

\vspace{-.2cm}

All later works build on these three important theories. As this paper represents a midway point in this intellectual history, an overview of Pollockian theory and practice would be helpful. Does his theory follow practice? The paper on D.D. Kosambi\index{Kosambi, D. D.} (2008a) represents the initial attempt to theorize philology\index{philology} where the importance of avoiding the application of Western models on Indian tradition is repeatedly stressed and for which Kosambi himself is faulted. A careful study of several papers of Pollock\index{Pollock, Sheldon} so far reveals the repeated application of Western theory on Sanskrit tradition. Kosambi was at least being honest and direct in applying Western models while Pollock hides behind the three–dimensional philology to “exhume” culture–power nexus within Sanskrit texts. Writing on Kosambi’s understanding of feudalism in medieval India, it is stated that

\begin{myquote}
Its dust–dry shastric exercises over tax or rent, peasant or serf, class or caste are often completely \textbf{a priori} and devoid of any engagement with \textbf{real empirical data} and actual texts. (2008a: 56)
\end{myquote}

Pollock questions Kosambi as to why we should even bother to study Indian texts when we know in advance what they should mean. Ironically, Pollock himself discards empirical data (both living traditions and texts) and always applies Western models. But the philological papers claim the opposite.

\begin{myquote}
...a rejection of systematicity and of the mechanical \textbf{a priori application of theory}, a willingness to challenge theory’s omnipotence and omniscience with the realities of the particularities and messiness of history...(2008a: 58)
\end{myquote}

We have repeatedly observed that theory’s predominance over \textit{pramāṇa}–s\index{pramana@\textit{pramāṇa}} is the basis of the Pollockian philology and it has remained unchanged for four decades. The “messiness of history” means that there is no data in the texts themselves that can be used for substantiating theories. Pollock attempts to equate philology with disciplines of mathematics and philosophy, just as Kosambi\index{Kosambi, D. D.} equated his methodology with that of Gauss, Faraday, and Darwin\index{Darwin, Charles}. Once again, it is stated that

\begin{myquote}
…if the past is studied in a \textbf{spirit of theoretical openness}–and not as if we knew beforehand what it was going to tell us–it might teach us something we do not already know, and make once–old resources, of culture or power, newly available to us. (2008a: 58)
\end{myquote}

The nine thesis statements analysed earlier [4.6] show that Pollock\index{Pollock, Sheldon} seems to know beforehand what the texts are going to teach: culture and power are invariably linked. One has to constantly realize that such statements (regarding culture and power) are neither found in any Sanskrit text nor preserved by the oral\index{oral tradition} traditions. The three passages above were cited to illustrate the vast difference between Pollock’s theory and practice. In the philological papers, the rejection of theory and the need for understanding tradition is stated only to impress the administrators and to save philology\index{philology} from its imminent death. In every paper of Pollock that is analysed in this \textit{prabandha }(and this can be extended to all of the other papers of his), the only “openness” that is seen is in choosing which Western theory should be applied to Indian tradition. Thus, Pollock’s papers on philology are meant to deceive the readers and are directly opposed to his own practice.

\vspace{-.3cm}

\section*{A1.9 \textit{Amara Bhāratī} (2001)}

(2001b) “The Death of Sanskrit”

This work is a response to the efforts of the then BJP government to popularize Sanskrit by declaring 1999–2000 as the year of Sanskrit. The claim here is that Sanskrit is dead in some historical sense implying that even though Sanskrit literature is still being created, it has very little or no social impact. That Sanskrit was very much alive in 1800 C.E. and greatly influenced European understanding of language which then led to a flowering of European thought is a history that needs to be told in greater detail. That the Sanskrit traditions of Āyurveda and Yoga are greatly influencing the world now also shows the vitality of Sanskrit in modern times.

\vspace{-.3cm}

\section*{A1.10 Early–Modern India (2000–2004)}

\vspace {-.2cm}

(2001a) “New Intellectuals in Seventeen Century India”

(2002c) “Sanskrit Knowledge–Systems on the Eve of Colonialism”

(2011b) “Introduction in Forms of Knowledge in Early Modern South Asia”

(2011c) “The Languages of Science in Early–Modern India”

Early–Modern is a term used to indicate Islamic rule (1550–1750) in India before the coming of the British. Even though innovations in Indian intellectual traditions can be noticed in all periods, Pollock\index{Pollock, Sheldon} limits it to the early–modern period implying that Islamic rule had something to do with it. A project called the \textit{Sanskrit Knowledge–Systems on the Eve of Colonialism} is started to collect and analyse manuscripts during this period to show innovation. An earlier paper on Indian intellectual history (1985c) had claimed that there could be no newness or innovation in Indian tradition. The problem once again is the lack of \textit{pramāṇa}–s\index{pramana@\textit{pramāṇa}}: the “new intellectuals” (such as Appayya\index{Appayya Diksita@Appayya Dīkṣita} Dīkṣita and Jagannātha\index{Jagannatha Pandita@Jagannātha Paṇḍita} Paṇḍita) of the early–modern period do not refer to Islamic contact as an inspiration for their works. There is not a single scholar in this period that attributes “innovation” to an external source (Persian or Islamic). In fact, no pundit is even viewing this period as innovative. However, the lack of \textit{pramāṇa}–s has never stopped Pollock from theorizing since Indian tradition is supposed to have “eliminated” all historical references. Words such as \textit{navya} or new (\textit{pracīna},\textit{ pracīnānuyāyin},\textit{ atinavīna},\textit{ ādhunika},\textit{ navyatara},\textit{ abhinava}) are used to infer that some sort of historical approach (“historicist periodisation”) came into existence for the first time. Obviously, Sanskrit writers themselves make no mention of any new historical approach in their works. Thus, this idea of a new historical consciousness must have an external source.

\begin{myquote}
If we accept the construction of modernity that judges it to be…\textbf{a different mode of structuring temporality} whereby the ‘continues present’ of tradition gives way to a world in which the past and the future are understood as discrete phenomena…(2001a: 22)
\end{myquote}

This passage outlines the framework used to understand “newness” (\textit{navya}) in Indian tradition and the footnote directs us to a sociological theorist (Anthony Giddens\index{Giddens, Anthony}, \textit{The Consequences of Modernity}, pp.104–105) whose ideas are borrowed (both verbally and conceptually). Structuring time is one way to distinguish modernity from an earlier period, and the use of the words such as \textit{navya }indicates a different way the early–modern writers represented time. This “newness” is then attributed to contact with Persian culture. But all this has no basis in Sanskrit texts and this is again a case of Western theory (sociological model) applied to Indian tradition. But what does Pollock\index{Pollock, Sheldon} really think of sociological thinkers?

\newpage

\begin{myquote}
Thinkers, especially \textbf{sociological thinkers}, (for whom… “history tends to be mildly annoying stuff which happens between one sociological model and another”), are far less readily inclined to bother with the boring task of excavating premodernity than to sit back and \textbf{simply imagine it}…(2011b: 2)
\end{myquote}

In the earlier paper (2001a), a social theory was applied to “imagine” premodernity in India, but sociological thinkers are now (2011) criticised for imagining such things. We had repeatedly seen that in philology\index{philology}, \textit{interpretation} and theorizing always supersedes the \textit{pramāṇa}–s\index{pramana@\textit{pramāṇa}}. In this case, not only are the traditional (plane 2)\index{Second Dimension/Traditional/Plane 2} and historical (plane 1)\index{First Dimension/Historical/Plane 1}\index{Second Dimension/Traditional/Plane 2}\index{Third Dimension/Presentist/Plane 3} meanings not possible (as they are based on a sociological theory), even the personalist meaning (plane 3)\index{Third Dimension/Presentist/Plane 3} is not possible here since Pollock’s rejection of sociology leads to a rejection of his own theorizing!

\vspace {-.4cm}

\section*{A1.11 Literary History (2003)}

(2003b) “Introduction in In Literary Cultures in History: Reconstructions from South Asia”

(2003c) “Sanskrit Literary Culture from the Inside Out”

As the title suggests, these papers do not discuss the history of Sanskrit literature, but how the Indian culture viewed Sanskrit literature.

\begin{myquote}
We cannot orient ourselves to a text without first grasping how its readers oriented themselves…(2003b: 14)
\end{myquote}

This brings us back to the primary question raised throughout this \textit{prabandha}: what \textit{pramāṇa}–s are used to grasp how the readers (Indians) oriented themselves?

\begin{myquote}
Of course, no audience, however primary, \textbf{is omnipotent in its capacity to understand its own culture}; texts can be thought to bear meanings—ideological meanings, for example—\textbf{that by definition are unavailable to primary readers}. Yet we cannot possibly know and make sense of what early readers could not see until we know what they did see. For this reason, too, the prior recuperation of historical reading practices is a \textbf{theoretical necessity} of scholarship. (2003b: 14)
\end{myquote}

This passage illustrates the essence of Pollockian philology\index{philology}. That the Indian culture does not have the capacity to understand itself is an idea repeatedly seen in the philological works. Again, certain ideological meanings (such as the political nature of the \textit{Rāmāyaṇa}) are not available to primary readers (which can also be extended to medieval readers). In other words, if ideology is textual, then the readers would grasp it. If they are unable to grasp it, then ideology is not based on the texts. Thus, there would be no textual evidence (\textit{śabda–pramāṇa}\index{pramana@\textit{pramāṇa}}\index{sabda@\textit{śabda}}) to support the ideological meanings (which are the thesis statement of Pollock\index{Pollock, Sheldon}). If there is no \textit{śabda–pramāṇa} and \textit{pratyakṣa–pramāṇa}\index{pratyaksa@\textit{pratyakṣa}} is denied (as living traditions are considered dead), then the only possibility to establish an inference (\textit{anumāna}\index{anumana@\textit{anumāna}}) would be by \textit{upamāna}\index{upamana@\textit{upamāna}} (comparing and applying a Western theory). And finally, it is stated that all this is actually a “theoretical necessity” which means that every paper will follow the same method: \textit{interpretation} and theory’s primacy over the \textit{pramāṇa}–s. Even the kind of historiography used to evaluate Indian tradition is non–traditional.

\begin{myquote}
Most important of all, this search would mean learning to think in a \textbf{historical–anthropological} spirit... (2003b: 14)
\end{myquote}

Pollock’s antagonism towards anthropology has been noted earlier [4.6.8]. One can clearly see the impossibly of the three–dimensional philology and the theories substantiated by Pollock will only be an application of Western models.

\vspace {-.4cm}

\section*{A1.12 Vernacularization\index{vernacularization} (2004)}

(1998b) “India in the Vernacular Millennium: Literary Culture and Polity, 1000–1500”

(2004c) “The Transformation of Culture–Power in Indo–Europe, 1000–1300”

(2004a) “A New Philology: From Norm–bound Practice to Practice–bound Norm in Kannada Intellectual History”

Pollock develops a theory of vernacularization, a dramatic shift that happened at the beginning of second millennium from Sanskrit to regional languages signifying the end of Sanskrit dominance. In one of the papers (2004a), Kannada is used as the prime example of this transformation and the grammatical texts composed in Kannada are supposed to signify this shift. This theory reaches its culmination in the \textit{Language of the Gods in the World of Men}. Once again, no vernacular writer discusses this momentous shift that occurred, but the lack of \textit{pramāṇa}–s\index{pramana@\textit{pramāṇa}} does not stop Pollock\index{Pollock, Sheldon} from theorizing (as Indian tradition does not have the capacity to completely comprehend itself). Kannada and Telugu writers make no mention of this shift of power from Sanskrit to the vernaculars, nor are there any court records that document this shift.

\begin{myquote}
The kind of ideational conjuncture presented by Oresme, famously echoed in the next century by Lorenzo de’ Medici for Tuscan and Antonio de Nebrixa for Castilian, \textbf{has never been directly expressed} in any Indian text before modernity. (1998b: 65)
\end{myquote}

European authors such as Orseme (French), Medici (Italian) and Nebrixa (Castilian) discuss the topic of vernacularization\index{vernacularization}, but such discussions are not found in Sanskrit texts.~If this concept of vernacularization is not mentioned in texts, then as repeatedly shown in the previous sections, only \textit{upamāna}\index{upamana@\textit{upamāna}}–\textit{pramāṇa} can establish this thesis. Even if vernacularization is accepted, Tamil poses a significant problem to this theory and thus it is called a historiographically difficult language and \textit{Tolkāppiyam}\index{Tolkappiyam@\textit{Tolkāppiyam}} is placed around the 12th century while the actual dating is around the beginning of the first millennium.

\begin{myquote}
While the \textit{Tolkāppiyam}is often dated to the early centuries of the first millennium, one sober assessment places it a few centuries before the appearance of its first commentaries in the thirteenth century…(2006a:399fn.) The \textbf{commentary mystifies} its own history by placing itself in the \textit{caṅkam} age. Its true date is probably a little before the beginning of the second millennium…(2006a: 385fn.)
\end{myquote}

Even though the Nakkīraṇār’s\index{Nakkiranar@Nakkīraṇār} commentary on \textit{Tolkāppiyam }places itself in the \textit{caṅkam} period, Pollock discards it and places it around 1000 C.E. Once again, \textit{interpretation} supersedes textual evidence and traditional view is discarded. The three–dimension philology\index{philology} is again a single dimension of Pollockian \textit{interpretation}.

%~ \newpage

\vspace{-.3cm}

\section*{A1.13 The Major Work (2006)}

(2006a)\textit{ The Language of the Gods in the World of Men}

This creative work is about two great moments of transformation in culture and power. The first occurs at the beginning of the first millennium when \textit{kāvya }(which had to be written) was first used for political expression. For the first moment to occur, the\textit{ ādikāvya }had to be written and not oral\index{oral tradition}. This, as we saw previously [4.6.6, 4.6.8], was inferred from Western sources. The second moment occurs at the beginning of the second millennium where vernacular languages challenge Sanskrit. Again, this inference was established from non–traditional sources as discussed in the previous papers on vernacularization\index{vernacularization}. Thus, these two great moments of transformation never occurred in Indian history and can only be attributed to Pollock’s\index{Pollock, Sheldon} political imagination. After hypothesizing the two great moments by \textit{a priori} application of Western theory, Pollock claims that there were parallel developments in Europe!

\begin{myquote}
Astonishingly \textbf{close parallels} to these processes, both chronologically and structurally, can be perceived in western Europe, with the rise of a new Latin literature and a universalist Roman Empire, and with the eventual displacement of both by regionalized forms. (2016a: 1)
\end{myquote}

European chronology and structure are initially superimposed on Indian history and then it is claimed that Indian and European chronologies and structures are similar! Comparing two objects of the same nature would obviously show similarity or parallels. Pollock suggests an alternative title to this book in the beginning: “A Study of Big Structures, Large Processes, and Huge Comparisons.” A more apt title would be to use singulars in place of plurals: “A Study of Big Structure, Large Process, and Huge Comparison”, as the superimposed European theory is being compared to itself making the plural redundant. Comparing an object with itself is called \textit{ananvayālaṅkāra}\index{Alankara@Alaṅkāra} in poetics: \textit{gaganaṁ gaganākāraṁ sāgaraḥ sāgaropamaḥ |rāma–rāvaṇayor yuddhaṁ rāma–rāvaṇayor iva || }(\textit{Rāmāyaṇa}\index{Ramayana@\textit{Rāmāyaṇa}} 6.110.23–24).

“The form (\textit{ākāra}) of the sky is similar only to the sky and the ocean can be compared only to the ocean. The battle between Rāma\index{Rama@Rāma} and Rāvaṇa\index{Ravana@Rāvaṇa} can only be compared to itself.” In other words, these objects cannot be compared to anything else (\textit{anupama}). Pollock derives his theories from his own (Western) tradition, applies it to Indian traditions and then compares it again with Western tradition. Essentially, this is a comparison of Western concepts of power and culture with itself.

The “invention” of written \textit{kāvya} is theorized to be different from the oral\index{oral tradition} Vedic tradition. \textit{Kāvya }is used to portray a new political consciousness which was not present before and this is called the “aestheticization of actually existing political power”.

\begin{myquote}
…we must be careful to \textbf{not make }\textit{\textbf{kāvya }}\textbf{a continuation of the Veda} by this–worldly means and must avoid incautious generalization about its “Vedic effect,” to which much\textit{ kāvya }anyway shows complete indifference. (2016a: 76–77)
\end{myquote}

\newpage

However, this interpretation contradicts an earlier one:

\begin{myquote}
We have already noted that the poet represents himself as developing a new formal vehicle for his work. Nothing of that sort really occurred, of course; the \textbf{elements of the metric revert to the vedic times} (and were in great measure an Indo–European heritage). (1986a: 43)
\end{myquote}

The two passages above represent the dramatic shift that occurs in Pollock’s\index{Pollock, Sheldon} intellectual career as a result of Edwards Said’s\index{Said, Edward} criticism. In the earlier work, \textit{kāvya }is considered a continuation of the Vedic tradition. But in the later works, it is theorized to be written as it is necessary for substantiating the first great moment that is supposed to have occurred in the beginning of the first millennium. \textit{Orientalism} actually had a detrimental effect on Pollock and the result has been documented in this \textit{prabandha}. The non–existence of the three–dimensional philology\index{philology} should be clear in this work also.

\vspace{-.5cm}

\section*{A1.14 {\it{\textbf{Upamāna}}}\index{upamana@\textit{upamāna}} (2007–2010)}

\vspace{-.3cm}

(2007e) “We Need to Find What We are Not Looking For”

(2010a) “Comparison without Hegemony”

The paper titled “We Need to Find What We are Not Looking For” discusses the issue of theoretical openness and non–application of \textit{a priori} reasoning suggesting that if we start looking for European categories such as “new sense of individual, a new scepticism, a new historical sensibility” in non–Europe, we would definitely find them since one usually finds what one is looking for. At the same time, we should not worry if European categories are not found in non–Europe. However, an analysis of several papers has shown that Pollock has a special ability to find what he is looking for (the connection between culture and power) by an application of European models.

Another paper (2010a) discusses the various possibilities of comparison in literary history. Accepting that a three–dimensional philology\index{philology} is possible, at least one should be able to faithfully present the view of the tradition along the second plane. But Pollockian philology constantly attempts to enter the Indian mind:\textit{ How did people experience these transformations in the realm of thought?} This necessarily means going beyond Sanskrit texts. If what Indians thought is not textualized, then what \textit{pramāṇa}–s\index{pramana@\textit{pramāṇa}} are used to know them? Such ideas must then have a non–traditional source and in the case of intellectual history, it is admitted that comparison (\textit{upamāna}\index{upamana@\textit{upamāna}}) is a necessity.

\begin{myquote}
Isn’t comparison something of a \textbf{cognitive inevitability}, making all intellectual history of necessity comparative intellectual history and all literary study comparative literary study?...But there are very suggestive hints in major European thought that \textbf{comparison} is fundamental to how we perceive the world.... (2010a: 196)
\end{myquote}

By now, Pollock\index{Pollock, Sheldon} has been working on Indian intellectual history for more than three decades and this is written at the time of formulating the theory of the three–dimensional philology. Since his views are from a European standpoint, Western paradigms cannot be expunged and thus it is unambiguous that on evaluating India, comparison cannot be avoided. It is further stated that to “compare or not is also not a choice” because of the cognitive necessity of comparison.

If we then apply this “cognitive necessity” to the three–dimensional philology, the meaning of a text (or textualized thought) along the three planes would necessarily have to be comparative and thus both the first and the second planes would only be \textit{a priori} application of Western theories. Thus, the attempt to formulate a three–dimensional philology would be futile and the “cognitive necessity” of comparison would result in single dimension personalist view of Indian tradition.

\vspace{-.3cm}

\section*{A1.15 {\it{\textbf{Rasa Śāstra}}}\index{sastra@\textit{śāstra}}\index{rasa@\textit{rasa}} (2009–2016)}

(2009a)\textit{ Rasamañjarī\index{Rasamanjari@\textit{Rasamañjarī}} and Rasataraṅgiṇī\index{Rasatarangini@\textit{Rasataraṅgiṇī}} of Bhānudatta}\index{Bhanudatta@Bhānudatta}

(2010b) “What was Bhaṭṭa Nāyaka\index{Bhattanayaka@Bhaṭṭa Nāyaka} Saying?”

(2012a) “From \textit{Rasa} Seen to \textit{Rasa} Heard”

(2012c) “\textit{Vyakti }and the History of\textit{ Rasa}”

(2016a) \textit{A Rasa\index{rasa@\textit{rasa}} Reader: Classical Indian Aesthetics}

These works on \textit{Rasa siddhānta} are a continuation of the work started in 1998 where it was stated that \textit{Rasa} was also used as a tool by kings to control the subjects. In Pollockian philology\index{philology}, transformation of some sort is necessary and essential to document change and conflict. Thus, the transformation from oral\index{oral tradition} to writing, from descriptive to prescriptive, from Sanskrit to Vernaculars and here a hermeneutical transformation that occurred in Bhaṭṭa Nāyaka’s\index{Bhattanayaka@Bhaṭṭa Nāyaka} thought from\textit{ Rasa} seen to \textit{Rasa} heard.

\newpage

\section*{A1.16 Philological Works (2008–2019)}

The works on philology represent the fifth and the final phase of this intellectual history. In these papers, a theoretical basis for philology supposedly based on practice is attempted. The primary reason for putting forth a theory is that philology, as rightly predicted by Auerbach\index{Auerbach, Erich}, is a dying field as it is essentially based on \textit{interpretation} which is always undefined due to its personal nature. A theoretical basis is meant to secure philology a permanent place in the academia along with other subjects such as philosophy and mathematics.

(2008a) “Towards a Political Philology: D. D. Kosambi\index{Kosambi, D. D.} and Sanskrit”

The definition of philology as making sense of texts is first seen in this paper. Here, philology includes political aspects also and is thus called political philology. The term philology is applied to Indian tradition thus enabling an always political interpretation.

(2009b) “Future Philology\index{Future Philology}: The Fate of a Soft Science in a Hard World”

(2014b) “Critical Philology”

(2014c) “Philology in Three Dimensions”

These three papers attempt to give a theoretical basis for philology. “Critical Philology” and “Philology in Three Dimensions” are identical with no new material. Both give two examples, one on the \textit{Rāmāyaṇa}\index{Ramayana@\textit{Rāmāyaṇa}} and the other on the \textit{Śākuntala}\index{Abhijnana-sakuntala@\textit{Abhijñāna-śākuntala}}, illustrating the approach to understanding Sanskrit texts. These have been discussed earlier. “Future Philology” tries to vaguely define the three dimensions or planes needed in understanding texts.

(2015a) “Philologia Rediviva”

(2015h) “Introduction in World Philology”\index{philology}

(2015g) “What was Philology in Sanskrit?”

(2015i) “Liberating Philology”

(2016b) “Philology and Freedom”


All the papers on philology would amount to about two–hundred pages, but is intended for university administrators and thus contain very little theorizing. Philology is defined as “making \textit{sense} of texts” and \textit{sense} here indicates the thoughts, i.e., looking into the minds of the “actors” (Indian tradition) to try to perceive the world they saw. The term philology itself becomes political philology, then critical philology, then liberating philology and is equated to freedom. The correct term would be \textit{sense} philology to reflect the definition.

To conclude, beginning with the 1985 paper on \textit{śāstra}\index{sastra@\textit{śāstra}}, Pollock’s\index{Pollock, Sheldon} effort has been to show how Indian tradition viewed itself. One would have expected \textit{pramāṇa–}s\index{pramana@\textit{pramāṇa}} to be shown from Sanskrit texts to substantiate thesis statements. However, in all the papers since 1985 covering a period of three decades, no \textit{pramāṇa}–s are shown to support his theories. Thus, in the final assessment, they remain only theories and the three–dimensional philology in reality is a one–dimensional imagination of Pollock.

