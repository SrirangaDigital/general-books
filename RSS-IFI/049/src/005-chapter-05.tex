
\chapter{Conclusion}\label{chapter5}

In the works on philology\index{philology}, Pollock\index{Pollock, Sheldon} attempts to formulate a theory of comprehending classical texts which includes the Sanskrit tradition. This had to be done to save philology from its imminent death as it did not produce any new knowledge. It was based on the undefined and untheorizable \textit{interpretation} which essentially was \textit{a priori} application of outdated Western theories and only meant that earlier speculations were being recycled and repackaged as new ones. As this \textit{prabandha} shows, philology could never become a “unified transregional and transhistorical discipline” as \textit{interpretation} is essentially personal and unique, and thus opposed to any generalization (theory). Some thoughtful University administrators seemed to have realized this and had correctly desired the end of this academic field. Pollock laments about this state of affairs:

\begin{myquote}
The core problem of philology today, as I see it, is whether it will survive at all; and it is philology’s survival that I care about and how this might be secured. (2009b: 931) I have long felt that, properly understood, philology in this large sense—\textbf{the application of interpretation }to the understanding of texts based on their original language—would naturally appear to \textbf{academic administrators and “decision–makers”} to be an intellectual activity as central to education as philosophy or mathematics. (2016a: 14)
\end{myquote}

For philology to survive and have a place on par with \textit{śāstra}\index{sastra@\textsl{śāstra}} or science, it should have certain rules based on universal principles that others could also apply in understanding texts. These rules should be based on \textit{pramāṇa}–s\index{pramana@\textsl{pramāṇa}} and be verifiable and open to \textit{parīkṣā}\index{pariksa@\textsl{parīkṣā}}. In the works of Pollock, who claims to have been living “life philologically” for a period of four decades, no such rules (and thus no theories) are to be found. The three–dimensional philology has been “consciously constructed” (\textit{kalpita}) only to impress administrators and not to define a system of knowledge. Only a single dimension is seen in the all the works and no \textit{pramāṇa}–s\index{pramana@\textsl{pramāṇa}} are shown from Sanskrit texts to substantiate thesis statements. The claim that “three–dimensional philology\index{philology} is actually the way human understanding works” (Pollock\index{Pollock, Sheldon} 2014c: 411) only shows Pollock’s misunderstanding of his own mind.

His intellectual development is closely connected to Edward Said’s\index{Said, Edward} devastating critique in \textit{Orientalism} and they both represent two opposites facets of philology. Said declared “that understanding literature and political commitment were things he did separately from each other” (2009b: 960). In stark contrast, Pollock declares that “a political project of one kind or another…has always informed and cannot but inform philology” (2009b: 946). What then is the point of philology if we already know that literature informs us of the political?

Said’s criticism drives Pollock to start a project of finding Western power structures deeply embedded in Sanskrit texts. We can only hope that the academic administrators and decision makers will now be more than convinced that philology will never produce any sort of new knowledge and this form of humanities in the end is dehumanizing.

A \textit{parīkṣā}\index{pariksa@\textsl{parīkṣā}} of his works has shown that Pollock has never established by valid \textit{pramāṇa}–s as to how Indians viewed themselves, but only how he himself viewed the Indian tradition from a Western perspective. Since the birth of philology, the exact nature of \textit{interpretation} has never been defined nor explained and the same is true in the works of Pollock. Thus, a \textit{pramāṇa-parīkṣā} of the three–dimensional philology has shown it to be one–dimensional based on deductive principles of \textit{a priori} application of outdated Western models. This has been the primary reason for the decline and imminent death of philology. The philological works (which are supposedly autobiographical) are only meant to mislead the public and are diametrically opposed to Pollock’s own practice.

\begin{myquote}
a readiness to expand our notion of scholarly truth and combine it with social hope...\textbf{My own autobiography}, for what it’s worth, tells me that…neither truth nor hope, can go it alone any longer. (2016d: 928)
\end{myquote}

\newpage

Tracing this philologist’s intellectual career spanning a period of forty years, my own biography tells me that it is hopeless to expect philology\index{philology} to change its inherent nature of \textit{interpretation,} and thereby reinvent itself. The only readiness is in the application of Western theories and the “scholarly truth” is far from both scholarship and truth. In conclusion,

\begin{myquote}
What this all demonstrates may be too obvious to mention: that [philology] as an [academic subject] in contemporary [America] is completely [dehumanized]. Its cultivation constitutes largely an exercise in nostalgia for those directly involved, and, for outsiders, a source of bemusement that such [academic work] takes place at all. Government feeding tubes and oxygen tanks may try to preserve [philology] in a state of quasi–animation, but most observers would agree that, in some crucial way, [philology] is dead. (adapted from 2001b: 393)
\end{myquote}

Only then will we have truly liberated philology. And “when we learn to free philology” which we have by liberating it, “we will at the same time be learning one more way to free ourselves” (2016b: 27).

