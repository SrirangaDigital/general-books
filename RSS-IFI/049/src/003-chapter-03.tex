
\chapter{\textit{Lakṣaṇa}\\ – An Enquiry into the \textit{Pramāṇa}–s}\label{chapter3}\index{laksana@\textit{lakṣaṇa}}

\section*{3.1 The Validity of \textit{Pramāṇa}–s}

\textit{Pramāṇa}–s are the essential means of cognizing and comprehending \textit{artha}–s\index{artha@\textit{artha}} (objects) and understanding the world around us. Vātsyāyana\index{Vatsyayana@Vātsyāyana}	begins his \textit{Nyāya–bhāṣya}\index{Nyaya-bhasya@\textit{Nyāya-bhāṣya}} with a profound observation on the interdependence of \textit{pramāṇa}–s\index{pramana@\textit{pramāṇa}} and \textit{artha}–s. Since this is essential for our understanding of \textit{pramāṇa}–s, the \textit{bhūmikā} (introductory portion) of the \textit{Nyāya–bhāṣya} is discussed at relevant places in this \textit{prabandha}. The first part of the \textit{bhūmikā} is translated and discussed here.

\textit{\textbf{Nyāya–bhāṣya}}[p.1]

\begin{myquote}
\textit{pramāṇato'rtha–pratipattau pravṛtti\index{pravrtti@\textit{pravṛtti}}–sāmarthyād\index{samarthya@\textit{sāmarthya}} arthavat pramāṇam |\break pramāṇam antareṇa nārtha–pratipattiḥ | nārtha–pratipattim antareṇa pravṛtti–sāmarthyam\index{samarthya@\textit{sāmarthya}} | pramāṇena khalv ayaṁ jñātā'rtham upalabhya tam īpsati jihāsati vā | tasya īpsā–jihāsā–prayuktasya samīhā pravṛttir ity ucyate | sāmarthyaṁ punar asyāḥ phalenā’bhisambandhaḥ\index{sambandha@\textit{sambandha}} | samīhamānas tam artham abhīpsan\break jihāsan vā tam artham āpnoti jahāti vā | }
\end{myquote}

\textit{\textbf{Anuvāda}}

By the \textit{pramāṇa}–s, the \textit{artha} (object) is comprehended and this results in \textit{pravṛtti–sāmarthya} (successful activity). Therefore, \textit{pramāṇa} is \textit{arthavat} or always connected with \textit{artha}. Without \textit{pramāṇa}–s, \textit{artha} cannot be comprehended. And without the comprehension of \textit{artha}, successful activity does not arise. By the \textit{pramāṇa}–s, the \textit{jñātṛ}\index{jnatr@\textit{jñātṛ}} (knower) comprehends \textit{artha} and either desires or avoids it. The \textit{samīhā} (effort) of the \textit{jñātṛ}\index{samarthya@\textit{sāmarthya}} in desiring or avoiding \textit{artha}\index{artha@\textit{artha}} is called \textit{pravṛtti}\index{pravrtti@\textit{pravṛtti}} (activity). \textit{Sāmarthya}\index{samarthya@\textit{sāmarthya}} is the capacity of the \textit{jñātṛ} to achieve \textit{phala} (result). By making an effort in desiring or avoiding an object, the \textit{jñātṛ} either obtains it or avoids it.

\textbf{\textit{Parīkṣā}\index{pariksa@\textit{parīkṣā}} of 3D Philology\index{philology}}

This \textit{prabandha} examines the three–dimension philology of the \textit{pūrvapakṣin} (Pollock)\index{Pollock, Sheldon} using the conceptual framework of Nyāya–śāstra\index{sastra@\textit{śāstra}}, and the \textit{pramāṇa}–s therein in particular. \textit{Nyāya–bhāṣya}\index{Nyaya-bhasya@\textit{Nyāya-bhāṣya}} begins with an observation on the acquisition of knowledge and this is equally applicable to Pollock. \textit{Pramāṇa–}s\index{pramana@\textit{pramāṇa}} have to be accepted at the outset, as denying it would essentially deny valid knowledge. The \textit{pramāṇa–}s\break used in his works are discussed in a later section [3.5]. The \textit{jñātṛ} in our case is Pollock, whose primary aim over his academic career has been to “exhume” deeply embedded cultural practices in Sanskrit texts. Since all objects are known only through \textit{pramāṇa–}s, the \textit{jñātṛ} will also use \textit{pramāṇa–}s to establish his theories and thus a \textit{parīkṣā} of the \textit{pramāṇa}–s used will be the primary aim of this \textit{prabandha}. The \textit{bhūmikā} of \textit{Nyāya–bhāṣya} describes a fundamental activity common to all living beings and thus cannot be denied or objected to by Pollock.

Since Pollock’s correct use of \textit{pramāṇa}–s is being evaluated, an objection could be raised regarding the validity of the \textit{pramāṇa}–s. Just by stating that \textit{pramāṇa}–s are used by all living beings does not establish it. Moreover, since the \textit{pramāṇa}–s have not been agreed upon by the \textit{pūrvapakṣin}, the basis of evaluating Pollock’s philology using the \textit{pramāṇa}–s could also be questioned. \textit{Without establishing and agreeing upon the pramāṇa–s, how is it possible to conduct a vakyārtha or discussion with anybody (Pollock included)?}

The beginning portion of Ācārya Vātsyāyana’s\index{Vatsyayana@Vātsyāyana} \textit{bhūmikā} discussed above and which forms an introduction to the first \textit{sūtra,} provides an answer to this question. \textit{Pramāṇa}–s are the valid means of comprehending \textit{artha}–s, and once the \textit{artha}\index{artha@\textit{artha}} is comprehended, the \textit{jñātṛ} acts accordingly. Without \textit{pramāṇa–}s, there will be no knowledge of the world around and thus no activity will be possible, and so \textit{pramāṇa–}s\break are necessarily connected with \textit{artha}. We can observe successful activity in the world around us. This is possible only if \textit{artha} is comprehended by the knower and not otherwise. This can be done only through \textit{pramāṇa–}s. Thus, \textit{pramāṇa–}s are said to be \textit{loka–siddha}\index{loka-siddha@\textit{loka-siddha}} or that which is established in the world itself.

\vspace{-.3cm}

\section*{3.2 \textit{Pramāṇa–s} are \textit{Loka–siddha}}

\vspace{-.2cm}

Ācārya Vātsyāyana\index{Vatsyayana@Vātsyāyana} begins the \textit{Nyāya–bhāṣya}\index{Nyaya-bhasya@\textit{Nyāya-bhāṣya}} by describing a fundamental human experience not limited by \textit{kāla }(time) and \textit{deśa }(place) implying its universality. Vātsyāyana implies that the relation between \textit{pramāṇa} and \textit{artha}\index{artha@\textit{artha}} is established. Where is this established? In the \textit{loka }itself. We can directly perceive all the four \textit{pramāṇa–}s, namely \textit{pratyakṣa}\index{pratyaksa@\textit{pratyakṣa}}, \textit{anumāna}\index{anumana@\textit{anumāna}}, \textit{upamāna}\index{upamana@\textit{upamāna}} and \textit{śabda}\index{sabda@\textit{śabda}}, being used in the world around us. No child is taught about the \textit{pramāṇa–}s but uses them instinctively.

Similarly, in discussions and debates participants use \textit{pramāṇa–}s without establishing them first. Further, if we must establish and agree upon the \textit{pramāṇa–}s\index{pramana@\textit{pramāṇa}} before a \textit{vakyārtha}, no discussions would be possible since no two \textit{darśana}–s agree on the \textit{pramāṇa–}s. In fact, an important discussion in Indian tradition is on the number of \textit{pramāṇa–}s.\break Even if the \textit{pramāṇa–}s are agreed upon, the definition is different even within a given tradition.\endnote{Vedāntadeśika’s \textit{Nyāya-pariśuddhi} (p.131–136) discusses several definitions within the Viśiṣṭādvaita tradition itself. Sri N.S. Ramanuja Tatacharya’s lucid comments trace the historical reasons for these differences.}

Further,\textit{ Nyāya–sūtra}\index{Nyaya-sutra@\textit{Nyāya-sūtra}} classifies \textit{kathā} (discussion) into three types: \textit{vāda}\index{vada@\textit{vāda}}, \textit{jalpa}\index{jalpa@\textit{jalpa}} and \textit{vitaṇḍā}\index{vitanda@\textit{vitaṇḍā}} and any form of discussion will be included under these three. But nothing is said either in the \textit{sūtra} or the \textit{bhāṣya }about the need for establishing the \textit{pramāṇa–}s before starting a discussion. A large portion of Sanskrit literature is in the form of a dialogue which can be included as, \textit{vāda }(a discussion between \textit{guru} and \textit{śiṣya} for establishing truth) and we do not see \textit{pramāṇa–}s being established in these texts before starting the dialogue. Such discussions are to be found in the \textit{Veda}–s\index{Veda-s}, the \textit{Upaniṣad}–s, the \textit{Āgama}–s and Pali literature.

Śrī Śaṅkara\index{Sankaracarya@Śaṅkarācārya} and Śrī Rāmānuja\index{Ramanujacarya@Rāmānujācārya}, too, have not established any \textit{pramāṇa–}s\break in their works. Still, many important topics in \textit{Vedānta} have been discussed and thoroughly examined. This is because \textit{pramāṇa–}s are \textit{loka–siddha}\index{loka-siddha@\textit{loka-siddha}} and thus there is no need to establish them. The relative strength of the \textit{pramāṇa–}s (such \textit{pratyakṣa} and \textit{śabda}) are discussed in the context of enquiring into the meaning of the \textit{Upaniṣad}–\textit{vākya}–s. The followers of both these traditions such as Dharmarājādhvarīndra\index{Dharmarajadhvarindra@Dharmarājādhvarīndra} and Vedāntadeśika\index{Vedantadesika@Vedāntadeśika} have defined the \textit{pramāṇa–}s based on the writings of their respective Ācārya–s much later but have not “established” them.

We can also observe this in polemical texts that criticize each other directly. Vyāsatīrtha\index{Vyasatirtha@Vyāsatīrtha} was a great logician belonging to the Mādhva tradition and wrote several works of which three are well known: \textit{Nyāyāmṛta}\index{Nyayamrta@\textit{Nyāyāmṛta}}, \textit{Tarka–tāṇḍava}\index{Tarka-tandava@\textit{Tarka-tāṇḍava}} and \textit{Tātparya–candrikā}\index{Tatparya-candrika@\textit{Tātparya-candrikā}} (collectively called \textit{Vyāsatraya}\index{Vyasatraya@\textit{Vyāsatraya}}). These three texts criticize other schools of thought. \textit{Nyāyāmṛta}, a brilliant analysis of the Advaita\index{Advaita@\textit{Advaita}} school was answered by Madhusūdana\index{Madhusudana-sarasvati@Madhusūdana-sarasvatī}–sarasvatī in his equally brilliant \textit{Advaita–siddhi}\index{Advaita-siddhi@\textit{Advaita-siddhi}}. Several scholars representing Dvaita\index{Dvaita} and Advaita continued writing commentaries and sub–commentaries on \textit{Nyāyāmṛta} and \textit{Advaita–siddhi} and this debate continues even today in traditional \textit{vakyārtha}–s.\break Neither \textit{Nyāyāmṛta} nor \textit{Advaita–siddhi} establish the \textit{pramāṇa–}s but begin by examining each other’s position on \textit{mithyā} or the falsity of the world. Dvaita school establishes the reality of the world, while Advaita denies the ultimate reality of the world. The former accepts only three \textit{pramāṇa–}s\index{pramana@\textit{pramāṇa}} while the later six \textit{pramāṇa–}s\endnote{ This is according to \textit{Vedanta-paribhāṣā}. Other works within the Advaita tradition itself could have a different number of \textit{pramāṇa}–s.} and the definition of \textit{pramāṇa–}s is also different. Many important philosophical issues were thoroughly discussed (and continue to be) in this debate, but all this was done without establishing the \textit{pramāṇa–}s. Now, if \textit{pramāṇa–}s had to be established in the beginning, this \textit{vakyārtha} would not have happened.

An objection could be raised here. In the above examples, only written texts were mentioned, while Indian traditions were and still are primarily oral\index{oral tradition}. Thus, when conducting a debate, \textit{pramāṇa–}s must have been established first and agreed upon, and only then the discussion would have been conducted.

It is true that the oral traditions will always be a central part of Indian knowledge systems. We can still observe \textit{vākyārtha}–s being conducted in traditional scholarly gatherings in places like Kashi, Kanchi and Sringeri. In Sringeri, the \textit{Gaṇapati Vakyārtha Sabhā} is conducted every year, where some of the best scholars in Vyākaraṇa\index{Vyakarana@Vyākaraṇa}, Nyāya, Mīmāṁsā\index{Mimamsa@Mīmāṁsā} and Vedānta participate. One can observe that scholars discuss their respective subjects in a \textit{pūrvapakṣa\index{purva-paksa@\textit{pūrva-pakṣa}}–siddhānta\index{siddhanta@\textit{siddhānta}}} format and come to a reasoned conclusion. There is no effort to establish any \textit{pramāṇa}–s before starting the \textit{vākyārtha}.

Moreover, there are many texts that examines the topic of \textit{kathā} (discussion). Some of them are Madhvācārya’s\index{Madhvacarya@Madhvācārya} \textit{Pramāṇa–lakṣaṇa}\index{laksana@\textit{lakṣaṇa}}\index{Pramana-laksana@\textit{Pramāṇa-lakṣaṇa}}, Jayatīrtha’s\index{Jayatirtha@Jayatīrtha} \textit{Pramāṇa–paddhati}\index{Pramana-paddhati@\textit{Pramāṇa-paddhati}}, Vedāntadeśika’s\index{Vedantadesika@Vedāntadeśika} \textit{Nyāya–pariśuddhi}\index{\textit{Nyāya-pariśuddhi}} and Hemacandrācārya’s\index{Hemacandracarya@Hemacandrācārya} \textit{Pramāṇa–mimāṁsā}\index{Pramana-mimamsa@\textit{Pramāṇa-mimāṁsā}}. In all these texts, \textit{pramāṇa–}s are defined and discussed in great detail, but there is no mention of establishing them. Discussions and debates take place even within a specific śāstric tradition itself such as Vyākaraṇa\index{Vyakarana@Vyākaraṇa}, Āyurveda\endnote{ \textit{Caraka–saṁhitā} also has a section devoted to discussion and debate.} and Jyotiṣa\index{jyotisa@\textit{jyotiṣa}}, and here also there is no attempt to establish \textit{pramāṇa–}s. All this clearly shows that \textit{pramāṇa–}s were never “established” and were always considered \textit{loka–siddha}\index{loka-siddha@\textit{loka-siddha}}.

In conclusion, simply by observing the world around us, we can perceive that discussions and debates take place without establishing the \textit{pramāṇa–}s since they are \textit{loka–siddha}.

It could be said that even if \textit{pramāṇa–}s\index{pramana@\textit{pramāṇa}} are \textit{loka–siddha}, they at least need to be agreed upon before conducting a debate. Nyāya–śāstra\index{sastra@\textit{śāstra}} comprises of Hindu, Buddhist\index{Buddhism/Buddhist} and Jaina\index{Jaina} traditions. Discussions and debates have been and still are an integral part of these traditions. For example, Dharmakīrti\index{Dharmakirti@Dharmakīrti} in his \textit{Nyāya–vinaya}\index{Nyaya-vinaya@\textit{Nyāya-vinaya}} criticizes some aspects of Nyāya–śāstra, but this criticism would not be possible if both these traditions had to agree upon the \textit{pramāṇa–}s. Nyāya–śāstra has four \textit{pramāṇa–}s (even within this tradition, some accept only three), namely \textit{pratyakṣa}\index{pratyaksa@\textit{pratyakṣa}}, \textit{anumāna}\index{anumana@\textit{anumāna}}, \textit{upamāna}\index{upamana@\textit{upamāna}} and \textit{śabda}\index{sabda@\textit{śabda}} while Dharmakīrti discusses only \textit{pratyakṣa} and \textit{anumāna}. Thus, there is no need to agree upon the number of \textit{pramāṇa–}s with the \textit{pūrvapakṣin} to conduct a discussion.

\vspace{-.3cm}

\section*{3.3 {\it {\bfseries Sarvatantra-Siddhānta}}}

\vspace{-.2cm}

The preceding section answers the question that was asked initially: \textit{Without establishing and agreeing upon the pramāṇa–s, how is it possible to conduct a vakyārtha or discussion with Pollock?\index{Pollock, Sheldon}}Therefore, this discussion with the \textit{pūrvapakṣin} will initially evaluate if the \textit{pramāṇa–}s are being used correctly and if so, discuss the issues (\textit{prameya})\index{prameya@\textit{prameya}} that are being speculated. Examining the \textit{pramāṇa–}s for validity would have to be agreed by the \textit{pūrvapakṣin}, the proponent of the three–dimensional philology\index{philology}.

Only examples from Indian tradition were given, but this is equally true for all discussions that take place. \textit{Pramāṇa}–s are universal and essential to any human activity and have to be accepted by all. When any topic or position is accepted by all schools of thought, it is called \textit{sarvatantra}–\textit{siddhānta}\index{siddhanta@\textit{siddhānta}} in Nyāya–śāstra and is defined as follows:

\newpage

\textit{\textbf{Nyāya–sūtra}}\index{Nyaya-sutra@\textit{Nyāya-sūtra}}[1.1.28]

\begin{verse}
\textit{sarvatantrāviruddhaḥ tantre'dhikṛto'rthaḥ sarvatantra–\\ siddhāntaḥ |}
\end{verse}

\newpage

An object that is \textit{aviruddha} (accepted) by other \textit{tantra}–s (schools of thought) and admitted in one's own school of thought is called \textit{sarvatantra-siddhānta} (position accepted by all schools of thought).

\textit{\textbf{Nyāya–bhāṣya}}\index{Nyaya-bhasya@\textit{Nyāya-bhāṣya}}[p.28]

\begin{verse}
\textit{yathā ghrāṇādīnīndriyāṇi gandhādayaḥ indriyārthāḥ pṛthivyādīni\\ bhūtāni pramāṇair arthasya grahaṇam iti |}
\end{verse}

\textit{\textbf{Anuvāda}}

For example, the five senses beginning with \textit{ghrāṇa }(sense of smell), the objects of the senses beginning with \textit{gandha }(smell), the five elements beginning with \textit{pṛthivī} (earth), and acquiring of valid knowledge of objects through the \textit{pramāṇa–}s\index{pramana@\textit{pramāṇa}}. (These are accepted by all the schools of thought.)

\textbf{\textit{Parīkṣā}\index{pariksa@\textit{parīkṣā}} of 3D Philology\index{philology}}

The \textit{Nyāya–bhāṣya} states that obtaining valid knowledge through \textit{pramāṇa}–s is acceptable to all schools of thought and this includes the proponent of the three–dimensional philology. That Pollock\index{Pollock, Sheldon} uses \textit{pramāṇa}–s can be known by a careful reading of his works and will be discussed in the section on the four \textit{pramāṇa}–s.

It could be further asked that if \textit{pramāṇa}–s are validated by \textit{loka} (world experience) itself, then what is the need and purpose of a \textit{śāstra}\index{sastra@\textit{śāstra}} that enquires into the \textit{pramāṇa}–s. If people in the world are having discussions and debates and coming to reasoned conclusions by using the \textit{pramāṇa}–s, why even have a \textit{śāstra}?

The subtler aspects of the first \textit{vākya} of \textit{Nyāya–bhāṣya }– \textit{pramāṇataḥ artha–pratipattau} – can be grasped by an illuminating analogy from Bhagavān Patañjali’s\index{Patanjali@Patañjali} \textit{Mahābhāṣya}\index{Mahabhasya@\textit{Mahābhāṣya}}. While commenting on Vararuci's\index{Vararuci} first \textit{vārtika}, Patañjali discusses in detail the \textit{loka–siddha}\index{loka-siddha@\textit{loka-siddha}} aspect and its importance in Vyākaraṇa\index{Vyakarana@Vyākaraṇa}. The essence of discussion is that since the relation between the \textit{śabda}\index{sabda@\textit{śabda}} (word) and \textit{artha}\index{artha@\textit{artha}} (meaning) is \textit{loka–siddha}, i.e., established in the world itself, the purpose of Vyākaraṇa is for \textit{dharma–niyama}, to distinguish between correct usage and incorrect usage. Similarly, even though the \textit{pramāṇa–}s are \textit{loka–siddha} in Nyāya–śāstra\index{sastra@\textit{śāstra}}, the purpose of this \textit{śāstra} is for \textit{dharma–niyama}, that is to distinguish between correct and incorrect use of \textit{pramāṇa–}s. Thus, our primary goal is to evaluate the works of Pollock\index{Pollock, Sheldon} to ascertain in order if the \textit{pramāṇa–}s are correctly being used to substantiate various theories.

The \textit{loka–siddha}\index{loka-siddha@\textit{loka-siddha}} aspect (of both Vyākaraṇa\index{Vyakarana@Vyākaraṇa}-śāstra and the \textit{pramāṇa–}s\index{pramana@\textit{pramāṇa}} of Nyāya–śāstra) is of paramount importance in relation to the theories of Pollock. In the 1985 paper on Indian intellectual traditions, it is claimed that \textit{śāstra}–s were descriptive in the beginning and later became prescriptive or normative (see section 4.6.1 for details). This transformation was used to infer that \textit{śāstra}–s such as Vyākaraṇa were used by kings to control the populace. Pollock extends this to \textit{kāvya}–s and the concept of \textit{rasa}\index{rasa@\textit{rasa}} also, both of which were “meant to control the subjects”. The \textit{loka–siddha }aspect means that Nyāya–śāstra and Vyākaraṇa-śāstra are by nature descriptive and could not have become prescriptive. Thus, a transition from descriptive to prescriptive nature could have never happened and this would suffice to show that \textit{śāstra}–s could not have been used for control and domination.


\section*{3.4 The Four \textit{Pramāṇa}–s in the Works of Pollock}

\textit{Nyāya–sūtra}\index{Nyaya-sutra@\textit{Nyāya-sūtra}} enumerates the four \textit{pramāṇa}–s and each one of them is defined and examined in subsequent \textit{sutra}–s. As mentioned before, there is no effort made to establish the \textit{pramāṇa}–s as such (no separate reasoning or proof is given) in either the \textit{sūtra} or the \textit{bhāṣya} or any of the later commentaries. In this section, the four \textit{pramāṇa}–s are stated, defined and illustrated with examples from Pollock’s works.

\textbf{\textit{Nyāya–sūtra} 1.1.3}

\vspace{-.3cm}

\begin{verse}
\textit{pratyakṣānumānopamāna–śabdāḥ\index{sabda@\textit{śabda}} pramāṇāni |}
\end{verse}

\vspace{-.3cm}

The four \textit{pramāṇa}–s are \textit{pratyakṣa}\index{pratyaksa@\textit{pratyakṣa}} (perception), \textit{anumāna}\index{anumana@\textit{anumāna}} (inference), \textit{upamāna}\index{upamana@\textit{upamāna}} (analogy) and \textit{śabda }(verbal authority).

%~ \newpage

\textit{\textbf{Vivaraṇa}}\index{vivarana@\textit{vivaraṇa}}

Vātsyāyana\index{Vatsyayana@Vātsyāyana} discusses the etymological meanings of the four \textit{pramā\-ṇa}–s in detail. In this \textit{sūtra}\index{pratyaksa@\textit{pratyakṣa}}, \textit{pratyakṣa} implies the \textit{pramāṇa} or the instrument of valid knowledge. It could also refer to the knowledge arising from \textit{pratyakṣa} or the objects of \textit{pratyakṣa} itself.~This is applicable to the three other \textit{pramāṇa}–s as well.

\textbf{\textit{Parīkṣā}\index{pariksa@\textit{parīkṣā}} of 3D Philology\index{philology}}

Pollock\index{Pollock, Sheldon} defines philology as making \textit{sense} of texts and a direct meaning would be to understand or comprehend texts through the senses. If senses were to mean \textit{indriya}–s, then the definition would imply understanding Sanskrit texts through the\textit{ pramāṇa}–s\index{pramana@\textit{pramāṇa}}.

\begin{myquote}
Texts, their history, their mode of material existence, their very textuality, and above all, \textbf{their content}, are the primary objects of study of philology…All flows from the study of language itself…(2016d: 916)
\end{myquote}

Thus, the content of Sanskrit texts should be included under \textit{śabda–pramāṇa}\index{sabda@\textit{śabda}} and used for substantiating theories. This meaning, however, is not intended by the definition. The word “sense”actually means “\textit{ati}–sense” or going beyond the senses or \textit{pramāṇa}–s. The three dimensions of Pollockian philology were briefly described earlier. The second dimension\index{Second Dimension/Traditional/Plane 2} is “the meaning of the text according to Indian tradition”. One would think that the textual content at least along the second dimension would be included as \textit{śabda–pramāṇa}, but Pollock states that

\begin{myquote}
…the historicism involved is of a sort that ancient and medieval traditions \textbf{never practiced or even conceptualized }in their own right, since this mode of thought is an invention of the early modern conceptual revolution. Yet it would be an act of extreme indigenism to forgo historicism because it did not conform with \textbf{traditional ways of knowing}…Seeing things their way has even greater implication for conceptual renovation…Such otherness cannot just be imagined; it must be\textbf{ laboriously exhumed} from the depths of the textual past. (2009b: 955)
\end{myquote}

The historical method (which amounts to application of Western models) that Pollock uses by his own admission was not practiced in Indian traditions. If the purpose of second dimension (plane 2 reading) is to portray tradition’s view–point, then only traditional ways of knowing should be adhered to. The whole purpose of the second dimension is defeated if the method used “does not conform with the traditional ways of knowing.” What then is the purpose of the second dimensional reading? If this type of historicism has never been conceptualized, then “seeing things their way” (plane 2)\index{Second Dimension/Traditional/Plane 2} is actually seeing things in one’s own way (plane 3)\index{Third Dimension/Presentist/Plane 3}. A \textit{parīkṣā}\index{pariksa@\textit{parīkṣā}} in the later sections would reveal that there is only one way of seeing in Pollockian philology.

When it is stated that tradition “never practiced or even conceptualized,” it means that there is nothing in Sanskrit texts (including inscriptions) that would support his philological speculations. It should also be carefully noted that “otherness” is to be “laboriously exhu\-med”\endnote{ The word “exhumed” is initially seen in the 1985 paper on \textit{śāstra} and more recently in \textit{Future Philology} (2009b) which lays the theoretical foundation of understanding texts. The word exhume originates from medieval Latin and means to dig up something buried (usually a dead body) or to reveal, disclose or unearth some thought. Said’s criticism had a reverse effect and drives Pollock to “exhume” Western theories deeply embedded in Sanskrit texts.} not from texts (as texts have nothing in them), but from textual past, i.e., a past that needs to be imagined (obviously from Western sources) and then linked to the text. For example, in Pollockian philology\index{philology}, Sanskrit grammar is not only an intellectual discipline, but is also linked to political power. In the innumerable grammatical texts, there would be nothing to support this inference. In the long tradition of grammarians, no one is discussing the political aspect of grammar. This is what is meant by stating that tradition “never practiced or even conceptualized” this sort of historicism. So, if tradition never documented the connection between grammar and the political, how does Pollock\index{Pollock, Sheldon} get this idea in the first place? The answer is discussed in the next chapter [section 4.6.8]. Once again, the impossibility of three–dimensional philology should be clear.

Some essential points are summarized here:

\vspace{-.3cm}

\begin{enumerate}
\itemsep=0pt
\item Philology is \textit{interpretation} which “actually precedes and informs all other aspects.”

 \item The \textit{interpretation} of grammar being connected to the political is “exhumed” as Sanskrit texts do not make any such connection.

 \item Thus, making \textit{sense} of texts is \textit{interpretation} superseding \textit{pramāṇa}–s\index{pramana@\textit{pramāṇa}}. This is in accordance with the nature of philology since its birth.

\end{enumerate}

\vspace{-.3cm}

Thus, Pollock’s effort to theorize philology amounts to nothing new. Giving philology a new prefix (three–dimensional) does not change its inherent nature of the predominance of \textit{interpretation}. Words can be changed and any number of new prefixes can be added (political, critical, liberating, etc.) to philology, but the meaning and process remains the same: \textit{interpretation} superseding the \textit{pramāṇa}–s.

The following section explains the \textit{pramāṇa}–s as they are used in the works of Pollock\index{Pollock, Sheldon}.

\newpage

\section*{3.5 Definition of {\it {\bfseries Pratyakṣa}}}\index{pratyaksa@\textit{pratyakṣa}}

\textbf{\textit{Nyāya–sūtra}\index{Nyaya-sutra@\textit{Nyāya-sūtra}} 1.1.4}

\vspace{-.3cm}

\begin{verse}
\textit{indriyārtha–sannikarṣotpannaṁ jñānam avyapadeśyam\\ avyabhicāri vyavasāyātmakaṁ pratyakṣam |}
\end{verse}

\vspace{-.3cm}

\textit{\textbf{Anuvāda }}(p.43)

The \textit{jñāna} (knowledge) arising from the contact of \textit{indriya}–s and \textit{artha}\index{artha@\textit{artha}} (object) and which is \textit{avyapadeśya }(not due to words), \textit{avyabhicārin }(not erroneous) and \textit{vyavasāyātmaka }(of a definite character) is \textit{pratyakṣa}.

\textbf{\textit{Vivaraṇa}}\index{vivarana@\textit{vivaraṇa}} (p.43–50)

The definition of \textit{pratyakṣa }is given first because all the \textit{pramāṇa}–s\index{pramana@\textit{pramāṇa}} are based on \textit{pratyakṣa }and thus it is called \textit{pramāṇa–jyeṣṭha} or the most important of the \textit{pramāṇa}–s. This shows that Nyāya–śāstra\index{sastra@\textit{śāstra}} is primarily based on the experience or observation of the world which in turn enabled it to be adopted according to the needs of the time and place. The innumerable texts in this tradition alone would attest to this. Just as the Sanskrit grammarians deal with the actual usages in the \textit{loka}, the \textit{Naiyāyika}–s enquiry about the validity of the \textit{pramāṇa}–s and thus it is also known as \textit{pramāṇa-śāstra}.

The commentaries discuss the three words in the definition, namely \textit{avyapadeśya}, \textit{avyabhicāri} and \textit{vyavasāyātmaka} in detail which is not elaborated here.

The definition of \textit{pratyakṣa} varies even within the \textit{Nyāya} tradition itself and one may give any number of definitions, but the underlying process of perceptual knowledge must be accepted. The \textit{pūrvapakṣin} may disagree with the definition but would have to agree with the existence of \textit{pratyakṣa}–\textit{pramāṇa} and the knowledge that arises from it. Nyāya–śāstra or \textit{pramāṇa-śāstra} forms the basis of all Indian śāstric traditions since\textit{ pramāṇa}–s are the basis for comprehending the world. \textit{Pratyakṣa}–\textit{pramāṇa }is primary and fundamental to all śāstric traditions.

%~ \newpage

\textbf{\textit{Parīkṣā}\index{pariksa@\textit{parīkṣā}} of 3D Philology}

\textit{Pratyakṣa} refers both to the \textit{pramāṇa} (instrument of valid knowledge) as well as knowledge that arises from perception, i.e. perceptual knowledge. The three planes of philology\index{philology} are historical, traditional and personal. Both the historical and the traditional meaning of the text cannot be known by \textit{pratyakṣa}–\textit{pramāṇa} as the author and the commentators are located in the past. Thus, \textit{pratyakṣa}\index{pratyaksa@\textit{pratyakṣa}}–\textit{pramāṇa}\index{pramana@\textit{pramāṇa}}is only relevant for the personalist (plane 3)\index{Third Dimension/Presentist/Plane 3} reading of the text. Pollock\index{Pollock, Sheldon} is sensitive to the events occurring in the world and some of them have a varying effect on his academic career. Any event directly perceived during his academic career spanning forty years can be included under \textit{pratyakṣa}–\textit{pramāṇa}.

\vspace{-.3cm}

\begin{enumerate}
\itemsep=0pt
\item Perceiving the devastation caused by Said’s\index{Said, Edward} \textit{Orientalism} which results in Pollock’s drastic change in interpreting Sanskrit texts (from 1985).

 \item Perceiving the demolition of Babri\index{Babri Masjid} Masjid and writing the paper, “\textit{Rāmāyaṇa}\index{Ramayana@\textit{Rāmāyaṇa}} and Political Imagination in India” in 1993.

 \item Perceiving the BJP government’s decision to celebrate Sanskrit year in 1999–2000 and penning “The Death of Sanskrit” in response to it in 2001.

 \item Perceiving the imminent death of philology (closure of departments, etc.), and the attempt to save it by writing papers and holding conferences from about 2009.

 \item Perceiving the Swadeshi Indology Conference and writing \textit{Areas, Disciplines, and the Goals of Inquiry} as a response in 2017.

 \item Refusing to perceive the living traditions of Sanskrit which are considered “dead” (in the historical sense) even though he claims to have studied with pundits.

\end{enumerate}

\vspace{-.3cm}

As these instances illustrate, \textit{pratyakṣa}–\textit{pramāṇa} is important in Pollock’s intellectual history. It is however necessary to ascertain the correctness of \textit{pratyakṣa}–\textit{pramāṇa}. For example, the \textit{Rāmāyaṇa} becomes primarily a political text after the Babri Masjid demolition.

\begin{myquote}
The ready availability to reactionary Indian politics of central cultural icons like the \textbf{Rāmāyaṇa text} has proved challenging to understand and explain. (1993b: 262)
\end{myquote}

The demolition of a mosque is perceived and then connected with the \textit{Rāmāyaṇa}\index{Ramayana@\textit{Rāmāyaṇa}} text. What actually needs to be perceived is the text of the \textit{Rāmāyaṇa }specifically being used for political mobilization. Is there some section or chapter or even a single \textit{śloka} in Vālmīki’s\index{Valmiki@Vālmīki} \textit{Rāmāyaṇa} that is being used to mobilize the populace? Various symbols are being used, but in \textit{Rāmāyaṇa and Political Imagination in India}, no evidence is shown as to how the text itself was used for mobilization. Even without perceiving the direct link between the \textit{Rāmāyaṇa }text and the demolition, Pollock\index{Pollock, Sheldon} assumes it to be so, and starts a project to document the inherent political nature of the \textit{Rāmāyaṇa}. This is an incorrect case of \textit{pratyakṣa}\index{pratyaksa@\textit{pratyakṣa}}–\textit{pramāṇa}\index{pramana@\textit{pramāṇa}}.

Another instance of \textit{pratyakṣa}–\textit{pramāṇa }relates to the death of Sanskrit thesis. In spite of perceiving the use of Sanskrit (written and spoken) in various ways and in spite of having studied with traditional scholars (directly perceiving the living traditions), Sanskrit is declared to be “dead.” Historicity is an important topic that repeatedly occurs in the works of Pollock and Sanskrit is considered dead in a historical sense. This implies that even though works are being written and people speak in Sanskrit, the impact on society and culture is non–existent and thus it is considered “historically” dead. Even if we accept this argument, a later perception of Pollock contradicts this one.

\begin{myquote}
Consider yoga and other forms of body discipline and meditation, which are now practiced, according to recent estimates, by more than twenty million people—nearly nine per cent of adults—in the US alone, or Ayurveda and the whole array of alternative medical practices from South Asia, which have increasingly been penetrating the Western medical establishment. (2014a: 6)
\end{myquote}

Surely a “historically dead language” cannot have such a profound effect on a country geographically situated halfway around the world. Many texts in Yoga and Āyurveda are in Sanskrit, and if Sanskrit was dead in some historical sense (the incapacity to respond to challenges or adapt to the surroundings), all this would not be possible. This is a case of a later perception that contradicts the earlier one. In other words, even in the earlier paper, the vitality of Sanskrit is perceived, but is theorized to be dead. As discussed previously, this is a case of \textit{interpretation} superseding all \textit{pramāṇa}–s including\textit{ pratyakṣa}. It should be noted that in Pollock’s career, perceptual knowledge is significant but is at times denied in favour of \textit{interpretation}.

\vspace{-.3cm}

\section*{3.6 Definition of {\it {\bfseries Anumāna}}}\index{anumana@\textit{anumāna}}

\vspace{-.2cm}

Inference is the process by which a theory (or assertion) is established. The three types of inferences are discussed in this \textit{sūtra}.

\textbf{\textit{Nyāya–sūtra}\index{Nyaya-sutra@\textit{Nyāya-sūtra}} 1.1.5}

\vspace{-.3cm}

\begin{verse}
\textit{atha tat–pūrvakaṁ trividham anumānaṁ – pūrvavat śeṣavat\\ sāmānyato dṛṣṭaṁ ca|}
\end{verse}

\textit{\textbf{Anuvāda}} (p.43)

Then, that which is preceded (by \textit{pratyakṣa}\index{pratyaksa@\textit{pratyakṣa}}) and is of three types is\textit{ anumāna}: \textit{pūrvavat}, \textit{śeṣavat} and \textit{sāmānyato–dṛṣṭa}. (The commentary states that one can infer from \textit{pratyakṣa}, from another \textit{anumāna}, from \textit{upamāna}\index{upamana@\textit{upamāna}} and from \textit{śabda}\index{sabda@\textit{śabda}}.)

\vspace{-.3cm}

\subsection*{3.6.1 {\it {\bfseries Pūrvavad–anumāna}}}

\vspace{-.2cm}

In the \textit{pūrvavad–anumāna}, the effect is inferred from its cause. From the dark clouds (cause), it is inferred that it will rain (effect).

\textbf{\textit{Parīkṣā}\index{pariksa@\textit{parīkṣā}} of 3D Philology\index{philology}}

We have observed (\textit{pratyakṣa–pramāṇa}\index{pramana@\textit{pramāṇa}}) many times, or some trusted person (\textit{śabda–pramāṇa}) had told us, that dark clouds are followed by rain. On seeing dark clouds, we infer that rain follows. Similarly, Islamic invasion and the resulting destruction which is known to us is considered the \textit{cause}. The effect according to Pollock\index{Pollock, Sheldon} is that Sanskrit texts were repurposed to respond to this event, or at least this is the inference that is attempted.

\begin{myquote}
The representation of invader as demon and defender as the divine king Rama\index{Rama@Rāma}, in royal temple cult, documentary inscriptions, and historiographical texts, arising as it seems to do first in the wake of the political events of the eleventh to fourteenth centuries, might suggest to some that it is a response to what was perceived to be a new and special sort of threat…because \textbf{data are devilishly hard to find}...I want to concentrate here on what I'll call textual events, certain new kinds of textual practices and the cultural processes they \textbf{presuppose} that may invite \textbf{wider social inference} (1993b: 284–285)
\end{myquote}

The \textit{effect} here is the “othering” of the Islamic invaders in royal temple cult, inscriptions and texts. And the \textit{cause} of this repurposing is Islamic invasion which is well known. Because of the invasion of the Islamic hordes, Indians started representing them as demons in temples and inscriptions. Pollock\index{Pollock, Sheldon} thus attempts to show that the \textit{Rāmāyaṇa}\index{Ramayana@\textit{Rāmāyaṇa}} is primarily a political text. If it were, the commentators of the \textit{Rāmāyaṇa} would be mentioning the Islamic invasions, but they have not a single word on this issue. Then, Pollock tries to use royal temple cult, inscriptions, etc. as \textit{pramāṇa}–s\index{pramana@\textit{pramāṇa}}, but even they have little to say about Islamic invasions. When an important historical event takes place, it is usually recorded. Even though this process of supposed repurposing (against Islam) is being carried out on a large scale around the country, the Indians themselves are not recording this event and thus “data are devilishly hard to find” in the “millions and millions” of texts and inscriptions. What we (including Pollock) are looking for is textual reference to this major event of repurposing. Thus, Pollock tries to support his inference by showing the following “textual events:”

\vspace{-.3cm}

\begin{enumerate}
\itemsep=0pt
\item Commentaries being written is a sign of repurposing even though commentators do not mention this process of repurposing.

 \item The \textit{Rāmāyaṇa }rendered in vernacular languages is another sign, but again Kamban\index{Kamban}, Tulasīdās\index{Tulasidas@Tulasīdās} and others are not discussing this repurposing.

 \item Rāma\index{Rama@Rāma} temples are being built is yet another sign, although the temple inscriptions make no mention of this repurposing.

\end{enumerate}

\vspace{-.3cm}

Pollock is seeing the dark clouds (Islamic invasion) and but he is unable to perceive the rains (textual evidence for repurposing). If the innumerable Sanskrit texts do not record this process of repurposing even once which is necessary to make a “wider social inference,” where is Pollock getting this \textit{cause–effect} idea in the first place? An analysis of “\textit{Rāmāyaṇa }and Political Imagination in India” provides us an answer which is discussed later [4.2.2]. It should not be difficult to see that this is an incorrect form of \textit{pūrvavad–anumāna}\index{anumana@\textit{anumāna}}.

%~ \newpage

\vspace{-.3cm}

\subsection*{3.6.2 {\it {\bfseries Śeṣavad–anumāna}}\index{anumana@\textit{anumāna}}}

\vspace{-.3cm}

In the \textit{śeṣavad–anumāna}, the cause is inferred from the effect. The flowing water in the river (effect) is directly perceived (by \textit{pratyakṣa}) to be different from the earlier flow because of its colour, speed and volume. Either we have perceived this (by \textit{pratyakṣa}\index{pratyaksa@\textit{pratyakṣa}}) many times or heard it from an authoritative source (\textit{śabda–pramāṇa}\index{sabda@\textit{śabda}}). Then, it is inferred that it has rained upstream.

\textbf{\textit{Parīkṣā}\index{pariksa@\textit{parīkṣā}} of 3D Philology\index{philology}}

This form of \textit{anumāna} can be seen when discussing the communicative status of Sanskrit. Was Sanskrit used for everyday communication? It was never used for this purpose according to Pollock\index{Pollock, Sheldon}.

\textbf{Effect:} There is no \textit{laukika} (this–worldly) literature other than the \textit{Rāmāyaṇa}\index{Ramayana@\textit{Rāmāyaṇa}} in the early period, whereas in the later period there is an abundance of this–worldly literature. (Similar to the difference in the flow of water at different times.)

\textbf{Cause:} Sanskrit was not an everyday spoken language in the early period. (There is no rain upstream.)

But this is an incorrect inference because even when there is abundance of \textit{laukika} literature available later on, it is still inferred that Sanskrit was not a spoke language.

\begin{myquote}
The basic question here, usually formulated as whether or to what degree Sanskrit was ever an everyday spoken language, has long been debated, and current \textbf{sociolinguistic} opinion seems rather muddled…It is significant that, with the exception of the \textit{Rāmāyaṇa}, no remains of a nonsacral, this–worldly Sanskrit are extant from the early epoch…It is not easy to believe that virtually every scrap of \textbf{early evidence of such a usage has been lost}…Moreover, all that we can \textbf{infer} about the sociality of the language from the moment we can glimpse it provides further counterevidence to the belief that Sanskrit ever functioned as an everyday medium of communication. (2016a: 48–49)
\end{myquote}

That Sanskrit is still spoken is not included as a \textit{pramāṇa}\index{pramana@\textit{pramāṇa}} since it is considered dead and the language being used for communication today is said to be “completely denaturalized.” Thus, living traditions (and \textit{pratyakṣa}–\textit{pramāṇa}) are denied. But what doesPatañjali’s\index{Patanjali@Patañjali}	 \textit{Mahābhāṣya}\index{Mahabhasya@\textit{Mahābhāṣya}} say about Sanskrit usage? The word \textit{loka} (referring to the world) is used around two–hundred times in the text of\textit{ Mahābhāṣya}. The usage \textit{yathā loke} (as seen or used in the world) appears frequently and one example is cited here\textit{:}

\vspace{.2cm}

\begin{myquote}
\textit{tad yathā loke – kañcit kaścit pṛcchati | grāmāntaraṁ gamiṣyāmi panthānaṁ me bhavān upadiśatu iti | saḥ tasmai ācaṣṭe | amuṣminn avakāśe hastadakṣiṇaḥ grahītavyaḥ amuṣmin avakāśe hastavāmaḥ iti |}\endnote{\textit{Vyākaraṇa Mahābhāṣya} of Patañjali\index{Patanjali@Patañjali}. Vol.1 edited by F. Kielhorn and K. V. Abhyankar. (Poona: Bhandarkar Oriental Research Institute, 1985):118.}
\end{myquote}

\vspace{.2cm}

\begin{myquote}
As is seen in the \textit{loka} Someone questions someone else. I will go to another village; you tell me the way. He answers him (the questioner). In this place, take the righthand side and in this place take the left–hand side.
\end{myquote}

Many such usages seen in the\textit{Mahābhāṣya}\index{Mahabhasya@\textit{Mahābhāṣya}} are an indication that in the early period, Sanskrit was used for everyday communication. Pollock\index{Pollock, Sheldon} is willing to forego even a text like \textit{Mahābhāṣya }(\textit{śabda–pramāṇa}\index{pramana@\textit{pramāṇa}}\index{sabda@\textit{śabda}}) and accept sociolinguistics (more theories) as \textit{pramāṇa}–s and conclude Sanskrit was not spoken. In the later period, even when there is an abundance of \textit{laukika} literature, Sanskrit is not considered a spoken language because George Cardona \index{Cardona, George} says (Sanskrit “had ceased to be truly a current language”) so. In essence, there no connection between cause and effect; this then is an incorrect use of the \textit{śeṣavad-anumāna}.

It was mentioned in the introduction that \textit{interpretation} (theories) had for Pollock greater authority than even the\textit{ pramāṇa}–s and this is one more such instance. There is no effort to present the view of the Indian tradition on this issue. Note that the inference is based on sociolinguistic theory and not from a Sanskrit text. \textit{The Language of the Gods} (from which the above passage was cited) was published in 2006, a few years before the three–dimensional philology\index{philology} is formulated. As there is no attempt to present the view of tradition (plane 2)\index{Second Dimension/Traditional/Plane 2}, the three–dimensional philology is non–existent at least till 2006. As the next chapter shows, the three–dimensional philology can only be \textit{kalpita} or “consciously constructed” with no connection to his own practice.

%%%%%%%%%%%%%%71%%%%%%%%%%%%%
\subsection*{3.6.3 {\it {\bfseries Sāmānyato–dṛṣṭa–anumāna}}}\index{anumana@\textit{anumāna}}

The \textit{sāmānyato–dṛṣṭa}–\textit{anumāna} includes instances of inferences where perception is not possible. There are certain things that cannot be directly perceived, such as the movement of the sun. We can see it in one place and then at a later period, in a different place. Thus, it is inferred that sun has movement even though movement is not known through\textit{ pratyakṣa–pramāṇa}\index{pratyaksa@\textit{pratyakṣa}}.

\textbf{\textit{Parīkṣā}\index{pariksa@\textit{parīkṣā}} of 3D Philology\index{philology}}

The primary thesis statement of the \textit{Language of the Gods in the World of Men} can be considered as an instance of this sort of inference.

\begin{myquote}
This book is an attempt to understand \textbf{two great moments} of transformation in culture and power in pre–modern India. The \textbf{first} occurred around the beginning of the Common Era, when Sanskrit, long a sacred language restricted to religious practice, was \textbf{reinvented} as a code for literary and political expression. This development marked the start of an amazing career that saw Sanskrit literary culture spread across most of southern Asia from Afghanistan to Java. (2006a: 1)
\end{myquote}

The influence of Sanskrit and Indian culture on the regions far and near has yet to be fully documented. Sanskrit knowledge enriched the cultures it came in contact with including the Persian, Arabic, Chinese and Japanese. Europe (and particularly Germany) in 1800 C.E. came alive with the coming of Sanskrit knowledge. This is a history of Sanskrit that needs to be told and retold in great detail. Pollock\index{Pollock, Sheldon} limits himself to literary culture of Sanskrit that spread across southern Asia. How and why did this happen? We inferred that the sun had movement (which cannot be perceived) because we see it in two different places after a period of time. Similarly, we see Sanskrit literary culture spread over a vast geographical area of South Asia over a period of time. There were no invasions, no large–scale missions for propagation and no kingly sanctions that can be attributed for this movement of Sanskritic culture. Thus, we do not perceive the reason for the spread of Sanskrit, but as in the case of the sun, we see it in many different places. Pollock theorizes that Sanskrit was “reinvented” which then made it possible to spread over vast regions.

However, this theory of reinvention (which depends on the beginning of \textit{Kāvya} and the invention of writing) is based on incorrect inferences. The beginning of \textit{Kāvya} and the invention of writing discussed previously [2.6] was shown to contradict both the \textit{pramāṇa}–s\index{pramana@\textit{pramāṇa}} and his own earlier views. Pollock attempts to use inscriptions (\textit{śabda–pramāṇa}\index{sabda@\textit{śabda}}) to show the political aspect of literature as the reason for the spread of Indian culture. An analysis in sections [4.6.6, 4.6.8] shows that no\textit{ pramāṇa}–s are shown to support the two great moments. This, then, is an incorrect type of \textit{sāmānyato–dṛṣṭa anumāna}\index{anumana@\textit{anumāna}}.

The three types of \textit{anumāna }were defined and their (incorrect) use shown. \textit{Anumāna} is needed to establish the concomitance (\textit{vyāpti}) between two objects. Any thesis statement has to be established by an \textit{anumāna}\index{pratyaksa@\textit{pratyakṣa}}. Inference can be based on any one of the four \textit{pramāṇa–}s: \textit{pratyakṣa}\index{pratyaksa@\textit{pratyakṣa}}, another \textit{anumāna}, \textit{upamāna}\index{upamana@\textit{upamāna}} or \textit{śabda}\index{sabda@\textit{śabda}}. In all papers, Pollock\index{Pollock, Sheldon} uses \textit{anumāna} (incorrectly) to establish his theories. The incorrect use is documented in the next chapter of this \textit{prabandha}.


\section*{3.7 Definition of {\it {\bfseries Upamāna}}}

\textit{Upamāna }is the most important \textit{pramāṇa}\index{pramana@\textit{pramāṇa}} in Pollockian philology\index{philology} as we shall shortly show. The definition of \textit{upamāna} is discussed below.

\textbf{\textit{Nyāya–sūtra}\index{Nyaya-sutra@\textit{Nyāya-sūtra}} 1.1.5}

\vspace{-.3cm}

\begin{verse}
\textit{prasiddha–sādharmyāt sādhya\index{sadhya@\textit{sādhya}}–sādhanam upamānam |}
\end{verse}

\vspace{-.3cm}

An object which is known through its similarity to another well–known object is \textit{upamāna}.

\textbf{\textit{Parīkṣā}\index{pariksa@\textit{parīkṣā}} of 3D Philology}

\textit{Upamāna }is unambiguous knowledge of an object sought to be known by its similarity to another object which is already well known. \textit{Nyāya–bhāṣya}\index{Nyaya-bhasya@\textit{Nyāya-bhāṣya}} gives several examples of \textit{upamāna}. There is an animal called the \textit{gavaya} (which lives in the forest) which is similar to the cow. A villager has knowledge of the cow but has not seen the \textit{gavaya}. An \textit{āpta}\index{apta@\textit{āpta}}	(trusted person) who has seen the \textit{gavaya} tells the villager that it is similar to the cow. The villager then goes to the forest, sees the \textit{gavaya}, recollects what was said by the \textit{āpta}, and then decides that the animal perceived by him is indeed the \textit{gavaya}.

In our case, Pollockian philology attempts to present the Indian traditional viewpoint (at least along the second plane) without Western bias. For this to happen, analogy with Western paradigms has to be avoided along the second plane. This means the one would have to follow Sanskrit texts closely when describing the Indian intellectual traditions. But Pollockian philology constantly attempts to go beyond the text: “How did people experience these transformations in the realm of thought? That is what we need to discover.” (Pollock 2007e: 4) Denying the validity of living traditions (\textit{pratyakṣa–pramāṇa}) and going beyond texts into textualized thought (essentially entering the Indian mind and superseding \textit{śabda–pramāṇa}) is the basis of this philology. If \textit{pratyakṣa– }and \textit{śabda–pramāṇa}–s are not used, inferences can only be from \textit{upamāna}. Pollock himself addresses this issue.

\begin{myquote}
In fact, I am becoming persuaded not only that we cannot not do intellectual history, but when we do it, \textbf{it must be comparative}. That \textbf{comparison is a cognitive necessity} is becoming increasingly obvious to scholars...If comparison is everywhere, we need to make our inevitable but implicit comparisons explicit and to try to explain what role they are playing in the \textbf{interpretation} of our primary object.(2007e: 4)
\end{myquote}

\textit{Upamāna}\index{upamana@\textit{upamāna}}–\textit{pramāṇa}\index{pramana@\textit{pramāṇa}} is used to show either similarity or dissimilarity between two objects, but if both \textit{pratyakṣa}\index{pratyaksa@\textit{pratyakṣa}} (living traditions) and \textit{śabda}\index{sabda@\textit{śabda}}(Sanskrit texts) are disregarded, then \textit{upamāna }will become a “cognitive necessity.” Thus, as shown previously, comparing with Western models would become a necessity. The paper “Comparison Without Hegemony” also repeats the same idea that comparison is inevitable. This passage is written in 2007, before the three–dimensional philology\index{philology} is constructed. If comparison is inevitable and if “cases that constitute the objects of our \textit{intellectual} history are forms of systematic thought that are found everywhere literate culture itself is found” (2007e: 4) then there certainly would be “remarkable parallels awaiting discovery.” Thus, the second dimension\index{Second Dimension/Traditional/Plane 2} (meant to represent the tradition without comparison) would only be a personal view with no regard for the Indian viewpoint. The same can be said for the meaning of a text along the first dimension\index{First Dimension/Historical/Plane 1}. Once comparison (\textit{upamāna}) becomes a cognitive necessity, then it implies that there is only a single dimension to understanding a Sanskrit text.

\vspace{-.3cm}

\section*{3.8 Definition of {\it {\bfseries Śabda}}}

A discussion of \textit{śabda–pramāṇa }or authoritative testimony follows.

\textbf{\textit{Nyāya–sūtra}\index{Nyaya-sutra@\textit{Nyāya-sūtra}} 1.1.5}

\vspace{-.3cm}

\begin{verse}
\textit{āptopadeśaḥ śabdaḥ |}
\end{verse}

\vspace{-.3cm}

\textit{Śabda} is the \textit{upadeśa} (communication) of an \textit{āpta}\index{apta@\textit{āpta}} (trustworthy person).

\textit{\textbf{Nyāya–bhāṣya}}\index{Nyaya-bhasya@\textit{Nyāya-bhāṣya}} (p.14)

\begin{myquote}
\textit{āptaḥ khalu sākṣātkṛta–dharmā yathādṛṣṭasyārthasya cikhyāpayiṣayā prayukta upadeṣṭā | sākṣāt–karaṇam arthasyāptiḥ | tayā pravartata ityāptaḥ | ṛṣyārya–mlecchānāṁ samānaṁ lakṣaṇam |}
\end{myquote}

\textit{\textbf{Anuvāda}}

\textit{Āpta} is one who has a direct knowledge of an object and who wants to communicate the object as it was directly known by him. Direct knowledge of an object is \textit{āpti} and one who uses it is an \textit{āpta}. This definition of \textit{āpta} is the same for \textit{ṛṣi}–s (seers), \textit{ārya}–s (generally Indians) and \textit{mleccha}–s (foreigners).

\textbf{\textit{Parīkṣā}\index{pariksa@\textit{parīkṣā}} of 3D Philology\index{philology}}

Vātsyāyana’s\index{Vatsyayana@Vātsyāyana} definition of an \textit{āpta}\index{apta@\textit{āpta}} can include anyone (including a Western scholar) as long as the \textit{āpta }has direct knowledge (known through the \textit{pramāṇa}–s\index{pramana@\textit{pramāṇa}}) and communicates it without falsifying. Scholars who propagate social theories cannot be \textit{āpta}–s because they do not have direct knowledge of the society they are theorising (by using anthropological and other models). In any case, substantiating a thesis based on a Western \textit{āpta} cannot be included under the second dimension\index{Second Dimension/Traditional/Plane 2} (plane 2) which should represent the traditional Indian viewpoint. One has to note the irony here as Vātsyāyana’s definition accommodates Western scholars as authoritative, while Pollock’s\index{Pollock, Sheldon} philology discounts the living Indian scholarly tradition. Nyāya–śāstra\index{sastra@\textit{śāstra}} is also known as \textit{pramāṇa śāstra }and forms the basis of all Indian śāstric traditions. In the \textit{bhāṣya} above, Vātsyāyana states that \textit{ṛṣi}–s, \textit{ārya}–s and \textit{mleccha}–s can all be \textit{āpta}–s as long as they have a direct knowledge of an object and communicate it as they have perceived it. There is no distinction between the three in this aspect. However, Pollock interprets this \textit{ārya–mleccha }distinction as a sign of fundamental inequality in Indian society and as an indication of racism.

\begin{myquote}
The binary pair \textit{\textbf{ārya/anārya}} is one of several discursive definitions by which the Sanskrit cultural order constitutes itself. It overarches the world of traditional Indian inequality...Another antithesis, \textit{\textbf{ā}\textbf{rya/mleccha}}, seems to add little new...From such factors as the semantic realm of the distinction \textit{ārya/anārya} and the biogenetic map of inequality…it may seem warranted to speak about a "\textbf{pre–form of racism}" in early India…\break (1993a: 107–108)
\end{myquote}

The terms \textit{ārya–anārya} and \textit{ārya–mleccha} are said to denote “biogenetic map of inequality.” Pollock quotes \textit{Manusmṛti}\index{Manusmrti@\textit{Manusmṛti}}\index{smrti@\textit{smṛti}} and Lakṣmīdhara\index{Laksmidhara@Lakṣmīdhara}, but this important \textit{sūtra }is not mentioned. An inherently unequal society would never have this concept and that too uttered by a \textit{Ṛṣi}: in all spheres of activities, a \textit{mleccha}’s word is as valid as a \textit{ṛṣi}’s or an\textit{ ārya}’s. But Pollock disregards this statement and interprets these terms as indicative of some form of racism.

%~ \newpage

\textit{Śabda–pramāṇa} is important since the basis of a textual philologist like Pollock should be Sanskrit texts. But, since 1985 (post–\textit{Orientalism} period) there would no textual basis for any Pollockian theories. One example clearly illustrates this aspect (others were discussed earlier).

\begin{myquote}
Voluntaristic vernacularism and non–coercive cosmopolitanism are values largely \textbf{unnamed and untheorized} in the classical tradition precisely because they were given and not produced… (2015b: 8)
\end{myquote}

Earlier, we saw that Indian philology\index{philology} was never theorized and now the same is said of vernacularism and cosmopolitanism. This implies that these were known to the Indian conscious but was never textually documented because they were taken for granted. So, if the texts are silent about these issues, then what is the source of this idea in the first place? \textit{Śabda}–\textit{pramāṇa}\index{pramana@\textit{pramāṇa}}\index{sabda@\textit{śabda}} (Sanskrit texts) is not possible as the texts are silent on this issue. \textit{Pratyakṣa}\index{pratyaksa@\textit{pratyakṣa}}–\textit{pramāṇa} has been discounted since traditions are not living any more. Then, it can only mean that \textit{upamāna}\index{upamana@\textit{upamāna}} (comparing and contrasting with Western models) is used for establishing this theory of vernacularism and cosmopolitanism. This, again, would contradict the definition of the second plane: representing tradition without comparison. An analysis in section [4.6] will show that Pollockian theories on Indian tradition will have no basis in Sanskrit texts.

\textit{\textbf{Nyāya–bhāṣya}}\index{Nyaya-bhasya@\textit{Nyāya-bhāṣya}} (p.14)

\begin{myquote}
\textit{tathā ca sarveṣāṁ vyavahārāḥ pravartanta iti | evam ebhiḥ pramāṇaiḥ deva–manuṣya–tiraścāṁ vyavahārāḥ prakalpante, nāto'nyatheti |}
\end{myquote}

\textit{\textbf{Anuvāda}}

In this way (through the four \textit{pramāṇa–}s), all interactions in the world takes place. The activities of \textit{devata}–s (gods), \textit{manuṣya}–s (humans) and \textit{tiryac} (animals) are done with the help of these \textit{pramāṇa}–s. They (activities of the world) are not possible otherwise.

\textbf{\textit{Parīkṣā}\index{pariksa@\textit{parīkṣā}} of 3D Philology}

\textit{Nyāya–bhāṣya} concludes the enquiry into\textit{ pramāṇa}–s with a universal observation: all interactions take place only through the \textit{pramāṇa}–s and without it no activity will take place. Animals also comprehend objects through the \textit{pramāṇa}–s indicating that obtaining knowledge through\textit{ pramāṇa}–s is a natural process. In this section, the four \textit{pramāṇa}–s as used in the works of Pollock\index{Pollock, Sheldon} have been documented. Since Pollock necessarily has to use \textit{pramāṇa}–s to establish any theory, a \textit{pramāṇa}\index{pramana@\textit{pramāṇa}}–\textit{parīkṣā}\index{pariksa@\textit{parīkṣā}} in the following section tests and establishes the validity of these theories.

