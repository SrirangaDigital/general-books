
\chapter{{\it {\bfseries Uddeśa}}\\ – Three–dimensional Philology and Nyāya Śāstra}\label{chapter2}

\vspace{-.3cm}

The primary purpose of this \textit{prabandha} is to analyse the three–dimen\-sional philology\index{philology} in the framework of Nyāya–śāstra\index{sastra@\textit{śāstra}}. Thus, initially philology’s historical birth is traced, and the three–dimensional philology is described by closely following Pollock’s\index{Pollock, Sheldon} own works on philology.

\vspace{-.3cm}

\section*{2.1 Historical Introduction to Philology}

\vspace{-.2cm}

The birth of philology as an academic science towards the end of the eighteenth century can be traced back to Germany. Europe was introduced to Indian knowledge systems after the famous speech of Williams Jones\index{Jones, Sir William} which resulted in large scale transfer of Sanskrit knowledge to Europe, and to Germany in particular.\endnote{ The \textit{Battle for Sanskrit} briefly addresses this issue in the section “Response: Ignoring how Sanskrit benefited the West”.

\begin{quote}
In the half–century that followed, almost every premier European university had started a Sanskrit and Indology department. It became important for intellectuals to learn this language, or at least be familiar with the contents of its texts. Western thought was transformed forever as a result of this encounter…It is unfortunate that when the European intellectual movement known as the Enlightenment is taught, the Indian influences are ignored or else reduced to mere footnotes. The European Enlightenment is made to look like an entirely internal development by Europeans, without any contribution from elsewhere. Nonetheless, during the time of that movement, there was great enthusiasm for appropriating Sanskrit ideas, digesting them into Western languages and frameworks, and using them in intellectually innovative ways. The benefits of this knowledge were felt in several modern fields, including philosophy, botany, psychology, linguistics, anthropology, ethics, mathematics, Christian theology and comparative religion. (Rajiv Malhtora. \textit{The Battle for Sanskrit} (India: HarperCollins, India, 2016), 172–73.)
\end{quote}} This new knowledge brought about fundamental changes in the German academia which later would influence the rest of Europe and America. Sanskrit language itself was to play a pivotal role as until then Europe did not have a scientific understanding of language (no grammar existed that was even remotely comparable to Pāṇini’s\index{Panini@Pāṇini} \textit{Aṣṭādhyāyī}\index{Astadhyayi@\textit{Aṣṭādhyāyī}}) and thus, for example, had no access to Greek texts which for them were essentially “dead.” This process of encountering the living language tradition of Sanskrit infused life into European notions of language which in turn led to the academic birth of philology.\endnote{ “Philology’s emergence as an independent form of knowledge is usually marked by the moment Friedrich August Wolf, future editor and critic of Homer and member of Humboldt’s new institution in Berlin, declared on enrolling in the University of Göttingen in 1777 that he was a “philologist” (\textit{studiosus philologiae}), thereby becoming the first official student of the subject in Europe.” (2015h: 6)} Pollock recounts the history of the birth of philology in the introduction to\textit{ World Philology}.\endnote{ Pollock accepts the institutional and academic birth of philology in Germany but claims that the conceptual history of philology can be traced back to others such as Spinoza. But Pollock admits that many scholars consider 1800 as the birth of philology: “For Michel Foucault, philology in the modern era began with the transformed understanding of the nature of language itself at the end of the eighteenth century. In the chapter “Labor, Life, Language” in \textit{The Order of Things}, Foucault attributes almost magic properties to what he calls the “discovery” or “birth” of this philology.” (Pollock 2009b: 936) “For Foucault…the invention of modern philology as historical–grammatical study is to be credited to Franz Bopp, whose \textit{Conjugationssystem der Sanskritsprache} (1816) demonstrated the morphological relationship among Sanskrit, Persian, Greek…” (Pollock 2009b: 938)}

Even though the term philology, literally meaning “love of learning”, was in use from earlier times, it was around 1800 C.E. that it gained institutional predominance in Germany, and came to be considered a separate subject and an independent form of knowledge. At that time, philology in Germany meant analysing a text based on linguistic principles learnt from Sanskrit. Learning and understanding Sanskrit lead to the understanding of European languages which then gave Europe access to texts (such as Greek and Latin) and thus to its past that until then had been greatly obscured. Thus, 1800 C.E. would become an important period in European intellectual history. Works of Franz Bopp\index{Bopp, Franz} and Wilhelm von Humboldt\index{Humboldt, Wilhelm von} based on Sanskrit\endnote{ Pollock briefly mentions their role in \textit{Philology and Freedom} (2016b: 5). Bopp’s \textit{Ausführliches Lehrgebäude der Sanskritsprache} (\textit{Detailed System of the Sanskrit Language}) and Wilhelm von Humboldt’s book \textit{On Language} were greatly influential. Friedrich Schlegel was another important figure during this time.} were widely read and greatly influenced the German and European scholars of that time. This is a history (the influence of Sanskrit on European thought) that needs to be documented and told in great detail which, however, is beyond the scope of this \textit{prabandha}. Pollock\index{Pollock, Sheldon} refers to the influence of Sanskrit on Western thought in \textit{The Language of the Gods in the World of Men}.

\begin{myquote}
…Franz Bopp, William Dwight Whitney\index{Whitney, W. D.}, Ferdinand de Saussure\index{de Saussure, Ferdinand}, Emile Benveniste, Leonard Bloomfield\index{Bloomfield, Leonard}, and Noam Chomsky\index{Chomsky, Noam}, learning both substantively and theoretically from Indian premodernity (being all of them Sanskritists or students of Sanskrit–knowing scholars), developed successively historical, structural, and transformational linguistics and, by these new forms of thought, invented some basic conceptual components of Western modernity itself. (2006a: 164)
\end{myquote}

That Sanskrit could travel such great distance without any governmental support and cause such a profound effect on Europe is itself an indication of its vitality. This could only mean that Sanskrit had, and still has, the capacity to make history: to decisively transform any culture that it encounters. In spite of Europe coming alive because of Sanskrit, Pollock claims that it actually died around 1800 C.E.!

With Sanskrit as its source, philology then became an important branch of learning which would profoundly influence and lead to the birth of many other disciplines at that time. Pollock explains the importance and centrality of philology\index{philology} and its wide-ranging influence.

\begin{myquote}
Philology was the \textbf{queen of the sciences} in the nineteenth century European university, bestriding that world like a colossus in its conceptual and institutional power. It set the standard of what scientific knowledge should be and influenced a range of other disciplines, from anthropology to zoology. (2015h: 2)
\end{myquote}

Philology initially meant a close analysis of texts based on linguistic principles (learnt from Sanskrit). But because of its undefined nature, over a period of time it came to mean various approaches (including political as in the case of Pollock) in understanding a text. What then was the exact nature of philology? Even though philology had its basis in Sanskrit, the early adherents were unable to formalize it which was necessary for philology to become a separate form of knowledge, and attain the status of a \textit{śāstra} or science. After its genesis, philology gradually became equated with \textit{interpretation} and the problem was that the exact nature of \textit{interpretation,} was never defined. At times, it meant \textit{interpretation} based on grammar, textual criticism and historical analysis. But from the beginning,\textit{ interpretation} was never bounded by any set of rules. The undefined nature of \textit{interpretation} continues to haunt philology to date. There were no rules framed on how to interpret texts, and the actual nature of \textit{interpretation} was never discussed. This problem can be seen in the philology of Pollock\index{Pollock, Sheldon} which depends completely on \textit{interpretation} which is never actually defined anywhere. Pollock himself, while tracing the history of philology\index{philology}, repeatedly states that the essential component of philology from its birth was \textit{interpretation}.

\begin{myquote}
...\textbf{philology and interpretation as such are identical}; interpretation actually precedes and informs all other aspects of philology, including grammar and criticism. (2015h: 7)
\end{myquote}

Pollock describes the profoundness of Schlegel’s\index{Schlegel, Friedrich} notion of philology: \textit{interpretation} would have greater authority than grammar or literary criticism. In other words, as Pollock’s own methodology will show, what can be known directly from a text is of lesser significance than \textit{interpretation}. This would lead to the eventual downfall of philology as in due course it becomes nothing other than \textit{interpretation}. Even for Schlegel and other earlier German philologists such as August Boeckh\index{Boeckh, August} (who defined philology as “knowing what has been known”), the nature of \textit{interpretation} was never defined. This would hold true for Nietzsche\index{Nietzsche} also, who is considered by Pollock to be one of the most critical and visionary philologists. For Nietzsche, the essence of philology was again \textit{interpretation} or hermeneutics:

\begin{myquote}
As he [Nietzsche] described it in one of his last published works, philology is \textit{\textbf{ephexis}}\endnote{ The Greek work \textit{ephexis}  {\greekchars (ἔφεξις)} means checking or stopping. Nietzsche even considers the “salvation of the soul” as coming under philology but no method is described for others to follow.}, \textbf{constraint} (or restraint), in \textbf{interpretation}, the means by which we learn to guard ourselves both from falsification and from the impulse to abandon caution, patience, and subtlety in the effort to understand. And this pertains not just to reading the Greek or Latin classics, but “whether one be dealing with books, with newspaper reports, with the most fateful events or with weather statistics...”\break (2015h: 8)
\end{myquote}

Thus, even for a critical and astute philologist like Nietzsche\index{Nietzsche}, philology was “constraint” in \textit{interpretation}, which was again never defined. As to how one could avoid falsehood and attain truth through \textit{interpretation} was not expounded by Nietzsche. And this “slow” or “patient” reading, as Nietzsche called philology, was extended to the understanding of all texts including the classics without defining the process. Would it be possible to have a set of rules that others could follow, and which would also act as a constraint on our \textit{interpretations}? Evidently, even the great Nietzsche was not able to formulate any such rules for philology\index{philology}. And so, even after a hundred years (Nietzsche died in 1900) since its birth, philology merely remained \textit{interpretation} with no governing principles.

This emphasis on \textit{interpretation} as the vital element in comprehending texts that were written over a span of several thousand years was also the approach taken by Erich Auerbach\index{Auerbach, Erich} whom Pollock\index{Pollock, Sheldon} quotes approvingly.

\begin{myquote}
In 1969\endnote{ Pollock claims that Auerbach warned of the death of philology in 1969, but Auerbach himself had passed away in 1957. The publication year of the Auerbach’s paper \textit{Philology and “Weltliteratur”} is mistakenly taken to be the year that Auerbach made this statement.}, the Romance philologist Erich Auerbach, widely viewed as the consummate practitioner of the discipline in the post–World War II era, warned of the imminent disappearance of philology, describing its loss as “[the loss of such a spectacle—whose appearance is thoroughly dependent on \textbf{presentation} and \textbf{interpretation}—would be]\endnote{ See 2016b: 7fn. The bracketed portion is from the following paper: Erich Auerbach, “Philology and “Weltliteratur”” Translated by Maire Said and Edward Said, \textit{The Centennial Review} 13, no. 1 (Winter 1969): 5.} an impoverishment for which there can be no possible compensation.” (2015h: 4)
\end{myquote}

Even Auerbach, a renowned figure in philological circles, who had correctly foreseen the death of philology, viewed it as a spectacle (performance associated with the arts) and not something rule based. Auerbach also equated philology with presentation and \textit{interpretation,} and this \textit{interpretation} was clearly of a personal nature and not a methodology that others could follow. Pollock admits that “even the great Auerbach never did” actually theorize what philology actually was. By Auerbach’s time, one–hundred and fifty years had passed since its birth, and philology still remained untheorized and based only on a personal \textit{interpretation}.

\newpage

Next in the line of eminent philologists discussed is Hans–Georg Gadamer\index{Gadamer, Hans-Georg}, who is referred to when discussing what a text would mean to a philologist.

\begin{myquote}
Gadamer...was therefore right to stress the role of the old hermeneutic stage of \textit{applicatio}...Discovering the meaning of such texts [laws or art] by understanding and \textbf{interpreting} them…And the principle here holds for all \textbf{interpretation}; \textit{applicatio} is not optional but integral to understanding. (2009b:957–958)
\end{myquote}

Gadamer was well known for his philosophical hermeneutics (the English translation of \textit{Truth and Methodology} was published in 1975) where he tried to provide a framework for understanding texts and art, while at the same time rejecting the idea that understanding could have a method or a set of rules. Thus, even for an astute philologist like Gadamer, \textit{applicatio} or \textit{interpretation} was of primary importance in understanding a text, but there could never be a methodology for this process. The interpreter was always bound by a “historical consciousness” of the tradition he belonged to and thus interpretations would necessarily be prejudiced by that tradition. Thus, even for Gadamer philology could never be theorized as \textit{applicatio} or \textit{interpretation} was not bounded by any rules.

While discussing the future of philology\index{philology}, Pollock\index{Pollock, Sheldon} refers to two more renowned philologists, Paul de Man\index{de Man, Paul} and Edward Said\index{Said, Edward}. An influential philologist, Paul de Man authored a work titled \textit{The Return to Philology} (published 1986), and defined philology as “mere reading…prior to any theory” and “attention to how meaning is conveyed” (2009b: 947). The most important figure in Pollock’s academic career was and continues to be Edward Said who defined philology generally as close reading and specifically as “a rigorous commitment to reading for meaning” (2009b: 960). According to Pollock, Said did not offer any theory (\textit{applicatio}) in his work also titled \textit{The Return to Philology} (published 2004) other than his own practice of separating the comprehension of literature from political leanings. Both Said and de Man defined philology with the intent that texts should be closely read for meaning and we should try as much as possible to avoid forcing our preconceived notions on the texts. Even though such an approach is laudable and humanist, it cannot be universal and would vary with individuals. Thus, an underlying theory for the \textit{interpretation} of a text would be impossible in both de Man’s and Said’s approach.

Pollock’s own survey of several eminent philologists over a period of two–hundred years illustrates that philology was always equated to \textit{interpretation} which was never defined and thus could only be of a personal nature. The same conclusion was reached by group of European philologists when they were asked to define their own field.

\begin{myquote}
In 1990...a cross–section of prominent European literary historians, premodern and modern, was assembled to ask what precisely was to be understood by the term “\textbf{philology},” and the very tentativeness of their answers demonstrates how little serious attention, of a theoretical, self–reflective sort, scholars had been paying up to that point. (2015a: 4)
\end{myquote}

Since \textit{interpretation} was the basis of philology\index{philology}, one could easily understand their tentativeness in arriving at any sort of definition. They were unable to define an object that was personal in nature and a generalization or a theory could never be possible as \textit{interpretations} would vary from person to person. Pollock\index{Pollock, Sheldon} finally gives one example of “pure philology” as practiced in the \textit{loka} (world). Internet site Rap Genius (now called Genius) provides a platform for listeners to annotate their favourite lyrics.

\begin{myquote}
Users, including original creators, provide annotation to the often complex lyrics of songs, as well as intertextual linkages and contextual material. The purpose of Rap Genius, originally named Rap Exegesis, is precisely to make sense of texts… The site seeks to “annotate the world,” “to help us all realize the richness and depth in every line of text.” This is \textbf{pure philology} in terms of practice. (2015a: 35)
\end{myquote}

In this website, listeners provide different interpretations for the same song lyrics which is natural as listeners come from varied backgrounds. The site declares that “interpretations can coexist and compete.” This would again mean that there could be no general approach in understanding a song lyric. Thus, even this example of “pure philology” belonging to the \textit{loka} would only amount to \textit{interpretation} which can neither be defined or theorized.

Finally, Pollock summarizes the nature of philology:

\begin{myquote}
...when turning to the \textbf{nature of interpretation}, which I take to be philology’s proper theory, I am only recovering its actual historical development, as Wilhelm Dilthey\index{Dilthey, Wilhelm} (like Schlegel\index{Schlegel, Friedrich} before him) made clear: (“This art of interpretation...originated in the personal and inspired virtuosity of the philologist, where it continues to flourish”). (2016b: 13–14)
\end{myquote}

Wilhelm Dilthey\index{Dilthey, Wilhelm} also equated philology with \textit{interpretation} and considered it an art (which depends on the skill of the interpreter) rather than science implying the impossibility of a theory of any sort. A distinctive theory would be needed to make philology a proper discipline, but the history of philology summarized from Pollock’s own works has shown that such a theory would be impossible as \textit{interpretation} can only be of a personal nature. Pollock\index{Pollock, Sheldon} equates \textit{interpretation} itself to theory failing to understand that\textit{ interpretation} is essentially opposed to theory, but finally admits that no “grand theory” is offered, but only his own approach to understanding Sanskrit texts which is referred to as the three–dimensional philology\index{philology}. Thus, even in a recent paper called \textit{Philology and Freedom} (most of the material here is recycled from earlier papers on philology), there is no theory that is presented other than equating it with \textit{interpretation}. Beginning with the paper on D.~D.~Kosambi\index{Kosambi, D. D.} (2008a), around ten papers have been written on the topic of philology (with much repetition), and in the end what is offered is only an autobiography of understanding texts which will be analysed later [4.2] in this \textit{prabandha}.

Pollock’s own historical survey of philology since its birth included some of the most distinguished philologists such as Schlegel\index{Schlegel, Friedrich}, Nietzsche\index{Nietzsche}, Auerbach\index{Auerbach, Erich}, Gadamer\index{Gadamer, Hans-Georg}, Dilthey\index{Dilthey, Wilhelm}, de Man\index{de Man, Paul}, Said\index{Said, Edward} and also a gathering of prominent European philologists. One instance was also cited from the real world. This historical assessment clearly and unambiguously established that philology was equated with \textit{interpretation} which could only be of a personal nature and thus could never be theorized. In spite of his own assessment, Pollock, in trying to impress the University administrators\endnote{ “I have long thought that if we spell out what is at stake as clearly as possible in every possible forum, surely decision makers in universities, foundations and governments will see that if we lose philology, we stand to lose something precious and irreplaceable.” (2014c: 398)}, attempts to theorize the untheorizable so that philology’s imminent death can be avoided or at least postponed. This leads to several contradictions in the papers on philology some of which are illustrated here.

The historical survey showed us that philology was untheorized, but now it is claimed that a theory existed for understanding texts!

\begin{myquote}
\textbf{Philology has such a theory}, as already noted, namely “interpretation,” or more grandly “hermeneutics.” \textbf{Such theory was developed} not only in Europe but in the Ancient Near East, the Arab world, India, China, and elsewhere in order to make sense of texts…(2016b: 16)
\end{myquote}

\newpage

By definition, \textit{interpretation} is personal and thus diametrically opposed to theory which is universal. But contradicting his own earlier views (in an attempt to impress administrators), it is said there was a theory developed not only in Europe, but across the world to make sense of texts! Earlier, it was admitted that no “grand theory” was being offered, now a “grand theory” across all time and space is proposed. Attempting to theorize and universalize an object that is inherently personal and untheorizable would only lead to such contradictions.

From no theory in the last two–hundred years to a grand theory, we return once again to no theory: that \textit{interpretation} cannot have a standard or method is finally admitted.

\begin{myquote}
The objective of philology\index{philology} is not to determine whether that interpretation is true or not according to some \textbf{transcendent standard}; (2016b: 24).
\end{myquote}

Contradicting his own statement discussed above, making \textit{sense} of texts is not about a “transcendent standard” or some universal approach, implying the impossibility of a theory. Thus, we are again left with \textit{interpretation} which can only be of a personal nature.

\vspace{-.3cm}

\section*{2.2 {\it {\bfseries Nirṇaya}}}\index{nirnaya@\textit{nirṇaya}}

The contradictory statements of Pollock\index{Pollock, Sheldon} are summarized here.

\begin{enumerate}
\itemsep=0pt
\item Historically, philology was equated with \textit{interpretation} and never theorized.

 \item Pollock offers no “grand theory”, but only his own practice.

 \item Philology is then said to have developed a theory globally contradicting [1].

 \item Finally, philology cannot have standard (or a theory) contradicting [3].

\end{enumerate}

In conclusion, a historical survey showed that since its inception, philology was always equated with \textit{interpretation} which could not be defined or theorized as it was of a personal nature. Pollock is no exception to this tradition which relies more on \textit{interpretation} rather than the \textit{pramāṇa–}s\index{pramana@\textit{pramāṇa}} (means of valid knowledge). The attempt to theorize would only be to impress administrators in an effort to save this dying field. The three–dimensional philology, on analysis, would only be a single dimensional view of Pollock. This emphasis on \textit{interpretation} will be noticed in all his works and is the essential tool in understanding Sanskrit texts. Attempting to “consciously construct” a three–dimensional philology would not change the inherited tendency of \textit{interpretation} superseding all the \textit{pramāṇa–}s\index{pramana@\textit{pramāṇa}}.

\vspace{-.3cm}

\section*{2.3 Scope of Philology}

\textit{Interpretation} was considered the essence of philology and its application to Indian Sanskrit traditions is discussed in following sections. Pollock’s\index{Pollock, Sheldon} philology\index{philology} belongs to the post–\textit{Orientalism} period which has been intent on “exhuming” various theories from within Sanskrit texts. Even though Pollock has been active and practising this interpretative method for the last four decades, it has taken a greater significance in recent times. Philology is a dying humanistic science that is now sought to be revived. A project called \textit{Zukunftsphilologie} (future philology)\index{Future Philology} aimed at rejuvenating the field has been initiated. The following passage broadly outlines the direction of this philology:

\begin{myquote}
But if such efforts are to be sustained, philologists must develop a new disciplinary formation, with a new intellectual core...texts exist in \textbf{social} and \textbf{political contexts}, after all–but they need to be complemented by a structure that acknowledges what unifies philologists, encourages comparison and synthesis of diverse traditions and their interpretive multiplicity, and \textbf{fosters larger generalization} from particular cases. It is through the disciplinization of philology that its real intellectual contribution–as the basic \textbf{science} of the humanities–can be realized. (2015a: 35)
\end{myquote}

The two important aspects in this passage that illustrates Pollock’s overall methodology are “political contexts” and “larger generalization.” Using Western political models to interpret Indian intellectual traditions is the basis of all the works beginning with the 1985 paper on \textit{śāstra}\index{sastra@\textit{śāstra}} (the significance of this date in Pollock’s intellectual history is discussed later) and this aspect is explored in this section. The introduction to the recently published \textit{World Philology} outlines the new direction of this interpretative philology that attempts at larger generalizations. The basic premise of \textit{Zukunftsphilologie} is that there were parallel and strikingly similar developments of philology in India, China and the West. But as previously elaborated in the historical introduction, philology as perceived by its own practitioners had no theoretical basis and essentially meant \textit{interpretation} since its inception in 1800 C.E. So, there could be nothing in philology (and in the European intellectual traditions) that was even remotely similar to śāstric traditions of Vyākaraṇa\index{Vyakarana@Vyākaraṇa}, Alaṅkāra\index{Alankara@Alaṅkāra} and Mīmāṁsā\index{Mimamsa@Mīmāṁsā}. Thus, Pollock’s\index{Pollock, Sheldon} theory of parallel development of philology in various traditions across the world has no basis in reality and cannot be substantiated by any \textit{pramāṇa}–s\index{pramana@\textit{pramāṇa}}. The introduction briefly mentioned Sanskrit’s profound influence on the West, whereas this theory of parallel traditions would deny any such influence. Pollock’s recent efforts are directed in making philology a branch of knowledge similar to that of science or mathematics so that its place in the university system could be secured. But philology’s most vital component was the undefined \textit{interpretation} which meant that it could never become a science or \textit{śāstra}\index{sastra@\textit{śāstra}}. This undefined nature of approaching and understanding texts can be clearly perceived when analysing Pollock’s own philological method.

Pollockian philology\index{philology} is based on deductive techniques (application of theories, always political, in assessing Indian tradition) with complete disregard for \textit{pramāṇa}–s; and the conclusions reached from applying such theories are diametrically opposed to how the Indian tradition perceived itself. An analysis of the scope of philology reveals the philological method\endnote{ The term “three–dimensional philology” is a recent construction and is not seen before 2009. Since 1985, the phrase “how the Indian tradition viewed itself” is frequently used.} used since the 1985 paper “The Theory of Practice and Practice of Theory in Indian Intellectual History”. The term philology itself is now applied to Indian Sanskrit traditions.

\begin{myquote}
The Indian tradition of “philology”–the term must finally be redefined to comprise the practices of making sense of texts, and so as to recover its true disciplinary dignity–is among the richest one in the world, and it therefore offers the richest archive anywhere in service of a \textbf{global theory} of philology. (2011f: 441)
\end{myquote}

In spite of the failing to formulate (in the last two–hundred years) even a basic theory for Western philology, it is now proposed to formulate a “global theory” for making \textit{sense} of texts across a time frame of several thousand years. This statement, however, contradicts a later statement that making \textit{sense} of texts is not about a “global theory.”

\begin{myquote}
…definition of philology–one that demands, not a specific set of \textbf{methodological or theoretical features} invariable across all time and space, but the broader concern with making sense of texts. (2015g: 114)
\end{myquote}

%~ \newpage

From a theory–less philology for two–hundred years, there now is a sudden effort to form a “global theory” in 2011; and by 2015 making \textit{sense} of texts is not about theory.~When it has been impossible to theorize philology in the Western context even by its most ardent adherents, how could one even think of formulating a “global theory?”~Then, the effort at global theorizing is dropped, and a broader humanistic concern becomes the focus of philology. Pollock’s\index{Pollock, Sheldon} confusion in explaining even basic issues is a reflection of philology’s untheorized nature. Even the term philology, which is now applied to Indian Sanskrit tradition, is explained in contradictory ways. Untheorized philology when applied in the Indian context becomes the “queen of the disciplines” covering a variety of \textit{śāstra}–s\index{sastra@\textit{śāstra}}.

\begin{myquote}
...in Hindustan, where \textbf{philology}—rather than mathematics or theology—had always been the queen of the disciplines and where as a result analyses of \textbf{grammar}, \textbf{rhetoric}, and \textbf{hermeneutics} were produced that were the most sophisticated in the ancient world. (2009b: 939)
\end{myquote}

The Indian śāstric traditions of grammar or Vyākaraṇa, rhetoric or Alaṅkāra\index{Alankara@Alaṅkāra} and hermeneutics or Mīmāṁsā\index{Mimamsa@Mīmāṁsā} are collectively considered as philology and that a history of Indian philology\endnote{ “A comprehensive account of the history of Sanskrit philology in that broad sense would address not only grammar but also lexicography, metrics, rhetoric (Alaṅkāraśāstra), and hermeneutics (Mīmāṁsā), among other things, all richly developed to a degree of complexity virtually unknown elsewhere in the ancient world.” (2015g: 115)} would include an analysis of these three\textit{ śāstra}–s. Thus, philology\index{philology} is equated with these \textit{śāstra}–s even though Indian tradition considers them as separate sciences. After defining Indian philology as including these śāstric traditions, it is declared that philology never existed in India!

\begin{myquote}
What is peculiar about the Indian case, given the case of argumentative Indian who above all else were very argumentative about language itself, is that philology was never explicitly theorized as a practice. Language analysis certainly was there, philosophy of language, too, and discourse analysis...and all well and clearly thematized. (2011f: 441)
\end{myquote}

Indian philology initially included the three \textit{śāstra}–s of Vyākaraṇa, Alaṅkāra and Mīmāṁsā, but now it is stated that philology does not encompass them, was never defined and no equivalent word exists in Sanskrit. It is further stated that “philology” as practised by the Indian Sanskrit scholars never became a \textit{śāstra} and thus never achieved the status of a \textit{vidyā–sthāna}\index{vidya-sthana@\textit{vidyā-sthāna}} (branch of knowledge). An earlier explanation of philology includes the three \textit{śāstra}–s (2009), while a later one (2011) excludes them. Such fundamental contradiction illustrates the reason why philology had declined as an academic subject.

%~ \newpage

\begin{myquote}
But most of the practices we think of as \textbf{philological} and that Indians cultivated as seriously as anyone in the premodern world, among them the constitution of texts and their correction...were \textbf{never thematized} at all; (2011: 441)
\end{myquote}

As to what these philological practices entailed (other than commentators indicating various readings) is nowhere described. Pollock\index{Pollock, Sheldon} cites the works of Aruṇagirinātha\index{Arunagirinatha@Aruṇagirinātha}, Sāyaṇa\index{Sayana@Sāyaṇa} and Madhvācārya\index{Madhvacarya@Madhvācārya} in “What was Philology in Sanskrit?”\textit{,} but the methods of these traditional commentators are very much in accordance with Sanskrit \textit{vyākhyā} (commentarial) tradition, which have nothing in common with Pollock’s own philology\index{philology}. For example, all these commentators directly comment on the Sanskrit texts (\textit{kāvya}–s and the \textit{Veda}–s\index{Veda-s}) with the help of various\textit{ śāstra}–s\index{sastra@\textit{śāstra}}, but Pollockian philology disregards Sanskrit texts and tries to enter and interpret the Indian mind. This can be known by a very significant admission which sheds light on the three–dimensional philology’s methodology–in the Indian tradition, the practices of philology “\textit{were never thematised}” and “\textit{was never explicitly theorized}”. This means that Indians never recorded or documented their philological practices, i.e., they never put it in writing. If it was never put in writing, how does Pollock know about them in the first place? He himself provides the answer to this important question.

\begin{myquote}
...indeed, the nature of literacy and its consequences were almost \textbf{completely ignored}...and this is a society that produced \textbf{millions upon millions} of written texts. What this means is that if we are to understand the principles of India’s philology will have to \textbf{deduce} them from the \textbf{empirical stuff} itself. (2011: 441)
\end{myquote}

A careful reading of the above passage indicates the deductive\endnote{ Deduction is a form of inference or \textit{anumāna} used in the west which has no basis in \textit{pratyakṣa} or \textit{śabda}–\textit{pramāṇa}–s. This form of inference is a result of \textit{a priori} application of Western theories and is not used in Indian traditions.} technique used by Pollock in understanding Sanskrit texts. The “nature of literacy and its consequences” refers to why texts were written rather than what was written in the texts. For example, it is claimed that the \textit{Rāmāyaṇa}\index{Ramayana@\textit{Rāmāyaṇa}} was written partly as a reaction to Buddhism\index{Buddhism/Buddhist} even though there is no such mention in any of the 24,000 \textit{śloka}-s. Again, the \textit{Rāmāyaṇa} is placed after the period of Aśhoka\index{Asoka@Aśoka} to show political influence, while Vālmīki\index{Valmiki@Vālmīki} himself says nothing about the Mauryan king. Thus, even though “millions and millions” of texts were written in Sanskrit and are available in printed and manuscript form, there would be nothing in these texts that directly support the theories of Pollockian philology. In essence, there is no \textit{śabda}–\textit{pramāṇa}\index{pramana@\textit{pramāṇa}}\index{sabda@\textit{śabda}} (nothing within the Sanskrit texts) to support any thesis of Pollock, and by his own admission. The “empirical stuff” does not refer to anything known by \textit{pramāṇa}-s\index{pramana@\textit{pramāṇa}} but to Western theories. When something is “completely ignored” in Sanskrit texts, then the “empirical stuff” cannot be Sanskrit texts themselves. It would have to be something external and thus the deduction would have to be from non–Sanskrit sources.

What kind of ideas would be used to evaluate Indian intellectual traditions? Culture and power are inseparable, and this is made clear in several papers.

\begin{myquote}
Although it may not always be possible to draw a perfectly straight–line between a philological method and a critical theory of culture and power, there is nothing odd in suggesting that \textbf{philology has political projects to achieve} and political lessons to teach...” (2008a:52–53)
\end{myquote}

Thus, to summarize, Pollockian philology\index{philology} would always be \textit{interpretation} (undefined and untheorized) of a personal nature, and it would always be political, and all of this would have no basis in the Indian tradition itself.

\vspace{-.3cm}

\section*{2.4 {\it {\bfseries Nirṇaya}}}\index{nirnaya@\textit{nirṇaya}}

\vspace{-.2cm}

The summary below shows how the political aspect is foisted on Indian tradition.

\begin{enumerate}
\itemsep=0pt
\item Philology is undefined and untheorized since its birth: it is of a personal nature

 \item Texts exists in political contexts, so philology now has a political aspect.

 \item The term philology is applied to Indian tradition, but it does not include any\textit{ śāstra}\index{sastra@\textit{śāstra}}.

 \item Philology was never thematised or theorized in Indian tradition, i.e., Sanskrit texts do not mention any political aspects.

 \item Pollock\index{Pollock, Sheldon} theorizes that Indian philology also has embedded political aspects.

\end{enumerate}

As illustrated previously, Pollock’s philology or the method used in making \textit{sense} of texts is “\textit{interpretation}” which is undefined and has no set of rules that could be applied universally. This is to be expected since philology as practised by all the earlier adherents was always about \textit{interpretation} and nothing more.

Philology is defined as making sense of texts\endnote{ This definition is first seen in the piece on D. D. Kosambi: “...the fullest use of the most human attribute, language, which occurs in the making sense of texts.” (2008a: 52). The same is seen in later writings on philology: “...the discipline of making sense of texts.” (2009b: 934), “the discipline of making sense of texts." (2014c: 398), (2015a: 34). “...but the broader concern with \textit{making sense of texts.}” (2015g: 114). “…the discipline concerned with making sense of texts.” (2018a: 125).} and Pollock\index{Pollock, Sheldon} constructs a three–dimensional model of philology, a rather simplistic method with no definite rules for understanding all classical literature including Sanskrit texts.

\begin{myquote}
…three different planes, that of (1) the text’s genesis; (2) its earlier readers; (3) me reading here and now. And this suggests that there are three dimensions of meaning—the author’s, the tradition’s, and my own…(2016b: 20)
\end{myquote}

The three–dimensions of Pollockian philology\index{philology} are:

\vspace{-.2cm}

\begin{enumerate}
\itemsep=0pt
\item \textbf{Historical} – This is the meaning of a text according to the author: for example, the meaning of the \textit{Rāmāyaṇa}\index{Ramayana@\textit{Rāmāyaṇa}} as intended by Vālmīki\index{Valmiki@Vālmīki}. The meaning along this plane should have the text itself as its basis, which can then be supported by other historical evidences.

 \item \textbf{Traditional} – This is the meaning of a text as understood by the readers: the meaning of Vālmīki’s \textit{Rāmāyaṇa} as understood by the commentators (tradition). Pollock’s claim is that along this plane, he is only presenting the Indian view in all his works.

 \item \textbf{Personal} – The meaning of the text according to the present–day reader. This is Pollock’s own interpretation of the \textit{Rāmāyaṇa} and other Sanskrit texts.

\end{enumerate}

\vspace{-.2cm}

These three dimensions are analysed in later sections [4.2 and 4.3]. But what does ‘making sense of texts’ actually consist of? After more than ten papers on philology and some two–hundred pages of much repetition, Pollock’s finally admits that there is no profound theory to offer, but only his own autobiography.

\begin{myquote}
Answering this question would seem to call for some heavyweight philological theory, but what I want to offer instead is rather \textbf{lightweight autobiography}… (2014c: 399) …indeed, what I offer you this evening is less a grand theory of reading than a \textbf{mere autobiography} of reading, made up of my three lives. (2016b: 399)
\end{myquote}

In continuing with philology’s long and unbroken tradition of theory–less \textit{interpretation}, Pollock neither offers a “heavyweight” nor a “grand” philological theory. The three dimensions of making sense of Sanskrit texts would in reality be a one–dimensional personalist understanding of Indian tradition and a single (and always political) life masquerading as three.

Pollock’s\index{Pollock, Sheldon} philological works offer a “lightweight autobiography” of how he had to reconcile conflicting modes of \textit{interpretation} along the three dimensions, and thus this \textit{prabandha} also offers a “lightweight biography” of how he does so in reality.

The \textit{Nyāya} framework is used for a \textit{parīkṣā}\index{pariksa@\textit{parīkṣā}} of the three–dimensional philology. Nyāya–śāstra\index{sastra@\textit{śāstra}} is necessary and essential to counter the Western deductive model since \textit{Nyāya} and all Indian śāstric traditions are based on \textit{pramāṇa}-s\index{pramana@\textit{pramāṇa}} for establishing their respective \textit{artha}-s\index{artha@\textit{artha}} (objects). The historical introduction showed that philology\index{philology} is \textit{interpretation}.\break From the Nyāya perspective, \textit{interpretation }(or deduction) is an incorrect form of \textit{anumāna}\index{anumana@\textit{anumāna}} (inference) from Western theoretical models. If one were to continue the debate on the deductive paradigm, then it would lead to a complete misunderstanding of Indian traditions which is what has happened for the last two–hundred years.

\vspace{-.3cm}

\section*{2.5 {\it {\bfseries Nyāya–śāstra Bhūmikā}}}

\vspace{-.2cm}

\textit{Bhāratīya Jñāna Paramparā} (Indian knowledge tradition) is based on the \textit{anubhava}-s or the inner experiences of \textit{ṛṣi}-s (including those of the Buddha\index{Buddha, the} and Mahāvīra\index{Mahavira@Mahāvīra}). Every generation was inspired to produce a vast body of literature expounding these \textit{anubhava}–s and its understanding and interpretation varied according to time and place. These oral\index{oral tradition} and written literary traditions spanning a period of several thousand years were not only in Sanskrit but also in the other languages such as Prakrit and Tamil.

The Indian knowledge tradition was divided into the fourteen \textit{vidyā–sthāna}–s\index{vidya-sthana@\textit{vidyā-sthāna}} and a well–known \textit{śloka} from \textit{Viṣṇupurāṇa}\index{Visnupurana@\textit{Viṣṇupurāṇa}} (3.6.27) illustrates the basic framework.

\vspace{-.3cm}

\begin{verse}
\textit{aṅgāni vedāś catvāro mīmāṁsā–nyāya–vistaraḥ | }\\\textit{purāṇaṁ dharmaśāstraṁ ca vidyā hy etāś caturdaśa ||}
\end{verse}

\vspace{-.3cm}

The fourteen \textit{vidyā–sthāna}–s include the four \textit{Veda}–s\index{Veda-s} along with the six \textit{Vedāṅga}–s\endnote{ The six are the following: \textit{śikṣā}, \textit{vyākaraṇa}, \textit{chandas}, \textit{nirukta}, \textit{kalpa} and \textit{jyotiṣa}.}, Mīmāṁsā\index{Mimamsa@Mīmāṁsā}, \textit{Nyāyavistara}, \textit{Purāṇa}–s and Dharmaśāstra.

%~ \newpage

\textit{Yājñavalkya–smṛti}\index{smrti@\textit{smṛti}}\index{Yajnavalkya-smrti@\textit{Yājñavalkya-smṛti}}(1.3) also has a similar division of \textit{vidyā–sthāna}–s:\index{vidya-sthana@\textit{vidyā-sthāna}}

%~ \index{Yajnavalkya-smrti@\textit{Yājñavalkya-smṛti}}

\vspace{-.3cm}

\begin{verse}
\textit{purāṇa–nyāya–mīmāṁsā\index{Mimamsa@Mīmāṁsā}–dharmaśāstrāṅga–miśritāḥ |}\\\textit{vedāḥ sthānāni vidyānāṁ dharmasya ca caturdaśa ||}
\end{verse}

\vspace{-.3cm}

The \textit{vidyā–sthāna}–s mentioned in the \textit{Viṣṇupurāṇa}\index{Visnupurana@\textit{Viṣṇupurāṇa}} are all mentioned by Yājñavalkya\index{Yajnavalkya@Yājñavalkya} Maharṣi and they are also referred to as \textit{dharma–sthāna}–s.\break Thus, each \textit{vidyā–sthāna} is also a \textit{dharma–sthāna}: knowledge is based on \textit{dharma} and knowledge is also bounded by \textit{dharma}.~These \textit{vidyā–sthāna}–s have been transmitted both orally\index{oral tradition} and textually from very early times\endnote{ In this \textit{prabandha}, issues related to the dating of Indian tradition are not discussed as this is only a \textit{pramāṇa-parīkṣā} of Pollock’s three–dimensional philology. However, dating is vital to many of Pollock’s theories.} and still have great relevance to modernity. Mahāmahopādhyāya N.S. Ramanuja Tatacharya\index{Tatacharya, N. S. Ramanuja} (1928-2017), a great Naiyāyika representing the living traditions of \textit{Bhāratīya Jñāna Paramparā}, commented\endnote{ \textit{Nyāya–pariśuddhi} of Vedāntadeśika with \textit{Nyāya–sārāsvādinī} of N.S. Ramanuja Tatacharya, Vol.1. (Chennai: Srirangam Srimad Andavan Ashramam, 2011):\textit{ prastāvanā, }ix.} on these \textit{śloka}–s: the fourteen branches are the basis of\textit{ vidyā }or knowledge; these fourteen are also the basis of \textit{dharma }which helps in attaining the four \textit{puruṣārtha}–s.

A knowledge tradition based on revealed texts and \textit{śāstra}–s\index{sastra@\textit{śāstra}} to understand them is by no means exclusive to the Hindu traditions. The Buddhist\index{Buddhism/Buddhist} and Jaina\index{Jaina} traditions too have “revealed sources” such as the \textit{vacana}–s of the Buddha\index{Buddha, the} and Mahāvīra\index{Mahavira@Mahāvīra}, and there is a large body of literature in Sanskrit, Prakrit and Pali to understand them. All the three traditions are part of \textit{Sanātana Dharma} and the close connection between them can be seen by the fact that the \textit{Rāmāyaṇa}\index{Ramayana@\textit{Rāmāyaṇa}} story appears in the Buddhist \textit{Jātaka}–s\index{Jataka@\textit{Jātaka}} and also in the Jaina Apabhraṁśa and Prakrit literature. Similarly, Nyāya–śāstra was studied and adapted in all three traditions.

Among the fourteen \textit{vidyā–sthāna}–s, \textit{Nyāyavistara} or Nyāya–śāstra is an enquiry into \textit{pramāṇa–}s\index{pramana@\textit{pramāṇa}} and their validity. Gautama’s\index{Gautama} \textit{Nyāya–sūtra}\index{Nyaya-sutra@\textit{Nyāya-sūtra}} (which had earlier sources and can be traced back to the \textit{Veda}–s\index{Veda-s}) is the basis of Nyāya–śāstra which was commented upon and adapted by many \textit{acārya}–s, including those of the Buddhist and Jaina traditions, resulting in the composition of innumerable texts. This long and unbroken \textit{Nyāya paramparā} continues even today.

Vātsyāyana\index{Vatsyayana@Vātsyāyana} composed a profound commentary on the \textit{Nyāya–sūtra} called \textit{Nyāya–bhāṣya}\index{Nyaya-bhasya@\textit{Nyāya-bhāṣya}} which contains a deep observation on our (which includes all living beings as indicated by the word \textit{prāṇa–bhṛt}) relationship with the world around us.

%~ \newpage

The \textit{Nyāya–sūtra}\index{Nyaya-sutra@\textit{Nyāya-sūtra}} contains more than five–hundred \textit{sūtra–}s, and along with the \textit{Nyāya–bhāṣya,}\index{Nyaya-bhasya@\textit{Nyāya-bhāṣya}} it is primarily concerned with the validity of \textit{pramāṇa–}s\index{pramana@\textit{pramāṇa}} in acquiring knowledge. The first two \textit{sutra}–s contains the essence and the scope of Nyāya–śāstra\index{sastra@\textit{śāstra}}. The profound observation at the beginning of the \textit{Nyāya–bhāṣya} can be considered as an introduction to all śāstric traditions and is discussed in section [3.1]. The very first \textit{vākya} of the \textit{Nyāya–bhāṣya} (p.1) is as follows\endnote{ Traditionally commentators begin a text with a \textit{maṅgala} for auspiciousness. In some printed books, \textit{Nyāya–bhāṣya} begins with the statement ‘\textit{oṁ namaḥ pramāṇāya}’ whereas the critical edition of the great Naiyāyika Anantalal Thakur does not include this. The inclusion of this \textit{vākya} in the beginning would be very appropriate since this text is about the validity of \textit{pramāṇa–}s and thus a ‘prostration to \textit{pramāṇa}’ would be suitable. Even otherwise, beginning the text with the statement ‘\textit{pramāṇataḥ}’ would be apt since that is the primary subject of \textit{Nyāya–bhāṣya}. Even though Anantalal Thakur’s critical edition does not include \textit{oṁ namaḥ pramāṇāya}, oral traditions play an important role and even if the manuscript does not have a certain reading, the authentic \textit{guru paramparā} could have possibly preserved it.}:

\vspace{-.3cm}

\begin{verse}
\textit{pramāṇato'rtha–pratipattau pravṛtti\index{pravrtti@\textit{pravṛtti}}–sāmarthyād\index{samarthya@\textit{sāmarthya}} arthavat\\ pramāṇam |}
\end{verse}

\vspace{-.3cm}

\section*{2.6 The Definition of Philology}

\vspace{-.2cm}

The validity of Sanskrit oral\index{oral tradition} tradition is completely denied in the works of Pollock\index{Pollock, Sheldon}. This, however, is only done for Indian tradition, whereas when defining philology, oral traditions are accepted as an object of philology.

\begin{myquote}
Philology has such an object, namely language as concretized in texts—all texts, “everything made of language” (Sanskrit’s lovely term \textit{vāṅmaya}), whether the texts are \textbf{oral}, written, printed, or electronic; expressive or prosaic; ancient or contemporary. (2016b: 16)
\end{myquote}

Again, in \textit{Philologia Rediviva}, philology\index{philology} includes oral traditions.

\begin{myquote}
These scholars were all philologists: contributors to the discipline of making sense of texts–all texts–whether \textbf{oral}, written, printed, or electronic, whether literary, religious, or legal, those of mass culture no less than those of elite culture.” (2015a: 34)
\end{myquote}

Even electronic texts, or whatever one may utter, is considered to come under the purview of philology. In practice though, the validity of Indian oral tradition is completely denied, and they are never considered as a \textit{pramāṇa} (oral traditions would be \textit{śabda}\index{sabda@\textit{śabda}}–\textit{pramāṇa}) as they should be. Further, Pollock claims to be an heir to the “brilliant tradition” of Sanskrit (that he has written nothing in Sanskrit and probably has no such capacity should not be considered a disqualification) and that he studied with many traditional scholars such as K. Krishnamoorthy\index{Krishnamoorthy, K.} (1923-1997), Pattabhirama\index{Sastry, Pattabhirama} Sastry (1908-1992) and Venkatachala\index{Sastry, Venkatachala} Sastry (1933-). But his theories (drawn from Western sources) on \textit{Rasa}\index{rasa@\textit{rasa}}, Mīmāṁsā\index{Mimamsa@Mīmāṁsā} and Kannada language goes “through” or beyond what is found in the texts themselves and these certainly would not have been taught by these scholars as we do not find it in their own writings. Again, in theory, the importance of oral\index{oral tradition} tradition is emphasized.

\begin{myquote}
More specific to India has been the \textbf{marginalization} of \textit{authentic} scholar–practitioners [pandits], the people from whom I myself was fortunate to learn and with whose disappearance one of the grandest traditions of scholarship the world has ever known is disappearing. (2016d: 927)
\end{myquote}

But what would these pundits say about the death of Sanskrit or grammar and its relation to political power? In the last forty–years, Pollock\index{Pollock, Sheldon} has never bothered to ask pandits any such questions and never will as he has considered Sanskrit to be “dead” in a historical sense. This means that even though there are traditional pundits today, their own self–understanding died (around 1800 C.E); and thus, for example, they would not be able to connect grammar with the political. Pundits have no understanding of their own grand tradition, while Pollock has an intuitive understanding of it! Thus, the living oral traditions are marginalized, even though it is depicted otherwise.

And what would these marginalized pundits say about his \textit{interpretation} of the \textit{Rāmāyaṇa}?\index{Ramayana@\textit{Rāmāyaṇa}} Oral traditions are not only disregarded, but some of Pollock’s theories are in direct conflict with what even Sanskrit texts say about orality. His own conflicting views on the \textit{Rāmāyaṇa }show how \textit{interpretation }(as explained in the introduction) has greater significance than even that of tradition (\textit{śabda}–\textit{pramāṇa}\index{pramana@\textit{pramāṇa}}\index{sabda@\textit{śabda}}). In an early paper, \textit{The Rāmāyaṇa Text and the Critical Edition}, the oral and written traditions of the \textit{Rāmāyaṇa} are discussed.

\begin{myquote}
The tradition thus represents Vālmīki’s\index{Valmiki@Vālmīki} \textit{Rāmāyaṇa }as an individual artistic elaboration of a pre–existing narrative, \textbf{composed and transmitted orally} in more or less memorized form. There is little in this account that is not in keeping with the unitary character of most of the poem and with what we can infer about its sources. That the \textit{Rāmāyaṇa }is an oral composition has now been \textbf{statistically demonstrated}, and indeed, as we shall see, our manuscript evidence implies a long antecedent period of oral transmission. (1984a: 82–83)
\end{myquote}

On grounds of traditional understanding, manuscript evidence, and modern research (statistical analysis), Pollock states that we can safely infer that the \textit{ādikāvya} was indeed an oral composition. In his introduction to the Ayodhyākāṇḍa\index{Ayodhyakanda@Ayodhyākāṇḍa}, Pollock has stated several times the oral nature the \textit{ādikāvya}.

\begin{myquote}
In the course of their [Indian epics] transmission, in \textbf{oral} and afterwards in written form…(1986b: 9). It [\textit{Ayodhyākāṇḍa}]\index{Ayodhyakanda@Ayodhyākāṇḍa} is in large part an \textbf{oral} composition…(1986b: 76)
\end{myquote}

Similarly, in his introduction to the Araṇyakāṇḍa\index{Aranyakanda@Araṇyakāṇḍa}, orality of the \textit{ādikāvya} is explicitly stated: “Perhaps as much as one–quarter of this vulgate did not form part of the monumental \textbf{oral poem} of Vālmīki.”\index{Valmiki@Vālmīki} (Pollock\index{Pollock, Sheldon} 1991b: 17) And further, the earlier papers do not equate the \textit{ādikāvya} with invention of writing.

\begin{myquote}
The history of \textit{Rāmāyaṇa}\index{Ramayana@\textit{Rāmāyaṇa}}\textbf{in its written form} effectively commences in the \textbf{eleventh century}. The probable date of our earliest exemplar, a palm leaf manuscript from Nepal\; representing\; the northwest tradition,\newpage is A.D. 1020. No earlier manuscript fragments have been discovered. (1984a: 82–83)
\end{myquote}

As manuscripts are found only from the eleventh century onwards, it is inferred that the \textit{Rāmāyaṇa }was first put in written form just at that time. Hence, in the earlier (1984) \textit{interpretation}, the \textit{Rāmāyaṇa }was an orally\index{oral tradition} transmitted text at least till the tenth century and was written down in the eleventh century. However, later interpretations are altered drastically to forcefully fit his fanciful theories. Almost every topic becomes “complex and theory laden”.

\begin{myquote}
It is \textbf{no simple thing}, however, to identify what is first about the first poem...Another dimension of newness may lie in its being one of the first major texts \textbf{committed to writing} after the invention of writing in the mid–third century B.C.E... The carefully constructed image of a purely oral culture in the prelude—\textbf{a text unquestionably dated later} than the main body of the work—\textbf{cannot mean} what it literally says. When Vālmīki is shown to compose his poem after meditating and to transmit it orally to two young singers, who learn and perform it exactly as he taught it to them, we are being given\textbf{ not a realist depiction }but a sentimental “fiction of written culture.” (2006a: 78)
\end{myquote}

In the earlier paper, it was stated that we had a good knowledge regarding the orality of the text, whereas now its origin has become complex. Manuscript evidence was previously used to establish that the \textit{Rāmāyaṇa} took a written form in the eleventh century, but this view is drastically altered to fit his theory that writing began in the beginning of the first millennium. In opposition to the \textit{śabda–pramāṇa}\index{pramana@\textit{pramāṇa}}\index{sabda@\textit{śabda}} (textual testimony) and his own thesis in the earlier paper, it is now stated that the \textit{Rāmāyaṇa }was committed to writing before the beginning of the first millennium. Further, the “invention of writing,” an event that is neither recorded in the \textit{Rāmāyaṇa }nor in any other Sanskrit text, is now established by inferential evidence such as the appearance of Sanskrit inscriptions. The \textit{Rāmāyaṇa} (1.4.1–3) clearly states that Vālmīki\index{Valmiki@Vālmīki} taught it orally to Lava and Kuśa, which is now interpreted as “fictional.”

\begin{verse}
\textit{cakāra caritaṁ kṛtsnaṁ vicitra–padam arthavat |}\\\textit{caturviṁśat–sahasrāṇi ślokānāṁ \textbf{uktavānṛṣiḥ} ||}\\\textit{kṛtvāpi sa mahāprājñaḥ sabhaviṣyaṁ sahottaram |}\\\textit{cintayāmāsa ko nvetat \textbf{prayuñjīyād } iti prabhuḥ || }
\end{verse}

\newpage

Vālmīki orally recited the complete \textit{Rāmāyaṇa}\index{Ramayana@\textit{Rāmāyaṇa}} of 24,000 \textit{śloka}–s, and both Lava and Kuśa were also taught orally to recite and sing the \textit{ādikāvya}. Therefore, all this is an instance of Pollock’s\index{Pollock, Sheldon} complete disregard for \textit{pramāṇa}–s\index{pramana@\textit{pramāṇa}} (textual evidence) and propounding theories that are diametrically opposed to them (and to his own earlier views).

In the 1984 paper on the \textit{Rāmāyaṇa}, the critical edition of the \textit{Rāmāyaṇa} (which included the prelude as part of the main text) was accepted as mostly authentic, whereas now disregarding the earlier view, the prelude (\textit{upodghāta}) is considered “unquestionably” later than the rest of work. One has to remember that Pollock was part of the Robert Goldman\index{Goldman, Robert} translation team and translated the Ayodhyākāṇḍa\index{Ayodhyakanda@Ayodhyākāṇḍa} and Araṇyakāṇḍa\index{Aranyakanda@Araṇyakāṇḍa}. It was considered that the critical version “puts us in possession of the most uniform, intelligible, and archaic version of the \textit{Vālmīki Rāmāyaṇa}” (1984a: 92) which includes the \textit{upodghāta}. This earlier view is now discarded as it is opposed to a later theory!

The commentaries on the\textit{ Rāmāyaṇa} including Govindarāja’s\index{Govindaraja@Govindarāja} \textit{Bhūṣaṇa}\index{Bhusana@\textit{Bhūṣaṇa}}, Nageśabhaṭṭa’s\index{Nagesabhatta@Nageśabhaṭṭa} \textit{Tilaka}\index{Tilaka@\textit{Tilaka}}, Maheśvaratīrtha’s\index{Mahesvaratirtha@Maheśvaratīrtha} \textit{Tattva-dīpikā}\index{Tattva-dipika@\textit{Tattva-dīpikā}}, Śivasahāya’s\index{Sivasahaya@Śivasahāya} \textit{Rāmāyaṇa–śiromaṇi}\index{Ramayana-siromani@\textit{Rāmāyaṇa-śiromaṇi}} and Mādhavayogīndra’s\index{Madhavayogindra@Mādhavayogīndra} \textit{Kataka}\index{Kataka@\textit{Kataka}} comment on the \textit{upodghāta, }and consider it to be a part of the main text. Even the \textit{pārāyaṇa} or recitation (representing the living oral\index{oral tradition} tradition) of the \textit{Rāmāyaṇa} is done along with the \textit{upodghāta, }and never without it, and thus by all traditional accounts the \textit{upodghāta }was always considered part of the \textit{Rāmāyaṇa }recitation since its composition. Again, if philology\index{philology} includes oral traditions, then Pollock could have asked any of the traditional reciters of the \textit{Rāmāyaṇa }(recitations of Vālmīki’s text are performed across the country, both publicly and privately) in order to ascertain if the\textit{ upodghāta }was part of the text. Obviously, Pollock has made no such effort throughout his academic career. Thus, including oral traditions while defining philology is purposefully and “carefully constructed” to mislead the University administrators, while in practice Indian oral\index{oral tradition} traditions are not only disregarded but Pollockian theories are directly opposed to them.

\textit{\textbf{Nirṇaya}}\index{nirnaya@\textit{nirṇaya}}

The above discussion is summarized to illustrate the contradictory nature of Pollockian philology\index{philology}. In the earlier papers:

\begin{enumerate}
\itemsep=0pt
\item The \textit{Rāmāyaṇa}\index{Ramayana@\textit{Rāmāyaṇa}} is an oral composition because it stated so in the text itself, and also as per statistical evidence.

 \item The \textit{Rāmāyaṇa }is written after the tenth century because of manuscript evidence.

 \item The prelude (\textit{upodghāta}) is part of the text as the critical edition includes it.

\end{enumerate}

In the later interpretation:

\begin{enumerate}
\itemsep=0pt
\item The \textit{Rāmāyaṇa}’s orality is fictional–which is against \textit{śabda–pramāṇa}\index{pramana@\textit{pramāṇa}}\index{sabda@\textit{śabda}} (textual evidence) and his own earlier view.

 \item The \textit{Rāmāyaṇa }is written at the beginning of the first millennium–which is opposed to both the manuscript tradition and his own earlier view.

 \item The prelude (\textit{upodghāta}) is a later composition–which is against both tradition and the critical edition (and thus his own earlier view).

\end{enumerate}

What would the marginalized pundits say about this \textit{interpretation} of the \textit{Rāmāyaṇa}? To forcefully establish theories, Pollock\index{Pollock, Sheldon} is not only willing to discard textual authority, but also to ignore contradictions that occur with his earlier views. The historical introduction documented that philology is \textit{interpretation} and that \textit{interpretation }always superseded the \textit{pramāṇa}–s. This aspect of philology is exemplified by the \textit{interpretation }of the \textit{Rāmāyaṇa}. This would be the essential method used from 1985: the primacy of \textit{interpretation} (application of theory) over what is known by \textit{pramāṇa}–s (both written texts and oral traditions).

Why is Pollock willing to discard not only textual evidence, but also his earlier views? And what is important about the year 1985? To answer these questions, it is essential to understand the dramatic shift that is evident in Pollock’s intellectual history. His works can generally be divided into two periods based on Edward Said’s\index{Said, Edward} \textit{Orientalism}. Said’s remarkable work succinctly portrayed the biases and the inherently political nature of Western understanding of the Orient and had a devastating effect on the humanities including Indology. Said and \textit{Orientalism} (published in 1978 which coincides with the beginning of Pollock’s\index{Pollock, Sheldon} academic career) left an indelible mark on his mind which continues to date. The early papers on the \textit{Rāmāyaṇa}\index{Ramayana@\textit{Rāmāyaṇa}} belong to the pre–\textit{Orientalism} or pre–Saidian period in which Western theories and models are readily applied in evaluating Sanskrit texts. The following passage is a sample from this period:

\begin{myquote}
Having assembled the essential building blocks, we are in a position to explore the \textbf{mythological map} of experience charted by the \textit{Rāmāyaṇa}, \textbf{to discover} what \textbf{Frye} might call the \textbf{myth’s} “authoritative social function,” how, that is, it “tells a culture what it is.” (1991a: 41)
\end{myquote}

The \textit{Rāmāyaṇa }is repeatedly referred to as mythology, and the views of Western theorists are applied and used to comprehend Sanskrit texts. Edward Said documented and critiqued such \textit{a priori} application of alien theories on other traditions. But even though theories were superimposed on Sanskrit texts, textual evidence was also included as \textit{pramāṇa}–s\index{pramana@\textit{pramāṇa}}, as shown in the previous discussion on the \textit{Rāmāyaṇa }(in Section 2.6). In the pre–\textit{Orientalism} period, Western theories were the basis for understanding Indian tradition with supporting statements from Sanskrit texts. This would be the normal Indological approach, but one that \textit{Orientalism }critiqued. In this period, there is no attempt to read “through” the texts or show as to how the Indian tradition viewed itself. In the post–\textit{Orientalism} period, the intent is to “exhume” power structures supposedly embedded within the Sanskrit texts. This drastic shift can be seen from his 1985 paper “The Theory of Practice and Practice of Theory in Indian Intellectual History”. Beginning with this paper, Pollock repeatedly states that his effort is to show how the Indian tradition viewed itself. Instead of acknowledging and rectifying the inherent Western bias, Said’s incisive criticism drives Pollock to start the process of “exhuming” Western models from deep within Sanskrit texts. This results in the forceful application of theories while discarding what is known through valid \textit{pramāṇa}–s (and sometimes his own earlier views). It should be noted that in the post–\textit{Orientalism} (from 1985) period, there would no\textit{ pramāṇa–}s\index{pramana@\textit{pramāṇa}} from Sanskrit texts to support any thesis statement (section 4.6 documents this aspect).

One must also carefully note the nonexistence of the three–dimensional philology\index{philology} in the pre–\textit{Orientalism} period even though it is claimed that the human mind naturally works in this way.

\begin{myquote}
Three–dimensional philology is actually the way human understanding works. (2014c: 41)
\end{myquote}

In conclusion, the frailty of the Western understanding of Indian tradition is exemplified by the approach taken by Pollock\index{Pollock, Sheldon}. \textit{Orientalism}, as Edward Said\index{Said, Edward} realized towards the end of his life, ultimately had little positive effect. But what he could not have known was that it also had a detrimental one, as in the case of our three–dimensional philologist. Pollock’s own statement–that he was “not aware that the author of \textit{Orientalism} ever tried to rein in the absurdities and abuses to which his theory gave rise” (Pollock 2009b: 960)–is applicable to himself! Said’s \textit{Return to Philology} perhaps should have been titled \textit{Return to Traditional Ways of Knowing}, as the only remedy to the deductive model would be the Nyāya–śāstra\index{sastra@\textit{śāstra}} which bases itself on the four \textit{pramāṇa}–s\break of \textit{pratyakṣa}\index{pratyaksa@\textit{pratyakṣa}} (perception), \textit{anumāna}\index{anumana@\textit{anumāna}} (inference), \textit{upamāna}\index{upamana@\textit{upamāna}} (analogy) and \textit{śabda} (authoritative testimony). Only an approach based on the \textit{pramāṇa}–s can end Pollock’s forty–years of and philology’s two–hundred years of “search for method” and theory for understanding texts.

