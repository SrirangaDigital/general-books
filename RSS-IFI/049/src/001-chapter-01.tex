
\chapter{Introduction}\label{chapter1}

\textit{Nyāya–śāstra}, also known as \textit{pramāṇa–śāstra}, is primarily an enquiry into the means of valid knowledge. \textit{Pramāṇa–}s are the natural means of obtaining knowledge of the world around us. \textit{Nyāya–sūtra} of Gautama Maharṣi (circa) is the oldest text available that makes a profound inquiry into the means of obtaining such valid knowledge. \textit{Nyāya–bhāṣya} of Vātsyāyana (circa) is an equally profound commentary on the \textit{sūtra}–s in lucid Sanskrit, a style that is similar to many of the early śāstric texts such as Patañjali’s \textit{Mahābhāṣya} and Śabarasvāmin’s \textit{Mīmāṁsā–bhāṣya}. Both \textit{Nyāya–sūtra }(abbreviated as \textit{\textbf{NS}}) and \textit{Nyāya–bhāṣya} have great relevance even today as they provide a dhārmic basis and framework for knowing and acting accordingly. This \textit{dhārmic} aspect is seen in the very first \textit{sūtra} itself which states that a knowledge of \textit{pramāṇa}–s (means of knowing) and \textit{prameya}–s (knowable objects) ultimately leads one to \textit{niḥśreyasa} or ultimate good.

\section*{1.1 The Tattvajñāna Sūtra – NS 1.1.1}

\begin{myquote}
\textit{pramāṇa–prameya–saṁśaya–prayojana–dṛṣṭānta–siddhānta–avayava–tarka–\break nirṇaya–vāda–jalpa–vitaṇḍā–hetvābhāsa–cchala–jāti–nigraha–sthānānāṁ\break tattva–jñānān niḥśreyasādhigamaḥ |}
\end{myquote}

\textit{\textbf{Anuvāda}}

The \textit{tattva jñāna} (true knowledge) of these \textit{padārtha}–s (objects) leads one to \textit{niḥśreyasa} (ultimate good): \textit{pramāṇa} – means of valid knowledge; \textit{prameya} – objects of valid knowledge; \textit{saṁśaya} – doubt; \textit{prayojana} – incentive; \textit{dṛṣṭānta} – example; \textit{siddhānta} – established doctrine; \textit{avayava} – the inference components; \textit{tarka} – hypothetical argument; \textit{nirṇaya} – final conclusion; \textit{vāda} – discussion for the final conclusion; \textit{jalpa} – debate to establish one’s own position; \textit{vitaṇḍā} – destructive criticism; \textit{hetvābhāsa} – fallacies in establishing the \textit{hetu}; \textit{chala} – purposeful distortion of the opponent; \textit{jāti} – futile rejoinder based on mere similarity or dissimilarity; \textit{nigraha–sthāna} – point of defeat.

All the \textit{padārtha}–s necessary for this\textit{ śāstra} are enumerated above and the pertinent ones used in this \textit{prabandha} are explained in relevant sections. Vātsyāyana’s \textit{Nyāya–bhāṣya} glosses on this \textit{sūtra} in great detail and explains that the correct knowledge of \textit{ātman} leads one to \textit{mokṣa}. The knowledge of the \textit{padārtha}–s mentioned in the first \textit{sūtra} are greatly helpful in understanding other objects (such as \textit{ātman}) and thus are an indirect cause for \textit{mokṣa}.

Therefore, a proper knowledge of the \textit{padārtha}–s ultimately leads to \textit{mokṣa}. Having \textit{mokṣa} as the ultimate goal is common to all Indian knowledge systems, and thus all of them are bounded by this \textit{dharma}. But this aspect never limited the innovations, and all śāstric systems continued to produce new knowledge according to the needs of time and place. Their adaptability was, and still is, remarkable. We can observe this even today with the adaptation of \textit{Āyurveda} and \textit{Yoga} throughout the world. \textbf{The purpose of this }\textit{\textbf{prabandha}} is to show that \textit{Nyāya–śāstra} is relevant even today and can be used for a \textit{parīkṣā} or evaluation of any object including the works of Professor Sheldon Pollock (hereafter Pollock).The \textit{pramāṇa}–s of \textit{Nyāya–śāstra} are used in analysing Pollock’s three–dimensional philology\endnote{ This \textit{prabandha} is a \textit{parīkṣā} of Sheldon Pollock’s works within the framework of \textit{Nyāya śāstra}. Authors such as Edward Said and others are mentioned only to explain Pollock’s own references to them.} which, Pollock claims, lays the theoretical foundation for understanding Sanskrit texts\endnote{ Many traditional \textit{Nyāya} pundits consider the \textit{sūtra} and the \textit{bhāṣya} to be the most difficult and profound among the numerous texts in this \textit{śāstra}. My understanding of these two texts has primarily been guided by the brilliant and insightful Bengali translation and commentary titled \textit{Nyāyadarśana} by Mahāmahopādhyāya Phaṇibhūṣaṇa Tarkavāgīśa (CE 1875–1941). Bengal or \textit{Vaṅgapradeśa} was an important centre for the study of \textit{Nyāyaśāstra} and Śrī Phaṇibhuṣaṇa Tarkavāgīśa belonged to this long and unbroken tradition. His work was published in five volumes in the beginning of the last century. An abridged and free translation of this work has been ably done by Professors Debiprasad Chattopadhyaya and Mrinalkanti Gangopadhyaya and published in several volumes. I have followed their translation of the \textit{Nyāya–sūtra} and \textit{bhāṣya} with some modifications. Their translation of Śrī Phaṇibhūṣaṇa commentary is also followed in many places and thus my own translation is indebted to their work. The text of \textit{Nyāya–sūtra} and \textit{bhāṣya }is from the critical edition of Śrī Anantalal Thakur. The translators have given the dates according the Bengali calendar: Bengali Samvad. 1282–1348. One can arrive at the Common Era by adding 593. Also, many Sanskrit words are non–translatable and thus the meaning of such words are usually, but not always, given in brackets.}.


\section*{1.2 Format of the Prabandha}

In this \textit{prabandha}, four sections are used for presentation and analysis:

\textit{\textbf{Nyāya–sūtra/Nyāya–bhāṣya}} – The Sanskrit text of the relevant portion of the \textit{Nyāya–sūtra }and \textit{Nyāya–bhāṣya} is presented in this section.

\textit{\textbf{Anuvāda}} – This is the translation of the relevant \textit{sūtra }or the appropriate portion of the\textit{bhāṣya}. The translation by Debiprasad Chattopadhyaya and Mrinalkanti Gangopadhyaya is followed with some modifications.

\textit{\textbf{Vivaraṇa}} – This section is based on Mahāmahopādhyāya Phaṇibhūṣaṇa Tarkavāgīśa’s commentary which has been translated and summarized by the two aforementioned scholars. Again, their explanations are followed with modifications suggested by me.

\textbf{\textit{A Parīkṣā of 3D Philology}} – In this section, the relevant portion of Pollock’s work is analysed. Throughout this \textit{prabandha}, only the works of Pollock are examined to maintain focus. The three–dimensional philology is abbreviated as “3D Philology” for brevity. The “\textit{pūrvapakṣa}–\textit{siddhānta} style” is used for presentation.

\textit{\textbf{Pūrvapakṣa}} – Throughout this \textit{prabandha}, the view of the \textit{pūrvapakṣin} (viz. Pollock) is quoted from his works exactly and without paraphrasing so as to avoid misrepresentation. These portions are in a different font and are also indented to distinguish it from the regular text. In such sections, certain words or phrases are highlighted (bold font is used) for emphasis. \textbf{This emphasis is mine} and is not representative of the original text or article.

\textit{\textbf{Siddhānta}} – The views of the \textit{pūrvapakṣin} are then assessed and the viewpoint of the \textit{siddhāntin,} representing the Indian tradition, is presented. This is called \textit{nirṇaya} in \textit{Nyāya–sūtra} and is defined in the following section:


\section*{1.3 The Nirṇaya Sūtra – NS 1.1.41}

\begin{verse}
\textit{vimṛśya pakṣa–pratipakṣābhyām arthāvadhāraṇaṁ nirṇayaḥ |}
\end{verse}

\textit{Nirṇaya} (establishing the \textit{tattva}) is the \textit{avadhāraṇa} or determination of the right nature of an \textit{artha} (object) after \textit{vimarśa} (evaluating it through the establishment) of the \textit{pakṣa} (thesis) and the refutation of the \textit{pratipakṣa} (the position that is being evaluated).

\textit{\textbf{Vivaraṇa}}

When listening to a discussion, how does one come to a \textit{nirṇaya}? This is done through \textit{pakṣa} and \textit{pratipakṣa}. \textit{Pakṣa} and \textit{pratipakṣa} literally mean two contradictory characteristics stated as belonging to the same object. In a discussion, one claims, for example, that sound is eternal, while the other claims that sound is not eternal. Here, eternality and non–eternality as characterising sound are the \textit{pakṣa} and \textit{pratipakṣa}. Vātsyāyana further explains that \textit{pakṣa} means the arguments which establish one's own thesis and \textit{pratipakṣa }means the refutation of the arguments thus advanced in support of the opponent's thesis.

\textbf{\textit{A Parīkṣā of 3D Philology}}

The śāstric tradition of debate and discussion continues in many scholarly gatherings and \textit{sabhā}–s across the country even to this day, and is not much different from those held several hundred years ago. In such oral discussions, a topic is discussed thoroughly which leads ultimately to a \textit{nirṇaya}. Discussions can occur both orally and in written form; and addressing the issues raised by Western Indologists would come under the latter category. Pollock has been writing and propounding theories for four decades, but there has been no response from the Indian tradition. The first person to write a comprehensive response and initiate a debate was Sri Rajiv Malhotra in \textit{The Battle for Sanskrit }which is an assessment of Pollock’s major works.

Thus, the purpose of this \textit{prabandha} is to analyse and ascertain the theories of Pollock and come to a \textit{nirṇaya} (conclusion) about their validity. The following issues are considered:

\begin{enumerate}
\itemsep=0pt
\item 
 Evaluating the theory of three–dimensional philology. Pollock claims that there are three perspectives in understanding a Sanskrit (or any classical) text and these are:

\begin{myquote}
Plane 1: Historical (\textit{aitihāsika}) – The meaning according to the author of a text.
\end{myquote}

\begin{myquote}
Plane 2: Traditional (\textit{sāmpradāyika}) – The meaning of a text according to tradition or the commentators.
\end{myquote}

\begin{myquote}
Plane 3: Presentist (\textit{svecchā}) – The meaning of the text according to the present–day reader (typically Pollock’s own interpretation).
\end{myquote}

 Pollock’s important claim is that along the second plane, only the view of the Indian tradition is presented without any Western bias in all his works.

 \item Ascertaining if the theory of the three–dimensional philology actually follows Pollock’s own practice.

 \item Tracing the intellectual history of Pollock’s philology showing the overall development and internal contradictions.

\end{enumerate}

A \textit{parīkṣā} of all these issues would lead us to the final \textit{nirṇaya}.


\section*{1.4 The Three Steps of Enquiry in Nyāya–śāstra}

This \textit{prabandha} is divided into three sections based on the division of \textit{Nyāya–śāstra} as discussed in the \textit{Nyāya–bhāṣya}.

\textit{\textbf{Nyāya–bhāṣya}}[p.8]

\begin{myquote}
\textit{trividhā cāsya śāstrasya pravṛttiḥ – uddeśo lakṣaṇaṁ parīkṣā ceti | tatra nāmadheyena padārtha–mātrasyābhidhānam uddeśaḥ | uddiṣṭasyātattva–vyavacchedako dharmo lakṣaṇam | lakṣitasya yathālakṣaṇam upapadyate na veti pramāṇair avadhāraṇaṁ parīkṣā |}
\end{myquote}

\textit{\textbf{Anuvāda}}

This \textit{śāstra} follows a three–fold procedure of \textit{uddeśa}, \textit{lakṣaṇa} and \textit{parīkṣā}. The process of referring to a \textit{padārtha} by its name is called \textit{uddeśa}. The distinguishing characteristic (or definition) of the \textit{padārtha} named is\textit{ lakṣaṇa}. Ascertaining the correctness of a \textit{lakṣaṇa} (of a \textit{padārtha}) through the \textit{pramāṇa}–s is \textit{parīkṣā}.

\textit{\textbf{Vivaraṇa}}

The first two \textit{sūtra}–s constitute the first \textit{prakaraṇa} (section). In these \textit{sūtra}–s, the \textit{abhidheya} (subject–matter) of \textit{Nyāya–śāstra}, its \textit{prayojana} (purpose) and the \textit{sambandha} (relation between the two) are stated as is usually done in the beginning of any śāstric text.

The subject–matter of the \textit{Nyāya–śāstra} consists of the sixteen \textit{padārtha}–s mentioned in the first \textit{sūtra}. But mentioning them alone cannot result in their true knowledge. It is further necessary to define and examine them and this is done in the subsequent \textit{sūtra}–s. To emphasis this, Bhagavān Vātsyāyana introduces the third \textit{sūtra} by stating that the \textit{Nyāya–śāstra} follows a three–fold procedure – of \textit{uddeśa}, \textit{lakṣaṇa} and \textit{parīkṣā} – of the sixteen \textit{padārtha}–s.

\textbf{\textit{A Parīkṣā of Pollock’s Three–dimensional Philology}}

This \textit{prabandha} also follows the method explained in the \textit{bhāṣya}. The \textit{padārtha} in our context is Pollock’s three–dimensional philology.

\textit{\textbf{Uddeśa }} – The first \textit{prakaraṇa} (chapter 2) includes a historical introduction to philology and Pollock’s three–dimensional philology, which is followed by a brief discussion of Gautama’s \textit{Nyāya–sūtra}.

\newpage

\textit{\textbf{Lakṣaṇa }} –The second \textit{prakaraṇa} (chapter 3) defines the \textit{pramāṇa}–s and illustrates their use in the works of Pollock.

\textit{\textbf{Parīkṣā }} –The third \textit{prakaraṇa} (chapter 4) is an evaluation of the three–dimensional philology. What \textit{pramāṇa}–s are used to substantiate the thesis statements in the various papers? Pollock’s theory of understanding Sanskrit texts is assessed to see if it is consistently followed in his major works. Does his theory follow his own practice? In the 1985 paper titled “The Theory of Practice and Practice of Theory in Indian Intellectual History”, he writes that

\begin{myquote}
The understanding of the relationship of \textit{śāstra} (“theory”) to \textit{prayoga} (“practical activity”) in Sanskritic culture is shown to be diametrically opposed to that usually found in the West. Theory is held always and necessarily to precede and govern practice; there is no dialectical interaction between them. (1985c: 499)\endnote{ “…the conception of the relationship between theory (\textit{śāstra}) and practice (\textit{prayoga}) in traditional India is diametrically opposed to that commonly found in Western formulations from the time of Aristotle.” [1989c:301]}
\end{myquote}

Even a basic understanding of \textit{Nyāya–śāstra} will show that it is based on \textit{lokānubhava} or worldly observation and experience (like any other science) and thus in this \textit{śāstra}, theory necessarily follows practice. Both Gautama and Vātsyāyana make a profound observation of the world around them, and present their observations in the form of the \textit{sūtra} andthe\textit{ bhāṣya}. If theory were to govern practice in Indian śāstric traditions, we would not be able to analyse the works of Pollock using \textit{Nyāya–śāstra} since his own works are written independently, and are not bound by this \textit{śāstra }(as he neither follows Indian tradition nor lives in India). This is an important aspect which shows that \textit{śāstra}–s such as \textit{Nyāya} and \textit{Vyākaraṇa} were always descriptive (based on observation of the \textit{loka}) and not prescriptive, and thus could never have been used for “domination” as theorized by Pollock. Further, an analysis of the three–dimensional philology would show that it does not follow Pollock’s own practice, and is \textit{kalpita} i.e., is a “conscious construction”\endnote{ “Conscious construction” is an expression used by Pollock regarding Nīlakaṇṭha Caturdhara’s version of \textit{Mahābhārata}. [2015f: 119]}, aimed to mislead readers, while only \textit{a priori} application of theory is seen in all the works.

