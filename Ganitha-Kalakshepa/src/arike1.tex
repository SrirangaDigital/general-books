\chapter*{ಅರಿಕೆ}

\phantom{a}

\vskip  -1.2cm

\quad 
{\fontsize{12}{12.5}\selectfont{
ನಮ್ಮ ರಾಜ್ಯದ ಜನರಲ್ಲಿ ವಿಜ್ಞಾನವನ್ನು ಪ್ರಚಾರ ಮಾಡಿ, ವೈಜ್ಞಾನಿಕ ಮನೋಭಾವದ ಬೆಳವಣಿಗೆಗೆ ಉತ್ತೇಜನ ನೀಡುವುದು ಕರ್ನಾಟಕ ರಾಜ್ಯ ವಿಜ್ಞಾನ ಪರಿಷತ್ತಿನ (ಕರಾವಿಪ) ಮುಖ್ಯ ಧ್ಯೇಯ. ರಾಜ್ಯದ ವಿವಿಧ ಸ್ಥಳಗಳಲ್ಲಿ ಸ್ವಯಂ ಪ್ರೇರಣೆಯಿಂದ ರೂಪುಗೊಂಡಿರುವ ಪರಿಷತ್ತಿನ ಘಟಕಗಳು ಹಾಗೂ ಜಿಲ್ಲಾ ವಿಜ್ಞಾನ ಸಮಿತಿಗಳು ಸ್ಥಳೀಯವಾಗಿ ಈ ಕೆಲಸದಲ್ಲಿ ನಿರತವಾಗಿವೆ. ಉಪನ್ಯಾಸಗಳು ವಿಚಾರ ಸಂಕಿರಣ\break ಗಳು, ವೈಜ್ಞಾನಿಕ ಪ್ರದರ್ಶನಗಳು ಮುಂತಾದುವನ್ನು ಏರ್ಪಡಿಸುವ ಮೂಲಕ ದಿನನಿತ್ಯದ ಸಮಸ್ಯೆಗಳಿಗೆ\break ವೈಜ್ಞಾನಿಕ ಪರಿಹಾರಗಳನ್ನು ಹುಡುಕುವಲ್ಲಿ ಜನತೆಗೆ ನೆರವು ನೀಡುವ ಮೂಲಕ ಪರಿಷತ್ತಿನ ಧ್ಯೇಯ\break ಗಳನ್ನು ಸಫಲಗೊಳಿಸುವ ಪ್ರಯತ್ನ ನಡೆದಿದೆ. 

\smallskip
\smallskip

ಪರಿಷತ್ತು  ಪ್ರಕಟಿಸುವ ನಿಯತಕಾಲಿಕೆಗಳು, ಕಿರುಹೊತ್ತಿಗೆಗಳೂ ಆ ಪ್ರಯತ್ನಕ್ಕೆ ಬೆಂಬಲ ನೀಡ\-ಲಿವೆ. ಈಗಾಗಲೇ ನಲವತ್ತನೇ ವರ್ಷಕ್ಕೆ ಕಾಲಿಟ್ಟಿರುವ “ಬಾಲ ವಿಜ್ಞಾನ” ಮಾಸಪತ್ರಿಕೆ ಈ ದಿಶೆಯಲ್ಲಿ ಸಾಕಷ್ಟು ಯಶಸ್ಸುಗಳಿಸಿ ಜನಪ್ರಿಯವಾಗಿದೆ. ವಿಜ್ಞಾನ ವಿಷಯಗಳನ್ನು ಕುರಿತ ಕಿರು ಹೊತ್ತಿಗೆಗಳನ್ನು ಪ್ರಕಟಿಸುವ ಕಾರ್ಯವನ್ನು ಪರಿಷತ್ತುಕೈಗೆತ್ತಿಕೊಂಡು ಈಗಾಗಲೇ {\rm 200}ಕ್ಕೂ ಹೆಚ್ಚು ಪುಸ್ತಕಗಳನ್ನು ಪ್ರಕಟಿಸಿದೆ. 

\smallskip

ಪುಸ್ತಕಗಳ ಪ್ರಕಟಣೆಗೆ  ಕರಾವಿಪ, ವಿಜ್ಞಾನದ ಎಲ್ಲಾ ಪ್ರಕಾರಗಳನ್ನು ತನ್ನ ವ್ಯಾಪ್ತಿಗೆ ತೆಗೆದು\break ಕೊಂಡಿದ್ದು ವಿದ್ಯಾರ್ಥಿಗಳ ಮತ್ತು ಜನಸಾಮಾನ್ಯರ ದೈನಂದಿನ ಜೀವನಕ್ಕೆ ಸಂಬಂಧಿಸಿದ ವಿಷಯ\break ಗಳಿಗೆ ಆದ್ಯತೆ ನೀಡಿ ಪುಸ್ತಕಗಳನ್ನು ಪ್ರಕಟಿಸುತ್ತಿದೆ. 

\smallskip

ಈ ದಿಸೆಯಲ್ಲಿ ಲೇಖಕರಾದ ಶ್ರೀ ಬಿ. ಕೆ. ವಿಶ್ವನಾಥರಾವ್‌ರವರು ಬರೆದಿರುವ “ಗಣಿತ ಕಾಲಕ್ಷೇಪ” ಎಂಬ ಪುಸ್ತಕವು ಮುದ್ರಣಗೊಳ್ಳುತ್ತಿರುವುದು ಸಂತೋಷದ ವಿಷಯ.


\medskip
\medskip

\noindent
\quad 
{\bf ಶ್ರೀ ಎಸ್. ಎಂ. ಕೊಟ್ರುಸ್ವಾಮಿ} \hfill  {\bf ಶ್ರೀ ಎಸ್. ವಿ. ಸಂಕನೂರ}\\
ಅಧ್ಯಕ್ಷರು, ಪುಸ್ತಕ ಪ್ರಕಟಣಾ ಸಮಿತಿ, \hfill ಅಧ್ಯಕ್ಷರು, ಕರಾವಿಪ\\
~\phantom{AAAAA~} ಕರಾವಿಪ 

\smallskip
\medskip

\noindent
ಬೆಂಗಳೂರು\\
ಅಕ್ಟೋಬರ್  {\rm  2019}}}\relax
