\chapter*{AI powered society}

\byline{%
\begin{center}
\uppercase{Rajendra K. Bera\footnote{Chief Mentor, email: rajendrabera@yahoo.com. Communicating author. The views expressed are those currently held by
the author.}}

Acadinnet Education Services India Pvt. Ltd., B1/S1 Ganga Chelston, Silver Spring Road, Varthur Road, Munnekolala, Bangalore 560037, India
\end{center}
}

\begin{multicols}{2}
\begin{center}
\bf{Abstract}
\end{center}

To understand the post-industrial world of the millennials and the challenges and opportunities it presents
is yet another chapter in the evolution of Homo sapiens. It will force humanity to reassess the meaning of
life, its place and significance in the Universe, and above all its ability to survive in a world that includes its
own creative creation - the super-intelligent, human-machine hybrid - the humanoid. The role of natural
humans and humanity's faith in spirituality if humanoids take charge will undergo a sea-change. The
millennials' ability to adapt to the new world by competing against the humanoids will face severe
limitations and may even lead to Homo sapiens becoming an endangered species within a century. A
biological evolution of intelligent life is waiting to happen, triggered by the Homo sapiens’ curiosity-driven
quest to understand the Universe within a rational, axiomatized framework.

\textit{Key words:} Artificial intelligence, post-industrial economy, millennials, rationalism.

\section{Introduction}

Recent advances in artificial intelligence (AI) and biotechnology are nothing short of sensational in
terms of the conceptual barriers they have overcome. In the last few years they have garnered a list
of achievements and synergistic integration that clearly indicate that \textit{Homo sapiens} are in for an
unprecedented upheaval in their life--the loss of its vaunted intellectual supremacy on Earth and the
rising role of humanoids in human affairs. This change will likely happen within a century if Kurzweil's
predictions about the future of AI come true (given his track record, it most likely will). Kurzweil, the
author of The Singularity is Near\footnote{Kurzweil (2005).} predicts: “By 2029, computers will have human-level intelligence.”\footnote{Fox News (20170316).}
He also says the future will provide opportunities of unparalleled human-machine synthesis. In a
communication to Futurism, Kurzweil said:
\begin{quote}
2029 is the consistent date I have predicted for when an AI will pass a valid Turing test and therefore
achieve human levels of intelligence. I have set the date 2045 for the `Singularity' which is when we
will multiply our effective intelligence a billion-fold by merging with the intelligence we have
created.\footnote{Galeon \& Reedy (2017). “Of his 147 predictions since the 1990s, Kurzweil claims an 86 percent accuracy rate.”}
\end{quote}

To this we add an observation by Richard Ogle in his book, \textit{Smart World:}
\begin{quote}
We are coming to understand that in making sense of the world, acting intelligently, and solving
problems creatively, we do not rely solely on our mind's internal resources. Instead, we constantly
have recourse to a vast array of culturally and socially embodied \textit{idea-spaces} that populate the
extended mind. These spaces-manifested in forms as various as myths, business models, scientific
paradigms, social conventions, practices, institutions, and even computer chips--are rich with
embedded intelligence that we have progressively offloaded into our physical, social, and cultural
environment for the sake of simplifying the burden on our own minds of rendering the world
intelligible. \textit{Sometimes the space of ideas thinks for us.} We live in a smart world.\footnote{Ogle (2007).} [Italics in the original.]
\end{quote}

This smart world also faces unprecedented demographic changes due to variations in mortality, life
expectancy, and a youthful population in countries where fertility is high. Overcrowding on Earth is a
recent phenomenon. In the next three or four decades, the overall population of the more developed
countries is likely to stagnate at about 1.2 billion (see Fig. 1.1). Their population is ageing and would
decline but for migration. The populations of Germany, Italy, Japan, and several states of the former

\begin{center}
{\bf Figure}
\end{center}

Soviet Union that broke away are also expected to decline by 2050.\footnote{See, e.g., UNPF (2017).} The world's flexibility to cope with such unprecedented socio-economic changes is untested. Hence the millennials are expected to face unprecedented challenges and novel opportunities in the future. It will force humanity to reassess the meaning of life, its place and significance in the Universe, and above all its ability to survive in a world that includes its own creative creation - the super-intelligent, human-machine hybrid - the humanoid. The role of natural humans (the \textit{Homo sapiens}) and humanity's faith in spirituality if humanoids take charge will undergo a sea-change. Humanity's ability to adapt to the new world by competing against humanoids will be severely tested and may even lead to the \textit{Homo sapiens} becoming an endangered species within a century. A biological evolution of intelligent life is waiting to happen, triggered by the \textit{Homo sapiens'} quest to understand the Universe not according to the scriptures but according to science.

\section{Life on Earth}

Life on Earth began some 3.8 billion years ago with single-celled prokaryotic cells, such as bacteria,
evolving to multi-cellular life over a billion years. It is only in the last 570 million years that life forms
we are familiar with began to evolve, starting with arthropods, followed by fish 530 million years ago
(Mya), land plants 475 Mya, and forests 385 Mya. Mammals evolved around 200 Mya, and \textit{Homo
sapiens} (the species to which we humans belong evolved from Homo erectus)\footnote{Davis (2018). \textit{Homo erectus} had bodies similar to modern humans, could make tools, and were possibly the first to cook. They may have been mariners and possibly had a language. They first appeared in Africa more than 1.8 mya and perhaps the first archaic human species to leave the continent.} arose only a mere 300,000 years ago.\footnote{BBC (2018). \textit{See also:} SNMNH (2018b).} Up until 2.4 billion years ago, there was no oxygen in the air. There is still much to learn about the \textit{Homo sapiens}.\footnote{Maropeng (20100409). \textit{See also:} Berger (2015); Berger \& Hawks (2017); Greshko (2017) (“After adding \textit{Homo naledi} to the human family tree, researchers reveal that the species is younger than it seems. ... Lee Berger—provides an age range for the [fossil] remains first reported in 2015: between 236,000 and 335,000 years old.”); Barras (2017a) (“The past 15 years have called into question every assumption about who we are and where we came from. Turns out our evolution is more baffling than we thought.”); Barras (2017b) (“Lee Berger’s stunning discoveries of huge caches of ancient bones are overturning ideas about our origins, but not everyone likes his methods.”).} Their origin is vastly different from what various religions tell us. They were not made by God in His own image; they evolved to their present image and they are still evolving in unknown ways. One day the \textit{Homo sapiens} will become the ancestor species of one or more species. No religion has ever alluded to that.

On the biological front, Darwin's theory\footnote{Darwin (1859).} of evolution of life was a remarkable eye-opener. He posited
that all life is related and that it descended from a common ancestor: the birds and the bananas, the
fishes and the flowers, the animals and the \textit{Homo sapiens, etc.} It presumes life developed from non-
life and that complex creatures evolve from less complex ancestors naturally and over time through a
random process of adaptation via “descent with modification” in which random genetic mutations
occur within an organism's genetic code; the beneficial mutations that aid an organism to survive are
thus passed on to the next generation while the weaker organisms die and are eliminated from
breeding--a process known as “natural selection” or survival of the fittest in a given environment.
Over time, enough beneficial mutations accumulate to trigger a phase transition and an entirely
different organism (not just a variation of the original) comes into existence. The supporting evidence
for Darwin's theory of evolution comes from morphological similarity among organisms (suggesting
shared descent), and that living species are similar to recent related fossils. The fossil record is good
and large enough for us to see relatives of clearly different species in it.\footnote{See, \textit{e.g.,} Dawkins (2010).}

Evolutionary biology by itself does not necessarily imply that God does not exist. But it admits the
plausible view: “If God does exist, however, existing is about the only thing He has ever done. God is
permanently unemployed, if, in the entire history of life, impersonal material forces were capable of
doing the whole job and did do it. So if one attempts to hold a view of God as creator, it is a very
attenuated view and one which tends to fade away into unreality.”\footnote{Provine \& Johnson (1995).}  There is still much to learn about the nature of biological diversity and its complexity. We do not yet know how genetic information, as encoded in the DNA, came into existence to start life out of single-celled predecessors. That this is what happened millions of years ago is a reasonable scientific conjecture not yet refuted. Man learnt thousands of years ago how to accelerate the natural evolution process through selective breeding, \textit{i.e.,} by reducing randomness in the selection process, say, dogs with specific characteristics (\textit{e.g.,} size, body color, hair type, demeanor, \textit{etc.}) or plants modified to taste, made sturdier, etc., within a few thousand years rather than hundreds of millions of years that natural selection would take. Early human-engineered breeding practices have now advanced to an extent where we can clone living organisms and directly modify a living or dead organism's DNA or even \textit{ab initio} design DNA to create new organisms in the lab. We call it biotechnology, the core of which is genetic engineering. Creation of living matter is no longer a mystery, but how non-living matter gets turned into life and vice-versa is still a mystery.

Man's association with dogs goes back to some 10,000 to 30,000 years ago. It is generally believed
that “all dogs, from low-slung corgis to towering mastiffs, are the tame descendants of wild ancestral
wolves.”\footnote{Yong (2016).} Wolves, living in and around human surroundings, grew tamer with each generation until
they became domesticated permanent companions. Dogs have cross-bred so often with wolves and
each other that we have a great variety of them. The process itself indicates that dog domestication
may have happened several times spread over different geographies and times. Some day, molecular
biologists may tell us how ancient canines relate to each other and to modern pooches. The
domestication of the grey wolf into dogs, man's reputed best friend, happened long before the
Industrial Revolution (1760-1840), before literature and mathematics, and before bronze, iron, and
agriculture. This ancient partnership between man and animal entwined the fate of the two species.
“The wolves changed in body and temperament. Their skulls, teeth, and paws shrank. Their ears
flopped. They gained a docile disposition, becoming both less frightening and less fearful. They learned
to read the complex expressions that ripple across human faces. They turned into dogs.”\footnote{Yong (2016).}

The domesticated dog is an outstanding example that man is the important factor on Earth in changing
the environment. It is he who could domestic a wild species through genetic breeding in a very small
fraction of the time that Nature would have required. And now that man has learnt the secret of
creating new species in a lab, the time is not too far when he would be doing it on a mass scale.

Interestingly, no religion talks about either the bacteria or the dinosaur or the evolution of man from
the great apes. The dog does get a mention. Till an explosion in scientific knowledge occurred and
directly began to affect the structure of human societies through new schemes of division of labor and
its institutions, religion had a stranglehold on man that transcended the powers of human rulers. Since
the 20$^{\rm{ th}}$ century, religion's power over educated man has begun to erode with increasing momentum. The simple dilemma of the millennials is: Should they enjoy and explore life on Earth or secure a place in heaven by servilely appeasing an unseen, undescribable God who communicates with men through messengers claiming to carry revealed messages whose contents are increasingly at variance from science. Science is progressive and open to correction, religion is regressive and dogmatic.

Religions are oblivious of millions of species that once existed on Earth and have since vanished and
millions that will emerge on Earth in the future.\footnote{\textit{See, e.g.,} Han (2013). \textit{See also:} CBD (), WWF Global (2017), Pearce (2015).} Indeed, we have no idea how and to what \textit{Homo
sapiens} will become ancestors to. For example, geology tells us how plate tectonics has changed the
location of continents through movements of plates on the Earth's surface. This has helped scientists
correlate in spatial-temporal terms the geographical distribution of animals and plants, both living and
fossil, with plate movements. This is amazing evidence for evolution by shared descent. It is generally
estimated that there were about fifty thousand species of vertebrates just 65-70 million years ago, to
the end of the Cretaceous. Of those, fewer than twenty gave rise to some one hundred thousand
species of vertebrates that exist now. The rest became extinct.\footnote{Provine \& Johnson (1995).}  One wonders why God let them all die instead of letting them live in harmony.

Life on earth began nearly four billion years ago. In another billion years life will become extinct
because the Sun's brightness is increasing nearly 10 percent per billion years, enough to extinguish
life on Earth by incineration.\footnote{Starr (2018).} In another 10 billion years from now, the Sun too will die in a spectacular display of fireworks. It does not matter if we devoutly and religiously believe the Sun to be a god or not.\footnote{With the death of the Sun, Hinduism, allegedly the oldest of religions, will lose one of its most important gods. Of course, his worshippers would have died long ago by then.} As William Provine says,

\begin{quote}
Let me summarize my views on what modern evolutionary biology tells us loud and clear - and these
are basically Darwin's views. There are no gods, no purposes, and no goal-directed forces of any kind.
There is no life after death. When I die, I am absolutely certain that I am going to be dead. That's the
end of me. There is no ultimate foundation for ethics, no ultimate meaning in life, and no free will
for humans, either.\footnote{Provine \& Johnson (1995).}
\end{quote}

This is quite the opposite of what religions preach us, namely, God created life and humans He created
separately. Humans and chimpanzees do not share a common ancestor. God gives life after death and
He gives us absolute foundation for ethics. He also gives us the ultimate meaning for life and gives us
free will to go astray or seek genuine understanding of Him and shoulder responsibility. No reason
was ever provided for not creating the perfect man and the perfect universe in the first place. Also,
free will gives man the opportunity to act as nastily and irresponsibly as God does, like putting people
in hell-like prisons and fanning vindictiveness. There is no Earthly reward for praising God and suffering
while doing so. The biologically evolving \textit{Homo sapiens} are indeed a small part of a complex, process;
they are not the final goal of evolution. “Think of us all as young leaves on this ancient and gigantic
tree of life - connected by invisible branches not just to each other, but to our extinct relatives and
our evolutionary ancestors.”\footnote{Chakrabarty (2018).}

\section{The evolving \textit{Homo sapiens}}

We, the \textit{Homo sapiens}, have been around for about 300,000 years.\footnote{SNMNH (2018b). \textit{See also} the original papers announcing the discovery in Nature: Hublin, \textit{et al.} (2017), and Richter, \textit{et al} (2017). Prior to these papers, \textit{Homo sapiens} were said to have been around for about 200,000 years.} Records of our civilization date back approximately 6000 years. About 12,000 years ago, after more than 99\% of mankind's life on Earth, man transitioned from being the nomadic hunter-gatherer to a pastoral-agricultural life of rearing animals and sowing seeds. This lasted till about 1500 AD. During this period society structured itself into families; women took care of the household, men earned a livelihood. From about 1500 AD to the later-half of 20$^{\rm th}$ century, an industrial economy developed with increasing growth of industrial activity and mechanization of agriculture. Within five centuries, the economy graduated from using animal power to steam power to fossil fuel power to electrical power. Along with changing sources of power, society too restructured itself into increasingly complex communities--extended families, cities, nations, alliances, institutions, modes of governance, dominions, \textit{etc.}--and economies that ranged from family businesses run locally to multinational corporations operating globally and employing millions of men and women. Women were gradually weaned away from the hearth to the power corridors of corporations, competing with men for power and success in all spheres of life--business, politics, arts, science, \textit{etc.} Since the late 20$^{\rm th}$ century the world began to transform rapidly into a post-industrial economy which expects to power its way to another future using wind, solar, and information power. This, in brief, is the progress of human civilization.

\section{The hunter-gatherer stage}





\end{multicols}
