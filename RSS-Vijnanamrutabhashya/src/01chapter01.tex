\chapter{}\label{chap01}

%\Authorline{B. Sankareswari}

\begin{center}
\dev{\large\bfseries श्रीसरस्वत्यै नमः}

\dev{\large\bfseries श्रीगणेशाय नमः}
\end{center}

\begin{center}
\begin{tabular}{l}
\dev{सर्वत्र यो यत्र सर्वं यश्च सर्वमतो भवेत् ।}\\
\dev{चिदचिच्छक्तये तस्मै नमश्चिन्मात्ररूपिणे ॥ १ ॥}\\[2pt]
\dev{अन्तर्यामिगुरूद्दिष्टज्ञानविज्ञानभिक्षुणा ।}\\
\dev{ब्रह्मसूत्रऋजुव्याख्या क्रियते गुरुदक्षिणा ॥ २ ॥}\\[2pt]
\dev{श्रुतिस्मृतिन्यायवच:क्षीराब्धिमथनोद्धृतम् ।}\\
\dev{ज्ञानामृतं गुरोः प्रीत्यै भूदेवेभ्यो नु दीयते ॥ ३ ॥}\\[2pt]
\dev{परिविषय्य सद्बुद्ध्या मोहिन्येवाथ दानवान्}\\
\dev{कुतर्कान् वञ्चयित्वेदं पीयताममृतेप्सुभिः ॥ ४ ॥}\\[2pt]
\dev{पीत्वैतद् बलवन्तस्ते पाखण्डासुरयूथपान् ।}\\
\dev{विजित्य ज्ञानकर्मभ्यां यान्तु श्रीमद्गुरोः पदम् ॥ ५ ॥}
\end{tabular}
\end{center}


Bhikṣu starts his work as is the custom with some benedictory verses.
\begin{enumerate}
\item He that is everywhere, where everything is, and He that becomes everything eventually, to That (Supreme) of the nature of pure conscienceness, having the powers of consciousness  (jīvas) and insentience (prakṛti)(cidacicchaktaye), I bow down.\endnote{In Bhikṣu's philosophypuruṣa and prakṛti are śaktis of Īśvara}

\item This Ṛjuvyākhyā interpretation of the Brahmasūtra is being offered as gurudakṣiṇā  by Bhikṣu who is a jñānavijñāna, keeping in mind his guru as
          the internal self.
          
\item This Jñānāmṛta which has been obtained after churning the ocean of śruti, smṛti and Nyāya reasoning is being offered to the devas on earth for pleasing (my) guru.

\item Examining well the demons through Mohinī in the form of truthful intelligence (and) tricking them through false reasoning those desiring nectar may drink this.

\item Having drunk this, those powerful ones after conquering the ignorant groups of demons and winning (liberation) through the means of both karma and knowledge may they reach the feet of the guru.
\end{enumerate}

\dev{ब्रह्मविदाप्नोति परम्, ब्रह्म वेद ब्रह्मैव भवति, तमेव विदित्वाति मृत्युमेति'' इत्यादिश्रुतिसिद्धपरमपुरुषार्थसाधनताके ब्रह्मज्ञाने विधिः श्रूयते --- ``आत्मेत्येवोपासीत, स म आत्मेति विद्यात् तमेव धीरो विज्ञाय प्रज्ञां कुर्वीत ब्राह्मणः'' इत्यादिरूपः । तत्र किं ब्रह्म, किं वा तस्य ब्रह्मतानिर्वाहकं गुणजातम्, कीदृशं वा तस्य ज्ञानम् , कीदृशं वा तस्य फलमित्यादिकं विशिष्य मुमुक्षूणां जिज्ञासितं भवति, श्रुतिष्वापाततोऽन्योन्यविरुद्धार्थतायाः शाखाभेदेन प्रतिभासनादिति। अतस्तन्निर्णयाय ब्रह्ममीमांसाशास्त्रमपेक्षितम्}

In order to attain the highest puruṣārtha which is the realization of Brahman established and stated in śruti as ``brahmviḍāpnoti param, brahma veda brahmaiva bhavati, etc'', one hears of imperative statements such as ``ātmetyevopā\-sīta, sa ma ātmeti vidyāt'', etc. Therein many questions arise such as: What is Brahman, what are the qualities that  determine its having the nature of Brahman (brahmatā\-nirvā\-hakam guṇajātam), what is the nature of its knowledge, what is the nature of its (knowledge's) result, all of these is desired to be known, especially for those who desire mokṣa; this is because in the sacred texts there appears in the first instance(āpātataḥ) mutually contradictory meanings (anyonyaviruddhārthatāyāḥ) due to the difference (of beliefs)in the śākhās (śākhābhedena) (they follow). Therefore there is a need for an authoritative text (śāstram) on brahmamīmāmsā, in order to resolve (tannirṇayāya) these (differences).

\dev{नन्व “थातो धर्मजिज्ञासे” त्यादिपूर्वमीमांसयैव ब्रह्मज्ञानरूपधर्मस्यापि मीमांसितत्वान्नास्ति पुनराकाङ्क्षा । ब्रह्मज्ञानस्य च धर्मत्वं चोदनालक्षणत्वात् सिद्धम् “अयं तु परमो धर्मो यद् योगेनात्मदर्शन” मित्यादिस्मृतेश्च । वक्ष्यति चाचार्यः । सर्वासु वेदान्तविद्यासु चोदनां “सर्ववेदान्तप्रत्ययं चोदनाद्यविशेषा” दिति सूत्रेण, तत्र कुतर्कजातस्य च पश्चान्निराकरिष्यमाणत्वादिति। मैवम् । सामान्यतो धर्मत्वादिना}

\dev{निरूपणेऽपि अशेषविशेषनिर्धारणार्थं कल्पसूत्रादिवद् ब्रह्ममीमांसाया अप्यपेक्षितत्वात् ।}
       
\dev{ननु तथापि “सत्यं ज्ञानमनन्तं ब्रह्म, बिज्ञानमानन्दं ब्रह्मे” त्यादिश्रुतिसिद्धत्वाद् ब्रह्मस्वरूपे जिज्ञासा नोपपद्यत इति चेन्न, तदेव ज्ञानं किं सांख्ये सिद्धं 
जीवचैतन्यं किं वा चैतन्यान्तरमित्येवंस्वरूपजिज्ञासासत्त्वादिति । तदेवमाकाङ्क्षितत्वाद् ब्र ह्ममीमांसाशास्त्रस्यारम्भं प्रतिजानीते भगवान् वेदव्यासः—}

\textbf{Ques:} But as through pūrvamīmāmsā sūtras such as “athāto dharmajijñāsā” etc., there has been reflection on the dharma of the nature of knowledge of Brahman there cannot be that desire again. The knowledge of Brahman being dharma is also established because of having the quality of being known through the Vedas; there are also statements in the smṛtis such as “ayam tu paramo…yogenātmadarśanam”\endnote{Bhikṣu’s pet preference for Yoga comes out very early in the work}. Ācārya Bādarāyaṇa will also mention the injunction common to all meditation pertaining to Vedānta (sarvāsu vedāntavidyāsu) through the sūtra “sarvavedānta…viśeṣa” (BS III.3.1) with the intention of rejecting the illogical reasoning later.  

\textbf{Ans:} It is not so; even though, in general, there is decisions based on dharma, in order to reflect on the many special features (aśeṣaviśeṣanirdhāraṇārtham), similar to that of the Kalpasūtras, there is the need for the Brahmamīmāsā śāstra as well (brahmamīmāṁsāyā\break apyapekṣitattvāt).

\textbf{Ques:} Even so such statements as “satyam jñānamanantam brahma, vijñānamānandam brahma” are established in śruti and so it cannot be said that the desire to know about the nature of Brahman is not proper (brahmasvarūpe jijñāsā nopapadyate it cet)

\textbf{Ans:} It is not so. Is it the same knowledge already determined in Sāmkhya (philosophy) or is the consciousness of jīva of a different nature? Such kind of desire to know the real nature (of Brahman) is there. Thus since there is such a desire Bhagavān Vedavyāsa declares (pratijānīte) the commencement of the śāstra of brahmamīmāṁsā.

\section*{\dev{अथातो ब्रह्मजिञासा}}

\dev{अत्राथशब्द उच्चारणमात्रेण मङ्गलरूपोऽधिकारवाचकः । अधिकारश्च प्रकरणं ग्र  ग्रन्थान्येन निरूपणमिति यावत् । अतो ब्रह्मशेषतयाऽन्येषामपि मीमांसनमर्थाक्षिप्तम् । तथा प्रत्यधिकरणं ब्रह्मशब्दाभावेऽपि प्रकरणितया ब्रह्मैव \hbox{लब्धव्यमित्यादिकमथशब्दस्यप्रयोजनम्} । अत इत्यत्रेदमा प्रकृतं सूत्रमुच्यते, पञ्चमी चावधी, तथा च इदं सूत्रमारभ्येत्यर्थः । अर्थज्ञानात् प्रागेव सूत्रस्योपस्थितत्वान्न तत्परामर्शानुपपत्तिः, वक्ष्यमाणस्यापीदमा परामर्शदर्शनात् उत्तरसूत्रमारभ्येत्यर्थस्वतिसमीचीनः । तथा च यथा ग्रन्थशेषेऽ“नावृत्तिः शब्दादि” ति समग्रसूत्रद्विरावृत्तिः शास्त्रस्योत्तरावधिसूचिका तथैवात इति शब्दोऽपि तत्पूर्वावधिवाचकः । एवं हि शास्त्रपूर्वापरान्तद्वयावधारणे सति ग्रन्थमहावाक्यार्थबोधाय वाक्यान्तराकाङ्क्षया शिष्याणां विलम्बो भवतीत्याशयेन शास्त्रकृद्भिराद्यन्तावधी परिच्छिद्येते । अवश्यं चा ‘थातः’ शब्दयोर्यथोक्तार्थतैव शास्त्रान्तरेष्वभ्युपेया, यथा “अथातो व्रतमीमांसे” त्यादिश्रुतौ, अथातो गोभिलोक्तानामन्येषां चैव कर्मणाम् । अस्पष्टानां विधिं सम्यग् दर्शयिष्ये प्रदीपवत् ।। इत्यादिस्मृतौ । न हि तत्राधिकारावधी विहाया ‘थातः’ शब्दयोरन्यार्थता सम्भवति ।}

Here by the very utterance of the word ‘atha’ is indicated auspiciousness and denotes the authority (to write the work). ‘adhikāra’ indicates mainly the examination of a topic (prakaraṇam). Therefore since Brahman is the only residue (brahmaśeṣatayā) there is the need for reflection on all the other subjects as well (anyeṣāmapi). Even if the word Brahman is absent in every adhikaraṇa (of the BS), since that is the topic (prakaraṇitayā) one understands, therefore the usage of the word ‘atha’ is to indicate that the purpose (prayojanam) of the (discussion) is Brahman alone. 

The word ‘ata’ declares (ucyate) that here is the first sūtra; the fifth case is to denote the limit; thus it means starting from this place onwards. Even though the sūtra comes even before understanding the meaning (of  Brahman), it is not  unreasonable to inquire (parāmarśa) about it (Brahman), since one sees that what is going to be discussed (in the coming sūtras) is the same (topic); therefore the meaning ‘from the next sūtra onwards’ is most appropriate. Thus just as at the end of the work there is the indication of the end of the śāstra through the repetition of the sūtra “anāvṛttiḥ śabdādanāvṛttiḥ śabdāt” (BS. IV.4.22), so also the word ‘ata’ also denotes the starting point (of the śāstra). Thus in this way, (even) when the two limits of the beginning and the end of the śāstra are stated (clearly), since there is a delay because disciples/students desire to understand the meaning of the mahāvakyas through different vākyas, with this in mind (āśayena) the authors of the śāstra have divided the beginning and the end. It is necessary that the meanings of the words ‘athātaḥ’ are understood in the same manner in the other śāstras as well like “athāto brahmamīmāṁsā” and in smṛtis like “athāto gobhilo…pradīpavat”(the first śloka in the Karmapradīpa written by Kātyāyana; cited in Tripathi p.2 fn.1). Therein it is not possible to have any other meaning for ‘athātaḥ’ than the rule of limits (adhikārāvadhī).

\dev{ब्रह्मणो ब्रह्मशब्दार्थस्य जिज्ञासा ब्रह्मजिज्ञासा, अतो विशिष्य पूर्वं ब्रह्मज्ञानाभावेऽपि शिष्याणां न सूत्रवाक्यार्थबोधासंभवः । ब्राह्मणवेदहिरण्यगर्भादिषु ब्रह्मशब्दस्य गौणत्वेन न ब्रह्मशब्दार्थतेति वक्ष्यामः । जिज्ञासा चात्र \hbox{विचारो} मीमांसापरनामकः, जिज्ञासाशब्दस्य मीमांसाशब्दवद् विचारे रूढत्वात्। अथातो धर्मजिज्ञासेत्याद्यनेकशास्त्रेषु \hbox{जिज्ञासां} प्रतिज्ञाय विचारकरणदर्शनात्। श्रुतावपि ब्रह्मज्ञानेच्छयोपसन्नं शिष्यं प्रत्यपि “तद्विजिज्ञासस्व तद्ब्रह्मेति” पुनर्जिज्ञासोपदेशाच्च ।}

\dev{अत एव “अजिज्ञासितसद्धर्मो गुरुं मुनिमुपव्रजेदि”त्यादिवाक्येषु विचार एव जिज्ञासाशब्दः प्रयुज्यमानो दृश्यते, तत्रेच्छार्थकत्वासम्भवात् । तस्माद् योगेन  रूढतया च प्रकरणभेदेन जिज्ञासाशब्देन विचारेच्छयोरुभयोरेव वाचक इति बोध्यम् । विचारश्च विवरणं निर्णयहेतुभूतं लिङ्गाद्यवधारणम् । निर्णयश्चोक्तो न्यायाचार्यैः– “विमृश्य पक्षप्रतिपक्षाभ्यामर्थावधारणं निर्णयः” इति । स च निर्णयः वेदान्तैरेवेति वक्ष्यति “शास्त्रयोनित्वादि” ति सूत्रेण । तथा चायं सूत्रार्थ:— इदं सूत्रमारभ्य प्राधान्येन ब्रह्मविचारः तच्छास्त्रमस्माभिः क्रियत इति । यदि च जिज्ञासाशब्देन तच्छास्त्रं न लक्ष्यते तदा आचार्येण पूर्वमेव ब्रह्मणो निर्णीतत्वाद् विचारप्रतिज्ञा नोपपद्यते नोपपद्यते  च विचारं प्रतिज्ञाय सूत्रपरम्परारचनमिति ।}

The desire to learn the meaning of the word Brahman is ‘brahmajijñāsā’; ‘ataḥ’= even if there is the absence of knowledge especially regarding Brahman earlier, it is not impossible to instruct the śiṣyas the meanings of the sūtra vākyas (sūtravākyārthabodhāsambhavaḥ). Since the word Brahman is used in a secondary sense (gauṇatvena) in the Brāhmaṇas, Vedas, in Hiraṇyagarbha etc., we consider that they do not convey the right meaning of the word Brahman (na brahmaśabdārthatā). ‘jijñāsā’ here stands for reflection (and) is another word for mīmāṁsā, as similar to the word mīmāmsā the word jijñāsā also conventionally means contemplation/reflection; in many such śāstra texts as “athāto dharmajijñāsā” (one sees) that after declaring (pratijñāya) the desire to know, one sees engagement in thought (about the subject) (vicārakaraṇadarśanāt). In śruti as well the disciple who has approached (the guru) with the desire to learn about Brahman is again advised to have the desire to know in such statements as “tadvijijñā\-sasva tadbrahmeti” etc. That is the reason why in such statements as “ajijñāsitasaddharmo gurum munimupavrajet” it is seen that the word jijñāsā is used for thought process alone, since there is no possibility of the meaning of desire therein. Thus both etymologically and by convention the word jijñāsā denotes both (just) thinking/thought as well as desire to know, according to the context (prakaraṇabhedena). Thinking or a thought process is explaining (vivaraṇam) the cause such as the sign etc., based on which leads to a decision or conclusion (nirṇayahetubhūtam liṅ\-gādyavadhāraṇam). And a conclusion has been mentioned by Nyāya ācāryas as “vimṛṣya pakṣapratipakṣābhyām…nirṇayaḥ” (NS. I.1.41). By the sūtra “śāstrayonitvāt” it will be said that decision (nirṇaya) is through Vedānta (utterances) alone. Thus the meaning of this sūtra is ‘starting with this sūtra the main topic in general is reflection on Brahman (and) we are writing  the śāstra connected with that. If by the word jijñāsā that śāstra is not meant (tacchāstram na lakṣyate) then since ācārya (Bādarāyaṇa) has already determined (the nature of) Brahman it does not stand to reason to declare reflection (again on it); having promised reflection it is also not right to not start composing a set of sūtras (nopapadyate ca vicāram pratijñāya sūtraparamparāracanamiti).

\dev{आधुनिकास्तु प्रौढ्या सूत्रमिदमेवं व्याचक्षते — अधीतस्वाध्यायैर्विचारितकर्मकाण्डैरप्यकर्तृ\-त्वादिरूपेणा\-त्मनोऽनव\-धृतत्वात् तन्निर्धारणे चाविद्यानिवृत्त्या पुरुषार्थसिद्धेस्तन्निर्धारणायात्मनो ब्रह्मणो जिज्ञासा तदुपलक्षितो विचारः शिष्याणां कर्तव्यतया शास्त्रस्यादौ विधीयते “अथातो ब्रह्मजिज्ञासे”ति । अथशब्दो नित्यानित्यवस्तुविवेकेहामुत्रफलभोगविरागशमदमादिसंपन्मुमुक्षुत्वरूपसाधनचतुष्टयानन्तर्यमाह । अतः-शब्दश्च वक्ष्यमाणहेतुवाचकः । ज्ञातुं साक्षात्कर्तुमिच्छा जिज्ञासा तद्धेतुको विचारः । तथा चायमर्थः— यस्मादग्निहोत्रादिकमनित्यफलकं ब्रह्मज्ञानं चानन्तफलकमतः सर्वकर्माणि संन्यस्य शमदमादिसाधनचतुष्टयसंपन्नेन विविदिषुणा ब्रह्मविचार. षड्विधलिङ्गैर्वेदान्ततात्पर्यावधारणरूपो ब्रह्मसाक्षात्काराय कर्तव्य इति । “तद्विजिज्ञासस्व तद् ब्रह्मे”ति श्रुतेः । लिङ्गानि च पूर्वाचार्यैरुक्तानि— }
\begin{verse}
\dev{उपक्रमोपसंहारावभ्यासोऽपूर्वता फलम् ।} \\
\dev{अर्थवादोपपत्ती च लिङ्गं तात्पर्यनिर्णये ॥ इति ॥}\\
\dev{तथा च विधिपरं सूत्रमिति वदन्ति । }
\end{verse}

Modern day (vedāntins) through arrogance (prauḍhyā) explain this sūtra as follows:  Through instruction under a guru (svādhyāyaiḥ) and through  performing karma/rituals since one does not come to realize the non-doership nature of the ātman, in order to  realize that and achieve the final goal in life through the cessation of avidyā, there is the desire of ātman to know about Brahman and reflection pertaining to that (tadupalakṣito vicāraḥ); therefore at the start of the śāstra the sūtra “athāto brahmajijñāsā” is prescribed as a duty for disciples. They also say that the word “atha” denotes coming after the four fold prerequisites like “niyānityavastuviveka…mumukṣutvam”. The word “ataḥ” (according to them) denotes the reason which will be mentioned later; “jijñāsā” is the desire to realize (Brahman) directly and thought/reflection pertaining to that is the reason (for the composition of the sūtras) (jñātum sākṣātkartumicchā jijñāsā taddhetuko vicāraḥ). Thus the meaning (of the sūtra according to them) is that since rituals such as agnihotra etc., lead to results which are impermanent and knowledge of Brahman leads to a permanent result (anantaphalakam) so abandoning all rituals and  equipped with the four requisites like ‘śamadama’ etc., the one desiring to know (vividiṣuṇā) should engage in reflection on Brahman based on the sixfold marks (ṣaḍvidhaliṅgaiḥ) for determining the intended meaning of Vedānta, for the direct realization of Brahman (brahmasākṣātkā\-rā\-ya). The authority for this is the śruti “tadivjijñāsasva tadbrahmeti”. The sixfold marks mentioned by earlier ācāryas are as follows: upakramopasahārā\-bhyāsa…\-tātparyanirṇa\-yane”. Thus they say it is a sūtra which indicates a prescription (vidhiparam sūtramiti vadanti).

\dev{तत्रेदमुच्यते—कथं पुनः सर्वतन्त्रसाधारणमपेक्षितं  चोक्तार्थं हित्वा सामान्याभ्यामथात:शब्दाभ्या\-मेतादृ\-शोऽर्थ\-विशेषोऽवधारित इति । साधनाध्यायसूत्रेभ्य इति चेत्, अलमत्र तत्सूत्रणेन, एकाक्षरलाघवा ह्याचार्याः पुत्रोत्सवं मन्यन्ते । न वा साधनाध्याये सर्वकर्मत्यागं वा विचाराङ्गत्वेन शमदमादीन् वा विधास्यति, किन्तु विद्याया अकर्मशेषतया पुरुषार्थहेतुत्वं प्रतिज्ञाय तत्साधकत्वेन समाधिनिष्ठानां न्यायसिद्धं कामतः कर्मत्यागमनुवदति “उपमर्दं चेति” सूत्रेण । तथा विद्याप्रधाने संन्यासाश्रमे विधिं व्यवस्थापयिष्यति न सर्वकर्मत्यागे । तथा सम्प्रज्ञातयोगब्रह्मसाक्षात्काररूपायामेव विद्यायां फलपर्यवसायिन्यां शमदमाद्यन्तरङ्गसाधनं वक्ष्यति “शमदमाद्युपेतः स्यात्तथापि तु तद्विधेस्तदङ्गतया तेषामवश्यानुष्ठेयत्वादि” ति सूत्रेण । “तस्मादेवंविच्छान्तो दान्त उपरतस्तितिक्षु: समाहितो भूत्वा आत्मन्येवात्मानं पश्यती”त्यादिश्रुतौ संप्रज्ञातजसाक्षात्काराङ्गतयैव शमदमादिविधानात् । नारदीये नित्यानित्यविवेकादिसाधनचतुष्टयप्रतिपादनानन्तरम् —}
\begin{verse}
\dev{चतुर्भिः साधनैरेतैर्विशुद्धमतिरच्युतम् ।}\\
\dev{सर्वगं भावयेद् विप्राः सर्वभूतदयापर: ॥}
\end{verse}

\dev{इति वाक्येन साधनचतुष्टयस्य ध्यानाङ्गताया एव लाभाच्च । तत्र च यदि अङ्गाङ्गतया बहिरङ्गैरग्नीन्धनादिभिः कर्माङ्गैर्विक्षेपात् शमादिर्न संभवति तदा अन्तरङ्गानुरोधेन बाह्याङ्गान्यग्नीन्धनादीनि नापेक्षणीयानीत्येव वक्ष्यति “अत एव चाग्नीन्धनाद्यनपेक्षे”ति सूत्रेण, न तु कर्मत्यागं तेनापि सूत्रेण विधास्यति । तथा सति “सर्वापेक्षा च यज्ञादिश्रुतेरश्ववत्, सहकारित्वेन चे” त्युत्तरसूत्रविरोधात् “कर्मानपेक्षे” त्यस्यैव वक्तव्यतौचित्याच्च ।}

In that connection this is being stated: How is it that abandoning the intended meaning accepted by all disciplines (sarvatantrasādhāraṇa\-mapekṣitam coktārtham hitvā) you have determined such a special meaning of the common words “atha” and “ataḥ”. If it is said that it is understood from the sūtras in the sādhanādhyāya, enough of this argument using sūtras. Ācāryas consider the saving of even one letter as equivalent to the birth of a son [the implication being that they will not be using the sūtras which are composed with more than many akṣaras, to further their case].\endnote{Ekamātralāghavam putrotsavam manyante vaiyākaraṇāḥ (\dev{एकमात्रालाघवं पुत्रोत्सवं मन्यन्ते वैया\-करणाः}).} In the chapter on sādhanā (BS 3rd  adhyāya), neither the total giving up of all rituals (sarvakarmatyāgam) nor śama, dama etc., have been mentioned as part of the reflection (on Brahman). On the other hand due to the absence of the residual karma it is assured (pratijñāya) as being a cause for the attainment of one’s goal (puruṣārthahetutvam); and that being attainable for those dedicated to samādhi (tatsādhakatvena samādhiniṣṭhānām) there follows of its own accord the cessation of karma (karmatyāgamanuvadati) as a logical end (nyāyasiddham); this is in accordance with the sūtra “upamardam ca” (BS.III.4.16). So with reference to the āśrama of saṁnyāsa with its emphasis on meditation/knowledge the injunction will be this (sūtra) and not on the abandonment of all karma. Thus the sūtra “śamadamādyupetaḥ syāt…avaśyānuṣṭheyatvāt” (III.4.27) will mention that in relation to the result which occurs through the practice of samprajñātayoga-meditation for the direct realization of Brahman\endnote{Bhikṣu loses no opportunity to bring in his partiality towards yoga whenever he gets a chance} the practice of śama, dama etc., are prescribed as subsidiaries of knowledge. Thus in such śruti statements as “tasmādevamvit…paśyati”, śama, dama etc., have been prescribed as part of the direct realization (of Brahman) through samprajñāta yoga. After mentioning the four fold means like nityānityaviveka there is the verse (vākyena) “caturbhiḥ sādhanaiḥ…sarvabhūta\-dayāparaḥ” in the Naradīya, which includes the fourfold means as part of dhyāna (Bṛhannāradīya.39.54.cited in Tripathi p.3.fn1).

In that context by rejection of the several parts of the outer means of rituals like kindling the fire etc., if śama, dama etc., do not occur  (śamadamādirna sambhavati) then in compliance with the internal requisites, there is no need for external requisites such as kindling of the fire etc., this will be mentioned as “ata eva cāgnīndhanādyanapekṣā” (BS.III.4.25). When that is so then since it is in contradiction to the next sūtra “sarvāpekṣā…sahakāritvena ca” (ibid. III.4.26) it would have been appropriate to mention ‘not requiring any karma’ (karmānapekṣā).

\dev{नन्वेवमपि सूत्रद्वयोक्तयोरग्न्याद्यपेक्षानपेक्षयोर्विरोध एवेति चेन्न, संन्यासिनां बाह्याग्न्याद्यभावेऽपि स्वसमारोपितानां बाह्याग्न्यादीनामन्तरग्निहोत्रकालेऽपेक्षणीयत्वेनाऽविरोधात् । तथा च विष्णुपुराणे संन्यासप्रकरणे— }
\begin{verse}
\dev{कृत्वाग्निहोत्रं स्वशरीरसंस्थं, शारीरमग्निं स्वमुखे जुहोति ।}\\
\dev{विप्रस्तु भैक्ष्योपचितैर्हविर्भिश्चिताग्निना स व्रजति स्वलोकान्  ।। इति}
\end{verse}
\dev{मोक्षधर्मे च संन्यासप्रकरणे}
\begin{verse}
\dev{प्रादेशमात्रे हृदि निष्ठितं यत् तस्मिन् प्राणानात्मयाजी जुहोति ।}\\
\dev{तस्याग्निहोत्रं हुतमात्मसंस्थं सर्वेषु लोकेषु सदैवकेषु ।।}\\
\dev{उत्तान आस्येन हविजुहोति लोकस्य नाभिर्जगतः प्रतिष्ठा ।}\\
\dev{तस्याङ्गभङ्गानि कृताकृतं च वैश्वानरः सर्वमिदं प्रपेदे  ।। इति}
\end{verse}
\dev{न्यायाचार्यैश्च “समारोपणादात्मन्यप्रतिषेध” इत्युक्तम् । अग्न्यादीनामात्मन्यारोपणात् कर्मणामप्रतिषेध इत्यर्थः । अत्र मोक्षधर्मवाक्ये अङ्गहानावप्यन्तर्यागो योगिनां न दुष्यतीत्युक्तम् ।}

\textbf{Ques:}  Even so if it is said that there is a conflict between the two sūtras one prescribing rituals and other not doing so, then (the answer is);

\textbf{Ans:} Even if samnyāsins have no external (rituals) like fire kindling etc., since there is the necessity of internalizing within themselves the external (ritual) fire at the time of the agnihotra sacrifice there is no contradiction (avirodhāt). Thus in the Viṣṇu Purāṇa Saṁnyāsa chapter it says “kṛtvāgnihotram…vrajati svalokān” (III.7.30 cited in Tripathi p.4.fn.1). So also in the Mokṣadharmaparvan, under the saṁnyāsa chapter, it says “prādeśamātrehṛdi niṣṭhitam…sarvamidam prapede”   (Mokṣa.254.27. cited in ibid p.4.fn.2). The Nyāyācāryas have said “samāropaṇādātmanyapratiṣedha”.\endnote{NS.IV.1.60} It means that since the sacrificial fires have been internalized (samāropaṇāt) in the ātman, liberation is assured (apratiṣedhaḥ). In the (above) statement of the Mokṣaparvan, even if there is the cessation of the limbs of yoga, the internal sacrifice is not futile for the saṁnyāsins.

\dev{एतेन संन्यासिनां सर्वकर्मत्यागोऽशास्त्रार्थः। तथा च श्रुतिरपि “अत ऊर्ध्वममन्त्रवदाचेरदि”ति तथा च श्रुत्यन्तरं “सन्धिं समाधावात्मन्याचरेदि”ति च । आत्मनि स्वशरीरे समाधिमात्रेण देवैः सह सन्धिं सन्ध्याख्यं कर्माचरेदित्यर्थः । सान्ध्यम् सन्ध्येति । परमात्मात्मनोरेकत्वे  विज्ञानेन तयोर्भेद एव विभग्नः । तथा मनौ— }
\begin{verse}
\dev{सर्वभूतस्थमात्मानं सर्वभूतानि चात्मनि ।}\\
\dev{समं पश्यन्नात्मयाजी स्वाराज्यमधिगच्छति ।।}
\end{verse}
\dev{इत्यनेन ज्ञानस्यात्मयागेन सह समुच्चय उक्तः । आत्मयागश्च स्वशरीरस्थानामेव सर्वदेवानां स्वकीयस्नानाहारादिकालेषु स्वभोगैरेव मन्त्रादिनैरपेक्ष्येण यजनम् ।}
\begin{verse}
\dev{एतानेके महायज्ञान् यज्ञशास्त्रविदो जनाः ।}\\
\dev{अनीहमानाः सततमिन्द्रियेष्वेव जुह्वति ।।}\\[5pt]
\dev{आत्मैव देवताः सर्वाः सर्वमात्मन्यवस्थितम् ।}\\
\dev{खं संनिवेशयेत् खेषु चेष्टनस्पर्शनेऽनिलम् ।।}
\end{verse}
\dev{इत्यादिमनुवाक्यान्तरात् । तस्य चात्मयागस्य मनोमात्रसाध्यत्वमाह वशिष्ठः वैदिकं कर्म प्रकृत्य— }
\begin{verse}
\dev{बाह्यामाभ्यन्तरं चेति प्रत्येकं मुक्तिसाधनम् ।}\\
\dev{बाह्यं बहिः क्रियाभिश्च यत्तद् विहितसाधनैः ॥}\\
\dev{आभ्यन्तरं तु मनसा विध्यनुष्ठानमात्मनि ।}\\
\dev{तयोरन्यतरत् कुर्यान्नित्यं कर्म यथाविधि ।।}
\end{verse}
\dev{मोक्षधर्मे च —}
\begin{verse}
\dev{शान्तियज्ञरतो नित्यं ब्रह्मयज्ञरतो मुनिः ।}\\
\dev{वाङ्मनः काययज्ञैश्च भविष्याम्युरुगायन ।। इति ।}
\end{verse}
\dev{गौतमीयतन्त्रे च —}
\begin{verse}
\dev{केवलम्  सततं श्रीमच्चरणाम्भोजभाजिनाम् ।}\\
\dev{संन्यासिनां मुमुक्षूणां मानसः कथितो विधिः ॥}
\end{verse}
\dev{इत्युक्तम् । केवलमित्यनेन बाह्यकर्मणामेव त्यागो लब्धः, मानसकर्मविधानात् । वसिष्ठेनात्मयागस्य प्रशंसा च कृता — }
\begin{verse}
\dev{यष्टुमात्मन्यशक्तश्चेत् यजेद् बाह्येषु सर्वदा ।}\\
\dev{स्वयमुत्पन्नलिङ्गे वा स्थापिते वा विशेषतः ।।}
\end{verse}
\dev{इत्यात्मयागाशक्तावेव बाह्यस्यावश्यकत्वमित्यर्थः । एवं च सति संन्यासिनामप्यन्तरग्निहोत्रं गायत्र्यर्थब्रह्मात्मताध्यानरूपसंध्यादिकं वास्तीति न “यावज्जीवमग्निहोत्रं जुहोती” त्यादिश्रुतिविरोधः। तथा —}


\begin{verse}
\dev{नियतस्य तु संन्यासः कर्मणो नोपपद्यते ।}\\
\dev{मोहात्तस्य परित्यागस्तामसः परिकीर्तितः ।।}\\[3pt]
\dev{कर्मणां नियतानान्तु त्यागो नैव विधीयते ।}\\
\dev{तेषां कर्मफलत्यागः संन्यास इति चोच्यते ॥}
\end{verse}
\dev{इत्येवंविधवक्यान्यपि संन्यासविधिना न विरुध्यन्त इति मन्तव्यम् ।}

All this means that giving up all karma for saṁnyāsins is not the intention of the śāstras (śāstrārthaḥ). Thus there is the śruti which says “ata ūrdhvamamantravadācaret”\endnote{Āruṇeyopaniṣad.2}   there is also another śruti statement “sandhim samādhāvātmanyācaret” . It means that within oneself or within one’s own body through samādhi alone “sandhim”=one should do karma known as sandhyā; it means the twilight sandhyā (ritual). 

Paramātman and ātman being declared to be identical the statement of their being different is an aberration/obstruction (vibhaṅgaḥ). Manu’s following statement “sarvabhūtamātmānam…adhigacchati”\endnote{Manu Smṛti 12.91} indicates that there is a confluence of the sacrifice of the ātman and jñāna (jñānasya ātmayāgena saha samuccaya uktaḥ). Ātmayoga is the sacrifice (yajanam) to all the devas situated in one’s body during all one’s experiences like bathing, eating etc., without any mantras (mantrādinairapekṣyeṇa). There are other Manusmṛti statements which express the same idea such as “etāneke mahāyajñān…juhvati”\endnote{Ibid.4.22} and “ātmaiva devatāḥ sarvāḥ…ceṣtanasparśane’nilam”\endnote{Ibid.12.119  (in Jagadīśalal Śāstri (JŚ. edition the second line is the first line of 12.120)}.

Vasiṣṭha mentions that this sacrifice of the ātman is possible only through the mind as follows “bāhyābhyantaram ceti pratyekam muktisādhanam…tayoranyatarat kuryānnityam karma yathāvidhi”. So also in the Mokṣadharmaparvan we have “śāntiyajñarato…bhaviṣyāmyurugāyana” (175.32 cited in Tripathi p.5. fn.1).  The Gautamīyantra mentions the same idea as follows “kevalam satata…kathito vidhiḥ”. The word “kevalam” indicates the giving up only the external activities (karma) and by prescribing mental activities Vasiṣṭha has praised the yoga of the self (ātmayogasya praśaṁsā kṛtā) in the following manner “yaṣṭumātmanyaśaktaścet…viśeṣataḥ) i.e. only when one is incapable of practising ātmayoga there is the need for external (karma). When, in this manner, there is mention of kindling the agnihotra sacrifice internally or performing the sandhyākarma like meditating internally on the meaning of the Gāyatrī mantra which denotes Brahman even in the case of saṁnyāsins, it is not in contradiction to the śruti statement “yāvajjīvamagnihotram juhoti” (the agnihotra sacrifice is performed as long as one lives). So also such statements as “niyatasya tu samnyāsaḥ…samnyāsa iti cocyate” should be considered as not contrary to the prescribed duties of saṁnyāsins\endnote{Bhikṣu has given this long justification for jñānakarasamuccaya quoting from many sacred texts; it probably indicates the life he chose as a samnyāsin/bhikṣu for himself.}.

\dev{मोहादकर्तव्यताज्ञानात् कर्मत्यागस्य वैधत्वे तु तत्तच्छ्रुतिस्मृतयः संकुच्येरन् । असंकोचेनोपपत्तौ संकोचश्चान्याय्यो वैधस्य रात्रिश्राद्धस्येव प्रतिषेधानुपपत्तिश्च । प्राप्ताप्राप्तविकल्पग्रासात् पर्युदासाश्रयणे च “रात्रौ श्रIद्धं न कुर्वीते” त्यादाविव कर्मत्यागवाक्येष्वपि लक्षणाभ्युपगमेन गौरवात् , कर्मत्यागवाक्यानाम् उपदर्शितविकल्पाद्यनुसारेण बाह्यकर्मत्यागपरत्वस्यैवौचित्याच्च । एतेन —}
\begin{verse}
\dev{सर्वाणि भूतानि सुखे रमन्ते, सर्वाणि दुःखेषु तथोद्विजन्ति ।}\\
\dev{तेषां भयोत्पादनजातखेदः, कुर्यान्न कर्माणि हि जातवेद:॥}
\end{verse}
\dev{इत्यादीनि जातवेदानां विदुषां कर्मत्यागविधायकवाक्यानि बाह्यकर्मत्यागपराण्येवावगन्तव्यानि, “आत्मक्रीड आत्मरतिः क्रियावानेष ब्रह्मविदां वरिष्ठ” इत्यादि श्रुतिभिरात्मारामस्यापि विदुषः कर्मावगमात् ।}
\begin{verse}
\dev{ज्ञानिनाऽज्ञानिना वापि यावद्देहस्य धारण}\\
\dev{तावद् वर्णाश्रमप्रोक्तं कर्तव्यं कर्म मुक्तये ॥}\\
\dev{ज्ञानेनैव सहैतानि नित्यकर्माणि कुर्वतः ।}\\
\dev{निवृत्तफलतृप्तस्य मुक्तिस्तस्य करे स्थिता ॥}
\end{verse}
\dev{इति वशिष्ठकौमार्दिस्मृतिभिर्विदुषोऽपि कर्मIवश्यकत्वस्मरणाच्च । तथा कर्मत्यागहेतोर्हिंसाविक्षेपादेरान्तरकर्मण्यसंभवाच्च । संन्यासिनां कर्माभावे स्वशरीरेऽग्न्यादीनां देवादीनां वारोपणस्य वैफल्यापत्तेश्च ।}

If out of delusion or ignorance as to knowing what should not be done there is prescription of giving up karma then it will narrow (samkucyeran) the meaning of śruti and smṛtis. As it is proper to avoid narrowing the meaning and narrowing the meaning of something prescribed (vaidhasya) is not in conformance to reason (anyyāyo) and it is also not reasonable (to do something that is prohibited like the performance of śrāddha in the night (rātriśrāddhasyeva pratiṣedhānupapattiśca). Divided between what is meant or not meant or meant optionally (prāptāprāptavikalpagrāsāt) to lean towards an exception (paryudāsāśrayaṇe) and like (the rule) “one should not perform śrāddha in the night” by taking recourse to a secondary meaning in the statements mentioning the giving up of karma is cumbersome (karmatyāgavākyeṣvapi lakṣaṇābhyupagamena gauravāt); following the option of giving up karma in statements dealing with giving up karma it is also appropriate to understand (that it is) with reference to external karma alone (bāhyakarmatyāgaparatvasyaucityācca). Thus the utterances of learned scholars (vidvāns) like Jātaveda etc., “sarvāni bhūtāni…karmāṇi hi jātavedaḥ” (Mbh.Mokṣa. 244.25 Tri.p.5 fn.2) prescribing the abandonment of karma are to be understood as towards external karma alone; through such śruti statements as “ātmakrīḍa ātmaratiḥ kriyāvāneṣa brahmavidām variṣṭha”\endnote{Muṇḍaka.Up.III.1.4} one understands that even one who is blissful in the realization of the ātman (ātmārāmasyāpi) perform karma (karmāvagamāt). Through such verses as “jñāninā’jñāninā vāpi yāvaddehasya dhāraṇam…muktistasya kare sthitā” in smṛtis such as Vaśiṣṭha, Kūrma etc., learned scholars have reminded about the necessity of karma. Also in the case of internal karma there is no need to cite the reason for giving up karma being the discarding of violence involved (in external karma). Since there is no karma (prescribed) for samnyāsins to internalize in their bodies either the devas or agni etc., may also have the danger of being useless. 

\dev{ननु भवतामप्यारोपणमद्दृष्टार्थमेवाग्न्यादीनां देहे सत्त्वादिति चेन्न, बाह्याग्निसूर्याद्याधारकोपासनापरित्यागजदोषपरीहारस्यैव बाह्यग्न्याद्यारोपफलत्वात् । एवं ह्यन्तरे समष्टिव्यष्टिभेदेन सर्वदेवानां बाह्ययागोऽपि संपद्यत इति । एवमन्यान्यपि वाक्यानि “किमर्था वयं यक्षामहे” “अग्निहोत्रं न जुहवाञ्चक्रिरे”, “तस्मात् कर्म न कुर्वन्ति यतयः पारदर्शिनः । यदिदं वेदवचनं कुरु कर्म त्यजेति च ।।” इत्येवमादीनि विदुषां हिंसाविशेषादिदोषदुष्टबाह्यकर्मत्यागविधायकान्येव, न त्वावश्यकभिक्षादावप्यन्तर्यागस्य नमस्कारजपादेर्वा प्रतिषेधकानि । कर्मत्यागस्य शमादिपालनाय “गुणलोपो न गुणिन” इति न्यायेन दृष्टार्थकत्वात् । अत एव,}
\begin{verse}
\dev{ऋतुं प्राप्य यथा वृक्षः पत्रं त्यजति निस्पृहः ।}\\
\dev{तत्त्वं प्राप्य तथा योगी त्यजेत् कर्मपरिग्रहम् ।।}
\end{verse}
\dev{इत्यादिस्मृतिष्वपि कर्मपरिग्रहस्य कर्मोपकरणस्यैव त्यागोऽवगम्यते, अन्यथा त्यजेत् कर्म, इत्येव वक्तुमौचित्यात् । प्रणवजपे च शिखायज्ञोपवीताद्यभावेऽप्यधिकारोऽस्ति।  तापनोये “अशिखा अयज्ञोपवीता” इति परमहंसं प्रकृत्य “प्रणवे एव पर्यवसिता” इति श्रवणात् । यत्तु—}
\begin{verse}
\dev{यस्त्वात्मरतिरेव स्यादात्मतृप्तश्च मानवः ।}\\
\dev{आत्मन्येव च सन्तुष्टस्तस्य कार्यं न विद्यते ।।}\\
\dev{न केवलेन योगेन प्राप्यते परमं पदम् ।}\\
\dev{ज्ञानं तु केवलं सम्यगपवर्गफलप्रदम् ।।}
\end{verse}
\dev{इति भगवद्गीताकूर्मयोर्वाक्यं तन्निष्पन्नसमाधियोगिपरम् , तेषां बाह्यसंवेदनाभावेन कर्मसामान्याभावेऽदोषात् । तदुक्तं गीतायाम्—}
\begin{verse}
\dev{सर्वधर्मान् परित्यज्य मामेकं शरणं व्रज ।}\\
\dev{अहं त्वा सर्वपापेभ्यो मोक्षयिष्यामि मा शुचः ।। इति}
\end{verse}
\dev{तस्मात् स्वयमुद्भवत्समाधिं परित्यज्यान्तरमपि कर्म न कर्तव्यं प्रधानानुरोधेन गुणत्यागस्य न्यायसिद्धत्वात् । अत एवानेकसंवत्सरं व्याप्यापि बाह्याभ्यन्तरसर्वकर्मपरित्यागेन समाधाववस्थानं परमर्षीणां श्रूयत इति ।  तैरपि च व्युत्थानकाले   यथाशक्तिभिक्षादावन्तर्यागादिः क्रियत एवेति ।}

\textbf{Ques:}  If it is said that even in your case the internalization of fire etc., present in the body is for the purpose of adṛṣṭa then, we say;  

\textbf{Ans:} It is not so; it is for the sake of removing the defects caused by giving up meditation on the external fire which has for its support the sun, that the kindling of the external fire etc., within oneself is done (bāhyāgnisūryādyādhārakopāsanaparityāgajadoṣaprīhārasyaiva bāhyāgnyādyāropaphalatvāt). When it is so, one accomplishes the external sacrifice for all devas both collectively and individually (samaṣṭivyaṣṭibhedena). Thus other statements also by the learned like “kimarthā vayam yakṣyāmahe”, “agnihotram…cakrire”, “tasmāt karma…tyajeti ca” are prescribed for the giving up of external karma which are contaminated by defects such as violence and not against practice of internal sacrifice such as namaskāra, prayer etc., even during bhikṣā etc., which is a necessity.

In keeping with the maxim “guṇalopo na guṇina” it has a seen result. Therefore in such smṛti statements like “ṛtum prāpya yathā vṛkṣaḥ…tyajet karmaparigraham” “karmaparigraha” is understood as the giving up of the tools of rituals, otherwise it would have been proper to say “tyajet karma”. The (samnyāsin) has the adhikāra to do japa with praṇava (Om) even when he has no śikhā, yajñopavīta etc.  With reference to the paramahamsa (samnyāsin) it is known that “praṇave eva paryavasitā”. Statements such as “yastvātmaratireva… samyagapavargaphalapradam” in the Bhagavadgītā (Gītā) and Kūrma Purāṇa are with reference to a yogī who has already achieved samādhi; in their case, as there is the absence of external feelings, no defect is there in the absence of common karma. Thus the Gītā says: “sarvadharmān parityajya…mā śucaḥ”\endnote{Gītā 18. 66}. Therefore, even in between, when there is abandonment of the automatic rise of samādhi (svayamudbhavatsamādhim) one need not do karma, as it is logical (in conformity to the principle) that the subsidiary is given up when the important one is present (pradhanānurodhena guṇatyāgasya nyāyasiddhatvāt). That is the reason why enveloping even the lapse of many years (anekasamvatsaram vyāpyāpi) it is heard of great ṛṣis being absorbed in samādhi having given up both external and internal karma.They also practice internal sacrifice during the time of coming out of samādhi (vyutthānakāle) and during bhikṣā etc., as much as possible (yathāśakti).

\dev{यत्तु—}
\begin{verse}
\dev{“यथोक्तान्यपि कर्माणि परिहाय द्विजोत्तमः ।}\\
\dev{आत्मज्ञाने शमे च स्याद्वेदाभ्यासे च यत्नवान् ॥”}
\end{verse}
\dev{इति मनुवाक्यं तदशक्तपरं बोध्यम्। तत्र हि कर्मत्यागो न विधीयते किन्तु रोगाद्यशक्त्या कर्माभावे प्रसक्ते विद्वानात्मज्ञानादितत्परो भवेदित्येव विधीयते ।}

\dev{केचित्तु—}
\begin{verse}
\dev{“लोकेऽस्मिन् द्विविधा निष्ठा पुरा प्रोक्ता मयानघ।}\\
\dev{ज्ञानयोगेन सांख्यानां कर्मयोगेन योगिनाम् ।।”}
\end{verse}
\dev{इत्यादिवाक्येभ्यो विद्वदविद्वद्विषयभेदेन कर्मत्यागकर्मणोर्व्यवस्थेति वदन्ति, तन्न, उत्पन्नज्ञानस्यापि कर्मविधिश्रवणेन ज्ञानयोगशब्दस्य समाधिवाचकत्वात् । तथा च एकाग्रतालक्षणज्ञानयोगेन समाध्याख्येन सांख्यानां विवेकाभ्यासिनां निष्ठा ज्ञाननिष्पत्तिः । कर्मयोगेन तु योगिनां योगाभ्यासिनां निष्ठा योगारूढता भवतीत्यर्थः । तथा चोक्तं गीतायामेव—}
\begin{verse}
\dev{आरुरुक्षोर्मुनेर्योगं कर्म कारणमुच्यते ।}\\
\dev{योगारूढस्य तस्यैव शमः कारणमुच्यते ।। इति ।}
\end{verse}
\dev{नन्वेवं कर्मत्यागस्याविधेयत्वे “विधिर्वा धारणवत्” इति साधनाध्यायस्थसूत्रेण किं विधेयमुक्तमिति चेत् , संन्यासाश्रम इत्यवेहि । कः पुनरयं संन्यासशब्दार्थः १ उच्यते, “आत्मन्यग्नीन् समारोप्य ब्राह्राणः प्रव्रजेद् गृहाद्” इति मन्वादिवाक्योक्तः स्वशरीरेऽग्न्याद्यारोपणपूर्वकः पुत्रगृहाद्यभिमानत्यागेन गृहात् प्रव्रजनरूप आश्रम इति । अपरश्च केवलकर्माभिमानकर्मफलत्यागादिः “कुर्यात् कर्मसंन्यासचिन्तनम् । काम्यानां कर्मणां न्यासं संन्यासं कवयो विदुः” इत्यादियाज्ञवल्क्यगीताद्युक्तो गौणसंन्यास एव । नैष्कर्म्यसिद्धिं परमां संन्यासेनाधिगच्छति” इति गीतावाक्येन कर्मफलसंन्यासफलस्य नैष्कर्मसिद्धिरूपस्य मन्वाद्युक्तसंन्यासाश्रमस्य पूर्वोक्तफलसंन्यासापेक्षया परमत्ववचनादिति । नैष्कर्म्यशब्दश्च अनुगीतादौ संन्यासविशेषवाची दृष्टो, यथा—}

\dev{अभयं सर्वभूतेभ्यो दत्त्वा नैष्कर्म्यमाचरेत् । सर्वभूतसमो मैत्रः सर्वेन्द्रिययतो मुनि:  इति कर्माभिमानतत्फलत्यागयोश्च गौणसंन्यासत्वेऽपि मुख्यकल्पत्वं मन्तव्यम् । “तयोस्तु कर्मसंन्यासात् कर्मयोगो विशिष्यते” इति गीतावाक्ये अभिमानफलत्यागपूर्वकस्य
कर्मयोगस्यैव सर्वकर्मत्यागापेक्षया श्रेष्ठत्ववचनात् । समाधिप्रयुक्ताऽशक्त्या रोगदारिद्र्याद्यशक्त्या चैव हि बाह्यकर्मत्यागसिद्धिरिति । अत एव श्रुतिः “आत्मरतिः क्रियावानेष ब्रह्मविदां वरिष्ठः” इति । अत एव जनकादिषु वसिष्ठादिषु च ज्ञानकर्मसमुच्चय एव दृश्यत इति । अधिकं तु साधनाध्याये वक्ष्यामः ।}

The statement of Manu: “yathoktāni karmāni…yatnavān” (MS. 12.92) that should be understood as referring to one who is incapable (of performing karma). Therein there is no prescription of giving up karma but in the event of illness etc., when there is the absence of (performance of) karma then the vidvān can apply himself to knowledge of the self. Some say that according to such verses as “loke’smin dvividhā…karmayogena yoginām” (Gītā.3.3) by dividing the two types of people as vidvān and avidvān there is the arrangement of the action of giving up karma; that is not so; even in the case of one who has attained knowledge (of the Self)  due to the prescription of karma (utpannajñānasyāpi karmavidhiśravaṇena), the word jñānayoga (in the above verse) denotes samādhi. Thus by jñānayoga\endnote{Bhikṣu interpres jñānayoga as samādhi. He is reluctant to give up the paramount importance of yoga for achievement of liberation to any other means.} characterized by one pointedness known as samādhi (ekāgratālakṣaṇajñānayogena samādhyakhyena) those followers of Sāṅkhya who are steadfast in the practice of discriminate discernment (vivekābhyāsinām) for them the purpose is the attainment of knowledge. On the other hand through karmayoga the yogīs who practice yoga have the purpose of climbing the heights of yoga. Thus there is the  statement in the Gītā: “ārurukśormuneryogam… śamaḥ kāraṇamucyate” (Gītā.6.3).

\textbf{Ques: But then, in this manner, when giving up of ritual is not prescribed then what is prescribed by the śūtra “vidhirvā dhāraṇavat” (BS.III.4.20) in the chapter on the means.}

\textbf{Ans:} Please know that it is only the saṁnyāsāśrama that is prescribed here (in the sūtra). 

\textbf{Ques:} What then is the meaning of this saṁnyāsa word; 

\textbf{Ans:} In accordance with Manu’s statement  “Ātmanyagnīn samāropya…gṛhād” (not traced) it means that it is the āśrama of the nature of wandering around after giving up the sense of possession (ego) towards the son, home etc., having internalized within oneself the sacred fires. The other (meaning ) according to Yājñavalkya, Gītā etc., is just the giving up of the sense of agency (ego) in the results of karma etc., (or) “one should think of giving up ritual action” (not traced) “as the sages (kavayo) understand (viduḥ) that saṁnyāsa is the giving up of ritual which has desired results” (kāmyānām karmaṇām nyāsam saṁnyāsam kavayo viduḥ ) (Gītā 18.2); this is known as a lower form of saṁnyāsa (gauṇa saṁnyāsa eva). By the Gītā statement “naiṣkarmyasiddhim paramām saṁnyāsenādhigacchati”(18.49) it  is known that by the use of the word ‘paramām’ the attainment of the result of total abandonment of karma due to abandoning the result of karma done, is superior to the saṁnyāsāśrama mentioned already by Manu.\endnote{Karmayoga where one does karma without any expectation of result is superior to Manu's restricted definition of giving up only the desired karma (kāmyakarma).} And the word naiṣkarmya has a special saṁnyāsa connotation as seen in the Anugītā etc. Thus “abhayam sarvabhūtebhyo…sarvendriyayato muniḥ”. (Aśva-Parvan: chap 45 cited in Tripathi.p.7.fn.3). Even though the giving up the expectation of the result of action as well as the sense of one’s agency regarding the action is present in gauṇa-saṁnyāsa as well one needs to understand it as of prime importance (mukhyakalpatvam mantavyam).  The Gītā statement “tayostu…karmayogo viśiṣyate” (5.2) mentions that as compared to the abandonment of all action (sarvakarmatyāgāpekṣayā) karmayoga preceded by giving up a sense of agency as well as expectancy of results is superior. There is also the giving up of external ritual due to the lack of  attention or being ill or not having the means (to perform external karma) (samādhiprayuktā’śaktyā rogadāridryādyaśaktyā caiva). Thus we have the śruti statement “ātmaratiḥ kriyāvāneṣa brahmavidām variṣṭhaḥ” (not traced). That is the reason one sees a combination of both ritual and knowledge in people such as Vasiṣṭha and Janaka. We shall discuss this further in the Sādhanapāda.

\dev{तस्मात् श्रवणार्थमविदुषां सर्वकर्मत्यागो बाह्यकर्मत्यागो वा शक्तानां न शास्त्रार्थः ।}

\dev{आतुरस्य तु बाह्यकर्माशक्तस्यान्तरकर्मानुष्ठानपूर्वकं जाबालोक्तसंन्यासं नैव निराकुर्मः। }
\begin{verse}
\dev{प्रकर्तुमसमर्थश्चेज्जुहोति-यजतिक्रियाः ।।}\\
\dev{अन्धः पङ्गुर्दरिद्रो वा विरक्तः संन्यसेद् द्विजः  ।। इति कूर्मवाक्यादिति।}
\end{verse}
\dev{यत्र श्रौतोऽयं विविदिषूणामपि सर्वकर्मसंन्यासः “परीक्ष्य लोकान् कर्मचितान् ब्राह्मणो निर्वेदमायान्नास्त्यकृतः कृतेन तद्विज्ञानार्थं स गुरुमेवाभिगच्छेत् समित्पाणिः श्रोत्रियं ब्रह्मनिष्ठं तस्मै स विद्वानुपसन्नाय सम्यक्प्रशान्तचित्ताय शमान्विताय येनाक्षरं पुरुषं वेद सत्यं प्रोवाच तां तत्त्वतो ब्रह्मविद्यामि” ति श्रुतेः । अत्र गुरुमेवेत्येवकारादिभिः सर्वकर्मत्यागावगमादिति । तत्र एवकारादगुरुभावेनोपगमव्यावृत्तेरेव लाभान्न तु कर्मव्यावृत्तेः । “क्रियावन्तः श्रोत्रिया ब्रह्मनिष्ठा स्वयं जुह्वन्त एक ऋषिं श्रद्धयन्तः तेषामेवैतां ब्रह्मविद्यां वदेदि” ति वाक्यशेषे क्रियावतामेवोद्देश्यत्वविधानात् समित्पाणित्वलिङ्गेन होमावगमाच्च । समित्पाणिशब्दस्योपहारपाण्यर्थकत्वोपवर्णनन्तु स्वाज्ञानवर्णनमेव । नास्त्यकृतः कृतेनेत्यनेन च कर्मणां मोक्षसाधनत्वमेव निराकृतम् , फलवत्सन्निधावफलस्यैवाङ्गत्वेन तस्य मोक्षासाधनत्वात् , न तु कर्मत्यागो विधीयत इति । प्रशान्तचित्तायेत्यनेनापि विद्याग्रहणोपयोग्येव प्रशम उक्तः न तु कर्मत्यागं, अर्जुनादिकर्मिभ्योऽपि ब्रह्मविद्योपदेशदर्शनादिति ।}

\dev{तथा “न कर्मणा न प्रजया न धनेन” इत्यादिश्रुतिरपि कर्मादीनामङ्गानामभावेऽपि एकेषां जडभरतादीनां मोक्षमेवानुवदति न तु कर्मत्यागं विदधाति । या च पुत्रैषणादित्यागश्रुतिः, साऽपि पुत्रादीच्छानामेव त्यागं विदधाति न तु कर्मणः ।}
\begin{verse}
\dev{त्यज धर्ममधर्मञ्च उभे सत्यानृते त्यज ।}\\
\dev{उभे सत्यानृते त्यक्त्वा येन त्यजसि तत्त्यज ।।}
\end{verse}
\dev{इति श्रुतिश्च धर्मादेस्तत्त्यागहेतुविवेकख्यातेश्च रागद्वेषादित्यागं विदधाति । “एतान्यपि तु कर्माणि सङ्गं त्यक्त्वा फलानि च । कर्तव्यानी” त्यादिगीताद्येकवाक्यत्वात् । अन्यथा जीवतां धर्माधर्मादित्यागासंभवात् , मूत्रोत्सर्गादावप्यन्ततः पिपीलिकादीनां विनाशात् ,}
\begin{verse}
\dev{न हि देहभृता शक्यं त्यक्तुं कर्माण्यशेषतः ।}\\
\dev{यस्तु कर्मफलत्यागी स त्यागीत्यभिधीयते ।।}
\end{verse}
\dev{इति गीतावाक्यादिति ।।}

Therefore, for the (understanding) of the common folk (aviduṣām) giving up all karma (sarvakarmatyāga) or giving up external karma (bāhyakarmatyāgo vā) in the case of those who can (śaktānām), is not the meaning of the śastra. For one who is ill, who is not capable of performing external karma we need not reject the saṁnyāsa preceded by observing internal karma mentioned by Jābāla. Thus there is this statement in the Kūrma. P “prakartumasamarthaścet…saṁnyased dvijaḥ” (Kūrma.P.adh.3.10 cited in Tri.p.8.fn.2) .  According to śruti regarding giving up karma even for those seeking to know about Brahman (vividuṣūṇāmapi) it says “parīkṣya lokān…brahmavidyām”(Muṇḍ.Up.I.2.12). 

\textbf{Ques:} In this case by stating ‘gurumevābhigacchet’ (approach only the guru) one understands the giving up of all karma. 

\textbf{Ans:} Therein the use of the word ‘eva’ only means to exclude approaching with the feeling of not having a guru and not of excluding ritual (karma). This is also understood from the remaining portion “kriyāvantaḥ śrotriyā…vadet” iti, and also by the use of the word ‘samit in the hand’ (samipāṇitvaliṅgena) one understands a homa (fire sacrifice). To understand the word ‘samitpāṇi’ as having some gift in hand only reveals one’s own ignorance. Through the words “nāstyakṛtaḥ kṛtena” (in the above śruti quotation) there is rejection of only karma being a means to mokṣa; since not having a result in the presence of the attainment of a result means (phalavat sannidhāvaphalasyaivāṅgatvena) it is not considered a means to mokṣa; however giving up karma is not prescribed; this is also demonstrated by the advice regarding knowledge of Brahman given even to those devoted to karma like Arjuna etc.

Even śruti words “na karmaṇā na prajayā na dhanena” etc., (Mahānā.Up. 8.14) only reiterate (anuvadati)  the mokṣa of persons such as Jaḍabharata etc., even when there is absence of ritual action etc., and do not advocate the giving up of ritual (karmatyāgam). The śruti statements such as giving up the desire for progeny etc., (putraiṣaṇādi tyāgaśrutiḥ) only mention the giving up the desire for putra etc., but not that of ritual karma. The śruti statement: “tyaja dharmamadharmañca…tyajasi tattyaja”(not traced), prescribe the giving up of attachment and hatred and also mention the giving up of discriminate discernment which is the cause of abandoning dharma etc. This is in concordance with the meaning given in the Gītā statement “etānyapi tu…tyaktvā phalāni ca” (Gītā 18.6). It is not possible while living to give up dharma and adharma, as even when one urinates or defecates there is destruction of ants etc.\endnote{The idea seems to be that since there is killing even in ordinary activities like urinating etc., one is subject to both dharma and adharma even while living. Adharma  is because of killing insects like ants etc., and dharma is prāyaścitta for such adhārmic acts I suppose. } This is made clear in the Gītā statement “na hi dehabhṛta…tyāgītyabhidīyate” (Gītā.18.11).

\dev{तस्मात् “विद्यां चाविद्यां च यस्तद्वेदोभयं सहे” त्यादिपरब्रह्मप्रकरणस्थवाक्यविरोधात् सर्वकर्माणि सन्न्यस्य श्रवणं कुर्यादित्याधुनिकानां विधिकल्पनं कुकल्पनमेवेति द्रष्टव्यम् । अतएव श्रवणानन्तरमेव मनुना संन्यासाश्रमो विहितः “वेदान्तान् विधिवत् श्रुत्वा संन्यसेदनृणो द्विज” इति । तस्मात् संन्यस्य श्रवणं कुर्यादित्यस्यायमेवार्थोऽज्ञस्याशक्त्या बाह्यकर्मत्यागरूपसंन्यासे सति श्रवणमवश्यं कर्तव्यमिति ।}

\dev{ज्ञानकर्मसमुच्चये चायं विशेषः “आरुरुक्षोर्मनेर्योगं कर्म कारणमुच्यते । योगारूढस्य तस्यैव शमः कारणमुच्यते” इत्यादिवाक्यानुसारेणावगन्तव्यः— आरुरुक्षुत्वेनोत्सर्गतो ब्रह्मचारिगृहस्थयोः कर्माधिक्येनानुष्ठेयं ज्ञानं तु तदुपसर्जनम् , युज्यमानस्य तथा च वानप्रस्थस्योभे एष समे समच्चिते, योगारूढतया तु परिव्राजकस्य ज्ञानं प्रधानम् आन्तरं कर्म तदुपसर्जनमिति । एतेषु प्रकारेषूत्तरोत्तरमायासबाहुल्याद् हिंसादिदोषह्रासाच्च मोक्षाख्यफलेऽप्याश्वाशुतरादिरूपेण विशेषो मन्तव्यः ।}
\begin{verse}
\dev{जन्मान्तरैरभ्यसतो मुक्तिः पूर्वस्य जायते ।}\\
\dev{विनिष्पन्नसमाधिस्तु मुक्तिम् तत्रैव जन्मनि ।।}
\end{verse}
\dev{प्राप्नोतीति विष्णुपुराणादिति । पूर्वस्य पूर्वोक्तस्य युज्यमानस्येत्यर्थः । यथोक्तव्यवस्थया च प्रत्येकप्राधान्यसमसमुच्चयबोधकवाक्यानां पूर्वोत्तरमीमांसयोश्चाविरोध इति । प्रत्येकप्राधान्यसमुच्चयवाक्यं च यथा मोक्षधर्मे—}
\begin{verse}
\dev{ज्ञाननिष्ठां वदन्त्येके मोक्षशास्त्रविदो जनाः ।}\\
\dev{कर्मनिष्ठां वदन्त्यन्ये यतयः सूक्ष्मदर्शिनः ।।}\\
\dev{प्रहायोभयमप्येतज्ज्ञानं कर्म च केवलम् ।}\\
\dev{तृतीयेयं समाख्याता निष्ठा पञ्चशिखेन मे ।।}\\
\dev{तेनाहं सांख्यमुख्येन स्वदृष्टार्थेन तत्त्वतः ।}\\
\dev{श्रावितस्त्रिविधं मोक्षं न च राज्यIद् विचालितः ।।}
\end{verse}
\dev{इति जनकेनोक्तम् । तत्र कर्मप्राधान्यसमसमुच्चयज्ञानप्राधान्यानां फलरूपं त्रिविधं मोक्षं क्रमेणाह तत्त्वसमासाख्यसांख्यस्य भाष्ये पञ्चशिखाचार्यः—}
\begin{verse}
\dev{आदौ तु मोक्षो ज्ञानेन द्वितीयो रागसंक्षयात् ।}\\
\dev{कृच्छ्रक्षयात्तृतीयस्तु व्याख्यातं मोक्षलक्षणम् ।।}
\end{verse}
\dev{इति । कृच्छ्रक्षय आत्यन्तिकदुःखाभावः ।}

Therefore, one should know that there is the mistaken view (kukalpanameveti) of the modern day Vedāntins misinterpreting the sentence in the chapter on the supreme Brahman like “vidyām cāvidyām ca yastadvedobhayam saha” (Īśa.Up.11; Mait.Up.7.9) as giving up all ritual karma and prescribing only listening to śruti (śravaṇam kuryāt). That is why Manu has prescribed the saṁnyāsāśrama only after listening to śruti etc.,\endnote{This is a reference to Manu’s prescribing the succession of the āśramas one after another (brahmacarya,
gṛhastha, vānaprastha and then saṁnyāsa} in the statement “vedāntān vidhivat śrutvā saṁyasedanṛṇo dvija” (not traced ).There is this special characteristic in combining knowledge and ritual (jñānakarma samuccaye) and it has to be understood according to the statement “ārurukṣormuneryogam karma kāraṇamucyate; yogārūdhasya tasyaiva śamaḥ kāraṇamucyate” (Gītā. 6.3)\endnote{Bhikṣu uses this quote generously in all his works and is a firm believer in the combination of karma and jñāna till the very last stage of yoga.}. Generally for a brahmacārī and a gṛhastha who desire to become saṁnyāsins (muneryogam) there are a number of rituals to be observed and knowledge is subsidiary to karma (jñānam tu tadupasarjaman); for one who is striving and for one in the vānaprastha stage both karma and jñāna are equally important; whereas for the saṁnyāsin who has reached the summit of yoga knowledge is important and internal karma is its subsidiary (āntaram karma tadupasarjanam). In this succession of stages since there is great exertion in later stages and there is a diminishing of violence (in later stages) and also a growing confidence in the attainment of mokṣa, it can be considered as having this special quality (mokṣākhyaphale’pyāśvāśutarādirūpeṇa viśeṣo mantavyaḥ).Thus from the Viṣṇu Purāṇa statement “janmāntarairabhyasato…muktim tatraiva janmani” we understand that one attains (mukti there itself). The word “pūrvasya” in the above quotation refers to the one engaged in yoga (yujyamānasya). By the above mentioned methodology of explaining each mainly as an equal combination (of jñāna and karma) (yathoktavyavasthayā ca pratyekaprādhānyasamasamuccayabodhakavākyānām) it is not in opposition to pūrva and uttaramīmāṁsā. Thus this equal combination mainly of (jñāna and karma) is stated by Janaka in the Mokṣadharmaparvan as follows: “jñananiṣṭhām vadantyeke mokṣaśāstravido janāḥ…śrāvitastrividham mokṣam na ca rājyād vicālitaḥ”. There the threefold result of liberation of mainly the equal combination of karma and jñāna is mentioned in sequence (krameṇāha)as follows by Pañcaśikhācārya in the commentary on the Sāmkhya work Tattvasamāsa: “ādau tu mokṣo…mokṣalakṣaṇam”. “kṛcchrakṣaya” in the above quotation means absolute absence of sorrow.\endnote{This is Bhikṣu’s idea of liberation which he also reiterates in many of his works like the Yogavārttika and the
Kaṭhopaniṣadaloka commentary.}  \dev{इदानीं समाधिरहितस्यापि विविदिषुतामात्रेण सर्वकर्मत्यागे परोक्तकुतर्का अपि परिह्रियन्ते—}

\dev{ननु श्रुतिस्मृतीनां कर्मत्यागाविधायकत्वेऽपि ज्ञानिनामर्थादेव कर्मत्यागो लभ्यते ज्ञानकर्मणोरेकदा विरोधादिति, मैवम्, यदि शानशब्देन समाधिरुच्यते तदा एकदा विरोधेऽपि व्युत्थानदशायां भिक्षाटनात्मयागसंभवात् ,  कालाद्यङ्गहानावपि योगिनामदोषस्योक्तत्वात् । यदि पुनर्ज्ञानशब्देनानुभवमात्रमुच्यते तदा तेन सह कर्माभिमानस्येव विरोधान्न त कर्मणः जडभरतादीनां भिक्षाटनादेर्भवतामप्यभ्युपगमात् , “कुर्याद् विद्वांस्तथाऽशक्तश्चिकीर्षुर्लोकसंग्रहमि” त्यादि वाक्यैर्वैदिककर्मणामपि विदुषिविधानाच्च । न चाभिमानाद्यभावेन ज्ञानिनां कर्म निष्फलमेवेति तत्त्याग एवोचित इति वाच्यम्, }
\begin{verse}
\dev{उत्सिदेयुरिमे लोका न कुर्यां कर्म चेदहम् ।}\\
\dev{सङ्करस्य च कर्ता स्यामुपहन्यामिमाः प्रजाः ।।}
\end{verse}
\dev{इत्यादिवाक्यतो लोकविनाशनजप्रत्यवायानुत्पत्तेरेव फलत्वसिद्धेः । “योगिनः कर्म कुर्वन्ति सङ्गं त्यक्त्वाऽऽत्मशुद्धये” इत्यादिवाक्यैः सत्त्वशुद्ध्याख्यपापक्षयहेतुतयैवाभिमानाख्यसङ्गशून्यकर्मणां साफल्याच्च । अन्यथा पापजन्यरोगादिना शरीरनाशादिभिर्ज्ञानपरिपाके प्रारब्धाक्षयेनैव प्रतिबन्धसंभवात् “कृतनियमलङ्घनादानर्थक्यमिति’’ सांख्यसूत्रादिभ्य आनर्थक्यं ज्ञानस्येति प्रकृतत्वाल्लभ्यते ।}

\dev{नन्वविद्याया अदृष्टहेतुत्वादविद्यानिवृत्तौ पापोत्पत्तिरेव न सम्भवतीति कुतो मोक्षप्रतिबन्धरशङ्केति चेन्न, ज्ञानिभ्योऽपि कर्मविधानाद् वृथाकर्मत्यागजपापातिरिक्तादृष्टेष्वेवाऽविद्याया हेतुत्वावधारणात् । }

\dev{अतएव “आप्रायणात् तथाहि दृष्टमि” ति सूत्रेण मरणपर्यन्तमेवानुभविनोऽपि विद्यावृत्तिं वक्ष्यति प्रतिबन्धनिरासार्थमिति अतोऽपि चानिर्दिष्टकालविशेषाणां ज्ञानाङ्गकर्मणां ज्ञानकालिकतया मरणपर्यन्तता लभ्यत इति ।}

Now the false argument of others of giving up all karma even by those not attaining samādhi (samādhirahitasyāpi) but just desiring (to attain knowledge of the Self) (vividiṣutāmātreṇa) is removed (as follows)—

\textbf{Ques:} But then, even though śruti and smṛti (texts) do not prescribe giving up of karma still by the very meaning of ‘those possessed of knowledge’ (jñānināmarthādeva) giving up of karma is implied, as there is a contradiction in both having karma and jñāna at the same time.

\textbf{Ans:} No it is not so; if by the word jñāna, samādhi is implied then even though there is a contradiction of both (karma and jñāna) coexisting simultaneously, during the intervals of activity (vyuthānadaśāyām) self-sacrifice is possible when wandering about for alms (bhikṣāṭanātmatyāgasaṁbhavāt); even when due to passage of time the limbs are weak no blame is there for the yogīs. If then by the word jñāna, only the experience (of realization) is intended then it is only in contradiction to the sense of agency of karma and not to karma per se. Since you also accept the wandering for alms of people like Jadabharata etc., and through such statements as “kuryād vidvāmstathā…lokasangraham” (Gītā.3.25) even in medical pertaining to the Vedas (vaidika) matters lack of blame is prescribed in one who knows.\endnote{Since the doctor treats with the idea of welfare for all he is not to blame if something goes wrong.}  It cannot be said that the actions of wise people (jñāninām karma) done without any pride (abhimānādyabhāvena) is not without a result. From such statements as “utsideyurime…prajāḥ” (ibid.3.24) it is understood that the result is the non-arising of impediments which lead to the destruction of the world.\endnote{The Vedic belief that the good actions of all humans contribute to the maintenance of dharma and stability
of the world is echoed here.} Statements such as “yoginaḥ…ātmaśuddhaye” (ibid.5.11) indicate the expected good results of deeds done without any sense of agency/pride due to the decline of pāpa due to the cause known as sattva-śuddhi (sattvaśuddhyākhyapāpakṣayahetutayaiva abhimānākhyasaṅgaśūnyakarmaṇām sāphalyācca)\endnote{Bhikṣu is committed to Sāṁkhya-Yoga metaphysics and epistemology. He therefore talks about the gradual
sattva-śuddhi of the intellect of selfless action. All schools of Indian philosophy believe in selfless action
(niṣkāma-karma) as a preliminary step in the quest for liberation.}. Otherwise because of the body weakening due to diseases caused by pāpa, since the fruition of knowledge can only happen when opposed by the weakening of the prārabdhakarma, one understands from Sāṁkhyasūtras like “kṛtaniyamalaṅghanādānarthakyam”etc., that knowledge which is understood from what is mentioned (prakṛtatvāllabhyate) is useless.

\textbf{Ques:} But then since avidyā is the cause of adṛṣṭa, when avidyā disappears (in the case of a jñānī) (avidyānivṛttau) there can be no rise of pāpa; then where is the doubt of having an obstacle for mokṣa.

\textbf{Ans:} That is not so; even in the case of those who are jñānīs since karma is prescribed, it is useless to assign avidyā as a cause for only adṛṣṭa apart from the pāpa arising from the giving up of karma. That is why by the sūtra “āprāyaṇāt tathāhi dṛṣtam” (BS.4.1.12) even for those who have experienced the truth (anubhavino’pi) the observance of vidyā will be mentioned up to the time of death for the sake of removal of obstacles. Moreover engaging in karma which are the limbs of knowledge, having no fixed times (anirdiṣṭakālaviśeṣam) one understands the limit for the rise of knowledge is up to the time of death. 

\dev{ननु अभिमानाभावेनात्मनि नियोज्यत्वप्रत्ययासम्भवाद् विदुषां विद्याकिंकरता न सम्भवतीति चेन्न, अभिमानाऽभावेऽप्युपाधौ नियोज्यत्वप्रत्ययसम्भवात् अन्यथा भिक्षादावपि नियोज्यत्वप्रत्ययो न स्यात् ।}

\dev{स्यादेतत्, प्रवृत्तौ स्वीयसुखादिसाधनताज्ञानं कारणं, परसुखादिषु तत्साधने वा हानोपादानादर्शनात् । तथा च सुखादीनामनात्मधर्मतया तेषु स्वीयत्वज्ञानाऽसम्भवात् कथं तत्साधने कर्मादौ विदुषां प्रवृत्तिः स्यादिति ? उच्यते, सांख्यादिभिः स्वभोग्यत्वमेव धनादिष्विव सुखदुःखयोरपि स्वीयत्वमुच्यते, न तु नैयायिकादिवत् स्वसमवेतत्वम् । स्वभोग्यत्वं च स्वसाक्ष्यत्वम्। साक्ष्यत्वं चोपाधिवृत्तिविषयत्वं विना भास्यत्वमतो न योगिनां परमेश्वरस्य वा परसुखादिषु स्वीयत्वम् । एतेन ज्ञानिनां स्वकृतिसाध्यतादिज्ञानमप्युपपादितम् । स्वोपाधिकृतावपि साक्ष्यत्वरूपस्वीयत्वसंभवात्, सांकारादी धनादौ च स्वीयसुखादिसाधनत्वात् स्वीयत्वम् । नन्वेचं नाहं सुखीत्यादिविवेको नोपपद्यतेति चेत् न, समवायसम्बन्धेन अहं सखीत्यादिप्रत्ययस्यैवाविद्यात्वेन समवायसम्बन्धावच्छिन्नतदभाचप्रत्ययस्यैव विवेकशब्दार्थत्वात्, अधिकारित्वेनैव शास्त्रेषु बुद्ध्यादिभ्य आत्मनो विवेचनाच्चेति । तस्माद् विदुषामपि स्वीयसुखादिसाधनताज्ञानं स्वोपाधिप्रवृत्तेर्हेतुः संम्भवत्येव ।}

\textbf{Ques:} If it is said that due to absence of ego there is no possibility of the idea of being one charged with any duty in the ātman so there is no possibility for knowledge  to have the capacity to accomplish its purpose for the wise (viduṣām kimkaratā na sambhavatiti) then (there is the following)

\textbf{Ans:} It is not so; even when a sense/knowledge of agency is absent,  in the limitation there is the sense of being charged with duty (upādhau niyojyatvapratyayasaṁbhavāt); otherwise even in the act of wandering for alms etc., there will not be the idea of being charged with a duty.

\textbf{Ques:} Let this be. In any activity the cause is the knowledge of its being a means to one’s own happiness (svīyasukhādisādhanatājñānam kāraṇam); one does not see either acceptance or rejection with regard to another’s pleasure or in its being an instrument for this (parasukhādiṣu tatsādhanevā hānopādānādarśanāt). Thus since pleasure etc., being qualities of the non-ātman it is not possible to have a sense of them belonging to oneself (svīyatvajñānā’saṁbhavāt); so how can the wise be engaged in karma which accomplishes such results (tatsādhane karmādau viduṣam pravṛttiḥ syāditi?).

\textbf{Ans:} It is said by Sāmkhya believers that both pleasure and pain are also to be enjoyed by oneself (svabhogyatvameva svīyatvamucyate) just as wealth etc., is fit to be enjoyed (by oneself. It is not like the view of the Naiyāyikas that it is in a relationship of inherence (svasamvetatvam). And being able to enjoy oneself means the quality of being a witness oneself. And the quality of being a witness is being able to possess knowledge without the quality of being an object of the modification of the limitation (i.e. the mind) (copādhivṛttiviṣayatvam vinā bhāsyatvam); thus neither the yogīs nor Parameśvara have the sense of the pleasure etc., of others belonging to itself. The result of this is that it also makes it possible for the wise to have knowledge etc., which accomplishes one’s own purpose (svakṛtisādhyatādijñanamapi upapāditam). Even though achieved by the limitation, as there is the possibility of possessing it as one’s own in the form of the witness (sākṣyatvarūpasvīyatvasaṁbhavāt), therefore both in the case of saṁskāras etc., or wealth and so on, since it is able to accomplish the idea of pleasure belonging to oneself, one calls it as belonging to oneself.

\textbf{Ques:} But then if it is said that there will not be the rise of the wisdom in the form ‘I am not the repository of pleasure’ etc.,(nāham sukhītyādiviveko nopapadyeta) then the answer is:

\textbf{Ans:}  No; due to the relationship of inherence, since the thought such as “I am happy” etc., are due only through having ignorance (avidyātvena), the word meaning ‘insight’ indicates the absence of thought which has the the delimitation of the relationship of inherence (samavāyasaṁbandhāvacchinnatadabhāvapratyayasyaiva vivekaśabdārthatvāt). Moreover in the authoritative texts (śāstreṣu) ātman has been differentiated from the intellect etc., as not being subject to any change (avikāritvenaiva buddhyādibhya ātmano vivecanācceti). Therefore even in the case of the wise the knowledge of the means of one’s own pleasure etc., is only possible due to the cause of the activity of one’s own limitation. 

\dev{अथ तथापि आत्मतृप्ततया ज्ञानिनां सुखादाविच्छैव नास्तीति चेन्न, रागरूपाया अविद्याजन्येच्छाया एव ज्ञानिनां शास्त्रेषु प्रतिषेधात् “दुखजन्मप्रवृत्तिदोषमिथ्याज्ञानानामुत्तरोत्तरापाये तदनन्तरापायादपवर्ग” इति न्यायसूत्रादिभिः, न तु लीलारूपादीच्छा प्रतिषिद्धा भिक्षादिदर्शनविरोधात् । ननु तथापि देहेन्द्रियादिभिः सम्बन्धाभावाद् विदुषां कथं प्रवृत्तिः ? न ह्यसङ्गस्यात्मनो देहादिभिः सहाभिमानातिरिक्तः सम्बन्धः सम्भवतीति चेन्न, विदुषां ज्ञानोपपत्यर्थम् असङ्गवाक्यैर्लोपाख्यस्य विकारहेतुसंयोगस्यैव निषेधात् न तु पुष्करपत्रे जलस्येवासङ्गेऽपि चेतने स्योपाधेः संयोगविशेषः प्रतिषिध्यते “आत्मेन्द्रियमनोयुक्तं भोक्तेत्याहुर्मनीषिणः” इति श्रुतेः । स च संयोगविशेषः प्रारब्धकर्मक्षयादेव नश्यतीति सर्वैरेवाभ्युपेयम् ; अन्यथा ज्ञानाद्यनुपपत्तरिति ।}

\dev{विषयैः प्रतिबिम्वरूपस्य बन्धस्यापि च ज्ञानिसाधारण्याद् विषयभोगोऽपि ज्ञानिनामुपपन्नः । स्वप्रतिबिम्बितस्योपाधिसुखस्य भानमेव भोग इति वक्ष्यमाणत्वात् ।}

\dev{[न च “सति मूले तद् विपाको जात्यायुर्भोगाः” इतिपातञ्जलसूत्रेण क्लेशसत्त्व एच कर्मविपाको भवतीति वचनात् कथं विदुषां भोगः स्यादिति वाच्यम् ‘‘क्लेशाभावे कर्मविपाकारम्भो¹ न भवती” ति तत्सूत्रभाष्यतो विपाकारम्भे क्लेशहेतुताया एव तत्सूत्रार्थत्वात्, ज्ञानिना च प्रारब्धविपाकमेव भुज्यत इति ।]}

\textbf{Ques:} But then if it is said that due to being satisfied within oneself (knowing the truth about the self), the wise have no desire for pleasure etc., (ātmatṛptatayā jñāninām sukhādāvicchaiva nāstiti cenna) then the answer is:

\textbf{Ans:} It is not so; it is only desire in the form of attachment which is desire that arises out of avidyā which is prohibited in the śāstras; through such Nyāyasūtras as “duḥkhajanma…tadanantarāpāyādapavarga” (I.1.2) there is no prohibition of desire which is of the nature of sport (na tu līlārūpādīcchā pratiṣiddhā), as that will be in contradiction to what we see as roaming around for alms.

\textbf{Ques:} But even then how can there be any activity of the wise in the absence of a relationship with the body, sense organs etc. If it is said that the non-attached ātman has a relationship apart from that associated with a sense of agency (abhimānātiriktaḥ saṁbandhaḥ saṁbhavatīti) then the answer is:

\textbf{Ans:} It is not so; for the sake of rise of knowledge in the wise through statements which denote detachment, there is rejection of any contact with the cause for change, due to what is known as any impurity (asaṅgavākyairlepākhyasya vikārasaṁyogasyaiva niṣedhāt); nor is there rejection of a special contact of caitanya with its limitation (cetane svopadheḥ saṁyogaviśeṣah niṣidhyate); even if there is no contact like a lotus leaf with the water on it (na tu puṣkarapatre jalasyevāsaṅge’pi), there is no rejection of a special contact of caitanya with its limiation . This is in accordance with the śruti statement: “ātmendriyamanoyuktam…manīṣiṇaḥ” (Kaṭh.Up.I.3.4). And all agree that special contact will only be destroyed when the prārabdha-karma comes to an end; otherwise there will be an absence of reasonable ground for (the rise of) knowledge etc.

Bondage through objects in the form of reflection being common, it is also logical for the wise to have experience (viṣayabhogo’pi jñānināmupapannaḥ). This is so since it will be mentioned that experience is only the knowing/illumination of pleasure of the limitation which is reflected by oneself (svapratibimbitasyopādhisukhasya bhānameva boga iti vakṣyamāṇatvāt).\endnote{This is in accordance with the epistemology of SY} Ques: Since through Patañjali’s sūtra “sati mūle tadvipāko jātyāyurbhogāḥ” (YS.II.13) we know that only when there is kleśa will there ensue the result of karma (the question is) how can there be experience in the case of those who are wise (who have realized the truth due to absence of kleśa). 

\textbf{Ans:} “kleśābhāve…na bhavati”—this bhāṣya on that sūtra clarifies that the meaning of that sūtra is that in the rise of vipāka the cause is kleśa; therefore it means that the vipāka of the prārabdhakarma alone is experienced by the knowing one (jñaninā).

\dev{ये स्वाभिमानमेव बुद्ध्यात्मनोः सम्बन्धं मन्यन्ते, तेषामेव ज्ञानेन सवासनाज्ञाननिवृत्त्या विदुषां भोगानुपपत्तिः । ज्ञानपरिपाकोत्तरमपि यावद्देहपातमविद्यालेशस्वीकारे च स लेशो न नश्येतैव, देहारम्भककर्मनाशस्य वासनानाशकत्वे प्रमाणाभावात्, अविदुषोऽपि पुनर्जन्मासम्भवाच्च । स्वयं विनाशे कदाचित् ज्ञानं विनाऽपि मुक्तिः स्यात् । अज्ञानज्ञानयोर्नाश्यनाशकभावे व्यभिचारप्रसङ्गाच्च । कार्यतावच्छेदकविशेषस्य च दुर्वचत्वात् ।}

\dev{किञ्च बुद्ध्यात्मनोरध्यासरूपः सम्बन्धः किमहङ्करोमीत्यादिविशिष्टबुधिनियामकतया कल्प्यते? किं वा बुद्धिप्रवृत्तिनियामकतया ? नाद्यः, अहङ्करोमीत्यादिप्रतीतेः सम्बन्धविधया स्वविषयत्वे कर्मकर्तृविरोधात्, स्वजनकत्वे तु आत्माश्रयात्,  अधिकाधिकाङ्गीकारे चान्योन्याश्रयचक्रकानवस्थादिप्रसङ्गात् , अहं कर्तेत्यादि विशिष्टबुद्धिविवेकज्ञानाऽनाश्यत्वप्रसङ्गाच्च मिथ्याज्ञानरूपसम्बन्धस्य समवायादिवत् पारमार्थिकत्वादिति ।}

\dev{न द्वितीयः, ईश्वरस्याज्ञानाभावेनोपाधिसम्बन्धाभावप्रसक्त्या तस्य विश्वनिर्मातृत्वानुपपतिः। ननु प्रपञ्चदर्शनानुपपत्त्येश्वरस्याप्यगत्या बाधितार्थाभिमान आहार्यज्ञानरूपः कल्पनीय इति चेन्न, चेतने प्रतिबिम्बितस्यैव विषयभानहेतुताया वक्ष्यमाणत्वान्न त्वज्ञानस्य । अन्यथेश्वरस्य क्लेशकर्मादिशून्यताप्रतिपादकश्रुतिस्मृतिविरोधापत्तेः । नन्वेवमसत्प्रपञ्चाकारा वृत्तिरेवेश्वरोपाधावज्ञानं स्यादेवेति चेन्न, प्रपञ्चस्याऽत्यन्ततुच्छताया निरकरिष्यमाणत्वात् । तस्मात् किमर्थमभिमानस्य सम्बन्धत्वमिति न विद्मः ।}

\dev{अत आत्मानात्मनोः परस्परप्रतिबिम्बो ज्ञाननियामकः सम्बन्धः । बुद्धेः प्रवृत्तिहेतुस्तु पद्मपत्रजलयोरिव संयोगविशेषः सम्बन्धः,}
\begin{verse}
\dev{यथाकाशस्थितो नित्यं वायुः सर्वत्रगो महान् ।}\\
\dev{तथा सर्वाणि भूतानि मत्स्थानीत्युपधारय ।।}
\end{verse}
\dev{इत्यादिवाक्यर्लेपसङ्गादिनियामकस्य स्वाश्रयविकारहेतोः संयोगविशेषस्यैव प्रतिषेधावगमादिति आत्मा- नात्मनोरभेदबुद्धिस्तु भ्रम एवेति न तदर्थं सम्बन्धापेक्षेति ।}

Those who believe that the relationship of the ātman and the intellect is (in) the sense of agency, for them alone, is there a contradiction in understanding logically the knowledge of experience of knowers due to the cessation of knowledge accompanied by subtle impressions (jñānena savāsanājñānanivṛttyā viduṣām bhogānupapattiḥ). Even after knowledge has reached fruition, when one accepts a trace of avidyā left, that trace is not destroyed till the time of the fall of the body as there is no authority which states that when there is the destruction of the subtle impressions of karma then the karma that gave rise to the body is destroyed (dehārambhakakarmanāśasya vāsanānāśakatve pramāṇābhavāt).\endnote{The destruction of the subtle impressions of karma will ensure that one has realized the truth; but that does not entail that the body will also fall since the prārabdhakarma which gave rise to that particular body has to
come to an end.} (In that case) even for those who have not realized the truth, reincarnation is not possible. When it is destroyed by itself, then even without realizing the truth, liberation can occur; when there is the relationship of destroyed and destroyer between ignorance and knowledge there can be a contingency regarding exceptions also (ajñānajyanayornāśyanāśakabhāve vyabhicāraprasaṅgācca). It is also difficult to describe the special quality of the delimitation of that which contains the effect (kāryatāvacchedakaviśeṣasya ca durvacatvāt)\endnote{Bhikṣu uses navya nyāya terminology in his works and which is evident here as well as in his other works as the Yogavārttikā for instance.}. Moreover is the relationship of the intellect and ātman of the nature of a superimposition imagined by the defining rule of special knowledge like “what is it that I am doing” etc., or is it defined by the activity of the intellect? It is not the first; as in the knowledge of the form “what is it that I am doing” etc., due to the rule of relationship when the object is for oneself (svaviṣayatve) there is the contradiction of the object and subject being the same; if given rise to by oneself it has the fallacy of being dependent on oneself; if one accepts more and more intermediaries then one is caught in the circle of mutual dependency and inconsistency of its being without any end (adhikādhikāṅgīkāre cānyonyāsrayacakrakānavasthādiprasaṅāt). Since there is also the contingency of the non-destruction of the special insightful knowledge as “I am the doer” (viśiṣṭabuddhivivekajñanā’nāsyatvaprasaṅgācca) the relationship in the form of an illusory knowledge is also existent just as the relation of inherence etc., (aham kartetyādi viśiṣṭabuddhivivekajñā’nāśyatvaprasṅgācca mithyājñānarūpasaṁbandhasya samvāyādivat pāramārthikatvāditi).

It cannot be the second; since there is absence of ignorance in Īśvara and due to the possibility of absence of relationship with a limitation there will be the incontingency of the creation of the universe (īśvarasyājñānābhāvenopādhisambandhābhāvaprasaktyā tasya viśvanirmātṛtvānupapattiḥ)\endnote{Bhikṣu believes that Īśvara is the efficient cause of the universe}

\textbf{Ques:} Since there is the contingency of the visibility of the universe so without any other explanation/resource available (for that) we have to imagine a contradictory meaning of agency in the form of a supporting knowledge even of Īśvara.

\textbf{Ans:} It is not so; since it will be mentioned that the cause for knowledge of the Object is its being reflected in the consciousness and not of ignorance. Otherwise there will be a contradiction in the declaration of the śruti and smṛti of Īśvara being devoid of kleśa, karma etc\endnote{The reference is to YS I.24 “kleśakarmavipākāśayairaparāmṛṣṭah Īsaraḥ”.}.

\textbf{Ques:} In that case the modification in the form of the shape of the illusory universe is itself ignorance in the limitation of Īśvara (astprapañcākārā vṛttireveśvaropādhānam syādeveti cet) then the answer is:

\textbf{Ans:} It is not so as the absolute illusory nature of the universe will be rejected. Therefore we do not understand the reason for a relationship of agency (mentioned earlier).

Herein the relationship is the mutual reflection (parasparapratibimbo) between the ātman (self) and anātman (not-self) which is the regulator of knowledge. The cause for the activity of the intellect is a special contact-relationship (saṁyogaviśeṣaḥ saṁbandhaḥ) like that between water and lotus leaf/leaves. In statements such as “yathākāśasthito…matsthānītyupadhāraya”(Gītā.9.6) one only learns the rejection of any special relationship which is the cause of change of that dependent on oneself which is the regulator of the contact of any stain (of ignorance) (lepasaṅgādiniyāmakasya svāśrayāvikārahetoḥ saṁyogaviśeṣasya pratiṣedhāvagamāt). The understanding of the relationship between the self and not-self as one of identity is a delusion only (bhrama eveti); thus there is no need for a relationship between them.

\dev{स्यादेतत् , ज्ञानेन शुक्तिरजतवत् प्रपञ्चस्य बाधे किंगोचरा किमर्था वा प्रवृत्तिः स्यात् ? न हि सुप्तोत्थितस्य स्वयमविषयगोचरा प्रवृत्तिर्दृश्यते । तस्मादविद्यावद्विषयाण्येव सर्वाणि प्रमाणानि शास्त्राणि प्रवृत्तिनिवृत्त्यादयश्चेति । अत्रोच्यते—ईश्वरस्य ज्ञानिनां च प्रवृत्तिभोगादिश्रवणादेव शुक्तिरजतादिविलक्षणव्यावहारिकसत्ता कार्याणामभ्युपगम्यते—}
\begin{verse}
\dev{“सद्भाव एषो भवते मयोक्तो, ज्ञानं यथा सत्यमसत्यमन्यत् ।}\\
\dev{एतच्च यत् संव्यवहारभूतं, तत्रापि चोक्तं भुवनाश्रितं ते  ॥”}
\end{verse}
\dev{इति विष्णुपुराणादिषु । सदसद्रूपत्वमेव च व्यावहारिकसत्त्वं प्रकृतितत्कार्यसाधारणम् । एकधर्मेण सत्तादशायामपि विकारिपदार्थानां धर्मान्तरेण सदैवासत्तानियमात् । कूटस्थनित्यस्य चात्मनो नास्ति धर्मतोऽप्यसत्त्वमिति स एव परमार्थसन्निति शास्त्रमर्यादा। तथा चोक्ताऽसत्तावधारणमेव बाधः, स च न प्रवृत्त्यादिविरोधी, तस्य विवेकवैराग्यमात्रहेतुत्वादिति ।}

\dev{“वाचारम्भणं विकारो नामधेयं मृत्तिकेत्येव सत्यमि” त्यादिश्रुतिरपि दृष्टान्तमुखेनेदृशे एव सत्त्वासत्वे ब्रह्मविकारयोः प्रतिपादयति । तस्याश्चायमर्थः—यतः प्रलये च ब्रह्मणि सर्वं विकारजातमव्याकृतरूपामसत्तां गच्छति, अतो वाचां कार्यो नाममात्रशेषश्चेत्यसत्यो भङ्गुरो विकारः, कारणं ब्रह्मैव सत्यं नित्यमिति । अतीतानागतावस्थयोः कार्याणां नाममात्रेणावशेषता तु “नामैवैनं न जहाती” ति श्रवणात् । ननु  (न तु) नामधेयशब्दस्यात्यन्ततुच्छत्वमर्थः मृदविकारस्य  तुच्छता (अतुच्छतायाः) शतशः साधितत्वेन दृष्टान्तत्वासंभवात् पक्षसमत्वात् । “अपागादग्नेरग्नित्वम्” इति वाक्यशेषे विनाशमात्रश्रवणाच्चेति स्वयमूह्यम् ।}

\textbf{Ques:} Let it be; when there is a refutation of the world through knowledge just as in the (false) knowledge of mother-of-pearl being silver, what is the object known and what is the purpose of the activity involved  (jñānena śuktirajatvat prapañcasya bādhe kimgocarā kimarthā va pravṛttiḥ syāt)? One does not witness activity in a person who has woken up from sleep towards a non-object of the senses (svayamaviṣayagocarā pravṛttirdṛśyate). Therefore all objects are of the nature of ignorance as also all the means of knowledge (sarvāṇi pramāṇāni śastrāṇi) as also activity and cessation from activity. 

\textbf{Ans:} It is said: it is only by listening to the activity and experience (pravṛttibhogādiśravaṇāt) of Īśvara and the realized persons (īśvarasya jñāninām ca) that one accepts the reality of unusual worldly objects like the mother of pearl appearing as silver. Thus there are (supporting) statements in the Viṣṇu Purāṇa like: “sadbhāva eṣo bhavate mayokto…tatrāpi coktam bhuvanāśritam te”. Worldly objects are of the nature of being both real and unreal (sadasadrūpatvameva ca vyāvahārikasattvam) which is common to prakṛti and its effects. Even when objects which are subject to change exist with one characteristic there is always the rule of their non-being/non-existence in other qualities (ekadharmeṇa sattādaśāyāmapi vikāripadārthānām dharmāntareṇa saddaivāsattāniyamāt). As for the immutable, permanent ātman, it has no non-being due to any dharma as it has absolute being/existence according to the established śāstra-rule (sa eva parmārthasanniti śāstramaryādā)\endnote{The dharma here refes to the three times i.e. past, present and future. Thus the absolute is trikālābādhita}. Thus it only refutes the said understanding of non-being, and that is not in contradiction to its activity etc., as its cause is only discriminate discernment and detachment (tasya vivekavairāgyamātrahetutvāditi).“vācāraṁbhaṇam vikāro nāmadheyam mṛttiketyeva satyam” (Chānd.Up. VI.1.4) such śruti statements also indicate through such examples the being and non-being of Brahman and the transformation. Its meaning is as follows: since during dissolution the entire collection of transformations (objects) attain the state of non-being of the nature of non-differentiation in Brahman, therefore the effects of śabdas (i.e. objects denoted by words) only remain in name and so (the resultant) change is non-real and perishable; the causal Brahman is alone real and permanent (kāraṇam brahmaiva satyam nityamiti). The stages of being past and not-as-yet-manifest of the effects (things) remains through names alone which is known through such sayings as “namaivainam na jahāti”. It is not that by the word nāmadheya the meaning understood is that it is totally useless; the change in clay being useless (or not useless) has been proved hundreds of times; since there is no example,  the logic is equally applicable to the other side as well (dṛṣṭāntāsaṁbhavāt pakṣasamatvam)\endnote{Whether says that the changes are useless or useful (atucchatā) the reasoning is the same.}. One infers that it is so from hearing only the destruction from such statements as “apāgādagneragnitvam” (Chānd.Up. 6.4.1).  \dev{“सत्ता सर्वपदार्थानां नान्या संवेदनादृते” “भूतं च सिद्धं च परेण यद्यत्तदेव तत्स्यादिति मे मनीषा” इत्यादिवाक्यानि चानुभवकारणाभ्यां विभागेनानुभूयमानकार्ययोरसत्त्वमेव बोधयन्ति न तु तयोरत्यन्तासत्त्वमेव, तथा सति ते न स्त इत्येवोच्येत न तु ताभ्यां सह तयोरभेद इति, सदसतोरभेदायेागात् । एवमेवान्या अप्येवंविधाः श्रुतिस्मृतयो व्याख्येया इति ।}

\dev{अपि च भवतु ज्ञानकर्मणोर्यथाकथञ्चिद् विरोधस्तथापि विविदिषूणां सर्वकर्मत्यागोऽनुचित एव । विचारेण सह कर्मणां विरोधस्य भवतामप्यनभ्युपगमात् , तदानीमभिमानस्य सत्त्वात् । प्रत्युतानुत्पन्नज्ञानस्य सत्यां शक्तौ देवताराधनरूपबाह्याभ्यन्तरसर्वकर्मत्यागे ज्ञानमेव नोत्पद्यत देवकृतविघ्नसम्भवात् , “तदेतद्देवानां न प्रियं यदेतन्मनुष्या विद्युरि” ति श्रुतेः ।}

\dev{ननु “विविदिषन्ति यज्ञेन दानेने” त्यादिवाक्याज्जिज्ञासाद्वारैव ज्ञाने कर्मणामुपयोग इति जिज्ञासानन्तरमेव तत्त्याज्यमिति चेन्न, तेन वाक्येन यज्ञादीनां ज्ञानसाधनत्वस्यैवावगमात् नत्विच्छासाधनत्वस्य, सर्वत्रैव नामधातुस्थले मूलधात्वर्थन सहैव कारकाणामन्वयव्युत्पत्तेः । अन्यथा रथेन जिगमिषतीत्यादौ रथादीनामिच्छा साधनत्वप्रतीतिप्रसङ्गात् , न चैवं संभवति ।}

\dev{न च,}
\begin{verse}
\dev{“कषायपक्तिः कर्माणि ज्ञानं च परमा गतिः }\\
\dev{कषाये कर्मभिः पक्वे ततो ज्ञानं प्रवर्तते ।।}\\
\dev{पापक्षयाच्छुद्धमतिर्वाञ्छति ज्ञानमुत्तमम् ।।”}
\end{verse}
\dev{इत्यादिवाक्येभ्यो वैराग्यजिज्ञासाद्वारैव कर्मणां ज्ञानाङ्गत्वमवसीयत इति वाच्यम्, तथाविधवाक्यानां वैराग्यजिज्ञासाद्वारपरत्वेऽपि द्वारान्तराप्रतिषेधकत्वात् ।}

The statements such as “sattā…saṁvedanādṛte” (Mahopaniṣad.Up.5.47), “bhūtam…me manīṣā” (not traced), by separation of the experience and one’s experience and the causes (that give rise to those effects) only leads to the conclusion of their non-being but not to their total non-existence (kāryayorasattvameva bodhayanti na tu tayoratyantāsattvameva); if that were so then one would have to say that they are not present\endnote{This seems to refer to the denial of both being and non-being of objects} and not that they are identical with them both (i.e. being and non-being) as being and non-being being identical is unreasonable. It is in this way that other similar statements in śruti and smṛti need to be explained.Moreover it is possible that there is opposition between knowledge and karma at times; even then it is improper for those who are seeking the truth (vividiṣūṇām) to give up all karma. You also do not accept that there is opposition of karma with knowledge as therein there is a sense of agency. However in one in whom knowledge has not as yet risen, when all external and internal karma  such as worshipping devas etc., is given up, knowledge itself will not arise due the possibility of obstruction caused by devas (superstition); thus there is the supporting  śruti statement “tadetaddevānām…vidyuḥ”.

\textbf{Ques:} If it is said that such statements as “vividiṣanti yajñena dānena” etc., indicate that the utility of karma in knowledge is only through the pathway of having the desire to know and after that desire (arises) it needs to be discarded\endnote{Once the mind is cleansed through karma (and bhakti) and the desire to know about Brahman arises karma can be discarded.} then the answer is:

\textbf{Ans:} That is not so; by that statement one only learns that sacrifices are the means for giving rise to knowledge (of Brahman) and not for the purpose of giving rise to the desire to know (Brahman). In all cases in place of nāmadhātu (a verb derived from a noun) it is customary to analyse the etymology/derivation with reference to the meaning of the main verb.\endnote{Thus it is not for the desire to know that sacrifices are done but to give rise to knowledge of Brahman.} Otherwise in such uses as ‘ by a chariot desires to go’ (to go desire) (literal meaning of ‘rathena jigamiṣati’) there will be the undesirable result of understanding its being  a means of the desire of the chariot; but it does not happen thus.Through such statements like: “kaṣāyapaktiḥ…jñānamuttamam” (Mbh.Mokṣa.264 cited in Tri. P.13 fn.1) one does not conclude that only through generating detachment and desire to enquire (about Brahman) karma serves as a limb of jñāna (vairāgyajijñāsādvāraiva karmaṇām jñānāṅgatvamavasīyata iti vācyam). Even if such statements are indicative of their partiality towards detachment and desire to enquire (about Brahman) they do not reject other means (as well).
\begin{verse}
\dev{कर्मणा सहिताज्ज्ञानात् सम्यग्योगोऽभिजायते ।}\\
\dev{ज्ञानं च कर्म सहितं जायते दोषवर्जितम् ।।}
\end{verse}
\dev{इति  कौर्मादिवाक्यैर्जिज्ञासोत्तरं जातेऽपि ज्ञाने सम्यग्योगाख्यसम्प्रज्ञातसमाधौ ज्ञाननिर्दोषत्वे च ज्ञानाभ्याससहभावेन कर्मोपयोगश्रवणात् । दोषाः रागद्वेषमोहाः पापानि च । तथा,}
\begin{verse}
\dev{ज्ञानमुत्पद्यते पुंसः क्षयात् पापस्य कर्मणः ।}\\
\dev{यथादर्शतलप्रख्ये पश्यत्यात्मानमात्मनि ।।}
\end{verse}
\dev{इत्यादिवाक्यैज्ञानप्रतिबन्धकपापक्षयद्वाराऽपि कर्मणां ज्ञानहेतुत्वं बोध्यते । तथा    “कर्माणीश्वरतुष्ट्यर्थं कुर्यान्नैष्कर्म्यमाप्नुयादि” ति वसिष्ठयाक्यान्नैष्कर्म्याख्यसमाधि-   हेतुत्वमीश्वरप्रीतिद्वाराऽपि कर्मणः सिद्धम्   ।}

\dev{ एवं ज्ञानसमुच्चयवाक्येभ्यो मोक्षप्रतिबन्धकपापक्षयद्वारा मोक्षहेतुत्वमपि कर्मणामनुमेयम्,    अन्यथा“सहकारित्वेन च, अग्निहोत्रादि तु तत्कार्यायैव” इति वक्ष्यमाणसूत्रद्वयविरोधात् । तथा च जिज्ञासामात्रेण न कर्मत्यागः, जिज्ञासोत्तरमपि ज्ञाने योगे मोक्षे च प्रतिबन्धकसम्भवादिति “जिज्ञासुरपि योगस्य शब्दब्रह्मातिवर्तते” इति वाक्यं च परम्परया विधिकिंकरत्वाभाचं फलं बोधयति “अनेकजन्मसंसिद्धस्ततो याति परां गतिमि” ति वाक्यशेषात् , ननु (न² तु) योगजिज्ञासामात्रेण कर्मत्यागम् । “सर्वं कर्माखिलं पार्थ ज्ञाने परिसमाप्यत” इति च कर्मणां ज्ञानाङ्गत्वं बोधयति “अभ्यासेऽप्यसमर्थोऽसि मत्कर्मपरमो भवे” ति वाक्यं चाऽभ्यासाऽसमर्थस्य केवलं कर्म विदधाति परमशब्दादिति । तस्मात् सर्वकर्माणि संन्यस्य श्रवणं कुर्यादित्यपसिद्धान्तः कलिकृत एव—}
\begin{verse}
\dev{पुंसां जटाधरणमौढ्यवतां वृथैव}\\
\dev{मोघाशिनामखिलशौचबहिष्कृतानाम् ।}\\
\dev{पिण्डप्रदानपितृतोयविवर्जितानां ।}\\
\dev{सम्भाषणादपि नरा नरकं प्रयान्ति  ।।}
\end{verse}
\dev{इति विष्णुपुराणात् । वृथा समाधिरोगाद्यशक्ति विनैवेत्यर्थः । तथा—}
\begin{verse}
\dev{दुःखमित्येव यत्कर्म कायक्लेशभयात् त्यजेत् ।}\\
\dev{स कृत्वा राजसं त्याग नैव त्यागफलं लभेत् ॥}
\end{verse}
\dev{इत्यादिस्मृतेश्चेत्यवधेयम् ।।}

\dev{यच्चान्यैरुक्तम्—विचारविधिपरमिदं सूत्रमिति, तदपि मन्दम्, श्रुतावेव “तद्विजिज्ञासस्व, आत्मा वा अरे श्रोतव्यः” इति विचारे निःसन्दिग्धविधिसत्वेन तत्र सूत्रवैफल्यात्, शास्त्रादावभिधेयप्रतिज्ञायामुन्मत्तप्रलापवदुपेक्षणीयतापत्तेः, सर्वशास्त्रदृष्टप्रतिज्ञापरत्वसम्भवे तत्त्यागानौचित्याच्च । अत एव पूर्वपक्षसूत्रवदस्य शिष्यप्रश्नसूत्रत्वमपि नोचितम्, अभ्यर्हिताया ग्रन्थारम्भप्रतिज्ञाया एव संभवादिति ।}

\dev{ननु यदि शमदमादिसंम्पन्नः सर्व कर्मसंन्यासी नात्राधिकारी तदाऽस्या ब्रह्ममीमांसाया कीदृशोऽधिकारीति वक्तव्यम् । उच्यते—शास्त्रार्थग्रहणोपयोगिशमादियुक्तो गुरुभक्तिनित्यकर्मतप आदिसम्पन्नो जिज्ञासुः कर्मफलविरक्त इति । “परीक्ष्य लोकान् कर्मचितान् ब्राह्मणो निर्वेदमायादि” त्याद्युक्तश्रुतेः,}
\begin{verse}
\dev{यस्य देवे परा भक्तिर्यथा देवे तथा गुरौ ।}\\
\dev{तस्यैते कथिता ह्यर्थाः प्रकाशन्ते महात्मनः ।।}
\end{verse}
\dev{इत्यादिश्रुतेश्च ।।}
\begin{verse}
\dev{इदं ते नातपस्काय नाभक्ताय कदाचन ।}\\
\dev{न चाशुश्रुषवे वाच्यं न च मां योऽभ्यसूयति ।।}
\end{verse}
\dev{इत्यादिस्मृतेश्चेति ।}

Even if such Kūrma Purāṇa statements like: “karmaṇā sahitājjñānāt…doṣavarjitām” (adh. 3.23 cited in ibid. p.13. fn.2) indicate the rise of knowledge after the desire to enquire (about Brahman) it is also learnt that karma is useful by being associated with the repeated practice of knowledge in removing the defects of knowledge in the state of saṁprajñāta samādhi known as samyagyoga.\endnote{Bhikṣu comes back to his favourite obsession which is yoga.} The defects are attachment, hatred and delusion which are vices (papāni). Thus according to sayings like “jñanamutpadyate…ātmani” (Mokṣa.204.8 cited in ibid. p.13. fn.3) it is learnt that karma is a cause for knowledge through weakening the vices which are obstacles to the rise of knowledge. Thus through the utterance of Vasiṣṭha “karmāṇīśvaratuṣṭyartham kuryānnaiṣkaryamāpnuyāt” it can be shown that karma is a means as being the cause for samādhi known as total detachment from karma through pleasing Īśvara.In this manner through sentences combining jñāna and karma one can infer that karma is also a cause for the rise of mokṣa through weakening the vices caused by the obstacles to mokṣa. Otherwise it will contradict the two sūtras “sahakāritvena ca (BS.3.4.33); and agnihotrādi…tatkāryāyaiva” (ibid.4.1.16) which will be mentioned. Thus one should not give up karma only because of the desire to enquire; even after having the desire to enquire there can be obstacles in knowledge, yoga and mokṣa. The sentence “jijñāsurapi…ativartate” (Gītā. 6.44) informs us of the absence of dependence on vidhi (injunction) for the result to follow (paramparayā vidhikimkaratvābhāvam phalam bodhayati) through the next sentence “anekajanmasaṁsiddha…gatim (ibid.6.45). However one cannot give up karma just by the desire to enquire about yoga.\endnote{Since the context in the Gītā is on yoga and it is also Bhikṣu’s attachment to yoga that is reflected in this sentence.} The statement “sarvam…parisamāpyate” (ibid. 4.33) also indicate that karma is a limb of jñāna. So also the statement “abhyāse…matkarmaparamo bhava” (ibid. 12.10) indicates that for one who is incapable of continuous meditation on God only karma is prescribed by the use of the word “parama” (in the above verse). Therefore the statement that giving up all karma one should just listen to śruti (Upaniṣads) is a wrong conclusion made in the Kaliyuga (tasmāt sarvakarmāṇi sanyasya śravaṇam kuryādityapasiddhāntaḥ kalikṛta eva)\endnote{Referenc to Advaita Vedānta as a false doctrine.}. The word “vṛthā” in the following statement from the Viṣ.P: “pumsām jaṭādhāraṇamauḍhyavatām vṛthaiva…narā narakam prayānti” (3.28.103 cited in Tri. p.14.fn.3) (refers to those who give up karma) excluding those incapable because of being in samādhi or being sick.  So also attention must be paid to the following smṛti statements : “yastu vidyābhimānena…tyāgaphalam labhet” (not traced) which say the same thing.

\dev{शास्त्रोक्तविद्याभ्यासे पुनरधिकारी योगशास्त्रोक्तयोगाङ्गादिसम्पन्नः श्रवणमननाभ्यां कोमलकण्टकन्यायेनोत्पन्नज्ञानो नारदीयोक्तनित्यानित्यविवेकादिसाधनचतुष्कवान् “शान्तो दान्त उपरत” इत्याद्युक्तश्रुतेः । तत्र चोपरतिर्योगविरोधिकर्मभ्य उपरम इति । तत्रापि मन्दाधिकारी गृहस्थादिस्रिदण्डिपर्यन्तः । उत्तमाधिकारी च परमहंसः,}
\begin{verse}
\dev{चतुर्विधा भिक्षवः स्युः कुटीचकबहूदकौ ।}\\
\dev{हंसः परमहंसश्च श्रेयांश्चैषां यथोत्तरम्।}\\
\dev{आत्मनिष्ठः स्वसंसक्तः त्यक्तसर्वपरिग्रहः ।}\\
\dev{चतुर्थोऽयं महानेषां ध्यानभिक्षुरुदाहृतः ।।}
\end{verse}
\dev{इति विष्णुधर्मसंहितादिवाक्यात् । परमश्चासौ हंसश्चेति परमहंसः परमात्मा “यमाहुः परमहंसमि” ति स्मृतेः । तन्निष्ठत्वाद् यतिरपि परमहंस उच्यते । हंसशब्दश्चात्मवाची—}
\begin{verse}
\dev{सकारेण बहिर्याति हकारेण विशेत् पुनः ।}\\
\dev{हंस हंसेति वै मन्त्रं जीवो जपति सर्वदा ॥}
\end{verse}
\dev{इतिस्मरणात् । हंसाश्रमस्तु केवलजीवोपयोगीति विवेकः । ते त्वाद्यास्त्रिदण्डिविशेषास्तेषाँ लिङ्गानि धर्माश्च तत्रैव विष्णुधर्मसंहितायामुक्ता विस्तरभयान्न लिख्यन्ते ।}

The statement by others that this sūtra (reference to BS. I.1.1) refers to the injunction of reflection  (vicāra) has no weight also need to be reflected upon has no weight (mandam). In śruti itself there are sayings like “tadvijiñāsasva” (Taitt.Up.III.1.1) “atmā vā are śrotavyaḥ” (Bṛh. Up. IV.5.6) which prescribe reflection clearly (niḥsandigdhavidhisattvena) so the sūtra (BS. I.1) is useless for that purpose. In śāstra texts if there is no deliberation on what has been declared as the topic, there is the danger of its being ignored like the ramblings of a madman; when it is possible to be engaged in what is declared (as the topic) as seen by all the śāstras, it is also improper to give that up (sarvaśātradṛṣṭapratijñāparatvasambhave tattyāgānaucityācca). That is why to consider, like a pūrvapakṣasūtra, to consider this sūtra as being the question of a disciple is also not correct;\endnote{Bhikṣu is perhaps referring to the four requisites laid down in Advaita in the BSBh as a question raised by a student for the competence of a disciple to pursue Advaita.} the competence (of a disciple) is decided by the declared suitable topic at the beginning of the grantha (book) itself (abhyarhitāyā granthārambhapratijñāyā eva sambhavāditi).

\textbf{Ques:} If one who is proficient in śama, dama etc., and who has given up all karma is not an adhikārī here (nātrādhikārī) then please tell me who is a fit adhikārī for this Brahmamīmāṁsā (knowledge pertaining to Brahman).\endnote{This seems a counter question by the Advaitin.}

\textbf{Ans:} One who is proficient in śama, dama etc., which enables one to grasp the meaning of the śāstras, who has accomplished successfully gurubhakti (devotion towards the guru), obligatory karma (nityakarma), austerity (tapaḥ) etc., one who has a deep desire to know (and) who is detached from the result of the action (karma).\endnote{Such a one is the competent adhikārī needs to be added to complete the answer.} This is in accordance with such śruti sayings like “parīkṣya…nirvedamayāt” (Muṇḍ.Up. I.2.12) and “yasya deve parā…prakāśante mahātmanaḥ” (Śvet. Up. 6.23) and the smṛti statement “idam te nātapaskāya…mām yo’bhyasūyati” (Gītā.18.67). Again with reference to learning in accordance with the instruction of the śāstras the adhikārī is one who is proficient in the yogāṅgas (limbs of yoga) mentioned in the YS;\endnote{Bhikṣu inserts his pet preference for yoga whenever he ggets an opportunity.} through listening (to the śāstras) and reflection (on them) when knowledge arises through reasoning similar to the komalakaṇḍakanyāya\endnote{Using a soft thorn to remove a painful thorn; in other words it is through the listening and reflection on the śāstras that knowledge arises from the same śāstras.} then he is one who has achieved the fourfold requisites of distinguishing between what is permanent and that which is temporal etc., (nityānityaviveka) mentioned by Nārada; he is one who is “śānto dānta uparata” etc., according to śruti.\endnote{The complete quotation “śānto dānta uparatastitikṣuḥ samāhito bhūtvā’’tmanyevā’’tmānam paśyati”
Subālopaniṣad 9.14}  Therein the word “uparati” means desisting from deeds that are against the dictates of yoga\endnote{This is a special meaning given by Bhikṣu because of his attachment to yoga. uparati in general means abstaining from prescribed deeds.}; even there (there are the) dull aspirants (mandādhikārī) such as householders and including those carrying the three daṇḍa whereas the best aspirants are the paramahaṁsas.\endnote{Bhikṣu must have been a ‘ekadaṇḍin’ i.e. carrying a single stick as all advaitins. He also shows contempt for the other sannyāsins who carry three sticks.} Thus the Viṣṇu Dharmasaṁhitā says “caturvidhā bhikṣavaḥ…dhyānabhikṣurudāhṛtaḥ”.  A pramahaṁsa is the great ātman and its etymology is ‘paramaścāsau hamsaśca iti paramahamsaḥ’ i.e. paramātmā in accordance with the smṛti statement “yamāhuḥ paramahaṁsam”. Because an ‘yati’ is also established in that state he is also called a ‘paramahaṁsa’. The word ‘haṁsa’ denotes ātman as one recalls the following: “sakāreṇa bahiryāti…jīvo japati sarvadā”. One understands that the āśrama of a haṁsa\endnote{There is no hamsāśrama as such but Bhikṣau probably calls living as a paramahaṁsa in the world as a
haṁsāśrama.} is only for the sake of leading one’s life (haṁsāsramastu kevalajīvopayogīti vivekaḥ). The characteristics and the duties (dharmās) of the first three who carry three daṇḍas (the kuṭīcaka, bahūdaka and the haṁsa) mentioned in the above verse) are mentioned in the Viṣṇu Dharmasaṁhitā; it is not given (here) out of fear of increasing   the length of the work (vistarabhayānna likhyante). \dev{तत्र विविदिषुसंन्यास आद्ययोरेव तयोस्तपःप्रधानत्वात् “तपसा ब्रह्म विजिज्ञासस्वे” ति श्रुतेः । मनौ—}
\begin{verse}
\dev{वेदसंन्यासिकानान्तु कर्मयोगं निबोधत ।}\\
\dev{संन्यस्य सर्वकर्माणि कर्मदोषानपानुदत् ।}\\
\dev{नियतो वेदमभ्यस्य पुत्रैश्वर्ये सुखं वसेत् ।।}
\end{verse}
\dev{इत्यादीनां वेदसंन्यासस्यापि स्मरणाच्च । विद्वत्संन्यासस्त्वन्त्ययोः जीवपरमात्मनिष्ठताभेदेनेत्यपि विवेक्तव्यम् ।}
\begin{verse}
\dev{भौतिकी भावना पूर्वे सांख्ये त्वक्षरभावना ।}\\
\dev{तृतीये चान्तिमा प्रोक्ता भावना पारमेश्वरी ।।}
\end{verse}
\dev{इत्यादिनाम्  कौर्मे त्रिविधयोगिकथनात् । वैश्वानरादिभावना भौतिकी, सा चाद्ययोः संन्यासिनोरिति तत्र परमहंसा जाबालश्रुतौ परिगणिताः ‘‘तत्र परमहंसा नाम संवर्तकारुणिश्वेतकेतुदुर्वासऋभुनिदाढ्यजडभरतदत्तात्रेयरैवतकप्रभृतय”इतिI ननु भरतस्य यज्ञोपवीतश्रवणादस्य कथं परमहंसत्वम् ? }
\begin{verse}
\dev{त्रिदण्डं कुण्डिकां चैव सूत्रं चापि कपालिकाम् ।}\\
\dev{जन्तूनां वारणं वस्त्रं सर्व भिक्षुरिदं त्यजेत् ।।}
\end{verse}
\dev{इति परमहंसप्रकरणस्थविष्णुधर्मवाक्यादिति । [ न]}
\begin{verse}
\dev{आत्मन्येवात्मना बुद्ध्या न्यस्तसर्वपरिग्रहः ।}\\
\dev{अव्यक्तलिङ्गोऽव्यक्तश्च चरेद् भिक्षुः समाहितः ।।}
\end{verse}
\dev{इति विष्णुधर्मवाक्योक्ताव्यक्तलिङ्गतासम्पादनाय मनसा यज्ञोपवीतादीनामात्मन्यारोपणेऽपि सूत्राभासधारणसंभवादिति । तस्मात् सूत्रस्य विपरीतार्थकल्पनया सर्वकर्मत्यागे शिष्यो न प्रवर्तनीयः। किन्तु मुमुक्षवे ज्ञानमुपदिश्य कर्मार्थं समाधिभङ्गो न कर्तव्यो “गुणलोपो न गुणिन” इति न्यायात्, जडभरतादिशिष्टाचाराच्चेत्येवोपदेश्यम् । समाध्याविर्भावे च कर्मत्यागस्तद्विरोधेन स्वयमेव क्रमेण भवति “एतद्ध स्म वै तद्विद्वांस आहुः ऋषयः कावषेयाः किमर्था वयमध्येष्यामहे किमर्था वयं यक्ष्यामह” इति श्रुतेः,}
\begin{verse}
\dev{न कर्माणि त्यजेद्योगी कर्मभिस्त्यज्यते ह्यसौ ।}\\
\dev{विदिते परतत्त्वे तु समस्तैनियमैरलम् ।।}\\
\dev{तालवृन्तेन किं कार्यं लब्धे मलयमारुते ।}\\
\dev{ज्ञानामृतरसो येन सकृदास्वादितो भवेत् ।।}\\
\dev{स सर्वकार्यमुत्सृज्य तत्रैव परिधावति । इत्यादि स्मृतेश्चेति ।}
\end{verse}
In that context the samnyāsin who has the desire to enquire (into the truth) is included in the first two (kuṭīcaka and bahūdaka) as for them there is the importance of austerity according to the śruti statement “tapasābrahma vijijñāsasva” (Taitt.Up. 3.2,4,5). One also hears about Vedasaṁnyāsa from the MS as “vedasamnyāsikanāntu…putraiśvarye sukham vaset” (MS.6.86)\endnote{Nowadays one does not hear about Vedasaṁnyāsa.}. Whereas a vidvatsaṁnyāsin needs to be distinguished from the last two (presumably the hamsa and paramahamsa) as they are intent on (realizing) jīva as ātman. The Kūrma Purāṇa mentions three kinds of yogīs as:  “bhautikī bhāvanā…bhāvanā pārameśvarī” (2.86; cited in Tripathi p.15.fn. 3). Meditation on Vaiśvānara etc., is of a gross nature (bhautikī); and that belongs to the first two samnyāsins (kuṭīcaka and bahūdaka). Paramahaṁsas are recounted in the Jābāla Up as “tatra paramahamsā…raivatakaprabhṛtaya” (Jābāla. 6).Ques: Since one hears of Bharata’s thread ceremony (yajñopavīta) how can he be a paramhaṁsa (this is obviously not the Rāmāyaṇa Bharata)\endnote{Bharata is not mentioned as a paramahaṁsa in the above quote from the Jābāla Up. But this must be well known for Bhikṣu to raise the issue here.}. That is not in accordance with the Viṣṇudharma statement:  “tridaṇḍam…bhikṣuridam tyajet”. In order (for a paramhamsa) to qualify by having no clear marks/symbols according to the Viṣṇudharma  statement: “ātmanyevātmanā…avyatkaliṅga…samāhitaḥ” even if one internalizes mentally yajñopavīta etc., it is possible that there will be the semblance of the thread/yajñopavīta.Therefore by imagining an opposite meaning of the sūtra (sūtrasya) the disciple should not engage in giving up all karma. But instructing about knowledge to one desiring mokṣa, there should be no breaking with samādhi for the sake of performance of karma (karmārtham samādhibhaṅgo na kartavyo); this follows the maxim of “guṇalopo na guṇina” (instruction should also emphasize the) excellent (spiritual) observances of ācāryas like Jaḍabharata etc. When samādhi comes into being then there is the giving up of karma gradually on its own accord as it is contradictory to it (samādhi); thus there is the saying “etaddha sma vai…vayam yakṣyāmahe”. Smṛti also says: “na karmāni tyajedyogī…tatraiva paridhāvati.”

\dev{यच्चान्यत् परैरुच्यते—अस्य शास्त्रस्य जीवब्रह्मैक्यं विषयः, तज्ज्ञानस्य च न कर्मशेषत्वमिति तदपि न, अस्य शास्त्रस्य जीवब्रह्मैक्यविषयत्वे लिङ्गाद्यभावात्: “ब्रह्मसूत्रपदैश्चैव हेतुमद्भिर्विनिश्चितै” रिति गीतावाक्येन सूत्राणां ब्रह्मविषयतामात्रावगमात्, तथा सत्य “थातो जीवनह्मैक्यजिज्ञासे” त्येव सूत्रणौचित्याच्च । शास्त्रमहावाक्यार्थ परित्यज्य तदेकदेशप्रतिज्ञानौचित्यात् । ब्रह्मसूत्ररूपैरधिकरणैर्यैर्युक्तिमद्भिरसंदिग्धैर्गीतमित्यर्थः । जीवनिरूपणं चात्र ब्रह्मशेषतयैव प्राणादिनिरूपणवदिति जीवप्रकरणे वक्ष्यामः । यद्यपि ब्रह्मात्मतैवान्न शास्त्रमहावाक्यार्थः, तथापि ब्रह्मत्वेनैवात्मत्वमाक्षिप्तमित्याशयः । “बृहत्त्वाद् बृंहणत्वाच्च आत्मा ब्रह्मेति गीयत” इति स्मृत्यादिभिरात्मब्रह्मशब्दयोरर्थैक्यं वा । अतोऽस्मन्मते सूत्रन्यूनता न शङ्कनीयेति । यश्चास्य विषयो ब्रह्म तज्ज्ञानस्य कर्मशेषत्वमप्यस्येव,}
\begin{verse}
\dev{ब्रह्मण्याधाय कर्माणि सङ्गं त्यक्त्वा करोति यः ।}\\
\dev{लिप्यते न स पापेन पद्मपत्रमिवाम्भसा ।।}
\end{verse}
\dev{इत्यादिवाक्येभ्य इति दिक् ।}

\dev{सपरिकरं ब्रह्म विचार्यमित्युद्दिष्टं, तत्र ब्रह्मलक्षणं ब्रह्मशब्दप्रवृत्तिनिमित्तञ्च प्रकृतिपुरुषादिव्यावृत्तमाह—}

\textbf{Ques:} There is this said by others: the subject matter of this śāstra is the identity of jīva and Brahman (and) for realizing it there is no need of karma, as there is no characteristic mark with reference to the identity of jīva and Brahman in this śāstra; by the Gītā statement “brahmasūtrapadaiścaiva…viniścitaiḥ” (Gītā 13.4)  one knows that its subject is only pertaining to Brahman.

\textbf{Ans:} In that context then it would be appropriate for the sūtra to be “athāto jīvabrahmaikyajijñāsā”.\endnote{Bhikṣu states if the identity of Brahman and jīva was the intention of the BS then the first sūtra should have been “athāto jīvabrahmaikyajijñāsā” and not “athāto brahmajijñāsā”.} Abandoning the meaning of the mahavākyas of the śāstras it is inappropriate to understand its meaning partially (in part).\endnote{The reference is perhaps to the mahāvākya “tat tvam asi” where one has to understand the meaning by giving up the identitynot wholly but partially by substraction and non-substraction of meaning called as “jahajjajjahajlakṣanā”.} It is thus spelt out clearly in the adhikaraṇas of the Brahmasūtras. We shall also mention later in the chapter dealing with jīva that there is the definition of the jīva as being a part of Brahman just as the definition of prāṇa etc.Even though the meaning of the śāstra-mahāvākya is the identity of Brahman and ātman, still the intention pointed out is that having the  characteristic of ātman (ātmatvam)is through having the characteristic of Brahman (brahmatvenaiva ātmatvamākṣiptam ityāśayaḥ).  Or this could be interpreted as the identity of the meaning of the words Brahman and ātman as mentioned in smṛti texts as “bṛhattvād bṛhaṇatvācca ātmā brahmeti gīyate”. Therefore in our understanding one need not doubt that the sūtra has omitted anything. Thus the subject matter of the sūtra is Brahman and its knowledge is the result of karma/karmayoga (tajjñānasya karmaśeṣatvamapyastyeva). This is understood from such statements as “brahmaṇyādhāya karmāni…pāpena padmapatramivāmbhasā”(Gītā 5.10). The intention (of the sutra) was to examine Brahman in all details (saparikaram brahma); in that context the characteristics of Brahman, the activity and purpose of the use of the word Brahman (which) differentiates it from prakṛti and puruṣa is mentioned (in the sūtra):

\section*{BS. I.1.2}

\begin{verse}
\dev{जन्माद्यस्य यतः ॥} I.\dev{२ ॥}
\end{verse}

\dev{अस्य जगतो नामरूपाभ्यां व्याकृतस्य चेतनाचेतनरूपस्य प्रतिनियतदेशकाल- संस्थाव्यापारादिमतोऽचिन्त्यरचनात्मकस्य जायतेऽस्ति वर्धते विपरिणमतेऽपक्षीयते विनश्यतीत्येवंरूपं जन्मादिषट्कं यतः परमेश्वरादन्तर्लीनप्रकृतिपुरुषाद्यखिलशक्तिकात् स्वतश्चिन्मात्राद् विशुद्धसत्त्वाख्यमायोपाधिकात् क्लेशकर्मविपाकाशयैरपरामृष्टाच्चेतनविशेषाद् भवति, आकाशादिव महावायुर्महाजलादिव च पृथिवी, प्रथिव्या इव च स्थावरजङ्गमादिकं, तद् ब्र ब्रह्मेति वाक्यशेषः । अत्र च ‘एतद्यत’ इत्यनुक्त्वा ‘जन्माद्यस्य यत’ इति वचनादव्यक्तरूपेण जगन्नित्यमेवेत्याचार्याशयोऽवगन्तव्यः । यत इति पञ्चमी चात्राधिष्ठानकारणत्वे महदाद्यखिलजगदधिष्ठानकारणत्वं च ब्रह्मण एव “आधारमानन्दमखण्डबोधं यस्मिन् लयं याति पुरत्रयं च, एतस्माज्जायते प्राणो मनः सर्वेन्द्रियाणि चे” त्यादिश्रुतेरिति न, (न) प्रकृतिपुरुषादिष्वतिव्याप्तेः (तिव्याप्तिः)।}

This world having various names and forms, which has both sentient and insentient entities, which has its own rules of space, time, place and action which is constructed in a manner which cannot be fathomed (acintyaracanātmakasya), has the nature of coming into existence, then exists, grows, changes and then declines and is (finally) destroyed; this world having the sixfold nature beginning with its birth comes into being by itself (svataḥ) from the powers of all the puruṣas and prakṛti hidden within Parameśvara through the special consciousness which is untained by kleśa, karma and vipāka (and) which has the limitation called māyā composed of pure sattva. The sentence needs to be completed by adding ‘just like the great wind from ākāśa, just like earth from the great waters, just like the inanimate and animate entities from the earth’ Brahman is that. By not mentioning “etadyata” (from that) through the words “janmādyasya yataḥ” one should understand that the ācārya’s (Bādarayaṇa) intention is clearly that the world is eternal.\endnote{This flows logically from the metaphysics of Sāṇkhya-Yoga.}  The word “yataḥ” in the fifth case (pañcamī) is used in the sense of being the causal support (adhiṣṭhānakāraṇa). It is Brahman alone who is the causal support of the whole world composed of mahat etc. Thus the śruti statement “ādhāram…sarvendriyāṇi ca” supports it; it indicates that there is no over-pervasion (ativyāptiḥ) with regard to prakṛti and puruṣa.\endnote{According to Bhikṣu there should not be any doubt regarding who is the final cause. But when both prakṛti and the puruṣas are declared to be vibhu and permanent it does raise some problems.} \dev{`किं पुनरधिष्ठानकारणत्वम् ? उच्यते—तदेवाधिष्ठानकारणं यत्राविभक्तं येनोपष्टब्धं च सदुपादानकारणं कार्याकारेण परिणमते । यथा सर्गादौ जलाविभक्ताः पार्थिवसूक्ष्मांशास्तन्मात्राख्याः जलेनैवोपष्टम्भात् पृथिव्याकारेण परिणमन्त इत्यतो जलं महापृथिव्या अधिष्ठानकारणमिति । तथा च स्मर्यते—}
\begin{verse}
\dev{यस्य यत् कारणं प्रोक्तं तस्य साक्षान्महेश्वरः।}\\
\dev{अधिष्ठानतया स्थित्वा सदैवोपकरोति हि ।। इति ।}
\end{verse}
\dev{तथा चैतादृशकारणत्वमेवाधिष्ठानकारणत्वमिति मूलकारणत्वमिति चोच्यते । ब्रह्मणश्च स्वाविभक्तप्रकृत्याद्युपष्टम्भकत्वं साक्षितामात्रेणेति जगत्कारणत्वेऽपि न ब्रह्मणो विकारित्वं न वा प्रकृतिपुरुषादिष्वतिप्रसङ्गः, सर्गात् पूर्वमन्येषां साक्षित्वासम्भवात् । अत एवाविकारचिन्मात्रत्वेऽपि ब्रह्मणो जगदुपादानत्वं जगदभेदश्चोपपद्यते । विकारिकारणवदधिष्ठानकारणस्याप्युपादानत्वव्यवहारात् ।   कार्याविभागाधारत्वस्यैवोपादानसामान्यलक्षणत्वात् । अविभागश्चाधारतावत् स्वरूपसम्बन्धविशेषोऽत्यन्तसंमिश्रणरूपो दुग्धजलाद्येकताप्रत्ययनियामकः । }

\dev{तत्र समवायसम्बन्धेन यत्राविभागस्तद्विकारिकारणम्। यत्र च कार्यस्य कारणाविभागेनाविभागस्तदधिष्ठानकारणम्, यथा जलं पृथिव्या इति । न हि जलस्य साक्षादेव पृथिवी विकारस्तन्मात्राणां भूतप्रकृतित्वश्रुतिस्मृतिविरोधात् । न च द्वयोरेवोपादानत्वम्, विजातीयानामनारम्भकत्वात् । एवमाकाशादीनां वाय्वाद्युपादानत्वमप्यधिष्ठानतयैव द्रष्टव्यम्। संभवत्यविरोधे सृष्टिप्रक्रियायां वैशेषिकसांख्ययोरुभयोरप्यत्र विरोधानौचित्यादिति वैशेषिकादिभिरपीदृशं। ब्रह्मणः कारणत्वमिष्यत एव । परन्तु तैरिदमपि निमित्तकारणतेति परिभाष्यते । अस्माभिस्तु समवाय्यसमवायिभ्यामुदासीनं निमित्तकारणेभ्यश्च विलक्षणतया चतुर्थमाधारकारणत्वमिति । तदेतत् सर्वं “तत्तु समन्वयादि” ति सूत्रेणाऽऽचार्यो वदिष्यति, शिष्यव्युत्पत्त्यर्थं त्वत्राप्यस्माभिः किञ्चिदुक्तमिति। इमं चार्थं तत्रैव प्रपञ्चयिष्यामः।ब्रह्मणः साक्षात् परिणामवादं विवर्तवादं च तत्रैव निराकरिष्यामः।}

In answer to the question as to what is a causal support he says: That is itself called a resting place wherein without separation and being closely connected with it, the material cause transforms itself into the form of effects. This is like the subtle elements of earth called as tanmātras are not separated from water at the start of evolution/manifestation, and staying within water change into the shape of earth and so water is (called) the causal support of the great earth. Thus one recalls “yasya yat kāraṇam…sadaivopakaroti hi” (not traced). Similarly the causal support is this kind of causal support and it is also called the principal support. And Brahman sees prakṛti etc., which stays inseparably within itself just as a witness; even in being the cause for the world there is no change in Brahman nor does it extend (have an unwarrantable stretch) to prakṛti and puruṣas (na vā prakṛtipuruṣādiṣvatiprasaṅgaḥ)\endnote{In other words all three Brahman, prakṛti and puruṣas who are vibhu stay without any change in close proximity is what Bhikṣu says. That seems to be a stretch of the imagination when all three are all pervasive.}. Prior to evolution there is no witnessing of anything. That is why even if Brahman of the form of pure consciousness has no change it is appropriate that Brahman is the material cause of the world as well as not being different from the world (jagadupādānatvam jagadabhedaścopapadyate). Just as the transforming cause (vikārikāraṇavat) the causal support can also possess the function of a material cause.  The general characteristic of a material cause is that it is a non-separable support of the effect (kāryāvibhāgādhāratvasyaivopādanasāmānyalakṣaṇatvāt). And non-separation, similar to that of a support, is a special relationship called svarūpa-sambandha; it is of a very mixed nature; it regulates the knowledge of identity (like that) between milk and water.\endnote{This makes no sense; when water and milk are mixed together they become one and they are not seen separately. But in the case of Brahman, prakṛti and the puruṣas, Bhikṣu has already stated that there is no close contact between them.} Therein when there is non-separation by the relationship of inherence that becomes the cause for change. Where the effect is non-separate due to being non-separate from the cause then it is a causal support (adhiṣṭhānakāraṇa) like water in the case of the earth.\endnote{One can see that Bhikṣu is not happy with the svarūpa sambhandha and Brahman being an adhiṣṭhāna kāraṇa he has advocated; he therefore tries to explain it in various ways.} The earth is not a direct transformation of water as it will contradict śruti and smṛti (statements which mention) the subtle elements of (water) being of the nature of earth. Nor can the two be the material cause as there cannot be the rise of (objects )with different properties (vijātīyānāmanārambhakatvāt). Thus one should also view ākāśa etc., being the material cause of wind etc., in the sense of being a support (adhiṣṭhānatayaiva draṣṭavyam). Since it is possible for both the Vaiśeṣika and Sāṁhkya philosophers to assent to this process of creation/evolution it is improper for them to oppose it.\endnote{Bhikṣu at heart is a syncretist and even though a staunch Yoga advocate would still like to reconcile his views with that of the other orthodox systems of philosophy. His awoved battle is mainllywith Śaṅkara and his advaita philosophy.} Even the Vaiśeṣikas desire this kind of cause as Brahman. But it is defined by them as an efficient cause. We are indifferent to causes such as samavāyī and asamavāyī and (also) being distinguished (different) from efficient causes as well (nimittakāraṇebhyaśca vilakṣaṇatayā) (and thus) this fourth cause is a causal support (caturthamādhārakāraṇatvamiti)\endnote{There is not one singular characteristic that can be mentioned for an efficient cause.}. All this will be mentioned by the ācārya under the sūtra “tattu samanvayāt” (BS.I.1.4). We also have mentioned it briefly here for the sake of the learning of the disciple (śiṣyavyutpattyartham). We shall explain this meaning there itself. We shall also reject the pariṇāma theory (real transformation) and vivartavāda theory (transformation as an appearance) of Brahman there itself. \dev{ ब्रह्मणश्च जगत्कर्तृत्वं स्वोपाधिमायोपाधिकम्, परिणामित्वरूपोपादानत्वं च }

\dev{ प्रकृतितत्कार्याद्यौपाधिकमपीष्यत एव । तथा चोक्तम्—}
\begin{verse}
\dev{सर्वशक्तिमयो ह्यात्मा शक्तिमण्डलताण्डवैः ।}\\
\dev{सम्सारं तन्निवृत्तिं च करोत्यविरतोदयम् ।। इति }
\end{verse}
\dev{“यस्मिन् यतो यर्हि येन यस्य यस्मै यद् यो यथा कुरुते कार्यते वा । परावरेषां परमं प्राक् स्वसिद्धं तद् ब्रह्म   तद्धेतुरनन्यदेकम्” इति च  अस्मिश्च कारणताद्वये कुलालोर्णनाभौ दृष्टान्ताविति । एवञ्च जगतः सर्वप्रकारकारणत्वमपि ब्रह्मलक्षणं कर्तुं शक्यते । प्रकृतिपुरुषादिषु शक्तिषु प्रत्येकमुपादानत्वादिरूपेण प्रतिनियतमेव कारणत्वम् । ब्रह्मणस्तु सर्वशक्तिकत्वात् तत्तदुपाधिभिः सर्वकारणत्वम् । यथा चक्षुरादीनां दर्शनादिकारणत्वं यत्प्रत्येकमस्ति तत्सर्वं सर्वाध्यक्षस्य जीवस्य भवतीति । एतेन जगतोऽभिन्ननिमित्तोपादानत्वं व्याख्यातम् । अस्मिंश्च शास्त्रे सृष्टिप्रक्रिया महदादिक्रमेणैव सांख्ययोगयोरिव वक्ष्यते वियदादिपादे सृष्टिप्रकरणे।}

The creation of the world by Brahman is dependent on its own limitation (called) māyā. Being the material cause in the form of having the quality of change is aso desired as belonging to the limitation of prakṛti and its effects. Thus it is said “sarvaśaktimayo hyātmā…karotyaviratodayam”. There is also the saying: “yasmin yato…tad brahma taddheturananyadekam”. In these two instances of having causal efficiency (asminśca kāraṇatādvaye) (mentioned in the quote “yasmin…brahma “) examples are that of the potter (kulāla) and the spider (ūrṇanābhi). In this manner even if the world has causes of all kinds (sarvaprakārakāraṇatvamapi) it can be established as a distinctive sign of Brahman. Being a cause is fixed (pratiniyatameva) in the powers centred in prakṛti and puruṣas in the form of being a material cause (prakṛtipuruṣādiṣu śaktiṣu pratyekamupādāntvādirūpeṇa pratiniyatameva kāraṇatvam).\endnote{In Bhikṣu’s philosophy puruśas and prakṛti are śaktis of Parameśvara and help in the process of evolution of the world.} Brahman possessed of all powers (sarvaśaktikatvāt) is a cause for everything (sarvakāraṇatvam) through their respective limitations (tatadupādibhiḥ);  this is like the eye etc.,(i.e. all the senses are implied here) each possessing the quality of being causes for the act of seeing etc., respectively, (but) all those activities belong to the all-supervising jīva. In this manner Brahman being both a material and also an efficient cause has been explained.\endnote{As the causal support in the form of adhiśṭhāna-kāraṇa closely associated with the śaktis, prakṛṭi and puruṣas, Brahman is the material cause (upādānakāraṇa) of the world. So also Brahman with its upādhi of pure sattva is the efficient cause of the world. Thus Brahman is both the material and efficient cause of jagat.}  It will be mentioned later in the chapter on viyat etc., dealing with creation, that in this śāstra (Bhikṣu’s avibhāga-vedānta) the function of creation happens with the rise of mahat etc as in Sāṁkhya and Yoga\endnote{Bhikṣu sticks to the cosmology of S/Y.}. \dev{विशेषस्त्वत्रोच्यते—प्रकृतिस्वातन्त्र्यवादिभ्यां सांख्ययोगिभ्यां पुरुषार्थप्रयुक्ता प्रवृत्तिः   (प्रकृतिः) स्वयमेव पुरुषेण आद्यजीवेन संयुज्यत इत्यभ्युपगम्यते अयस्कान्तेन लोहवत् । अस्माभिस्तु प्रकृतिपुरुषसंयोग ईश्वरेण क्रियत इत्यभ्युपगम्यते, “आदिः स संयोगनिमित्तहेतुः परस्त्रिकालादकलोऽपि दृष्ट” “प्रकृतिं पुरुषं चैव प्रविश्यात्मेच्छया हरिः ।क्षोभयामास सम्प्राप्ते सर्गकाले व्ययाव्ययौ  ।। इति स्मृतेश्चेति ।}
\begin{verse}
\dev{पुरुषोऽत्र जीवः “चित्यात्मा गृह्यते यस्तु बुद्ध्यवस्थित आत्मनः ।}\\
\dev{पुरुषाख्यः स विज्ञेयो भोक्तृभावः स उच्यते ।।}
\end{verse}
\dev{ इति योगियाज्ञवल्क्यात् । परमेश्वरे च पुरुषशब्दः उपाधिसम्बन्धमात्रेण गौणः ।}

\dev{ननु संयोगविशेषहेतुः क्रियाविशेषः क्षोभो विभ्वोः प्रकृतिपुरुषयोर्न संभवतीति चेन्न, प्रकृतेर्गुणत्रयरूपतया परिच्छिन्नगुणांशेन क्षोभसंभवात्, पुरुषस्य च तदौपाधिकक्षोभात् आकाशस्य वाय्वौपाधिकक्षोभवत् । अथवा संयोगोन्मुखत्वेन पुरुषे क्षोभोपचारः अत- एव “गुणेभ्यः क्षोभ्यमाणेभ्यस्त्रयो देवा विजज्ञिरे” इत्यादिश्रुतिषु गुणानामेव क्षोभः श्रूयत इति, न तु पुरुषस्येति । प्रकृतिपुरुषयोश्चेश्वरस्य प्रवेशः शास्त्रवदवधानमात्रमिति । विभावपि वेश्वरोपाधौ विभोरीश्वरस्य नित्य एव संयोग इत्यगत्याभ्युपेयम् ! नित्यसंयुक्तयोरपि वैधर्म्यात् जीवतदुपाधिदृष्टान्तेन भेदसिद्धिरिति ।}

Something special is mentioned here:

The Sānkhya and Yoga philosophers who believe in the independence of prakṛti believe that prakṛti used for the purpose of accomplishing the goal of puruṣa gets connected by itself with the first jīva, like iron getting connected to a magnet.  In our view, on the other hand, contact between prakṛti and puruṣa is brought about by Īśvara (in accordance with the śruti saying “ādiḥ sa samyoganimittahetuḥ parastrikālādakalo’pi dṛṣṭa” (Śvet. 6.5). Smṛti also mentions the following: “prakṛtim puruṣam caiva…sargakāle vyayāvyayau” (Kūrma.P 4.13; also Viṣ.P. 1.2.27 cited in Tripathi p.18.fn.3). The creation of the world by Brahman is dependent on (belongs to) its own limitation (called) māyā. Puruṣa here (in the above verse), is the same as jīva; this is understood from Yogi Yājñavalya’s statement: “cityātmā gṛhyate…bhokṛbhāvaḥ sa ucyate”. When puruṣa is used for Parameśvara it is in a secondary sense (and) only when it is associated with its limitation.

\textbf{Ques:} The cause for the special contact (between Hari, puruṣa and prakṛti after entering them mentioned in the verse above) is activity of a special nature i.e disturbance (kṣobhaḥ); that is not possible between the all pervading prakṛti and puruṣa.

\textbf{Ans:} That is not so. Since prakṛti is of the nature of three guṇas it is possible to have disturbance in the portion which is divided;\endnote{What Bhikṣu means by this is not clear at all. Prakṛti which is all-pervading is of the nature of the three constituents. How can there be a partial division of that?} (there is disturbance in puruṣa) due to the disturbance of the limitation of puruṣa just as there is the disturbance of ākāśa due to (the disturbance of) the limitation of wind (vāyvaupādhikakṣobhavat). Or due to having the intent of contact, there is this activity of disturbance in puruṣa.\endnote{Bhikṣu has no confidence in what he stated earlier and by this only weakens his stand. Contrast this style with the definiteness with which Śaṅkarācārya states his theories in the BSBh} That is why in such śruti sayings as “guṇebhyaḥ kṣobhymāṇebhyastrayo devāvijajñire” one hears only the disturbance of the guṇas and not that of puruṣa.  According to śāstra, the entry of Īśvara into prakṛti and puruṣa is only an intention (śāstravadavadhānamātramiti). Thus one necessarily understands (agatyābhyupeyam) that even though Īśvara and the limitation of Īśvara are (both) all-pervasive there is a permanent contact with the all-pervading Īśvara. Even when there is eternal contact between the two, through the example of the jīva and its upādhi, the difference is established of the permanent entities\endnote{Bhikṣu uses the contact of the conditional upādhi of the jīva with the jīva, to justify the contact with the nitya puruṣa and prakṛti with Īśvara. But one can see that it is a very weak argument.}. \dev{ एतच्च जगज्जन्मादिकारणत्वं ब्रह्मशब्दप्रवृत्तिनिमित्तमपि बोध्यम् । मूलकारणस्यैव निरतिशयबृहत्त्वात् । ब्रह्मशब्दश्च पङ्कजादिवद् योगरूढः । अतो न जीवादिर्मुख्यो ब्रह्मशब्दार्थः । तथा चोक्तं नारसिंहे—}
\begin{verse}
\dev{आदिसर्गमहं तावत् कथयामि द्विजोत्तम।}\\
\dev{रहस्यैर्ज्ञायते येन परमात्मा सनातनः ।।}\\
\dev{प्राक्सृष्टेः प्रलयादूर्ध्वं नासीत् किञ्चद्द्विजोत्तम।}\\
\dev{ब्रह्मसंज्ञमभूदेकं ज्योतिर्यत् सर्वकारणम् ।।}\\
\dev{नित्यं निरञ्जनं शान्तं निर्गुणं नित्यनिर्मलम्।}\\
\dev{आनन्दसागरं स्वच्छं यत्काङ्क्षन्ति मुमुक्षवः ।।}
\end{verse}
\dev{सर्वज्ञं ज्ञानरूपत्वादच्युतं व्यापक महत्। सर्गकाले तु संप्राप्ते ज्ञात्वा तं कालरूपकम् अन्तर्लीनविकारञ्च तत्स्रष्टुमुपचक्रमे। तस्मात् प्रधानमुद्भूतं ततश्चापि महानभूत् ।। इति । रहस्यैः संगुप्तशक्तिवर्गैः सहेत्यर्थः । नासीदिति विरतव्यापारतया कारणरूपेण गर्तस्थमृतसर्पवद् विलीनमासीदित्यर्थः । अन्यथाऽन्तर्लीनविकारं चेत्युत्तरासङ्गतेः, अत्यन्तासत्त्वे “सत्त्वाच्चावरस्ये” त्यागामिसूत्रविरोधापत्तेश्च । शान्तम् रागादिरहितम् औपाधिकव्यापारशून्यं च, न तु सुषुप्तवद् विषयसंवेदनरहितम्, सर्वज्ञमित्युत्तरात् । निर्गुणं नित्यमेव गुणानभिमानेन गुणासङ्गेन च गुणातीतं, गुणानां विलयाद्वा निर्गुणत्वम् । नित्यनिर्मलमिति जीवव्यावृत्तिजीवानामौपाधिककादाचित्कमालिन्यात् । मलाश्च क्लेशकर्मविपाकाशयाः । सर्वज्ञं ज्ञानरूपत्वात् इत्यनेन साधननैरपेक्ष्यमीशस्य सर्वाकारवृत्तेरुक्तम्। ज्ञानरूपत्वमत्र निरावरणसत्त्वमूर्तिकत्वं विवक्षितम् । प्रधानस्योत्पत्तिश्च प्रकृतिपुरुषसंयोगेनाभिव्यक्तिर्गौणीति बोध्यम्। “संयोगलक्षणोत्पत्तिः कथ्यते कर्मज्ञानयोः” इति मात्स्यात् ।}

One needs to understand that this cause for the rise of the world is also the reason for the activity denoted by the word Brahman. Only the primary cause has unsurpassed expansion. The word Brahman has an etymological and conventional meaning (of unsurpassed expansion) like that of the word pañkaja (denoting a lotus). Therefore jīva etc., is not the main meaning of the word Brahman.\endnote{Bhikṣu understands that the word Brahman itself implies the meaning that it gives rise to the world etc. It therefore cannot mean the jīva as the advaitin claims.} Thus it is said in the Nāradasimha: “ādisargamaham tāvat kathayāmi dvijottamarahasyaiḥ…tasmāt pradhanamudbhūtam tataścāpi mahānabhūt”. “rahasyaiḥ” in the above verses means with the varieties of powers which are hidden within Īśvara. “nāsīt” (above) means, in the form of a cause through activity without attachment, it was hidden like a dead serpent in a hole. Otherwise the hidden change will be in contradiction to what is said later; if absolutely non-existent then there is the danger of its being incompatible with the later sūtra “sattvāccāvarasya” (BS. II.1.16).\endnote{This is satkāryavāda where the effect is believed to be existent in the cause before it comes into existence.} “śāntam” (above) means devoid of qualities such as attachment etc., and also devoid of activity associated with its limitation. It is not like deep sleep devoid of being conscious of objects; this has been made clear by the word “sarvajñam” (above). “nirguṇam”= it is eternally without qualities; it surpasses the guṇas (guṇātītam) by not being attached to the guṇas and by not having any sense of pride in the guṇas; or because of the absorption of the guṇas it has no characteristic of the guṇas. “nityanirmalam”=It (Brahman) is different from jīva since the jīvas are subject to defects due to being associated at times with their upādhis (limitations). And the defects are deposits (of karma) such as kleśa, karma and vipāka. “sravajñam”=being of the very nature of consciousness Īśa, without the aid of any instruments (of knowledge) has modifications (of the mind) of all shapes (sarvākāravṛtteruktam). Here the idea of the nature of consciousness is desired as being a personification of sattva without any obstacles. One should know that the manifestation of the evolution existing in pradhāna (pradhānasthotpattiśca) is due to the contact between puruṣa and prakṛti which is secondary. Thus it is said in the Matsya Purāṇa “samyogalakṣaṇotppattiḥ kathyate karmajñānayoḥ” (MBh. Mokṣa.216.11; cited in Tripathi p.19, fn.2). “anayoḥ” means of the above mentioned puruṣa and prakṛti. \dev{अनयोः पूर्वोक्तपुरुषयोः । तथाऽनयोर्लयोऽपि वियोगरूप एव कौर्मे प्रोक्तः—}
\begin{verse}
\dev{वियोजयत्यथान्योन्यं प्रधानपुरुषावुभौ ।}\\
\dev{प्रधाप्नपुंसोरनयोरेष संहार ईरितः  ॥}
\end{verse}
\dev{अत्र वाक्ये ब्रह्मसंज्ञमित्यनेन परब्रह्मण्येव रूढिः ब्रह्मशब्दस्योक्ता। तथा विष्णुपुराणेऽपि परमेश्वरे एव ब्रह्मशक्तिरुक्ता—}
\begin{verse}
\dev{न सन्ति यत्र सर्वेशे नामजात्यादिकल्पनाः ।}\\
\dev{सत्तामात्रात्मके ये ज्ञानात्मन्यात्मनः परे ।।}\\
\dev{तद्ब्रह्म परमं धाम स चात्मा परमेश्वरः ।}\\
\dev{स विष्णुः सर्वमेवेदं यतो नावर्तते यति  ।। इति ।}
\end{verse}
\dev{आत्मनः पर इत्यनेन जीवस्याब्रह्मत्वमुक्तम् । विष्णुर्महाविष्णुः । “बृहत्वाद् बृंहणत्वाच्च आत्मा ब्रह्मेति गीयत” इत्यादिवाक्यं चांशांश्यविभागेनात्मसामान्यपरं परमात्मपरं वा, जीवे सर्वशक्तिबृंहितत्वाभावादिति । यस्मात् परमेश्वर एव मुख्यो ब्रह्मशब्दार्थः, “परं जैमिनिर्मुख्यत्वादि” त्यागामिसूत्रात् । हिरण्यगर्भे त्वपरब्रह्मणि ब्रह्मात्मन्यूनशक्तितया तदव्यवहितकार्यत्वादिना ब्रह्मशब्द गौण इति वक्ष्यति “सामीप्यात्तु तद्व्यपदेशः” इति सूत्रेण । अत एव मनौ—}
\begin{verse}
\dev{यत्तत् कारणमव्यक्तं नित्यं सद्सदात्मकम् ।}\\
\dev{तद्विसृष्टः स पुरुषो लोके ब्रह्मेति गीयते ।। इति ।}
\end{verse}
\dev{“एतद्वै तद्ब्रह्म परमपरं चे” त्यादौ श्रुतौ गौणमुख्यभेदेन ब्रह्मद्वयवचनं बोध्यम् । अन्यथा “यन्मनसा न मनुते येनाहुर्मनो मतम्, तदेव ब्रह्म त्वं विद्धि नेदं यदिदमुपासत” इत्यादिश्रुतिविरोधात् । अन्यजीवेषु तु ब्रह्मशब्दप्रयोगोंऽशांश्यभेदाद् विभुत्वसर्वाधारत्वादिगुणयोगाद् वेति बोध्यम् ।}

“anayoḥ” means, the above mentioned puruṣa and prakṛti. Similarly their dissolution is mentioned in the Kūrma P. as their separation (viyogarūpa eva): “viyojayatyathānyonyam…eṣa samhāra īritaḥ” (Uttarārdha.48.19; cited in ibid p.19. fn.3). Similarly in the Viṣṇu P. as well the power of Brahman is mentioned as situated in Parameśvara alone: “na santi yatra sarveśe nāmajātyādikalpanāḥ…sa viṣṇuḥ sarvamevedam yato nāvartate yatiḥ” (6.4.36; ibid. p.20. fn.1). “ātmanaḥ para”= through this (phrase) it is indicated that the jīva is other than Brahman.  “viṣṇuḥ”=denotes Mahā Viṣṇu. The sentence: “bṛhattvād bṛhaṇattvācca ātmā brahmeti gīyate” denotes in general the ātman or the paramātman in the sense of non-separation due to the nature of being part and whole, as there is the absence of increase of all powers in jīva. That is why Parameśvara alone is primarily denoted by the word Brahman by the following BS “param jaiminirmukhyatvāt” (BS. 4.3.12).

But with reference to Hiraṇyagarbha , the other (inferior) Brahman, which being less powerful than Brahman and also due to its being directly connected to effects (such as the world) the word Brahman has been used in a secondary sense. This will be mentioned in the sūtra  “sāmīpyāttu tadvyapadeśaḥ” (BS.4.3.9). That is why Manu states: “yattat kāraṇamayaktam…brahmeti gīyate” (I.11; the reading in my edition has “kīrtyate” instead of “gīyate” as the last word). In such śruti statements as “etadvai tadbrahma paramaparam ca” one understands a twofold Brahman different (from each other) as primary and secondary. Otherwise it will contradict such śruti statements as  “yanmanasā na manute…nedam yadidamupāsate” (Kena.Up. 1.6). The use of the word Brahman when referring to other jīvas is understood in the sense of non-difference between the part and whole or in the sense of all pervading (ātman) and as the support of everything (Brahman).

\dev{आधुनिकास्तु जीवब्रह्मणोरखण्डतया जीवेऽपि ब्रह्मशब्दो मुख्य एव आकाशशब्द इव घटाकाशे । जीवस्याब्रह्मत्वं त्वज्ञानकल्पितम्। तथाहि—“तत्त्वमसि, अहं ब्रह्मास्मि, अनेन जीवनात्मनाऽनुप्रविश्य नामरूपे व्याकरवाणि, नान्यदतोऽस्ति द्रष्टा”}

\dev{इत्याद्यभेदश्रुतिशतेभ्यो जीवोऽपि ब्रह्मैव चिन्मात्रत्वाविशेषात् । ऐश्वर्यबन्धयोश्चोपाधिद्वयधर्मत्वात् । न च “द्वा सुपर्णा सयुजा सखाया समानं वृक्षं परिषस्वजाते, नित्यो नित्यानां चेतनश्चेतनानामेका बहूनां यो विदधाति कामान् । तमात्मस्थं येऽनुपश्यन्ति धीरास्तेषां शान्तिः शाश्वता नेतरेषाम्,  आत्मनि तिष्ठन्  आत्मनोऽन्तरः स मे आत्मेति विद्यात्, त्रिषु धामसु यद् भोग्यं भोक्ता भोगश्च यद्भवेत् । तेभ्यो विलक्षणः साक्षी चिन्मात्रोऽहं सदाशिवः ।।” इत्यादिभेदश्रुतिशतानुपपत्तिरिति वाच्यम्, औपाधिकभेदानुवादकत्वेन तादृशवाक्योपपत्तेः । यथा हि घटाकाशादुपाधिपरिछिन्नमहाकाशोऽन्य इति व्यवह्रियते तथैव बुद्ध्यवच्छिन्नचैतन्याज्जीवादन्यः परमेश्वर इति श्रुतिषु व्यवह्रियते । अथवा यथा मायाविनः खड्गचर्मधरात् सूत्रेणाकाशमधिरोहतः सकाशात् स एव मायावी परमार्थभूतो भूमिष्ठोऽन्यस्तथैवाविद्याकल्पितात् कर्तृभोक्तृलक्षणाज्जीवादन्यः परमेश्वरोऽस्तु । तदवमवच्छेदभेदेन बिम्बम्प्रतिबिम्वरूपेण वा जीवेश्वरयोर्भदः । तथा श्रूयतेऽपि—“आकाशमेकं हि यथा घटादिषु पृथग्भवेत्। तथात्मैको ह्यनेकस्थो जलाधारेष्विवांशुमान्” इत्यादिष्विति वदन्ति ।}

\textbf{Ques:} Modern advaitins\endnote{Bhikṣu castigates Śaṅkara as a modern day Vedāntin which is the same as advaita for Bhikṣu. This also can
mean that there is an implicit approval of the other schools of Vedānta.} consider that the word Brahman is used in a primary sense due to the immutable nature of both jīva and Brahman, like the word ākāśa with reference to ākāśa in the delimited pot. The idea of jīva not being Brahman is imagined due to ignorance (according to them). Thus through hundreds of śruti statements like “tattvamasi” (Chānd.UP.6.8.7; 6.9.4; 6.14.3??), “aham brahmāsi” (Bṛ.Up. i.4.10??), “anena jīvenātmanā’nupraviśya namarūpe vyākaravāṇi” (Chānd.Up. 6.3.2), “nānyadato’sti draṣṭā”(Bṛ.Up.3.7.23) (it is established that) the jīva is also Brahman alone, as there is the common characteristic of consciousness (cinmātratvāviśeṣāt). And both have upādhidharmas (characterised by) power and bondage (aiśvaryabandhayoścopādhidvayadharmatvāt). They also say it is not in opposition to other hundreds of śruti statements that declare difference such as “dvā suparṇā…netareṣām”, “ātmani tiṣṭhan…vidyāt” (not traced in śruti; however according to Tripathi p.20.fn.2 it is in the Liṅga.P) , “triṣu dhāmasu…sadāśivaḥ”(Kaivalya.Up.18). By explaining the possession of different limitations the above sentences can make sense. Just as the great ākāśa is considered different from the limited pot-ākāsa so also in śrutis, Parameśvara is considered to be different from the jīva-consciousness limited by the intellect. Or just as the same māyāvī (magician) who ascends to the sky through a string from the shield holding the sword of the māyāvin, appears to be different from the one standing in truth on the earth, so also let Parameśvara be different from the jīva characterized by being an agent, an experiencer etc., which comes about because of the imagination of avidyā (ignorance). Thus in this manner due to   different limitation(s), or like the mirror and the reflection in it, there is a difference between the jīva and Īśvara. Thus one hears the saying “ākāśamekam hi yathā ghaṭādiṣu…jalādhāreṣvivāṁśumān” (not traced). \dev{तत्रोच्यते— अभेदवाक्यानुरोधेन भेदवाक्यानामौपाधिकभेदपरत्यं यथा कल्यते,}

\dev{तथा भेदवाक्यानुरोधेनाभेदवाक्यानामविभागादिलक्षणाभेदपरत्वं कथं न कल्प्यते ? अविरोधस्योभयथैध सम्भवात् । श्रूयते चाविभागादिरूपाभेदोऽपि “यथोदकं शुद्धे शुद्धम: क्षिप्तं तादृगेव भवति एवं मुनेर्विजानत आत्मा भवति गौतम, न तु तद्द्वितीयमस्ति ततोऽन्यद् विभक्तमित्यादिश्रुतिषु । स्मृतिषु च—}
\begin{verse}
\dev{अविभक्तं च भूतेषु विभक्तमिव च स्थितम् ।}\\
\dev{व्यक्तं स एव वाऽव्यक्तं स एव पुरुषः परः  ॥ इत्यादिषु ।}
\end{verse}
\dev{प्रत्युतविभागादिलक्षणाभेदस्य पारमार्थिकतया तत्परत्वमेवोचितम् । औपाधिकभेदस्य तु मिथ्यात्वेन तत्परत्वं नोचितमिति । न चाविभागपरत्वे सत्यभेदशब्दे लक्षणाऽस्ति, भिदिर्विदारणे इति विभागेऽपि भिदिधातोरनुशासनात् ।}

\dev{ननु जीवेऽपीश्वरभेदस्य स्वानुभवसिद्धतया तत्र श्रुतेर्न प्रामाण्यं किन्तूपासनार्थेऽनुवादतामात्रम् । प्रमाणान्तरानधिगतत्वात्तु तयोरभेद एव तात्पर्यमिति चेन्न, अभेदोपासनवाक्येन प्रसक्तस्य भेदाभिभवस्यैव विवेकवाक्यैः प्रतिषेधस्यास्माभिरभ्युपगमात्, दुःखभोगादिदोषाणामीश्वरे प्रतिषेधात्।}

\dev{ननु मोक्षफलश्रवणादभेदवाक्यानामेव सम्यग् ज्ञानपरत्वमसङ्कोचश्चेति युक्तमिति चेन्न “पृथगात्मानं प्रेरितारञ्च मत्वा जुष्टस्ततस्तेनामृतत्वमेती” त्यादि श्रुतिभिर्भेदज्ञानस्यापि मोक्षहेतुत्वश्रवणात् । भेदज्ञानस्य विवेकज्ञानतया अविद्यानिवर्तकस्यैव “तमेव विदित्वाऽति मृत्युमेती” त्यादिभेदवाक्येष्वाधिक्येन मोक्षफलश्रवणात्। किं च सम्यग्ज्ञानत्वेन हेतुना भेदाख्यविवेकज्ञानस्यैव साक्षान्मोक्षहेतुत्वं श्रुतिसिद्धम्। “अस्थूलमनणु अह्रस्वम् , न तदश्नाति किंचन, त्रिषु धामसु यद् भोग्यं भोक्ता भोगश्च यद्भवेत् तेभ्यो विलक्षणः, अथात आदेशो नेति नेति तर्ह्येतस्मादिति नेत्यन्यत् परमस्ती” त्यादिश्रुतिषु,}
\begin{verse}
\dev{“प्रधानपुरुषव्यक्तकालानां परमं हि यत् ।}\\
\dev{पश्यन्ति सूरयः शुद्धं तद् विष्णोः परमं पदम् ।।}
\end{verse}
\dev{इत्यादिस्मृतिषु च भेदज्ञानस्यैव सम्यग्ज्ञानत्वं गम्यते ।}

Ans: It is said: that just as in accordance with the statements supporting identity (between Brahman and ātman) one imagines that the statements that mention difference  is due to the difference in limitations (of the two), so also with regard to the statements supporting difference/non-identity why cannot one imagine identity to be in the form of non-separation (of the two) in statements supporting identity? Non-contradiction is possible both ways. One hears of identity in the form of non-separation in śruti sayings like “yathodakam…gautama” (Kaṭho.Up. II.1.15); “na tu taddvidīyamasti…vibhaktam” (Bṛ.Up.4.3.27), and also in smṛtis like: “avibhaktam ca bhūteṣu…sa eva puruṣaḥ paraḥ” (Viṣ.P. 6.4.44 cited in Tripathi. p.21.fn. 1). On the contrary, it is proper that identity in the form of non-separation is what (these statements) are inclined towards in truth (pratyuta avibhāgādilakṣaṇābhedasya pāramārthikatayā tatparatvamevocitam). The difference in limitations by being false is not in favour of that. Also when the word abheda is interpreted as non-separation it is not in a secondary sense (na cāvibhāgaparatve satyabhedaśabde lakṣaṇā’sti). Even when the root “bhid” denotes breaking (bhidirvidhāraṇe) there is (also) instruction regarding the root “bhid” to mean separation\endnote{Bhikṣu thus justifies that even grammatically the root bhid has the meaning of separation. So abheda can also mean non-separation.}.

\textbf{Ques:} Even in the jīva one experiences the difference from Īśvara, so there is no need for the authority of the śruti (for this purpose).  However,to say that  since there is no other pramāṇa available for this purpose, so its intention is only identity,

\textbf{Ans:} is not right. We accept only what is rejected through sentences that reflect and overpower statements of difference (bheda) which are connected to identity statements which have devotion as intent, this is because there is rejection of defects such as sorrow, experience etc., in Īśvara.

\textbf{Ques:} If it is said that since freedom is declared as the result, sentences of identity alone lend themselves amenably to knowledge without any restriction (then) the answer is: 

\textbf{Ans:} It is not so; śruti statements such as “pṛthagātmānam…tenāmṛtatvameti” (Śvet. Up.I.6) declare that even statements of difference can be the cause for freedom (mokṣahetutvaśravaṇāt). Through discrimination of knowledge pertaining to knowledge of difference (bhedajñānasya vivekajñānatayā) one hears mainly of the result of freedom through the removal of ignorance in sentences declaring difference such as: “tameva viditvā’ti mṛtyumeti” (Śvet. Up.III.8; VI.15). Moreover  since the cause  is having right knowledge, only discriminating knowledge known as difference is the direct cause of freedom which is established by śruti.\endnote{Bhikṣu seems to stress the fact that discrimination assumes a difference between two entities for it to function. This seems to also support the S/Y theory of insight into the difference between puruṣa and prakṛti being the criterion for moka/apavarga.} In such śruti satements as: “asthūlamanaṇu ahrasvam…” (Bṛ.Up.3.8.8), “natadaśnāti kiṁcana…” (Bṛ.Up. 3.8.8; Subāla. Up. 3.2), “triṣu dhāmasu yadbhogyam bhoktā bhogaśca yad bhavet tebhyo vilakṣaṇaḥ…(Kaiv. Up.18), “athāta ādeśo…netyanyat paramasti” (Bṛ. Up. 2.3.6) and in smṛti statements as “pradhānapuruṣavyaktakālānām…viṣṇoḥ paramam padam” only statements of difference lead to correct knowledge. \dev{“सत्येन लभ्यस्तपसा ह्येष आत्मा सम्यग्ज्ञानेन ब्रह्मचर्येण नित्यमित्यादिश्रुतिभ्यः, ततो मां तत्त्वतो ज्ञात्वा विशते तदनन्तरमि” त्यादिस्मृतिभ्यश्च सम्यग् ज्ञानादेव मोक्षः श्रूयत इति अभेदवाक्यान्यपि साक्षादविद्यानिवर्तकत्वासम्भवेन ब्रह्मात्मतावाक्यानमेव शेषभूतानि । न ह्यभेदज्ञानं साक्षादेवाहं दःखीत्यादिलक्षणामविद्यामुच्छेत्तुमर्हति । एकास्मिन्नेवाकाशेऽवच्छेदभदेन शब्दतदभाववदेकस्मिन्नेवात्मनि कार्यकारणलक्षणावच्छेदभेदेन दुःखादितदभावसंभवादिति । किं च ब्रह्माभेदस्य जडेष्वपि श्रवणान्न तेन दुःखादिशून्यतासिद्धिः। तस्माद् विवेकवाक्यरूपतया भेदवाक्यान्येव बलवन्ति, तद्विरोधेन चाभेदवाक्यान्यविभागपरतयैव संकोच्यानि ।}

\dev{ननु “य एतस्मिन्नुदरमन्तरं कुरुतेऽथतस्य भयं भवति” त्यादि श्रुतौ}
\begin{verse}
\dev{“तस्यात्मपरदेहेषु सतोऽप्येकमयं हि यत् ।}\\
\dev{विज्ञानं परमार्थोऽसो द्वैतिनोऽतशयदर्शिनः ॥}
\end{verse}
\dev{इत्यादिस्मृतौ च भेदनिन्दाश्रवणान्न भेदपरत्वं श्रुतीनां सम्भवतीति चेन्न, अभेदवाक्यानामविभागपरतया भेदनिन्दावाक्यानामपि विभागलक्षणमेदपरत्वात् प्रतिपाद्यविपरीतस्यैव निन्दार्हत्वात् । अन्यथा “मनसैवेदमाप्तव्यं नेह नानास्ति किंचन, मृत्योः स मृत्युमाप्नोति य इह नानेव पश्यतो” त्यादिश्रतिषु जडवर्गेष्वपि भेदनिन्दनादभेदः स्यात् विवेकादिवाक्यान्न भवतीति चेत्, तुल्यं जीवेऽपि । जीवादपि ब्रह्मणो विवेकस्योक्तत्वात्। “तस्यात्मपरदेहेष्वि” त्यादिवाक्यं च “एकमयमि” ति शब्दादवैधर्म्यलक्षणभेदपरं प्रकरणाद् ब्रह्मात्मैक्यपरमेव वेति ।}

From such śruti sayings like: “satyena labhyastapasā…brahmacaryeṇa nityam” (Muṇḍ. Up.3.1.5) and also from smṛti statements like: “tato mām tatvato jñātva viśate tadanantaram” one hears that mokṣa is attained only by correct/right knowledge. Statements of identity also, by not being able to remove directly ignorance, only end up by meaning Brahman of the nature of ātman (brahmātmatāvākyānāmeva śeṣabhūtāni). The knowledge of identity cannot directly remove ignorance of the form ‘I am hungry’ etc. Just as in one ākāśa it is possible to have sound and its absence so also in one ātman, it is possible through different limitations characterized by cause and effect to have the form of sorrow and its absence. Moreover, as one hears the identity of Brahman even with regard to inanimate things it is not proven that there is absence of duḥkha through that (kiṁca brahmābhedasya jaḍeṣvapi śravaṇānna tena duḥkhādiśūnyatāsiddhiḥ). Therefore  having the nature of discrimination alone, statements which advocate difference are more powerful; as against them sentences that advocate identity need to be narrowed down in the sense of non-separation (tadvirodhena cābhedavākyānyavibhāga paratayaiva saṁkocyāni).

\textbf{Ques:} In such śruti statements as:  “ya etasmin…bhayam bhavati” (not traced) and in smṛti sayings like: tasyātmaparadeheṣu…dvaitino’tathyadarśinaḥ” (Viṣ.P. 2.14.31; cited in Tripathi p.22.fn.1) since difference is found fault with, it cannot be said that it is not possible to interpret śruti statements as favouring difference (between Brahman and ātman). 

\textbf{Ans:} The fault needs to be removed by pointing out that statements of identity are in the sense of non-separation and statements  which find fault with difference are (to be understood) in the sense of difference which has the characteristic  of separation; in this way it is the contradiction that needs to be blamed (viparītasyaiva nindārhatvāt). Otherwise in such śruti statements as: “manasaivedamāptavyam…iha nāneva paśyati” (Kaṭho. Up. II.1.11) by faulting difference even in inanimate things one will have to accept identity with them. 

\textbf{Ques:} If it is said that because of statements of difference it will not occur with regard to them (inanimate things), then with reference to them the answer is:

\textbf{Ans:} the same logic applies to the jīva as well (tulyam jīve’pi); there are statements that proclaim the difference of Brahman from jīvas. Sayings like: “tasyātmaparadeheṣu” (not traced), “ekamayam” (not traced) etc., are inclined towards identity characterized by difference (avaidharmyalakṣaṇābhedaparam) or towards identity of Brahman and ātman depending on the context (prakaraṇād brahmātmaikyaparameva veti).

\dev{नन्वेवमपि लाघवमैकात्म्यश्रुतेर्बलमस्त्विति चेन्न,     बन्धमोक्षव्यवस्थानुपपत्तिप्रमाणसिद्धतया  आत्मनानात्वगौरवस्यैवादर्तव्यत्वात् । या च प्रतिबिम्बावच्छेदरूपाभ्यां बन्धमोक्षादिव्यवस्था रचिता सा  स्वशिष्यमोहनमात्रं प्रतिबिम्बस्य तुच्छतया बन्धमोक्षानौचित्यात् । ज्ञानेनोपाधिवियोगे जीवनाशप्रसङ्गात्, तत्त्वमसीत्यादिवाक्यार्थतयाऽभ्युपगतस्याखण्डत्वस्य विरोधाच्च सदसतोरभेदानुपपत्तेः । न च प्रतिबिम्बोपाधिना बिम्बस्यैव जीवत्वं वाच्यम्, तथा सत्यवच्छेदभेद एव पर्यवसानात् किमिति प्रतिबिम्बवादः पृथङ् निर्मीयते ।}

\dev{अथैवमुच्यते बिम्बप्रतिबिम्बयोः परमार्थतो नास्ति भेदः किन्तु द्विचन्द्रदर्शनवदेकस्मिन्नेव वस्तुनि भेदभ्रममात्रमिति, तदपि न विचारसहम्, बन्धमोक्षानुपपत्तितादवस्थात् । व्यवस्थातत्प्रतिपादकश्रुत्यादिकं च सर्वमेवात्मातिरिक्तम् अज्ञानकल्पितमेव वक्तव्यमिति चेन्न, एवमपि प्रमाणस्यापि बाधेन ब्रह्मातिरिक्तनिषेधस्य श्रुतस्यापि पुनः संशयापत्तेः, स्वाप्नशब्दस्य जाग्रति बाधे   पुनस्तच्छब्दबोधितार्थसंशयवत्। तथा बन्धमोक्षादिक सर्वथा नास्तीति श्रवणानन्तरं मननादिषु प्रवृत्त्यनुपपत्तेश्च, आप्तवाक्यतः पुरुषार्थाभावनिर्णयादित्यादीन्यनेकानि दूषणानि । किं चात्मभेदादिकमज्ञानकल्पितं चेत्तदज्ञानं कस्येत्युच्यताम् । ब्रह्मणो भ्रान्तत्वे “ज्ञाज्ञौ द्वावजावीशानीशौ, नाविद्यानुभवात्मनि स्वप्रकाशे, अभयं भ्रान्तिरहितमनिद्रमजरामरम्, }
\begin{verse}
\dev{परः पराणां सकला न यत्र ।}\\
\dev{क्लेशIदयः सन्ति परावरेशे ।}
\end{verse}
\dev{इत्यादिवाक्यैर्ब्रह्मण्यज्ञानप्रतिषेधस्य विरोधात् । जीवस्य भ्रान्तत्वे चान्योन्याश्रयात् । भ्रमेण बिम्बप्रतिबिम्बभेदसिद्धौ जीवसिद्धिर्जीवसिद्धौ च तदाश्रयस्य भ्रमस्य सिद्धिरिति ।}

\textbf{Ques:} Even then in the interest of simplicity let the śruti statements of identity be accepted then the answer is, 

\textbf{Ans:} No; due to the proof of contradiction of states of bondage and mokṣa (existing together) one needs to support the  heavier theory of the existence of many jīvas/ātmans.\endnote{The standard criticism that one does not see all jīvas attaining mokṣa when one jīva attains it. This automatically will support the theory of many jīvas} As for the explanation of the states of bondage and liberation being of the nature of reflection or delimitation (pratibimbāvacchedarūpābhyām) that is only for the sake of confusing one’s disciples; a reflection being without substance, bondage and liberation are inappropriate.  Through knowledge when the limitation is disjointed there is the danger of the destruction of the jīva; through statements like “tatvamasi” etc., the meaning obtained of being complete/identical is also contradicted, as identity between what exists and what is non-existent is illogical. Nor can one aver jīvatva (the quality of having the property of jīva) of the reflection like the prototype (bimbasyeva). Thus since what results is only the difference of limitation, why is this theory of reflection manufactured as different.Ques: If then it is said that there is no difference between the prototype and the reflection in truth but like seeing two moons it is an illusion of division in one object alone, 

\textbf{Ans:} even then it is not fit for discussion (vicārasaham) due to the contradiction of bondage and freedom of those states.

\textbf{Ques:} If it is said that the state of the ātman and śruti sayings which point that out (vyavasthātatpratipādakaśrutyādikam ca) mention that everything other than ātman is to be known as imagined by ignorance, then the answer is:

\textbf{Ans:} it is not so. In this way when pramāṇa (testimony) also is contradicted, even śruti which rejects anything other than Brahman will be subject to doubt. Just as there is a contradiction of the use of the word ‘dream state’ when in the ‘waking state’ so also there arises the doubt as to  the meaning of what the word itself denotes (svāpnaśabdasya jāgrati bādhe punastacchabdabodhitārthasaṁśayavat). Thus having heard that (in truth) there is no bondage and liberation at all times, it will be illogical to engage in practices such as reflection etc., since from the words of the trustworthy, one decides that there is an absence of any puruṣārtha (goal to work for).\endnote{If the goal/puruṣārtha is liberation since the śruti says that there is in reality no bondage or liberation then there is nothing to work for.} Thus there are many defects in this approach.Moreover if the difference in ātmans is imagined due to ignorance, then one needs to answer the question as to whose is ignorance. If Brahman has ignorance it contradicts statements such as “jñājñau dvāvajāvīsānīśau” (Śvet. Up. 1.9), “nāvidyānubhavātmani svaprakāśe” Nṛsim. 9.6), “abhayam bhrāntirahitamanidramjarāmaram” (not traced), “paraḥ parāṇām…parāvareśe” (not traced) which reject ignorance situated in Brahman. If ignorance belongs to jīva it will suffer from the defect of ‘anyonyāśraya’ (mutual dependence on each other.\endnote{Jīva itself comes into being due to avidyā and avidya is dependent on jīva; thus there is the defect of anyonyāśrayabhāva.} If the difference as prototype and reflection is established due to ignorance, then even in the case of the existence or non-existence of the jīva there is the establishment of ignorance on which it depends. \dev{अथ ब्रह्मण्येवाज्ञानम्, अज्ञानप्रतिषेधवाक्यानि चानिर्वचनीयाज्ञानप्रतिषेधेन परमार्थपराणीति चेन्न, व्यवहारभूमावपि ब्रह्मण्यज्ञानासम्भवात्। “ज्ञाज्ञौ द्वावजावीशानीशावित्यादिवाक्यैर्जीवेऽज्ञान- व्यवहारदशायामेव ब्रह्मण्यज्ञानप्रतिषेधात् । किं च प्रतिविम्बवादे अंशश्रुतिस्मृतिसूत्राणां विरोधः स्यात्, प्रतिबिम्बे- ऽम्शव्यवहाराभावात् । यत्त्वा “भास एवे” ति जीवप्रकरणस्थसूत्रं, तत्राभासशब्दो न प्रतिबिम्बवाची तथा प्रयोगादर्शनात् “अंशो नानाव्यपदेशादि” ति सूत्रेणांशत्वमुक्त्वा पुनः सूत्रान्तरेण तद्विरुद्धप्रतिबिम्बतायाः सूत्रणानौचित्याच्च। किन्तु प्रकाशवाची “सर्वेन्द्रियगुणाभासमिति । अस्तु वा हेत्वाभासवदात्माभासवाची जीवस्यापि पारमार्थिकात्मत्वस्य प्रतिषेध्यमानत्वात् प्रकाशे प्रयोगदर्शनादिति, न तत्स्वाभासं दृश्यत्वादिति च । अथैवं प्रतिबिम्बदृष्टान्तः कथमुपपद्येतेति चेत्, अंशांश्यविभागेन किरणसूर्ययोरिव जीवब्रह्मणोरेकपिण्डीभावेन जीवेन अंशैः परमात्मनो नानाबुद्धिप्रतिबिम्बनादित्यवेहि }

Then let ignorance be in Brahman.\endnote{If it is not in jīva then it can only be in Brahman} 

\textbf{Ques:} Then if sayings that reject ignorance and due to rejection of ignorance which is ‘anirvacanīya’ (indefinable) they are said to be leaning towards Brahman, then the answer is: 

Ans: it is not so; even then in the world of activity (vyavahārabhūmau) it will not be possible to have ignorance regarding Brahman.  Through statemens such as “jñājñau dvāvajau…” (Śvet.Up. I.9) only in the state of (worldly) activity through ignorance in jīva, is there rejection of ignorance in Brahman.\endnote{Jīva with its apparatus of avidyā-limitation (upādhi)strives to gt rid off avidyā and attain Brahman and realize
the true nature of Brahman without ignorance.} Moreover in the theory of reflection there is opposition of the śruti and smṛti sūtras which proclaim that (jīva is a) part (of Brahman), as there is no partial activity in a reflection;  the word “ābhāsa” in the sūtra “ābhāsa eva ca” (BS.2.3.50) in the section dealing with the jīva does not denote a reflection, as one does not see it being used in that sense; having mentioned its (jīva’s) being a part through the sūtra “amśo nānāvyapadeśat…” (2.3.43) then by another sūtra to mention reflection that is opposed to it is not correct. However it denotes illumination (prakāśavācī) as in the saying “sarvendriyaguṇābhāsam…” (Gītā.13.14). Or like a fallacy (hetvābhāsavat/ semblance of reason) it can denote a semblance of the jiva to the ātman;  since there is rejection of jīva being of the nature of the ultimate ātman, it points to its use in the act of illumination\endnote{Jīvātman resembling paramātman has usefulness in the act of illuminating objects.}; it is also not self-illuminating (svābhāsam) as it is an object of experience (dṛśyatvāt).   Then how can the example of reflection fit in? Ans: By non-separation of the parts and the whole similar to the sun and its rays, jīva and Brahman by uniting as one jīva, through parts is reflected in the many intellects of paramātman (jīvabrahmaṇorekapiṇḍībhāvena); it can be understood thus.

\dev{नन्वेवं मा भवतु प्रतिविम्बवादोऽवच्छेदवादस्तु स्यादिति, मैवम्, अवच्छेदवादेऽपि “तयोरन्यः पिप्पलं स्वाद्वत्त्यनश्नन्नन्योऽभिचाकशी” त्यादिविभागानुपपत्तेः, धर्मिण एकत्वात् । अथोपाधिविशिष्टयोरेव जीवेश्वरत्वे वाच्ये, तथा च विशेषणभेदाद् भेदः स्यादिति चेन्न, विशिष्टस्यातिरेकानतिरेकयोरुभयतः पाशात् । विशिष्टस्यातिरेके भवदभिमतस्य “तत्त्वमसी” त्यादिवाक्यार्थस्याखण्डत्वस्यानुपपत्तेः । न च लक्षणया विशेषणद्वयं परित्यज्य केवलचैतन्यपरत्वं तत्त्वंपदयोर्वक्तव्यमिति वाच्यम्, वाक्यार्थयोर्विशिष्टयोरतिरिक्ततया केवलचैतन्ये तटस्थलक्षणापत्तौ जीवस्य देहाद्यभिमाननिवृत्त्यसम्भवात्, सर्वपदार्थपरित्यागेन लक्षणया “तत्त्वमसी”- त्यभेद- वाक्यस्य जीवब्रह्मभेदवादिनाप्युपपादयितुं शक्यत्वाच्च।}

\dev{किं च मोक्षावस्थायां विशेषणनाशेन जीवनाशप्रसङ्गः, ब्रह्माणि प्रपञ्चाध्यारोपा- पवादयोरविवेकरूपयोर्वैयधिकरण्यापत्तिश्च । तथा लक्षणां विनैवास्माभिस्तत्त्वमस्यादिवाक्यानां व्याख्येयतया लक्षणानौचित्यं च ।}

\dev{यदि च विशिष्टमनतिरिक्तमुच्यते तदा एकस्मिन्नेवात्मन्यवच्छेदभेदेन बन्धमोक्षैश्वर्यादिप्रसक्त्या “ये तद्विदुरमृतास्ते भवन्त्यथेतरे दुःखमेवापियन्ति” समाने वृक्षे पुरुषो निमग्नोऽनीशया शोचति  मुह्यमानः । जुष्टं यदा पश्यत्यन्यमीशमस्य महिमानमिति वीतशोक” इत्यादिविभागानुपपत्तिः । न कस्मिन्नेव वृक्षेऽवच्छेदभेदेन कपिसंयोगतदभाववति एको वृक्षः कपिसंयोगवानन्यश्च नेत्यमूढैः प्रयुज्यते । न च लौकिकभेदानुवादेन तादृशवाक्यान्युपपादयितुं शक्यन्ते, ज्ञानफलस्याज्ञानदोषस्य चापारमार्थिकत्वे तदर्थकप्रवृत्याद्यनौचित्यात् । अस्मन्मते चात्मनि दुःखभोगतदभावयोः कालभेदेन पारमार्थिकत्वस्योपपाद्यत्वात् ।}

\textbf{Ques:} If it is then said: let the reflection theory be discarded and let us accept the theory of limitation; then the answer is: 

\textbf{Ans:} even in the theory of delimitation it contradicts statements of separation such as “tayoranyaḥ pippalam…abhicākaśīti” (Muṇḍ.Up. 3.1.1) as the dharmī (one possessing the characteristic) is one (dharmiṇaḥ ekatvāt). 

\textbf{Ques:} If it is said that both jīva and Īśvara qualified by limitations are only meant (athopādhiviśiṣṭayoreva jīveśvartve vācye)  and then, due to the difference in characteristics  there is difference (between the two)\endnote{This is a reference to the jahadajajallakṣaṇā device used to distinguish Brahman from ātman. For a detailed discussion on this lakṣaṇā see \textit{The Naiṣkaymyasiddhi of Sureśvara} by R.Balasubramanian, pp.163-164; 225-228} then the answer is:Ans: It is not so; the defect of having excellence or not affects both the qualified entities.  When the qualified (entities) have excellence then in your preferred meaning (bhavadabhimatasya) of the sentence “tatvamasi”, there will be a contradiction in knowing it as without division (as complete) (vākyārthasyākhaṇḍatvasyānupapattiḥ). Nor can it be said that by giving up the two qualifiers as having secondary meaning, the words “tat” and “tvam” stand only for ‘consciousness’, as, apart from the qualified meaning of the sentences, in the complete singular consciousness (kevalacaitanye), there is the difficulty of having other qualifications (taṭasthalakṣaṇāpattau);\endnote{Taṭasthalakṣaṇā is a property of a thing which is distinct from its nature and still it is known by that property like the property of smell in the case of earth (Apte’s Practical Sanskrit English Dictionary).} (and) in the case of the jīva, it is (also) not possible to get rid of the sense of agency/pride of the body etc., by giving up all things in a secondary sense;  however, it is possible to reconcile the identity of the sentence “tattvamasi” with those who advocate the difference between Brahman and jīva.\endnote{It is easier to accommodate these ideas when Brahman and ātman are viewed differently.} Moreover in the stage of mokṣa by giving up qualifications there will be the contingency of destruction of the jīva; there is also the danger of the superimposition of the world and its refutation being in different case relations   in Brahman (brahmaṇi prapañcādhyaropapavādayoravivekarūpayorvaiyadhikaraṇyapattiśca). Thus since we can explain sentences like “tattvamasi” even without (resorting to) any secondary sense (tathā lakṣaṇām vinā) it is not correct/proper to use a secondary sense.

\textbf{Ques:} If then one says that there is no reduction of excellence in the qualified one then the answer is:

\textbf{Ans:} in one ātman itself, due to the difference in limitations there will be the inconsistency of connection with bondage, liberation, possession of power etc. It will also not be consistent with statements of separation such as “ye tadviduramṛtāste…   duḥkhamevāpiyanti” (not traced), “samāne vṛkṣe…muhyamānaḥ” (Muṇḍ.Up.3.1.2; Śvet.Up. 4.7), “juṣṭam yada…vītaśokaḥ” (ibid). When due to difference in limitation in one tree itself, it has both contact with a monkey and also it has no contact (with it), then those who know (amūḍhaiḥ) do not say that the one tree has contact with the monkey and the other has no contact.\endnote{There is only one tree and when in contact with the monkey it cannot be said to be another tree not having contact with the monkey.} Nor can such sayings be made intelligible by explaining it as following the meaning of difference common in worldly usage. As the result of knowledge and the defect of ignorance have great significance of meaning, (apāramārthikatve), activity for the sake of achieving it is not appropriate. In our view the experience of sorrow and its absence in ātman is due to the change in time (and) thus is applicable to (achieving) the highest truth (asmanmate cātmani duḥkhabhogatadabhavayoḥ kālabhedena pāramārthikasyopapādyatvāt).\endnote{Bhikṣu seems to be still dealing with the different limitations on the same tree with reference to the monkey. Since both truth and ignorance cannot be compatible with the real truth he uses that for demolishing the avacchedakavāda of Śaṅkara wo is his main obsession.} \dev{किं चाखण्डैकात्म्ये सति मुक्तस्य पुनर्बन्धापत्तिः, एकान्तःकरणवियोगेऽपि मुक्तांश एवान्तःकरणान्तरसम्भवात् । यथैकघटावच्छिन्नाकाशस्य तद्धटभङ्गेऽपि घटान्तरेण पुनः सम्वन्धो भवति तद्वत् । न च तेनावच्छेदेनान्तःकरणसम्बन्ध एव न भवतीति वाच्यम्, तथा सति योगिनां सर्वगतत्वानुपपत्तेरिति । अन्तःकरण-}

\dev{गणोऽस्माभिरेव कल्पित इति न शङ्कनीयम् । कपिलादिभिरपि श्रुतिद्वैधे यथोक्तर्काणामेव निर्णायकत्वेनोक्तत्वात् । यथा कपिलसूत्राणि—“जन्मादिव्यवस्थातः पुरुषबहुत्वम्, उपाधिभेदेऽप्येकस्य नानायोग आकाशस्येव घटादिभिः, उपाधिर्भिद्यते न तु तद्वान्, एवमेकत्वेन परिवर्तमानस्य न विरुद्धधर्माध्यासः, नाद्वैतश्रुतिविरोधो जातिपरत्वात्” इति । एतेषां पञ्चसूत्राणामयमर्थः—}

\dev{आत्मैक्ये सत्यौपाधिकानां जन्ममरणादीनामनौपाधिकानां च भोगादीनां श्रुतिस्मृतिसिद्धा आश्रयविभागव्यवस्था न स्यात् “अयं जातोऽयं मृत” इत्यादिरूपा । अतश्चेतना बहव एव न तु लाघवादाकाशवदेकत्वमित्याद्यसूत्रार्थः । श्रुतौ च भेदवदभेदस्याप्यवगमात् तर्केणैवात्र व्यवस्थेत्याशयः कपिलाचार्याणाम् ।}

\dev{ननूपाधिभेदेन व्यवस्था कर्तव्या लाघवतर्कानुग्रहेणाभेदश्रुतेर्बलवत्त्वादित्याशङ्कां समाधत्ते—द्वितीयसूत्रेण । उपाधिभेदे सत्यपि एकस्यैवात्मनो नानोपाधियोगः स्यादतो न व्यवस्थेत्यर्थः ।}

\dev{नन्वेवं विशिष्टमतिरिक्तमेव वक्तव्यं तत्राह—तृतीयसूत्रम् उपाधिरित्यादि । उपाधिरेव भिन्नो वक्तव्यो नतूपाधिमान् उपाधिनाशे जीवनाशप्रसङ्गात्, विशिष्टाऽहं पदार्थे बुद्धिविवेकानुपपत्तेश्च ।}

\dev{उक्तदूषणमुपसंहरति चतुर्थसूत्रेण “एवमि” त्यादिना । एवमेकत्वेनावस्थितस्यात्मनो न विरुद्धधर्माध्यासः औपाधिकानोपाधिकविरूद्धधर्मसम्बन्धः संभवति, विरोधस्यासंभवात्। अविरोधे चाव्यवस्थेत्यर्थः ।}

Moreover when everything is one complete single ātman (akhaṇḍaikātmye) there is the contingency of the liberated ātman being subject to bondage; even when the one internal organ is separated (ekāntaḥkaraṇaviyoge’pi) it is possible for the liberated portion to be (connected) to another internal organ. This is similar to (what happens) to the space limited in one pot that can get connected with another pot when the earlier one breaks. Nor can it be said that limitation has no connection whatsoever with the internal organ; if it is so then there will be inconsistency with the yogīs’ capacity of travelling to all places (mentioned in the YS).\endnote{This is a strange example and only confirms Bhikṣu’s commitment to yoga and its siddhis.} There is no need to doubt that the collection of internal organs (mentioned by the YS)  is imagined by us. This has been stated by even sages like Kapila who have determined this after following prescribed logic. Thus it is said in the Kapilasūtras: “janmādivyavasthātaḥ puruṣabahutvam, upādhibhede’pyekasya nānāyoga ākāśasyeva ghaṭādibhiḥ, upādhirbhidyate na tu tadvān, evamekatvena parivartamānasya na viruddhadharmādhyāsaḥ, nādvaitaśrutivirodho jātiparatvāt”. The meaning of the (above) five sūtras are as follows: When there is identity of ātman   which is established by śruti and smṛti, birth and death of those with limitations and the experiences of those without limitations will not be possible in the form ‘this person is born, this person is dead’ (ātmaikye satyaupādhikānām janmamaraṇādīnāmanaupādhikānām ca bhogādīnām śrutismṛtisiddhā āśrayavibhāgavyavasthā na syāt “ayam jāto’yam mṛta” ityādirupā)\endnote{The inconsistence of some jīvas having experience even efter the destruction of the limitation of those who are dead.}. Therefore consciousness is many; it cannot be one; for the sake of brevity of expression (lāghavāt) the expression (one) is used like ākāśa; that is the meaning of the first sūtra. In śruti since one understands difference as well as identity (of ātman and Brahman) it needs to be established by logic according to ācārya Kapila (tarkeṇaivātra vyavasthetyāśayaḥ kapilācāryāṇām).Ques:. If it is said that one has to establish (the fact) through the difference in limitation (of the ātman) and the śruti utterances declaring identity are more powerful due to brevity of expression (in logic) then the answer is: 

\textbf{Ans:} that doubt is settled through the second sūtra. Even if there is a difference in limitation there will be many limitations for the one single ātman, so there will be no definiteness, (that is the meaning of the second sūtra).

\textbf{Ques:} In that case one can talk about something that surpasses what is qualified; 

\textbf{Ans:} so the third sūtra says “upādhiḥ…”. One needs to only say that only the limitation is different and not the one which has the limitation. 

\textbf{Ques:} then there is the inconsistency of the destruction of the jīva once the upādhi is destroyed; then in the qualified ‘aham padārtha’ (tattva/principle) of the I-sense there will be the inconsistency of difference of the intellect (from itself).

\textbf{Ans:} The fourth sūtra resolves the defect as follows: “evam…”. Thus when the ātman is established as one, there cannot be superimposition of contradictory qualities; it is not possible to have a relation with contradictory qualities such as having a limitation and not having a limitation; opposite (qualities) do not coexist. If there is no contradiction then there will be confusion (cāvyavasthetyarthaḥ).

\dev{आत्माद्वैतश्रुतिमुपपादयति पञ्चमेन “नाद्वैते” त्यादिना । जातिपरत्वात् चित्सामान्याद्वैतपरत्वात् श्रुतीनामित्यर्थः । सर्ववस्तूनां सामान्यविशेषात्मकत्वेन विशेषरूपं धर्मं परित्यज्य सामान्यरूपेण धर्मिणामात्माद्वैतं प्रतिपाद्यते, वैधर्म्याणामवास्तवत्वप्रतिपादनायेति । यथा च विशिष्टयोस्तत्त्वंपदार्थतामभ्युपगम्य परैर्लक्षणया केवलविशेष्यार्थकत्वं पदयोरुच्यते, तद्वत् सांख्यैरपि । विशेषस्त्वियान् यत्तैर्लक्षणया न क्रियते “स एवायं गकारः” इत्यादाविवेति बोध्यम् । साँख्यैरीश्वरानभ्युपगमात् सामान्याभेद एवोक्तः । ब्रह्म मीमांसायान्तु अग्निस्फुलिङ्गवदंशांश्यभेदोऽप्यविभागलक्षणो वक्ष्यते “प्रकाशाश्रयवद्वा तेजस्त्वात्, अविभागेन दृष्टत्वात्” इत्यादिसूत्रैरिति । तस्मादैकात्म्ये लाघवेऽपि बन्धमोक्षादिव्यवस्थानुपपत्त्यात्मनानात्वगौरवमाश्रयणीयमिति । अपि चौपाधिकमात्रभेदेन श्रतिस्मृती अपि नोपपद्यते । तथाहि, “निरञ्जनः परमं साम्यमुपैति” “यथाग्निरग्नौ संक्षिप्तः समानत्वमनुव्रजेत् । तथात्मसाम्यमस्येति योगिनः परमात्मना” इत्यादौ मोक्षकालेऽपि भेदघटितं साम्यं श्रूयते, तदानीं चौपाधिकभेदो नास्तीति ।}

\dev{‘‘कालः स्वभावो नियतिर्यदृच्छा   भूतानि योनिः पुरुष इति चिन्त्यम्” इत्यनेन मूलकारणे चिन्तां प्रकृत्याम्नायते—}
\begin{verse}
\dev{ते ध्यानयोगानुगता अपश्यन् देवात्मशक्तिं स्वगुणैर्निगूढाम् ।}\\
\dev{यः करणानि निखिलानि तानि कालात्मयुक्तान्यधितिष्ठत्येक: ।। इति ।}
\end{verse}
\dev{अत्र सृष्टेः प्रागपीश्वराधिष्ठेयो जीवोऽस्तीत्यवगम्यते ।  तथा,}
\begin{verse}
\dev{प्रकृतिं पुरुषं चापि प्रविश्यात्मेच्छया हरिः ।}\\
\dev{क्षोभयामास सम्प्राप्ते सर्गकाले व्ययाव्ययौ ॥}
\end{verse}
\dev{इति स्मृतेरपि ।}

\dev{न चायमुपाधिसंबन्धात् पूर्वमधिष्ठेयाधिष्ठातृभावो निरंशस्यात्मनः स्वरूपभेदं विनोपपद्यत   इति।}

By the fifth sūtra “nādvaita” etc. he explains that śruti mentioning advaita is due to its favouring a common generic property (jātiparatvāt) and being intent on non-duality based on a commonality of consciousness (citsāmānyādvaitparatvāt). Since all things have a common and particular property, giving up the particular property he points out the intrinsic non-dual nature common to all qualified things, in order to point out the false nature of contradictory qualities. Just as by accepting only the word meaning of the qualified words   “tat” and “tvam” and through the  secondary meaning of the other words one mentions only the qualified meaning of the words, so do the Sāṁkhyas as well.\endnote{Bhikṣu’s partiality towards S/Y comes up very often.} There is this difference alone: they do not take recourse to a secondary meaning but just as saying “sa evāyam gakāraḥ” they explain it as lack of understanding (aviveketi bodhyam). Since Īśvara is not accepted by the Sāṁkhyas they mention nonduality as only a common property\endnote{The common property (jāti) of consciousness is taken to be in the sense of non-duality.}. In Brahmamīmāṁsā through such sūtras as “prakāśāśrayavadvā tejastvāt” (BS. 3.2.28), “avibhāgena dṛṣṭatvāt” (BS.4.4.4) it will also be mentioned that like fire and its sparks there is non-difference like a whole and its parts of the nature of non-separation. Therefore even if there is parsimony (of logic) in  (accepting) identity, due to lack of consistency regarding the state of bondage and liberation one needs to depend on the heavier argument of many ātmans. Nor can śruti and smṛti statements fit in with just difference in limitations (api caupādhikamātrabhedena śrutismṛtī api nopapadyete). Thus in such statements as “nirañjanaḥ paramam sāmyamupaiti” (Muṇḍ. Up. 3.1.3), “yathāgniragnau saṁkṣiptaḥ samānatvamanuvrajet, tathātmasāmyamasyeti yoginaḥ paramātmanā” (not traced); even at the time of liberation one hears of similarity associated with difference; at that time there is no difference of limitation\endnote{In liberation there are no limitations and hence this cannot be explained due to difference in limitation. Bhikṣu seems to suggest that S/Y is in truth not against the Upaniṣadic philosophy of Brahman being One alone.}. Thus in the Śvet.Up. by the saying: “kālaḥ svabhāvo… iti cintyam” (Śvet.Up.1.2; Nāra.Pa. Up. 9.1) there is instruction to think about the main cause in a natural manner (pravṛttyāmnāyate seems correct). So also from “te dhyānayogānugatā apaśyan…kālātmayuktānyadhitiṣṭhatyekaḥ” (Śvet.Up.1.3) one understands that even before creation/manifestation Īśvara is the support of jīva. The same is said in the smṛti as: “prakṛtim puruṣam cāpi…vyayāvyayau”. Nor is it logical to think (reasonable to think) that before this connection with the limitation, the relationship of supporter and supported can happen without a change in the partless ātman itself. 
\begin{verse}
\dev{अन्यश्च राजन् प्रवरस्तथान्यः पञ्चविंशकः ।}\\
\dev{तच्छ्रुत्वा चानुपश्यन्ति एक एवेति साधवः” ॥}
\end{verse}
\dev{इति मोक्षधर्मादौ स्वरूपभेदमुक्त्वाऽधिष्ठेयाधिष्ठात्रोरविभागलक्षणमैक्यमुक्तम् । तथा,  “सर्वगत्वादनन्तस्य स एवायमहं स्थितः।” इति विष्णुपुराणादावपि । }

\dev{अतोऽवगम्यते— }

\dev{श्रुतयोऽप्यभेदमविभागलक्षणमेव बोधयन्ति, एकवाक्यत्वात् “ऐतदात्म्यमिदं सर्वमि” ति श्रुतौ जडवर्गाभेदस्याविभागादिरूपस्यैव वक्तव्यतया जीवाभेदस्याप्यविभागादिरूपत्वसिद्धेश्च, अन्यथाऽर्धजरतीयन्यायापत्तेः ।}

\dev{व्यतिरिक्तं न यस्यास्ति व्यतिरिक्तोऽखिलस्य यः   । इति विष्णुपुराणादिभ्योऽप्यत्यन्ताभेदोऽपि न शास्त्रार्थः ब्रह्मणि पुरुषव्यतिरेकानुपपत्तेः ।}

\dev{तथा विष्णुपुराण एव—}
\begin{verse}
\dev{परस्य ब्रह्मणो रूपं पुरुषः प्रकृतेः परः । }\\
\dev{व्यक्ताव्यक्ते तथैवान्ये रूपे कालस्तथा परः  ॥ इति ।}
\end{verse}
\dev{व्यवहारमुक्त्वा परमार्थमाह—}
\begin{verse}
\dev{प्रधानपुरुषव्यक्तकालानां परमं हि यत् ।}\\
\dev{पश्यन्ति सूरयः शुद्धं तद्विष्णोः परमं पदम् ॥}
\end{verse}
\dev{इत्यादिनापि नात्यन्ताभेदः । नन्वेवं }
\begin{verse}
\dev{तस्यैव नित्यतृप्तस्य सदानन्दमयात्मनः ।}\\
\dev{अवच्छिन्नस्य जीवस्य संसृतिः कथ्यते बुधैः” ।।}
\end{verse}
\dev{इति वसिष्ठसंहितासूक्तोऽवच्छेदोऽनुपपन्न इति चेन्न, किरणसूर्यादिवदंशांशिनोर्जीव ब्रह्मणोरेकपिण्डीभावेन तस्य जीवरूपैरंशैरवच्छेदवादस्यापि प्रतिबिम्बवादवदेवोपपत्तेः । यश्च “आकाशमेकं हि यथा घटादिषु पृथग्भवेदि” त्यादिदृष्टान्तः, सोऽप्यविभागरूपैक्यमात्रांशेन पुनरखण्डत्वेऽपि न्यायानुग्रहेण बलवत्तरस्य सखण्डतादृष्टान्तस्यैवादर्तव्यत्वात् }

Thus in the Mokṣadharma : “anyaśca rājan…eka eveti sādhavaḥ” (Mokṣa.P. 318.78, cited in Tripathi p.25.fn.1)  having mentioned the change in oneself, the relationship of identity of the nature of supporter and supported characterized by non-separation is mentioned.  This is also said in the Viṣṇu.P as “sarvagatatvādanantasya sa evāyamaham sthitaḥ” (Viṣ.P.I.19.885 cited in ibid.p.25. fn.2).   Thus one knows that even śruti instructs identity of the nature of non-separation alone, due to having consistency in meaning (ekavākyatvāt). In the śruti statement: “aitadātmyamidam sarvam” (Chānd.Up.6.8.7) since the identity of inanimate things is mentioned as of the nature of non-separation it follows that there is identity of the nature of non-separation of jīvas as well; otherwise this will suffer from the defect of the principle of being a half-widow\endnote{Well known as the ardhajarjarī nyāya i.e one cannot be a half widow. Similarly what applies to the inanimate things must equally apply to the animate world as well; otherwise it will suffer from the defect known as ‘ardhajarjarī-nyāya’}. Thus such statements in the Viṣṇu P. like: “vyatiriktam na yasyāsti…akhilasya yaḥ” ( Viṣ.P. I.19.78. cited in ibid.p.25.fn.3) also do not pronounce total identity as the meaning of the śāstra, since puruṣa being separate from Brahman it (total identity) is illogical (brahmaṇi puruṣavyatirekānupapatteḥ). Similarly in the Viṣ. P again it is said: “parasya Brahmano rūpam…kālastathā paraḥ” (ibid.1.2.15 cited in ibid. p.25.fn.4). Having spoken about the world he mentions the ultimate truth as: “pradhānapuruṣavyaktakalānām…viṣṇoḥ paramam padam; viṣṇoḥ svarūpāt…pradhāna puruṣaśca vipra” (ibid. 1.2.24; cited in ibid. p.25.fn.5); all the others such as the manifested (world) and kāla (time) are different from the true nature of Viṣṇu and so are pradhāna and puruṣa; by this also it says that there is no absolute identity. 

\textbf{Ques:} If it is said that the statement in the Vas.Saṁ: “tasyaiva nityatṛptasya…samsṛtiḥ kathyate budhaiḥ” (not traced) mentioning the worldly life of the conditioned jīva is not correct then the answer is: 

\textbf{Ans:} it is not so. Just like the sun and its rays being whole and possessed of parts, the jīva and Brahman being united as one whole, the argument of limitation in the form of the jīva parts is only as reasonable as the theory of reflection.

As for the simile: “ākāśamekam hi yathā ghaṭādiṣu pṛthagbhavet” that also denotes a unity only in the form of non-separation of the part, and can logically be supportive of the stronger simile of having parts (so’pyavibhāgarūpaikyamātrāśena punarakhaṇḍatve’pi nyāyānugraheṇa balavattarasya sakhaṇḍatādṛśṭāntasyaivādartavyatvā

\dev{यथा,}
\begin{verse}
\dev{“वायुर्यथैको भुवनं प्रविष्टो रूपं रूपं प्रतिरूपो बभूव ।}\\
\dev{एकस्तथा सर्वभूतान्तरात्मा रूपं रूपं प्रतिरूपो बहिश्च।।}
\end{verse}
\dev{“अग्नियर्थैको भुवनं प्रविष्ट” इत्यादिः सखण्डतादृष्टान्तः । वायुतदवयवाग्निस्फुलिङ्गादिष्वप्यवयवावयविनोरन्योन्याभावलक्षणो भेदः अविभागलक्षण एव चाभेद इति जीवब्रह्मणोः साम्यमिति ।}

\dev{अपि च चन्द्रजलचन्द्राकाशघटाकाशाग्निविस्फुलिङ्गच्छायातपस्त्रीपुरुषादिदृष्टान्तैः प्रतिबिम्बा परछेदांशादिवादाः परस्परविरोधेन सर्वे न सम्भवन्तीत्येक एव वाद आश्रयणीयः । इतरास्तु विवक्षिततत्तदंशमात्रे दृष्टान्ता इत्यभ्युपेयम् । तथा च सति अंशवाद एवाश्रयितुं युक्तः “अंशो नानाव्यपदेशादि” त्यादिसूत्रेणाचार्यैरंशत्वस्यैव न्यायतो मीमांस्यत्वात् । प्रतिबिम्बादिभावेनाखण्डत्वे स्पष्टसूत्राभावात् । प्रत्युत प्रत्यधिकरणं जीवब्रह्मभेदस्यैव स्पष्टं सूत्रणीयत्वात् “अधिकं तु भेदनिर्देशादि” त्यादिभिः । अंशसूत्रे च भेदाभेदयोर्वक्ष्यमाणत्वादिति । तस्माद् भेदाभेदाभ्यां जीवब्रह्मणोरशांशिभाव एव ब्रह्ममीमांसासिद्धान्तोऽवधारणीयः, “अंशो नानाव्यपदेशात्, अन्यथा चापि दाशकितवादित्वमधीयत एक” इति वक्ष्यमाणसूत्रादिति ।}

It is like the saying: “vāyuryathaiko bhuvam praviṣṭo…pratirūpo bahiśca” (Kaṭh. Up. II.2.10). The line “agniryathaiko bhuvanam praviṣṭa” (ibid. II.2.9) etc., is a simile denoting having parts.  In vāyu and its parts as also in fire and its sparks which denote parts and whole, there is only the difference in the form of mutual absence, and non-difference is non-separation alone; thus it is similar to that between the jīva and Brahman.

Those who argue for theories of reflection and and limited parts (pratibimbāvacchedāmśādivādāḥ) by such examples as moon reflected in water and moon in the sky, space limited in a pot, fire and its sparks, shadow and sunlight, woman and man, (will know) all that is not possible as they are mutually contradictory. Therefore one needs to adopt only one argument. The other arguments should be understood as only examples for those respective parts which is intended (by them). When it is so, it is appropriate to depend on the argument of parts (and whole).  By the sūtra “amśo nānāvyapadeśāt” (BS. 2.3.43) possessing parts has been argued logically by the ācāryas (aṁśatvasyaiva nyāyato mīmāmsyatvāt). There is absence of any definite sūtra stating unity through being a reflection etc. On the other hand every adhikaraṇa (section) has sūtras clearly stating the difference between jīva and Brahman through such sūtras as “adhikam tu bhedanirdeśāt” (2.1.22). And in the sūtra “amśa…” (BS. 2.3.43) the difference/non-difference relationship will be  mentioned. Therefore one should conclude from the upcoming sūtra “amśo nānāvypadeśāt…eke”(BS.2.3.43) that the siddhānta (conclusion) of Vedānta (Brahmamīmāmsā) is that jīva and Brahman which are different have a relationship of part and whole.

\dev{अंशांशिनोश्च भेदाभेदौ विभागाविभागरूपौ कालभेदेनाविरुद्धौ । अन्योन्याभावश्च जीवब्रह्मणोरात्यन्तिक एव, तथा शक्तिशक्तिमदविभागोऽपि नित्य एवेति मन्तव्यम् । तथा च स्मृति—“पृथग् विभक्ता प्रलये च गोप्ते” ति, ‘तदात्मकं तदन्यत् स्याद्यद्रूपं भासकं विदुरि’ ति चेति । तदात्मकं पूर्ववाक्योक्तप्रधानपुरुषकालात्मकमित्यर्थः । अंशत्वं च सजातीयत्वे सति अविभागप्रतियोगित्वम् तदनुयोगित्वं चांशित्वम् । येन च रूपेणांशता यत्र विवक्ष्यते तेनैव रूपेण साजात्यं तत्र ग्राह्यं, यथा आत्मांशलक्षणे आत्मत्वेनैव साजात्यं सदंशादिलक्षणेषु च सत्त्वादिरूपेणैवेत्यतो नातिप्रसङ्गः । अथवा द्रव्यत्वसाक्षाद्व्याप्यजात्या साजात्यं ग्राह्यम् । विभागश्च लक्षणान्यत्वम् अभिव्यक्तधर्मभेद इति यावत् । तद्भावश्चाविभागः । अथवास्तु अविभागः संयोगविशेषः स्वरूपसम्बन्धो वा आधेयत्वादिवत्। जलस्य दध्नि लवणस्य समुद्रे अविभागव्यवहारस्यापलपितुमशक्यत्वात् । “न तु तद्वितीयमस्ति ततोऽन्यद् विभक्तमि”ति श्रुत्या}

\dev{“सति सम्पद्यते  न विदुः सति सम्पद्यामह” इत्यादिश्रुत्या च जीवस्यापि ब्रह्मण्यविभागश्रवणात् । “या प्रकृतिः पुरुषश्चोभौ लीयेते परमात्मनी” त्यादिस्मृतेश्च । “पदगतौ, लिङ् श्लेषण” इत्याद्यनुशासनेभ्यःसम्पत्तिलयाप्ययादिशब्दानामविभागार्थत्वात्। तथा “अविभागो वा वचनादि” त्यागामिसूत्राच्चेति । }

\dev{ननु निरवयवस्य ब्रह्मणः कथं मुख्योंऽशः स्यादिति चेन्न, यथोक्तलक्षणांशत्वस्यावयवत्वाभावेऽपि दर्शनात् । यथा शरीरस्य केशादिरंशो, राशेश्चैकदेशोंऽशः, पितुश्च पुत्र इति । सर्वे च जीवाः पितरि पुत्रचेतना इव चिन्मात्रे ब्रह्मणि नित्यसर्वावभासके विषयभासनरूपं स्वलक्षणं विहाय प्रलये लक्षणानन्यत्वं गच्छन्ति । सर्गकाले च तदिच्छया तत एव लब्धचैतन्यफलोपधाना आविर्भवन्ति पितुरिव पुत्राः । अतो जीवा ब्रह्मांशा भवन्ति । “आत्मा वै जायते पुत्रः” इति श्रुत्या पुत्रे पितुरविभागलक्षणाभेदवज्जीवेऽपि ब्रह्मणोऽविभागलक्षणाभेदस्य “बहुस्यां प्रजायेये” त्यादिश्रुत्या सिद्धेरिति । अतो जीवा ब्रह्मांशा मुख्या एव भवन्ति ।}

The difference and non-difference relationship of part and whole is of the nature of separation and non-separation and is non-contradictory due to change of time. Mutual negation between the jīva and Brahman is absolute; similarly one should know that non-separation between power and the one possessing power is also eternal. Thus there is the smṛti: “pṛthag vibhaktā pralaye ca goptā” (not traced), as also: “tadātmakam tadanyat…bhāsakam viduḥ” (not traced). The word “tadātmakam” (in the above quotation) refers to the nature of definite time of pradhāna and puruṣa. Having parts when belonging to the same generic property is the counter-part of non-separation, and its adjunct is being its whole (tadanuyogitvam cāmśitvam). In which ever form one desires to speak of being a part, one needs to understand its having the same generic property in that form itself; in the  characteristic mark of the ātman, the part has the common generic property of the nature of ātmatva (being ātman), then in the characteristics of the parts also it will be in the form of sattva etc., and thus there is no contradiction in that.\endnote{In other words the amśas also will have the same characteristics of sattva etc., as the amśī (whole).} Or one can understand the common generic quality as that pervaded directly by the characteristic of dravyatva. Vibhāga (separation) is a different characteristic and is another manifested quality. Having that nature is non-separation. Or let avibhāga (non-separation) be a special kind of contact or an intrinsic relationship (svarūpasambadho vā) like that which has the action of a support (ādheyatvādivat). It is not possible to deny the phenomenon of non-separation of water in milk, and salt in the ocean. Through the śruti statement:  “na tu…vibhaktam” (Bṛ.Up. 4.3.24) and by the śruti saying: “sati sampadya…sampadyāmaha” one hears of non-separation of the jīva in Brahman. This is also stated in such smṛti statements as: “yāprakṛtiḥ…paramātmani” (not traced).   Through such (grammatical) rules as: “padagatau, liṅ śleṣaṇe” avibhāga has the meanings of words like stability, absorption and increase. So also the later sūtra “avibhāgo vā vacanāt” (support avibhāga) (BS.4.2.16).

\textbf{Ques:} If it is said how can the partless Brahman have important parts then the answer is: 

\textbf{Ans:} It is not so. The said characteristic of having parts is felt even in the absence of having parts. (It is like) hair etc., being  a part of the body, like being a part of a whole collection and like the son being part of the father. Like the cetanā of the son in the father, all jīvas giving up one’s own characteristic of illuminating objects during pralaya get absorbed totally without any characteristic in Brahman of the nature of pure consciousness, which eternally is the illuminator of all (objects).\endnote{The use of cetanā suggests that the son here is still very young and cannot take decisions on his own.} And during the time of evolution through his (Brahman’s) desire, having obtained therein the excellent result of consciousness (jīvas) come into being, like sons of the father. Therefore jīvas are parts of Brahman. By the śruti statement: “ātmā vai jāyate putraḥ”,  like the identity of the nature of non-separation of the father in the son, in the jīva also the identity of the nature of non-separation of Brahman is established by the śruti: “bahusyām prajāyeya” (Chānd.Up.6.2.3; Taitt.Up. 2.6). Therefore jīvas are mainly parts of Brahman. \dev{नन्वंशस्तत्र स्यात्, आकाशवदंशांशिभावो नार्थः (किन्तु अग्निविस्फुलिङ्ग पितापुत्रादिवदेवांशांशिभावोनार्थः किं तु अग्निविस्फुलिङ्गपितापुत्रादिवदेवांशांशिभावोऽभिप्रेत इति कथं निर्धारणीयमिति चेत्, उच्यते—आकाशवदंशांशिभावाश्रयणेंऽशशब्दस्य गौणत्वं स्यात् । घटाकाशो ह्याकाशाद् विभक्तो न भवति, लक्षणान्यत्वाभावात् घटाकाशधर्माणामप्याकाशधर्मत्वात् अवच्छिन्ने चांशशब्दो गौणोऽवयवादिशब्दवदिति । किं च एवमपि जीवश्चिन्मात्रस्यैवांशः स्यान्न तु सर्वकर्तुरीश्वरस्य, परैरुपाधिविशिष्टस्यैवेश्वरत्वकल्पनात् । न चेश्वरोपाधेरंशो जीवोपाधिरीश्वरस्यापि तद् द्वारा भवन्मते संसारप्रसंगात् कायव्यूहादाविवेति । अपि च पितापुत्रवदंशांशिभावो व्यासाभिप्रेत इति मोक्ष धर्मस्थाच्छिष्यवैशम्पायनवाक्यादेवावधार्यते । अतो नैवाकाशवदंशत्वं ब्रह्ममीमांसार्थः यथा, मोक्षधर्मे—}

\dev{ “बहवः पुरुषा ब्रह्मन्नुताहो एक एव तु” इति प्रश्ने—}
\begin{verse}
\dev{बहवः पुरुषा राजन् सांख्ययोगविचारिणाम् ।}\\
\dev{नैवमिच्छन्ति पुरुषमेकं कुरुकुलोद्वह ।।}
\end{verse}
\dev{इत्यनेन पुरुषनानात्वमेव विचारतो व्यवस्थाप्य व्यासोकं पुरुषैक्यं पितापत्रवदविभागेनोपपादितम्—}
\begin{verse}
\dev{समासतस्तु यद्व्यासः पुरुषैकत्वमुक्तवान् ।}\\
\dev{तत्राहं संप्रवक्ष्यामि प्रसादादमितौजसः ।।}\\
\dev{बहूनां पुरुषाणां हि यथैका योनिरिष्यते ।}\\
\dev{तथा  तम्पुरुषं विश्वमाख्यास्यामि गुणाधिकम् ॥ इति ।}
\end{verse}
\dev{अस्यार्थः—यत्तु व्यासः समासतो जीवजातस्य ब्रह्मणि प्रक्षेपत आत्मैक्यमुक्तवानर्थात् ब्रह्ममीमांसादौ तत्तुभ्यं जनमेजयाय तच्छिष्यवैशम्पायन आह, तदुपदेशप्रसादाद्वक्ष्यामि । तदेवाह ‘बहूनामिति श्लोकेनेति । उपसंहारे च “एवं बहुविधः प्रोक्तः पुरुषस्ते यथाक्रममि” त्यनेन पुरुषबहुत्वमेव सिद्धान्तितम् ।}

\dev{यत्तु तत्रैव प्रसंगान्तरे वाक्यान्तरम्- }
\begin{verse}
\dev{बहवः पुरुषाः पुत्र यत्त्वया समुदाहृताः ।}\\
\dev{एवमेतर्हि संवृत्तं द्रष्टव्यं नैवमित्यपि ।। इति,}
\end{verse}
\dev{तेन च पुरुषविभागस्य विकारवत् वाचारम्भणत्वमभिप्रेत्य जीवात्मताप्रतिषेधतः एकस्यैव ब्रह्मणः पारमार्थिकात्मत्वमुक्तं, न तु जीवात्मत्वं स्थापयित्वा पुरुषबहुत्वं निराकृतम् । तस्मात् पितापुत्रवदंशत्वमेवागाम्यंशसूत्रस्यार्थः । तथा च स्पष्टे श्रतिस्मृती अंशत्व एव भवतः ।}
\begin{verse}
\dev{“मायां तु प्रकृतिं विद्यान्मायिनं तु महेश्वरम् ।}\\
\dev{अस्यावयवभूतैस्तु व्याप्तं सर्वमिदं जगत् ।।”}
\end{verse}
\dev{तत्सां ( तस्यां ) शोऽयं यश्चैतामात्राः प्रतिपुरुषं क्षेत्रज्ञाः”}
\begin{verse}
\dev{“यथा सुदीप्तात् पावकाद् विस्फुलिङ्काः सहस्रशः प्रभवन्ते सरूपाः}\\
\dev{तथाक्षराद् विविधाः सौम्य भावाः प्रजायन्ते तत्र चैवापियन्ति ।।}
\end{verse}
\dev{इत्यादिश्रुतिः । स्मृतिश्च—}
\begin{verse}
\dev{“ममैवांशो जीवलोके जीवभूतः सनातनः” इत्यादिरिति ।}
\end{verse}
\textbf{Ques:}  But then let there be parts, but (you say that) the meaning cannot be that it is like part and whole similar to space. However how can it be determined that the part and whole is like fire and its sparks or like that of father and son? Then the answer is: 

\textbf{Ans:} when one depends on the meaning of part and whole as like that of space, then the meaning of the part becomes secondary. The pot-space is not divided from space (as such) since it does not possess another (different) quality (lakṣaṇānyatvābhāvāt); the characteristics of the pot-space also have the same characteristic as space (itself). When the word denoting a part is limited, it has a secondary meaning like the words denoting parts. Moreover in this state the jīva becomes a part of consciousness alone and not of Īśvara who is the creator of all; others imagine īśvaratva (state of being īśvara) as that qualified by a limitation. The limitation of jīva is also not a part of the limitation of Īśvara, as through that, in your view, there will be the contingency of Īśvara having (connection with) samsāra  through (having) an assemblage of body etc. Moreover the relationship of part and whole, like that between father and son, is desired by Vyāsa which is known from the words of his disciple Vaiśampāyana in the Mokṣadharma section (of the MBh). Therefore the meaning of being a part with reference to the jīva in the BS, is not like being a part like that of space (ākāsa) limited by pot.Thus in answer to the question “bahavaḥ puruṣā brahmannutāho eka eva tu” it is said “bahavaḥ puruṣā rājan sāmkhyayogavicāriṇām naivamicchanti puruṣamekam kurukulodvaha”, thus establishing, after reflection, the existence of many puruṣas; it is (also) shown by Vyāsa that the identity of puruṣas is a non-separation (relationship) similar to that of father and son as “samāsatastu yadvyāsaḥ puruṣaikatvamuktavān…guṇādhikam”.

The meaning of this is: “that which has been mentioned briefly by Vyāsa projecting the identity of the collection of jīvas in Brahman in such works as Brahmamīmāṁsā (Vedāntasūtras) that has been conveyed to you, Janamejaya, by his disciple Vaiśampāyana, (and) I shall tell you that through the grace of his teaching” (tadupadeśaprasādādvakṣyāmi). That same (idea) is mentioned through the verse (given above) “bahūnām” etc. In conclusion the existence of many puruṣas is established through the verse: “evam bahuvidhaḥ proktaḥ puruṣaste yathākramam”. In that same context there is the statement at the end of the discussion (prasaṅgāntare): “bahavaḥ puruṣāḥ…naivamityapi”; desiring through that (verse) the separation of puruṣas to be (only) a matter of speech (and) rejecting the jīvas’ ātman-nature (jīvatmatāpratiṣedhataḥ) the absolute ātman-nature of the single Brahman is mentioned; nor has the existence of many puruṣas been abandoned after having established the ātman-nature of the jīvas.\endnote{In other words there is a hierarchy accepted. While the jīvas share the ātman-nature with paramātman they are not the ultimate absolute ātman; similarly, having accepted the ātman-nature of the jīvas does not preclude from accepting the existence of many puruṣas as well. Brahman is the only supreme singular Ātman. The use of ātman for jīvas is only in a secondary sense.} Therefore the meaning of the upcoming sūtra “aṁśo” etc., (BS. 2.3.43) is that the relationship between Brahman and the jīvas is like that between a father and son. Thus the śruti and smṛti statements like: “māyām tu prakṛtim vidyāt…vyāptam sarvamidam jagat” (Śvet.Up 4.10), “tasyāmśo’yam…kṣetrajñānāḥ”, “yathā sudīptāt…prajāyante atra caivāpiyanti” (Muṇḍ.Up.2.1.1), and the smṛti: “mamaivāṁśo jīvaloke jīvabhūtaḥ sanātanaḥ”are clearly about (jīvas) being a part (of Brahman). \dev{यद्यपि जीवा अपि ब्रह्मवदेव  विभुचिन्मात्ररूपास्तथाप्युपाध्यवच्छेदेनैवाभिव्यक्तपरिच्छिश्नचैतन्यतया विस्फुलिङ्गतुल्या भवन्ति “बुद्धेर्गुणेनात्मगुणेन चैव ह्याराग्रमात्रो ह्यवरोऽपि दृष्टः” “बालाग्रशतभागस्य शतधा कल्पितस्य च । भागो जीवः स विज्ञेयः स चानन्त्याय कल्पत” इत्यादिश्रुतेरिति ।}

\dev{अस्य चांशस्य भेद ( अभेद ) प्रतिपादनस्य फलम्, अंशांशिनोरुत्सर्गतः एकरूपतया जीवस्याप्यसंसारित्वविभुत्वसर्वाधारत्वादिज्ञापनम्, ब्रह्मणश्चोपाधिविवेकेन चिन्मात्रत्वज्ञापनं, तथा जीवेंऽशेऽभेददृष्ट्या ब्रह्मोपासनं तथा ब्रह्मात्मत्वोपपादनं चेत्यादि । एतेषु व्रात्मत्वोपपादनं मुख्यं फलं ब्रहात्मतापरत्वात् सर्ववेदान्तानाम् । अत एवाचार्यो वक्ष्यति—“आत्मेति तूपयन्ति ग्राह्यन्ति च” इति । ब्रह्मात्मज्ञानादेवौपाधिकसंसारवज्जीवमारभ्य स्थूलदेहपर्यन्तेष्वभिमाननिवृत्तेः । अत इदं ब्रह्मात्मज्ञानं विविक्त- जीवज्ञानातू सांख्योक्तादपि श्रेष्ठं, नातोऽधिकं ज्ञानमस्ति ।}

\dev{नन्वंशांशिभावज्ञानात् कथं ब्रह्मात्मज्ञानं स्यात्? उच्यते—यो यत आगत्य यदधिष्ठितं यत्र जीवित्वा यत्र लीयते समुद्रतरंगादिवत् जीवात् तद्बुद्ध्यादिवच्च स तस्यामा भवति, }
\begin{verse}
\dev{यच्चाप्नोति यदादत्ते यच्चात्ति विषयानिह ।}\\
\dev{यच्चास्य सन्ततोभावस्तस्मादात्मेति कथ्यते ।।}
\end{verse}
\dev{इति आत्मलक्षणस्मृतेः । यथा हि देहेन्द्रियादीनां बुद्धिपर्यन्तानामुत्पत्तिलयाधारतया तत्साक्षित्वेनाधिष्ठातृत्वादिना च जीवस्तेषामात्मा तत्स्वरूपज्ञानाच्च तेष्वहमित्यभिमानो निवर्तते “नाहं बुद्ध्यादिरि” ति विद्ययेति सांख्यसिद्धान्तः । तथैव जीवानां चिन्मात्रस्वरूपाणामप्युत्पत्तिलयाधारतया तत्साक्षित्वेन तदधिष्ठातृत्वादिना चेश्वरस्तेषामप्यात्मा तत्स्वरूपज्ञानादेव च जीवेष्वहमित्यभिमानो निवर्तते ब्रह्मात्मविद्ययेति ब्रह्ममीमांसासिद्धान्तः । “तमेवैकं जानथ आत्मानं, नान्योऽतोऽस्ति द्रष्टा नान्योऽतोऽस्ति श्रोता, ¹यदेव साक्षादपरोक्षाद् ब्रह्म, य आत्मा सर्वान्तरः, य आत्मनि तिष्ठन्नात्मानमन्तरो यमयति, यस्यात्मा शरीरम्, नित्यो नित्यानां चेतनश्चेतनानामि” त्यादिश्रुतेः। }
\begin{verse}
\dev{तत्त्वैः सम्पादितं भुङ्क्ते पुरुषः पञ्चविंशकः ।}\\
\dev{ईश्वरेच्छावशात् सोऽपि जडात्मा कथ्यते बुधैः ।।}\\
\dev{तवान्तरात्मा मम च ये चान्ये देहसंज्ञिताः ।}\\
\dev{सर्वेषां साक्षिभूतोऽसौ न ग्राह्यः केनचित् क्वचित् ।।}\\
\dev{अन्यश्च राजन् प्रवरस्तथान्यः पञ्चविंशकः ।}\\
\dev{तत्स्थत्वाच्चानुपश्यन्ति एक एवेति साधवः ।।}\\
\dev{ते चैनं नाभिनन्दन्ति पञ्चविंशकमप्युत ।}\\
\dev{षड्विंशमनुपश्यन्तः शुचयस्तत्परायणाः ।।}\\
\dev{तमो रजश्च सत्त्वञ्च विद्धि बुद्धिगुणानिमान् ।}\\
\dev{बुद्धिमात्मगुणं विद्यादात्मानं परमात्मनः ।।}
\end{verse}
\dev{इत्यादिस्मृतेश्च । “आत्मेति तूपयन्ती” त्यादिसूत्राच्च,  तथा “आत्मा भोक्तुरि” ति पूर्वाचार्यवाक्याच्चेति ।।}

Even though the jīvas are all pervading (and) of the nature of pure consciousness still, since it is a limited consciousness manifested through the delimitation by conditions they are like sparks (of fire). Thus śruti says: “buddherguṇenātmaguṇena…dṛṣṭaḥ” (Śvet.Up. 5.8), “bālāgraśatabhāgasya…kalpate” (not traced). The result of pointing out the difference (or identity) between the part and the whole is in order to inform about the non-worldly, all-pervasive, all-supportive nature of the jīva as well, and to inform that because of the difference in limitation, Brahman is of the nature of consciousness alone (brahmaṇaścopādhivivekeṇa cinmātratvajñāpanam). Thus in the part that is the jīva, there is devotion to Brahmankeeping the focus on identity as also establishing the āmatva of Brahaman (tathā jīvem’śe’bhedadṛṣṭyā brahmopāsanam tathā brahmātmatvopapādanam cetyādi). In these statements the  establishment of Brahman as having ātmatva (nature of ātman) is the main purpose, since all the Vedānta systems are in favour of Brahman being ātman.Thus Śaṅkarācārya  will mention: “ātmeti tūpayanti grāhayanti ca”.\endnote{Available readings have “ātmeti tūpagacchanti grāhayanti ca” ( BS. 4.1.3)}   Starting a jīva’s life in saṁsāra (and) possessing (one’s) limitation throughout one’s gross bodies, till the end of the sense of agency, there is an end to it only through the knowledge of Brahman as being ātman (brahmātmajñānādevaupādhikasamsāravajjīvamārabhya sthūladehaparyanteṣvabhimānanivṛtteḥ). Thus this knowledge of Brahman being ātman is better than the knowledge of jīva’s being different (from prakṛti) which is mentioned by Sāmkhya as there is no more (important) knowledge than that (to be achieved).\endnote{Bhikṣu is willing to concede that Sāṁkhya’s apavarga/mokṣa is inferior to that of Vedānta knowledge. This is inspite of his maintenance of the cosmogony of Sāṁkhya and the Yoga sādhana for attaining mokṣa.}  

\textbf{Ques:} If it is said how can the knowledge of Brahman being ātman be got by knowledge of (jīvas being) part of the complete (Brahman), then the answer is:

\textbf{Ans:} that which comes from some source on which it is supported, (and) having lived therein gets absorbed there itself just like the waves of an ocean, is like the intellect etc., that comes from jīva,. Thus the Ātmalakṣaṇasmṛti states: “yaccāpnoti yadādatte…tasmādātmeti kathyate”(not traced).

Just as the jīva is the ātman of the body, the sense organs etc., right upto the intellect etc., due to its being the support for their rise and absorption (therein), and by being the witness (of all that happens) and by being their support and when it realizes its true nature due to (correct) knowledge vidyayā), the sense of identity in them is destroyed in the form “nāham buddhyādiḥ” (I am not the intellect etc.). This is the Sāṁkhya doctrine. In a similar manner by being the support for the rise and absorption of the jīvas of the nature of pure consciousness, by being the witness (of everything), by being their supporter etc., Īśvara is also their ātman and only by realizing its true nature will the sense of agency as ‘I’ in the jīvas disappear through the knowledge of Brahman as ātman; this is the doctrine of Brahmamīmāṁsā (Vedānta).

Thus the śruti statements say: “tamevaikam jānatha ātmānam” (Muṇḍ.Up.2.2.5), “nānyato’sti drāṣṭā…śrotā” (Bṛ.Up.3.7.23), “yadeva sākṣādaparokṣāt” (Bṛ.Up 3.4.2), “ya ātmā sarvāntaraḥ” (Bṛ.Up. 3.4.1)”,”ya ātmani tiṣṭhan…gamayati”, “yasyatmā śarīram”, “nityo nityānām cetanaścetanānām” (Śvet.Up.6.13). Smṛti statements like “tatvaiḥ sampāditam bhuṅkte puruṣaḥ pañcavimśakaḥ…buddhimātmaguṇam vidyādātmānam paramātmanaḥ” also support this. Sūtras such as “ātmeti tūpayanti” (BS.4.1.3) and the ācārya’s statement “ātmā sa bhoktuḥ” support this. 

\dev{आत्मनानात्वं च चैतन्ययोग्यतामात्रेण शास्त्रेषुक्तम् । बहिर्जीवस्य चित्शक्तिमत्त्वरूपमात्मत्वं नाभ्युपगम्यते “आदरादलोप” इति सूत्रविरोधात् सुखदुःखभोगानुपपत्तेश्च। किन्तु जीवेषु चैतन्यफलोपधानं कादाचिकतया वाचारम्भणमात्रमीश्वरपरतन्त्रं सदल्पं चेति। जीवाश्चिच्छक्तिगुणयोगाद् गौणात्मान एव, यथाऽध्यक्षत्वगुणयोगेन प्राणः करणानामात्मा, तद्वत्। मुख्यस्त्वात्मेश्वर एव, सदा सर्वज्ञत्वापरतन्त्रत्वाच्चेत्येवाभ्युपगम्यते “नान्योऽतोऽस्ति द्रष्टे” त्यादिश्रुतेरिति । तथा च नारदीये गौणमुख्यभेदेनात्मद्वयमुक्तम्—}
\begin{verse}
\dev{आत्मानं द्विविधं प्राहुः परापरविभेदतः ।}\\
\dev{परस्तु निर्गुणः प्रोक्तो ह्यहङ्कारयुतोऽपरः ।। इति ।}
\end{verse}
\dev{परापरौ श्रेष्ठाश्रेष्ठौ मुख्यगौणत्वाभ्यामिति भावः ।}

\dev{इदमेव खण्डैकात्म्यमबुद्ध्वा आधुनिका अखण्डैकात्म्युपपत्तये जीवानां प्रति बिम्बावच्छेदादिरूपैः कुकल्पनां कुर्वन्ति । सांख्यनैयायिकादिभिश्चैतन्यफलयोग्यता रूपमात्मत्वं जीवस्योच्यत् इति न तदुविरोधः । यथा   चातेक्षिकात्मनो जीवस्य तत्वज्ञानादप्यहंकतॆत्याराभिमाननिवृत्त्या मोक्षोभवति तथा विद्यपादे वक्ष्यामः । यदि वा शेषशेषिरूपेणात्मद्वयमेवाभ्युपगम्यते तदा ब्रह्मात्मताज्ञानस्याविद्यानिवृत्तिहेतुत्वं न साक्षादभ्यु पगन्तुं शक्यते ब्रह्मात्मताज्ञानानिवर्तकत्ववद् देहाद्यात्मताज्ञानानवर्तकत्वस्याप्यौचित्यात् । अपि त्वदृष्टेश्वरानुग्रहादिद्वारैवाविद्यानिवृत्तौ मोक्षे च हेतुत्वमभ्युपगन्तव्यम् । तथा च सति —}
\begin{verse}
\dev{परमात्मा हरिः स्वामी दासोऽहं तस्य सर्वदा ।}\\
\dev{स्वेच्छया विनियोक्तातस्तस्यैवात्मेश्वरस्य हि ।।}\\
\dev{अहंकृतिर्मकारः स्यान्नकारस्तन्निषेधकः ।}\\
\dev{तस्मात्तन्तमसैवातश्चाङ्कारविमोचनम् ।।}
\end{verse}
\dev{इति यमपुराणयुक्त्या ईश्वरात्मज्ञानात् साक्षात् सङ्गाताभिमाननिवृत्तिर्जीवात्मत्वाभावश्च न घटेतेति । एतेन “एष त आत्मान्तर्याम्यमृतः स म आत्मेति विद्यात्, तत्त्वमसी” त्यादिश्रुतयो व्याख्याताः जीवभेदेऽपीति मन्तव्यम् ।}

The statement of many ātmans/puruṣas in the śāstras is only due to the capacity of consciousness; the ātmatva of the external power of  jīva, due to having the nature of consciousness is not appropriate, as it is in contradiction to the sūtra “ādarādalopa”(BS.3.3.40) and it is also incompatible with the experience of pleasure and pain.\endnote{The meaning of the sūtra is: “There can be no omission (of the performance of the agnihotra to Prāṇa) on account of the respect shown (in the Upaniṣad)” (trans.Swami Gambhirananda). Similarly the use of the word ‘ātman’ for jīva/puruṣa is not appropriate but is used out of respect for the term used in the Upaniṣads.} However, the bestowing of the result of consciousness being intermittent is only meant for worldly purposes, and being dependent on Īśvara it is meagre (kintu jīveṣu caitanyaphalopādhānam kādācitkadayā vacāraṁbhaṇamātram īśvaraparatantram sadalpam ceti). Jīvas by association with the quality of the power of consciousness are only subsidiary ātmans; this is like prāṇa being the ātman of the senses by having the quality of being the supervisor. The main ātman is Īśvara alone, as he is at all times omniscient and independent; this is in accordance with śruti statements such as “nānyato’sti draṣṭā” etc (Bṛ.Up.3.7.23). So also in the Nārada P. two ātmans have been mentioned by dividing them as subsidiary and primary, due to being superior and inferior, excellent and not excellent as well as being primary and subsidary: “ātmānam dvidham…aparaḥ”(Nār.P. 31.57 cited in Tripathi p.30. fn.1). Not knowing this partial identity, modern Vedāntins argue wrongly for a single ātman in the form of the theory of reflection and limitation. The Sāmkhyas and Naiyāyikas mention the ātman nature of the jīva in the form of having the capacity of the result of consciousness which is not contradictory. We shall mention in the section on vidyā how jīva of the nature of seeing/knowing, through giving up the sense of agency such as ‘I am the doer’ etc., attains liberation. If one accepts the two fold ātman in the form of śeṣa (servant/subsidiary) and śeṣī (master/primary) then knowledge of Brahman as ātman acquired by removal of ignorance is not possible directly (tadā brahmātmatājñānasyāvidyānivṛttihetutvam na sākṣādbhyupagantum śakyate); just as the non-removal of the knowledge of Brahman as ātman, it is proper that there is non-removal of knowledge of the body being ātman. However, only when, through the blessings of the unseen Īsvara, ignorance is removed can one understand its cause for the achievement of mokṣa.\endnote{} 

Following the statement in the Yama Purāṇa : “paramātmā hariḥ svāmī dāso’ham tasya sarvadā…tasmāttantamasaivātaścāhaṅkāravimocanam” (see also Tripathi.p.30 fn.3) the statement that, through the knowledge of Īśvara as ātman, there is the direct removal of the collective sense of ego/agency and the absence of the ātmatva of the jīva does not make sense. The śruti statements like “eṣa ta ātmāntaryāmyamṛtaḥ” (Bṛ.Up 3.7.3; 4.2-3), “tattvamasi” (Chānd.Up 6.8.7 and in many other places) one should know that they have also been explained in terms of difference in jīvas.

\dev{ननु “तत्त्वमसि” वाक्यस्यांशांश्यविभागेनाप्युपपत्तेः कथमस्य ब्रह्मतापरत्वमवधारणीयमिति चेत्, अविद्यानिवर्तकतयाऽभ्यर्हितत्वेन बाधकाभावे सर्वत्रैवयाभेदवाक्येषु ।}

\dev{ब्रह्मात्मतापरत्वस्यौत्सर्गिकत्वात् “तं त्वौपनिषदं पुरुषं पृच्छामी” तिश्रुतेः, वेदा ब्रह्मात्मविषया इति स्मृतेश्च । अंशांश्यभेदस्यापि ब्रह्मात्मतावगतिफलत्वात् । न च “ऐतदात्म्यमिदम् ” सर्वं स आत्मे” ति पूर्वभागेनैव ब्रह्मात्मत्वं लब्धमिति वाच्यम्, यत “ऐतदात्म्यमि” त्यनेन प्रपञ्चरस्य ब्रह्मणि स्वरूपत्वलक्षणमेवात्मत्वमविभागलक्षणाभेदो वा घटस्य मृदात्मत्ववत्, न त्वध्यक्षत्वरूपं चेतनत्वरूपं वा, “स आत्मे” त्यनेन वा साक्षित्वरूपमात्मत्वमुक्तम् । ताभ्यां च हेतुभ्यां “तत्त्वमसी” त्यनेन चाध्यक्षत्वरूपमात्मत्वमिति विभागः। अध्यक्षत्वरूपात्मत्वस्यैव त्वमहंशब्दार्थत्वात्। अहंममेति शब्दाभ्यां स्वस्वामिभावावगमात् । अत एव पातञ्चले “स्वस्वामिशक्त्योः स्वरूपोपलब्धिहेतुः संयोगः” इति स्वस्वामिशब्दाभ्यां बुद्ध्यात्मानावुक्ताविति । अतएव च “अथात आत्मादेशोऽथातोऽहङ्कारादेश” इति श्रुत्योर्न पौनरुक्त्यम्, अर्थभेदात् । अत एव च “नास्मि न मे नाहमित्यपरिशेषम्” इति सांख्यकारिकायामपि न पौनरुक्त्यं च, नास्मीत्यनेन साक्षित्वस्य नाहमित्यनेन स्वाम्यस्य प्रतिषेधादिति चेन्न एवं च कार्यकारणसंघाताध्यक्षो भवतीति कृत्वा स एवाहमित्युच्यते, यथा प्राणा इन्द्रियाध्यक्षतयैव तेषामात्मोच्यते । न चेश्वरस्य सम्बोध्यत्वप्रयोक्तृत्वाभवात् कथं त्वमहंशब्दार्थता स्यादिति वाच्यम्, वागिन्द्रियद्वारा जीवस्य प्रयोक्तृत्ववज्जीवाख्यकरणद्वारा ब्रह्मण एव संबोध्यत्वादिसकलव्यवहारप्रतिपादनायैव तत्त्वमसीत्युपदेशादिति । तथा च श्रुत्यन्तरम्— “नान्योऽतोऽस्ति द्रष्टा श्रोता मन्ता बोद्धे” त्यादिना परमात्मन एव दर्शनश्रवणादिसर्वव्यवहारकर्तृत्वमाह ।}

\textbf{Ques:} If the sentence “tattvamasi” is explained as the non-separation of the parts and whole how can one understand the non-separate nature of Brahman? Then the answer is: 

\textbf{Ans:} By the removal of ignorance. Since that is the most suited (abhyarhitatvena) in all statements which lean towards identity of Brahman being ātman in the absence of obstacles (bādhakābhāve sarvatraivābhedavākyeṣu brahmātmatāparatvasyautsargikatvāt). Thus the śruti statement “tam tvaupaniṣadam puruṣam pṛcchāmi” (Bṛ.Up.3.9.26) supports this. Smṛti also says that the Vedas deal with the subject of Brahman and ātman. The purpose of the non-difference between the parts and the whole is also to understand the ātman nature of Brahman. Nor can it be said that through “aitadātmyamidam sarvam sa ātmā…” (Chānd.Up 6.8.7 and in many other places in Chānd.Up) the ātman nature of Brahman is understood from the initial statement itself, since by saying “āitadātmyam” it can mean only that the intrinsic nature of the world in Brahman has ātmatva, or is of the nature of identity in the sense of non-separation like the pot possessing the essence of clay, and not of the nature of superintendence or of the nature of consciousness; nor by the statement “sa ātmā” is there a reference to ātmatva in the sense of being a witness. It is because of those reasons that by the statement “tattvamasi” there is a division of its having ātmatva in the sense of superintendence. The meaning of the words “tvam” and “aham” denotes ātmatva in the sense of having the power of superintendence. The meaning of the word “aham” denotes being ātman having the nature of superintendence. One understands the words “aham”, “mama” to have the sense of ‘being the possessed’ and ‘being the possessor’. That is why in Patañjali’s YS “svasvāmiśaktyoḥ…samyogaḥ” (YS.II.23) the intellect and the ātman have been denoted by the words “sva” and “svāmī”. That is why one does not have the repetition of the śruti in the form “athāta ātmādeśo’thāto’haṅkārādeśaḥ” (Chānd.Up. 7.25.2) due to the difference in meaning.\endnote{Bhikṣu seems to suggest that if the little self (ahaṁkāra) was intended to be the purport then the ādeśa should also repeat the ādeśa for ahamkāra (little self) as well; but it does not do so.} That is why again in the Sāṁkhyakārikā (SK) “nāsmi…apariśeṣam” (not traced) is not repeated; since by the phrase “nāsmi” the state of being a witness, and by the phrase “nāham” the state of being a master (svāmyasya) is denied.\endnote{The words ‘cenna’ seems to be out of place according to Tripathi. I also found that it does not connect so I have ignored it.} In this manner by being the superintendent of the collection of cause and effect it is said ‘I am he’. This is like saying that prāṇa by being in charge of the sense-organs is their self (teṣāmātmocyate). Ques:. Nor can it be said that since Īśvara cannot be addressed or be used as an agent, how can it denote the meaning of the word “aham”. A: just as the jīva has agency through the sense organ of speech, for the sake of indicating all worldly activity such as being addressed etc., it is Brahman alone that is being addressed through the instrument known as the jīva; this is the (meaning of) the instruction “tattvamasi”. Thus another śruti states through “nānyato’ato’sti…boddhā” (Bṛ.Up.3.7.23) that paramātman alone has agency of worldly activities such as seeing, hearing etc.

\dev{यत्तु आधुनिकास्तत्त्वमसीत्यादिवाक्ये जीवस्यैव त्वमहंपदार्थ निर्णये सति “को न आत्मे” त्यादिश्रुत्यन्तरानुसारिण्या लोकानुसारिण्याश्चाकाङ्क्षाया अनुपपत्तेः ।लोके हि कोऽहमित्याकाङ्क्षयैवामुकस्त्वमसीत्युत्तरं दृश्यते नान्यथेति । अस्मन्मते तु अहंशब्दार्थत्वेन प्रयोक्तृसंघाताध्यक्षत्वादिना सामान्यरूपेणैवा काङ्क्षायां तदित्यादिविशेषरूपेणोपदेश इति लौकिकी शब्दमर्यादा न हीयत इति । यदपि तत् त्वमेव त्वमेव तत् इति परस्परव्यतिहारवाक्यं तदन्योन्यवैधर्म्यलक्षणं भेदं निवर्तयति जीवस्यासंसारित्वप्रतिपादनाय परमात्मस्वरूपप्रदर्शनाय च, अन्यथा व्यतिहारवैयर्थ्यात्। “यच्चाप्येवं सकलं जातमपि सर्वं प्रतिष्ठितम् , स एव जीवः सुखदु:खभोक्ते” त्यादिवाक्यं तदविभागेनोपासनां विदधाति । यच्च— }
\begin{verse}
\dev{विभेदजनकेऽज्ञाने नाशमात्यन्तिकं गते ।}\\
\dev{आत्मनो ब्रह्मणा भेदमसन्तं कः करिष्यति ।।  इति विष्णुपुराणं.}
\end{verse}
\dev{तस्यायमर्थः— धर्माधर्मादिद्वारा विभाजनकेऽज्ञाने देहाद्यभिमाने ज्ञानेनात्यन्तमुत्सन्ने सति असन्तं कदाचित्कत्वेन वाचारम्भणमात्रं जीवब्रह्मविभागं पुनः कः करिष्यति कारणनाशादिति}

\dev{यच्च—}
\begin{verse}
\dev{यदि जीवः पराद् भिन्नः कार्यतामेति सुव्रत ।}\\
\dev{अचित्त्वं च प्रसज्येत घटवत् पण्डितोत्तम ।।}
\end{verse}
\dev{इति गौतमीयतन्त्रम्, तस्याप्ययमर्थः— यदि जीवः परा भिन्नः परादत्यन्तं भिन्नः परस्यानंश इति यावत् तथा प्रलयकालीनचेतनाद्वैतश्रुत्यनुरोधेन जीवस्य कार्यत्वं स्यात् प्रकृत्यादिवदभेदे च घटवज्जडत्वमेव प्रसज्यत इत्यर्थः । “क्षेत्रज्ञं चापि मां विद्धि” इत्यादिवाक्यं चांशांश्यविभागपरम् । एवमन्यान्यपि जीवब्रह्माभेदवाक्यान्यनयैव दिशा यथायोग्यं प्रकरणानुसारेणीशांश्यभेद- ब्रह्मात्मत्व-सामान्याभेदैस्तृतीयसूत्रे वक्ष्यमाणेन शक्तिशक्तिमदभेदेन वा व्याख्येयानीति दिक् ।}

However when modern (vedāntins) in making the decision in the meaning of the words ‘tvam’ and ‘aham’ as referring to    the word jīva itself is the meaning of “aham” in the sentence “tattvamasi” then it is in contradiction to other śruti statements such as “ko na ātmā”\endnote{This quote is probably from the Aitareya Up (Ait.Up 3.1.1)} (not traced) as also to the desire of people to know (about the ātman)  which is in accordance with worldly behaviour (śrutyantarānusāriṇyā lokānusāriṇyāścākāṅkṣāyā anupapatteḥ). In the world it is with the desire to know who ‘I am’ the answer ‘you are so and so’ is given not otherwise. In our view the meaning of the word ‘aham’ does not abandon the worldly usage (laukikī śabdamaryādā) of being a superintendent who directs a collective in general; thus with that desire the instruction (upadeśa) is given in the special form as “tat”\endnote{The one who supervises and directs all the functions is denoted by the word “tat” in “tattvamasi”} Even though such sentences as “tat tvameva”, “tvameva tat” are mutually interchangeable it only accomplishes the division of the nature of mutually different properties in order to indicate the unworldly nature of the jīva and the intrinsic nature of paramātman; otherwise the reciprocity will be useless. The statement: “yaccāpyevam sakalam…sukhaduḥkhabhoktā” (not traced) accomplishes worship through (the idea of) non-separation.The meaning of the verse “vibhedajanake’jñane…kaḥ kariṣyati” in the Viṣṇu P  (6.7.94 cited in Tripathi p.31.fn 3) is as follows: Due to dharma and adharma when ignorance having the sense of agency in the body etc., gives rise to separation , then when the separation of jīva and Brahman which is just in the form of worldly activity (kādācittkatvena vācāraṁbhaṇamātram) intermittently, which is not exhausted, is destroyed totally by (correct) knowledge then who will bring about the separation of jīva and Brahman as the cause (for the separation) is destroyed.

Also the verse in the Gautamīyatantram: “yadi jīvaḥ…paṇḍitottama” has the same meaning. Its meaning is as follows: If the jīva is totally different i.e. not a part of the absolute (parād bhinnaḥ), then in keeping with the advaita texts, which mention consciousness at the time of dissolution (pralayakālīnacetanādvaitaśrutyanurodhena) jīva will become a kārya (effect); if there is identity like prakṛti (during dissolution) then like pots etc., there is the danger of its being insentient\endnote{When different it can suffer the result of becoming like any other effect/object (kāryatvam) during pralaya. If identical like prakṛti in pralaya then there is the contingency of its being insentient like a pot etc.}. Such statements as “kṣetrajñam cāpi mām viddhi” (Gītā.13.3) are inclined towards non-separation like a whole and its parts. So also in the same way statements of identity between jīva and Brahman are mentioned in the same manner appropriately, depending on the context, in the upcoming third sūtra (BS.I.1.3) as identity like that between the whole and its parts; or they can be explained as the identity relationship between power and the one who possesses power (śaktiśaktimadabhena vā vyākhyeyānīti dik). \dev{ये तु तार्किका अविभागलक्षणाभेदमपि त्यक्त्वा उपासनामात्रपरत्वेनैव अभेद वाक्यानि नयन्ति, तन्मते भेदनिन्दाश्रुत्यनुपपत्तिः, अन्योन्याभावस्य पारमार्थिकत्वाद् विभागाविभागरूपयोश्च भेदाभेदयोः श्रुत्यर्थत्वानभ्युपगमादिति । अधिकं वा भेदसाधकभेदसूत्रेषु वक्ष्यामः । तस्मात् सिद्धौ जीवेश्वरयोरंशांशिभावेन भेदाभेदौ विभागाविभागरूपौ । तत्राप्यविभाग एव आद्यन्तयोरनुगतत्वात् स्वाभाविकत्वात् नित्यत्वाच्च सत्यः । विभागस्तु मध्ये स्वल्पावच्छेदेन नैमित्तिको विकारान्तरवद् वाचारम्भणमात्रमिति विशेषः ।} 

\dev{तदेवमात्माद्वैतं व्याख्यातं सामान्यतो ब्रह्माद्वैतवाक्यानि च तृतीयसूत्रे व्याख्यास्यामः । तदेवमन्योन्याभावलक्षणभेदेन जीवादत्यन्तभिन्न एवेश्वरो ब्रह्मशब्दार्थ इति सिद्धम् ।}

\dev{तत्राप्ययं विशेष—जीवव्यावर्तनायैव जगज्जन्मादिकारणं तथा नित्येच्छादितन्मायाख्यशक्त्यौपाधिकमैश्वर्यं चोपलक्षणमेव ब्रह्मणः, ननु ( न तु ) ऐश्वर्यमपि ब्रह्मशब्दार्थान्तर्गतं “सत्यं ज्ञानमनन्तं ब्रह्म, तदेव ब्रह्म त्वं विद्धि नेदं यदिदमुपासते, साक्षी चेता केवलो निर्गुणश्च, अथात आदेशो नेति, अकर्ता चैतन्यं चिन्मानं सदि” इत्यादिस्मृतिभिः, ज्ञानमेव परं ब्रह्म ज्ञानं बन्धाय वै ( ने ) ष्यते । ज्ञानात्मकमिदं विश्वं न ज्ञानाद् भिद्यते परम् ।। इत्यादिस्मृतिभिः, “चितितन्मात्रेण तदात्मकत्वादित्यौडुलोमि” रित्यागामिसूत्रेण च चिन्मात्रस्यैव ब्रह्मजीवशब्दार्थत्वावगमात् उपाधिविशिष्टे शक्ति कल्पयित्वा केवले लक्षणाकल्पनायां गौरवाच्च। ब्रह्मणीच्छादिव्यवहारस्य तु स्वस्वामितासम्बन्धेन अविवेकन चोपपत्तेः । एवमेव परमात्मपरमेश्वरादिशब्दा अपि चैतन्यविशेष एव शक्ताः ।}

\dev{“वदन्ति तत्तत्त्वविदः तत्त्वं यज्ज्ञानमद्वयं ब्रह्मेति परमात्मेति भगवानिति शब्द्यते” इत्यादिश्रुतिभ्यः।}

\dev{“अनामरूपश्चिन्मात्र” इत्यादिवाक्यैस्तु जन्मानिमित्तकानामेव निषिद्धमिति ।}

Those logicians (tārkikāḥ) abandoning identity as a non-separation relationship, take the identity sentences to be inclined towards meditation/devotion (upāsanāmātraparatvena); in their view it is in contradiction to śruti that condemns difference; (according to them) since mutual absence is true (anyonyābhāvasya pāramārthikatvāt), to understand śruti utterances of difference and non-difference as of the nature of separation and non-separation is not reasonable. We will explain this further in the bheda-sūtras that (try to) establish difference. Thus the difference-nondifference relationship of jīva and Īśvara as part and whole is of the nature of being that of separation and non-separation. There also since non-separation alone follows both the beginning and the end, it being natural and eternal, it is the truth. Separation happens in between due to some minor delimitation; it is like some change due to some cause and is only meant for worldly usage (vibhāgastu madhye svalpāvacchedena naimittiko vikārāntavad vācāraṁbhaṇamātramiti viśeṣaḥ).

Thus the identity of the non-dual ātman has been generally explained; we will explain under the third sūtra (BS.I.1.3) sentences mentioning the non-duality of Brahman. In this manner, through difference of the nature of mutual absence it is established that Īśvara which is the meaning of the word Brahman, is absolutely different from jīva.\endnote{For Bhikṣu Īśvara is Brahman and the supreme ātman.} Even there, there is this special purpose: in order to separate jīva alone, Brahman (is mentioned) as the cause of the origin of the world, and its limitation, known as the power of māyā of the nature of eternal desire etc. having supremacy as being, is only an implication (jīvavyāvartanayaiva jagajjanmādikāraṇam tathā nityecchāditanmāyākhyaśaktyaupādhikamaiśvaryam copalakṣaṇameva brahmaṇaḥ).

\textbf{Ques:} But since the word ‘aiśvaryam” is included in the meaning of the word Brahman in such śruti sentences “satyam jñānamanantam brahma” (Taitt. Up. 2.1.1), “tadeva brahma…yadidamupāsate”  (Kena.Up. I.5-9), “sākṣī cetā kevalo nirguṇaśca” (Śvet. Up 6.11), “athāta ādeśo neti” (Bṛ.Up. 2.3.6), “akartā caitanyam cinmātram sat” (not traced) and in smṛti statements like “jñānameva param brahma…na jñānād bhidyate param” and in the upcoming sūtra: “cititanmātreṇa tadātmakatvādityaudulomi” (BS.4.4.6) one understands the meaning of the words Brahman and jīva as only consciousness. Imagining power in one who  is qualified by a limitation (and) imagining an indication (of power) in the (isolated) One (kevale) only makes it cumbersome (kevale lakṣaṇākalpanāyām gauravācca). The worldly usage of desire etc., in Brahman due to lack of insight) is reasonable (avivekena copapatteḥ, due to the relationship of master/possessor and servant/possessed (and) In this manner alone, can the words Paramātman, Parameśvara etc., be powerful having consciousness as an attribute. Thus śruti says: “vadanti tattatvavidaḥ tattvam…bhagavāniti śabdyate” (Bhā.P I.2.11).\endnote{It is interesting to note that Bhikṣu mentions this quotation from the Bhā.P as śruti statement. This is in keeping with the belief of 16 th century Bengal School of Vaiṣṇavism that the Bhā.P is a śruti. Even
Madhusūdana Sarasvati accords an exalted position to the Bhā.P.} By such sentences as “anāmarūpaścinmātra”, even its being an efficient cause for the birth (of the world) is rejected.\endnote{According to Tripathi this sentence is not in the original text (Tripathi p.32 fn.2)} \dev{चैतन्यं चात्मनो न गुणः, किन्तु द्रव्यविशेष एव धर्मधर्मिविभागशून्यश्चेतन इति चैतन्यमिति चोच्यते चैतन्यधर्मकत्वप्रतिषेधाय । यथा तेजो द्रव्यं प्रकाशकं प्रकाश इति चोच्यते, सदा सदा सहोपलम्भेन लाघवादेकत्वस्येव न्याय्यत्वात् । अहं जानामीति प्रत्ययस्यैव लौकिकत्वेनाधाराधेयभावांशे प्रमात्वमेव  न भ्रमत्वं, लोकानां संघातेष्वेवाहमिति भ्रमात् । संघातस्य च चित्प्रकाशधर्मकत्वमस्त्येव, यथोल्मूकस्य तेजोधर्मकत्वं घटादेर्वा छिद्रधर्मकत्वमिति । विवेकिनां त्वहं जानामीति प्रत्ययो न भवत्येव, तत्प्रत्ययस्य श्रुतिस्मृतिमाया¹नुसारित्वात् । विवेकिनामपि तथा व्यवहारस्तु राहोः शिर इतिवद् विकल्पमात्रो लोकानुसारीति न काप्यनुपपत्तिः । अधिकं तूपदेशरत्नमालाख्यप्रकरणे द्रष्टव्यम् । एवमेव वा शास्त्रेषु सर्वज्ञत्वादिवचनं व्यवहारानुसारेणोपपन्नमिति । तथा च पातञ्जलसूत्रम्—“द्रष्टा दृशिमात्र” इति । तदेतत्तृतीयाध्याये स्वयं वक्ष्यति ‘‘प्रकाशादिवच्चावैशेष्यमि” ति सूत्रेणेति । एतेन ब्रह्मण आनन्दरूपत्वमपास्तम् “नैकस्यानन्दचिद्रूपत्वे विरोधादि” ति सांख्यसूत्रोक्तन्यायाच्च । आनन्दो हि दुःखवत् स्वगोचरवृत्तिं विनाऽपि दृश्यत्वादप्रकाशरूपः, चैतन्यन्तु धर्मिग्राहकमानेन प्रकाशरूपतयैव सिद्धमिति प्रकाशाप्रकाशरूपतयोभयोर्विरोधः चैतन्यस्य तु बुद्धिवृत्तिद्वारैव स्वविषयत्वं चैतन्यगोचरचैतन्यान्तराङ्गीकारेऽनवस्थानात्, साक्षात् स्वविषयत्वे च कर्मकर्तृविरोधादतश्चैतन्यस्य प्रकाशत्वमुपपन्नमिति । यदि वा चैतन्यवत् सुखस्यापि वृत्तिद्वारैव भानमभ्युपगम्यायं विरोधः परिह्रियते तथाऽपि एकज्ञानेऽन्याज्ञानाद् विरोधः स्यात् ज्ञातत्वाज्ञातत्वयोरेकदा विरोधात्, दुःखानुव्यवसायकाले सुखाज्ञानात् सानन्दसमाध्यादौ च चैतन्याज्ञानात् ज्ञानत्वसुखत्वरूपप्रकारभेदश्च त्वयाऽपि नेष्यते एकरसत्वश्रुतिविरोधात् । किं चैवं दुःखत्वमप्यात्मनः स्याल्लाघवादिति ।}

Consciousness is not an attribute of ātman, but it is a special substance. ‘Cetana’ is devoid of the separation as qualifier and qualified; cetane is also called caitanya (dharmadharmivibhāgaśūnyaścetana iti caitanyamiti cocyate) in order to reject (the idea) that it has the property of caitanya (caitanyadharmakatvapratiṣedhāya).\endnote{In other words cetana and caitanya are the same. Caitanya does not mean it has the quality of cetana but is
a special substance and can be called cetana or caitanya.} It is like light which is a substance that has the property of illumination and is called light; since it is always accompanied (by illumination) for the sake of being non-cumbersome there is the logic of using just one word to denote the same thing. When there is the worldly thought like ‘I know’ it is correct knowledge as far as the part of support and supporter is concerned and it is not an illusion;\endnote{There is one who knows (supporter) and the knowledge (supported); this is given in worldly experience argues Bhikṣu} (however) there is illusion in the minds of people regarding the notion of ‘I’ in the collection (buddhi, ahaṅkāra,  manas etc.) The collection does have the quality of revealing consciousness (saṅghātasya ca citprakāśadharmakatvamastyeva), just as fire has the quality of heat, or like pots have the quality of holes. As for the wise the thought like ‘I know’ does not happen in accordance with the belief in māyā of śruti and smṛti. Even in the case of the wise, when there is such worldly behaviour it is a misapprehension like the saying ‘Rāhu’s head’\endnote{Probably a reference to the headless Rāhu after the head was cut off by Viṣṇu but which survived without the body because he had tasted a little bit of the nectar.} and  there is no contradiction as it follows worldly behaviour. More on this can be seen in the work Upadeśaratnamālā.\endnote{Thus the Upadeśaratnamālā has been written prior to this bhāṣya.} In a similar way the expressions omniscient etc., (sarvajñatvādivacanam) mentioned in the śāstras are appropriate in accordance with worldly usage (vyavahārānusāreṇopapannam). Thus we have Patañjali’s sūtra “draṣṭā dṛśimātra” (YS. 2.20). This will be mentioned by the author himself in the third chapter by the sūtra “prakāśādivaccāvaiśeṣyam” (BS.3.2.25).By the above statement, Brahman being of the nature of ānanda has been rejected; this is also because of the logic given in the Sāṁkhyasūtra: “naikasyānandacidrūpatve virodhāt”. Ānanda (bliss) is like pain and is capable of being experienced without being a modification of one’s own  mind (and) is of the nature of non-illumination. Consciousness, on the other hand, is of the nature of illumination alone, having the purpose of grasping the object (dharmigrāhakamānena prakāśarūpatayaiva siddhamiti) and is established as of the nature of illumination; therefore there is a contradiction between being of the nature of having the quality of illumination and not having the quality of illumination (parkāśāprakāśarūpatayobhayorvirodhaḥ).\endnote{By its intrinsic nature caitanya grasps or knows the object and so it does not have illumination and non-illumination as qualities} It is only through the modification of the intellect (citta) that consciousness experiences its object (and so) possesses the object as its own; if one were to accept consciousness as an internal object of consciousness there will be lack of certainty (it will lead to arguing ad infinitum); also if one has one’s own self directly as an object there will the contradiction of the subject and object being the same; therefore it is correct to accept consciousness of the nature of illumination. Ques: If it is said that, like consciousness, by admitting the knowledge of pleasure also through a modification of the mind this contradiction can be removed, then the answer is: 

\textbf{Ans:} even then there will be a contradiction, as within one knowledge of an object (there will be the presence of) another unknown object, leading to the contradiction of (something) being known and something being unknown at the same time. Since at the time of apperception of pain (time of consciousness of pain ) one is not conscious of pleasure, (so also)  in sānanda-samādhi (samādhi accompanied by bliss) etc., at the start, consciousness is not known.\endnote{The four types of saṁprajñāta samādhi has been explained in detail in the Yogavārttika under sūtra I.17.
(See Rukmani 1981:104ff)} You (the opponent) also do not  desire difference of qualities of the nature of consciousness, pleasure etc., as it contradicts śruti admitting having only one sentiment (ekarasatvaśrutivirodhāt). Moreover in this way even pain can be admitted in ātman in the interest of parsimony of reasoning (caivam duḥkhatvamapyātmanaḥ syāllāghavāditi). \dev{ननु “आनन्दो ब्रह्मेति व्यजानाद् विज्ञानमानन्दं ब्रह्म, आनन्दाद्ध्येव खल्विमानि भूतानि जायन्त” इत्यादिश्रुतिबलत्वात् तर्कस्याप्रयोजकत्वं स्यादिति चेन्न, आनन्दादात्मनि भेदस्यापि … श्रवणात् तर्कस्यैवादर्तव्यत्वात् । “यस्तर्केणानुसन्धत्ते स धर्मं वेद नेतर” इति मनुवाक्येन संशयस्थले तर्कं विनाऽर्थावधारणस्य निन्दितत्वात् । भेदश्रुतयश्च “आनन्दं ब्रह्मणो विद्वान् न विभेति कुतश्चन, स एको ब्रह्मण आनन्दः, विज्ञानमयादन्योऽन्तर आत्मा आनन्दमय” इत्याद्याः किं बहुना, साक्षादेवानन्दरूपत्वप्रतिषेधोऽपि श्रूयते “नानन्दं न निरानन्दं, विद्वान् हर्षशोकौ जहाति” इत्यादिश्रुतिषु, स्मृतिषु च—}
\begin{verse}
\dev{“अदुःखमसुखं ब्रह्म भूतभव्यभवात्मकम् ।}\\
\dev{तत्सन्तु चेतस्यथवापि देहे}\\
\dev{सुखानि दुःखानि च किं ममात्र ।}\\
\dev{मनसः परिणामोऽयं सुखदुःखोपलक्षणम् ।” इत्याद्यासु ।}
\end{verse}
\dev{अत्र नानन्दमित्यानन्दरूपताप्रतिषेधः, न निरानन्दमिति चौपाधिकानन्दधर्मकत्वानुमतिः । विद्वानिति वाक्ये च यत् सुखहानं श्रूयते तदात्मनः सुखरूपत्वे सति न घटते, आत्मनो हानोपादानासम्भवात् । यद्यपि विद्वानिति वाक्ये जीवस्यैव सुखहानं गम्यते तथाप्यंशांशिनोरेकरूपत्वात् “तत्त्वमेव त्वमेव तत, इत्यादिवाक्यैर्जीवब्रह्मणोरत्यन्तमवैधर्म्य- प्रतिपादनाच्च जीवस्य सुखप्रतिषेधेन ब्रह्मण्यपि सुखप्रतिषेधोऽवगम्यते । अदुःखमसुखमित्यत्र च कर्मधारय एव, बहुव्रीहौ लक्षणाप्रसङ्गात् ।—सुखदुःखोपलक्षणः सुखदुःखधर्मक इत्यर्थः । अथवा सुखदुःखप्रभृतिरित्यर्थः ।}

\dev{एतेन आत्मन आनन्दरूपताप्रतिषेधान्मोक्षकाले सुखप्रतिपादकं वाक्यजातं दुःखनिवृत्तौ गौणं बोध्यम्, “तृष्णाक्षयसुखस्यैते नार्हतः षोडशीं कलामि” त्यादिप्रयोगदर्शनेन दु:खनिवृत्तौ सुखशब्दस्य निरूढलक्षणासिद्धेः । तथा चोतं कपिलचार्यैः “दु:खनिवृत्तेर्गौणः, विमुक्तः (क्त) प्रशंसा मन्दानामि” ति सूत्राभ्यामिति । अथवा “सुखं दुःखसुखात्यय” इति परिभाषया सुखशब्दोऽत्र न गौणः । तस्मादिच्छिादिवदेव नित्य आनन्दोऽपीश्वरे मायोपाधिक एव जीव इव बुद्ध्यौपाधिकः । आत्मनो निरुपाधिप्रियत्वं वा, सुखत्ववदात्मत्वस्यापि प्रेमप्रयोजकत्वात् दु:खनिवृत्तिरूपत्वाद् वा बोध्यम् ।}

\textbf{Ques:} But if it is said that due to the strong statements in śruti like “ānando brahmeti vyajānāt…imāni bhūtāi jāyante” (Taitt. Up. III.6.1), logic is without purpose, then the answer is: 

%word file, page no. 68, pdf 65

\theendnotes
