\chapter{ಮಹಾಪುರುಷ ಶ್ರೀರಾಮ}

(ಹೇವಿಳಂಬಿನಾಮ ಸಂವತ್ಸರದ ಚೈತ್ರ ಶುದ್ಧ ನವಮೀ ದಿನಾಂಕ ೯-೪-೫೭ರಂದು, ಶ್ರೀರಾಮನವಮಿಯ ಮಹಾಪರ್ವದಂದು ಶ್ರೀಗುರುವು ಅನುಗ್ರಹಿಸಿದ ಪಠ ಇದರ ಸಂಗ್ರಾಹಕರು ಶ್ರೀ ವರದದೇಶಿಕಾಚಾರ್ಯರಂಗಪ್ರಿಯರು.)

\section*{ಆತ್ಮಾರಾಮರ ಹೃದಯದಲ್ಲಿ ಗೋಚರಿಸುವ ಪರಮಾತ್ಮ - ಶ್ರೀರಾಮ}

ನಾರಾಯಣನು ನರನಾಗಿ ಅವತರಿಸಿ ನರರೊಡನೆ ತಾನೂ ಬೆರೆತು ಅವರನ್ನು ತನ್ನ ಪರಮಪದಕ್ಕೆ ಕರೆದುಕೊಂಡು ಹೋಗಲು ಇಳಿದು ಬಂದ ದಿವಸವಿದಾಗಿದೆ. ದಿವ್ಯಪುರುಷನು ಭುವಿಯ ಪೆಣ್ಣೊಡನೆ ವಿವಾಹ ಮಾಡಿಕೊಂಡು ಭುವಿಯ ಜನರಿಗೆ ಆದರ್ಶವನ್ನು ತೋರಿಸಿದನು. ದಿವಿಭುವಿಗಳ ಸಂಯೋಗವನ್ನುಂಟುಮಾಡುವುದೇ ರಾಮಾಯಣದ ಕಥೆ. ಇದು ಕೇವಲ ಭೌತಿಕವಾದ ಕಥೆಯಲ್ಲ. ಇದರ ಮೂಲವನ್ನು ಹರಪ್ಪಾದಲ್ಲೋ, ಈಜಿಪ್ಟಿನಲ್ಲೋ ಪಳೆಯುಳಿಕೆಗಳಲ್ಲೋ ಹುಡುಕುವ ಸಾಹಸ ಮಾಡುತ್ತಿದ್ದಾರೆ. ಆದರೆ ವಾಲ್ಮೀಕಿಗಳ ರಾಮ - ಆತ್ಮಾರಾಮರ ಹೃದಯದಲ್ಲಿ ಧ್ಯಾನಯೋಗ (ಸಮಾಧಿ)ದಲ್ಲಿ ಗೋಚರನಾಗುವ ಪರಮಾತ್ಮ.

\section*{ರಾಮಾಯಣವು ಕೇವಲ ಲೌಕಿಕವಾದ ಒಂದು ಕಥೆಯೇ?}

ಈ ಹಿನ್ನೆಲೆಯಿಲ್ಲದಿದ್ದರೆ ಇದೊಂದು ಲೌಕಿಕ ಕಥೆಯಾಗುತ್ತೆ. ರಾಮಾಯಣದಲ್ಲಿ ನಮ್ಮ ಕಥೆಯಲ್ಲದೆ ಬೇರೇನೂ ಇಲ್ಲ- ಪುತ್ರನ ವಿರಹದಿಂದ ಕೊರಗುವುದು, ಪುತ್ರಕಾಮೇಷ್ಟಿ , ಅಶ್ವಮೇಧ. ನಾಗಪ್ರತಿಷ್ಠೆ ಮುಂತಾದ್ದು ಮಾಡುವುದು ನಮ್ಮಲ್ಲೂ ಇದೆ. ಹೆಂಡತಿಯನ್ನು ಕಳೆದುಕೊಂಡರೆ ಯಾರು ಅಳುವುದಿಲ್ಲ? ಆದರೆ ರಾಮನಂತೆ `ಗಿಡವೇ ನೀನು ಕಂಡೆಯಾ? ಮರವೇ ನೀನು ಕಂಡೆಯಾ?' ಎಂದು ಗಿಡ, ಬೆಟ್ಟ, ಜಿಂಕೆ ಮೊದಲಾದವುಗಳನ್ನು ಕೇಳದಿರಬಹುದು. `ಏನೋ ಯಾಕಯ್ಯ, ಅಷ್ಟು ಫೀಲ್ ({\eng feel}) ಮಾಡಿಕೊಳ್ಳುವುದು.' ಎಂದು ಸಮಾಧಾನ ಮಾಡಿದರೆ ಸುಮ್ಮನಾಗಬಹುದು. ಆದರೆ ರಾಮಾಯಣಕ್ಕಿರುವ ಭೂಮಿಕೆಯಾದರೂ ಏನು? ಅದನ್ನು ಹತ್ತಬೇಕು. ಬರೀ ಲೋಕದಥೆಯೇ ಆಗಿದ್ದರೆ ದೇವತೆಗಳು ರಾವಣನ ಉಪದ್ರವದಿಂದ ನಾರಾಯಣನ ಹತ್ತಿರ ಹೋದುದು, ಅಲ್ಲಿ ಭಗವಂತನ ಅಭಯ ಮುಂತಾದ್ದು ಏಕೆ? ಅದು ಆತ್ಮಾರಾಮರ ಹೃದಯದಲ್ಲಿ ಸಮಾಧಿಯಲ್ಲಿ ಪ್ರತ್ಯಕ್ಷವಾಗುವ ರಾಮನ ಕಥೆ. ಅದಕ್ಕೇ ಪ್ರಾಚೀನಾಗ್ರವಾದ ದರ್ಬೆಯಲ್ಲಿ ಕಣ್ಮುಚ್ಚಿ ಕುಳಿತು ಧ್ಯಾನಮಾಡಿ ನೋಡಿ ಬರೆಯಬೇಕಾಯಿತು. ಲೋಕದಲ್ಲೂ ಡೀಪ್ ({\eng Deep}) ಆಗಿ ಥಿಂಕ್ ({\eng Think})ಮಾಡಿ ಬರೆಯುವುದೂ ಇದೆ. ಆದರೆ ವಾಲ್ಮೀಕಿಗಳು ನೋಡಿದುದು ಈ ರೀತಿ ಡೀಪ್({\eng Deep})ಆಗಿ ಅಲ್ಲ. ಅವರು ಸೃಷ್ಟಿಯ ಮೂಲದವರೆವಿಗೂ ಹೋಗಿ ಅಲ್ಲಿಂದ ಪ್ರಕೃತಿಯಲ್ಲಿಳಿದು ಬಂದು ಪುರುಷನ ಕಥೆಯನ್ನು ಬರೆದರು. ಪುರುಷನದು ಮಾತ್ರವಲ್ಲ, ಪ್ರಕೃತಿಯ ಕಥೆಯೂ ಇದೆ. ಏಕೆಂದರೆ ಪ್ರಕೃತಿಯಲ್ಲಿರುವ ಮನುಷ್ಯರ ಉದ್ಧಾರಕ್ಕಾಗಿರುವ ಕಥೆಯಿದು.

\section*{ಭಗವದ್ಗೀತೆಯೊಡನೆ ರಾಮಾಯಣಪಾರಾಯಣ ಮಾಡುವ ಸಂಪ್ರದಾಯದ ಹಿನ್ನೆಲೆ.}

ಭಗವದ್ಗೀತೆಯೊಡನೆ ರಾಮಾಯಣವನ್ನು ಪಾರಾಯಣಮಾಡಬೇಕೆಂದು ಸಂಪ್ರದಾಯವಿದೆ. ಆದರೆ ನಾನು ಹೇಳುವುದು ಆ ಸಂಪ್ರದಾಯಕ್ಕಲ್ಲ, ಗೀತೆಯು ಹೇಳುವುದಾದರೂ ಏನು? ಆ ಪರಮ ತತ್ತ್ವವನ್ನೇ. ಅದನ್ನು ಪದೇಶಿಸುವವನೂ ನರನ ರಥದಲ್ಲಿರುವ ನಾರಾಯಣ. ಅವರನ್ನು ಕೊಂಡಾಡಿರುವುದಾದರೂ ಹೇಗೆ?

\begin{shloka}
ಕರಕಮಲನಿದರ್ಶಿತಾತ್ಮಮುದ್ರಃ\label{217}\\
ಪರಿಕಲಿತೋನ್ನತಬರ್ಹಿಬರ್ಹಚೂಡಃ|\\
ಇತರಕರಗೃಹೀತವೇತ್ರತೋತ್ರಃ\\
ಮಮ ಹೃದಿ ಸನ್ನಿಧಿಮಾತನೋತು ಶೌರಿಃ||
\end{shloka}

ತನ್ನ ಒಳಗಿನ ಸ್ಥಿತಿಯನ್ನು ಸೂಚಿಸಲು ಹೃದಯದ ನೇರದಲ್ಲಿ `ಕರಕಮಲ ನಿದರ್ಶಿತಾತ್ಮಮುದ್ರಃ'. ಹಾಗೇ, ಎಲ್ಲವನ್ನೂ ಊರ್ಧ್ವಮುಖಕ್ಕೆ ಕೊಂಡೊಯ್ಯಲು ಶಿರಸ್ಸಿನಲ್ಲಿ `ಪರಿಕಲಿತೋನ್ನತ ಬರ್ಹಿಬರ್ಹಿಚೂಡಃ'. ಮನಸ್ಸು ಎಲ್ಲಾದರೂ ಕೆಳಕ್ಕೆ ಬಂದುಬಿಟ್ಟೀತೆಂದು ಒಡನೆಯೇ `ಉನ್ನತಬರ್ಹಿಬರ್ಹ'. ಅದೂ ಬಲಭಾಗದಲ್ಲಿ. ಪುರುಷನು ತನ್ನ ನಾದವನ್ನು ಕೇಳುವುದು ದಕ್ಷಿಣಕರ್ಣದಲ್ಲಿ. ಅದೇ ಭಾವವನ್ನೇ ಎಡಗೈಯಲ್ಲಿ ತೋರಿಸುತ್ತಾನೆ. ಇಂದ್ರಿಯಾಶ್ವಗಳು ಬೇರೆಕಡೆಗೆ ಧಾವಿಸಿ (ಪ್ರಕೃತಿಯ ಪರವಾಗಿ) ದಾರಿ ತಪ್ಪದಿರಲೆಂದು ಹತೋಟಿಯಲ್ಲಿಡುವ ಚಾವಟಿ - `ಇತರ ಕರಗೃಹೀತವೇತೋತ್ರಃ'. ಇಂತಹ ಶೌರಿಯು ನಮ್ಮ ಹೃದಯದಲ್ಲಿರಲಿ ಎಂದು ಹೇಳಲ್ಪಟ್ಟಿರುವ ಪರಮಪುರುಷನ ಕಥೆಯೊಡನೆ ಪ್ರಕೃತಿಯ ಕಥೆಯನ್ನೂ ಸೇರಿಸಿ ಹೆಣೆದಿದ್ದಾರೆ.  

\section*{ದಿವಿಭುವಿಗಳ ಸೇತು ರಾಮರಾಜ್ಯ}

ರಾಮಾಯಣದಲ್ಲಿ-

\begin{shloka}
ಪ್ರಾಣಾಪಾನೌ ಸಮಾವಾಸ್ತಾಂ ರಾಮೇ ರಾಜ್ಯಂ ಪ್ರಶಾಸತಿ|
\end{shloka}
ಎಂದು ಹೇಳಿರುವಂತೆ ಪ್ರಾಣ ಮತ್ತು ಅಪಾನಗಳನ್ನು ಸಮನಾಗಿರುವ ಸ್ಥಿತಿಯಲ್ಲಿರಿಸುವ ರಾಜ್ಯವಾಗಿದೆ ರಾಮರಾಜ್ಯ. ಮೇಲ್ಮುಖವಾಗಿರುವ ಅಪಾನವು ಅಧೋಮುಖವಾಗಿಯೂ ಹೋಗುವ ಸ್ವಭಾವ. `ಎತ್ತು ಏರಿಗೆಳೆದರೆ ಕೋಣ ನೀರಿಗೆಳೆಯತ್ತೆ' ಎಂಬಂತೆ. ಇವೆರಡೂ ಒಂದು ದಾರಿಗೆ ಬರುವ ಒಪ್ಪಂದವಾಗಬೇಕು. ನಮ್ಮಲ್ಲೂ ಒಂದು ಬಗೆಯ ಒಪ್ಪಂದವಾಗಿದೆ. ಎರಡೂ ಕೆಳಮುಖವಾಗಿ ಕೆಲಸಮಾಡುವ ಒಪ್ಪಂದವಾಗಿ ಪ್ರಕೃತಿಯ ಪ್ರವಾಹಕ್ಕೆ ಬಂದಿದ್ದೇವೆ. ಆದರೆ ಎರಡೂ ಮೇಲ್ಮುಖವಾಗಿ ಹೊರಟು ಆ ಒಂದೇ ದಾರಿಯನ್ನು ಹಿಡಿಯುವ ಪ್ರಯತ್ನವೇ ರಾಮರಾಜ್ಯ. ಅದಕ್ಕೇ ಅವನಿಗೆ ಅಭಿಷೇಕವನ್ನು ಮಾಡುವವರು ವಸಿಷ್ಠರೇ ಮುಂತಾದ ಬ್ರಹ್ಮರ್ಷಿಗಳು. ಅದೂ ಮೂರ್ಧಾಭಿಷೇಕ. ಕೈಗೆ ಕಾಲಿಗೆ ಮಾತ್ರ ಅಭಿಷೇಕವಲ್ಲ. ಎಲ್ಲವೂ ಹೋಗಿ ಸೇರುವ ಭಾಗವೂ ನೆಲೆಯೂ ಆದ ಶಿರಸ್ಸಿನಲ್ಲಿ ಅಭಿಷೇಕ ಆ ಪರಮಪುರುನಿಗೆ.

\begin{shloka}
ಸಹಸ್ರಶ್ರೀರ್ಷಾ ಪುರುಷಃ| ಸಹಸ್ರಾಕ್ಷಃ ಸಹಸ್ರಪಾತ್|\\
ಸ ಭೂಮಿಂ ವಿಶ್ವತೋ ವೃತ್ವಾ| ಅತ್ಯತಿಷ್ಠದ್ದಶಾಂಗುಲಮ್|\\
ಪುರುಷ ಏವೇದಗ್ಂ ಸರ್ವಂ| ಯದ್ಭೂತಂ ಯಚ್ಚ ಭವ್ಯಮ್|\\
ಉತಾಮೃತತ್ವಸ್ಯೇಶಾನಃ| ಯದನ್ನೇನಾತಿರೋಹತಿ|\\
ಏತಾವಾನಸ್ಯ ಮಹಿಮಾ|| ಅತೋ ಜ್ಯಾಯಾಗ್ಂಶ್ಚ ಪೂರುಷ್ಃ|\\
ಪಾದೋಽಸ್ಯ ವಿಶ್ವಾ ಭೂತಾನಿ|  ತ್ರಿಪಾದಸ್ಯಾಮೃತಂ ದಿವಿ|\\
ತ್ರಿಪಾದೂರ್ಧ್ವ ಉದೈತ್ಪುರುಷಃ| ಪಾದೋಽಸ್ಯೆಹಾ ಭವಾತ್ಪುನಃ|\\
ತತೋ ವಿಶ್ವಙ್ ವ್ಯಕ್ರಾಮತ್| ಸಾಶನಾನಶನೇ ಅಭಿ|\\
ತಸ್ಮಾದ್ವಿರಾಡಜಾಯತ||
\end{shloka}
ಎಂದು ಆತ್ಮಾರಾಮರ ಹೃದಯದಲ್ಲಿ ಕಾಣುವ ಪ್ರಭುವಿಗೆ ಪುಷ್ಪಾಂಜಲಿಯ ಸಮರ್ಪಣೆ. ಅದನ್ನು ಹೇಳುವ ಶ್ಲೋಕವಾದರೂ ಹೇಗಿದೆ?

\begin{shloka}
ಅಭ್ಯಷಿಂಚನ್ ನರವ್ಯಾಘ್ರಂ ಪ್ರಸನ್ನೇನ ಸುಗಂಧಿನಾ|\\
ಸಲಿಲೇನ ಸಹಸ್ರಾಕ್ಷಂ ವಸವೋ ವಾಸವಂ ಯಥಾ||
\end{shloka}

ಶ್ಲೋಕದ ಮೊದಲನೆಯ ಅರ್ಧದಲ್ಲಿ ಭುಮಿಯಲ್ಲಿ ಬಾಳಿದ ಒಬ್ಬ ರಾಜನಿಗೆ ಅಭಿಷೇಕವನ್ನು ಹೇಳಿದರೂ ದಿವಿಗೆ ನಮ್ಮನ್ನು ಸೇರಿಸಲು ಒಂದು ದಿವಿಯ ಉಪಮಾನ ಪಕ್ಕದಲ್ಲೇ ಇದೆ.

\begin{shloka}
ಸಲಿಲೇನ ಸಹಸ್ರಾಕ್ಷಂ ವಸವೋ ವಾಸವಂ ಯಥಾ|\\
\end{shloka}
ವಾಸವನಿಗೆ (ಇಂದ್ರನಿಗೆ) ವಸುಗಳು ಮಾಡಿದ ಅಭಿಷೇಕ. ಹೀಗೆ ಭುವಿಯನ್ನು ದಿವಿಯೊಡನೆ ಸೇರಿಸುವ ಗ್ರಂಥವಾಗಿದೆ ರಾಮಾಯಣ. ನವರಸನಾಯಕನಾದ ರಾಮನನ್ನು ನಾಯಕನನ್ನಾಗಿ ಹೊಂದಿರುವ ಆದಿಕಾವ್ಯ ರಾಮಾಯಣ. 

\section*{ಸರ್ವಸದ್ಗುಣಧಾಮ ಶ್ರೀರಾಮ}

ಒಬ್ಬೊಬ್ಬರಲ್ಲಿ ಒಂದೊಂದು ಒಳ್ಳೆಯ ಗುಣವಿರಬಹುದು. ಕೆಲವರಿಗೆ ರೂಪವಿದೆ. ಕೆಲವರಿಗೆ ರೂಪವೂ ಇದೆ, ಧನವೂ ಇದೆ. ಮದುವೆಯೂ ಆಗಬಹುದು. ಎರಡೂ ಗಂಡು ಅಷ್ಟೇ! ಎನ್ನುವ ಹಾಸ್ಯೊಕ್ತಿಗೆ ವಿಷಯವೂ ಇರಬಹುದು. ಹಾಗಾಗದೇ ಎಲ್ಲಾಸದ್ಗುಣಗಳಿಗೂ ನೆಲೆಯಾಗಿರುವ ಮಹಾಪುರುಷನೇ ಶ್ರೀರಾಮ. ಆದ್ದರಿಂದಲೇ-

\begin{shloka}
ಬಹವೋ ದುರ್ಲಭಾಶ್ವೈವ ಯೇ ತ್ವಯಾ ಕೀರ್ತಿತಾ ಗುಣಾಃ|\\
ಮುನೇ ವಕ್ಷ್ಯಾಮ್ಯಹಂ ಬುದ್ಧ್ವಾ ತೈರ್ಯುಕ್ತಃ ಶ್ರೂಯತಾಂ ನರಃ||
\end{shloka}
ಎಂದು ದೇವರ್ಷಿ ನಾರದರು ಹೇಳುತ್ತಾರೆ. ಎಲೆಯು ಧರ್ಮ, ಹೂವಿನಧರ್ಮ, ಹಣ್ಣಿನಧರ್ಮ ಎಲ್ಲವನ್ನೂ ಒಂದೆಡೆಯಲ್ಲಿ ನೋಡಬೇಕಾದರೆ ಎಲ್ಲಿ? ಬೀಜದಲ್ಲಿ. ಅಂತೆಯೇ ಎಲ್ಲಾ ಸದ್ಗಣಗಳಿಗೂ ನೆಲೆಯಾದವನು ಸೃಷ್ಟಿಗೇ ಮೂಲಭೂತವಾದ ಶಕ್ತಿ ಶ್ರೀರಾಮ. ಆತ್ಮಾರಾಮರ ಹೃದಯದಲ್ಲಿ ಪ್ರಭುವಾಗಿ, ಪ್ರಕೃತಿಯಲ್ಲಿಳಿದು ಬಂದು ತನ್ನ ಅಪ್ರಾಕೃತಭಾವಕ್ಕೆ ಕೊಂಡೊಯ್ಯುವ ನವರಸನಾಯಕನಾದ ರಾಮನ ಸ್ತುತಿಯನ್ನು ಆತನ ಜನ್ಮದಿನದಲ್ಲಿ ಕೊಂಚ ಮಾಡೋಣ. (ಈ ಕೆಳಕಂಡ ಶ್ಲೋಕಗಳನ್ನು ಗಾನಮಾಡಿದರು)

\begin{shloka}
ಶೃಂಗಾರಂ ಕ್ಷಿತಿನಂದನಾವಿಹರಣೇ ವೀರಂ ಧನುರ್ಭಂಜನೇ\\
ಕಾರುಣ್ಯಂ ಬಲಿಭುಙ್ಮುಖೆಽದ್ಭುತರಸಂ ಸಿಂಧೌ ಗಿರಿಸ್ಥಾಪನೇ|\\
ಹಾಸ್ಯಂ ಶೂರ್ಪಣಖಾಮುಖೇ ಭಯವಹೇ ಭೀಭತ್ಸಮಾಜೇರ್ಮುಖೇ\\
ರೌದ್ರಂ ರಾವಣಭಂಜನೇ ಮುನಿಜನೇ ಶಾಂತಂ ವಪುಃ ಪಾತು ನಃ||\\
ರಾಜಾಧಿರಾಜರಾಜಾಯ ರಮಭದ್ರಾಯ ಮಂಗಳಂ||\\
ಕಾಯೇನ ವಾಚಾ ಮನಸೇಂದ್ರಿಯೈರ್ವಾ ಬುದ್ಧಾತ್ಮನಾ ವಾ ಪ್ರಕೃತೇಃ ಸ್ವಭಾವಾತ್|\label{219}\\
ಕರೋಮಿ ಯದ್ಯತ್ಸಕಲಂ ಪರಸ್ಮೈ ನಾರಾಯಣಾಯೇತಿ ಸಮರ್ಪಯಾಮಿ||
\end{shloka}
