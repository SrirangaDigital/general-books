\chapter{ದರ್ಶನ-ಹಿನ್ನೆಲೆ}

(೧೦-೪-೬೨ ರಂದು ನಡೆದ ಪಾಠ. ಉಪಸ್ಥಿತರಿದ್ದವರು, ಶ್ರೀಯುತರಾದ ಛಾಯಾಪತಿಗಳು, ಶೇಷಾಚಲಶರ್ಮರು ಹಾಗು ಎಸ್. ವಿ. ನಾರಾಯಣರು)

\section*{ನೋಡುವವರ ನೋಟವೆಲ್ಲ ಸರಿಯಾಗಿದೆಯೆಂದೇನೂ ಇಲ್ಲ.}

ಲೋಕದಲ್ಲಿ ಒಂದು ವಸ್ತುವನ್ನು ಅನೇಕರು ನೋಡಿದರೂ ಎಲ್ಲರೂ ಒಂದೇ ವಿಧವಾಗಿ ನೋಡೂವುದಿಲ್ಲ. ಒಬ್ಬೊಬ್ಬರು ಒಂದೊಂದಂಶನ್ನು ನೋಡುತ್ತಾರೆ. ಆದರೆ ನೋಡುವರ ನೋಟವೆಲ್ಲವೂ ಸರಿಯಾದ ನೋಟವೇ ಎಂದೇನೂ ಇಲ್ಲ. ಉದಾಹರಣೆಗೆ-ಭೂತಗನ್ನಡಿಯಲ್ಲಿ ಒಬ್ಬ ಮನುಷ್ಯನು ತನ್ನ ಆಕಾರವನ್ನು ನೋಡಿಕೊಂಡರೆ ಇವನ ಆಕಾರ ಇವನಿಗೇ ಭಯವಾಗುವಂತೆ ಪ್ರತಿಬಿಂಬಿಸಿರುತ್ತದೆ. ಅದು ಆ ಲೆನ್ಸಿನ ಯೋಗ್ಯತೆ. ಅಲ್ಲಿ ತನ್ನ ಪ್ರತಿಬಿಂಬವನ್ನು ಪಡೆದು ಅಂತೆಯೇ ತಾನಿರುವೆನೆಂದು ಭಾವಿಸಿದರೆ ಇದು ಭ್ರಮೆಯೇ. ಅದೇ ಯೋಗ್ಯಪ್ರತಿಬಿಂಬನ ಶಕ್ತಿಯಿರುವ ಕನ್ನಡಿಯಲ್ಲಿ ನೋಡಿಕೊಂಡಾಗ ಇವನ ಸರಿಯಾದ ಆಕಾರ ಸಿಗುತ್ತದೆ. ಕನ್ನಡಿ ಎನ್ನುವ ಶಬ್ದವೇ ಪ್ರತಿಬಿಂಬಿಸುವ ವಸ್ತು ಎನ್ನುವ ಅರ್ಥವನ್ನು ಕೊಡುತ್ತದೆ. ಆದ್ದರಿಂದಲೇ ಗಿಳಿಗೆ ``ಕನ್ನಡವಕ್ಕಿ" ಎಂಬ ಹೆಸರು ಬಂದಿದೆ.

\section*{ಶುದ್ಧವಾದ ಬುದ್ಧಿದರ್ಪಣದಲ್ಲಿ ಮಾತ್ರ ಪುರುಷನ ಯಥಾವತ್ತಾದ ದರ್ಶನ}

ಭೂತಗನ್ನಡಿಯಲ್ಲಿ ತನ್ನನ್ನು ನೋಡಿಕೊಂಡ ಮನುಷ್ಯನು ಕನ್ನಡಿಯ ಯೋಗ್ಯತೆಯನ್ನರಿಯದೇ ತನ್ನ ಆಕಾರವೇ ಇಂತಿರುವುದೆಂದು ತಿಳಿದರೆ ತನ್ನ ಆಕಾರಕ್ಕೆ ತಾನೇ ಹೇಸಿ ಭ್ರಮಿಸಬೇಕಾಗುತ್ತದೆ. ಭೂತಗನ್ನಡಿ, ನಿಲುಗನ್ನಡಿ ಎರಡೂ, ಒಂದೇ ಮರಳು ಮೊದಲಾದ ವಸ್ತುಗಳಿಂದ ಮಾಡಲ್ಪಟ್ಟಿದ್ದರೂ ಗಾಜಿನಲ್ಲಿರುವ ಉಬ್ಬು ತಗ್ಗುಗಳು ಪ್ರತಿಬಿಂಬನ ಶಕ್ತಿಯನ್ನು ಮಾರ್ಪಡಿಸುತ್ತವೆ. ಅಂದರೆ ಕನ್ನಡಿಯು ತೋರಿಸುವ ವಿಕರ ಸ್ವರೂಪವನ್ನು ನಮ್ಮ ಮೇಲೆ ಹಾಕಿಕೊಳ್ಳದೇ ಕನ್ನಡಿಯ ಯೋಗ್ಯತೆಯ ಮೇಲೆ ಹಾಕಿದಾಗ ಈ ಭ್ರಮೆಯಿರುವುದಿಲ್ಲ. ಹೀಗಾಗಬೇಕಾದರೆ ಶುದ್ಧ ದರ್ಪಣದಲ್ಲಿ ಇವನು ತನ್ನ ಆಕಾರವನ್ನು ನೋಡಿಕೊಂಡೂ ನಿಶ್ಚಯಿಸಿಕೊಂಡಿರಬೇಕು. ಹೀಗೆಯೇ ಪ್ರಕೃತಿಯ ಏರು ಪೇರುಗಳಿಂದ ವಿಕೃತವಾದ ಬುದ್ಧಿದರ್ಪಣವು ತನ್ನ ವಿಕಾರವನ್ನು ಪುರುಷನ ಮೇಲೆ ಹಾಕಿ ವಿಕೃತವಾದ ಸ್ವರೂಪವೇ ಸಹಜವೆಂದು ತೋರಿಸುತ್ತದೆ. ಆದರೆ ಅಂತಹ ಬುದ್ಧಿದರ್ಪನವನ್ನು ಶುದ್ಧಗೊಳಿಸಿ ನೋಡಿದಾಗ ತಾನೇ ನಿಜಸ್ಥಿತಿಯ ಅರಿವಾಗುತ್ತದೆ.

\section*{ಭ್ರಮೆ ಪರಿಹಾರವಾದರೆ ನಿರ್ಭಯನಾಗಿರಬಹುದು}

ರೈಲಿನಲ್ಲಿ ಪ್ರಯಾಣ ಮಾಡುವಾಗ ನಮ್ಮ ರೈಲು ನಿಂತಿದ್ದು ಎದುರು ರೈಲು ಚಲಿಸುತಿದ್ದರೂ ನಮ್ಮ ರೈಲು ಚಲಿಸುತ್ತಿರುವಂತೆ ಕಾಣುತ್ತದೆ. ಆದರೆ ಅದು ಭ್ರಮೆ. ಏಕೆ? ಸಂಚರಿಸುತ್ತಿರುವುದು ಎದುರಿನ ರೈಲೇ ಹೊರತು ನಮ್ಮ ರೈಲಲ್ಲ. ಇದನ್ನು ನಿರ್ಧರಿಸಿಕೊಳ್ಳುವುದು ಯಾವಾಗ? ಪಕ್ಕದಲ್ಲಿರುವ ನಿಲ್ದಾಣವನ್ನು ನೋಡಿದ ಮೇಲೆ. ಅಂತೆಯೇ ಚಲಿಸದಿರುವ ಇತರ ಅಕ್ಕಪಕ್ಕದ ಪದಾರ್ಥಗಳನ್ನು ನೋಡಿ ನಮ್ಮ ರೈಲುನಿಂತಿದೆ ಎಂದು ನಿರ್ಧರಿಸಿಕೊಳ್ಳಬೇಕು. ಅಂತೆಯೇ ಚಲವಾದ ಪ್ರಕೃತಿಚಕ್ರವು ಚಲಿಸುತಿರುವುದನ್ನು ನೋಡಿ ತಾನೇ ಚಲಿಸುತ್ತಿರುವೆನೆಂದು ಜೀವನು ಭ್ರಮಿಸುತ್ತಾನೆ. ಪ್ರಕೃತಿಯಲ್ಲಿ ತನ್ನನ್ನು ನೋಡಿಕೊಳ್ಳುತ್ತಿರುವವರೆಗೂ ಈ ಭ್ರಮೆಯೇ ಆತನಿಗೆ. ಅಂದರೆ ತನ್ನನ್ನು ತಾನು ಪ್ರಕೃತಿಗಿಂತ ಹೊರತಾಗಿ ನೋಡಿಕೊಂಡಾಗ ತಾನೇನೂ ಚಲಿಸಿಲ್ಲ, ಚಲಿಸುತ್ತಿರುವುದು ಪ್ರಕೃತಿ ಎಂಬುದು ನಿರ್ಣಾಯಕ್ಕೆ ಬರುತ್ತದೆ. ಆಗ ಭ್ರಮೆಯಲ್ಲ, ಪ್ರಮೆ.

\begin{shloka}
``ಅಚ್ಛೇದ್ಯೋಽಯಂ ಅದಾಹ್ಯೋಽಯಂ" | ``ಅಚಲೋಽಯಂ ಸನಾತನಃ"\label{36}
\end{shloka}

ಈ ಸನಾತನವಾದ ವಸ್ತುವು ಅcಅಲವಾಗಿಯೇ ಇದೆ. ತಾನು ಪ್ರಕೃತಿಯೆಂಬ ರೈಲಿನಲ್ಲಿ ಕುಳಿತು ಅದರ ಚಲನೆಯನ್ನು ತನ್ನ ಚಲನೆಯಲ್ಲಿ ಆರೋಪ ಮಾಡಿಕೊಂಡು ಒದ್ದಾಡುತ್ತಾನೆ ಈ ಜೀವ. ಆದರೆ ತನ್ನ ಕಡೆಗೆ ತಾನು ನೋಡಿಕೊಂಡು ಅಚಲನೆಂದು ನಿರ್ಧಿರಿಸಿಕೊಂಡಾಗ ನಿರ್ಭಯನಾಗಿರಬಹುದು.

\section*{ಶುದ್ಧವಾದ ಬುದ್ಧಿದರ್ಪಣದಲ್ಲಿ ಮಾತ್ರ ಪುರುಷನ ಯಥಾವತ್ತಾದ ದರ್ಶನ}

ಭೂತಗನ್ನಡಿಯಲ್ಲಿ  ತನ್ನನ್ನು ನೋಡಿಕೊಂಡ ಮನುಷ್ಯನು ಕನ್ನಡಿಯ ಯೋಗ್ಯತೆಯನ್ನರಿಯದೇ ತನ್ನ ಆಕಾರವೇ ಇಂತಿರುವುದೆಂದು ತಿಳಿದರೆ ತನ್ನ ಆಕಾರಕ್ಕೆ ತಾನೇ ಹೇಇಸ್ ಭ್ರಮಿಸಬೇಕಾಗುತ್ತದೆ. ಭೂತಗನ್ನಡಿ, ನಿಲುಗನ್ನಡಿ ಎರಡೂ, ಒಂದೇ ಮರಳು ಮೊದಲಾದ ವಸ್ತುಗಳಿಂದ ಮಾಡಲ್ಪಟ್ಟಿದ್ದರೂ ಗಾಜಿನಲ್ಲಿರುವ ಉಬ್ಬು ತಗ್ಗುಗಳು ಪ್ರತಿಬಿಂಬನ ಶಕ್ತಿಯನ್ನು ಮಾರ್ಪಡಿಸುತ್ತವೆ. ಅಂದರೆ ಕನ್ನಡಿಯ ಯೋಗ್ಯತೆಯ ಮೇಲೆ ಹಾಕಿದಾಗ ಈ ಭ್ರಮೆಯಿರುವುದಿಲ್ಲ. ಹೀಗಾಗಬೇಕಾದರೆ ಶುದ್ಧ ದರ್ಪಣದಲ್ಲಿ ಇವನು ತನ್ನ ಆಕಾರವನ್ನು ನೋಡಿಕೊಂಡು ನಿಶ್ಚಯಿಸಿಕೊಂಡಿರಬೇಕು. ಹೀಗೆಯೇ ಪ್ರಕೃತಿಯ ಏರು ಪೇರುಗಳಿಂದ ವಿಕೃತವಾದ ಬುದ್ಧಿದರ್ಪಣವು ತನ್ನ ವಿಕಾರವನ್ನು ಪುರುಷನ ಮೇಲೆ ಹಾಕಿ ವಿಕೃತವಾಅ ಸ್ವರೂಪವೇ ಸಹಜವೆಂದು ತೋರಿಸುತ್ತದೆ. ಆದರೆ ಅಂತಹ ಬುದ್ಧಿದರ್ಪಣವನ್ನು ಶುಧಗೊಳಿಸಿ ನೋಡಿದಾಗ ತಾನೆ ನಿಜಸ್ಥಿತಿಯ ಅರಿವಾಗುತ್ತದೆ.

\section*{ಭ್ರಮೆ ಪರಿಹಾರವಾದರೆ ನಿರ್ಭಯನಾಗಿರಬಹುದು}

ರೈಲಿನಲ್ಲಿ ಪ್ರಯಾಣ ಮಾಡುವಾಗ ನಮ್ಮ ರೈಲು ನಿಂತಿದ್ದು ಎದುರು ರೈಲು ಚಲಿಸುತ್ತಿದ್ದರೂ ನಮ್ಮ ರೈಲು ಚಲಿಸುತ್ತಿರುವಂತೆ ಕಾಣುತ್ತದೆ. ಆದರೆ ಅದು ಭ್ರಮೆ. ಏಕೆ? ಸಂಚರಿಸುತ್ತಿರುವುದು ಎದುರಿನ ರೈಲೇ ಹೊರತು ನಮ್ಮ ರೈಲಲ್ಲ. ಇದನ್ನು ನಿರ್ಧರಿಸಿಕೊಳ್ಳುವುದು ಯಾವಾಗ? ಪಕ್ಕದಲ್ಲಿರುವ ನಿಳಾಣವನ್ನು ನೋಡಿದ ಮೇಲೆ. ಅಂತೆಯೇ ಚಲಿಸದಿರುವ ಇತರ ಅಕ್ಕಪಕ್ಕದ ಪದಾರ್ಥಗಳನ್ನು ನೋಡಿ ನಮ್ಮ ರೈಲು ನಿಂತಿದೆ ಎಂದು ನಿರ್ಧರಿಸಿಕೊಳ್ಳಬೇಕು. ಅಂತೆಯೇ ಚಲವಾದ ಪ್ರಕೃತಿಚಕ್ರವು ಚಲಿಸುತ್ತಿರುವುದನ್ನು ನೋಡಿ ತಾನೇ ಚಲಿರುತ್ತಿರುವೆನೆಂದು ಜೀವನು ಭ್ರಮಿಸುತ್ತಾನೆ. ಪ್ರಕೃತಿಯಲ್ಲಿ ತನ್ನನ್ನು ನೋಡಿಕೊಳ್ಳುತ್ತಿರುವವರೆಗೂ ಈ ಭ್ರಮೆಯೇ ಆತನಿಗೆ. ಅಂದರೆ ತನ್ನನ್ನು ತಾನು ಪ್ರಕೃತಿಗಿಂತ ಹೊರತಾಗಿ ನೋಡಿಕೊಂಡಾಗ ತಾನೇನೂ ಚಲಿಸಿಲ್ಲ, ಚಲಿಸುತ್ತಿರುವುದ್ ಪ್ರಕೃತಿ ಎಂಬುದು ನಿರ್ಣಯಕ್ಕೆ ಬರುತ್ತದೆ. ಆಗ ಭ್ರಮೆಯಲ್ಲ, ಪ್ರಮೆ.

\begin{shloka}
``ಅಚ್ಛೇದ್ಯೋಽಯಂ ಅದಾಹ್ಯೋಽಯಂ"|``ಅಚಲೋಽಯಂ ಸನಾತನಃ"
\end{shloka}

ಈ ಸನಾತನವಾದ ವಸ್ಸ್ತುವು ಅಚಲವಾಗಿಯೇ ಇದೆ. ತಾನು ಪ್ರಕೃತಿಯೆಂಬ ರೈಲಿನಲ್ಲಿ ಕುಳಿತು ಅದರ ಚಲನೆಯನ್ನು ತನ್ನ ಚಲನೆಯಲ್ಲಿ ಆರೋಪ ಮಾಡಿಕೊಂಡು ಒದ್ದಾಡುತ್ತಾನೆ ಈ ಜೀವ. ಆದರೆ ತನ್ನ ಕಡೆಗೆ ತಾನು ನೋಡಿಕೊಂಡು ಅಚಲನೆಂದು ನಿರ್ಧರಿಸಿಕೊಂಡಾಗ ನಿರ್ಭಯುನಾಗಿರಬಹುದು.

\section*{ಜ್ಞಾನಿಯೊಬ್ಬನಿದ್ದಾಗ ಭ್ರಮೆಗೆ ಪರಿಹಾರ}

ತಲೆ ತಿರುಗುವಾಗ ಮನೆಯೇ ತಿರುಗಿದಂತೆ ಕಾಣುತ್ತದೆ. ಈ ವಿಷಯದ ಅನುಭವವಿಲ್ಲದಿದ್ದರೆ ತಲೆ ತಿರುಗುವಾಗ `ನೋಡಿ! ಮನೆ ತಿರುಗುತ್ತಿದೆ, ಕಟ್ಟಿಹಾಕಿ', ಎಂದು ಹೇಳ ಹೊರಡುತ್ತಾನೆ. ನೂರು ಜನ ಇಂತಹವರೇ ಆಗಿಬಿಟ್ಟರೆ ಹಗ್ಗ ತೆಗೆದುಕೊಂಡು ಕಟ್ಟಿದ್ದೂ ಕಟ್ಟಿದೇ ಮನೆಯನ್ನು. ಅದು ನಿಲ್ಲುವಂತಿಲ್ಲ, ಇವರು ಬಿಡುವಂತಿಲ್ಲ. ಅಂತೆಯೇ ತನ್ನ ತಲೆಯಲ್ಲಿನ ತಿರುಗುವಿಕೆಯನ್ನು ಪಿತ್ತವು ಮನೆಯಲ್ಲಿರುವಂತೆ ತೋರಿಸಿ ಭ್ರಮೆಪಡಿಸುತ್ತದೆ. ಆದರೆ ಆರೋಗ್ಯವಂತನಾದವನೊಬ್ಬನಿದ್ದು, ಅವನು ತಲೆ ತಿರುಗುವಿಕೆಯನ್ನೂ, ಅದರ ವಿಕಾರವನ್ನೂ ಅರಿತಿದ್ದರೆ `ತಿರುಗುತ್ತಿರುವುದ್ ತಲೆ, ಮನೆಯಲ್ಲ, ಎಂಬುದನ್ನು ಅರಿಯುಬಲ್ಲನು. ಅಂತೆಯೇ ಪ್ರಕೃತಿಯು ತನ್ನ ತಿರುಗುವಿಯನ್ನು ಅಚಲನಾದ ಪುರುಷನಲ್ಲಿ ಆರೋಪಿಸಿಬಿಡುತ್ತದೆ. ಆದರೆ ಜ್ಞಾನಿಯಾದವನೊಬ್ಬನಿದ್ದಾಗ ಈ ಭ್ರಮೆಗೆ ಅವಕಾಶವಿಲ್ಲ.

\section*{ಬುದ್ಧಿಯಲ್ಲಿ ವೇಗವಿದ್ದರೆ ಭ್ರಮೆಗೆ ಅವಕಾಶ}
ತನ್ನ ಪ್ರತಿಬಿಂಬವನ್ನು ವಿಕೃತಗೊಳಿಸಿದ ನೋಟ, ಅಂತೆಯೇ ತನ್ನೊಳಗಿನ ಚಲನೆಯನ್ನು ಹೊರಗೆ ಗ್ರಹಿಸಿದ ನೋಟ, ಮತ್ತೊಂದರ ಚಲನೆಯನ್ನು  ತನ್ನಮೇಲೆ ಹಾಕಿಕೊಂಡ ನೋಟ-ಈ ಮೂರೂ ನೋಟವೇ ಆದರೂ ಇದರಿಂದಾಗುವುದು ಭ್ರಮೆಯೇ.

ಕೆರೆಯ ಕಡೆಗೆ ಹೋದ. ಅಲ್ಲಿದ್ದ ಶುಕ್ತಿಯನ್ನು (ಕಪ್ಪೆಚಿಪ್ಪು) ನೋಡಿದ. ಅದರ ಹೊಳಪಿನಿಂದ ರಜತ (ಬೆಳ್ಳಿ) ವೆಂದು ತಿಳಿದು ಅದನ್ನು ಸಂಗ್ರಹಿಸಿ ಮನೆಗೆ ಬಂದು ನೋಡಿದಾಗ ಅದು ಶುಕಕ್ತಿಯಾಗಿತ್ತು. ``ಬೆಳ್ಳಿ ಕಪ್ಪೇಚಿಪ್ಪಾಗಿಬಿಡ್ತಲಾ"! ಎಂದುಕೊಳ್ಳುತ್ತಾನೆ. ಆಗ ಬೆಳ್ಳಿಯು ಕಪ್ಪೇಚಿಪ್ಪೇನೂ ಆಗಲಿಲ್ಲ. ಮೊದಲಿನಿಂದಲೂ ಕಪ್ಪೇಚಿಪ್ಪಾಗಿಯೇ ಇದೆ. ಆದರೆ ಇವನು ನೋಡಿದಾಗ ಅದರ ಹೊಳಪು ಬೆಳ್ಳಿಯೆಂದು ತಿಳಿಯುವಂತೆ ಮಾಡಿತು. ಬೆಳ್ಳಿಯೆಂದು ಮನಸ್ಸಿಗೆ ಬಂದ ತಕ್ಷಣ ಇವನಿಗೆ ಆಶೆಯುಂಟಾಗಿ ಅದರ ಸಂಗ್ರಹದಲ್ಲಿ ಪ್ರವೃತ್ತನಾದ. ಆದರೆ ಇವನಿಗೆ ಆಶೆ ವೇಗವಾಗಿ ಹರಿಯುತ್ತಿದ್ದುದರಿಂದ ಆ ವೇಗ ವಿಚಾರವನ್ನು ಮರೆಸಿತು. ಆ ವೇಗೆ ಇಳಿದು ನಿಧಾನವಗಿ ನೋಡಿದಾಗ ಅದರ ವಾಸ್ತವ ಚಿತ್ರ ತಿಳಿಯಿತು. ಅಂತೆಯೇ ಹಗ್ಗವನ್ನು ಹಾವೆಮ್ದು  ಭಮಿಸುವುದು. ಹಾವು ಎಂದಂದುಕೊಂಡ ಕೂಡಲೆ ಭಯದ ವೇಗೆ ಜಾಸ್ತಿಯಾಗಿ ನಿಜಾಂಶವು ಮರೆಯಾಗುತ್ತದೆ. ನಂತರ ಭ್ರಮಕ್ಕಾರಂಭ. ಆದರೆ ವಿಚಾರಸಿ ನಿಧಾನವಾಗಿ ನೋಡಿದಾಗ, ಅದು ಹಗ್ಗವೆಂದು ತಿಳಿದ ಮೇಲೆ, `ಅಯ್ಯೋ ! ಇದು ಹಗ್ಗ, ಹಾವೆಂದು ಭ್ರಮಿಸಿದೆನಲ್ಲಾ' ಎಂದು ತನಗೇ ತನ್ನ ಕ್ರಿಯೆ ಹಾಸ್ಯವಾಗಿ ಕಾಣುತ್ತದೆ.

ಆಹಾರವನ್ನು ತೆಗೆದುಕೊಳ್ಳುವಾಗ ಷಡ್ರಸೋಪೇತವಾದ ಆಹಾರವನ್ನು ಬಡಿಸಿದರೂ ಇತರ ವೇಗಗಳಿದ್ದಾಗ ಅದರ ಸವಿ ತಿಳಿಯದು. `ಉಪ್ಪು ಸಾಕೇ?' ಎಂದರೆ `ಅದು ನೋಡಲಿಲ್ಲ, ಇನ್ನೊಂದು ಸಲ ಬಡಿಸಿ' ಎಂದು ಹೇಳುವ ಹಾಗೆ ಇರುವ ರಸಕ್ಕೆಲ್ಲಾ ಒಂದೊಂದು ಸಲ ಬಡಿಸಿಕೊಳ್ಳಬೇಕಾಗುತ್ತೆ. ಆದರೆ ಪದಾರ್ಥವನ್ನು ತೆಗೆದುಕೊಳ್ಳುವಾಗಲೇ ರಸದ ಕೇಂದ್ರಗಳು ಮಾತ್ರ ತಾವು ಗ್ರಹಿಸಿಬಿಟ್ಟಿರುತ್ತವೆ. ಆದರೆ ಇವನಿಗೆ ಇನ್ಫಫರ್ಮೇಷನ್ ಇರುವುದಿಲ್ಲ.

\section*{ಈ ಪಾಠದ ಉದ್ದೇಶ}

ಒಬ್ಬ ಯಜಮಾನ ಆಳುಗಳಿಗೆ ಹಣಕೊಟ್ಟ `ಮಾರ್ಕೆಟ್ಟಿನಲ್ಲಿ ಹಣ್ಣು  ತೆಗೆದುಕೊಂಡು ಬನ್ನಿ' ಎಂದು. ಆಳುಗಳು ಹಣ್ಣು ತೆಗೆದುಕೊಂಡು ತಾವೇ ತಿಂದುಬಿಟ್ಟರು. ಹೀಗೇಯೇ ಇಂದ್ರಿಯಗಳು ಆತ್ಮನಿಂದ ಶಕ್ತಿಯನ್ನು ಪಡೆದು ತಮ್ಮ ಕೆಲಸವನ್ನು ಸಾಗಿಸಿಕೊಳ್ಳುತ್ತವೆ. ಹೀಗಾಗಿ ಯಜಮಾನನಿಗೆ ಏನೂ ಇಲ್ಲದಂತಾಗಿದೆ, ಮೂಲವನ್ನು ಮರೆಸಿವೆ, ವಿಶ್ವಾಸದ್ರೋಹ ಮಾಡುತ್ತಿವೆ, ಸತ್ಯವನ್ನು ಮರೆಸಿ ಭ್ರಮಣವುಂಟುಮಾಡುತ್ತಿವೆ. ಹೀಗೆ ವಿವಿಧ ಕಾರಣಗಳಿಂದ ವಸ್ತುವಿನ ವಾಸ್ತವಿಕ ಸ್ವರೂಪ ತಿಳಿಯುವುದಿಲ್ಲ. ಯಥಾರ್ಥಗ್ರಹಣವಿಲ್ಲ. `ಯಥಾರ್ಥ' ವೆಂದರೆ ವಸ್ತು ಹೇಗಿದೆಯೋ ಅದನ್ನು ಹಾಗೆಯೇ ಗ್ರಹಿಸುವುದು. ದೃಷ್ಟಿಕೆಟ್ಟಿದೆ. ಆದ್ದರಿಂದ ದೃಷ್ಟಿ ಅಥವಾ ಧರ್ಶನ ಹೇಗಿರಬೇಕು? ಎನ್ನುವುದರ ಬಗ್ಗೆಯೇ ಇಲ್ಲಿ ಪಾಠ. ಕಣ್ಣಿನಲ್ಲಿ ಪೊರೆ ಬೆಳೆದುಕೊಂಡಿದ್ದರೆ ಆ ಪೊರೆಯನ್ನು ತೆಗೆದು ಮೊದಲಿನ ದೃಷ್ಟಿಯನ್ನು ಕೊಡುವಂತೆ  ವೈ ದ್ಯರಲ್ಲಿ ಕೇಳುವ ಹಾಗೆ, ಪುರಾತನವೂ ಸನಾತನವೂ ಆದ ನೋಟವನ್ನು ಪಡೆಯುವುದೇ ಈ ಪಾಠದ ಉದ್ದೇಶ. ವೇಗವಾಗಲೀ ಇಂದ್ರಿಯದೋಷವಾಗಲೀ ಇತರ ಯಾವುದೇ ಕಾಲದೇಶಗಳಲ್ಲಿನ ವ್ಯತ್ಯಾಸವಾಗಲೀ ಇದ್ದಲ್ಲಿ ಅದು ವಸ್ತುವಿನ ಸಹಜದರ್ಶನವನ್ನು ಕೊಡಲಾರದು. ಆದರೆ ವಸ್ಸ್ತುವಿನ ಸಹಜವಾದ ದರ್ಶನವನ್ನು ಶುದ್ಧವೂ ಸಹಜವೂ ಆದ ಬುದ್ಧಿಯಿಂದ ಮಾತ್ರ ಪಡೆಯಬಹುದು. ಆದ್ದರಿಂದ ನಿರ್ವಿಕಾರವಾದ ಸಹಜವಾದ  ದರ್ಶನದ ಬಗ್ಗೆ ಈ ಪಾಠವನ್ನು ಇಡಲಾಗಿದೆ.
  

