\chapter{Verse 8}

\begin{moolashloka}
\dev{स पर्यगात् शुक्रमकायम् अव्रणमस्नाविरं शुद्धमपापविद्धम् ।}\\
\dev{कविर्मनीषी परिभूः स्वयंभूः याथातथ्यतः\\ अर्थान्व्यदधात् शाश्वतीभ्यः समाभ्यः ॥~8~॥}
\end{moolashloka}

\textbf{Word to word Meaning}: He (\dev{स}) is the one who has gone around (\dev{पर्यगात्}), is bright (\dev{शुक्रम्}), bodiless (\dev{अकायम्}), without scar (\dev{अव्रणम्}), without muscles (\dev{अस्नाविरं}), pure (\dev{शुद्धम्}), untouched by sin (\dev{अपापविद्धम्}), is the Seer (\dev{कवि:}), is the Thinker/the Wise (\dev{मनीषी}), is transcendent (\dev{परिभूः}), self-existing (\dev{स्वयंभूः}); duly according to their nature (\dev{याथातथ्यतः}) has allotted the duties (\dev{अर्थान्व्यदधात्}) to the eternal(\dev{शाश्वतीभ्यः}) time-principles (\dev{समाभ्यः}).

\textbf{Meaning}: He (the \emph{Ātma/Brahman}) is the one who has gone around (all the sides), is bright, without body, without any scars or muscles, is pure and untouched by sin. He is the \emph{Seer} and the \emph{Thinker}, who is transcendent and self-existent and who has duly allotted the duties to all the eternal time principles (\emph{Prajāpati-s}) according to their natures.

\textbf{Analysis:} In this verse, the \emph{Upaniṣad} returns to the theme of ``nature of \emph{Brahman/Ātma}''.

\dev{स}- \emph{He}. Here `He' refers to the `\emph{Ātma/Brahman}' that has been referred to in previous verses. The reference to ``He'' should not be taken to mean that \emph{Ātma} is masculine. Here ``He'' has been merely used to refer to \emph{Brahman} who has no gender in his absolute state.

\dev{पर्यगात्}-\emph{who has gone around all the sides.} `Gone around' refers to the fact that, \emph{Ātma} engulfs everything that is present in the universe. In whatever direction one may look, one will find \emph{Brahman} there; in whatever direction one may travel, one will reach \emph{Brahman} through it. The whole manifestation and everything inside it, is totally pervaded and inhabited by \emph{Ātma}. In fact, all the numerous forms and names are only external and apparent, and in truth all that is present is \emph{Ātma} alone, nothing other than it exist. The \emph{Kaṭhopaniṣad} says (1) ``There is no multiplicity here''. Hence, here \dev{पर्यगात्} refers to all pervading nature of \emph{Brahman}.
\newpage

\dev{शुक्रम्}-\emph{Bright,atrractive} \emph{Ātma} is pure, without any imperfections, without any impurities. Here, impurity or \emph{`aśuddhi'} refers to the property of an object to cause a feeling of good, bad or disgust towards it. The objects of the Universe, as they are able to cause feelings of happiness when attained, of sorrow when one is unable to attain and of disgust when one has no liking for it are called \emph{`aśuddha'}.On the other hand, `\emph{Ātma}' is `\emph{śuddha} /pure' i.e. free from impurities, free from any dualities. It is \emph{ānandasvarūpa}, the supreme bliss. Hence, it causes neither a sense of pleasure nor a sense of pain or disgust. Instead, everyone feels drawn towards it. It is the supreme attraction for all beings. Hence, one of the names of \emph{Brahman} is `\emph{Kṛṣṇa} -all attractive'. Because, \emph{Ātma} is free from any imperfections, it is `pure' and hence `bright and attractive'.
\vskip 1.5pt

\dev{अकायम्}-\emph{without body.} Here \emph{Ātma} is being described as ``bodiless''. Every person has 3 bodies or \emph{Śarīra-s}: \emph{sthūla śarīra} (gross body), \emph{sūkṣma / liṅga} \emph{śarīra} (subtle body) and \emph{kāraṇa} \emph{śarīra} (causal body). Even though \dev{अकायम्} refers to the absence of all the three bodies for \emph{Ātma}, in this verse it specifically refers to the absence of \emph{sūkṣma śarīra}, the subtle body (as absence of gross and causal body is described by words ``without muscles'', ``without sin'', etc.).
\vskip 1.5pt

Subtle body (2) is composed of the five great elements called ``\emph{pañca mahābhūta-s} (3)'' present in a state before they undergo \emph{pañcīkaraṇa} (4) (grossification/quantuplication process). It comprises of seventeen items called ``\emph{tattva-s}''- \emph{pañca jñānendriya-s} (five faculties of perception) (5), \emph{pañca karmendriya-s} (five faculties of action) (6), \emph{pañca prāṇa-s} (five kinds of internal forces)(7), \emph{antaḥkaraṇa} comprising of \emph{manas} (mind) and \emph{buddhi} (intellect). Here, the \emph{Upaniṣad} is describing \emph{Ātma} to be without subtle body i.e. without the faculties of perception and actions and without the mind, intellect and the elements. Subtle body and gross body are instruments to experience the fruits of \emph{Karma}. \emph{Ātma} being without these bodies is untouched by \emph{karma-s} and hence untouched by \emph{Māyā}.
\vskip 1.5pt

\dev{अव्रणमस्नाविरं}-\emph{Without scar, without muscles.} Here by using the terms ``without muscles'' and ``without scar (imperfections)'', the \emph{Upaniṣad} is saying that \emph{Ātma} is without \emph{sthūla śarīra}, the gross body. The \emph{Upaniṣad} first described \emph{Ātma} as being without the subtle body, so it follows that \emph{Ātma} is without gross body as gross body is formed from subtle body through a process of \emph{pañcīkaraṇa}.

The gross body (8) is composed of five great elements which have undergone \emph{pañcīkaraṇa}. It has five organs of perceptions corresponding to the \emph{pañca jñānendriya-s} and five organs of action corresponding to the \emph{pañca karmendriya-s}. These ten organs are together called ``\emph{golaka-s}''. The gross body is subjected to \emph{ṣaḍ vikāra-s} (six modifications) --existing in potential form, birth, growth, transformation, decay and death. By saying \emph{Ātma} is without gross body, \emph{Upaniṣad} is again stressing that, \emph{Ātma} is beyond the modifications of birth and death and it is beyond the influence of \emph{Karma}. Hence, there are no scars, no imperfections in \emph{Ātma}. It is one indivisible whole without the bodies and the limitations placed by the bodies.

\dev{शुद्धमपापविद्धम्}- \emph{Pure, without sin.} As explained before \dev{शुद्धम्} refers to the fact that \emph{Ātma} does not cause pleasure in some, pain and disgust in some others. A \emph{jñānī} who has realized \emph{Ātma} will always feel \emph{ānanda} -the supreme bliss. This is possible because, \emph{Ātma} is beyond the influence of \emph{Karma}. Good and bad, virtue and sin are present only in the manifested universe. Only those who are under the influence of \emph{Karma} commit meritorious or condemnable actions and hence, face corresponding karmic results. \emph{Ātma} being beyond the reach of Karmic influence is called ``\dev{अपापविद्धम्}''-without sin. \emph{Manusmṛti} explains that there are three kinds of actions which are considered as \emph{pāpam} /sin: condemnable actions committed through body like committing murder, stealing, etc., condemnable actions committed through speech like speaking falsehood, backbiting, etc., and condemnable actions committed through mind like desiring others wealth, thinking bad about others, etc (9). \emph{Ātma} is pure and without sin because it is devoid of \emph{Karma}, i.e. \emph{Ātma} does not have \emph{kāraṇa śarīra} -the causal body. Causal body (10) is the store house of one's \emph{Karma-s}. It is called ``\emph{kāraṇa śarīra}'' because it causes the creation of subtle and gross bodies. Hence, \emph{Ātma} which is without the gross, subtle and casual bodies is pure and untouched by \emph{Karma}.

\dev{कविर्मनीषी} - \emph{The Seer and the Thinker (the wise one)}. The \emph{seer} is the one who perceives \emph{pāramārthika satya} alone. He does not perceive the universe through faculty of mind, but he `\emph{sees'} them, first hand, i.e. through direct illumination/knowledge. On the other hand, the \emph{thinker} is the one who works through his mind. He understands the whole universe by perception through his mind. Hence, he perceives \emph{vyāvahārika satya}. The seer perceives oneness by direct illumination and the thinker perceives multiplicities and understands everything through indirect knowledge.

Here, \emph{Ātma} is being described as the \emph{seer} who perceives everything, the past, the present and the future and as the \emph{thinker} who knows everything, every aspect, every form, and every name of this universe of multiplicities. It is the same \emph{Brahman} who exists in \emph{pāramārthika daśā} and manifests the apparent \emph{vyāvahārika daśā} and hence, it is He alone who perceives Himself to be one undivided whole and also perceives Himself as millions of names and forms. Hence \emph{Brahman} is called both the seer and the thinker, the one who perceives the \emph{pāramārthika} and \emph{vyāvahārika satya} simultaneously.

\dev{परिभूः}- \emph{Transcendant.} Here, \emph{Brahman} is described as transcendent, i.e. being above everything else. In the beginning of the verse, \emph{Brahman} was declared to be pervading everything, being present everywhere and now the \emph{Upaniṣad} declares that \emph{Brahman} is above everything else. That is, thought \emph{Brahman} pervades everything in the universe as its substratum, the multiplicities of names and forms are not its true nature. These names and forms are merely appearances that come and go and \emph{Brahman} in reality is verily the \emph{existence absolute} and hence beyond impermanent appearances.

\dev{स्वयंभूः}- \emph{Self Existing.} \emph{Ādi Saṅkarācārya} defines \dev{स्वयंभूः} as He who exists everywhere on his own (11). This is one of the very significant descriptions of the nature of \emph{Brahman}. In the manifested universe, for the production of any new thing, one always requires some raw-materials from which the final product is achieved. Every object in the universe undergoes through the cycle of birth, growth, decay and death. Hence, there is a beginning and an end. On the other hand, \emph{Brahman} is \emph{self-existing}. He has no beginning and hence no end. He exists always on His own. As \emph{Brahman} is not bound by the cycle of birth and death, He needs no other object to support himself. He exists on his own, \emph{alone} as \emph{existence absolute} --the only reality that exists.

\dev{याथातथ्यतः अर्थान्व्यदधात् शाश्वतीभ्यः समाभ्यः}- \emph{Who has duly, according to their nature, allotted the respective duties to the eternal time principles.} This can be understood in two ways. The term \dev{अर्थान्} can be taken to mean `\emph{duties'} or `\emph{objects'}. When the meaning of ``duties'' is taken, the \emph{Upaniṣad} is saying that it is \emph{Ātma} who has duly allotted their respective duties to the \emph{Prajāpati-s}, the eternal time principles. \emph{Prajāpati-s} are the time-principles called ``years'' whose function is the creation of various objects and various realms in the universal movement. And the \emph{Ātma}, by allotting the respective duties to various \emph{Prajāpati-s} accomplishes the tasks of creation, sustainment and destruction of various objects of the universe. When \dev{अर्थान्} is taken to mean `\emph{objects'}, the meaning of the verse would become ``It is \emph{Ātma} who has duly ordered the objects of the universe according to their inherent nature from time immemorial''. In other words, in whatever way one interprets the verse, the \emph{Upaniṣad} is saying that it is \emph{Brahman} who has created and ordered the whole universe and all the objects inside it according to their respective inherent nature. Hence, the \emph{Upaniṣad} is declaring that \emph{Brahman} itself is the material as well as the intelligent cause of the \emph{Universe}.

\textbf{Summary:} After speaking about the \emph{jñānī} and the state of \emph{jñāna}, the \emph{Upaniṣad} returns to describing the nature of \emph{Brahman}. In this verse, it makes certain important declarations about the nature of \emph{Brahman}. It describes \emph{Brahman} as being without the \emph{śarīra-s} (bodies). Every \emph{jīva}, who is in the \emph{saṁsāra} has three kinds of bodies- the gross, the subtle and the causal termed as ``\emph{sthūla}'', ``\emph{sūkṣma/ liṅga}'' and ``\emph{kāraṇa śarīra-s}''. The causal body is the storehouse of all the \emph{Karma-s} (12) ever done by a \emph{jīva} in all the lives. It is the causal body that gives rise to the subtle and the gross bodies of an individual in order to make the \emph{jīva} experience the fruits of his \emph{Karma-s}. Hence, the causal body is called the ``\emph{kāraṇa śarīra}'' meaning that which causes the rise of other bodies, or that which causes an Individual to faces his Karmic fruits. The subtle and the gross bodies are merely an instrument, through which a \emph{jīva} experiences his Karmic fruits. In other words, one who has the \emph{śarīra-s} is subjected to the Karmic cycle of action and result. He is always under the influence of \emph{Karma}, bound to it. But, the \emph{Upaniṣad} declares that \emph{Brahman} is `without bodies', i.e. \emph{Brahman} is not under the influence of \emph{Karma} or \emph{avidyā}. Instead, \emph{Brahman} is \emph{absolute, ever-free existence}, without any imperfections, without any divisions and without any movement. Even though He pervades everything, in this \emph{saṁsāra}, yet He is not bound by the limitations of \emph{saṁsāra}. He is above all limitations. Hence, the \emph{Upaniṣad} calls him as \emph{transcendent}.

Further, as \emph{Īśvara}, it is \emph{Brahman} alone who has created all the objects and ordered them according to their natures, giving them their rightful place in His scheme of cosmic play. The space and time principles that govern the \emph{saṁsāra}, is created and given their duties by \emph{Brahman} alone. He is both the \emph{Seer} and the \emph{Thinker} as He alone perceives both the \emph{pāramārthika satya} and the \emph{vyāvahārika satya}. As a Seer, He perceives only Himself. There is nothing other than Him. It is He alone, who has manifested the apparent names and forms and inhabits each of such name and form. And as a thinker, He is all-knowing. Even in the \emph{vyāvahārika daśā}, it is He alone who knows everything. In short, the \emph{Upaniṣad} describes \emph{Ātma} as omniscient and transcendent existence (\emph{Sat}) that pervades everything.

\section*{References}

\begin{enumerate}
\itemsep=0pt
\item
  \emph{Kaṭhopaniṣad} 2. 1. 11.
\item
  \emph{Ātmabodha} Verse 13 and \emph{Tattvabodha} Verse11.2
\item
  The \emph{pañca mahābhūta-s} refers to five elements- earth (\emph{pṛthvī}), water (\emph{āpa}), fire (\emph{agni}), air (\emph{vāyu}) and space (ākāśa). The elements represent the states of matter. The earth represents the solid state; the water represents the liquid state. The Fire represents the force that brings about change from one state to another; from solid to liquid and vice versa. The space is where all the objects in different states exist. The whole of the physical universe is made of these five elements that had undergone \emph{pañcīkaraṇa} - grossification/quantuplication process.
\item
  \emph{Pañcīkaraṇa} is the process of grossification/quantuplication from which the physical universe is created. The five great elements in subtle form undergo \emph{pañcīkaraṇa} process and create the gross elements and hence the physical universe. The \emph{tamas} portion of each of the five subtle elements divides itself into two halves. The one half is reserved for that element and the other half is further divided into four parts to be associated with the other elements. Hence, each gross element is formed by combination wherein its one half is made up of its own element and the other half consist of one-eighth of all the other four elements. \emph{Pañcadaśī} Verse no-26-27.
\item
  The five faculties of perception are- \dev{श्रोत्रं} (faculty of hearing) \dev{त्वक्} (of touch) \dev{चक्शुः} (of sight) \dev{रसना} (of taste) \dev{घाणम्} (of smell). These faculties exist in subtle body; whereas their corresponding organs- ears, skin, eyes, tongue, and nose exist in physical body. \emph{Tattvabodha} Verse 11.3 and 11.5
\item
  The five faculties of action are- \dev{वाक्} (Faculty of speech) \dev{पाणि} (of grasping) \dev{पाद} (of moving) \dev{पायु} (of excretion) \dev{उपस्थ्म्} (of procreation). These faculties exist in Subtle body, where as their corresponding organs- vocal-chord, hands, feet, anus and genitals exist in physical body. \emph{Tattvabodha} Verse 11.6 and 11.8
\item
  \emph{Pañcaprāṇa-s} are- \dev{प्राण:} (force that causes respiration) \dev{अपान:} (that causes excretion) \dev{व्यानः} (that causes circulation) \dev{उदानः} (at death that carries subtle body out of physical body) \dev{समानः} (that causes digestion and assimilation). Tattva Bodha 14.3
\item
  \emph{Ātmabodha} Verse 12 and \emph{Tattvabodha} Verse 10.2
\item
  \emph{Manusmṛti} Verses 12.5-7
\item
  \emph{Ātmabodha} Verse 14 and \emph{Tattvabodha} Verse 12.2
\item
  \emph{Upaniṣad} \emph{Bhāṣya} of \emph{Ādi Śaṅkarācārya} on \emph{Īśopaniṣad}.
\item
  A person has 3 types of \emph{Karma}- \emph{sañcita}, \emph{āgāmī} and \emph{prārabdha}. \emph{Sañcita} \emph{Karma} refers to all the \emph{Karma-s} ever performed by an Individual in general and to those \emph{Karma-s} which are yet to bear fruit in particular. \emph{Āgāmī Karma} refers to the \emph{Karma-s} performed by an Individual in his present life adding to his account of \emph{sañcita karma}. And \emph{Prārabdha} \emph{Karma} refers to that portion of \emph{sañcita Karma-s} that have been allotted to the present life of an Individual to face its karmic fruits.
\end{enumerate}


