\chapter{Verse 10}

\begin{moolashloka}
\dev{अन्यदेवाहुर्विद्यया अन्यदाहुरविद्यया ।}\\
\dev{इति शुश्रुम धीराणां ये नस्तद्विचचक्षिरे ॥ 10 ॥}
\end{moolashloka}

\textbf{Word to word Meaning}: Distinct (\dev{अन्यत्}) verily (\dev{एव}) they say (\dev{आहु}) from Vidya (\dev{विद्यया}); distinct (\dev{अन्यत्}) they say (\dev{आहु}) from \emph{Avidyā} (\dev{अविद्यया}). This (\dev{इति}) we have heard (\dev{शुश्रुम}) from the wise (\dev{धीराणां}) who (\dev{ये}) to us (\dev{नः}) about that (\dev{तत्}) have explained (\dev{विचचक्षिरे}).

\textbf{Meaning:} Distinct verily they say, (the results obtained) from \emph{vidyā} (\emph{upāsanā}); distinct they say, (the results obtained) from \emph{avidyā} (\emph{upāsanā}). Thus (i.e. the teaching) we have heard from the wise, who have explained to us about that (i.e. about the \emph{devatopāsana} and \emph{karmānuṣṭhāna}).

\textbf{Analysis:} Having stated that \emph{sakāma karma} or \emph{apara-bhakti} practiced alone will lead to more bondage to \emph{saṁsāra}; the \emph{Upaniṣad} now further elaborates about why it is so.

\dev{अन्यदेवाहुर्विद्यया}-\emph{Distinct verily, they say, (is obtained) from vidyā.} The \emph{Upaniṣad} is saying that, distinct is the result obtained, when one performs \emph{vidyā upāsanā} i.e. devatopāsana. \emph{Bṛhadāraṇyakopaniṣad} says that through \emph{vidyā} one attains \emph{devaloka} (1). A person who performs worship of a deity, he attains the realm of the deity. Hence, a person who desires to attain \emph{devaloka} must perform the \emph{upāsanā} of his \emph{iṣṭa devatā} (his beloved deity). In other words, a person who performs \emph{vidyā upāsanā} in a \emph{sakāma} way (with a desire to attain the result), he will attain \emph{devaloka}. On the other hand, those who practice \emph{devatopāsana} out of pure love for the deity will develop a sense of surrender and sacrifice (\emph{samarpaṇa bhāva}) and will achieve `\emph{ekāgra citta}'- One pointed concentration of the mind.

As noted in the previous verse, a householder is enjoined by the \emph{śāstra-s} to attain all the 3 worlds. Hence, a person who practices \emph{sakāma vidyā upāsanā} alone, while completely giving up \emph{karmānuṣṭhāna}, will only increase his bondage to \emph{saṁsāra}.

\dev{अन्यदाहुरविद्यया} - \emph{Distinct verily, they say, (is obtained) from avidyā.} Those people who perform \emph{avidyā} \emph{upāsanā} i.e. practice of \emph{Karma-s} according to \emph{śāstra-s}, they attain a result different than that attained by those who perform \emph{devatopāsana}. People who practice rituals that are prescribed in the scriptures in a \emph{sakāma} way and lead their life practicing only the \emph{vihita karma-s} and avoiding the \emph{niṣiddha karma-s}, such people attain `\emph{pitṛ loka}', the realm of the manes, the ancestors. That is, a person who desires to attain `\emph{pitṛ loka}' must perform \emph{vihita karmānuṣṭhāna} (in \emph{sakāma} way) (2). On the other hand, those who perform \emph{vihita karma} with a sense of duty will develop \emph{niṣkāma daśā} (detachment).

\dev{इति शुश्रुम धीराणां ये नस्तद्विचचक्षिरे}-\emph{Thus (i.e. the teaching) we have heard from the wise, who have explained to us about that.} The \emph{Upaniṣad} is saying here that, the teaching, about \emph{karmānuṣṭhāna} and \emph{devatopāsana} leading to different results has been taught by all those wise people who have travelled this path and have attained the ultimate \emph{jñāna}.

\textbf{Summary:} Continuing with the explanations given in the previous verse, the \emph{Upaniṣad} here explains further about why practice of \emph{karma} (\emph{vihita}) or \emph{bhakti} (\emph{apara}) alone will lead to more bondage to \emph{saṁsāra}. It notes that the results obtained from \emph{karmānuṣṭhāna} are distinct from the results obtained from the \emph{devatopāsana}. Therefore, a person who wishes to achieve spiritual progress can do so only by simultaneous practice of \emph{vidyā} and \emph{avidyā upāsanā}. A practice of \emph{vihita karma} alone will lead to \emph{pitṛ loka} or when done with a sense of duty, it will lead to \emph{niṣkāma daśā}. Similarly, a practice of \emph{devatopāsana} alone will lead to \emph{deva loka} or when done with love and surrendering, it will lead to \emph{ekāgra citta}. But, without the practice of both together, one will neither achieve purification of mind (\emph{citta śuddhi}) that would lead to \emph{jīvanmukti}, nor will he transcend the limitations imposed by the physical universe and travel the northern path (3) towards \emph{kramamukti}. Hence, the \emph{Upaniṣad} is suggesting a person in a \emph{pravṛttimārga}, a householder, to practice both \emph{vidyā} and \emph{avidyā} \emph{upāsanā}.

\section*{References}

\begin{enumerate}
\item
  \emph{Bṛhadāraṇyakopaniṣad} 1.5.16.
\item
  \emph{Bṛhadāraṇyakopaniṣad} 1.5.16.
\item
  Northern path refers to \emph{devayāna}.
\end{enumerate}


