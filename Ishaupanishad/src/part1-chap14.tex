\chapter{Verse 13.}

\begin{moolashloka}
\dev{अन्यदेवाहु: सम्भवात् अन्यदाहुरसम्भवात् ।}\\
\dev{इति शुश्रुम धीराणां ये नस्तद्विचचक्षिरे ॥ 13 ॥}
\end{moolashloka}

\textbf{Word to word Meaning}: Distinct (\dev{अन्यत्}) verily (\dev{एव}) they say (\dev{आहु}) from the manifested (\dev{सम्भवात्}); distinct (\dev{अन्यत्}) they say (\dev{आहु}) from the unmanifested (\dev{असम्भवात्}). This (\dev{इति}) we have heard (\dev{शुश्रुम}) from the wise (\dev{धीराणां}) who (\dev{ये}) to us (\dev{नः}) about that (\dev{तत्}) have explained (\dev{विचचक्षिरे}).

\textbf{Meaning:} Distinct verily they say, (the results obtained) from (the worship of) the manifested (i.e.\ \emph{Kārya Brahman}); distinct they say, (the results obtained) from (the worship of) the unmanifested (i.e.\ \emph{Kāraṇa Brahman}/ \emph{mūla prakṛti}). Thus (i.e.\ the teaching) we have heard from the wise, who have explained to us about that (i.e.\ about the worship of \emph{Kārya} and \emph{Kāraṇa Brahman}).

\textbf{Analysis:} The \emph{Upaniṣad} continues with its exposition about the worship of \emph{Kārya} and \emph{Kāraṇa Brahman}.

\dev{अन्यदेवाहु: सम्भवात् अन्यदाहुरसम्भवात्} - \emph{Distinct verily, they say (is obtained) from (the worship) of the manifested; distinct (is obtained) from (the worship) of the unmanifested.} The worship of the \emph{Kārya Brahman} leads a person to acquire control over various aspects of nature and attain powers like \emph{aṇimā, garimā}, \emph{mahimā}, etc.\ Further, it facilitates one to attain various higher realms and finally reach `\emph{Hiraṇyagarbha} i.e.\ \emph{Brahmaloka}. On the other hand, the worship of \emph{Kāraṇa} \emph{Brahman}, i.e.\ the \emph{mūla prakṛti} will result in a person to lose his Individuality (subtle and gross) and get absorbed into the unmanifested \emph{prakṛti}.

Hence, the results obtained from the \emph{saṁbhūti} and \emph{asaṁbhūti} are distinct from each other. The former leads to the attainment of knowledge and control over subtle realms and the latter leads to temporary attainment of unmanifested state. Due to this, a person who practices any one by giving up the other will not achieve \emph{mokṣa} but instead will be stuck in \emph{avidyā}.

\dev{इति शुश्रुम धीराणां ये नस्तद्विचचक्षिरे}-\emph{Thus (i.e.\ the teaching) we have heard from the wise, who have explained to us about that.} The \emph{Upaniṣad} is saying here that, the teaching, about the worship of \emph{Kārya Brahman} and \emph{Kāraṇa Brahman} leading to different results has been taught by all those wise people who have travelled this path and have attained the results the paths lead to.

\textbf{Summary:} In the previous verse, the \emph{Upaniṣad} explained how a person who practices the worship of the manifested universe or the unmanifested \emph{prakṛti} in isolation to the exclusion of the other will increase his bondage to subtle existence and hence is unable to transcend the \emph{vyāvahārika daśā} and reach the \emph{pāramārthika} state. In this verse, the \emph{Upaniṣad} further reiterates that the practice of either the worship of \emph{saṁbhūti} or that of \emph{asaṁbhūti} alone will not help a spiritual aspirant as the fruits given by them are distinct. The worship of manifested gives a distinct fruit and the worship of \emph{Kāraṇa} \emph{Brahma} gives a distinct fruit.


