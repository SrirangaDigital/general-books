\chapter{Verse 1}

\begin{moolashloka}
\dev{ईशा वास्यमिदँ सर्वं यत्किञ्च जगत्यां जगत् ।}\\
\dev{तेन त्यक्तेन भुञ्जीथा मा गृधः कस्यस्विद्धनम् ॥ 1 ॥}
\end{moolashloka}

\textbf{Word to Word Translation}: By God (\dev{ईशा}) is inhabited here everywhere (\dev{वास्यमिदँ सर्वं}), whatever is there (\dev{यत्किञ्च}), all the created objects in this creation (\dev{जगत्यां जगत्}). Renouncing that (\dev{तेन त्यक्तेन}), one must enjoy/protect (\dev{भुञ्जीथा}); one must become dispassionate (\dev{मा गृधः}), to whom does the wealth belong? (\dev{कस्यस्विद्धनम्}).

\textbf{Meaning}: All the objects in this cosmos, all that exists in the entire creation are inhabited by \emph{Brahman} (as their innermost-Self/\emph{Ātma}). One must renounce (attachment to sensory world, attachments to names and forms) and enjoy (the bliss of \emph{Ātma}) by developing dispassion towards worldly objects as they do not belong to anyone (but \emph{Brahman}, the cause and ruler of universe).

\textbf{Analysis}: The \emph{Upaniṣad} begins with a bold and crystal clear declaration that ``God/ \emph{Īśvara} inhabits everywhere and those who wish to realize Him and enjoy the bliss of Him, must renounce attachment to the sensory world''.

\dev{ईशा}- \emph{Lord}. The word ``\emph{Īśā}'' means `He who rules'. What does He rule? He being the creator of the whole cosmos, rules everything, every object, phenomenon, realms that exist in this universe. God is both the \emph{upādāna kāraṇa} (material cause) and the \emph{nimitta kāraṇa} (intelligent cause) for the universe. Being \emph{upādāna kāraṇa}, God himself becomes the \emph{jagat} and exists in all the objects of the manifested universe as their \emph{Ātma}-true Self. Being \emph{nimmita kāraṇa}, He rules and controls all the activities that happen in the universe. Hence, the name- ``\emph{Īśā}''-the one who is the creator and controller of the universe.

\dev{वास्यम्}- \emph{Inhabits or covers}. \emph{Brahman} being the \emph{Ātma} of all manifested objects inhabits everywhere. The world as it appears-the multiplicities of names and forms (\emph{nāmarūpa}), the diversities are all only an apparent reality, a temporary appearance. It is \emph{Brahman} who is one, appears as many by his power of \emph{Māyā}. Just as waves rise and fall back into ocean, so also the numerous names and forms come out and fall back into \emph{Brahman}. It is \emph{Brahman} alone who exist beneath every form, behind every name. He inhabits every entity in this universe. An individual/\emph{jīva} who is in \emph{vyāvahārika} state --a state of duality, a state under the influence of \emph{Māyā}-- will not be able to recognize this \emph{Brahman} who is the one eternal (\emph{nitya}) behind all temporary (\emph{anitya}) manifestation. But, such a \emph{jīva} when he achieves \emph{cittaśuddhi} --purification of his mind/intellect-- he will understand that it is the God alone, who inhabits the whole cosmos as their indwelling \emph{Ātma}.

\dev{इदं सर्वं} - \emph{Here everywhere}. \emph{Brahman} or \emph{Īśvara} is not someone who is sitting high up in the heaven or atop some mountain. He is present ``\emph{here}'', in this universe, the air, the water, the fire, the mud everything that has life and everything that does not have life. \emph{Brahman} is present in every object, every physical state (1); every state of mind (2); every realm of existence (3). The word ``\dev{इदं}'' signifies that wherever a person may be, he will find \emph{Brahman} there. When the \emph{asura} king \emph{Hiraṇyakaśipu} addressed his son \emph{Prahlāda} thus-

``\emph{O most unfortunate Prahlāda, you have always described a supreme being other than me, a supreme being who is above everything, who is the controller of everyone, and who is all-pervading. But where is He? If He is everywhere, then why is He not present before me in this pillar?}'' (4)

The all-pervading \emph{Īśvara} then took a form of Lord \emph{Narasiṁha}, half human and half lion and came out of the very same pillar and killed the \emph{asura}. The gist of this story is explained in the \emph{Upaniṣad} by the words ``\dev{इदं सर्वं}''- that God is present everywhere. It only takes a \emph{jīva} to develop the power of \emph{viveka} (discrimination) to see Him.

\dev{यत्किञ्च जगत्यां जगत्}- \emph{Whatever are there, all the objects in this cosmos}. \dev{जगत्} refers to ``\emph{universal motion} (5)''. The word signifies the fact that, the whole cosmos, the entire manifestation is always in a state of motion- either in the state of evolution (\emph{sṛṣṭi}) or in the state of dissolution (\emph{laya}). \dev{जगत्यां} refers to all the objects, all the individual movements that occur inside in this universal motion. Inside the whole manifestation, multiple universes arise and die-out, the \emph{yugas} and \emph{kalpas} begin and end. Inside each of such universes, the \emph{jīva}'s take birth, undergo many modifications and then die. Hence, there is always movement in every object, every realm, every universe present in this manifestation, which itself is in continuous motion. So, the \emph{Upaniṣad} is declaring that, the \emph{Īśvara} who is the source and creator of the entire movement also inhabits each and every movement. Every object exists in Him in potential form, then takes birth from Him and undergoes modifications in Him and finally dissolves back into Him. He is present everywhere, in every object- small or big, living or non-living, matter or energy. From the tiniest molecule to the sum-total of all manifestation, it is \emph{Īśvara} alone who exists. He is the one \emph{Ātma} who is behind all names and forms.

\dev{तेन त्यक्तेन भुञ्जीथा}- \emph{Renouncing which, one must enjoy (or protect)}. A person, who has achieved \emph{citta-śuddhi} and has developed \emph{viveka}-\emph{śakti}, will discriminate between the real and the apparent, between the eternal and the temporary. Such a person will come to an understanding that the external sensory world being a temporary manifestation can only offer temporary happiness. An individual attached to such sensory objects out of desire for the pleasures it offers is very likely to end up in sorrow and frustration. Knowing thus, he will renounce the world. What does renouncing mean? Does it mean non-action or running away from one's duties? The answer is No. ``\emph{Tyāga}'' does not mean abandoning actions or duties. It means giving up the attachment to sensory objects. An Individual will have to not only give up the fruits of actions but also the sense doership of actions. A true \emph{tyāga} is when a person does \emph{karma} by surrendering- the action, the doership and the fruits- all the three to \emph{Īśvara}.

Why will a person renounce the worldly objects? He does it, so that he can enjoy the bliss of \emph{Brahman}, who is one's own innermost \emph{Ātma}. Alternately, one renounces the worldly objects to protect \emph{Ātma}. ``\dev{भुञ्जीथा}'' can be understood as enjoying the bliss of Self by becoming established in one's true Self or as protecting the Self by being established in one's true Self. A person having thus renounced the worldly attachments will contemplate on his own \emph{Ātma} and try to achieve \emph{ātma-jñāna} and attain \emph{mukti} or freedom (6), the ultimate goal of life. Hence, the \emph{Upaniṣad} is saying that, the discriminating person who understands that it is God alone who inhabits every-object in this cosmos as their innermost Self, will renounce his attachments to the external names and forms and instead engages himself in enjoying the bliss of \emph{Ātma}.

\dev{मा गृधः कस्यस्विद्धनम्}- \emph{Becoming dispassionate towards wealth as it belongs to none}. In what manner should a person renounce the world? As explained earlier, one must only give up the attachment to the objects in the world. This is being described here as ``\dev{मा गृधः}'' to be disinterested/dispassionate. ``\dev{धनम्}'' refers to not only the material wealth, but also to all the pleasures and temptations that sensory world offers. All these pleasures and wealth does not belong to any as nobody can possess them forever. These objects are all nothing but temporary projections of \emph{Māyā}. People waste away their entire life running after the material benefits and may end up losing even that which they had in the beginning. And even if they attain more wealth, it is all relevant only for this one life which itself is fleeting. Hence an individual must realize that these material objects, the wealth and prosperity belongs to none but \emph{Īśvara}, who either gives it to individuals or takes it away from them based on the respective \emph{karmas} of the individuals. Such a person will develop \emph{vairāgya}/dispassion towards such objects and renounce the attachments to them.

\textbf{Summary}: The \emph{Upaniṣad} speaks in the very first verse about the path that leads to \emph{mokṣa}, the `path of \emph{nivṛtti'}. \emph{Nivṛtti} literally means ``turning away'' (from sensory objects). A person in this path turns away from sensory pleasures towards the inner Self so as to achieve \emph{mokṣa}. A person desiring to achieve \emph{mokṣa}- the freedom from the karmic cycle of birth and death must first achieve \emph{ātma-jñāna} because only \emph{ātma-jñāna} will lead to \emph{mokṣa}. Such a person must first develop four-fold qualities of \emph{viveka}, \emph{vairāgya}, \emph{ṣaṭka}-\emph{sampatti} and \emph{mumukṣutva} termed as ``\emph{Sādhanā - Catuṣṭaya} (7)'' meaning ``four-fold competencies required for \emph{jñāna}- \emph{sādhanā}''.

Viveka represents the knowledge to differentiate between \emph{Nitya} \emph{Brahman} which is eternal self-existing reality without a beginning or an end and the \emph{anitya} \emph{jagat} which is the ever changing world continuously undergoing cycles of creation and destruction. A person must first learn to discriminate between the two. \emph{Vairāgya} refers to dispassion/detachment towards the sensory objects that arise as a consequent to \emph{Viveka}. When a person realizes that sensory objects give only temporary happiness and not eternal contentment, then he develops dispassion towards those objects. Only \emph{vairāgya} can make a person pursue the path towards \emph{Brahman}.

\emph{Ṣaṭka}-\emph{sampatti} refers to the six fold qualities of \emph{śama}, \emph{dama}, \emph{uparati}, \emph{titikṣā}, \emph{śraddhā} and \emph{samādhāna}. \emph{Śama} refers to control of mind and \emph{dama} refers to control of the five senses. Only when one develops sufficient restraint over the mind and the senses that he will be able to withdraw the mind from the worldly objects and instead direct it towards \emph{Īśvara}. \emph{Uparati} comes when \emph{śama} and \emph{dama} are perfected. \emph{Uparati} is a state when the mind remains drawn away from the external world and remains fixed on \emph{Īśvara} but with effort. \emph{Titikṣā} refers to forbearance. It is a calm and controlled state of mind wherein a person patiently endures all the pain and the sufferings that life throws at him without letting his mind become disturbed in anyway, nor giving in to the feelings of anger or revenge. This is possible only when one gets an understanding that happiness and sorrow are results of one's own past actions and hence it is futile to blame others. \emph{Śraddhā} refers to faith in scriptures and in one's Guru. \emph{Samādhāna} refers to the capacity for one-pointed spontaneous fixing of mind on \emph{Brahman}. Finally \emph{mumukṣutva} refers to the burning desire for \emph{mokṣa}.

A person who has developed \emph{viveka} that, it is \emph{Brahman} alone who pervades the whole cosmos, the One reality behind all names and forms, will embrace the path of \emph{nivṛtti} in order to achieve the bliss of \emph{Ātma} (i.e. \emph{ātma-jñāna}).

But, in what manner should a person, having developed four-fold competencies practice the path of \emph{nivṛtti}? The \emph{Upaniṣad} answers- ``\dev{त्यक्तेन भुञ्जीथा}'', that is ``renounce the world and enjoy the Self''. The first step is to renounce the path of \emph{pravṛtti}. The sensory attachments and the desires must be completely given up. The second step is to ``enjoy the Self''. How will a seeker ``enjoy the Self''? The \emph{śāstra-s} explain that to enjoy the Self, one must contemplate on the Self, the Ātma. One must practice \emph{jñāna}-\emph{sādhana} also called \emph{śravaṇa}-\emph{catuṣṭayaṃ} (the four fold steps starting with hearing). It includes- \emph{śravaṇa}, \emph{manana}, \emph{nididhyāsana} and \emph{ātma}-\emph{sākṣātkāra}.

\emph{Śravaṇa} means `Hearing'. It refers to listening to the instructions of one's \emph{guru}. A student must sincerely and with complete faith listen to and understand all the instructions and teaching that the \emph{guru} is imparting. Then comes \emph{manana}-the reflection. A student must continuously involve himself in reflecting on the teachings of his \emph{guru} with an intention to gain understanding into the multiple layers of meaning and then he should get solved any doubts that may arise in him. \emph{Nididhyāsana} refers to `Contemplation'. After intellectually understanding the \emph{guru}'s words, the student must practice \emph{Nididhyāsana} as instructed by his \emph{guru}. A student who thus practices śravaṇa, \emph{manana}, and \emph{nididhyāsana} will attain \emph{ātma-jñāna}. He will become \emph{jīvanmukta}, the ever free, the one who is always enjoying the bliss of \emph{Ātma}.

\section*{References}

\begin{enumerate}
\item
  Physical state- solid, liquid, gaseous.
\item
  State of mind- wakefulness, dreaming, deep sleep.
\item
  The seven higher \emph{lokas}- \emph{bhū}, \emph{bhuvaḥ}, \emph{svaḥ}, \emph{mahaḥ, janaḥ, tapaḥ,} and \emph{satya}. The seven lower \emph{loka}s- \emph{atala, vitala, sutala, talātala, mahātala, rasātala,} and \emph{pātāla}.
\item
  \emph{Śrīmad Bhāgavatam} 7.8.12
\item
  The movement referred here is called \emph{vikāra} or modifications. Any manifestation proceeds through six types of modifications called `\emph{vikāra'}. They are six in number- \emph{asti} (existing in potential form), \emph{jāyate} (birth), \emph{vardhate} (grows), \emph{vipariṇamate} (transforms), \emph{apakṣīyate} (decays) and \emph{vinaśyati} (perishes). {[}See \emph{Tattvabodhaḥ}, Verse 3.1{]}
\item
  \emph{Mukti} can be of two types- \emph{jīvanmukti} and \emph{videhamukti}. \emph{Jīvanmukti} refers to being free even while living in the body. \emph{Videhamukti} refers to the absolute merging with \emph{Brahman} by fully discarding the body.
\item
  \emph{Vivekacūḍāmaṇi}, Verse 17
\end{enumerate}
