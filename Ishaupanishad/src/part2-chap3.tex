\chapter[Section 3 \emph{Pravṛttimārga}: The Path to \emph{Kramamukti}]{Section 3\\ \emph{Pravṛttimārga}: The Path to \emph{Kramamukti}}
\lhead[\small\thepage]{\small \emph{Pravṛttimārga}: The Path to \emph{Kramamukti}}

\section*{3.1. Introduction}

`\emph{Pravṛtti}' literally means `to turn towards'. It refers to people who are attached to worldly pleasures. People who have desires for physical, sexual, psychological, worldly and other-worldly happiness are called `\emph{pravṛttimārgī-}s'. Scriptures classify these desires of a worldly man into three categories- desire for wealth, desire for progeny and desire for attaining worlds. These people are also called ``\emph{saṁsārī}'' or ``householders''- the people who are attached to world/\emph{saṁsāra}.

In the world, one thing that every person strives for is to attain `\emph{sukha}' or happiness. Every action of man is directly or indirectly aimed at achieving happiness. People work hard, pursue careers, have family, etc.\ all aimed at fulfilling one's desires. It is for such people that the present \emph{Upaniṣad} instructs to live life for hundred years by performing \emph{karma-}s (1). But, these \emph{karma-}s should be performed by adhering to the tenets of \emph{dharma}. Otherwise, one will never make spiritual progress. Hence, the `\emph{pravṛttimārga}' is the path of \emph{karma}. Scriptures speak of two kinds of activity- \emph{karma} and \emph{upāsanā} through which a person can fulfill his desires and also make spiritual progress.

\emph{Karmānuṣṭhāna} involves practicing one's life according to the tenets of \emph{dharma} and performing all the duties as prescribed in the scriptures. \emph{Upāsanā} on the other hand includes practice of \emph{bhakti} in order to attain higher spiritual goals. It is only by a combination of them either simultaneously or successively (i.e.\ one after the other) that one will be able to achieve spiritual progress. If a person performs \emph{karma} and \emph{bhakti} in a \emph{niṣkāma} way without desiring for the fruits of actions, then he would gain \emph{citta} \emph{śuddhi} / purification of the mind and will become eligible for \emph{jñāna} \emph{sādhanā} and \emph{jīvanmukti}. On the other hand, performance of \emph{karma} and \emph{bhakti} in a \emph{sakāma} way with a desire for fulfillment of material and spiritual goals, one would attain \emph{mokṣa} at the end of cycle of manifestation (2) through the gradual path of \emph{kramamukti}.

Hence, just as the \emph{path of sannyāsa} leads one to \emph{jīvanmukti}, the \emph{path of Karma and upāsanā} leads to \emph{kramamukti}.

\section*{References}

\begin{enumerate}
\itemsep=0pt
\item
  \emph{Īśopaniṣad}- Verse 2
\item
  End of manifestation refers to end of Life of \emph{Brahmā}, when whole \emph{jagat} including the \emph{Mūla}-\emph{Prakṛti} merges back into \emph{Brahman}.
\end{enumerate}
\newpage

\section*{3.2. Nature and Creation of the universe}

It has been stated before, that \emph{Brahman} though without any movement, without any duality in reality, through his \emph{Māyā}, He Himself manifests this universe and then pervades all the objects in it as their innermost Self/\emph{Ātman} which is the witness (i.e.\ \emph{cit} /consciousness) of all activity that happens in them. \emph{Taittirīyopaniṣad} (1) says-

\begin{verse}
\emph{yato vā imāni bhūtāni jāyante}\\
\emph{yena jātāni jīvanti}\\
\emph{yatprayantyabhisamviśanti}\\
\emph{tadvijijñāsasva tadbrahmeti \dev{॥}}
\end{verse}

\emph{``From which all the creatures (bhūta) are born, being born by which they sustain and into which they merge back, Know that as Brahman''.}

The world with all its multiplicities are not separate from \emph{Brahman} but have their origin in Him and they are sustained in Him and finally merge back into Him. \emph{Brahman} is both the material and intelligent cause for this universe. To accomplish any work (Karya) --for example, creating a pot, a base material (soil) called `\emph{upādāna}' in scriptures and a person (the potter) who accomplishes the task called `\emph{kartā}' is needed. \emph{śāstra-s} term these as `\emph{upādāna} \emph{kāraṇa}/material cause' and `\emph{nimitta} \emph{kāraṇa}/intelligent or efficient cause'.

In case of the Universe, \emph{Brahman} is both the \emph{upādāna} \emph{kāraṇa} and the \emph{nimitta} \emph{kāraṇa}. That is, \emph{Brahman} does not create any universe separate from Him, but only has manifested the universe out of His own-Self. In other-words, \emph{Brahman} has himself become the cosmos. If it is asked ``How exactly does \emph{Brahman} manifest the universe?'' The answer is that there are two ways in which an already existing entity (\emph{kāraṇa}/cause) can become another entity (\emph{kārya} /effect) --through \emph{pariṇāma} or through \emph{vivarta}. `\emph{Pariṇāma}' refers to a real transformation of an object into another object, ex. milk turning in curd, mud becoming pot, etc.\ `\emph{Vivarta}' refers to an apparent transformation of one object into another, ex. a rope being mistaken as a snake, a mirage in the desert, etc.

In the case of the universe, the `\emph{pariṇāma} \emph{vāda}' does not hold. If \emph{Brahman} has indeed undergone complete and real transformation and become the cosmos the way milk becomes curd, then there would have been no difference between \emph{Brahman} and \emph{jagat} in the \emph{vyāvahārika} state. Further, in the \emph{vyāvahārika} state, there would have been only the sensory universe and no \emph{Brahman}/\emph{Īśvara}. Just as the milk that turns into curd ceases to be milk and cannot turn back into milk, so also \emph{Brahman} will cease to exist once it transforms completely into universe if such a transformation were a real transformation. However, such a materialistic view is neither supported by Hindu \emph{śāstra-s} nor supported by logic. Hindu \emph{śāstra-s} instead proclaim \emph{Brahman} as being: \emph{nitya} (eternal), \emph{niṣkalaṁ} (without parts/divisions), \emph{niṣkriyaṁ} (without action) \emph{śāntaṁ} (calm/without movement), \emph{niravadyaṁ} (flawless), \emph{nirañjanaṁ} (stainless)(2). \emph{Vedānta} \emph{Darśana} explains that \emph{Brahman} manifests the universe, just as a rope appears as a snake. That is, there is only an apparent transformation of \emph{Brahman} into universe. In the case of snake, rope was the \emph{upādāna} \emph{kāraṇa}/material cause and in the case of universe, \emph{Brahman} is the \emph{upādāna} \emph{kāraṇa}. This phenomenon of \emph{apparent manifestation} is termed as `\emph{Māyā}' and the power of \emph{Brahman} to achieve it is termed as `\emph{Māyā}-\emph{śakti}'. \emph{Ādi} \emph{Śaṇkarācārya} in \emph{Māyā} \emph{Pañcakaṁ} (3) says-

\begin{verse}
\emph{nirupamanityaniraśaṁke'pyakhaṇḍe}\\
\emph{mayi citi sarvavikalpanādiśūnye \dev{।}}\\
\emph{ghaṭayati jagadīśajīvabhedaṃ}\\
\emph{tvaghaṭitaghaṭanāpaṭīyasī māyā \dev{॥}}
\end{verse}
\newpage

\emph{In Brahman who is unique, eternal, without parts, absolute, it is Māyā which creates the distinctions of Man, World and God. There is nothing impossible for Māyā}.

Similarly, Lord \emph{Kṛṣṇa} says (4)-

\begin{verse}
\emph{ajo'pi sannavyayātma bhūtānāmīśvaro'pi san \dev{।}}\\
\emph{prakṛtim svāmadhiṣṭhāya saṁbhavāmyātmamāyayā \dev{॥}}
\end{verse}

\emph{Though I am birthless, un-decaying by nature, and the Lord of beings, (still) by subjugating My Prakṛti, I take birth by means of My own Māyā.}

Hence, it is clear that \emph{Brahman} Himself is both the material and the intelligent cause of the universe and He manifests the universe as an appearance or magic through his power of \emph{Māyā}. That is, there is no real transformation of \emph{Brahman} into Jagat but only an `apparent manifestation'. This \emph{Brahman} in His absolute state devoid of all dualities and all movement is called `\emph{Nirguṇa} \emph{Brahman}' and the same \emph{Brahman} in His state of manifestation when associated with \emph{Māyā} is called `\emph{Saguṇa} \emph{Brahman}'.

In the state of manifestation, \emph{Saguṇa} \emph{Brahman} further appears to exist in two ways- as \emph{Kāraṇa} \emph{Brahman} associated with \emph{Māyā}/\emph{Mūla} \emph{Prakṛti}/unmanifested seed and as \emph{Kārya Brahman} associated with manifested cosmos. The whole universe with its all gross and subtle realms of existence together is called `\emph{saṁbhūtim}' or `manifested cosmos'. The scriptures speak about seven higher \emph{loka-s} (realms of existence) and seven lower \emph{loka-s}. The higher \emph{loka-s} are- \emph{bhū}, \emph{bhuvaḥ}, \emph{svaḥ}, \emph{mahaḥ, janaḥ, tapaḥ,} and \emph{satya}. The lower \emph{loka-s}- \emph{atala, vitala, sutala, talātala, mahātala, rasātala,} and \emph{pātāla}. The \emph{bhū loka} refers to the gross physical universe that we humans live in. \emph{Bhuvaḥ} and \emph{svarloka-s} refer to subtle realms of existence. `Mahat' is the realm of `\emph{Hiraṇyagarbha}' or `\emph{Brahmā}' who is also called `\emph{Prathamaja}' or first-born as he was the very first to be born out of \emph{Mūla} p\emph{rakṛti} or \emph{Kāraṇa} \emph{Brahman}. This \emph{Hiraṇyagarbha} who creates the gross and subtle universes and has this whole manifested cosmos/\emph{saṁbhūtim} as his body and different realms as his limbs is called `\emph{Kārya} \emph{Brahman}'.

However, for this manifested cosmos or \emph{saṁbhūtim} to exist, it must have a source/seed from which it has originated just as a tree that grows out of a seed. This `cause' or \emph{Kāraṇa} from which the cosmos is manifested is called `\emph{Mūla} \emph{prakṛti}' or `\emph{asaṁbhūtim}/unmanifested seed'. Just as the tree stays inside a seed in an unmanifested/potential state, so also the whole \emph{kārya} or `manifested cosmos' exist in an unmanifested, undifferentiated seed state. The three highest \emph{loka-s} (\emph{Janaḥ, Tapaḥ,} and \emph{Satya}) together constitute this \emph{Mūla} \emph{prakṛti} which is also called `\emph{Māyā}' in the \emph{Vedānta} literature. And \emph{Saguṇa} \emph{Brahman} who is associated with this \emph{Māyā} or \emph{Mūla} \emph{prakṛti} is called `\emph{Kāraṇa} \emph{Brahman}'. Hence, \emph{Brahman} who is \emph{Nirguṇa} in His absolute state, exist both as `\emph{Kārya}' and `\emph{Kāraṇa}' in the state of manifestation by inhabiting the whole universal movement, both the ``\emph{saṁbhūtim''} and a\emph{saṁbhūtim} as its innermost Self.

Regarding the creation of the universe, \emph{Manusmṛti} (5) says-

\begin{verse}
\emph{so'bhidhyāya śarīrāt svāt sisṛkṣurvividhāḥ prajāḥ \dev{।}}\\
\emph{apa eva sasarjādau tāsu vīryamavāsṛjat \dev{॥}}\\
\emph{tadaṇḍamabhavad haimam sahasrāṁśusamaprabhaṁ~\dev{।}}\\
\emph{tasmin jajñe svayaṁ brahmā sarvalokapitāmahaḥ \dev{॥}}
\end{verse}

\emph{``He (i.e.\ Brahman) desiring to produce many beings from His own being/body, through Thought/Will first created `apa/waters' (i.e.\ Mūla Prakṛti) and then placed his seed in them.}

\emph{That (i.e.\ seed/vīryam) became a Golden egg of brilliance equal to thousand suns (i.e.\ the Seed manifested into Hiraṇyagarbha). In it (i.e.\ the egg) was born, Brahmā himself, the father, the creator of whole universe}.''

\emph{Brahman} wishing to create universe from His own body, first created \emph{Mūla Prakṛti} (which acts as His body from which everything else originates), and then placed His seed in them (i.e.\ He Himself became associated with it). \emph{Mūla} \emph{Prakṛti} is called `waters' because in it everything exists in undifferentiated state. In other words, \emph{Nirguṇa} \emph{Brahman} with an intention to create universe, first manifested His \emph{Māyā} which acts as His body and then himself became associated with it. \emph{Māyā} is the seed for the objects and \emph{Brahman} who became associated with \emph{Māyā} by inhabiting it (i.e.\ \emph{Kāraṇa} \emph{Brahman}) is the seed for consciousness and intelligence. In other words, \emph{Brahman} through \emph{Māyā}-\emph{śakti} creates the objects of the universe and then through his \emph{cit śakti} inhabits the universe as intelligence.

\emph{Śrī Gauḍapādācārya} (6) says-
\vskip -10pt

\begin{verse}
\emph{prabhavaḥ sarvabhāvānāṁ satāmiti viniścayaḥ \dev{।}}\\
\emph{sarvaṁ janayati prāṇaścetoṁśūnpuruṣaḥ pṛthak \dev{॥}}
\end{verse}
\vskip -10pt

\emph{Origination belongs to all entities that exist. This is a well-established fact. Prāṇa created all (objects); puruṣa creates separately the rays of consciousness.}

Here, `\emph{prāṇa}' refers to \emph{Mūla} \emph{Prakṛti}/\emph{Māyā}. From \emph{Māyā} is created the objects of the manifested cosmos. And from the \emph{puruṣa} who is associated with \emph{Māyā} are created the rays of consciousness that inhabits the manifested cosmos as consciousness and intelligence (i.e.\ as subject).

At the cosmic level, the \emph{jagat} can be divided into \emph{sthūla} (gross), \emph{sūkṣma} (subtle) and \emph{kāraṇa} (causal) states. Likewise, at an individual, the \emph{jīva} has three bodies: gross, subtle, and causal through which it experiences the three respective states of the universe.

\emph{Brahman} when associated with the gross realms alone He is called `\emph{Virāṭ}' or `\emph{Vaiśvānara}''' \emph{Brahman} when associated with both gross and subtle realms, He is called `\emph{Hiraṇyagarbha}', `\emph{Brahmā}', `\emph{Kārya} \emph{Brahman}' or `\emph{Satya} \emph{Brahman}'. And when \emph{Brahman} is associated with \emph{Māyā} or \emph{Mūla} \emph{Prakṛti} which is the seed of the manifested universe, He is called `\emph{Kāraṇa} \emph{Brahman}' or `\emph{Īśvara}' or `\emph{Nārāyaṇa} (6)'. \emph{Brahman} in his absolute state/ \emph{pāramārthika daśā}, without any association with either manifested cosmos or the unmanifested seed, He is called `\emph{Para} \emph{Brahman}' or `\emph{Nirguṇa} \emph{Brahman}''. And the same \emph{Brahman} is called `\emph{Apara} \emph{Brahman}' or `\emph{Saguṇa} \emph{Brahman}' when the reference is made with respect to both \emph{kāraṇa} and \emph{kārya}.

It is important to understand that, various names for \emph{Brahman} does not signify presence of multiple \emph{Brahman-}s or various divisions within \emph{Brahman}. Instead various names are used to signify whether \emph{Brahman} being referred is in His Absolute aspect or in one of His relative aspects associated with \emph{Māyā}.

\section*{References}

\begin{enumerate}
\itemsep=0pt
\item
  \emph{Taittirīyopaniṣad} 3.1.1
\item
  \emph{Śvetāśvataropaniṣad} 6.19.
\item
  \emph{Māyā} Panchakam Verse 1
\item
  \emph{Bhagavad Gītā} 4.6
\item
  \emph{Manusmṛti} 1.8 and 1.9
\item
  \emph{Gauḍapāda} \emph{Māṇḍukya} \emph{Kārikā} 1.6
\item
  \emph{Manusmṛti} 1.10
\end{enumerate}
\newpage

\section*{3.3. \emph{Karmānuṣṭhāna}}

`\emph{Karma}' or `action' is the very foundation of our existence. A second does not go by without anybody performing any action. People are continuously involved in performing one or the other tasks --eating, drinking, talking, driving etc.\ But action does not refer to physical activity alone. Every word we speak, every thought that arise in the mind represents an action being performed. The whole universe works on the principle action and reaction.

`\emph{Karmānuṣṭhāna}' refers to all those \emph{karma-s}/actions that are prescribed to be performed in the scriptures for the spiritual and material welfare of the individuals and the society. This means, that \emph{karmānuṣṭhāna} not only involves practice of rites and rituals as enjoined in the scriptures but also involves living life by the tenets of \emph{dharma} by fulfilling all the duties and responsibilities as applicable to each individual. It means implies that one adheres to \emph{non-performance} of `\emph{niṣiddha}/prohibited' \emph{karma-s}. Scriptures divide actions into `\emph{vihita}' and `\emph{niṣiddha}'. \emph{Vihita} \emph{karma-s} refer to actions that are prescribed and/or permitted in the scriptures that cause welfare of an individual and the society (Ex: performance of \emph{sandhyā} \emph{upāsanā} for those who are competent). \emph{niṣiddha} \emph{karma-}s are those actions that are prohibited in scriptures because they are harmful to an individual and the society (Ex: murder, rape, etc.). Practice of \emph{vihita} \emph{karma-s} is \emph{dharma} and practice of \emph{niṣiddha} \emph{karma-s} is \emph{adharma}. To properly understand `\emph{karmānuṣṭhāna}', it is vital to first understand the concept of \emph{dharma}.

\textbf{\emph{Dharma}:}

In popular language \emph{dharma} has been given various meanings like- Duty, Charity, Righteousness, Justice, Morality etc.\ But, all of them define \emph{dharma} only partially. `\emph{dharma}' literally means `that which upholds'. Sri \emph{Kṛṣṇa} says (1)-
\newpage

\emph{dhāraṇāt dharmam ityāhuh dharmo dhārayate prajāḥ \dev{।}}

\emph{That which upholds is called dharma. dharma upholds all beings.}

The word `\emph{prajā}' refers to not only human beings, but to all objects (both animate and inanimate) in the universe. \emph{Dharma} is that which supports and sustains whole cosmos.

Again, who is it that supports the universe? It is \emph{Brahman}. It is \emph{Brahman} who creates the universe and all its objects and then sustains the universe by allotting each object its respective duties according to its nature (2). Hence, it is the \emph{guṇa-s} and \emph{karma-s} which are created by \emph{Brahman} that upholds the universe and establishes cosmic order. Hence, \emph{dharma} refers to the inherent qualities and the corresponding duties that are assigned to each object by \emph{Brahman} Himself. Hence, the \emph{dharma} of the physical sun is to give heat and light and that of wind is to blow. Similarly, \emph{Brahman} has also allotted various duties i.e.\ \emph{karma-s} to be carried out by human beings. So, the next question that arises would be- What \emph{karma-s} are considered dharmic and why? \emph{Kaṇāda} in his \emph{Vaiśeṣika} \emph{Sūtra} (3) says-

\emph{yato abhyudaya niḥśreyasa siddhiḥ saḥ dharamaḥ \dev{।}}

\emph{That which yields material and spiritual attainments (of all) is dharma.}

The same has been stated also by \emph{Ādi} \emph{Śaṇkarācārya} (4)-

\emph{jagataḥ sthitikaraṇaṁ prāṇināṁ sākṣāt abhyudaya niḥśreyasahetuḥ yaḥ sa dharmo brāhmaṇādyai varṇibhiḥ āśramibhiḥ śreyo'rthabhiḥ anuṣṭhīyamānaḥ \dev{॥}}

\emph{Dharma is that which causes the stability of the universe, which grants both material and spiritual welfare to all living beings and which is practiced by people of all classes and stages of life who aspire for higher good. }

Hence, it is only those \emph{karma-}s by performance of which an individual attains material and spiritual welfare that is designated as \emph{dharma}. In other words, all those \emph{karma-}s, thoughts, words and actions that result in imbalance of the material and spiritual stability of an individual, society and the cosmos as a whole are deemed as `\emph{adharma}'. Hence, performance of \emph{dharma} is termed as `\emph{puṇya} /virtue' as it will grant `\emph{sukha} /happiness' and non-adherence to \emph{dharma} i.e.\ performance of \emph{adharma} is termed as `\emph{pāpa} /sin' as it would lead `\emph{duḥkha} /sorrow'. To the question, what are the sources from which one must learn \emph{dharma}? \emph{Jaimini} in his \emph{Mīmāṁsā}-\emph{Sūtra} (5) says-

\emph{codanālakṣaṇo 'rtho dharmaḥ \dev{।}}

\emph{Dharma is that which is indicated by Injunctions (of vedas).}

Hence, \emph{Veda-s} are the ultimate source of \emph{dharma}. \emph{Manu} (6) further elaborates it thus-

\begin{verse}
\emph{vedaḥ smṛtiḥ sadācāraḥ svasya ca priyamātmaaḥ \dev{।}}\\
\emph{etaccaturvidhaṁ prāhuḥ sākṣāddharmasya lakṣaṇam \dev{॥}}
\end{verse}

\emph{Veda, Smṛtis, conduct of good persons (i.e.\ saints) and self-satisfaction are said to be the four indications of dharma.}

\emph{Veda} refer to the whole set consisting of \emph{Saṁhitā-s}, \emph{Brāhmaṇa-s}, \emph{Āraṇyaka-s} and \emph{Upaniṣad-}s. \emph{Smṛti-}s refer to various \emph{dharmaśāstra-s} written by \emph{ṛṣi-s} and \emph{jñānī}s like \emph{Manu}, \emph{Yājñvalkya}, \emph{Āpastamba} etc.\ In short, \emph{dharma} is that which is established by \emph{Veda}, which is explained and stipulated in \emph{dharmaśāstra-s} and practiced by saints and other righteous people and finally which satisfies one's inner-conscience. Hence, a person must learn the essence of \emph{dharma} from the \emph{Veda} and practice his life according to the injunctions of \emph{Veda} and \emph{dharmaśāstra-s}. If ever he gets any doubts regarding his duties or conduct, he must consult with those who are living life according to \emph{dharma} and finally perform his action that is satisfactory and fulfilling to his inner-conscience.

\textbf{\emph{Sāmānya} and \emph{Viśeṣa-Dharma}:}
\newpage

\emph{Dharma} is broadly divided into two divisions- \emph{sāmānya} and \emph{viśeṣa}. \emph{Sāmānya-dharma} refers to all those tenets of \emph{dharma} that are common to all human beings. All people are bound to follow these duties for overall welfare. \emph{Viśeṣa-dharma} or Special duties refer to those actions that differ from individual to individual and is context specific. \emph{Dharma} for a particular individual (i.e.\ \emph{svadharma}) consists of \emph{Sāmānya-dharma} that is common to all and \emph{Viśeṣa-dharma} that are specifically applicable to him. This adherence of a person to \emph{svadharma} by the performance of both common duties and specific duties constitutes `\emph{karmānuṣṭhāna}'. Hence, it is necessary to understand the actions and duties that come under `\emph{sāmānya}' and `\emph{viśeṣa}' categories.

Explaining about the tenets of \emph{dharma} that are common to all, \emph{Manu} (7) says-

\begin{verse}
\emph{ahiṁsā satyamāsteyaṁ śaucamindriyanigrahaḥ \dev{।}}\\
\emph{etaṁ sāmāsikaṁ dharmaṁ cāturvarṇye'bravīnmanuḥ~\dev{॥}}
\end{verse}

\emph{Non-injury, truth, non-stealing, cleanliness, control of mind and senses- these are the dharma that is common to people of all varṇa-s.}

\emph{Ahiṁsā} refers to practice of not harming anyone else out of revenge or with selfish motive. `Hiṁsā' refers to not only the `physical violence' committed through the body but also the violence committed through words and thoughts. Hence, every person must live a life that is free from greed, jealousy, hatred, vengefulness, etc.\ as they breed violence in thoughts, words and actions. This manner of living life by following the tenets of non-injury in every thought, speech and action is termed as `\emph{ahiṁsā} \emph{dharma}' (8). \emph{Satya}m refers to practice of truthfulness in thoughts and speech. A person must practice what he speaks and he must speak only what he thinks. If a person thinks one thing in mind and speaks out something else it is considered as `falsehood'. And it is also falsehood if one says something that is contrary to reality. `\emph{Satya}m' is the practice of speaking in accord with reality. Further, scriptures dictate that, a person should only speak that truth which is useful and does not cause hurt to the person who hears it. In other words, when truth is to be spoken, it must be spoken in a proper manner and not in a way to hurt the other person (9). Such a practice of truthful life is termed as `\emph{satya} \emph{dharma}'. `\emph{Āsteya}' refers to `non-stealing'. It not only refers to a thief who steals others wealth, but also to those who steal (i.e.\ have relationship with) others' spouses. A person should be satisfied and content whatever \emph{Īśvara} has given him with. If he has any desires to be fulfilled, it must be through honesty and hard-work and never by theft. The very thought about stealing others wealth or having relationship with others spouses also amounts to \emph{pāpa}/sin. Hence, a practice of `\emph{āsteya} \emph{dharma}' entails a practice of non-stealing in thoughts, deeds and words. Similarly, `\emph{śauca}' refers not only to bodily cleanliness but also to verbal and mental purity. A person whose mind is afflicted with lust, greed, delusion, anger, hatred, jealousy etc.\ is considered `Impure'. Practice of ``\emph{śauca dharma}'' includes maintenance clean mind, clean speech, clean body and clean actions. Finally, `\emph{indrīya nigrahaṁ}' refers to control of the mind and the sense organs. There are eleven \emph{indrīya-s}. They are: the five faculties of perception (i.e.\ sight, touch, taste, hearing and smelling), the five faculties of action (hands, feet, anus, phallus, speech) and the mind. Control of them implies living a simple life with contentment. By controlling the mind and the senses, one controls the passions of lust, desires, jealousy, anger, etc.\ This has been reiterated in \emph{Bhāgavata Purāṇa} (10) which says that nonviolence, truthfulness, honesty, desire for the happiness and welfare of all others and freedom from lust, anger and greed constitute duties for all members of society. Hence, the scriptures enjoin upon all people to live life by following the tenets of \emph{Sāmānya-dharma}.


`\emph{Viśeṣa-Dharma}' refers special duties that are unique to each individual and depends upon different factors like individual's temperaments and inner callings (\emph{varṇa}), station in life (\emph{āśrama}), time (\emph{kāla}), location (\emph{deśa}), emergency situations (\emph{āpad dharma}) etc.\ Among all the above factors, \emph{varṇa} and \emph{āśrama} are two factors, which are central to the practice of \emph{Viśeṣa-Dharma}, because without a determination of an individual's temperaments and station in life, a tailor made `\emph{Viśeṣa-Dharma}' cannot be proposed for every individual. In other words, it is in the context of righteous duties (\emph{svadharma}) that the concept of \emph{varṇa} and \emph{varṇa} \emph{vyavasthā} (system of \emph{varṇa}) must be understood.
\vskip 3pt

\textbf{\emph{Varṇa} \emph{Dharma}:}

The term `\emph{varṇa}' is derived from the verbal root word `\emph{vṛ}', which means to choose, to cover, or color and it refers to the \emph{svadharma} (personal duty/purpose of life) chosen by each individual in his/her life according to his/her inherent nature (\emph{svabhāva}, \emph{guṇa}) or it may refer to the \emph{svabhāva} itself that drives him/her to spontaneously choose particular actions as his/her \emph{svadharma}.

\emph{Ṛg Veda} (11) speaks about how different \emph{varṇa-s} are nothing but designations for different \emph{svabhāva-s} of people by symbolically describing different \emph{varṇa-s} as emerging from different limbs of \emph{Puruṣa} (\emph{Brahman}). In this verse, \emph{Ṛg Veda} employs the model of the human body to describe a conception of human society rooted in \emph{svabhāva} and \emph{svadharma} in an organic manner. \emph{Manusmṛti} (12) describes about how \emph{Brahman} allotted different \emph{svadharma-s} (personal duties) to people born with different \emph{svabhāva-s} (inherent nature) for the sake of protecting and sustaining the universe. Similarly, \emph{Bhagavad} \emph{Gītā} also speaks about creation of four \emph{varṇa-s} based on \emph{guṇa} (natural qualities and tendencies) and \emph{karma} (personal duties) (13) and that the duties have been allotted based on the \emph{guṇa-s} that arise from \emph{svabhāva} (14). \emph{Bhāgavata} \emph{Purāṇa} (15) stresses that the four \emph{varṇa}'s that originated from the Supreme \emph{Puruṣa} are to be recognized/designated by their \emph{Ātma} \emph{ācāra} (natural activities or personal duties according to inherent nature i.e.\ \emph{svadharma}). \emph{Mahābhārata} (16) assigns a color to each \emph{varṇa} that symbolically represents the attributes/\emph{svabhāva} associated with that \emph{varṇa}, reflecting the three qualities of the nature (\emph{Prakṛti}): \emph{sattva}, \emph{rajas}, and \emph{tamas}.

From the above, it follows that \emph{varṇa} is a designation that refers to \emph{svabhāva} (inherent nature) of an individual; \emph{varṇa} classification is a conceptual classification of people based on their \emph{svabhāva-s} and whose sole aim is to identify what \emph{svadharma-s} are applicable to whom, what duties will grant overall welfare to people with which \emph{svabhāva-s}; and \emph{varṇa} \emph{vyavasthā} or \emph{varṇa} system is any practical model, any framework that facilitates each individual to choose and practice the \emph{svadharma} that is applicable to them (or allotted to them) based on their inherent \emph{svabhāva}, without any ambiguities or hindrances.

There are three elements or stages in the implementation of the ideals of \emph{varṇa}: identification of the \emph{varṇa} of any individual, classification of the \emph{varṇa-s}, and assignment of different duties to individuals exhibiting different \emph{varṇa}.

\emph{\textbf{Identification:}} If, the conceptual \emph{varṇa} classification has to be implemented on the ground, the very first thing that is required is a mechanism to identify the \emph{varṇa} of an individual. \emph{Bhagavad} \emph{Gītā}, as quoted before, says that \emph{varṇa} classification is based on differences in the \emph{guṇa-s} that arise from \emph{svabhāva} and the duties that have been adjoined according to those \emph{guṇa-s}. Thus, identification of \emph{svabhāva}/\emph{guṇa} becomes crucial for designating a particular \emph{varṇa} to an Individual. The scriptures have elaborated upon what temperaments constitutes what \emph{varṇa}. But, before getting into it lets briefly look into the factors that cause an individual to get a particular \emph{svabhāva}.

The \emph{svabhāva} of an individual is determined by two components: the \emph{svabhāva-s} (\emph{guṇa}/\emph{varṇa}- natural tendencies) of the parents that one inherits, and the \emph{saṁskāra-s} (mental impressions) that one inherits from past lives and both components are again determined by \emph{prārabdha} \emph{karma-s}- the allotted fruits of previously committed actions that determines where one takes birth, what \emph{svabhāva} one exhibits, etc.\ It is for this reason, `birth' or `\emph{janma}' was used as an identifying factor for determining \emph{varṇa}. But, here the reference is to the `\emph{prārabdha} \emph{karma}' and the \emph{svabhāva} one inherits due to \emph{prārabdha} and need not be a reference to being born in a tribe, caste, class, or family. \emph{Varṇa} refers to the \emph{jīvātma} (the individual soul) and not simply to the body or social responsibility.

\emph{\textbf{Classification:}} In regards to the classification of people into various \emph{varṇa-s}, based on their \emph{guṇa}-\emph{svabhāva}, Hindu scriptures say that there are only four \emph{varṇa-s}, i.e.\ clear cut divisions of \emph{svabhāva} (17): \emph{brāhmaṇa-s}, \emph{kṣatriya-s}, \emph{vaiśya-s}, and \emph{śūdra-s}. \emph{Ādi} \emph{Śaṇkarācārya} (18) while commenting on \emph{Bhagavad} \emph{Gītā} says that \emph{brāhmaṇa} is a designation given to one in whom there is a predominance of \emph{sattva}; \emph{kṣatriya} is one in whom there is both \emph{sattva} and \emph{rajas}, but \emph{rajas} predominates; in \emph{vaiśya}, both \emph{rajas} and \emph{tamas} exist, but \emph{rajas} predominates; and \emph{śūdra} is one in whom both \emph{rajas} and \emph{tamas} exist, but \emph{tamas} predominates. These \emph{Guṇa-s} are revealed by the natural temperaments and behavior exhibited by the person.

Elaborating on this, \emph{Bhāgavata} \emph{Purāṇa} (19), lists what temperaments and behavior indicates what \emph{varṇa} designation to be assigned to a person. It says: the control of mind and senses, austerity, cleanliness, satisfaction, tolerance, simple straightfor\-wa\-rdness, devotion to God, mercy, and truthfulness are the natural qualities of the \emph{brāhmaṇa-s}; dynamic power, bodily strength, determination, heroism, forbearance, generosity, great endeavor, steadiness, devotion to the \emph{brahmaṇa-s} and leadership are the natural qualities of the \emph{kṣatriya-s}; faith in God and \emph{Veda-s}, dedication to charity, freedom from hypocrisy, service to the \emph{brahmaṇa-s} and perpetually desiring to accumulate more money are the natural qualities of the \emph{vaiśya-s}, service without duplicity to others, cows and gods and complete satisfaction with whatever income is obtained in such service, are the natural qualities of\break \emph{śūdra-s}.

Therefore, the scriptures clearly give a wide framework by which people can be designated and classified according to their inherent temperaments. But, this four-fold classification is essentially a conceptual classification based on four ideal \emph{svabhāva} conditions (i.e.\ having a clear cut predominant \emph{guṇa} and a secondary \emph{guṇa}) and may not always reflect a ground situation, especially in \emph{Kaliyuga} in general and at present times in particular, as society stratified along caste, profession, and political lines (rooted mostly in colonial discourse), and the concept of \emph{guṇa} and \emph{svadharma} no longer drives the society.

Nonetheless, this four-fold \emph{guṇa} based \emph{varṇa-s} and the assignment of ideal duties that a person with a particular \emph{svabhāva} must practice, will act as general guidelines, which would not only help societies to evolve their own practical models according to their own unique social conditions, it will also help each individual to understand his/her place in life and \emph{dharma}, such that each person may choose his/her \emph{svadharma} according to his/her \emph{svabhāva} and attain material and spiritual success.

Notably, the \emph{varṇa} model places knowledge, particularly spiritual knowledge (\emph{adhyātma} \emph{vidyā}) or transcendent Knowledge (\emph{ātmavidyā}) at the top, like the head of a human being and a whole framework has been conceived wherein all other mundane activities be it politics, commerce, or labor, act as tools to facilitate individuals to eventually reach the ultimate goal of transcendent knowledge and liberation. In fact, a clear correlation between the four \emph{varṇa-s} and the four \emph{puruṣārtha-s} (goals of life) can be established. Though, the four \emph{puruṣārtha-s} are equally applicable to all human beings irrespective of their \emph{varṇa}, there is also a clear correlation between the \emph{svabhāva} of a person and the \emph{puruṣārtha} he is most likely to consider as central to his life.

\emph{Śūdra-s} who are simple minded common folks have a mundane and worldly outlook. Thus, their primary concern is often limited to their everyday life, family, children, security, and happiness. In other words, their primary goal is `\emph{Kāma}' or fulfillment of worldly desires of themselves and their families. It is for this reason, \emph{Śūdra} \emph{varṇa} is also considered to have only one \emph{āśrama} (stage in life) of \emph{gṛhasta} (householder), wherein they can satisfy their worldly desires including sexual ones. Similarly, \emph{Vaiśya} \emph{varṇa} is associated with the \emph{puruṣārtha} of `\emph{artha}' (gathering of wealth), because their \emph{svabhāva} drives them to pursue wealth and prosperity; \emph{kṣatriya} is associated with \emph{dharma}, because their foremost duty is the protection of \emph{dharma} and the welfare of citizens, and not pursuance of personal desires or wealth; and \emph{Brāhmaṇa-s} are associated with \emph{Mokṣa}, because it is the ultimate calling of the \emph{brāhmaṇa} and they are by \emph{svabhāva} spiritual in outlook. In fact, \emph{Vajrasucikopaniṣad} (20) says, a true \emph{brāhmaṇa} is one who has established himself in \emph{Brahman} (i.e.\ attained \emph{Mokṣa}).

Therefore, unlike the materialistic world order that predominates the western thought, the \emph{varṇa} model honors \emph{karma},\break \emph{dharma}, and higher consciousness, and \emph{varṇa} combined with\break \emph{āśrama} (stages in life) system facilitates each individual to accomplish all the \emph{puruṣārtha-s} (life purposes- righteousness\break (\emph{dharma}), wealth (artha), desires (\emph{kāma}), and liberation (\emph{mokṣa})) in life.

\emph{\textbf{Assignment:}} After successful identification and classification of the \emph{varṇa-s} of people, the final stage is the assignment of duties or \emph{svadharma} for each person according to his/her own inherent temperaments. \emph{Bhagavad} \emph{Gītā} (21) assigns following duties to people exhibiting different \emph{varṇa} \emph{svabhāva}. \emph{Brāhmaṇa-s} are assigned: control of the internal and external organs, austerity, purity, forgiveness, straightforwardness, \emph{jñāna} (knowledge of the scriptures), \emph{vijñāna} (experiential understanding of what is presented in the scriptures) and \emph{āstikyaṁ} (faith and conviction in the truth expounded in the scriptures regarding God, etc.), as their duties. Similarly \emph{kṣatriya-s} are assigned: heroism, boldness, fortitude, capability, and also not retreating from battle, generosity and lordliness; \emph{Vaiśya-s} are assigned: agriculture, cattle-rearing and trade; and \emph{Śūdra-s} are assigned service as their duty.

\emph{Manusmṛti} (22) has further elaborated the duties for people having the four \emph{varṇa} \emph{guṇa-s} thus- teaching and studying, sacrificing for their own benefit and for others, giving and accepting (of alms) as duties of \emph{Brāhmaṇa-s}; protection of the people, giving charity, to offer sacrifices (\emph{yajña-s}), to study (the \emph{Veda}), and to abstain from attaching himself to sensual pleasures as duties of \emph{kṣatriya-s}, to tend cattle, giving charity, to offer sacrifices, to study (the \emph{Veda}), to trade, to lend money, and to cultivate land, are the duties of \emph{Vaiśya}; and serving the other \emph{varṇa-s}, i.e.\ rest of the society by means of various professions like arts, sculpture making, wood carving, etc.\ (23). Hence, the practice of knowledge, of power, of wealth and of service, all practiced by adhering to the principles of \emph{dharma} constitutes the \emph{Viśeṣa-Dharma} of the four \emph{varṇa-s}.
\vskip 2pt

It is clear from above that the duties assigned to people are a) in sync with their inherent temperaments, b) duties further seek to reinforce and strengthen the already present inner talents and temperaments, c) through performance of these duties, though different for different persons, all will attain complete success and overall welfare (24)
\vskip 2pt

It is also clear that the duties assigned are more of a general nature and do not as such refer to any particular profession. A \emph{brāhmaṇa} \emph{varṇa} person may be a teacher teaching wide range of subjects, or a priest at a temple, or a \emph{ṛtvik}, etc.\ who performs \emph{yajña-s}, or simply a researcher, or a scientist. Similarly, a \emph{Śūdra} may well have been a painter, wood carver, architect, sculptor, laborer, artisans, etc.\ In other words, \emph{varṇa} grouping is not same as Kula-groupings or groupings according to clans and /or skills/professions (some of which have attained the character of caste in modern society). Similarly, \emph{varṇa} grouping is not as such related to ethno-cultural \emph{jāti} groupings as well (though the term \emph{jāti} as used in \emph{Manusmṛti} is a reference to \emph{guṇa} and at times synonymous with \emph{varṇa} and not as a reference to ethno-cultural identity). More importantly, \emph{varṇa} has no correlation to the concept of caste or caste identities as understood in present society, which is mostly a colonial construct and superimposition on Indian society.
\vskip 2pt

It is important to understand that, no \emph{varṇa} is to be considered superior or inferior to another \emph{varṇa}. People of each \emph{varṇa} have their own specific duties based on their \emph{guṇa-s} and by the performance of those duties each person would contribute to the harmonious functioning of the society and at the same time attain overall wellbeing. This division of people into different \emph{varṇa-s} was solely done for the proper division of duties so as to create a prosperous and harmonious society. For, if there is no division of works in society, then there would be confusion of duties followed by enormous and unhealthy competition for certain jobs while certain other jobs would lie completely neglected.
\bigskip

\textbf{\emph{Āśrama} \emph{Dharma}:}

The second element of \emph{Viśeṣa-Dharma} is- \emph{āśrama} \emph{dharma}. \emph{Āśrama} refers to different stages in a person's life --\emph{brahmacarya}, \emph{gṛhasta}, \emph{vānaprastha} and \emph{sannyāsa}. \emph{Brahmacarya} or the stage of studenthood is the very first stage in the life of a person. The children were sent to \emph{gurukula} or taught at home from a very young age. While the children of \emph{dvija-s} (25), especially the children of \emph{Brāhmaṇa-s}, were initiated into the study of \emph{Veda} by their \emph{guru-s}, others were initiated into study of various branches of knowledge often in sync with their \emph{varṇa}, individual interest \emph{kula}, and \emph{jāti}. For example, while \emph{kṣatriya} education focused more on warfare and administration, a \emph{Vaiśya}'s education often involved learning the nuances of economy and trade. \emph{Śūdra-s} were initiated into different crafts and skills, often those practiced by their families and communities. The \emph{brāhmaṇa} education was focused on study of \emph{Veda} and \emph{Śāstra-s}. It is said that one would require at least 12 years to learn one \emph{Veda}, 24 years to learn two \emph{Veda-s}, 36 years for three \emph{Veda-s} and 48 years to learn all four \emph{Veda-s} (26). \emph{Śāstra} study included study of variety of subjects including medicine, science, and philosophy. After initiating the students, the \emph{guru} will teach them about cleanliness, food habits and other disciplines that have to be followed (27). For the \emph{brāhmaṇa} students, the \emph{guru} would teach about conducting daily fire ritual, \emph{sandhyā}-worship (28) and over the years, they will learn both \emph{Vedic} chanting and \emph{Vedic} \emph{karma-s} thoroughly.

At the end of their stay at \emph{gurukula}, an individual can either choose to enter \emph{nivṛttimārga} or \emph{pravṛttimārga}. The student who after his study of \emph{Veda} and \emph{Śāstra} has developed discrimination, dispassion etc.\ and wishes to enter \emph{nivṛttimārga}, is given \emph{sannyāsa} \emph{dīkṣā} by the \emph{guru} and such a student renounces the world and enters \emph{sannyāsa} \emph{āśrama}. \emph{Sannyāsa} \emph{āśrama} involves renouncing of one's house, wealth, rituals, and all of one's material desires. It involves constant movement from one place to another place all the while immersed in contemplation on \emph{Brahman}. \emph{Sannyāsin-s} are supposed to consume only that food which they receive in \emph{bhikṣā} and only so much as is absolutely necessary to sustain the body. A \emph{sannyāsī} must remain completely unattached and indifferent to the pleasures of this and the next world (29) He should further practice silence whenever possible and maintain same sightedness towards everyone. He should speak only truth and renounce all anger, hatred etc.\ (30) He should remain naked (31) or wear minimal clothing.

However, if a student is still attached to the world and has desires to attain welfare in this world as well as the next, then he should embrace the \emph{pravṛttimārga} by entering into \emph{gṛhasta}-\emph{āśrama} or householder life. Such, a person should practice \emph{gṛhasta} \emph{āśrama} \emph{dharma} and all the enjoined \emph{karma-}s sincerely and then when he attains old age, he should go to the forest and practice \emph{Devatopāsana}. This going to forest is called `\emph{vānaprastha} \emph{āśrama}'. When a person by such practice of \emph{karma} and \emph{upāsanā} in \emph{gṛhasta} and \emph{vānaprastha} \emph{āśrama} develop sufficient degree of dispassion towards worldly pleasures and also develop a burning desire for \emph{Mokṣa}, then he can also renounce the world and enter \emph{sannyāsa} \emph{āśrama}.

However, among these four \emph{āśrama}, \emph{gṛhasta} \emph{āśrama} is most vital as far as \emph{karmānuṣṭhāna} is concerned.
\vskip 5pt

\textbf{\emph{Gṛhasta} \emph{āśrama}:}

The \emph{Bṛhadāraṇyakopaniṣad} describes three principal duties for a \emph{brāhmaṇa} Householder i.e.\ the person in \emph{pravṛttimārga}. They are: studying and chanting of \emph{Veda-s}, performance of \emph{yajña-s} and other \emph{karma-s}, and the attainment of all the three worlds (32).

Practice of one's branch of \emph{Veda} that one has learned during \emph{brahmacarya} is the most vital of the duties of a \emph{brāhmaṇa}. A \emph{brāhmaṇa} must dedicate his whole life to practice and propagation of \emph{Vedic} \emph{mantras} and \emph{Vedic} \emph{karma-}s. Hence, continuous study, practice and teaching of \emph{Vedic} \emph{Mantra Saṁhitā} to others forms the first duty mentioned by the \emph{Upaniṣad}.

The second duty mentioned in the \emph{Upaniṣad} is the performance of \emph{Vedic} \emph{karma-s}. The scriptures speak about 41 \emph{saṁskāra-s}/ritual ceremonies that must be performed in a \emph{brāhmaṇa}'s life. Of these 16 are more prominent and most of these 16 are practiced not only by \emph{Brāhmaṇa-s}, but also by the other three \emph{varṇa-s}.

Regarding the third duty of attainment of the three words, \emph{Bṛhadā\-raṇya\-kopaniṣad} itself further says: ``There are indeed three worlds, the world of men (\emph{manuṣyaloka}), the world of the manes (\emph{pitṛloka}) and the world of the deities (\emph{devaloka}). This world of men is to be won through the offspring alone, and by no other rite; the world of the \emph{Pitṛ-s} through action (\emph{karma}); and the world of the \emph{deva-s} through meditation (\emph{vidyā}/ \emph{upāsanā}).''

Let us now take a deeper look into the second duty: the performance of \emph{karmānuṣṭhāna} that includes \emph{saṁskāra-s}, \emph{yajña-s}, and other Dharmic actions.

In the \emph{Śabara} \emph{Bhāṣya} on \emph{Jaimini} \emph{Sūtra-}s (33), \emph{saṁskāra} is defined as-

\begin{verse}
\emph{saṁskāro nāma sa bhavati yasmiñjāto padārtho bhavati yogyaḥ kasyacidarthasya \dev{॥}}
\end{verse}

\emph{Saṁskāra is an act which makes a certain thing or person fit for a certain purpose.}

Hence, \emph{saṁskāra} is a ritual of purification. It removes the mental impurities like anger, exultation, grumbling, covetousness, perplexity, doing injury, hypocrisy, lying, gluttony, calumny, envy, lust, secret hatred, neglect to keep the senses in subjection and neglect to concentrate the mind and grants competency. Further, it helps one to attain qualities like compassion on all creatures, forbearance/patience, freedom from jealousy, cleanliness, mental-calmness, auspiciousness, generosity and freedom from desires (34) that make a householder competent to take up \emph{nivṛttimārga}. Hence, \emph{saṁskāra-s} play a very important role in a householder's life as it gives him both material and spiritual welfare. It helps him to attain happiness in this world and attain higher worlds after death.

There are a total of forty-one \emph{saṁskāra-s} that an individual undergoes starting with pre-birth rituals and ending in post-death rituals. They are-

\begin{enumerate}
\itemsep=0pt
\item
  Fourteen \emph{saṁskāra-s} from pre-birth till (including) marriage.
\item
  \emph{Pañchamahāyajñá}
\item
  Twenty-one \emph{yajña-s}
\item
  \emph{Antyeṣṭi} \emph{saṁskāra}
\end{enumerate}

\textbf{1.\ Fourteen \emph{saṁskāra-s} from pre-birth till (including) Marriage:}

The \emph{saṁskāra-s} for a child starts right at the time of its conception. There are 13 such ceremonies that are done for a person before he enters \emph{gṛhastha}-\emph{āśrama}, with marriage/\emph{vivāha} forming the 14\textsuperscript{th} \emph{saṁskāra} that provides one competency to practice \emph{gṛhasta} \emph{dharma}. \emph{Gautama} \emph{Dharmaśāstra} (35) gives the list of 14 \emph{saṁskāra-s} in following order- \emph{Garbhādhānaṁ}, \emph{Puṁsavana}, \emph{Sīmanta}, \emph{Jātakarma}, \emph{Nāmakaraṇa}, \emph{Annaprāśana}, \emph{Cūḍā}, \emph{Upanayana}, the four rites called \emph{Vedavrata-s} (performed during stay at \emph{gurukula}), \emph{Snānaṁ} and \emph{Vivāha}.

\begin{enumerate}
\itemsep=0pt
\item
  \textbf{\emph{Garbhādhānaṁ}:} It is a conception ritual wherein the husband transfers his semen to his wife with chanting of \emph{mantras} for the purpose of conceiving a child.

  After the marriage ceremony is over, when the couple returns to bridegroom's house the bride is made to sit on a bull's hide facing east or north. The husband and wife together lights a fire and perform a simple fire ritual giving oblations to \emph{Agni}, \emph{Vāyu} and \emph{Sūrya}. Later in the evening, the husband takes her out and shows her the pole star and other stars and both make a commitment to have a firm, stable and happy life. For three nights they do not have conjugal relationship. Further they should sleep on floor and should not consume saline food (36). They can eat boiled rice with curd (37). Then, on the last part of the fourth night, they will make a fire sacrifice with expiatory oblations of ghee/clarified butter to Agni, \emph{Vāyu} and \emph{Sūrya}. These expiatory oblations are given to remove any `\emph{doṣa}' (38) or karmic faults in the \emph{prārabdha} \emph{karma-s} of the bride that may become obstacle in their marriage. Then he makes her sit down to the west of the fire, facing the east, and pours ghee that remains from the oblations on her head four times. During the first pouring, he chants the \emph{vyāhṛti} `\emph{bhūḥ}'. During second and third pouring, he respectively chants the \emph{vyāhṛti-s} `\emph{bhuvaḥ}' and `\emph{svaḥ}'. And finally, during the fourth time, he pours by chanting all the three \emph{vyāhṛti-s} together, and hence purifying her. Then, the husband and the wife become sexually intimate with each other with the \emph{mantra} (39)-

  \emph{\dev{सं नाम्नः सं हर्दयानि सं नाभिः सं त्वचः ।}} \emph{\dev{सं त्वा कामस्य योक्त्रेण युञ्जान्यविमोचनाय ।}}

  \emph{United in name, united in our hearts, united in our navel, united in our skin. I will bind thee with the bond of love; that shall be insoluble. }

  For the purpose of conception, when the husband and wife cohabit with each other on the fourth night after the monthly menstrual cycle, the husband should chant \emph{mantra} for \emph{Viṣṇu}, \emph{Tvaṣṭā}, \emph{Prajāpati} and \emph{Dhātṛ} for preparing the womb of the mother, for the proper formation of frame (fetus) of the child upon conception, for the healthy and potent sperm that can cause conception, and finally for successfully causing the conception respectively (40). Then after chanting a few more \emph{mantras}, the husband must enter the wife by praying to \emph{Prajāpati} by uttering the \emph{vyāhṛti-s} for an offspring (41).

  This cohabiting of a couple for the conception of a child by purifying the act with \emph{mantras} is called `\emph{Garbhādhānaṁ}'. The \emph{saṁskāra-s} like \emph{Garbhādhānaṁ}, \emph{Jātakarma}, \emph{Cūḍakarma} and \emph{Upanayanaṁ} helps to remove the faults that may be present in the father's seed and mothers egg (i.e.\ faults can be biological/genetic or/and subtle/karmic ) and hence making sure that the child born is physically, mentally and spiritually healthy (42). The scriptures suggest that the husband and wife to conceive between the 4\textsuperscript{th} day of the monthly cycle and 16\textsuperscript{th} day of the monthly cycle except on the 4\textsuperscript{th}, 11\textsuperscript{th} and 13\textsuperscript{th} day (43). The \emph{Garbhādhāna} procedure i.e.\ reciting of \emph{mantras} before cohabiting is to be practiced, not only when the couple gets intimate for the first time after marriage but also after every monthly periods (44) till conception happens. Some also opine that, it should be performed every time the couples get intimate for the sake of conception (45).
\item
  \textbf{\emph{Puṁsavana}:} Though commonly considered as a ceremony for the birth of a male child, it is actually a ceremony done for the sake of facilitating the entry of \emph{Jīvātma} into the \emph{hṛdaya} of the fetus. \emph{Hṛdaya} here refers to the seat of consciousness and not to physical heart. The name \emph{Puṁsavana} itself literally means `quickening a being' referring to the entry of \emph{Jīvātma} into the fetus which makes the fetus self-aware and results in its first movements. However, since, the first child was usually preferred by Hindu parents to be male because the duties of the householder were transferred from father to son and the son would perform the final rites of the parents, this \emph{saṁskāra} became exclusively identified with the birth of the male child. As a result, the ceremony and the Ayurvedic regimen associated with it also came increasingly associated with prescriptions designed for having a male child. However, the same ritual and the associated Ayurvedic regimen have been designed for having female child as well.

  In the third month (46) of pregnancy of the wife, when the head and feet of the fetus have already been formed and the portions of stomach and hip (i.e.\ the genitals that determine the gender) are yet to be formed (47), a day in the fortnight of increasing moon (i.e.\ the fortnight between new moon and full moon) is chosen. The day is chosen such that the moon is under a male constellation like \emph{Puṣya} or \emph{Śravaṇa} (48). After the wife fasts and baths, the husband gives the wife one barley seed with two beans mixed with curd to eat (49). Then, he sprinkles into her right nostrils, either the powder prepared from the shoot of Banyan tree (that contains two fruits) mixed with ghee or a splinter/ash from fire sacrifice or sap prepared from \emph{kuśa} needle or \emph{soma} stalk (50) while chanting the required \emph{mantras} requesting a healthy and vigorous child (51). The presence of this ritual should not be taken that Hindu scriptures were biased against females. It should be understood that, the duties and responsibilities of men and women were different. In an ideal \emph{Vedic} society, men were given the duties to perform the \emph{Vedic} \emph{karma-s} including the last rites for the parents. Hence, this ritual was given for those who wished a son as their first child and it was performed only during the first pregnancy. Apart from \emph{Puṁsavana}, a ritual called ``\emph{Garbha Rakṣaṇa}'' for the protection of the fetus (be it male or female) and the mother is performed in the fourth month of pregnancy. Here, prayers from \emph{Ṛg Veda} addressed to \emph{Agni} for protecting the fetus and the mother are chanted (52).
\item
  \textbf{\emph{Sīmantonnayana}:} This is the ritual of parting the hair of the pregnant mother performed in her fourth month of pregnancy (53). Some also suggest that it could be done in sixth (54), seventh (55) or even eighth (56) month of pregnancy. This is the ritual of worship for Goddess \emph{Rākā}, the full-moon goddess for the protection of fetus and the \emph{jīva} that is to enter the fetus. The fetus develops various body parts till seven months. \emph{jīva} becomes identified with the fetus in stages. In the fourth month, the \emph{Jīvātma} enters the \emph{hṛdaya} (the seat of consciousness) of the fetus. In the fifth and sixth month, the mind and intellect become fully develop and the \emph{jīva} becomes completely self-aware and conscious by the end of seventh month. The whole process of fetus development becomes complete in the eighth month (57). The time period between fourth and eighth month of pregnancy is considered very critical. And during this course of pregnancy, many negative forces and ethereal spirits may try to attack the mother and the fetus, and may also try to enter the womb (58). Hence, to prevent any such unpleasant influences from harming the mother and the child, the \emph{saṁskāra} of \emph{Sīmantonnayana} is performed. It is to be noted that it is performed only during the first pregnancy (59) of the wife as the protection provided is believed to be present throughout the life.

  In the fourth month of pregnancy, an auspicious day in the fortnight of increasing moon (\emph{śukla} \emph{pakṣa}) is chosen. On the day, the husband conducts the fire sacrifice with oblations made with \emph{vyāhṛti-s} and other \emph{mantras} dedicated to \emph{Dhātṛ} (60). Then, the wife sits to the west of the fire, facing east. The husband first ties to her neck an \emph{udumbara} branch with an even number of unripe fruits on it (61). Then he parts the hairs of his wife upwards (i.e.\ beginning from the front, he parts the hair into two portions). This parting is done six times. The first, second and third parting is done with a bunch of three \emph{darbha} blades with \emph{bhūḥ}, \emph{bhuvaḥ}, \emph{svaḥ} respectively (62). The parting for the fourth time is done with \emph{viratara} wood with a \emph{mantra} dedicated to Aditi (63). The fifth and sixth parting is done with a full spindle and a porcupine's quill with three white spots respectively and it is performed with \emph{mantras} that invoke Goddess \emph{Rākā} (64). After the ritual, the \emph{vīṇā} is played and the husband ties the barley grains with young shoots to the hairs of his wife. The wife keeps silence till evening and then breaks her silence after chanting \emph{vyāhṛti-s} by touching a calf (65).
\end{enumerate}

The \emph{Garbhādhānaṁ}, \emph{Puṁsavana} and \emph{Sīmantonnayana} forms the pre-natal \emph{saṁskāra-s} that are done before the birth of the child for the sake of proper conception, protection of fetus and the safe birth of the child.

\begin{enumerate}
\itemsep=0pt
\item[4.]
  \textbf{\emph{Jātakarma}:} This ritual is performed by the father right after the birth of the child and before the umbilical cord is cut. The main features of this ritual are the production of long life, intelligence, strength, and character in child and also protection of the child and mother. The mother when enters labor is confined in a place which is smeared with pounded mixtures of various roots like that of eggplant and indigo plant for the protection of mother from negative beings like rakshasas (66). When the child is born, before anybody touches the child, the father touches the child with \emph{vatsapra} hymn (67) praying for the long life for the child. Then he lifts the child and lays it over stone, axe and gold placed one over the other with chanting of the \emph{mantra}- ``\emph{Be a stone, be an axe, and be insuperable gold}'' (68) or the father simply touches the child by his shoulder and chants the verses (69). By this, the father prays that the child develops strength like a stone, sharpness of an axe and shining character like that of gold.

  The \emph{Sūtika Agni} i.e.\ fire of confinement is lighted in the confinement room and sacrifice is done in it for fumigation. A mixture of rice-chaff and mustard seeds are given as sacrifice with \emph{mantras} for driving out certain negative forces and beings that become active around mothers who have just given birth and attempt to hurt them. Then, the father places the child in his lap and performs the procedure for the production of Intelligence in the child. A piece of gold/ gold coin tied to \emph{darbha} grass is dipped in a mixture of ghee and honey and few drops are fed to the child with chanting of \emph{vyāhṛti-s}- ``\emph{bhūḥ! I sacrifice the ṛk-s over thee! bhuvaḥ! I sacrifice the yajus over thee! suvaḥ! I sacrifice the sāman-s over thee! bhūr bhuvaḥ svaḥ! I sacrifice the Atharvan and Aṅgiras hymns over thee}'' (70). Then the father murmurs in the ear of the child, a prayer to \emph{Sāvitrī} and \emph{Sarasvatī} for granting intelligence to the child (71). Then, the father bathes the child while accompanied by chanting of various \emph{mantras} to protect the child from chronic diseases, negative influences from others and destruction. Then the umbilical cord is cut (72) and the father then places the child on mothers lap and then washes her right breast while chanting the \emph{mantras} that protect both the mother and child and then the mother feeds the child. A covered water pot is kept near the head of the mother-child (73) and a \emph{nakṣatra}/birth-star name is given to the child which is known only to the parents. The mother and child are made to rest in confinement room for a period of 10 days.
\item[5.]
  \textbf{\emph{Nāmakaraṇa}:} On the 10\textsuperscript{th} day (74) the mother comes out of her confinement chamber and takes bath, thus ending her confinement. The \emph{Sūtika Agni} is replaced by \emph{Aupāsana} \emph{Agni} and the husband makes oblations in the \emph{Aupāsana} fire. The father and mother first pronounce the name they have chosen for their child. It is suggested that names could be kept based on constellations (75) or can begin with `\emph{su}' (76) which means `good'. The girls are given odd syllabled name (77) and the boys are given even syllabled names preferably two syllabled or four syllabled (78). The child is addressed by this name that is kept during \emph{Nāmakaraṇa} ceremony. The name that was previously kept right after birth is treated as secret name. Further, during the ceremony another name is given for the purpose of society (as a surname) denoting special accomplishments done in the family (example- \emph{Somayāgin} - performer of \emph{Soma yajña} (79) or \emph{Dvivedī} - knower of two \emph{Veda-s}) or denoting the person's \emph{varṇa} (\emph{Śarma} for \emph{brāhmaṇa}, \emph{Varma} for \emph{kṣatriya}, etc.\ (80)).
\item[6.]
  \textbf{\emph{Annaprāśanam}:} This is the ceremony of feeding solid food to the child for the first time. When the child has become six months old, the ceremony is conducted on an auspicious day in the fortnight of increasing moon (\emph{śukla}-\emph{pakṣa}). After offering oblations to the fire with proper \emph{mantras}, a mixture of curd, honey, ghee and boiled rice is fed to the child (81). The feeding is performed with chanting of \emph{vyāhṛti-s} (\emph{bhūḥ}, \emph{bhuvaḥ}, Suvah) and a prayer to water and plant (82). It is suggested that use of boiled rice imparts divine brilliance to the child. If instead of boiled rice- goat's flesh, flesh of partridge or fish is used then it would impart nourishment, divine luster or swiftness to the child respectively (83).
\item[7.]
  \textbf{\emph{Cūḍakarma}:} This is the ceremony of first shaving of the child's head. This is performed in the third year of the child for \emph{brāhmaṇa-s}, in fifth year for the \emph{kṣatriya-s} and in the seventh year for the \emph{vaiśya-s} (84). After performing the fire rituals, the child is made to sit to the west of the fire facing east. The father combs his hair with porcupine quills with three white spots, with three \emph{darbha} blades, and with a bunch of unripe \emph{udumbara} fruits (85). The locks of hair are then arranged and the hairs of the child are moistened with a mixture of warm and cold water. Then, the father applies the herbs to the hairs and shaves them leaving only the locks on the child's head (86). The whole procedure is performed with proper \emph{mantras} as prescribed by the \emph{Gṛhya} \emph{Sūtra}s.
\item[8.]
  \textbf{\emph{Upanayanaṁ}:} This is the ritual of initiation into \emph{Vedic} study. This is one of the most important of the \emph{saṁskāra-s}. \emph{Upanayanaṁ} is considered as a second birth, a birth into \emph{Vedic} learning and \emph{Vedic} \emph{karma-s}. A person becomes eligible for the study of \emph{Veda} and performance of \emph{Vedic} \emph{karma-s} only after \emph{Upanayanaṁ}. The ceremony itself is very elaborate and involves various rituals. It starts with preparatory rituals like \emph{Nāndī Śrāddha} (for ancestors), \emph{udaka śānti and puṇyāhavācanaṁ} (for purification). It is followed by ritual of \emph{yajñopavita dhāraṇaṁ} and vāpaṁ. \emph{Yajñopavita dhāraṇaṁ} refers to wearing of sacred thread. It consists of three strands that symbolize the body, the mind and the speech and three knots are put in it. By wearing the sacred thread, the child vows to live a life of \emph{brahmacārī} /student by practicing all the \emph{karma-s} as prescribed in the scriptures by adhering to the tenets of \emph{dharma} in his mind, body and speech. \emph{Vāpaṁ} refers to removing of hairs on the head leaving only a `\emph{śikhā} /tuft'. This is followed by rituals of offering \emph{samidh} to the fire called \emph{samidh- ādhānaṁ} and climbing on stone (\emph{aśmārohanaṁ}). The stone signifies ``firmness'' and the ritual imparts firmness and steadiness of the mind to the child. The child is then made to wear new clothes, girdle and deer skin. This is followed by \emph{añjali-tīrtha-prokṣaṇaṁ} and \emph{ācārya varaṇaṁ}. In \emph{añjali-tīrtha-prokṣaṇaṁ}, the \emph{ācārya} or father pours the water from his hand into the hands of the child with recitation of \emph{mantras} and blesses the child to become fit for the practice of \emph{brahmacarya}. In \emph{ācārya varaṇaṁ}, the student chooses the \emph{ācārya} as his teacher and the teacher accepts him as his student. After accepting the child as his student, the \emph{guru} initiates him in the \emph{mantra} dedicated to \emph{Savitṛ} /Sun first \emph{pada} by \emph{pada}, then hemistich by hemischtch and then in full (87). This ritual is called \emph{Brahma upadeśa}. A \emph{brāhmaṇa} child is initiated with \emph{Gāyatrī} \emph{Mantra}, a \emph{kṣatriya} child is initiated in \emph{Triṣṭup} \emph{mantra} and a \emph{Vaiśya} child is initiated into \emph{Jagatī} \emph{Mantra} (88). Further, a \emph{Brāhmaṇa} child is to be initiated in his 8\textsuperscript{th} year, a \emph{kṣatriya} child in 11\textsuperscript{th} year and a \emph{Vaiśya} child in 13\textsuperscript{th} year (89). A child thus initiated into \emph{brahmacarya} \emph{āśrama}, must daily practice \emph{Agni Samidhānaṁ} and \emph{Sandhyopāsanā}. He must practice strict discipline and concentrate on \emph{Vedic} studies and practice.
\item[9.]
  \textbf{\emph{Veda Vrataṁ} (Four in number):} During the study of \emph{Veda}, at the beginning and at the end of study of each \emph{Kāṇḍa}m, worship of the \emph{ṛṣi}/seer and \emph{Devatā}/deity of the \emph{Kāṇḍam} is performed. This worship is called \emph{Veda Vrataṁ}. For \emph{Kṛṣṇa} \emph{Yajurveda}, there are four such \emph{Veda Vrataṁ} corresponding to four \emph{Kāṇḍam}- \emph{Prājāpatya}, \emph{Saumya, Āgneya and Vaiśvadeva Kāṇḍaṁ}. Similarly, for the study of \emph{Ṛg Veda}, four \emph{Vrataṁ} are prescribed (\emph{Mahānāmnī, Mahopaniṣad and two Godāna Vrataṁ}).
\item[10.]
  \textbf{\emph{Saṁvartanaṁ}:} This is a ceremony that marks the end of a student's \emph{Vedic} studies, after which he returns home from the \emph{gurukula}. Such a person is considered ready for entering the life of a \emph{gṛhastha} (householder).
\item[11.]
  \textbf{\emph{Vivāha}:} It is the ceremony of marriage which is considered as a \emph{saṁskāra} or consecration that facilitates a couple to enter the \emph{gṛhastha} \emph{āśrama}. This is perhaps among the most important \emph{saṁskāra-s}, one that is practiced by every community. \emph{Vivāha} enables the couples to pursue \emph{Rati} (conjugal love), \emph{Prajā} (offspring), and \emph{dharma} (duties of householder). The couple also bear the responsibility of helping other \emph{āśrama-s} such as the \emph{sannyāsin-s} and \emph{brahmacārin-s}. It is an elaborate ceremony with multiple rituals and procedures. \emph{Madhuparka}, \emph{Kanyādāna}, \emph{Pāṇigrahaṇa}, \emph{Saptapadī}, \emph{Māṅgalya Dhāraṇa}, and \emph{Aśmārohaṇaṁ} are some of the important stages in \emph{vivāha} ceremony.
\end{enumerate}

\textbf{2. \emph{Pañca Mahāyajña-s}}

\emph{Taittirīya} Aranyaka (90) says: ``These are the five great sacrifices which are to be performed on a daily basis and completed. They are \emph{Deva} \emph{Yajña}, \emph{Pitṛ} \emph{Yajña}, \emph{Bhūta} \emph{Yajña}, \emph{Manuṣya} \emph{Yajña} and \emph{Brahma} \emph{Yajña}.'' Enunciating upon these five sacrifices, \emph{Manusmṛti} (91) notes thus:

\begin{quote}
``A householder has five slaughter-houses (as it were, viz.) the hearth, the grinding-stone, the broom, the pestle and mortar, the water-vessel, by using which he is bound (with the fetters of sin). In order to successively expiate (the offences committed by means) of all these (five) the great sages have prescribed for householders the daily (performance of the five) great sacrifices. Teaching (and studying) is the sacrifice (offered) to \emph{Brahman} (\emph{Brahma}-\emph{Yajña}), the (offerings of water and food called) \emph{tarpaṇa} the sacrifice to the manes (\emph{Pitṛ} \emph{Yajña}), the burnt oblation the sacrifice offered to the gods (\emph{Deva} \emph{Yajña}), the \emph{bali} offering that offered to the \emph{bhūta-s} (\emph{Bhūta} \emph{Yajña}), and the hospitable reception of guests the offering to men (\emph{Manuṣya} \emph{Yajña}). He who neglects not these five great sacrifices, while he is able (to perform them), is not tainted by the sins (committed) in the five places of slaughter, though he constantly lives in the (order of) householders. But he who does not feed these five: the gods, his guests, those whom he is bound to maintain, the manes, and himself, lives not, though he breathes. They call (these) five sacrifices also, \emph{Ahuta, Huta, Prahuta, Brāhmyahuta} and \emph{Prāśita}. \emph{Ahuta} (not offered in the fire) is the muttering (of \emph{Vedic} texts), \emph{Huta} the burnt oblation (offered to the gods), \emph{Prahuta} (offered by scattering it on the ground) the \emph{bali} offering given to the \emph{bhūta-s}, \emph{Brāhmyahuta} (offered in the digestive fire of \emph{Brāhmaṇa-s}), the respectful reception of \emph{Brāhmaṇa} (guests), and \emph{Prāśita} (eaten) the (daily oblation to the manes, called) \emph{tarpaṇa}. Let (every man) in this (second order, at least) daily apply himself to the private recitation of the \emph{Veda}, and also to the performance of the offering to the gods; for he who is diligent in the performance of sacrifices, supports both the movable and the immovable creation.''
\end{quote}

These five \emph{Mahāyajña-s} are thus central to the practice of \emph{gṛhasta}-\emph{āśrama} \emph{dharma}.

\textbf{3. Twenty One \emph{Yajña-s}}

The 21 \emph{yajña-s} that a householder is expected to perform at specific times can be classified into three groups: \emph{Pāka} \emph{yajña-s} wherein cooked food are offered in the sacrificial fire, \emph{Havir} \emph{Yajña} wherein offerings of \emph{Havis} are given, and \emph{Soma} \emph{Yajña} wherein \emph{Soma} is offered into the sacrificial fire (92).
\medskip

\begin{longtable}{|l|l|l|}
\hline
\emph{\textbf{Pāka Yajña}} & \emph{\textbf{Havir Yajña}} & \emph{\textbf{Soma Yajña}}\tabularnewline
\hline
\emph{Aṣṭaka} & \emph{Agnyādheya} & \emph{Agniṣṭoma}\tabularnewline
\hline
\emph{Sthālīpāka} & \emph{Agnihotra} & \emph{Atyagniṣṭoma}\tabularnewline
\hline
\emph{Vārṣika Śrāddha} & \emph{Nirūḍha Paśubandha} & \emph{Ukthya}\tabularnewline
\hline
\emph{Śrāvaṇī} & \emph{Iṣṭi} & \emph{Ṣoḍaśī}\tabularnewline
\hline
\emph{Āgrahāyaṇī} & \emph{Āgrāyaṇī} & \emph{Vājapeya}\tabularnewline
\hline
\emph{Caitrī} & \emph{Darśa Pūrṇamāsa} & \emph{Aptoryāma}\tabularnewline
\hline
\emph{Āśvayujī} & \emph{Sautrāmaṇi} & \emph{Atirātra}\tabularnewline
\hline
\end{longtable}

Regarding the performance of these 21 \emph{yajña-s}, \emph{Kāñcī Paramā\-cā\-rya} (93) notes: ``\emph{Aupāsana} and \emph{Agnihotra} are part of the daily religious routine. Though a pakayajna, \emph{Aupāsana} is not included in the group of seven \emph{Pākayajña-s} mentioned above, while \emph{Agnihotra} is one of the seven \emph{Havir Yajña-s}. \emph{Darśa Pūrṇmāsa} is a \emph{Havir Yajña} to be performed once in fifteen days. The other five \emph{Havir Yajña-s} and the seven \emph{Somayajña-s} are to be performed once a year or, at least, once in a lifetime. As if out of consideration for us, the \emph{Smṛti-s} have granted us this concession: that the difficult \emph{Somayajña-s} need be undertaken only once in a lifetime.''

\textbf{4. \emph{Antyeṣṭi Saṁskāra}}

\emph{Antyeṣṭi} literally means the `last sacrifice'. It refers to the funeral rites wherein the physical body of the dead person is offered to the fire. It is the last of the 41 \emph{saṁskāra-s} that a person has to undergo in life. While the exact ritual procedure varies depending upon region, community, \emph{varṇa}, age, and many other factors, it usually lasts for 12 days. It is usually performed by the son or in the absence of the son, some other relative, to facilitate the journey of the \emph{jīva} that has just left the body, but still has attachment and self-identification with it to migrate from its status as a \emph{preta} attached to its old body to the abode of \emph{Pitṛ-s} by taking recourse to \emph{pitṛyāna}. It is also called \emph{Antima} \emph{saṁskāra}, \emph{Antya- kriyā}, \emph{Anvarohanya}, or as \emph{Vahni} \emph{saṁskāra}.

\emph{Karmānuṣṭhāna} is very vital for the welfare of an individual. Hindu scriptures provide a broad framework of \emph{varṇa}-\emph{āśrama} that enunciates various \emph{dharma-s} or righteous duties for people with different temperaments and competencies. \emph{saṁskāra-s} are important part of \emph{karmānuṣṭhāna} that are enjoined by the \emph{śāstra-s}. Each person is expected to understand and perform his/her own \emph{svadharma} to attain material and spiritual welfare.

\section*{References}

\begin{enumerate}
\itemsep=1pt
\item
  \emph{Mahābhārata} \emph{Karṇa Parva} 69.58
\item
  \emph{Īśopaniṣad} Verse 8
\item
  \emph{Vaiśeṣika} \emph{Sūtra} 1.1.2
\item
  \emph{Ādi} \emph{Śaṇkarācārya}, \emph{Gītā} \emph{Bhāṣya}, Introduction to Chapter 1
\item
  \emph{Jaimini} \emph{Mīmāṁsā} \emph{Sūtra}- 1.1.1
\item
  \emph{Manusmṛti} 2.12
\item
  \emph{Manusmṛti} 10.63
\item
  \emph{Kūrma} \emph{Purāṇa}, Uttara-Bhaga, Chapter 11, Verse 14
\item
  \emph{Manusmṛti} 4.139
\item
  \emph{Bhāgavata} Puran 11.17.21
\item
  \emph{Ṛg Veda} \emph{Puruṣasūkta} Verse 12
\item
  \emph{Manusmṛti} (1.87)
\item
  \emph{Bhagavad} \emph{Gītā} 4.13
\item
  \emph{Bhagavad} \emph{Gītā} 18.41
\item
  \emph{Bhāgavata} \emph{Purāṇa} 11.17.13
\item
  \emph{Mahābhārata} 12.188
\item
  \emph{Manusmṛti} 10.4
\item
  \emph{Ādi} \emph{Śaṇkarācārya}, \emph{Gītā} \emph{Bhāṣya} on verses 4.13 and 18.41
\item
  \emph{Bhāgavata} \emph{Purāṇa} 11.17.16-19
\item
  \emph{Vajrasucikopaniṣad} verse 10
\item
  \emph{Bhagavad} \emph{Gītā} 18.42-44
\item
  \emph{Manusmṛti} 1.88-91
\item
  \emph{Manusmṛti} 10.100
\item
  \emph{Bhagavad} \emph{Gītā} 18.45
\item
  \emph{Dvija} means ``Twice Born''. It refers to \emph{Brāhmaṇa-s}, \emph{Kṣatriya-s} and \emph{Vaiśya-s} who have rights for \emph{Upanayanaṁ} Initiation. This Initiation ceremony is considered a second Birth and hence they are called ``Dvija''.
\item
  \emph{Āpastamba} \emph{Dharmaśāstra} 1.1.2.12 to 1.1.2.16. \emph{Manusmṛti} 3.1.
\item
  Disciplines like not wasting time in watching dance etc., should not involve in gossiping, should have control over mind and senses, should not get angry or jealous about others, should be gentle, and should serve his \emph{guru} etc.\ \emph{Āpastamba} \emph{Dharmaśāstra} 1.1.3.11 to 1.1.3.24
\item
  \emph{Manusmṛti} 2.69
\item
  \emph{Āpastamba} \emph{Dharmaśāstra} 2.9.21.10
\item
  \emph{Manusmṛti} 6.41, 6.48, 6.49
\item
  \emph{Āpastamba} \emph{Dharmaśāstra} 2.9.21.11
\item
  \emph{Bṛhadāraṇyakopaniṣad} 1.5.16-17
\item
  \emph{Śabara} Bhasya on Verse 3.1.3 of \emph{Jaimini} \emph{Mīmāṁsā} \emph{Sūtra}. Quoted in Page 16, \emph{Hindu} \emph{Saṁskāra-s}: Socio-religious Study of the Hindu Sacraments by Rajbali Pandey.
\item
  \emph{Gautama} \emph{Dharmaśāstra} 8.23
\item
  \emph{Gautama} \emph{Dharmaśāstra} 8.14 to 8.16
\item
  \emph{Āpastamba} \emph{Gṛhya} \emph{Sūtra} 3.8.8
\item
  \emph{Śāṅkhāyana} \emph{Gṛhya} \emph{Sūtra} 1.17.7
\item
  The \emph{doṣa}'s that are expiated are that of causing death of husband, being without offsping, destruction of cattle and wealth ( \emph{Śāṅkhāyana} \emph{Gṛhya} \emph{Sūtra} 1.18.3), any other faults that are terrible and blameful. (\emph{Hiraṇyakeśin} \emph{Gṛhya} \emph{Sūtra} 1.7.24.1)
\item
  \emph{Hiraṇyakeśin} \emph{Gṛhya} \emph{Sūtra} 1.7.24.4
\item
  \emph{Hiraṇyakeśin} \emph{Gṛhya} \emph{Sūtra} 1.7.25.1
\item
  \emph{Hiraṇyakeśin} \emph{Gṛhya} \emph{Sūtra} 1.7.25.2
\item
  \emph{Manusmṛti} 2.27
\item
  \emph{Manusmṛti} 3.46 and 3.47
\item
  \emph{Pāraskara} \emph{Gṛhya} \emph{Sūtra} 1.11.7 \emph{Hiraṇyakeśin} \emph{Gṛhya} \emph{Sūtra} 1.7.25.4
\item
  \emph{Hiraṇyakeśin} \emph{Gṛhya} \emph{Sūtra} 1.7.25.3
\item
  \emph{Śāṅkhāyana} \emph{Gṛhya} \emph{Sūtra} 1.20.1.
\item
  \emph{Garbhopaniṣad} Section 3
\item
  \emph{Śāṅkhāyana} \emph{Gṛhya} \emph{Sūtra} 1.20.2.
\item
  \emph{Āśvalāyana} \emph{Gṛhya} \emph{Sūtra} 1.13.2
\item
  \emph{Śaṅkhāyana} \emph{Gṛhya} \emph{Sūtra} 1.20.3, Hiranyakeshi \emph{Gṛhya} \emph{Sūtra} 2.1.2.6
\item
  \emph{Śāṅkhāyana} \emph{Gṛhya} \emph{Sūtra} 1.20.5. Four verses from \emph{Ṛg}-\emph{Veda}- 1.1.3, 3.4.9, 5.37.2 and 2.3.9 with \emph{svāhā} at end of each verse.
\item
  For \emph{Garbha}-\emph{Rakṣaṇa} refer- \emph{Śāṅkhāyana} \emph{Gṛhya} \emph{Sūtra} 1.21. The verses for protection of fetus is from \emph{Ṛg}-\emph{Veda} 10.162 and verses for protection of mother (of fetus) is from \emph{Ṛg}-\emph{Veda} 10.163
\item
  \emph{Āpastamba} \emph{Gṛhya} \emph{Sūtra}s 6.14.1
\item
  \emph{Gobhila} \emph{Gṛhya} \emph{Sūtra}s 2.7.2
\item
  \emph{Śāṅkhāyana} \emph{Gṛhya} \emph{Sūtra} 1.22.1
\item
  \emph{Gobhila} \emph{Gṛhya} \emph{Sūtra}s 2.7.2
\item
  \emph{Garbhopaniṣad} Section 3
\item
  Need Citations
\item
  \emph{Āpastamba} \emph{Gṛhya} \emph{Sūtra}s 6.14.1
\item
  \emph{Hiraṇyakeśin} \emph{Gṛhya} \emph{Sūtra}s 2.1.1.2.
\item
  \emph{Gobhila} \emph{Gṛhya} \emph{Sūtra}s 2.7.4.
\item
  \emph{Gobhila} \emph{Gṛhya} \emph{Sūtra}s 2.7.5
\item
  \emph{Gobhila} \emph{Gṛhya} \emph{Sūtra}s 2.7.6
\item
  \emph{Gobhila} \emph{Gṛhya} \emph{Sūtra}s 2.7.7-8. The \emph{mantras} invoking Goddess \emph{Rākā} that are used are from \emph{Ṛg}-\emph{Veda} 2.32.4-5
\item
  \emph{Āpastamba} \emph{Gṛhya} \emph{Sūtra}s 6.14. 4-8
\item
  \emph{Śāṅkhāyana} \emph{Gṛhya} \emph{Sūtra} 1.23.1
\item
  \emph{Āpastamba} \emph{Gṛhya} \emph{Sūtra} 6.15.1, the \emph{vatsapra} hymn appears in \emph{Yajurveda} \emph{Taittirīya} \emph{Saṁhitā} 4.2.2)
\item
  \emph{Hiraṇyakeśin} \emph{Gṛhya} \emph{Sūtra} 2.1.3.2
\item
  \emph{Āśvalāyana} \emph{Gṛhya} \emph{Sūtra} 1.15.3
\item
  \emph{Hiraṇyakeśin} \emph{Gṛhya} \emph{Sūtra} 2.1.3.9
\item
  \emph{Āśvalāyana} \emph{Gṛhya} \emph{Sūtra} 1.15.2
\item
  Gobila \emph{Gṛhya} \emph{Sūtra} 2.7.22
\item
  \emph{Hiraṇyakeśin} \emph{Gṛhya} \emph{Sūtra} 2.1.4.3-5
\item
  \emph{Āpastamba} \emph{Gṛhya} \emph{Sūtra} 6.15.8
\item
  \emph{Hiraṇyakeśin} \emph{Gṛhya} \emph{Sūtra} 2.1.4.13
\item
  \emph{Āpastamba} \emph{Gṛhya} \emph{Sūtra} 6.15.10
\item
  \emph{Āpastamba} \emph{Gṛhya} \emph{Sūtra} 6.15.11
\item
  \emph{Āpastamba} \emph{Gṛhya} \emph{Sūtra} 6.15.9
\item
  \emph{Hiraṇyakeśin} \emph{Gṛhya} \emph{Sūtra} 2.1.4.15
\item
  \emph{Manusmṛti} 2.32
\item
  \emph{Āpastamba} \emph{Gṛhya} \emph{Sūtra} 6.16.1
\item
  \emph{Hiraṇyakeśin} \emph{Gṛhya} \emph{Sūtra} 2.1.6.2-3
\item
  \emph{Śāṅkhāyana} \emph{Gṛhya} \emph{Sūtra} 1.27.2-5
\item
  \emph{Śāṅkhāyana} \emph{Gṛhya} \emph{Sūtra} 1.28.2-4
\item
  \emph{Āpastamba} \emph{Gṛhya} \emph{Sūtra} 6.16.6
\item
  \emph{Hiraṇyakeśin} \emph{Gṛhya} \emph{Sūtra} 2.1.6.6-12
\item
  \emph{Āpastamba} \emph{Gṛhya} \emph{Sūtra}s 4.11.11-12
\item
  \emph{Śāṅkhāyana} \emph{Gṛhya} \emph{Sūtra} 2.5.4-6
\item
  \emph{Āpastamba} \emph{Gṛhya} \emph{Sūtra} 4.10.2-3
\item
  \emph{Taittirīya} Aranyaka 2.10
\item
  \emph{Manusmṛti} 3.67-75. The translation reproduced is by George Buhler.
\item
  The table has been reproduced from an article by Hariprasad Nellitheertha titled `\emph{Ṣoḍaśa} \emph{saṁskāra-s} -- The Sixteen Rites To Mark The Passage Of Life', Indic Today {[}\url{http://www.indictoday.com/perspective/sixteen-samskaras/}{]}
\item
  Sri Chandrasekharendra Saraswathi, `Names of \emph{saṁskāra-s}', in `Hindu \emph{Dharma}', Bharatiya Vidya Bhavan {[}\url{http://www.kāmakoti.org/hindudharma/part16/chap8.htm}{]}
\end{enumerate}
\newpage

\section*{3.4. \emph{Devatopāsana}}

\emph{Upāsanā} is one of the central practices of almost all Hindu spiritual traditions, regardless of their philosophical or theistic orientations. Often translated as worship or meditation it comes from the Sanskrit root `\emph{upa}' and `\emph{āsana}' which means `to sit near' or `to stay close by'.

`\emph{Devatopāsana}' would thus imply being near the \emph{Devatā} by continuously either fixing the mind on the form of the \emph{Devatā} or directing the mind to meditate on the formless \emph{Brahman}, who is the substratum of all manifestation and who Himself manifests as different \emph{Devatā-}s.

As Swami Achalananda describes, ``\emph{upāsanā} is the process of practicing the proximity of God and of progressively feeling his presence till one merges in Him'' (1).

While \emph{upāsanā} in its primary sense refers to Dhyana or meditation, especially in the \emph{Upaniṣad-s} either on formless \emph{Brahman}, or on various aspects of \emph{Brahman}, or through metaphors indicating \emph{Brahman}, or lastly on different \emph{Devatā-s} that are manifestations of \emph{Brahman}; in its expanded sense, it actually came to be associated with two interrelated but distinct concepts of \emph{pūjā} and \emph{bhakti} \emph{yoga}.

Let us now explore a bit more into all the three streams of \emph{upāsanā} as \emph{dhyāna, pūjā} and \emph{bhakti}.
\medskip

\textbf{\emph{Upāsanā} as \emph{dhyāna}}

The first and the primary stream of \emph{upāsanā} was its understanding as Dhyāna or meditation. It is in this sense that the \emph{Upaniṣad-s} understood the term \emph{upāsanā}. It is in this sense that \emph{Āraṇyaka-s} have been classified as \emph{upāsanā}-\emph{kāṇḍa}- the portion of the \emph{Veda} that describes various modes of meditation. \emph{Upāsanā} retains this primary meaning as \emph{dhyāna} even when it developed into inter-related but distinct streams of \emph{pūjā} and \emph{bhakti}. \emph{Dhyāna} finds a central place in \emph{Itihāsa - Purāṇa - Tāntrika - Āgamika} traditions as well.

\emph{Dhyāna} means one-pointed concentration or meditation with the mind fixed upon the object of meditation. In the case of \emph{upāsanā}, the object of meditation is \emph{Brahman}, often meditated using various symbols, metaphors, names, or forms. \emph{Vidyā} \emph{upāsanā} or \emph{Devatopāsana} involves meditation on any particular aspect of \emph{Brahman} either in abstract way or in a more personalized way as the case may be. While \emph{vidyā} literally means `to know', \emph{Devatā} means `the shining one'. Thus, they are both related to knowledge. However, while \emph{vidyā} \emph{upāsanā} seems to convey meditation on knowledge of \emph{Brahman} in a more abstract sense, \emph{Devatopāsana} conveys a meaning of meditation on knowledge in a more theistic sense.

The \emph{Upaniṣad}s prescribe many different forms of \emph{upāsanā}. \emph{Īśopaniṣad} (2) provides a beautiful \emph{upāsanā} of \emph{Satya} \emph{Brahman}, the \emph{Puruṣa} who inhabits the \emph{Sūrya} \emph{Maṇḍala}, who is concealed by a golden vessel as it were, and who is non-different from the \emph{Puruṣa} who inhabits the individual \emph{jīva}. \emph{Kenopaniṣad} (3) provides four meditations on \emph{Brahman}, two in the cosmic context (\emph{adhidaivika}) and two in the context of individual self (\emph{adhyātmika}). It says that one must meditate on \emph{Brahman} as if it is like the flash of the lightning or as fast as the blinking of the eye in the cosmic context. At \emph{adhyātmika} level, it says, one must meditate on \emph{Brahman} as that towards which the mind seems to repeatedly go and repeatedly remembers; or \emph{Brahman} must be meditated with the help of the name `\emph{Tadvanaṁ}'- the most adorable one. \emph{Taittirīyopaniṣad} and \emph{Māṇḍukyopaniṣad} provide meditation on the sacred \emph{Oṁ}.

In his book, `The Thirty-Two \emph{Vidyā}-s', K Narayanaswami Aiyar (4) lists thirty-two different kinds of meditation described in various \emph{Upaniṣad}s, the more famous among them include: \emph{Madhu} \emph{Vidyā}, \emph{Prāṇavidyā}, \emph{Śāṇḍilya} \emph{Vidyā}, \emph{Vaiśvānara} \emph{Vidyā} and \emph{Dahara} \emph{Vidyā} from the \emph{Chāndogyopaniṣad}, \emph{Pañcāgni Vidyā} is mentioned in both the \emph{Chāndogyopaniṣad} and the \emph{Bṛhadāraṇyakopaniṣad}, \emph{Ānandamaya Vidyā} from \emph{Taittirīyopaniṣad}, Akṣara Vidyā, Jyotiṣāṁ \emph{Jyotir Vidyā} from \emph{Bṛhadāraṇyakopaniṣad}, \emph{Naciketāgni} \emph{Vidyā} and \emph{Aṅguṣṭmātra Vidyā} from \emph{Kaṭhopaniṣad}, to name a few.

Among these, \emph{Śāṇḍilya} \emph{Vidyā}, \emph{Pañcāgni Vidyā}, \emph{Bhūma Vidyā}, \emph{Dahara} \emph{Vidyā}, \emph{Maitreyī} \emph{Vidyā}, and \emph{Vaiśvānara} \emph{Vidyā} are perhaps among the more famous forms of \emph{Upaniṣad}ic meditations.

\emph{Śāṇḍilya} \emph{Vidyā} is a meditation taught by \emph{ṛṣi} \emph{Śāṇḍilya}. He instructs the practitioner to meditate upon Self as being \emph{Manomaya} with \emph{Prāṇa} as its body and effulgence as its form; being smaller than an atom and bigger than the universe, containing all works, desires, odours, tastes, etc.\ which is situated within the heart.

\emph{Pañcāgni} \emph{Vidyā} is the meditation of five fires wherein five acts of the universe are conceived of as five symbolic sacrificial fires with its five components of fuel, smoke, flame, coals and sparks. The five symbolic fires which act as the object of meditation are the \emph{svarloka} (heaven), \emph{bhuvarloka} (intermediate space), and bhūloka (earth), man and woman.

A table representing the five-fire meditation as enunciated in the \emph{Bṛhadāraṇyakopaniṣad} is given below.

\emph{Dahara} \emph{Vidyā} is a type of meditation wherein one meditates upon his Heart, the innermost Self. Swami Sivananda (5) says: ``This is one of the greatest of the \emph{Vidyā-s}. The all-pervading and all-inclusive nature of the Self is stressed upon in this \emph{Vidyā}. In this meditation, the meditator feels the whole universe as his Self and excludes nothing from the One Self. This \emph{Vidyā} further explains the identity of the external and the internal, the objective and the subjective, the macrocosmic and the microcosmic, the universal and the individual, \emph{Brahman} and \emph{Ātman}.''
\newpage

{\tabcolsep=4pt\fontsize{7}{9}\selectfont
\begin{longtable}{|l|l|l|l|l|l|l|l|l|}
\hline
 & \textbf{Fire} & \textbf{Fuel} & \textbf{Smoke} & \textbf{Flame} & \textbf{Cinder} & \textbf{Spark} & \textbf{Oblation} & \textbf{Fruit}\tabularnewline
 & & & & & & & &\textbf{of the}\tabularnewline
 & & & & & & & &\textbf{Sacrifice}\tabularnewline
\hline
\textbf{1} & \emph{Svarloka}/ & \emph{Āditya} / & Light- & Day & Four- & Inter- & \emph{Śraddhā} /  & Soma\tabularnewline
& Heaven & Sun & rays & & Quarters & mediate & Faith &\tabularnewline
&  &  &  & &  & Quarters &  &\tabularnewline
\hline
\textbf{2} & \emph{Parjanya}/ & Year & Clouds & Light- & Thunder & Rumblings & \emph{Soma} & Rain\tabularnewline
& the Deity &  &  & ning &  &  &  &\tabularnewline
& of Rain &  &  &  &  &  &  & \tabularnewline
\hline
\textbf{3} & \emph{Bhūloka} & Earth & Fire & Night & Moon & Stars & Rain & Food\tabularnewline
\hline
\textbf{4} & Man & Open & \emph{Prāṇa} & Speech & Eyes & Ear & Food & Semen\tabularnewline
& & Mouth &  &  &  &  &  &\tabularnewline
\hline
\textbf{5} & Woman & Phallus & Pubic & \emph{Yoni}/ & Sexual & Sexual  & Semen & Offspring\tabularnewline
&  &  & Hair & Vulva & Inter- & delight &  &\tabularnewline
& &  &  &  & course & (orgasm) &  &\tabularnewline
\hline
\end{longtable}}



\emph{Vaiśvānara} \emph{Vidyā} involves meditating upon \emph{Brahman} as \emph{Vaiśvānara} \emph{Ātman} whose head is the heaven, His eyes \emph{Sūrya}, His breath \emph{Vāyu}, His trunk \emph{Akāśa}, His bladder the \emph{Rayi}, His feet the earth, His chest the sacrificial altar, His hairs the sacrificial grass, His heart the \emph{Gārhapatya} fire, His mind the \emph{Anvāhāryapacana} fire (\emph{Dakṣiṇa\-gni}), His mouth the \emph{Āhavanīya} fire. Here while the \emph{Vaiśvānara} is understood as pervading the three words, He is described as constituting the three fires from the standpoint of \emph{Agni} and \emph{upāsanā} is also conducted from this standpoint of \emph{Agni} \emph{Vaiśvānara} only~(6).

This wealth of \emph{upāsanā} techniques enunciated in the \emph{Upaniṣad}s were later further developed in different Hindu traditions. \emph{Patañjali} \emph{Yoga} \emph{Sūtra}s, for example, provides an eight-limbed framework for spiritual emancipation, wherein Dhyāna plays a central role. Similarly \emph{Tantric} traditions explored different forms of meditation with texts like \emph{Vijñāna Bhairava} \emph{Tantra} belonging to \emph{Trika} School of Kashmir Shaivism enunciating 112 techniques of meditation. The \emph{Itihāsa}-\emph{Purāṇa} tradition also provided many different techniques of meditation, especially those related to different \emph{Devatā-s}, which later developed into full-fledged procedure of worship.
\newpage

These \emph{upāsanā} methods were enunciated in the \emph{Upaniṣad}s primarily for spiritual practitioners who did not have competency to pursue \emph{Vedānta} \emph{sādhanā}. The primary aim was to help them develop one-pointed concentration and purification of the mind. Later traditions further built upon this and provided elaborate mechanisms to accomplish the twin goals.
\medskip

\textbf{\emph{Upāsanā} as \emph{pūjā}}

The second stream of \emph{upāsanā}, which especially developed in the \emph{Paurāṇika, Āgamika} and \emph{Tāntrika} traditions, was the understanding of \emph{upāsanā} as pūjā or ritual worship.

\emph{Pūjā} literally means worship, invocation, or showing reverence. \emph{Kulārṇava Tantra} (7) defines \emph{pūjā} as a spiritual action that quells the \emph{karmic} burden flowing from previous lives, thereby putting an end to the cycle of birth and death leading to complete fulfilment. \emph{Mahānirvāṇa Tantra} (8) notes that \emph{pūjā} is the oneness of the \emph{jīva} and \emph{Ātman}.

\emph{Pūjā} has a wider connotation of worship which includes not only internal elements of concentration and meditation, but also elements of external worship like purification (\emph{Śuddhi}), divinization (\emph{nyāsa}), invocation of the deity (\emph{āvahana}), and various external offerings to a deity (\emph{upacāra}) (9). \emph{Devī} \emph{Bhāgavataṁ} (10), in fact, classifies worship into external and internal and further classifies external worship into \emph{Vedic} and \emph{Tantric}. \emph{Mahānirvāṇa} \emph{Tantra} (11) comments that the highest form of worship is abiding in the state of \emph{Brahman}-consciousness; next comes abiding in the state of meditation; practicing \emph{japa} and chanting of \emph{mantras} are lower forms of worship; and lowest form of worship is external worship. Here, the gradation is not used to depreciate the importance of external worship, but only to show how external rituals have been designed to slowly lead the individual to the higher states of meditation and inner steadfastness and ultimately lead to the highest goal of worship: being one with \emph{Brahman}.

On the relationship between \emph{Vedic} and \emph{Tantric} forms of external worship, Dr. Mahanamabrata Brahmachari (12) notes: ``From ancient times, the \emph{Vedic} philosophic truths and the \emph{Tantric} rites like worship, etc.\ have been conjoined and have thus been fulfilling each other --each compensating for the other's limitations\ldots{} On certain auspicious occasions, the performance of both the (\emph{Vedic}) \emph{Yajña} as well as (\emph{Tāntrika}) \emph{pūjā} becomes essential. \emph{Yajña} is a special contribution of the \emph{Veda-s}; \emph{pūjā}, of the \emph{Tantra-s}. \emph{Veda-s} and \emph{Tantra-s} are complementary to each other.''

Thus, the \emph{Vedic} \emph{Yajña} and the \emph{Tāntrika} \emph{pūjā} are complementary external forms of \emph{upāsanā} designed to lead a worshipper to higher meditative stages of \emph{upāsanā}.

The standard procedure of \emph{pūjā} involves 16 offerings to the deity called `\emph{Ṣoḍaśa} \emph{upacāra}'. The \emph{pūjā} usually begins with \emph{śuddhi} (purification), followed by \emph{prāṇa}yama (regulation of breath), and \emph{saṅkalpa} (intention). Then, commences the 16-staged process that involves: \emph{dhyāna} (meditation on the form of the deity), \emph{āvāhana} (invocation of the deity in \emph{vigraha}, \emph{yantra}, or \emph{Agni}), \emph{āsana} (offering the seat), \emph{pādya} (washing the deity's feet with clean water), \emph{arghya} (offering water to rinse hands and mouth), \emph{ācamana} (Offering water to drink), \emph{snāna} (bathing the deity), \emph{vastra} (offering new clothes), \emph{yajñopavīta} (offering the sacred thread), \emph{gandha} (applying sandalwood paste), \emph{puṣpa} (offering fresh flowers), \emph{dhūpa} (Incense is burned before the deity), \emph{dīpa} (waving a lamp before the deity), \emph{naivedya} (offering food and fruits to the deity), \emph{tāṁbūla} (Offering the deity a refreshing mix of betel nut and leaves), and finally \emph{pradakṣiṇā} and \emph{namaskāra} (offering prayers by doing circumambulation followed by salutation and bidding farewell to the deity).

A shorter version involves offering only 5 \emph{Upacāra-s} called \emph{Pañca}-\emph{upacāra} \emph{pūjā}, while there is also an elaborate \emph{pūjā} procedure which involves 64 offerings called \emph{Catuṣṣaṣṭi upacāra} \emph{pūjā}. In certain \emph{Tantric} traditions, \emph{Pañcamakāra} or the five M's are offered to the deities: \emph{madya} (alcohol), \emph{māṁsa} (meat), \emph{matsya} (fish), \emph{mudrā} (gesture), \emph{maithuna} (sex). In the \emph{Āgamika} tradition, this notion of invocation and worship of the deity as well as the ritual structure of \emph{Vedic} \emph{Yajña} transformed into the concept of temple and temple worship.
\vskip 2pt

A good example of how the external ritual elements are designed to bring about internal transformation and eventually lead one to higher meditation can be understood from \emph{Mahānirvāṇa} Tantra (13) that provides an internal counterpart of each of the external \emph{Upacāra-s}. It says one should offer the heart-lotus as the seat of the deity; the nectar that emanates from the \emph{sahasrāra} in the head must be utilized for \emph{padya}, \emph{arghya}, \emph{ācamana} and \emph{snāna}; the \emph{ākāśa} \emph{tattva} as the \emph{vastra}; the \emph{gandha} \emph{tattva} as the sandal-paste; the \emph{citta} or mind as the flowers; \emph{prāṇa} or the vital energy as the \emph{dhūpa}; the \emph{tejas} \emph{tattva} as the \emph{dīpa}; and the ocean of nectar as the \emph{naivedya}. It further says that one should offer the ten flowers of: lack of ignorance, lack of pride, detachment, non-bragging, absence of delusion, absence of arrogance, lack of hatred, unagitated composure, lack of envy, and absence of greed. It suggests offering five more flowers of \emph{ahiṁsā}, sense-control, compassion, forbearance and knowledge.
\vskip 2pt

In other words, the external \emph{pūjā} \emph{upacāra-s} and ritual procedures are intended to create inner transformation leading one to the development of virtues mentioned above, which in-turn will make one capable of performing inner \emph{pūjā} or meditative stages of \emph{upāsanā},, ultimately leading one to \emph{citta}-\emph{śuddhi} and \emph{ekāgra}-\emph{citta} (purification of the mind and one-pointed concentration, respectively) needed for practicing \emph{Vedānta} or \emph{jñāna} \emph{sādhanā}.
\medskip

\textbf{\emph{Upāsanā} as \emph{Bhakti}-\emph{yoga}}

The third stream of \emph{upāsanā}, which has its roots in the \emph{Upaniṣad}s, but developed its distinct flavour in the \emph{Itihāsa}-\emph{Purāṇa} tradition, especially \emph{Mahābhārata} and \emph{Bhāgavata} \emph{Purāṇa}, and was later further developed by different devotional \emph{saṁpradāya-s} was the understanding of \emph{upāsanā} as \emph{bhakti} \emph{yoga}.

Different definitions of \emph{bhakti} are available in Hindu tradition.

\emph{Nārada} \emph{Bhakti} \emph{Sūtra} (14) gives following definition of \emph{bhakti}: It is of the nature of supreme love for \emph{Īśvara}; and is of the nature of \emph{amṛta} or immortal bliss. A similar definition is found in \emph{Śāṇḍilya} \emph{Bhakti} \emph{Sūtra} (15), another authoritative text on \emph{bhakti} which says: It is the supreme unshakable attachment to \emph{Īśvara}. From the above definitions, we can discern two elements in understanding \emph{bhakti}: that it is of the nature of \emph{supreme bliss}, and that it is \emph{supreme love directed towards Īśvara}.

Enunciating on the second element, \emph{Ādi} \emph{Śaṇkarācārya} in \emph{Śivānandalaharī} (16) defines \emph{bhakti} as: ``When the modification or thought-patterns of the mind reaches the lotus feet of \emph{Paśupati} and becomes seated there permanently, then that condition is called \emph{bhakti}.'' Madhusudhana Saraswati provides a more elaborate definition in \emph{Bhagavad}-\emph{Bhakti} \emph{Rasāyana} (17) where he says: ``The \emph{manasa} \emph{vṛtti} or the modification of the mind melted by the (practice of) \emph{bhāgavad}-\emph{dharma} that has become a continuous, stream-like flow directed toward \emph{Sarveśa} or the lord of all is called \emph{bhakti}.''

In short, at the highest level, \emph{bhakti} is the constant abidance in the supreme bliss of \emph{Īśvara} by a person with the melted mind. It is a practice of meditation by those with melted mind wherein the object of meditation is \emph{Īśvara} or one of his manifestations as \emph{devatā-s} with whom a personalized relationship of love is cultivated and the goal is complete merger with \emph{Īśvara}. \emph{Bhakti} is thus a spontaneous extension, a logical conclusion of the \emph{Upaniṣad}ic \emph{Saguṇopāsanā}. However, this practice of \emph{bhakti} or \emph{bhakti} \emph{yoga} is different from both yogic \emph{dhyāna} and Vedantic \emph{nididhyāsana} in one significant way. While \emph{bhakti} \emph{yoga} can be practiced only by the melted mind, melting of the mind is not required for the practice of the other two.

Madhusudhana Saraswati (18) says that in such a mind melted by devotion as described above, \emph{Īśvara} himself enters and becomes established permanently just as a colour added to the melted wax becomes forever established in the wax!

However, that is only the description of \emph{bhakti} at the highest level. To reach there, a practitioner has to first go through different stages of devotion and practice lower forms of devotion. In \emph{Bhāgavata} \emph{Purāṇa} (19), \emph{Prahlāda}, the famous devotee of Lord \emph{Narasiṁha} enunciates the nine-kinds of devotional practices thus: ``Hearing, chanting, remembering, serving the feet, offering ritual worship, offering prayers, serving as a servant, becoming the best friend and surrendering one's ownself to \emph{Viṣṇu}.''

The last practice called `\emph{Ātma} \emph{nivedanaṁ}' or the surrendering of one's Self completely to \emph{Īśvara} implies complete merger with \emph{Īśvara}, the highest goal of \emph{bhakti}. Regarding this, Madhusudhana Saraswati says in his commentary to \emph{Bhagavad} \emph{Gītā} called \emph{Gūḍhārthadīpikā} (20) thus: ``With the maturity of spiritual practice, three types of surrender to God come about- `I belong to him indeed', `He belongs to me indeed' and `I am He indeed'.'' These are the three stages of \emph{bhakti} through which a devotee invariably goes through as his \emph{bhakti} matures more and more. But to reach the mature stage one must first practice various devotional practices like hearing, singing etc.\ that help in developing one-pointed concentration and meditation on \emph{Īśvara}.

Thus, \emph{upāsanā} as \emph{bhakti} \emph{yoga} is a special kind of meditation practice rooted in devotion and \emph{Saguṇa} worship. It intends to purify the mind of the practitioner and help him develop one-pointed concentration through devotion and development of personal connection with \emph{Īśvara} who responds to the devotee by becoming his \emph{Iṣṭa} \emph{Devatā} or favorite deity. Here the path starts with singing bhajans, chanting \emph{mantras}, listening to stories of \emph{purāṇa-s}, and slowly leads one to higher meditative stages of \emph{upāsanā} ultimately culminating with the manifestation of \emph{Īśvara} in one's own \emph{hṛdaya} and eventual merging of \emph{jīva} with \emph{Īśvara}.

Through these three streams of \emph{dhyāna}, \emph{pūjā}, and \emph{bhakti}-\emph{yoga}, \emph{devatopāsana} intends to take a spiritual practitioner to higher practices of meditation by facilitating \emph{ekāgra}-\emph{citta} or one-pointed concentration and \emph{citta}- \emph{śuddhi} or purification of the mind and eventually lead one to either \emph{jīvanmukti} or \emph{kramamukti}.

\section*{References}

\begin{enumerate}
\itemsep=0pt
\item
Swami Achalananda, Introduction, `Meaning and Significance of Worship', Page 8, Sri Ramakrishna \emph{āśrama}, Mysuru.
\item
\emph{Īśopaniṣad} Verse 15-16
\item
\emph{Kenopaniṣad} 4.4-6
\item
K Narayanaswami Aiyar, `The Thirty-Two \emph{Vidyā}-s', The Adyar Library Series.
\item
Swami Sivananda, `Vidyas from the \emph{Upaniṣad}s', The Divine Life Society. [\url{http://sivanandaonline.org/public_html/?cmd=}
\url{displaysectionandsection_id=759}]
\item
K Narayanaswami Aiyar, `The Thirty-Two \emph{Vidyā}-s', Page 68, The Adyar Library Series.
\item
\emph{Kulārṇava} \emph{Tantra} 17.70
\item
\emph{Mahānirvāṇa} \emph{Tantra} 14.123
\item
Swami Muktidananda, A Word with the Worshipper, `Meaning and Significance of Worship', Page 12, Sri Ramakrishna \emph{āśrama}, Mysuru.
\item
\emph{Devī} \emph{Bhāgavataṁ} 7.39.3
\item
\emph{Mahānirvāṇa} \emph{Tantra} 14.122
\item
Dr.\ Mahanamabrata Brahmachari, Candi Cinta (Bengali), Page 9; Cited from `Meaning and Significance of Worship', Page 155, Sri Ramakrishna \emph{āśrama}, Mysuru
\item
\emph{Mahānirvāṇa} \emph{Tantra} 5.143-149.
\item
\emph{Nārada} \emph{Bhakti} \emph{Sūtra}, Verse 2-3
\item
\emph{Śāṇḍilya} \emph{Bhakti} \emph{Sūtra} Verse 2
\item
\emph{Śivānandalaharī} Verse 61
\item
\emph{Bhagavad}-\emph{Bhakti}- \emph{Rasāyana} Verse 3
\item
\emph{Bhagavad}-\emph{Bhakti}- \emph{Rasāyana} Verses 8-10
\item
\emph{Bhāgavata} \emph{Purāṇa} Verse 7.5.23
\item
\emph{Gūḍhārthadīpikā} on \emph{Bhagavad} \emph{Gītā} Verse 18.66
\end{enumerate}
\newpage

\section*{3.5. \emph{Kramamukti}}

While those in the path of \emph{nivṛtti} with intense dispassion and other required competencies practice \emph{jñāna} \emph{sādhanā} and attain liberation here and now even while living in the body (i.e.\ jīvanmukti); others who are on the path of \emph{pravṛtti}, who are householders, who still have desires for worldly pleasures intact, have to take refuge in \emph{karma} and \emph{upāsanā} and attain liberation in a gradual manner. This gradual process of liberation is called \emph{kramamukti}.

In previous chapters, we saw in some depth what constitutes the practice of \emph{karma} and \emph{upāsanā}. The \emph{Upaniṣad-}s give a detailed account about what are the fruits of such practices. They have enunciated about how a life lived through a persistent practice of \emph{karma} and \emph{bhakti} determines one's further journey after the death of the physical body. They note that after death, a \emph{jīva} would either travel through the path of \emph{devayāna} to \emph{devaloka} and eventually to \emph{Brahmaloka}; or through the path of \emph{pitṛyāna} to \emph{pitṛloka} and to \emph{candraloka}.

\emph{Bṛhadāraṇyakopaniṣad} (1) says: ``There are indeed three worlds, the world of men (\emph{manuṣyaloka}), the world of the manes (\emph{pitṛloka}) and the world of the Gods (\emph{devaloka}). This world of men is to be won through the son alone, and by no other rite; the world of the \emph{Pitṛ-s} through action (\emph{karma}); and the world of the \emph{deva-s} through meditation (\emph{Vidyā}/\emph{upāsanā}). The world of the Gods is the best of the worlds. Therefore they praise meditation.''

Therefore, through the practice of \emph{karma} and \emph{dharma} \emph{anuṣṭhāna} one takes the path of \emph{pitṛyāna} after death and attains \emph{pitṛloka}. On the other hand, those who practice \emph{upāsanā} on \emph{Īśvara} or on \emph{devatā-s}, attain \emph{devaloka} by travelling the path of \emph{devayāna}. However, these two paths do not by themselves lead to one to the ultimate \emph{mokṣa}.
\newpage

\emph{Bṛhadāraṇyakopaniṣad} further elaborates about the journey through \emph{devayāna} and \emph{pitṛyāna}. About \emph{devayāna}, it says (2): ``Those who know this as such, and those others who meditate with faith upon the \emph{Satya} \emph{Brahman} in the forest, reach the deity identified with the flame, from him the deity of the day, from him the deity of the fortnight in which the moon waxes, from him the deities of the six months in which the sun travels northward, from them the deity identified with the world of the gods, from him the sun, and from the sun the deity of lightning. (Then) a being created from the mind (of \emph{Hiraṇyagarbha}) comes and conducts them to the \emph{Brahmaloka}-s. They attain perfection and live in those \emph{Brahmaloka-s} for a great many superfine years. They no more return to this world.'' Then, about \emph{pitṛyāna}, it says (3): ``While those who conquer the worlds through sacrifices, charity and austerity, reach the deity of smoke, from him the deity of the night, from him the deity of the fortnight in which the moon wanes, from him the deities of the six months in which the sun travels southward, from them the deity of the world of the Manes, and from him the moon. Reaching the moon they become food. There the gods enjoy them as the priests drink the shining \emph{Soma} juice (gradually, saying, as it were), flourish, dwindle. And when their past work is exhausted, they reach (become like) this ether, from the ether air, from air rain, and from rain the earth. Reaching the earth they become food. Then they are again offered in the fire of man, thence in the fire of woman, whence they are born (and perform rites) with a view to going to other worlds. Thus do they rotate.''

While the \emph{pitṛyāna} implies that one must necessarily have to return to human realm and the \emph{karmic} cycle of birth and death continues; even in the case of \emph{devayāna} one may have to return to human or other realms if the method of meditation or its intensity is of lower kind, or even if they do not return in this cycle, they may eventually return in a different cycle of creation. Note that in the context of \emph{devayāna}, the \emph{Bṛhadāraṇyakopaniṣad} quoted above mentions \emph{Brahmaloka-s} i.e.\ in plural. \emph{Ādi} \emph{Śaṇkarācārya} in his commentary on the verse notes that the plural has been used to denote that there are higher and lower planes in \emph{Brahmaloka} based on different gradations in meditation. That is, those who practice \emph{upāsanā}, based on the method of meditation (i.e.\ whether it is on a particular form, or a particular \emph{tattva}, or is it on \emph{Satya} \emph{Brahman}), the intensity and maturity of meditation, one attains different realms of \emph{Brahmaloka}. The \emph{Upaniṣad} says that those who meditate using \emph{Satya} \emph{Brahman} or meditate using \emph{Pañcāgni} \emph{Vidyā} and such attain \emph{Brahmaloka} such that they do not return to this world. \emph{Ādi} \emph{Śaṇkarācārya} further notes that they may return to some other realms, or even to human realm in some other cycle, with the `non-return' being in reference to only this human realm in this cycle. From this it follows that others who practice a different, but lower forms of meditation or meditation of lower intensity may even return to human existence within this cycle as well.

Hence, \emph{Īśopaniṣad} (4) suggests a combination of \emph{karma} and \emph{upāsanā} as a means for \emph{kramamukti}. By such a combination, one attains the highest realm of \emph{Brahmaloka} and stays there until the end of the life of \emph{Brahma}. In the \emph{paurāṇika} tradition, the highest realms of \emph{Brahmaloka} (or sometimes conceived as higher than \emph{Brahmaloka}) have been variously named as \emph{Vaikuṇṭha}, \emph{Kailāsa}, etc.\ depending upon which theistic philosophy the particular text is speaking about. \emph{Ādi} \emph{Śaṇkarācārya} in his commentary on \emph{Brahmasūtra} (5) notes that the \emph{Jīva}s who stay in \emph{Brahmaloka}, having attained the Knowledge of the Self (\emph{Ātma} \emph{jñāna}) by the time the end of life of \emph{Brahmā} nears, they along with \emph{Brahmā} attain the highest state i.e.\ the final liberation or establishment in \emph{Para}-\emph{Brahman}. \emph{Ādi} \emph{Śaṇkarācārya} terms this kind of liberation by successive stages as \emph{kramamukti}.

Thus, the attainment of \emph{Brahmaloka} or the realms of respective \emph{Iṣṭa} \emph{Devatā-s} constitutes the first stage of the multi-staged process of \emph{kramamukti}. The later \emph{paurāṇika-tāntrika-bhakti} traditions actually identify four-stages in \emph{kramamukti}. They are: \emph{sālokya, sāmīpya, sārūpya} and \emph{sāyujya} (6).

\emph{Sālokya} refers to attaining the \emph{loka} of the \emph{Saguṇa} \emph{Brahman} or one's \emph{Iṣṭa} \emph{Devatā} after the death of physical body. Here, the \emph{jīva} is expected to continue with its \emph{upāsanā} and as it matures, the \emph{jīva} will attain the next stage, which is \emph{sāmīpya}, wherein one develops proximity to \emph{Īśvara}. The third-stage is \emph{sārūpya}, wherein as a fruit of further maturation of meditation, one assumes the form of the deity. The final stage is \emph{sāyujya}, wherein one completely merges with the deity/\emph{Īśvara}.

Here, the merger could be with a particular form, in which case, the \emph{jīva} who has now become identified with a \emph{Devatā} or with \emph{Saguṇa} \emph{Brahman} attains final liberation at the end of life of \emph{Brahmā}, during which time \emph{Brahmaloka} itself dissolves and \emph{Saguṇa} \emph{Brahman} becomes reabsorbed into \emph{Nirguṇa} \emph{Para}-\emph{Brahman}. However, the \emph{sāyujya} or union with \emph{Īśvara} can also be \emph{kaivalyaṁ} or the complete cessation of duality owing to attainment of \emph{ātmajñāna}, in which case, the \emph{jīva} attains absolute liberation. In the latter case, \emph{Īśvara} or one's \emph{Iṣṭa Devatā} itself performs the role of \emph{guru} and instructs the devotee in the \emph{Upaniṣad}ic \emph{Mahāvākya}s facilitating the devotee attain Self-Knowledge and attain \emph{Mokṣa}.

This is the path of \emph{kramamukti} or the gradual process of liberation.

\section*{References}

\begin{enumerate}
\itemsep=0pt
\item
\emph{Bṛhadāraṇyakopaniṣad} Verse 1.5.16
\item
\emph{Bṛhadāraṇyakopaniṣad} Verse 6.2.15
\item
\emph{Bṛhadāraṇyakopaniṣad} Verse 6.2.16
\item
\emph{Īśopaniṣad} Verse 11
\item
\emph{Ādi} \emph{Śaṇkarācārya}'s \emph{BrahmaSūtra} \emph{Bhāṣya} on Verse 4.3.10
\item
Some like Sri Bhaskaracharya, in his \emph{Tripura} \emph{Upaniṣad} \emph{Bhāṣya} (Verse 1) adds `\emph{kaivalya'} or \emph{jīvanmukti} as the fifth and final stage of Liberation. But, \emph{kaivalya} can be understood as the highest stage of \emph{sāyujya} itself as far as \emph{kramamukti} is considered. \emph{Śrī Caitanya Caritāmṛta} (Verse 6.266) belonging to \emph{Vaiṣṇava} tradition mentions five kinds of \emph{mukti}, adding ``\emph{sārṣṭi}'' to the list of four. \emph{Śivānandalaharī} (Verse 28) attributed to \emph{Ādi} \emph{Śaṇkarācārya} also makes a mention of the four-kinds of \emph{mukti}.
\end{enumerate}
