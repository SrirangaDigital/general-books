\chapter{Verse 14}

\begin{moolashloka}
\dev{सम्भूतिं च विनाशं च यस्तद्वेदोभयं सह ।}\\
\dev{विनाशेन मृत्युं तीर्त्वा सम्भूत्याsमृतमश्नुते ॥ 14 ॥}
\end{moolashloka}

\textbf{Word to word Meaning}: The (un)manifested and the destruction (\dev{सम्भूतिं च विनाशं च}) who (\dev{यः}) knows (\dev{वेद}) that (\dev{तत्}) both together (\dev{उभयं सह}); By the destruction (\dev{विनाशेन}) death (\dev{मृत्युं}) transcends (\dev{तीर्त्वा}); by the (un)manifest (\dev{सम्भूत्या}) immortality (\dev{अमृतम्}) attains (\dev{अश्नुते}).

\textbf{Meaning:} He who knows both the (un)manifested and the destruction (i.e.\ manifested) together, he transcends death (i.e.\ limitations) by the manifested (whose nature is to get destroyed at the end, hence called \dev{विनाश}) and attains (relative) immortality by the unmanifested (whose nature is to manifest hence called \dev{सम्भूत्या}).

\textbf{Analysis:} Having explained about how the \emph{sakāma} worship of \emph{Kārya} or \emph{Kāraṇa Brahman} alone will not help a spiritual seeker as the results obtained by each of them are distinct from the other, now the \emph{Upaniṣad} explains the manner in which the \emph{upāsanā} of \emph{Kārya} and \emph{Kāraṇa Brahman} must be performed to attain \emph{mokṣa} through \emph{kramamukti}.

\dev{सम्भूतिं च विनाशं च यस्तद्वेदोभयं सह}- \emph{He who knows the (un)manifested and the destruction (manifested) both together.} In this verse, the usage of the words \dev{सम्भूतिं} and \dev{विनाशं} is significant. The word \dev{विनाशं} means `\emph{that which gets destroyed}'. In other words, \dev{विनाशं} refers to the manifested cosmos, which was born out of \emph{mūla prakṛti} and which will ultimately undergo destruction (\emph{laya}). Hence, \dev{विनाशं} means `\emph{Kārya} \emph{Brahman'}- the manifested. Similarly, \dev{सम्भूतिं} (1) in this context \emph{does not} mean `the manifested'. Instead it means `\emph{that which manifests the cosmos}' i.e.\ \dev{सम्भूतिं} means `\emph{Kāraṇa Brahman}', the unmanifested \emph{prakṛti} which manifests the whole cosmos. The meaning of \dev{विनाशं} to be \emph{Kārya} \emph{Brahman} is clearly understood from the next line of the verse, where it says that from \dev{विनाशं} one overcomes death and other limitations. As only a worship of \emph{Kārya} \emph{Brahman} will grant control over the forces of nature and hence overcome limitations, the term \dev{विनाशं} in this context means `\emph{Kārya} \emph{Brahman}', the manifested cosmos. Hence, the \emph{Upaniṣad} is saying that, a person who having thus understood that the worship of the \emph{Kārya} \emph{Brahman} and that of \emph{Kāraṇa Brahman} will bear distinct results, will begin practicing both of them together in order to attain \emph{mokṣa}.

\dev{विनाशेन मृत्युं तीर्त्वा सम्भूत्या अमृतमश्नुते} \emph{--By the (worship of) manifested, one overcomes death and by the (worship of) the (un)manifested one attains Immortality}. Now the \emph{Upaniṣad} explains that a person who practices the worship of \emph{Kārya} and \emph{Kāraṇa Brahman} together in a \emph{sakāma} way with a desire to attain both, will overcome limitations of subtle existance (\emph{sūkṣma śarīra}) by the worship of \emph{Kārya} \emph{Brahman} and will attain immortality by the worship of \emph{Kāraṇa Brahman}. Here also, the term \dev{अमृतम्} is significant. It refers to `\emph{relative immortality}' and not `absolute Immortality' (\emph{mokṣa}/\emph{jñāna}).

As explained previously, absolute immortality is attained, only by the realization of \emph{Ātman} (i.e.\ \emph{Ātmajñāna} or \emph{Brahmajñāna}). Only a person who has developed \emph{viveka}, etc.\ understands that it is the same \emph{Brahman} who exists as both \emph{Kārya} and \emph{Kāraṇa}; \emph{Brahman} alone exists both as the unmanifested source and as the manifested cosmos. Due to the development of such \emph{viveka}, a person would worship in a detached manner both the \emph{Kārya} and \emph{Kāraṇa}, by making \emph{Brahman} alone as his sole object of worship. Such a person would overcome the limitations of both \emph{Kārya} and \emph{Kāraṇa} by attaining \emph{Brahman} through both of them. In other words, he achieves \emph{Brahmajñāna} by worship of \emph{Brahman} who pervades everywhere. This is called `absolute immortality'.

On the other hand, he who performs the worship of \emph{Kārya} and \emph{Kāraṇa} \emph{Brahman} in a \emph{sakāma} way, i.e.\ with a desire to attain both the \emph{Kārya} and \emph{Kāraṇa} \emph{Brahman}, he will first acquire the \emph{siddhi-s} by the worship of \emph{Kārya} \emph{Brahman} by which he will overcome the limitations imposed on the individual by \emph{sūkṣma śarīra} i.e.\ the limitations imposed by subtle existence like grief, sorrow, limited powers, illness like dumbness, blindness, etc.\ These unlimited-powers will lead him to overcome his limited personality. Then, due to his worship of \emph{Kāraṇa Brahman}, he will lose his subtle-individuality and merge into the unmanifested \emph{prakṛti}. But, in this case, owing to his overcoming of the limitations of \emph{Kārya} \emph{Brahman}, he would stay in the unmanifested \emph{prakṛti} in an undifferentiated form without having to take birth into manifested cosmos. This absorption into unmanifested \emph{prakṛti}, without having to take rebirth ever again is called \emph{prakṛti} \emph{laya}. This absorption, \emph{prakṛti} \emph{laya}, is referred as \dev{अमृतम्} or immortality in the verse. This is only a relative immortality. As explained before, it is called \dev{अमृतम्} because of its nearness to absolute \emph{Brahman}. Such a person, who has overcome his subtle and gross existences and has become absorbed into \emph{mūla prakṛti} is still limited by his causal existence. He then overcomes his causal existence and attains \emph{mokṣa} (absolute liberation) only at the end of the life of \emph{Brahmā} when everything merges back into \emph{Brahman}. This is the path of \emph{kramamukti}.

\textbf{Summary:} After explaining the futility of practicing the \emph{sakāma} worship of the \emph{Kārya} or \emph{Kāraṇa} \emph{Brahman} alone, the \emph{Upaniṣad} now gives instructions about the correct manner in which the \emph{devatopāsana} has to be done and how it would lead to \emph{mokṣa}.

Those people, who worship the \emph{Kārya} and \emph{Kāraṇa} \emph{Brahman} together with a desire to attain the \emph{Kārya} and \emph{Kāraṇa} \emph{Brahman}, would, by the worship of \emph{Kārya} \emph{Brahman}, acquire the \emph{siddhi-s} to control different aspects of subtle existence. By these powers they would overcome the limitations imposed by subtle existence. Further, by the worship of \emph{Kāraṇa Brahman}, they would attain (get absorbed into) \emph{Kāraṇa Brahman} having already overcome their subtle individuality (\emph{sūkṣma śarīra}). There, in that \emph{mūla prakṛti} they will exist in an undifferentiated state, without re-birth, without having to come to \emph{saṁsāra} till the end of life of \emph{Brahmā}. This is the relative immortality. They then overcome causal existence and attain \emph{Mokṣa} at the end of life of \emph{Brahmā}. Thus, the \emph{Upaniṣad} explains to the \emph{pravṛttimārgī-s} (Householders) who are still attached to worldly desires, the manner in which \emph{Kārya} and \emph{Kāraṇa} \emph{Brahman} must be worshipped in order to attain \emph{kramamukti}.

\section*{References}

\begin{enumerate}
\itemsep=0pt
\item
  Some are also of the opinion that, in the first line, `\emph{asaṁbhūti}' has been mentioned as `\emph{saṁbhūti}' by removing the letter `\dev{अ}' (a) according to the rule `\dev{एङः पदान्तात् अति}' of \emph{Pāṇini Sūtra-s} (6.1.109). And in the second line, the `\dev{अ}' (a) of `\emph{asaṁbhūti}' has merged with the previous word and hence appears as `\emph{saṁbhūti}' according to the rule `\dev{अकः सवर्णे दीर्घः}' of \emph{Pāṇini Sūtra-s} (6.1.101). Hence, \emph{vināśa} and \emph{saṁbhūti} in the verse refers to `\emph{Kārya Brahman}' and `\emph{Kāraṇa Brahman}' respectively.
\end{enumerate}
