\chapter{Foreword}

The \emph{Īśopaniṣad} is one of the shortest \emph{upaniṣads}, yet it is one of the most important inasmuch as it encapsulates within it the entire schema of the pursuit of \emph{para}-\emph{vidyā}, the Supreme Knowledge. Today, teaching about Hinduism has become almost synonymous with teaching \emph{vedānta} with the result that Hinduism is being taught in a highly skewed manner with an overriding emphasis on \emph{vedānta} to the exclusion of the \emph{dharma}-\emph{kṣetra}, consisting of both \emph{pravṛtti-dharma} and \emph{nivṛtti-dharma}, within which \emph{vedānta} appears as its supreme revelation. But as \emph{Śrī Kṛṣṇa} says in the \emph{Bhagavad Gītā}, the attainment of Supreme Knowledge is possible for only a select few.

\begin{quote}
\emph{Out of many thousands among men, one may endeavor for perfection, and of those who have achieved perfection, hardly one knows Me in truth. (Bhagavad Gītā 7.3)}
\end{quote}

The pursuit of \emph{nivṛtti-dharma} is meant for only those few people in whom the desire for worldly objects has been extinguished. For others, the chain that binds them to the world has to be gradually disentangled by the pursuit of \emph{pravṛtti-dharma}, the \emph{dharma} of action, before they can acquire the degree of mental purity required for treading the path to \emph{mukti}. This aspect of \emph{Sanātana Dharma} has been largely ignored today. There is thus an acute need to bring about a correction in perspective among the people so that all and sundry do not neglect \emph{pravṛtti-dharma} in favour of pursuing the highest goal of \emph{jīvanmukti} even when they would be lacking in the necessary qualification for pursuing the path of \emph{nivṛtti-dharma}. Nithin Sridhar's commentary on the \emph{Īśopaniṣad} fulfils this need in an admirable way. Taking the inner kernel of the meanings of the first three verses of the \emph{Īśopaniṣad}, Nithin Sridhar fleshes out their meanings to give us a panoramic view of the paths to liberation as laid out in \emph{Sanātana Dharma}. The author does this in two ways: first by expounding the meanings of the individual verses of the \emph{upaniṣad} and second by presenting, in the second half of the book, the philosophy of the text of the \emph{Īśopaniṣad} in a comprehensive manner.

The second part of the book explains in a succinct and concise way the entire terrain \emph{of Sanātana Dharma}. Beginning with the theme of the \emph{upaniṣad} and the \emph{puruṣārtha}-s- the four goals of human existence - it moves on to an exposition of \emph{nivṛtti-dharma}, the path to \emph{jīvanmukti} or immediate liberation (\emph{sadyomukti}). It explains the nature of Brahman and the four qualifications - known as \emph{sādhanā} \emph{catuṣṭaya} --required for pursuing the path of \emph{nivṛtti-dharma} as well as the method of \emph{śravaṇa}, \emph{manana}, and \emph{nididhyāsana} leading to \emph{ātma-jñāna} / \emph{aparokṣa}-\emph{jñāna}. The state of a \emph{jīvanmukta}, when the residual \emph{prārabdha} continues to hold sway until the final fall of the body, and the consequent attainment of \emph{videhamukti} is also explained. In the next section, the author takes up \emph{pravṛtti-dharma} in all its varied aspects. This section is indeed the high point of the book. While the path of \emph{nivṛtti-dharma} has been explained by most modern commentators of the \emph{upaniṣads}, the path of \emph{pravṛtti-dharma} / \emph{kramamukti} is an area that has been sadly neglected by most modern scholars as well as commentators. Nithin Sridhar fills this vacuum with a comprehensive coverage of the path. The author builds up the background for the path of \emph{kramamukti} by an exposition of \emph{Brahman} as the creator and as both the material and the efficient cause of the universe, creating, sustaining and dissolving the universe by His inscrutable \emph{māyā}-\emph{śakti}. The sheer scope and vastness of the conception of creation is brought out by a description of the various \emph{lokas}, or spheres of creation. The author then moves on to a full-fledged exposition of the path of \emph{pravṛtti-dharma} / \emph{kramamukti} starting from the relation between \emph{karma} and \emph{dharma}, the explanations of \emph{sāmānya} and \emph{viśeṣa dharma}, the classification of \emph{varṇa} and \emph{āśrama} as the pivots around which \emph{karma} revolves, the distinction between various types of \emph{karma} such as \emph{vihita} \emph{karma}, \emph{kāmya} \emph{karma}, \emph{niṣiddha} \emph{karma}, etc. and the importance of \emph{vihita} \emph{karma} for obtaining the required mental purity for transitioning to the paths of \emph{nivṛtti}. This section serves as a very useful compendium on \emph{karma} with explanations of the fourteen \emph{saṁskāras} from conception to marriage, the five \emph{mahāyajñas} or great sacrifices to be performed on a daily basis, the twenty one \emph{yajña}s that a householder is expected to perform at various times, and the \emph{antyeṣṭi} \emph{saṁskāra} or the last rite to be performed upon death.

In the case of a few persons, due to the most assiduous performance of \emph{vihita} \emph{karmas} and the avoidance of \emph{niṣiddha} \emph{karmas}, it results in the rise of supreme \emph{viveka} along with all the four qualifications and makes the person fit to tread the path of \emph{nivṛtti} or \emph{saṃnyāsa}. In most people however, the performance of \emph{karma} will lead to the path of \emph{kramamukti}. This is the path in which \emph{upāsanā} forms the central motif. The author has explained how \emph{upāsanā} may take the forms of \emph{dhyāna, pūjā} or \emph{bhakti}-\emph{yoga} and ultimately to the stage of leaving the body by the `northern path' upon death to travel to \emph{Brahmaloka} and reside therein in bliss until the end of the \emph{kalpa} and then to be absorbed totally in \emph{Brahman}. The last section of the book deals with the third path -- the path to suffering -- in which the \emph{jīva} by neglecting the path of virtuous action is hurled again and again into the cycle of births and deaths.

Nithin Sridhar's commentary on the \emph{Īśopaniṣad}, as well as the exposition of the \emph{Upaniṣad}'s philosophy, is written from the perspective of \emph{Advaita} \emph{Vedānta}. But there is no polemics in the work; it focuses entirely on shedding light upon some of the most important and often neglected aspects of Hinduism. This book is yet another great addition to the five books authored by Nithin Sridhar and it cannot but be highly recommended.
\bigskip

\begin{flushright}
\textbf{---\emph{Śrī} Chittaranjan Naik}\\
\textbf{Director of \emph{Gautama Academy of Indian Intellectual Traditions} and the author of `\emph{Natural Realism and Contact Theory of Perception}',}\\
\textbf{Mumbai, India.}
\end{flushright}
