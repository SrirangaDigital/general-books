\chapter{Verse 15}

\begin{moolashloka}
\dev{हिरण्मयेन पात्रेण सत्यस्यापिहितं मुखम ।}\\
\dev{तत्त्वं पूषन्नपावृणु सत्यधर्माय दृष्टये ॥ 15 ॥}
\end{moolashloka}

\textbf{Word to word Meaning}: By the golden (\dev{हिरण्मयेन}) vessel (\dev{पात्रेण}) is covered/concealed (\dev{अपिहित}) the face (\dev{मुखम}) of the Truth (i.e.\ \emph{Satya Brahman}) (\dev{सत्यस्य}); that (\dev{तत्}) you (\dev{त्व}) Pūṣan/Sun (\dev{पूषन}) do open/remove (\dev{अपावृणु}) for me who has practiced the tenets/law of Truth (\dev{सत्यधर्माय}) to see (\dev{दृष्टये}).

\textbf{Meaning:} The face of the Truth (i.e.\ \emph{Satya Brahman} / \emph{Hiraṇyagarbha}) is covered/ concealed by a golden (brilliant/resplendent) vessel. Oh \emph{Pūṣan} (The Sun who nourishes everything), please open that (vessel), so that I who have practiced the \emph{Satya Dharma} (upāsanā of \emph{Satya Brahman}) can see (the \emph{Satya Brahman}).

\textbf{Analysis:} Having explained about need for the householder to practice both \emph{karma} and \emph{vidyā upāsanā} together and the general manner in which he must practice \emph{vidyā upāsanā} by worshipping both the \emph{Kārya} and \emph{Kāraṇa} \emph{Brahman} together, the \emph{Upaniṣad} now gives specific instruction about the worship of \emph{Hiraṇyagarbha}, through a prayer to the Sun of a person who having practiced \emph{Karma} and the worship of \emph{Hiraṇyagarbha} all his life is now on the verge of death.

\dev{हिरण्मयेन पात्रेण सत्यस्यापिहितं मुखम}- \emph{The face of the Truth (Satya Brahman) is covered by a golden vessel.} \dev{सत्य} literally means `\emph{Truth'}. In this context, it refers \emph{Hiraṇyagarbha} or \emph{Kārya Brahman} called `\emph{Satya Brahman}'. \emph{Hiraṇyagarbha} is called \dev{सत्य} because he includes both the gross and the subtle entities (1). He projects both the gross and the subtle universes. Further, the scriptures clearly call \emph{Satya Brahman} as `\emph{Prathamajam}' (2) meaning `First Born' and goes on to explain that he was the first to manifest from Unmanifested \emph{Prakṛti} (3). This clearly establishes that, the \dev{सत्य} here refers to \emph{Kārya Brahman} / \emph{Hiraṇyagarbha}.

So, the \emph{Upaniṣad} is saying that, the face of this \emph{Satya Brahman} is covered by a vessel. What is this vessel? It is the \emph{sūrya maṇḍala}, the solar orb. It is \emph{Satya Brahman}, who exists as `Sun' (\emph{Sūrya} / \emph{Āditya} / \emph{Pūṣan}) in the solar orb (4). This Solar orb, along with the rays that come out of the Sun, acts as a vessel which hides the face of \emph{Hiraṇyagarbha}. \dev{हिरण्मयेन} means `Golden'. The \emph{sūrya maṇḍala} is being referred as `Golden' because of its brightness of its rays that cloud the vision from seeing the Truth.
So, the \emph{Upaniṣad} is saying that, the face of this \emph{Satya Brahman} is covered by a vessel. What is this vessel? It is the \emph{sūrya maṇḍala}, the solar orb. It is \emph{Satya Brahman}, who exists as `Sun' (\emph{Sūrya} / \emph{Āditya} / \emph{Pūṣan}) in the solar orb (4). This Solar orb, along with the rays that come out of the Sun, acts as a vessel which hides the face of \emph{Hiraṇyagarbha}. \dev{हिरण्मयेन} means `Golden'. The \emph{sūrya maṇḍala} is being referred as `Golden' because of its brightness of its rays that cloud the vision from seeing the Truth.

\dev{तत्त्वं पूषन्नपावृणु सत्यधर्माय दृष्टये}- \emph{Oh Pūṣan, please remove/open that (vessel), so that I who have practiced Satya Dharma (i.e.\ worship of Satya Brahman) can see (Satya Brahman).} \dev{पूषन्} literally means `he who nourishes all beings'. It refers to the solar deity in the solar orb. It refers to \emph{Satya Brahman} Himself who exists as \emph{Pūṣan}. \dev{सत्यधर्माय} literally means ``tenets of Truth'' and in this context refers to the practice of the tenets of worship of \emph{Satya Brahman}. What are those tenets? The \emph{Bṛhadāraṇyakopaniṣad} explains this in depth. It instructs that a person should meditate on \emph{Satya Brahman} as being present in the \emph{sūrya maṇḍala} and also in the right eye (i.e.\ in an individual himself) (5). The deity who is present in the solar orb and he who is present in the right eye is \emph{Satya Brahman} alone but in two different forms. It further says that, the former rests on the latter through its rays and the latter rests on the former through the functions of the eyes (6). And when the death engulfs a person, he would see the solar orb clearly as the rays have receded from him. It is this meditation that is being expressed by the present \emph{Upaniṣad} through the prayer of a person who has approached death. The \emph{Upaniṣad} is saying that, a person who having worshipped the \emph{Kārya Brahman}, the \emph{Hiraṇyagarbha} as being present in the Sun and in his own right eye for his whole life, has now approached death and he requests the \emph{Pūṣan}, the solar deity to remove the rays that are clouding his vision and show him the real face of \emph{Satya Brahman}. \emph{Muṇḍakopaniṣad} also speaks about this. It says that those people who have practiced tapas (i.e.\ \emph{karma}) and \emph{śraddhā} (i.e.\ \emph{devatopāsana}), travel through the door of the Sun to the immortal \emph{Hiraṇyagarbha} (7). That is, those people who practice Karma and Vidya both together and worship the \emph{Kārya Brahman} as \emph{Satya} who resides in the Solar orb and also in the right eye; such a person at death, will travel through the door of the sun to \emph{Satya Brahman} (i.e.\ \emph{Brahmaloka} or \emph{Hiraṇyagarbha}). This ``\emph{sūryadvāra}'' or door of the Sun is referred by this \emph{Upaniṣad} as \dev{हिरण्मयेन पात्रेण}. Hence, a person on the verge of death is requesting \emph{Satya Brahman} who exists as \emph{Pūṣan}, the solar deity to remove his rays and open the door to Himself, so that he, who has practiced \emph{karma} and \emph{upāsanā} of \emph{Satya Brahman}, may `See' Him, by attaining Him.

\textbf{Summary:} In this verse, the \emph{Upaniṣad} gives a prayer of a \emph{pravṛttimārgī}, the householder who has approached death. Through this prayer, the \emph{Upaniṣad} gives instructions about the manner in which the \emph{Hiraṇyagarbha} has to be meditated upon. The \emph{Hiraṇyagarbha} is called `\emph{Satya'}. He exists both in the \emph{sūrya maṇḍala} and in the right eye. He who knowing thus has practiced both \emph{Karma} and the worship the \emph{Satya Brahman} all his life, now having approached death, requests the \emph{Pūṣan}, the Sun deity to remove his rays that clouded his vision and show him the face of Satya-\emph{Brahman}. He is requesting the \emph{Pūṣan} to lead him (through the northern path) to the abode of Satya-\emph{Brahman}.

\section*{References}

\begin{enumerate}
\itemsep=1.5pt
\item
  `\emph{Satya Brahman}' means \emph{Brahman} that is `Sat' and that is `Tyat'. `Sat' refers to gross entities and ``Tyat'' refers to subtle entities. (Taittirīyopaniṣad 2.6.1). Hence, \emph{Kārya Brahman} is called ``Satya'' as it includes both the gross and the subtle realms.
\item
  \emph{Bṛhadāraṇyakopaniṣad} 5.4.1
\item
  The verse calls unmanifested \emph{prakṛti} as `water' \emph{Bṛhadā\-raṇya\-kopani\-ṣad} 5.5.1
\item
  \emph{Bṛhadāraṇyakopaniṣad} 5.5.2
\item
  \emph{Bṛhadāraṇyakopaniṣad} 5.5.2
\item
  The `Sun' on a physical plane represents the `physical Sun' that by its heat and light nourishes the earth and its inhabitants. The physical Sun, by its rays makes the physical eyes ``see'' the gross universe. And the eye in turn, perceives the physical Sun as the nourisher of the earth and its beings. On a subtle plane, `Sun' represents the divine light, the divine illumination that is present inside every individual. This divine light reveals the universe, through thoughts and intuition. And the mind in turn, perceives the divine Sun, through the thoughts and intuition. The mind is first clouded by the thoughts induced by the ignorance. But, later these thoughts are organized and replaced by intuition that takes one to Truth.
\item
  \emph{Muṇḍakopaniṣad} 1.2.11
\end{enumerate}


