\chapter{Verse 4}

\begin{moolashloka}
\dev{अनेजदेकं मनसो जवीयो नैनद्देवा आप्नुवन् पूर्वमर्षत् ।}\\
\dev{तद्धावतो अन्यानत्येति तिष्ठत्तस्मिन्नपो मातरिश्वा दधाति~ ॥ 4 ॥}
\end{moolashloka}

\textbf{Word to word Meaning}: Unmoving (\dev{अनेजत्}), one (\dev{एकं}), faster than the mind (\dev{मनसो जवीयो}), beyond the grasp of the senses (\dev{नैनद्देवा आप्नुवन्}) for it moves ever in the front (\dev{पूर्वमर्षत्}), always surpasses (\dev{अत्येति}), others who run (\dev{तद्धावतो अन्यान्}). In that which is Sitting (\dev{तिष्ठत् तस्मिन्}) the cosmic activity (\dev{अपो}) is caused by the cosmic life force (\dev{मातरिश्वा दधाति})

\textbf{Meaning}: \emph{Brahman}/\emph{Ātman} is without motion, without division, one which is swifter than the mind and is beyond the grasp of senses; it is ever in the front, always surpassing beyond those in movement, as it is the source as well as the destination. In this \emph{Ātma}, which is by itself eternally inactive --without any movement, the Cosmic Life Force causes and sustains all cosmic activity.

\textbf{Analysis}: After explaining the three paths and their results, the \emph{Upaniṣads} now take up the subject of nature of \emph{Brahman}, which is also the Self or \emph{Ātman} of all.

\dev{अनेजत्}- \emph{Unmoving, without movement, inactive}. The very first description of \emph{Brahman}, that \emph{Upaniṣad} enunciates is that ``it is unmoving''. The world as we understand, the \emph{jagat}- as the name suggest is full of movement. It is full of modifications, subject to time and space. The six modification/ \emph{vikāra-s} that scriptures mention are- unmanifest form, birth, growth, transform, decay and death. People continuously take birth, live life, and die. The animals also take birth and die. Even the stars and universes have a birth and a death. All the objects within the manifestation are subjected to this time-space process called `\emph{movement'}. \emph{Brahman} is being described as `\emph{unmoving}', that is beyond and unaffected by this time-space process of movement. \emph{Brahman} is the source of this movement, but itself unaffected by it. It is without birth and hence without death. \dev{अनेजत्} refers to eternal existence of \emph{Brahman}, beyond the state of action-inaction, moving-unmoving. The `movement' referred here is not the normal activity we see in the external world, which is opposite of inaction. The world is composed of dual principles of motion and motionlessness, activity and inactivity. But the term `movement' --the Universal Movement, the \emph{jagat} --includes both the motion and the rest. The activity and inactivity we see in the sensory world happen within this movement, within the \emph{vyāvahārika daśā}. On the other hand, in the \emph{pāramārthika daśā}, absolute state, there is neither action, nor inaction. \emph{Brahman} simply exists. Hence, \emph{Brahman} is described as `without movement', without the dual principle of activity and inactivity.

\dev{एकं}- \emph{One}. Next the \emph{Upaniṣad} describes \emph{Brahman} as ``One''. This term is very significant. The ``One'' does not refer to there being only One God, but instead it is saying that \emph{Brahman} alone exists. The whole universe, the multiplicities all are \emph{Brahman} alone, they being only an apparent manifestation of \emph{Brahman}. Just as a snake becomes perceived in place of a rope appearing as if being manifested by the rope itself, when the rope is mistaken as snake due to illusion, so also the cosmos is manifested by \emph{Brahman} as an appearance using Its \emph{Māyā-śakti}. \emph{Brahman} is the source, the material and intelligent cause of the universe. In \emph{pāramārthika} state, \emph{Brahman} just is, without movement, without multiplicity. He exists as infinite eternal truth. In \emph{vyāvahārika} also, He alone exists, but here He manifests many apparent names and forms with His \emph{Māyā}.

\emph{Pāramārthika} is a state of \emph{jñāna}, where One Eternal Infinite Truth is perceived, \emph{vyāvahārika} is a state of duality, where a person perceives multiplicity. \emph{Brahman} encompasses everything, the known and the Unknown, the one and the many. \emph{Upaniṣad} does not mean a numerical figure when it declares \emph{Brahman} as \dev{एकं}, instead it refers to the fact that ``Whatever there is, it is \emph{Brahman}'', the sum total of all that exists, the known manifested universe and the unknown un-manifested source.

\dev{मनसो जवीयो:} \emph{Faster than the mind}. In the manifested universe, the mind travels the fastest. The speed of thought is held to be faster than that of light. A trained mind can travel back and forth through different realms within few moments. \dev{मनः} not only refer to the lower faculty of mind but also to the higher faculty of \emph{buddhi} or intellect. Our mind is full of hundreds of thoughts at any moment. And we jump from one thought to another to yet another. So, in the manifested universe, mind travels the fastest. \emph{Upaniṣad} calls \emph{Brahman} as \emph{faster than the mind} i.e.\ beyond the reach/perception of the faculty of mind. The mind with all its rational analysis cannot comprehend \emph{Brahman}. \emph{Brahman} being the innermost Self/\emph{Ātman} of everything including the mind, the mind cannot reach it. The Self travels faster than the mind.

\dev{नैनद्देवा आप्नुवन्} -\emph{Beyond the grasp of the gods}. `\emph{Deva'} here refers to `\emph{indriya-s}/senses' of an individual, or more precisely to the deities that preside over these sense functions. The \emph{Upaniṣad} is declaring that, \emph{Brahman} is beyond the reach of faculty of senses. Neither the mind, nor the senses can comprehend it. Why is it so? The \emph{manas} and the \emph{indriya-s} are by their very nature move outwards. They are the faculties that function in the external world, they are faculties of the subject to grasp and comprehend the object. But, \emph{Brahman} being beyond the duality of subject and object and containing the both within Him, is beyond the reach of the mind and senses. \emph{Brahman} is the \emph{Ātman} of all there is in \emph{vyāvahārika} \emph{daśā} --both the knower and the known. Hence, the senses and the deities which preside over them cannot reach \emph{Brahman}.

\dev{पूर्वमर्षत्} --\emph{It moves ever in front}. When the mind or the senses grasp any object, mind or the senses are said to have travelled to them. For example, the eyes seeing a flower means, the faculty of sight has travelled to the flower. \emph{Brahman} being the object and the subject, He exists everywhere. Hence, it is as if, He is always ``in front of'' the mind and the senses. Wherever the mind or the senses travels, on reaching the destination, they find that the Self/\emph{Ātman} as being already present there.

\dev{तद्धावतो अन्यानत्येति} -- \emph{It surpasses all those who are in continuous motion (the mind and the senses).} \emph{Brahman} who is without activity appears as being faster than the mind, beyond the grasp of all senses. \emph{Brahman}, though motionless, is the motion of the \emph{universal} \emph{movement}, the source of the whole movement, the innermost Self of all those in movement. It is \emph{Brahman} which causes the movement. Hence, \emph{Brahman} always surpasses all those in movement. He being the source and destination both, He being omniscient and omnipresent, is present everywhere. Hence, \emph{Brahman} appears to be surpassing the others in their movement.

\dev{तिष्ठत्तस्मिन्नपो मातरिश्वा दधाति} --\emph{In that which is ``Sitting'' without activity, the Mātariśvā-Vāyu causes all activity}. \dev{तिष्ठत} or `\emph{Sitting'} refers to \emph{Brahman} being inactive/without movement. \emph{Upaniṣad} again declares \emph{Brahman} as being beyond the duality of motion and rest. Even though He is \emph{sitting without activity}, in the \emph{vyāvahārika} \emph{daśā} He appears to be ever in motion as He inhabits everything in the universe, being the source of all activity and inactivity in the world is present in both of them. Hence, the \emph{Upaniṣad} is reminding us, that in absolute reality, \emph{Brahman} just exists, without movement, without multiplicity.

\dev{मातरिश्वा} \emph{that which move in/fills the mother}. Here the mother is a reference to the \emph{ākāśa}, the space which is filled by \emph{Vāyu} -the life force. \emph{Vāyu} is the force which both causes and sustains all activity in the cosmos. \emph{Ādi Saṅkara} compares it with ``\emph{sūtra} (1)'' the thread that supports all \emph{karma-s}, all causes and effects in the universe. It presides over all cosmic activity. It is \emph{Hiraṇyagarbha} in its aspect of \emph{kriyā śakti}. \dev{अपो} may refer either to `works/cosmic activity' or to `waters/cosmic planes activity (2)'. In both cases it refers to all the activity of all the objects of the universe. The whole karmic cycle of cause and effect is addressed as \dev{अप:} Hence, in \emph{Brahman} who is without movement, the cosmic life force- \emph{Vāyu} induces/causes cosmic activity. That is, \emph{Brahman} by His \emph{Māyā}-\emph{śakti} creates apparent activity in Himself through the medium of \emph{Vāyu}.

\textbf{Summary:} In the \emph{pāramārthika daśā}, the absolute state, \emph{Brahman} just is. It is neither with form nor without form; It is neither in motion nor in rest; It is neither in the state of manifestation nor in the state of dissolution (3). There is no duality in It. Hence, It is beyond description, beyond the capture of prose and poetry. \emph{Brahman} just is. And a \emph{jīva} can realize \emph{Brahman} by becoming Him. Hence, in \emph{vyāvahārika daśā}, all the descriptions of \emph{Brahman} the \emph{śāstra-s} give are just pointers towards \emph{Brahman}. The words as such do not describe \emph{Brahman} by themselves, but only point towards Him with respect to the dual universe we see and understand.

After speaking about three ways an individual can live his life, the \emph{Upaniṣad} speaks about the nature of \emph{Brahman}. \emph{Brahman} is described as `one' and as being `without movement'. These are descriptions that repeatedly appear in the \emph{Upaniṣads}. The \emph{Śvetāśvataropaniśad} (4) describes \emph{Brahman} as ``\emph{niṣkalam}'' and ``\emph{niskriyam}'' meaning `without parts' and `without activity'. The same has been mentioned in this \emph{Upaniṣad} as ``\dev{अनेजत्}'' and ``\dev{एकं}''. It is very important to understand what exactly the terms convey. By `one' are the sages reducing the ultimate reality into a numerical entity? Or are they pointing to something else? Does `unmoving' refer to a state of rest as different from state of motion? Or does it mean something else?

In \emph{Bṛhadāraṇyakopaniṣad} (5), When \emph{Vidagdha}, son of \emph{Śākala} asks Sage \emph{Yājñavalkya} about how many Gods actually are there. He first gives the number as `\emph{three hundred and three, and three thousand and three}', then he gives the number as `\emph{thirty-three}', then `\emph{six'}, `\emph{three'}, `\emph{two'}, `\emph{one and half}' and finally says God is `\emph{one'}. What the Sage is trying to say here is, \emph{Brahman} being everything, He is beyond numerical conceptualization. \emph{Brahman} is not a numerical entity. \emph{Kenopaniṣad} (6) states: ``\emph{Brahman} is distinct from what is known and also distinct from what is unknown''. Hence, the term `one' does not refer to `one God', not to a numerical entity. Instead `one' refers to the fact that \emph{only} \emph{Brahman} is. \emph{Brahman} \emph{alone} is. There is no divisions, no parts within Him. There is no second to Him. He just is and He alone exists.

As explained earlier, the whole universe is called `\emph{movement'} as they are in continuous motion from manifestation to dissolution to re-manifestation. This cosmic movement contains within itself, both the state of motion and the state of rest. The \emph{Upaniṣad} is not describing \emph{Brahman} as being in this state of rest. If it were so, then the description that \emph{Brahman} runs faster than the mind and senses would be void. The \emph{Upaniṣad} in the same verse calls \emph{Brahman} both as `without motion' and `as moving faster than all those in movement'. \emph{Kaṭhopaniṣad} (7) describes \emph{Brahman} as ``While sitting, it travels far away; while sleeping, it goes everywhere''. This shows that the term ``unmoving'' does not refer to the temporary state of rest. Instead \emph{Brahman} is being described as \emph{without universal movement}, without both the periods of rest and motion. As \emph{Gauḍapādācārya} (8) says, in \emph{pāramārthika}, \emph{Brahman} has neither manifestation, nor dissolution; it simply is without cosmic movement.

Hence, in \emph{pāramārthika daśā}, \emph{Brahman} just is. He is one and without universal movement. Yet in the \emph{vyāvahārika daśā, Brahman} appears as being swifter than the mind, beyond the grasp of the senses and the deities presiding the senses, running always in front of the others, always overtaking others in the movement. The whole universe is manifested by \emph{Brahman} as an appearance. And in this too, He appears as faster than the fastest, always in front of all others, beyond the grasp of them. The mind travels fastest in the manifested universe. Then there are faculties of senses. Yet, \emph{Brahman} is beyond the grasp of all of them. The eyes, the ears, the speech or the \emph{Agni}, the \emph{Vāyu}, the \emph{Āpaḥ} - none can reach Him. Not one of them is able to understand Him or surpass Him.

\emph{Brahman} is that which eyes cannot see, but which makes the yes see, which the speech cannot speak about, but that which makes the faculty of speech function, which cannot be understood by the mind, but which makes the mind to understand everything (9). The \emph{indriyaḥ} (senses) and the mind are by their very nature outward looking. They travel only outwards. They can grasp only the things outside of them. And hence, they cannot grasp \emph{Brahman}, who is the innermost Self, the \emph{Ātman} of everything.

In such \emph{Brahman}, who is `one' and `without movement', the cosmic activities are caused by the \emph{Vāyu} -Cosmic life force that is called ``\dev{मातरिश्वा}''. \dev{मातरिश्वा} means that which fills the mother/ container. But who is the mother that is being referred to? This question is answered by the \emph{Taittirīyopaniṣad} (10) which says ``from \emph{Ātman} is born the \emph{ākāśa} (cosmic space), from \emph{ākāśa} is born the \emph{vāyu} (cosmic life force)''. That is, \emph{vāyu} is the life force that fills the cosmic space (\emph{ākāśa}). And what is the function for this \emph{vāyu} It is the source of all activity in the manifested universe. It supports and presides over all \emph{karma-s}: all the causes and the effects. Hence, \emph{Brahman} manifests the \emph{vāyu} through whom all the cosmic activity is caused and supported.

\section*{References}

\begin{enumerate}
\itemsep=0pt
\item
  \emph{`Sūtrāman'} refers to \emph{Hiraṇyagarbha}, the \emph{Kārya Brahman}. Here, \emph{Mātariśvā} refers to \emph{Hiraṇyagarbha} who exists as \emph{Vāyu}-life force that causes all activity.
\item
  Planes of activity refer to the three middle realms- \emph{Bhū}, \emph{Bhuvaḥ} and \emph{Svaḥ} \emph{loka}.
\item
  ``In \emph{pāramārthika}, there is no creation, no dissolution; there is no bondage, no seeker of liberation, no liberated'' (Verse 32, \emph{Vaitathya Prakaraṇa}, \emph{Gauḍapāda Kārikā} on \emph{Māṇḍukyopaniṣad})
\item
  \emph{Śvetāśvataropaniṣad} 6.19
\item
  \emph{Bṛhadāraṇyakopaniṣad} Part 8. Chapter 3.9.1
\item
  \emph{Kenopaniṣad} Part 1.3
\item
  \emph{Kaṭhopaniṣad} 1.2.21
\item
  Verse 32, \emph{Vaitathya Prakaraṇa}, \emph{Gauḍapāda Kārikā} on \emph{Māṇḍu\-kyo\-pani\-ṣad}
\item
  \emph{Kenopaniṣad} Part 1.4 to 1.8
\item
  \emph{Taittirīyopaniṣad, Brahmānandavalli}, Chapter 1.
\end{enumerate}
