\chapter{\textbf{Section 1\\ The Theme of the \emph{Upaniṣad}}}

\emph{Īśopaniṣad} is one of the most concise and important of the \emph{Upaniṣad-s} that forms the last portion of the \emph{Yajurveda Saṁhitā}. \emph{Veda} can be divided into three portions based on the subject matter they deal with. The \emph{Saṁhitā-s} and \emph{Brāhmaṇa-s} collectively constitute the `\emph{karma kāṇḍa}' that deal with \emph{karmānuṣṭhāna} i.e. performance of various rituals to attain various ends. The \emph{Āraṇyaka-s} constitute `\emph{upāsanā} \emph{kāṇḍa}' that deal with vidyā \emph{upāsanā} i.e. practice of various worship and meditations. The \emph{Upaniṣad-}s constitute the `\emph{jñāna kāṇḍa}' that deals with the nature of \emph{Ātman} / \emph{Brahman} and the means of attaining \emph{ātmajñāna} and hence \emph{mokṣa}.
\vskip 2pt

The central theme of \emph{Īśopaniṣad} is three-fold. Like other \emph{Upaniṣad}s, it also lucidly explains about \emph{Ātma}, the state of \emph{jñāna} (i.e. \emph{pāramārthika} \emph{satya}) and the means to attain \emph{jīvanmukti} (\emph{nivṛttimārga}). But, along with this, this \emph{Upaniṣad} also addresses the concerns of the householders (i.e. the \emph{pravṛttimārgī-s}) and explains the means for them to attain \emph{mokṣa} through \emph{kramamukti}. This \emph{Upaniṣad} further broaches the issue of people who are neither eligible for \emph{mokṣa} (\emph{jñāna}) \emph{sādhanā} nor practice \emph{karmānuṣṭhāna}. They live life according to their whims and fancies without a care for \emph{dharma}. To properly understand the paths of \emph{pravṛttimārga} and \emph{nivṛttimārga} and the role of \emph{karma}, \emph{bhakti} (i.e. \emph{upāsanā}) and \emph{jñāna} (i.e. \emph{jñāna} \emph{sādhanā}) in them, one must first have a clear understanding about the tenets on which the scheme of Hindu life is designed.
\vskip 2.3pt

The Hindu scheme of life is based on a firm understanding that there exists a meaning and a purpose to the life of a being, howsoever small or big the being may be. From the intelligent humans down to the animals, plants and even the inanimate entities, everything that exist in the universe has a unique place in the universe and a specific purpose in the grand scheme of \emph{Brahman}. Among all those that have life, human life is considered most precious because it is humans alone who have `\emph{free-will}'. It is they alone, who can discriminate between good and bad, between \emph{dharma} and \emph{adharma} and can practice \emph{karma} and \emph{bhakti} by which they can achieve spiritual progress as well as ensure better births for themselves in future. All other animals do not have this faculty of free-will. They are completely under the control of their inherent and instinctive nature as well as the forces of their \emph{prārabdha} \emph{karma}. It is humans alone who have the capacity to overcome their inherent nature and elevate themselves to the higher states of existence.
\vskip 4pt

Hence, in the Hindu scheme of life, the human life occupies a very significant position. Hindu texts (\emph{śāstra-s}) have elaborately dealt with the meaning and purpose of human life, which they have termed as `\emph{puruṣārtha-s}' --the goals of human life.
\vskip 4pt

The \emph{śāstra-s} speak about four-fold goals of human life -- \emph{dharma}, \emph{artha}, \emph{kāma} and \emph{mokṣa}. `\emph{Dharma'} has a wide range of meanings ranging from ethics and justice to law and duty. In the context of an individual, \emph{dharma} refers to living one's life according to the tenets mentioned in the \emph{śāstra-s} like \emph{satya}/truthfulness, \emph{āsteya}/non-stealing, \emph{ahiṁsā}/non-injury, \emph{śauca} /cleanliness, etc. and performing \emph{karma}s and duties towards family and society as enjoined in the \emph{śāstra-s} to the best of one's ability and avoiding those actions that are considered unethical and hence, prohibited in the \emph{śāstra-s} like homicide, stealing, etc. `\emph{artha'} refers to the wealth (both material and heavenly) one must attain through proper \emph{dharmic} means, free from corruption and dishonesty. \emph{`Kāma'} refers to desires and dreams an individual has. These desires must also be fulfilled through \emph{dharmic} means and not by resorting to \emph{adharmic} means. These three --\emph{dharma}, \emph{artha}, and \emph{kāma} --refer to the immediate goals pertaining to one's life. The ultimate goal of life however is `\emph{mokṣa}' or liberation from the karmic cycle of birth and death.
\vskip 4pt

Hindu \emph{dharma}, thus, provides a framework wherein a person works through the objectives of \emph{artha} and \emph{kāma} and towards \emph{mokṣa} with the help of \emph{dharma}. Only a person devoid of desires for worldly pleasures is eligible to pursue \emph{mokṣa} exclusively, since, why would a worldly person with worldly attachments ever want liberation from the cycle of birth and death? As long as desires remain, one would wish to attain them. Hence, only a person who has a burning desire to attain \emph{mokṣa} will renounce the worldly attachments and involve himself in the practice of \emph{jñāna} \emph{sādhanā} (1). Such people are called `\emph{sannyāsin-s}' or `\emph{nivṛttimārgī-s}'.

On the other hand, those people in whom desires are still present, those who wish to have sons and daughters, and attain wealth and prosperity both in this world and the next are called `\emph{pravṛttimārgī-s}'. Most householders come under this category. \emph{Śāstra-s} speak about three kinds of desire in the context of householders: desire for progeny, desire for wealth, and desire to attain all three worlds. But, a person is advised to fulfill all his desires only through \emph{dharmic} means. Only such a person is a \emph{pravṛttimārgī} who lives his life according to the tenets of \emph{dharma} and fulfills all his desires through \emph{dharma}. Practice of rituals and rites are suggested for those who wish to attain the world of \emph{pitṛ-s}/ancestors after death and later return to human life. Various meditations on deities are prescribed for those who wish to attain heaven or the realms of deities. The former constitute \emph{karmānuṣṭhāna}, while the latter form the practice of \emph{apara}-\emph{bhakti}. Further, the \emph{śāstra-s} state that, if a \emph{pravṛttimārgī} wishes to overcome his desires and become nivṛttimārgī, then he must practice \emph{karma} and \emph{bhakti} in a detached manner by surrendering everything to \emph{Brahman}. A \emph{gṛhastha} /householder is advised to practice both \emph{karma} and \emph{bhakti}, while a \emph{vānaprasthī} (one who has gone to forest at old-age) is advised to practice \emph{bhakti}/\emph{upāsanā} alone. After attaining \emph{citta} \emph{śuddhi} and overcoming all desires by such practice, a person becomes eligible for \emph{jñāna} \emph{sādhanā} (\emph{sannyāsa}).

However, those people who are neither eligible for \emph{nivṛttimārga} nor practice \emph{pravṛttimārga} by living according to \emph{dharma}, but instead live life selfishly and recklessly, such people are bound to suffer the \emph{karmic} punishments for the wrongs they have done. The \emph{nivṛttimārga}, \emph{pravṛttimārga} and the `\emph{third} \emph{path}' constitute the three-fold paths which is the subject matter of this \emph{Īśopaniṣad}.

\section*{References}

\begin{enumerate}
\itemsep=0pt
\item
  \emph{Jñāna} \emph{sādhanā} includes: \emph{śravaṇa} (listening to the teachings from a teacher), \emph{manana} (internalizing the teachings by getting the doubts cleared by resorting to reasoning), \emph{nididhyāsana} (meditating upon the import of the teachings) and finally the process ends in \emph{ātmasākṣātkāra} / \emph{ātmajñāna} /enlightenment.
\end{enumerate}
