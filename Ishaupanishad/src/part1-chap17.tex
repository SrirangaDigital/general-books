\chapter{Verse 16}

\begin{moolashloka}
\dev{पूषन्नेकर्षे यम सूर्य प्राजापत्य व्यूह रश्मीन् समूह तेजः ।}\\
\dev{यत् ते रूपं कल्याणतमं तत् ते पश्यामि योऽसावसौ पुरुषः सोsहमस्मि ॥ 16 ॥}
\end{moolashloka}

\textbf{Word to word Meaning}: The Nourisher (\dev{पूषन्}), the Sole traveler (\dev{एकर्षे}), the Controller (\dev{यम}), the Acquirer (\dev{सूर्य}), the son of \emph{Prajāpati} (\dev{प्राजापत्य}), orderly arrange (\dev{व्यूह}) your rays (\dev{रश्मीन्}), gather together (\dev{समूह}) your light (\dev{तेजः}); Which (\dev{यत्}) form (\dev{रूपं}) of yours (\dev{ते}) which is most auspicious (\dev{कल्याणतमं}), that (\dev{तत्}) I will see (\dev{पश्यामि}) by you (\dev{ते}); He who is (\dev{यः}) there (\dev{असावसौ}) Person (\dev{पुरुषः}), that (\dev{सः}) I am (\dev{अहमस्मि}).

\textbf{Meaning:} (Oh) Nourisher, the Sole traveler, the Controller, the Acquirer, the son of \emph{Prajāpati}, (please) arrange your rays and gather them together, so that, by your (grace), I will perceive your most auspicious form. He who is that Person (i.e. the Person in the solar orb), that (Person) I am.

\textbf{Analysis:} This verse continues with the prayer to \emph{Satya Brahman} (Hiranyagarbha) who exists as the Solar deity.

\dev{पूषन्}- (\emph{the Sun who is) the nourisher.} The Solar deity is called \dev{पूषन्} because he nourishes all life. The Solar deity in the physical plane represents the `Sun' that gives heat and light. It supports and nourishes all life through this heat and light. Through its rays, it makes an individual to perceive the physical universe through his eyes. On the subtle plane, the solar deity represents the `\emph{divine light}' that illuminates the mind and grants \emph{viveka} and \emph{dharmajñāna} to an individual. Through thoughts and intuition, it reveals the subtle realms to an individual. Further, it is the Solar deity alone, who through His divine illumination will remove the thoughts rooted in multiplicity (ignorance) and replace it by intuition that would lead to `\emph{Satya Brahman}'. Hence, the Solar deity is called \dev{पूषन्}- he who nourishes life in both gross and subtle realms. Here, the prayer is actually dedicated to `Satya' i.e. \emph{Kārya Brahman} who inhabits the solar orb as the Sun. \emph{Kārya Brahman} is called `Satya' because he includes in him both the gross and subtle existence. Hence, here this `\emph{Satya Brahman}' is called \dev{पूषन्} who as the Solar deity nourishes and supports both the gross and the subtle existences.

\dev{एकर्षे यम} - \emph{the Sole traveler, the Controller.} `\emph{Satya Brahman} who exists as Solar deity is called \dev{एकर्षे} (1) the sole traveler, because he alone exists behind all thoughts and all movements in both subtle and gross universes. Further, he is referred as \dev{यम} because he controls everything in the subtle and gross universes as he inhabits them.

\dev{सूर्य}- \emph{the Acquirer.} `\emph{Satya Brahman}' who manifests as Solar Deity is called \dev{सूर्य} because he absorbs into him all the rays, all the life forces and the liquids (2). Further, the epithet `Acquirer' is used because \emph{Satya Brahman} secures to himself all objects.

At cosmic level, \emph{Hiraṇyagarbha} begins the cosmic creation by bringing out the entire universe from within Himself. Then, at the end of the cosmic creation, he causes \emph{laya} or destruction of the universe by withdrawing and absorbing unto Himself all the objects of the universe. At an individual level, the Solar deity that exists as the power of illumination is the one who illuminates all objects to an individual. In the waking state, the divine light travels from an individual through his mind-sense organ complex and illuminates the presence of objects in the external world. When an individual goes to a state of deep sleep, this divine illumination is withdrawn from the objects of the world and hence, the objects cease to exist for the individual. It is for this reason that \emph{Satya Brahman} is being described as `\emph{Sūrya}' --the one who acquires for Himself all the rays (the \emph{vṛtti-s} of the mind that perceive objects), the vital airs (that sustain an individual as well as the universe), and various gross and subtle matters.

\dev{प्राजापत्य}- \emph{Son of Prajāpati.} `\emph{Prajāpati}' means `\emph{master of the beings}' and it refers to \emph{Hiraṇyagarbha}/\emph{Kārya Brahman}. He is called \emph{Prajāpati} because being the first born and creator of whole cosmos (gross and subtle), He is the master and controller of all objects and beings. In this verse, the Solar deity is called `\emph{son'} of \emph{Prajāpati}, because it was the `Sun' who was first created by \emph{Hiraṇyagarbha}. The \emph{Praśnopaniṣad} states that, the \emph{Prajāpati} desiring to manifest universes, created the Sun (the energy) and Moon (the matter) (3). Hence, the Solar deity is called \dev{प्राजापत्य}- the son of \emph{Prajāpati} or the power of \emph{Prajāpati}.

\dev{व्यूह रश्मीन् समूह तेजः}- \emph{Orderly arrange your rays and gather together your light.} Now the \emph{Upaniṣad} speaks about the manner in which \emph{Pūṣan} --the Solar deity will remove the rays and lead a person to \emph{Satya Brahman}. The individual is clouded by the thoughts rooted in \emph{avidyā}. The Solar deity, who exists as divine light inside each individual, first arranges the thoughts in orderly manner so as to reveal the intuitions that are beneath them. By this, the thoughts rooted in \emph{avidyā} are replaced (4) by intuitions rooted in the Light of knowledge. Then, the Solar deity will gather together i.e. withdraws to Himself this Light (i.e. intuitions) revealing the \emph{Satya Brahman} who inhabits the Solar orb. Sri Aurobindo writes thus: ``By the revelation of the vision of \emph{Sūrya} the true knowledge is formed. In this formation the \emph{Upaniṣad} indicates two successive actions. First, there is an arrangement or marshaling of the rays of \emph{Sūrya}, that is to say, the truths concealed behind our concepts and percepts are brought out by separate intuitions of the image and the essence of the image and arranged in their true relations to each other So we arrive at totalities of intuitive knowledge and can finally go beyond to unity This is the drawing together of the light of \emph{Sūrya}'' (5). A person in \emph{saṁsāra}, bound by \emph{avidyā} cannot perceive \emph{Satya Brahman} directly. Hence, he is requesting the Solar deity, to help him develop the \emph{intuition} that would make him able to perceive \emph{Satya Brahman}.

This verse also reminds one of \emph{Yoga} as defined in \emph{Yogasūtra} of \emph{Patañjali}. \emph{Patañjali} defines Yoga as `\emph{citta vṛtti nirodha}' meaning `\emph{removing of the fluctuations of the mind}'. Here, the thought patterns/\emph{vṛtti-s} of the mind are being described as raśmin/rays and the \emph{Upaniṣad} is suggesting that by a meditation on \emph{Satya Brahman} as enunciated in the previous verse, one is able to gather the mental \emph{vṛtti-s} scattered in thoughts about various objects of the world and then refocus it inwards towards the object of meditation i.e. \emph{Satya Brahman}. This refocusing inwards is being described as `gathering together of the light'.

\dev{यत् ते रूपं कल्याणतमं तत् ते पश्यामि}\emph{- I will See, by you (your grace), that form of yours which is most auspisious}. The person, who is on the verge of death, is thus praying to the Solar deity that, he has practiced the tenets of worship of \emph{Satya Brahman} (i.e. \emph{Satya Dharma}) and now he wishes to directly perceive \emph{Satya Brahman} by attaining him. He requests the Solar deity to please remove his rays and grant him intuition, so that he could perceive \emph{Satya Brahman} who has the most auspicious form (\dev{कल्याणतमं रूपं}). The form of \emph{Satya Brahman} is auspiscious because it is devoid of any limations, be it physical limitations like hot and cold, or subtle limitations like pleasure and sorrow, and \emph{Satya Brahman} can grant such a state of bliss to his devotees as well.

The devotee on the verge of death is requesting his object of worship: \emph{Satya Brahman} that the deity should facilititate him to leave his body in a state of meditation such that he can withdraw his mind from the worldly thoughts and redirect one-pointedly towards \emph{Satya Brahman} and thus attain Him.

\dev{योऽसावसौ पुरुषः सोsहमस्मि}- \emph{He who is that Person (i.e. the Person in the solar orb), that (Person) I am.} The \emph{Upaniṣad} is saying that the \emph{Puruṣa} who inhabits the solar orb is the same \emph{Puruṣa} who inhabits a \emph{Jīva}. That is, it is \emph{Satya Brahman} who exists both as the Solar deity and as an individual \emph{jīva}. \emph{Satya Brahman} is called \dev{पुरुषः} or Person, because he has for his body limbs the three \emph{loka}- \emph{bhū} \emph{loka} as head, \emph{bhuvaḥ} \emph{loka} as arms and \emph{svar} \emph{loka} as his feet. And this \emph{Satya Brahman} when it exists in the solar orb, it is called `\emph{Ahar'} and when it exist in an individual it is called `\emph{Aham'} (6). This portion of the verse upholds non-duality as the ultimate reality. The devotee on the verse of death is stressing this non-difference between the devotee and the deity by noting how \emph{Satya Brahman} who exists as his beloved Solar deity is also his own Self/ \emph{Ātma} and as such, his prayer is a prayer to attain non-duality, the ultimate state of \emph{mokṣa}.

\textbf{Summary:} The \emph{Upaniṣad} continuous here with the prayer that started in the previous verse. The \emph{Upaniṣad} describes \emph{Satya Brahman} who exists as the Solar deity inhabiting the \emph{Sūrya} \emph{Maṇḍala} as \emph{Pūṣan}, \emph{Ekarṣe}, \emph{Sūrya}, \emph{Yama} and the son of \emph{Prajāpati} indicating that it is \emph{Satya Brahman} alone who is the creator, nourisher, controller and absorber of both the gross and subtle universes. The devotee on the verse of death then prays to his beloved Solar deity to help him die in a state of meditation by focusing his mind one-pointedly towards \emph{Satya Brahman}. The verse ends with the devotee noting the non-difference between the devotee and the deity and thus revealing his prayer as an appeal to eventually attain \emph{mokṣa}.

\section*{References}

\begin{enumerate}
\item
  \dev{एकर्षे} can also mean `the Sole Seer', the Sun is called Seer/\emph{Kavi} because he is the sole witness to all thoughts and all movements. It is \emph{Satya Brahman} alone, who exists behind all thoughts and all movements in both gross and subtle universes as their inner Self, as their witness. Hence, the \emph{Satya Brahman} is called \dev{एकर्षे}- Sole traveler or Sole Seer.
\item
  \emph{Śaṅkara Bhāṣya} on \emph{Īśopaniṣad}.
\item
  \emph{Praśna} Up 1.4 and 1.5
\item
  \dev{व्यूह} can also mean disperse or remove indicating the removal of thoughts rooted in \emph{avidyā} thereby revealing the intuitions (rooted in light) that were present beneath them.
\item
  Page 73 and 74, Volume 17- \emph{Īśopaniṣad}, The Complete Works of Sri Aurobindo.
\item
  \emph{Bṛhadāraṇyakopaniṣad} 5.5.3 and 5.5.4
\end{enumerate}


