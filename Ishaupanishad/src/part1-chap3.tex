\chapter{Verse 2}

\begin{moolashloka}
कुर्वन्नेवेह कर्माणि जिजीविषेच्छतं समाः ।\\
एवं त्वयि नान्यथेतोsस्ति न कर्म लिप्यते नरे ।।2।।
\end{moolashloka}

\textbf{Word to word Meaning}: One must perform (\dev{कुर्वन्}) action (\dev{कर्माणि}) verily (\dev{एव}) here (\dev{इह}), those who wish to live (\dev{जिजीविषेत्}) one hundred years (\dev{शतम् समाः}). For them (\dev{त्वयि}) who are living like that (\dev{एवं}), no way (\dev{न अन्यथा}) other than this (\dev{इतः}) exist (\dev{अस्ति}) in which the action (\dev{कर्म}) will not (\dev{न}) stick to (\dev{लिप्यते}) people (\dev{नरे}).

\textbf{Meaning}: People (who have not achieved \emph{citta śuddhi} and hence) who wish to live in this world for hundred years, must do so by verily performing actions (prescribed in the \emph{śāstra-s} in a detached way). There is no way, other than this (detached action) in which the fruits of action will not bind them.

\textbf{Analysis}: After addressing the people who have achieved \emph{citta śuddhi}, now the \emph{Upaniṣad} addresses those who are still bound by the Karmic cycle, but have a desire to overcome it. These are the people who are yet to gain control over their mind and the senses and are still attached to their desires, but also at the same time recognize that there exists a higher truth and wish to proceed towards it.

\dev{जिजीविषेच्छतं समाः}- \emph{Those who wish to live a hundred years}. People who have achieved \emph{citta śuddhi}, having realized that the \emph{jagat} is temporary and an apparent manifestation will renounce their attachments to sensory objects and desires. But those people who have not purified their mind and are still attached to sensory pleasures, they wish to live a long life so that they can fulfill their dreams and desires.

\dev{कुर्वन्नेवेह कर्माणि}- \emph{Must verily perform actions here itself}. Such people who are bound by the \emph{pāśa-s} (1) of \emph{saṁsāra} / transactional world but wish to make spiritual progress, they must perform actions here in this \emph{saṁsāra} itself. What kind of actions must they perform? It is the Dharmic actions as prescribed in the \emph{śāstrā-s} called ``\emph{vihita karma}'' that must be performed. How should it be performed? In a \emph{niṣkāma} /detached way. Every action performed will not lead to \emph{mokṣa}. If a person keeps performing actions to fulfill his desires and pleasures and without a concern for Dharma or ethical duties, such actions will not lead him to \emph{mokṣa}. They instead lead to more bondage to \emph{saṁsāra}. A person who wishes to achieve \emph{mokṣa} must first implement \emph{vihita karmānuṣṭhāna} (2) i.e. he must practice \emph{karma-s} like japa, agnihotra, etc., practice \emph{upāsanā} and also fulfill all the duties prescribed by the \emph{śāstra-s} (3). If one does such a \emph{vihita karmānuṣṭhāna} in a \emph{niṣkāma} way with a sense of duty, he will achieve \emph{citta śuddhi} (purification of mind) and it will lead eventually lead him to \emph{jīvanmukti} (liberation even while being in the body). On the other hand, those in \emph{pravṛttimārga}, who are still strongly bound to their desires and hence cannot perform actions in a detached way, should do so in a \emph{sakāma} way (attached way), because non-performance of duties and prescribed karmas is ``\emph{pāpam}'' and hence will increase bondage to \emph{saṁsāra}. Hence, such a person, should practice \emph{vihita karma-s} in \emph{sakāma} way and avoid the \emph{niṣiddha karma-s} that are prohibited in the \emph{śāstra-s}. By doing so, he/she would slowly achieve liberation through \emph{kramamukti} (the path of gradual or step-wise liberation).

The word \dev{एव इह}~highlights that people in \emph{pravṛttimārga} must definitely perform the \emph{vihita karma-s} here, in the world itself. People who still have desires and duties to fulfill, will not purify their minds by either abandoning their families, giving up their duties, or going to the forest. They can achieve \emph{citta śuddhi} only by staying in the world and performing their duties with as much detachment as possible.

\dev{एवं त्वयि}- \emph{For such people}. For people who are in \emph{pravṛttimārga} and hence are bound by Karmic cycle but desiring \emph{mokṣa} who have begun to perform \emph{vihita karma-s} as prescribed in the \emph{śāstra-s.}

\dev{नान्यथेतो ऽस्ति}- \emph{There is no path other than what was mentioned before}. For those people in \emph{pravṛttimārga}, if they wish to attain \emph{mokṣa} either through \emph{ātmajñāna} or through \emph{kramamukti}, there is no other path than the practice of \emph{vihita karmānuṣṭhāna}, preferably in a detached manner.

\dev{न कर्म लिप्यते नरे}- \emph{In which the karma does not bind you}. Why is there no other path than detached action for people in \emph{pravṛttimārga}? Because, only when a person performs detached action will the karmic fruits not bind him. When a person does any activity with a desire for the fruits of it, he increases attachment to the action as well as its fruits. Further, he will jump from fulfilling one desire to another. There is no contentment, only indulgence. This will increase the attachment to \emph{saṁsāra}, to the pleasures that \emph{saṁsāra} offers. Hence, such a person will ever be trapped in this Karmic cycle of birth and death ever fulfilling his desires and facing fruits of his actions. However, a person who performs \emph{vihita karma-s} and avoids \emph{niṣiddha karma-s}, even though he is still full of desires, he fulfills his desires by adhering to \emph{Dharma}. For, such a person ``\emph{pāpam} / sin'' i.e. the fruits of bad actions which may have lead him to greater darkness and attachment to \emph{saṁsāra}, will not bind him.

\textbf{Summary}: Having dealt with the \emph{nivṛttimārga}, the \emph{Upaniṣad} in the second verse deals with the \emph{pravṛttimārga}. \emph{Sukha} or happiness is the one thing that every person in this world strives for. Every action of man is directly or indirectly aimed at achieving happiness. There are two ways a person can live his life. He can either spend his whole life pursuing desires, ambitions and happiness in the external sensory world or he can turn-away from the sensory world which is temporary in a search for the eternal ever-lasting bliss.

People in the world, are those who work hard, pursue careers, earn money, earn fame, make family, etc. so that they can lead a happy life. But, in this approach one never finds contentment. Without contentment, one is always running behind one object after another which he perceives as a source of happiness. But the physical-sensory world being temporary and ever changing, happiness is never accompanied by contentment. Hence, a person in this world often experiences disappointment and sorrow (\emph{duḥkha}). This \emph{saṁsāra} is a never ending cycle of \emph{sukha} and \emph{duḥkha}. Often, people bound by \emph{saṁsāra}, pursue only two of the \emph{puruṣārtha-s} (the goals of life) namely- \emph{kāma} and \emph{artha}. They, involve in indulgence / \emph{bhoga} of the sensory pleasures without regard for \emph{dharma} or \emph{mokṣa}.

If such people, who are deeply engrossed in \emph{saṁsāra}, wish to work towards \emph{mokṣa}, then they must do so by the practice of only those \emph{karma-s} that are prescribed in the \emph{śāstra-s} (\emph{vihita karma}) and avoid those \emph{karma-s} that are prohibited (\emph{niṣiddha karma-s}), i.e. they must learn to lead their life according to the Dharma. This performance of \emph{vihita karmas} and avoidance of \emph{niṣiddha karma-s} is called ``\emph{karma-anuṣṭhāna}''. This \emph{karma-anuṣṭhāna}, when done in a detached way will lead a person to \emph{citta śuddhi}.

\emph{Karma-s} are of four kinds- \emph{nitya, naimittika, kāmya} and \emph{niṣiddha}. \emph{Nitya karma-s} refer to regular activities that a person is supposed to follow like \emph{sandhyāvandana}. \emph{Naimittika} refers to \emph{karma-s} performed on specific occasions. \emph{Kāmya} refers to \emph{karma-s} done to fulfill specific desires. \emph{Niṣiddha karma-s} refers to \emph{Karmas} that are prohibited like stealing, murder, etc. \emph{Nitya} and \emph{naimittika} \emph{karma-s} are called ``\emph{vihita karma-s}'', those actions which are obligatory for a person to follow. Non-performance of such karmas will lead to bad karmic results. \emph{Vihita karma} does not mean only rituals and rites like \emph{Agnihotra} and \emph{Sandhyā} ritual. It includes every action, every speech and every thought that must be done by adhering to \emph{Dharma}. Every person must fulfill various obligations towards his spouse, his children, his parents, his career etc. He must perform all such actions; fulfill all his desires and duties in an ethical manner by strictly adhering to \emph{Dharma}.

Hence, a person who wishes to achieve \emph{citta śuddhi} must perform \emph{vihita karma-s} and he must stay away from \emph{niṣiddha karma}. He must perform all his actions including the \emph{kāmya karma-s}, in a detached manner with a sense of surrender towards God.

Any action can be performed in two ways- \emph{sakāma} and \emph{niṣkāma}. ``\emph{Sakāma}'' refers to performing an action with an eye towards the fruits that action will bore. Such actions may lead a person to \emph{sukha} or to \emph{duḥkha} depending upon whether the karmic fruit was as per the expectation of the person or not. On the other hand, the \emph{karma-s} performed in \emph{niṣkāma} way, i.e. performing an action with the sense of duty without expecting any result, such a person will find inner contentment irrespective of the Karmic fruits.

\emph{Sakāma karma} increases attachment to the sensory world. As there is no end to the desires of the person, he will eternally be pursuing one desire after another performing countless number of \emph{karma-s}. He will be ever-struck in this Karmic cycle of \emph{sukha - duḥkha}. \emph{Brahmavaivarta Purāṇa} (4) says thus-

\emph{A person will definitely enjoy the fruits of his action; it may be good or bad; for without giving the results, an action does not die out even after billions of years.}

However, if such binding actions are performed in \emph{niṣkāma} way surrendering the fruits of action to \emph{Iśvara} and giving up one's sense of doership of action, then such \emph{karma-s} themselves will pave way for \emph{mokṣa}.

Hence, the \emph{Upaniṣad} is instructing a person in the \emph{pravṛttimārga} that he should avoid the \emph{niṣiddha karma-s} and practice \emph{vihita karma-s} alone by which the ``\emph{pāpam}'' the fruits of bad actions will not bind him. He should then try to develop detachment and surrendering that would lead to citta śuddhi and development of qualities like \emph{viveka} and \emph{vairāgya}. There is no other way than this to transcend the limitations and the bondages imposed by \emph{saṁsāra}.

\section*{References}

\begin{enumerate}
\item
  \emph{Pāśa} means ``to bind''. \emph{Kulārṇava Tantra} enurnerates eight kind of \emph{pāśa-s} -- namely, pity (\emph{dayā}), ignorance and delusion (\emph{moha}), fear (\emph{bhaya}), shame (\emph{lajjā}), disgust (\emph{ghṛṇā}), family (\emph{kula}), custom (\emph{śīla}), and caste (\emph{varṇa}). {[}Sir John Woodrofe, Introduction and Preface, \emph{Mahānirvāṇa Tantra} \url{https://www.sacred-texts.com/tantra/maha/maha00.htm}{]}
\item
  \emph{Vihita karmānuṣṭhāna} means ``practice of \emph{karma-s} as prescribed in \emph{śāstra-s}''.
\item
  Three duties that a householder must fulfill are study the scriptures, perform the rites and rituals and attain all the three worlds. \emph{Bṛhadāraṇyakopaniṣad} 1.5.17.
\item
  \emph{Brahmavaivarta Purāṇa} 1.44.74
\end{enumerate}
