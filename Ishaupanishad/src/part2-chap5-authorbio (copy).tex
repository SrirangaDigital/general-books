\chapter{\textbf{Author Bio}}

Nithin Sridhar has a degree in Civil Engineering, and having worked in the construction field, he passionately writes about various issues from development, politics, and social issues, to religion, spirituality, and ecology.

He is currently the Chief Curator of Advaita Academy -- a video portal dedicated to the dissemination of Advaita Vedanta. He is also a Consulting Editor to the Indic Today Magazine. He was formerly the Editor of IndiaFacts- a portal on Indian history and culture.

He is the author of six books. His first book `\emph{Musings on Hinduism}' provided an overview of various aspects of Hindu philosophy and society. His books `\emph{Sri Dakshinamurthy}' and `\emph{Candika'} enunciates upon the two prominent deities of Hinduism. His book \emph{'Samanya Dharma}' gives an overview of the general tenets of ethics as available in Hindu texts. However, his most widely read book is `\emph{Menstruation across Cultures: A Historical Perspective}' that examines menstruation notions and practices prevalent in different cultures and religions from across the world.

He is based in Mysuru, Karnataka and tweets at @nkgrock.



\chapter{Section 4\\ The Third Path: The Path to Suffering}

After discussing about \emph{nivṛttimārga}- the path of \emph{jñāna} and \emph{pravṛttimārga}- the path of \emph{karma} and \emph{upāsanā}, the \emph{Upaniṣad} next explains the fate of those people who are neither eligible for \emph{nivṛtti} \emph{dharma} nor perform \emph{pravṛtti} \emph{dharma}.

\emph{Chāndogya} \emph{Upaniṣad} (1) says that those people who neither practice \emph{karma} nor practice \emph{upāsanā}, after death, they do not travel through either \emph{pitṛyāna} or \emph{devayāna}; instead they are born as ``small creatures'' that continuously undergo birth and death. \emph{Bṛhadāraṇyakopaniṣad} (2) says that such people become insects, moths, and such creatures that bite things. The present \emph{Īśopaniṣad} explains that those people who ``slay their Self'' (i.e. those who forgetting the true nature of \emph{Ātma} are recklessly indulged in worldly enjoyments) go to \emph{asura} \emph{loka} that is filled with darkness of ignorance.

Compared to the state of \emph{Ātma}, the whole world and all its realms are product of ignorance and are engulfed in ignorance. But, people who live life by practicing \emph{karma} and \emph{upāsanā} are at least living life according to \emph{dharma}. After death, they would attain \emph{pitṛloka} or \emph{devaloka}, and eventually they would attain \emph{kramamukti}. On the other hand, people who live life according to their whims and fancies without a care for \emph{dharma} and always indulged in sensual gratifications will dive deeper and deeper into ignorance (3). As a result, after death, they would have to suffer for the various \emph{adharmic} activities that constitute \emph{pāpa} -sin that they have performed while in body through attaining various kinds of \emph{naraka-s}-hells and then taking rebirth as various creatures (4).

In \emph{Gītā}, Lord \emph{Kṛṣṇa} speaks about two kinds of people- those who have divine nature and those who have \emph{asuric}/undivine nature. He says that those people who have the qualities like truthfulness, fearlessness, charity, control of senses, purity, austerity, non-injury etc. are considered to be of divine nature (5) and those who have the qualities like religious ostentation, pride, self-conceit, ignorance and anger are considered to be of \emph{asuric} nature (6). He further says that, those with divine nature proceed towards liberation, where as those with \emph{asuric} nature whose supreme aim is enjoyment of desires(7), will dive deeper in Bondage (8). Similarly, \emph{Manusmṛti} says that, people with \emph{sāttvic} qualities like austerity, knowledge, purity, control of sense and practice of \emph{dharmic} works attain divine nature after death; people with \emph{rājasic} qualities like desire for enjoyment of objects and lack of firmness of mind attain a human birth again; And those with \emph{tāmasic} qualities like greed, cruelty, laziness and practice of prohibited (\emph{adharmic}) actions end up taking birth in the womb of various beasts (9). Hence, it must be understood that those people in whom the \emph{tāmasic}/\emph{asuric} qualities predominates, they live life recklessly and without any care for \emph{dharma} or welfare of others. Driven by greed and delusion and with the sole intention of attaining their desired objects, they perform all kinds of adharmic activities that constitute \emph{pāpa} /sin (10).

\emph{Chāndogya} \emph{Upaniṣad} (11) says that he who steals gold, drinks alcohol, has sexual relationship with teacher's wife, and kills a \emph{brāhmaṇa} as well as those who associate with such a person fall. The \emph{dharmaśāstra-s} speaks elaborately about various sins that are committed by people. \emph{Manusmṛti} broadly classifies these sins under three categories as those done through body, through speech, and through mind. Grabbing of others property, indulging in physical violence and having sexual relationship with other's spouse constitute the \emph{adharma} committed through the body (12). Speaking rudely (so as to hurt others), speaking lies, speaking slander/ill of others and speaking irrelevantly (example: gossiping) constitute the \emph{adharma} committed through the speech (13). Desiring for others wealth, wishing harm to others, having liking for falsehood and hypocrisy constitute the \emph{adharma} committed through the mind (14). \emph{Manu} further says that those who commit sins through body, speech and mind will respectively face their \emph{karmic} punishments in the form of sufferings experienced through the body, the speech and the mind (15). After death, these people will attain realms of various Hells/\emph{naraka-s} and later take rebirth in various wombs of various creatures to face their \emph{karmic} fruits.

\emph{Bhāgavata} \emph{Purāṇa} (16) speaks about the presence of 28 types of \emph{naraka-s} for the purpose of allotting different \emph{karmic} punishments to different \emph{adharmic} activities. They are: \emph{Tāmisra}, \emph{Andhatāmisra, Raurava, Mahāraurava, Kumbhīpāka, Kālasūtra, Asi-patravana, Sūkaramukha, Andhakūpa, Kṛmibhojana, Sandaṁśa, Taptasūrmi, Vajrakaṇṭaka-śālmalī, Vaitaraṇī, Pūyoda, Prāṇarodha, Viśasana, Lālābhakṣa, Sārameyādana, Avīci, Ayaḥpāna, Kṣārakardama, Rakṣogaṇa-bhojana, Śūlaprota, Dandaśūka, Avaṭa-nirodhana, Paryāvartana} and \emph{Sūcīmukha}. Here is a summary of what kind of activities will lead a person to fall into a particular type of \emph{naraka} (17):

\begin{quote}
``A person who steals another's money, wife or possessions is put into the hell known as \emph{Tāmisra}. A man who tricks someone and enjoys his wife is put into the extremely hellish condition known as \emph{Andhatāmisra}. A foolish person absorbed in the bodily concept of life, who on the basis of this principle maintains himself or his wife and children by committing violence against other living entities, is put into the hell known as \emph{Raurava}. There the animals he killed take birth as creatures called \emph{ruru-s} and cause great suffering for him. Those who kill different animals and birds and then cook them are put by the agents of \emph{Yamarāja} into the hell known as \emph{Kumbhīpāka}, where they are boiled in oil. A person who kills a \emph{brāhmaṇa} is put into the hell known as \emph{Kālasūtra}, where the land, perfectly level and made of copper, is as hot as an oven. The killer of a \emph{brāhmaṇa} burns in that land for many years. One who does not follow scriptural injunctions but who does everything whimsically or follows some rascal is put into the hell known as \emph{Asipatravana}. A government official who poorly administers justice, or who punishes an innocent man, is taken by the assistants of \emph{Yamarāja} to the hell known as \emph{Sūkaramukha}, where he is mercilessly beaten.
\end{quote}

\begin{quote}
``God has given advanced consciousness to the human being. Therefore he can feel the suffering and happiness of other living beings. The human being bereft of his conscience, however, is prone to cause suffering for other living beings. The assistants of \emph{Yamarāja} put such a person into the hell known as \emph{Andhakūpa}, where he receives proper punishment from his victims. Any person who does not receive or feed a guest properly but who personally enjoys eating is put into the hell known as \emph{Kṛmibhojana}. There an unlimited number of worms and insects continuously bite him.
\end{quote}

\begin{quote}
``A thief is put into the hell known as \emph{Sandaṁśa}. A person who has sexual relations with a woman who is not to be enjoyed is put into the hell known as \emph{Taptasūrmī}. A person who enjoys sexual relations with animals is put into the hell known as \emph{Vajrakaṇṭaka}-\emph{śālmalī}. A person born into an aristocratic or highly placed family but who does not act accordingly is put into the hellish trench of blood, pus and urine called the \emph{Vaitaraṇī} River. One who lives like an animal is put into the hell called \emph{Pūyoda}. A person who mercilessly kills animals in the forest without sanction is put into the hell called \emph{Prāṇarodha}. A person who kills animals in the name of religious sacrifice is put into the hell named \emph{Viśasana}. A man who forces his wife to drink his semen is put into the hell called \emph{Lālābhakṣa}. One who sets a fire or administers poison to kill someone is put into the hell known as \emph{Sārameyādana}. A man who earns his livelihood by bearing false witness is put into the hell known as \emph{Avīci}.
\end{quote}

\begin{quote}
``A person addicted to drinking wine is put into the hell named \emph{Ayaḥpāna}. One who violates etiquette by not showing proper respect to superiors is put into the hell known as \emph{Kṣārakardama}. A person who sacrifices human beings to \emph{Bhairava} is put into the hell called \emph{Rakṣogaṇa}-\emph{bhojana}. A person who kills pet animals is put into the hell called Śūlaprota. A person who gives trouble to others is put into the hell known as \emph{Dandaśūka}. One who imprisons a living entity within a cave is put into the hell known as \emph{Avaṭanirodhana}. A person who shows unwarranted wrath toward a guest in his house is put into the hell called \emph{Paryāvartana}. A person maddened by possessing riches and thus deeply absorbed in thinking of how to collect money is put into the hell known as \emph{Sūcīmukha}.''
\end{quote}

After suffering punishment in these \emph{naraka-s} for various \emph{adharmic} activities, a \emph{jīva} will take birth in the earthly realm. The birth on the earth is again determined by the previous actions of the \emph{jīva}. In \emph{Manusmṛti} (18) we find an exhaustive list about what kind of future births in what kind of wombs a person will have based on \emph{adharmic} activities he performs in current life.

\begin{quote}
``Those who committed mortal sins having passed during large numbers of years through dreadful hells, obtain, after the expiration of (that term of punishment), the following births. The slayer of a \emph{brāhmaṇa} enters the womb of a dog, a pig, an ass, a camel, a cow, a goat, a sheep, a deer, a bird, a \emph{caṇḍāla}, and a \emph{pukkasa}. A \emph{brāhmaṇa} who drinks (the spirituous liquor called) \emph{sura} shall enter (the bodies) of small and large insects, of moths, of birds, feeding on ordure, and of destructive beasts. A \emph{brāhmaṇa} who steals goes a thousand times through the bodies of spiders, snakes and lizards, of aquatic animals and of destructive \emph{piśāca-s}. The violator of a \emph{guru}'s bed (enters) a hundred times (the forms) of grasses, shrubs, and creepers, likewise of carnivorous (animals) and of (beasts) with fangs and of those doing cruel deeds. Men who delight in doing hurt (become) carnivorous (animals); those who eat forbidden food, worms; thieves, creatures consuming their own kind; those who have intercourse with women of the lowest communities, \emph{preta-s}. He who has associated with outcasts, he who has approached the wives of other men, and he who has stolen the property of a \emph{brāhmaṇa} become \emph{brahmarākṣasa-s}. A man who out of greed has stolen gems, pearls or coral, or any of the many other kinds of precious things, is born among the goldsmiths. For stealing grain (a man) becomes a rat, for stealing yellow metal a \emph{haṁsa}, for stealing water a \emph{plava}, for stealing honey a stinging insect, for stealing milk a crow, for stealing condiments a dog, for stealing clarified butter an ichneumon; For stealing meat a vulture, for stealing fat a cormorant, for stealing oil a winged animal of the kind called \emph{tailapaka}, for stealing salt a cricket, for stealing sour milk a bird of the kind called \emph{balākā}. For stealing silk a partridge, for stealing linen a frog, for stealing cotton-cloth a crane, for stealing a cow an iguana, for stealing molasses a flying-fox; For stealing fine perfumes a musk-rat, for stealing vegetables consisting of leaves a peacock, for stealing cooked food of various kinds a porcupine, for stealing uncooked food a hedgehog. For stealing fire he becomes a heron, for stealing household-utensils a mason-wasp, for stealing dyed clothes a francolin-partridge; For stealing a deer or an elephant a wolf, for stealing a horse a tiger, for stealing fruit and roots a monkey, for stealing a woman a bear, for stealing water a black-white cuckoo, for stealing vehicles a camel, for stealing cattle a he-goat. That man who has forcibly taken away any kind of property belonging to another, or who has eaten sacrificial food (of) which (no portion) had been offered, inevitably becomes an animal. Women, also, who in like manner have committed a theft, shall incur guilt; they will become the females of those same creatures (which have been enumerated above). But (men of the four) castes who have relinquished without the pressure of necessity their proper occupations, will become the servants of \emph{dasyu-s}, after migrating into despicable bodies. A \emph{brāhmaṇa} who has fallen off from his duty becomes an \emph{Ulkāmukha Preta}, who feeds on what has been vomited; and a \emph{kṣatriya}, a \emph{Kaṭapūtana Preta}, who eats impure substances and corpses. A \emph{vaiśya} who has fallen off from his duty becomes a \emph{Maitrākṣajyotika Preta}, who feeds on pus; and a \emph{śūdra}, a \emph{Kailāsaka Preta}, who feeds on moths.''
\end{quote}

By being subjected to go through such a difficult journey through \emph{naraka-s} and series of births in the wombs of different creatures, those people who have abandoned both \emph{nivṛtti} and \emph{pravṛtti} \emph{dharma-s} suffer. By taking refuge in desire, selfishness, and \emph{adharma}, these people undergo immense suffering after the death of their human body. It may take a very long time for many to exhaust the fruits of their demerits and regain human life again. It is for this reason, the \emph{Upaniṣad}s and \emph{dharmaśāstra-s} stress that one should live a dharmic life through performance of \emph{karmānuṣṭhāna} and \emph{devatopāsana}, such that one travels on the upward journey towards \emph{mokṣa}, rather than spiral into deeper and deeper darkness of \emph{tamas} and the associated suffering.

\section*{Reference:}

\begin{enumerate}
\itemsep=0pt
\item
\emph{Chāndogya} \emph{Upaniṣad} 5.10.8
\item
\emph{Bṛhadāraṇyakopaniṣad} 6.2.16
\item
\emph{Manusmṛti} 12.52
\item
\emph{Manusmṛti} 12.54, \emph{Bhāgavata} \emph{Purāṇa} 5.26.3
\item
\emph{Bhagavad} \emph{Gītā} 16.1-3
\item
\emph{Bhagavad} \emph{Gītā} 16.4
\item
\emph{Bhagavad} \emph{Gītā} 16.11
\item
\emph{Bhagavad} \emph{Gītā} 16.5
\item
\emph{Manusmṛti} 12.40, For description of \emph{sāttvic}, \emph{rājasic} and \emph{tāmasic} qualities- \emph{Manusmṛti} 12.31-33
\item
\emph{Pāpa} is different from Abrahamic notion of Sin. \emph{Pāpa} simply means `demerit' accrued as a result of \emph{adharmic} activities that leads one to suffering. The term sin has been used in this sense alone.
\item
\emph{Chāndogya} \emph{Upaniṣad} 5.10.9
\item
\emph{Manusmṛti} 12.7
\item
\emph{Manusmṛti} 12.6
\item
\emph{Manusmṛti} 12.5
\item
\emph{Manusmṛti} 12.8
\item
\emph{Bhāgavata} \emph{Purāṇa} 5.26
\item
This summary has been taken from \emph{Bhaktivedānta} Vedabase website {[}\url{https://www.vedabase.com/en/sb/5/26}{]}
\item
\emph{Manusmṛti} 12.54-72. The excerpt is taken from translation of \emph{Manusmṛti} by George Buhler. {[}\url{http://www.sacred-texts.com/hin/manu/manu12.htm}{]}
\end{enumerate}

