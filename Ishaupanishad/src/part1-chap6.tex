\chapter{Verse 5}

\begin{moolashloka}
\dev{तदेजति तन्नेजति तद्दूरे~तद्वन्तिके ।}\\
\dev{तदन्तरस्य सर्वस्य तदु सर्वस्यास्य बाह्यतः ॥ 5 ॥}
\end{moolashloka}

\textbf{Word to word Meaning}: That moves (\dev{तदेजति}), that moves not (\dev{तन्नेजति}); That is far (\dev{तद्दूरे}), that is near (\dev{तद्वन्तिके}); That (\dev{तत्}) is inside (\dev{अन्तरस्य}) everything (\dev{सर्वस्य}), that (\dev{तत्}) is outside (\dev{सर्वस्य}) of everything (\dev{बाह्यतः}).

\textbf{Meaning}: That (\emph{Brahman}) is in universal movement, (yet) it is not in universal movement. That is very far, (yet) it is very near (as it is the innermost Self of everything). It is inside of everything (in the manifested Universe) and also outside of everything (i.e. beyond the manifested universe).

\textbf{Analysis}: In this verse, the \emph{Upaniṣad} continues with its description of \emph{Brahman}.

\dev{तदेजति तन्नेजति}- \emph{That moves, that moves not}. In the previous verse, it has been explained that in the \emph{pāramārthika daśā}, the absolute state, there is neither motion, nor rest. The universal movement (\emph{jagat}) is absent. \emph{Brahman} simply exists- without division and without universal movement. Yet, in the \emph{vyāvahārika daśā}, He moves swifter than the mind and the senses and all those which are in motion. In other words, \emph{Brahman} who is without any movement, any modifications, He himself appears as universal movement using His power of \emph{Māyā}. \emph{Brahman} is called ``\emph{Nirguṇa Brahman} (without gunas)'' in the former state and the state/condition itself is referred as ``\emph{pāramārthika daśā}'' and he is called ``\emph{Saguna Brahman} (with gunas)'' in the latter state/condition and the state itself is called ``\emph{vyāvahārika daśā}''.

This again has been explained in this verse as \dev{तदेजति तन्नेजति}. There is no \emph{movement} (\emph{jagat}) in \emph{Brahman} in the absolute state. Yet He appears as the universal movement in the \emph{vyāvahārika}, the state of duality. Hence, a person in \emph{vyāvahārika daśā} can only perceive the universe. But one who attains the state of \emph{jñāna}, the \emph{pāramārthika daśā}, he perceives \emph{Brahman} as the One who is without movement and without division yet who appears as the cosmos with all its names and forms.

\dev{तद्दूरे~तद्वन्तिके}- \emph{That is far, that is near}. Further, in \emph{vyāvahārika daśā}, \emph{Brahman} appears to be very far. A person who is constantly deluded by \emph{Māyā} cannot perceive anything but the temporary world of names and forms. For such a person, realization of \emph{Brahman} as the One who is beyond the names, beyond the forms, beyond every duality, every movement is simply not possible. As the \emph{Kenopaniṣad} (1) says, \emph{Brahman} is beyond the perception of the mind and the senses. People in the \emph{vyāvahārika daśā}, cannot comprehend the \emph{pāramārthika satya} using logic, scientific enquiry or using the senses. They cannot perceive or realize \emph{Brahman}. Hence, in the \emph{vyāvahārika daśā}, \emph{Brahman} appears to be very far away.

But, at the same time, \emph{Brahman} being the innermost Self, the \emph{Ātma} of everything, every entity, every movement in this cosmos, is very close to everyone. A person under the influence of \emph{avidyā}, believes that \emph{Brahman} is present somewhere far away. But, all he has to look is deep within his own-Self to find \emph{Brahman}. \emph{Brahman} is the \emph{Ātma} of all beings- living and non-living. He is present in all the objects, all life forms and lifeless entities of the cosmos. \emph{Brahman} is the very cosmos itself. Hence, a person who has developed \emph{viveka} (discrimination), does not go around searching for \emph{Brahman}, but instead he contemplates on his own inner-Self. Only such contemplation will ultimately lead to \emph{Ātmajñāna}.

\dev{तदन्तरस्य सर्वस्य तदु सर्वस्यास्य बाह्यतः-} \emph{That is inside everything, that is outside of everything}. The term \dev{सर्वस्य} refers to everything in the universe, all the entities, all the worlds of names and forms that are present in this universal movement. \emph{Upaniṣad} is saying that, \emph{Brahman} is inside all that exists, all that is in the \emph{jagat}. The very first verse of this Upaniṣad proclaims that \emph{Brahman} pervades everything in the \emph{jagat}. The same is being reiterated here again. The whole universal movement is \emph{Brahman} itself, who exists as \emph{Ātma}, the innermost Self of all entities present in this movement. But, at the same time, \emph{Brahman} is outside of, that is beyond all this manifestation. \emph{Brahman}, who is without movement (in \emph{pāramārthika daśā}) exists outside of the \emph{jagat}. In other words, the same \emph{Brahman} exists as endowed with movement and as being without movement.

The \emph{Upaniṣad} is re-stressing the fact that, all that exists is \emph{Brahman} alone. There is nothing other than \emph{Brahman}. The known and the unknown, the manifested and the un-manifested, the movement and beyond the movement, everything is \emph{Brahman} alone. People often assume that \emph{Nirguṇa Brahman} and \emph{Saguṇa Brahman} are different. And they try to argue, which of the two is superior. But, here the \emph{Upaniṣad} is proclaiming that no such distinction really exists. \emph{Brahman} who is inside the \emph{vyāvahārika} -- \emph{Saguṇa Brahman}, He himself exists outside of the \emph{vyāvahārika} --\emph{Nirguṇa Brahman}. This is the \emph{pāramārthika satya}.

\textbf{Summary:} Having explained \emph{Brahman} as being without movement, one, who is swifter than all objects in the Universe in the last verse, the \emph{Upaniṣad} continues its enunciation on the nature of \emph{Brahman} in this verse as well.

Every word in this verse has two layers of meaning. A person who is under the influence of \emph{avidyā} but on the path of \emph{sādhanā}, at first perceives that God is both in the state of motion and in the state of rest. The words- \dev{तदेजति तन्नेजति} in the verse being a reference to the temporary state of motion and rest within \emph{jagat} or universal movement. So, a person at first may perceive God as either being in motion or in a state of rest. Further, he comes to believe that God must be someone sitting atop some hill or enjoying in heaven very far away. He may even worship God to be someone confined in body and mind like himself, but with much more strength and power. He perceives a God who is far away, above and unconcerned towards the human world. As the person progresses on the path of spiritual evolution, he begins to understand, God is not as such present in some hill or heaven very far away completely unconcerned with the world, but instead He is the life-force, the essence that inhabits every object around him. \emph{Brahman} is present both in the state of rest and that of motion. He is present in the far off objects, so also in the objects surrounding one. One then perceives \emph{Brahman} as being present inside every person, every object and also outside in the surrounding environment, surrounding space. The person begins to understand that, the whole of manifested universe is inhabited by \emph{Brahman}. This is the first layer of understanding.

As the person evolves further spiritually, as his mind attains \emph{citta śuddhi}, he begins to understand that, the universe of names and forms, the entire cosmos, the manifestation is not the absolute reality. They are temporary in nature. They have a beginning and hence will have an end too. On other hand, \emph{Brahman} is permanent; He is beyond time and space. Such a person will understand that the cosmos as we perceive is the temporary state, a state of apparent reality (\emph{vyāvahārika daśā}) caused by \emph{Brahman} Himself by His power of \emph{Māyā}. But, beyond this \emph{vyāvahārika satya} (Relative Truth), there is a state of absolute reality (\emph{pāramārthika} \emph{daśā}). This state of absolute reality is the state of \emph{jñāna}. A \emph{sādhaka} having thus attained the state of \emph{jñāna}, perceives neither motion nor rest, neither forms nor formlessness. He perceives one \emph{Brahman} everywhere. He sees that it is \emph{Brahman} itself, who is though without any movement, is the material and intelligent cause of this universal movement. In other words, it is same \emph{Brahman} who appears to move as universal movement (\dev{तदेजति}) in \emph{vyāvahārika daśā}, but who does not have universal movement (\dev{तन्नेजति}) in \emph{pāramārthika daśā}. He appears very far for a person in \emph{vyāvahārika} as he is deluded by \emph{avidyā} hence unable to perceive the \emph{pāramārthika} \emph{satya}. But for a \emph{jñānī}, \emph{Brahman} is very near, as he sees \emph{Brahman} in everything, he perceives \emph{Brahman} as the innermost Self of everything. He is inside everything here in the \emph{vyāvahārika daśā}, but is also outside of this \emph{vyāvahārika daśā}. That is, it is same \emph{Brahman} who is present in both the relative and the absolute states.

Just as with a dream wherein the individual is both the material and instrumental cause of his dream and hence present both inside the dream and outside of the dream state. So is \emph{Brahman} the material and instrumental cause of Universe and is present both in \emph{vyāvahārika} and outside of it. Just as the dream state is temporary and relative reality when compared to waking state, so is the \emph{vyāvahārika}, which is a temporary and relative state in comparison to \emph{pāramārthika} \emph{daśā}. A \emph{jñānī} having woken up from the cosmic sleep, sees only the \emph{pāramārthika satya}, the truth of non-duality (2). For him, there is no separate \emph{Iśvara} different from \emph{jīva} and \emph{Jagat}. He perceives \emph{Ātma} alone in everything.

\section*{References}

\begin{enumerate}
\item
  \emph{Kenopaniṣad} Part 1.3
\item
  \emph{Kenopaniṣad} Part 2.5
\end{enumerate}
