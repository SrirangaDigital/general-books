\chapter{Blessings from \emph{Svāmī/Svāminī-s}}

\emph{It is not an easy task to write a~vyākhyāna~on the~Iśāvāsyopaniṣad, however Nithin Sridhar has made a valiant attempt to decode the message of this Upaniṣad. Adi Shankara once said ``asampradāyavidmūrkhavadupekṣaṇiyaḥ,'' which means, the person who does not value the teaching tradition has to be considered as foolish and ignored. From reading Nithin's commentary, one thing becomes very clear ---he is clearly a~sampradāyavit, a person who values the traditional teaching pedagogy of~Advaita-Vedānta handed down through ancient lineages, rather than a~sampradāyakṛt, a person who is desirous of giving new meanings and starting a new lineage. The success of this commentary primarily rests on its ability to follow Adi Shankara's exposition closely. In reading this commentary, one can be assured that one is not being misled by newfangled ideas or notions that have no basis in the teaching tradition. I congratulate Nithin on his diligence and loyalty to the sacred parampara of gurus in the tradition.~}
\medskip

\begin{flushright}
\textbf{---\emph{Svāminī} Svatmavidyananda,}\\
\textbf{Arsha Vijnana Gurukulam,}\\
\textbf{Eugene, OR, USA}
\end{flushright}
\medskip

\emph{I am happy that I had gone through this book written by Sri Nithin Sridhar and he had beautifully explained the thoughts of Isavasya Upanishad in a very simple language that would make even toddlers in spiritual realm walk easily and get the glimpses of this~Upanishad. It is said, `upanishad shabdena brahma vidya uchyate', Upanishad means Brahma Vidya or the knowledge of~Brahman. This Upanishad speaks about the nature of reality in verse 8 and it is meant for contemplation for all seekers. Author Sri Nithin had touched upon all the important points of the Upanishad. I am sure whoever studies this book will get a~great clarity on the subject matter and since it is presented in a simple language without losing the flow of beauty of the Upanishad itself, I am sure it will inspire all seekers whoever reads it.}

\emph{I congratulate Sri Nithin for this remarkable work and invoke Lord's blessings on him for his continued journey on spiritual path}
\medskip

\begin{flushright}
\textbf{---\emph{Svāmī} Chidrupananda,}\\
\textbf{\emph{Ācārya}, Chinmaya Mission,}\\
\textbf{Noida, India}
\end{flushright}
\medskip

\emph{The Upanishads represent humanity's first and most profound attempt to understand and to interpret the deepest truths of~life and existence. They form the foundation of Hindu spiritual and philosophical tradition. The Iśāvāsyopaniṣad is chronologically placed first among the ten Upanishads interpreted by Sri Adi Shankaracharya. It tells the story of man's spiritual ascent and destiny in 18 verses and describes not only the highest spiritual philosophy of~Vedanta but also its practical application in everyday life through the practice of renunciation and non-attachment.~}

\emph{The approach of this English commentary is simple, modern, unconventional and meant for the general reader.~The beginners of Upanishadic philosophy will find this commentary interesting because of the lucidity of style. They will derive great benefit and pleasure especially because of its direct and unconventional treatment of the subject. Written with an earnest conviction and reverence for India's spiritual heritage, this book takes a beginner on a path to spiritual self-discovery.}
\medskip

\begin{flushright}
\textbf{---\emph{Svāmī} Tattwamayananda,}\\
\textbf{Minister-in-Charge}\\
\textbf{Vedanta Society of Northern California,}\\
\textbf{California, USA.}
\end{flushright}
\medskip

\emph{I had a look at your work on "Isopanishad". It is monumental, to say the least. The work follows Acharya's commentary truthfully, while charting out its own individuality. Those interested in work of this type will surely enjoy it and benefit hugely from it. I have delivered talks on this great Upanishad and I believe that every Hindu should read and try to understand it, for, it contains the blueprint of Hindu religious practices and ideals. I am sure that your work will find go a long way in bringing light of Indian spiritual wisdom to many. Congratulations and thank you for such a great work that requires sustained effort of years!}
\medskip

\begin{flushright}
\textbf{---\emph{Svāmī} Samarpananda,}\\
\textbf{Monk,}\\
\textbf{Ramakrishna Mission,}\\
\textbf{Vivekananda educational and Research Centre,}\\
\textbf{Belur Math, Kolkata, India.}
\end{flushright}
\medskip

\emph{I congratulate Sri Nithin Sridhar for writing the book, Īśopaniṣad, An English Commentary. I have read the book, and find that it is written well. The first part contains each verse presented well with word for word meaning, general translation and a commentary on the verse. The second part deals with all the general aspects of ātmajñana and the pursuit of mokṣa. This book would not be complete without that. }

\emph{The word Upaniṣad in its yaugikārtha means brahmavidyā. By its definition it is the one that loosens the bondage of ignorance and removes the ignorance totally and makes the one who pursues this knowledge, gain this knowledge that, `I am the limitless vastu, Brahman.' This knowledge is presented in every Upaniṣad. And Nithin has done well by presenting all the aspects of this pursuit, the various sadhanas needed to assist the pursuit, and the all the gatis for a human being in the second part of the book making it complete. }

\emph{I wish him all the best and wish that he writes more such books.}
\medskip

\begin{flushright}
\textbf{---\emph{Svāminī} Brahmaprakasananda,}\\
\textbf{Arsha Vijnana Gurukuam,}\\
\textbf{Nagpur, India. }
\end{flushright}

