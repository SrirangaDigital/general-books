\chapter{Śānti-mantra}

\begin{moolashloka}
\dev{ॐ पूर्णमदः पूर्णमिदं पूर्णात् पूर्णमुदच्यते ।}\\
\dev{पूर्णस्य पूर्णमादाय पूर्णमेवावशिष्यते ॥}\\
\dev{ॐ शान्ति: शान्ति: शान्ति: ॥}
\end{moolashloka}
\vskip -10pt

\textbf{Word to Word Translation:} \emph{Oṃ} (\dev{ॐ}) That is Full/Complete (\dev{पूर्णमदः}), This is Full/Complete (\dev{पूर्णमिदं}), From (That) Fullness (\dev{पूर्णात्}), (This) Fullness (\dev{पूर्णम्}), has emerged (\dev{उदच्यते}); of (this) Fullness (\dev{पूर्णस्य}), (when) Fullness is taken/perceived (\dev{पूर्णमादाय}), Fullness (\dev{पूर्णम्}) alone (\dev{इव}) remains (\dev{अवशिष्यते}).

\textbf{Meaning:} \emph{Oṃ}, That (i.e. Supreme \emph{Brahman} in His \emph{nirguṇa} / unconditioned state) is Full/Complete. This (i.e. the universe/\emph{jagat} of names and forms which is inhabited by \emph{Brahman} as its Innermost Self/ \emph{ātma} i.e. \emph{saguṇa} or Conditioned \emph{Brahman}) is Full/Complete. From That Fullness (i.e. \emph{Nirguṇa} \emph{Brahman}), This Fullness (i.e. \emph{jagat}) has been projected. When the Fullness (i.e. unconditioned Reality) of this Fullness (i.e. the \emph{jagat}/Universe) is realized, Fullness (i.e. Unconditioned \emph{Brahman}) alone remains.

\textbf{Analysis}: The \emph{śānti-mantra} of the \emph{Upaniṣad} beautifully explains the gist of all the \emph{Upaniṣad-s} in a single verse. It explains about the God* (\emph{Brahman}), the world (\emph{jagat)} and their mutual relationship.

\dev{ॐ}- \emph{Oṃ.} \emph{Oṃ} is \emph{Brahman}. It is the symbol of \emph{Brahman}. It represents the whole existence (1).

\dev{पूर्णमदः}- \emph{That is Full/Complete}. \emph{Adaḥ} means `That'. The pronoun `that' is used in the sense of being far away or which is not empirically perceived. Here it refers to \emph{Brahman} who** seems to be transcendent and beyond the grasp of mind and senses. The \emph{mantra} is describing \emph{Brahman} as \emph{pūrṇam} - complete. The word `complete' denotes multiple things. It means \emph{Brahman} is \emph{one whole} without a second and without any parts or divisions (2). It further signifies that, \emph{Brahman} alone exists. Everything that exists is \emph{Brahman}. \emph{Brahman} is everything and everywhere and there is nothing outside of Him. He does not have any hankering, goals or desires to be achieved as He is already complete. He is \emph{self-existing}, without causation or dissolution (3). In other words, He is limitless, without any limitation of time, space, attributes, form or ignorance. He is `complete' in every sense of the word. Hence, He is described as \emph{nirguṇa}-without \emph{guṇa-s} (4), \emph{niṣkala}-without parts and \emph{śānta}-without modifications like birth (5) etc.

\dev{पूर्णमिदं}- \emph{This is Full/Complete.} \emph{Idaṃ -} means ``Here''. It refers to the world of names and forms. The \emph{mantra} is saying that, the whole manifestation, the cosmos is `complete'. For, the question how can the world with multiplicities, with various names and forms that undergo modifications like birth, transformation and death can be called \emph{pūrṇam}? It is because, \emph{jagat}/universe is nothing but \emph{Brahman} itself, which seems to become limited by limiting principles (\emph{upādhi}) of name, form, time and space and becomes empirically available to perception as \emph{jagat}/universe.

\dev{पूर्णात् पूर्णमुदच्यते}- \emph{From (That) Fullness, (this) Fullness is projected.} This world, the whole cosmos is manifested/projected by \emph{Brahman}. \emph{Taittirīyopaniṣad} says- ``Know that as \emph{Brahman}, from which all the objects of the universe are born, by which they live and into which they merge back'' (6). Therefore, this whole cosmos and all the objects in it have manifested from \emph{Brahman}. \emph{Brahman} which is unconditioned and infinite, limiting itself through limiting principles appears as \emph{jagat}/universe. Hence, the universe has no existence independent of \emph{Brahman}. It emerges from, sustains itself in and merges back into \emph{Brahman}. Whatever is perceived as universe is \emph{in} \emph{reality} \emph{Brahman} alone, and the apparent multiplicities of names and forms are just a product of \emph{Māyā} (7).

\dev{पूर्णस्य पूर्णमादाय पूर्णमेवावशिष्यते}- \emph{When the Fullness of this Fullness is taken/ realized, Fullness (i.e. Brahman) alone remains.} Here the word \emph{ādāya} means `having taken' i.e. having understood/perceived. The \emph{mantra} is saying that, when the completeness of the \emph{jagat}/universe is realized i.e. when it is realized that the universe with its myriad of names and forms is in actuality \emph{non-different} from \emph{Brahman} or to put it differently, when \emph{Brahman} who exists as innermost Self/\emph{Ātma} of Jagat is perceived after having discarded the limitations imposed by names and forms as being non-Self (a product of illusion); then completeness alone remains i.e. unconditioned \emph{Brahman} alone is perceived.

\dev{ॐ शान्ति: शान्ति: शान्ति:}- \emph{Oṃ. Peace, Peace, Peace}. Here `peace' is repeated thrice to indicate that peace and balance be established at physical, vital, and mental levels.

\textbf{Summary}: In the Upaniṣads, \emph{Brahman} or God is described as being `\emph{ananta}-infinite' (8). Being infinite, He is without any limitations of time, space, name or form. In other words, He is not limited by any limiting principles (\emph{upādhi-s}). Hence, He is called \emph{pūrṇam}. His completeness includes completeness in both time and space. Hence, it is said that `\emph{Brahman} alone exist' or `All that exist is \emph{Brahman'} because He is everywhere and there is nothing outside of Him or other than Him. From such a \emph{Brahman} who is Infinite, this \emph{jagat} of names and forms has emerged out. In other words, \emph{Brahman} limited by limiting principles of name and form, itself manifests empirically as \emph{jagat}. Hence, this \emph{jagat} has \emph{Brahman} as its substratum, as its innermost Self and hence is also `complete'. But, because such a manifestation is only a product of \emph{Māyā}, an appearance and not a real transformation or extension, \emph{Brahman} even after being manifest as \emph{jagat} does not lose its completeness/infiniteness. When a person, by means of \emph{vicāra}/self-enquiry, has discarded the outer names and forms as products of \emph{Māyā}, and perceives \emph{Brahman} who exists as innermost Self, then \emph{Brahman} alone remains. That is, he*** attains \emph{jñāna} and perceives \emph{Brahman} alone. Just as in a desert, a mirage is first perceived as real and on introspection is revealed as only an appearance; and even after such a realization, the appearance of mirage continues, but, the person now understands that what appears as water is sand only. Similarly, a person having realized that \emph{jagat} is in reality non-different from \emph{Brahman}, he perceives \emph{Brahman} alone even in the universe. Hence, completeness alone remains.

\emph{*Note: In the entire book, the English term `God' has been used interchangeably with Sanskrit term `Brahman'. It must be noted that `God' has been used in the sense of `Ultimate Reality' as understood in Hindu philosophy and not in the sense of Abrahamic God. Likewise, whenever the term `sin' is used it is used not in the Abrahamic sense of original sin, but only in the Hindu sense of Karmic demerit that one accrues by performing Adharmic activities.}

\emph{**Note: In the entire book, Brahman is is referred by using both pronouns `It' and `He' and consequently using words like both `who' and `which'. It must be noted that Brahman is beyond gender in its absolute state. However, we as devotees and seekers can relate to Brahman either as He or She or It depending upon the temperament of an individual.}
\newpage

\emph{***Note: In the entire book when referring to single person, seeker or otherwise, the pronoun `he' is used for convenience. The pronoun should be taken as a reference to a person of any gender, and not just to males.}

\section*{References}

\begin{enumerate}
\itemsep=0pt
\item
	\emph{Māṇḍūkyopaniṣad} Verse 1 and 2.
\item
  Īśopaniṣad Verse 4.
\item
  Īśopaniṣad Verse 8.
\item
  \emph{Śvetāśvataropaniṣad} 6.12
\item
  \emph{Śvetāśvataropaniṣad} 6.19
\item
  \emph{Taittirīyopaniṣad} 3.1.3.
\item
  \emph{Gauḍapāda Kārikā} on \emph{Māṇḍūkyopaniṣad}, \emph{Āgama Prakāraṇa}, Verse 17.
\item
  \emph{Taittirīyopaniṣad} 2.1.1
\end{enumerate}
