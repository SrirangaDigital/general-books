\chapter{Verse 18}

\begin{moolashloka}
\dev{अग्ने नय सुपथा राये अस्मान् विश्वानि देव वयुनानि विद्वान् ।}\\
\dev{युयोध्यस्मज्जुहुराणमेनो भूयिष्ठां ते नमउक्तिं विधेम ॥ 18 ॥}
\end{moolashloka}

\textbf{Word to word Meaning:} O \emph{Agni} (\dev{अग्ने}) lead (\dev{नय}) through the good path (\dev{सुपथा}) to wealth/Karmic fruits (\dev{राये}) us (\dev{अस्मान्}), all (\dev{विश्वानि}) god (\dev{देव}) deeds/\emph{Karma-s} (\dev{वयुनानि}) knower (\dev{विद्वान्}); Remove from us (\dev{युयोध्यस्मत्}) the devious sin (\dev{जुहुराणम् एनः}). Many (\dev{भूयिष्ठां}) to you (\dev{ते}) verbal salutations (\dev{नमउक्तिं}) I offer (\dev{विधेम}).

\textbf{Meaning:} Oh \emph{Agni}/Fire, O \emph{Deva}, the knower of all our \emph{karma-s} (both \emph{karmānuṣṭhāna} and \emph{vidyā upāsanā}), (please) lead us through the good path to (enjoyment of divine) wealth (i.e. the Karmic fruits of \emph{karmānuṣṭhāna} and \emph{vidyā upāsanā}) and (Please) remove from us the devious sins (that may obstruct our journey). I offer many verbal salutations to you.

\textbf{Analysis:} Now the \emph{Upaniṣad} continues with the prayer to \emph{Agni}.

\dev{अग्ने नय सुपथा राये अस्मान्}\emph{- O Agni (the Divine will), lead us through the good path to (enjoyment of divine) wealth}. \dev{अग्नि:} - \emph{Agni} refers to Divine will that is present in every individual. It is the `\emph{power of will}' which is the source of all activity. It is the power of will that accomplishes all actions. This Divine will that brings about all activity and presides over them is called \dev{अग्नि:}. The good path or \dev{सुपथा}- refers to \emph{devayāna} or the northern path through the \emph{Sūrya} that takes one to \emph{Hiraṇyagarbha} or \emph{Satya Brahman}. It is called `\emph{good'} because it is a path of `no-return' i.e. a person who once attains Hiranyagarbha will not return to \emph{saṁsāra}. Hence, he will be free from gross and subtle limitations. Hence, this path towards \emph{Satya Brahman} is called \dev{सुपथा. राये}- literally means `\emph{wealth'}. It refers to the divine wealth i.e. the enjoyment of Karmic fruits of attaining \emph{Satya Brahman} due to practice of \emph{karmānuṣṭhāna} and \emph{Satya Brahma Upāsanā}. A person who attains \emph{Satya Brahman} is free from the limitations that bound him when he had \emph{sthūla} (gross) and sūkṣma (subtle) bodies (1). He will be free from Karmic cycle of birth, death and re-birth. He will neither feel cold nor hot. He will neither feel grief nor happiness. He will be free from all the limitations that bind a person in physical and subtle universes. It is this wealth of `relative immortality' that one enjoys by attaining \emph{Satya Brahman} through the worship of \emph{Agni} (i.e. \emph{karmānuṣṭhāna}) and \emph{Sūrya} (i.e. \emph{vidyā upāsanā}).

\dev{विश्वानि देव वयुनानि विद्वान्}- \emph{The god who is the Knower of all actions}. \emph{Agni} is the Divine Will. He presides over all the Karmas and accomplishes them. Hence, he is called \dev{विश्वानि वयुनानि विद्वान्}- the Knower of all the Karmas in the Cosmos. In this context, it specifically refers to Agni as being Knower of all the \emph{vidyā} and \emph{Avidyā} \emph{upāsanā} one has performed. The person on the verge of death is praying to \emph{Agni} who is the Knower of all Karmas and hence knows that the person has performed all the \emph{karmānuṣṭhāna} and \emph{vidyā} \emph{upāsanā} and hence, requesting him to lead him to \emph{Satya Brahman}.

\dev{युयोध्यस्मज्जुहुराणमेनो}- \emph{(Please) remove from us any devious sins}. The person is praying \emph{Agni} who is the knower of all our Karmas to not only lead him through the northern path to \emph{Satya Brahman}, but also to remove any deviousness, any sins, any obstacles that are blocking his path. \dev{जुहुराणम् एनः} refers to sensory desires and attachment that binds one to \emph{saṁsāra}. They refer to the \emph{ariṣaḍ varga-s} i.e. the impurities of mind like lust, anger, conceit etc. The person is praying to \emph{Agni} to remove all these impurities of mind.

\dev{भूयिष्ठां ते नमउक्तिं विधेम}- \emph{I offer many verbal salutations to you.} The person is saying that, O Agni, I who have worshipped you all my life by performing \emph{Agnihotra-s} (fire-sacrifices) and all other \emph{karmānuṣṭhāna} properly, now have approached death. Please remove any flaws in me and lead me through the Northern path to \emph{Satya Brahman}. I surrender myself completely to you. I salute you verbally again and again. The salutation is verbal because having approached death the devotee can no longer perform any physical ritual.

\textbf{Summary:} The person who is on the verge of death continuous with his prayer to \emph{Agni}. After having prayed to Sun, to lead him to \emph{Satya Brahman}, the person prays to \emph{Agni} to help him to go through the good path of the Sun to \emph{Satya Brahman}. The person is offering his prayers for the last time to \emph{Agni} and requests the deity to remove any faults, any deficiency that are still present in him, so that at death, he could overcome the limitations of the gross body through \emph{Agni} and attain \emph{Satya Brahman} through \emph{Sūrya}.

\section*{References}

\begin{enumerate}
\itemsep=0pt
\item
  \emph{Bṛhadāraṇyakopaniṣad} 5.10.1
\end{enumerate}


