\chapter{Verse 12}

\begin{moolashloka}
\dev{अन्धं तमः प्रविशन्ति ये असम्भूतिमुपासते् ।}\\
\dev{ततो भूय इव ते तमो य उ सम्भूत्यां रताः ॥ 12 ॥}
\end{moolashloka}

\textbf{Word to word Meaning:} Into the blinding (\dev{अन्धं}) darkness (\dev{तमः}) enter (\dev{प्रविशन्ति}) they who (\dev{ये}) worship (\dev{उपासते}) the unmanifested (\dev{असम्भूतिम्}); into even greater (\dev{भूय}) darkness (\dev{तमः}) as it were (\dev{इव}) than that (\dev{ततो}) they (enter) (\dev{ते}), those who (\dev{य उ}) delight/indulge (\dev{रताः}) in the manifested (\dev{सम्भूत्यां}).

\textbf{Meaning:} Those who worship the \emph{unmanifested} (i.e. the \emph{Kāraṇa} \emph{Brahman} or \emph{mūla prakṛti}), they enter into blinding darkness (of ignorance); and those who worship the \emph{manifested} (i.e. \emph{Kārya Brahman} or \emph{Hiraṇyagarbha}), they enter, as it were into a greater darkness (of bondage and limitation).

\textbf{Analysis:} After explaining that a person in \emph{pravṛttimārga} must practice \emph{karmānuṣṭhāna} and \emph{devatopāsana} together, the \emph{Upaniṣad} now speaks about the manner in which \emph{devatopāsana} must be carried out.

\dev{अन्धं तमः प्रविशन्ति ये असम्भूतिमुपासते}-\emph{Those who worship the unmanifested (alone) will enter blinding darkness.} \dev{असम्भूतिम्} refers to the \emph{unmanifested} \emph{prakṛti}, the source of all existence, the source of \emph{vyāvahārika daśā}. It is also called `\emph{avyākṛta}' or `\emph{Kāraṇa Brahman} (1)'. The whole manifestation, the whole movement, exists in this `\emph{mūla prakṛti}' in an undifferentiated form. \emph{Brahman} with the movement i.e. the \emph{Saguṇa Brahman} exists as \emph{Kāraṇa Brahman} and \emph{Kārya Brahman}- the unmanifested source and the manifested cosmos. The unmanifested cause, the \emph{mūla prakṛti} which is the source of whole cosmos --the source of whole \emph{avidyā} --is called \dev{असम्भूतिम्}. Hence, being the source of \emph{avidyā}, She is herself \emph{avidyā}. She is the combination of the three-Gunas that exist in her in undifferentiated equilibrium.

Those people who practice \emph{vidyā upāsanā} by the worship \emph{Kāraṇa Brahman} alone, will attain \emph{Kāraṇa Brahman} i.e. will lose their existence in the universe (i.e. lose both their gross and subtle bodies) and be absorbed into \emph{mūla prakṛti}. But, they fail to attain the \emph{Supreme Brahman} (\emph{pāramārthika satya}), because the duality of \emph{Kāraṇa} and \emph{Kārya Brahman} exist within \emph{vyāvahārika daśā} and hence, attaining the state of \emph{Kāraṇa} \emph{Brahman} does not result in transcending of the duality.

Further, this absorption into \emph{unmanifested} \emph{prakṛti} will be only temporary as they have not overcome the limitations of the \emph{manifested}. Hence, even in absorption in \emph{prakṛti}, they will maintain their individual \emph{saṁskāra-s}. Due to this, one will eventually return to \emph{saṁsāra} (2). Hence, by attaining \emph{Kāraṇa Brahman} alone, one cannot attain \emph{pāramārthika satya} (\emph{mokṣa}). Further, \emph{mūla prakṛti} being the seed of all activity and desires, is \emph{avidyā} Herself and hence one who attains \emph{mūla prakṛti}, though he loses his existence in the cosmos, he will still be stuck in \emph{avidyā}. Hence, the \emph{Upaniṣad} says that, he who worships the \emph{unmanifested} alone; he would enter blinding darkness of Ignorance.

\dev{ततो भूय इव ते तमो य उ सम्भूत्यां रताः}- \emph{Those who take delight in worship of the manifested (alone), they enter, as it were into a greater darkness.} \dev{सम्भूत्यां} refers to the `\emph{Kārya Brahman} (3)', the manifested cosmos. A person who worships \emph{Kārya Brahman} attains realms of various deities and achieves control over various aspects of nature. As previously mentioned, the \emph{vidyā upāsanā} leads a person to \emph{devaloka}, wherein the \emph{vidyā upāsanā} specifically referred to the \emph{upāsanā} of \emph{Hiraṇyagarbha} (\emph{Kārya Brahman}). Hence, a person by worshipping \emph{Hiraṇyagarbha} (the first born) conquers all the \emph{loka-s} and eventually attains \emph{Hiraṇyagarbha} (i.e. \emph{Kārya Brahman} or \emph{Brahmaloka}). Such a person will gain control over the limitations imposed by his subtle body. Just as a \emph{jīva} present in physical realm in his gross body experiences limitations imposed by the physical realm so also a \emph{jīva} in his subtle body experiences various limitations. Such a person by the worship of \emph{Kārya Brahman}, the \emph{Hiraṇyagarbha} will achieve the \emph{siddhi-s} like \emph{aṇimā, garimā}, etc. (4) and attains \emph{Brahmaloka}. But, such a control over various aspects of nature, control over the limitations of one's subtle body does not lead one to \emph{mokṣa} by itself. Instead, it increases one's attachment to these accomplishments. The \emph{Upaniṣad} is saying that, such a person goes, as it were into a greater darkness. Why is it, as it were a greater darkness? Because by the attainment of various \emph{siddhi-s} and \emph{loka-s} one often ends up mistaking it to be the ultimate accomplishment. This would bind him more strongly to \emph{saṁsāra}. Hence, the \emph{Upaniṣad} says that such a person who indulges in attaining subtle realms and gaining of \emph{siddhi-s} alone will end up in greater bondage to \emph{saṁsāra}.

\textbf{Summary:} In the previous three verses, the \emph{Upaniṣad} has clearly explained how the \emph{vihita karma} and \emph{apara-bhakti} must be combined together by a householder, by which he would achieve \emph{mokṣa} through \emph{kramamukti}. Now, the \emph{Upaniṣad} further explains about the manner in which \emph{bhakti} or \emph{vidyā upāsanā} must be practiced in order to attain \emph{kramamukti}.

\emph{Brahman} is eternal, without division and without movement. In the \emph{pāramārthika daśā}, He is called `\emph{Nirguṇa Brahman}'. The same \emph{Brahman}, when he manifests (appears as) this \emph{jagat}-universal movement using His power of \emph{Māyā}, i.e. in the \emph{vyāvahārika daśā}, He is called `\emph{Saguṇa Brahman}'.

\emph{Saguṇa Brahman} is both the cause and the effect. He is the \emph{seed} that manifests this cosmos as well as the manifested cosmos itself. The seed or the \emph{unmanifested source} is called ``\emph{Kāraṇa Brahman}'' (i.e. \emph{mūla prakṛti}) and the manifested cosmos is called ``\emph{Kārya Brahman}'' (i.e. \emph{Hiraṇyagarbha}). \emph{Kāraṇa} and \emph{kārya} are thus two aspects of same \emph{Saguṇa Brahman}. The \emph{Upaniṣad} is saying that a person who having transcended his physical existence (\emph{sthūla śarīra}) through \emph{karma}, if as part of his practice of \emph{bhakti} he worships the \emph{kārya} or the \emph{Kāraṇa Brahman} alone to the exclusion of the other, he will increase his bondage to subtle- \emph{saṁsāra} as he would not be able to transcend them.

A person who worships the unmanifested alone, attains the unmanifested source. However, the \emph{Kāraṇa Brahman} being the source of this cosmos, source of this \emph{avidyā} is itself \emph{avidyā}. Hence, a person who attains \emph{Kāraṇa Brahman}, even though he loses his existence in the manifestation he would still get stuck in the unmanifested \emph{avidyā}. Further, it would only be a temporary attainment, as due to the exclusion of the manifested he has not overcome the limitations of the manifested. Hence, he would regain individual personality (either subtle personality alone or subtle and gross both depending upon whether he has practiced \emph{karmānuṣṭhāna} or not) and will return to \emph{saṁsāra}. On the other hand, those who devote themselves to acquiring one \emph{siddhi} after another, in attaining one \emph{loka} after another, by the worship of \emph{Kārya Brahman} alone are completely lost in this \emph{saṁsāra}. They would be continuously running behind \emph{siddhi-s} without an end to it. They would mistake their accomplishment as the highest truth. Hence, they fall into a greater bondage of ignorance.

\section*{References}

\begin{enumerate}
\itemsep=0pt
\item
  \emph{Kāraṇa Brahman} or \emph{asaṁbhūti} refers to the world beyond the `\emph{mahat}'. That is, the three higher realms- \emph{jana loka}, \emph{tapa loka} and \emph{satya loka} all come under \emph{asaṁbhūti} as they exist in unmanifested state. The three \emph{guṇa}-s are in dynamic equilibrium. Such \emph{Kāraṇa Brahman} is called `\emph{avyākṛta}' because they are beyond the perception of mind and senses. It is called `\emph{Kāraṇa Brahman}' because it causes the creation of the subtle and gross universes.
\item
  A person, who has practiced \emph{Karma} and \emph{asaṁbhūti} \emph{upāsanā} without practicing the \emph{upāsanā} of the \emph{Hiraṇyagarbha}, will return to subtle- \emph{saṁsāra} by regaining his subtle body. On the other hand, a person who practices \emph{asaṁbhūti} \emph{upāsanā} alone without even doing Karma, he would return to gross- \emph{saṁsāra}, by regaining both subtle and gross bodies to face Karmic fruits.
\item
  \emph{Kārya Brahman} or \emph{Hiraṇyagarbha} is also called `\emph{mahat}'. He is the first born, i.e. the first to be manifested out of the \emph{mūla prakṛti}. The realm of \emph{Hiraṇyagarbha} is called `\emph{mahat-loka}'. This \emph{Kārya Brahman} projects out the whole manifestation, both the subtle (\emph{bhuvar loka} and \emph{svar loka}) and the gross realms (\emph{bhu loka}). Hence, \emph{Hiraṇyagarbha} is described as having the \emph{bhu loka} as his head, the \emph{bhuvar} and \emph{svar loka-s} as his hands and feet.
\item
  \emph{Patañjali} mentions 8 major Siddhis (or powers). They are- \emph{aṇimā} (becoming very small), \emph{mahimā} (becoming very large), \emph{laghimā} (becoming very light in weight), \emph{garimā} (becoming very heavy), \emph{prāpti} (obtaining/ grasping things very far away), \emph{prākāmya} (fulfilling all desires), \emph{īśitva} (ability to control things at Will), \emph{vaśitva} (mastery over everything). \emph{Patañjali} \emph{Yoga Sūtra-s}, \emph{Vibhūti Pāda}.
\end{enumerate}

