\chapter{Verse 7}

\begin{moolashloka}
\dev{यस्मिन्सर्वाणि भूतानि आत्मैवाभूत् विजानतः ।}\\
\dev{तत्र को मोहः कः शोक एकत्वमनुपश्यतः ॥ 7 ॥}
\end{moolashloka}

\textbf{Word to word Meaning}: When (\dev{यस्मिन्}) all (\dev{सर्वाणि}) the objects/beings (\dev{भूतानि}) have verily (\dev{एव}) become (\dev{अभूत्}) the (innermost) Self (\dev{आत्म}) to the knower (\dev{विजानतः}); Then (\dev{तत्र}), what delusion/attachment (\dev{को मोहः})? What sorrow (\dev{कः शोक})? For him who (\dev{त्वम्}) perceives (\dev{अनुपश्यतः}) Oneness (\dev{एक:}).

\textbf{Meaning}: When, all the created objects have become his own innermost Self (\emph{Ātma}) for a \emph{jñānī} (the knower), then what delusion? What sorrow (to him)? He perceives oneness (alone).

\textbf{Analysis:} In this verse, the \emph{Upaniṣad} continues with the theme of \emph{jñānī} and the state of \emph{jñāna}.

\dev{यस्मिन्}- \emph{When, in whom}. Here, the \emph{Upaniṣad} is referring to the \emph{jñānī-s}, the \emph{jīvanmukta-s} who have attained the \emph{ātmajñāna}. It is saying, in those who have become \emph{jñānī-s} or when a person attains \emph{jñāna}.

\dev{सर्वाणि भूतानि आत्मैवाभूत्}- \emph{All the objects/entities become Ātma}. When a person attains \emph{Ātma} \emph{jñāna}, then all the multiplicities, all the names and forms merge into his own Self. He will perceive nothing other than \emph{Ātma}. He will perceive his Self in all names, all entities. Hence, the state of \emph{jñāna} is the state of \emph{pāramārthika satya}. In \emph{vyāvahārika} state, a person deluded by \emph{Māyā} perceives everything as different from his Self. He falsely identifies his Self with his body and mind and hence perceives this universe of names and forms as distinct entity. On the other hand, when a person attains \emph{jñāna}, he becomes free from the influences of \emph{Māyā} and hence, will perceive his \emph{Ātma} alone everywhere and in everything. Hence, for a \emph{jñānī}, \emph{Ātma} alone exists. All the beings, all the objects-living and non-living, the known and the unknown, the whole manifestation has merged into his own Self i,e. they have all become non-different from his innermost Self.

\dev{विजानतः}- \emph{The Knower.} Here the \emph{Upaniṣad} is not referring to someone who has a theoretical understanding of the philosophy or those who have studied and rationally understood the scriptures. It is not even referring to those who have practiced \emph{sādhanā} and have gained some siddhis and have experienced different realms of existence. Here, the `knower' specifically refers to a \emph{jīvanmukta} --one who has realized his Self. The \emph{Upaniṣad} is describing the state of \emph{jñāna}, the \emph{pāramārthika satya} that is perceived by the \emph{jñānī}.

\dev{तत्र को मोहः कः शोक}- \emph{Then, What delusion? What sorrow?} \dev{तत्र} means ``then'', that is when one has attained the state of \emph{jñāna} and hence perceives \emph{Ātma} alone in all the objects and all the beings, `\emph{then'}. \dev{को मोहः कः शोक}- What delusion? What sorrow? Only a person in the \emph{saṁsāra} bound by the eight \emph{pāśa-s} (1) is in deluded condition. \dev{मोहः} refers to attachment to and indulgence in the temptations of the world. Only a person bound by \emph{saṁsāra} is always chasing behind one or the other desires continuously. He gets attached to the pleasures that life offers and tries to run away from the sorrows and frustrations it causes, without understanding that \emph{sukha} and \emph{duḥkha} are both products of the bondage to Karmic cycle. But, for a \emph{jñānī} who has freed his own-Self from Karmic bondage having realized that the world is only an apparent reality and \emph{Brahman} alone exists, there is neither attachment nor aversion from anything as every object is his own Self. \emph{Kaṭhopaniṣad} (2) says- ``the wise having meditated on the Self (\emph{Ātma}) as one who is present as `bodiless' in all the bodies (\emph{śarīra-s}), who is present as `permanent eternal Self' in all the impermanent objects; having thus meditated on ``great'' and ``all-pervasive'' Self, he does not have any sorrow, he does not grieve for anything''. Hence, such a \emph{jñānī} desires nothing nor he turns away anything that comes his way. He is beyond the delusion and attachment caused by the \emph{saṁsāra}. Hence, he is untouched by the \emph{sukha} and \emph{duḥkha} that \emph{saṁsāra} offers him. For this reason, the present \emph{Upaniṣad} is saying that, to such a \emph{jñānī} in whom all the objects have become his own Self, no attachment, no sorrow is present in him as he perceives his own Self everywhere. Hence, he neither indulges in nor hates the worldly objects. A \emph{jīvanmukta} stays in the \emph{saṁsāra} the way a swan stays in the mud water- ever detached and untouched, unaffected by the temptations and sorrows of \emph{saṁsāra}.

\dev{एकत्वमनुपश्यतः}- \emph{He perceives Oneness alone}. A person having become a \emph{jñānī} perceives \emph{Ātma/Brahman} alone. He does not perceive multiplicities; he does not perceive the universe of names and forms. He perceives only \emph{Ātma} that pervades the entire manifestation. A \emph{jñānī} having removed the \emph{ajñāna} of duality perceives \emph{oneness} alone. A person in \emph{vyāvahārika daśā}, perceives all names, forms, all movements. The duality exists only for those who are bound by \emph{avidyā}. But, for a \emph{jñānī} only oneness exists, \emph{Ātma} alone exists. Here, `one' should not be taken to mean a numerical quantity. The `one' only indicates the absence of multiplicities, the absence of any being and any entity other than \emph{Brahman}. Hence, a \emph{jñānī} having realized his true Self, having realized that `\emph{jīvo brahmaiva nāparaḥ}', i.e. \emph{jīva} is \emph{Brahman} and not different, perceives his \emph{Ātma} alone everywhere and in everything.

\textbf{Summary:} Continuing with the theme about the \emph{jñānī} and the state of \emph{jñāna} of the last verse, the \emph{Upaniṣad} now again reiterates that, a realized person perceives only \emph{oneness}. He perceives his own Self, his \emph{Ātma} alone everywhere. For a \emph{jīvanmukta} the world of names and forms is only an apparent reality, a \emph{vivarta} which has been manifested by \emph{Brahman} Himself. These multiplicities exist as long as \emph{ajñāna} exist for it is through \emph{avidyā} and \emph{ajñāna} that \emph{Brahman} manifests the cosmos. The moment an individual attains \emph{ātmajñāna}, all the dualities, all movement merges in him. He will perceive only non-duality (i.e. oneness), he will perceive \emph{Ātma} alone.

The state of \emph{jñāna} is the state of \emph{pāramārthika} - without multiplicity and without movement. Hence, \emph{jñānī} having attained that state perceives \emph{Brahman} alone as \emph{Sat} --pure existence without any creation or dissolution, with neither motion nor rest, with neither form nor formlessness. A \emph{jīvanmukta}, while living in his body may still perceive the world of names and forms with his mind and the senses, but this vision of duality would be superseded/overridden by the awareness of non-duality/oneness behind the apparent multiplicities, the awareness of one \emph{Ātma} that pervades everything.

\section*{References}

\begin{enumerate}
\item
  Pasha means ``to bind''. \emph{Kulārṇava Tantra} enurnerates eight kind of \emph{pāśa-s} -- namely, pity (\emph{dayā}), ignorance and delusion (\emph{moha}), fear (\emph{bhaya}), shame (\emph{lajjā}), disgust (\emph{ghṛṇa}), family (\emph{kula}), custom (\emph{śīla}), and caste (\emph{varṇa}). (Sir John Woodrofe, Introduction and Preface, \emph{Mahānirvāṇa Tantra})
\item
  \emph{Kaṭhopaniṣad} 1.2.22
\end{enumerate}
