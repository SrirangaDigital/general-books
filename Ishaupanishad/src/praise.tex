\chapter{Praise for the Book}

\emph{Isavasya Upanishad is small in size, but quite comprehensive in its analysis of human priorities. It does not go into the ontological issues, like the Mandukya Upanishad does, but discusses the question of ideal human life and human action which would be conducive to self-knowledge. The origins of the concepts of karma yoga, the role of meditations and bhakti yoga, which we see in the Bhagavad Gita, can be found in this Upanishad. It also discusses the nature of the Supreme Reality, called Brahman in Vedanta.}

\emph{The present commentary by Sri Nithin Sridhar captures all these issues precisely and discusses in all thoroughness and in shastra tradition. The format adopted is very convenient to the reader. After the word to word meaning and the meaning of the mantra, the writer has made an analysis of the mantra and has given a summary at the end. The summary would provide the link to the next mantra. At the end, he has also provided the foot notes, not merely confined to the traditional commentary of Sri Shankaracharya, but to several texts, based on his own study. The commentator has presented the purport of the text in very clear language. The format is traditional but in some places the author seems to have given preference to the overall purport of the mantra and not much to the grammatically precise meanings of certain words.}

\emph{The part II of the text is perhaps the real text which the writer desired to write. The author liberates himself from the limitations of the text and examines it in the context of the whole gamut of Vedanta. It discusses the path of action (pravritti marga) and the path of withdrawal (nivritti marga). We find an exhaustive discussion on different levels of dharma such as the varna dharma, the ashrama dharma and so on. We find elaborate references to other texts such as Bhagavatam, Manu smriti and such other texts. There is also an elaborate discussion on the samskaras which we find in the dharma shastra texts. Similarly the author has examined the state of the jivanmukta, the person who is liberated while being alive. Section 4 of Part 2 of the book is unexpected but that is relevant to the modern man who stands disinherited from tradition and who is at a loss. The author discusses the pitfalls of the karma prohibited in the Vedas. The discussion is about the person who has no positive contribution either to the world or to himself but who is lost in prohibited actions.}

\emph{The work of Sri Nithin Sridhar is a good refresher to a general reader who wants to know the relative importance of various paths and traditions mentioned in the Upanishads.}
\medskip

\begin{flushright}
\textbf{---Dr.\ Aravinda Rao K,}\\
\textbf{Former DGP, Member of Council of Indian Council of Philosophical Research and a teacher of \emph{Vedānta},}\\
\textbf{Hyderabad, India.}
\end{flushright}
\medskip

\emph{I have known Shree Nithin Sridhar for a long time as Chief Curator of Advaita Academy, Editor of India Facts, Consulting Editor of India Today, and also an active participant in the Facebook, communicating about Hinduism to a larger number of audiences. I have met him personally also during the recent Oneness Conference organized by Indic Academy and Chinmaya University in Kerala.}

\emph{Nithin has written many articles on Hinduism, and this commentary on Isha or Ishaavaasya Upanishad happened to be his first book on Vedanta. It so happens that of the ten or eleven Upanishads that Adi Shankara has written bhaashyaas, Isha is generally taken as the first Upanishad for spiritual study. The very first sloka of this Upanishad starts with Ishaavaasyam indicates the essential truth that the whole universe is pervaded by the Lord, and to see or recognize Him, one has to open his wisdom-eye. The intense study of this Upanishad helps to accomplish that. This commentary is written in a lucid style that an earnest seeker can easily understand and visualize the truth expounded by the Upanishad.}

\emph{Hearty congratulations to Nithin and wish him all the best to author many more books on Vedanta.}
\medskip

\begin{flushright}
\textbf{---\emph{Ācārya} Sadananda,}\\
\textbf{Material Scientist and a \emph{Vedāntin},}\\
\textbf{Chennai, India}
\end{flushright}
\medskip

\emph{Nithin Sridhar has produced one of the most indepth and detailed studies of this key Upanishad. Notably he has placed it in a traditional context in the greater field of Vedic knowledge and culture, which is seldom properly examined. Provides a new and insightful view of the Vedic basis of the Upanishadic mind and its approach to truth and reality.}
\medskip

\begin{flushright}
\textbf{---Dr.\ David Frawley (\emph{Paṇḍita} Vamadeva Shastri),}\\
\textbf{\emph{Vedācārya}, Author of books on \emph{Yoga}, \emph{Āyurveda} and \emph{Vedānta},}\\
\textbf{United States.}
\end{flushright}
\medskip


\emph{On the face of this earth and in the history of mankind, there are no works that span the grandeur and plumb the depths of Consciousness, both inner and universal, as do the Upanishads. Among them, the Ishavasya Upanishad is one of the smallest but it is also one of the most profound and all-inclusive, giving it the exalted status of being one of the first Upanishads taken up for study, in the Upanishadic range of spiritual treasures.}

\emph{Sri Nithin Sridhar is a young, brilliant and devout seeker whose passion for spiritual reading and writing is already evident from his earlier works.}

\emph{In this commentary too, he is thorough and painstaking in his efforts to delve into the deeper import of each verse. He also embellishes his book with an added orientation towards a subliminal yet effective motivation to the readers to apply the lofty truths of this Upanishad in their life.}

\emph{His committed effort is truly laudable, considering his young age. In a world of so many responsibilities and preoccupations, that Sri Nithin Sridhar is making such a wholehearted effort to comment on an Upanishad reassures us that the future of Sanatana Dharma is in safe hands.}

\emph{May he be blessed by this sacred 'jnana yajna' of the dissemination of the Vedantic truths that lead to the realisation of the purpose of the human birth. I wish him the very best for all his future endeavors too.}
\medskip

\begin{flushright}
\textbf{---\emph{Vidvān} V Subramanian,}\\
\textbf{\emph{Vedānta} \emph{Ācārya},}\\
\textbf{Bengaluru, India}
\end{flushright}

