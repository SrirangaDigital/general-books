\chapter{Verse 9}

\begin{moolashloka}
\dev{अन्धं तमः प्रविशन्ति ये अविद्यामुपासते् ।}\\
\dev{ततो भूय इव ते तमो य उ विद्यायां रताः ॥ 9 ॥}
\end{moolashloka}


\textbf{Word to word Meaning}: Into the blinding (\dev{अन्धं}) darkness (\dev{तमः}) enter (\dev{प्रविशन्ति}) they who (\dev{ये}) worship (\dev{उपासते}) ignorance (\dev{अविद्याम्}); into even greater (\dev{भूयः}) darkness (\dev{तमः}) as it were (\dev{इव}) than that (\dev{ततो}) they (enter) (\dev{ते}), those who (\dev{य उ}) delight/indulge (\dev{रताः}) in knowledge (\dev{विद्यायां}).

\textbf{Meaning:} Those who worship \emph{ignorance} (i.e. \emph{Karmānuṣṭhāna}, practice of rituals, actions and duties alone), they enter into blinding darkness (of ignorance and karmic bondage); and those who worship \emph{knowledge} (i.e. meditation on the deities, \emph{saguṇopāsanā} alone), they enter, as it were into a greater darkness (of bondage and limitation).

\textbf{Analysis:} After discussing about the \emph{jñānī}, the state of \emph{jñāna}, and the nature of \emph{Brahman}, the \emph{Upaniṣad} now addresses the \emph{pravṛttimārgī-s}, the householders, those people who still have attachment to sensory objects and desires but wish to make spiritual progress.

\dev{अन्धं तमः प्रविशन्ति ये अविद्यामुपासते:} \emph{Into the blinding darkness they enter, who worship Ignorance.} \dev{अविद्या} literally means `\emph{ignorance'}. Ignorance here refers to the whole universe, which is a result of \emph{ajñāna}. The whole cosmos is an apparent manifestation, manifested by \emph{Brahman} using his \emph{Māyā}-\emph{śakti}. In other words, the world of names and forms exist as long as \emph{avidyā} exists. That is, the whole manifestation is \emph{avidyā}. It is nothing but the Karmic bondage, the Karmic cycle of birth and death in which a \emph{jīva} is trapped. So, the worship of \emph{avidyā} by a person who is bound by \emph{avidyā} means `\emph{practicing karma-s}' to achieve worldly desires and goals in general and the practice of ``\emph{vihita karmānuṣṭhāna}'' as prescribed in the scriptures to attain the results mentioned in them in particular.

\emph{Śāstra-s} divide the \emph{Karma-s} into those that are to be practiced by the individual called `\emph{vihita karma-s}' and those that are prohibited for an individual called `\emph{niṣiddha karma-s}'. A householder is enjoined/prescribed by the \emph{Vedas} to practice \emph{vihita karmānuṣṭhāna} and \emph{vidyā} upāsanā / \emph{devatopāsana} (\emph{apara-bhakti} / \emph{saguṇopāsanā}) to achieve attain \emph{pitṛ loka} (the realm of ancestors) and the \emph{deva loka} (realm of the gods) respectively and to produce an offspring to fulfill the duty towards the \emph{manuṣya loka} (physical universe). The attainment of these three worlds- \emph{manuṣya}, \emph{pitṛ} and \emph{devaloka} are the duties prescribed for a householder (1). Hence, a person who does not perform both \emph{vihita karma} and \emph{devatopāsana} together and instead chooses to perform only one of them will incur `\emph{pāpam} /sin' and it will increase his bondage to \emph{saṁsāra}, in this case the bondage to physical existence (\emph{sthūla śarīra}).

A person who practices the \emph{karmānuṣṭhāna}, the practice of rituals and rites prescribed by the scriptures in a \emph{niṣkāma},/detached way and \emph{devatopāsana} / \emph{apara-bhakti} with \emph{samarpaṇa bhāva} (an attitude of surrendering) and \emph{ekāgra citta} (One pointed concentration), he will attain \emph{citta śuddhi} /purification of mind and this in-turn would lead him to \emph{ātmajñāna} and \emph{mokṣa}. The \emph{Upaniṣad} had also stressed this in the beginning itself, that for a householder who wishes to live in this world, he has no other way than practicing \emph{vihita karmānuṣṭhāna} prescribed in scriptures in a detached manner by which the \emph{Karma-s} wont bind him. (2).

On the other hand, a householder who is yet to achieve \emph{niṣkāma daśā} /detachment, will perform the \emph{Karma-s} in a \emph{sakāma} way i.e. he will perform permissible actions not with a sense of duty, but instead with a desire to attain the results mentioned in the \emph{śāstra-s}. If such a person performs rites and rituals to attain \emph{pitṛ loka} (realm of manes) alone to the exclusion of \emph{devatopāsana}, owing to this non-practice of \emph{devatā} (\emph{vidyā}) \emph{upāsanā} he would not develop the attitude of surrendering and one-pointed concentration. And further, because of the neglect of his \emph{devatopāsana}, which is a duty prescribed by the \emph{śāstra-s} for the householder; such a person will increase his bondage to \emph{saṁsāra}.

Hence, such a performance of \emph{Karma-s} alone without \emph{devatopāsana}, will not lead one to a state of detachment, surrendering and purification of mind. Instead, it will lead a person to pursue one desire after another and hence, increase his attachment to \emph{saṁsāra}. That person would become trapped in the karmic cycle of birth and death, as there is no end to desires and he will be running behind them, one after the other continuously.

Therefore, the \emph{Upaniṣad} is saying \emph{karmānuṣṭhāna} in a \emph{sakāma} way, i.e. practice of \emph{Karma-s} with a desire to achieve the said results, when practiced alone, would make the person fall into blinding darkness. \dev{अन्धं तमः} means `\emph{blinding darkness'}. Here, `darkness' refers to the darkness of gross- \emph{saṁsāra}, the bondage to physical universe and its objects. It refers to person being more deeply bound to his \emph{sthūla śarīra}, the gross existence. It is called `blinding' because, a person who is indulging in the pleasures of \emph{saṁsāra}, is deluded by it and hence he cannot perceive his true nature. A person bound by the temptations of the sensory objects becomes blind towards \emph{Brahman}, his own innermost Self. He will become convinced that the external world (physical universe perceived in waking state) of names and forms, the multiplicities are the ultimate truth and by this he will become entangled in the karmic cycle till eternity. After death, such a person will attain \emph{pitṛ loka}, where he will stay till his karmic fruits mature and then he would return to \emph{manuṣya loka} (physical universe) to experience his karmic fruits. Hence, such a person has no freedom from the cycle of birth and death, but instead the practice of \emph{Karma-s} alone in a \emph{sakāma} way will lead him deeper and deeper into bondage to physical \emph{saṁsāra}.

\dev{ततो भूय इव ते तमो य उ विद्यायां रताः}- \emph{Into a greater darkness, as it were, they enter, those who worship knowledge.} Now, the \emph{Upaniṣad} is saying that, those who worship \emph{vidyā} alone, they enter as if into a greater darkness. \emph{Vidyā} \emph{upāsanā} is nothing but \emph{devatopāsana}, the worship (meditation) of deities. It is the practice of \emph{apara-bhakti} (3), or \emph{dvaita bhakti} wherein there is a devotee and a deity, the worshipper and the worshipped.

The \emph{deva-s} or the gods are the powers which preside over various aspects of the universe (\emph{vyāvahārika daśā}). They represent various aspects of \emph{Saguṇa-Brahman} (i.e. \emph{Brahman} with the universal movement). For example, the \emph{Apas} represent all cosmic activities and \emph{Vāyu} represents the cosmic life force that governs all activities, The \emph{Prajāpati-s} represents the eternal time principles that are responsible for creation of various objects. At an individual level various \emph{devas} represent various \emph{jñānendriya-s} (4), \emph{karmendriya-s} (5) and \emph{antaḥkaraṇa} (6). In the Puranic literatures, the \emph{trimūrti} - \emph{Brahmā}, \emph{Viṣṇu} and \emph{Śiva} represent the three aspects of creation, sustainment and destruction of universe. In the Tantric literatures, the sixty-four \emph{Yoginī-s} represents sixty-four aspects of the \emph{Saguṇa Brahman} / \emph{Śakti}''

When a person does \emph{upāsanā} of a particular deity and attains them, he attains \emph{knowledge} about that aspect of \emph{Brahman}. And it is for this reason, that \emph{devatopāsana} is called \emph{vidyā upāsanā} by the \emph{Upaniṣad}, because, a worship of \emph{devas} will lead one to attain knowledge about that Deity, i.e. knowledge about that aspect of \emph{Brahman}. \emph{Śāstra-s} clearly state that one who wishes to attain \emph{devaloka}, he must practice \emph{vidyā - upāsanā}. It is hence, a duty of a householder to practice \emph{devatopāsana} along with \emph{vihita karmānuṣṭhāna} in order to attain both \emph{pitṛ} and \emph{deva loka-s} as enjoined by the \emph{śāstra-s}. But, when a householder practice worship of the deities alone neglecting his \emph{karmānuṣṭhāna}, then such a person would be committing a ``\emph{pāpam} /sin'' due to neglect of his duty (to try to achieve all worlds) and hence, it will increase his bondage to \emph{saṁsāra}.

Further, because a person practices \emph{apara-bhakti} with an intention/ desire to attain \emph{devaloka}, etc. and not with an attitude of surrendering, such a person will not develop \emph{citta śuddhi}. Hence, such a person because of his non-performance of rituals and other \emph{Karma-s} as prescribed in the \emph{śāstra-s} and at the same time performance of the \emph{devatopāsana} in a \emph{sakāma} way, will increase his bondage towards \emph{saṁsāra} and hence the \emph{Upaniṣad} says ``he enters, as it were, into a greater darkness''. Here the words \dev{भूय इव ते तमो} is very significant. It means ``as it were into greater darkness''. The use of \dev{इव}- as it were, signify that, \emph{devatopāsana} will not lead to the kind of karmic bondage, that exclusive performance of will lead to. Exclusive practice of \emph{devatopāsana} will lead to bondage caused by the attachment to limited knowledge (of the universe), whereas exclusive practice \emph{karmānuṣṭhāna} will lead to bondage caused by the attachment to ignorance (manifested as sensory objects).

A person who, having neglected his \emph{Karma-s}, practices only \emph{devato\-pā\-sana} with an intention to gain access to higher realms of existence and hence attain the knowledge of those realms; such a person owing to the impurity of his mind (7) will mistake his neglect of \emph{Karma-s} as renouncing of them, and becomes attached to his accomplishments i.e. his attainment of knowledge of deities or higher \emph{lokas}. In the words of Sri Aurobindo, such a person being attached to the higher state (subtle existence, \emph{sūkṣma śarīra}), will mistake exclusion of lower-realms as transcending them and hence fall into a greater darkness of ignorance. Such people ignore by the choice of knowledge, whereas people practicing \emph{Karma} alone ignore by compulsion of error (8). That is, such a person owing to his attainment of knowledge of certain aspects of this cosmos becomes attached to his knowledge. He will consider his limited Knowledge to be the ultimate knowledge (\emph{Brahmajñāna}) and his conception of god to be the ``only true God''. And hence, this would lead him to a greater bondage of \emph{saṁsāra}. Such a person may attain \emph{deva-loka} after death, but this would not be a permanent abode for him. He has to return to \emph{manuṣya loka} owing to his neglect of the performance of \emph{vihita karma-s} and hence not having overcome the limitations of physical \emph{saṁsāra}.

\textbf{Summary:} After addressing the \emph{nivṛttimārgī-s} and explaining to them about the nature of \emph{Brahman}, the \emph{jñānī} and the state of \emph{jñāna}, the \emph{Upaniṣad} now addresses the \emph{pravṛttimārgī-s} i.e. the householders. A \emph{pravṛttimārgī} is basically a person who is still attached to sensory objects. He has unfulfilled desires towards worldly objects and obligatory duties towards the society and the world. The \emph{Bṛhadāraṇyakopaniṣad} speaks about three kinds of desires a householder has- desire for an offspring, desire for wealth and desire for worlds (9) and three duties that he must fulfill- study the scriptures, perform the rites and rituals and attain all the worlds (10). The \emph{Upaniṣad} describes that the worlds that are to be attained are three in number- world of gods, of manes and of men (11).

The world of Men (\emph{manuṣya loka}) is attained by having an offspring and by no other means; the world of manes (of the ancestors- \emph{pitṛ loka}) is attained by the practice of rites i.e. \emph{vihita karmānuṣṭhāna} and by no other means; and the world of gods (\emph{deva loka}) is attained by the practice of \emph{apara-bhakti} i.e \emph{vidyā upāsanā} and by no other means (12). Hence, when a householder practices exclusively either \emph{bhakti} or \emph{karma}, he will be able to attain either the \emph{devaloka} or the \emph{pitṛ loka}, respectively. In both the cases, he would attain only one of the \emph{Loka-s} and hence thereby not fulfilling the duties that are enjoined on him by the \emph{śāstra-s}. This non-performance of the one or the other duties will increase the Karmic bondage tying the person more strongly to the \emph{saṁsāra}. Further, \emph{citta-śuddhi} /purification of mind happens in a person, only when he performs \emph{karma} and \emph{bhakti} together and develop \emph{niṣkāma daśā}, \emph{samarpaṇa bhāva} and \emph{ekāgra citta}. Hence, a person who performs only \emph{avidyā upāsanā} or only \emph{vidyā upāsanā} in a \emph{sakāma} way, with a desire to achieve the worldly goals, he would not achieve \emph{citta-śuddhi} and hence would not be able to attain \emph{jīvanmukti}. It is for this reason, the \emph{Upaniṣad} says, that those who practice \emph{avidyā upāsanā} or \emph{vidyā upāsanā} alone will enter blinding darkness. Because, practice of one by excluding the other, will not ``free'' a person, but instead ``binds'' him more strongly to the \emph{saṁsāra} (physical existence) (13).

\section*{References}

\begin{enumerate}
\itemsep=1.2pt
\item
  \emph{Bṛhadāraṇyakopaniṣad} 1.5.17.
\item
  \emph{Īśopaniṣad} Verse No- 2
\item
  There are two stages of \emph{bhakti}- \emph{apara} /lower stage and \emph{parā} -higher stage. \emph{Apara-bhakti} involves worshipping God as separate from One-self. There is a devotee and a deity. \emph{Parā} \emph{bhakti} involves contemplation of God as one's own \emph{Ātma}.
\item
  ``\emph{Dik-devatā}'', \emph{Vāyu}, \emph{Sūrya}, \emph{Varuṇa} and \emph{Aśvinīkumāra-s} are the deities of Faculty of hearing, touch, sight, taste and smell respectively. \emph{Tattvabodha} Verse 11.4
\item
  \emph{Agni}, \emph{Indra}, \emph{Viṣṇu}, \emph{Yama} and \emph{Prajāpati} are deities of Faculty of speech, grasping (hands), movement (legs), excretion (anus) and procreation (genitals) respectively. \emph{Tattvabodha} Verse11.7
\item
  \emph{Antaḥkaraṇa} includes \emph{manas}, \emph{buddhi}, \emph{ahaṁkāra} and \emph{citta.} \emph{Candra}, \emph{Brahmā}, \emph{Rudra} and \emph{Vāsudeva} are respectively their deities. \emph{Tattvabodha} Verse 21.6
\item
  Impurities of mind are called ``\emph{ariṣaḍvarga}''-the six passions or ``\emph{ṣaḍ ripu-s}''- the six enemies. They are \emph{kāma} (Lust), \emph{krodha} (Anger), \emph{lobha} (Greed), \emph{moha} (Delusion/Attachment), \emph{mada} (Pride) and \emph{mātsarya} (Jealousy).
\item
  Sri Aurobindo's exact words are- ``They enter into some special state and accept it for the whole, mistaking exclusion in consciousness for transcendence in consciousness. They ignore by choice of Knowledge, as others are ignorant by compulsion of error.'' Page 53, Volume 17- \emph{Īśopaniṣad}, The Complete Works of Sri Aurobindo.
\item
  \emph{Bṛhadāraṇyakopaniṣad} 4.4.22.
\item
  \emph{Bṛhadāraṇyakopaniṣad} 1.5.17.
\item
  \emph{Bṛhadāraṇyakopaniṣad} 1.5.16.
\item
  \emph{Bṛhadāraṇyakopaniṣad} 1.5.16.
\item
  \emph{Saṁsāra} as a whole includes the gross, subtle and causal realms. A \emph{jīva} has 3 \emph{śarīra-s} /bodies- \emph{sthūla} /gross, \emph{sūkṣma}/subtle and \emph{kāraṇa}/causal and these bodies are the \emph{upādhi-s}, the limiting entities. In the waking state, one interacts with physical universe with gross body. In the dream state, one interacts with the subtle universe with subtle body and in deep sleep one is present his causal body.

  ­
\end{enumerate}
