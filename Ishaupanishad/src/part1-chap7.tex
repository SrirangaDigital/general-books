\chapter{Verse 6}

\begin{moolashloka}
\dev{यस्तु सर्वाणि भूतानि आत्मन्येवानुपश्यति ।}\\
\dev{सर्वभूतेषु चात्मानं ततो न विजुगुप्सते ॥ 6 ॥}
\end{moolashloka}

\textbf{Word to word Meaning}: He who (\dev{यस्तु}) verily (\dev{एव}) perceives (\dev{अनुपश्यति}) the (innermost) Self (\dev{आत्मनि}) in all (\dev{सर्वाणि}) objects/entities (\dev{भूतानि}) and (\dev{च}) all entities (\dev{सर्वभूतेषु}) in the (innermost) Self (\dev{आत्मानं}), due to that (\dev{ततो}) he does not (\dev{न}) have aversion towards anything (\dev{विजुगुप्सते}).

\textbf{Meaning}: (For a \emph{jñānī}) who verily perceives his (innermost) Self-\emph{Ātma} (alone) in all entities (of the cosmos) and all the entities (of the cosmos) in the \emph{Ātma}, (Such a \emph{jñānī}) by virtue of that (perception/awareness) has no aversion towards anything.

\textbf{Analysis:} After explaining the nature of \emph{Brahman}, the \emph{Upaniṣad} now speaks about the \emph{jñānī} who is a \emph{jīvanmukta}.

\dev{यस्तु सर्वाणि भूतानि आत्मन्येवानुपश्यति}- \emph{Who verily sees/perceives the Ātma in all entities.} In the previous verse, it was explained how \emph{Brahman} inhabits everything, how \emph{Brahman} alone exist. A person who has attained the state of \emph{jñāna}, i.e. a \emph{jñānī} or \emph{jīvanmukta} perceives only the \emph{pāramārthika satya} (1). People under the influence of \emph{Māyā}, perceive only duality i,e \emph{vyāvahārika satya}. But, a \emph{jñānī} having removed the \emph{ajñāna} and gone beyond the influence of \emph{Māyā} perceives \emph{Brahman} alone. \dev{अनुपश्यति} literally means `\emph{to see}'. But the vision of a \emph{jīvanmukta} would be very different from the common man. He sees not through the physical eyes but through his divine eye of \emph{jñāna}. He perceives his own innermost Self, the \emph{Ātma} as the essential truth of all the entities around him. He sees that, it is his \emph{Ātma} alone that exist underneath all names and forms, underneath the innumerable entities, and underneath all the movements in the universe. The \emph{śāstra-s} say ``A knower of \emph{Brahman}, verily becomes \emph{Brahman}'' (2). Hence, a \emph{jīvanmukta} having attained \emph{Brahman} (due to \emph{ātmajñāna}) perceives his own Self everywhere. Such a \emph{jñānī}, as long as he stays in his body may still perceive the names and forms, the multiplicities of the Universe through his senses and mind. But, his inner vision, the eye of \emph{jñāna} remains unclouded. He will see the countless forms, yet not see them; he may interact with numerous people, yet will not see them as separate from his Self. In other words, a \emph{jñānī} will be ever aware of his true identity, the \emph{pāramārthika satya} that is devoid of duality, devoid of influence of \emph{Māyā}. He will ever be aware that the whole universe, all the names and forms are not distinct from himself as \emph{Ātma}, and in fact, \emph{Ātma} alone pervades everything, the \emph{Ātma} alone exist and it is \emph{Ātma} alone which has appeared as forms and names.

\dev{सर्वभूतेषु चात्मानं}- \emph{And all entities in the Ātma.} A \emph{jñānī} will further perceive all the entities as being present inside his own Self. In other words, a \emph{jñānī} perceives no difference between his own Self and the world. He sees his \emph{Ātma} in all the entities and all the entities in his \emph{Ātma}. That is, he perceives \emph{Ātma} alone. \emph{Brahman} was described as `\emph{ekam}/the One' and a \emph{jīvanmukta} perceives this \emph{One} alone. A \emph{jīvanmukta} is not the one who simply recognizes the existence of one reality behind all names and forms, but he is the one who actually perceives this oneness. He perceives the \emph{pāramārthika satya}- reality as being non-duality. This is not to say that he does not see the multiplicities of universe with his naked eye, it only means that every such vision of multiplicities perceived by the physical eye is superseded by the vision of the oneness perceived by the eye of \emph{jñāna}. In other words, a \emph{jñānī} does not perceive various names and forms as being distinct from \emph{Brahman}, instead he perceives them as \emph{Brahman} itself.

\dev{ततो न विजुगुप्सत}- \emph{By virtue of that (perception), he has no aversion towards anything}. \dev{विजुगुप्सत} means ``shrinking away from'' or ``rejecting''. The \emph{Upaniṣad} is saying that a jñānī who perceives the whole universe in his Self and his Self in whole universe, owing to this state of \emph{jñāna}, owing to his perception of \emph{pāramārthika satya} does not show either like or dislike towards any person or a thing. A \emph{jñānī} is one who has a `\emph{sama dṛṣṭi} or same sightedness. He perceives every person as his own Self, hence he neither avoids anybody nor does he favour anybody. He neither pursues any object or pleasure nor avoids the objects or sorrows that come his way. A Jnani is simply above the duality of \emph{Sukha} and \emph{duḥkha}. A \emph{jīvanmukta} stays in the world only for the benefit of the world. He does not pursue worldly desires and goals nor does he shrink away from his duties. He maintains same-sightedness towards everyone and everything without any attraction or repulsion towards them.

\textbf{Summary:} The \emph{jñānī} and the state of \emph{jñāna} are the subjects of this and the next verse. The state of \emph{jñāna} is the state of \emph{pāramārthika}. A \emph{jñānī} perceives only \emph{pāramārthika satya}. A person in the \emph{saṁsāra}, owing to \emph{ajñāna} and \emph{avidyā} cannot perceive \emph{Brahman}. In \emph{vyāvahārika daśā}, the dual state, a person can perceive only through his mind and the senses. The \emph{manas} and the \emph{indriya-s} by their very nature travel only outwards. That is, they can only perceive external objects and not the \emph{Ātma}, the innermost Self which is beyond the duality of object and subject.

But a person who has attained \emph{ātmasākṣātkāra} has removed the \emph{ajñāna} that had bound him. Hence, such a \emph{jñānī} having overcome the \emph{vyāvahārika daśā}, perceives \emph{pāramārthika satya}-absolute non-dual truth alone. He perceives \emph{Brahman} as being both within and without the universe. He perceives his own \emph{Ātma} as being present in all the entities, all the names and forms, all the modifications (\emph{vikāra-s}). Further, he perceives the entire cosmos as being present in him. He perceives the absolute truth that \emph{Ātma} alone exists.

By the virtue of such an understanding of the universe, where he perceives nothing other than his own Self, he feels neither attraction nor revulsion towards any object or people of the universe. A \emph{jīvanmukta} may still live in the world for the benefit of the world, but he remains unaffected by it. Neither the good nor the bad, neither happiness nor the sorrow, neither the temptations nor the fear affects him. He simply lives for the sake of others. He neither pursues any objects nor shrinks away from his duty. He becomes a `\emph{sama dṛṣṭi}', a person who perceives everything and everyone as his own.

\section*{References}

\begin{enumerate}
\item
  \emph{Kenopaniṣad} 2.5
\item
  \emph{Muṇḍakopaniṣad} (3.2.9)
\end{enumerate}

