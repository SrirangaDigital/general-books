\chapter{Verse 11}

\begin{moolashloka}
\dev{विद्यां चाविद्यां च यस्तद्वेदोभयं सह् ।}\\
\dev{अविद्यया मृत्युं तीर्त्वा विद्यया अमृतमश्नुत ॥ 11 ॥}
\end{moolashloka}
\vskip 1pt

\textbf{Word to word Meaning}: Vidya and \emph{Avidyā} (\dev{विद्यां चाविद्यां च}) who (\dev{यः}) knows (\dev{वेद}) that (\dev{तत्}) both together (\dev{उभयं सह}); By \emph{Avidyā} (\dev{अविद्यया}) death (\dev{मृत्युं}) (one) transcends (\dev{तीर्त्वा}), by Vidya (\dev{विद्यया}) immortality (\dev{अमृतम्}) attains (\dev{अश्नुते}).
\vskip 1.1pt

\textbf{Meaning:} He who knows both \emph{vidyā} and \emph{avidyā} together, he transcends death (i.e. limitations imposed by physical existence) by \emph{avidyā} and attains immortality (i.e. freedom from cycle of birth and death) by \emph{vidyā}.
\vskip 1.1pt

\textbf{Analysis:} In this verse, the manner in which the practice of both \emph{sakāma karmānuṣṭhāna} and \emph{sakāma devatopāsana} together will lead a person towards \emph{kramamukti} is explained.
\vskip 1.1pt

\dev{विद्यां चाविद्यां च यस्तद्वेदोभयं सह}- \emph{He who knows vidyā} \emph{and avidyā} \emph{both together.} After explaining the futility of practicing \emph{vidyā} or \emph{Karma} alone, the \emph{Upaniṣad} is speaking about the right path to be adopted by the householders. A \emph{pravṛttimārgī} who still has desires for worldly objects, but who also wishes to attain \emph{mokṣa} must practice both \emph{devatopāsana} and \emph{karmānuṣṭhāna} together. Those who practice \emph{Karma} alone will achieve knowledge of the physical \emph{saṁsāra} and its multiplicities and will attain \emph{pitṛ loka} through the \emph{southern} route. But, after their karmic fruits in \emph{pitṛ loka} is exhausted, they will return to \emph{bhū loka} (physical realm) and take birth to face their Karmic fruits. This cycle of birth and death keeps going on. Similarly, those who practice \emph{devatopāsana} alone, they would gain access to higher realms and attain knowledge of different deities. They would travel the \emph{northern} path and attain \emph{devaloka}, but even their stay in \emph{devaloka}, though long, would be temporary and they would eventually return to \emph{bhū loka} to fulfill their duties and face their Karmic fruits. He who knowing this, performs both \emph{karma} and \emph{bhakti}, both \emph{avidyā} and \emph{vidyā upāsanā} together, only he would be able to attain spiritual upliftment.

\dev{अविद्यया मृत्युं तीर्त्वा}- \emph{By avidyā one transcends death}. Now, the \emph{Upaniṣad} starts explaining how, a person who practices both \emph{vihita karmānuṣṭhāna} and \emph{devatopāsana} in a \emph{sakāma} way will be able to attain liberation. \dev{मृत्युं} literally means `\emph{death'}. Here, by death, the \emph{Upaniṣad} is referring to all the limitations imposed by the gross \emph{saṁsāra} (the physical universe) on an individual. `Death' refers to an individual being trapped in the cycle of birth and death, a \emph{jīva} who is trapped in the limitations imposed by his \emph{sthūla śarīra} (gross body). Ādi Śaṅkarācārya defines \dev{मृत्युं} as knowledge and action induced by \emph{svabhāva} (one's nature) (1). That is, `death' in this context denotes the limitations (in the form of \emph{Karma} and sensory knowledge) imposed by the \emph{avidyā} on a \emph{jīva} who is bound to physical existence. The \emph{Upaniṣad} is saying that, a person who practices \emph{vidyā} and \emph{avidyā upāsanā} together, he will overcome the limitations imposed by the physical universe (in the form of gross body) through the practice of \emph{karmānuṣṭhāna}. In other words, the individual will overcome his gross/physical existence (2).

A person, who practices \emph{vihita karma-s} and \emph{apara-bhakti} together in a \emph{sakāma} way, i.e. with a desire to attain \emph{pitṛ loka} and \emph{deva loka}, such a person will burn away his \emph{saṁskāra-s, vāsanā-s} and \emph{prārabdha karma-s} that are binding him to physical realm (\emph{bhū loka}), further he would burn away the Karmic \emph{ṛṇa-s} (debts) that he owe to his ancestors (\emph{pitṛ-s}). Hence, such a practice of \emph{karmānuṣṭhāna} when done along with \emph{devatopāsana}, the \emph{karmānuṣṭhāna} will lead a person to overcome the limitations placed by the gross- \emph{saṁsāra}. It would burn the \emph{Karma-s} that are binding him to physical existence and help him to overcome the limitations induced by the \emph{bhū loka}. Hence, the \emph{Upaniṣad} is saying that by \emph{karmānuṣṭhāna} one will overcome the limitations like birth, death, physical disability, ill health, etc. that are associated with gross- \emph{saṁsāra}.

\dev{विद्यया अमृतमश्नुते}- \emph{By vidyā, one will attain Immortality.} The \emph{Upaniṣad} is saying that a person who performs, \emph{vihita karma} and \emph{apara-bhakti} together, will overcome physical limitations by \emph{Karma} and will attain immortality by \emph{bhakti}. Here the word \dev{अमृतम्} is significant. The word immortality here refers \emph{not} to `\emph{absolute immortality}' but \emph{only} to `\emph{relative immortality}'.

Immortality or \emph{mokṣa} means freedom from \emph{saṁsāra} /cycle of birth and death. One can achieve complete freedom from the entire \emph{saṁsāra} only by becoming one with \emph{Brahman}, because it is \emph{Brahman} alone who is \emph{real} and \emph{eternal} with the world being merely an impermanent appearance. And this is possible only through the attainment of \emph{ātmajñāna} (i.e. realization of Self as \emph{Brahman}). \emph{Muṇḍakopaniṣad} says ``A knower of \emph{Brahman} verily becomes \emph{Brahman}'' (3). \emph{Śvetāśvataropaniṣad} says ``Only those who know Him, transcend the limitations placed by \emph{saṁsāra}, there is no other way'' (4). The whole world of names and forms are nothing but apparent manifestations that are caused by \emph{avidyā} and \emph{ajñāna}. Hence, by removal of ignorance, a \emph{jīva} attains the \emph{jñāna} that he is not separate than \emph{Brahman}. The separation was only present due to \emph{avidyā} and hence in the absence of \emph{avidyā}, there is no more any duality. Hence, \emph{mukti} or \emph{mokṣa} is attainment of `\emph{Brahmajñāna}' also called `\emph{ātmajñāna}'. This ``\emph{ātmajñāna}'' will dawn only on a person who having achieved the required competencies called `\emph{sādhanā catuṣṭaya}' like \emph{viveka}, etc. (5) will approach a Guru and under his guidance practice \emph{śravaṇa catuṣṭaya} (6) also called `\emph{jñāna sādhanā}' or `\emph{parā bhakti}' by contemplating on his own innermost \emph{Ātma} as \emph{Brahman}. To attain the competencies required for \emph{jñāna sādhanā}, one must first develop \emph{citta śuddhi} (purification of mind) and \emph{ekāgra citta} (one pointed concentration of mind) by the practice of \emph{karma} and \emph{apara-bhakti} in a \emph{niṣkāma} way with \emph{samarpaṇa bhava}. Only such a person will be able to attain \emph{ātmajñāna} and hence \emph{absolute immortality}.

On the other hand, one who worship \emph{vidyā} and \emph{avidyā} together in a \emph{sakāma} way, will overcome the limitations of his \emph{sthūla śarīra} (gross body) by \emph{Karma} and will attain the realms of the deities, the \emph{deva loka} in his \emph{sūkṣma śarīra} (subtle body) as a result of \emph{vidyā upāsanā} by travelling through the \emph{northern path of no-return}. Once a person attains \emph{devaloka}, he would be devoid of both the limitations of gross and subtle existence. A person, who had practiced \emph{devatopāsana} alone attains \emph{devaloka} only temporarily. But, one who practices both \emph{Karma} and \emph{devatopāsana} together, he would by overcoming the limitations and karmic bondages of gross- \emph{saṁsāra} and subtle- \emph{saṁsāra} will attain the \emph{devaloka-s} up to even \emph{Brahma Loka} permanently without having to return to physical existence. There in the \emph{devaloka}, he will remain until the end of the life of \emph{Brahmā} (7). This is the \emph{path of gradual liberation} called `\emph{Kramamukti}'.

This attainment of \emph{devaloka} is called here as \dev{अमृतम्}- immortality, because one who attains the \emph{devaloka} permanently, does not return to the cycle of birth and death (i.e. to physical existence). But this is the `relative immortality' since a person is still in \emph{vyāvahārika daśā} and yet to realize his Self (\emph{ātmajñāna}). \emph{Brahmasūtras} state that, on account of \emph{Saguṇa Brahman's} (i.e. \emph{Brahmaloka}) nearness to Supreme \emph{Brahman} (\emph{pāramārthika} \emph{daśā}), it is also designated as (absolute) \emph{Brahman} (8). Similarly, the \emph{Upaniṣad} calls the permanent attainment of \emph{devaloka} (relative immortality) as \dev{अमृतम्} on account of its nearness to \emph{mokṣa} (absolute immortality). The \emph{kramamukti} has four-stages- \emph{sālokya, sāmipya, sārūpya} and \emph{sāyujya}. \emph{Sālokya} refers to living in the realm of the deity. \emph{Sāmīpya} refers to living close to the deity. \emph{Sārūpya} refers to attaining a form identical to the deity and finally \emph{sāyujya} refers to totally merging into the deity. A person, who has attained \emph{devaloka} in his \emph{sūkṣma śarīra} has attained the stage of \emph{sālokya}. He will further progress to \emph{sāmīpya} and \emph{sārūpya}. And finally, he may either completely merge with \emph{Brahman} (\emph{sāyujya} with \emph{Brahman}) by attaining \emph{ātmajñāna} by the practice of \emph{parā-bhakti}, or he may attain \emph{sāyujya} with his beloved deity by merging with the form of the deity. In the second case, he would merge with \emph{Brahman} only at the end of the life of \emph{Brahmā} (9), when \emph{Brahmā} and all other deities will attain \emph{jñāna} and hence merge back into \emph{Brahman}. This is the path of \emph{kramamukti} (10) or the gradual path of liberation.

Hence, a householder who is still attached to worldly objects will overcome his physical existence through the performance of \emph{vihita karma-s} and attain \emph{devaloka} through worship of the \emph{devas}.

\textbf{Summary:} After explaining in the previous two verses that a person who worships \emph{vidyā} or \emph{avidyā} alone will increase his bondage to physical existence as the results given by each is distinct, the \emph{Upaniṣad} now explains the results obtained by those who practice the \emph{sakāma} \emph{upāsanā} of \emph{vidyā} and \emph{avidyā} together.

Those people who perform both the \emph{karma} and \emph{bhakti} in a \emph{sakāma} way with a desire to attain all the worlds, they would fulfill their debts to the \emph{pitṛ-s} and overcome the limitations of the physical existence through \emph{karma} and would travel through the northern path (\emph{devayāna}) and attain the realms of the deities by \emph{devatopāsana}. They may even attain the \emph{Brahma-Loka}. There, they will stay till the end of \emph{Brahmā} (\emph{Hiraṇyagarbha}). When, at the end the whole cosmos everything including \emph{Brahmā} merges back in to \emph{Brahman}, these people will attain \emph{brahmajñāna} and merge with \emph{Brahman} (12). This is the path of \emph{kramamukti} and attainment of the \emph{devaloka} or \emph{Brahma Loka} without having to return to physical existence is called `relative immortality'.

Thus, the \emph{Upaniṣad} explains the path of \emph{kramamukti} for the \emph{pravṛtti\-mārgī-s} who are still attached to worldly desires and wish to remain in the world.

\section*{References}

\begin{enumerate}
\itemsep=0pt
\item
  \emph{Śaṅkara Bhāṣya} on \emph{Īśopaniṣad}.
\item
  Every person has three bodies gross, subtle and casual through which he interacts with physical, subtle and causal realms.
\item
  \emph{Muṇḍakopaniṣad} 3.2.9
\item
  \emph{Śvetāśvataropaniṣad} 3.8
\item
  \emph{Sādhanā Catuṣṭaya} are- \emph{viveka}, \emph{vairāgya}, \emph{ṣaṭ sampatti} and \emph{mumu\-kṣutva}. Refer to explanations in first verse.
\item
  \emph{Śravaṇa Catuṣṭaya} includes \emph{śravaṇa}, \emph{manana}, \emph{nididhyāsana}, \emph{ātma\-sākṣāt\-kāra}. Refer to explanations in first verse.
\item
  A day of \emph{Brahmā} consists of 14 \emph{manvantara-s} and is equal to 4.32 billion human years. A night of \emph{Brahmā} is equally long and \emph{Brahmā} lives 100 years.
\item
  \emph{Brahma-Sutras} 4.3.9
\item
  \emph{Brahmā} lives 100 \emph{Brahmā} Years. One day of \emph{Brahmā} is equal of 4.32 billion human years called \emph{Kalpa}. And one night of \emph{Brahmā} is also equally long.
\item
  It is important to understand that attaining \emph{mokṣa} through ``\emph{kramamukti}'' does not mean that one attains \emph{mokṣa} without \emph{ātmajñāna}. \emph{Mokṣa} is not possible without \emph{ātmajñāna}, because \emph{ātmajñāna} is \emph{mokṣa}. The bondage and limitations in \emph{saṁsāra} are due to \emph{avidyā} and hence, \emph{jñāna} alone can remove them. The only difference between \emph{jīvanmukti} and \emph{kramamukti} is that, a person who by practice of \emph{jñāna} \emph{sādhanā} attains \emph{jñāna} and hence \emph{mukti} even while alive in physical body is called `\emph{jīvanmukta}'. On the other hand, people who still have desires left and are yet to attain \emph{citta śuddhi}, travel the longer or gradual path in which they travel stage by stage from gross-existence to subtle-existence to \emph{Hiraṇyagarbha} and finally at the end of cycle attain \emph{ātmajñāna} and become liberated.
\item
  \emph{Brahmasūtra} 4.3.10
\end{enumerate}

\textbf{Alternate explanation for the verses 9-11}

In the above explanations, we saw how \emph{avidyā} refers to \emph{karmānuṣṭhāna} and vidyā to \emph{devatopāsana}. But, if one were to consider these words in their primary meaning, then \emph{avidyā} will mean `ignorance about true nature of \emph{Brahman}' and \emph{vidyā} will mean `knowledge about the true nature of \emph{Brahman}'. In other words, \emph{avidyā} refers to \emph{bandhana} (bondage) and \emph{vidyā} refers to \emph{mokṣa} (Liberation).

From this it follows that \emph{avidyā} includes both \emph{karmānuṣṭhāna} and \emph{devatopāsana}. Further, \emph{avidyā} is the root-cause of whole \emph{saṁsāra}, whether it is manifested cosmos, or unmanifested \emph{prakṛti}. Hence, one who keeps practicing \emph{karmānuṣṭhāna} and \emph{devatopāsana} be it exclusively or simultaneously, but without developing dispassion (\emph{vairāgya}), then such a person will remain in darkness of ignorance.

On the other hand, a person who is not dispassionate and is attached to his body and yet chooses to abandon his Karmic duties and instead involves in \emph{jñāna sādhanā} (or \emph{vicāra}) will fall into a greater darkness owing to his incompetence for \emph{jñāna sādhanā} and his abandonment of \emph{karma} and \emph{bhakti}. Hence, a practice of either \emph{avidyā} alone or \emph{jñāna sādhanā} alone will not bear any fruit.

But, when both are combined in succession, one overcomes the limitations of \emph{desire} (which itself is death) by \emph{karma} and \emph{bhakti}, thus gaining competence to practice \emph{jñāna sādhanā} (\emph{vidyā}). Then, by practice of \emph{jñāna sādhanā}, one attains \emph{mokṣa} or knowledge of \emph{Brahman}. In this case, it will be absolute immortality.

This explanation of the verses is not contradictory to the previously given interpretation. It is just another layer of understanding. Much of the Hindu scriptures, especially the \emph{Upaniṣad-s} have multiple layers of meaning and each of them is equally valid. It is important to have an integrated understanding of all such different layers to arrive at a full picture.


