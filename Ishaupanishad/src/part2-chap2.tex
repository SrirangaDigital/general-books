\chapter[Section 2 \emph{Nivṛttimārga}: The Path to \emph{Jīvanmukti}]{Section 2\\ \emph{Nivṛttimārga}: The Path to \emph{Jīvanmukti}}

\section*{2.1. Introduction}

``\emph{Nivṛtti}'' literally means `\emph{to turn away}'. It refers to people who have become devoid of any worldly desires. Those people who have \emph{turned away} from the material and sensory attachments and have only a desire for \emph{mokṣa}/liberation are called `\emph{nivṛttimārgī-s}'. They are also called `\emph{sannyāsin-s}' or renunciates, because having overcome their desires, they have `given up' their attachments and attractions to sensory objects and pleasures.

\emph{Mokṣa} or final liberation refers to cessation of the \emph{karmic} cycle of birth and death. This is only possible when one overcomes the limitations imposed on him/her by the \emph{saṁsāra}. `\emph{Karma}' or `action' is the very foundation of our existence. A second does not go by without anybody performing any action. People are continuously involved in performing one or the other tasks --eating, drinking, talking, driving etc.\ But action does not refer to physical activity alone. Every word that we speak and every thought that arise in the mind also represents actions being performed. If, a person does something good, he would ultimately get a favorable \emph{karmic} fruit and if a person intends to harm someone, he would ultimately end up suffering for that. \emph{Brahmavaivarta Purāṇa} (1) says thus-

\begin{verse}
\emph{avaśyameva bhoktavyaṁ kṛtaṁ karma śubhāśubham \dev{।}}\\
\emph{nābhuṅkte kṣīyate karma kalpakoṭiśatairapi \dev{॥}}
\end{verse}
\newpage

\emph{A person will definitely enjoy the fruits of his action; it may be good or bad; for without giving the results, an action does not die out even after millions of years.}
\vskip 2pt

Hence, a person is forever trapped in this action-reaction cycle of \emph{karma} and as long as he is trapped here, he would experience both \emph{sukha}/happiness and \emph{duḥkha}/sorrow. It is for this reason that the scriptures speak about \emph{mokṣa}. \emph{Mokṣa} is a state of ever-lasting happiness without sorrow. It is a state of knowledge and bliss. It has been variously described as `realizing \emph{Brahman}' or `merging with \emph{Īśvara}', etc.\ How can an ordinary individual attain this state of everlasting happiness? The scriptures say that, it is through the knowledge of the Self/\emph{ātmajñāna} also called `\emph{Brahmajñāna}' (i.e.\ knowledge of true nature of \emph{Brahman}) alone that one attains \emph{Brahman} (i.e.\ \emph{mokṣa}). \emph{Śvetāśvataropaniṣad} says~(2)-

\begin{verse}\emph{tameva viditvā atimṛtyumeti nānyaḥ panthā vidyate'yanāya~\dev{॥}}\end{verse}

\emph{``Only those who know Him (i.e.\ Brahman/Ātma) will surpass death (i.e.\ the karmic cycle of birth and death). There is no other way than this.}
\vskip 2pt

The \emph{Upaniṣad} says that, only those who have \emph{Brahmajñāna} would attain \emph{Mokṣa} because, the whole universe and all its limitations are due to one's ignorance / \emph{avidyā} about the true nature of \emph{Brahman} and its non-difference from the innermost Self. Hence, the limitations are removed by acquiring this knowledge of the true Self (i.e.\ \emph{ātmajñāna}). But to attain this knowledge, one must first purify one's mind (i.e.\ \emph{cittaśuddhi}) by getting it rid of impurities like desire, pride, jealousy etc.\ and develop the qualities like discrimination, dispassion, control of mind and senses, forbearance, faith, one pointed concentration and a burning desire for liberation (i.e.\ \emph{sādhanā} \emph{catuṣṭaya}). Such, a person having developed the above mentioned qualities, must approach an enlightened \emph{guru}, who would initiate him into \emph{jñāna} \emph{sādhanā} and show him the path towards \emph{Brahman}. This constitutes the path of \emph{nivṛttimārga}.

\section*{References}

\begin{enumerate}
\itemsep=0pt
\item
  \emph{Brahmavaivarta Purāṇa Prakṛti Khaṇḍa} 65.39. Also in \emph{Prakṛti Khaṇḍa} 59.46-55; 37.17
\item
  \emph{Śvetāśvataropaniṣad} 3.8
\end{enumerate}
\newpage

\section*{2.2. Nature of \emph{Brahman}}
\vskip -5pt

\emph{Brahman} or \emph{Ātman} has been variously described as formless, unmoving, all-pervading, etc.\ The \emph{Īśopaniṣad} describes \emph{Brahman} as being unmoving, yet moving faster than mind and senses; inhabiting everything in the universe, yet staying above/beyond the universe, among other things. To have a proper understanding regarding the descriptions given in the scriptures about the nature of \emph{Brahman} and its relationship to the universe, one must first have a clear understanding about the \emph{`frame of reference'} with respect to which, the descriptions have been made. The scriptures speak about two different frames of reference using which one can understand the relationship between the individual, the world and \emph{Brahman}. They are called: \emph{vyāvahārika} \emph{daśā} and the \emph{pāramārthika} \emph{daśā}.

The \emph{vyāvahārika} state refers to the \emph{dual} (\emph{dvaita}) state or relative state of reference. It is from this plane of \emph{vyāvahārika}, do people in this \emph{saṁsāra} who are bound by \emph{avidyā} perceive the world. That is, every person who is not a \emph{jñānī} perceives \emph{vyāvahārika} \emph{daśā} alone. They perceive the duality of object and the subject. There is a \emph{jīva} / individual who is the enjoyer i.e.\ subject and a \emph{jagat} / world, which is the object of his enjoyment. And distinct from these two, there is a \emph{Īśvara} / God who is the creator of the universe. Hence, in \emph{vyāvahārika} \emph{daśā}, there exist \emph{jīva}, \emph{jagat} and \emph{Īśvara} --all separate. On the other hand, the \emph{pāramārthika} \emph{daśā} is a state of \emph{non-duality} (\emph{advaita}) where the `\emph{One'} alone exists --call it \emph{Brahman} or \emph{Ātman} or \emph{Īśvara}. Nothing other than that exists. The distinctions of \emph{Īśvara}, \emph{jīva} and \emph{jagat} cease to exist in this state. The \emph{śāstra-s} also call this state as `\emph{turīya}' --the fourth state of \emph{jñāna}. While the former (i.e.\ \emph{vyāvahārika} state) is a temporary and relative state of existence, the latter is the absolute-permanent state of existence. As long as a person is bound by \emph{avidyā} and \emph{karma}, he cannot perceive \emph{pāramārthika} \emph{satya} i.e.\ absolute truth. But, a \emph{jñānī} who has attained \emph{ātmajñāna} perceives \emph{pāramārthika} \emph{satya} alone.

In \emph{Bṛhadāraṇyakopaniṣad} (1), Sage \emph{Yājñavalkya} while discussing with his wife \emph{Maitreyī} about the nature of this cosmos says:

\emph{yatra hi dvaitamiva bhavati taditara itaraṁ jighrati taditara itaraṁ paśyati taditara itaraṁ śṛṇoti taditara itaramabhivadati taditara\break itaraṁ manute taditara itaraṁ vijānāti yatra vā asya sarvamātmai\-vābhūttatkena kaṁ jighrettatkena kaṁ paśyettatkena kaṁ śṛṇuyāttatkena kamabhivadettatkena kaṁ manvīta tatkena kaṁ vijānīyāt} \dev{॥}
\vskip 2.5pt

Translation: \emph{``Where there is duality, as it were, then one smells something, one sees something, one hears something, one speaks something, one thinks something, and one knows something. (But) when to the knower of Brahman everything has become the Self, then what should one smell and through what, what should one see and through what, what should one hear and through what, what should one speak and through what, what should one think and through what, what should one know and through what?}
\vskip 2.5pt

The first part of the address refers to the \emph{vyāvahārika} state. Here, an individual perceives himself to be different from \emph{Ātman}. He perceives himself to be body and mind, and perceives \emph{Īśvara} and world as separate from him. Here, \emph{Īśvara} is the creator and the \emph{jīva} and the \emph{jagat} is the created. The \emph{jīva} is the subject and the world the object. Due to this multiplicity in perception, a person can experience the sensory objects. He can see, hear, taste, smell and feel the touch. He experiences both the pains and pleasures of the sensory world. On the other hand, the second part of the \emph{Yājñvalkya's} address beautifully captures the essence of \emph{pāramārthika} state. In the absolute state of non-duality, \emph{no question of a second entity arises}. When the object and the subject have merged together, what remains to be perceived? The \emph{knower} and the \emph{known} and the \emph{process of knowing} have \emph{become one} in this state. \emph{Yatra tvasya} \emph{sarvaṁ} \emph{ātmaivābhūt} - where only \emph{Ātman} pervades everything, there is neither a creator nor any creature, neither enjoyer nor object of enjoyment.
\vskip 2.5pt

Thus, the \emph{śāstra-s} have spoken about two frames of references with respect to which \emph{Brahman} \dev{।} \emph{Ātman} can be understood. Though birthless, eternal and without any movement in the \emph{pāramārthika} \emph{daśā}, \emph{Brahman}, by its power of \emph{māyā} manifests the whole universe as an appearance. Hence, in \emph{vyāvahārika} \emph{daśā}, \emph{Brahman} is both the material cause and the intelligent cause of the universe. The universe is called `\emph{jagat}' or `universal movement', because it is subjected to modifications like birth, growth, transformation, decay and death. Though in reality \emph{Brahman} is without modifications, It appears (through Its \emph{māyā}) as if undergoing modifications, as if manifesting this \emph{jagat} and inhabiting it. Hence, when this \emph{Upaniṣad} says, that \emph{Brahman} though ``sitting, runs faster than the mind and the senses'' (2), it means, that though \emph{Brahman} is without movement, without any real manifestation in the absolute state, yet by His power of \emph{māyā} He appears in \emph{Vyāvahārika} \emph{daśā} to be manifesting this universe and inhabiting in all its movements. Likewise, on the one hand, this \emph{Upaniṣad} says that \emph{Brahman} pervades the whole universe by inhabiting every object in them (3), and on the other hand, it says that \emph{Brahman} is devoid of all flesh, all bodies and all \emph{karma-s} that constitute this Universe and it exist `\emph{above'} the universe (4). By this, the \emph{Upaniṣad} means that, \emph{Brahman} though devoid of all duality, all limitations and devoid of any \emph{jagat}, yet by \emph{māyā} it manifests and pervades the entire \emph{jagat} as its \emph{Ātman}. Hence, though in absolute sense \emph{Brahman} is devoid of all movement and hence \emph{above} them, in the relative plane, He does exist as the innermost-Self of all objects of the universe.

When \emph{Brahman} is described as \emph{nirguṇa} (devoid of \emph{guṇa-s}), \emph{nirākāra} (formless), \emph{ekaṁ} (one without a second and without parts), \emph{anejat} (without movement), etc., it must be understood that the reference is to the \emph{pāramārthika} \emph{satya}/absolute truth i.e.\ the state of \emph{turīya} \dev{।} \emph{jñāna}.

The \emph{jagat} consists of unmanifested source/ \emph{mūla prakṛti} and the manifested Universe (both subtle and gross realms). The manifested world contains both the forms and the formless entities. It is all that could be known. The unmanifested \emph{prakṛti}, the source/ mūla which is unknown contains the whole manifestation in it in an unmanifested state. The word `\emph{nirākāra} -formlessness' does not refer to the formless entities present in the manifestation as distinct from the forms, nor does it refer to the `\emph{formfullness}' of \emph{mūla} \emph{prakṛti} which contains the whole of manifestation inside it in an undifferentiated form. Even the formless entities in a sense have a form, even though that form ever keeps changing. The term \emph{nirākāra}/without form instead refers to \emph{Brahman} as being distinct from both forms and the formless, from both the manifested forms and the unmanifested formfullness. `\emph{Ākāra}' or `form' refers to an entity that is subjected to the space-time principles. The whole universe starting from the unmanifested source to the manifested realms --all are subjected to these space-time principles. Further, the entities of the manifested universe are subjected to the limitations of `form/shape (\emph{rūpa})'. However, \emph{Brahman} despite being the material and instrumental cause of this universe, is Himself devoid of space-time-form limitations. Hence, He is called `\emph{Nirākāra}'- the one who is devoid of any form or space limitations. Similarly, \emph{anejat}/unmoving refers to the absence of time principle in \emph{Brahman}. The \emph{jagat} as explained before is subjected to modifications like birth, growth, decay and death. Further, the manifested universe contains alternate states of rest and motion. These modifications constitute the limitations caused by time-principle. `\emph{Anejat}' does not refer to temporary state of rest, instead it refers to absence of time-principle in \emph{Brahman}.

\emph{Brahman} is also called `\emph{nirguṇa}' and `\emph{saguṇa}'. \emph{Nirguṇa} refers to `devoid of \emph{guṇa-s}' and \emph{saguṇa} refers to `endowed with \emph{guṇa-s}'. \emph{Brahman} in \emph{pāramārthika} state is referred as `\emph{nirguṇa}' and the same \emph{Brahman} when He is manifesting the universe as an appearance or mirage through his \emph{māyā}, He is referred as ``\emph{saguṇa}''. It is \emph{Brahman} alone who actually is \emph{nirguṇa}, by his \emph{māyā} appears as \emph{saguṇa}. A firm understanding of this, that `\emph{Brahman}' alone exists constitutes `\emph{Brahmajñāna}/\emph{ātmajñāna}'.

It is important to understand that the \emph{pāramārthika} \emph{satya} / absolute truth is beyond any description or logical comprehension. Hence, all the descriptions we find in the scriptures are merely pointers about the \emph{pāramārthika} \emph{satya} that are explained in relation to the \emph{vyāvahārika} \emph{daśā}. And they aim at teaching one thing: in \emph{pāramārthika} \emph{daśā} \emph{Brahman} `\emph{just is}', \emph{Brahman} `\emph{alone is}', \emph{Brahman} `\emph{exists}'. It is `\emph{sat}' --existence. No other descriptions are possible about \emph{Brahman} because He is beyond all words, all thoughts, all descriptions and all duality. \emph{Brahman} can only be directly realized (\emph{aparokṣa} \emph{jñāna}/immediate knowledge).

\section*{References}

\begin{enumerate}
\itemsep=0pt
\item
  \emph{Bṛhadāraṇyakopaniṣad} 2.4.14, translation taken from\hfil\break Swami Madhavananda
\item
  \emph{Īśopaniṣad} Verse 4
\item
  \emph{Īśopaniṣad} Verse 1 and 8
\item
  \emph{Īśopaniṣad} Verse 8
\end{enumerate}
\newpage

\section*{2.3. \emph{Sādhanā} \emph{Catuṣṭaya}}

The \emph{śāstra-s} speak about various necessary qualities and attitudes that must be imbibed in an individual in order to become eligible for \emph{jñāna} \emph{sādhanā}. A person of impure mind, who is bound by impurities like anger, lust, etc.\ cannot grasp the true nature of \emph{Brahman}. Hence, before practicing \emph{jñāna} \emph{sādhanā}/\emph{mokṣa} \emph{sādhanā}, one must achieve `\emph{citta} \emph{śuddhi}' or purification of the mind and attain certain qualities collectively called `\emph{sādhanā} \emph{catuṣṭaya}' or four-fold qualifications required for (\emph{jñāna}) \emph{sādhanā}.

The present \emph{Upaniṣad} (1) opens with the statement that only those who have understood that the sensory objects are temporary and mere products of \emph{avidyā} and that \emph{Brahman} alone inhabits all the objects as their \emph{Ātma}, will renounce all their desires and travel \emph{nivṛttimārga}. Hence, without the development of discrimination (\emph{viveka}) between what is \emph{nitya}/permanent i.e.\ \emph{Brahman} and what is \emph{anitya} /temporary i.e.\ the world, one cannot practice \emph{jñāna} \emph{sādhanā}.

\emph{Kaṭhopaniṣad} (2) says-

\begin{verse}
\emph{nāvirato duścaritān nāśānto nāsamāhitaḥ \dev{।}}\\
\emph{nāśāntamānaso vāpi prajñanenainamāpnuyāt \dev{॥} }
\end{verse}

\emph{``One who has not desisted from bad conduct, whose senses are not under control, who has not achieved one pointed concentration, whose mind is not restrained, cannot attain Self through Knowledge.''}

The term `\emph{duścaritaṁ} /bad conduct' here refers to a person who lives his life without a care or concern for \emph{dharma}. He is completely under the influence of impurities of mind like anger, hate, delusion, desire, etc.\ and acts in a selfish and adharmic manner. Hence, a person who cannot live his life by the tenets of \emph{dharma} like truth, nonviolence, non-stealing, etc.\ is not qualified for \emph{nivṛttimārga}. Similarly, a person whose mind and the senses are ever indulged in sensory pleasures and temptations and without restraint (\emph{śāntaḥ}) and hence devoid of one pointed concentration (\emph{samāhitaḥ}) are also not qualified for the path.

In another verse, the same \emph{Upaniṣad} (3) says that only those people who are associated with `discriminating intellect/\emph{vijñāna}', `controlled mind (\emph{samanaḥ})' and are `purity (\emph{śuciḥ})' can attain \emph{mokṣa}. The same has been repeated in \emph{Muṇḍakopaniṣad} (4) that the knowledge of \emph{Brahman} must be taught by a teacher to only those whose mind is pure (\emph{praśānta} \emph{citta}) and their senses are under control (\emph{śama}). Similarly, the \emph{Taittirīyopaniṣad} (5) also speak about control of mind (\emph{śama}) and control of senses (\emph{dama}) and to that it adds righetousness (\emph{ṛtaṁ}), truth (\emph{satya}), austerity (\emph{tapas}), practice of rites and rituals (\emph{Agnihotra}), learning and teaching (\emph{svādhyāya-pravacanaṁ}) and performance of other duties.

Thus, the \emph{Upaniṣad}s at various places have listed many qualities like truth, austerity, righteousness, control of mind and senses, discriminating intellect, purity, one pointed concentration etc.\ as the necessary eligibility criteria for practicing the path of renunciation. \emph{Ādi Śaṅkarācārya} has systematically arranged these qualities into four categories and has termed it as `\emph{sādhanā} \emph{catuṣṭaya}'- the four-fold qualifications/competencies required to make one eligible for \emph{jñāna} \emph{sādhanā}. In his \emph{Vivekacūḍāmaṇi} (6), \emph{Śaṅkarācārya} explains these competencies thus:

\begin{verse}
\emph{ādau nityānityavastuvivekaḥ parigaṇyate \dev{।}}\\
\emph{ihāmutraphalabhogavirāgastadanantaram \dev{।}}\\
\emph{śamādiṣaṭkasampattirmumukṣutvamiti sphuṭam \dev{॥}}
\end{verse}

\emph{First, the discrimination between real and unreal is counted. (Next comes) dispassion towards enjoyment of fruits of action here and the next world. After this comes, the six qualities like ``śama /control of mind''. (And finally) a burning desire for liberation.}

Hence, the four-fold competencies are- \emph{viveka}, \emph{vairāgya}, \emph{ṣaṭka} \emph{sampatti} and \emph{mumukṣutva}. `\emph{viveka}' or ``discrimination'' is described as the quality of the intellect to differentiate between what is real and permanent i.e.\ \emph{Brahman} and what is unreal and temporary i.e.\ sensory world. This is the most important of the qualities because every other competency follows from this. It is for this reason alone, the \emph{Īśopaniṣad} begins by enumerating about it. Once a person has developed \emph{viveka} and understood that \emph{Brahman} alone is \emph{real}, he would develop dispassion towards the pleasures and temptations offered by the sensory world. This `dispassion' towards enjoyment of pleasures like offspring, wealth etc.\ in this world and enjoyment of heaven, etc.\ in the next world is termed as ``\emph{vairāgya}''.

\emph{Ṣaṭka} \emph{sampatti} refers to the six fold qualities of \emph{śama}, \emph{dama}, \emph{uparati}, \emph{titikṣā}, \emph{śraddhā} and \emph{samādhāna}. \emph{Śama} refers to \emph{antar indrīya nigraha} /control of mind and \emph{Dama} refers to \emph{bahir indrīya nigraha} /control of five senses. One must develop control over one's five senses and the mind only then the mind will be able to direct itself towards \emph{Brahman}. Otherwise the mind will always be indulging in sensory objects. It is the very nature of mind and the sense organs to proceed outwards to grasp and enjoy the sensory objects. Hence, in order to turn the mind inwards towards \emph{Ātma}, one must first attain complete restraint of the mind and the senses. The complete withdrawal of the mind and the senses through effort is called `\emph{uparati}' and it comes when \emph{śama} and \emph{dama} are perfected. In his \emph{Tattva Bodha} (7), \emph{Śaṇkarācārya} defines `\emph{uparati} / \emph{uparama}' as `\emph{svadharma} \emph{anuṣṭhāna}m /practice of one's \emph{dharma}'. Hence, a person by the practice of \emph{karma} (i.e.\ rites and rituals) as prescribed in the scriptures and living a life of righteousness, truth, etc.\ attains mental purity and will be able to withdraw his mind and the senses from the sensory objects. \emph{Titikṣā} refers to `absence of anger/revenge'. It is a state where a person does not feel anger or revenge towards anybody else. \emph{Śaṇkarācārya} defines \emph{titikṣā} as: ``\emph{sahanaṁ sarvaduḥkhānāmapratīkārapūrvakam \dev{।}} \emph{cintāvilāparahitaṁ sā titikṣānigadyate \dev{॥}} (8)'' meaning ``one must patiently bear all the sorrows without developing hate or a sense of revenge and without any worries''. This is possible only when one gets an understanding that happiness and sorrow are results of one's own past actions and hence, it is futile to blame others. \emph{Titikṣā} is one of the very important qualities because, just as a person must be dispassionate towards worldly pleasures, similarly he must be unaffected by the worldly sorrows. \emph{Śraddhā} refers to faith in scriptures and in one's \emph{guru}. Without this, no amount of \emph{sādhanā} will bear fruit. \emph{Samādhāna} refers to one-pointed fixing of the mind on \emph{Brahman} spontaneously. Finally \emph{mumukṣutva} refers to the intense burning desire for \emph{mokṣa}.

Without any of these qualities, a person cannot overcome his attachment to desires and hence, will not be able to renounce those sensory objects and practice \emph{nivṛttimārga} (i.e.\ \emph{jñāna} \emph{sādhanā}). But, to attain these qualities, a person must achieve mental purification (\emph{citta} \emph{śuddhi}) by getting rid of mental impurities. The scriptures speak about six types of impurities that bind a person in \emph{saṁsāra}. They are- \emph{kāma} (lust), \emph{krodha} (anger), \emph{lobha} (greed), \emph{moha} (delusion/attachment), \emph{mada} (pride) and \emph{mātsarya} (jealousy). These are collectively called `\emph{ariṣaḍvarga} - six passions' or `\emph{ṣaḍripu-s} six enemies' of an individual. \emph{Kāma} and \emph{lobha} makes a person continuously hanker for material pleasures. He will be ever running behind his desires. As soon as a desire is fulfilled, he would be occupied with new desires. \emph{Lobha} further makes a person commit crimes and corruption to attain wealth and other pleasures through illegal, immoral, and adharmic means. Such people would become frustrated and develop \emph{krodha}, if they fail to fulfill their desires. They would blame others and will spend time and energy with an intention to take revenge. \emph{Mada} and \emph{mātasarya} would make a person blind towards their own weaknesses. Such people will always be comparing themselves with other's wealth and achievements and will ever indulge in trying to bring down their competitors. In this way, such a a person would be completely occupied with and always running behind the worldly desires and pleasures born out of attachment to them i.e.\ \emph{moha}. Thus, a person will become completely bound to \emph{saṁsāra}, alternatively facing pleasure and pain, and he will ever remain trapped in the \emph{karmic} cycle of birth and death.

On the other hand, a person who develops \emph{vairāgya} completely destroys \emph{kāma}. Similarly, \emph{viveka} leads to destruction of \emph{moha}, \emph{titikṣā} leads to destruction of \emph{krodha}, and \emph{indrīya nigraha} destroys the remaining three internal enemies. However, in order to attain the four-fold competencies by getting rid of internal enemies, one must first develop detachment and surrendering. One must practice \emph{niṣkāma} \emph{karma}/detached \emph{karma} with a sense of duty by surrendering the action, the doership of action and the fruits of action to \emph{Īśvara}. Only by developing \emph{niṣkāma} \emph{daśā} and \emph{samarpaṇa bhāva} (i.e.\ attitude of surrendering) does one attain purification of the mind and the destruction of all the internal enemies, thereby attaining the required competencies

\section*{References}

\begin{enumerate}
\itemsep=0pt
\item
  \emph{Īśopaniṣad} Verse 1
\item
  \emph{Kaṭhopaniṣad} 1.2.24
\item
  \emph{Kaṭhopaniṣad} 1.3.8
\item
  \emph{Muṇḍakopaniṣad} 1.2.13
\item
  \emph{Taittirīyopaniṣad} 1.9.1
\item
  \emph{Vivekacūḍāmaṇi} Verse 19
\item
  \emph{Tattva Bodha} Verse 5.5
\item
  \emph{Vivekacūḍāmaṇi} Verse 24
\end{enumerate}
\newpage

\section*{2.4. \emph{Jñāna} \emph{Sādhanā} and \emph{Jīvanmukti}}

A person who has purified his mind and has acquired the four-fold qualities of discrimination, dispassion, control of the mind and the senses, and a burning desire for liberation, only such a person becomes competent enough to practice \emph{jñāna} \emph{sādhanā} (also called \emph{Ātman} \emph{sādhanā} or \emph{mokṣa} \emph{sādhanā}). Such, a person should approach a competent \emph{guru}, who would then instruct him in \emph{ātmavidyā} and take him towards \emph{Brahman}/\emph{Ātman}. \emph{Muṇḍakopaniṣad} says (1)-

\begin{verse}
\emph{parīkṣya lokān karmacitān brāhmaṇo nirvedamāyānnāstyakṛtaḥ kṛtena \dev{।}}\\
\emph{tadvijñānārthaṁ sa gurumevābhigacchet samitpāṇiḥ śrotriyaṁ brahmaniṣṭham \dev{॥}}
\end{verse}

\emph{A brāhmaṇa should resort to renunciation after examining the worlds acquired through karma, with the help of this maxim ``There is nothing (here) that is not the result of karma; so what is the need of (performing) karma (i.e karmānuṣṭhāna)? For knowing that reality, he should go, with sacrificial faggots in hand, only to a teacher versed in the Veda-s and absorbed in Brahman.}

A person who has purified his mind by the practice of \emph{karmānu\-ṣṭhāna} (including \emph{devatopāsana}) will renounce all his desires by understanding that \emph{karma} or \emph{upāsanā} done with desires will lead at the most to Hiranyagarbha or the unmanifested \emph{prakṛti} and will not bestow \emph{mokṣa} (this understanding is called \emph{viveka}). Such a person would renounce all the desires and attachments for worldly objects (i.e.\ \emph{vairāgya}) and with a single pointed desire to attain \emph{mokṣa} (i.e.\ \emph{mumukṣu}) will approach a \emph{guru} who can lead him to \emph{ātmajñāna}. The \emph{Muṇḍaka} says that, the \emph{guru} must be one who is well versed in \emph{Veda-s} and ever absorbed in \emph{Brahman}. It is so because only a person who has achieved \emph{ātmajñāna} will be able to guide others on the path. If a \emph{guru} is himself ignorant of \emph{Veda-s} and \emph{Brahman}, how would he be able to teach \emph{Ātma}-\emph{tattva} (i.e.\ tenets or teachings related to \emph{Ātma}/\emph{Brahman}) and lead one to \emph{ātmasākṣātkāra} /self-realization? Hence, a sincere student desiring to attain \emph{mokṣa} must approach only that teacher who is well versed in \emph{Veda-s}, who practices \emph{Vedic} tenets in his own life, and who is ever absorbed in \emph{Brahman}. \emph{Ādi} \emph{Śaṇkarācārya}, in his \emph{Vivekacūḍāmaṇi} (2) elaborates on the qualities of \emph{guru}.

\begin{verse}
\emph{śrotriyo'vṛjino'kāmahato yo brahmavittamaḥ \dev{।}}\\
\emph{brahmaṇyuparataḥ śānto nirindhana ivānalaḥ \dev{।}}\\
\emph{ahetukadayāsindhurbandhurānamatāṃ satām \dev{॥ 33॥}}
\end{verse}

\emph{(A teacher is one) who is well-versed in the Veda-s, sinless (i.e.\ pure), un-afflicted by desires and a knower of Brahman par excellence, who always abides in Brahman; who is calm like a fire that has consumed its fuel, who is a boundless reservoir of compassion that knows no reason and an intimate friend to all good people who surrender to him.}

Hence, a \emph{guru} must be one who is well versed in the \emph{Veda-s} and is ever abiding in \emph{Brahman}. He must be one who is completely devoid of desires so that he remains pure in body and mind, for it is the desires that drive a man to commit \emph{adharma}. A \emph{guru}'s mind is ever established in \emph{Brahman} without any desires and impulses that would result in external activity like a fire that has consumed its fuel. He is the very abode of compassion who manifests unselfish love towards everyone and is a friend and guide to all those who seek refuge in him. Hence, a person who desires \emph{mokṣa} must approach such a \emph{guru} and under his guidance the seeker should practice \emph{Ātma-}s\emph{ādhanā} also called `\emph{śravaṇa}-\emph{catuṣṭaya}'.

\emph{Śravaṇa} \emph{Catuṣṭaya} includes four-stages: \emph{śravaṇa}, \emph{manana}, \emph{nididhyāsana}, and \emph{Ātma}- \emph{sākṣātkāra}. \emph{Śravaṇa} and \emph{manana} refers to listening and internalizing the teachings imparted by one's \emph{guru}. A \emph{śiṣya} is expected to do further reading of the \emph{śāstra-s} on the said subject and get cleared of the doubts that arises in his mind. The \emph{guru} will guide the disciple slowly towards \emph{jñāna} by clearing one by one all the doubts that arise inside the disciple. After a disciple understands the general import of the scriptures, the \emph{guru} will teach him the \emph{Mahāvākya}'s enunciated in the \emph{śruti} (i.e.\ \emph{Veda-s}).The disciple then contemplates and meditates on the three-fold meanings of the \emph{mahāvākya}. This is called \emph{nididhyāsana}.\break Such, contemplation leads to complete understanding / realization of the Self. This is called \emph{Ātma}-\emph{sākṣātkāra} or \emph{ātmajñāna}. About the three-fold meanings of a sentence, \emph{Sureśvarācārya} in his \emph{naiṣkarmya siddhi} (3) says-

\begin{verse}
\emph{sāmānādhikaraṇyaṁ ca viśeṣaṇaviśeṣyatā \dev{।}}\\
\emph{lakṣyalakṣaṇasaṁbandhaḥ padārthapratyagātmaām \dev{॥}}
\end{verse}

\emph{Co-ordination, subject-predicate relation and indirect indication are the ways governing the terms, their meanings and the inner-Self.}

For example, in the \emph{mahāvākya} - `\emph{tat tvam asi}' meaning `\emph{thou art that}', the `\emph{thou}' and `\emph{That}', i.e.\ the \emph{jīva} and \emph{Brahman} which 	are different are being equated. In other words, the \emph{mahāvākya} is equating the \emph{jīva} that is limited and temporary as body-mind complex with \emph{Brahman} that is infinite and birthless. This form of meaning derivation is called `\emph{sāmānādhikaraṇam}' or `\emph{co-ordination'}. On the other hand, if we are to take that the \emph{mahāvākya} is implying that `\emph{thou}' is being qualified by the term `\emph{That}' and vice versa, then it means that to \emph{jīva} is attributed the \emph{Brahman}-hood (i.e.\ the qualities like birthlessness, infiniteness) and to \emph{Brahman} is attributed the \emph{jīva}-hood (i.e.\ the qualities like impermanence and limitations). This kind of interpretation is called ``\emph{viśeṣaṇa-viśeṣya bhāva}'' or \emph{`subject-predicate relation'}.

But, it is visible that in the case of the \emph{mahāvākya}, both \emph{sāmānādhi\-karaṇam} and \emph{viśeṣaṇa-viśeṣya bhāva} does not apply because, neither the limited and temporary \emph{jīva} that is subjected to \emph{karma} and bondage can be equated to \emph{Brahman} who is eternal, birthless and ever free, nor can the qualities of the one be attributed to another. \emph{Gauḍapāda} in his \emph{Māṇḍukya} \emph{Kārikā} (4) says that: ``The immortal cannot become mortal. Similarly, the mortal cannot become immortal. The mutation of one's nature will take in no way whatsoever''. Hence, in the case of this \emph{mahāvākya}, the co-ordination and subject-predicate relationship does not apply. Instead, one must take recourse to the third option: `\emph{lakṣya-lakṣaṇa sambandha}' i.e.\ `\emph{indirect indication}'. Here, the `thou' refers not to the \emph{jīva} who is limited by body and mind, but to the \emph{Ātma}, the inner-most Self who is ever-free and the \emph{lone witness} of the body, mind, and the senses. This eternal and ever-free \emph{Ātman} is then identified as being non-different from \emph{Brahman}, the substratum of the universe which is also eternal and birthless. In other words, the \emph{mahāvākya} implies that, it is \emph{Brahman} which is infinite and eternal inhabits all the objects of the universe as their \emph{Ātma}/innermost Self (5) and hence, there is identity and non-difference between \emph{Brahman} and \emph{Ātman}. This direct realization of \emph{Brahman} as immediate and inner-most Self is called `\emph{aparokṣa jñāna}' or `\emph{ātmajñāna}'.

Thus, a disciple who hears the \emph{mahāvākya}s from the \emph{guru}, internalizes the message and contemplates on it, will eventually attain \emph{Brahmajñāna}, i.e.\ the direct knowledge of the identity of \emph{Ātman} with \emph{Brahman}, thereby destroying ignorance that had created an appearance of difference/duality in its entirety. Such a realized person will become free immediately and they are called `\emph{jīvanmukta}'. But, he may or may not leave his body immediately. The \emph{karma-s} can be classified as past \emph{karma-s} that are yet to bear results (\emph{sañcita karma}), the past \emph{karma-s} whose results are bearing the fruit in the form of present birth, etc.\ (\emph{prārabdha karma}), and the present \emph{karma-s} (as well as future \emph{karma-s}) that will bear fruit in the future (\emph{āgāmī karma}). A person when he attains \emph{ātmajñāna}, the Self-knowledge would destroy all the bondages that arise due to \emph{avidyā}/ignorance. Hence, it would completely destroy the past \emph{karma-}s (\emph{sañcita}) and also nullifies the future \emph{karma-}s (\emph{āgāmī}). But, it will not destroy those \emph{karma-}s which have already begun to bear fruit i.e.\ \emph{prārabdha}. In \emph{Brahmasūtra-s} (6), \emph{Śrī Veda Vyāsa} says-

\begin{verse}
\emph{tadadhigama uttarapūrvāghayoraśleṣavināśau tad vyapadeśāt~\dev{॥}}
\end{verse}

\emph{When That (i.e.\ Brahman) is realized, then there is non-clinging of and destruction of future and past sins (i.e.\ actions) respectively, for this is the teaching.}

Hence, a person who attains \emph{jñāna} immediately discards his body-mind complex only when his \emph{prārabdha} \emph{karma-}s have been already exhausted at the time of attaining \emph{jñāna}. On the other hand, if he has \emph{prārabdha} \emph{karma-}s left to be exhausted at the time of \emph{jñāna}, then he will live in the body as \emph{jīvanmukta}, till the time the \emph{prārabdha} \emph{karma-s} are worked out and become completely exhausted. Just as an uprooted tree perishes by withering away and drying up, so also the \emph{jīvanmukta} lives in the body till the exhaustion of \emph{prārabdha} \emph{karma-}s (7).

Such a \emph{jīvanmukta} stays in the world and his body and mind would perform actions according to \emph{prārabdha}, but he is not affected by it. A \emph{jīvanmukta} neither experiences sorrow, nor experiences happiness. He is ever abiding in the blissful state of \emph{jñāna}. He perceives his \emph{Ātman} everywhere and in all objects and all objects are perceived in his \emph{Ātman} (8). Though through mind, he may perceive multiplicities of the world, but such perception would be superseded by the \emph{vision of oneness}, of \emph{Ātman} (9). His body stays in the world, works out its \emph{karma-}s, but his Self remains ever untouched by those actions like a cow that is not affected by the garland it is made to wear (10). He will be ever established in \emph{Brahman} and neither feels anger nor hate nor delusion nor attachment. He will always have a \emph{sama dṛṣti} /equanimity and same-sightedness towards everyone. Such a \emph{jīvanmukta} will have no duties, no obligations towards anybody. Yet he stays in the world, only for the welfare of the world and to guide others towards \emph{mukti} and when his \emph{prārabdha} is exhausted, all the three bodies would get dissolved for ever.

\section*{References}

\begin{enumerate}
\itemsep=0pt
\item
  \emph{Muṇḍakopaniṣad}- 1.2.12
\item
  \emph{Vivekacūḍāmaṇi}- Verse -- 34-35
\item
  Naishkarmya-Siddhi- 3.3
\item
  Gaudapada \emph{Kārikā}- 3.21
\item
  \emph{Īśopaniṣad}- Verse 1
\item
  \emph{Brahmasūtra-s}- 4.1.13
\item
  \emph{Naiṣkarmya Siddhi} - 4.16
\item
  \emph{Īśopaniṣad} Verse 6
\item
  \emph{Śataślokī} Verse 12
\item
  \emph{Vivekacūḍāmaṇi} Verse- 416
\end{enumerate}
