\chapter{Verse 17}

\begin{moolashloka}
\dev{वायुरनिलममृतमथेदं भस्मांतँ शरीरम् ।}\\
\dev{ॐ क्रतो स्मर कृतँ स्मर क्रतो स्मर कृतँ स्मर ॥ 17 ॥}
\end{moolashloka}

\textbf{Word to word Meaning}: (Let) the \emph{prāṇa}/vital air/life-force (\dev{वायु:}) (merge in) the air immortal (\dev{अनिलम् अमृतम्}). Now (\dev{अथ}) (let) this (\dev{इदं}) reduced to ashes (\dev{भस्मांतँ}) body (\dev{शरीरम्}). Oṃ (\dev{ॐ}) `Divine will' (\dev{क्रतो}) remember (\dev{स्मर}) my deeds (\dev{कृतँ}) remember (\dev{स्मर}) Divine will (\dev{क्रतो}) remember (\dev{स्मर}) my deeds (\dev{कृतँ}) remember (\dev{स्मर}).

\textbf{Meaning:} Now (that I am on the verge of death), (let) this gross-body be reduced to ashes, (and/so that) my \emph{prāṇa}/life-force i.e. the subtle body can be (merged) with `\emph{immortal air}' (i.e. \emph{Satya Brahman} who exists as \emph{Mātariśvā Vāyu}). Oṃ, (\emph{Agni}) --the \emph{`Divine will'} --remember my deeds remember (i.e. remember the \emph{karmānuṣṭhāna} and \emph{Satya Brahma Upāsanā} I have performed my whole life).

\textbf{Analysis:} After the prayer to Sun, for guiding one towards \emph{Satya Brahman}, by overcoming one's subtle limitations, the \emph{Upaniṣad} now gives a prayer to Agni (the fire).

\dev{वायुरनिलममृतमथेदं भस्मांतँ शरीरम्}- \emph{Now, let this body be reduced to ashes, (so that) my life-force be (merged) with `immortal air'.} \dev{अथ} means `\emph{Now'}. Here, it refers to a devotee who having practiced \emph{karmānuṣṭhāna} and \emph{Satya Brahma Upāsanā} has approached death. Such a person after praying to \emph{Sūrya} that he be led to \emph{Satya Brahman} by removing his subtle limitations now prays to \emph{Agni} to burn down his body to ashes, so that he can attain \emph{Satya Brahman} by merging his life force i.e. subtle body in \emph{Satya Brahman}. \dev{इदं भस्मांतँ शरीरम्}- means literally `reduce this body into ashes'. Here it refers to removing the limitations imposed by the gross existence. The devotee is requesting \emph{Agni} to remove the limitations of gross existence by burning them down, by reducing them to ashes. Why? Because, for a person to attain \emph{Satya Brahman} (\emph{Hiraṇyagarbha}) by overcoming his subtle existence, he must first overcome the limitations of his gross existence. Hence, the devotee is praying to \emph{Agni}, to burn down his gross-body to ashes and remove the limitations placed by it. \dev{अनिलम् अमृतम्}- `\emph{immortal air}' refers to \emph{Satya Brahman}/\emph{Sūtrāman} who exists as cosmic life-force i.e. \emph{Mātariśvā Vāyu}, and \dev{वायु:} represents the \emph{prāṇa} or the individual life-force that makes up the subtle body. By merging one's life-force in cosmic life-force, one attains \emph{Satya Brahman} by overcoming the limitations of subtle existence. Hence, here the merging of individual \emph{Vāyu} with cosmic \emph{Vāyu} refers to merging of the subtle body of a \emph{jīva} with \emph{Hiraṇyagarbha}. Hence, the person is praying to \emph{Agni} to remove the limitations of gross existence, so that he can overcome the limitations of subtle existence and attain \emph{Satya Brahman} through \emph{Sūrya}.

\dev{ॐ क्रतो स्मर कृतँ स्मर क्रतो स्मर कृतँ स्मर}- \emph{Oṃ, Agni --the Divine will --remember by deeds, remember Divine will, remember my deeds, remember}. \dev{ॐ} refers to \emph{Brahman} (1). It is the symbol of \emph{Brahman}. Here, it refers to \emph{Brahman} who exist in Gross \emph{saṁsāra} (physical universe) as \emph{Agni} --the Divine will. \dev{क्रतो}- means `\emph{will'}. It refers to \emph{Agni}. Just as \emph{Sūrya} represents the \emph{Divine light} in an individual, the \emph{Agni} represents the \emph{Divine will} in the individual. The person, who has practiced \emph{karmānuṣṭhāna} and \emph{Satya Brahma Upāsanā}, having approached death, is praying to Agni to remember all the Karmas he has done. \dev{कृतँ}- here refers to performance of \emph{Karmānuṣṭhāna} and \emph{devatopāsana}. The person is praying to \emph{Agni} to remember that he has performed both \emph{karmānuṣṭhāna} and \emph{Satya Brahma Upāsanā} and hence, free him of gross existence by burning all the \emph{karma-s} that are binding him to gross-existence, so that he can follow the path of \emph{Sūrya} and attain \emph{Satya Brahman} by merging his subtle body in \emph{Satya Brahman}.

\textbf{Summary:} After a prayer to Sun, the \emph{Upaniṣad} now gives a prayer to \emph{Agni}. The Sun represents the Divine light and the \emph{Agni} represents Divine will, both of which are present within an individual. The worship of the \emph{Sūrya} i.e. \emph{vidyā upāsanā} removes the limitations of subtle existence, by removing the thoughts rooted in ignorance and redirecting it towards \emph{Satya Brahman}. The worship of \emph{Agni} i.e. \emph{karmānuṣṭhāna} removes the limitations of gross existence by burning down the \emph{karma-s} that are binding one to gross \emph{saṁsāra}. Hence, a prayer to \emph{Sūrya} is incomplete without a prayer to \emph{Agni}. Only together, would a worship of \emph{Sūrya} and \emph{Agni} will take one towards \emph{mokṣa}. So, after praying to \emph{Sūrya}, the devotee on the verge of death, now prays to \emph{Agni} to remember all the actions -- \emph{karmānuṣṭhāna} and \emph{devatopāsana} --that he has performed, so that upon his death, the limitations of gross existence be removed from him and he be facilitated to travel the northern path, the path of light and attain \emph{Satya Brahman} by merging his life-force in cosmic \emph{Vāyu} i.e. \emph{Hiraṇyagarbha}.

\section*{References}

\begin{enumerate}
\item
  \emph{Māṇḍukyopaniṣad} Verse Number-1, 2 and 8.
\end{enumerate}


