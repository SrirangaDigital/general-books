\chapter{Preface}

\begin{center}
\emph{The very first stanza of this matchless Upanishad is in itself a miniature philosophical textbook\ldots{}Its mantras are the briefest exposition on philosophy and each one is an exercise in contemplation} (1) --\emph{\textbf{Swami Chinmayananda}}
\end{center}

Though a short \emph{Upaniṣad} with only 18 \emph{mantras, Īśopaniṣad} is infused with deep insights into the mysteries of life and covers the entire gamut of Vedic worldview. It will not be a stretch to state that the \emph{Upaniṣad} speaks about all there is to be spoken about and provides a bird's-eye-view of Hindu philosophy and practice.

The \emph{Īśopaniṣad,} also called \emph{Īśāvāsyopaniṣad} gets its name from its opening words. M. Hiriyanna in his introduction to his translation of this \emph{Upaniṣad} notes that it is also called \emph{Samhitopaniṣad} as it forms the concluding chapter of the \emph{Saṃhitā} of the \emph{Suklayajurveda}, unlike other \emph{Upaniṣads} which generally find their place in the \emph{Brāhmaṇas} (2).

The present work is divided into two parts. Part I contains the translation and verse-by-verse commentary on each of the eighteen \emph{mantra-s} of the \emph{Īśopaniṣad} and part 2 enunciates the overall philosophy of the text and provides a bird's-eye-view of the \emph{Vedāntic} vision of life.

My enquiry into this wonderful text began in 2012 when I had come across some arguments on the internet regarding how different commentators have interpreted some of the \emph{mantras} of this \emph{Upaniṣad} in a vastly divergent manner. In particular, a long discussion on the topic with my friend Rajarshi Nandy, who is a scholar and a \emph{Tantra} practitioner, led me to embark upon an in-depth study and reflection of this \emph{Upaniṣad} along with its commentaries. In particular, I focussed my attention towards the Sanskrit commentary of \emph{Śrī} Shankaracharya Bhagavadpada and the English commentary of \emph{Śrī} Aurobindo, as well as many recent translations of the text.

2012-13 were also the years when I became a recipient of the immense \emph{anugraha} of my \emph{Guru}, \emph{Jagadguru} \emph{Śrī} Bharati Tirtha Mahaswamiji --the \emph{Śaṅkarācārya} of Sringeri \emph{pīṭha} and became properly introduced to \emph{Advaita Siddhānta} through a series of lectures delivered by him during his \emph{Vijayayātrā}. Then, in February 2013, I received an \emph{ādeśa} from my divine mother and \emph{iṣṭa}, \emph{Mahākālī} directing me to write a commentary in English for \emph{Īśopaniṣad}.

Significant sections of Part I of the book were written during 2013-14 and Part II during 2016-17. They were further revised and polished during 2018-19. This commentary could not have been completed without the \emph{anugraha} of my mother \emph{Mahākālī} and my \emph{Guru}. My \emph{sāṣṭāṅga-namaskāra} to both of them.

I would also like to thank Rajarshi Nandy for that discussion back in 2012 which set me upon the path which has led me here today. I am very thankful to all the \emph{Svāmīs, Svāminī-s} and the teachers of \emph{Vedānta} who have blessed the book and shared their positive comments. I must make special mention of \emph{Svāminī} Svatmavidyananda ji, \emph{Ācārya} V Subramanian ji and Prof. K Ramasubramanian ji for going through the manuscript and suggesting corrections and improvements that has enriched the book.

I am especially grateful to \emph{Śrī} Chittaranjan Naik ji for writing the Foreword to the book. I also thank Praveen Nair for his help with diacritic work and Rithwik Subramanya of Subbu publication for publishing the book. Last, but not the least, I want to thank my family, especially my wife Pratyasha, without whose constant inspiration, advice, and support, I could not have completed this work.

It is my hope that this book would reach far and wide, and would inspire many to not only take up a deeper study of \emph{Vedānta}, but also to align their lives to Hindu values and worldviews.

\section*{References:}

\begin{enumerate}
\item Swami Chinmayananda, Preface, `\emph{Īśāvāsya Upaniṣad} - God in and as everything', Chinmaya Prakashan, Mumbai.
\item M. Hiriyanna (tr.), Introduction, `\emph{Īśāvāsyopaniṣad with the Commentary of Śrī Śaṅkarāchārya}', Vani Vilas Press, Srirangam
\end{enumerate}
