\chapter{Verse 3}

\begin{moolashloka}
\dev{असुर्या नामा ते लोका अन्धेन तमसा आवृताः ।}\\
\dev{ताँस्ते प्रेत्याभिगच्छंति ये के च आत्म-हनो जनाः~॥ 3 ॥}
\end{moolashloka}

\textbf{Word to word Meaning}: Undivine (\dev{असुर्या}) are called (\dev{नामा}) those (\dev{ते}) worlds (\dev{लोका}) which are engulfed (\dev{आवृताः}) by blinding (\dev{अन्धेन}) darkness (\dev{तमसा}). Into those worlds (\dev{तान्}) after death (\dev{प्रेत्य}) they (\dev{ते}) go (\dev{अभिगच्छंति}) those people who (\dev{ये के जनाः}) are slayers of the Self (\dev{आत्मा-हनः}).

\textbf{Meaning}: The worlds which are engulfed by blinding darkness (of ignorance and bondage) are called ``undivine'' (because they lack of light of Truth and Freedom). People who slay the (True) Self, the \emph{Ātma} (by indulging in \emph{saṁsāra} and forgetting their true nature), will go to such undivine-worlds (i.e. return to \emph{saṁsāra}) after their death.

\textbf{Analysis}: The \emph{Upaniṣad} now moves to explain about the condition of those people who are completely indulged in the \emph{Moha's}/attractions offered by the \emph{saṁsāra}; those who have neither achieved \emph{citta śuddhi}, nor have any desire to achieve it; those who are completely immersed in materialistic life without care for either \emph{Dharma} or \emph{mokṣa}.

\dev{असुर्या नामा ते लोका}- \emph{``Undivine'' are called those worlds}. \dev{लोका} refers to the ``births'' an individual \emph{jīva} takes into various worlds/realms of \emph{saṁsāra} in order to enjoy the fruits of his \emph{Karma}. Every person who is bound in the Karmic cycle of birth and death will endlessly take birth again and again to face the fruits of his action, both good and bad. Dharmic actions may lead a person towards better world, a better state of existence, a better birth in this world itself, or may be an entry into higher realms wherein he enjoys the good fruits and Adharmic actions will lead a person to lower states of existence, a lower birth may be that of some beasts or an entry into lower realms wherein he faces the fruits of bad karmas. The worlds-High and Low refer to the amount of \emph{light of Truth} expressed/manifested in those worlds. As one moves higher and higher, one gets closer to the \emph{Truth of Unity}, the \emph{Truth of Onesses}- the \emph{pāramārthika satya} (1). As one moves lower and lower, the bondage to \emph{saṁsāra} increases, the divisions, the multiplicities increases, and one's indulgence in \emph{vyāvahārika satya} (2) deepens. The term \dev{लोका} hence refers to various births in various worlds that a \emph{jīva} takes, in order to face the Karmic fruits.

All the worlds --the higher worlds like \emph{Satya, Tapa, Jana}, and \emph{Maha}; the lower worlds like \emph{Atala, Vitala, Sutala}, \emph{Talātala}, \emph{Mahatala}, \emph{Rasātala} and \emph{Pātāla}; and the middle worlds like \emph{Bhū} (the physical universe), \emph{Bhuvar} and \emph{Svar} --are called \dev{असुर्या}. The realms of the \emph{Deva-s:} the deities, of the \emph{Pitṛ-s}: the forefathers and that of \emph{manuṣya}- the humans are all referred as \dev{असुर्या} because they all exist only in the \emph{vyāvahārika daśā} - the state of duality and exhibit different degrees of \emph{avidyā} /ignorance. Hence, compared to the \emph{pāramārthika satya} that \emph{Brahman} alone exist, all the states of dualities are called \dev{असुर्या}- Undivine, in the sense that it is Unreal, Untruth, only an apparent manifestation, a play, a \emph{līlā} of \emph{Brahman}.

\dev{अन्धेन तमसा आवृताः}-\emph{Engulfed in the blinding darkness}. The worlds, the entire manifestation are engulfed in blinding darkness of ignorance. For those, who have understood that the world is impermanent and that \emph{Brahman} alone is permanent, the world becomes an instrument of \emph{Brahman}. He will understand that \emph{Brahman} alone inhabits the world; He alone is its creator and the ruler. It is His play. Hence, such a renounces the world, renounces his desires and attachments to the worldly objects. But, those who are attached to sensory objects and are bound by \emph{avidyā}, they mistake the apparent reality as absolute truth, they mistake the impermanent world to be permanent truth and are ever indulgent in the pleasures of the world. Such people are ever in the \emph{darkness of} \emph{ignorance} \emph{and} \emph{inertia}. This ignorance binds them to the Karmic cycle and blinds them to the reality of \emph{Brahman}. Hence, the \emph{Upaniṣad} calls the \emph{saṁsāra} as ``undivine worlds which are engulfed in darkness''

\dev{ताँस्ते प्रेत्याभिगच्छंति}- \emph{There into those worlds, they go after death.} People who are ever indulgent in the \emph{saṁsāra}, after death come back to \emph{saṁsāra} in order to face the results of previous actions. Such individuals will be eternally stuck in the Karmic cycle, sometimes gaining better births to enjoy the fruits of good actions and at other times gaining disadvantaged births to enjoy fruits of bad actions. Here \dev{प्रेत्य} refers to the death of the Physical body. Every individual has three kinds of bodies (\emph{śarīra-s}) (3) - \emph{sthūla} (gross), \emph{sūkṣma} (subtle) and \emph{kāraṇa} (causal/the storage of one's \emph{karma-s} and \emph{vāsanā}-s). After the death of \emph{sthūla śarīra}, a person may either directly take a rebirth in the Physical universe or first enter different worlds like that of forefathers (\emph{pitṛ-s}), the realm of spirits (the \emph{preta-s}), the heavens (\emph{devaloka}), or the hells (\emph{narakaloka}) depending upon the karmic fruits that awaits to bear fruit and then take a rebirth in this physical universe. In both cases, a \emph{jīva} will be stuck eternally in this cycle of birth-death-rebirth.

\dev{ये के च आत्मा-हनो जनाः}- \emph{Those who are slayers of the Self}. People who are always immersed in the sensory world, enjoying in its pleasures and sobbing in its sorrows, taking the apparent world to be the only Truth, slay the \emph{Ātma}, the Self, the Truth by forgetting, by denying that the whole world is a manifestation of \emph{Brahman} alone, that an individual's true Self, is \emph{Brahman} itself. The word \dev{आत्मा-हनः} does not literally mean ``killing of the \emph{Ātma}'', as the \emph{Ātma} is beyond birth and death. Here \dev{हनः} refers to denial --of \emph{Brahman}, of unity of existence, of ultimate reality (\emph{pāramārthika satya}). A person immersed completely in the sensory world is completely bound by the eight pashas. His outlook is completely materialistic, always pursuing primal instincts, never satisfied, always hungry for more. Such people who are selfish, greedy, adharmic; those who deny \emph{Brahman}, deny their own true Self, are the slayers of \emph{Ātma}.

\textbf{Summary}: Those people who have purified their minds and have developed an understanding that \emph{Brahman} alone is truth, He alone is beauty and bliss-eternal will renounce all desires, all attachments to worldly things that cannot grant them the \emph{ānanda} - the eternal bliss. Renouncing the attachments thus, one will contemplate on \emph{Ātma}, one's true Self with a single pointed dedication, concentration and surrendering.

Then there are those people who have faith in the \emph{śāstra-s} and tradition but lack the purification of mind; those who have a desire to reach \emph{Brahman} but also harbor desires for worldly pleasures; those that find themselves unable to free themselves from worldly attractions but wish to attain a permanent state of happiness; such people are instructed by the \emph{Upaniṣad} to perform their \emph{Karma-s}, their spiritual as well as worldly obligations towards their family and society with a sense of commitment and responsibility. And they must do so in a manner that they are within the boundaries of \emph{Dharma}/Righteousness. Such \emph{Karma-s} performed will eventually lead a person to develop \emph{niṣkāma daśā} (detached state) paving way for the purification of his mind.

But the majority of people neither have the purification of mind and the resulting dispassion towards worldly desires and single-pointed contemplation on \emph{Brahman}, nor have the any faith, surrendering, detachment, contentment and a sense of virtue. They are completely bound by the ``\emph{aṣṭa (eight) pāśa-s}'' and afflicted by ``\emph{ṣaḍ ṛpu-s} (six enemies)(4)''. They set their minds only on self-serving desires and means of achieve it, without regard for truth or \emph{Dharma}. They are ever indulged in satisfying their fantasies and ego. Such people are neither in \emph{nivṛttimārga} --the path towards \emph{mokṣa} nor in \emph{pravṛttimārga} --the path of \emph{Dharma}. They are only indulged in \emph{kāma} and \emph{artha} without any regard for truth, duty, knowledge, righteousness, love or God. Hence, these people who slay truth in every thought and action by resorting to untruth and dishonesty, who slay love with hatred and apathy, who slay knowledge with conceit and slay righteousness with selfishness --such people are called \dev{आत्मा-हनो जनाः}. \emph{Brahman} who inhabits everything in the universe, who is the source of all the things, who exists in all beings as their \emph{Ātma}, is `\emph{Slayed}' by these people who are completely indulgent in sensory objects, regarding untruth as truth, temporary as permanent, bad as good and ignorance as knowledge. They kill the \emph{Ātma}, by denying It, by refusing to acknowledge that it is \emph{Brahman} which has become the universe, that there is a divine spark in every person and every object, and by refusing to travel from untruth to truth, from darkness to light.

Hence, such people, upon death are not liberated from the Karmic cycle, nor are they able to achieve better birth within \emph{saṁsāra}; Instead they will be reborn again and again into various worlds and as various beings- man, animal, spirit, etc., to face the fruits of their Karmic action.

\section*{References}

\begin{enumerate}
\item
  The \emph{Pāramārthika satya} refers to the absolute non-dual Truth. An absolute state of reference, where only \emph{Brahman}/\emph{Ātma} is. There is no difference between \emph{Brahman} or individual or the world. \emph{Gauḍapādācārya} says- ``There is neither creation, nor destruction; There is neither a \emph{sādhaka} (performer of action) nor a \emph{mumukṣu} (seeker of \emph{mokṣa}); There is neither bondage, nor liberation; This is \emph{pāramārthika}'' (Verse-32, \emph{Vaithathya Prakāraṇa}, \emph{Gauḍapāda Kārikā} on \emph{Māṇḍukyopaniṣad})
\item
  The \emph{vyāvahārika} state refers to the dual (\emph{dvaita}) state of reference. The perception of Reality is through the duality of object and the subject. There is the world (\emph{jagat}) and there is Individual (\emph{jīva}) and the God (\emph{Iśvara}) all separate. Sage \emph{Yājñavalkya} while discussing with his wife \emph{Maitreyī} about the nature of this Cosmos says- ``Where the Duality is present, there one can smell the fragrance, one can speak to others, one can listen to others, one will pay respect to others, and one can think and understand. (\emph{vyāvahārika daśā}) But, where there is only \emph{Ātma} everywhere, what will he smell? Whom will he ask? What will he listen to? Whom will he pay respect to? What will he think about and understand? (\emph{pāramārthika daśā})'' (\emph{Bṛhadāraṇyakopaniṣad} 2.4.14)
\item
  \emph{Śarīra-s} have been explained during explanation of Verse 8.
\item
  The \emph{ṣaḍ ṛpu-s} /six enemies of man are: \emph{kāma} (lust), \emph{lobha} (greed), \emph{krodha} (anger), \emph{mada} (pride), \emph{moha} (attraction), \emph{mātsarya} (jealousy).
\end{enumerate}
