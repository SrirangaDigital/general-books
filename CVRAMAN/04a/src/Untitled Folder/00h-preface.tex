
\chapter*{ಮೊದಲ ಮಾತು}


ಭಾರತವು ಜಗತ್ತಿಗೆ ನೀಡಿದ ಅತಿ ಶ್ರೇಷ್ಠ ವಿಜ್ಞಾನಿಯೆಂದರೆ ಸರ್ ಸಿ. ವಿ. ರಾಮನ್. ಅವರು ತಮ್ಮ\break ವೈಜ್ಞಾನಿಕ ಕಾರ್ಯಕ್ಕಷ್ಟೇ ಅಲ್ಲದೆ ಭಾರತವನ್ನು ವಿಜ್ಞಾನ ರಂಗದಲ್ಲಿ ಪ್ರತಿಷ್ಠಾಪಿಸಿದ್ದಕ್ಕೂ ಖ್ಯಾತಿ\break ಹೊಂದಿದರು. ಭಾರತದ ಪ್ರಥಮ ವಿಜ್ಞಾನದ ವಕ್ತಾರರೂ, ವರ್ಣರಂಜಿತ ವ್ಯಕ್ತಿತ್ವ ಉಳ್ಳವರೂ\break ಆಗಿದ್ದರು. ವೈಜ್ಞಾನಿಕ ಪರಂಪರೆಯಿಲ್ಲದ ದೇಶದಲ್ಲಿ, ವಿಜ್ಞಾನದ ಅಧ್ಯಯನದ ಆಯ್ಕೆಗೂ\break ಉತ್ತೇಜನವಿರಲಿಲ್ಲ. ಉಪಕರಣಗಳಂತೂ ಇಲ್ಲವೇ ಇಲ್ಲದಂತಹ ಕಾಲದಲ್ಲಿಯೂ, ವೈಜ್ಞಾನಿಕ ಶೋಧನೆಯ ಉತ್ತುಂಗಕ್ಕೇರಿದವರು ರಾಮನ್. ವಿಜ್ಞಾನ ಕ್ಷೇತ್ರದಲ್ಲಿ ಮುಂದುವರೆಯಬೇಕೆಂದು ಪ್ರತಿಷ್ಠಿತ ಸರ್ಕಾರಿ ಕೆಲಸವನ್ನು ತೊರೆದು, ಅಲ್ಪ ಮೊತ್ತದ ಸಂಬಳವಾದರೂ ಪ್ರಾಧ್ಯಾಪಕ ಸ್ಥಾನವನ್ನು\break ಆಯ್ದುಕೊಂಡರು. ಇದರ ಪ್ರತಿಫಲವಾಗಿ ಅವರು ಕಂಡು ಹಿಡಿದ, ಬೆಳಕಿನ ಚದುರುವಿಕೆಯ\break “ರಾಮನ್ ಪರಿಣಾಮ” ಕ್ಕೆ 1930 ರಲ್ಲಿ ಅವರಿಗೆ ನೊಬೆಲ್ ಬಹುಮಾನ ಸಂದಿತು. 

ವೈಜ್ಞಾನಿಕ ಸಂಶೋಧನಾ ವಿಧಾನದಲ್ಲಿ ಅನೇಕ ಪೀಳಿಗೆಗಳನ್ನು ತಯಾರು ಮಾಡಿದ ಮೇರು ವ್ಯಕ್ತಿತ್ವ ಅವರದು. ಮೊದಲಿಗೆ ಕಲ್ಕತ್ತದಲ್ಲಿ ಭೌತಶಾಸ್ತ್ರ ಅಧ್ಯಯನ ಪೀಠ ತೆರೆದರು. ಅಲ್ಲಿಗೆ ದೇಶದ ಎಲ್ಲಾ ಕಡೆಗಳಿಂದಲೂ ಶಿಷ್ಯರು ಬಂದರು. ಈ ಶಿಷ್ಯರಲ್ಲಿ ವೈಜ್ಞಾನಿಕ ಕುತೂಹಲವನ್ನೂ ಅಧ್ಯಯನಾಸಕ್ತಿ\break ಯನ್ನೂ ಬೆಳೆಸಿ ವಿಜ್ಞಾನ ಸಂವಹನವನ್ನೂ ಸಂಶೋಧನೆಯ ಆನಂದವನ್ನೂ ಮನದಟ್ಟು ಮಾಡಿ\break ಕೊಟ್ಟರು. ಇವರಲ್ಲಿ ಅನೇಕ ಮಂದಿ ವಿದ್ಯಾರ್ಥಿಗಳು ತಮ್ಮ ಮುಂದಿನ ಜೀವನವನ್ನು ವಿಜ್ಞಾನಕ್ಕಾಗಿ ಮುಡುಪಾಗಿಟ್ಟರು. ಅಲ್ಲದೆ ವಿವಿದೆಡೆಗಳಲ್ಲಿ ವಿಜ್ಞಾನ ಪೀಠಗಳನ್ನು ಪ್ರಾರಂಭಿಸಿದರು. 

ರಾಮನ್‌ರವರು ಅತಿಶ್ರೇಷ್ಠ ಭಾಷಣಕಾರರೂ ಉಪನ್ಯಾಸಕಾರರೂ ಆಗಿದ್ದರು. ಅವರ\break ಉಪನ್ಯಾಸಗಳಲ್ಲಿ ಪ್ರಯೋಗ\enginline{-}ಪ್ರದರ್ಶನಗಳೂ ಇದ್ದು ಕೇಳುಗರನ್ನು ಮಂತ್ರಮುಗ್ಧವಾಗಿಸುತ್ತಿದ್ದವು. ಅವರ ಮಾತುಗಳನ್ನು ಕೇಳಿದವರಿಗೆ ಅವರು ವಿಷಯವನ್ನು ಮಂಡಿಸುವ ರೀತಿಯೂ ಅವರು ಸೃಷ್ಠಿಸುವ ಉತ್ಸಾಹವೂ ಮರೆಯಲಾಗದ ಸಂಗತಿಗಳಾಗಿದ್ದವು. ಅವರ ಸಂಶೋಧನಾ ಪ್ರಬಂಧಗಳು ಅತಿ ಜಾಗರೂಕತೆಯಿಂದ ಸಿದ್ಧಪಡಿಸಿದವುಗಳಾಗಿದ್ದು ವಿಷಯ ಮಂಡನೆಗಾಗಿ ಅಲ್ಲಲ್ಲಿ ಲ್ಯಾಟಿನ್ ಭಾಷೆಯ\break ಹೇಳಿಕೆಗಳಿಂದ ತುಂಬಿರುತ್ತಿದ್ದವು. ಅವರ ಇಂಗ್ಲಿಷ್ ಭಾಷೆಯೂ ಸಹ ಅತಿ ಸರಳವಾಗಿ ಗಂಭೀರವಾಗಿ ಹರಿಯುತ್ತಿತ್ತು. ಅವರ ವೈಜ್ಞಾನಿಕ ಪ್ರಬಂಧಗಳನ್ನು ಓದುವುದೆಂದರೆ ಸಾಹಿತ್ಯ ಕೃತಿಯೊಂದನ್ನು\break ಆಸ್ವಾದಿಸಿದಂತೆ ಇರುತ್ತಿತ್ತು.


\eject

ಅವರು ಅನೇಕ ವೈಜ್ಞಾನಿಕ ನಿಯತಕಾಲಿಕೆಗಳನ್ನು ಶುರು ಮಾಡಿ, ಅವನ್ನು ಬೆಳೆಸಿದರು. ಅವುಗಳ ಗುಣಮಟ್ಟವನ್ನು ಕಾಪಾಡಿಕೊಂಡು ಬಂದರು. ಗುಣಮಟ್ಟದ ವಿಜ್ಞಾನ ಕಾರ್ಯವಿದ್ದಂತೆಯೇ ಗುಣಮಟ್ಟದ ಪತ್ರಿಕಾ ವ್ಯವಸಾಯವೂ ಜೊತೆಯಾಗಿ ನಡೆಯಬೇಕೆಂದು ನಂಬಿದರು. ಅವರ ಬಹಳಷ್ಟು ಸಂಶೋಧನಾ ಪ್ರಬಂಧಗಳನ್ನು ಈ ನಿಯತಕಾಲಿಕಗಳಲ್ಲೇ ಪ್ರಕಟಿಸುತ್ತಿದ್ದರು. ಇವುಗಳಲ್ಲೇ\break ಅವರ ಸಹೋದ್ಯೋಗಿಗಳಿಗೂ, ವಿದ್ಯಾರ್ಥಿಗಳಿಗೂ ಪ್ರಕಟಿಸಲು ಉತ್ತೇಜಿಸುತ್ತಿದ್ದರು. ಅವರು ವೈಜ್ಞಾನಿಕ ಆವಿಷ್ಕಾರಗಳನ್ನು ಚರ್ಚಿಸಲು ಬೆಂಗಳೂರಿನಲ್ಲಿ ಸೈನ್ಸ್ ಅಕಾಡೆಮಿ ಸ್ಥಾಪಿಸಿದರು. ಇದರಲ್ಲಿನ\break ಸಭೆಗಳ ಮೂಲಕವೂ ಪ್ರಕಟಣೆಗಳು ಮತ್ತು ಉಪನ್ಯಾಸಗಳ ಮೂಲಕವೂ ವಿಜ್ಞಾನವನ್ನು\break ಸಹಕಾರ್ಯಕರ್ತರಿಗೂ ಅಕಾಡೆಮಿಯ ಫೆಲೋಗಳಿಗೂ ಸಂವಹನ ಮಾಡಲು ಉದ್ದೇಶಿಸಿದರು. ಅವರು ಅಕಾಡೆಮಿಯ ವ್ಯವಹಾರಗಳಲ್ಲಿ ತೀವ್ರ ಆಸಕ್ತಿ ತೋರಿದರು. ಯುವ ವಿಜ್ಞಾನಿಗಳು ಉನ್ನತ ಕಾರ್ಯ ಮಾಡಿ ಪ್ರಸಿದ್ಧರಾಗುವ ಮೊದಲೇ, ಅಕಾಡೆಮಿಯ ಫೆಲೋಗಳಾಗಿ ಚುನಾಯಿತರಾಗುವಂತೆ ಮಾಡುತ್ತಿದ್ದರು. ಯಾರೊಬ್ಬರ ಸಾಮರ್ಥ್ಯವನ್ನೂ, ಕೌಶಲವನ್ನೂ ಶೀಘ್ರವಾಗಿ ಕಂಡುಕೊಳ್ಳುವ ಕಲೆ ಅವರಿಗೆ ಸಿದ್ಧಿಸಿತ್ತು.


ಇಂಡಿಯನ್ ಇನ್ಸ್ಟಿಟ್ಯೂಟ್ ಆಫ್‌ ಸೈನ್ಸ್ ನಿಂದ ನಿವೃತ್ತರಾದ ಮೇಲೆ, ವಿಜ್ಞಾನ ಕಾರ್ಯದಿಂದ\break ವಿಮುಖರಾಗವುದು ಅವರಿಗೆ ಅಸಾಧ್ಯವೆನಿಸಿ, ರಾಮನ್ ರಿಸರ್ಚ್ ಇನ್ಸ್ಟಿಟ್ಯೂಟ್ ಅನ್ನು ಸ್ಥಾಪಿಸಿದರು. ವಿಜ್ಞಾನದ ಕೆಲಸವೆಂದರೆ ಅವರಿಗೆ ಆನಂದದ ಕಾರ್ಯವಾಗಿತ್ತು. ಅವರಿಗೆ ವಿಜ್ಞಾನವೇ ಉಸಿರಾಗಿತ್ತು. ಭೌತಶಾಸ್ತ್ರದಿಂದ ಜೀವಶಾಸ್ತ್ರದವರೆಗೆ ಅವರ ವಿಜ್ಞಾನಾಸಕ್ತಿ ಹರಡಿತ್ತು. ಅವರನ್ನು ನಿಜವಾದ ನ್ಯಾಚುರಲ್ ಫಿಲಾಸಫರ್ ಎಂದು ಕರೆಯಬಹುದಾಗಿತ್ತು. ಇಂದಿನ ದಿನಗಳ ವಿಶೇಷಜ್ಞರ (ಸ್ಪೆಷಲೈಸೇಷನ್) ಹಾವಳಿಯಲ್ಲಿ ಇಂತಹ ಸರ್ವಜ್ಞರು ಬೆರಳೆಣಿಕೆಯ ಮಂದಿಯಷ್ಟೆ  ಸಿಗುತ್ತಾರೆ.


ರಾಮನ್ ಅವರು ತಮ್ಮ ಆಲೋಚನೆಗಳಲ್ಲೂ ಕೆಲಸದಲ್ಲೂ ಪೂರ್ಣ ಸ್ವತಂತ್ರರಂತೆ ವರ್ತಿಸು\break  ತ್ತಿದ್ದರು. ಅವರು ತಮ್ಮ ಅನಿಸಿಕೆಯನ್ನು ನೇರವಾಗಿ ಹೇಳಲು ಹಿಂಜರಿಯುತ್ತಿರಲಿಲ್ಲ. ಅವರಿಗೆ\break ರಾಜಮಹಾರಾಜರು, ರಾಜಕಾರಣಿಗಳು, ಮತ್ತು ಸಾರ್ವಜನಿಕರು ಬಹಳ ಮರ್ಯಾದೆ ಕೊಡುತ್ತಿದ್ದರು. ಭಾರತದ ಉದ್ದಕ್ಕೂ ಸಿ.ವಿ.ರಾಮನ್ ಅವರ ಹೆಸರಿಗೆ ಗೌರವ ಮತ್ತು ಮೆಚ್ಚುಗೆ ವ್ಯಕ್ತವಾಗುತ್ತಿತ್ತು.

ರಾಮನ್ ಅವರು ತಮ್ಮ ಸೌಂದರ್ಯ ಪ್ರಜ್ಞೆಯನ್ನು ಹರಿತಗೊಳಿಸಿಕೊಂಡಿದ್ದರು. ಪ್ರಕೃತಿಯನ್ನು ಪ್ರೀತಿಸುತ್ತಿದ್ದರು. ವರ್ಣಗಳೆಂದರೆ ಇಷ್ಟ. ಅದು ಮರಗಿಡಗಳಲ್ಲಿ, ತೋಟಗಳಲ್ಲಿ, ಹೂಗಳಲ್ಲಿ, ಬೆಟ್ಟಗಳಲ್ಲಿ, ಸರೋವರಗಳಲ್ಲಿದ್ದರೂ ಸರಿಯೇ. ಪ್ರಾಕೃತಿಕ ವಿದ್ಯಮಾನಗಳನ್ನೂ ಹಾಗೂ ಪ್ರಕೃತಿಯನ್ನೂ\break ಸಹ ಬಹಳ ಕುತೂಹಲದಿಂದ ವೀಕ್ಷಿಸುತ್ತಿದ್ದರು. ವಿಷಯವೊಂದು ಅವರ ಗಮನಕ್ಕೆ ಬಂದರೆ, ಯಾವುದೇ ಪೂರ್ವಾಗ್ರಹವಿಲ್ಲದೆ ಅದರ ಹಿಂದೆ ಬೀಳುತ್ತಿದ್ದರು. ಆ ವಿಷಯದ ಬಗ್ಗೆ ಅದುವರೆವಿಗೂ ತಿಳಿದಿದ್ದ ಎಲ್ಲ ಜ್ಞಾನವನ್ನೂ ಅರಿತುಕೊಂಡು ಪ್ರಶ್ನಿಸುತ್ತಿದ್ದರು. ಒಂದು ಸಂಕೀರ್ಣ ಸಮಸ್ಯೆಯನ್ನು ಅದರ ಮೂಲಭೂತ ಮತ್ತು ಸರಳ ಪ್ರಶ್ನೆಗಳಾಗಿ ವಿಭಜಿಸಿಕೊಳ್ಳುವ ಕಲೆ ಅವರಿಗಿತ್ತು. ಸಂಶೋಧಕನೊಬ್ಬ\break ಕಾಡಿನಲ್ಲಿ ಹೊರಟಾಗ ಎಲೆಗಳ ದಟ್ಟಣೆಯಲ್ಲಿ ಕಳೆದು ಹೋಗದೆ, ಮೂಲ ವೃಕ್ಷಗಳನ್ನು ಅರಸ\break ಬೇಕೆಂದು ಹೇಳುತ್ತಿದ್ದರು. ರಾಮನ್ ಅವರು ಪ್ರಯೋಗಗಳಲ್ಲಿ ನಿಷ್ಣಾತರು. ಒಂದು ಭೌತಿಕ ಪ್ರಕ್ರಿಯೆಯನ್ನು ಅರಿಯುವಾಗ ಅವರ ಅಂತರ್ದೃಷ್ಟಿಯು ಗಣಿತದ ಲೆಕ್ಕಾಚಾರಗಳ ಬಂಧದಿಂದ ಹಲವು ಪಟ್ಟು ದೂರಕ್ಕೆ ಜಿಗಿಯುತ್ತಿದ್ದಿತು.


ತಮ್ಮ ಸಹೋದ್ಯೋಗಿಗಳೊಂದಿಗೂ, ವಿದ್ಯಾರ್ಥಿಗಳೊಂದಿಗೂ ರಾಮನ್ ಅವರು ಮುಕ್ತವಾಗಿ\break ತೆರೆದುಕೊಳ್ಳುತ್ತಿದ್ದರು. ಅವರು ಕರಣಾಮಯಿ ಕೂಡ. ಒಳ್ಳೆಯ ಕೆಲಸಗಳಿಗೆ ಶೀಘ್ರ ಪ್ರಶಂಸೆ\break ವ್ಯಕ್ತಪಡಿಸುತ್ತಿದ್ದರು. ಅಗತ್ಯ ಘಳಿಗೆಯಲ್ಲಿ ಅತೀವ ಉತ್ತೇಜನ ನೀಡಬಲ್ಲವರಾಗಿದ್ದರು. ಆದರೆ ಅವರನ್ನು ಕೆಣಕಿದರೆ ರೌದ್ರಾವತಾರ ತಾಳುತ್ತಿದ್ದರು. ತಮ್ಮ ಜೀವಿತಾವಧಿಯಲ್ಲಿ ಅನೇಕ ಸಂದಿಗ್ಧ\break ಪರಿಸ್ಥಿತಿಗಳನ್ನು ಎದುರಿಸಿದ್ದರು. ಅದು ವಿಜ್ಞಾನ ರಂಗದಲ್ಲೂ ಹೌದು, ಸಾರ್ವಜನಿಕ ರಂಗದಲ್ಲೂ ಹೌದು. ಆದರೆ ಎಂದಿಗೂ ನಿರಾಶೆ ಹೊಂದಲಿಲ್ಲ. ವಿಜ್ಞಾನ ಕಾರ್ಯದಲ್ಲಿ ಮಗ್ನರಾಗಿ ಉತ್ಸಾಹದಿಂದ ಪುಟಿಯುತ್ತಿದ್ದರು. 

\medskip

ವಿಷಮ ಪರಿಸ್ಥಿತಿಗಳಿದ್ದಾಗ್ಯೂ ಅಸಾಮಾನ್ಯ ಸಾಧನೆ ಮಾಡಲು  ರಾಮನ್ ಅವರಿಗೆ ಸಾಧ್ಯ\break ವಾದದ್ದು ಹೇಗೆ? ಹಿನ್ನೆಲೆಗೆ ಯಾವ ಪ್ರೇರಣೆ ಇದ್ದಿತು? ಅವರ ಗೆಲುವಿನ ಗುಟ್ಟೇನು? ಈ ಪ್ರಶ್ನೆಗಳಿಗೆ ಸರಳ ಉತ್ತರಗಳಿಲ್ಲ. ಬಹುಷಃ ಹುಡುಕ ಹೊರಡುವ ಎಷ್ಟು ಮನಸ್ಸುಗಳಿವೆಯೋ ಅಷ್ಟೇ ಉತ್ತರಗಳೂ ಇವೆಯೆನ್ನಿಸುತ್ತದೆ. ಆದರೂ ಅವರ ಜೀವನದ ಸಂಗತಿಗಳೂ, ದಾಖಲೆಗಳೂ ಇವೆಯಲ್ಲ \enginline{-}ಇವನ್ನು\break ನಾವು ಅವಲೋಕಿಸಬಹುದು. ಇವುಗಳು ನಮ್ಮನ್ನು ಬೆರಗುಗೊಳಿಸುವುದು. ಹಾಗೆಯೇ ಹೆಚ್ಚಿನ\break ಅರಿವಿನತ್ತ ನಡೆಸುವುವು.

\medskip

1949 ರಿಂದ 1960ರ ವರೆಗಿನ ಹನ್ನೊಂದು ವರ್ಷಗಳಲ್ಲಿ ಅವರ ಹತ್ತಿರದ ಒಡನಾಟವಿದ್ದದ್ದು ನನ್ನ ಅದೃಷ್ಟವೆಂದೇ ತಿಳಿದಿದ್ದೇನೆ. ಪ್ರತಿ ದಿನವೂ ನನಗೆ ಅವರ ಸಂಪರ್ಕವಿತ್ತು. ನನ್ನ ಬಗ್ಗೆ ಅವರು ತೀವ್ರ ಅನುಕಂಪ ಮತ್ತು ಔದಾರ್ಯ ತೋರುತ್ತಿದ್ದರು. ನನ್ನಲ್ಲಿ ಅವರಿಗೆ ಬಹಳ ನಂಬಿಕೆಯಿತ್ತು. ನನ್ನನ್ನು ತಮ್ಮ\break ಮಗನಂತೆಯೇ ನೋಡುತ್ತಿದ್ದರು. ಅವರ ಆಲೋಚನೆಗಳನ್ನೂ ಅಭಿಪ್ರಾಯಗಳನ್ನೂ ಹಾಗೆಯೇ\break ಕನಸುಗಳನ್ನೂ ನನ್ನಲ್ಲಿ ಹಂಚಿಕೊಳ್ಳುತ್ತಿದ್ದರು. ನನ್ನ ವಿಜ್ಞಾನ ರಂಗದ ಪೂರ್ಣ ಶಿಕ್ಷಣ ಮತ್ತು ಅನುಭವವನ್ನು ಈ ಕಾಲಾವಧಿಯಲ್ಲೇ ಪಡೆದುಕೊಂಡಿದ್ದೇನೆ. ನಾನು ಈಗ ತೊಡಗಿಸಿಕೊಂಡಿರುವ ಉದ್ಯೋಗವೂ ಸಹ ಇದರಿಂದಲೇ ಆಗಿದೆ. 

\medskip

ರಾಮನ್ ಅವರ ಜೀವನ ಚರಿತ್ರೆಯ ಪುಸ್ತಕಗಳು ಕೆಲವೇ ಇವೆ. \enginline{-}ಜಿ. ವೆಂಕಟರಾಮನ್‌‌ರವರದ್ದೂ\break ಸೇರಿ. ರಾಮನ್ರವರ ಹತ್ತಿರದ ನೋಟವು ಕೆಲವರಿಗೆ ಇಷ್ಟವಾಗಬಹುದು ಎಂದು ಭಾವಿಸಿದ್ದೇನೆ. ಅದರಲ್ಲೂ ವಿಜ್ಞಾನವನ್ನು ತಮ್ಮ ಜೀವನೋದ್ಯಮವಾಗಿಸಿಕೊಳ್ಳಲು ಇಚ್ಛಿಸುವ ಯುವಕರಿಗೆ, ನನ್ನ ಈ ಪುಸ್ತಕವು ವೈಯಕ್ತಿಕ ನೆಲೆಯಲ್ಲಿನ ಅನೇಕ ಸಂಗತಿಗಳನ್ನು ಒಳಗೊಂಡಿದೆ. ವಿಜ್ಞಾನದ ಸ್ವಲ್ಪ ಅರಿವಿದ್ದ\break ವರಿಗೂ ತಿಳಿಯುವಂತಹದಾಗಿದೆ. ನನ್ನ ಈ ನೆನೆಪಿನ ಗುಚ್ಛವನ್ನು ರಾಮನ್ ಅವರ ಪೂರ್ಣಜೀವನಕ್ಕೆ ವಿಸ್ತರಿಸಿದ್ದೇನೆ. ರಾಮನ್ ಅವರೊಂದಿಗೆ ನನ್ನ ಸಹವಾಸವು ಮುಗಿದು ಇದರ ಅನಂತರದ\break ವರ್ಷಗಳಿಗೆ ನಾನು ಇಲ್ಲಿ ಹೆಚ್ಚಿನ ಪುಟಗಳನ್ನು ವ್ಯಯಿಸಿದ್ದೇನೆ. ಏಕೆಂದರೆ ದೂರ ನಿಂತು ಹತ್ತಿರದ\break ದೃಶ್ಯಗಳನ್ನು ಕಂಡಾಗ ನಿಮಗೆ ಒಬ್ಬ ವ್ಯಕ್ತಿಯ ಸಂಪೂರ್ಣ ಚಿತ್ರಣವು ಸಿಗಬಲ್ಲುದು. ನಾನು ನನ್ನ ಮಟ್ಟಿಗೆ ಯಾವುದೇ ದಿನಚರಿಯನ್ನು ಇಡಲಿಲ್ಲ. ನನ್ನ ಬರಹವೆಲ್ಲಾ ನೆನಪಿನಿಂದಲೇ ಹೊರಬಂದಿದೆ.

\newpage	

ಈ ನೆನಪುಗಳನ್ನು ಬರೆಯುವಾಗ ನಾನು ಕೆಳಗಿನ ಕೃತಿಗಳ ಮೊರೆಹೋಗಿದ್ದೇನೆ\enginline{-}
\medskip

\begin{enumerate}
\item 1971 ರಲ್ಲಿ ಎಲ್. ಎ. ರಾಮದಾಸ್ ಅವರು ಇಂಡಿಯನ್ ಜರ್ನಲ್ ಆಫ್‌ ಫಿಸಿಕ್ಸ್ ಎಜುಕೇಶನ್ ನಲ್ಲಿ ರಾಮನ್ ಅವರ ಬಗ್ಗೆ ಬರೆದ ಎರಡು ಚೇತೋಹಾರಿ ಪ್ರಬಂಧಗಳು.

\item ಪ್ರೊ|| ಭಗವಂತಂ ರವರಿಂದ ರಚಿತವಾದ ರಾಮನ್ ಕುರಿತು ಸಂಕ್ಷಿಪ್ತವಾದರೂ ಅತಿನಿಖರ ಮತ್ತು ಸತ್ಯನಿಷ್ಠ ಜೀವನಾವಲೋಕನ. ಇದು ಆಂಧ್ರಪ್ರದೇಶ್ ಅಕಾಡೆಮಿ ಆಫ್‌ ಸೈನ್ಸಸ್‌ನಿಂದ ಪ್ರಕಟವಾಗಿದೆ.

\item ಅಕಾಡೆಮಿ ಆಫ್ಸೈನ್ಸಸ್ ನ ಪ್ರಕಟಣೆ ~“ದಿ ಫಸ್ಟ್ ಫಿಫ್ಟಿಯಿರರ್ಸ್” ನಲ್ಲಿನ ಉದ್ಧೃತ\break ಭಾಗಗಳು (1994)

\item ಇಂಡಿಯನ್ ಇನ್ಸ್ಟಿಟ್ಯೂಟ್ ಆಫ್ಸೆಯನ್ಸ್ ಬೆಂಗಳೂರಿನಲ್ಲಿ, ‘ಸರ್ ಸಿ.ವಿ ರಾಮನ್ ಮೆಮೋರಿಯಲ್ ಲೆಕ್ಟರ್’ ನಲ್ಲಿ ನೀಡಿದ ಪ್ರೊ. ಎಸ್.ರಾಮಶೇಷನ್ ರವರ ಭಾಷಣ ಮತ್ತು ಬೆಂಗಳೂರಿನ ಮ್ಯಾಕ್ಸ್ ಮುಲ್ಲರ್ ಭವನದ ರಜತ ಮಹೋತ್ಸವ ಸಂದರ್ಭದಲ್ಲಿ ಇವರೇ ಬರೆದ ‘ಸಿ.ವಿ.ರಾಮನ್ ಮತ್ತು ಜರ್ಮನ್ ಕನೆಕ್ಷನ್’ ಲೇಖನ. 

\item ಜುಲೈ 4, 1931ರ ಕಲ್ಕತ್ತಾ ಮುನಿಸಿಪಲ್ ಗೆಜೆಟ್ ನ ಉದ್ಧೃತ ಭಾಗಗಳು

\item  ದಿ ಜರ್ನಲ್ ಆಫ್ರಾಮನ್ ಸ್ಪೆಕ್ಟ್ರೋಸ್ಕೊಪಿಯಲ್ಲಿ ಪ್ರಕಟವಾದ ಸೊಮ್ಮರ್ ಫಿ಼ೕಲ್ಡ್ ರವರು, ರಾಮನ್ ಅವರ ಜೊತೆಗೆ ಸಂಭಾಷಿಸಿದ ವರದಿ\enginline{-} ಡಾ. ಜಿ. ತೊರ್ಕರ್ ಬರೆದದ್ದು

\item ಆಂಧ್ರ ಪ್ರದೇಶ್ ಅಕಾಡೆಮಿ ಆಫ್ಸೈನ್ಸಸ್ ನಲ್ಲಿ, ದಿವಂಗತ ಪ್ರೊ. ಕೆ.ಆರ್. ರಾಮನಾಥನ್ ರವರು “ದಿ ಗೋಲ್ಡನ್ ಜ್ಯೂಬಲಿ ಆಫ್ದಿ ಡಿಸ್ಕವರಿ ಆಫ್ರಾಮನ್ ಎಫೆಕ್ಟ್ 28/ಫೆ/1978” ಎಂಬ ಲೇಖನ. 

\item ಭವನ್ಸ್ ಜರ್ನಲ್ ಡಿಸೆಂಬರ್ 1970

\item  ‘ಪಸಾಡೆನ ಸ್ಟಾರ್’ ಪತ್ರಿಕೆಯ ಪ್ರಕಟಣೆಗಳನ್ನು ಕ್ಯಾಲಿಫೊರ್ನಿಯಾ ಇನ್ಸ್ಟಿಟ್ಯೂಟ್ ಆಫ್‌\break ಟೆಕ್ನಾಲಜಿಯ, ರಾಬರ್ಟ್ ಎ. ಮಿಲ್ಲಿಕನ್ ಲೈಬ್ರರಿಯ ಆರ್ಕೈವ್ಸ್ ನಿಂದ ಪೌಲಾ ಅಗ್ರನಾಟ್ ಹುರ್ ವಿಟ್ಸ್  ತಂದುಕೊಟ್ಟರು.

\item  “ಭಾರತೀಯ ಸಂಗೀತದಲ್ಲಿ ಚರ್ಮ ವಾದ್ಯಗಳು ಮತ್ತು ರಾಮನ್” \enginline{--}ಪ್ರೊ|| ಬಿ.~ಎಸ್. ರಾಮಕೃಷ್ಣರವರು ಸೈನ್ಸ್ ಟುಡೇಯಲ್ಲಿ ಬರೆದ ಲೇಖನಗಳು 1970.
\end{enumerate}

ಇವೆಲ್ಲ ಆಕರ ಗ್ರಂಥಗಳು/ಲೇಖನಗಳು ನನಗೆ ಬಹಳ ಉಪಯುಕ್ತವಾಗಿವೆ. ನಾನು ಇವರೆಲ್ಲರಿಗೂ ಆಭಾರಿಯಾಗಿದ್ದೇನೆ. ಪ್ರೊ|| ಎಸ್.ಚಂದ್ರಶೇಖರ್ ಅವರ ಅನೇಕ ವಾಕ್ಯಗಳನ್ನು ಬಳಸಲು ಅವರು ಅನುಮತಿ ನೀಡಿದ್ದಾರೆ. ಅವರಿಗೆ ನಾನು ಕೃತಜ್ಞ. ಹಲವಾರು ಬಗೆಗಳಲ್ಲಿ ನಾನು\break ಪ್ರೊ|| ಎ. ಕೆ. ರಾಮದಾಸ್ ಅವರಿಗೆ ಋಣಿ. ಅವರು ತಾಳ್ಮೆಯಿಂದ ನನ್ನ ಕೈಹೊತ್ತಿಗೆಯನ್ನು ಓದಿ ಅದಕ್ಕೆ ಮೌಲ್ಯಯುತವಾದ ಟೀಕೆ, ಟಿಪ್ಪಣಿಗಳನ್ನು ಮಾಡಿದ್ದಾರೆ. ಅವರ ತೀರ್ಥರೂಪರ (ಎ. ಎಲ್. ರಾಮದಾಸ್) ಬರಹಗಳು ಮತ್ತು ಸಂಗ್ರಹಗಳನ್ನು ನನಗೆ ಒದಗಿಸಿ ಈ ಪುಸ್ತಕಕ್ಕೂ ಸಹ ಮುನ್ನುಡಿ\break ಯನ್ನು ಬರೆದು ಉಪಕರಿಸಿದ್ದಾರೆ.

\eject

ಈ ಕೈಬರಹದ ಪ್ರತಿಯನ್ನು ಸಿದ್ಧಪಡಿಸುವಾಗ ನನ್ನ ಜೊತೆಗಾರರಾದ ರಾಲ್ಫ್ ಜಿ. ಮೈನ್ಸ್ ರವರು ಬಹಳಷ್ಟು ಸಹಾಯ ಮಾಡಿದ್ದಾರೆ. ನಾನು ಇಲ್ಲಿ ಅವರನ್ನು ನೆನೆಯಲೇಬೇಕು. ಅವರಿಗೆ ಅಭಿನಂದನೆಗಳು ಶ್ರೀಮತಿ ಅಲೈನ್ ಇ. ಬೊನ್ನೆಲ್ ರವರು ನಮ್ಮ ಟೆಕ್ಸ್ಟ್ ಪ್ರೊಸೆಸ್ಸಿಂಗ್ ಸೆಂಟರ್ ನಲ್ಲಿ ಇದ್ದಾರೆ. ನನ್ನ ಕೆಲಸವನ್ನು ಬಹಳ ಆಸಕ್ತಿಯಿಂದಲೂ, ಜತನದಿಂದಲೂ ಟೈಪ್ ಮಾಡಿ ಮುಗಿಸಿದ್ದಾರೆ ನಾನು ಅವರಿಗೆ ಋಣಿ.

\medskip

ಪತ್ನಿ ಕಮಲಳು ನನ್ನ ಜೀವನದುದ್ದಕ್ಕೂ ನನಗೆ ಉತ್ತೇಜನ ನೀಡಿದ್ದಾಳೆ. ನನ್ನ ಕೈಬರಹದ\break ಪ್ರತಿಯನ್ನು ಓದಿ ಹಲವು ಟೀಕೆ ಟಿಪ್ಪಣಿಗಳನ್ನು ಮಾಡಿದ್ದಾಳೆ. ಹೀಗಾಗಿ ವಿಜ್ಞಾನ ಓದಿಲ್ಲದವರ ನೋಟ ನನಗೆ ಲಭ್ಯವಾಯಿತು. ಇದಕ್ಕೂ ನಾನು ಆಭಾರಿಯಾಗಿದ್ದೇನೆ. AT\&T ರವರ ಸಹಕಾರ ಮತ್ತು ಬೆಂಬಲವಿಲ್ಲದಿದ್ದರೆ ಇದು ಅಸಾಧ್ಯವಾಗುತ್ತಿತ್ತು. ನಾನಿಲ್ಲಿ ಕಳೆದ ಇಪ್ಪತ್ತೈದು ವರ್ಷಗಳಿಂದ ಕೆಲಸ\break ಮಾಡುತ್ತಿದ್ದೇನೆ.

\begin{flushright}
\textbf{ಎ. ಜಯರಾಮನ್}\\
18 ಮೇ 1989\\
AT\&T ಬೆಲ್ ಲ್ಯಾಬೊರೇಟರೀಸ್\\
ಮುರ್ರೆ ಹಿಲ್. ನ್ಯೂ ಜೆರ್ಸಿ\enginline{-}07974\\
ಯು.ಎಸ್.ಎ.
\end{flushright}
