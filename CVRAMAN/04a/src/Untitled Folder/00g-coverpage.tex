
\newpage
~ 

\thispagestyle{empty}

~\phantom{a}



\noindent
\begin{minipage}[c]{6cm}
\textbf{\large{ಗ್ರಂಥಕರ್ತರು}}

\medskip
\medskip

{\parfillskip=0pt
ರಾಮನ್~ರಿಸರ್ಚ್~ಇನ್ಸ್ಟಿಟ್ಯೂಟ್~ಸ್ಥಾಪನೆಗೊಂಡಂದಿ\enginline{-}\break ನಿಂದಲೂ ಡಾ।। ಎ. ಜಯರಾಮನ್‍‌ರವರು ಪ್ರೊ. ರಾಮನ್ ರವರೊಂದಿಗೆ ನಿಕಟ ಸಂಪರ್ಕದಲ್ಲಿದ್ದರು. ರತ್ನ, ವಜ್ರ ಮತ್ತು ಖನಿಜಗಳ ಭೌತಶಾಸ್ತ್ರ ಸಂಶೋಧನೆಗಳಲ್ಲಿ ಸುಮಾರು \enginline{11} ವರ್ಷಗಳ ಕಾಲ ಕೆಲಸ ಮಾಡಿದರು. \enginline{1960} ರಲ್ಲಿ \enginline{UCLA} ದಲ್ಲಿನ ಇನ್ಸ್ಟಿಟ್ಯೂಟ್ ಆಫ್ ಜಿಯೋಫಿಸಿಕ್ಸ್‌ನಲ್ಲಿ ಫೆಲೋಶಿಪ್ ‍ದೊರಕಿಸಿಕೊಂಡು ಪ್ರೊಫೆ಼ಸರ್ ಜಾರ್ಜ್ ಕೆನ್ನಡಿಯವರೊಂದಿಗೆ ಕೆಲಸ ಮಾಡಲು ಹೊರಟರು. ಅಲ್ಲಿ ಅವರಿಗೆ ಅಧಿಕ ಒತ್ತಡ ಭೌತಶಾಸ್ತ್ರ ಸಂಶೋಧನೆಗೆ ಪ್ರವೇಶ ಸಿಕ್ಕಿತು. \enginline{1963} ರಲ್ಲಿ ಡಾ. ಜಯರಾಮನ್ ರವರು ಭೌತಶಾಸ್ತ್ರ ಸಂಶೋಧನಾ ವಿಭಾಗದಲ್ಲಿ ಬೆಲ್ ಲ್ಯಾಬೊರೇಟರೀಸ್‌ಗೆ ಸೇರಿದರು ಮತ್ತು 1990 ರಲ್ಲಿ ನಿವೃತ್ತರಾದರು. ಆ ಅವಧಿಯಲ್ಲಿ \enginline{(1970)} ಭಾರತಕ್ಕೆ ದೀರ್ಘ ರಜೆಯಲ್ಲಿ ಆಗಮಿಸಿ ದೇಶದಲ್ಲಿ ಪ್ರಥಮ ಬಾರಿಗೆ ಅಧಿಕ ಒತ್ತಡ ಭೌತಶಾಸ್ತ್ರದ ಅಧ್ಯಯನಕ್ಕೆ ಬುನಾದಿ ಹಾಕಿದರು. ಹಾಗೆಯೇ \enginline{1991-92} ರಲ್ಲಿ ಭಾರತೀಯ ವಿಜ್ಞಾನ ಸಂಸ್ಥೆಯಲ್ಲಿನ ಭೌತಶಾಸ್ತ್ರ ವಿಭಾಗದಲ್ಲಿ ಅಧಿಕ ಒತ್ತಡ ರಾಮನ್ ರೋಹಿತ ಅಧ‍್ಯಯನಕ್ಕಾಗಿ ಪ್ರಯೋಗ ಶಾಲೆಯನ್ನು ಸಜ್ಜುಗೊಳಿಸಿದರು. ಇದರಲ್ಲಿ ಡೈಮಂಡ್ ಅನ್ವಿಲ್ ಸೆಲ್ ಬಳಸಿದರು. \enginline{1992} ರಲ್ಲಿ ಹೊನಲೂಲು ದ್ವೀಪದ ಮನೋವ ನಗರದ ಹವಾಯಿ ವಿಶ್ವವಿದ್ಯಾಲಯದಲ್ಲಿ ಹವಾಯಿಯನ್ ಇನ್ಸ್ಟಿಟ್ಯೂಟ್ ಆಫ್ ಜಿಯೋಫಿಸಿಕ್ಸ್ ಅಂಡ್ ಪ್ಲಾನೆಟಾಲಜಿ ಸಂಶೋಧನಾ ಪ್ರೊಫೆ಼ಸರ್ ಆಗಿ \enginline{5} ವರ್ಷ ಸೇವೆಗೈದರು. ಇದರ ಬಳಿಕ ವಾಷಿಂಗ್ಟನ್ ಡಿಸಿ ನಗರದ ಕಾರ್ನೇಗೀ ಸಂಸ್ಥೆಯ ಜಿಯೋಪಿಸಿಕ್ಸ್ ಲ್ಯಾಬ್‍ನಲ್ಲಿ ದುಡಿದರು (\enginline{1997} ರಿಂದ \enginline{2000}).\par} 
\end{minipage}

\newpage
~ 

\thispagestyle{empty}

~\phantom{a}



\noindent
\begin{minipage}[c]{6cm}
ಇದರ ನಂತರ ಪೂರ್ಣ ನಿವೃತ್ತರಾಗಿ ಅರಿಝೋನಾ ರಾಜ್ಯದ ಫೀನೀಕ್ಸ್ ನಗರದ ವಾಸಿಯಾಗಿದ್ದಾರೆ.  ಡಾ।। ಜಯರಾಮನ್‌ರವರಿಗೆ ಇಂಡಿಯನ್ ಜಿಯೋಫಿಸಿಕಲ್ ಯೂನಿಯನ್ನಿನ ಕೃಷ್ಣನ್ ಗೋಲ್ಡ್ ಮೆಡಲ್ \enginline{1968} ರಲ್ಲಿ ಸಂದಿದೆ. ನ್ಯೂ ಯಾರ್ಕ್‌ನ  ಜಾನ್ ನ್ಯೂಮನ್ ಗುಗೇನ್‌ಹೀಮ್ ಫೌಂಡೇಶನ್ನಿನ ಗ್ಯುಗೇನ‌ಹೀಮ್\break ಅವಾರ್ಡ್ \enginline{(1969)}, ಸಟ್‍ಗಾರ್ಟ್  ಜರ್ಮನಿಯಲ್ಲಿ ಮ್ಯಾಕ್ಸ್ ಪ್ಲಾಂಕ್ ಇನ್ಸ್ಟಿಟ್ಯೂಟ್‌ನಲ್ಲಿ ಸಂಶೋಧನೆ\break ಗೈಯಲು ಹಂಬೋಲ್ಟ್ ಸೀನಿಯರ್ ಯು.~ಎಸ್.\break ಸೈಂಟಿಸ್ಟ್ ಅವಾರ್ಡ್‌ಗಳು ಬಂದಿವೆ. ಹೀಗೆಯೇ ಡಿಸ್ಟಿಂಗ್ವಿಷ್ಡ್ ಸೈಂಟಿಸ್ಟ್ ಅವಾರ್ಡ್ (ಬೆಲ್ ಲ್ಯಾಬ್‍‌\break ನಿಂದ) 1983 ರಲ್ಲೂ, ರಾಮನ್ ಸೆಂಟಿನರಿ ಮೆಡಲ್\break \enginline{1988} ರಲ್ಲಿ ಸಿಕ್ಕಿವೆ. ಅಮೇರಿಕನ್ ಫಿಸಿಕಲ್ ಸೊಸೈಟಿ ಮತ್ತು ಇಂಡಿಯನ್ ಅಕಾಡೆಮಿ ಆಫ್ ಸೈನ್ಸಸ್‌ನ ಫೆಲೋ ಆಗಿದ್ದಾರೆ. 

\medskip
\medskip

\noindent
\textbf{ಹಿಂದಿನ ರಕ್ಷಾಪುಟ: }

\medskip

\noindent
ಲಾಬ್ರಡೋರೈಟ್ ಖನಿಜ ಮತ್ತು ರಾಮನ್‌ರವರ ನಿಧನದ  ಬಳಿಕ, ಅವರ ದೇಹವನ್ನು ದಹನ ಮಾಡಿದ ಸ್ಥಳದಲ್ಲಿರುವ ಬದರೀವೃಕ್ಷ (ಟೆಬೆಬ್ಯುಯ ಡೊನ್ನೆಲ್\enginline{--}ಸ್ಮಿಥಿ).  
\end{minipage}
